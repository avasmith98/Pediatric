\documentclass{article}
\usepackage[letterpaper, portrait, margin=1in]{geometry}
\usepackage{hyperref}\usepackage[utf8]{inputenc}
\hypersetup{colorlinks=true}
\title{Drug Safety Output}
\begin{document}
\maketitle
\section*{Clindamycin Phosphate}
\subsection*{Result}
\subsubsection*{Answer}

Yes (Neonates and infants <1 year, with appropriate dosing and monitoring)
Yes (Children 1–14 years)
Yes (Adolescents 12+ years, topical use)

\subsubsection*{{Explanation}}
\hypertarget{Clindamycin Phosphate}
Based on the abstracts available, several targeted studies have evaluated the safety of clindamycin phosphate in children across different age ranges and indications:

1. **Neonates and Infants (<1 year, including premature and term infants):**
   - Multiple pharmacokinetic and safety studies have been conducted in neonates and infants. One study of 62 infants with postnatal ages <121 days (median gestational age 28 weeks) found no adverse events related to clindamycin use, with dosing regimens adjusted for age and weight [\hyperlink{pmid_26926644}{PMID: 26926644}, Daniel Gonzalez et al., 2016]. Another large retrospective cohort of 4089 infants (median dose 15 mg/kg/d) found that clindamycin exposure was not associated with increased odds of death or serious adverse events, though there was a marginal increase in necrotizing enterocolitis at the highest exposures [\hyperlink{pmid_31725114}{PMID: 31725114}, Rachel G Greenberg et al., 2020]. Additional studies recommend dose adjustments for neonates due to slower clearance and longer half-life, but do not report significant safety concerns [\hyperlink{pmid_6470871}{PMID: 6470871}, M J Bell et al., 1984; \hyperlink{pmid_3737273}{PMID: 3737273}, G Koren et al., 1986].

2. **Children (1–14 years):**
   - Several studies have evaluated clindamycin phosphate (oral and intravenous) for various infections (osteomyelitis, malaria, streptococcal pharyngitis, and skin infections) in children. In a study of 29 children with osteomyelitis (age not specified but described as children), clindamycin was well tolerated with no cases of diarrhea or enterocolitis, even with high doses for up to nine weeks [\hyperlink{pmid_910760}{PMID: 910760}, W Rodriguez et al., 1977]. For malaria, studies in children aged 1–14 years found clindamycin (in combination with fosmidomycin) to be generally well tolerated, though some concerns about neutropenia and hemoglobin drops were noted [\hyperlink{pmid_16870763}{PMID: 16870763}, Steffen Borrmann et al., 2006]. In children aged 7–14 years, the combination was safe and well tolerated with no serious adverse events [\hyperlink{pmid_15478056}{PMID: 15478056}, Steffen Borrmann et al., 2004; \hyperlink{pmid_14976608}{PMID: 14976608}, Steffen Borrmann et al., 2004]. For streptococcal pharyngitis, clindamycin palmitate was as effective as penicillin, with rash as the main adverse event (8/52 patients) [\hyperlink{pmid_4208902}{PMID: 4208902}, M Stillerman et al., 1973].

3. **Adolescents (12 years and older):**
   - Multiple large randomized controlled trials of topical clindamycin phosphate (alone or in combination with benzoyl peroxide or tretinoin) for acne in patients aged 12 years and older (including adolescent subgroups) found a favorable safety and tolerability profile, with adverse event rates similar to vehicle or other comparators and most reactions being mild [\hyperlink{pmid_26345297}{PMID: 26345297}, Guy Webster et al., 2015; \hyperlink{pmid_24704684}{PMID: 24704684}, Brigitte Dréno et al.; \hyperlink{pmid_18805603}{PMID: 18805603}, Diane Thiboutot et al., 2008]. One study specifically notes that younger patients tended to have milder reactions [\hyperlink{pmid_26345297}{PMID: 26345297}, Guy Webster et al., 2015].

**Summary:** 
- For neonates and infants (<1 year), targeted studies affirm that clindamycin phosphate is generally safe when dosed appropriately, though monitoring for necrotizing enterocolitis at high exposures is warranted.
- For children (1–14 years), multiple studies support the safety of clindamycin phosphate for various indications, with rash and mild hematologic changes as the main adverse events.
- For adolescents (12+), topical clindamycin phosphate is affirmed as safe in large studies.

No studies were found that specifically demonstrate clindamycin phosphate is unsafe in children. The available evidence supports its safety in the studied pediatric age groups.

\subsection*{Abstracts}
\hypertarget{pmid_910760}{C}lindamycin phosphate was used in the treatment of 29 children with osteomyelitis of whom 25 had an acute and four a chronic type of infection. The usual dose was 50 mg/kg/day intravenously for approximately three weeks followed by oral clindamycin palmitate at home in a dose of 30 mg/kg/day for an additional six weeks. Staphylococcus aureus was isolated in 22 of 29 cases: 96\% of strains were penicillin resistant. The clinical and bacteriologic results in the present series were good to excellent. There was prompt clinical and bacteriologic response shortly after initiation of clindamycin therapy. Good bone penetration of the drug was observed. Long-term evaluation revealed satisfactory clinical and roentgenographic progress in all patients. No diarrhea or manifestations of enterocolitis appeared in any patient in spite of high doses of the drug for intervals up to nine weeks. [\hyperlink{Clindamycin Phosphate}{PMID: 910760}, W Rodriguez et al., 1977]

\hypertarget{pmid_26926644}{C}lindamycin may be active against methicillin-resistant Staphylococcus aureus, a common pathogen causing sepsis in infants, but optimal dosing in this population is unknown. We performed a multicenter, prospective pharmacokinetic (PK) and safety study of clindamycin in infants. We analyzed the data using a population PK analysis approach and included samples from two additional pediatric trials. Intravenous data were collected from 62 infants (135 plasma PK samples) with postnatal ages of <121 days (median [range] gestational age of 28 weeks [23 to 42] and postnatal age of 17 days [1 to 115]). In addition to body weight, postmenstrual age (PMA) and plasma protein concentrations (albumin and alpha-1 acid glycoprotein) were found to be significantly associated with clearance and volume of distribution, respectively. Clearance reached 50\% of the adult value at PMA of 39.5 weeks. Simulated PMA-based intravenous dosing regimens administered every 8 h (≤32 weeks PMA, 5 mg/kg; 32 to 40 weeks PMA, 7 mg/kg; >40 to 60 weeks PMA, 9 mg/kg) resulted in an unbound, steady-state concentration at half the dosing interval greater than a MIC for S. aureus of 0.12 μg/ml in >90\% of infants. There were no adverse events related to clindamycin use. (This study has been registered at ClinicalTrials.gov under registration no. NCT01728363.). [\hyperlink{Clindamycin Phosphate}{PMID: 26926644}, Daniel Gonzalez et al., 2016]

\hypertarget{pmid_6470871}{T}he pharmacokinetics of intravenously administered clindamycin phosphate was studied in 40 children less than 1 year of age. Mean peak serum concentrations were 10.92 micrograms/ml in premature infants less than 4 weeks of age, 10.45 micrograms/ml in term infants greater than 4 weeks, and 12.69 micrograms/ml in term infants less than 4 weeks of age. Mean trough concentrations were 5.52, 2.8, and 3.03 micrograms/ml, respectively, in the same groups. Serum half-life was significantly longer (8.68 vs 3.60 hours) in premature compared with term infants less than 4 weeks of age. Both premature and term infants less than 4 weeks had significantly decreased clearance when compared with infants greater than 4 weeks (0.294 and 0.678, respectively, vs 1.58 L/hr). Clearance was significantly greater (1.919 vs 0.310 L/hr) and serum half-life less (1.75 vs 7.57 hours) in infants with body weight greater than 3.5 kg. On the basis of these data it is recommended that in infants greater than 4 weeks or greater than 3.5 kg, intravenous clindamycin dosage be 20 mg/kg/day in four divided doses. In premature neonates less than 4 weeks, the dose should be reduced to 15 mg/kg/day in three divided doses. Term infants greater than 1 week of age may also receive 20 mg/kg/day in four doses. [\hyperlink{Clindamycin Phosphate}{PMID: 6470871}, M J Bell et al., 1984]

\hypertarget{pmid_16870763}{F}osmidomycin plus clindamycin was shown to be efficacious in the treatment of uncomplicated Plasmodium falciparum malaria in a small cohort of pediatric patients aged 7 to 14 years, but more data, including data on younger children with less antiparasitic immunity, are needed to determine the potential value of this new antimalarial combination. We conducted a single-arm study to improve the precision of efficacy estimates for an oral 3-day fixed-ratio combination of fosmidomycin and clindamycin at 30 and 10 mg/kg of body weight, respectively, every 12 hours for the treatment of uncomplicated P. falciparum malaria in 51 pediatric outpatients aged 1 to 14 years. Fosmidomycin plus clindamycin was generally well tolerated, but relatively high rates of treatment-associated neutropenia (8/51 [16\%]) and falls of hemoglobin concentrations of > or =2 g/dl (7/51 [14\%]) are of concern. Asexual parasites and fever were cleared within median periods of 42 h and 38 h, respectively. All patients who could be evaluated were parasitologically and clinically cured by day 14 (49/49; 95\% confidence interval [CI], 93 to 100\%). The per-protocol, PCR-adjusted day 28 cure rate was 89\% (42/47; 95\% CI, 77 to 96\%). Efficacy appeared to be significantly reduced in children aged 1 to 2 years, with a day 28 cure rate of only 62\% for this small subgroup (5/8). The inadequate efficacy in children of <3 years highlights the need for continued systematic studies of the current dosing regimen, which should include randomized trial designs. [\hyperlink{Clindamycin Phosphate}{PMID: 16870763}, Steffen Borrmann et al., 2006]

\hypertarget{pmid_4208902}{C}lindamycin palmitate and potassium phenoxymethyl penicillin were evaluated in 103 children with upper respiratory illnesses and pharyngeal group A streptococci, from November 1970 to July 1971. The children were assigned randomly by weight to one of the antibiotic regimens given orally for 10 days. Clindamycin palmitate and potassium phenoxymethyl penicillin dosages were 75 and 125 mg, respectively, in 5 ml tid for children weighing less than 25 kg, and 150 and 250 mg, respectively, in 10 ml bid for children weighing 25 kg or more. Recurrences of the original streptococcal group A, M, and T types within 3 weeks after the end of treatment were classified as failures. The failure rates were: clindamycin palmitate, 10\% (5 of 52), and potassium phenoxymethyl penicillin, 18\% (9 of 51). Possible drug-related rashes were observed in 8 of 52 clindamycin palmitate-treated patients. The geometric mean minimal inhibitory concentrations of clindamycin and penicillin against 103 isolates of group A streptococci were 0.033 and 0.007 mug/ml, respectively. The serum concentrations about 70 min after ingesting 150 mg of clindamycin palmitate averaged 3.8 mug/ml and after 250 mg of potassium phenoxymethyl penicillin averaged 0.9 mug/ml. Clindamycin palmitate was as effective as potassium phenoxymethyl penicillin in eradicating group A streptococci from the pharynx in tid and bid regimens. Nevertheless, because of its rash-producing tendency in some patients and higher cost, clindamycin palmitate should not be preferred to penicillin for treatment of streptococcal sore throat in the non-penicillin-allergic patient. [\hyperlink{Clindamycin Phosphate}{PMID: 4208902}, M Stillerman et al., 1973]

\hypertarget{pmid_28339377}{C}lindamycin is an effective antibiotic in the treatment of infections caused by certain gram-positive and gram-negative anaerobic microorganisms. While manufactured forms of the drug for pediatric use are available, there are instances when a compounded liquid dosage form is essential to meet unique patient needs. The purpose of this study was to determine the chemical stability of clindamycin hydrochloride in the PCCA base SuspendIt, a sugar-free, paraben- free, dye-free, and gluten-free thixotropic vehicle containing a natural sweetener obtained from the monk fruit. It thickens upon standing to minimize settling of any insoluble drug particles and becomes fluid upon shaking to allow convenient pouring during administration to the patient. A robust stability-indicating high-performance liquid chromatographic assay for the determination of clindamycin hydrochloride in SuspendIt was developed and validated. This assay was used to determine the chemical stability of the drug in SuspendIt. Samples were prepared and stored under three different temperature conditions (5°C, 25°C, and 40°C), and assayed using the high-performance liquid chromatographic assay at pre-determined intervals over an extended period of time as follows: 7, 14, 30, 45, 60, 91, 120, and 182 days at each designated temperature. Physical data such as pH, viscosity, and appearance were also monitored. The study showed that drug concentration did not go below 90\% of the label claim (initial drug concentration) at all three temperatures studied, barring isolated experimental errors. Viscosity and pH values also did not change significantly. Some variations in viscosity were attributed to the thixotropic nature of the vehicle. This study demonstrates that clindamycin hydrochloride is physically and chemically stable in SuspendIt for 182 days in the refrigerator and at room temperature, thus providing a viable, compounded alternative for clindamycin hydrochloride in a liquid dosage form, with an extended beyond-use date to meet patient needs. [\hyperlink{Clindamycin Phosphate}{PMID: 28339377}, Yashoda V Pramar et al., ]

\hypertarget{pmid_143716}{C}lindamycin phosphate was administered intravenously to 41 patients with different types of infections including osteomyelitis, septicaemia and soft tissue infections. All bacterial strains tested showed low MIC values for clindamycin. Maximum serum concentrations after 600 mg intravenously were 6.0--29.0 microgram/ml, after 300 mg intravenously 2.6--26.0 microgram/ml. The therapeutic effect of the drug was considered good in 26 of 31 patients with proven or probable bacterial aetiology. Side effects were noted in 16 of the 41 patients. However, in only 5 of these the treatment had to be terminated, all due to pruritic rashes. In the 7 cases with diarrhoea as side effect, the symptoms were mild and of short duration. [\hyperlink{Clindamycin Phosphate}{PMID: 143716}, H Hugo et al., 1977]

\hypertarget{pmid_31725114}{D}espite the absence of adequate safety or efficacy data, clindamycin is widely prescribed in the neonatal intensive care unit. We evaluated the association between clindamycin exposure and adverse events, as well as antibiotic effectiveness in infants. This was a retrospective cohort study of infants receiving clindamycin before postnatal day 121 who were discharged from a Pediatrix Medical Group neonatal intensive care unit (1997-2015). Using a previously developed pharmacokinetic model, we performed simulations to predict clindamycin exposure based on available dosing data. We used multivariable logistic regression to evaluate the association between clindamycin exposure and safety outcomes during and after clindamycin therapy. We reported the proportion of infants with methicillin-resistant Staphylococcus aureus (MRSA) bacteremia and clearance of MRSA bacteremia. A total of 4089 infants received clindamycin at a median (25th-75th percentile) dose of 15 mg/kg/d (12-16). Clearance increased with older gestational age. Infants with the highest total clindamycin exposure had marginally increased odds of necrotizing enterocolitis within 7 days (adjusted odds ratio = 1.95 [1.04-3.63]), but exposure was not associated with death, sepsis, seizures, intestinal perforation or intestinal strictures. Of 25 infants who had MRSA bacteremia, 19 (76\%) cleared the infection by the end of the clindamycin course. Higher clindamycin exposure was not associated with increased odds of death or nonlaboratory adverse events. The use of pharmacokinetic models combined with available electronic health record data offers a valuable, cost-effective approach to analyzing the safety and effectiveness of drugs in infants when large-scale trials are not feasible. [\hyperlink{Clindamycin Phosphate}{PMID: 31725114}, Rachel G Greenberg et al., 2020]

\hypertarget{pmid_2292542}{T}he Cystic Fibrosis Clinic at the Royal Belfast Hospital for Sick Children has treated 31 children with ciprofloxacin, for serious pseudomonas infection in cystic fibrosis, and carefully monitored the safety and acceptability of the drug. Initially, eight very ill children were treated on a named-patient basis, with an encouraging clinical response and few adverse effects. Children aged 10-18 years were included in a study of four consecutive exacerbations of respiratory disease, comparing (i) oral ciprofloxacin in each episode with (ii) ciprofloxacin alternating with intravenous azlocillin and tobramycin. Other children with cystic fibrosis were subsequently treated with ciprofloxacin, as the need arose. In all the groups very few adverse reactions were found; in particular only one child developed arthralgia. A total of 202 children in the UK have been treated with ciprofloxacin on a named-patient basis, and their clinicians have reported 46 adverse events that may have been drug-related. Overall ciprofloxacin appears to be safe and effective in children but concern about the possible occurrence of arthropathy remains and long term follow-up of these children may be necessary. [\hyperlink{Clindamycin Phosphate}{PMID: 2292542}, A Black et al., 1990]

\hypertarget{pmid_24949994}{C}lindamycin is commonly prescribed to treat children with skin and skin-structure infections (including those caused by community-acquired methicillin-resistant Staphylococcus aureus (CA-MRSA)), yet little is known about its pharmacokinetics (PK) across pediatric age groups. A population PK analysis was performed in NONMEM using samples collected in an opportunistic study from children receiving i.v. clindamycin per standard of care. The final model was used to optimize pediatric dosing to match adult exposure proven effective against CA-MRSA. A total of 194 plasma PK samples collected from 125 children were included in the analysis. A one-compartment model described the data well. The final model included body weight and a sigmoidal maturation relationship between postmenstrual age (PMA) and clearance (CL): CL (l/h) = 13.7 × (weight/70)(0.75) × (PMA(3.1)/(43.6(3.1) + PMA(3.1))); V (l) = 61.8 × (weight/70). Maturation reached 50\% of adult CL values at \textasciitilde{}44 weeks PMA. Our findings support age-based dosing.  [\hyperlink{Clindamycin Phosphate}{PMID: 24949994}, D Gonzalez et al., 2014] The combination of fosmidomycin and clindamycin (F/C) is effective in adults and older children for the treatment of malaria and could be an important alternative to existing artemisinin-based combinations (ACTs) if proven to work in younger children. We conducted an open-label clinical trial to assess the efficacy, safety, and tolerability of F/C for the treatment of uncomplicated P. falciparum malaria in Mozambican children <3 years of age. Aqueous solutions of the drugs were given for 3 days, and the children were followed up for 28 days. The primary outcome was the PCR-corrected adequate clinical and parasitological response at day 28. Secondary outcomes included day 7 and 28 uncorrected cure rates and fever (FCT) and parasite (PCT) clearance times. Fifty-two children were recruited, but only 37 patients were evaluable for the primary outcome. Day 7 cure rates were high (94.6\%; 35/37), but the day 28 PCR-corrected cure rate was 45.9\% (17/37). The FCT was short (median, 12 h), but the PCT was longer (median, 72 h) than in previous studies. Tolerability was good, and most common adverse events were related to the recurrence of malaria. The poor efficacy observed for the F/C combination may be a consequence of the new formulations used, differential bioavailability in younger children, naturally occurring variations in parasite sensitivity to the drugs, or an insufficient enhancement of their effects by naturally acquired immunity in young children. Additional studies should be conducted to respond to the many uncertainties arising from this trial, which should not discourage further evaluation of this promising combination. [\hyperlink{Clindamycin Phosphate}{PMID: 24949994}, Miguel Lanaspa et al., 2012]

\hypertarget{pmid_9194107}{M}ore data on the efficacy and safety of ciprofloxacin in pediatric cystic fibrosis patients are needed. One hundred eight pediatric cystic fibrosis patients (ages 5 to 17 years) with acute bronchopulmonary exacerbations entered a randomized multicenter trial designed to compare the safety and efficacy of antipseudomonas therapy with oral ciprofloxacin (15 mg/kg twice daily; maximum dosage 750 mg twice daily) or intravenous ceftazidime plus tobramycin (CAZ/TM) for 14 days. Clinical improvement was observed in 93\% of patients treated with oral ciprofloxacin and in 96\% of those receiving parenteral therapy. Transient suppression of Pseudomonas aeruginosa was achieved in 63\% of patients at the end of the course of iv CAZ/TM therapy and in 24\% receiving ciprofloxacin. Ultrasound examination and nuclear magnetic resonance imaging scans showed no evidence of cartilage toxicity in any of the ciprofloxacin-treated patients. Musculoskeletal adverse events were reported with similar frequency in the two groups of patients (7\% in the group receiving ciprofloxacin therapy and 11\% in the IV CAZ/TM group). The only sustained musculoskeletal symptom was a case of synovitis in a patient receiving parenteral CAZ/TM. Ciprofloxacin thus appears to be safe and effective for use in young patients with bronchopulmonary exacerbation of cystic fibrosis. [\hyperlink{Clindamycin Phosphate}{PMID: 9194107}, D A Richard et al., 1997]

\hypertarget{pmid_634877}{C}lindamycin phosphate is an antibiotic which is effective against both Staphylococcus aureus and the anaerobic organisms. In thirteen patients, its concentration following joint replacement was measured by the agar diffusion method. In bone, the concentration was (mean +/- s.e. mean) 5.01 microgram/ml +/- 1.16, N=10; in capsule, 3.29 microgram/ml +/- 0.71, N=12; measured between 1.75 and 3.75 hr after intramuscular and intravenous injections, and in drainage fluid it amounted to 4.61 microgram/ml +/- 0.38, N=11 in 48 hr. Two patients developed diarrhoea which settled within a short period. [\hyperlink{Clindamycin Phosphate}{PMID: 634877}, P Baird et al., 1978]

\hypertarget{pmid_3737273}{W}e studied 12 newborn infants (gestational ages 26-39 wk [mean +/- SD, 30.6 +/- 4.7]; birth weight 640-2700 g, [mean, 1,322 +/- 688]; postnatal age 1-24 days [mean, 9.6 +/- 8.5]) who received clindamycin phosphate for suspected or proven necrotizing enterocolitis (ten patients) or suspected anaerobic septicemia (two patients) in doses of 3.2-11 mg/kg every six hours. Range of mean serum concentration of clindamycin at steady state was between 12.7 and 40 micrograms/ml (therapeutic range = 2-10 micrograms/ml). High concentrations could be attributed to elimination T1/2 (6.3 +/- 2.1 hr) 100\% longer than in older children or adults. Clindamycin clearance (61.6 +/- 31.6 hr ml/kg/hr) was lower than in older children or adults. Because of the observed prolongation in T1/2 and correspondingly lower clearance, the IV dose of clindamycin in newborn infants should be reduced to 15-20 mg/kg/day given in four daily doses. [\hyperlink{Clindamycin Phosphate}{PMID: 3737273}, G Koren et al., 1986]

\hypertarget{pmid_15478056}{I}t has been demonstrated that fosmidomycin has good tolerability and rapid onset of action, but late recrudescences preclude its use alone; in vitro, clindamycin has been shown to act synergistically with fosmidomycin against Plasmodium falciparum. We conducted a study in pediatric outpatients with P. falciparum malaria in Gabon to evaluate the efficacy and safety of an oral combination of fosmidomycin-clindamycin of 30 mg/kg and 10 mg/kg of body weight, respectively, every 12 h. Patients 7-14 years old were recruited in cohorts of 10. The first 10 patients were treated for 5 days. The duration of treatment was then incrementally shortened in intervals of 1 day if >85\% of the patients in a cohort were cured by day 14. All dosing regimens were well tolerated, and no serious adverse events occurred. Asexual parasites and fever rapidly cleared in all patients. Cure ratios of 100\% on day 14 were achieved with treatment durations of 5 (10/10 patients), 4 (10/10 patients), 3 (10/10 patients), and 2 days (10/10 patients); 1 day of treatment led to a cure ratio of 50\% (5/10 patients). Fosmidomycin-clindamycin is safe and well tolerated, and short-course regimens achieved high efficacy in children with P. falciparum malaria. Fosmidomycin-clindamycin is a promising novel treatment option for malaria. [\hyperlink{Clindamycin Phosphate}{PMID: 15478056}, Steffen Borrmann et al., 2004]

\hypertarget{pmid_26345297}{T}o investigate the cutaneous safety and tolerability of clindamycin phosphate 1.2\%/benzoyl peroxide 3.75\% gel in moderate-to-severe acne patients. A safety assessment of 498 patients with moderate-to-severe acne receiving clindamycin phosphate 1.2\%/benzoyl peroxide 3.75\% gel or vehicle for 12 weeks. The vast majority (80-95\%) of patients reported no cutaneous safety or tolerability problems throughout the study. Mean scores for both active and vehicle were all <1 (where l=mild) and reduced over the duration of the study. When scaling, erythema, itching, burning, or stinging was reported it was generally mild. Moderate or severe reactions to clindamycin phosphate 1.2\%/benzoyl peroxide 3.75\% gel were rare and generally seen early in treatment. There were eight reports (3.3\%) of moderate erythema, four reports (1.7\%) of moderate scaling, three reports (1.2\%) of moderate itching, and one report of moderate burning (0.4\%) at Week 4. There was one report (0.4\%) of severe erythema and one report (0.4\%) of severe burning (both at Week 4), with one report (0.4\%) of severe stinging at Week 12. There were no substantive differences seen in cutaneous tolerability among treatment groups and younger patients tended to have milder reactions. It is not possible to determine the contributions of the individual active ingredients. Clindamycin phosphate 1.2\%/benzoyl peroxide 3.75\% gel has a favorable safety and tolerability profile with very low incidence of moderate or severe reactions. [\hyperlink{Clindamycin Phosphate}{PMID: 26345297}, Guy Webster et al., 2015]

\hypertarget{pmid_1191404}{C}lindamycin phosphate, a new semisynthetic antibiotic that is effective in the treatment of toxoplasmosis and of infections caused by Gram-positive bacteria, was found to be highly concentrated in the choroid, iris, and retina of the pigmented rabbit eye after a single intramuscular injection of 75 mg/kg. Drug levels considered adequate for the control of most ocular infections were detectable in the iris, choroid, and retina 24 hours after injection, at which time serum levels were negligible. Subconjunctival injection of clindamycin phosphate also produced sustained high levels of drug in the choroid, iris, and retina; but when 150 mg was injected in a volume of 1 ml, corneal edema and severe inflammation of the conjunctiva resulted. Lesser amounts (15 to 35 mg) injected subconjunctivally produced adequate ocular tissue levels without damage to the conjunctiva or cornea. [\hyperlink{Clindamycin Phosphate}{PMID: 1191404}, K F Tabbara et al., 1975]

\hypertarget{pmid_14976608}{F}osmidomycin is a new antimalarial drug with a novel mechanism of action. Studies in Africa that have evaluated fosmidomycin as monotherapeutic agent demonstrated its excellent tolerance, but 3-times-daily treatment regimens of >or=4 days were required to achieve radical cure, prompting further research to identify and validate a suitable combination partner to enhance its efficacy. We conducted a randomized, controlled, open-label study to evaluate the efficacy and safety of fosmidomycin combined with clindamycin (n=12; 30 and 5 mg/kg body weight every 12 h for 5 days, respectively), compared with fosmidomycin alone (n=12; 30 mg/kg body weight every 12 h for 5 days) and clindamycin alone (n=12; 5 mg/kg body weight every 12 h for 5 days) for the clearance of asymptomatic Plasmodium falciparum infections in schoolchildren in Gabon aged 7-14 years. Asexual parasites were rapidly cleared in children treated with fosmidomycin-clindamycin (median time, 18 h) and fosmidomycin alone (25 h) but slowly in children treated with clindamycin alone (71 h; P=.004). However, only treatment with fosmidomycin-clindamycin or clindamycin alone led to the radical elimination of asexual parasites as measured by day 14 and 28 cure rates of 100\%. Asexual parasites reappeared by day 28 in 7 children who received fosmidomycin (day 14 cure rate, 92\% [11/12; day 28 cure rate, 42\% [5/12]). All regimens were well tolerated, and no serious adverse events occurred. The combination of fosmidomycin and clindamycin is well tolerated and superior to either agent on its own with respect to the rapid and radical clearance of P. falciparum infections in African children. [\hyperlink{Clindamycin Phosphate}{PMID: 14976608}, Steffen Borrmann et al., 2004]

\hypertarget{pmid_24704684}{T}he efficacy and safety of clindamycin phosphate 1.2\%/tretinoin 0.025\% (Clin-RA) were evaluated in three 12-week randomised studies. To perform a pooled analysis of data from these studies to evaluate Clin-RA's efficacy and safety in a larger overall population, in subgroups of adolescents and according to acne severity. 4550 patients were randomised to Clin-RA, clindamycin, tretinoin and vehicle. Evaluations included percentage change in lesions, treatment success rate, proportions of patients with ≥50\% or ≥80\% continuous reduction in lesions, adverse events and cutaneous tolerability. In the overall population, the percentage reduction in inflammatory, non-inflammatory and total lesions and the treatment success rate were significantly greater with Clin-RA compared with clindamycin, tretinoin and vehicle alone (all p<0.01). The percentage reduction in all types of lesions was also significantly greater with Clin-RA in the adolescent subgroup (2915 patients, p<0.002) and in patients with mild/moderate acne (3662 patients, p<0.02) versus comparators. In patients with severe acne (n = 880), the percentage reduction in all lesion types was significantly greater with Clin-RA versus vehicle (p<0.0001). A greater proportion of Clin-RA treated patients had a ≥50\% or ≥80\% continuous reduction in all types of lesions at week 12 compared with clindamycin, tretinoin and vehicle. Adverse event frequencies in the active and vehicle groups were similar. Baseline-adjusted mean tolerability scores over time were <1 (mild) and similar in all groups. Clin-RA is safe, has superior efficacy to its component monotherapies and should be considered as one of the first-line therapies for mild-to-moderate facial acne. [\hyperlink{Clindamycin Phosphate}{PMID: 24704684}, Brigitte Dréno et al., ]

\hypertarget{pmid_557118}{T}wenty eight patients were treated with parenteral clindamycin-2-phosphate in the field of surgery, and good response was obtained in a series of superficial soft tissue infection, especially caused by staphylococci, with a daily dose of 300 mg. Serum level and urinary excretion were also investigated in four healthy male volunteers. [\hyperlink{Clindamycin Phosphate}{PMID: 557118}, Y Shiraha et al., 1977]

\hypertarget{pmid_18805603}{W}e sought to evaluate efficacy, safety, and tolerability of a combination of clindamycin phosphate 1.2\% and benzoyl peroxide 2.5\% (clindamycin-BPO 2.5\%) aqueous gel in moderate to severe acne vulgaris. A total of 2813 patients, aged 12 years or older, were randomized to receive clindamycin-BPO 2.5\%, individual active ingredients, or vehicle in two identical, double-blind, controlled 12-week, 4-arm studies evaluating safety and efficacy (inflammatory and noninflammatory lesion counts) using Evaluator Global Severity Score and subject self-assessment. Clindamycin-BPO 2.5\% demonstrated statistical superiority to individual active ingredients and vehicle in reducing both inflammatory and noninflammatory lesions and acne severity. Visibly greater improvement was observed by patients with clindamycin-BPO 2.5\% as early as week 2. No substantive differences were seen in cutaneous tolerability among treatment groups and less than 1\% of patients discontinued treatment because of adverse events. Data from controlled studies may differ from clinical practice. Clindamycin-BPO 2.5\% provides statistically significant greater efficacy than individual active ingredients and vehicle with a highly favorable safety and tolerability profile. [\hyperlink{Clindamycin Phosphate}{PMID: 18805603}, Diane Thiboutot et al., 2008]

\hypertarget{pmid_1494233}{C}efprozil (CFPZ, BMY-28100) was evaluated for its efficacy, safety and pharmacokinetics in children. CFPZ was effective against streptococcal pharyngitis, pneumococcal lower respiratory tract infections, staphylococcal skin infections and Escherichia coli urinary tract infections, but was less effective against lower respiratory tract infections and otitis media due to Haemophilus influenzae. No adverse reactions were encountered in 46 cases treated with CFPZ. With a premeal administration of 7.5 mg/kg, the Cmax was approximately 3.2 micrograms/ml and the T 1/2 beta was 1.4 hours. From the present study, CFPZ appears to be safe and effective against community-acquired childhood infections. [\hyperlink{Clindamycin Phosphate}{PMID: 1494233}, H Meguro et al., 1992]

\hypertarget{pmid_9797245}{C}lindamycin, which is usually used in combination with pyrimethamine, has been proven effective in the treatment of cerebral toxoplasmosis in human immunodeficiency virus-infected patients. However, it is not known if clindamycin achieves inhibitory concentrations at the site of infection. Also, it has been hypothesized that the activity of clindamycin against Toxoplasma gondii may be due, at least in part, to a metabolite. We evaluated the penetration of clindamycin and its major metabolite, N-demethylclindamycin (NDC), into cerebrospinal fluid (CSF) of AIDS patients undergoing lumbar puncture for diagnostic purposes. A single, 1,200-mg dose of clindamycin was administered as a 45-min intravenous infusion beginning at 1.5 or 2.5 h before CSF sampling. The concentrations of clindamycin in CSF ranged from 0.091 to 0.429 mg/liter at 1.5 h and from 0.120 to 0.283 mg/liter at 2.5 h following the beginning of the infusion. The concentrations of clindamycin in CSF were well above the 50\% inhibitory concentration of 0.001 mg/liter and the parasiticidal concentration of 0.006 mg/liter. NDC was undetectable both in plasma and in CSF. Our study provides a pharmacokinetic rationale for the clinical efficacy of clindamycin in the treatment of cerebral toxoplasmosis. [\hyperlink{Clindamycin Phosphate}{PMID: 9797245}, G Gatti et al., 1998]

\hypertarget{pmid_26322295}{T}he aim of the present study was to evaluate a non-destructive fabrication method in for the development of sustained-release poly (L, D-lactic acid)-based biodegradable clindamycin phosphate implants for the treatment of ocular toxoplasmosis. The rod-shaped intravitreal implants with an average length of 5 mm and a diameter of 0.4 mm were evaluated for their physicochemical parameters. Scanning electron microscopy (SEM), differential scanning calorimetry (DSC), Fourier-transform infrared (FTIR), and nuclear magnetic resonance (1H NMR) studies were employed in order to study the characteristics of these formulations. Drug content uniformity test confirmed the uniformity in different implant batches. Furthermore, the DSC, FTIR, and 1H NMR studies proved that the fabrication process did not have any destructive effects either on the drug or on the polymer structures. These studies showed that the developed sustained-release implants could be of interest for long-term sustained intraocular delivery of clindamycin, which can provide better patient compliance and also have good potential in terms of industrial feasibility. [\hyperlink{Clindamycin Phosphate}{PMID: 26322295}, Lana Tamaddon et al., 2015]

\hypertarget{pmid_15696985}{C}lindamycin phosphate is the most widely used topical antibacterial agent for acne treatment. Treatment of patients with mild to moderate acne vulgaris with a new foam formulation (clindamycin foam, CF) for 12 weeks was at least as effective as clindamycin gel (CG) based on the Investigator's Static Global Assessment (ISGA) score. CF was superior to CG based on the reduction from baseline in total (P = .0014), inflammatory (P = .0478), and noninflammatory (P = .0037) acne lesion counts. Additionally, CF achieved efficacy that was superior to that of vehicle foam based on ISGA score (P = .0025) and all 3 lesion counts (all P < .05). Adverse experiences in the active treatment groups were mild or moderate and transient in nature. Thus the foam formulation of clindamycin is a safe and effective acne treatment; the unique foam delivery vehicle may offer cosmetic benefits to the patient and thus increase compliance. [\hyperlink{Clindamycin Phosphate}{PMID: 15696985}, Alan R Shalita et al., ]

\section*{Fabior}
\subsection*{Result}
\subsubsection*{Answer}

Unknown

\subsubsection*{{Explanation}}
\hypertarget{Fabior}
After reviewing all available abstracts, there is no abstract specifically about Fabior (tazarotene) or its use in children. None of the abstracts mention Fabior by name, nor do they present targeted studies evaluating the safety of Fabior in pediatric populations of any age range. Therefore, based on the abstracts provided, there is not enough data to determine whether Fabior is safe or unsafe for use in children. The safety of Fabior in children is unknown according to the available evidence.

\subsection*{Abstracts}
\hypertarget{pmid_8132376}{L}ike all fluoroquinolones, ciprofloxacin causes articular damage in juvenile animals. Consequently, this drug was not recommended for children or pregnant women. However, due to its antibacterial effectiveness and convenience of oral administration, ciprofloxacin is now increasingly used for the treatment of certain infectious conditions in children and adolescents aged less than 18 years. In this paper the published literature on this subject is reviewed. Up to now, data are available on more than 1,500 paediatric patients who were given ciprofloxacin, two-thirds of whom were suffering from acute infectious bronchopulmonary exacerbations of cystic fibrosis, mainly due to Pseudomonas aeruginosa. The effectiveness of oral ciprofloxacin for this indication compared well to that of standard intravenous combination regimens. The majority of the remaining published trials was conducted in children with multiresistant typhoid fever; the administration of ciprofloxacin was successful in up to 100\% of the cases. The safety profile of ciprofloxacin in children and adolescents was very similar to that observed in adult patients. Adverse events were noted in 5-15\%, with gastrointestinal, skin and central nervous system reactions being the most common. Reversible arthralgia occurred in 36 out of 1,113 patients with cystic fibrosis, and in no case could cartilage damage be demonstrated by radiographic procedures. Thus, publication data clearly suggest that the administration of ciprofloxacin to children is effective and safe, but there is a need for further prospective, well-controlled clinical trials. [\hyperlink{Fabior}{PMID: 8132376}, R Kubin et al., ]

\hypertarget{pmid_16028153}{B}ecause of concerns about arthrotoxicity, fluoroquinolones are restricted for use in children. This study describes the safety and efficacy of gatifloxacin when used for treatment of children with recurrent acute otitis media (ROM) or acute otitis media (AOM) treatment failure (AOMTF). We performed an analysis of 867 children included in 4 clinical trials who had ROM and/or AOMTF and were treated with gatifloxacin (10 mg/kg once daily for 10 days). Gatifloxacin had adverse event rates that were similar overall to those of a comparator antibiotic (amoxicillin-clavulanate), except for increased diarrhea in children <2 years old receiving amoxicillin-clavulanate. There was no evidence of arthrotoxicity, hepatotoxicity, alteration of glucose homeostasis, or central nervous system toxicity acutely or during 1 year follow-up in any child. Regarding efficacy, in 2 noncomparative trials, the gatifloxacin cure rate of AOM was 89\% (95\% confidence interval [CI], 83\%-95\%) at the test of cure (TOC) visit, 3-10 days after completion of therapy. In 2 comparative trials of gatifloxacin versus amoxicillin-clavulanate, the efficacy of gatifloxacin was 88\% (95\% CI, 82\%-94\%). Gatifloxacin led to better clinical outcomes than amoxicillin-clavulanate for AOMTF (91\% vs. 81\%; P=.029), for AOMTF and age <2 years old (89\% vs. 69\%; P=.009), and for severe AOM in children <2 years old (90\% vs. 75\%; P=.012). Among children with AOMTF previously treated with amoxicillin-clavulanate or ceftriaxone injections, gatifloxacin cure rates were high (88\% and 75\%, respectively). Gatifloxacin appears to be safe for children, with no evidence of producing arthrotoxicity in 867 children exposed to the antibiotic when used as treatment for ROM and AOMTF. [\hyperlink{Fabior}{PMID: 16028153}, Michael E Pichichero et al., 2005]

\hypertarget{pmid_33106892}{P}ediatric patients with advanced chronic kidney disease (CKD) are often prescribed oral phosphate binders (PBs) for the management of hyperphosphatemia. However, available PBs have limitations, including unfavorable tolerability and safety. This phase 3, multicenter, randomized, open-label study investigated safety and efficacy of sucroferric oxyhydroxide (SFOH) in pediatric and adolescent subjects with CKD and hyperphosphatemia. Subjects were randomized to SFOH or calcium acetate (CaAc) for a 10-week dose titration (stage 1), followed by a 24-week safety extension (stage 2). Primary efficacy endpoint was change in serum phosphorus from baseline to the end of stage 1 in the SFOH group. Safety endpoints included treatment-emergent adverse events (TEAEs). Eighty-five subjects (2-18 years) were randomized and treated (SFOH, n = 66; CaAc, n = 19). Serum phosphorus reduction from baseline to the end of stage 1 in the overall SFOH group (least squares [LS] mean ± standard error [SE]) was - 0.488 ± 0.186 mg/dL; p = 0.011 (post hoc analysis). Significant reductions in serum phosphorus were observed in subjects aged ≥ 12 to ≤ 18 years (LS mean ± SE - 0.460 ± 0.195 mg/dL; p = 0.024) and subjects with serum phosphorus above age-related normal ranges at baseline (LS mean ± SE - 0.942 ± 0.246 mg/dL; p = 0.005). Similar proportions of subjects reported ≥ 1 TEAE in the SFOH (75.8\%) and CaAc (73.7\%) groups. Withdrawal due to TEAEs was more common with CaAc (31.6\%) than with SFOH (18.2\%). SFOH effectively managed serum phosphorus in pediatric patients with a low pill burden and a safety profile consistent with that reported in adult patients. [\hyperlink{Fabior}{PMID: 33106892}, Larry A Greenbaum et al., 2021]

\hypertarget{pmid_3430711}{F}lomoxef (FMOX, 6315-S), a new parenteral oxacephem antibiotic, was evaluated for its safety, efficacy and pharmacokinetics in children. Twenty-six patients with bacterial infections were treated with FMOX. Clinical efficacy rate was 92\% and bacteriological cure rate was 85\%. Three cases of infections due to methicillin-resistant Staphylococcus aureus were cured with FMOX therapy. No severe adverse reactions or abnormalities of laboratory test data were associated with FMOX therapy, although loose stools and diarrhea occurred frequently (23\%). Serum half-lives of FMOX after a single bolus injection of 9 infants and children were 0.77 +/- 0.31 hour and excretion into urine was rapid. From these experiences, FMOX appeared to be a safe and effective antibiotic when used in children with susceptible bacterial infections. [\hyperlink{Fabior}{PMID: 3430711}, H Meguro et al., 1987]

\hypertarget{pmid_1784081}{B}asic and clinical studies of flomoxef (6315-S, FMOX) were performed in the pediatric surgical field. The results obtained are summarized as follows: 1. FMOX was administered to 7 pediatric patients with biliary atresia (FMOX 20 mg/kg, i.v.d.). Peak biliary levels of FMOX were obtained at 1 hour after finishing administration by drip infusion, and were higher than those in blood 1 hours after finishing administration by drip infusion. 2. Urinary excretion was excellent, and urinary recovery rates were 57.8-97.8\%. 3. FMOX was administered to 5 patients in the pediatric surgical field. One case was phlegmon, and other 4 cases were premature babies for postoperative prophylactic use. Clinical results were excellent in 1 case, good in 4 cases, with an overall efficacy rate of 100\%. No clinical and laboratory side effects due to the administration FMOX were observed. It was concluded that FMOX was a safe and effective antibiotic in the pediatric surgical field. [\hyperlink{Fabior}{PMID: 1784081}, J Yura et al., 1991]

\hypertarget{pmid_25135766}{R}ecently, an association between childhood growth stunting and aflatoxin (AF) exposure has been identified. In Ghana, homemade nutritional supplements often consist of AF-prone commodities. In this study, children were enrolled in a clinical intervention trial to determine the safety and efficacy of Uniform Particle Size NovaSil (UPSN), a refined calcium montmorillonite known to be safe in adults. Participants ingested 0.75 or 1.5 g UPSN or 1.5 g calcium carbonate placebo per day for 14 days. Hematological and serum biochemistry parameters in the UPSN groups were not significantly different from the placebo-controlled group. Importantly, there were no adverse events attributable to UPSN treatment. A significant reduction in urinary metabolite (AFM1) was observed in the high-dose group compared with placebo. Results indicate that UPSN is safe for children at doses up to 1.5 g/day for a period of 2 weeks and can reduce exposure to AFs, resulting in increased quality and efficacy of contaminated foods.  [\hyperlink{Fabior}{PMID: 25135766}, Nicole J Mitchell et al., 2014] Exercise-induced bronchoconstriction (EIB) is common, particularly in children. To compare the protective effect of single doses of formoterol fumarate via Aerolizer with placebo and albuterol in children with EIB. In this randomized, double-blind, double-dummy, crossover trial, 23 children (aged 4-11 years) received formoterol, 12 or 24 microg; albuterol, 180 microg; or placebo at 4 separate visits. Protection against EIB was evaluated as the maximum percentage decrease in forced expiratory volume in 1 second (FEV1) from the preexercise value after exercise challenge tests (6-minute treadmill) conducted 15 minutes and 4, 8, and 12 hours after administration of the dose. The maximum percentage decrease in FEV1 after the 4-hour exercise test (primary efficacy variable) was significantly less for formoterol, 12 and 24 microg, vs placebo (P < .001 for both) or albuterol (P = .016 and .010, respectively); albuterol was not significantly different from placebo. Formoterol, 12 and 24 microg, differed from placebo at 8 hours (P = .002 and .001, respectively), with a smaller difference between albuterol and placebo (P = .045). Rescue medication use and a high dropout rate may have biased treatment differences at later time points. Protection against EIB (<20\% maximum decrease in FEV1) across all time points was observed for 17 (77\%) of 22 and 17 (74\%) of 23 children with formoterol, 12 and 24 microg, respectively, compared with 8 (35\%) of 23 with albuterol and 6 (27\%) of 22 with placebo. Single doses of formoterol, 12 or 24 microg, are effective in protecting against EIB in children, affording a statistically significantly greater protective effect than placebo or albuterol. [\hyperlink{Fabior}{PMID: 25135766}, David Pearlman et al., 2006]

\hypertarget{pmid_10632911}{F}lexible fibreoptic bronchoscopy (FOB) has become a useful diagnostic and therapeutic procedure in children. We investigated 26 patients (3-14 years) for FOB using a new sedation strategy. All patients received oral premedication and inhalation of topical anaesthetic. Sedation for bronchoscopy was achieved with a continuous infusion of remifentanil and intermittent boluses of propofol. Propofol injection was repeated if sedation was inadequate. Sedation could be successfully performed in all children without adverse effects. Endtidal CO2 concentration and arterial oxygen saturation remained stable throughout the study. All children were awake 5+/-1.3 min after stopping remifentanil infusion. Sedation with remifentanil/propofol is a new sedation strategy for diagnostic flexible paediatric bronchoscopy in children with spontaneous ventilation. [\hyperlink{Fabior}{PMID: 10632911}, M Reyle-Hahn et al., 2000]

\hypertarget{pmid_18164990}{T}o evaluate the safety and efficacy of once-daily (QD) fluticasone furoate (FF) nasal spray in children with perennial allergic rhinitis (PAR). A global, randomized, double-blind, placebo-controlled study. Pediatric patients (aged 2-11 years; n = 558) with PAR received once-daily placebo, FF 110 microg, or FF 55 microg for 12 weeks. Efficacy was evaluated by nasal symptom scores. General safety and corticosteroid-specific safety (nasal and ophthalmic examinations, and hypothalamic-pituitary-adrenal assessments) were assessed. No findings of clinical concern were identified from the safety assessments. For primary efficacy analysis of mean change from baseline over the first 4 weeks of treatment in daily reflective total nasal symptom score, FF 55 microg demonstrated significant improvement (P = 0.003) compared with placebo; however, the improvement for FF 110 microg versus placebo did not reach statistical significance (P = 0.073). FF QD was well tolerated and demonstrated efficacy in children aged 2 to 11 years with PAR. [\hyperlink{Fabior}{PMID: 18164990}, Jorge F Máspero et al., 2008]

\hypertarget{pmid_28866468}{T}he Stopping Cavities Trial investigated effectiveness and safety of 38\% silver diamine fluoride in arresting caries lesions. The study was a double-blind randomized placebo-controlled superiority trial with 2 parallel groups. The sites were Oregon preschools. Sixty-six preschool children with ≥1 lesion were enrolled. Silver diamine fluoride (38\%) or placebo (blue-tinted water), applied topically to the lesion. The primary endpoint was caries arrest (lesion inactivity, Nyvad criteria) 14-21days post intervention. Dental plaque was collected from all children, and microbial composition was assessed by RNA sequencing from 2 lesions and 1 unaffected surface before treatment and at follow-up for 3 children from each group. Average proportion of arrested caries lesions in the silver diamine fluoride group was higher (0.72; 95\% CI; 0.55, 0.84) than in the placebo group (0.05; 95\% CI; 0.00, 0.16). Confirmatory analysis using generalized estimating equation log-linear regression, based on the number of arrested lesions and accounting for the number of treated surfaces and length of follow-up, indicates the risk of arrested caries was significantly higher in the treatment group (relative risk, 17.3; 95\% CI: 4.3 to 69.4). No harms were observed. RNA sequencing analysis identified no consistent changes in relative abundance of caries-associated microbes, nor emergence of antibiotic or metal resistance gene expression. Topical 38\% silver diamine fluoride is effective and safe in arresting cavities in preschool children. The treatment is applicable to primary care practice and may reduce the burden of untreated tooth decay in the population. [\hyperlink{Fabior}{PMID: 28866468}, Peter Milgrom et al., 2018]

\hypertarget{pmid_10682335}{F}luoride supplements have been used for years to prevent dental caries; nevertheless, there are three reasons why their use is inappropriate today among infants and young children in the United States. Evidence for the efficacy of fluoride supplements when used from birth or soon after is weak, supplements are a risk factor for dental fluorosis, and fluoride has little preeruptive effect in caries prevention. While there are many reports on the caries-preventive efficacy of supplements, few meet standards for acceptability as clinical trials, and those that do have tested chewable tablets or lozenges under supervision in school-aged children. North American children today are exposed to fluoride from many sources--drinking water, toothpaste, gels, rinses, and in processed foods and beverages. The additional cariostatic benefits that accrue from using supplements are marginal at best, while there is strong risk of fluorosis when young children use supplements. Available evidence suggests that the public is more aware of the milder forms of fluorosis than was previously thought; thus, it is prudent for caries-preventive policies to aim to maximizing caries reductions while minimizing the risk of fluorosis. It is therefore concluded that the risks of using supplements in infants and young children outweigh the benefits. Because alternative forms of fluoride for high-risk individuals exist, fluoride supplements should no longer be used for young children in North America. [\hyperlink{Fabior}{PMID: 10682335}, B A Burt et al., 1999]

\hypertarget{pmid_24391411}{F}lexible fiberoptic bronchoscopy (FFB) is one of the most important procedures in paediatric pulmonology. To the best of our knowledge there is no review - audit summarising the experience with FFB in children in Greece. We therefore analysed retrospectively all FFBs performed by the paediatric pulmonology team in our hospital in order to analyse indications for bronchoscopy in our population, explore diagnostic yield for each indication and highlight potential complications. Material - Methods: Three hundred and sixteen (316) diagnostic FFBs performed in 305 children during a six years period were retrospectively analysed. Seventy five (75) \% of bronchoscopies had a meaningful outcome. Diagnostic yield for individual indications ranged from 41\% to 91\%. Stridor was the most rewarding indication (91\%). Fever was the most common side effect (7\%). The rest of complications were in small numbers and easily reversible. Bronchoscopy is a safe procedure and in our diverse population the overall diagnostic yield was 75\%. [\hyperlink{Fabior}{PMID: 24391411}, F Kirvassilis et al., 2011]

\hypertarget{pmid_34064721}{S}ilver Diammine Fluoride (SDF) is an emerging caries preventive treatment option that is inexpensive, safe, and easily accessible. The evidence is clear that the use of SDF at concentrations of 38\% is effective for arresting caries in primary teeth. However, the determination of an optimal SDF application frequency for a cavitated lesion in pragmatic settings is warranted especially among high dental caries risk groups. Hence, the primary objective of this clinical trial is to compare the effectiveness of annual, bi-annual, and four times a year application of 38\% SDF application in arresting active coronal dentinal carious lesions on primary teeth among tribal preschool children aged 2-6 years. Methods and Analysis: This study is designed as a randomized, controlled trial consisting of three parallel arms with an allocation ratio of 1:1:1. The trial will enroll 480 preschool tribal children with a cavitated carious lesion (2-6 years) attending a primary health care Centre in Wayanad district, India. Each arm will receive 38\% SDF application on an annual (baseline), bi-annual (baseline and 6 months), and four times a year (baseline, 2nd, 4th, and 8th week), respectively. The analysis will be performed both at the tooth- and person-level. Ethics and Dissemination: This trial will be conducted following the principles of the Declaration of Helsinki and local guidelines (Indian Council of Medical Research). The protocol has been approved by Institutional Review Committee (IRB). This trial has been registered prospectively with the Clinical Trial Registry of India [Registration No: CTRI/2020/03/024265]. [\hyperlink{Fabior}{PMID: 34064721}, Chandrashekar Janakiram et al., 2021]

\hypertarget{pmid_32725954}{F}iber-optic bronchoscopy (FOB) of the lower airways is a routine examination performed for investigating varying respiratory complaints in children. A common side effect is a transient high fever on the day of the FOB. Such episodes are usually unrelated to an infectious process but may cause clinical uncertainty and parental anxiety. We have previously shown that a single dose of systemic dexamethasone significantly reduces the rate of fever postbronchoscopy (FPB). To prospectively analyze the effect of a prophylactic dose of ibuprofen upon the FPB. Children presenting for elective FOB and broncho-alveolar lavage (BAL) were randomized, in a double-blind fashion, to receive a single dose of ibuprofen syrup 10 mg/kg or placebo prior to the procedure. Parents were contacted the next day to record the presence or absence of fever. Sixty-one children were included in the final analysis. Thirty-one children were in the treatment group and 30 in the placebo group. FPB occurred in 40 children (65\%). There was no difference in the rate of FPB between placebo (63\%) and treatment (67\%) groups (P = .717). Fifty (82\%) children had a positive BAL culture. Among them, 38 had FPB (76\%) compared with only 2 of 11 (18\%) of those with negative culture (P = .00026, relative risk 4.18). About 80\% of positive cultures grew Haemophilus influenza. There was no significant difference between the number of BALs with a positive culture between the treatment and placebo groups (87\% vs 77\%, P = .35). FPB occurs in around twothirds of children when BAL is performed. Fever occurred significantly more frequently when BAL culture is positive. A single standard dose of the nonsteroidal anti-inflammatory drug ibuprofen administered before a FOB does not prevent FPB. [\hyperlink{Fabior}{PMID: 32725954}, Leon Joseph et al., 2020]

\hypertarget{pmid_15690910}{T}he efficacy of the fluoroquinolone levofloxacin in the treatment of 35 children with bronchopulmonary disease exacerbation was practically the same as that of amoxycillin/clavulanate, cefotaxime or ceftriaxone. The clinical and bacteriological results were favourable. The eradication of the pathogens responsible for the bronchopulmonary inflammations in 86\% of the patients was stated. There is no doubt that fluoroquinolones should not be widely used in pediatrics. They should be considered as reserve drugs for the treatment of severe cases when the routine agents fail. Their use is justified when the situation is risky and the data on the pathogen susceptibility to the drugs are available. Still, levofloxacin is the most safe fluoroquinolone with low hepatotoxicity and lower effect on the central nervous system. The episodes of its negative cardiovascular action are less frequent. Moreover, the most frequent side effects of fluoroquinolones such as nausea, diarrhea or vomiting are less frequent with the use of levofloxacin. No signs of arthropathy in the patients treated with levofloxacin were observed in our trial. [\hyperlink{Fabior}{PMID: 15690910}, I K Volkov et al., 2004]

\hypertarget{pmid_35083718}{P}arenteral iron is generally considered safe in adults, and severe adverse events are extremely rare. Ferric carboxymaltose (FCM), a third-generation parenteral iron product, is not licensed for pediatric use. The aim of this study was to present our data on the safety of FCM in children with iron deficiency (ID) and/or iron deficiency anemia (IDA) and to investigate through a systematic literature review articles reporting on the safety of FCM use in children with ID/IDA. Safety data regarding children treated with FCM for ID/IDA from four pediatric departments in Greece over a 26-month period are presented. Additionally, a literature search was performed in PubMed, Scopus, and Google Scholar on December 4, 2021 for articles reporting on the use of FCM in children with ID/IDA. Review articles, guidelines, case reports/case series, and reports on the use of FCM for conditions other than ID/IDA were excluded. Identified articles were screened for all reported adverse events (AE) that were graded according to the Common Terminology Criteria for Adverse Events, version 5.0. In our cohort, 37 children with ID/IDA received 41 FCM infusions. All infusions were tolerated well. In addition, 11 articles reporting 1231 infusions of FCM in 866 children were identified in the literature. Among them, 52 (6\%) children developed AE that were graded as mild or moderate (grades I-III). Our patient cohort and this literature review provide further evidence for the good safety profile of FCM in children, although well-designed prospective clinical trials with appropriate safety endpoints are still required. [\hyperlink{Fabior}{PMID: 35083718}, Paraskevi Panagopoulou et al., 2022]

\hypertarget{pmid_9241926}{D}ue to its exceedingly high fluoride content, 40\% silver fluoride solution has the potential to cause fluorosis when used in young children. In vitro testing conducted in the present investigation indicates that application of 40\% silver fluoride to deep carious lesions or its use as a 'spot' application agent could result in 3 to 4 mg of fluoride reaching the systemic circulation. As scientifically-based clinical trials on the safety of 40\% silver fluoride have not been conducted, it would be appropriate for it to be withdrawn from further clinical use until proper testing and evaluation have been carried out. In view of the possibility that lower strength solutions of silver fluoride (1-4\%) may be just as effective as 40\% in 'arresting' deep caries, testing should focus on such solutions, particularly as the potential for toxicity from their fluoride content would be reduced by a factor of 10-40. [\hyperlink{Fabior}{PMID: 9241926}, T Gotjamanos et al., 1997]

\hypertarget{pmid_17727139}{C}urrent data, although incomplete, suggest that pediatric administration of a fluoroquinolone, especially the best-studied ciprofloxacin, is safe. However, many experts have raised concerns regarding the emergence of fluoroquinolone-resistant pathogens such as pneumococcus if more children are treated with fluoroquinolones. Examination of the available data suggests that these concerns remain valid. Therefore, most experts continue to advise against expanded pediatric use of fluoroquinolones, except in those selected clinical situations where the benefits clearly outweigh the risks of therapy and there are few other antibiotic choices. [\hyperlink{Fabior}{PMID: 17727139}, Thomas S Murray et al., 2007]

\hypertarget{pmid_12417880}{I}nformation on the dose of steroid infants inhale from spacer devices and its potential effect on adrenal suppression is limited. We sought to determine the total dose of fluticasone propionate (FP) inhaled from a spacer device (Babyhaler) with face mask attachment by infants recovering from acute bronchiolitis and the effect of inhaled FP on the infants' overnight urinary cortisol/creatinine ratios (UCCRs). Infants studied were recovering from acute bronchiolitis. In study 1, 22 infants inhaled 150 microg of FP through the Babyhaler. The likely inhaled dose was estimated by trapping it on a filter held within the face mask. In study 2, 40 infants had UCCRs measured before and during 3 months of treatment with either FP (150 microg twice daily, n = 20) or placebo (n = 20). In study 1 the mean +/- SD dose of captured FP was 12.8 +/- 6.9 microg (ie, 2.1 +/- 1.2 microg/kg). In study 2 the pretreatment UCCR medians (interquartile ranges) were as follows: FP, 22.8 (23.0) nmol/mmol; placebo, 24.0 (28.3) nmol/mmol. Within-group UCCR changes (median and interquartile range DeltaUCCR) were significantly different in the FP group (-8.9 and -20.6 nmol/mmol at 6 weeks and -12.6 and -25.9 nmol/mmol at 12 weeks, respectively; P =.0008) but not in the placebo group ( -5.8 and -10.7 nmol/mmol at 6 weeks and +0.3 and -17.9 nmol/mmol at 12 weeks, respectively; P =.45). Intergroup changes were insignificant in the follow-up period (6 weeks, P =.52; 12 weeks, P =.19). After bronchiolitis, infants are likely to inhale approximately 8 \% of the nominal steroid dose from the Babyhaler. UCCRs can be used to monitor the bioavailability of inhaled steroids in young infants. [\hyperlink{Fabior}{PMID: 12417880}, Jackson Wong et al., 2002]

\hypertarget{pmid_33370847}{G}lobally, has been an increase in the use of silver fluoride products to arrest carious lesions and a variety of products are available. To examine differences in caries arrest and lesion colour of primary tooth carious lesions. A four-armed, parallel-design cluster-randomised controlled trial which investigated four protocols for caries arrest at 6m and 12m. Children in Group 1 and Group 2 received Rivastar Silver Diammine Fluoride (SDF), and children in Group 3 and Group 4 received a stabilised aqueous silver fluoride solution (AgF). Children in Group 2 and Group 4 received an additional application of KI immediately after the fluoride. Differences in caries arrest and lesion appearance were examined at 6m and 12m using two level logistic regression modelling. The arrest rate varied by group membership; group 1 and group 3 had higher arrest rates (77.3\% and 75.3\% respectively) than group 2 and group 4 (65.4\% and 51.2\% respectively). The use of KI was also associated with lower odds of arrest (12m OR 0.25; CI 0.19, 0.34) and higher odds of avoiding black discolouration (12m OR 6.08; 2.36, 15.67). Globally, has been an increase in the use of silver fluoride products to arrest carious lesions and a variety of products are available. This study demonstrated that both AgF and SDF can effectively arrest carious lesions on primary teeth. The use of KI is associated with poorer caries control but better aesthetic outcomes. [\hyperlink{Fabior}{PMID: 33370847}, Bathsheba Turton et al., 2021]

\hypertarget{pmid_25726705}{W}ith the increasing resistance to antibiotics among common bacterial pathogens, challenges associated with the use of fluoroquinolones (FQs) in paediatrics have emerged. The majority of FQs have favourable pharmacokinetic properties, although these properties can differ in children compared with adults. Moreover, all FQs have broad antimicrobial activity both against Gram-positive and Gram-negative bacteria. However, only some FQs for which adequate studies are available have been approved for use in children in a limited number of clinical situations owing to the supposed risk of development of severe musculoskeletal disorders, as demonstrated in juvenile animals. Recent short- and long-term evaluations appear to indicate that, at least for levofloxacin, this risk, if present at all, is marginal. This marginal risk could lead to more frequent use of FQs in children, even to treat diseases for which several other drugs with documented efficacy, safety and tolerability are considered the first-line antibiotics. However, for most of the FQs, adequate long-term studies of safety are not available. This indicates that the use of FQs should be limited to selected respiratory infections (including tuberculosis), exacerbation of lung disease in cystic fibrosis, central nervous system infections, enteric infections, febrile neutropenia, as well as serious infections attributable to FQ-susceptible pathogen(s) in children with life-threatening allergies to alternative agents. When considering diseases that could benefit from the use of FQs, particular attention must be paid to the choice of drug and its dosage, considering that not all of the FQs have been evaluated in different diseases.  [\hyperlink{Fabior}{PMID: 25726705}, Nicola Principi et al., 2015] We evaluated the safety of ciprofloxacin administered in a dose of 15-25 mg/kg for 9-16 days, in a case series of 58 children who were between 8 months and 13 years of age. No arthropathy was observed during therapy and follow-up. Blinded evaluation of 22 pairs of nuclear magnetic resonance scans obtained before and between day 10 and 15 of therapy did not reveal any cartilage damage. After the first dose of ciprofloxacin (10 mg/kg), serum fluoride levels increased at 12 h in 15 of 19 (79\%) patients; 24-h urinary fluoride excretion was higher on day 7 compared with basal values in 16 of 18 (88.9\%) patients. Height z scores of 53 patients at a mean of 22.5 months of follow-up were not significantly different from basal scores (p = 0.12). In conclusion, ciprofloxacin may be recommended for use in children for short duration when effective alternative antibacterials are unavailable. However, there is a need for further studies to evaluate the tissue accumulation of fluoride and its potential to cause toxic effects. [\hyperlink{Fabior}{PMID: 25726705}, K M Pradhan et al., 1995]

\hypertarget{pmid_16102652}{T}he use of fluoroquinolones in children is limited because of the potential of these agents to induce arthropathy in juvenile animals and to potentiate development of bacterial resistance. No quinolone-induced cartilage toxicity as described in animal experiments has been documented unequivocally in patients, but the risk fro rapid emergence of bacterial resistance associated with widespread, uncontrolled fluoroquinolones use in children is a realistic threat. Overall, the fluoroquinolones have been safe and effective in the treatment of selected bacterial infections in pediatric patients. There are clearly defined indications for these compounds in children who are ill. [\hyperlink{Fabior}{PMID: 16102652}, Urs B Schaad et al., 2005]

\hypertarget{pmid_10917383}{T}o review the pharmacokinetics, efficacy, and safety of fluoroquinolones in children. A MEDLINE search (January 1966-March 1998) was conducted for relevant literature. Data from compassionate use and published studies were reviewed for the assessment of pharmacokinetics, efficacy, and safety of fluoroquinolones in children. Fluoroquinolones have a broad spectrum coverage of gram-positive and gram-negative bacteria, including Pseudomonas aeruginosa and intracellular organisms. Fluoroquinolones are well absorbed from the gastrointestinal tract, have excellent tissue penetration, low protein binding, and long elimination half-lives. These antibiotics are effective in treating various infections and are well tolerated in adults. However, the use of fluoroquinolones in children has been restricted due to potential cartilage damage that occurred in research with immature animals. Fluoroquinolones have been used in children on a compassionate basis. Ciprofloxacin is the most frequently used fluoroquinolone in children, most often in the treatment of pulmonary infection in cystic fibrosis as well as salmonellosis and shigellosis. Other uses include chronic suppurative otitis media, meningitis, septicemia, and urinary tract infection. Safety data of fluoroquinolones in children appear to be similar to those in adults. Fluoroquinolones are associated with tendinitis and reversible arthralgia in adults and children. However, direct association between fluoroquinolones and arthropathy remains uncertain. Fluoroquinolones have been found to be effective in treating certain infections in children. Additional research is needed to define the optimal dosage regimens in pediatric patients. Although fluoroquinolones appear to be well tolerated, further investigations are needed to determine the risk of arthropathy in children. However, their use in children should not be withheld when the benefits outweigh the risks. [\hyperlink{Fabior}{PMID: 10917383}, A A Alghasham et al., 2000]

\hypertarget{pmid_36681802}{A}nti-influenza treatment is important for children and is recommended in many countries. This study assessed safety, clinical, and virologic outcomes of baloxavir marboxil (baloxavir) treatment in children based on age and influenza virus type/subtype. This was a post hoc pooled analysis of two open-label non-controlled studies of a single weight-based oral dose of baloxavir (day 1) in influenza virus-infected Japanese patients aged < 6 years (n = 56) and ≥ 6 to < 12 years (n = 81). Safety, time to illness alleviation (TTIA), time to resolution of fever (TTRF), recurrence of influenza illness symptoms and fever (after day 4), virus titer, and outcomes by polymerase acidic protein variants at position I38 (PA/I38X) were evaluated. Adverse events were reported in 39.0 and 39.5\% of patients < 6 years and ≥ 6 to < 12 years, respectively. Median (95\% confidence interval) TTIA was 43.2 (36.3-68.4) and 45.4 (38.9-61.0) hours, and TTRF was 32.2 (26.8-37.8) and 20.7 (19.2-23.8) hours, for patients < 6 years and ≥ 6 to < 12 years, respectively. Symptom and fever recurrence was more common in patients < 6 years with influenza B (54.5 and 50.0\%, respectively) compared with older patients (0 and 25.0\%, respectively). Virus titers declined (day 2) for both age groups. Transient virus titer increase and PA/I38X-variants were more common for patients < 6 years. The safety and effectiveness of single-dose baloxavir were observed in children across all age groups and influenza virus types. Higher rates of fever recurrence and transient virus titer increase were observed in children < 6 years. Japan Pharmaceutical Information Center Clinical Trials Information JapicCTI-163,417 (registered 02 November 2016) and JapicCTI-173,811 (registered 15 December 2017). [\hyperlink{Fabior}{PMID: 36681802}, Nobuo Hirotsu et al., 2023]

\section*{Halobetasol Propionate}
\subsection*{Result}
\subsubsection*{Answer}

Unknown

\subsubsection*{{Explanation}}
\hypertarget{Halobetasol Propionate}
A review of the available abstracts reveals several clinical studies and reviews on the safety and efficacy of halobetasol propionate (HP) in the treatment of psoriasis and other dermatoses. However, none of these studies specifically mention the inclusion of children or provide age ranges that would indicate pediatric subjects were studied.

For example, the phase 3 studies evaluating halobetasol propionate 0.01\% lotion for moderate-to-severe plaque psoriasis [\hyperlink{pmid_30365586}{PMID: 30365586}, Lawrence J Green et al., 2018; \hyperlink{pmid_30893392}{PMID: 30893392}, Jeffrey L Sugarman et al., 2019; \hyperlink{pmid_33683083}{PMID: 33683083}, Seemal R Desai et al., 2021; \hyperlink{pmid_32845589}{PMID: 32845589}, Fran E Cook-Bolden et al., 2020] describe the study population as "subjects" or "participants" without specifying pediatric age groups or including data on children. Similarly, studies on halobetasol propionate 0.05\% cream or ointment [\hyperlink{pmid_1757613}{PMID: 1757613}, H I Katz et al., 1991; \hyperlink{pmid_1757614}{PMID: 1757614}, C A Guzzo et al., 1991] do not mention children or provide pediatric safety data.

One review of halobetasol propionate and tazarotene lotion [\hyperlink{pmid_32606876}{PMID: 32606876}, Vidhatha Reddy et al., 2020] explicitly states that the combination is approved for adults, with no mention of pediatric use or safety.

None of the abstracts provide evidence from targeted studies on the safety of halobetasol propionate in children of any age range. Therefore, based on the abstracts available, the safety of halobetasol propionate in children is unknown.

\subsection*{Abstracts}
\hypertarget{pmid_30365586}{T}opical corticosteroids (TCS) are the mainstay of psoriasis treatment; long-term safety concerns limiting consecutive use of potent TCS to 2-4 weeks. Investigate safety and efficacy of halobetasol propionate 0.01\% lotion in moderate-to-severe plaque psoriasis. Two multicenter, randomized, double-blind, vehicle-controlled phase 3 studies (N=430). Subjects randomized (2:1) to halobetasol propionate 0.01\% lotion or vehicle once-daily for 8 weeks, 4-week posttreatment follow-up. Primary efficacy assessment: treatment success (at least a 2-grade improvement from baseline in Investigator Global Assessment [IGA] score and 'clear' or 'almost clear') at week 8. Safety and treatment emergent adverse events (AEs) evaluated throughout. Halobetasol propionate 0.01\% lotion demonstrated statistically significant superiority over vehicle as early as week 2. By week 8, 36.5\% (Study 1) and 38.4\% (Study 2) of subjects were treatment successes compared with 8.1\% and 12.0\% on vehicle (P less than 0.001). Halobetasol propionate 0.01\% lotion was also superior in reducing psoriasis signs and symptoms, body surface area (BSA), and improving quality of life. Halobetasol propionate 0.01\% lotion was well-tolerated with no treatment-related AEs greater than 1\%. Study did not include subjects with BSA greater than 12. Halobetasol propionate 0.01\% lotion was associated with significant reductions in the severity of the clinical signs of psoriasis, without the safety concerns of a longer treatment course. J Drugs Dermatol. 2018;17(10):1062-1069. [\hyperlink{Halobetasol Propionate}{PMID: 30365586}, Lawrence J Green et al., 2018]

\hypertarget{pmid_6937455}{H}aloperidol is safe and effective in children for relieving psychotic symptoms associated with childhood autism, schizophrenia and mental retardation. It is the drug of choice for Tourette's syndrome, and may be useful in nonpsychotic hyperactive or aggressive children to control acute episodes, or when the stimulants normally useful in hyperactive children are ineffective. Such children taking haloperidol not only become calmer, but are often better able to respond to other modalities of therapy and to school instruction. Dosage, initially low, is increased gradually to minimize drowsiness and extrapyramidal symptoms, the most common side effects. Haloperidol in children is usually well-tolerated. [\hyperlink{Halobetasol Propionate}{PMID: 6937455}, A C Serrano et al., 1981]

\hypertarget{pmid_30893392}{P}otent topical corticosteroids (TCSs) are the mainstay of psoriasis treatment. Safety concerns have limited use to 2 to 4 weeks. The objective of our study was to investigate the safety and efficacy of once-daily halobetasol propionate (HP) lotion 0.01\% in moderate to severe plaque psoriasis through 2 multicenter, randomized, double-blind, vehicle-controlled phase 3 studies (N=430). Participants were randomized (2:1) to HP lotion 0.01\% or vehicle once daily for 8 weeks, followed by 4 weeks of follow-up. The primary efficacy assessment was treatment success (at least a 2-grade improvement in baseline investigator global assessment [IGA] score and a score of 0 [clear] or 1 [almost clear]). Additional assessments included improvement in psoriasis signs and symptoms, body surface area (BSA), and a composite score of IGA×BSA. Safety and treatment-emergent adverse events (AEs) were evaluated throughout. We found that HP lotion 0.01\% demonstrated statistically significant superiority over vehicle as early as week 2 and also was superior in reducing psoriasis signs and symptoms and BSA involvement. [\hyperlink{Halobetasol Propionate}{PMID: 30893392}, Jeffrey L Sugarman et al., 2019]

\hypertarget{pmid_33683083}{P}soriasis is a chronic, inflammatory disease that may differ in prevalence and clinical presentation among patients from various racial and ethnic groups. Two phase 3 studies demonstrated efficacy and safety of halobetasol propionate (HP) 0.01\% lotion in the treatment of moderate-to-severe plaque psoriasis (NCT02514577, NCT02515097). These post hoc analyses evaluated HP 0.01\% lotion in Hispanic participants. Participants were randomized (2:1) to receive once-daily HP or vehicle lotion for 8 weeks, with a 4-week posttreatment follow-up. Post hoc efficacy assessments in Hispanic participants (HP, n=76; vehicle, n=43) included treatment success (\&ge;2‑grade improvement in Investigator\&rsquo;s Global Assessment and score of \&lsquo;clear\&rsquo; or \&lsquo;almost clear\&rsquo;), psoriasis signs, and affected body surface area (BSA). Treatment-emergent adverse events (TEAEs) were evaluated. At week 8, 38.8\% of participants achieved treatment success with HP versus 10.3\% on vehicle (P=0.001). HP‑treated participants achieved greater improvements in psoriasis signs, compared with vehicle (P\&lt;0.01 all). HP group had a greater reduction in affected BSA versus vehicle (P=0.001). Treatment-related TEAEs with HP were application site infection and dermatitis (n=1 each). Once-daily HP 0.01\% lotion was associated with significant reductions in disease severity in Hispanic participants with moderate-to-severe psoriasis, with good tolerability and safety over 8 weeks. J Drugs Dermatol. 2021;20(3):252-258. doi:10.36849/JDD.5698. [\hyperlink{Halobetasol Propionate}{PMID: 33683083}, Seemal R Desai et al., 2021]

\hypertarget{pmid_30659785}{P}ropranolol is an effective method of treatment for infantile hemangiomas (IH). A recent concern is a shift of the therapy into outpatient settings. The aim of the study was to evaluate the safety of initiating and maintaining propranolol therapy for IH. The study involved 55 consecutive children with IH being treated with propranolol. The patients were assessed in the hospital at the initiation of the therapy and later in outpatient settings during and after the therapy. Each time, the following monitoring methods were used: physical examination, cardiac ultrasound (ECHO), electrocardiography (ECG), blood pressure (BP), heart rate (HR), and biochemical parameters: blood count, blood glucose, aspartate transaminase (AST), alanine transaminase (ALT), and ionogram. The therapeutic dose of propranolol was 2.0 mg/kg/day divided into 2 doses. Four children were excluded during the qualification or the initiation of propranolol; a total of 51 patients were subject to the final analysis. All the children presented clinical improvement. There was a significant reduction in the mean HR values only at the initiation of propranolol. There were no changes in HR during the course of the therapy. Blood pressure values were within normal limits. Both systolic and diastolic values decreased in the first 3 months. Bradycardia and hypotension were observed sporadically, and they were asymptomatic. Electrocardiography did not show significant deviations. The pathological findings of the ECHO scans were not a contraindication to continuing the therapy. There were no changes in biochemical parameters. Apart from 1 symptomatic case of hypoglycemia, other low glucose episodes were asymptomatic and sporadic. The observed adverse effects were mild and the propranolol dose had to be adjusted in only 6 cases. Propranolol is effective, safe and well-tolerated by children with IH. The positive results of the safety assessment support the strategy of initiating propranolol in outpatient settings. Future studies are needed to assess the benefits of the therapy in ambulatory conditions. [\hyperlink{Halobetasol Propionate}{PMID: 30659785}, Lidia Babiak-Choroszczak et al., 2019]

\hypertarget{pmid_1757613}{T}he efficacy and safety of halobetasol propionate 0.05\% cream, an ultra high-potency corticosteroid preparation, was evaluated in a double-blind, vehicle-controlled, paired comparison study. Patients' psoriatic lesions were evaluated before treatment and after 1 and 2 weeks of twice-daily treatment with halobetasol propionate and vehicle. Response measures (plaque elevation, erythema, scaling, and pruritus) were evaluated with a 4-point severity scale whereby the sum provided a total score. Patient self-assessment measures were obtained at the 2-week visit by categorizing his or her global responses to queries about each treatment's "effectiveness" and "overall rating." All efficacy parameters, as judged by the physician, showed statistically significant (p = 0.0001) treatment differences favoring halobetasol propionate at both week 1 and week 2 evaluations. Patient global responses for "effectiveness" and "overall rating" favored halobetasol propionate 0.05\% cream over vehicle after 2 weeks of use. No systemic adverse drug effects were reported during the study. No patient was discontinued from the study because of an adverse event, and there was no evidence of skin atrophy after 2 weeks of treatment with either agent. Patient reports of "stings" or "burns" were equally distributed between the active and vehicle treatment groups. This trial demonstrates that halobetasol propionate 0.05\% cream is clinically beneficial and without evidence of significant risk in the treatment of plaque psoriasis. [\hyperlink{Halobetasol Propionate}{PMID: 1757613}, H I Katz et al., 1991]

\hypertarget{pmid_34918994}{T}o evaluate the safety of initiating and maintaining propranolol therapy for infantile hemangioma (IH) and the safety of different doses. The retrospective analysis included 336 consecutive cases of infants with IH treated between January 2016 and October 2017. The patients were assessed in the hospital at the initiation of the therapy and later in outpatient settings during the therapy. The monitoring included blood pressure (BP), heart rate (HR), blood glucose, hepatic and renal function, myocardial enzymes and serum lipids. Cardiac examinations in the outpatient follow-up included electrocardiography, ultrasound echocardiography, height, weight and head circumference. Propranolol decreased BP and HR at the initiation of treatment. The incidences of sinus bradycardia and hypoglycemia increased with the time of administration. Mean height, weight and head circumference were not affected during the treatment. The incidence of PR prolongation was 0\%-5.7\%. The effect of propranolol on the cardiovascular system, metabolism and physical development was not affected by its dose. Oral propranolol is a safe treatment for IH. Serious side effects were not observed. Attention should be paid to the side effects during clinical treatment. [\hyperlink{Halobetasol Propionate}{PMID: 34918994}, Lu Yu et al., 2022]

\hypertarget{pmid_32845589}{I}ntroduction: Psoriasis is a chronic, immune-mediated skin disease that is associated with sex-related differences. Two double-blind, vehicle-controlled, phase 3 studies evaluated halobetasol propionate (HP) 0.01\% lotion for the treatment of moderate-to-severe localized plaque psoriasis; pooled post hoc analyses investigated efficacy and safety in male and female subgroups. Methods: Participants were randomized (2:1) to once-daily HP or vehicle lotion for 8-weeks of double-blind treatment, with a 4-week posttreatment follow-up. Post hoc efficacy assessments in male (n=253) and female (n=177) subgroups included treatment success (≥2‑grade improvement in Investigator's Global Assessment [IGA] score and score of 'clear' or 'almost clear'), treatment success in psoriasis signs (erythema, plaque elevation, and scaling) at the target lesion, and change in affected body surface area (BSA). Treatment-emergent adverse events (TEAEs) were evaluated. Results: At week 8, rates of IGA-rated treatment success were significantly greater for HP versus vehicle in males (34.0\% vs 6.4\%) and females (42.7\% vs 14.6\%; P<0.001 both). Treatment success in each psoriasis sign approached or exceeded 50\% for HP-treated males and females, with all differences versus vehicle statistically significant (P<0.001). Percent reduction in affected BSA was significantly greater for HP versus vehicle in males (34.9\% vs 6.7\%) and females (35.6\% vs 4.6\%; P<0.001 both). Five HP treatment-related TEAEs (all application site-related) were reported through week 8. Conclusions: HP lotion was associated with significant reductions in disease severity in male and female participants with moderate-to-severe psoriasis, with good tolerability and safety over 8 weeks of once-daily use. In the overall pooled population, results were similar. J Drugs Dermatol. 2020;19(8): doi:10.36849/JDD.2020.5250. [\hyperlink{Halobetasol Propionate}{PMID: 32845589}, Fran E Cook-Bolden et al., 2020]

\hypertarget{pmid_1757614}{T}he efficacy and safety of 0.05\% halobetasol propionate ointment were evaluated in patients with chronic atopic or other eczematous dermatoses in two vehicle-controlled, double-blind studies: a paired-comparison study in 124 patients (study A) and a parallel-group study in 100 patients (study B). In study A, patients applied both treatments twice daily for 2 weeks and were evaluated by investigators on days 0, 7, and 14 with 0 to 3 severity scales and by self-assessment with two 5-step end-of-treatment rating scales. In study B, patients applied treatments twice daily for 2 weeks, and investigators made evaluations on days 0, 3, 7, and 14 with 0 to 6 scales and also made a 5-step end-of-treatment physician's global assessment. In study A, both severity scores and patient ratings favored halobetasol propionate significantly on days 7 (p less than or equal to 0.0013) and 14 (p less than 0.0001); in study B, severity scores on days 3 (p less than or equal to 0.045, pruritus, erythema, and overall lesion severity), 7, and 14 (p less than 0.001, all comparisons) also favored halobetasol propionate significantly, and global assessments showed complete resolution or marked improvement for 83\% of patients using halobetasol propionate versus 28\% of those using vehicle (p less than 0.0001). No instances of systemic effects or skin atrophy were reported in either study. We conclude that 0.05\% halobetasol propionate ointment is highly effective and well tolerated in the treatment of the conditions studied, with the rapid action and high degree of clearing associated with superpotent corticosteroid formulations. [\hyperlink{Halobetasol Propionate}{PMID: 1757614}, C A Guzzo et al., 1991]

\hypertarget{pmid_27688361}{G}iven the widespread use of propranolol in infantile hemangioma (IH) it was considered essential to perform a systematic review of its safety. The objectives of this review were to evaluate the safety profile of oral propranolol in the treatment of IH. We searched Embase and Medline databases (2007-July 2014) and unpublished data from the manufacturer of Hemangiol/Hemangeol (marketed pediatric formulation of oral propranolol; Pierre Fabre Dermatologie, Lavaur, France). Selected studies included ≥10 patients treated with oral propranolol for IH and that either reported ≥1 adverse event or effect (AE) or planned to capture AEs. Data capture was standardized and extracted study design, demographic characteristics, IH characteristics, intervention, and safety outcomes. AEs were assigned a system organ class and preferred term. A total of 83 of 398 identified literature records met the inclusion criteria, covering 3766 propranolol-treated patients. The manufacturer's data for 3 pooled clinical trials (435 propranolol-treated patients) and 1 Compassionate Use Program (1661 patients) were included. AE data were reported for 1945 of 5862 propranolol-treated patients. The most frequently reported AEs included a range of sleep disturbances, peripheral coldness, and agitation. The most serious AEs (atrioventricular block, bradycardia, hypotension, bronchospasm/bronchial hyperreactivity, and hypoglycemia-related seizures) were managed by decreasing doses or temporary/permanent discontinuation of propranolol. Limitations included the variety of included study designs; monitoring, collection, and reporting of AE data; small sample sizes for some articles; and the wide scope of review. Oral propranolol is well tolerated if appropriate pretreatment assessments and within-treatment monitoring are performed to exclude patients with contraindications and to minimize serious side effects during treatment. [\hyperlink{Halobetasol Propionate}{PMID: 27688361}, Christine Léaute-Labrèze et al., 2016]

\hypertarget{pmid_31688260}{B}eta-blocker (Propanolol or Timolol maleate) treatment of infantile hemangiomas (IH) is a safe and effective treatment in the outpatient setting. The authors report a single surgeon's initial experience with setting up an outpatient service of beta-blocker treatment for head and neck IH at a tertiary children's hospital. A prospective study of children with head and neck IHs commenced in January 2015 with the end point being December 2018. Each child started either oral propranolol (2 mg/kg/day) or topical Timolol 0.5\%. Thirty-eight patients commenced a beta-blocker during the study duration. The mean age at time of starting therapy was 9 months (range 3 weeks to 116 months). Four patients were older than 12 months at commencement. The mean duration of treatment was 9 months. The response to treatment was excellent or complete in 29\% (n = 11), good in 50\% (n = 18) and mild in 10\% (n = 4). The non response rate was 10\% (n = 4). No major adverse effects occurred but 29\% (n = 11) experienced minor side effects. Low dose propranolol and topical Timolol is been safe and easy to use for surgeons who may not be regular prescribers or unfamiliar with treating children with IHs with beta-blocker therapy. In patient monitoring is unnecessary and parents can be taught easily to recognise side effects. Treating children from the start builds a trusting relationship with the family before the child requesting cosmetic revision of the fibro-fatty remnant. [\hyperlink{Halobetasol Propionate}{PMID: 31688260}, Shiba Sinha et al., ]

\hypertarget{pmid_28043186}{O}ral propranolol has been recently approved for infantile hemangiomas (IHs), but potential side effects stay a challenge. We sought to make an additional assessment on oral propranolol safety for this indication. Prospective study included 108 infants consecutively treated for IHs at the University Children's Hospital Tirsova, Belgrade from January 2010 to December 2013. Propranolol was administered orally at a daily dose of 0.5 mg/kg and doubled every 48 hours in the absence of side effects until reaching the maximum dose of 2 mg/kg daily. Systolic and diastolic blood pressure and heart rate were measured every 48 hours with clinical observation. Heart rate was monitored by standard electrocardiogram (ECG) and 48-hour Holter ECG. Statistically significant, but asymptomatic decreases in systolic blood pressure and heart rate recorded by Holter ECG were observed during the first doubling of dose and then remained stable. Arrhythmias were not detected. Despite mild sleep disturbance observed in 31\% of infants in the hospital milieu, Holter monitoring indicated circadian rhythm maintenance. Oral propranolol for IHs does not remarkably affect heart rhythm including circadian variations throughout hospital initiation. Therefore, there is no necessity for Holter monitoring in additional safety assessment. [\hyperlink{Halobetasol Propionate}{PMID: 28043186}, Jelena Petrovic et al., 2017]

\hypertarget{pmid_29149854}{P}ropranolol has become the first-line treatment for complicated Infantile Hemangioma (IH), showing so far a good risk-benefit profile. We report the case of a toddler, on propranolol, who suffered cardiac arrest during an acute viral infection. She had a neurally-mediated syncope that progressed to asystole, probably because of concurrent factors as dehydration, beta-blocking and probably individual susceptibility to vaso-vagal phenomena. In fact a significant history of breath-holding spells was consistent with vagal hyperactivity. The number of patients treated with propranolol for IHs will increase and sharing experience will help to better define the safety profile of this drug. [\hyperlink{Halobetasol Propionate}{PMID: 29149854}, Alvise Tosoni et al., 2017]

\hypertarget{pmid_23082876}{P}ediatric patients undergoing hematopoietic stem cell transplantation (HSCT) are at high risk of acquiring fungal infections. Antifungal prophylaxis shortly after transplantation is therefore indicated, but data for pediatric patients under 12 years of age are scarce. To address this issue, we retrospectively assessed the safety, feasibility, and initial efficacy of prophylactic posaconazole in children. 60 consecutive pediatric patients with a median age of 6.0 years who underwent allogeneic HSCT between August 2007 and July 2010 received antifungal prophylaxis with posaconazole in the outpatient setting. 28 pediatric patients received an oral suspension at 5 mg/kg body weight b.i.d., and 32 pediatric patients received the suspension at 4 mg/kg body weight t.i.d. The observation period lasted from start of treatment with posaconazole until its termination (maximum of 200 days post-transplant). Pediatric patients who received posaconazole at 4 mg/kg body weight t.i.d. had a median trough level of 383 μg/L. Patients who received posaconazole at 5 mg/kg body weight b.i.d. had a median trough level of 134 μg/L. Both regimens were well tolerated without severe side effects. In addition, no proven or probable invasive mycosis was observed. Posaconazole was a well-tolerated, safe, and effective oral antifungal prophylaxis in pediatric patients who underwent high-dose chemotherapy and HSCT. Posaconazole at a dosage of 12 mg/kg body weight divided in three doses produced consistently higher morning trough levels than in patients who received posaconazole 5 mg/kg body weight b.i.d. Larger prospective trials are needed to obtain reliable guidelines for antifungal prophylaxis in children after HSCT. [\hyperlink{Halobetasol Propionate}{PMID: 23082876}, Michaela Döring et al., 2012]

\hypertarget{pmid_8712442}{A}n epidural test dose containing epinephrine does not reliably produce hemodynamic responses in children under halothane anesthesia. The purpose of this study was to determine hemodynamic responses to intravenous isoproterenol in both awake and halothane-anesthetized children. After obtaining institutional review board approval and parental informed consent, 72 ASA physical status 1 or 2 children (2.8 +/- 1.7 yr) undergoing elective minor surgery were studied before and during anesthesia with 1.2 minimum alveolar concentration halothane. A bolus containing 0.25 mg/ kg bupivacaine and 0.05 microgram/kg, 0.075 microgram/kg, or 0.1 microgram/kg isoproterenol, or bupivacaine and saline was injected via a peripheral arm vein to simulate intravascular injection of an epidural test dose. Before induction of anesthesia, all patients showed a positive test response after isoproterenol injection (heart rate increase > or = 20 beats/min). During anesthesia, 79\% of patients receiving 0.05 microgram/kg, 89\% of patients receiving 0.075 microgram/kg, and 100\% of patients receiving 0.1 microgram/kg met the criterion of a positive test response. Among each treatment group, all infants showed a positive test response. Blood pressure did not differ among the groups at any time. Transient benign dysrhythmias occurred in only one patient under halothane anesthesia receiving 0.075 microgram/kg isoproterenol. Isoproterenol at a dose of 0.1 microgram/kg is a sensitive indicator for intravascular injection of a test dose in children anesthetized with halothane and nitrous oxide. Isoproterenol at a dose of 0.05 microgram/kg approximates a minimal effective dose in awake children and in infants. After detailed studies on neural toxicity, isoproterenol could be of value as an epidural test agent in children. [\hyperlink{Halobetasol Propionate}{PMID: 8712442}, S Kozek-Langenecker et al., 1996]

\hypertarget{pmid_27043724}{O}ral propranolol is now established as the first-line treatment for infantile haemangiomas, and used in up to 20 \% of all cases. Propranolol use in infants is most commonly instigated in a controlled environment to monitor for potential serious adverse events such as hypoglycaemia and hypotension. Two test doses are recommended, the first one of 300 μg/kg followed by 2-hourly monitoring. On the subsequent day, a further dose of 650 μg/kg is administered with the same monitoring. A dose of 2 mg/kg divided into three is started from the next day. Parents/carers need to be warned of common adverse effects, of which disturbed sleep is the commonest. Treatment is recommended for up to a year to avoid rebound growth and the need to restart the treatment.  [\hyperlink{Halobetasol Propionate}{PMID: 27043724}, Robert H Taylor et al., 2016] Halobetasol propionate and tazarotene lotion 0.01\%/0.045\% (HP/TAZ) is a topical medication approved for the treatment of plaque psoriasis in adults. As a treatment modality, HP/TAZ has a combinatory therapeutic effect because it contains both a corticosteroid (HP) and a retinoid (TAZ) component. Here, we review the important clinical efficacy and safety data derived from pivotal clinical trials for HP/TAZ in the treatment of plaque psoriasis. We also discuss the mechanism of action, dosage guidelines, pharmacokinetics/pharmacodynamics, and clinical considerations for HP/TAZ, including why HP/TAZ should be avoided in pregnant patients. [\hyperlink{Halobetasol Propionate}{PMID: 27043724}, Vidhatha Reddy et al., 2020]

\hypertarget{pmid_31170512}{H}alobetasol propionate (HB) is considered a super potent drug in the group of topical corticosteroids. HB has anti-inflammatory activity, vasoconstriction properties, and due to its high skin penetration, it can cause systemic side effects. To improve its characteristics, enhance topical effectiveness and reduce penetration to systemic circulation, a study to optimize and characterize a HB-loaded lipid nanocarrier (HB-NLC) has been made by high-pressure homogenization method. The formulation is composed by HB, surfactant, glyceryl distearate and capric glycerides. The optimized HB-NLC containing 0.01\% of HB and 3\% of total lipid shows an average size below 200 nm with a polydispersity index ≪0.2 and an encapsulation efficiency ≫90\%. The in vitro and in vivo tests indicate that the HB-NLC is not toxic, is well tolerated and has an anti-inflammatory effect because they decrease the production of Interleukins in keratinocytes and monocytes. HB-NLC is considered an alternative treatment for skin inflammatory disorders. [\hyperlink{Halobetasol Propionate}{PMID: 31170512}, Paulina Carvajal-Vidal et al., 2019]

\hypertarget{pmid_25753275}{P}ropranolol has been recently approved by health authorities to treat infantile haemangiomas (IH). Propranolol is indicated in infants less than 5months of age with an IH requiring systemic therapy: IH at life-threatening and/or functional risk, painful ulcerated IH and IH that may cause permanent disfigurement. Propranolol should be initiated by physicians who have expertise in the diagnosis, treatment and management of IH. In addition, the first intake and every escalation should be administrated in a controlled clinical setting where adequate facilities for handling of adverse reactions, including those requiring urgent measures, are available. Then a monthly monitoring with dose adjustment weight is mandatory by the family doctor. Parents should be informed of the risk of hypoglycaemia and bronchoconstriction, especially during respiratory infectious outbreaks. The recommended duration of treatment is 6months without tapering. Relapses are possible necessitating a second course of 3 to 6months of treatment.  [\hyperlink{Halobetasol Propionate}{PMID: 25753275}, C Léauté-Labrèze et al., 2015] Infantile haemangiomas (IH) are the most common benign tumours in children. They are characterised by rapid growth during the first year of life followed by spontaneous regression during childhood. Indications for treatment are functional impairment, bleeding/ulceration, rapid growth and severe aesthetic risk. Recently, systemic treatment with propranolol has become the first-line therapy. The objective of this study was to assess the efficacy of propranolol in the treatment of IH and to investigate whether treatment with a low dose of 1 mg/kg/day was sufficient. This study was retrospective and based on a review of children treated for IH with propranolol from the 2010-2012 period at Rigshospitalet. Overall, propranolol was effective in all but one child (97\%). The majority of the children (84\%) were treated with an initial dose of 1 mg/kg/day, which was considered sufficient in most cases (71\%). Children who started treatment before five months of age had a significantly better response than children who started treatment at a later age. No relation was found between location of IH and the effect of treatment. There were only few and mild side effects. Propranolol is effective in the treatment of IH and it has only few and mild side effects. In most cases, a low dose of 1 mg/kg/day was sufficient. Early initiation of treatment is recommended as the response to treatment was better in younger children and because early initiation helps prevent large residual changes. not relevant. not relevant. [\hyperlink{Halobetasol Propionate}{PMID: 25753275}, Ida Gillberg Andersen et al., 2014]

\hypertarget{pmid_24849505}{T}o evaluate the safety and efficacy of our institutional beta-blocker protocol for treatment of complicated infantile hemangiomas (IH). A retrospective descriptive study of 76 infants/children with IH treated with oral propranolol at the Children's Hospital of Philadelphia between June 2008 and August 2010 was performed, assessing both the safety and efficacy of propranolol. Based on preliminary data showing hemangioma recrudescence off-treatment, we reviewed 9 additional patients with recrudescence between August 2010 and December 2011. Mild adverse events included asymptomatic bradycardia, gastrointestinal symptoms, asymptomatic hypotension, cool hands/feet, asymptomatic hypoglycemia, and sleep disturbance. Sixteen patients had recrudescence of IH off-treatment, with propranolol discontinued at a median age of 14 months (interquartile range 10-15 months). Propranolol appears to be associated with minor, not severe symptomatic adverse events. Propranolol appears to be effective in treating complicated IH. Recrudescence can occur off-treatment, even with discontinuing propranolol as late as 15 months of age. [\hyperlink{Halobetasol Propionate}{PMID: 24849505}, Derek H Chu et al., 2014]

\hypertarget{pmid_23680605}{T}here has been widespread interest surrounding the use of beta-blockers (i.e. propranolol, timolol, nadolol, acebutolol) in the treatment of infantile hemangiomas (IH). To review literature evaluating treatment of IH with propranolol. We conducted a literature search on PubMed and investigated for case reports, case series, and controlled trials by using search terms including hemangioma and propranolol. Data suggest that beta-blockers are efficacious in cutaneous, orbital, subglottic, and hepatic hemangiomas and assist in the resolution of ulcerated hemangiomas. Improvement has also been documented in children with PHACE syndrome. Propranolol produces favorable results in children who do not respond to steroids and with no long-term adverse effects. Propranolol should be administered with caution due to rare but serious side effects including hypoglycemia, wheezing, hypotension, and bradycardia. Additionally, recurrence of lesions following the cessation of treatment has been documented. Although large-scale randomized controlled trials must be conducted in order to further evaluate the safety and the possible role of propranolol in the treatment of IH, the reviewed literature suggests that propranolol carries promise as a potential replacement for corticosteroids as first-line therapy or as a part of a multimodal approach. [\hyperlink{Halobetasol Propionate}{PMID: 23680605}, Nivedita Gunturi et al., 2013]

\hypertarget{pmid_22129321}{D}ata regarding the use of propranolol in pediatrics are limited despite its widespread use in adults. Since 1984, Propranolol has been used for the prevention of portal hypertensive hemorrhage in pediatric patients. Recently it has been also used for the management of hemangiomas in addition to other indications. The purpose of this review is to evaluate safety and efficacy of propranolol use in the pediatric population, highlighting the most important reported side effects, warnings and precautions. [\hyperlink{Halobetasol Propionate}{PMID: 22129321}, Mortada El-Shabrawi et al., 2011]

\hypertarget{pmid_25385271}{I}nfantile haemangiomas (IH) are neoplastic proliferations of endothelial cells which occur with an incidence of 10-12\%. IH rapidly growing and found in cosmetically sensitive areas or complicated with ulcerations are of special concern of parents. A review of medical charts was performed for newborns treated with propranolol because of IH between 2012 and 2013. There were two boys and two girls, referred to our department at the age of 2-3 weeks. Children were commenced on propranolol 0.5 mg/kg daily and closely monitored. The dosage was increased up to a maximum of 2 mg/kg/d and was maintained until the lesion had involuted or showed good result. The minimal dosage required to achieve involution was 1.5-2.0 mg/kg/d. No rebound growth or complications were observed. Three patients showed excellent response with resolution of the lesion. Fourth patient showed good result with >50\% reduction of IH. Propranolol at 1.5-2.0 mg/kg/d is effective and safe for treating IH in our series of newborn patients. Treatment should be maintained until the lesion is involuted or shows good cosmetic result. Still there is need for larger scale studies confirming the safety and efficacy of propranolol in treatment of haemangiomas in newborns. [\hyperlink{Halobetasol Propionate}{PMID: 25385271}, Marzanna Oksiuta et al., 2016]

\hypertarget{pmid_23340697}{W}e aimed to assess the efficacy and safety of low-dose propranolol for treatment of infantile hemangiomas (IHs) in China. Our prospective study included data from 89 patients with IH, aged 1-12 months. Plasma renin activity, angiotensin II, and aldosterone were measured before initiation of propranolol therapy. Patients were administered propranolol (0.75-1 mg/kg/day) under close observation. The volume, texture, and color of lesions were used to evaluate efficacy. Safety endpoints included heart rate, systolic and diastolic blood pressures, alanine transaminase, aspartate transaminase, thyroid function tests, and fasting blood glucose. Adverse effects were recorded. Mean plasma angiotensin II concentration in patients with IH was higher than that in age-matched healthy children, whereas mean plasma renin activity was lower. Mean aldosterone level was higher at 1-3 months but lower at 4-12 months, than values reported previously. After propranolol therapy for 6 months, IH regression was classed as grade IV in 44 patients (49.4 \%), grade III in 21 patients (23.6 \%), and grade II in 24 patients (27.0 \%); none were grade I. Mild adverse effects, including diarrhea, restless sleep, nausea, cold extremities, and hypoglycemia, occurred in 12 patients (13.5 \%). Slight decreases in heart rate and blood pressure occurred in all patients (p < 0.05). The IHs of four patients (4.5 \%) relapsed after treatment cessation at 4-5 months. Low-dose propranolol is effective and safe for Chinese children with IH, and larger-scale studies are merited. Mechanisms underlying IH pathogenesis, and possible involvement of the renin-angiotensin-aldosterone system, deserve study. [\hyperlink{Halobetasol Propionate}{PMID: 23340697}, Xiaorong Ma et al., 2013]

\section*{Advair HFA}
\subsection*{Result}
\subsubsection*{Answer}

Unknown

\subsubsection*{{Explanation}}
\hypertarget{Advair HFA}
A review of the available abstracts reveals that none of them are specifically about Advair HFA (fluticasone propionate and salmeterol HFA) in children. While there is an abstract evaluating the safety and efficacy of fluticasone propionate HFA in children aged 1 to <4 years, this study does not include the combination product Advair HFA, which contains both fluticasone and salmeterol. Therefore, there is no targeted study in the provided abstracts that directly assesses the safety of Advair HFA in children of any age group. As such, the safety of Advair HFA in children is unknown based on the abstracts available.

\subsection*{Abstracts}
\hypertarget{pmid_18978514}{T}he avian influenza A (H5N1) virus is considered to be a potential cause of the next influenza pandemic. Children may be particularly vulnerable to the pandemic virus, and they may react differently than adults to vaccines. We report the results of the first clinical trial of an H5N1 vaccine in children. Twelve healthy children (mean age +/- SD: 12.73 +/- 2.77 years) received a single dose of 6 microg of the inactivated whole virus vaccine Fluval. Twenty-one days after vaccination, immunogenicity was assessed by hemagglutination inhibition and microneutralization assays. Safety information was collected for 180 days. No side-effects were observed, and the vaccine fulfilled all applicable U.S. and European immunogenicity criteria for licensure. The post/prevaccination geometric mean titer ratio was 16.95, the rate of seroconversion was 75\% and the rate of seroprotection was also 75\% 21 days after vaccination. We confirmed our earlier findings of the present vaccine in adults showing encouraging safety and immunogenicity properties in children. Studies with the present vaccine in elderly subjects are underway. [\hyperlink{Advair HFA}{PMID: 18978514}, Zoltan Vajo et al., 2008]

\hypertarget{pmid_18276803}{T}here is a continued need for safe and effective treatments for children and adolescents with chronic hepatitis B. Adefovir dipivoxil (ADV) is a licensed treatment for chronic hepatitis B in adults. This study was designed to characterize the pharmacokinetic profile of adefovir following the administration of 0.14 mg/kg and 0.3 mg/kg of ADV (oral solution) in children aged 2 to 11 years and of ADV 10 mg in adolescents aged 12 to 17 years. Forty-five subjects were included in the pharmacokinetic and safety evaluations. Adefovir was rapidly absorbed. Adefovir levels rose rapidly in the first hour and then declined in a biphasic manner. Dose-proportional pharmacokinetics was observed in the 0.14-mg/kg and 0.3-mg/kg groups. The 0.3-mg/kg dose in children aged 2 to 6 and the 10-mg dose in adolescents resulted in exposures that were comparable to those seen previously in adults given ADV 10 mg. Adefovir dipivoxil was well tolerated at the doses evaluated in this study. Adverse events were generally mild and reported as being unrelated to study medication. There was 1 serious adverse event reported that was not related to study medication. No patient discontinued the study prematurely due to an adverse event related to the study drug. [\hyperlink{Advair HFA}{PMID: 18276803}, Etienne M Sokal et al., 2008]

\hypertarget{pmid_10722509}{T}he acyclic phosphonate analog adefovir is a potent inhibitor of retroviruses, including human immunodeficiency virus (HIV) type 1, and, unlike some antiviral nucleosides, does not require the initial phosphorylation step for its activity. Two oral dosages of the adefovir prodrug adefovir dipivoxil were evaluated in a phase I study with children with HIV infection. A total of 14 patients were stratified into age groups ranging from 6 months to 18 years of age. Eight patients received 1.5 mg of adefovir dipivoxil per kg of body weight, and six patients received 3.0 mg of adefovir dipivoxil per kg. Serum samples were obtained at intervals during the 8 h postdosing and were analyzed for adefovir concentrations. Patients were monitored for adverse effects. All samples collected resulted in quantifiable levels of adefovir (lower limit of quantitation, 25 ng/ml) from each patient. The areas under the concentration-versus-time curves (AUCs) were similar (P = 0.85) for the 1.5- and 3.0-mg/kg doses, while the apparent oral clearance (CL/F) was significantly higher (P = 0.05) for the 3-mg/kg dose. Pharmacokinetic parameters differed by patient age. In comparing those children older and younger than the median age of 5.1 years, AUC (P = 0.03), maximum concentration of drug in serum (P = 0.004), and the concentration at 8 h postdosing (P = 0.02) were significantly lower for the younger children. There were no significant differences for apparent volume of distribution and CL/F normalized to body surface area, but there was a suggestive difference in half-life (P = 0.07) among the subjects in the older and younger age groups. No significant adverse events were encountered. These data provide the basis for a multidose phase II study of adefovir dipivoxil in HIV-infected infants and children. [\hyperlink{Advair HFA}{PMID: 10722509}, W T Hughes et al., 2000]

\hypertarget{pmid_23798623}{D}ata on the efficacy of hydroxyurea (HU) in Indian children with sickle cell anaemia (SCA) is limited. Hence, we have evaluated the efficacy of fixed low dose HU in Indian children. The study cohort consisted of 144 children (<18 years of age) with SCA having severe manifestations (≥ 3 episodes of vasocclusive crisis or blood transfusions, or having ≥ 1 episode of acute chest syndrome or cerebrovascular stroke or sequestration crisis) who were started on fixed low dose HU (10 mg/kg/day). They were followed up for two years and monitored for the hematological and clinical efficacy and safety. There was significant increase in the fetal hemoglobin level (HbF\%), total hemoglobin and mean corpuscular volume. Vasoocclusive crises, blood transfusions, acute chest syndrome, sequestration crises and hospitalizations decreased significantly. Baseline HbF\% had significant positive correlation with HbF\% at 24 months. There was significant negative correlation between baseline HbF\% and change in HbF\% from baseline to 24 months. No significant correlation was found between HbF\% at baseline and clinical event rates per year after HU. No major adverse events occurred during the study period. Fixed low dose HU is effective and safe in Indian children with SCA. [\hyperlink{Advair HFA}{PMID: 23798623}, Dipti L Jain et al., 2013]

\hypertarget{pmid_12712854}{P}revalence of infection with hepatitis C virus (HCV) is lower in children than in adults. The detection rate of anti-HCV antibodies in Western countries is 0.1-0.4\% among children and adolescents. Prevalence of serologic response is higher in risk groups. HCV infection in children is usually asymptomatic, most of them have variations in serum levels of alanine aminotransferase (ALA). The laboratory exams for children are the same as those for adults. Histological progression may be faster in children than in adults. In this age group, HCV infection is considered as a special category, in which case it's possible to maintain the patient in observation without antiviral therapy. However, some studies with monotherapy showed that a regime with 1.75-3 MU/m2 of interferon alpha during 6-12 months induces a sustained viral response in 33-56\% of the children. Although ribavirin hasn't yet been accepted for pediatric use, there have been several clinical tests in small groups with oral doses of 15 mg/kg a day, combined with interferon, during 12 months. The results are good. Pegylated interferon alpha is not authorized for pediatric use. [\hyperlink{Advair HFA}{PMID: 12712854}, Solange Heller Rouassant et al., 2002]

\hypertarget{pmid_17143952}{T}o study the safety and efficacy of hepatitis A vaccine (HAV) in children with chronic liver disease of various etiologies. Eleven children with chronic liver disease and thirteen age- and sex-matched controls negative for HAV antibodies were vaccinated against hepatitis A after they gave their informed consent. Children with uncontrolled coagulopathy or signs of hepatic decompensation were excluded. The vaccine (Havrix: 720 ELISA units in 0.5 mL, from GlaxoSmithKline Biologicals) was given intramuscularly in the deltoid in 2 doses 6 mo apart. Children were tested for HAV antibodies one and six months after the 1st dose and one month after the 2nd dose. Total serum bilirubin, alanine aminotransferase (ALT), and aspartate aminotransferase (AST) were determined immediately before and after one month of the 1st dose of the vaccine. Only 7 out of the 11 patients were positive for HAV antibodies after the 1st dose of the vaccine, as compared to 100\% of the controls. One month after the 2nd dose, all patients tested were positive for HAV antibodies. No deterioration in liver functions of patients was noted after vaccination. No adverse events, immediate or late, were reported by the mothers after each dose of the vaccine. Hepatitis A vaccine is both safe and effective in this small studied group of children with chronic liver disease. Given the high seroconversion rate, post-vaccination testing for HAV antibodies is not needed. [\hyperlink{Advair HFA}{PMID: 17143952}, Hanaa-Mostafa El-Karaksy et al., 2006]

\hypertarget{pmid_33219694}{T}o summarize evidence regarding efficacy of anti-TNFα in childhood chronic uveitis, refractory to common DMARDs. An updated systematic search was conducted between November 2012 and January 2020. Studies investigating the efficacy of anti-TNFα therapy, in children of ages <16 years, as the first biologic treatment for childhood chronic uveitis, refractory to topical and/or systemic steroid and at least one DMARD were eligible for inclusion. The primary outcome measure was the improvement of intraocular inflammation according to Standardization of Uveitis Nomenclature Working Group criteria. A combined estimate of the proportion of children responding to etanercept (ETA), infliximab (INF), and adalimumab (ADA) was determined. We identified 1677 articles of which 37 articles were eligible. Three were randomized controlled trials, one on ETA and two on ADA, and were excluded from pooled analysis. From the observational studies, a total of 487 children were identified: 226 received ADA, 213 INF and 48 ETA. The proportion of responding children was 86\% (95\% CI: 76\%, 95\%) for ADA, 68\% (95\% CI: 50\%, 85\%) for INF and 36\% (95\% CI: 9\%, 67\%) for ETA. Pooled analysis showed clear differences (χ2 = 32.2, P < 0.0001): ADA and INF were both significantly superior to ETA (χ2 = 26.8, P < 0.0001, and χ2 = 7.41, P < 0.006, respectively), ADA significantly superior to INF (χ2 = 13.4, P < 0.0002). This meta-analysis, consistent with recent randomized controlled trial data, suggests the efficacy of ADA and INF in childhood chronic uveitis treatment. However, ADA results were superior to those of INF in this clinical setting. [\hyperlink{Advair HFA}{PMID: 33219694}, Ilaria Maccora et al., 2021]

\hypertarget{pmid_16834769}{A}ntiretroviral treatment (ART) in children has special features and consequently, results obtained from clinical trials with antiretroviral drugs in adults may not be representative of children. Nelfinavir (NFV) is an HIV-1 Protease Inhibitor (PI) which has become as one of the first choices of PI for ART in children. We studied during a 3-year follow-up period the effects of highly active antiretroviral therapy with nelfinavir in vertically HIV-1 infected children. Forty-two vertically HIV-infected children on HAART with NFV were involved in a multicentre prospective study. The children were monitored at least every 3 months with physical examinations, and blood sample collection to measure viral load (VL) and CD4+ cell count. We performed a logistic regression analysis to determinate the odds ratio of baseline characteristics on therapeutic failure. Very important increase in CD4+ was observed and VL decreased quickly and it remained low during the follow-up study. Children with CD4+ <25\% at baseline achieved CD4+ >25\% at 9 months of follow-up. HIV-infected children who achieved undetectable viral load (uVL) were less than 40\% in each visit during follow-up. Nevertheless, HIV-infected children with VL >5000 copies/ml were less than 50\% during the follow-up study. Only baseline VL was an important factor to predict VL control during follow-up. Virological failure at defined end-point was confirmed in 30/42 patients. Along the whole of follow-up, 16/42 children stopped HAART with NFV. Baseline characteristics were not associated with therapeutic change. NFV is a safe drug with a good profile and able to achieve an adequate response in children. [\hyperlink{Advair HFA}{PMID: 16834769}, Salvador Resino et al., 2006]

\hypertarget{pmid_30431383}{S}afety and immunogenicity data from 5 clinical trials conducted in the US in children 12-to-23 months old where HAVi was administered alone or concomitantly with other pediatric vaccines (M-M-R®II, Varivax®, TRIPEDIA®, Prevnar®, ProQuad®, PedvaxHIB®, and INFANRIX®) were combined. Among 4,374 participants receiving ≥ 1 dose of HAVi, 4,222 (97\%) had safety follow-up and the proportions reporting adverse events (AE) were comparable when administered alone (69.4\%) or concomitantly with other pediatric vaccines (71.1\%). The most common solicited injection-site AEs were pain/tenderness (Postdose 1: 25.8\%; Postdose 2: 26.1\%) and redness (Postdose 1: 13.6\%; Postdose 2: 15.1\%). The most common vaccine-related systemic AEs were fever (≥ 100.4ºF, 12.2\%) and irritability (8.1\%). Serious AEs (SAEs) were observed at a rate of 0.4\%; 0.1\% were considered vaccine-related. No deaths were reported within 14 days following a dose of HAVi. These integrated analyses also showed that protective antibody concentrations were elicited in 100\% of toddlers after two doses and 92\% after a single dose, regardless of whether HAVi was given concomitantly with other vaccines or alone. These results demonstrate that HAVi was well-tolerated whether given alone or concomitantly with other vaccines, with a low incidence of vaccine-related SAEs. HAVi was immunogenic in this age group regardless of whether administered with or without other pediatric vaccines and whether 1 or 2 doses were administered. HAVi did not impact the immune response to other vaccines. These data continue to support the routine use of HAVi with other pediatric vaccines in children ≥ 12 months of age. [\hyperlink{Advair HFA}{PMID: 30431383}, Maria Petrecz et al., 2019]

\hypertarget{pmid_27523719}{M}any children struggle with the use of albuterol hydrofluoroalkane (HFA) inhalers. Albuterol multidose dry powder inhaler (MDPI) may simplify rescue bronchodilator use in children. To compare the pharmacokinetics (PK), pharmacodynamics (PD), and tolerability of albuterol MDPI and albuterol HFA after a single inhaled dose in children with asthma. This single-center, open-label, two-period crossover study randomized children to albuterol MDPI or HFA 180 μg on two treatment days with a 4- to 14-day washout. Plasma albuterol concentrations were measured before the dose and up to 10 hours after the dose to determine the primary PK values of area under the plasma concentration-versus-time curve from time 0 to the last measurable concentration (AUC0-t), maximum observed concentration (Cmax), and AUC from time 0 extrapolated to infinity (AUC0-inf). Heart rate and blood pressure before the dose and after the dose were monitored for PD effects, and adverse events (AE) were monitored for overall safety. Fifteen children, ages 6-11 years, were included (PK, n = 13 for time to Cmax and terminal half-life of elimination; n = 12 for AUC and Cmax due to incomplete data). AUC0-t (geometric mean ratio [GMR] 1.056 [90\% confidence interval \{CI\}, 0.88-1.268]) and AUC0-inf (GMR 0.971 [90\% CI, 0.821-1.147]) were comparable between treatments. Cmax was larger for albuterol MDPI versus HFA (GMR 1.340 [90\% CI, 1.098-1.636]). PD parameters between the treatments were comparable. No deaths, serious AEs, treatment-emergent AEs, or withdrawals due to AEs were reported for either treatment. Albuterol MDPI and albuterol HFA had comparable PK and PD in children after a single 180-μg dose. ClinicalTrails.gov identifiers NCT01899144 and NCT02126839. [\hyperlink{Advair HFA}{PMID: 27523719}, Anoshie Ratnayake et al., 2016]

\hypertarget{pmid_36317914}{T}his study aimed to evaluate the efficacy and safety of high-frequency oscillation ventilation combined with volume guarantee (HFOV-VG) in preterm infants with acute hypoxemic respiratory failure (AHRF) after patent ductus arteriosus ligation. We retrospectively analyzed the clinical data of 41 preterm infants, who were ventilated for AHRF after patent ductus arteriosus ligation between January 2020 and January 2022. HFOV alone was used in 20 of the 41 infants, whereas HFOV-VG was used in the other 21 infants. There was no statistically significant difference in the demographic information and baseline characteristics of preterm infants included in the study. The average frequency tidal volume (VThf) of the HFOV-VG group was lower than that of the HFOV group (2.6 ± 0.6 mL versus 1.9 ± 0.3 mL, P < .001). In addition, the incidence of hypocapnia and hypercapnia in infants supported with HFOV-VG was significantly lower (15 versus 8, P < .001; 12 versus 5, P < .001). Furthermore, the duration of invasive ventilation in the HFOV-VG group also was lower than in the HFOV group (3.7 ± 1.2 days versus 2.1 ± 1.0 days, P < .01). Compared with HFOV alone, HFOV-VG decreases VThf levels and reduces the incidence of hypercapnia and hypocapnia in preterm infants with acute hypoxic respiratory failure after patent ductus arteriosus ligation. [\hyperlink{Advair HFA}{PMID: 36317914}, Hui-Zi Lin et al., 2022]

\hypertarget{pmid_24457365}{T}he aim of the study was to assess the long-term efficacy and tolerability of tumour necrosis factor α (TNFα) inhibitors in the therapy of children with refractory antinuclear antibody (ANA)-associated chronic anterior uveitis. Retrospective analysis of 31 children with ANA-associated uveitis, treated with TNFα inhibitors with a follow-up period of at least 2 years. The outcome measures included: control of inflammation, corticosteroid-sparing potential and side effects. Twenty-three children (74\%) were treated with adalimumab, five children (16\%) with infliximab and three children (10\%) with etanercept. Control of uveitis, defined as 0 anterior chamber cells while on ≤2 drops/day topical corticosteroids, was achieved in 22 of 31 patients (71\%) after 1 year (95\% CI 52\% to 86\%), and in 21 of 29 patients (72\%) after 2 years of treatment (95\% CI 53\% to 87\%). Control of uveitis was observed in 18 of 23 children (78\%) treated with adalimumab, and in two of five children (40\%) treated with infliximab. In all children treated with etanercept, no sufficient inflammatory control was found. Systemic corticosteroids could be discontinued in 71\% (12/17 children) and topical corticosteroids in 55\% (17/31) of the patients. Treatment-related side effects were found in nine children (29\%, rate: 0.10/patient-year). Our data show that adalimumab and infliximab have beneficial effects in the therapy of severe ANA-associated anterior uveitis in children. [\hyperlink{Advair HFA}{PMID: 24457365}, Deshka Doycheva et al., 2014]

\hypertarget{pmid_33964188}{T}his study aimed to evaluate the efficacy and safety of high-frequency oscillation ventilation combined with volume guarantee (HFOV-VG) compared with the safety and efficacy of HFOV alone in infants with acute hypoxemic respiratory failure (AHRF) after congenital heart surgery. We retrospectively analyzed the clinical data of 44 infants who were ventilated for AHRF after congenital heart surgery between January 2020 and January 2021. HFOV alone was used in 23 of the 44 infants, whereas HFOV-VG was used in the other 21 infants. The average frequency tidal volume (VThf) of the HFOV-VG group was lower than that of the HFOV group, and the proportion of VThf exceeding the target range of infants in the HFOV-VG group was also lower (p < .01). In addition, the incidence of hypocapnia and hypercapnia in infants supported with HFOV-VG was significantly lower (p < .01). Furthermore, the duration of invasive ventilation and the median ventilator adjustment per hour in the HFOV-VG group was also lower than that in the HFOV group (p < .01). Compared with HFOV alone, HFOV-VG decreases the fluctuation of VThf and the incidence of hypercapnia and hypocapnia. Moreover, it reduces the workload of bedside medical staff. Further studies are needed to confirm the efficacy and safety of HFOV-VG as a routine respiratory support strategy for congenital heart disease during the perioperative period. [\hyperlink{Advair HFA}{PMID: 33964188}, Yi-Rong Zheng et al., 2021]

\hypertarget{pmid_24740981}{T}o summarize evidence regarding the effectiveness of anti-tumor necrosis factor α (anti-TNFα) treatments in childhood autoimmune chronic uveitis (ACU), refractory to previous disease-modifying antirheumatic drugs (DMARDs). A systematic search between January 2000 and October 2012 was conducted using EMBase, Ovid Medline, Evidence-Based Medicine (EBM) Reviews: American College of Physicians Journal Club, Cochrane libraries, and EBM Reviews. Studies investigating the efficacy of anti-TNFα therapy, in children ages ≤16 years, as the first treatment with a biologic agent for ACU, refractory to topical and/or systemic steroid therapy and at least 1 DMARD, were eligible for inclusion. The primary outcome measure was the improvement of intraocular inflammation, as defined by the Standardization of Uveitis Nomenclature Working Group criteria. We determined a combined estimate of the proportion of children responding to anti-TNFα treatment, including etanercept (ETA), infliximab (INF), or adalimumab (ADA). We initially identified 989 articles, of which 148 were potentially eligible. In total, 22 retrospective chart reviews and 1 randomized clinical trial were deemed eligible, thus including 229 children (ADA: n = 31, ETA: n = 54, and INF: n = 144). On pooled analysis of observational studies, the proportion of responding children was 87\% (95\% confidence interval [95\% CI] 75-98\%) for ADA, 72\% (95\% CI 64-79\%) for INF, and 33\% (95\% CI 19-47\%) for ETA. There was no difference in the proportion of responders between ADA and INF (χ(2) = 3.06, P = 0.08), although both showed superior efficacy compared with ETA (ADA versus ETA: χ(2) = 20.9, P < 0.001 and INF versus ETA: χ(2) = 20.9, P < 0.001). Although randomized controlled trials are needed, the available evidence suggests that INF and ADA provide proven similar benefits in the treatment of childhood ACU, and they are both superior to ETA. [\hyperlink{Advair HFA}{PMID: 24740981}, Gabriele Simonini et al., 2014]

\hypertarget{pmid_17033526}{C}hronic hepatitis C virus (HCV) infection in children is a problem affecting thousands of children worldwide. Although standard interferon (INF) has better efficacy in pediatric patients than in adults, results in children with genotype 1 are poor; response rates to combination treatment with standard INF and ribavirin are better but the treatment requires thrice-weekly injections. The improved antiviral efficacy of weekly pegylated interferons relative to standard interferons in adults with chronic HCV infection suggests that pegylated interferons may also improve antiviral efficacy in children. We therefore investigated the pharmacokinetics, efficacy and safety of peginterferon alpha2a (pegINF-alpha2a) (40 kd) in 14 children ages 2 to 8 years with chronic hepatitis C (13 genotype 1, 1 non-1 genotype). Drug dose was calculated from each patient's body surface area (BSA) according to the formula BSA (m2)/(1.73 m2) x 180 microg, and patients were administered once-weekly subcutaneous injections for 48 weeks. Viral load and pharmacokinetic parameters were determined from blood drawn throughout the study and during follow-up. At week 24, the mean trough concentration was about 20\% below values obtained from adults treated with pegINF-alpha2a, and the area under the curve from 0 to 168 hours was about 20\% above adult values, suggesting that drug doses calculated from BSA achieved therapeutically adequate concentrations. Six of 14 patients (43\%), all infected with genotype 1, achieved a sustained virological response. Adverse events were those commonly associated with INF-based treatment, and none was deemed serious. In conclusion, our findings provide a basis for larger studies evaluating the efficacy and safety of pegINF-alpha2a as monotherapy as well as in combination with ribavirin in pediatric patients with chronic hepatitis C. [\hyperlink{Advair HFA}{PMID: 17033526}, Kathleen B Schwarz et al., 2006] (1) There is a far lower seroprevalence of hepatitis C virus (HCV) infection in children and adolescents (0.2\% to 0.4\%) than in adults. In childhood, the principal route of infection is mother-child transmission during pregnancy, while in adolescence transmission is mainly through certain at-risk behaviour (piercing, tattooing and drug injection). In adults with HCV infection, the standard treatment is a combination of peginterferon alfa and ribavirin. (2) 125 children aged 3 to 16 years were treated for 48 weeks in two non comparative trials. HCV RNA was undetectable in plasma in 46\% of children six months after treatment cessation (36\% for genotype 1 infection, 81\% for other genotypes), a proportion similar to that generally seen in adults. It is not known whether the interferon alfa-2b + ribavirin combination slows the progression of histological lesions or prevents clinical complications of HCV infection. (3) Psychological disorders, particularly depression and suicidal tendencies, are the main adverse effects of treatment, especially in children. Growth retardation can also occur, mainly due to gastrointestinal disorders linked to interferon alfa-2b (loss of appetite, nausea and vomiting, diarrhoea). Catch-up growth appears to occur during the six months after treatment cessation. (4) The combination of interferon alfa-2b and ribavirin appears to have similar virological efficacy in children to that seen in adults. Adverse effects, especially those of a psychological nature, remain frequent. [\hyperlink{Advair HFA}{PMID: 17033526}, Interferon alfa-2b and ribavirin: new indication. In children: more risks than in adults., 2007]

\hypertarget{pmid_33377568}{H}igh-flow is increasingly used in children with acute hypoxaemic respiratory failure (AHRF), despite limited evidence. The primary feasibility endpoint for this pilot-study was the proportion of treatment failure, secondary outcomes being intensive care unit (ICU) admissions and proportion of patients requiring escalation of care. We measured duration of hospital stay, duration of oxygen therapy and rates of ICU admission. An open-labelled randomised controlled trial feasibility design was used in two tertiary children's hospitals in the emergency department and general wards. Children aged 0-16 years with AHRF were randomised (1:1) to either high-flow or standard-oxygen. Children on standard-oxygen received rescue high-flow in general wards if failure criteria were met. Of 563 randomised, 283 received high-flow and 280 standard-oxygen with no adverse events. The proportion of children who failed treatment and receiving escalation of care was 11.7\% (32/283 children) on high-flow and 18.1\% (50/280 infants) on standard-oxygen (odds ratio 0.68, 95\% confidence interval 0.38-1.00). In children with obstructive airway disease, 9.7\% on high-flow and 17.4\% on standard-oxygen required escalation (risk-difference -7.7\% percentage points; 95\% confidence interval -14.3, -1.1); in children with non-obstructive disease no difference was observed. Neither difference in ICU admissions nor any difference in length of hospital stay was observed. Sixty percent of children who failed standard-oxygen responded to rescue high-flow. High-flow outside ICU appears to be feasible in children with AHRF and the required proportion of escalation was lower compared to standard-oxygen. The trial design can be applied in a future large randomised controlled trial. [\hyperlink{Advair HFA}{PMID: 33377568}, Donna Franklin et al., 2021]

\hypertarget{pmid_515801}{C}linical studies in the treatment of 54 children suffering from DHF with a combination of dipyridamole and ASA as an adjuvant of our standard therapy consisted of fluid, electrolytes, blood, plasma and plasma expanders were evaluated. Heparin was administered in cases of DIC. It appeared that dipyridamole and ASA did not change the mortality significantly, but it prevented the progress of the severity of the disease from grade I and II to grade III and IV. [\hyperlink{Advair HFA}{PMID: 515801}, L K Kho et al., 1979]

\hypertarget{pmid_17095339}{T}o evaluate the efficacy and tolerability of fluticasone propionate (FP) hydrofluoroalkane (HFA) in children age 1 to < 4 years with asthma. Children were assigned (2:1) to receive FP HFA 88 mug (n = 239) or placebo HFA (n = 120) twice daily through a metered-dose inhaler with a valved holding chamber and attached facemask for 12 weeks. The primary efficacy measure was mean percent change from baseline to endpoint in 24-hour daily (composite of daytime and nighttime) asthma symptom scores. The FP-treated children had significantly greater (P < or = .05) reductions in 24-hour daily asthma symptom scores (-53.9\% vs -44.1\%) and nighttime symptom scores over the entire treatment period compared with the placebo group. Daytime asthma symptom scores and albuterol use were slightly more decreased with FP than with placebo; however, the differences were not statistically significant. Increases in the percentage of symptom-free days were comparable. The percentage of patients who experienced at least 1 adverse event was similar in the 2 groups. Baseline median urinary cortisol excretion values were comparable between the groups, and there was little change from baseline at endpoint. FP plasma concentrations demonstrated that systemic exposure was low. FP HFA 88 mug twice daily was effective and well tolerated in pre-school-age children with asthma. [\hyperlink{Advair HFA}{PMID: 17095339}, Paul Y Qaqundah et al., 2006]

\hypertarget{pmid_36158836}{T}o compare the safety and efficacy of completely zero-fluoroscopy radiofrequency ablation (RFA) with that of conventional RFA guided by three-dimensional mapping in Chinese children with paroxysmal supraventricular tachycardia (PSVT). The study had a single-center observational design and included 46 children aged 6-14 years who underwent RFA for PSVT at the Second Hospital of Hebei Medical University between March 2019 and September 2021. The children were divided according to whether they underwent zero-fluoroscopy RFA (zero-fluoroscopy group,  The children had a median age of 12 years (interquartile range 10, 13), 47.8\% (22/46) were boys, and 52.2\% (24/46) were girls. The mean body weight was 48.75 ± 15.26 kg. There was no significant between-group difference in the baseline data ( Zero-fluoroscopy catheter ablation can completely avoid fluoroscopy exposure in children without affecting the safety and efficacy of RFA. [\hyperlink{Advair HFA}{PMID: 36158836}, Xiaoran Cui et al., 2022]

\hypertarget{pmid_27678432}{T}he efficacy and safety of atorvastatin in children/adolescents aged 10-17 years with heterozygous familial hypercholesterolemia (HeFH) have been demonstrated in trials of up to 1 year in duration. However, the efficacy/safety of >1 year use of atorvastatin in children/adolescents with HeFH, including children from 6 years of age, has not been assessed. To characterize the efficacy and safety of atorvastatin over 3 years and to assess the impact on growth and development in children aged 6-15 years with HeFH. A total of 272 subjects aged 6-15 years with HeFH and low-density lipoprotein cholesterol (LDL-C) ≥4.0 mmol/L (154 mg/dL) were enrolled in a 3-year study (NCT00827606). Subjects were initiated on atorvastatin (5 mg or 10 mg) with doses increased to up to 80 mg based on LDL-C levels. Mean percentage reductions from baseline in LDL-C at 36 months/early termination were 43.8\% for subjects at Tanner stage (TS) 1 and 39.9\% for TS ≥2. There was no evidence of variations in the lipid-lowering efficacy of atorvastatin between the TS groups analyzed (1 vs ≥2) or in subjects aged <10 vs ≥10 years, and the treatment had no adverse effect on growth or maturation. Atorvastatin had a favorable safety and tolerability profile, and only 6 (2.2\%) subjects discontinued because of adverse events. Atorvastatin over 3 years was efficacious, had no impact on growth/maturation, and was well tolerated in children and adolescents with HeFH aged 6-15 years. [\hyperlink{Advair HFA}{PMID: 27678432}, Gisle Langslet et al., ]

\hypertarget{pmid_12960646}{A}cute hepatitis A superimposed on chronic liver disease has been associated with a more severe course of disease and development of fulminant hepatitis. The aim of this study was to evaluate the immunogenicity and safety of an inactivated hepatitis A virus vaccine in children with chronic liver disease. This was an open, prospective, and controlled trial with 89 anti-HAV negative children between 1 and 16 years of age studied at a pediatric liver disease and transplantation referral center. Inactivated HAV vaccine (Havrix), from GlaxoSmithKline Biologicals containing 720 Elisa units of alum-adsorbed hepatitis A antigen per 0.5 ml dose was used. Thirty-four pediatric patients with chronic liver disease (mean age: 7.0 +/- 4.86 years) and 55 healthy controls (mean age: 4.8 +/- 2.7 years) received two doses of Havrix vaccine in months zero and six. Seroconversion and anti-HAV titers expressed as geometric mean titers (GMT) in mIU/ml were measured at months one and seven, by a modified Hepatitis A virus antibodies (HAVAB) assay. Seroconversion rates at four weeks after primary immunization were 76\% and 94\% and the GMT 107.77 and 160.77 mIU/ml in the patient and control groups, respectively. One month after second dose the seroconversion rates were 97\% and 100\% in the groups with GMT of 812.40 and 2,344.90 mIU/ml. Both doses were well tolerated with no significant adverse events observed. Local injection-site symptoms were the most common reactions reported in both groups. Although GMTs were significantly lower in children with chronic liver disease compared to healthy controls, the overall seroconversion rates were not different. Hepatitis A virus vaccine was safe, well-tolerated, and immunogenic in children with chronic liver disease. [\hyperlink{Advair HFA}{PMID: 12960646}, Cristina Targa Ferreira et al., 2003]

\hypertarget{pmid_16250032}{C}hronic hepatitis C virus (HCV) infection is usually asymptomatic in children, but significant liver disease may occur. We evaluated the efficacy, safety, and pharmacokinetics of interferon alfa-2b and ribavirin in children with chronic HCV. We determined the optimal ribavirin dose in an initial cohort of a phase 1 study and then subsequently used it, in combination with interferon alfa-2b, in a second cohort of this study and a phase 3 trial. The primary efficacy endpoint in all studies was sustained virological response, defined by undetectable serum HCV RNA 24 weeks after completion of therapy. All efficacy and safety analyses were performed on the intent-to-treat population. Children receiving interferon alfa-2b plus ribavirin 15 mg/kg/d in the phase 1 study had the maximum reduction in serum HCV RNA at treatment weeks 4 and 12 with an acceptable safety profile. This ribavirin dose was selected as optimal and used in all subsequent studies. In all, 46\% (54/118) of optimally treated children achieved sustained virological response. Sustained virological response was significantly higher in children with HCV genotype 2/3 (84\%) than in those with HCV genotype 1 (36\%). Adverse events led to dose modification in 37 (31\%) and discontinuation in 8 (7\%). Multiple-dose interferon alfa-2b and ribavirin peak and trough concentrations and area-under-the-curve were similar between children and adults. In conclusion, interferon alfa-2b in combination with ribavirin is effective and safe in children with chronic hepatitis C virus. [\hyperlink{Advair HFA}{PMID: 16250032}, Regino P González-Peralta et al., 2005]

\hypertarget{pmid_37324759}{R}adiofrequency ablation (RFA) is the standard method of treatment for tachyarrhythmias in school children, and it leads to complete recovery in children without structural heart disease. However, RFA in young children is limited by the risk of complications and unstudied remote effects of radiofrequency lesions. To present the experience of RFA of arrhythmias and the results of follow-up of younger children. RFA procedures ( The overall effectiveness of RFA, considering the repeated procedures performed due to the primary ineffectiveness and recurrencies, was 94.7\%. There was no mortality associated with RFA in patients, including young patients. All cases of "major" complications are associated with RFA of the left-sided accessory pathway and tachycardia foci and are represented by the mitral valve damage in three patients (1.4\%). Tachycardia and preexcitation recurred in 44 (21\%) patients. There was a correlation between recurrences and parameters of RFA (odds ratio 0.894; 95\% confidence interval: 0.804-0.994;  The use of the minimum effective parameters of RFA in children reduces the risk of complications, but increases arrhythmia recurrence rate. [\hyperlink{Advair HFA}{PMID: 37324759}, Liliya I Svintsova et al., 2023]

\hypertarget{pmid_30586314}{A}pproximately 2.1 million of the estimated 36 million infected with HIV are children or adolescents. International guidelines for HIV-1 Infection suggest starting antiretrovirals (ARV) at the moment of diagnosis. Many factors limit the optimization of antiretroviral therapy in children and adolescents: lack of pediatric formulations, poor adherence, metabolic and pharmacokinetic changes associated withnormal child development and puberty. Areas covered: Three integrase inhibitors are approved by the US Food and Drug Administration and by European Medical Agency for children and adolescents with HIV-1 infection. Raltegravir is approved for children aged 4 weeks to 18 years, while dolutegravir and elvitegravir co-formulated with cobicistat, emtricitabine, and tenofovir alafenamide (E/C/FTC/TAF) are approved for children from 6 years of age. This article evaluates E/C/FTC/TAF as a treatment option. Expert opinion: E/C/FTC/TAF was well tolerated, and the antiretroviral activity and tolerability data of this combination support the use in children and adolescents. However, the studies regarding E/C/FTC/TAF in children and adolescents are scant. Consequently, additional studies investigating its safety and efficacy in children are paramount. [\hyperlink{Advair HFA}{PMID: 30586314}, Vania Giacomet et al., 2019]

\section*{Asmanex HFA}
\subsection*{Result}
\subsubsection*{Answer}

Unknown

\subsubsection*{{Explanation}}
\hypertarget{Asmanex HFA}
A review of the available abstracts reveals the following:

- Only one abstract specifically mentions "mometasone furoate-formoterol (MF/F)" and "MF monotherapy" in children with asthma, using a metered-dose inhaler (MDI) [\hyperlink{pmid_32022483}{PMID: 32022483}, Cindy L J Weinstein et al., 2020]. This study was a phase 3, multicenter, randomized controlled trial in children aged 5 to 11 years. The study evaluated the efficacy and long-term safety of MF/F and MF monotherapy. The results showed that both treatments were well tolerated, with approximately 50\% of participants experiencing one or more treatment-emergent adverse events, and fewer adverse events in the MF/F group. The abstract concludes that the addition of formoterol to mometasone furoate was well tolerated and provided significant, rapid, and sustained improvement in lung function compared with MF alone in children 5 to 11 years of age.

- However, this study used a "metered-dose inhaler" (MDI) and does not specify the use of "Asmanex HFA" (the hydrofluoroalkane-propelled inhaler formulation of mometasone furoate). The product "Asmanex HFA" is a specific brand and formulation, and the abstract does not confirm that this was the product studied.

- No other abstracts mention "Asmanex HFA" or provide targeted safety data for this specific product in children of any age group.

- There are abstracts about other inhaled corticosteroids (such as fluticasone HFA, beclomethasone HFA) and their safety in children, but per your instructions, safety data for other products cannot be extrapolated to Asmanex HFA.

- Therefore, based strictly on the abstracts available, there is no targeted study that affirms the safety or unsafety of Asmanex HFA in children of any age group. The safety of Asmanex HFA in children is therefore unknown.

\subsection*{Abstracts}
\hypertarget{pmid_23798623}{D}ata on the efficacy of hydroxyurea (HU) in Indian children with sickle cell anaemia (SCA) is limited. Hence, we have evaluated the efficacy of fixed low dose HU in Indian children. The study cohort consisted of 144 children (<18 years of age) with SCA having severe manifestations (≥ 3 episodes of vasocclusive crisis or blood transfusions, or having ≥ 1 episode of acute chest syndrome or cerebrovascular stroke or sequestration crisis) who were started on fixed low dose HU (10 mg/kg/day). They were followed up for two years and monitored for the hematological and clinical efficacy and safety. There was significant increase in the fetal hemoglobin level (HbF\%), total hemoglobin and mean corpuscular volume. Vasoocclusive crises, blood transfusions, acute chest syndrome, sequestration crises and hospitalizations decreased significantly. Baseline HbF\% had significant positive correlation with HbF\% at 24 months. There was significant negative correlation between baseline HbF\% and change in HbF\% from baseline to 24 months. No significant correlation was found between HbF\% at baseline and clinical event rates per year after HU. No major adverse events occurred during the study period. Fixed low dose HU is effective and safe in Indian children with SCA. [\hyperlink{Asmanex HFA}{PMID: 23798623}, Dipti L Jain et al., 2013]

\hypertarget{pmid_18534230}{I}n study 1, to compare the effect on growth in healthy infants of a new amino acid-based formula (AAF) and a control extensively hydrolyzed formula (EHF), with both docosahexaenoic acid (DHA) and arachidonic acid (ARA) at levels similar to those in human milk worldwide. In study 2, to evaluate the hypoallergenicity of this new AAF in infants and children with confirmed cow's milk allergy (CMA). In study 1, a total of 165 healthy, full-term, formula-fed infants randomly received the new AAF or control formula. Anthropometric measurements, tolerance, and adverse events were recorded throughout the study. Plasma amino acid profiles were evaluated in a subset of the infants. In study 2, the hypoallergenicity of the new AAF was evaluated in 32 infants and children using a double-blind, placebo-controlled food challenge; an open challenge; and a 7-day feeding. In study 1, overall growth, tolerance, and safety outcomes were similar in both groups. In study 2, 29 of the 32 subjects completed both challenges; no allergic reaction was seen in any of the 32 subjects. The new AAF with DHA and ARA at levels similar to those in human milk worldwide is hypoallergenic. It also is safe and supports growth in healthy, term infants. [\hyperlink{Asmanex HFA}{PMID: 18534230}, Wesley Burks et al., 2008]

\hypertarget{pmid_17095339}{T}o evaluate the efficacy and tolerability of fluticasone propionate (FP) hydrofluoroalkane (HFA) in children age 1 to < 4 years with asthma. Children were assigned (2:1) to receive FP HFA 88 mug (n = 239) or placebo HFA (n = 120) twice daily through a metered-dose inhaler with a valved holding chamber and attached facemask for 12 weeks. The primary efficacy measure was mean percent change from baseline to endpoint in 24-hour daily (composite of daytime and nighttime) asthma symptom scores. The FP-treated children had significantly greater (P < or = .05) reductions in 24-hour daily asthma symptom scores (-53.9\% vs -44.1\%) and nighttime symptom scores over the entire treatment period compared with the placebo group. Daytime asthma symptom scores and albuterol use were slightly more decreased with FP than with placebo; however, the differences were not statistically significant. Increases in the percentage of symptom-free days were comparable. The percentage of patients who experienced at least 1 adverse event was similar in the 2 groups. Baseline median urinary cortisol excretion values were comparable between the groups, and there was little change from baseline at endpoint. FP plasma concentrations demonstrated that systemic exposure was low. FP HFA 88 mug twice daily was effective and well tolerated in pre-school-age children with asthma. [\hyperlink{Asmanex HFA}{PMID: 17095339}, Paul Y Qaqundah et al., 2006]

\hypertarget{pmid_3134034}{A}n open multicentric pediatric clinical trial has been performed to evaluate the safety and efficacy of astemizole (HISMANAL suspension) at a 1 ml (2 mg) dosage per 10 kg body weight once daily in children with chronic urticaria. A good and very good effect has been observed in 63.2\% of the children at day 7, in 79.2\% after 2 weeks and in 88.9\% at the end of the treatment (4 weeks). A marked improvement of the symptoms was reported in 63.8\% of children at day 7. There was no decrease of efficacy at the end of the treatment. Tolerance was good in 87.\% of the cases. Astemizole seems to be a useful treatment for chronic urticaria in pediatrics. [\hyperlink{Asmanex HFA}{PMID: 3134034}, M T Guinnepain et al., 1987]

\hypertarget{pmid_17075274}{H}ydrofluoroalkane-134a (HFA) has been shown to be a safe replacement for chlorofluorocarbons (CFCs) as a pharmaceutical propellant, with the advantage that it has no ozone-depleting potential. This is the first report of the pharmacokinetics of beclomethasone dipropionate (BDP) delivered from a pressurized solution formulation using an HFA propellant system (HFA-BDP) in Japanese children with bronchial asthma. Plasma concentrations of beclomethasone 17-monopropionate (17-BMP),a major metabolite of BDP, following an inhaled dose of HFA-BDP (200 microg as four inhalations from 50 microg/actuation) in five Japanese children with bronchial asthma were quantified and analyzed by a non-compartmental analysis to obtain pharmacokinetic parameters. The area under the concentration-time curve from time zero to the last quantifiable time (AUC(0-t)) was 1659 +/- 850 pg x h/mL (arithmetic mean +/- standard deviation (SD)), the maximum concentration observed (C(max)) was 825 +/- 453 pg/mL and the apparent elimination half-life (t(1/2)) was 2.1 +/- 0.7 hours. The time to reach Cmax Tmax was 0.5 hours in all patients. No special relationship was observed between these parameters and age or body weight. These parameters were compared with the previously reported parameters of American children with bronchial asthma. The Japanese/American ratio of the geometric means of each parameter was 1.36 for AUC(0-t), 1.04 for Cmax and 1.4 for t(1/2). The median of Tmax was 0.5 hours in American patients as well as Japanese patients. The pharmacokinetics of HFA-BDP in Japanese children with bronchial asthma are reported for the first time and a similarity to those in American children is suggested. [\hyperlink{Asmanex HFA}{PMID: 17075274}, Takahide Teramoto et al., 2006]

\hypertarget{pmid_15360067}{T}he incidence of asthma in children under age 5 is higher than in any other segment of the population. Current NAEPP guidelines recommend treatment of some asthmatics in this age group with the combination of an inhaled corticosteroid and a long-acting beta2-agonist even though this practice has never been studied with children younger than 4. This retrospective study analyzes the efficacy and safety of a combination of fluticasone propionate (FP) and salmeterol (SA) in children under 5. Fifty patients who started using FP/SA before the age of 60 months were included in the analysis. To determine efficacy, we tracked the change in emergency room visits, hospitalizations, and the frequency of wheezing as a result of treatment. Emergency room visits were reduced from 78 to 5 (p<0.001), hospitalizations were reduced from 43 to 2 (p<0.001) and frequency of wheezing, daily, weekly, or monthly, was also reduced significantly (p<0.003). In terms of safety, there was only a 3.4\% reduction in height percentile (p=0.37). Combination therapy is highly efficacious and safe for asthmatics under the age of 5. A well-designed prospective study is necessary to further evaluate the benefits and risks of this treatment method. [\hyperlink{Asmanex HFA}{PMID: 15360067}, Sudhir Sekhsaria et al., 2004]

\hypertarget{pmid_29490769}{T}he safety of a novel intranasal formulation of azelastine hydrochloride (AZE) and fluticasone propionate (FP) has been established in adults and adolescents with allergic rhinitis but not in children <12 years old. To evaluate the safety and tolerability of an intranasal formulation of AZE and FP in children ages 4-11 years with allergic rhinitis. The study was a randomized, 3-month, parallel-group, open-label design. Qualified patients were randomized in a 3:1 ratio to AZE/FP (n = 304) or fluticasone propionate (FP) (n = 101), one spray per nostril twice daily, and to one of three age groups: ≥4 to <6 years, ≥6 to <9 years, and ≥9 to <12 years. Safety was assessed by child- or caregiver-reported adverse events, nasal examinations, vital signs, and laboratory assessments. The incidence of treatment-related adverse events (TRAEs) was low in both the AZE/FP (16\%) and FP-only (12\%) groups after 90 days' continuous use. Epistaxis was the most frequently reported TRAE in both groups (AZE/FP, 9\%; FP, 9\%), followed by headache (AZE/FP, 3\%; FP, 1\%). All other TRAEs in the AZE/FP group were reported by ≤1\% of the children. The majority of TRAEs were of mild intensity and resolved spontaneously. Results of nasal examinations showed an improvement over time in both groups, with no cases of mucosal ulceration or nasal septal perforation. There were no unusual or unexpected changes in laboratory parameters or vital signs. The intranasal formulation of AZE and FP was safe and well tolerated after 3 months' continuous use in children with allergic rhinitis.The study was registered on <ext-link xmlns:xlink="http://www.w3.org/1999/xlink" ext-link-type="uri" xlink:href="http://ClinicalTrials.gov">ClinicalTrials.gov</ext-link> (NCT01794741). [\hyperlink{Asmanex HFA}{PMID: 29490769}, William Berger et al., 2018]

\hypertarget{pmid_18582451}{E}stimation of free polyunsaturated fatty acids (PUFAs) in blood and evaluation of behavior of autistic children before and after taking fish oil (Efalex) were performed. 30 autistic children (18 males and 12 females) aged 3-11 years and 30 healthy children as control group were included in this study. Tandem mass spectrometry and CARS were used to estimate the free PUFAs from dried blood spot and to evaluate the autistic behavior respectively. Before taking Efalex, linolenic acid showed a significant reduction (71\%), followed by docosahexaenoic acid (65\%) and arachidonic acid (45\%), while linoleic acid was the least affected PUFA (32\%). After taking Efalex, 66\% of autistic children showed clinical and biochemical improvement, linolenic acid and docosahexaenoic acid showed the highest levels after Efalex supplementation. PUFA supplementation may play an important role in ameliorating the autistic behavior. [\hyperlink{Asmanex HFA}{PMID: 18582451}, Nagwa A Meguid et al., 2008]

\hypertarget{pmid_11343044}{T}o determine the hypoallergenicity and efficacy of a pediatric amino acid-based formula (AAF), EleCare, for children with cow's milk allergy (CMA) and multiple food allergies (MFA). Hypoallergenicity was determined by performing blinded oral food challenges in 31 consecutive children with documented CMA. Growth, tolerance, and biochemical response were evaluated during a nonrandomized feeding study with each child serving as his or her own control. Thirty-one children (median age, 23.3 months; range, 6 months to 17.5 years) were recruited; 29 had MFA, 17 had acute reactions and cow's milk-specific IgE antibody, and 14 had allergic eosinophilic gastroenteritis. At study entry, 23 were receiving another AAF; 13 had not tolerated extensively hydrolyzed formula. Eighteen subjects with allergic eosinophilic gastroenteritis and/or MFA were followed up while receiving AAF for a median of 21 months (range, 7 to 40 months), with biochemical analysis performed at 4 months. No statistically significant differences were observed in the change in weight or height National Center for Health Statistics z scores from entry; the percent of expected growth exceeded 90\%. There was a small decline in percent eosinophils and increase in hemoglobin, hematocrit, and serum ferritin level (P < .05). Except for small increases in plasma leucine and valine levels (P < or = .006), the remaining biochemical markers were unchanged. The AAF was hypoallergenic and effective in maintaining normal growth for children with CMA and MFA. [\hyperlink{Asmanex HFA}{PMID: 11343044}, S H Sicherer et al., 2001]

\hypertarget{pmid_25624403}{A}rachidonic acid (ARA), an omega-6 fatty acid, is a potent schistosomicide that displayed significant and safe therapeutic effects in Schistosoma mansoni-infected schoolchildren in S. mansoni low-prevalence regions. We here report on ARA efficacy and safety in treatment of schoolchildren in S. mansoni high-endemicity areas of Kafr El Sheikh, Egypt. The study was registered with ClinicalTrials.gov (NCT02144389). In total, 268 schoolchildren with light, moderate, or heavy S. mansoni infection were assigned to three study arms of 87, 91, and 90 children and received a single dose of 40 mg/kg praziquantel (PZQ), ARA (10 mg/kg per day for 15 days), or PZQ combined with ARA, respectively. The children were examined before and after treatment for stool parasite egg counts and blood biochemical, hematological, and immunological parameters. ARA, like PZQ, induced moderate cure rates (50\% and 60\%, respectively) in schoolchildren with light infection and modest cure rates (21\% and 20\%, respectively) in schoolchildren with high infection. PZQ and ARA combined elicited 83\% and 78\% cure rates in children with light and heavy infection, respectively. Biochemical and immunological profiles were either unchanged or ameliorated after ARA therapy. Combination of PZQ and ARA might be useful for treatment of children with schistosomiasis in high-endemicity regions. [\hyperlink{Asmanex HFA}{PMID: 25624403}, Rashida Barakat et al., 2015]

\hypertarget{pmid_33798062}{T}o investigate the safety and efficacy of remifentanil combined with dexmedetomidine in fast-track cardiac anesthesia (FTCA) for transthoracic device closure of atrial septal defect (ASD) in pediatric patients. A retrospective analysis was performed on 61 cases of children undergoing ASD closure through a small thoracic incision from January 2018 to January 2020. According to whether FTCA was administered, they were divided into group F (fast-track anesthesia, n = 31) and group R (routine anesthesia, n = 30). There was no significant difference in general preoperative data, perioperative hemodynamics, or postoperative pain scores between the 2 groups (P > .05). The postoperative sedation score of group F was higher than that of group R 1 and 4 hours after extubation. Meanwhile, duration of mechanical ventilation and length of postoperative intensive care unit (ICU) stay of group F were significantly shorter than those of group R (P < .05). No serious anesthesia-related complications occurred. Remifentanil combined with dexmedetomidine in FTCA for transthoracic device closure of ASD in pediatric patients is safe and effective, is worthy of clinical promotion, and can benefit more children. [\hyperlink{Asmanex HFA}{PMID: 33798062}, Ling-Shan Yu et al., 2021]

\hypertarget{pmid_22294512}{C}hildren with sickle cell anemia (SCA) often develop hyposthenuria and renal hyperfiltration at an early age, possibly contributing to the glomerular injury and renal insufficiency commonly seen later in life. The Phase III randomized double-blinded Clinical Trial of Hydroxyurea in Infants with SCA (BABY HUG) tested the hypothesis that hydroxyurea can prevent kidney dysfunction by reducing hyperfiltration. 193 infants with SCA (mean age 13.8 months) received hydroxyurea 20 mg/kg/day or placebo for 24 months. (99m) Tc diethylenetriaminepentaacetic acid (DTPA) clearance, serum creatinine, serum cystatin C, urinalysis, serum and urine osmolality after parent-supervised fluid deprivation, and renal ultrasonography were obtained at baseline and at exit to measure treatment effects on renal function. At exit children treated with hydroxyurea had significantly higher urine osmolality (mean 495 mOsm/kg H(2) O compared to 452 in the placebo group, P = 0.007) and a larger percentage of subjects taking hydroxyurea achieved urine osmolality >500 mOsm/kg H(2) O. Moreover, children treated with hydroxyurea had smaller renal volumes (P = 0.007). DTPA-derived glomerular filtration rate (GFR) was not significantly different between the two treatment groups, but was significantly higher than published norms. GFR estimated by the Chronic Kidney Disease in Children (CKiD) Schwartz formula was the best non-invasive method to estimate GFR in these children, as it was the closest to the DTPA-derived GFR. Treatment with hydroxyurea for 24 months did not influence GFR in young children with SCA. However, hydroxyurea was associated with better urine concentrating ability and less renal enlargement, suggesting some benefit to renal function. [\hyperlink{Asmanex HFA}{PMID: 22294512}, Ofelia Alvarez et al., 2012]

\hypertarget{pmid_9317198}{C}hildren with sickle cell anemia provide the best opportunity to assess the efficacy of hydroxyurea (HU) in preventing complications and progressive organ damage. The possibility of treating infants with sickle cell disease (SCD) to inhibit the development of organ dysfunction may be the most important future use of HU. The possibility even exists that instituting HU in the neonate may stop the fetal-to-adult globin chain switch and thus markedly change the clinical phenotype of SCD. Recent data suggest HU may also be especially beneficial in children not only by increasing hemoglobin F (HbF), but also by altering the adhesive receptors expressed on red blood cells and vascular endothelium, further increasing the possibility that vasculopathy can be prevented. Six pediatric trials that included small numbers of severely ill patients have been reported recently. All patients received relatively standard HU doses. All studies reported a significant improvement in HbF and mean corpuscular volume and a mild to marked increase in hemoglobin. The clinical response to HU in children with SCD seems to be consistent. The National Institutes of Health pediatric multicenter trial should help answer the question of short-term HU toxicity; however, questions remain concerning long-term risks, such as carcinogenesis, gametogenesis, marrow toxicity, growth retardation, and chromosomal damage. Long-term studies are needed to answer these questions. The future treatment of most children with SCD with HU alone or in combination with other agents looks promising, and long-term trials are warranted. [\hyperlink{Asmanex HFA}{PMID: 9317198}, E P Vichinsky et al., 1997]

\hypertarget{pmid_32417537}{A}sfotase alfa is an enzyme replacement therapy approved for treatment of patients with pediatric-onset hypophosphatasia (HPP), a rare, inherited, systemic disease causing impaired skeletal mineralization, short stature, and reduced physical function in children. The role of dual X-ray absorptiometry (DXA) in the assessment of children with HPP has been insufficiently explored. This post hoc analysis included pooled DXA data from 2 open-label, multicenter studies in 19 children with HPP. The study population was aged ≥5 to <18 years and had received asfotase alfa for ≤6.6 years at enrollment (male: 79\%; median age at enrollment: 10.4 y [range: 5.9-16.7]; treatment duration: 6.3 y [range: 0.1-6.6]. Baseline height Z-scores indicated short stature (median [min, max]: -1.26 [-6.6, 0]); mean [SD]: -2.30 [1.97]), thus requiring height adjustment of DXA Z-scores. At Baseline, few patients had height-adjusted bone mineral density (BMD [\hyperlink{Asmanex HFA}{PMID: 32417537}, Jill H Simmons et al., 2020] Perfluoroalkyl acids (PFAAs) and brominated flame retardants (BFRs) are widely used and present in human food. Due to the increased susceptibility to pollutants of the young children, we conducted a total diet study focusing on this population. Around 200 baby and common food composite samples, prepared "as consumed", have been analysed for PFAAS, hexabromocyclododecanes, polybrominated biphenyls, polybrominated diphenyl ethers and tetrabromobisphenol A. The dietary exposure of 705 children aged 1-36 months was assessed. PFAAS were detected only in one fish sample. Detection rates varied from 4 to 93\% for BFRs, depending on the congeners. Regarding the provisional health-based guidance values set by EFSA in 2018 for PFOA and PFOS at 0.8 and 1.8 ng kg bw [\hyperlink{Asmanex HFA}{PMID: 32417537}, Gilles Rivière et al., 2019] Allergic rhinitis (AR) is a common chronic condition in children and may impact a child's quality of life. Increasing treatment compliance may improve quality of life. An oral suspension of fexofenadine hydrochloride (HCl) has been developed to ease administration to children and may, therefore, improve treatment compliance. The purpose of this study was to assess the pharmacokinetic behavior, safety, and tolerability of a single dose of fexofenadine HCl oral suspension administered to children aged 2-5 years with allergic rhinitis. Children (aged 2-5 years) with AR were recruited in a multicenter, open-label, single-dose study. Fexofenadine HCl (30 mg) was administered as a 6-mg/mL suspension (5 mL). Plasma samples were collected up to 24 hours postdose. Adverse events (AEs); electrocardiograms (ECGs); vital signs; and clinical laboratory tests for hematology, blood chemistry, and urinalysis were analyzed to evaluate safety and tolerability. Fifty subjects completed the study. Mean maximum plasma concentration of fexofenadine was 224 ng/mL, and mean area under the plasma concentration curve was 898 ng . hour/mL. Treatment-emergent AEs were mild in intensity and reported in a total of seven subjects. No trends or clinically meaningful changes in mean ECG, vital sign, or clinical laboratory test data occurred during the study. In children aged 2-5 years, the exposure after a 30-mg dose of fexofenadine HCl suspension was similar to the exposures previously seen after a 30- and 60-mg dose of fexofenadine HCl in children aged 6-11 years and in adults, respectively. The suspension was also well tolerated. [\hyperlink{Asmanex HFA}{PMID: 32417537}, Nathan Segall et al., ]

\hypertarget{pmid_9256830}{T}o observe the safety and efficacy of hydroxyurea (HU), a drug that stimulates fetal hemoglobin (Hb F) production, in previously severely ill children with sickle cell disease. HU was given in an uncontrolled study to 35 children with sickle cell disease, aged from 3 to 20 years, suffering from frequent painful crises. Mean duration of treatment was 32 months (range: 12-59 months). HU induced an increase in Hb F levels in all children out one; this increase was maximal after 9 months of treatment, was largely sustained thereafter, and was related to HU dose and inversely to patients' age. We also noted an apparent reduction in crisis, which occurred principally after 3 months of therapy and did not seem strictly correlated with the rise in Hb F level. No serious hematopoietic complication was observed. Growth curves and sexual development were not modified. Our data support the efficacy of HU in reducing painful events in children with sickle cell disease. Short- and middle-term tolerances are good. Thus, we think that HU can be given to children affected by frequent and severe painful crises. We recommend, however, very cautious use of this drug, because its long-term effects in children are still unknown. [\hyperlink{Asmanex HFA}{PMID: 9256830}, M de Montalembert et al., ]

\hypertarget{pmid_16114180}{O}ur aim was to assess the efficacy and safety of hydroxyurea (HU) in children with severe forms of sickle cell anemia followed in a Portuguese hospital. We carried out an open-label uncontrolled prospective study, which included children with severe forms of sickle cell anemia. Hydroxyurea was started at 15 mg/kg/day and increased to a maximum dose of 25 mg/kg/day. Patients were monitored to assess compliance, clinical and hematological response and toxicity. Nine children and adolescents, five girls and four boys, with a median age of 13 years (range 8 to 16) were enrolled in the study during a period of 24 months. All patients completed at least 15 months of therapy. Hb F was significantly increased, from a mean of 7.0 +/- 3.9\% to 13.7 +/- 5.3\% (p = 0.028). Clinically, all patients responded significantly with a reduction of 80\% in the number of vaso-occlusive crises (VOC), 69\% in hospital admissions, 76\% in hospitalization days and 67\% in transfusion requirements, without significant toxicity. We concluded that, in our population, HU proved to be effective in increasing Hb F levels, and in decreasing hospitalizations for VOC and transfusion requirements with no major side effects. Long-term clinical follow-up is important to certify benefit maintenance. [\hyperlink{Asmanex HFA}{PMID: 16114180}, Lígia Barbosa Braga et al., 2005]

\hypertarget{pmid_2970744}{R}eaferon, the analog of human alpha 2-interferon obtained by gene engineering techniques, was studied with a view to its use for the prevention of hepatitis A. The study involved children of preschool age in Tashkent. In a strictly controlled trial children aged 2-6 years received the preparation orally in a dose of 1 X 10(6) I. U. or the diluent alone used as placebo. The preparation was administered to 1,100 children and the placebo to 1,078 children. The preparation and placebo were administered twice a week for two months. On the whole, during that period hepatitis A morbidity in both test and control groups of children was the same (5.1\% and 4.9\% respectively), but among children of nursery age receiving Reaferon the incidence of hepatitis A and acute respiratory viral infections was lower than among those receiving the placebo, though this difference was statistically significant only for cases of acute respiratory infections. [\hyperlink{Asmanex HFA}{PMID: 2970744}, A K Iuldashev et al., 1988]

\hypertarget{pmid_33377568}{H}igh-flow is increasingly used in children with acute hypoxaemic respiratory failure (AHRF), despite limited evidence. The primary feasibility endpoint for this pilot-study was the proportion of treatment failure, secondary outcomes being intensive care unit (ICU) admissions and proportion of patients requiring escalation of care. We measured duration of hospital stay, duration of oxygen therapy and rates of ICU admission. An open-labelled randomised controlled trial feasibility design was used in two tertiary children's hospitals in the emergency department and general wards. Children aged 0-16 years with AHRF were randomised (1:1) to either high-flow or standard-oxygen. Children on standard-oxygen received rescue high-flow in general wards if failure criteria were met. Of 563 randomised, 283 received high-flow and 280 standard-oxygen with no adverse events. The proportion of children who failed treatment and receiving escalation of care was 11.7\% (32/283 children) on high-flow and 18.1\% (50/280 infants) on standard-oxygen (odds ratio 0.68, 95\% confidence interval 0.38-1.00). In children with obstructive airway disease, 9.7\% on high-flow and 17.4\% on standard-oxygen required escalation (risk-difference -7.7\% percentage points; 95\% confidence interval -14.3, -1.1); in children with non-obstructive disease no difference was observed. Neither difference in ICU admissions nor any difference in length of hospital stay was observed. Sixty percent of children who failed standard-oxygen responded to rescue high-flow. High-flow outside ICU appears to be feasible in children with AHRF and the required proportion of escalation was lower compared to standard-oxygen. The trial design can be applied in a future large randomised controlled trial. [\hyperlink{Asmanex HFA}{PMID: 33377568}, Donna Franklin et al., 2021]

\hypertarget{pmid_9243686}{W}e studied the efficiency of a heat and moisture exchanging filter (HMEF; Pall BB25) as a means of compensating for the heat and moisture loss during anaesthesia in young children using cold and dry gas supplied from open circuits. Forty ASA I children (mean age: 48 months +/- 20; mean weight: 16 +/- 3.5 kg) were randomized into two groups: Group I without HMEF/Group II with HMEF. The two groups did not show any significant differences for morphometric data or ventilation parameters. Relative humidity and temperature measurements in anaesthetic gases were taken using a combined temperature/humidity probe introduced into the circuit. Absolute humidity in the circuit was calculated from these measurements. In Group II, a significant increase (P < 0.001) in absolute humidity was demonstrated (Group I: 12 mg H2O.1(-1) vs Group II: 22 mg H2O.1(-1). This increase appeared immediately after introduction of the HMEF in the circuit and remained constant throughout the duration of the operation. Thus, the use of the device is recommended for young children, even for operations of short duration. [\hyperlink{Asmanex HFA}{PMID: 9243686}, J P Monrigal et al., 1997]

\hypertarget{pmid_32022483}{A}sthma affects over 6 million children in the United States alone. This study investigated the efficacy and long-term safety of mometasone furoate-formoterol (MF/F) and MF monotherapy in children with asthma. This phase 3, multicenter, randomized controlled trial evaluated metered-dose inhaler twice daily (BID) dosing with MF/F 100/10 µg or MF 100 µg in children, aged 5 to 11 years, with a history of asthma for greater than or equal to 6 months and confirmed bronchodilator reversibility, who were adequately controlled on inhaled corticosteroid/long-acting beta-agonist combination therapy for greater than or equal to 4 weeks. After a 2-week run-in on MF 100 µg BID, eligible patients received 24 weeks of double-blind treatment and were followed for safety up to 26 weeks. The primary efficacy endpoint was the change from baseline in AM postdose 60-minute AUC \%predicted FEV1\% across 12 weeks of treatment. A total of 181 participants received at least one dose of MF/F (n = 91) or MF (n = 90). MF/F was superior to MF across the 12-week evaluation period, with a treatment advantage of 5.21 percentage points (P < .001). Superior onset of action with MF/F over MF was achieved as early as 5 minutes postdose on day 1. Overall, approximately 50\% of participants experienced one or more treatment-emergent adverse events, with fewer occurring in the MF/F group. In children 5 to 11 years of age with persistent asthma, the addition of F to MF was well tolerated and provided significant, rapid, and sustained improvement in lung function compared with MF alone. [\hyperlink{Asmanex HFA}{PMID: 32022483}, Cindy L J Weinstein et al., 2020]

\hypertarget{pmid_26064413}{W}e report a case of serious anaphylactic shock in a 5-year-old child undergoing scheduled surgery blank space of a right femoral intramedullary nail removal. The boy had undergone right femoral elastic intramedullary nail fixation surgery 14 months prior, but had no history of allergies. Within 5 minutes of intravenous bonus injection of hemocoagulase agkistrodon (HCA) 1 unit, a widespread transient diffuse erythema was seen on the front of his chest. After 20 minutes, sudden, profound cardiovascular collapse occurred. The child was treated effectively and sent to a ward 5 hours later. In this period, he received intravenously infused 200 ml hydroxyethyl starch solution and epinephrine at a rate of 0.05-0.01 μg kg(-1) min(-1). Total amount of dexamethasone sodium phosphate 14 mg was used. To the best of our knowledge, few case reports of HCA-induced anaphylactic shock in children exist. Our report will, therefore, increase awareness of the allergic potential of HCA among pediatric anesthesiologists.  [\hyperlink{Asmanex HFA}{PMID: 26064413}, Ying-Yi Xu et al., 2015] Chronic hepatitis B in children is mainly asymptomatic, but they are at life long risk for severe complications. Treatment is considered to suppress the virus and to prolong the survival by preventing the progression to cirrhosis and HCC. Therapeutic options for children are interferon-alpha (IFN-alpha) with antiviral, antiproliferative and immuno-modulatory effects and lamivudine (LAM) which inhibits HBV replication and increases cellular immune response. IFN-alpha, 5 MU/m(2), thrice weekly for 6 months is used in patients with high ALT levels which is associated with virologic response rate of 30-40\%. Predictors of response are high ALT levels, low HBVDNA levels and high histological activity index. The response is sustained in 85\%-90\% of responders. Adverse events include flu-like syndrome, bone marrow suppression, hair loss, and psychiatric side effects, induction of autoimmunity and temporarily suppression of weight gain and growth velocity. LAM, a nucleoside anolog, leads to a virologic response rate of 20-30\% when used for 12 months. High ALT levels, low HBVDNA levels and high histological activity index predict better response. Maintenance of HBeAg seroconversion is 56-80\%. Longer courses of treatment with LAM increases the seroconversion rate but with high mutation rate and viral resistance. Except for causing mutations, LAM doesn't have serious adverse events. Different timing and durations of combination treatment with IFN and LAM were also evaluated without any significant superiority over monotherapy. In conclusion, the best approach for treatment of chronic HBV infection in children haven't been determined yet. Future developments concerning new drugs and different treatment strategies are needed. [\hyperlink{Asmanex HFA}{PMID: 26064413}, Aydan Kansu et al., 2008]

\hypertarget{pmid_25526197}{P}revious studies have found that children with multiple exposures to anesthesia at an early age are at increased risk of learning and memory impairment. Sevoflurane is the most commonly used inhalational anesthetic for general anesthesia in children. Multiple exposures to sevoflurane have been shown to induce neuroinflammation, inhibit neurogenesis, and cause subsequent learning and memory impairments in fetal mice. Histone-tail acetylation has been implicated in memory formation. In this study, we employed suberanilohydroxamic acid (SAHA), an inhibitor of histone deacetylases, to treat sevoflurane-induced learning and memory impairments. Six-day-old C57BL/6 mice were exposed to sevoflurane for 2 h daily for 3 days. Morris water maze test performed to evaluate learning and memory impairments and the expression of genes related in to synaptic remodeling/plasticity, or regulated by neuronal activity or the cell cycle were detected by real-time PCR. We found that SAHA attenuated sevoflurane-induced learning and memory impairments in fetal mice. Our findings suggest that SAHA may have potential as a therapeutic agent for preventing or treating the neurotoxicity associated with anesthesia.  [\hyperlink{Asmanex HFA}{PMID: 25526197}, X F Lin et al., 2014]

\section*{Epifoam}
\subsection*{Result}
\subsubsection*{Answer}
Unknown
\subsubsection*{{Explanation}}
\hypertarget{Epifoam}
A review of the available abstracts reveals that none of them are about Epifoam. The abstracts provided focus on the safety and efficacy of various anesthetic and sedative agents (such as propofol, sevoflurane, midazolam, and others) in children, but there is no mention of Epifoam or studies specifically evaluating its safety in pediatric populations. Therefore, based on the abstracts available, there is not enough data to determine whether Epifoam is safe or unsafe for use in children in any age range.

\subsection*{Abstracts}
\hypertarget{pmid_31110954}{V}arious publications on the use of sedation and anesthesia for diagnostic procedures in children have demonstrated that no ideal agent is available. Although propofol has been widely used for sedation during esophagogastroduodenoscopy in children, adverse events including hypoxia and hypotension, are concerns in propofol-based sedation. Propofol is used in combination with other sedatives in order to reduce potential complications. We aimed to analyze whether the administration of midazolam would improve the safety and efficacy of propofol-based sedation in diagnostic esophagogastroduodenoscopies in children. We retrospectively reviewed the hospital records of children who underwent diagnostic esophagogastroduodenoscopies during a 30-month period. Demographic characteristics, vital signs, medication dosages, induction times, sedation times, recovery times, and any complications observed, were examined. Baseline characteristics did not differ between the midazolam-propofol and propofol alone groups. No differences were observed between the two groups in terms of induction times, sedation times, recovery times, or the proportion of satisfactory endoscopist responses. No major procedural complications, such as cardiac arrest, apnea, or laryngospasm, occurred in any case. However, minor complications developed in 22 patients (10.7\%), 17 (16.2\%) in the midazolam-propofol group and five (5.0\%) in the propofol alone group ( The sedation protocol with propofol was safe and efficient. The administration of midazolam provided no additional benefit in propofol-based sedation. [\hyperlink{Epifoam}{PMID: 31110954}, Ulas Emre Akbulut et al., 2019]

\hypertarget{pmid_24586585}{T}his study was performed to analyse the effects of different sevoflurane concentrations on the incidence of epileptiform EEG activity during induction of anaesthesia in children in the clinical routine. It was suggested in the literature to use sevoflurane concentrations lower than 8\% to avoid epileptiform activity during induction of anaesthesia in children. 100 children (age: 4.6±3.0 years, ASA I-III, premedication with midazolam) were anaesthetized with 8\% sevoflurane for 3 min or 6\% sevoflurane for 5 min in 100\% O2 via face mask followed by 4\% sevoflurane until propofol and remifentanil were given for intubation. EEGs were recorded continuously and were analysed visually with regard to epileptiform EEG patterns. From start of sevoflurane until propofol/remifentanil administration, 38 patients (76\%) with 8\% sevoflurane had epileptiform EEG patterns compared to 26 patients (52\%) with 6\% (p = 0.0106). Epileptiform potentials tended to appear later in the course of the induction with 6\% than with 8\%. Up to an endtidal concentration of 6\% sevoflurane, the number of children with epileptiform potentials was similar in both groups (p = 0.3708). The cumulative number of children with epileptiform activity increased with increasing endtidal sevoflurane concentrations. The time from start of sevoflurane until loss of consciousness was similar in patients with 8\% and 6\% sevoflurane (42.2±17.5 s vs. 44.9 s ±14.0 s; p = 0.4073). An EEG stage of deep anaesthesia with continuous delta waves <2.0 Hz appeared significantly earlier in the 8\% than in the 6\% group (64.0±22.2 s vs. 77.9±20.0 s, p = 0.0022). The own analysis and data from the literature show that lower endtidal concentrations of sevoflurane and shorter administration times can be used to reduce epileptiform activity during induction of sevoflurane anaesthesia in children. [\hyperlink{Epifoam}{PMID: 24586585}, Ines Kreuzer et al., 2014]

\hypertarget{pmid_15166555}{P}ropofol is commonly used to anesthetize children undergoing esophagogastroduodenoscopy. Opioids are often used in combination with propofol to provide total intravenous anesthesia. Because both propofol and remifentanil are associated with rapid onset and offset, the combination of these two drugs may be particularly useful for procedures of short duration, including esophagogastroduodenoscopy. The authors previously demonstrated that the median effective concentration (C50) of propofol during esophagogastroduodenoscopy in children is 3.55 microg/ml. The purpose of this study was to describe the pharmacodynamic interaction of remifentanil and propofol when used in combination for esophagogastroduodenoscopy in pediatric patients. The authors studied 32 children aged between 3 and 10 yr who were scheduled to undergo esophagogastroduodenoscopy. Propofol was administered via a target-controlled infusion system using the STANPUMP software based on a pediatric pharmacokinetic model. Remifentanil was administered as a constant rate infusion of 25, 50, and 100 ng.kg(-1).min(-1) to each of three study groups, respectively. A sigmoid Emax model was developed to describe the interaction of remifentanil and propofol. There was a positive interaction between remifentanil and propofol when used in combination. The concentration of propofol alone associated with 50\% probability of no response was 3.7 microg/ml (SE, 0.4 microg/ml), and this was decreased to 2.8 microg/ml (SE, 0.1 microg/ml) when used in combination with remifentanil. A remifentanil infusion of 25 ng.kg(-1).min(-1) reduces the concentration of propofol required for adequate anesthesia for esophagogastroduodenoscopy from 3.7 to 2.8 microg/ml. Increasing the remifentanil infusion yields minimal additional decrease in propofol concentration and may increase the risk of side effects. [\hyperlink{Epifoam}{PMID: 15166555}, David R Drover et al., 2004]

\hypertarget{pmid_12883300}{T}o evaluate the efficacy and safety of propofol and meperidine plus midazolam for sedation during esophagogastroduodenoscopy (EGD) in children. Data were collected prospectively and retrospectively from neurologically intact children (0.2-17.7 years of age) who underwent ambulatory diagnostic EGD during a 4-year period. Data were included from 155 consecutive patients receiving propofol with or without premedication with midazolam (PM group). One hundred five consecutive patients who received sedation with a midazolam plus meperidine combination served as a comparison (MM group). Outcome variables were: time required for induction of sedation, length of procedure, time for recovery, need for additional supportive measures, and need for physical restraint. The onset of sedation was faster and the length of procedure and recovery were significantly shorter in the PM group as compared with the MM group (P < 0.01). Patients in the MM group required restraint more often than in the PM group. A higher dose of meperidine and midazolam was used in the prospective study. This led to deeper sedation but increased need for additional support. Propofol is safe and effective for facilitating EGD in children. [\hyperlink{Epifoam}{PMID: 12883300}, Vikram Khoshoo et al., 2003]

\hypertarget{pmid_26858095}{S}edation is increasingly used to facilitate procedures on children in emergency departments (EDs). This overview of systematic reviews (SRs) examines the safety and efficacy of sedative agents commonly used for procedural sedation in children in the ED or similar settings. We followed standard SR methods: comprehensive search; dual study selection, quality assessment, data extraction. We included SRs of children (1 month to 18 years) where the indication for sedation was procedure-related and performed in the ED. Fourteen SRs were included (210 primary studies). The most data were available for propofol (six reviews/50,472 sedations) followed by ketamine (7/8,238), nitrous oxide (5/8,220), and midazolam (4/4,978). Inconsistent conclusions for propofol were reported across six reviews. Half concluded that propofol was sufficiently safe; three reviews noted a higher occurrence of adverse events, particularly respiratory depression (upper estimate 1.1\%; 5.4\% for hypotension requiring intervention). Efficacy of propofol was considered in four reviews and found adequate in three. Five reviews found ketamine to be efficacious and seven reviews showed it to be safe. All five reviews of nitrous oxide concluded it is safe (0.1\% incidence of respiratory events); most found it effective in cooperative children. Four reviews of midazolam made varying recommendations. To be effective, midazolam should be combined with another agent that increases the risk of adverse events (upper estimate 9.1\% for desaturation, 0.1\% for hypotension requiring intervention). This comprehensive examination of an extensive body of literature shows consistent safety and efficacy for nitrous oxide and ketamine, with very rare significant adverse events for propofol. There was considerable heterogeneity in outcomes and reporting across studies and previous reviews. Standardized outcome sets and reporting should be encouraged to facilitate evidence-based recommendations for care. [\hyperlink{Epifoam}{PMID: 26858095}, Lisa Hartling et al., 2016]

\hypertarget{pmid_25956274}{I}n paediatric patients, esophagogastroduodenoscopy (EGD) is commonly performed with the use of sedation. The aim of the study was to compare the effectiveness of propofol and midazolam in providing procedural amnesia and controlling behaviour in children undergoing diagnostic EGD. Children (9-16 years), classified to the first or second class of the American Society of Anaesthesiologists' physical status classification referred for EGD, were randomly assigned to receive propofol with alfentanyl or midazolam with alfentanyl for sedation during the procedure. Within 120 min after the procedure, patients were repeatedly investigated for memory of the procedure and for memory of pain intensity during EGD with the use of the visual analogue scale. Activity and cooperation of the patient during the procedure was assessed with the relative adequacy scale. Of the 51 children, 48 completed the study. Propofol was significantly better than midazolam in inducing amnesia of procedural pain (mean difference 11.53 mm; 95 \% confidence interval [CI] 0.96 to 22.10), loss of memory of the procedure (relative risk 0.4; 95 \% CI 0.21 to 0.59) and controlling behaviour (relative risk 2.12; 95 \% CI 1.33 to 3.36). In children sedated for EGD, propofol is significantly better than midazolam at providing procedural amnesia and controlling behaviour during the procedure. [\hyperlink{Epifoam}{PMID: 25956274}, Edyta Sienkiewicz et al., 2015]

\hypertarget{pmid_29913339}{I}n pediatric patients, anaesthesia induction is often performed with intravenous Propofol or Sevoflurane inhalation. Although epileptiform discharges have been observed during inductions with Sevoflurane, their occurrence has not been investigated for i.v. Propofol inductions. The aim of this study is to compare the incidence of epileptiform discharges in children during anaesthesia induction using Propofol versus Sevoflurane. Prospective, observational cohort study in children aged 0.5-8 years undergoing elective surgery. Children were anaesthetized with either Propofol or Sevoflurane. Bi-frontal electroencephalograms electrodes were placed before start of anaesthesia. Visual electroencephalogram analysis was performed from start of anesthetic agent administration until Intubation with regard to identify epileptiform patterns, i.e. delta with spikes; rhythmic polyspikes; periodic, epileptiform discharges; or suppression with spikes. 39 children were anaesthetized with Propofol, and 18 children with Sevoflurane. Epileptiform discharges were seen in 36\% of the children in the Propofol group, versus 67\% in the Sevoflurane group (p = 0.03). Incidence of the distinct types of epileptiform discharge differed for periodic, epileptiform discharges (Sevoflurane group 39\% vs. Propofol group 3\%; p < 0.001). Higher concentration of Remifentanil (≥0.15 µg/kg/min) was associated with less frequent epileptiform discharges (Exp 5.8; CI 95\% 1.6/21.2; p = 0.008). Propofol i.v. induction of anaesthesia in children triggers epileptiform discharges, whereas to a lesser extent than Sevoflurane does. Presuming that epileptiform discharges have an impact on postoperative brain function, it is advisable to use Propofol rather than Sevoflurane and higher level of Remifentanil for anaesthesia induction in children. [\hyperlink{Epifoam}{PMID: 29913339}, Susanne Koch et al., 2018]

\hypertarget{pmid_3261948}{T}he induction characteristics of propofol were studied and compared with thiopentone in children aged 3-14 years who received either no premedication or pethidine-atropine or trimeprazine. Anaesthesia was maintained with nitrous oxide in oxygen, and isoflurane. The induction doses of propofol and thiopentone were 2.9 mg/kg and 6.5-7.1 mg/kg respectively; premedication had no significant effect on the induction doses of either agent. Spontaneous movement and hypertonus occurred in about 20\% of children with both agents. The use of propofol was associated with a high incidence of pain on injection (injections were mostly in veins on the dorsum of the hand), but this was reduced by mixing lignocaine with propofol. Cardiovascular effects were not clinically significant with either agent. Apnoea occurred in 35\% of patients given propofol and in 50\% of those given thiopentone. Children anaesthetised with propofol awoke significantly earlier after cessation of all anaesthesia. It is concluded that the use of propofol is safe in children and may have advantages where early recovery from anaesthesia is desirable, but offers no advantage over thiopentone for routine induction of anaesthesia. [\hyperlink{Epifoam}{PMID: 3261948}, R K Mirakhur et al., 1988]

\hypertarget{pmid_26290263}{P}ropofol is an intravenous agent used commonly for the induction and maintenance of anesthesia, procedural, and critical care sedation in children. The mechanisms of action on the central nervous system involve interactions at various neurotransmitter receptors, especially the gamma-aminobutyric acid A receptor. Approved for use in the USA by the Food and Drug Administration in 1989, its use for induction of anesthesia in children less than 3 years of age still remains off-label. Despite its wide use in pediatric anesthesia, there is conflicting literature about its safety and serious adverse effects in particular subsets of children. Particularly as children are not "little adults", in this review, we emphasize the maturational aspects of propofol pharmacokinetics. Despite the myriad of propofol pharmacokinetic-pharmacodynamic studies and the ability to use allometrical scaling to smooth out differences due to size and age, there is no optimal model that can be used in target controlled infusion pumps for providing closed loop total intravenous anesthesia in children. As the commercial formulation of propofol is a nutrient-rich emulsion, the risk for bacterial contamination exists despite the Food and Drug Administration mandating addition of antimicrobial preservative, calling for manufacturers' directions to discard open vials after 6 h. While propofol has advantages over inhalation anesthesia such as less postoperative nausea and emergence delirium in children, pain on injection remains a problem even with newer formulations. Propofol is known to depress mitochondrial function by its action as an uncoupling agent in oxidative phosphorylation. This has implications for children with mitochondrial diseases and the occurrence of propofol-related infusion syndrome, a rare but seriously life-threatening complication of propofol. At the time of this review, there is no direct evidence in humans for propofol-induced neurotoxicity to the infant brain; however, current concerns of neuroapoptosis in developing brains induced by propofol persist and continue to be a focus of research.  [\hyperlink{Epifoam}{PMID: 26290263}, Vidya Chidambaran et al., 2015] To document the safety and efficacy of an anaesthetic technique in paediatric patients undergoing transoesophageal echocardiography (TOE). Prospective descriptive study performed in a children's hospital with all patients undergoing TOE. Topical analgesia of the pharynx was achieved with lidocaine. Anaesthesia was induced with midazolam (25 microg.kg-1), fentanyl (1 microg.kg-1), and propofol (0.5-1 mg.kg-1), followed by a continuous infusion of propofol (5-10 mg.kg-1.h-1). Thirty patients are reported. The mean age was 11.4 +/- 5.1 years (range 1-22) and weight 40.5 +/- 22.1 kg (range 10-110). All the patients tolerated the procedure well. Two patients experienced brief oxygen desaturations during induction, 10 patients coughed during the procedure, and six patients had significant muscle activity requiring supplemental doses of propofol. None of the patients experienced nausea or vomiting. We conclude that our anaesthetic technique in spontaneously breathing paediatric patients during TOE is effective and appears to be safe in children with heart disease. [\hyperlink{Epifoam}{PMID: 26290263}, C M Heard et al., 2001]

\hypertarget{pmid_17474953}{T}he aim of this study was to evaluate the safety and efficacy of a combination of propofol and remifentanil deep sedation in spontaneously breathing children less than 7 years of age undergoing upper and/or lower gastrointestinal endoscopy. The effect of propofol and remifentanil sedation was prospectively studied in 42 unpremedicated children undergoing gastrointestinal endoscopy. Anesthesia was induced with a combination of sevoflurane, nitrous oxide and oxygen. Anesthesia was maintained with an infusion of propofol (50-80 microg x kg(-1) x min(-1)) and remifentanil (0.1 microg x kg(-1) x min(-1)). Demographic data, heart rate, blood pressure, respiratory rate, and oxygen saturation were recorded every 5 min for each child. In addition, recovery and discharge times were recorded. All 42 procedures were completed with no complications. The combination of propofol and remifentanil resulted in a decrease in heart rate, blood pressure, and respiratory rate. There was no respiratory depression or oxygen desaturation in any child. A bolus of propofol (1 mg x kg(-1)) was necessary in one child for excessive movement. No patient experienced any side effects in the recovery period. The combination of propofol and remifentanil for sedation in children undergoing gastrointestinal endoscopy can be considered safe, effective and acceptable. [\hyperlink{Epifoam}{PMID: 17474953}, Ibrahim Abu-Shahwan et al., 2007]

\hypertarget{pmid_17042840}{T}here is an ongoing debate as to whether propofol exhibits pro- or anticonvulsant effects, and whether it should be used in patients with epilepsy. We prospectively assessed the occurrence of seizure-like phenomena and the effects of intravenous propofol on the electroencephalogram (EEG) in 25 children with epilepsy (mean (SD) age: 101 (49) months) and 25 children with learning difficulties (mean (SD) age: 52 (40) months) undergoing elective sedation for MRI studies of the brain. No child demonstrated seizure-like phenomena of epileptic origin during and after propofol sedation. Immediately after stopping propofol, characteristic EEG changes in the epilepsy group consisted of increased beta wave activity (23/25 children), and suppression of pre-existing theta rhythms (11/16 children). In addition, 16 of 18 children with epilepsy and documented EEG seizure activity demonstrated suppression of spike-wave patterns after propofol sedation. In all 25 children with learning difficulties an increase in beta wave activity was seen. Suppression of theta rhythms occurred in 11 of 12 children at the end of the MRI study. In no child of either group was a primary occurrence or an increase in spike-wave patterns seen following propofol administration. The occurrence of beta wave activity (children with learning difficulties and epilepsy group) and suppression of spike-wave patterns (epilepsy group) were transient, and disappeared after 4 h. This study demonstrates characteristic, time-dependent EEG patterns induced by propofol in children with epilepsy and learning difficulties. Our data support the concept of propofol being a sedative-hypnotic agent with anticonvulsant properties as shown by depression of spike-wave patterns in children with epilepsy and by the absence of seizure-like phenomena of epileptic origin. [\hyperlink{Epifoam}{PMID: 17042840}, S Meyer et al., 2006]

\hypertarget{pmid_22829896}{A} high incidence of epileptiform activity in the electroencephalogram (EEG) was reported in children undergoing mask induction of anaesthesia with administration of high doses of sevoflurane for 5 minutes and longer. This study was performed to investigate whether reducing the time of exposure to a high inhaled sevoflurane concentration would affect the incidence of epileptiform EEG activity. It was hypothesized that no epileptiform activity would occur, when the inhaled sevoflurane concentration would be reduced from 8\% to 4\% immediately after the loss of consciousness. 70 children (age 7-96 months, ASA I-II, premedication with midazolam) were anaesthetized with 8\% sevoflurane in 100\% oxygen via face mask. Immediately after loss of consciousness, the sevoflurane concentration was reduced to 4\%. EEGs were recorded continuously and were later analyzed visually with regard to epileptiform EEG patterns. Sevoflurane at a concentration of 8\% was given for 1.2 ± 0.4 min (mean ± SD). In 14 children (20\%) epileptiform EEG patterns without motor manifestations were observed (delta with spikes (DSP), rhythmic polyspikes (PSR), epileptiform discharges (PED) in 10, 10, 4 children (14\%, 14\%, 6\%)). 38 children (54\%) had slow, rhythmic delta waves with high amplitudes (DS) appearing on average before DSP. The hypothesis that no epileptiform potentials would occur during induction of anaesthesia with a reduction of the inspired sevoflurane concentration from 8\% to 4\% directly after LOC was not proved. Even if 8\% sevoflurane is administered only briefly for induction of anaesthesia, epileptiform EEG activity may be observed in children despite premedication with midazolam. [\hyperlink{Epifoam}{PMID: 22829896}, Barbara Schultz et al., 2012]

\hypertarget{pmid_24138461}{T}he aim of this study was to determine the safety and the efficacy of paediatrician-administered propofol in children undergoing different painful procedures. We conducted a retrospective study over a 12-year period in three Italian hospitals. A specific training protocol was developed in each institution to train paediatricians administering propofol for painful procedures. In this study, 36,516 procedural sedations were performed. Deep sedation was achieved in all patients. None of the children experienced severe side effects or prolonged hospitalisation. There were six calls to the emergency team (0.02\%): three for prolonged laryngospasm, one for bleeding, one for intestinal perforation and one during lumbar puncture. Nineteen patients (0.05\%) developed hypotension requiring saline solution administration, 128 children (0.4\%) needed O2 ventilation by face mask, mainly during upper endoscopy, 78 (0.2\%) patients experienced laryngospasm, and 15 (0.04\%) had bronchospasm. There were no differences in the incidence of major complications among the three hospitals, while minor complications were higher in children undergoing gastroscopy. This multicentre study demonstrates the safety and the efficacy of paediatrician-administered propofol for procedural sedation in children and highlights the importance of appropriate training for paediatricians to increase the safety of this procedure in children. [\hyperlink{Epifoam}{PMID: 24138461}, Antonio Chiaretti et al., 2014]

\hypertarget{pmid_30614153}{I}n this prospective study, we describe the electroencephalographic (EEG) profiles in children anesthetized with sevoflurane or propofol. Seventy-three subjects (11 years, range 5-18) were included and randomly assigned to two groups according to the anesthetic agent. Anesthesia was performed by target-controlled infusion of propofol (group P) or by sevoflurane inhalation (group S). Steady-state periods were performed at a fixed randomized concentration between 2, 3, 4, 5, and 6 μg.ml Under propofol, BIS decreased monotonically and EEG slowed down as concentrations increased from 2 to 6 μg.ml Under deep anesthesia, the BIS and electroencephalographic profiles differ between propofol and sevoflurane. For high concentrations of sevoflurane, an elevated BIS value may be interpreted as a sign of epileptoid patterns or EEG fast oscillations rather than an insufficient depth of hypnosis. [\hyperlink{Epifoam}{PMID: 30614153}, Agnes Rigouzzo et al., 2019]

\hypertarget{pmid_23964333}{T}o evaluate the efficacy and safety of propofol and midazolam for sedation during esophagogastroduodenoscopy (EGD) in children. We retrospectively reviewed the hospital records of 62 children who underwent ambulatory diagnostic EGD during 1-year period. Data were collected from 34 consecutive patients receiving propofol alone. Twenty-eight consecutive patients who received sedation with midazolam served as a comparison group. Outcome variables were length of procedure, time to recovery and need for additional supportive measures. There were no statistically significant differences between the two groups in age, weight, sex, and the length of endoscopic procedure. The recovery time from sedation was markedly shorter in propofol group (30±16.41 minutes) compared with midazolam group (58.89±17.32 minutes; p<0.0001). During and after the procedure the mean heart rate was increased in midazolam group (133.04±19.92 and 97.82±16.7) compared with propofol group (110.26±20.14 and 83.26±12.33; p<0.0001). There was no localized pain during sedative administration in midazolam group, though six patients had localized pain during administration of propofol (p<0.028). There was no serious major complication associated with any of the 62 procedures. Intravenous administered propofol provides faster recovery time and similarly safe sedation compared with midazolam in pediatric patients undergoing upper gastrointestinal endoscopy. [\hyperlink{Epifoam}{PMID: 23964333}, Ji Eun Oh et al., 2013]

\hypertarget{pmid_15764631}{A}fter alarming reports concerning deaths after sedation with propofol, infusion of this drug was contraindicated by the US Food and Drug Administration in children <18 yr receiving intensive care. We describe our experiences with propofol 6\%, a new formula, during postoperative sedation in non-ventilated children following craniofacial surgery. In a prospective cohort study, children admitted to the paediatric surgical intensive care unit following major craniofacial surgery were randomly allocated to sedation with propofol 6\% or midazolam, if judged necessary on the basis of a COMFORT behaviour score. Exclusion criteria were respiratory infection, allergy for proteins, propofol or midazolam, hypertriglyceridaemia, familial hypercholesterolaemia or epilepsy. We assessed the safety of propofol 6\% with triglycerides (TG) and creatine phosphokinase (CPK) levels, blood gases and physiological parameters. Efficacy was assessed using the COMFORT behaviour scale, Visual Analogue Scale and Bispectral Index monitor. Twenty-two children were treated with propofol 6\%, 23 were treated with midazolam and 10 other children did not need sedation. The median age was 10 (IQR 3-17) months in all groups. Median duration of infusion was 11 (range 6-18) h for propofol 6\% and 14 (range 5-17) h for midazolam. TG levels remained normal and no metabolic acidosis or adverse events were observed during propofol or midazolam infusion. Four patients had increased CPK levels. We did not encounter any problems using propofol 6\% as a sedative in children with a median age of 10 (IQR 3-17) months, with dosages <4 mg kg(-1) h(-1) during a median period of 11 (range 6-18) h. [\hyperlink{Epifoam}{PMID: 15764631}, S A Prins et al., 2005]

\hypertarget{pmid_19207899}{P}ropofol is a sedative-hypnotic drug commonly used to anesthetize children undergoing esophagogastroduodenoscopy (EGD). Dexmedetomidine is a highly selective alpha-2 adrenergic receptor agonist that has been utilized in combination with propofol to provide anesthesia. There is currently no information regarding the effect of intravenous dexmedetomidine on the propofol plasma concentration-response relationship during EGD in children. This study aimed to investigate the pharmacodynamic interaction of propofol and dexmedetomidine when used in combination for children undergoing EGD. A total of 24 children undergoing EGD, ages 3-10 years, were enrolled in this study. Twelve children received dexmedetomidine 1 microg x kg(-1) given over 10 min as well as a continuous infusion of propofol delivered by a computer-assisted target-controlled infusion (TCI) system with target plasma concentrations ranging from 2.8 to 4.0 microg x ml(-1) (DEX group). Another group of 12 children undergoing EGD also received propofol administered by TCI targeting comparable plasma concentrations without dexmedetomidine (control group). We used logistic regression to predict plasma propofol concentrations at which 50\% of the patients exhibited minimal response to stimuli (EC50 for anesthesia). The EC50 +/- SE values in the control and DEX groups were 3.7 +/- 0.4 microg x ml(-1) and 3.5 +/- 0.2 microg x ml(-1), respectively. There was no significant shift in the propofol concentration-response curve in the presence of dexmedetomidine. The EC50 of propofol required to produce adequate anesthesia for EGD in children was unaffected by a concomitant infusion of dexmedetomidine 1 microg x kg(-1) given over 10 min. [\hyperlink{Epifoam}{PMID: 19207899}, Gregory B Hammer et al., 2009]

\hypertarget{pmid_14962329}{P}ropofol is frequently used for the induction and maintenance of anaesthesia in children aged 3 years and older. The present study compared the clinical and chemical effects of propofol containing disodium edetate (Diprivan) with that of sevoflurane in children younger than 3 years of age. This was an open-label, comparative, parallel-group study. Fifty-six healthy children were randomly assigned to receive either propofol (n=28; mean age 14.7 months) or sevoflurane (n=28; mean age 13.2 months) for ambulatory surgical procedures. Anaesthesia was induced with nitrous oxide (60\%), oxygen and sevoflurane (8\%). In the propofol group, it was followed by an intravenous infusion of propofol at a rate of 200 microg.kg(-1).min(-1). For the sevoflurane group, anaesthesia was maintained with sevoflurane (1.5-2.5\%). Haemodynamic measurements, recovery time and side-effects were recorded. Ionized calcium and magnesium concentrations in blood were measured. Statistical analysis was performed using ancova and the Fisher's exact test. The effects of propofol were similar to those of sevoflurane with respect to haemodynamic profile, recovery times (20 min vs 19.4 min) and side-effects (i.e. vomiting 10.7\% vs 7.1\%). Throughout the study, there were no significant differences between the mean ionized calcium and ionized magnesium concentrations in the two groups. In children younger than 3 years of age, propofol containing ethylenediaminetetraacetic acid has a similar profile to sevoflurane with respect to haemodynamic effects, recovery times, side-effects, ionized calcium and ionized magnesium levels. [\hyperlink{Epifoam}{PMID: 14962329}, Ira T Cohen et al., 2004]

\hypertarget{pmid_12453929}{S}evoflurane is a methyl ether anaesthetic commonly used for induction and maintenance of general anaesthesia in children. Sevoflurane is a non-irritant and acts quickly so induction is usually calm. However, inhalation induction with high concentrations of sevoflurane can cause convulsion-like movements and seizure-like changes in the electroencephalogram (EEG). Little is known about the EEG during maintenance of anaesthesia with sevoflurane, so we planned a prospective trial of sevoflurane maintenance after i.v. induction with benzodiazepine and barbiturate, which is another common induction technique in children. EEG recordings were made before premedication with midazolam (0.1 mg kg(-1) i.v.), during induction of anaesthesia with thiopental (5 mg kg(-1)), and during maintenance with sevoflurane (2\% end-tidal concentration in air/oxygen without nitrous oxide) in 30 generally healthy, 3- to 8-year-old children having adenoids removed. Noise-free EEG data of good quality were successfully recorded from all 30 children. Two independent neurophysiologists did not detect epileptiform discharges in any of the recordings. Premedication with midazolam, i.v. induction with thiopental and maintenance of anaesthesia with 2\% sevoflurane in air does not cause epileptiform EEG patterns in children. [\hyperlink{Epifoam}{PMID: 12453929}, K Nieminen et al., 2002]

\hypertarget{pmid_29071560}{A}nesthesia-induced neurotoxicity in immature animals has raised concerns about similar effects occurring in young children. Our study investigated two commonly used anesthetics-sevoflurane and propofol-for neurotoxicity in young children. Forty-seven children (aged 12-36 months) undergoing hypospadias repair surgery were randomized to receive sevoflurane (SG, n = 24) or propofol (PG, n = 23) general anesthesia. Venous blood was collected at three different times-immediately after induction, 2 h, and 3 days after surgery. The cellular portion was assessed for antioxidant defense and DNA damage, using enzyme assay kits and qRT-PCR, respectively, while serum was used to treat cultured neural stem cells (NSCs). MTT assay and TUNEL staining were performed, and the mRNA levels of antioxidant enzymes and apoptosis indicators were evaluated by qRT-PCR. Antioxidant defense and apoptosis status in the SG group were significantly higher than in the PG group at 2 h after surgery. Additionally, exposure of NSCs to postoperative serum of the SG group resulted in decreased cell density and viability, increased TUNEL-positive cells, elevated mRNA levels of antioxidant enzymes, and cleaved caspase-3 expression. Our data shows for the first time that in young children, administration of sevoflurane, but not propofol, leads to temporally increased antioxidant defense and apoptosis status as well as damage of NSCs. [\hyperlink{Epifoam}{PMID: 29071560}, Xue Zhou et al., 2018]

\hypertarget{pmid_27276432}{P}ropofol is a safe, well-tolerated anesthetic that is labeled as contraindicated in patients with egg or soy allergy. This contraindication has become increasingly problematic given the rising incidence of food allergy and eosinophilic esophagitis (EoE). To address this issue, we studied practice patterns of propofol use for esophagogastroduodenoscopies in children with EoE and food allergies at our institution. A retrospective observational study of 1365 esophagogastroduodenoscopies from January 2013 to June 2014 was performed. Data were analyzed using Student t tests, chi square tests, Fisher exact tests, and multivariable logistic regression. We found that propofol was used significantly less in patients with egg or soy allergy, and in patients with EoE, even after adjusting for the presence of food allergy. There was no difference in complication rates relative to propofol use. Propofol was used safely in pediatric patients with EoE and food allergy in this limited single-center review. [\hyperlink{Epifoam}{PMID: 27276432}, Pooja Mehta et al., 2017]

\hypertarget{pmid_25365612}{E}lectroencephalography (EEG) is a useful diagnostic tool in the diagnosis of seizure and differentiating it from seizure-like attacks. Cooperation and immobility of the patient is crucial and in children who do not naturally sleep, pharmacological agents and procedural sedation should be used for sleep inducement. The purpose of this study was to compare efficacy and safety of melatonin and intravenous solution of midazolam administered orally in sedation induction for EEG of children. In a parallel single-blinded randomized clinical trial, sixty 1 - 8 year old children who were referred to EEG Unit of Shahid Sadoughi Hospital, Yazd, Iran from September 2011 to March 2012 were evaluated. The Children were randomly assigned into two groups to receive orally 0.3 mg/kg melatonin or 0.75 mg/kg ampoule of midazolam. The primary outcome was efficacy in adequate sedation (Ramsay sedation score of four) and recording of EEG. Secondary outcome was clinical side effects. Nineteen girls (31.7\%) and 41 boys (68.3\%) with the mean age of 2.8 ± 1.8 years were evaluated. Adequate  sedation  and  recording  of EEG was  achieved in  36.7\% of  midazolam  group and  in 73.3\%  of  melatonin group, (p = 0.004). Transient agitation was seen in 6.6\% of midazolam group. No significant difference was observed from the viewpoint of side effects frequency between the two drugs, (p = 0.15).   Melatonin is a safe and an effective drug in sedation induction for EEG in children. [\hyperlink{Epifoam}{PMID: 25365612}, Razieh Fallah et al., 2014]

\hypertarget{pmid_16719889}{E}pidural analgesia in children is highly effective and safe; however, it has not enjoyed great popularity in surgery that requires cardiopulmonary bypass. A major concern is the possibility of damage to blood vessels with the epidural needle or catheter and epidural hematoma formation. There seems to be a low incidence of epidural hematoma if certain guidelines are followed, so that in children, epidural analgesia can be used in selected patients, with safety, when surgical repair requires cardiopulmonary bypass. Epidural morphine has been used for clinical pain relief in pediatric cardiac surgery. Improved pulmonary function, suppressed hormonal and metabolic stress responses, easy early tracheal extubation, and good analgesia and sedation that allows neurological examination to alert any possibles hidden complications, are the advantages. A dedicated medical team is essential in the perioperative management to achieve maximum benefit for these patients. [\hyperlink{Epifoam}{PMID: 16719889}, Ramon Vilà et al., 2006]

\hypertarget{pmid_30930979}{A}pplication of propofol in preventing emergence agitation after sevoflurane anesthesia in children was evaluated. Clinical data of 200 children who received sevoflurane anesthesia in Children's Hospital of Xuzhou Medical University were retrospectively analyzed. Among them, 120 patients who received inhaled sevoflurane for pediatric anesthesia and intravenous infusion of propofol (2 mg/kg) were included in observation group. The remaining 80 cases who were directly anesthetized by sevoflurane alone were the control group. T PAED scores, modified Aldrete scores, extubation time, PACU time and adverse reactions (gastrointestinal tract and respiratory response) were analyzed and compared between the control and observation group. PAED scores, extubation time, PACU time and incidence of adverse reactions were significantly lower in observation than in control group, and the modified Aldrete scores were higher in observation than in control group (P<0.05). Spearman's correlation analysis showed that the PAED scores were negatively correlated with modified Aldrete scores and positively correlated with extubation time. There was positive correlation between the PACU time and incidence of adverse reactions and between the PAED scores and extubation time. There was negative correlation between PACU time and incidence of adverse reactions and between Aldrete scores and extubation time (P<0.05). Therefore, we conclude that propofol can be used to prevent agitation after sevoflurane anesthesia in children. [\hyperlink{Epifoam}{PMID: 30930979}, Xiaole Wu et al., 2019]

\section*{Erythromycin}
\subsection*{Result}
\subsubsection*{Answer}

Neonates (0–14 days): No  
Infants (15–42 days): No  
Infants (43–90 days): Unknown  
Infants (>3 months) and Children (up to 18 years): Yes  
Preterm infants (prokinetic use): Unknown  

\subsubsection*{{Explanation}}
\hypertarget{Erythromycin}
A review of the available abstracts reveals multiple targeted studies evaluating the safety of erythromycin in children across various age groups. Below is a summary by age range:

Neonates and Infants (0–3 months):
- Several studies specifically address the use of erythromycin in neonates and young infants. One large retrospective cohort study found that exposure to oral erythromycin in the first 14 days of life is associated with a significantly increased risk of infantile hypertrophic pyloric stenosis (IHPS), with an adjusted odds ratio (aOR) of 13.3 (95\% CI, 6.80-25.9). The risk remains elevated, though lower, for exposures between 15 and 42 days of life (aOR 4.10, 95\% CI, 1.69-9.91). No association was found for exposures between 43 and 90 days of life. The study concludes that ingestion of oral erythromycin places young infants at increased risk of developing IHPS, especially in the first 6 weeks of life [\hyperlink{pmid_25687145}{PMID: 25687145}, Matthew D Eberly et al., 2015].
- A review of randomized controlled trials (RCTs) on erythromycin as a prokinetic agent in preterm infants found that oral erythromycin can be effective for gastrointestinal dysmotility, with no sinister adverse effects (including IHPS or fatal cardiac arrhythmia) reported in the RCTs reviewed. However, the review notes that long-term outcomes have not been fully evaluated and recommends cautious and selective use [\hyperlink{pmid_19218823}{PMID: 19218823}, Pak C Ng et al., 2009].
- Another study in preterm neonates (birth weight ≤1500g, ≤15 days old) treated for Ureaplasma urealyticum found erythromycin to be well-tolerated with no adverse effects observed in the small sample [\hyperlink{pmid_8036045}{PMID: 8036045}, K B Waites et al., 1994].
- A double-blind, placebo-controlled trial in infants and young children hospitalized with gastroenteritis (age not precisely specified, but includes infants) reported no significant adverse effects [\hyperlink{pmid_6349401}{PMID: 6349401}, R M Robins-Browne et al., 1983].
- A study of 12 infants less than one year old treated with erythromycin ethyl succinate for 7–10 days found no implantation of potentially pathogenic or resistant bacteria and did not report adverse effects [\hyperlink{pmid_3534749}{PMID: 3534749}, M J Butel et al., 1986].

Children (>3 months to 18 years):
- Multiple randomized controlled trials and comparative studies in children (ranging from 2 months to 15 years) with respiratory tract infections, pneumonia, skin infections, and other conditions consistently report that erythromycin is effective and generally well-tolerated. Adverse effects, when reported, are primarily mild gastrointestinal symptoms (e.g., diarrhea, vomiting, abdominal pain) [\hyperlink{pmid_9124837}{PMID: 9124837}, J J Roord et al., 1996; \hyperlink{pmid_1337553}{PMID: 1337553}, R Manfredi et al., 1992; \hyperlink{pmid_7667050}{PMID: 7667050}, S Block et al., 1995; \hyperlink{pmid_3149884}{PMID: 3149884}, J Goldfarb et al., 1988; \hyperlink{pmid_8195854}{PMID: 8195854}, R Salzberg et al., 1993; \hyperlink{pmid_18327427}{PMID: 18327427}, Ping-Ing Lee et al., 2008].
- A double-blind, placebo-controlled, crossover study in children aged 4–13 years with refractory chronic constipation found erythromycin to be useful and reported no erythromycin-related side effects [\hyperlink{pmid_14502372}{PMID: 14502372}, M A Bellomo-Brandão et al., 2003].
- A review of erythromycin’s prokinetic effects in children notes that, except for rare fatal reactions following intravenous administration in neonates at antibiotic doses, no serious adverse effects have been reported in studies using erythromycin for prokinetic purposes [\hyperlink{pmid_11328252}{PMID: 11328252}, J I Curry et al., 2001].
- A case report of an 18-year-old girl (adolescent) with reversible sensorineural hearing loss after high-dose oral erythromycin in the setting of severe renal failure suggests ototoxicity is rare and associated with high doses and/or renal impairment [\hyperlink{pmid_7359612}{PMID: 7359612}, P Thompson et al., 1980].

Summary:
- For neonates and infants (especially <6 weeks), there is strong evidence of increased risk of IHPS with oral erythromycin, indicating it is not safe for routine use in this age group.
- For infants older than 6 weeks, children, and adolescents, multiple studies affirm the safety of erythromycin, with adverse effects being generally mild and self-limited.
- For preterm infants and as a prokinetic agent, short-term safety appears acceptable in RCTs, but long-term safety is not fully established.

\subsection*{Abstracts}
\hypertarget{pmid_792407}{E}rythromycin continues to be a valuable and useful antimicrobial agent in children. Its low index of toxicity, freedom from sensitization, and reliable absorption and when administered orally contribute to make it an attractive agent in the treatment of a variety of minor respiratory and skin infections, especially in those situations where real or potential allergy to penicillin exists. Additional major uses are in the eradication of the carrier state in whooping cough and in diphtheria, especially in those instances when oral therapy can be tolerated. Dispite use over more than two decades, resistance developing in formerly susceptible organisms has not been a problem and thus seems unlikely to become so in the future. [\hyperlink{Erythromycin}{PMID: 792407}, C M Ginsburg et al., 1976]

\hypertarget{pmid_31321320}{A}zithromycin is widely used in children not only in the treatment of individual children with infectious diseases, but also as mass drug administration (MDA) within a community to eradicate or control specific tropical diseases. MDA has also been reported to have a beneficial effect on child mortality and morbidity. However, concerns have been raised about the safety of azithromycin, especially in young children. The aim of this review is to systematically identify the safety of azithromycin in children of all ages. MEDLINE, PubMed, Cochrane Central Register of Controlled Trials, Embase, CINAHL, International Pharmaceutical Abstracts and adverse drug reaction (ADR) monitoring systems will be systematically searched for randomised controlled trials (RCTs), cohort studies, case-control studies, cross-sectional studies, case series and case reports evaluating the safety of azithromycin in children. The Cochrane risk of bias tool, Newcastle-Ottawa and quality assessment tools, and The Joanna Briggs Institute Critical Appraisal tools will be used for quality assessment. Meta-analyses will be conducted to the incidence of ADRs from RCTs if appropriate. Subgroup analyses will be performed in different age and azithromycin dosage groups. Formal ethical approval is not required as no primary data are collected. This systematic review will be disseminated through a peer-reviewed publication. CRD42018112629. [\hyperlink{Erythromycin}{PMID: 31321320}, Peipei Xu et al., 2019]

\hypertarget{pmid_11328252}{E}rythromycin has been used as an antibiotic for more than four decades, but only in the last 10 years have other therapeutic benefits of this agent been exploited. Animal and human studies have demonstrated a prokinetic effect on the gastrointestinal tract at sub-antimicrobial doses (typically a quarter or less of the antibiotic dose). A limited number of studies have been performed in children to investigate this action. A review of this literature is particularly pertinent given the frequency of clinical problems related to gastrointestinal dysmotility in children and the limited availability of prokinetic agents in paediatric practice, compounded by the recent withdrawal of cisapride. The prokinetic effects of erythromycin have been investigated in infants with dysmotility associated with prematurity, in low birth-weight infants recovering from abdominal surgery, and in older children with a variety of other gastrointestinal disorders. Only one randomized placebo-controlled trial has been conducted. All except one of these studies have shown a beneficial effect of erythromycin in either promoting tolerance of enteral feeds or enhancing a measured index of gastrointestinal motility. Erythromycin appears to be equally effective when given orally (as ethylsuccinate or estolate) or intravenously (as lactobionate). Significantly, no serious adverse effects have been reported in studies in which erythromycin has been used for its prokinetic effects, although fatal reactions have followed the intravenous administration of erythromycin to neonates in antibiotic doses. [\hyperlink{Erythromycin}{PMID: 11328252}, J I Curry et al., 2001]

\hypertarget{pmid_18789096}{C}hronic bullous disease of childhood is the commonest acquired blistering disorder of children. Erythromycin has been reported to be beneficial for this condition. A three question survey was e-mailed to all members of the British Society for Paediatric Dermatology to assess the incidence, preferred treatments and experience of oral erythromycin in treating chronic bullous disease of childhood. A second, more detailed questionnaire was sent to members who had used erythromycin. Forty patients were reported to have been treated over the previous 2 years. The preferred treatment was dapsone. Erythromycin alone had been used in five children as first-line oral treatment. In three of these patients the initial improvement was graded as either "good" or "complete resolution." This benefit was only sustained in one child, with the other two relapsing between 4 and 12 weeks. In a further eight children, erythromycin had been used with other oral agents. In five of these children, erythromycin was associated with long-term benefit. These results suggest that erythromycin is unlikely to produce sustained improvement in chronic bullous disease of childhood when used as a sole first-line agent. However, erythromycin can cause an initial improvement, which may be useful whilst awaiting results of diagnostic tests and may confer benefit when used with other systemic treatments. [\hyperlink{Erythromycin}{PMID: 18789096}, Paul Farrant et al., ]

\hypertarget{pmid_7359612}{E}rythromycin is considered one of the safest antibiotics in common use today. In its otolaryngologic use, the authors have found it effective in treating acute suppurative sinusitis and occasionally otitis media, when combined with sulfonamides. There are few complications of erythromycin administration. Probably the least generally acknowledged of these is ototoxicity. There have been three reports of six cases with ototoxic complications from erythromycin, primarily from administration of its intravenous form. The authors present a case study of an 18 year old girl in severe renal failure, who suffered a reversible sensorineural hearing loss from high doses of an oral erythromycin preparation. The clinical manifestations of this case are compared to those previously reported. [\hyperlink{Erythromycin}{PMID: 7359612}, P Thompson et al., 1980]

\hypertarget{pmid_34447818}{R}espiratory infections in children are common pediatric diseases caused by pathogens that invade the respiratory system. Children are considerably susceptible to  To analyze the clinical efficacy of different antibiotics in treating pediatric respiratory mycoplasma infections. We included 106 children with a confirmed diagnosis of respiratory mycoplasma infection who were admitted to our hospital from April 2017 to July 2019 and grouped them using a random number table. Among them, 53 children each received clarithromycin or erythromycin. The clinical efficacy of both drugs was evaluated and compared. We performed the multiplex polymerase chain reaction (MP-PCR) test and determined the MP-PCR negative rate in children after the end of the treatment course. We compared the incidence of toxic and side effects, including nausea, diarrhea, and abdominal pain; further, we recorded the length of hospitalization, antipyretic time, and drug costs. Additionally, we evaluated and compared the compliance of the children during treatment. The erythromycin group showed a significantly higher total effective rate of clinical treatment than the clarithromycin group. MP-PCR test results showed that the clarithromycin group had a significantly higher MP-PCR negative rate than the erythromycin group. Moreover, children in the clarithromycin group had shorter fever time, shorter hospital stays, and lower drug costs than those in the erythromycin group. The clarithromycin group had a significantly higher overall drug adherence rate than the erythromycin group. The incidence of toxic and side effects was significantly lower in the clarithromycin group than in the erythromycin group ( Our findings indicate that clarithromycin has various advantages over erythromycin, including higher application safety, stronger mycoplasma clearance, and higher medication compliance in children; therefore, it can be actively promoted. [\hyperlink{Erythromycin}{PMID: 34447818}, Mei-Ying Zhang et al., 2021]

\hypertarget{pmid_36827282}{E}rythromycin is a macrolide antibiotic that is also prescribed off-label in premature neonates as a prokinetic agent. There is no oral formulation with dosage and/or excipients adapted for these high-risk patients. Clinical studies of erythromycin as a prokinetic agent were reviewed. Capsules of 20 milligrams of erythromycin were compounded with microcrystalline cellulose. Erythromycin capsules were analyzed using the chromatographic method described in the United States Pharmacopoeia which was found to be stability-indicating. The stability of 20 mg erythromycin capsules stored protected from light at room temperature was studied for one year. 20 mg erythromycin capsules have a beyond use date not lower than one year. 20 milligrams erythromycin capsules can be compounded in batches of 300 unities in hospital pharmacy with a beyond-use-date of one year at ambient temperature protected from light. [\hyperlink{Erythromycin}{PMID: 36827282}, Patrick Thevin et al., 2023]

\hypertarget{pmid_3534749}{E}rythromycin ethyl succinate is an antibiotic frequently administered in pediatrics. According to some authors, this drug sharply decreases the fecal count of enterobacteria. The fecal flora of 12 infants less than one year old, treated by erythromycin ethyl succinate for 7 to 10 days was studied by differential count. A variable effect was observed on enterobacteria: a 10(3) to 10(5) fold reduction in 9 cases with a final count superior or equal to 10(4) per gram of feces, with or without coming back to the initial count; in 3 cases no modification. MIC of enterobacteria and concentrations of erythromycin in feces were not predictives of flora variation. Anaerobic flora was weakly modified. No implantation of potentially-pathogenic bacteria or multi-resistant or highly erythromycin resistant enterobacteria occurred. Thus, erythromycin ethyl succinate is valuable in pediatrics as it does not disturb barrier effects. But its use for selective decontamination of gut must be discussed depending on pharmacologic form and posology administered. [\hyperlink{Erythromycin}{PMID: 3534749}, M J Butel et al., 1986]

\hypertarget{pmid_14502372}{T}he efficacy of erythromycin was assessed in the treatment of 14 children aged 4 to 13 years with refractory chronic constipation, and presenting megarectum and fecal impaction. A double-blind, placebo- controlled, crossover study was conducted at the Pediatric Gastroenterology Outpatient Clinic of the University Hospital. The patients were randomized to receive placebo for 4 weeks followed by erythromycin estolate, 20 mg kg-1 day-1, divided into four oral doses for another 4 weeks, or vice versa. Patient outcome was assessed according to a clinical score from 12 (most severe clinical condition) to 0 (complete recovery). At enrollment in the study and on the occasion of follow-up medical visits at two-week intervals, patient score and laxative requirements were recorded. During the first 30 days, the mean SD clinical score for the erythromycin group (N = 6) decreased from 8.2+/-2.3 to 2.2+/-1.0 while the score for the placebo group (N = 8) decreased from 7.8+/-2.1 to 2.9+/-2.8. During the second crossover phase, the score for patients on erythromycin ranged from 2.9+/-2.8 to 2.4+/-2.1 and the score for the patients on placebo worsened from 2.2+/-1.0 to 4.3+/-2.3. There was a significant improvement in score when patients were on erythromycin (P < 0.01). Mean laxative requirement was lower when patients ingested erythromycin (P < 0.05). No erythromycin-related side effects occurred. Erythromycin was useful in this group of severely constipated children. A larger trial is needed to fully ascertain the prokinetic efficacy of this drug as an adjunct in the treatment of severe constipation in children. [\hyperlink{Erythromycin}{PMID: 14502372}, M A Bellomo-Brandão et al., 2003]

\hypertarget{pmid_9124837}{T}he efficacies and safeties of a 3-day, 3-dose course of azithromycin (10 mg/kg of body weight per day) and a 10-day, 30-dose course of erythromycin (40 mg/kg/day) for the treatment of acute lower respiratory tract infections in children were compared in an open randomized multicenter study. Sixty-eight of 85 evaluable patients (80\%) had radiologically proven pneumonia, and 20\% had bronchitis. Treatment success defined as cure or major improvement was achieved in 42 of 45 (93\%) azithromycin recipients versus 36 of 40 (90\%) erythromycin recipients. Adverse events were reported in 12 of 45 and 6 of 40 of the patients treated with azithromycin and erythromycin, respectively, a difference which was not statistically significant. In conclusion, a 3-day course of azithromycin is as effective as a 10-day course of erythromycin in the treatment of community-acquired lower respiratory tract infections in children, with comparable safety and acceptability profiles. This shorter treatment course might have a beneficial effect on compliance, especially in the pediatric age group. [\hyperlink{Erythromycin}{PMID: 9124837}, J J Roord et al., 1996]

\hypertarget{pmid_6349401}{A} double-blind placebo-controlled trial of erythromycin ethylsuccinate was conducted in 65 infants and young children hospitalized with acute nonspecific gastroenteritis. Etiologic agents included rotaviruses (29\%), Campylobacter jejuni (17\%), "classical" enteropathogenic Escherichia coli (12\%), enterotoxigenic E. coli (11\%), Salmonella (9\%), Shigella (2\%), and Giardia lamblia (2\%). No pathogens were obtained from 25 (38\%) children. Treatment with erythromycin had no effect on the course of the illness in terms of the time required for hydration, stool frequency and temperature to return to normal, or for vomiting to be abolished. Children treated with erythromycin, however, experienced a marginally, but significantly (P less than 0.05), shorter period of abnormal stool consistency compared with control subjects. This effect was most pronounced in children from whom no enteropathogens were isolated. [\hyperlink{Erythromycin}{PMID: 6349401}, R M Robins-Browne et al., 1983]

\hypertarget{pmid_7049959}{F}ollowing a study in which the etiology of nearly 70\% of 142 cases of pneumonia in children could be determined using a combination of bacteriological and serological methods, the effect of erythromycin ethylsuccinate was compared with that of amoxicillin in a randomized study on 120 cases of pneumonia. We first examined the tracheal secretion microbiologically and determined other serological parameters and clinical data. The tracheal secretion was sterile in only 19\% of the cases. We were able to identify the etiology in 64\% of the cases using a combination of microbiological and serological methods. A discontinuation of therapy and acceptable side-effects were considerably more frequent with amoxicillin than with erythromycin ethylsuccinate (75 mg/kg body weight). The advantages of erythromycin, especially for the initial therapy of pneumonia, and the improvements in diagnosis resulting from the examination of the tracheal secretion will be discussed. [\hyperlink{Erythromycin}{PMID: 7049959}, H Ruhrmann et al., 1982]

\hypertarget{pmid_1820902}{E}rythromycin pharmacokinetics was studied in neonates (less than 1 month), infants (1-12 months) and other children (1-12 years) after the drug rectal and intravenous administration. The areas under the erythromycin serum concentration-time curves (AUC) were practically independent on children's age following the intravenous drug administration, but not its rectal administration. There was a distinct age dependency of the AUC parameter in the latter case. The increase of children's age was resulted in enhancement of the erythromycin total clearance, reduction of the steady-state volume of distribution and of the mean residence time. The extent of absolute bioavailability of rectally administered erythromycin was increased from 28 per cent in neonates to 36 per cent in infants and to 54 per cent in children greater than 1 year. Alteration of the mean absorption time parameter was reflected the delayed absorption of erythromycin in neonates. [\hyperlink{Erythromycin}{PMID: 1820902}, L S Stratchunsky et al., 1991]

\hypertarget{pmid_3149884}{T}he safety and efficacy of a new topical antiinfective agent, mupirocin, was compared with that of oral erythromycin ethylsuccinate in the treatment of impetigo in children. Sixty-two children aged 5 months to 13 years with impetigo were assigned to be treated with either mupirocin in three daily applications or erythromycin ethylsuccinate (40 mg/kg of body weight per day divided into four doses) according to a randomized treatment schedule. On the initial visit, exudate or cleansed infected sites or both were cultured and therapy was begun. All patients were treated for 8 days. Patients were seen again on days 4 to 5 of therapy, at the end of therapy, and 7 days after the end of therapy. Sites of infection were comparable between the groups, as were bacteriologic responses. At the first visit, 24 of 30 children in the mupirocin group and 14 of 32 children in the erythromycin group were cured or had at least a 75\% reduction in size of the lesions. At the end of the study, all 29 of the children in the mupirocin group who came to follow-up, compared with 27 of 29 in the erythromycin group, were cured. Side effects were few. Five children in the erythromycin group developed mild diarrhea. Thus, mupirocin appears to be safe and effective in treating impetigo in children. Our data show a trend toward more rapid clinical response with mupirocin than with erythromycin. [\hyperlink{Erythromycin}{PMID: 3149884}, J Goldfarb et al., 1988]

\hypertarget{pmid_8036045}{E}rythromycin is receiving renewed attention as an alternative for treatment of neonatal infections caused by Ureaplasma urealyticum because of recently proved abilities of this organism to produce systemic disease in this population. Although erythromycin has been used clinically for almost 40 years, very little is known about its activity in the preterm neonate. Fourteen neonates, birth weights < or = 1500 g and < or = 15 days of age, from whom U. urealyticum was isolated from the lower respiratory tract were randomized to receive erythromycin lactobionate either 25 or 40 mg/kg/day in four divided doses at 6-hour intervals scheduled for a total of 10 days. Blood samples collected at multiple time points after initial and steady state doses were assayed for erythromycin by liquid chromatography. Minimal inhibitory concentrations (MICs) of erythromycin for the U. urealyticum isolates were determined. MICs ranged from 0.031 to 2 micrograms/ml; MIC90 = 2 micrograms/ml. Serum erythromycin concentrations met or exceeded most MICs, with peak values of 3.05 to 3.69 and 1.92 to 2.9 micrograms/ml for the 40- and 25-mg/kg/day dosage groups, respectively. Pharmacokinetic parameters were calculated after the initial dose and at steady state for both dosage groups and compared. No adverse effects thought to be related to administration of erythromycin were observed. These preliminary findings showed that erythromycin is well-tolerated, has favorable pharmacokinetic activity in the preterm neonate and should be further investigated for treatment of ureaplasmal infections. [\hyperlink{Erythromycin}{PMID: 8036045}, K B Waites et al., 1994]

\hypertarget{pmid_1337553}{T}he efficacy and tolerability of azithromycin and erythromycin in the treatment of acute respiratory tract infections in children were compared in an open, multicenter, randomized trial. A total of 151 children, aged from 2 months to 14 years, suffering from upper airways infections (60), or lower respiratory tract infections (91), were randomized to be treated either with azithromycin, 10 mg/Kg/day per os once daily for 3 or 10 mg/Kg/day 1 and 5 mg/Kg/days 2-5 (77 patients) or with erythromycin, 50 mg/Kg/day thrice daily for at least 7 days (74 patients). The two treatment groups did not significantly differ as to sex, age, weight, type and severity of infection, and infecting pathogens. Clinical evaluation was performed prior to therapy, on treatment days 1, 3, 5 and 7, and on day 10. Microbiological and laboratory assessment were carried out at baseline and after the end of therapeutic course. Chest X-ray and serologic assays for Mycoplasma pneumoniae infection were obtained in patients suspected to have lower respiratory tract infections. At the end of therapy, clinical cure was achieved in 73 out of 77 patients (94.8\%) in the azithromycin group, and in 60/72 evaluable subjects (83.3\%) in the erythromycin group. A significantly more rapid remission of several illness-related signs and symptoms was observed in patients treated with azithromycin. A total of 75 bacterial pathogens were isolated at baseline microbiological examination; at the end of the therapeutic course bacteriological eradication was obtained in 34/34 cases (100\%) treated with azithromycin, and in 40/41 children (97.5\%) treated with erythromycin.(ABSTRACT TRUNCATED AT 250 WORDS) [\hyperlink{Erythromycin}{PMID: 1337553}, R Manfredi et al., 1992] In spite of vaccination programmes, whooping cough epidemics continue to occur. The disease affects all age groups, although its severity is greatest in the young, with infants being particularly vulnerable. Erythromycin is generally accepted as the drug of choice both for treatment and for prophylaxis during epidemics. Roxithromycin is a macrolide with pharmacokinetic advantages over erythromycin; it is well absorbed, produces high serum concentrations, has a long half-life and penetrates respiratory secretions well. There are no accepted standards for testing the sensitivity of Bordetella pertussis to antibiotics, and reports of the activity of roxithromycin and erythromycin are variable. Using Isosensitest agar supplemented with 5\% horse blood and an inoculum of 10(4) cfu, 88 strains of B. pertussis were tested for their sensitivity to roxithromycin, erythromycin, rifampicin and trimethoprim/sulphamethoxazole. The range of MICs was 0.12-0.5 mg/L for both roxithromycin and erythromycin. Roxithromycin was bactericidal, with an MBC of 1 mg/L (as compared with 0.5 mg/L for erythromycin). Since roxithromycin is well tolerated by children when used for respiratory tract infections, the good in-vitro activity against B. pertussis, combined with its favourable pharmacokinetics, suggest it may be a good candidate for use in the treatment and prophylaxis of whooping cough. [\hyperlink{Erythromycin}{PMID: 1337553}, M Brett et al., 1998]

\hypertarget{pmid_3429384}{R}oxithromycin sachets of 50 mg were given to 304 infants and children, aged 2 months to 14 years, suffering from respiratory and skin infections treated in 25 hospitals in France and one in Greece. The dosage range was from 2.5 to 5.0 mg/kg/12 h and the mean duration of therapy was 8.9 days. The cure rate was 89\% of the 266 children evaluable for clinical efficacy and 90.3\% of the 50 bacteriologically identified cases. The overall bacteriological efficacy was 82\%. The antibiotic was well accepted by the 90\% of the 304 children, while in 6.9\% an adverse effect was reported, mainly vomiting. There were no toxic effects. Roxithromycin should be considered as an effective and safe oral antibiotic to treat children with upper and lower respiratory tract and skin infections due to common pathogens. [\hyperlink{Erythromycin}{PMID: 3429384}, D A Kafetzis et al., 1987]

\hypertarget{pmid_25687145}{U}se of oral erythromycin in infants is associated with infantile hypertrophic pyloric stenosis (IHPS). The risk with azithromycin remains unknown. We evaluated the association between exposure to oral azithromycin and erythromycin and subsequent development of IHPS. A retrospective cohort study of children born between 2001 and 2012 was performed utilizing the military health system database. Infants prescribed either oral erythromycin or azithromycin as outpatients in the first 90 days of life were evaluated for development of IHPS. Specific diagnostic and procedural codes were used to identify cases of IHPS. A total of 2466 of 1 074 236 children in the study period developed IHPS. Azithromycin exposure in the first 14 days of life demonstrated an increased risk of IHPS (adjusted odds ratio [aOR], 8.26; 95\% confidence interval [CI], 2.62-26.0); exposure between 15 and 42 days had an aOR of 2.98 (95\% CI, 1.24-7.20). An association between erythromycin and IHPS was also confirmed. Exposure to erythromycin in the first 14 days of life had an aOR of 13.3 (95\% CI, 6.80-25.9), and 15 to 42 days of life, aOR 4.10 (95\% CI, 1.69-9.91). There was no association with either macrolide between 43 and 90 days of life. Ingestion of oral azithromycin and erythromycin places young infants at increased risk of developing IHPS. This association is strongest if the exposure occurred in the first 2 weeks of life, but persists although to a lesser degree in children between 2 and 6 weeks of age. [\hyperlink{Erythromycin}{PMID: 25687145}, Matthew D Eberly et al., 2015]

\hypertarget{pmid_18327427}{T}his study aimed to evaluate the efficacy and safety of clarithromycin and erythromycin in the treatment of community-acquired pneumonia in children. Children with community-acquired pneumonia were randomly assigned to receive 10-day regimens of either clarithromycin 15 mg/kg/day, twice a day, or erythromycin 30-50 mg/kg/day, four times daily. A total of 97 children entered this study, including 26 with Mycoplasma pneumoniae infection, 15 with Chlamydia pneumoniae infection, and 6 with mixed mycoplasma and chlamydia infections. Fifty and 47 children received clarithromycin and erythromycin treatment, respectively. Three children withdrew from the study because the identified pathogens were resistant to the study drugs. All 47 children with mycoplasma or chlamydia infection were cured clinically. Delayed defervescence, defined as a fever lasting for more than 72 h after treatment, was observed in 4 of 22 clarithromycin-treated children (18\%) and in 3 of 15 erythromycin-treated children (20\%) [p>0.05]. Gastrointestinal side effects, including vomiting, abdominal pain and diarrhea, were observed in 3 of 50 children (6\%) receiving clarithromycin and in 11 of 49 children (22\%) receiving erythromycin (p=0.039). Excluding children with abnormal pretreatment liver function, abnormal liver function after treatment was observed in only one child, treated with erythromycin. Post-treatment eosinophil and platelet counts were significantly elevated after treatment in both groups. Clarithromycin showed efficacy equivalent to erythromycin for the treatment of mycoplasma or chlamydia pneumonia in children. However, the tolerability of clarithromycin was superior to that of erythromycin. [\hyperlink{Erythromycin}{PMID: 18327427}, Ping-Ing Lee et al., 2008]

\hypertarget{pmid_19218823}{M}ilk intolerance due to functional gastrointestinal (GI) dysmotility is a common problem in preterm infants. In the past decade, erythromycin has been used for its motilinomimetric effect to facilitate enteral feeding in preterm infants. Although earlier studies suggested that erythromycin is an effective prokinetic agent, recent randomized control trials (RCTs) reveal conflicting findings. This review assesses the evidence from all RCTs performed to date on erythromycin for preterm infants. The results suggest that oral erythromycin administered in intermediate or high doses as a rescue treatment is associated with a shorter time to attain full enteral feeding and decrease in the duration of requirement for parenteral nutrition. More importantly, the outcome study further indicates that oral erythromycin can reduce the incidence of parenteral nutrition-associated cholestasis by almost 50\% and decreases the incidence of recurrent septicemia. None of the RCTs reported any sinister adverse effects, in particular, hypertrophic infantile pyloric stenosis or fatal cardiac arrhythmia. Nonetheless, as long-term outcomes have not been fully evaluated, neonatologists should use this treatment cautiously and selectively in preterm infants with moderately severe GI dysmotility. [\hyperlink{Erythromycin}{PMID: 19218823}, Pak C Ng et al., 2009]

\hypertarget{pmid_7667050}{W}e evaluated 260 previously healthy children ages 3 through 12 years who had clinical signs and symptoms of pneumonia, radiographically confirmed. Patients were randomized 1:1 to a 10-day course of either clarithromycin suspension 15 mg/kg/day divided twice a day or erythromycin suspension 40 mg/kg/day divided twice a day or three times a day. Evidence of infection with Chlamydia pneumoniae was detected in 28\% (74) of patients: 13\% (34) by nasopharyngeal culture and 18\% (48) by serology with the microimmunofluorescence assay. Evidence of infection with Mycoplasma pneumoniae was detected in 27\% (69) of patients: 20\% (53) by nasopharyngeal culture or polymerase chain reaction and 17\% (44) by serology with the use of enzyme-linked immunosorbent assay. Serologic confirmation of infection was observed in 23\% (8) and 53\% (28) of patients with bacteriologically detected C. pneumoniae and M. pneumoniae, respectively. Treatment with clarithromycin vs. erythromycin, respectively, yielded the following outcomes: clinical success 98\% (121 of 124) vs. 95\% (105 of 110); radiologic success 98\% (109 of 111) vs. 94\% (92 of 110); and eradication by pathogen, C. pneumoniae 79\% (15 of 19) vs. 86\% (12 of 14) and M. pneumoniae 100\% (9 of 9) vs. 100\% (4 of 4). Adverse events were primarily gastrointestinal occurring in almost one-fourth of patients in both groups, and were mild to moderate in severity. Clarithromycin and erythromycin were similarly effective and safe for the treatment of radiographically proved, community-acquired pneumonia in children older than 2 years old.(ABSTRACT TRUNCATED AT 250 WORDS) [\hyperlink{Erythromycin}{PMID: 7667050}, S Block et al., 1995] Erythromycin is recommended for secondary prophylaxis in children with rheumatic heart disease, who are allergic to penicillin. A 9-year-old girl, with rheumatic heart disease, on secondary prophylaxis with erythromycin 250 mg BD, presented with acute rheumatic fever. Responded to steroids and started on a higher dose (250 mg TDS) of erythromycin for secondary prophylaxis. There is need to document the resistance of group A streptococci to erythromycin. [\hyperlink{Erythromycin}{PMID: 7667050}, Dinesh Kumar Yadav et al., 2013]

\hypertarget{pmid_8195854}{T}he objective of the study was the comparison of the efficacy and tolerability of brodimoprim to those of erythromycin in children with acute tonsillitis or bronchitis. 50 children aged 0.5 to 9.3 years were included in the study, 25 treated either with brodimoprim or with erythromycin. The evaluation of the therapeutic response was based exclusively on clinical criteria. In the brodimoprim group the therapy was successful in 24 patients (one failure), in the erythromycin group the therapy was also successful in 24 children (one failure). Side effects: three patients treated with brodimoprim reported adverse reactions (stomatitis, vomiting, skin rash), whereas only one patient in the erythromycin group developed a skin rash. both therapeutic regimens were equally effective against bronchitis and tonsillitis in children. The tolerability was good in both groups. [\hyperlink{Erythromycin}{PMID: 8195854}, R Salzberg et al., 1993]

\hypertarget{pmid_7782112}{I}n two multicentre, non-comparative studies, a total of 477 children, aged 2 months to 15 years, suffering from respiratory tract infections or skin and soft tissue infections were treated with roxithromycin (50 mg sachets). The mean duration of treatment was 9 days and the mean daily dose was 6 mg/kg/day administered b.i.d. The overall safety of roxithromycin was assessed by analysing the adverse events reported by the patients or investigators, the discontinuation of treatment because of adverse events, and the laboratory data. Adverse events reported by 20 (4\%) children were considered to be possibly drug related. These adverse events were mainly digestive in nature and mild or moderate in severity. Treatment was discontinued because of adverse events in six (1\%) children. Analysis of laboratory data revealed a few variations without any clinical significance. [\hyperlink{Erythromycin}{PMID: 7782112}, P Bégué et al., 1995]

\section*{Omeprazole}
\subsection*{Result}
\subsubsection*{Answer}

Infants (<1 year): Unknown  
Children (1-16 years): Yes  

\subsubsection*{{Explanation}}
\hypertarget{Omeprazole}
Based on the abstracts reviewed, there is evidence from multiple targeted studies evaluating the safety of omeprazole in children across various age ranges. Here is a summary by age group:

Infants (<1 year):
- Several studies have evaluated omeprazole in infants, particularly for gastroesophageal reflux disease (GERD) and esophagitis. One randomized study in infants (6-12 weeks postnatal) with GERD found that both rectal and oral omeprazole were effective in increasing intraesophageal and gastric pH, with no significant safety concerns reported in the abstract [\hyperlink{pmid_32594305}{PMID: 32594305}, Petra Bestebreurtje et al., 2020]. Another study in infants (mean age \textasciitilde{}3 months) with peptic esophagitis reported marked improvement in symptoms and no adverse events during 6 weeks of omeprazole therapy [\hyperlink{pmid_9506656}{PMID: 9506656}, P Alliët et al., 1998]. However, a literature review noted that while omeprazole affects gastric acidity in infants, there is limited evidence for symptom improvement and few reported side effects, suggesting that safety data are present but not extensive [\hyperlink{pmid_20719016}{PMID: 20719016}, Robert G T Blokpoel et al., 2010]. Another study developing a pediatric suppository formulation for infants stated that clinical studies are still needed to establish safety in this population [\hyperlink{pmid_32594306}{PMID: 32594306}, Petra Bestebreurtje et al., 2020].

Children (1-16 years):
- Multiple studies specifically targeted children aged 1-16 years with erosive esophagitis or peptic ulcer disease. A large open multicenter study (n=57) found omeprazole to be well tolerated, highly effective, and safe for treatment of erosive esophagitis in children, including those with neurologic impairment or prior surgery [\hyperlink{pmid_11113836}{PMID: 11113836}, E Hassall et al., 2000]. Another pharmacokinetic study in the same age group (1-16 years) reported that omeprazole was well tolerated, with pharmacokinetics similar to adults, and no significant safety concerns [\hyperlink{pmid_11095324}{PMID: 11095324}, T Andersson et al., 2000]. A randomized controlled trial in children with peptic ulcer and H. pylori infection (age not specified, but at a children's hospital) found no significant difference in adverse reactions between treatment groups, supporting safety [\hyperlink{pmid_33235586}{PMID: 33235586}, Shaohui Zhang et al.]. A study of intravenous omeprazole in critically ill children (1 month to 14 years) found no significant hemodynamic changes or adverse events, supporting its safety in this context [\hyperlink{pmid_22818224}{PMID: 22818224}, M J Solana et al., 2013]. Additional studies in children with severe esophagitis and those with bile reflux also reported good tolerability and no significant safety issues [\hyperlink{pmid_8320610}{PMID: 8320610}, T S Gunasekaran et al., 1993; \hyperlink{pmid_16641575}{PMID: 16641575}, Rok Orel et al., 2006].

Adverse Events:
- While most studies report good tolerability, there are rare case reports of serious adverse events, such as omeprazole-induced hepatitis in a child [\hyperlink{pmid_16096600}{PMID: 16096600}, Wael El-Matary et al., 2005]. However, this appears to be an isolated case and not representative of the general pediatric population.

Summary:
- For infants (<1 year), there is some evidence of safety from targeted studies, but the data are less extensive and some sources call for further research.
- For children (1-16 years), there is strong evidence from multiple targeted studies affirming the safety of omeprazole for short-term use in the treatment of GERD, esophagitis, and peptic ulcer disease.
- For long-term use, some studies note the need for further research, particularly regarding chronically elevated gastrin levels.

Therefore, based on the available abstracts, omeprazole is affirmed as safe for short-term use in children aged 1-16 years, with more limited but generally positive safety data in infants.

\subsection*{Abstracts}
\hypertarget{pmid_21694842}{O}meprazole is a proton-pump inhibitor indicated for gastroesophageal reflux disease and erosive esophagitis treatment in children. The aim of this review was to evaluate the efficacy of delayed-release oral suspension of omeprazole in childhood esophagitis, in terms of symptom relief, reduction in reflux index and/or intragastric acidity, and endoscopic and/or histological healing. We systematically searched PubMed, Cochrane and EMBASE (1990 to 2009) and identified 59 potentially relevant articles, but only 12 articles were suitable to be included in our analysis. All the studies evaluated symptom relief and reported a median relief rate of 80.4\% (range 35\%-100\%). Five studies reported a significant reduction of the esophageal reflux index within normal limits (<7\%) in all children, and 4 studies a significant reduction of intra-gastric acidity. The endoscopic healing rate, reported by 9 studies, was 84\% after 8-week treatment and 95\% after 12-week treatment, the latter being significantly higher than the histological healing rate (49\%). In conclusion, omeprazole given at a dose ranging from 0.3 to 3.5 mg/kg once daily (median 1 mg/kg once daily) for at least 12 weeks is highly effective in childhood esophagitis. [\hyperlink{Omeprazole}{PMID: 21694842}, Alice Monzani et al., 2010]

\hypertarget{pmid_30966193}{O}meprazole (OME) is employed for treating ulcer in children, but is unstable and exhibits first pass metabolism via the oral route. This study aimed to stabilize OME within mucoadhesive metolose (MET) films by combining cyclodextrins (CD) and l-arginine (l-arg) as stabilizing excipients and functionally characterizing for potential delivery via the buccal mucosa of paediatric patients. Polymeric solutions at a concentration of 1\%  [\hyperlink{Omeprazole}{PMID: 30966193}, Sajjad Khan et al., 2018] Omeprazole is a proton pump inhibitor that is used in acid suppression therapy in infants. Infants cannot swallow the oral tablets or capsules. Since, infants require a non-standard dose of omeprazole, the granules or tablets are often crushed or suspended in water or sodium bicarbonate, which may destroy the enteric coating. In this study we explore the efficacy and pharmacokinetics of rectally administered omeprazole in infants with gastroesophageal reflux disease (GERD) due to esophageal atresia (EA) or congenital diaphragmatic hernia (CDH) and compare these with orally administered omeprazole. Infants (6-12 weeks postnatal and bodyweight > 3 kg) with EA or CDH and GERD were randomized to receive a single dose of 1 mg/kg omeprazole rectally or orally. The primary outcome was the percentage of infants for whom omeprazole was effective according to predefined criteria for 24-h intraesophageal pH. Secondary outcomes were the percentages of time that gastric pH was < 3 or < 4, as well as the pharmacokinetic parameters. Seventeen infants, 4 with EA and 13 with CDH, were included. The proportion of infants for whom omeprazole was effective was 56\% (5 of 9 infants) after rectal administration and 50\% (4 of 8 infants) after oral administration. The total reflux time in minutes and percentages and the number of reflux episodes of pH < 4 decreased statistically significantly after both rectal and oral omeprazole administration. Rectal and oral administration of omeprazole resulted in similar serum exposure. A single rectal omeprazole dose (1 mg/kg) results in consistent increases in intraesophageal and gastric pH in infants with EA- or CDH-related GERD, similar to an oral dose. Considering the challenges with existing oral formulations, rectal omeprazole presents as an innovative, promising alternative for infants with pathological GERD. ClinicalTrials.gov Identifier: NCT00226044. [\hyperlink{Omeprazole}{PMID: 30966193}, Petra Bestebreurtje et al., 2020]

\hypertarget{pmid_8320610}{O}meprazole, a potent inhibitor of acid secretion, is effective in adults with severe gastroesophageal reflux, but no such data are available on children. We studied 15 children in whom treatment with histamine (type 2) blockers and prokinetic agents had failed; 4 had also had one or more fundoplications. Their ages were 0.8 to 17 years (mean, 8.1 years) and weights were 7.5 to 30.7 kg (mean, 18.6 kg). Of the 15 children, 8 were neurologically handicapped. All patients had endoscopic and histologic evidence of esophagitis; most had esophagitis grade 3 to 4. Patients were initially given omeprazole at 10 to 20 mg; the dose was titrated upward until results of a subsequent 24-hour intraesophageal pH study was normal. Symptoms and signs abated and evidence of esophagitis diminished in all patients. Omeprazole was given for periods of 5.5 to 26 months (mean, 12.2 months). The effective total dose was 20 to 40 mg (0.7 to 3.3 mg/kg) in 11 patients, 10 mg (0.7 mg/kg) in 1 patient, and 60 mg (1.9 to 2.4 mg/kg) in 3 patients. The dosage range was 0.7 to 3.3 to mg/kg per day (mean, 1.9 mg/kg). Mildly elevated transaminase values in 7 patients and elevated fasting gastrin levels in 11 patients were present; in 6 of the 11, gastrin levels were 3 to 5.5 times the upper limit of normal. We found omeprazole to be highly effective in this group of patients with severe esophagitis refractory to other measures. We recommend a starting dose of 0.7 mg/kg as a single morning dose; the adequacy of reflux control is then determined by follow-up 24-hour intraesophageal pH studies. Omeprazole appears to be safe for short-term use, but further studies are needed to assess long-term safety because the significance of chronically elevated gastrin levels in children is unknown. [\hyperlink{Omeprazole}{PMID: 8320610}, T S Gunasekaran et al., 1993]

\hypertarget{pmid_11095324}{T}he aim of this study was to examine the pharmacokinetics of orally administered omeprazole in children. Plasma concentrations of omeprazole were measured at steady state over a 6-h period after administration of the drug. Patients were a subset of those in a multicenter study to determine the dose, safety, efficacy, and tolerability of omeprazole in the treatment of erosive reflux esophagitis in children. Children were 1-16 yr of age, with erosive esophagitis and pathological acid reflux on 24 h-intraesophageal pH study. The "healing dose" of omeprazole was that at which subsequent intraesophageal pH study normalized. Children remained on this dose for 3 months, and during this period the pharmacokinetics were measured. A total of 57 children were enrolled in the overall healing phase of the study. Pharmacokinetic study was optional for subjects and was performed in 25 of the 57 enrolled. The doses of omeprazole required were substantially higher doses per kilogram of body weight than in adults. Values of the pharmacokinetic parameters of omeprazole were generally within the ranges previously reported in adults. However, the plasma levels, area under the plasma concentration versus time curve (AUC), plasma half-life (t(1/2)), and maximal plasma concentration (Cmax), were lower in the younger age group, when the AUC and Cmax were normalized to a dose of 1 mg/kg. Furthermore, within the group as a whole, these values showed a gradation from lowest in the children 1-6 yr of age to higher in the older age groups. The pharmacokinetics of omeprazole in children showed a trend toward higher metabolic capacity with decreasing age, being highest at 1-6 yr of age. This may explain the need for higher doses of omeprazole on a per kilogram basis, not only in children overall compared with adults but, in many cases, particularly in younger children. [\hyperlink{Omeprazole}{PMID: 11095324}, T Andersson et al., 2000]

\hypertarget{pmid_22818224}{C}ritical patients usually have hemodynamic disturbances which may become worse by the administration of some drugs. Omeprazole is a drug used in the prophylaxis of the gastrointestinal bleeding in these patients, but its cardiovascular effects are unknown. The objective was to study the hemodynamic changes produced by intravenous omeprazole in critically ill children and to find out if there are differences between two different doses of omeprazole. A randomized prospective observational study was performed on 37 critically ill children aged from 1 month to 14 years of age who required prophylaxis for gastrointestinal bleeding. Of these, 19 received intravenous omeprazole 0.5mg/kg every 12 hours, and 18 received intravenous omeprazole 1mg/kg every 12 hours. Intravenous omeprazole was administered in 20 minutes by continuous infusion pump. Heart rate, systolic, diastolic and mean arterial blood pressure, central venous pressure and ECG were recorded at baseline, and at 15, 30, 60 and 120 minutes of the infusion. There were no significant changes in the electrocardiogram, heart rate, blood pressure and central venous pressure. No patients required inotropic therapy modification. There were no differences between the two doses of omeprazole. Intravenous omeprazole administration of 0.5mg/kg and 1mg/kg is a hemodynamically safe drug in critically ill children. [\hyperlink{Omeprazole}{PMID: 22818224}, M J Solana et al., 2013]

\hypertarget{pmid_20719016}{T}o determine the role of omeprazole treatment in crying infants under the age of 1 year in whom acid gastroesophageal reflux is suspected and to study the evidence for efficacy, prescribing behaviour and side effects of this medicine, which is not registered for use in infants. Literature study. To assess efficacy we conducted a study of the literature using PubMed with the search terms 'gastro-esophageal reflux disease', 'crying', 'adverse drug reactions' and 'omeprazole', in the age category 'all infants 0-23 months' We used the medicine prescription database Interactie DataBase to assess prescribing data and studied reports of suspected side effects of omeprazole in children younger than 1 year to the Lareb Netherlands pharmacovigilance centre. We found 139 articles including 32 clinical trials. In only 3 of these was the efficacy of omeprazole studied in children under the age of 1 year. These studies showed that there was an effect on the acidity of the stomach, but not on symptoms. Although many side effects may occur during the use of omeprazole, few suspected side effects were reported to the Lareb Netherlands pharmacovigilance centre. Omeprazole is supplied in 10 mg amounts and it is therefore difficult to adjust dose to weight. Pharmacoepidemiological data show therefore that nearly all children receive 10 mg or multiples thereof. Given the age and corresponding weights we expected doses of 4-20 mg/day to be prescribed. It is uncertain whether acid reflux is the cause of crying in babies and, if reflux is suspected, whether omeprazole is the preferred treatment. [\hyperlink{Omeprazole}{PMID: 20719016}, Robert G T Blokpoel et al., 2010]

\hypertarget{pmid_32594306}{O}meprazole is a proton pump inhibitor (PPI) that is used in acid suppression therapy in infants. In this study we aimed to develop a pediatric omeprazole suppository, with good physical and chemical stability, suitable for pharmaceutical batch production. The composition of the suppository consisted of omeprazole, witepsol H15 and arginine (L) base. To achieve evenly distributed omeprazole suspension suppositories, the temperature, stirring rate, and arginine (L) base amount were varied. A previously validated quantitative high-performance liquid chromatography-ultraviolet method was modified and a long-term stability study was performed for one year. Evenly distributed omeprazole suspension suppositories were obtained by adding 100 mg arginine (L) base and pouring at a temperature of 34.7 °C and a stirring speed of 200 rpm. The long-term stability study showed no signs of discoloration and a stable omeprazole content between 90 and 110\% over 1 year if stored in the dark at room temperature. We developed a pediatric omeprazole suppository. This formulation may provide a good alternative to manipulated commercial or extemporaneously compounded omeprazole oral formulations for infants. Clinical studies are needed to establish efficacy and safety in this young population. [\hyperlink{Omeprazole}{PMID: 32594306}, Petra Bestebreurtje et al., 2020]

\hypertarget{pmid_16096600}{O}meprazole; the first proton pump inhibitor (PPI) showing an effective acid inhibitory ability, provides the satisfactory therapy either in gastro-esophageal reflux symptom relief or in healing of erosive esophagitis. It's also effective in peptic ulcer disease. Up to date, omeprazole efficacy and safety are well established in many trials. Omeprazole-related hepatotoxicity is not very well recognized especially in pediatric population. We report a child who developed hepatitis following omeprazole intake. We believe that this is the first case report of omeprazole-induced hepatitis in pediatric population. [\hyperlink{Omeprazole}{PMID: 16096600}, Wael El-Matary et al., 2005]

\hypertarget{pmid_8877348}{F}ollowing failure of conventional therapy for reflux oesophagitis, 15 children were treated with omeprazole 20 mg daily for a period of up to three months initially. Treatment resulted in a marked symptomatic improvement as measured by incidence of pain, vomiting, dysphagia and haematemesis. Four children failed treatment and required fundoplication. No complications from the use of omeprazole were recorded and some children have continued long-term treatment. [\hyperlink{Omeprazole}{PMID: 8877348}, P B Martin et al., 1996]

\hypertarget{pmid_33235586}{T}o compare curative effect and safety of omeprazole under different treatment courses in treatment of children with peptic ulcer (PU, diameter≤1.0cm) and helicobacter pylori (HP) infection and its influence on inflammatory cytokines. The study was a randomized controlled study and conducted at Baoding children's hospital from June 2015 to June 2018. In this study 100 PU children with positive HP were chosen and classified into two groups at random. The 58 cases in the observation group were given omeprazole + amoxicillin + clarithromycin, and the antibiotics were not used two weeks later. Then, omeprazole was used to treat for two weeks. 42 cases in the control group were given omeprazole + amoxicillin + clarithromycin for two weeks. Curative effect, HP eradication rate, clinical symptoms, incidence of adverse reactions, level of serum inflammatory cytokine interleukin-6 (IL-6) and level of tumor necrosis factor-a (TNF-a) in two groups were compared. Total effective rate, HP eradication rate and clinical symptom relief of observation group were better than those of control group, and the differences showed statistical significance (P>0.05). The differences of two groups in the incidence of adverse reactions had no statistical significance (P>0.05). Serum IL-6 level and TNF-a level of observation group were significantly lower than those of control group and before the treatment, and the differences had statistical significance (P>0.05). The application of omeprazole in treatment of PU patients with positive HP for four weeks can significantly improve PU cure rate and HP eradication rate, relieve clinical symptoms and reduce inflammatory response, so it deserves to be promoted clinically. [\hyperlink{Omeprazole}{PMID: 33235586}, Shaohui Zhang et al., ]

\hypertarget{pmid_9506656}{T}welve neurologically normal infants (age 2.9+/-0.9 months) with peptic esophagitis (grade 2) who did not respond to cimetidine (in addition to positioning, cisapride, and Gaviscon) were treated with omeprazole, 0.5 mg/kg once a day, for 6 weeks. The effectiveness of omeprazole was evaluated in all infants by clinical assessment and endoscopy before and after treatment and by 24-hour gastric pH monitoring during treatment in seven infants. Omeprazole therapy led to a marked decrease in symptoms, endoscopic and histologic signs of esophagitis, and intragastric acidity. [\hyperlink{Omeprazole}{PMID: 9506656}, P Alliët et al., 1998]

\hypertarget{pmid_12970637}{T}o assess the efficacy of omeprazole in treating irritable infants with gastroesophageal reflux and/or esophagitis. Irritable infants (n=30) 3 to 12 months of age met the entry criteria of esophageal acid exposure >5\% (n=22) and/or abnormal esophageal histology (n=15). They completed a 4-week, randomized, double-blind, placebo-controlled crossover trial of omeprazole. Cry/fuss diary (minutes/24 hours) and a visual analogue scale of infant irritability as judged by parental impression were obtained at baseline and the end of each 2-week treatment period. The reflux index fell significantly during omeprazole treatment compared with placebo (-8.9\%+/-5.6\%, -1.9\%+/-2.0\%, P<.001). Cry/fuss time decreased from baseline (267+/-119), regardless of treatment sequence (period 1, 203+/-99, P<.04; period 2, 188+/-121, P<.008). Visual analogue score decreased from baseline to period 2 (6.8+/-1.6, 4.8+/-2.9, P=.008). There was no significant difference for both outcome measures while taking either omeprazole or placebo. Compared with placebo, omeprazole significantly reduced esophageal acid exposure but not irritability. Irritability improved with time, regardless of treatment. [\hyperlink{Omeprazole}{PMID: 12970637}, David John Moore et al., 2003]

\hypertarget{pmid_14749542}{S}tudies of the pharmacokinetics of omeprazole in children with gastroesophageal reflux disease (GERD) remain scarce despite the vast number of reports on its efficacy. The objectives of this study were to assess the pharmacokinetics of omeprazole in healthy adults and in children with GERD. Omeprazole (Losec, delayed-release capsules) was administered orally to 18 healthy adults (mean age 36.8 years) and 12 children with GERD (mean age 6.1 years). Blood samples were collected over 5 hours, and plasma concentrations were assessed using liquid chromatography. Population pharmacokinetic parameters were calculated using NONMEM. A 1-compartment model with zero-order absorption and a lag time was used. The population approach was well suited to the limited number of samples available, and residual variability was low. Oral clearance (CL/F) and apparent volume of distribution (V(ss)/F) in healthy adults (Mean +/- SD: 0.62 +/- 0.27 L/h/kg and 0.76 +/- 0.26 L/kg, respectively) were not significantly different than those in children with GERD (0.51 +/- 0.34 L/h/kg and 0.66 +/- 0.25 L/kg, respectively). Healthy adults displayed a statistically significantly longer delay in drug absorption (Lag time: 0.62 +/- 0.15 hours) as compared with that observed in children with GERD (0.12 +/- 0.03 hours, P < 0.05). On the basis of these findings, omeprazole dosings on a milligram-per-kilogram basis are recommended with no further adjustments for the treatment of GERD in children. [\hyperlink{Omeprazole}{PMID: 14749542}, Jean-Francois Marier et al., 2004]

\hypertarget{pmid_9161946}{S}evere esophagitis is a rare complication of gastroesophageal reflux in children. In adults, omeprazole therapy of severe erosive esophagitis has become the gold standard short-term treatment of the disease. In children, data on its use are limited, and problems about the dosage are unresolved. The aim of this study was to evaluate the efficacy of a simplified, body-weight-based daily dosage of omeprazole in children with severe esophagitis. Ten children (median age 75.6 months; range 25-109 months) with severe esophagitis were prospectively investigated. All patients were evaluated by endoscopy, histology, and 24-h pH-metry study before and after 3 months of omeprazole. The starting dose of omeprazole was 20 mg as a single daily dose in children weighing less than 30 kg, and 40 mg daily for those weighing over 30 kg. A significant improvement in all the children was demonstrated after 3 months of treatment by clinical, endoscopic, and pH-metry assessment. However, histologic study failed to show significant improvement of both inflammatory and hyperplastic findings. Relapse occurred in six of 10 patients after discontinuation of therapy. Omeprazole is effective in the short-term treatment of severe oesophagitis in children. The daily dose of the drug could be easily based on the body weight. The persistence of histologic features of esophagitis in spite of clinical and endoscopic healing could be an indicator of poor outcome. [\hyperlink{Omeprazole}{PMID: 9161946}, C De Giacomo et al., 1997]

\hypertarget{pmid_7859807}{T}his study was undertaken to define the pharmacokinetics of omeprazole in children and included 13 patients, heterogeneous in terms of age (0.3 to 19 years), underlying disease and biological constants, indication of omeprazole administration and associated therapy. The dose administered ranged from 36.9 to 139 mg.1.73 m-2. The pharmacokinetic parameters of omeprazole were: systemic clearance, 0.23 l.kg-1.h-1; volume of distribution, 0.45 l.kg-1; elimination half life 0.86 h; but were highly variable between individuals. Dosage, differences in hepatic and renal function and associated therapy may contribute to inter-individual variability. Within the range of doses administered, the pharmacokinetic parameters were similar to those reported in adults. The drug has been well tolerated in all children. [\hyperlink{Omeprazole}{PMID: 7859807}, E Jacqz-Aigrain et al., 1994]

\hypertarget{pmid_9952234}{M}any children with esophagitis demonstrate histological changes without gross evidence of esophagitis by esophagoscopy. The effect of omeprazole on the histological healing of esophagitis in children is unknown. Therefore, the aim of this study was to determine the effect of omeprazole on refractory histological esophagitis in pediatric patients. Eighteen patients with histological evidence of esophagitis and recurrent symptoms despite therapy with H2-receptor antagonists and prokinetic agents were prospectively treated with omeprazole. Dosing was adjusted by monitoring intragastric pH, and esophagoscopy was repeated after 8-12 weeks of omeprazole treatment. Two patients did not complete the study due to either worsening symptoms or hypergastrinemia. Of the remaining patients, 76\% were asymptomatic with omeprazole treatment and 24\% reported improvement in their symptoms. Approximately 40\% demonstrated complete histological healing of their esophagitis. Three patients (17\%) had persistent elevations in serum gastrin levels while on omeprazole treatment, which was associated with both younger patient age and higher omeprazole dosing; however, all elevated gastrin levels returned to normal after discontinuation of the medication. All patients had recurrence of their symptoms after completing a course of omeprazole, even patients with complete histological healing. Omeprazole is efficacious in treating children with esophagitis refractory to H2-receptor antagonist and prokinetic agents. However, none of the patients were able to discontinue acid suppressive therapy even after documented healing of their esophagitis. [\hyperlink{Omeprazole}{PMID: 9952234}, R S Strauss et al., 1999]

\hypertarget{pmid_26398674}{A}lthough, omeprazole is widely used for treatment of gastric acid-mediated disorders. However, its pharmacokinetic and chemical instability does not allow simple aqueous dosage form formulation synthesis for therapy of, especially child, these patients. The aim of this study was at first preparation of suspension dosage form omeprazole and second to compare the blood levels of 2 oral formulations/dosage forms of suspension \& granule by high performance liquid chromatography (HPLC). The omeprazole suspension was prepared; in this regard omeprazole powder was added to 8.4\% sodium bicarbonate to make final concentration 2 mg/ml omeprazole. After that a randomized, parallel pilot trial study was performed in 34 pediatric patients with acid peptic disorder who considered usage omeprazole. Selected patients were received suspension and granule, respectively. After oral administration, blood samples were collected and analyzed for omeprazole levels using validated HPLC method. The mean omeprazole blood concentration before usage the next dose, (trough level) were 0.12±0.08 µg/ml and 0.18±0.15 µg/ml for granule and suspension groups, respectively and mean blood level after dosing (C2 peak level) were 0.68±0.61 µg/ml and 0.86±0.76 µg/ml for granule and suspension groups, respectively. No significant changes were observed in comparison 2 dosage forms 2 h before (P=0.52) and after (P=0.56) the last dose. These results demonstrate that omeprazole suspension is a suitable substitute for granule in pediatrics. [\hyperlink{Omeprazole}{PMID: 26398674}, S Karami et al., 2016]

\hypertarget{pmid_11113836}{T}o determine the efficacy, safety, and tolerability of omeprazole in children and to determine the doses required to heal chronic, severe esophagitis. Open multicenter study in children aged 1 to 16 years with erosive reflux esophagitis. The healing dose of omeprazole used was that with which the duration of acid reflux was <6\% of a 24-hour intraesophageal pH study. Follow-up endoscopy was performed after 3 months of treatment with the healing dose. At entry, two thirds of 57 patients who completed the study had esophagitis grade 3 or 4 (scale 0-4); some 50\% had neurologic impairment or repaired esophageal atresia. Of the 57 patients, 54 healed; 3 did not heal and left the study, and 3 healed with a second course. Doses required for healing were 0.7 to 3.5 mg/kg/d: 0.7 mg/kg/d in 44\% of patients and 1.4 mg/kg/d in another 28\%. Healing dose correlated with grade of esophagitis but not with age or underlying disease. Reflux symptoms improved dramatically in almost all of the 57 patients, including the unhealed patients. Omeprazole is well tolerated, highly effective, and safe for treatment of erosive esophagitis and symptoms of gastroesophageal reflux in children, including children in whom antireflux surgery or other medical therapy has failed. On a per-kilogram basis, the doses of omeprazole required to heal erosive esophagitis are much greater than those required for adults. [\hyperlink{Omeprazole}{PMID: 11113836}, E Hassall et al., 2000]

\hypertarget{pmid_15773802}{I}n last years the use in the pediatric area of proton pump inhibitors (omeprazole, lansoprazole, pantoprazole, rabeprazole and esomeprazole) is more often, nevertheless the clinical trials carried out are poor. The aim of this work is to analyse the bibliography published about this kind of drugs in children and to make a revision of its use in the last seven years. More studies with omeprazole and lansoprazole have been developed, to be exact omeprazole and lansoprazole is present in 122 bibliographic appointments and 34 for lansoprazole, which include studies that demonstrate a good tolerance and efficacy. The remaining proton pump inhibitors count with very few studies. The main therapeutic indications were the eradication of Helicobacter pylori, gastroesophageal reflux disease and esophagitis. The number of patients included in the reviewed studies is quite heterogeneous, from 8 to 122 and the age range between 8 days and 17 years. On the other hand, it could be highlighted the non-existence of formulations adapted to the pediatric population and the difficulty of administration specially in the youngest patients. As in many other drugs, it would be necessary to carry out clinical trials in order to determinate the pharmacologic parameters at difference ages, which will allow a safe and effective administration, and its authorization by all Health Authorities. [\hyperlink{Omeprazole}{PMID: 15773802}, J Carcelén Andrés et al., ]

\hypertarget{pmid_18797857}{I}n some cases of drug therapy, the available evidence might be sufficient to extend the indications to children without further clinical studies. We reviewed the available evidence for one of the categories of drugs most frequently used off-label in children: proton pump inhibitors (PPIs) used for the treatment of gastroesophageal reflux disease (GERD). A classification of the appropriateness of off-label use of PPIs in children with GERD was also performed. Of the five PPIs evaluated, only omeprazole has a paediatric indication in Europe. Overall, 19 clinical trials were retrieved and evaluated on the basis of pharmacokinetics, efficacy and safety data. The off-label use of omeprazole, esomeprazole and lansoprazole in children was evaluated as appropriate given the consistent available evidence retrieved in literature. This study demonstrates the existence of a large body of clinical evidence on the use of PPIs in children. Regulatory agencies and ethical committees should cope with this issue for ethical reasons to avoid unnecessary trial replication. [\hyperlink{Omeprazole}{PMID: 18797857}, Giovanni Tafuri et al., 2009]

\hypertarget{pmid_32866648}{O}meprazole (OME) is often used to treat disorders associated with gastric hypersecretion in children but a liquid pediatric formulation of this medicine is not currently available. The aim of this study is to develop OME loaded nanoparticles with a view to the obtention of a liquid pharmaceutical dosage form. Eudragit® RS100 was selected as the skeleton material in the inner core and pH-sensitive Eudragit® L100-55 was used as the outer coating of the nanoparticles prepared by the nanoprecipitation method. Pharmacological activity was evaluated by induction of ethanol ulcers in mice. The OME nanoparticles exhibited mean diameters of 174 nm (±17), polydispersity index of 0.229 (±0.01), zeta potential values of -13 mV (±2.60) and encapsulation efficiency of 68.1\%. The in vivo pharmacological assessment showed the ability of nanoparticles to protect mice stomach against ulcer formation. The prepared suspension of OME nanoparticles represents effective therapeutic strategy in a liquid pharmaceutical form with the possibility of pediatric administration. [\hyperlink{Omeprazole}{PMID: 32866648}, Helissara Silveira Diefenthaeler et al., 2020]

\hypertarget{pmid_34607935}{T}he over-the-counter nasal decongestant oxymetazoline (eg, Afrin) is used in the pediatric population for a variety of conditions in the operating room setting. Given its vasoconstrictive properties, it can have cardiovascular adverse effects when systemically absorbed. There have been several reports of cardiac and respiratory complications related to use of oxymetazoline in the pediatric population. Current US Food and Drug Administration approval for oxymetazoline is for patients ≥6 years of age, but medical professionals may elect to use it short-term and off label for younger children in particular clinical scenarios in which the potential benefit may outweigh risks (eg, active bleeding, acute respiratory distress from nasal obstruction, acute complicated sinusitis, improved surgical visualization, nasal decongestion for scope examination, other conditions, etc). To date, there have not been adequate pediatric pharmacokinetic studies of oxymetazoline, so caution should be exercised with both the quantity of dosing and the technique of administration. In the urgent care setting, emergency department, or inpatient setting, to avoid excessive administration of the medication, medical professionals should use the spray bottle in an upright position with the child upright. In addition, in the operating room setting, both monitoring the quantity used and effective communication between the surgeon and anesthesia team are important. Further studies are needed to understand the systemic absorption and effects in children in both nonsurgical and surgical nasal use of oxymetazoline. [\hyperlink{Omeprazole}{PMID: 34607935}, Richard Cartabuke et al., 2021]

\hypertarget{pmid_2691312}{O}meprazole is a very potent inhibitor of gastric acid secretion and has proven to be efficacious in the healing of peptic ulcer and reflux oesophagitis. A search for adverse events during short-term treatment with omeprazole has been made, based on data from published comparative trials, data on file at the manufactor's (Hässle Research Laboratories, Mölndal, Sweden) and personal series. Omeprazole does not show more adverse events than drugs currently widely in use for the treatment of acid-related disorders. A change in a wide range of laboratory parameters has not been observed, except for a rise in basal and meal-stimulated serum gastrin which can be ascribed directly to the inhibition of acid secretion. For short-term treatment omeprazole can be considered as a safe drug. [\hyperlink{Omeprazole}{PMID: 2691312}, G F Nelis et al., 1989]

\hypertarget{pmid_16641575}{R}eflux of duodenal juice into the oesophagus has a role in the pathogenesis of both oesophageal and laryngopharyngeal inflammatory and neoplastic lesions. As little is known about effective therapy, we studied the effect of proton pump inhibitor therapy on oesophageal bile reflux in children. Twenty-nine children with moderate to severe erosive oesophagitis and abnormal oesophageal bile reflux were studied before and after treatment with omeprazole 1 mg/kg per day. Outcomes included a clinical symptom score, oesophageal acid and bile reflux (simultaneous 24-hour pH and Bilitec 2000 monitoring), and mucosal healing. After 8 weeks of therapy, 17 (59\%) of the patients were symptom-free, and 5 (17\%) had minimal symptoms. Mucosal healing or reduction to low-grade oesophagitis was achieved in 25 children (86\%; P < 0.0005). Mean percentages of total, upright, and supine time with oesophageal pH less than 4 were reduced from 17.0\%, 16.8\%, and 19.2\% before treatment, to 2.83\%, 3.17\%, and 2.07\%, respectively, after treatment (all P < 0.00001). Similarly, mean percentages of total, upright, and supine time with bile reflux were reduced from 16.96\%, 12.67\%, and 22.0\%, to 2.27\%, 1.91\%, and 2.23\%, respectively (P < 0.000001, P < 0.0001, and P < 0.000001, respectively). Omeprazole 1 mg/kg per day is an effective therapy for the majority of children with severe erosive oesophagitis due to abnormal isolated bile reflux or combined acid and bile reflux. It remains unclear how patients with treatment-resistant bile reflux should be managed. [\hyperlink{Omeprazole}{PMID: 16641575}, Rok Orel et al., 2006]

\section*{Bafiertam}
\subsection*{Result}
\subsubsection*{Answer}

Unknown

\subsubsection*{{Explanation}}
\hypertarget{Bafiertam}
Based on the abstracts available, there is no evidence regarding the safety of Bafiertam in children. None of the abstracts mention Bafiertam or present targeted studies evaluating its safety in pediatric populations. Therefore, the safety of Bafiertam for use in children is unknown according to the data provided.

\subsection*{Abstracts}
\hypertarget{pmid_2810744}{C}efteram pivoxil (CFTM-PI, T-2588), a new oral cephalosporin antibiotic, was evaluated for its safety and efficacy in children. Fifty-three patients with bacterial infections were treated with 8.3 to 21.8 mg/kg/day of CFTM-PI. The drug was very effective (efficacy rate 98.1\%) in pharyngitis, otitis media, tracheobronchitis, pneumonia and skin infections. The antibacterial potency was very good against Haemophilus influenzae, Streptococcus pyogenes and Streptococcus pneumoniae, and was comparable with cephalexin against Staphylococcus aureus. No severe adverse reaction was encountered with the CFTM-PI therapy. The data suggest that CFTM-PI is a safe and effective antibiotic when used in children with susceptible bacterial infections. [\hyperlink{Bafiertam}{PMID: 2810744}, H Meguro et al., 1989]

\hypertarget{pmid_2810745}{T}he usefulness of a new cephem antibiotic, cefteram pivoxil (CFTM-PI, T-2588), was evaluated in the field of pediatrics. 1. Thirty-one patients were enrolled in the study. They included 20 boys and 11 girls with ages 8 months to 8 years 7 months. 2. The patients were treated with CFTM-PI at a dose levels ranging 3.0-4.2 mg/kg in 3 divided portions. The administration was done orally for a duration between 3 and 13 days, with a total dose between 27.3 and 130.0 mg/kg. 3. The patients were diagnostically classified into the following categories: 11 with acute pharyngitis, 1 with acute nasopharyngitis, 6 with acute tonsillitis, 9 with acute bronchitis, 2 with scarlet fever, 1 with purulent parotitis and 1 with purulent cervical lymphadenitis. 4. Clinical responses to the treatment were excellent in 14, good in 13, fair in 1 and poor in 3 patients with an overall efficacy rate of 87.1\%. 5. Bacteriological responses were as follows: of 29 bacterial strains presumed to be pathogens, 23 were eradicated, 2 unchanged and 4 unknown, with an eradication rate of 92.0\%. 6. Neither adverse reactions nor abnormal changes in laboratory test values were observed with the medication in any patients during and after the end of the treatment. None of the patients refused the medication. CFTM-PI was found very effective and safe in treating pediatric infections: it is a useful drug in the field of pediatrics. [\hyperlink{Bafiertam}{PMID: 2810745}, H Hirosawa et al., 1989]

\hypertarget{pmid_2810756}{C}efteram pivoxil (CFTM-PI, T-2588), a new oral cephem antibiotic of ester type, was evaluated for its safety, efficacy and pharmacokinetics. 1. One child, 4 years of age (18 kg body weight), was administered orally 3 mg/kg after meal. The peak serum level of CFTM was 0.78 microgram/ml after 2 hours, and cumulative urinary excretion rate during the first 6 hours was 15.0\%. 2. Clinical studies on CFTM-PI were carried out in 17 pediatric patients; 1 with acute pharyngitis, 2 with acute tonsillitis, 1 each with pertussis, acute bronchitis, 2 with broncho-pneumonia, 4 with scarlatina, 3 with acute otitis media, and 1 each with lymphadenitis, acrobystitis and urinary tract infection. Clinical responses were excellent in 9, good in 6, fair in 1, poor in 1, and the overall clinical efficacy rate was 88.2\%. 3. Bacteriological efficacy was investigated with 10 strains of 5 species (Streptococcus pyogenes 4, Streptococcus pneumoniae 2, Haemophilus influenzae 2, Enterococcus and Bacteroides 1) isolated from 9 patients. All strains were eradicated. 4. As to adverse reactions, mild diarrhea was observed in 1 patient. But therapy had to be continued without procedure and the diarrhea disappeared after 6 days. No adverse hematological, renal or hepatic effects were noted. [\hyperlink{Bafiertam}{PMID: 2810756}, M Minamitani et al., 1989]

\hypertarget{pmid_2810758}{T}he clinical efficacy and the safety of cefteram pivoxil granule (CFTM-PI, T-2588), a newly prepared drug for pediatric use, were performed. A total of 60 patients with ages between 6 months and 14 years 3 months with pediatric infections were medicated with CFTM-PI at dose levels of 3.2-9.9 mg/kg 3 times daily for 3-11 days. Clinical responses to the drug were excellent in 3 of 3 patients with acute pharyngitis, excellent in 14, good in 5 and poor in 2 of 21 patients with acute purulent tonsillitis, excellent in 1 and good in 2 of 3 patients with acute bronchitis, excellent in 16 and good in 8 of 24 patients with acute pneumonia, excellent in 3 and good in 1 of 4 patients with acute urinary tract infection and excellent in 2 of 2 patients with acute purulent lymphadenitis, hence the overall clinical efficacy rate was 96.5\% in a total of 57 patients. Bacteriological responses to the drug were as follows: Eradicated, 8 strains of Streptococcus pyogenes, 3 strains of Streptococcus pneumoniae, 19 strains of Haemophilus influenzae (beta-lactamase positive; 7, beta-lactamase negative; 12), 1 strain of Haemophilus parainfluenzae (beta-lactamase positive) and 4 strains of Escherichia coli (beta-lactamase positive; 1, beta-lactamase negative; 3), decreased, 1 strain of S. pyogenes, hence the eradication rate was 97.2\%. No side effects were encountered in any of the patients but for 3 who had diarrhoea and 1 who had loose stool, though these changes were slight. As abnormal laboratory test data, elevation of GOT was noted in 1 case, thrombocytosis and elevation of GPT in another. Also, none of the patients refused or complained of difficulty in intaking of the drug via oral route. In conclusion, CFTM-PI appeared to be a safe and highly effective antibiotic against pediatric infections. [\hyperlink{Bafiertam}{PMID: 2810758}, N Iwai et al., 1989]

\hypertarget{pmid_12943481}{I}n the US, 6\% sulfur in petrolatum has been the most frequently administered treatment for infantile scabies. It appears to be safe but there is no literature containing a large series of patients on which to base that determination. In the UK, benzyl benzoate is the approved product. Benzyl benzoate is rarely used in the US at the present time. 5\% Permethrin is an excellent substitute and has many advantages. It appears to be quite safe in infants, although it is more expensive than other products. It remains present on the skin for several days, therefore protecting against reinfestation. Ivermectin is a systemic drug which is assumed to be safe in infants, although it requires repeated doses and does not protect against reinfestation. In the opinion of the author, 5\% permethrin is the best treatment for scabies in infants and young children. [\hyperlink{Bafiertam}{PMID: 12943481}, Mervyn L Elgart et al., 2003]

\hypertarget{pmid_9045581}{B}eforal was used for anesthesia in 21 children aged 4 to 14 years and weighing 15 to 65 kg. Surgical interventions lasted for 2.5 to 9 hours. The basic dose of beforal was 0.5 to 1 mg/kg. Anesthesia was maintained with half or third of the basic dose in combination with inhalations of nitrogen oxide and relanium (diazepam) in a dose of 0.3 to 0.4 mg/kg. Beforal as the only analgesic component of anesthesia effectively protected the organism of a child from surgical shock and its use was associated with stability of the studied hemodynamic parameters. [\hyperlink{Bafiertam}{PMID: 9045581}, E A Medvedeva et al., ]

\hypertarget{pmid_22928704}{O}romucosal midazolam (Buccolam™), a benzodiazepine, is approved in the EU for the treatment of paediatric patients (aged 3 months to <18 years) with acute, prolonged, convulsive seizures. In clinical trials in children with prolonged, acute, convulsive seizures, oromucosal midazolam was at least as effective as rectal diazepam and as effective as intravenous diazepam in the treatment of seizures and was generally well tolerated. It has several advantages over rectal diazepam, the previous gold standard of treatment, such as having a more convenient and socially acceptable administration route. [\hyperlink{Bafiertam}{PMID: 22928704}, Lesley J Scott et al., 2012]

\hypertarget{pmid_3970344}{I}ntra-operative intramuscular injections of either papaveretum 0.3 mg/kg or nefopam 0.4 mg/kg were given to alternate patients to promote smooth emergence from ENT anaesthesia in a consecutive series of 40 children. Observations over a period of 30 minutes following completion of surgery showed that emergence was satisfactory in 19 out of 20 children given papaveretum, and in 15 out of 20 children given nefopam. The study confirms that this indication for papaveretum is justifiably popular and that nefopam is a useful alternative mainly because it does not cause respiratory depression. There are no previous reports of the use of nefopam in children. [\hyperlink{Bafiertam}{PMID: 3970344}, J G Hannington-Kiff et al., 1985]

\hypertarget{pmid_2810739}{P}harmacokinetic and clinical studies of cefteram pivoxil (CFTM-PI, T-2588) fine granules in children were performed and the following results were obtained. 1. Peak serum concentrations in 4 children given orally a dose of 3 mg/kg and 2 children given orally a dose of 6 mg/kg after meal were reached in 3 to 4 hours and the concentration curves were dependent on dosage levels. The urinary recovery rates up to 8 hours were 29.7\% in children given a dose of 3 mg/kg and 29.7\% in children given a dose of 6 mg/kg. 2. Clinical efficacies were evaluated in 38 patients with bacterial infections. Twenty seven patients were given each doses of 3 mg/kg in 3 times a day and other 11 patients each doses of 6 mg/kg in 3 times. Clinical effects of CFTM-PI were excellent in 18, good in 19, fair in 1 case, hence the overall clinical efficacy rate was 97.4\%. 3. Bacteriologically, 24 strains of causative organisms were isolated. The overall bacteriological eradication rate was 81.8\%. Antimicrobial activities were excellent especially against Streptococcus pyogenes, Streptococcus pneumoniae, and Haemophilus influenzae. 4. As for the side effects, slight loose stools were observed in 2 cases, and in laboratory tests, elevations of GOT and GPT were observed in 1 case and an elevation of eosinophil was observed in 1 case. But no one needed any treatment. 5. CFTM-PI is a useful and safe oral antibiotic for the treatment of bacterial infections in pediatrics. [\hyperlink{Bafiertam}{PMID: 2810739}, G Suzuki et al., 1989]

\hypertarget{pmid_36324821}{T}o assess and summarize current evidence on the effectiveness and safety of ertapenem for treatment of childhood infections, in consideration of high infection prevalence in children and wide use of ertapenem. The following 8 databases were searched on 13th May 2021: Web of Science, Embase via Ovid SP, PubMed, The Cochrane Library (CENTRAL), Chinese BioMedical Literature Database (CBM), China National Knowledge Infrastructure (CNKI), VIP and Wanfang. The primary outcome was treatment success rate. Risk ratios (RRs) and 95\% confidence interval (CI) were estimated using random-effect models. Subgroup analysis was conducted where heterogeneity was found. Fifteen studies (8 randomized controlled trials, 1 observational comparative study, and 6 before and after studies) involving 2,528 patients were included in the final review. Ertapenem had similar treatment success rates with β-lactam antibiotics [relative risk (RR) = 1.08, 95\% CI: 0.99-1.19]. In a subgroup analysis, similar efficacy (RR = 1.08, 95\% CI: 0.97-1.20) between ertapenem and other carbapenems. Compared with β-lactam antibiotics, ertapenem did not increase the risk of any adverse events (RR = 1.02, 95\%CI: 0.71-1.48), drug-related diarrhea (all non-Asian children, RR = 0.62, 95\%CI: 0.31-1.25), or injection site pain (all non-Asian children, RR = 1.66, 95\%CI: 0.59-4.68). Subgroup analysis showed no obvious difference between ertapenem group and carbapenems or non-carbapenems group on risk of adverse events. Our findings suggest that ertapenem is effective and safe in treatment for children with infection. Further comparative real-world data is needed to supplement clinical evidence on the overall benefits of ertapenem in this population. [\hyperlink{Bafiertam}{PMID: 36324821}, Ruiqiu Zhao et al., 2022]

\hypertarget{pmid_2810736}{T}wenty-six children were treated with cefteram pivoxil (CFTM-PI) and the clinical efficacy and side effects were evaluated. Ages of the patients ranged from 8 months to 9 years. Doses of CFTM-PI ranged 7.5-20.1 mg/kg/day for 4 to 19 days. The twenty-six patients including 10 patients with tonsillitis, 1 pharyngitis, 3 otitis media, 2 scarlet fever, 1 bronchopneumonia, 1 lymphadenitis, 6 urinary tract infections, 1 vaginitis and 1 staphylococcal scalded skin syndrome were evaluated for clinical efficacy. Results were excellent in 11, good in 13, and fair in 2 patients. Out of the 26 patients, one case showed elevated GOT and GPT, and another case showed elevated GOT. The pharmacokinetic study of CFTM-PI was performed in 9 fasting patients whose ages ranged from 2 to 11 years. Serum peak concentrations of CFTM were 0.92 to 1.05 micrograms/ml (mean 0.99 microgram/ml) at 1 to 2 hours after a dose of 1.5 mg/kg each to 3 patients, 1.12 to 1.38 micrograms/ml (mean 1.25 micrograms/ml) after a dose of 3 mg/kg each to 3 patients and 0.66 to 2.1 micrograms/ml (mean 1.17 micrograms/ml) after a dose of 6 mg/kg each to 3 patients. The portions of the drug excreted into urine within 8 hours were 8.9\% and 14.7\% in 2 patients, each given a dose of 1.5 mg/kg, from 13.0 to 23.1\% (mean 18.4\%) in 3 patients, each given a dose of 3 mg/kg, and 6.3\% and 8.7\% in 2 patients, each given a dose of 6 mg/kg. [\hyperlink{Bafiertam}{PMID: 2810736}, H Sakata et al., 1989]

\hypertarget{pmid_25535540}{P}roviding a safe and efficient dental treatment for a young patient is a challenge for the dentist and the child. The purpose of this study was to investigate the effectiveness, safety and acceptability of buccal midazolam in dental pediatric patients and to compare it with oral Midazolam. Eighteen uncooperative healthy children aged 2.5-6 years were randomized to each of buccal midazolam (0.3mg/kg) or oral midazolam (0.5mg/kg) at the first visit, the alternative has been used at the second visit in a cross-over manner. The study took place at pediatric dentistry clinic of Shahed University, Tehran, from November 2011 to June 2012. The patients' vital signs and behavioral scores were recorded. The patient, the operator and the observer were blinded to the applied medication. Post operatively, patients' and parents' satisfaction were assessed by Visual Analogue Score and a questionnaire respectively. The P-value was set at 0.05 for significance level. There were no significant differences in physiologic factors in the medication groups at time 0, 10, 20, 30 minutes and discharge. There was also no significant difference between the two groups in behavioral parameters. The majority of parents rated both sedative agents as "effective" or "very effective" and their children mostly were without anxiety or with minor anxiety. Buccal midazolam may be safely and efficiently used in sedation of pediatric dental patients. [\hyperlink{Bafiertam}{PMID: 25535540}, Sara Tavassoli-Hojjati et al., 2014]

\hypertarget{pmid_28333952}{E}nvironmental factors play a major role on atopic dermatitis (AD) which shows a constant rise in prevalence in western countries over the last decades. The Hygiene Hypothesis suggesting an inverse relationship between incidence of infections and the increase in atopic diseases in these countries, is one of the working hypothesis proposed to explain this trend. This study tested the efficacy and safety of oral administration of the bacterial lysate OM-85 (Broncho-Vaxom®, Broncho-Munal®, Ommunal®, Paxoral®, Vaxoral®), in the treatment of established AD in children. Children aged 6 months to 7 years, with confirmed AD diagnosis, were randomized in a double-blind, placebo-controlled trial to receive, in addition to conventional treatment with emollients and topical corticosteroids, 3.5mg of the bacterial extract OM-85 or placebo daily for 9 months. The primary end-point was the difference between groups in the occurrence of new flares (NF) during the study period, evaluated by Hazard Ratio (HR) derived from conditional Cox proportional hazard regression models accounting for repeated events. Among the 179 randomized children, 170 were analysed, 88 in the OM-85 and 82 in the placebo group. As expected most children in both treatment groups experienced at least 1 NF during the study period (75 (85\%) patients in the OM-85 group and 72 (88\%) in the placebo group). Patients treated with OM-85 as adjuvant therapy had significantly fewer and delayed NFs (HR of repeated flares = 0.80; 95\% confidence interval (CI): 0.67-0.96), also when potential confounding factors, as family history of atopy and corticosteroids use, were taken into account (HR = 0.82; 95\% CI: 0.69-0.98). No major side effect was reported, with comparable and good tolerability for OM-85 and placebo. Results show an adjuvant therapeutic effect of a well standardized bacterial lysate OM-85 on established AD. [\hyperlink{Bafiertam}{PMID: 28333952}, Christine Bodemer et al., 2017]

\hypertarget{pmid_2810747}{C}efteram pivoxil (CFTM-PI, T-2588) was administered to pediatric patients with acute infectious diseases, and a summary of the results obtained are as follows. 1. Pharmacokinetic parameters determined in 1 patient appears to be comparable to those of adults. 2. Clinical efficacy was studied on 16 children with acute urinary tract infections, acute upper respiratory infections, acute sinusitis, acute otitis media and acute cervical lymphadenitis. Responses to the treatment were excellent in 7 (43.8\%) and good in 9 (56.3\%). 3. No adverse reactions were noted in this study. [\hyperlink{Bafiertam}{PMID: 2810747}, M Fujita et al., 1989]

\hypertarget{pmid_16023510}{R}ectal diazepam and buccal midazolam are used for emergency treatment of acute febrile and afebrile (epileptic) seizures in children. We aimed to compare the safety and efficacy of these drugs. A multicentre, randomised controlled trial was undertaken to compare buccal midazolam with rectal diazepam for emergency-room treatment of children aged 6 months and older presenting to hospital with active seizures and without intravenous access. The dose varied according to age from 2.5 to 10 mg. The primary endpoint was therapeutic success: cessation of seizures within 10 min and for at least 1 hour, without respiratory depression requiring intervention. Analysis was per protocol. Consent was obtained for 219 separate episodes involving 177 patients, who had a median age of 3 years (IQR 1-5) at initial episode. Therapeutic success was 56\% (61 of 109) for buccal midazolam and 27\% (30 of 110) for rectal diazepam (percentage difference 29\%, 95\% CI 16-41). Analysing only initial episodes revealed a similar result. The rate of respiratory depression did not differ between groups. When centre, age, known diagnosis of epilepsy, use of antiepileptic drugs, prior treatment, and length of seizure before treatment were adjusted for with logistic regression, buccal midazolam was more effective than rectal diazepam. Buccal midazolam was more effective than rectal diazepam for children presenting to hospital with acute seizures and was not associated with an increased incidence of respiratory depression. [\hyperlink{Bafiertam}{PMID: 16023510}, John McIntyre et al., ]

\hypertarget{pmid_28292340}{T}he purpose of this study was to evaluate, using a randomized, double-blind methodology: (1) the safety of phentolamine mesylate (Oraverse) in accelerating the recovery of soft tissue anesthesia following the injection of two percent lidocaine plus 1:100,000 epinephrine in two- to five-year-olds; and (2) efficacy in four- to five-year-olds only. One hundred fifty pediatric dental patients underwent routine dental restorative procedures with two percent lidocaine plus 1:100,000 epinephrine with doses based on body weight. Phentolamine mesylate or a sham injection (two to one ratio) was then administered. Subjects were monitored for safety and, in four- to five-year-olds, for efficacy during the two-hour evaluation period. There were no significant differences in adverse events between the phentolamine and sham injections. Compared to sham, phentolamine was not associated with nerve injury, increased analgesic use, or abnormalities of the oral cavity. Phentolamine was associated with transient decreased blood pressure in some children. In four- and five-year-olds, phentolamine induced more rapid recovery of lip anesthesia by 48 minutes (P<0.0001). Phentolamine was well tolerated and safe in three- to five-year-olds; in four- to five-year-olds, a statistically significant more rapid recovery of lip sensation compared to sham injections was determined. [\hyperlink{Bafiertam}{PMID: 28292340}, Elliot V Hersh et al., 2017]

\hypertarget{pmid_16028153}{B}ecause of concerns about arthrotoxicity, fluoroquinolones are restricted for use in children. This study describes the safety and efficacy of gatifloxacin when used for treatment of children with recurrent acute otitis media (ROM) or acute otitis media (AOM) treatment failure (AOMTF). We performed an analysis of 867 children included in 4 clinical trials who had ROM and/or AOMTF and were treated with gatifloxacin (10 mg/kg once daily for 10 days). Gatifloxacin had adverse event rates that were similar overall to those of a comparator antibiotic (amoxicillin-clavulanate), except for increased diarrhea in children <2 years old receiving amoxicillin-clavulanate. There was no evidence of arthrotoxicity, hepatotoxicity, alteration of glucose homeostasis, or central nervous system toxicity acutely or during 1 year follow-up in any child. Regarding efficacy, in 2 noncomparative trials, the gatifloxacin cure rate of AOM was 89\% (95\% confidence interval [CI], 83\%-95\%) at the test of cure (TOC) visit, 3-10 days after completion of therapy. In 2 comparative trials of gatifloxacin versus amoxicillin-clavulanate, the efficacy of gatifloxacin was 88\% (95\% CI, 82\%-94\%). Gatifloxacin led to better clinical outcomes than amoxicillin-clavulanate for AOMTF (91\% vs. 81\%; P=.029), for AOMTF and age <2 years old (89\% vs. 69\%; P=.009), and for severe AOM in children <2 years old (90\% vs. 75\%; P=.012). Among children with AOMTF previously treated with amoxicillin-clavulanate or ceftriaxone injections, gatifloxacin cure rates were high (88\% and 75\%, respectively). Gatifloxacin appears to be safe for children, with no evidence of producing arthrotoxicity in 867 children exposed to the antibiotic when used as treatment for ROM and AOMTF. [\hyperlink{Bafiertam}{PMID: 16028153}, Michael E Pichichero et al., 2005]

\hypertarget{pmid_7799115}{T}he safety and efficacy of a new sedation technique for children with facial injuries in the emergency department were prospectively evaluated. Thirty-seven children between the ages of 12 months and 7 years old who required sedation for minor surgical procedures were administered an intramuscular injection of ketamine (3 mg/kg), midazolam (0.05 mg/kg), and glycopyrrolate (0.005 mg/kg). A second 1-mg/kg intramuscular injection of ketamine alone was given if needed. Pulse rate, cardiac rhythm, respiratory rate, oxygen saturation, side effects, and behavior were recorded. Satisfactory sedation was achieved after a single injection in 32 children; five others required a second ketamine injection (1 mg/kg). Onset of anesthesia occurred within 6 minutes in 73\% of the children who received one injection, and there were generally adequate working conditions for 30 minutes. The average time from initial injection to discharge was 76 minutes. Results of physiologic monitoring, behavioral ratings, and side effects are reported. Emergence delirium and hallucinations were not observed. Ketamine reliably produced dissociative anesthesia without loss of respiratory drive or protective airway tone. Midazolam reduced the incidence of ketamine-induced dysphoric reactions and muscular hypertonicity. The use of intramuscular ketamine, midazolam, and glycoyrrolate is a safe, effective, and practical approach to managing selected pediatric injuries in the emergency department. Advanced airway management proficiency is recommended for use of this technique. [\hyperlink{Bafiertam}{PMID: 7799115}, J W Pruitt et al., 1995]

\hypertarget{pmid_22378696}{A}nti-seizure prophylaxis is routinely utilized during busulfan administration for HSCT. We evaluated the feasibility and efficacy of levetiracetam in children undergoing HSCT. A total of 28 children and young adults received levetiracetam during HSCT and the outcomes and costs were compared to a historical, but similar cohort of 25 patients who had received fosphenytoin. Levetiracetam was well tolerated and was efficacious in preventing seizures. Cost of drug, administration, and monitoring were also similar among the two groups. Due to non-induction of the hepatic cytochrome P450 enzymes, levetiracetam may lead to better dose assurance of busulfan in targeted dose regimens for HSCT. [\hyperlink{Bafiertam}{PMID: 22378696}, Sandeep Soni et al., 2012]

\hypertarget{pmid_3877456}{A}lthough it is used extensively in Europe, there is a limited amount of published data concerning pediatric clinical experience with cefuroxime in the United States. Thirty-six children, ranging from 3.5 to 57 months of age, received intravenous cefuroxime (75 mg/kg/day in three divided doses) for soft-tissue infections of the face or epiglottis. Infections treated included preseptal (19 patients) and buccal (13 patients) cellulitis and epiglottitis (four patients). Blood cultures were positive in 22 patients, yielding Haemophilus influenzae type b in 17 (four were beta-lactamase-positive), Streptococcus pneumoniae in four; and beta-lactamase-positive, nontypable H influenzae in one. An additional five patients with buccal cellulitis had negative blood cultures but H influenzae type b antigenuria. A satisfactory clinical response was noted in all patients, and repeated blood cultures performed in initially bacteremic patients were sterile. Cefuroxime therapy was well tolerated, and abnormal laboratory results were infrequent, except for absolute granulocytopenia (granulocytes, less than 1,500/cu mm), which occurred in six patients but could not be ascribed to a drug effect because of the uncontrolled design of our study. Treatment with cefuroxime appears to be a safe and effective therapy for pediatric soft-tissue infections due to H influenzae and S pneumoniae. [\hyperlink{Bafiertam}{PMID: 3877456}, W J Barson et al., 1985]

\hypertarget{pmid_12145915}{P}remedication in children continues to be a problem in anaesthesiology. Apart from the application of premedication drugs, a strongly committed anaesthetist and sound psychological work are decisive prerequisites for adequate premedication. In this randomized study, 329 children aged between 1 and 14 who were scheduled for small surgical procedures were examined. Based on the results of a preliminary trial using different dosages of midazolam, the children were premedicated either orally with 0.4 mg/kg BW midazolam or with 0.3 mg/kg BW midazolam as a rectal solution and the premedication effect was recorded according to the classification of Lindgren. These reduced doses of midazolam, which were much lower than those described in the literature, led to very good premedication effects, too. Regarding cardiovascular behaviour, an increasing anxiolysis and the effect of premedication led to a significant decline in mean arterial blood pressure and heart frequency. The circulatory changes that occurred were at no time clinically relevant in any child, and therefore no treatment was needed. Based on the measured oxygen saturation, no significant respiratory changes were seen during the observation period. The frequency and extent of the registered side-effects were classified as slight. Side-effects such as singultus, salivation and ataxia/gesticulations were of no danger during the entire perioperative period. [\hyperlink{Bafiertam}{PMID: 12145915}, B Pohl et al., 2002]

\hypertarget{pmid_2810760}{W}e have carried out laboratory and clinical studies on cefteram pivoxil (CFTM-PI, T-2588). The results are summarized as follows. CFTM-PI was given through oral administration to 2 children each at dose levels of 1.5 mg/kg, 3 mg/kg and 6 mg/kg. After administration, mean peak serum levels of CFTM obtained for the 3 dose levels were 0.66 +/- 0.01 microgram/ml, 1.26 +/- 1.05 micrograms/ml and 2.28 +/- 0.95 micrograms/ml at 2 hours, respectively, and mean half-lives were 1.07 +/- 0.52 hours, 1.32 +/- 0.76 hours and 2.53 +/- 1.70 hours, respectively. Mean urinary excretion rates of CFTM were 19.0 +/- 4.0\%, 9.4 +/- 1.5\% and 19.9 +/- 4.0\% in the first 8 hours after administration of 1.5 mg/kg, 3 mg/kg, 6 mg/kg, respectively. Treatment with CFTM-PI was made in 36 cases of pediatric bacterial infections including 20 cases of tonsillitis, 3 cases of bronchitis, 6 cases of scarlet fever, 3 cases of UTI and 1 case each of bronchopneumonia, abscess, staphylococcal scalded skin syndrome and vaginitis. Results obtained were excellent in 22 cases, good in 14 cases. No significant side effect due to the drug was observed in any cases. [\hyperlink{Bafiertam}{PMID: 2810760}, T Nishimura et al., 1989]

\hypertarget{pmid_37862}{P}ethidine 1 mg kg-1, diazepam 0.25 mg kg-1 and flunitrazepam 0.02 mg kg-1 i.m. wer compared as premedicants in a double-blind study in 145 children undergoing otolaryngological surgery. Both flunitrazepam and pethidine had an anxiolytic effect in the children of less than 5 yr whereas diazepam had little effect. All of the drugs were anxiolytic in the children aged 5 yr and older. Sleep following thiopentone was restless more often in the younger than in the older children. Cardiovascular responses to thiopentone and to tracheal intubation were most obvious following benzodiazepines in children of less than 5 yr. After anaesthesia 10--33\% of the older children could not recall pictures shown to them before anaesthesia. Forty-five (+/-SD 13) min after injection, the concentration of diazepam in serum was similar in both age groups; after 90 min it decreased in the younger and increased in the older children. All concentrations of flunitrazepam were significantly greater in the older compared with the younger children. [\hyperlink{Bafiertam}{PMID: 37862}, L Lindgren et al., 1979]

\hypertarget{pmid_28139636}{T}o evaluate the efficacy of cogitum in the treatment of asthenoneurotic disorders in children after bacterial meningitis (BM) or brain injury (BI). Twenty-four patients were examined. Group 1 included 14 patients with BM, 8 boys and 6 girls, aged 7 - 12 years, mean age 9,91 ± 1,71 years; group 2 consisted of 10 patients with BI, 6 boys and 4 girls, aged 7-12 years, mean age 10,4 ± 2,36 years. All patients received cogitum in dose of 250- 500 mg daily during 8 weeks. Neurological and neuropsychological (Bourdon's test, Luria's tests) examinations, EEG, MRI were performed before and after treatment. The study of cognitive functions showed a decrease in the accuracy and speed during performance of Bourdon's and Luria's tests. After the beginning of treatment with cogitum, 80\% of the patients in both groups demonstrated a significant improvement in the accuracy of Bourdon's test (р<0,01) and verbal retention (р<0,001). Repeated EEG (3-6 months after treatment) showed the distinct theta- , alpha-, beta- rhythms and clear zonal differences. Given a spectrum of clinical effects of cogitum, the authors recommend it for active use in pediatric practice. [\hyperlink{Bafiertam}{PMID: 28139636}, O A Milovanova et al., ]

\hypertarget{pmid_2214592}{S}ince children's intellectual perception is limited, the preoperative visit by an anesthesiologist alone can rarely help to free the small patients from fear and restlessness prior to elective surgery. In order to relieve anxiety which should be the primary goal of premedication in any patient - children need anxiolytic premedication agents. Drugs for premedication administered by intramuscular or rectal route in children often cause pain, fear and discomfort. The present study was performed in order to investigate oral given midazolam in the premedication of children with special regard to the practical suitability of this method. 100 children, 0.5 to 10 years of age (group A: 0.5-4 years, group B: 5-10 years) undergoing elective urological surgery received 0.4 mg/kg midazolam orally about 20 minutes prior to the arrival in the operation unit. After insertion of a venous cannula into a forearm vein anesthesia was induced with thiopental and maintained by inhalation with Isoflurane, nitrous oxide and oxygen (fi O2:0.3). Degree of sedation, state of mind and behaviour (for 100 children) as well as blood pressure and heart rate (separately for group A and B) were registered preoperatively at defined, comparable and representative circumstances. Side effects prior and during induction phase of anesthesia were documented. The personal data are representative for a normal population of children with typical urological diseases. Oral administered midazolam had only a mild or non sedative effect in 76-84\% of the children 70-84\% of the small patients showed an indifferent or euphoric state of mind and 67-88\% behaved cooperatively or passively.(ABSTRACT TRUNCATED AT 250 WORDS) [\hyperlink{Bafiertam}{PMID: 2214592}, G Molter et al., ]

\section*{Dimethyl Fumarate}
\subsection*{Result}
\subsubsection*{Answer}

Ages 10–17: Yes  
Ages 6–9: Unknown  
Ages <6: Unknown  

\subsubsection*{{Explanation}}
\hypertarget{Dimethyl Fumarate}
Based on the abstracts available, several studies have specifically evaluated the safety of Dimethyl Fumarate (DMF) in children with multiple sclerosis:

1. A retrospective review of 13 children (≤18 years) treated with DMF for a median of 15 months found that DMF was generally well tolerated. Common side effects included facial flushing, gastrointestinal discomfort, rash, and malaise. Three children discontinued due to side effects, but no laboratory abnormalities or infections were reported. The authors concluded that DMF appears to be safe and generally well tolerated in children with multiple sclerosis, though they note that formal clinical trials are ongoing [\hyperlink{pmid_26996405}{PMID: 26996405}, Naila Makhani et al., 2016].

2. The FOCUS phase 2 multicenter study enrolled patients aged 10 to 17 years with relapsing-remitting multiple sclerosis. Over 24 weeks, 20 of 22 patients completed the study. Adverse events (mainly gastrointestinal and flushing) and pharmacokinetics were consistent with adult findings. No serious adverse events were considered related to DMF. The study concluded that DMF treatment was associated with a reduction in MRI activity and that the safety profile was consistent with adults [\hyperlink{pmid_29681490}{PMID: 29681490}, Raed Alroughani et al., 2018].

3. A case report described a 3-year-old boy with MS who began DMF at age 6. After increasing the dose at age 7, no relapses were observed for over three years, and no severe side effects or sequelae were reported. The authors suggest DMF may be useful and safe in young children, but this is a single case [\hyperlink{pmid_35058083}{PMID: 35058083}, Naoya Saijo et al., 2022].

Other abstracts either focus on adult populations, animal studies, or do not address pediatric safety directly.

Summary by age range:
- Ages 10–17: Safety of DMF is supported by a targeted phase 2 clinical trial and a retrospective review, both indicating DMF is generally safe and well tolerated in this age group.
- Ages 6–9: Evidence is limited to a single case report, which is not sufficient to definitively affirm safety.
- Ages <6: No targeted safety studies; only a single case report exists.

Therefore, based on the abstracts:
- For ages 10–17, there is evidence from targeted studies affirming safety.
- For ages 6–9 and <6, safety is unknown due to insufficient data.

\subsection*{Abstracts}
\hypertarget{pmid_26996405}{F}irst-line injectable therapies for multiple sclerosis in children may be ineffective or not well-tolerated. There is therefore an urgent need to explore oral medications for pediatric multiple sclerosis. We review our dual-center experience with oral dimethyl fumarate. This study was a retrospective review of children 18 years of age or less with multiple sclerosis treated with dimethyl fumarate at Yale University and the University of Colorado. Clinical, demographic, and magnetic resonance imaging parameters were analyzed. We identified 13 children treated with oral dimethyl fumarate for a median of 15.0 months (range, 1 to 25). Dimethyl fumarate was utilized as first-line therapy in five children (38\%). Ten children (77\%) tolerated dose escalation to the usual adult dose of 240 mg twice daily. Nine children had ≥12 months of follow-up on treatment. Eight of nine (89\%) displayed stabilized or reduced relapse rates and disability scores on treatment. Nine children underwent brain magnetic resonance imaging performed after 12 or more months of therapy. New T2 lesions were observed in three children (33\%), one of whom had been nonadherent to treatment. Common side effects included facial flushing (8/13, 62\%), gastrointestinal discomfort (7/13, 54\%), rash (3/13, 23\%), and malaise (2/13, 15\%). Three children (23\%) discontinued treatment because of side effects. No patients displayed laboratory abnormalities including lymphopenia or abnormal liver transaminases. There were no reported infections. Oral dimethyl fumarate appears to be safe and generally well tolerated in children with multiple sclerosis. Formal clinical trials to evaluate efficacy are ongoing. [\hyperlink{Dimethyl Fumarate}{PMID: 26996405}, Naila Makhani et al., 2016]

\hypertarget{pmid_29681490}{N}o therapies have been formally approved by the Food and Drug Administration for use in pediatric multiple sclerosis, a rare disease. We evaluated the safety, efficacy, and pharmacokinetics of dimethyl fumarate in pediatric patients with multiple sclerosis. FOCUS, a phase 2, multicenter study of patients aged 10 to 17 years with relapsing-remitting multiple sclerosis, comprised an eight-week baseline and 24-week treatment period; during treatment, patients received dimethyl fumarate (120 mg twice daily on days one to seven; 240 mg twice a day thereafter). Magnetic resonance imaging scans were obtained at week -8, day 0, week 16, and week 24. The primary end point was the change in T2 hyperintense lesion incidence from the baseline period to the final 8 weeks of treatment. Secondary end points were pharmacokinetic parameters and adverse event incidence. Twenty of 22 enrolled patients completed the study. There was a significant reduction in T2 hyperintense lesion incidence from baseline to the final eight weeks of treatment (P = 0.009). Adverse events (most commonly gastrointestinal events and flushing) and pharmacokinetic parameters were consistent with adult findings. No serious adverse events were considered dimethyl fumarate related. Dimethyl fumarate treatment was associated with a reduction in magnetic resonance imaging activity in pediatric patients; pharmacokinetic and safety profiles were consistent with those in adults. Dimethyl fumarate is a potential treatment for pediatric multiple sclerosis. [\hyperlink{Dimethyl Fumarate}{PMID: 29681490}, Raed Alroughani et al., 2018]

\hypertarget{pmid_35058083}{E}arly disease control with disease-modifying drugs is important for improving the prognosis of multiple sclerosis (MS) in children. Dimethyl fumarate (DMF) is an oral disease-modifying drug for MS in adults with relatively stable disease; however, its use in young children has not been heavily documented in the current literature. We report the case of a pediatric patient with relapsing-remitting MS who was treated with DMF. A 3-year-old boy with a history of common cold symptoms developed unsteadiness and somnolence. Magnetic resonance imaging revealed multiple white matter lesions. Symptoms were recurrent, and DMF was prescribed at 6 years of age due to a relapse episode with oculomotor disability and facial paralysis. However, disease progression continued, and new lesions were noted at age 7; thus, the dose of DMF was increased to 240 mg/day. No relapse has been observed for over three years; sequelae or severe side effects were absent. DMF may be a useful oral disease-modifying drug for preventing recurrence in young children with MS. [\hyperlink{Dimethyl Fumarate}{PMID: 35058083}, Naoya Saijo et al., 2022]

\hypertarget{pmid_31591676}{F}umaric acid esters are recommended in European guidelines for induction and maintenance treatment of patients with moderate to severe plaque psoriasis. A systemic medication with pure dimethyl fumarate without monoethyl fumarate salts was recently licensed in Europe. The efficacy and safety of pure dimethyl fumarate were assessed in patients with severe (physician global assessment) plaque psoriasis in Austria in the BRIDGE trial. In this double blind, randomized, placebo-controlled trial patients received 16-week treatment with pure dimethyl fumarate in a head to head comparison with dimethyl fumarate with monoethyl fumarate salts, which is licensed in Germany. In this post hoc analysis the efficacy and safety were assessed in patients with severe psoriasis in Austria. Efficacy measures significantly improved in both active treatment arms compared to placebo in 65 patients after 16 weeks of treatment. Physician global assessment of clear/almost clear in the dimethyl fumarate group was non-inferior to the dimethyl fumarate with monoethyl fumarate salts group 2 months after end of treatment. No serious adverse reaction occurred in patients with dimethyl fumarate in contrast to the second active treatment. Efficacy outcome was paralleled by quality of life improvements. This is the first report of dimethyl fumarate in a severely affected population with plaque psoriasis. Dimethyl fumarate is effective and safe in the systemic treatment of adults with severe psoriasis (physician global assessment). [\hyperlink{Dimethyl Fumarate}{PMID: 31591676}, Paul Sator et al., 2019]

\hypertarget{pmid_9428981}{T}he efficacy and tolerability of dimethindene maleate (CAS 3614-69-5, DMM, Fenistil) as drops in the treatment of pruritus in children suffering from chicken-pox were investigated in a study with two different doses of dimethindene maleate and placebo. 128 children, 1 to 6 years of age, were included in a double blind, randomized, placebo controlled, multi-center clinical trial. Patients received either a dosage of DMM of 0.1 mg/kg x d, or 0.05 mg/kg x d, or placebo, respectively. All patients received a commercially available astringent lotion for topical treatment of skin lesions. The primary efficacy criterion which was the change in the itching severity score from baseline to the end of the treatment assessed as area under the baseline (AUB) showed for both treatments with DMM a statistically significant superiority versus placebo in reducing the severity of itching. There was no statistically proven difference between the two verum groups. [\hyperlink{Dimethyl Fumarate}{PMID: 9428981}, W Englisch et al., 1997]

\hypertarget{pmid_8087216}{D}iphemanil methylsulfate is an atropin-like drug used in some infants suffering from vagal bradycardia. Its pharmacokinetic parameters are known for adults but not for infants. The report describes these parameters in six infants. Five infants aged 35 to 109 days (mean: 62 +/- 28) and weighing 3.5 to 5.3 kg (mean: 4.3) were included in the study with the formal consent of their parents. All suffered from vagal hyperreactivity. The sixth younger full-term infant was aged 10 days and weighed 4 kg. They were given a single dose (3 mg/kg) of diphemanil methylsulfate orally, after a minimal fast of 4 hours. Blood samples were collected at T0 and 3, 6, 8, 12 and 24 hours after administration. Urines were also collected from 1 hour before drug administration to 24 hours after. Plasma concentrations of diphemanil methylsulfate were measured by gas-exchange chromatography. The peak plasma concentration in the five infants occurred at 3.9 +/- 2.3 hours (range: 2.9-8 hours). Half-life was 8.6 +/- 2.4 hours and tended to decrease with age. All the other parameters were identical to those found in adults. The peak plasma concentration occurred in the sixth younger infant at 2.9 hours, with a half-life of 17.2 hours. Renal clearance was high (0.3 l/h/kg). The relatively long half-life of diphemanil methylsulfate allows this drug to be given every 8 hours. This longer interval is more comfortable for the patients and their parents. The high renal clearance suggests that this drug is excreted by both glomerular filtration and tubular secretion. [\hyperlink{Dimethyl Fumarate}{PMID: 8087216}, G Chéron et al., 1994]

\hypertarget{pmid_27277955}{S}timulant medications are approved to treat attention deficit hyperactivity disorder (ADHD) in children over the age of 6 years. Fatal ingestion of stimulants by children has been reported, although most ingestions do not result in severe toxicity. Lisdexamfetamine dimesylate, a once daily long-acting stimulant, is a prodrug requiring conversion to its active form, dextroamphetamine, in the bloodstream. Based on its unique pharmacokinetics, peak levels of d-amphetamine are delayed. We describe a case of accidental ingestion of lisdexamfetamine dimesylate in an infant. A previously healthy 10-month-old infant was admitted to the hospital with a 5-h history of tachycardia, hypertension, dyskinesia, and altered mental status of unknown etiology. Confirmatory urine testing, from a specimen collected approximately 16 h after the onset of symptoms, revealed an urine amphetamine concentration of 22,312 ng/mL (positive cutoff 200 ng/mL). The serum amphetamine concentration, from a specimen collected approximately 37 h after the onset of symptoms, was 68 ng/mL (positive cutoff 20 ng/mL). Urine and serum were both negative for methamphetamine, methylenedioxyamphetamine (MDA), methylenedioxymethamphetamine (MDMA, Ecstasy), and methylenedioxyethamphetamine (MDEA). During the hospitalization, it was discovered that the infant had access to lisdexamfetamine dimesylate prior to the onset of symptoms. Amphetamine ingestions in young children are uncommon but do occur. Clinicians should be aware of signs and symptoms of amphetamine toxicity and consider ingestion when a pediatric patient presents with symptoms of a sympathetic toxidrome even when ingestion is denied. [\hyperlink{Dimethyl Fumarate}{PMID: 27277955}, Kelly E Wood et al., 2016]

\hypertarget{pmid_21490354}{D}imenhydrinate is an over-the-counter drug that is commonly used for the treatment of nausea and vomiting. Many of my adult patients use it, but is it safe and useful in the pediatric population? Dimenhydrinate appears to be safe for use in the pediatric population. While little literature has been published about adverse effects of this medication, family physicians need to identify the cause of the vomiting before considering if the drug will be effective and need to ensure that patients safely use the medication and avoid potential interaction of the drug with other products. [\hyperlink{Dimethyl Fumarate}{PMID: 21490354}, Paul Enarson et al., 2011]

\hypertarget{pmid_29948245}{D}imethyl-fumarate (DMF) demonstrated efficacy and safety in relapsing-remitting multiple sclerosis (MS) in randomized clinical trials. To track and evaluate post-market DMF profile in real-world setting. Patients receiving DMF referred to Italian MS centres were enrolled and prospectively followed, collecting demographic clinical and radiological data. Among the 735 included patients, 45.4\% were naïve to disease-modifying therapies, 17.8\% switched to DMF because of tolerance, 27.4\% switched to DMF because of lack of efficacy, and 9.4\% switched to DMF because of safety concerns. Median DMF exposure was 17 months (0-33). DMF reduced the annual relapse rate (ARR) by 63.2\%. At 12 and 24 months, 85 and 76\% of patients were relapse-free. NEDA-3 status after 12 months of DMF treatment was maintained by 47.5\% of patients. 89 and 70\% of patients at 12 and 24 months regularly continued DMF. Most frequent adverse events (AEs) were flushing (37.2\%) and gastro-enteric AEs (31.1\%). Our post-market study corroborated that DMF is a safe and effective drug. Additionally, the study suggested that naïve patients strongly benefit from DMF and that DMF improved ARR also in patients who were horizontally switched from injectable therapies due to tolerability and efficacy issues. [\hyperlink{Dimethyl Fumarate}{PMID: 29948245}, Giulia Mallucci et al., 2018] (1) Respiratory distress and seizures developed in an 18-month-old boy following brief exposure to low-strength (17.6\%) N,N-diethyl-m-toluamide (DEET). A review of the literature revealed 17 reports of DEET-induced encephalopathy in children. The objective of this study was to test the hypothesis that the potential toxicity of DEET is high and that available repellents containing DEET, irrespective of their strength, are not safe when applied to children's skin. (2) Although this is a case report, we used the features of published reports of DEET-induced encephalopathy in children to support the diagnosis, since the evidence that the child's illness was caused by DEET was circumstantial. In the following case analysis, clinical reports of children < 16 years old have been reviewed and analyzed in an effort to relate direct DEET toxicity to various clinical, demographic, and toxic compound exposure factors (Fisher's exacttest and logistic regression analysis). (3) DEET-induced encephalopathy in children (56\% girls) followed not only ingestion or repeated and extensive application of repellents, but also a brief exposure to DEET (45\%). Of those who reported a dermal exposure, 33\% reported an exposure to a product containing DEET < 20\%. Seizures, the most prominent symptom (72\%), were significantly more frequent when DEET solutions were applied to the skin (P<0.01). Mortality (16.6\%) did not correlate significantly with the concentration of the DEET liquid used, duration of skin exposure, pattern of use, age, or sex. (4) Data of this case analysis suggest that repellents containing DEET are not safe when applied to children's skin and should be avoided in children. Additionally, since the potential toxicity of DEET is high, less toxic preparations should be probably substituted for DEET-containing repellents, whenever possible. [\hyperlink{Dimethyl Fumarate}{PMID: 29948245}, G Briassoulis et al., 2001]

\hypertarget{pmid_27128459}{T}his multi-center, randomized, double-blind, placebo-controlled, two-way crossover study characterized the safety, tolerability, pharmacokinetics, and pharmacodynamics of fluticasone furoate (FF) in children (5-11 years) with persistent asthma. Twenty-seven children received inhaled FF 100 µg or placebo via the ELLIPTA™ dry powder inhaler once daily for 14 days, with a ≥7 day washout period. Adverse events (AEs) were reported by eight (31\%) and four (16\%) subjects during FF 100 µg and placebo treatment, respectively. Headache was reported by three subjects during FF 100 µg treatment and by no subjects during placebo treatment, all other AEs were reported by only one subject on either treatment; there were no serious AEs. Following repeat dosing, the arithmetic mean (SD) FF Cmax was 26.71 pg/mL (9.16) at 31 minutes post-dose. Arithmetic mean (SD) FF AUC(0-t) was 121.44 pg h/mL (83.04). Arithmetic mean values for weighted mean (SD) serum cortisol (0-12 hours) on day 14 were 56.49 (16.51) and 67.57 (20.66) ng/mL for FF 100 µg and placebo, respectively. No clinically significant effect of FF on serum cortisol levels was observed. FF was well tolerated. Pharmacokinetic profiles were well defined and did not differ between age groups in the study population, and no clinically significant suppression of serum cortisol was observed.  [\hyperlink{Dimethyl Fumarate}{PMID: 27128459}, Amanda Oliver et al., 2014] Dialkyl phthalates are plasticizers used in household products made from polyvinyl chloride (PVC). Diisononyl phthalate (DINP) is the principal phthalate in soft plastic toys. Because DINP is not tightly bound to PVC, it may be released when children mouth PVC products. The potential chronic health risks of phthalate exposure to infants have been under scrutiny by regulatory agencies in Europe, Canada, Japan, and the U.S. This report describes a risk assessment of DINP exposure from children's products, by the U.S. Consumer Product Safety Commission (CPSC) staff. This report includes the findings of a CPSC Chronic Hazard Advisory Panel (CHAP) which: (1) concluded that DINP is unlikely to present a human cancer hazard and (2) recommended an acceptable daily intake (ADI) level of 120 microg/kg-d, based on spongiosis hepatis in rats. The risk assessment incorporates new measurements of DINP migration rates from 24 toys and a new observational study of children's mouthing activities, with a detailed characterization of the objects mouthed. Probabilistic methods were used to estimate exposure. Mouthing behavior and, thus, exposure depend on the child's age. Approximately 42\% of tested soft plastic toys contained DINP. Estimated DINP exposures for soft plastic toys were greatest among children 12-23 months old. The mean exposure for this age group was 0.08 (95\% confidence interval 0.04-0.14) microg/kg-d, with a 99th percentile of 2.4 (1.3-3.2) microg/kg-d. The authors conclude that oral exposure to DINP from mouthing soft plastic toys is not likely to present a health hazard to children. The opinions expressed by the authors have not been reviewed or approved by, and do not necessarily reflect the views of, the U.S. Consumer Product Safety Commission. Because this material was prepared by the authors in their official capacity, it is in the public domain and may be freely copied or reprinted. [\hyperlink{Dimethyl Fumarate}{PMID: 27128459}, Michael A Babich et al., 2004]

\hypertarget{pmid_25511835}{T}he efficacy and safety of the three oral agents approved by the Food and Drug Administration for the treatment of relapsing-remitting multiple sclerosis (RRMS) are reviewed. Limitations to parenteral disease-modifying therapies (DMTs) (interferon beta-1a, interferon beta-1b, and glatiramer acetate) for the treatment of RRMS have been addressed by the approval of three oral DMTs: fingolimod, teriflunomide, and dimethyl fumarate. In clinical trials, each of the oral DMTs was superior to placebo in annualized relapse rate, a key indicator of clinical efficacy, and in neuroradiological efficacy. A reduction in disability progression was evident with higher doses of teriflunomide but was not consistently demonstrated with fingolimod or dimethyl fumarate. Each of the oral DMTs demonstrated acceptable safety in clinical trials, with adverse-effect profiles that differ from injectable agents. The safety of both teriflunomide and dimethyl fumarate is supported by long-term use of related agents for other diseases; however, postmarketing surveillance studies are needed to determine the safety of each of the oral DMTs in patients with RRMS. Dimethyl fumarate seems to have the most innocuous safety profile of the three agents. Fingolimod requires first-dose inpatient monitoring due to cardiac safety concerns and multiple laboratory tests prior to initiation of therapy, while teriflunomide has been associated with hepatotoxicity and teratogenicity. With the approval of three oral drugs for RRMS-fingolimod, teriflunomide, and dimethyl fumarate-the therapeutic strategy for RRMS has evolved to include options that are efficacious and appear to have administration advantages over established parenteral treatments. [\hyperlink{Dimethyl Fumarate}{PMID: 25511835}, Rachel Hutchins Thomas et al., 2015]

\hypertarget{pmid_25800129}{D}imethyl fumarate (DMF), a fumaric acid ester, is a new orally available disease-modifying agent that was recently approved by the US FDA and the EMA for the management of relapsing forms of multiple sclerosis (MS). Fumaric acid has been used for the management of psoriasis, for more than 50 years. Because of the known anti-inflammatory properties of fumaric acid ester, DMF was brought into clinical development in MS. More recently, neuroprotective and myelin-protective mechanism actions have been proposed, making it a possible candidate for MS treatment. Two Phase III clinical trials (DEFINE, CONFIRM) have evaluated the safety and efficacy of DMF in patients with relapsing-remitting MS. Being an orally available agent with a favorable safety profile, it has become one of the most commonly prescribed disease-modifying agents in the USA and Europe.  [\hyperlink{Dimethyl Fumarate}{PMID: 25800129}, Duvyanshu Dubey et al., 2015] Dimethyl fumarate (DMF) has immune-modulatory and neuro-protective characteristics that can be used for treatment of acute ischemic stroke. To investigate the therapeutic effects of DMF on histological and functional recovery of rats after transient middle cerebral artery (MCA) occlusion. 22 Sprague-Dawley male rats weighing 275-300 g were randomized into three groups by block randomization. In the sham group (n = 7), the neck was opened, but neither MCA was occluded, nor any drug was administered.The control group (n = 7) was treated with vehicle (methocel) by gavage for 14 days after MCA occlusion. In the DMF-treated group (n = 8), treatment was performed with 15 mg/kg body weight dimethyl fumarate twice a day for 14 days after MCA occlusion. Transient occlusion of the right MCA was performed by intraluminal thread method in the DMF-treated and the control group. Neurological deficit score (NDS), pole test, and adhesive removal test were performed before the surgery, and on post-operative Days 0, 3, 5, 7, 10, and 14. After the final behaviour test, the animals' brains were perfused and removed. Brains were frozen and sectioned serially and coronally using a cryostat. Infract volume and brain volume were estimated by stereology. The percentage of infarct volume was significantly lower in DMF-treated animals (5.76\%) than in the control group (22.39\%) (P < 0.0001). Regarding behavioural tests, the DMF-treated group showed better function in NDS on Days 7 (P = 0.041) and 10 (P = 0.046), but not in pole and adhesive removal tests. There was no significant correlation between behavioural tests and histological results. Dimethyl fumarate could be beneficial as a potential neuroprotective agent in the treatment of stroke. [\hyperlink{Dimethyl Fumarate}{PMID: 25800129}, Anahid Safari et al., 2017]

\hypertarget{pmid_22364032}{A}cute respiratory infections are the second leading cause of morbidity in children under 18 years. Several drugs have been used with variable efficacy and safety, trying to reduce the associated symptoms and improve quality of life. To evaluate the efficacy and safety of buphenine, aminophenazone and diphenylpyraline hydrochloride when compared with placebo for the control of symptoms associated with common cold in children 6-24 months of age. Randomized clinical trial, double blind, placebo controlled, in 100 children < 24 months of any gender, with symptoms associated to common cold. They received the drug under study vs. placebo for seven days. Both groups received acetaminophen. The change on common cold related symptoms were evaluated. Statistic analysis was made with STATA 11.0 for Mac. Fifty-three children were randomized to study drug and forty-seven to placebo. Age of children in each group was similar (12.2 +/- 5.8 months vs. 12.7 +/- 5.8 months, p NS). There were significant differences between groups in relation to rhinorrea and sneezing resolution, with better results in Flumil group and no adverse events observed. The results in this study indicates that Flumil is a safe and effective drug for control of symptoms present in the common cold in children aged 6-24 months. [\hyperlink{Dimethyl Fumarate}{PMID: 22364032}, Ericka Montijo-Barrios et al., ]

\hypertarget{pmid_37103520}{M}edications for treating bipolar disorder (BD) are limited and can cause side effects if used chronically. Therefore, efforts are being made to use new agents in the control and treatment of BD. Considering the antioxidant and anti-inflammatory effects of dimethyl fumarate (DMF), this study was performed to examine the role of DMF on ketamine (KET)-induced manic-like behavior (MLB) in rats. Forty-eight rats were randomly divided into eight groups, including three groups of healthy rats: normal, lithium chloride (LiCl) (45 mg/kg, p.o.), and DMF (60 mg/kg, p.o.), and five groups of MLB rats: control, LiCl, and DMF (15, 30, and 60 mg/kg, p.o.), which received KET at a dose of 25 mg/kg, i.p. The levels of total sulfhydryl groups (total SH), thiobarbituric acid reactive substances (TBARS), nitric oxide (NO), and tumor necrosis factor-alpha (TNF-α), as well as the activity of antioxidant enzymes including catalase (CAT), superoxide dismutase (SOD), and glutathione peroxidase (GPx) in the prefrontal cortex (PFC) and hippocampus (HPC), were measured. DMF prevented hyperlocomotion (HLM) induced by KET. It was found that DMF could inhibit the increase in the levels of TBARS, NO, and TNF-α in the HPC and PFC of the brain. Furthermore, by examining the amount of total SH and the activity of SOD, GPx, and CAT, it was found that DMF could prevent the reduction of the level of each of them in the brain HPC and PFC. DMF pretreatment improved the symptoms of the KET model of mania by reducing HLM, oxidative stress, and modulating inflammation. [\hyperlink{Dimethyl Fumarate}{PMID: 37103520}, Shiva Saljoughi et al., 2023]

\hypertarget{pmid_9132194}{T}o evaluate the safety and efficacy of intranasal diamorphine as an analgesic for use in children in accident and emergency (A\&E). A prospective, randomised clinical trial with consecutive recruitment of patients aged between 3 and 16 years with clinically suspected limb fractures. One group received 0.1 mg/kg intranasal diamorphine, and the other group received 0.2 mg/kg intramuscular morphine. At 0, 5, 10, 20, and 30 minutes pain scores, Glasgow coma score, and peripheral oxygen saturations were recorded; parental acceptability was assessed at 30 minutes. 58 children were recruited, with complete data collection in 51 (88\%); the median summed decrease in pain score was better for intranasal diamorphine than intramuscular morphine (9 v 8), though this was not significant (P = 0.4, Mann-Whitney U test). The episode was recorded as "acceptable" in all parents whose child received intranasal diamorphine, compared with only 55\% of parents in the intramuscular morphine group (P < 0.0001, Fisher's exact test). There was no incidence of decreased peripheral oxygen saturation or depression in the level of consciousness in any patient. Intranasal diamorphine is an effective, safe, and acceptable method of analgesia for children requiring opiates in the A \& E department. [\hyperlink{Dimethyl Fumarate}{PMID: 9132194}, J A Wilson et al., 1997]

\hypertarget{pmid_33193814}{D}imethyl fumarate (DMF) is approved for the treatment of relapsing-remitting multiple sclerosis. It is unknown whether DMF or its primary metabolite monomethyl fumarate (MMF) are excreted into human milk. We present two cases of lactating patients who donated milk samples to study the transfer of MMF into human milk following a week of 2 × 240 mg daily oral dose. Samples were analyzed using liquid chromatography mass spectrometry. The calculated relative infant dose was 0.019\% and 0.007\%. This is the first study to demonstrate that MMF is transferred into human milk, with only limited exposure to an infant. [\hyperlink{Dimethyl Fumarate}{PMID: 33193814}, Andrea I Ciplea et al., 2020]

\hypertarget{pmid_28150703}{P}ulmonary arterial hypertension (PAH) is a fatal condition for which there is no cure. Dimethyl Fumarate (DMF) is an FDA approved anti-oxidative and anti-inflammatory agent with a favorable safety record. The goal of this study was to assess the effectiveness of DMF as a therapy for PAH using patient-derived cells and murine models. We show that DMF treatment is effective in reversing hemodynamic changes, reducing inflammation, oxidative damage, and fibrosis in the experimental models of PAH and lung fibrosis. Our findings indicate that effects of DMF are facilitated by inhibiting pro-inflammatory NFκB, STAT3 and cJUN signaling, as well as βTRCP-dependent degradation of the pro-fibrogenic mediators Sp1, TAZ and β-catenin. These results provide a novel insight into the mechanism of its action. Collectively, preclinical results demonstrate beneficial effects of DMF on key molecular pathways contributing to PAH, and support its testing in PAH treatment in patients. [\hyperlink{Dimethyl Fumarate}{PMID: 28150703}, Agnieszka P Grzegorzewska et al., 2017]

\hypertarget{pmid_23013261}{D}imethylacetamide (DMAC) and dimethylformamide (DMF) continue to be important, widely used solvents involved in a wide variety of industrial applications. As liquids with relatively low vapor pressures, contact with both the integumentary and respiratory systems is the main source of human exposure. Although airborne control levels for the workplace have been established and industrial hygiene practices to limit dermal contact have been put in place, use of these chemicals has been associated with occupational illness, mainly in Asia where new and expanded uses have led to overexposures. Thus an update of the basic toxicology data including tables indicating the dose/exposure response characteristics of both DMAC and DMF is currently important. Both chemicals are similar from a toxicology perspective. Human experience has generally shown the materials to be without adverse effect except under conditions where airborne and dermal controls were not properly applied. The use of urinary metabolite monitoring has successfully been employed to measure integrated dermal and inhalation worker exposure. The chemicals are not particularly toxic following acute exposure but high doses can produce damage to the liver, the organ which is first affected by these two chemicals. Repeated dose/exposure studies have characterized both the targets of toxicity and the doses required to produce changes by various routes of exposure. Higher doses of these materials can produce changes in developing systems, infrequently in experiments at doses in which the maternal animal is unaffected, thus care needs to be taken when exposures are to women of child-bearing age. The chemicals appear to be low in genetic activity and inhalation exposures have not shown the materials to produce tumors in rodents except with DMF in a situation in which aerosol formation was encountered. This presentation extends the two previous reviews and, like those, includes updated information on acetamide and formamide and their monomethyl derivatives as well as the commercially important DMAC and DMF. Since a large portion of the newer information deals with effects in humans and biomonitoring, these sections are presented at the start of this review. [\hyperlink{Dimethyl Fumarate}{PMID: 23013261}, Gerald L Kennedy et al., 2012]

\hypertarget{pmid_32974794}{D}imethyl fumarate and fingolimod are oral disease modifying treatments (DMTs) that reduce relapse activity and slow disability worsening in relapsing-remitting multiple sclerosis (RRMS). To compare the effectiveness of dimethyl fumarate and fingolimod in a real-world setting, where both agents are licensed as a first-line DMT for the treatment of RRMS. We identified patients with RRMS commencing dimethyl fumarate or fingolimod in the Swiss Federation for Common Tasks of Health Insurances (SVK) Registry between August 2014 and July 2019. Propensity score-matching was applied to select subpopulations with comparable baseline characteristics. Relapses and disability outcomes were compared in paired, pairwise-censored analyses. Of the 2113 included patients, 1922 were matched (dimethyl fumarate, n = 961; fingolimod, n = 961). Relapse rates did not differ between the groups (incident rate ratio 1.0, 95\%CI 0.8-1.2, p = 0.86). Moreover, no difference in the hazard of 1-year confirmed disability worsening (hazard ratio [HR] 0.9; 95\%CI 0.6-1.6; p = 0.80) or disability improvement (HR 0.9; 95\%CI 0.6-1.2; p = 0.40) was detected. These findings were consistent both for treatment-naïve patients and patients switching from another DMT. Dimethyl fumarate and fingolimod have comparable effectiveness regarding reduction of relapses and disability worsening in RRMS. [\hyperlink{Dimethyl Fumarate}{PMID: 32974794}, Johannes Lorscheider et al., 2021]

\hypertarget{pmid_20593906}{T}he high prevalence of asthma in pediatric patients underscores the need for effective and safe treatment in this population. Current treatment guidelines recommend inhaled corticosteroids (ICSs) as a preferred treatment for the control of mild to moderate persistent asthma in patients of all ages, including young children. Clinical efficacy, systemic safety, and ease of use are desirable attributes of an ICS used to treat children with persistent asthma. Recently, mometasone furoate administered via a dry powder inhaler (MF-DPI) 110 microg once daily in the evening (delivered dose of 100 microg) was approved by the US FDA for the maintenance treatment of asthma in children 4-11 years of age. Data from the clinical trial program for MF-DPI that establish the efficacy, long-term safety, and absence of systemic effects of the approved dosage in children with mild to moderate persistent asthma are reviewed. These findings indicate that once-daily dosing of MF-DPI in children aged 4-11 years significantly improves lung function and health-related quality of life while reducing rescue medication use and exacerbations despite previous treatment with other ICSs. MF-DPI is also well tolerated in children. Clinical trial results showed that, at the approved dosage, there are no effects on growth velocity or the hypothalamic-pituitary-adrenal axis. Results of pediatric studies are consistent with the clinical development program for adults and adolescents. In addition, once-daily dosing, established safety, and ease of use of MF-DPI may help to improve asthma management by addressing issues that inhibit proper adherence. [\hyperlink{Dimethyl Fumarate}{PMID: 20593906}, Henry Milgrom et al., 2010]

\hypertarget{pmid_15258101}{T}oxicology studies are typically performed on single compounds, which we hypothesized would miss adverse synergies from chemical mixtures. This hypothesis was tested using an insect repellant and sunscreens because both groups include known permeation enhancers, with prior pediatric reports of toxicity from highly concentrated DEET (N,N-diethyl-m-toluamide). Using real-time mass spectroscopy in a hairless mouse skin model, we confirmed substantial penetration of a 20\% DEET standard. Despite a lower (10\%) DEET content, a commercially marketed sunscreen formulation had a 6-fold more rapid detection (5 versus 30 min) and 3.4-fold greater penetration at steady state. We also tested the efficacy of DEET microemulsion products and confirmed that one successfully slowed the onset of absorption, but not the steady-state permeation. Risks from mixtures of potential toxins are worthy of routine testing, which can be accomplished by simple assays, and are of utmost importance for pediatric applications. [\hyperlink{Dimethyl Fumarate}{PMID: 15258101}, Edward A Ross et al., 2004]

\hypertarget{pmid_19752076}{V}omiting is a common symptom in children with infectious gastroenteritis. It contributes to fluid loss and is a limiting factor for oral rehydration therapy. Dimenhydrinate has traditionally been used for children with gastroenteritis in countries such as Canada and Germany. We investigated the efficacy and safety of dimenhydrinate in children with acute gastroenteritis. We performed a prospective, randomized, placebo-controlled, multicenter trial. We randomly assigned 243 children with presumed gastroenteritis and vomiting to rectal dimenhydrinate or placebo. Children with no or mild dehydration were included. All children received oral rehydration therapy. Primary outcome was defined as weight gain within 18 to 24 hours after randomization. Secondary outcomes were number of vomiting episodes, fluid intake, parents' assessment of well-being, number of diarrheal episodes, and admission rate to hospital. We recorded potential adverse effects. Change of weight did not differ between children who received dimenhydrinate or placebo. The mean number of vomiting episodes between randomization and follow-up visit was 0.64 in the dimenhydrinate group and 1.36 in the placebo group. In total, 69.6\% of the children in the dimenhydrinate group versus 47.4\% in the placebo group were free of vomiting between randomization and the follow-up visit. Hospital admission rate, fluid intake, general well-being of the children, and potential adverse effects, including the number of diarrhea episodes, were similar in both groups. Dimenhydrinate reduces the frequency of vomiting in children with mild dehydration; however, the overall benefit is low, because it does not improve oral rehydration and clinical outcome. [\hyperlink{Dimethyl Fumarate}{PMID: 19752076}, Ulrike Uhlig et al., 2009]

\section*{Mycapssa}
\subsection*{Result}
\subsubsection*{Answer}

Unknown

\subsubsection*{{Explanation}}
\hypertarget{Mycapssa}
After reviewing all available abstracts, there is no evidence from any abstract that a targeted safety study of Mycapssa (oral octreotide capsules) has been conducted in children. None of the abstracts mention Mycapssa or provide data on its use, safety, or efficacy in pediatric populations of any age range. Therefore, based on the abstracts provided, the safety of Mycapssa in children is unknown.

\subsection*{Abstracts}
\hypertarget{pmid_8239401}{T}he efficacy and safety of ear drops containing phenazone and lidocaine hydrochloride (Otipax) for the treatment of congestive myringitis were evaluated in 18 infants and children aged 1 to 10 years. Relief of pain was evident 5 minutes after instillation and significant after 15 to 30 minutes. Serial photographs of the tympanic membrane demonstrated prompt improvement of inflammation. Congestion was significantly reduced after five minutes and overall ear drum color was significantly improved after 15 to 30 minutes. No adverse effects were recorded. These data suggest that Otipax is effective and safe for the treatment of painful congestive myringitis in infants and children. [\hyperlink{Mycapssa}{PMID: 8239401}, M François et al., 1993]

\hypertarget{pmid_16566570}{S}nap-caps are marketed as a relatively safe pyrotechnic (explosive) device for children 8 years and older. Individually, the snap-caps pose very little threat because the amount of explosive compounds contained in each is limited to 1 mg. However, the accidental explosion of numerous snap-caps may cause significant burns. This study highlights a series of pediatric patients who presented with severe second- and third-degree burns as a result of accidental explosion of snap-caps. Seven patients with snap-caps-related injuries were treated at the University of California, San Diego Regional Burn Center from January 1996 to April 1999. Study foci included 1) mode and extent of injury, 2) management, 3) associated morbidity, and 4) functional outcome. Six patients (84\%) required hospital admission. Four patients (57\%) underwent split-thickness skin grafting to repair mean TBSA burns of 4.1\% (range, 2-8\%). Three patients (43\%) received aggressive management of burns with topical medications and dressing changes. The nature and extent of snap-cap injuries support the contention that snap-caps have the potential to harm children to whom they are marketed. [\hyperlink{Mycapssa}{PMID: 16566570}, Raffy L Karamanoukian et al., ]

\hypertarget{pmid_15283801}{M}ycotic scalp infection caused by Microsporum canis is one of the more recalcitrant disorders, with increasing incidence during the last decade. We report our experience with administration of itraconazole in 163 children (86 girls, 77 boys) with M. canis tinea capitis. Fifty-five patients had previous treatment with terbinafine without success. In all children, the dosage of itraconazole was adjusted according to body weight, with 5 mg/kg/day given in a continuous regimen either as a capsule (116 patients) or an oral suspension (47 patients). In all children, there was both clinical and mycologic cure after a mean treatment period of 39 +/- 12 days (range 10-77 days). Eleven children (6.7\%) had side effects: diarrhea in five children, cutaneous eruption in four, and abdominal pain in two. Itraconazole was effective and safe for the treatment of M. canis tinea capitis. [\hyperlink{Mycapssa}{PMID: 15283801}, Gabriele Ginter-Hanselmayer et al., ]

\hypertarget{pmid_36865690}{T}he efficacy and tolerability of  In an open-label, randomized clinical trial (EudraCT number 2011-002652-14), children aged 1-5 years suffering from AB received EPs 7630 syrup or solution for 7 days. Safety was assessed by frequency, severity, and nature of adverse events (AE), vital signs, and laboratory values. Outcome measures for evaluating the health status were the intensity of coughing, pulmonary rales, and dyspnea, measured by the short version of the Bronchitis Severity Scale (BSS-ped), further symptoms of the respiratory infection, general health status according to the Integrative Medicine Outcomes Scale (IMOS), and satisfaction with treatment according to the Integrative Medicine Patient Satisfaction Scale (IMPSS). 591 children were randomized and treated with syrup ( Both pharmaceutical forms, EPs 7630 syrup and oral solution, were shown to be equally safe and well tolerated in pre-school children suffering from AB. Improvement of health status and of complaints were similar in both groups. [\hyperlink{Mycapssa}{PMID: 36865690}, Wolfgang Kamin et al., 2023]

\hypertarget{pmid_30045280}{T}his study investigated the effectiveness and safety of montelukast combined budesonide (MCB) treatment for children with chronic cough-variant asthma (CCVA).In total, 82 cases of children with CCVA, aged 4 to 11 years were included in this study. All cases received either MCB or budesonide alone between May 2015 and April 2017. The primary outcome was lung function, measured by the peak expiratory flow rates (PEFRs) and forced expiratory volume in 1 second (FEV1). The secondary outcome was measured by the clinical assessment score. Furthermore, adverse events (AEs) were also recorded in this study. All outcomes were measured after 8-week treatment.After 8-week treatment, MCB showed greater effectiveness than did budesonide alone in improving the lung function, measured by PEFR V1 (P = .02), and FEV1 (P < .01). Similarly, the clinical assessment score also demonstrated significant difference between the 2 groups (P < .05). In addition, no serious AEs occurred in both groups.The results of this study demonstrate that the effectiveness of MCB is superior to budesonide alone in the treatment of children with CCVA. [\hyperlink{Mycapssa}{PMID: 30045280}, Xiu-Ping Wang et al., 2018]

\hypertarget{pmid_14632810}{E}xposure to environmental microorganisms is associated with variations in the prevalence and severity of atopic diseases. We have previously shown that administration of a Mycobacterium vaccae suspension significantly reduced the severity of atopic dermatitis (AD) in children aged 5-18 years. This study aimed to extend these observations to younger children. Fifty-six children aged 2-6 years with moderate to severe AD were enrolled in a randomized, double-blind, placebo-controlled trial and given one intradermal injection of either killed M. vaccae suspension or buffer solution (placebo). Skin surface area affected and dermatitis severity score were assessed before and 1, 3 and 6 months after treatment. Although a 38-54\% reduction in surface area affected by dermatitis was noted at all time points after M. vaccae administration (P = 0.005), this improvement was not significantly different from that observed in the placebo group. Meta-analysis of this and our previous cohort (97 children aged 2-18 years) showed that M. vaccae was associated with a significant improvement in clinical severity at all ages, whereas within the placebo group, younger but not older children showed a similar improvement. Despite a reduction in clinical severity associated with M. vaccae at all ages, no benefit could be found after administering M. vaccae to children with AD aged 2-6 years when compared with placebo. M. vaccae may offer greater benefit in children over 5 years old, whose AD appears less likely to regress spontaneously. [\hyperlink{Mycapssa}{PMID: 14632810}, P D Arkwright et al., 2003]

\hypertarget{pmid_18616067}{L}ead toxicity is an ongoing concern worldwide, and children, the most vulnerable to the long-lasting effects of lead exposure, are in urgent need of a safe and effective heavy metal chelating agent to overcome the heavy metals and lead exposure challenges they face day to day. This clinical study was performed to determine if the oral administration of modified citrus pectin (MCP) is effective at lowering lead toxicity in the blood of children between the ages of 5 and 12 years. Hospitalized children with a blood serum level greater than 20 microg/dL, as measured by graphite furnace atomic absorption spectrometry (GFAAS), who had not received any form of chelating and/or detoxification medication for 3 months prior were given 15 g of MCP (PectaSol) in 3 divided dosages a day. Blood serum and 24-hour urine excretion collection GFAAS analysis were performed on day 0, day 14, day 21, and day 28. This study showed a dramatic decrease in blood serum levels of lead (P = .0016; 161\% average change) and a dramatic increase in 24-hour urine collection (P = .0007; 132\% average change). The need for a gentle, safe heavy metal-chelating agent, especially for children with high environmental chronic exposure, is great. The dramatic results and no observed adverse effects in this pilot study along with previous reports of the safe and effective use of MCP in adults indicate that MCP could be such an agent. Further studies to confirm its benefits are justified. [\hyperlink{Mycapssa}{PMID: 18616067}, Zheng Yan Zhao et al., ]

\hypertarget{pmid_29498372}{S}cientific literature data on the experience of use of Proteflazid® (drops) and Immunoflazid® (syrup) for the treatment of viral diseases in children of the first six years of life are analysed in the article. A systematic review was conducted on the basis of postmarketing comparative clinical trials and long-term follow-up (during the period of 2002 to 2016) that involved about 1500 children (the intent-to-treat population comprised more than 800 of them). The safety and efficacy of the Proteflazid® (drops) and Immunoflazid® (syrup) usage in children for the treatment of viral infections have been proven. [\hyperlink{Mycapssa}{PMID: 29498372}, Galina Beketova et al., 2018]

\hypertarget{pmid_11801811}{P}iracetam is an effective symptomatic treatment for some types of myoclonus in adults. To survey the efficacy and safety of piracetam in pediatric opsoclonus-myoclonus, we conducted an open, randomized, two-period, dose-ranging, double-blind, crossover, clinical trial of five children comparing the antimyoclonic properties of oral piracetam to placebo. We devised and validated a new rating scale, specifically for pediatric opsoclonus-myoclonus. Two parents while blinded were able to identify the active phase by improvement in behavior, but another thought the behavior was worse. None of the patients showed improvement in myoclonus. The adult-equivalent dose of piracetam used in this study, which is threefold higher than that used in previous pediatric studies, was well tolerated and safe. We found our rating scale to be a reliable and useful tool for future studies of opsoclonus-myoclonus in children. [\hyperlink{Mycapssa}{PMID: 11801811}, M R Pranzatelli et al., ]

\hypertarget{pmid_16210843}{M}ethylxanthine therapy reduces the frequency of apnea and the need for mechanical ventilation. Recent research has raised concerns about the safety of methylxanthines in very preterm infants. Possible adverse effects include poor growth, worsening of hypoxic-ischemic brain damage and abnormal childhood behavior. Over 2,000 infants with birth weights 500-1,250 g have been randomized in the international placebo-controlled Caffeine for Apnea of Prematurity (CAP) trial to examine the long-term efficacy and safety of methylxanthine therapy for the management of apnea of prematurity. Additional therapies such as continuous positive airway pressure were used as necessary to control apneic attacks. At 18 months we measure the combined rate of death or survival with one or more of the following impairments: cerebral palsy, cognitive deficit, blindness and deafness. This outcome was chosen because of the need to evaluate the impact of common neonatal therapies beyond discharge from the intensive care unit. However, several potential long-term consequences of methylxanthine therapy may not become apparent until the study cohort reaches pre-school age. We will therefore extend the follow-up to age 5 years. The main outcome at 5 years will be a composite of death or survival with severe disability in at least one of six domains: cognition, neuromotor function, vision, hearing, behavior, and general health. Once this project is completed, caffeine will be one of the most rigorously evaluated neonatal therapies. [\hyperlink{Mycapssa}{PMID: 16210843}, Barbara Schmidt et al., 2005]

\hypertarget{pmid_2873768}{P}oison exposures in children less than 1 year old and the safety and efficacy of syrup of ipecac in children 9 to 12 months old were evaluated in a prospective eight-month study conducted at the Massachusetts Poison Control Center. Poison exposures in children less than 1 year old represented approximately 9\% of the 38,080 calls received. Mobile children (in walkers, crawling, or walking) were at the greatest risk of poisoning. The majority of children (94\%) were asymptomatic and none were hospitalized or died. The products involved were primarily plants (38\%) and  household products (30\%). All 21 patients, ages 9 to 12 months, were given 10 mL syrup of ipecac under medical supervision and vomited within one hour. The mean time to vomit was 21.7 (SEM +/- 2.8) minutes. The patients vomited 3.3 (SEM +/- 0.3) times and all episodes of vomiting abated by 26.4 (SEM +/- 6.6) minutes. No significant side effects were noted. The use of the syrup of ipecac in the 9- to 12-month-old child appears to be safe and effective. [\hyperlink{Mycapssa}{PMID: 2873768}, P Gaudreault et al., 1986]

\hypertarget{pmid_23958810}{M}icafungin is an echinocandin with proven efficacy against a broad range of fungal infections, including those caused by Candida spp. To evaluate the safety and pharmacokinetics of once-daily 3 mg/kg and 4.5 mg/kg micafungin in children with proven, probable or suspected invasive candidiasis. Micafungin safety and pharmacokinetics were assessed in 2 phase I, open-label, repeat-dose trials. In Study 2101, children aged 2-16 years were grouped by weight to receive 3 mg/kg (≥25 kg) or 4.5 mg/kg (<25 kg) intravenous micafungin for 10-14 days. In Study 2102, children aged 4 months to <2 years received 4.5 mg/kg micafungin. Study protocols were otherwise identical. Safety was analyzed in 78 and 9 children in Studies 2101 and 2102, respectively. Although adverse events (AEs) were experienced by most children (2101: n=62; 2102: n=9), micafungin-related AEs were less common (2101: n=28; 2102: n=1), and the number of patients discontinuing due to AEs was low (2101: n=4; 2102: n=1). The most common micafungin-related AEs were infusion-associated symptoms, pyrexia and hypomagnesemia (Study 2101), and liver function abnormalities (Study 2102). The micafungin pharmacokinetic profile was similar to that seen in other studies conducted in children, but different than that observed in adults. In this small cohort of children, once-daily doses of 3 mg/kg and 4.5 mg/kg micafungin were well tolerated. Pharmacokinetic data will be combined in a population pharmacokinetic analysis to support US dosing recommendations in children. [\hyperlink{Mycapssa}{PMID: 23958810}, Daniel K Benjamin et al., 2013]

\hypertarget{pmid_12641681}{T}he bitter taste of midazolam is more acceptable to children when the drug is mixed with fruit juice or syrup. We use a thick grape syrup (Syrpalta), and children are sedated in 10-15 min. A premixed cherry-flavoured midazolam solution (Roche), 2 mg.ml (-1), is currently available. It has been our impression that the premixed midazolam has a slower onset of action. Our aim was to evaluate the effects of the midazolam mixtures (midazolam 0.5 mg.kg (-1), 2 mg.ml (-1)) on children's anxiety, sedation, separation anxiety, mask acceptance, and recovery time. Seventy-six healthy children, 1-4 years of age, scheduled for elective placement of ear tubes, were enrolled. The trial was double-blinded and randomized. For premedication, one group received the premixed midazolam, and a second group received the midazolam/Syrpalta mixture. An independent blinded observer evaluated the children, using anxiety and sedation scales at baseline, at 5, 10 and 15 min and at parental separation. Mask acceptance and awakening time were evaluated. Children who received the midazolam/Syrpalta mixture had less anxiety at 15 min (P = 0.046) and at parental separation (P < 0.001) than those who received the premixed midazolam solution. Mask acceptance was not different. We concluded that the midazolam/Syrpalta mixture has a faster onset of action than the premixed midazolam solution. [\hyperlink{Mycapssa}{PMID: 12641681}, Samia N Khalil et al., 2003]

\hypertarget{pmid_16184028}{T}o assess the safety and distribution of a cellulose acetate 1,2-benzenedicarboxylate (CAP) gel formulation in rhesus macaques as part of the development process for its use as a vaginally administered product in humans. The similarities between the reproductive physiology, anatomy and vaginal microflora of human and non-human primates makes non-human primates a relevant animal model to assess the safety and distribution of candidate anti-HIV microbicides. CAP gel was instilled once or once daily for 4 days into the vaginal vault of rhesus macaques. Colposcopy and magnetic resonance imaging were performed to detect adverse effects and spread of CAP, respectively. Additionally, vaginal pH and composition of the vaginal micorflora in macaques before, during and after CAP instillations were determined, and vaginal biopsies obtained following repeated CAP exposures were examined to further document its safety. CAP is safe for repeated use and exhibits a favorable distribution profile, showing no evidence of penetration into cells that line the vaginal epithelium. Further, the presence of CAP has no adverse effect on vaginal pH or the composition of the vaginal microflora, and does not induce vaginal epithelial thinning or inflammation. CAP gel shows minimal toxicity in vivo, supporting its use as a candidate vaginal microbicide in humans. [\hyperlink{Mycapssa}{PMID: 16184028}, Marion Ratterree et al., 2005]

\hypertarget{pmid_20885413}{C}onscious sedation for young children is a rapidly developing area of clinical activity. Many studies have shown positive results using oral midazolam on children. These case series investigated oral midazolam conscious sedation as an alternative to general anaesthesia in a clinical service setting. The purpose of this work was to determine the safety and efficacy of oral midazolam for conscious sedation in children undergoing dental treatment. Patients were selected by colleagues for treatment under oral sedation. The main general criteria were weight below 36 kilos and ASA I, II, or III. Midazolam 0.5 mg/kg was administered orally. A pulse oximeter was applied to a finger to monitor vital signs and the Houpt scale was used to assess behaviour. A total of 510 children aged between 13 months and 11 years were included. The behaviour of 379 (74\%) was excellent or very good. The pulse rate and peripheral oxygenation were within the normal range for all patients. The main adverse effects were diplopia and post-sedation dysphoria. Oral midazolam is a safe and effective method of sedation although some children were agitated and distressed either during or after treatment. Parents need to be warned about this. [\hyperlink{Mycapssa}{PMID: 20885413}, L Lourenço-Matharu et al., 2010]

\hypertarget{pmid_20527137}{O}nly a few corticosteroids for topical use have proven safe and effective in pediatric populations down to 3 months of age. The authors report the results of a study designed to assess the efficacy and safety of hydrocortisone butyrate (HCB) 0.1\% in lipocream (LCr) vehicle in infants and children. A total of 264 boys and girls 3 months to less than 18 years old, with stable, mild to moderate atopic dermatitis affecting at least 10\% body surface area applied HCB 0.1\% in LCr or LCr alone twice daily for up to 1 month without occlusion. Primary end-points included: percent of patients who achieved treatment success based on physician global assessments. Secondary endpoint included: difference in pruritus and Eczema Area and Severity Index (EASI) at day 29. Treatment was significant (P < 0.001) for HCB 0.1\% LCr over vehicle. No serious nor significant adverse events were reported. Results are representative of a short duration treatment for a chronic disease. HCB 0.1\% in LCr is more effective than its vehicle in pediatric populations down to 3 months of age without significant adverse events when used twice a day for up to 1 month. [\hyperlink{Mycapssa}{PMID: 20527137}, William Abramovits et al., ]

\hypertarget{pmid_6214182}{T}he clinical efficacy and safety of the new oxacephalosporin moxalactam disodium were evaluated in 54 children with a variety of pediatric infections. Except for a terminally ill neutropenic leukemic patient with pneumonia and sepsis due to Pseudomonas aeruginosa who  died shortly after initiation of therapy, moxalactam treatment was effective in all patients. No recurrent infections were observed. The rate of clinical response to moxalactam appeared to be at least comparable to that of patients treated with traditional antibiotics. In vitro sensitivity testing demonstrated that all bacteria isolated except P aeruginosa were sensitive to moxalactam while Haemophilus influenzae was exquisitely sensitive. Side effects included thrombocytosis (five patients), transient SGPT elevations and eosinophilia (three each), fever with rash (one), and neutropenia (one). In one patient, superinfection with Streptococcus faecalis developed. We conclude that moxalactam may be a useful antibiotic in pediatrics, particularly for the treatment of infections due to H. influenzae and Enterobacteriaceae. Its role in infections caused by group B streptococcus and Pseudomonas awaits further studies. [\hyperlink{Mycapssa}{PMID: 6214182}, R Yogev et al., 1982]

\hypertarget{pmid_10851644}{C}iprofloxacin clinical and bacteriological efficacies, as well as tolerability mainly with respect to chondrotoxicity were evaluated in the treatment of children with mucoviscidosis. The drug was shown to have high clinical and moderate bacteriological efficacies. As for its tolerability, adverse reactions chiefly associated with affection of the gastrointestinal tract, i.e. nausea, stomach pain, diarrhea, increased transaminase levels were recorded. The arthrotoxicity episode was single and transitory. The use of ciprofloxacin had no negative effect on the children growth. No chondrotoxic effect of ciprofloxacin in the treatment of children was observed which is explained in the paper. It is concluded that ciprofloxacin is in general an efficient and safe antibiotic useful for the treatment of children. [\hyperlink{Mycapssa}{PMID: 10851644}, S S Postnikov et al., 2000]

\hypertarget{pmid_31198261}{C}hildren who have experienced previous hospital admission, operation, procedures, and needle pricks are more reactive to subsequent anesthetic procedures. Many sedative agents have been used for the purpose of premedication, but few of them can be given orally, thus avoiding the pricks. Midazolam, being one such choices, can be given orally, intranasally, and parenterally but has unpredictable response. Triclofos, available as sweet syrup, is a phosphorylated derivative of chloral hydrate, has been proven to be effective within 30 min in doses of 25-75 mg/kg. Hence, this study compares triclofos hydrochloride with midazolam oral to know the efficacy of both the drugs as premedication. This study aims to assess sedation score, level of anxiety/resistance, and behavior of the child in the preoperative period. After parental and institutional approval, a total of 70 children were studied based on computer-generated randomization and divided into groups M and T of 35 each. Group M patients received oral midazolam 0.5 mg/kg. Group T patients received commercially available triclofos syrup containing 100 mg/ml of drug in dose of 75 mg/kg. The response of children to taste of premedication was noted, whether completely ingested or not. In case of vomiting, the child was excluded from further study. Numerical variables were analyzed using Student's paired  Sedation score at 5 min interval from 0 to 30 min showed  From data obtained, it can be concluded that parenteral formulation of either midazolam or triclofos can be safely used as premedicant in children. [\hyperlink{Mycapssa}{PMID: 31198261}, Ankesh Gupta et al., ]

\hypertarget{pmid_29498605}{O}BJECTIVE The aim of this paper was assess the efficacy and safety of using the MynxGrip arterial closure device in pediatric neuroendovascular procedures where the use of closure devices remains off-label despite their validation and widespread use in adults. METHODS A retrospective review of all pediatric patients who underwent diagnostic or interventional neuroendovascular procedures at the authors' institution was performed. MynxGrip use was predicated by an adequate depth of subcutaneous tissue and common femoral artery (CFA) diameter. Patients remained on supine bedrest for 2 hours after diagnostic procedures and for 3 hours after therapeutic procedures. Patient demographics, procedural details, hemostasis status, and complications were recorded. RESULTS Over 36 months, 83 MynxGrip devices were deployed in 53 patients (23 male and 30 female patients; mean age 14 years) who underwent neuroendovascular procedures. The right-side CFA was the main point of access for most procedures. The mean CFA diameter was 6.24 mm and ranged from 4 mm to 8.5 mm. Diagnostic angiography comprised 46\% of the procedures. A single device failure occurred without any sequelae; the device was extracted, and hemostasis was achieved by manual compression with the placement of a Safeguard compression device. No other immediate or delayed major complications were recorded. CONCLUSIONS MynxGrip can be used safely in the pediatric population for effective hemostasis and has the advantage of earlier mobilization. [\hyperlink{Mycapssa}{PMID: 29498605}, Tahaamin Shokuhfar et al., 2018]

\hypertarget{pmid_17901791}{L}evofloxacin has established efficacy and safety in the treatment of community-acquired pneumonia (CAP) in adults, and its use as an alternative therapy for children with CAP has been proposed. Assess the clinical efficacy and safety of levofloxacin compared with standard of care antibiotic therapy in the treatment of CAP in children aged 6 months to 16 years. In an open-label, multicenter, noninferiority trial, children with CAP were randomized 3:1 to receive levofloxacin or comparator antimicrobial therapy (0.5 to <5 years: amoxicillin/clavulanate or ceftriaxone; > or =5 years: clarithromycin or ceftriaxone with clarithromycin or erythromycin lactobinate) for 10 days. The primary outcome was cure rates at the test-of-cure visit (10-17 days after completing treatment) as determined by symptoms, physical examination, and chest radiography. Seven hundred and thirty-eight children were enrolled and 539 (405 levofloxacin-treated, 134 comparator-treated) were clinically evaluable at test-of-cure visit. Clinical cure rates were 94.3\% (382 of 405) in levofloxacin-treated and 94.0\% (126 of 134) in comparator-treated children. Cure rates were also similar for levofloxacin and comparator for each age group (<5 years, 92.2\% versus 90.8\%; > or =5 years, 96.5\% versus 97.1\%; respectively) and for children categorized as being at higher risk for severe disease. Mycoplasma pneumoniae was the most frequently identified cause of pneumonia (230 children). Levofloxacin was as well tolerated as comparators, with similar type and incidence of adverse events. Levofloxacin was as well tolerated and effective as standard-of-care antibiotics for the treatment of CAP in infants and children. [\hyperlink{Mycapssa}{PMID: 17901791}, John S Bradley et al., 2007]

\hypertarget{pmid_22150581}{T}he modified Yale Preoperative Anxiety Scale (m-YPAS) is an observational behavioral checklist that has been widely used as an indicator of pre-operative anxiety in children. The present study describes the translation process of m-YPAS into Swedish and the testing of its reliability and validity when used with Swedish children. The questionnaire was translated using standard forward-back-forward translation technique. The validation process was divided into two phases: a pilot study with 61 children as a first version and a test of a final version with 102 children. The reliability tested with Cronbach's alpha was acceptable to good. Interrater reliability analyzed with weighted kappa was acceptable to good with Students Registered Nurse Anesthetists and Certified Registered Nurse Anesthetist (CRNA) as evaluators (phase 1) and good to excellent with CRNA's very experienced in child anesthesia (phase 2). Both concurrent and constructed validity could be demonstrated. This validation study of the Swedish version of the m-YPAS shows good consistency, interrater validity, and construct validity when used by experienced assessors. [\hyperlink{Mycapssa}{PMID: 22150581}, M Proczkowska-Björklund et al., 2012]

\hypertarget{pmid_17482021}{P}ediatric patients present unique concerns in the field of medical toxicology. First, there are medicines that are potentially dangerous to small children, even when they are exposed to very small amounts. Clinicians should be wary of these drugs even when young patients present with accidental ingestions of apparently insignificant amounts. Next, over-the-counter laxatives and syrup of ipecac, although not commonly considered abused substances, may be misused in both the setting of Munchausen's syndrome by proxy and in adolescents who have eating disorders. Their use should be considered in any gastrointestinal illness of uncertain origin. Finally, as the use of syrup of ipecac at home now has been discouraged by many, some have explored using activated charcoal at home as a new method of prehospital gastrointestinal decontamination. The literature examining activated charcoal and its use in this capacity is discussed. [\hyperlink{Mycapssa}{PMID: 17482021}, David L Eldridge et al., 2007]

\hypertarget{pmid_19438539}{T}he prevalence of atopic diseases in the Western world is rising while infectious diseases decline. The 'hygiene hypothesis' suggests that reduced exposure to microbes such as mycobacteria in early life is associated with increased atopic disease. Recent research showed that Mycobacterium vaccae reduced the severity of atopic dermatitis (AD) in children. To evaluate the efficacy of a derivative of heat-killed M. vaccae in children with AD. In total, 129 children, aged 5-16 years old with moderate to severe AD participated in this randomized, double-blind, placebo-controlled trial. Participants received an intradermal injection of either M. vaccae or placebo three times at 2-weekly intervals. The two groups were compared for changes in severity and extent of AD from baseline to 3 and 6 months after treatment. There was no significant difference between the two groups for change in severity of AD at 3 and 6 months (P = 0.77 and P = 0.70, respectively) or in extent of disease at 3 months (P = 1.0). Local injection-site reactions occurred in 47\% of participants, of whom 75\% received M. vaccae. In this study, M. vaccae did not improve AD significantly in children with moderate to severe disease. [\hyperlink{Mycapssa}{PMID: 19438539}, S Brothers et al., 2009]

\hypertarget{pmid_21420573}{M}icafungin is an echinocandin approved by the European Medicines Evaluation Agency for the treatment of invasive candidiasis in children, including premature infants born before 29 weeks of pregnancy, and as prophylaxis in children undergoing hematopoietic stem-cell transplantation or patients at risk of prolonged neutropenia. This drug has good activity in several Candida spp., including those resistant to fluconazole. Although micafungin is active against Aspergillus spp., it has been used mainly in combination therapy for invasive aspergillosis. There is ample information on the use of micafungin in children, including neonates, and this drug is the only echinocandin approved for use in infants aged less than 3 months. The efficacy, pharmacokinetics and safety of micafungin have been evaluated in phase II and III clinical trials in children, in which its efficacy and safety were demonstrated in comparison with liposomal amphotericin B and fluconazole. The pharmacokinetic profile of micafungin in children allows once daily intravenous administration, with greater clearance than in adults, and consequently pediatric doses are relatively higher. The most appropriate dose in children weighing less than 40 kg is 2 mg/kg/day in the treatment of invasive candidiasis and 1 mg/kg/day as prophylaxis in children undergoing hematopoietic stem-cell transplantation. Doses in neonates should be higher. In premature infants, the most appropriate doses to achieve levels in the brain parenchyma are 7 mg/kg/day and 10 mg/kg/day in those weighing more and less than 1,000 g, respectively. Micafungin has few drug-drug interactions and an acceptable safety profile. Withdrawal of this drug due to adverse effects is rare, although transaminase monitoring is recommended during treatment, as well as evaluation of the risk-benefit balance in patients with liver disease or concomitant administration of hepatotoxic drugs. [\hyperlink{Mycapssa}{PMID: 21420573}, José Tomás Ramos Amador et al., 2011]

\section*{Talicia}
\subsection*{Result}
\subsubsection*{Answer}

Unknown

\subsubsection*{{Explanation}}
\hypertarget{Talicia}
After reviewing all available abstracts, there is no abstract specifically about Talicia or its use in children. None of the abstracts mention Talicia, nor do they present targeted studies evaluating the safety of Talicia in pediatric populations of any age range. Therefore, based on the abstracts provided, there is not enough data to determine whether Talicia is safe or unsafe for use in children. The safety of Talicia in children is unknown according to the evidence available in these abstracts.

\subsection*{Abstracts}
\hypertarget{pmid_24799812}{T}o assess the level of homocysteine (tHcy) in children taking AEDs and to study whether daily oral supplementation of folic acid for 1 month will reduce the tHcy level. This was a double-blinded, randomized control trial conducted in Institute of Maternal and Child Health, Kozhikode, India. Totally 60 children were recruited and of them, 48 were enrolled. Of these children, 32 were assigned to the experimental group and 16 to the control group. Baseline data collection and tHcy estimation were done. One mg folic acid tablets were given to the experimental group and placebo tablets to the control group for 30 days. tHcy levels were re-estimated after 1 month follow-up. Statistical significance was tested by χ(2) test, and paired and unpaired t-tests, as appropriate. Correlation was tested by Pearson correlation test and P value less than 0.05 was taken as the cut-off for statistical significance. Baseline plasma tHcy concentrations in both groups were comparable [11.90 (6.3) and 13.02 (2.4) μmol/l, respectively]. During the follow-up period, no increase in seizure episodes or no serious adverse reactions were noticed in either group. The reduction of tHcy in the experimental group was 1.92 μmol/l (P = 0.04) and in the control group, there was an increase of 1.05 μmol/l (P = 0.16). In children on AED treatment, folic acid supplementation may reduce tHcy level and thus reduce CVD risk. [\hyperlink{Talicia}{PMID: 24799812}, Mathummal Cherumanalil Jeeja et al., 2014]

\hypertarget{pmid_36681802}{A}nti-influenza treatment is important for children and is recommended in many countries. This study assessed safety, clinical, and virologic outcomes of baloxavir marboxil (baloxavir) treatment in children based on age and influenza virus type/subtype. This was a post hoc pooled analysis of two open-label non-controlled studies of a single weight-based oral dose of baloxavir (day 1) in influenza virus-infected Japanese patients aged < 6 years (n = 56) and ≥ 6 to < 12 years (n = 81). Safety, time to illness alleviation (TTIA), time to resolution of fever (TTRF), recurrence of influenza illness symptoms and fever (after day 4), virus titer, and outcomes by polymerase acidic protein variants at position I38 (PA/I38X) were evaluated. Adverse events were reported in 39.0 and 39.5\% of patients < 6 years and ≥ 6 to < 12 years, respectively. Median (95\% confidence interval) TTIA was 43.2 (36.3-68.4) and 45.4 (38.9-61.0) hours, and TTRF was 32.2 (26.8-37.8) and 20.7 (19.2-23.8) hours, for patients < 6 years and ≥ 6 to < 12 years, respectively. Symptom and fever recurrence was more common in patients < 6 years with influenza B (54.5 and 50.0\%, respectively) compared with older patients (0 and 25.0\%, respectively). Virus titers declined (day 2) for both age groups. Transient virus titer increase and PA/I38X-variants were more common for patients < 6 years. The safety and effectiveness of single-dose baloxavir were observed in children across all age groups and influenza virus types. Higher rates of fever recurrence and transient virus titer increase were observed in children < 6 years. Japan Pharmaceutical Information Center Clinical Trials Information JapicCTI-163,417 (registered 02 November 2016) and JapicCTI-173,811 (registered 15 December 2017). [\hyperlink{Talicia}{PMID: 36681802}, Nobuo Hirotsu et al., 2023]

\hypertarget{pmid_16630377}{I}n many African countries, trimethoprim-sulfamethoxazole (TS) is recommended for the treatment of children with malaria and pneumonia - in accordance with the guidelines for the integrated management of childhood illness (IMCI) - and, in some settings, for the home management of febrile illnesses. There have been few studies, however, of the risk of failure of treatment with this drug combination in children with acute, Plasmodium falciparum malaria. The factors that identify children at risk of treatment failure after being given TS were therefore evaluated in 101 children with acute, symptomatic, uncomplicated, P. falciparum malaria, in a hyper-endemic area of south-western Nigeria. Overall, 11\% of the children failed treatment by day 14. In a multivariate analysis, two factors were found to be independent predictors of the failure of treatment with TS: an age of <3 years (adjusted odds ratio=0.1; 95\% confidence interval=0.02-0.53; P=0.007); and a body temperature of >or=38 degrees C 2 days after the commencement of treatment (adjusted odds ratio=4.9; 95\% confidence interval=1.2-21.3; P=0.03). These findings may have implications for control efforts in some sub-Saharan African countries, where TS is recommended for the management of malaria in children, with or without pneumonia. [\hyperlink{Talicia}{PMID: 16630377}, A Sowunmi et al., 2006]

\hypertarget{pmid_26732220}{R}TS,S is the first vaccine to have received a positive opinion from the European Medicines Agency (EMA) under Article 58, for vaccination of young children aged from 6 weeks up to 17 months against malaria caused by Plasmodium falciparum and against hepatitis B. Although vaccine efficacy is modest and wanes rapidly, a substantial number of cases of clinical malaria can be averted, particularly in settings with high disease burden. Further evaluations are needed regarding safety, and more specifically regarding efficacy against severe malaria and mortality. The current formulation, however, is a milestone as a gold standard and represents a basis for further required improvements. Evaluation of the benefits, risks and feasibility are anticipated at global and national levels.  [\hyperlink{Talicia}{PMID: 26732220}, Robert W Sauerwein et al., 2015] Trauma is the leading cause of morbidity and mortality in children in the United States. The antifibrinolytic drug tranexamic acid (TXA) improves survival in adults with traumatic hemorrhage, however, the drug has not been evaluated in a clinical trial in severely injured children. We designed the Traumatic Injury Clinical Trial Evaluating Tranexamic Acid in Children (TIC-TOC) trial to evaluate the feasibility of conducting a confirmatory clinical trial that evaluates the effects of TXA in children with severe trauma and hemorrhagic injuries. Children with severe trauma and evidence of hemorrhagic torso or brain injuries will be randomized to one of three arms: (1) TXA dose A (15 mg/kg bolus dose over 20 min, followed by 2 mg/kg/hr infusion over 8 h), (2) TXA dose B (30 mg/kg bolus dose over 20 min, followed by 4 mg/kg/hr infusion over 8 h), or (3) placebo. We will use permuted-block randomization by injury type: hemorrhagic brain injury, hemorrhagic torso injury, and combined hemorrhagic brain and torso injury. The trial will be conducted at four pediatric Level I trauma centers. We will collect the following outcome measures: global functioning as measured by the Pediatric Quality of Life (PedsQL) and Pediatric Glasgow Outcome Scale Extended (GOS-E Peds), working memory (digit span test), total amount of blood products transfused in the initial 48 h, intracranial hemorrhage progression at 24 h, coagulation biomarkers, and adverse events (specifically thromboembolic events and seizures). This multicenter trial will provide important preliminary data and assess the feasibility of conducting a confirmatory clinical trial that evaluates the benefits of TXA in children with severe trauma and hemorrhagic injuries to the torso and/or brain. ClinicalTrials.gov registration number: NCT02840097 . Registered on 14 July 2016. [\hyperlink{Talicia}{PMID: 26732220}, Daniel K Nishijima et al., 2018]

\hypertarget{pmid_29661382}{O}ne major criticism of prolonged propofol-based total i.v. anaesthesia (TIVA) in children is the prolonged recovery time. As target-controlled infusion (TCI) obviates the need to manually calculate the infusion rate, the use of TCI may better match clinical requirements, reduce propofol dose, and shorten recovery time. Children of ASA grade 1, aged 1-12 yr, were recruited and randomly assigned to TCI or manual infusion. Children in the TCI group had propofol delivered by TCI. Children for manual infusion had a loading dose of 2.5 mg kg Seventy-four children completed the study. The time taken to extubate the trachea after cessation of propofol was 15.1 (5.5) and 16.2 (6.1) min for children who had TCI and manual infusion, respectively (P=0.42). The mean propofol infusion rate was 16.7 [standard deviation (sd) 4.2] mg kg Use of TCI led to higher propofol doses but not prolonged recovery time in children compared with manual infusion. It was associated with a greater percentage of time when the BIS was in the desired range and it may be an easier method for titration of propofol administration during anaesthesia or sedation. ChiCTR-IOD-16010147. [\hyperlink{Talicia}{PMID: 29661382}, J Mu et al., 2018]

\hypertarget{pmid_31882189}{T}hallium (TI) is one of the most toxic heavy metals and priority pollutant metals. The emerging TI environmental pollution worldwide has posed a great threat to human health. However, based on the World Health Organization (WHO), the risk and severity of adverse health effects of TI in the range of 5-500 μg/L are uncertain. Moreover, evidence regarding the adverse impacts of TI on children's health is still insufficient. Herein, we aim to investigate the early adverse effects of TI on children's health and provide references for the WHO to establish stricter safety limits of TI. From 2015 to 2019, urinary TI and many clinical laboratory parameters related to blood routine, hepatic, renal, myocardial, coagulation function and serum electrolyte were measured in six children aged 1-9 years. The urinary TI concentration ranged from 13.4 μg/L to 60.1 μg/L with a mean of 36.1 μg/L and a median of 34.8 μg/L in six children in 2015. Although only four children felt a little poor appetite, several laboratory abnormalities indicated early damage in liver, renal, and myocardial functions in all children in 2015. After treatment and following up for four years, although the children's TI concentration decreased below 5 μg/L, their liver and renal functions did not completely recover, and their myocardial function worsened. Results indicated that impaired liver, renal, and myocardial functions were closely associated with elevated urinary TI concentration in children. Considering the increasing use of TI in high-technology industries and emerging TI environmental-contamination zones worldwide, establishing stricter safety limits of TI and paying more attention to the adverse health effects of TI on children are urgently required. SUMMARY: We found that a relatively low concentration of thallium (13.4 μg/L to 60.1 μg/L) impaired liver, renal, and myocardial function in six children. After treatment and following up these children for four years, although their urinary TI concentration decreased below 5 μg/L, their liver and renal functions did not completely recover, and their myocardial function worsened. [\hyperlink{Talicia}{PMID: 31882189}, Weixia Duan et al., 2020]

\hypertarget{pmid_7594705}{T}he safety, tolerability, and pharmacokinetics of zalcitabine (ddC) in a single oral dose (0.02 mg/kg) was evaluated in 23 mildly symptomatic human immunodeficiency virus-infected children (mean age, 4.2 years). After administration of ddC, blood samples were obtained at 0.5, 1, 1.5, 2, 4, 6, and 8 h for analysis. The drug was well tolerated and no side effects were noted. Plasma ddC levels were determined by ion spray liquid chromatography/tandem mass spectrometry. ddC was rapidly absorbed, with a mean maximum plasma concentration of 9.3 ng/mL (range, 3.2-14.1) attained within a mean of 1 h (range, 0.5-2.0). Mean elimination half-life was 1.4 h (range, 1.0-3.5), mean area under the plasma concentration-time curve was 25 ng.h/mL (range, 11-37), and mean total body clearance was 14.6 mL/min/kg (range, 8.9-30.6). Plasma concentrations were lower and the half-life shorter in these children than in adults given comparable doses, suggesting that ddC may be cleared more rapidly in children than adults. [\hyperlink{Talicia}{PMID: 7594705}, E G Chadwick et al., 1995]

\hypertarget{pmid_9099329}{A}cute percutaneous salicylate intoxication is a rare event in children but can happen with a skin disease where salicylic acid, used as a keratolytic ointment, can be absorbed transcutaneously. Until now, few cases of transcutaneous salicylate intoxication have been reported in the literature. Our case report is about a 5-year-old girl with lamellar ichthyosis and an acute salicylate transcutaneous intoxication after the application of a skin ointment. The child had a fever, hyperpnoea with respiratory alkalosis, comatose state and oculogyric crisis. We would like to emphasize the danger of applying salicylic acid in children with extensive skin diseases and, therefore, it is advisable to measure the plasma salicylic levels so as to prevent eventual salicylate toxicity. [\hyperlink{Talicia}{PMID: 9099329}, A Chiaretti et al., 1997]

\hypertarget{pmid_33245023}{S}uspected pediatric ingestions of greater than or equal to one teaspoon topical salicylate analgesic are recommended by poison control centers to be managed at healthcare facilities. This cutoff is applied for both liquid and non-liquid (cream, ointment, gel) formulations. California poison control cases involving topical salicylate exposures in children less than 6-years-old who were evaluated at a health care facility between 2003 and 2018 were analyzed. Of 599 patient cases, the majority described no or minor symptoms, with gastrointestinal distress being the most common. Signs of salicylate toxicity (metabolic acidosis, tachypnea) occurred in six cases. Seven patients were hospitalized, six of whom were exposed to liquid preparations. In line with previous research, liquid salicylate preparations were more frequently associated with the signs of salicylate toxicity and hospitalization. There was a low frequency of severe side effects and low hospitalization rates among those referred to a healthcare facility, especially for non-liquid topical salicylate ingestions. [\hyperlink{Talicia}{PMID: 33245023}, Kate Arriola et al., 2021]

\hypertarget{pmid_22210663}{T}o assess the safety and efficacy of US-guided CS injection done by a paediatric rheumatologist into the TM joints (TMJs) in children with JIA. Children with JIA presenting to the rheumatology clinic were assessed for TMJ arthritis. Triamcinolone hexacetonide was injected in children with active arthritis assessed by MRI using US guidance under general anaesthesia by the same paediatric rheumatologist trained in the procedure. Efficacy and safety were assessed post-injection by patient-guided symptoms and physical examination. Thirty-eight children (34 girls) with JIA who had TMJ injection done between January 2009 and January 2011 were included in the analysis. Mean age was 12.25 (± 3.55) (range 5-18) years. The mean disease duration was 4.54 (± 2.73) (range 1.5-11.1) years. Symptoms pre-injection were pain in 17/38 (44.7\%), jaw deviation in 14/38 (36.8\%), restricted jaw movement in 13/38 (34.2\%) and chewing dysfunction in 7/38 (18.4\%). Five (12.5\%) children had micrognathia. A total of 63 joints were injected. The injection was efficacious in 58/63 (92.06\%) joints. All 17 (100\%) children had resolution of pain, and chewing dysfunction improved in 5/7 (71.4\%). Jaw deviation improved in 13/14 (92.8\%). In the 5/63 (7.9\%) injections that were not efficacious, two children with both TMJs injected (four joints) had persisting stiffness with chewing dysfunction and one had persistent jaw deviation. One child developed a scar at the site of injection. US-guided CS injection into the TMJ done by a paediatric rheumatologist trained in the procedure is safe with a high rate of success. [\hyperlink{Talicia}{PMID: 22210663}, Shabina Habibi et al., 2012]

\hypertarget{pmid_8415309}{A} child affected by giant recurrent aphthous ulcers was treated successfully over the long term with thalidomide, with no adverse reactions or reduction of therapeutic efficacy. The use of thalidomide in children for serious aphthosis is proposed. [\hyperlink{Talicia}{PMID: 8415309}, S Menni et al., 1993]

\hypertarget{pmid_22477803}{S}afety and efficacy issues regarding over-the-counter cough and cold (CAC) products for use in children have surfaced. Late in 2007 the FDA began reviewing CAC product status for use in children under 6 years old. In regards to CAC products for children < 6 years old, to determine pharmacists: 1) comfort level in recommending; 2) attitudes towards behind-the-counter (BTC) status; and 3) level of support for BTC status. An additional objective was to determine how frequently pharmacists were asked for CAC product recommendations for children Georgia Pharmacy Association members (2,045) were invited to anonymously participate in a self-administered online survey from January 3 - Feb 6, 2008. Topic areas included demographics, comfort in recommending CAC, and BTC status. Most responding pharmacists (99.1\%) feel pediatric CAC medicine safety problems are due to inappropriate use. More than 50\% of chain and independent pharmacists were asked to recommend CAC medicines for children during cold/flu season once a day or less, and 79\% reported counseling on less than 50\% of total CAC sales. The majority of pharmacists felt comfortable recommending CAC medications when thinking of both safety and efficacy. Most pharmacists supported a BTC condition of sale for children under two for decongestants, antihistamines, and antitussives, and for decongestants and antitussives for children between 2 and 5 years old. Most pharmacists indicate comfort in recommending CAC despite lack of evidence for safety or efficacy and support BTC status. Pharmacist education on this topic would be useful. [\hyperlink{Talicia}{PMID: 22477803}, Sally A Huston et al., 2010]

\hypertarget{pmid_31476938}{T}he topical calcineurin inhibitors (TCIs), tacrolimus (Protopic) and pimecrolimus (Elidel), were approved in the early 2000s and were widely used as a nonsteroid treatment for atopic dermatitis (AD) in adult and pediatric populations. In 2005, the addition of a boxed warning was mandated for TCIs based on a potential risk of malignancy, and there was subsequently a substantial decline in their use. Since then, evidence has mounted to support the safety of this class of medications and suggests that the increased risk of malignancy remains theoretical. This review aims to dispel some of the common myths surrounding the safety of TCIs by evaluating the key evidence regarding their safety and tolerability in adult and pediatric populations. Four major themes are addressed in a practical question-and-answer format: the risk of harm associated with TCIs including common and serious adverse events; warnings and precautions for their use including the risk of systemic absorption, immunosuppression, and malignancy; the comparative safety of TCIs; and suggestions for counselling patients about the risk of harm with TCIs. Based on the available evidence, international professional dermatological organizations and regulatory authorities have concluded that the benefits of TCIs outweigh their potential risks when used in the appropriate patient populations for the recommended duration of time. [\hyperlink{Talicia}{PMID: 31476938}, Sam Hanna et al., ]

\hypertarget{pmid_34798685}{T}opical tacrolimus is used off-label in young children, but data are limited on its use in children under 2 years of age and for long-term treatment. To compare safety differences between topical tacrolimus (0.03\% and 0.1\% ointments) and topical corticosteroids (mild and moderate potency) in young children with atopic dermatitis (AD). We conducted a 36-month follow-up study with 152 young children aged 1-3 years with moderate to severe AD. The children were followed up prospectively, and data were collected on infections, disease severity, growth parameters, vaccination responses and other relevant laboratory tests were gathered. There were no significant differences between the treatment groups for skin-related infections (SRIs) (P = 0.20), non-SRIs (P = 0.20), growth parameters height (P = 0.60), body weight (P = 0.81), Eczema Area and Severity Index (EASI) (P = 0.19), vaccination responses (P = 0.62), serum cortisone levels (P = 0.23) or serum levels of interleukin (IL)-4, IL-10, IL-12, IL-31 and interferon-γ. EASI decreased significantly in both groups (P < 0.001). In the tacrolimus group, nine patients (11.68\%) had detectable tacrolimus blood concentrations at the 1-week visit. There were no malignancies or severe infections during the study, and blood eosinophil counts were similar in both groups. Topical tacrolimus (0.03\% and 0.1\%) and topical corticosteroids (mild and moderate potency) are safe to use in young children with moderate to severe AD, and have comparable efficacy and safety profiles. [\hyperlink{Talicia}{PMID: 34798685}, A Salava et al., 2022]

\hypertarget{pmid_21487326}{T}he tuberculin skin test (TST) is often used to screen for latent tuberculosis infection (LTBI) in school children, many of whom were bacille Calmette-Guérin (BCG)-vaccinated in infancy. The reliability of the TST in such children is unknown. TSTs performed in low-risk BCG-vaccinated and -nonvaccinated grade 1 and grade 6 First Nations (North American Indian) school children in the province of Alberta, Canada, were evaluated retrospectively. To further assess the specificity of the TST, BCG-vaccinated children with a positive TST (≥10 mm of induration) and no treatment of LTBI were administered a QuantiFERON-TB Gold In-Tube test (QFT-GIT, Cellestis International). A total of 3996 children, 2063 (51.6\%) BCG-vaccinated and 1933 (48.4\%) BCG-nonvaccinated, were screened for LTBI. Vaccinated children were more likely than nonvaccinated children to be TST positive (5.7\% vs. 0.2\%, P < 0.001). Vaccinated children with a positive TST were more likely to have a recent past TST as compared with those with a negative TST (6.8\% versus 2.8\%, P = 0.01). Among 65 BCG-vaccinated TST-positive children who underwent a QFT-GIT, only 5 (7.7\%; 95\% CI: 2.5\%, 17.0\%) were QFT-GIT positive. A TST of ≥15 mm was more likely to be associated with a positive QFT-GIT than a TST of 10 to 14 mm, 16.0\% (95\% CI: 4.5\%, 36.1\%) versus 2.5\% (95\% CI: 0.1\%, 13.2\%), P = 0.047. The TST is unreliable in school children, BCG-vaccinated in infancy, and who are at low risk of infection. The QFT-GIT is a useful confirmatory test for LTBI in BCG-vaccinated TST-positive school children. [\hyperlink{Talicia}{PMID: 21487326}, Sandy Jacobs et al., 2011]

\hypertarget{pmid_33832544}{I}n children, up to 30\% of viral respiratory tract infections (RTIs) develop into bacterial complications associated with pneumonia, sinusitis or otitis media to trigger a tremendous need for antibiotics. This study investigated the efficacy of Echinacea for the prevention of viral RTIs, for the prevention of secondary bacterial complications and for reducing rates of antibiotic prescriptions in children. Echinaforce® Junior tablets [400 mg freshly harvested Echinacea purpurea alcoholic extract] or vitamin C [50 mg] as control were given three times daily for prevention to children 4-12 years. Two × 2 months of prevention were separated by a 1-week treatment break. Parents assessed respiratory symptoms in children via e-diaries and collected nasopharyngeal secretions for screening of respiratory pathogens (Allplex® RT-PCR). Overall, 429 cold days occurred in N Our results support the use of Echinacea for the prevention of RTIs and reduction of associated antibiotic usage in children. Trial registration clinicaltrials.gov, NCT02971384, 23th Nov 2016. [\hyperlink{Talicia}{PMID: 33832544}, Mercedes Ogal et al., 2021]

\hypertarget{pmid_25338496}{T}o define the efficacy and safety of low-dose rasburicase in children from south India with hematologic malignancies. This study is a retrospective analysis of data on 41 children with hematologic malignacies with laboratory evidence of tumor lysis syndrome (TLS) or clinical features indicating high risk for developing TLS. Patients were treated with rasburicase in doses of 0.1-0.15 mg/kg dose, repeated when necessary. Male : Female ratio was 32:9. Thirty-six children had laboratory evidence of TLS and 5 were at risk for TLS. Diagnoses were T-cell acute lymphoblastic leukemia (ALL), 19; Pre-B ALL, 17; B-non-Hodgkin lymphoma (NHL), 2; T-NHL, 2; and acute myeloid leukemia (AML), 1. Initial plasma uric acid (PUA): median, 8.5 mg/dl (range, 4.3 to 45.5). Six had creatinine levels of >2 mg/dl on admission; and 10 had peak PO4 levels of >10 mg/dl. Dose of rasburicase used: median, 0.12 mg/kg (range, 0.08-0.24). Median reduction of PUA at 6 h: 80 \% (range 40 to 98 \%). Twenty-seven needed only one dose; 12 needed 2 or 3 doses; and two needed 5 doses each. One child required dialysis. None of the children developed anaphylaxis or hemolysis and there were no deaths from TLS. Low-dose rasburicase (0.1-0.15 mg/kg) is safe and effective in reducing PUA in Indian children with lymphoid malignancies, and thus it may reduce the risk of renal failure from TLS. [\hyperlink{Talicia}{PMID: 25338496}, Somasundaram Jayabose et al., 2015]

\hypertarget{pmid_26861518}{I}nfantile hemangioma is the most common benign vascular tumor of childhood that has a tendency for spontaneous involution. The aim of this study was to evaluate the efficacy of topical timolol maleate in the treatment of superficial infantile hemangioma and associated side effects during the course of treatment. Four boys and five girls with a median age of 5 months were reviewed at 2-week intervals for a period of 16 weeks. A decrease in size, color, and consistency were noted. Adverse effects caused by timolol maleate were noted and managed. Of nine cases, two patients showed excellent response, five showed good response, one showed partial response, and one had poor response. Topical timolol maleate is safe and effective in the treatment of infantile hemangioma.  [\hyperlink{Talicia}{PMID: 26861518}, Abhijeet Kumar Jha et al., ] Sildenafil (Revatio®) and tadalafil (Adcirca®) are specific inhibitors of the phosphodiesterase-5 enzyme and produce pulmonary vasodilation by inhibiting the breakdown of cyclic guanosine monophosphate (cGMP) in the walls of pulmonary arterioles. We focus on the efficacy and safety of sildenafil and tadalafil in the treatment of pulmonary hypertension (PH) in children through a PubMed literature search. Although used since 1999 in the treatment of PH in children, it is only in the past few years that robust evidence for the use of sildenafil has emerged principally in the pivotal STARTS-1 study. The open-label extension of this study, STARTS-2, has revealed safety concerns substantiated by FDA post marketing surveillance leading to recommendations to use lower doses. More recently, tadalafil has been introduced allowing once daily dosing with apparently similar efficacy to sildenafil in children. Recently there have been suggestions that sildenafil and tadalafil may have a place in treating muscular dystrophy. [\hyperlink{Talicia}{PMID: 26861518}, Alan G Magee et al., 2015] 92.3\% schoolchildren aged 6-13 years of a mexican rural village, suspected foci of Taenia solium cysticercosis were screened for intestinal parasites with the main purpose to know the infection rate by taeniasis. An stool sample was collected to schoolchildren of the village and 95.4\% of a urban private school as comparative group. Laboratory examinations were performed with the most accurate technics, included microscopies with an ocular micrometer. The general parasitation rate was 4 times higher in the rural village, but the percentages of Taenia spp. infection were 0.6\% both of them. Entamoeba histolytica was observed 1.8\% and 7.2\% in the city and rural village, respectively. All the cases with taeniasis passed T. saginata after treatment with niclosamide. Negative results were obtained with the same chemotherapy in a randomly selected group of 112 schoolchildren which previous stool examination was reported negative. Neither taeniasis were demonstrated in 94 adult persons. These data are suggestive of the great variability on the transmission rates of T. solium cysticercosis in endemic areas and illustrate the faced methodological problems to confirm the diagnosis of taeniasis. By other hand support the hypothesis that estimates of infection rates with E. histolytica have been overdiagnosed in the country. Taeniasis-cysticercosis; schoolchildren; Taenia saginata; amebiasis. [\hyperlink{Talicia}{PMID: 26861518}, R Lara-Aguilera et al., 1990]

\hypertarget{pmid_14693538}{E}mtricitabine (FTC; Emtriva), a potent deoxycytidine nucleoside reverse transcriptase inhibitor, has recently been approved by the U.S. Food and Drug Administration for the treatment of human immunodeficiency virus (HIV) infection. In adults, FTC has demonstrated linear kinetics over a wide dose range, and FTC 200 mg once a day (QD) is the recommended therapeutic dose. A phase I open-label trial was conducted in children to identify an FTC dosing regimen that would provide comparable plasma exposure to that observed in adults at 200 mg QD. Two single oral doses of FTC (60 and 120 mg/m(2), up to a maximum of 200 mg, in solutions) were evaluated in HIV-infected children aged <18 years old. Children >/=6 years old also received a third dose of approximately 120 mg/m(2) in capsules. A total of 25 children (two <2 years old, eight 2 to 5 years old, eight 6 to 12 years old, and seven 13 to 17 years old) received at least two doses of FTC. Single escalating oral doses of FTC were well tolerated and produced dose-proportional plasma drug concentrations in children. The FTC pharmacokinetics was comparable between adults and children 22 months to 17 years of age. The capsule formulation provided approximately 20\% higher plasma FTC exposure than the solution formulation. Using plasma area under the concentration-time curve (AUC) data at the 120-mg/m(2) dose, it is projected (based on dose proportionality) that a 6-mg/kg dose (up to a maximum of 200 mg) of FTC would produce plasma AUCs in children comparable to those in adults given a 200-mg dose (i.e., median of approximately 10 h. micro g/ml). This pediatric FTC dose is being evaluated in long-term phase II therapeutic trials in HIV-infected children. [\hyperlink{Talicia}{PMID: 14693538}, Laurene H Wang et al., 2004]

\hypertarget{pmid_29442362}{S}aliva, as a matrix, offers many benefits over blood in therapeutic drug monitoring (TDM), in particular for infantile TDM. However, the accuracy of salivary TDM in infants remains an area of debate. This review explored the accuracy, applicability and advantages of using saliva TDM in infants and neonates. Databases were searched up to and including September 2016. Studies were included based on PICO as follows: P: infants and neonates being treated with any medication, I: salivary TDM vs. C: traditional methods and O: accuracy, advantages/disadvantages and applicability to practice. Compounds were assessed by their physicochemical and pharmacokinetic properties, as well as published quantitative saliva monitoring data. Twenty-four studies and their respective 13 compounds were investigated. Four neutral and two acidic compounds, oxcarbazepine, primidone, fluconazole, busulfan, theophylline and phenytoin displayed excellent/very good correlation between blood plasma and saliva. Lamotrigine was the only basic compound to show excellent correlation with morphine exhibiting no correlation between saliva and blood plasma. Any compound with an acid dissociation constant (pKa) within physiological range (pH 6-8) gave a more varied response. There is significant potential for infantile saliva testing and in particular for neutral and weakly acidic compounds. Of the properties investigated, pKa was the most influential with both logP and protein binding having little effect on this correlation. To conclude, any compound with a pKa within physiological range (pH 6-8) should be considered with extra care, with the extraction and analysis method examined and optimized on a case-by-case basis. [\hyperlink{Talicia}{PMID: 29442362}, Laura Hutchinson et al., 2018]

\hypertarget{pmid_10961786}{T}he objective of this study was to determine the safety and tolerability of the immunomodulatory agent thalidomide as adjunct therapy in children with tuberculous meningitis. Children with stage 2 tuberculous meningitis received oral thalidomide for 28 days in a dose-escalating study, in addition to standard four-drug antituberculosis therapy, corticosteroids, and specific treatment of complications such as raised intracranial pressure. Clinical and laboratory evaluations were carried out. Fifteen patients (median age, 34 months) were enrolled. Thalidomide was administered via nasogastric tube in a dosage of 6 mg/kg/day, 12 mg/kg/day, or 24 mg/kg/day. The only adverse events possibly related to the study drug were transient skin rashes in two patients. Levels of tumor necrosis factor-alpha in the cerebrospinal fluid decreased markedly during thalidomide therapy. Clinical outcome and neurologic imaging showed greater improvement than that experienced with historical controls. Thalidomide appeared safe and well tolerated in children with stage 2 tuberculous meningitis and could have important anti-inflammatory effects. These promising results have led us to embark on a randomized, double-blind, placebo-controlled trial of the efficacy of thalidomide in tuberculous meningitis. [\hyperlink{Talicia}{PMID: 10961786}, J F Schoeman et al., 2000]

\hypertarget{pmid_29654485}{E}fficacy and safety of tocilizumab (TCZ), an interleukin-6 receptor inhibitor, were demonstrated in juvenile idiopathic arthritis (JIA) with polyarticular course (pJIA) in the CHERISH trial. This observational, III phase study evaluated long-term treatment of TCZ in pJIA patients was conducted by members of the Pediatric Rheumatology International Trials Organization (PRINTO) from Poland and Russia. Forty-one patients, who had completed the CHERISH core study (104 weeks), were extensionally treated with TCZ (8 mg/kg, intravenous infusion every 4 weeks). Total treatment time was from 131 to 193 weeks. The long-term safety (the primary endpoint) and efficacy were evaluated. All patients achieved ACR70 response in the core study and continued to achieve at least ACR50 response up to week 24 of this study. The safety population comprised 46.41 patient-years (PY). Rates per 100 PY of adverse (AEs) and serious events (SAEs) were 181.0 and 6.46, respectively. Pharyngitis and respiratory tract infections were the most common AEs. Except one AE (severe neutropenia), all others were classified as mild (24.4\%) or moderate (29.3\%). The incidence of SAEs was low (7.3\%). No new safety findings were observed. The safety profile of over 2.5-year treatment with TCZ is consistent with the pre-marketing CHERISH clinical trial. Presented data and continued efficacy response support the use of TCZ in pJIA. EUDRACT No: 2011-001607-12. https://clinicaltrials.gov/ct2/show/study/NCT01575769?term=ML27783. [\hyperlink{Talicia}{PMID: 29654485}, Violetta Opoka-Winiarska et al., 2018]

\section*{Aspirin and Dipyridamole}
\subsection*{Result}
\subsubsection*{Answer}

Ages 1–23 years (mechanical aortic valves): Yes  
Other pediatric populations/age ranges: Unknown

\subsubsection*{{Explanation}}
\hypertarget{Aspirin and Dipyridamole}
To determine if aspirin and dipyridamole is safe for use in children, I reviewed all available abstracts for targeted studies on the safety of this combination in pediatric populations.

Relevant Abstracts:

1. One study specifically evaluated children (ages 1 to 23 years, mean 12.9 years) with mechanical aortic valves treated with aspirin or aspirin with dipyridamole for anticoagulation. Of 51 patients (6 received aspirin with dipyridamole), followed for a mean of 36.5 months, there were no deaths related to thrombosis or embolus, and no postoperative thromboembolic events. Minor hemorrhagic complications (nosebleeds, one upper GI hemorrhage) were reported, but these were easily treated. The authors concluded that children with mechanical aortic valves in normal sinus rhythm can be safely treated with aspirin (or aspirin with dipyridamole) with little risk of thromboembolic events, valve thrombosis, or valve failure, and that hemorrhagic complications were minor and easily treated [\hyperlink{pmid_3784585}{PMID: 3784585}, Verrier et al., 1986].

2. Another study evaluated the use of dipyridamole and aspirin as an adjuvant therapy in 54 children with dengue hemorrhagic fever. The combination did not significantly change mortality but appeared to prevent progression to more severe disease. However, the abstract does not provide detailed safety data or adverse event rates [\hyperlink{pmid_515801}{PMID: 515801}, Kho et al., 1979].

3. Several large meta-analyses and clinical trials of aspirin and dipyridamole for secondary stroke prevention do not specify inclusion of children or provide pediatric subgroup data [\hyperlink{pmid_23871093}{PMID: 23871093}, Xia Li et al., 2013; \hyperlink{pmid_18535024}{PMID: 18535024}, Halkes et al., 2008; \hyperlink{pmid_20884870}{PMID: 20884870}, Dippel et al., 2010; \hyperlink{pmid_10493274}{PMID: 10493274}, Hervey et al., 1999]. These studies are therefore not directly relevant to pediatric safety.

Summary by Age Range:
- Ages 1–23 years: There is targeted evidence (albeit with a small sample size for the combination therapy group) that aspirin and dipyridamole can be used safely in children with mechanical aortic valves, with only minor, manageable hemorrhagic complications reported [\hyperlink{pmid_3784585}{PMID: 3784585}, Verrier et al., 1986].
- Other pediatric populations or age ranges: No targeted safety studies of aspirin and dipyridamole combination therapy were identified. The dengue hemorrhagic fever study included children but did not provide sufficient safety data to draw conclusions.

Conclusion:
- For children aged 1–23 years with mechanical aortic valves, there is evidence supporting the safety of aspirin and dipyridamole combination therapy.
- For other pediatric populations and age ranges, the safety of aspirin and dipyridamole is unknown due to lack of targeted studies.

\subsection*{Abstracts}
\hypertarget{pmid_6359862}{F}ever and pain are the most common issues in pediatric patient management. Acetaminophen, aspirin, and dipyrone are the most commonly used drugs and are equivalent in their efficacy. Dipyrone, used in many parts of the world, but not in the United States, is an effective agent; however, it has been implicated in producing agranulocytosis and anaphylactic shock. The salicylates have anti-inflammatory effects making them appropriate for the treatment of patients with juvenile rheumatoid arthritis, but they are gastric irritants, may impair clotting, and, because of saturable kinetics, may lead to accumulation and toxicity. Acetaminophen is an effective antipyretic and analgesic with few side effects that is toxic only in massive overdose. [\hyperlink{Aspirin and Dipyridamole}{PMID: 6359862}, E Gladtke et al., 1983]

\hypertarget{pmid_3784585}{T}he optimal method of anticoagulation in children with mechanical heart valves is controversial. Between 1975 and 1986, aspirin or aspirin with dipyridamole has been used for anticoagulation in children receiving a mechanical aortic valve at the University of California, San Francisco. Fifty-one patients (ages 1 to 23 years, mean 12.9 years) were treated with aspirin (n = 45) or aspirin with dipyridamole (n = 6) and observed a mean of 36.5 months (range 3 to 100 months). There were four late deaths: two from endocarditis and two from other medical problems, but none related to thrombosis or embolus. Follow-up was accomplished by direct contact with the patient, parent, or referring physician. Two patients (3.9\%) were lost to late follow-up. One minor neurologic event occurred perioperatively and resolved spontaneously. There were no postoperative thromboembolic events. Eleven asymptomatic children were recently studied by magnetic resonance imaging or computed axial tomography of the brain and had no evidence of prior silent cerebral thromboembolic defects. There were four patients (5.9\%) who had minor hemorrhagic complications: Three patients had nosebleeds and one patient had an upper gastrointestinal hemorrhage. Five patients were changed to warfarin anticoagulation: the patient with upper gastrointestinal hemorrhage and four older patients because of physician preference, all after uncomplicated aspirin therapy. There were no mechanical valve failures, although one patient required reoperation 9 months later for perivalvular leak. All children have remained in normal sinus or paced rhythm during follow-up. These results show that children with mechanical aortic valves in normal sinus rhythm can be safely treated with aspirin (or aspirin with dipyridamole) with little risk of thromboembolic events, valve thrombosis, or valve failure. Hemorrhagic complications resulting from aspirin are minor and easily treated. [\hyperlink{Aspirin and Dipyridamole}{PMID: 3784585}, E D Verrier et al., 1986]

\hypertarget{pmid_22364032}{A}cute respiratory infections are the second leading cause of morbidity in children under 18 years. Several drugs have been used with variable efficacy and safety, trying to reduce the associated symptoms and improve quality of life. To evaluate the efficacy and safety of buphenine, aminophenazone and diphenylpyraline hydrochloride when compared with placebo for the control of symptoms associated with common cold in children 6-24 months of age. Randomized clinical trial, double blind, placebo controlled, in 100 children < 24 months of any gender, with symptoms associated to common cold. They received the drug under study vs. placebo for seven days. Both groups received acetaminophen. The change on common cold related symptoms were evaluated. Statistic analysis was made with STATA 11.0 for Mac. Fifty-three children were randomized to study drug and forty-seven to placebo. Age of children in each group was similar (12.2 +/- 5.8 months vs. 12.7 +/- 5.8 months, p NS). There were significant differences between groups in relation to rhinorrea and sneezing resolution, with better results in Flumil group and no adverse events observed. The results in this study indicates that Flumil is a safe and effective drug for control of symptoms present in the common cold in children aged 6-24 months. [\hyperlink{Aspirin and Dipyridamole}{PMID: 22364032}, Ericka Montijo-Barrios et al., ]

\hypertarget{pmid_23871093}{S}troke is becoming a common disease worldwide, and has an increased rate of recurrence yearly after a transient ischemic attack (TIA) or stroke. Aspirin, dipyridamole, clopidogrel and aspirin plus dipyridamole combination therapy have been recommended for the secondary prevention of stroke in Americans. We performed meta-analyses to assess the effectiveness and safety of combination therapy with aspirin and dipyridamole (A+D) versus aspirin (A) alone in secondary prevention after transient ischemic attack (TIA) or stroke of presumed arterial origin within one week and six months. Medline, Embase, and the Cochrane Library. Eligible studies were completed randomized controlled trials investigating the effect of aspirin plus dipyridamole versus aspirin in patients with previous TIA or stroke. Five trials involving the use of aspirin and dipyridamole were included, 4318 allocated to A+D and 4304 to A alone. Meta-analysis of trials showed a significant protective effect of reducing or preventing recurrence of stroke (P=0.01), and ischemic event (P=0.003). The statistics showed no significant difference in vascular event, death from all cause and myocardial infarction (P>0.05). There were similarities with all bleeding events, major bleeding and intracranial hemorrhage was significant (P>0.05) between two groups. Aspirin plus dipyridamole combination therapy was beneficial in reducing the recurrence of stroke, and did not increase the bleeding event. Hence, aspirin plus dipyridamole combination therapy is effective and safe for the secondary prevention of stroke. [\hyperlink{Aspirin and Dipyridamole}{PMID: 23871093}, Xia Li et al., 2013]

\hypertarget{pmid_23078168}{P}aracetamol (acetaminophen) and ibuprofen are the most frequently purchased over-the-counter (OTC) medicines for children. Parents purchase these medicines for the treatment of fever and pain. In some countries other NSAIDs such as aspirin (acetylsalicylic acid) and dipyrone are available. We aimed to perform a narrative review of the efficacy and toxicity of OTC analgesic medicines for children in order to give guidance to health professionals and parents regarding the treatment of pain in a child. Neither aspirin nor dipyrone are recommended for OTC use because of the association with Reye's syndrome for the former and the risk of agranulocytosis for the latter. Both paracetamol and ibuprofen are effective for the treatment of mild pain in children. Adverse effects with both medicines are infrequent. Ibuprofen is an NSAID and therefore there is a greater risk of gastrointestinal adverse effects and hypersensitivity. Aspirin and dipyrone should be avoided. Paracetamol is the drug of first choice for mild pain in children because of its favourable safety profile. For the treatment of significant musculoskeletal pain, ibuprofen is the drug of first choice. [\hyperlink{Aspirin and Dipyridamole}{PMID: 23078168}, Zeina Bárzaga Arencibia et al., 2012]

\hypertarget{pmid_29024184}{D}ipyrone has analgesic, spasmolytic, and antipyretic effects and is used to treat pain. Due to a possible risk of agranulocytosis with the use of dipyrone, it has been banned in a number of countries. The most commonly used data for the use of dipyrone are related to adults. Information relating to the use of dipyrone in children is scarce. Given the potential added value of dipyrone in the treatment of pain, a review of the literature was conducted to obtain more insight into the analgesic efficacy of dipyrone in children as well as the safety of dipyrone in terms of adverse events. A literature search was done for original articles (in English, German, or Spanish language) which met the following criteria: the use of dipyrone for pain and children up to the age of 17 years old. All titles and abstracts retrieved were reviewed, independently, by two of the authors, for their suitability for inclusion. The references of the selected articles were also checked for additional relevant papers. The publications were categorized into case reports, observational studies, or randomized controlled trials. To assess the methodological quality of the studies, the Jadad score was used. In the limited available data, the analgesic efficacy of intravenous dipyrone appears similar to that of intravenous paracetamol. Evidence is lacking to support the claim that dipyrone is equivalent or even superior to Non-Steroid-Anti-Inflammatory-Drugs in pediatric pain. While the absolute risk of agranulocytosis with dipyrone in children, based on available literature, cannot be determined, case reports suggest that this risk is not negligible. [\hyperlink{Aspirin and Dipyridamole}{PMID: 29024184}, Thomas G de Leeuw et al., 2017]

\hypertarget{pmid_20884870}{T}he combination of low-dose aspirin and dipyridamole is more effective than aspirin alone in reducing the risk of recurrent stroke and other major cardiovascular events in patients with a recent transient ischemic attack or minor stroke. It is unknown whether this also applies to patients with a disabling stroke. We reanalyzed the data of 5700 patients from ESPRIT and ESPS-2 to study the effect of aspirin and dipyridamole according to modified Rankin scale (mRS) score at baseline. Primary outcome was vascular events (stroke, myocardial infarction, or vascular death). We used proportional hazards regression to estimate the treatment effect across mRS strata at baseline, and we tested for interactions with treatment. In total, 426 patients (7.5\%) had mRS score of 4 or 5 at baseline. The risk of an outcome event increased with mRS score. The relative risk associated with the combination of aspirin and dipyridamole compared to aspirin alone in patients with mRS score 0 to 5 was 0.79 (95\% confidence interval, 0.69-0.91). The relative risk according to mRS subcategory score 0 to 4 at baseline varied between 0.73 and 0.96 for vascular events and between 0.62 and 0.96 for stroke. The number of patients with mRS score 5 was too small for reliable estimates, but the data suggest a beneficial effect. There was no evidence of interaction between treatment effect and mRS score at baseline. The beneficial effect of the combination of low-dose aspirin and dipyridamole was present in all subcategories of the mRS score. [\hyperlink{Aspirin and Dipyridamole}{PMID: 20884870}, Diederik W J Dippel et al., 2010]

\hypertarget{pmid_7008732}{A}ntipyretics should be employed in the pediatric population whenever it is the clinical judgment of the attending physician that fever should be lowered. Aspirin and acetaminophen are equally effective as antipyretics. The efficacy and safety of these two most common antipyretic agents are examined, and various studies with these agents are critically reviewed. Since acetaminophen has a greater margin of safety at antipyretic dosages, it is preferred to aspirin when an anti-inflammatory effect is not required. The efficacy and safety of combination therapy with acetaminophen and aspirin in pediatric patients--or an alternative treatment regimen with both these drugs--warrant investigation. [\hyperlink{Aspirin and Dipyridamole}{PMID: 7008732}, S J Yaffe et al., 1981]

\hypertarget{pmid_1780077}{I}n order to assess the usefulness of a combination of low-dose aspirin (25 mg b.i.d.) with dipyridamole (200 mg b.i.d.) in the prevention of major coronary events in patients with acute unstable angina, we performed a prospective, double-blind, placebo-controlled study involving 88 consecutive patients admitted to three Hospital Departments of Cardiology. The patients entered the study as soon as possible after hospital admission, and were treated and followed up to one year. There was no appreciable difference in side effects and adverse reactions between the treatment and control group. The incidence of cardiac death and/or nonfatal myocardial infarction during the whole period of observation was 14\% (6/44) in the treatment group and 25\% (11/44) in the placebo group by "intention-to-treat" analysis; 16\% (4/25) and 32\% (10/31), respectively, by "drug-efficacy" analysis (p = 0.21 by Fisher's exact test, non significant difference). However, when considering the only events occurred in the first month (2/44 in the treatment group and 9/44 in the placebo group, amounting to 4.5 and 20 percent, respectively), the combination of dipyridamole with low-dose aspirin reached a statistically significant protective effect (p = 0.04). The results of this pilot study provide strong evidence for a beneficial effect of the regimen tested in patients with acute unstable angina, at least in the first weeks of treatment, while at the same time suggesting a safe alternative for patients with contraindications to higher doses of aspirin. [\hyperlink{Aspirin and Dipyridamole}{PMID: 1780077}, P Prandoni et al., ]

\hypertarget{pmid_21464191}{A}s many as 1 in every 110 children in the United States has an autism spectrum disorder (ASD). Many medical treatments for ASDs have been proposed and studied, but there is currently no consensus regarding which interventions are most effective. To systematically review evidence regarding medical treatments for children aged 12 years and younger with ASDs. We searched the Medline, PsycInfo, and ERIC (Education Resources Information Center) databases from 2000 to May 2010, regulatory data for approved medications, and reference lists of included articles. Two reviewers independently assessed each study against predetermined inclusion/exclusion criteria. Studies of secretin were not included in this review. Two reviewers independently extracted data regarding participant and intervention characteristics, assessment techniques, and outcomes and assigned overall quality and strength-of-evidence ratings on the basis of predetermined criteria. Evidence supports the benefit of risperidone and aripiprazole for challenging and repetitive behaviors in children with ASDs. Evidence also supports significant adverse effects of these medications. Insufficient strength of evidence is present to evaluate the benefits or adverse effects for any other medical treatments for ASDs, including serotonin-reuptake inhibitors and stimulant medications. Although many children with ASDs are currently treated with medical interventions, strikingly little evidence exists to support benefit for most treatments. Risperidone and aripiprazole have shown benefit for challenging and repetitive behaviors, but associated adverse effects limit their use to patients with severe impairment or risk of injury. [\hyperlink{Aspirin and Dipyridamole}{PMID: 21464191}, Melissa L McPheeters et al., 2011]

\hypertarget{pmid_515801}{C}linical studies in the treatment of 54 children suffering from DHF with a combination of dipyridamole and ASA as an adjuvant of our standard therapy consisted of fluid, electrolytes, blood, plasma and plasma expanders were evaluated. Heparin was administered in cases of DIC. It appeared that dipyridamole and ASA did not change the mortality significantly, but it prevented the progress of the severity of the disease from grade I and II to grade III and IV. [\hyperlink{Aspirin and Dipyridamole}{PMID: 515801}, L K Kho et al., 1979]

\hypertarget{pmid_23801256}{A}ripiprazole and risperidone are the only FDA approved medications for treating irritability in autistic disorder, however there are no head-to-head data comparing these agents. This is the first prospective randomized clinical trial comparing the safety and efficacy of these two medications in patients with autism spectrum disorders. Fifty nine children and adolescents with autism spectrum disorders were randomized to receive either aripiprazole or risperidone for 2 months. The primary outcome measure was change in Aberrant Behavior Checklist (ABC) scores. Adverse events were assessed. Aripiprazole as well as risperidone lowered ABC scores during 2 months. The rates of adverse effects were not significantly different between the two groups. The safety and efficacy of aripiprazole (mean dose 5.5 mg/day) and risperidone (mean dose 1.12 mg/day) were comparable. The choice between these two medications should be on the basis of clinical equipoise considering the patient's preference and clinical profile.  [\hyperlink{Aspirin and Dipyridamole}{PMID: 23801256}, Ahmad Ghanizadeh et al., 2014] Tonsillectomy is the 2nd most common outpatient surgery performed on children in the United States of America. Its main complication is pain, which varies in intensity from moderate to severe. Dipyrone is one of the most widely used painkillers in the postoperative period in children. Its use, however, is controversial in the literature, to the point that it is banned in many countries due to its potential severe adverse effects. Because of this controversy, reviewing the analgesic use of dipyrone in the postoperative period of tonsillectomy in children is essential. The aim of this study was to review the analgesic use of dipyrone in the postoperative period of tonsillectomy in children. Systematic review of the literature, involving an evaluation of the quality of articles in the databases MEDLINE/Pubmed, EMBASE and Virtual Health Library, selected with a preestablished search strategy. Only studies with a randomised clinical trial design evaluating the use of dipyrone in the postoperative period of tonsillectomy in children were included. Only 2 randomised clinical trials were found. Both compared dipyrone, paracetamol, and placebo. We were unable to carry out a metanalysis because the studies were too heterogenous (dipyrone was used as pre-emptive analgesic in one and only postoperatively in another). The analgesic effect of dipyrone, measured by validated pain scales in childhood, was shown to be superior to placebo and similar to paracetamol. It appears that dipyrone exhibits a profile suitable for use in children. However, the scarcity of randomised clinical trials evaluating its analgesic effect in this age group leads to the conclusion that more well-designed studies are still needed to establish the role of dipyrone in the postoperative period of tonsillectomy in children. [\hyperlink{Aspirin and Dipyridamole}{PMID: 23801256}, Maira Isis S Stangler et al., ]

\hypertarget{pmid_1429411}{T}he pharmacological management of anxiety in children primarily has used antidepressants, such as imipramine. Buspirone, an atypical anxiolytic, has been shown to be of benefit in both adults and children. It has relatively few side effects and is generally well tolerated. Two cases are reported here involving children treated for anxiety with buspirone who subsequently suffered a possible psychotic deterioration. [\hyperlink{Aspirin and Dipyridamole}{PMID: 1429411}, P Soni et al., 1992]

\hypertarget{pmid_33235453}{A}utism spectrum disorder (ASD) is a debilitating neurodevelopmental disorder with high heterogeneity and no clear common cause. Several drugs, in particular risperidone and aripiprazole, are used to treat comorbid challenging behaviors in children with ASD. Treatment with risperidone and aripiprazole is currently recommended by the Food and Drug Administration (FDA) in the USA for children aged 5 and 6 years and older, respectively. Here, we investigated the use of these medications in younger patients aged 4 years and older. This retrospective case series included 18 children (mean age, 5.7 years) with ASD treated at the Kids Neuro Clinic and Rehab Center in Dubai. These patients began treatment with risperidone or aripiprazole at the age of 4 years and older, and all patients presented with comorbid challenging behaviors that warranted pharmacological intervention with either risperidone or aripiprazole. All 18 children showed objective improvement in their ASD core signs and symptoms. Significant improvement was observed in 44\% of the cases, and complete resolution (minimal-to-no-symptoms) was observed in 56\% of the cases as per the Childhood Autism Rating Scale 2-Standard Test (CARS2-ST) and the Clinical Global Impression (CGI) scales. Our findings indicate that the chronic administration of antipsychotic medications with or without ADHD medications is well tolerated and efficacious in the treatment of ASD core and comorbid symptoms in younger children when combined with standard supportive therapies. This is the first report to suggest a treatment approach that may completely resolve the core signs and symptoms of ASD. While the reported outcomes indicate significant improvement to complete resolution of ASD, pharmacological intervention should continue to be considered as part of a multi-component intervention in combination with standard supportive therapies. Furthermore, the findings support the critical need for double-blind, placebo-controlled studies to validate the outcomes. [\hyperlink{Aspirin and Dipyridamole}{PMID: 33235453}, Hamza A Alsayouf et al., 2020]

\hypertarget{pmid_24144215}{A} multimodal and preventative approach to providing postoperative analgesia is becoming increasingly popular for children and adults, with the aim of reducing reliance on opioids. We conducted a prospective, randomized double-blind study to compare the analgesic efficacy of intravenous paracetamol and dipyrone in the early postoperative period in school-age children undergoing lower abdominal surgery with spinal anesthesia. Sixty children scheduled for elective lower abdominal surgery under spinal anesthesia were randomized to receive either intravenous paracetamol 15 mg/kg, dipyrone 15 mg/kg or isotonic saline. The primary outcome measure was pain at rest, assessed by means of a visual analog scale 15 min, 30 min, 1 h, 2 h, 4 h and 6 h after surgery. If needed, pethidine 0.25 mg/kg was used as the rescue analgesic. Time to first administration of rescue analgesic, cumulative pethidine requirements, adverse effects and complications were also recorded. There were no significant differences in age, sex, weight, height or duration of surgery between the groups. Pain scores were significantly lower in the paracetamol group at 1 h (P = 0.030) and dipyrone group at 2 h (P = 0.010) when compared with placebo. The proportion of patients requiring rescue analgesia was significantly lower in the paracetamol and dipyrone groups than the placebo group (vs. paracetamol P = 0.037; vs. dipyrone P = 0.020). Time to first analgesic requirement appeared shorter in the placebo group but this difference was not statistically significant, nor were there significant differences in pethidine requirements, adverse effects or complications. After lower abdominal surgery conducted under spinal anesthesia in children, intravenous paracetamol appears to have similar analgesic properties to intravenous dipyrone, suggesting that it can be used as an alternative in the early postoperative period. Clinical Trials.gov. Identifier: NCT01858402. [\hyperlink{Aspirin and Dipyridamole}{PMID: 24144215}, Esra Caliskan et al., 2013]

\hypertarget{pmid_33921933}{R}isperidone and aripiprazole are approved by the USA Food and Drug Administration for the treatment of irritability and aggression in children from the ages of 5 and 6 years, respectively. However, there are no approved medications for the treatment of autism spectrum disorder (ASD) core signs and symptoms. Nevertheless, early intervention is recognized as key to improving long-term outcomes. This retrospective case study included 10 children (mean age, 2 years 10 months) with ASD who presented with persistent irritability and aggression before 4 years of age that was unresponsive to behavioral interventions and sufficiently severe to consider pharmacological intervention with risperidone or aripiprazole combined with standard supportive therapies. Besides ameliorating comorbid behaviors, improvement was observed in ASD core signs and symptoms for all patients, with minimal-to-no symptoms observed in 60\% of patients according to the Childhood Autism Rating Scale 2-Standard Test and Clinical Global Impression scales. Excessive weight gain in two patients was the only adverse effect observed that required intervention. This is the first study to suggest that ASD can potentially be treated in very young children (<4 years). Clinical trials are urgently required to validate these findings among this pediatric population. [\hyperlink{Aspirin and Dipyridamole}{PMID: 33921933}, Hamza A Alsayouf et al., 2021]

\hypertarget{pmid_34430426}{M}igraine is the most common primary headache among children and adolescents. The aim of this meta-analysis was to evaluate the efficacy and safety of antiepileptic drugs in the prevention of pediatric migraine. PubMed, Cochrane Library, EMBASE and Chinese National Knowledge Infrastructure (CNKI) databases were searched for eligible published RCTs from January 1970 to June 2020. Migraine frequency and ≥50\% response rate were measured as the efficacy outcomes. We used "Risk of Bias" tool for quality assessment and RevMan5.3 software for statistical analysis. Four articles containing 7 RCTs with 794 participants compared the efficacy of AEDs with placebo. Four RCTs assessed topiramate  Topiramate can reduce monthly headache days in children and adolescents with migraine. However, it failed to improve the ≥50\% response rate. DVPX ER showed no difference from placebo in the prophylactic treatment pediatric migraine. Side effects seemed to be more frequent in topiramate and DVPX ER treated children but generally well-tolerated. [\hyperlink{Aspirin and Dipyridamole}{PMID: 34430426}, Guoyong Jia et al., 2021]

\hypertarget{pmid_23503913}{T}onsillectomy is associated with severe postoperative pain for which, several drugs are employed for management. In this double-blind, placebo-controlled study we aimed to evaluate the efficacy of intravenous paracetamol and dipyrone when used for post-tonsillectomy analgesia in children. 120 children aged 3-6 yr, undergoing tonsillectomy with or without adenoidectomy and/or ventilation tube insertion were randomized to receive intraoperative infusions of paracetamol (15 mg/kg), dipyrone (15 mg/kg) or placebo (0.9\% NaCl). Evaluation was carried out at 0.25, 0.50, 1, 2, 4, 6h postoperatively. Pethidine 0.25 mg/kg was utilized as rescue analgesic. Cumulative pethidine requirement was the primary outcome. Pain intensity measurement, pain relief, sedation level, nausea and vomiting, postoperative bleeding and any other adverse effects were noted. No significant difference was found in pethidine requirement between paracetamol and dipyrone groups. Cumulative pethidine requirement was significantly less in paracetamol and dipyrone groups vs. placebo. No significant difference was observed between groups in postoperative pain intensity scores throughout the study. Intravenous paracetamol is found to have a similar analgesic efficacy as intravenous dipyrone and they both help to reduce the opioid requirement for postoperative analgesia in pediatric day-case tonsillectomy. [\hyperlink{Aspirin and Dipyridamole}{PMID: 23503913}, Aysu Inan Kocum et al., ]

\hypertarget{pmid_367358}{A} review is given on the clinical studies performed with aspirin in patients with chronic vascular occlusions of the limbs and on studies in cerebral ischemia using aspirin and sulfinpyrazone. Aspirin reduces the risk of reocclusions in patients after vascular surgery and also reduces the risk of peripheral vascular occlusions in diabetic patients. In doses of 1.2-1.5 g/day it also reduces the frequency of transient ischemic attacks. Conclusive results of similar studies with sulfinpyrazone and dipyridamole can be expected of the ongoing studies. Aspirin has no effect on the course of glomerulonephritis in children. Warfarin plus dipyridamole seem to have some effect in patients renal allografts. Sulfinpyrazone and ASA reduced the incidence of shunt thromboses in hemodialyzed patients. Several case reports in patients with thrombocytemia or Raynaud's syndrome made it likely that treatment with antiplatelet drug reduces the incidence of vascular occlusions. [\hyperlink{Aspirin and Dipyridamole}{PMID: 367358}, K Breddin et al., 1977]

\hypertarget{pmid_28063133}{T}he antipyretic analgesics, paracetamol, and non-steroidal anti-inflammatory agents NSAIDs are one of the most widely used classes of medications in children. The aim of this review is to determine if there are any clinically relevant differences in safety between ibuprofen and paracetamol that may recommend one agent over the other in the management of fever and discomfort in children older than 3 months of age. [\hyperlink{Aspirin and Dipyridamole}{PMID: 28063133}, Dipak J Kanabar et al., 2017]

\hypertarget{pmid_3898670}{T}he epidemiological effectiveness of dipyridamol, an interferon-inducing agent used for the prevention of influenza and viral acute respiratory diseases, was tested in 4 epidemiological trials, 3 of them carried out as double blind trials. Observations were made in groups of adults (a research institute, a factory) and children (a kindergarten, a school), comprising 1040 subjects in the test groups and 771 subjects in the control groups. The drug was used during the whole epidemic period (January--March 1983) according to the following schedule: 1 oral administration in 8 days, in doses of 8 mg for adults, 50 mg for schoolchildren and 24 mg for children in the kindergarten. The epidemiological effectiveness of the drug was evaluated by comparing the total morbidity rates in influenza and acute respiratory diseases in the test and control groups. The results of 4 trials showed a pronounced epidemiological effectiveness of dipyridamol. The values of the epidemiological effectiveness index of the drug were 2.38 in the kindergarten, 1.55 at the school, 7.42 at the factory and 2.16 at the research institute. The results of the study of dipyridamol suggest that further investigations should be made with a view to use it for the mass prevention of influenza and acute respiratory diseases. [\hyperlink{Aspirin and Dipyridamole}{PMID: 3898670}, K Kuzmov et al., 1985]

\hypertarget{pmid_18535024}{T}o study the effect of combination therapy with aspirin and dipyridamole (A+D) over aspirin alone (ASA) in secondary prevention after transient ischaemic attack (TIA) or minor stroke of presumed arterial origin and to perform subgroup analyses to identify patients that might benefit most from secondary prevention with A+D. The previously published meta-analysis of individual patient data was updated with data from ESPRIT (n = 2,739); trials without data on the comparison of A+D versus ASA were excluded. A meta-analysis was performed using Cox regression, including several subgroup analyses and following baseline risk stratification. A total of 7612 patients (five trials) were included in the analyses, 3800 allocated to A+D and 3812 to ASA alone. The trial-adjusted hazard ratio (HR) for the composite event of vascular death, non-fatal myocardial infarction and non-fatal stroke was 0.82 (95\% confidence interval (CI) 0.72 to 0.92). HRs did not differ in subgroup analyses based on age, sex, qualifying event, hypertension, diabetes, previous stroke, ischaemic heart disease, aspirin dose, type of vessel disease and dipyridamole formulation, nor across baseline risk strata as assessed with two different risk scores. A+D were also more effective than ASA alone in preventing recurrent stroke; HR 0.78 (95\% CI 0.68 to 0.90). The combination of aspirin and dipyridamole is more effective than aspirin alone in patients with TIA or ischaemic stroke of presumed arterial origin in the secondary prevention of stroke and other vascular events. This superiority was found in all subgroups and was independent of baseline risk. [\hyperlink{Aspirin and Dipyridamole}{PMID: 18535024}, P H A Halkes et al., 2008]

\hypertarget{pmid_27144151}{A}lthough pharmacotherapy with atypical antipsychotics is common in child psychiatry, there has been little research on this issue. To compare the efficacy and safety of risperidone and aripiprazole in the treatment of preschool children with disruptive behavior disorders comorbid with attention deficit-hyperactivity disorder (ADHD). Randomized clinical trial conducted in a university-affiliated child psychiatry clinic in southwest Iran. Forty 3-6-year-old children, diagnosed with oppositional defiant disorder comorbid with ADHD, were randomized to an 8-week trial of treatment with risperidone or aripiprazole (20 patients in each group). Assessment was performed by Conners' rating scale-revised and clinical global impressions scale, before treatment, and at weeks 2, 4, and 8 of treatment. The data were analyzed by SPSS version 16. Mean scores between the two groups were compared by analysis of variance and independent and paired t-test. Mean scores of Conners rating scales were not different between two groups in any steps of evaluation. Both groups had significantly reduced scores in week 2 of treatment (P = 0.00), with no significant change in subsequent measurements. Rates of improvement, mean increase in weight (P = 0.894), and mean change in fasting blood sugar (P = 0.671) were not significantly different between two groups. Mean serum prolactin showed a significant increase in risperidone group (P = 0.00). Both risperidone and aripiprazole were equally effective in reducing symptoms of ADHD and oppositional defiant disorder, and relatively safe, but high rates of side effects suggest the cautious use of these drugs in children.  [\hyperlink{Aspirin and Dipyridamole}{PMID: 27144151}, Parvin Safavi et al., ] The fixed-dose combination of extended-release dipyridamole/aspirin (Aggrenox/Asasantin Retard) combines 2 antiplatelet agents with different mechanisms of action. The combination reduced thrombus formation in human and animal models. Coadministration of extended-release dipyridamole and aspirin in healthy volunteers had no significant effects on the plasma concentrations of either agent. Twice-daily oral extended-release dipyridamole/aspirin (400/50 mg/day) was twice as effective as either agent alone in the secondary prevention of stroke in a large clinical trial involving patients with prior stroke or transient ischaemic attack. The rate of the combined end-point of stroke and death tended to be lower with the combination than with other treatments. The incidence of death was not significantly reduced by any treatment. Most adverse events with extended-release dipyridamole/aspirin were mild and similar to those with either agent alone. Bleeding was more common with the combination than with extended-release dipyridamole alone, as was headache when compared with aspirin alone. Limited pharmacoeconomic analyses suggest that treatment with extended-release dipyridamole/aspirin was cost saving and was cost effective compared with aspirin monotherapy for the secondary prevention of stroke. [\hyperlink{Aspirin and Dipyridamole}{PMID: 27144151}, P S Hervey et al., 1999]

\section*{Effexor XR}
\subsection*{Result}
\subsubsection*{Answer}

Unknown

\subsubsection*{{Explanation}}
\hypertarget{Effexor XR}
A review of the provided abstracts reveals that none of them are targeted studies evaluating the safety of Effexor XR (venlafaxine extended-release) specifically in children. The only mention of venlafaxine in children is in an abstract describing two case reports of neonatal seizures attributed to maternal use of venlafaxine during pregnancy, with both infants experiencing adverse neurological events shortly after birth [\hyperlink{pmid_17044905}{PMID: 17044905}, Ravi K Pakalapati et al., 2006]. This report does not constitute a targeted safety study of Effexor XR in children for any indication, nor does it provide evidence of safety or lack thereof for pediatric use.

No abstracts present data from controlled trials or systematic studies assessing the safety of Effexor XR in children of any age range (e.g., neonates, infants, children, or adolescents). Therefore, based on the available abstracts, the safety of Effexor XR in children is unknown.

\subsection*{Abstracts}
\hypertarget{pmid_17044905}{V}enlafaxine (Efexor), a selective noradrenergic reuptake inhibitor, is an important therapeutic option in the treatment of perinatal depression, but its effects on the newborn are uncertain. We present a report of two infants with neonatal seizures attributed to maternal use of venlafaxine. The first infant was hypotonic and required resuscitation at birth. The second was born in a good condition but developed clinically apparent seizures after the second day of life. Both infants responded rapidly to treatment with phenobarbitone that was weaned uneventfully by the first and second week of life. Both remain well at 1 year of age. Other causes of neonatal seizures were excluded and neurological investigations on these two infants were unremarkable. We suggest that all infants exposed to maternal venlafaxine, no matter their condition at birth, be monitored in hospital for at least 3 to 4 days in order to preempt and treat adverse neurological events. [\hyperlink{Effexor XR}{PMID: 17044905}, Ravi K Pakalapati et al., 2006]

\hypertarget{pmid_16554175}{T}o evaluate the long-term efficacy, tolerability, and safety of oxcarbazepine (OXC) in children with epilepsy. We enrolled 36 patients (median age 7.75) with new diagnosis of partial epilepsy in an open prospective study. All type of epilepsy were included: 25 patients were affected by idiopathic epilepsy, eight by symptomatic epilepsy and three by cryptogenic epilepsy. Patients were then scheduled to come back for controls at 3 months (T1), 12 months (T2) and 24 months (T3) after the beginning of OXC-monotherapy (T0). At each control we evaluated patients through their seizure diary, a questionnaire on side effects, their level of 10-monohydroxy (MHD) metabolite and laboratory analysis. At T1, 21/36 patients (58.3\%) were seizure-free, 3/36 patients (8.3\%) showed an improvement higher than 50\%, 3/36 (8.3\%) lower than 50\%, while 2/36 worsened (5.6\%). In 7/36 (19.5\%) patients, no improvement was reported. At T2 13/18 patients (72.2\%) were seizure-free, 1/18 showed a response to therapy higher than 50\% while 2/18 worsened (11\%). In two patients no improvement was reported. A correspondence between MHD plasmatic levels and clinical response (r=0.49; p<0.05) was only registered at T1. An EEG normalization was observed in 25\% of cases. Side effects were reported in 25\% of cases, but symptoms progressively disappeared at follow-up. We can therefore conclude that OXC can be considered, for its efficacy and safety, as a first line drug in children with epilepsy. [\hyperlink{Effexor XR}{PMID: 16554175}, E Franzoni et al., 2006]

\hypertarget{pmid_36036696}{T}he chest x-ray (CXR) was the gold standard in the diagnosis of pneumonia in children. However, CXR has limitations and cannot discriminate in etiology. Current guidelines recommend against routine use of CXR in children with uncomplicated lower respiratory tract infections (LRTI). We used routine care data from a multicentre RCT including 597 children with LRTI symptoms, to evaluate the influence of CXR on antibiotic prescription in the emergency department (ED). CXR remains frequently performed in non-complex children suspected of LRTI in the ED (18\%). Children who underwent CXR were more likely to receive antibiotics, even when adjusted for symptoms, hospital and CXR results. Our study highlights the inferior role of CXR in treatment decisions for children with LRTI as CXR, regardless of its results, is independently associated with more antibiotic prescriptions. [\hyperlink{Effexor XR}{PMID: 36036696}, Daniella P Garcia Pérez et al., 2022]

\hypertarget{pmid_19740527}{E}noxaparin, a low molecular weight heparin (LMWH), is frequently used for the prevention and treatment of thromboembolic complications in infants and children (Sutor et al., 2004 [1]). Injection pain and the fear and anxiety associated with needle phobia in the pediatric population are well documented. Best practice pediatric pain management standards of care recommend mitigating the child's pain experience whenever possible. The use of topical anesthetics such as liposomal-lidocaine 4\% results in a rapid onset of anesthesia, minimal blanching, without vasoconstriction (Koh et al., 2004 [2]) or risk of methemoglobinemia. Topical lidocaine has been used to reduce the injection pain of enoxaparin, but there is no data available examining whether it will interfere with the absorption of LMWH. To determine if the topical lidocaine, Maxilene, interferes with enoxaparin absorption as measured by peak anti-Xa levels. Infants and children clinically prescribed enoxaparin were eligible for this study. Children in group 1 were pre-treated with Maxilene prior to enoxaparin injection on day 1 with no Maxilene pre-treatment on day 2. For group 2, the order was reversed. Peak anti-Xa levels were measured following each enoxaparin dose and were compared between the groups. 26 children of ages 14d-16 y (median 6.7 months) were enrolled. Anti-Xa levels following topical lidocaine administration were 0.070 U/mL (95\%CI 0.025; 0.114) lower than without prior topical lidocaine administration. Anti-Xa levels on the second day were on average 0.013 U/mL (95\%CI -0.066; 0.040) higher compared to day one regardless of the order of topical lidocaine administration. There were no reported incidences of local reactions such as redness, hives or blanching. Topical lidocaine (Maxilene) administration before enoxaparin injection results in a small, clinically non-significant, reduction in anti-Xa levels. [\hyperlink{Effexor XR}{PMID: 19740527}, S M Duncan et al., 2010]

\hypertarget{pmid_2391756}{W}e administered norfloxacin (NFLX) to 16 children aged 3 to 14 year-old at the dose of 5.2 to 17.2 mg/kg/day. We evaluated the efficacy and safety of NFLX in 6 children with respiratory tract infections, 8 urinary tract infections, and 2 gastrointestinal tract infections. Efficacy rate of NFLX was 93.8\% and eradicated rate was 92.9\%. Any adverse effects were not observed. These results suggested that NFLX could be used safely to the children. [\hyperlink{Effexor XR}{PMID: 2391756}, T Ihara et al., 1990]

\hypertarget{pmid_32145737}{I}ntranasal dexmedetomidine (DEX), as a novel sedation method, has been used in many clinical examinations of infants and children. However, the safety and efficacy of this method for electroencephalography (EEG) in children is limited. In this study, we performed a large-scale clinical case analysis of patients who received this sedation method. The purpose of this study was to evaluate the safety and efficacy of intranasal DEX for sedation in children during EEG. This was a retrospective study. The inclusion criteria were children who underwent EEG from October 2016 to October 2018 at the Children's Hospital affiliated with Chongqing Medical University. All the children received 2.5 μg·kg A total of 3475 cases were collected and analysed in this study. The success rate of the initial dose was 87.0\% (3024/3475 cases), and the success rate of intranasal sedation rescue was 60.8\% (274/451 cases). The median sedation onset time was 19 mins (IQR: 17-22 min), and the sedation recovery time was 41 mins (IQR: 36-47 min). The total incidence of adverse events was 0.95\% (33/3475 cases), and no serious adverse events occurred. Intranasal DEX (2.5 μg·kg [\hyperlink{Effexor XR}{PMID: 32145737}, Hang Chen et al., 2020] Several studies have reported the use of dexmedetomidine (DEX) plus opioids for flexible bronchoscopy in both adults and children. To determine whether DEX plus sufentanil (SF) is safe for children, 142 children undergoing flexible bronchoscopy were assigned to one of three groups, each of which received the same SF loading dose and similar DEX and SF maintenance doses, but different loading doses of DEX: DS1 (DEX 0.5 μg·kg-1), DS2 (DEX 1.0 μg·kg-1), and DS3 (DEX 1.5 μg·kg-1). The Ramsay sedation scale was maintained at 3 in all groups. Results showed that anesthesia onset time was shorter, and the perioperative hemodynamic profile was more stable, in the DS3 group. The number of intraoperative movements was also lowest in the DS3 group. The time to first dose of rescue midazolam and lidocaine was significantly longer, but the total corresponding accumulated doses were lower in the DS3 group. Although the time to recovery prior to discharge from the post anesthesia care unit was longer, the overall incidence of tachycardia was lower in the DS3 group, and it received the highest bronchoscopist satisfaction score among the three groups. We therefore conclude that high-dose DEX plus SF can be safely and efficaciously used in children undergoing flexible bronchoscopy. [\hyperlink{Effexor XR}{PMID: 32145737}, Xiujing Dang et al., 2017]

\hypertarget{pmid_30040807}{T}he paper summarizes the results of the studies on the efficacy and safety of a new form of controlled release levetiracetam XR (Lev XR) compared to standard tablet immediate release form in treatment of resistant partial seizures. The authors present the data on the bioequivalence and absorption of LevXR related to the meal and therapeutic doses in the range between 1000 to 3000 mg/day. It has been concluded that LevXR has high efficacy and safety due to its stable plasma concentration during the day. The results meet FDA bioequivalence criteria and, in authors opinion, can be recommended as drug of choice in additional treatment of partial seizures in patients above 12 years of age. [\hyperlink{Effexor XR}{PMID: 30040807}, N A Ermolenko et al., ]

\hypertarget{pmid_17941284}{T}he safety of fexofenadine has been examined extensively in adults and school-age children. However, the safety of fexofenadine in children younger than 6 years has not been reported to date. To compare the safety and tolerability of twice-daily fexofenadine hydrochloride, 30 mg, and placebo in preschool children aged 2 to 5 years with allergic rhinitis. This was a multicenter, double-blind, randomized, placebo-controlled, parallel-group study, conducted between February 29, 2000, and June 14, 2001. Participants were randomized to either fexofenadine hydrochloride, 30 mg, or placebo twice daily for a 2-week period. To facilitate dosing, capsule content was mixed with applesauce (approximately 10 mL). Safety assessments depended on date of entry into the study because of an amendment to the protocol. Before the amendment, assessments included physical examination, vital signs reporting (oral temperature, heart rate, and respiratory rate), and adverse event (AE) reporting. After the amendment, safety assessments included laboratory testing (blood chemistry and hematology profiles), physical examination, 12-lead electrocardiography, and vital signs (oral temperature, blood pressure, heart rate, and respiratory rate) and AE reporting. Treatment-emergent AEs were observed in 116 of 231 participants receiving placebo and 111 of 222 receiving fexofenadine. These AEs were possibly related to study medication in 19 (8.2\%) and 21 (9.5\%) of the participants receiving placebo and fexofenadine, respectively, and most frequently involved the digestive system. No clinically relevant differences in laboratory measures, vital signs, and physical examinations were observed. The findings show that fexofenadine hydrochloride, 30 mg, is well tolerated and has a good safety profile in children aged 2 to 5 years with allergic rhinitis. [\hyperlink{Effexor XR}{PMID: 17941284}, Henry Milgrom et al., 2007]

\hypertarget{pmid_36043350}{W}e investigated the efficacy and safety of fluoxetine, a selective serotonin reuptake inhibitor, for treating refractory primary monosymptomatic nocturnal enuresis in children. Children 8-18 years old with severe primary monosymptomatic nocturnal enuresis unresponsive to alarm therapy, desmopressin, and anticholinergics were screened for eligibility. After excluding children with daytime urinary symptoms, constipation, underlying urological, neuropsychiatric, endocrinological, or cardiac conditions, patients were randomly and equally assigned to 10 mg fluoxetine once daily or placebo for 12 weeks. The primary outcome was treatment response according to the International Children's Continence Society terminology. Treatment-related adverse effects and nighttime arousal were secondary outcomes. A total of 150 children were enrolled, of whom 110 (56 in fluoxetine group and 54 in placebo group) with a mean age of 11.8 (SD 2.46) years were finally analyzed. After 4 weeks, 7.1\% and 66.1\% of the fluoxetine group achieved complete response and partial response (defined as 50\%-99\% reduction of the number of wet nights), respectively, versus 0\% and 16.7\% of the placebo group ( Fluoxetine is safe treatment for refractory primary monosymptomatic nocturnal enuresis in children with good initial response that declines at 12 weeks. [\hyperlink{Effexor XR}{PMID: 36043350}, Mohamed Hussiny et al., 2022]

\hypertarget{pmid_3430711}{F}lomoxef (FMOX, 6315-S), a new parenteral oxacephem antibiotic, was evaluated for its safety, efficacy and pharmacokinetics in children. Twenty-six patients with bacterial infections were treated with FMOX. Clinical efficacy rate was 92\% and bacteriological cure rate was 85\%. Three cases of infections due to methicillin-resistant Staphylococcus aureus were cured with FMOX therapy. No severe adverse reactions or abnormalities of laboratory test data were associated with FMOX therapy, although loose stools and diarrhea occurred frequently (23\%). Serum half-lives of FMOX after a single bolus injection of 9 infants and children were 0.77 +/- 0.31 hour and excretion into urine was rapid. From these experiences, FMOX appeared to be a safe and effective antibiotic when used in children with susceptible bacterial infections. [\hyperlink{Effexor XR}{PMID: 3430711}, H Meguro et al., 1987]

\hypertarget{pmid_24338578}{E}mergence delirium (ED) is a leading problem in children after general anesthesia. Dexmedetomidine (DEX) can be administered prior to general anesthesia to decrease ED, although wide ranges of dose are used. This study was conducted to investigate the proper dosages of DEX to attenuate children's ED after sevoflurane anesthesia. Twenty-five children, aged 3 to 9, undergoing repair of epiblepharon were studied. A chosen dosage of DEX was infused for 10 minutes in the preoperative holding area. The dose of DEX started from 0.25 µg/kg, and then was increased or decreased by 0.25 µg/kg depending on the response of the previous patient, using the Dixon up-and-down method. After the surgery under general anesthesia with sevoflurane, ED was assessed by the Cravero 5-point emergence agitation scale (5-point scale) at the postanesthesia care unit. The 50\% and 95\% effective concentrations (EC50 and EC95 ) of DEX to attenuate ED were calculated by isotonic regression estimators. The EC50 to attenuate ED was 1.0 (95\% confidence interval [CI] 0.29 to 1.71) and EC95 was 1.43 µg/kg (95\% CI -1.73 to 4.60). No patient failed parental separation while the Modified Observer's Assessment of Alertness/Sedation Scale at the end of the infusion was scattered from 1 to 5. One child who received 1.50 µg/kg had brief desaturation but recovered soon after being given a verbal command. Dexmedetomidine can be safely used between 1.0 and 1.43 µg/kg to attenuate children's ED after sevoflurane anesthesia. [\hyperlink{Effexor XR}{PMID: 24338578}, Sohee Yang et al., ]

\hypertarget{pmid_7961355}{T}he objective of this open study was to determine the efficacy and safety of fluoxetine for the treatment of children and adolescents with anxiety disorders. Twenty-one patients with overanxious disorders, social phobia, or separation anxiety disorder, who were unresponsive to previous psychopharmacological and psychotherapeutic interventions, were treated openly with fluoxetine for up to 10 months. Patients with lifetime histories of obsessive-compulsive disorder (OCD) or panic disorder, or with current major depression, were excluded. Beneficial and adverse effects of fluoxetine were ascertained using the improvement and severity subscales of the Clinical Global Impression Scale (CGIS) in two ways: (1) independent chart reviews by two child psychiatrists and (2) prospective assessments by the treating nurses and the patients' mothers. Eighty-one percent (n = 17) of patients showed moderate to marked improvement in anxiety symptoms. The severity of anxiety as measured by the CGIS was also significantly reduced from marked to mild (effect size: 2.3). There were no significant side effects. These results suggest that fluoxetine may be an effective and safe treatment for nondepressed children and adolescents with anxiety disorders other than OCD and panic disorder. Future investigations using double-blind, placebo-controlled methodologies are warranted. [\hyperlink{Effexor XR}{PMID: 7961355}, B Birmaher et al., 1994]

\hypertarget{pmid_19597919}{F}ollowing a previous preliminary report on a group of children suffering from partial epilepsies, we present the final considerations on the same group in order to evaluate the long-term efficacy, tolerability and safety of oxcarbazepine (OXC). We enrolled 36 patients (mean age 8.5), between January 2003 and December 2004, with new diagnosis of partial epilepsy: 25 patients were affected by idiopathic partial epilepsy, eight by symptomatic epilepsy and three by cryptogenic epilepsy. Each patient was scheduled to attend the center four times after the initial examination: 3 months (T1), 12 months (T2), 24 (T3) months and 36 (T4) months after the beginning of OXC-monotherapy (T0). At the end of our study, 20 patients were seizure free (SF): nine stopped OXC because of SF for at least 2 years, 11 were still on therapy. One patient showed a reduction of seizure frequency >or=50\%, three were non responders (but still on therapy), nine stopped OXC due to a non-responder condition during follow-up before T4 and one because of adverse effects. At the end of the study no EEG focal abnormalities became generalized because of treatment. Normalization of EEG was observed in ten patients. Our preliminary findings have been confirmed. OXC can be considered an effective and well tolerated first line drug for long-term monotherapy in children with epilepsy, both for idiopathic and symptomatic/cryptogenic forms. [\hyperlink{Effexor XR}{PMID: 19597919}, Emilio Franzoni et al., 2009]

\hypertarget{pmid_12182376}{F}luoroquinolones (FQs) have been infrequently used in children, largely because of concern that these agents can cause lesions of the cartilage in juvenile animals. However, the relevance of this laboratory observation to children treated with FQs is unknown. A retrospective, observational study was conducted to assess the incidence and relative risk of tendon or joint disorders (TJDs) that occur after use of selected FQs compared with azithromycin (AZ), a drug with no known effect on cartilage or tendons in humans or animals. An automated database was searched to identify patients younger than 19 years who had been prescribed ofloxacin (OFX), levofloxacin, ciprofloxacin (CPX), or AZ. Potential cases of TJD occurring within 60 days of a prescription of one of the study drugs were identified based on assignment of a claims diagnosis consistent with a TJD within this period. Verified cases were identified by a blinded review of abstracts of medical records from subjects identified as potential cases. The incidence of verified TJD was 0.82\% for OFX (13 of 1593) and CPX (37 of 4531) and was 0.78\% for AZ (118 of 15,073). The relative risk of TJD for OFX and CPX compared with AZ was 1.04 (95\% confidence interval, 0.55 to 1.84) and 1.04 (95\% confidence interval, 0.72 to 1.51), respectively. The distributions of claims diagnoses and time to onset of TJD were comparable for all groups. The most frequently reported category of TJD involved the joint followed by tendon, cartilage and gait disorder. In this observational study involving more than 6000 FQ-treated children, the incidence of TJD associated with selected FQ use in children was <1\% and was comparable with that of the reference group, children treated with AZ. [\hyperlink{Effexor XR}{PMID: 12182376}, Chuen L Yee et al., 2002]

\hypertarget{pmid_33505889}{T}he long-term efficacy and safety of infliximab (IFX) in children with ulcerative colitis (UC) have not been well-evaluated. Here, we reviewed the long-term durability and safety of IFX in our single center pediatric cohort with UC. This retrospective study included 20 children with UC who were administered IFX. For induction, 5 mg/kg IFX was administered at weeks 0, 2, and 6, followed by every 8 weeks for maintenance. The dose and interval of IFX were adjusted depending on clinical decisions. Corticosteroid (CS)-free remission without dose escalation (DE) occurred in 30\% and 25\% of patients at weeks 30 and 54, respectively. Patients who achieved CS-free remission without DE at week 30 sustained long-term IFX treatment without colectomy. However, one-third of the patients discontinued IFX treatment because of a primary nonresponse, and one-third experienced secondary loss of response (sLOR). IFX durability was higher in patients administered IFX plus azathioprine for >6 months. Four of five patients with very early onset UC had a primary nonresponse. Infusion reactions (IRs) occurred in 10 patients, resulting in discontinuation of IFX in four of these patients. No severe opportunistic infections occurred, except in one patient who developed acute focal bacterial nephritis. Three patients developed psoriasis-like lesions. IFX is relatively safe and effective for children with UC. Clinical remission at week 30 was associated with long-term durability of colectomy-free IFX treatment. However, approximately two-thirds of the patients were unable to continue IFX therapy because of primary nonresponse, sLOR, IRs, and other side effects. [\hyperlink{Effexor XR}{PMID: 33505889}, Hirotaka Shimizu et al., 2021]

\hypertarget{pmid_26949403}{F}ood allergy is the most common cause of anaphylaxis in children. Intramuscular delivery of epinephrine auto-injectors (EAI) is the standard of care for the treatment of anaphylaxis. We examined if children and adolescents at risk of anaphylaxis weighing 15-30 kg and >30 kg would receive epinephrine into the intramuscular space with the currently available EAI in North America and Europe. The distance from skin to muscle (STMD) and skin to bone (STBD) on the mid third anterolateral area of the right thigh was measured by ultrasound applying either high pressure (max) or slight pressure (min) in 102 children weighing 15-30 kg (group 1) and 100 children and adolescents, weighing more than 30 kg (group 2). Using a high pressure EAI (HPEAI), Epipen Jr(®) and Auvi-Q(®)/Allerject(®) 0.15 mg, 11/102 (11 \%) children in group 1 and 38/102 (38 \%) using another HPEAI, Jext(®), had a STMDmax that showed a risk of intraosseous injection. There was a 1 \% risk of subcutaneous injection with these devices. There was no risk of intraosseous injection using a low pressure EAI (LPEAI), Emerade(®). In group 2, the risk of intraosseous injection using a HPEAI was 3 \% and no risk using a LPEAI. However, the risk of subcutaneous injection using HPEAI was 9 \% and using LPEAI was 2 \%. There is a risk of intraosseous injection using HPEAI (Epipen(®)/Epipen Jr(®), Auvi-Q(®)/Allerject(®) and especially Jext(®)) in children at risk of anaphylaxis. There was also a risk of subcutaneous injection using the currently available HPEAI in children and adolescents. [\hyperlink{Effexor XR}{PMID: 26949403}, Sten Dreborg et al., 2016]

\hypertarget{pmid_2391760}{W}e have evaluated norfloxacin (NFLX) tablets for therapeutic effectiveness and safety in children. The results are summarized as follows. 1. A clinical study was performed on 14 children with infections, including 12 with urinary tract infections and 2 with acute bronchitis. Doses ranging from 1.7 to 5.4 mg/kg body weight were given b.i.d. or t.i.d.. Lengths of treatment ranged from 3 to 15 days. The therapeutic responses were considered "excellent" in 8 and "good" in 5, with an efficacy rate of 93\%. 2. Side effects were observed in 2 cases, one with light-headed feeling and one with vomiting. In clinical laboratory tests, eosinophilia was found in 2 cases and GOT was slightly elevated in 1 case. It has been concluded that NFLX is a usable drug for the treatment of bacterial infections in children. [\hyperlink{Effexor XR}{PMID: 2391760}, H Morita et al., 1990]

\hypertarget{pmid_28827252}{C}eftriaxone is widely used in children in the treatment of sepsis. However, concerns have been raised about the safety of ceftriaxone, especially in young children. The aim of this review is to systematically evaluate the safety of ceftriaxone in children of all age groups. MEDLINE, PubMed, Cochrane Central Register of Controlled Trials, EMBASE, CINAHL, International Pharmaceutical Abstracts and adverse drug reaction (ADR) monitoring systems will be systematically searched for randomised controlled trials (RCTs), cohort studies, case-control studies, cross-sectional studies, case series and case reports evaluating the safety of ceftriaxone in children. The Cochrane risk of bias tool, Newcastle-Ottawa and quality assessment tools developed by the National Institutes of Health will be used for quality assessment. Meta-analysis of the incidence of ADRs from RCTs and prospective studies will be done. Subgroup analyses will be performed for age and dosage regimen. Formal ethical approval is not required as no primary data are collected. This systematic review will be disseminated through a peer-reviewed publication and at conference meetings. CRD42017055428. [\hyperlink{Effexor XR}{PMID: 28827252}, Linan Zeng et al., 2017]

\hypertarget{pmid_7334587}{C}linical evaluation was carried out on cefroxadine dry syrup (containing 100 mg of cefroxadine per 1 g) for child use, and the following results were obtained. 1. Serum levels: Peak serum levels at 1 hour after single administration of CXD 100 mg (9.1 mg/kg) to a 4-year old child (11kg) and 300 mg (12.8 mg/kg) to a 8-year old child (23.5 kg) were 20.32 microgram/ml and 18.75 microgram/ml, respectively. They declined to 0.78 microgram/ml and 0.88 microgram/ml respectively after 6 hours and to undetectable levels after 8 hours. Half-life was 1 hour and 1.2 hours, respectively. CXD has shown the same concentration pattern as CEX, except for the fact that serum levels were peaked after 30 minutes and not detectable after 6 hours. 2. Clinical responses: CXD was administered, for 7 days, to 33 children with scarlet fever in the dosage of greater than or equal 20 approximately less than 60 mg/kg/day (7 children in greater than or equal to 20 approximately 30 mg/kg/day, 21 in greater than or equal to 30 approximately less than 40 mg/kg/day and 5 in greater than or equal to 40 approximately less than 60 mg/kg/day). Clinical responses were excellent in 19 cases and good in 14 cases, with an efficacy rate of 100\%. All strains of group A Streptococcus isolated from the pharynx of 22 children were eradicated within 24 hours. In 1 case each of acute pharyngitis, acute tonsillitis, acute laryngotracheitis and staphylococcal scalded skin syndrome, the dosage of greater than or equal to 30 approximately less than 45 mg/kg/day produced a 100\% good clinical response and eliminated the causative pathogens. 3. Side effect: Only 2 cases of eosinophilia were observed in hematologic study as well as in hepatic and renal function tests before and after administration. [\hyperlink{Effexor XR}{PMID: 7334587}, M Minamitani et al., 1981]

\hypertarget{pmid_3531565}{P}harmacokinetic and clinical studies of cefixime (CFIX) in children were done and the following results were obtained. Serum and urinary concentrations of CFIX were determined in 6 children aged 5 to 14 years given single doses of 1.5 or 6.0 mg/kg. Mean serum concentrations peaked at 4 hours after the administration of either 1.5 or 6.0 mg/kg, and respective peak values were 0.71 and 4.46 micrograms/ml. Biological half-lives for the low and the high doses were 5.28 and 4.45 hours, respectively. The 12-hours urinary recovery ranged from 7.0 to 13.8\% after administration of 1.5 mg/kg, and the 8-hours urinary recovery was 18.1\% after administration of 6.0 mg/kg. Therapeutic responses were recorded as excellent or good in 43 (97.7\%) of the children, comprising 13 with tonsillitis and 31 with scarlet fever. The microbiological effectiveness of CFIX on identified pathogens comprising 29 strains of S. pyogenes and 2 strains of S. aureus was satisfactory as evidence by a high eradication rate of 93.5\%. No clinical side effects were observed. Abnormal laboratory findings were elevation of GOT and/or GPT in 4 patients and eosinophilia in 1 patient. In conclusion, CFIX was found to be efficacious and safe for the treatment of bacterial infections in children. [\hyperlink{Effexor XR}{PMID: 3531565}, T Nishimura et al., 1986]

\hypertarget{pmid_3761541}{W}e used cefixime (CFIX), a newly developed oral cephalosporin antibiotic, to treat 21 children with various infections. The results are summarized as follows. The serum half-lives of CFIX after an administration of 6 mg/kg to each of 2 children were 2.56 and 2.79 hours. The serum concentrations were high enough to ensure the therapeutic response. The clinical response was "excellent" in 16 children and "good" in 5, with a 100\% efficacy rate. No side effects were recorded. The only abnormal finding was slight eosinophilia in 1 child. [\hyperlink{Effexor XR}{PMID: 3761541}, S Furukawa et al., 1986]

\hypertarget{pmid_3430725}{A} new oxacephem antibiotic, flomoxef sodium (FMOX, 6315-S), was studied for its clinical efficacy in the field of pediatrics. The treated patients were infants and children ranging from 6 months to 14 years old suffering from bacterial pneumonia in 3 cases, acute tonsillitis in 2 cases, acute enterocolitis in 2 cases, and cellulitis and urinary tract infection in 1 case each, a total of 9 cases. FMOX was administered at (levels of) 57-150 mg/kg in daily dose with durations of treatment ranging from 5 to 18 days. Clinical efficacies of good or excellent results were obtained in all cases (excellent in 4, good in 5). As an adverse reaction, eosinophilia was observed in 1 patient. This elevation is, however, normalized with the cessation of the treatment. [\hyperlink{Effexor XR}{PMID: 3430725}, H Ogura et al., 1987]

\hypertarget{pmid_25593242}{R}ituximab (RTX) has been used to treat many pediatric autoimmune conditions. We investigated the safety and efficacy of RTX in a variety of pediatric autoimmune diseases, especially systemic lupus erythematosus (SLE). Retrospective study of children treated with RTX. Effectiveness data was recorded for patients with at least 12 months of followup; safety data was recorded for all subjects. The study included 104 children; 50 had SLE. Improvements in corticosteroid dosage, physician's global assessment of disease activity, and SLE-associated markers of disease activity were seen. The incidence of hospitalized infections was similar to previous studies of patients with childhood-onset SLE. RTX can be safely administered to children and appears to contribute to decreased disease activity and steroid burden. [\hyperlink{Effexor XR}{PMID: 25593242}, Ajay Tambralli et al., 2015]

\hypertarget{pmid_28293110}{T}o assess the efficacy and safety of oxcarbazepine (OXC) in the treatment of children with epilepsy. Randomized controlled trials (RCTs) published in PubMed, Embase, Web of Science, Cochrane Library, Scopus, SinoMed (Chinese BioMedical Literature Service System, China), and Chinese National Knowledge Infrastructure (China) database were systematically reviewed. Eligible studies were those that compared the efficacy and safety of OXC with other antiepileptic drugs in epilepsy. Risk ratio (RR) with 95\% confidence intervals (95\% CIs) was calculated using fixed-effects or random-effects model. Eleven RCTs with a total of 1,241 patients met the inclusion criteria and were included in this meta-analysis. Compared with other antiepileptic drugs (sodium valproate, levetiracetam, phenytoin, and placebo), OXC was associated with similar seizure-free rate (RR =1.06, 95\% CI: 0.94, 1.20;  OXC showed similar effects and safety as other antiepileptic drugs in the treatment of children with epilepsy. Further well-conducted, large-scale RCTs are needed to validate these findings. [\hyperlink{Effexor XR}{PMID: 28293110}, Hua Geng et al., 2017]

\section*{Trospium Chloride}
\subsection*{Result}
\subsubsection*{Answer}

Children (as per study, likely school-aged): Yes
Infants (<1 year): Unknown

\subsubsection*{{Explanation}}
\hypertarget{Trospium Chloride}
To determine the safety of Trospium Chloride in children, I reviewed all available abstracts for evidence of targeted safety studies in pediatric populations.

One abstract specifically addresses the use of Trospium Chloride in children:

- [\hyperlink{pmid_14532838}{PMID: 14532838}, P Lopez Pereira et al., 2003]: This multicenter, randomized, single-blind clinical study evaluated the efficacy and safety of Trospium Chloride in 58 children with bladder instability. Children were randomized to receive various doses of Trospium Chloride (10, 15, 20, or 25 mg daily) or placebo for 21 days. All adverse events were recorded. The study found that Trospium Chloride was well tolerated, with only 10\% of patients experiencing adverse effects, and concluded that Trospium Chloride is an effective option for the management of detrusor instability in children.

Other abstracts on Trospium Chloride focus on adult populations or do not specify pediatric data. Therefore, based on the available abstracts, only the study above provides direct evidence regarding safety in children.

The age range in the cited pediatric study is not explicitly stated in the abstract, but as it refers to "children" and is a targeted pediatric study, it provides evidence for safety in the pediatric population as defined by the study's inclusion criteria.

No abstracts were found that specifically address safety in infants (under 1 year), so safety in this age group remains unknown.

In summary:
- For children (age range as defined by the study, likely school-aged): There is evidence from a targeted clinical trial supporting the safety of Trospium Chloride.
- For infants (<1 year): No data available; safety is unknown.

\subsection*{Abstracts}
\hypertarget{pmid_14532838}{W}e assessed the efficacy and most appropriate dosage of trospium chloride (TCl) for managing bladder instability in children as compared with a placebo. A total of 58 patients with bladder instability were allocated at random to 1 of 5 groups-10, 15, 20 or 25 mg TCl, or placebo administered daily in a multicenter, randomized, single-blind clinical study. Patients were treated for 21 days, and current symptoms, voiding diary and urodynamic values were collected at the beginning and end of the treatment period. All adverse events were recorded at the last visit. Of 50 patients treated with TCl 41 (82\%) had a positive therapeutic result (excellent, good or fair) versus only 3 of 8 patients with improvement in the placebo group (37.5\%, p = 0.006). In all responding patients clinical symptoms either resolved or decreased markedly, and in 37 (74\%) this improvement was accompanied by urodynamic improvement. In these 37 children the average number of uninhibited contractions decreased by 54.3\% (p <0.0001) and the volume at first contraction increased by 71.4\% (p = 0.001). There were no statistically significant differences with regard to therapeutic efficacy between TCl dosages. Fourteen patients (9 with TCl, 5 with placebo) showed no clinical improvement, although some had improved urodynamic parameters. Furthermore, TCl was well tolerated with few patients (10\%) experiencing adverse effects. Trospium chloride (10 to 25 mg total daily dosage, split into 2 doses) is an effective option for the management of detrusor instability in children. [\hyperlink{Trospium Chloride}{PMID: 14532838}, P Lopez Pereira et al., 2003]

\hypertarget{pmid_15126811}{T}rospium chloride is an anticholinergic agent with predominantly peripheral nonselective antimuscarinic activity lacking central nervous system effects. It has no known drug-drug interactions, an advantage for patients taking many medications. Because these qualities may provide added benefit when treating patients with symptoms associated with overactive bladder (OAB) and urge incontinence, we studied the effectiveness of trospium in treating these conditions. Patients with OAB with urge incontinence were randomized 1:1 to 20 mg trospium twice daily or placebo in this 12-week, multicenter, parallel, double-blind, placebo controlled trial. Dual primary end points were change in average number of toilet voids and change in urge incontinent episodes per 24 hours. Secondary efficacy variables were change in average of volume per void, voiding urge severity, urinations during day and night, time to onset of action and change in Incontinence Impact Questionnaire. A total of 523 patients were entered at 51 sites. Trospium significantly decreased average frequency of toilet voids and urge incontinent episodes compared to placebo. It significantly increased average volume per void, and decreased average urge severity and daytime frequency. All effects occurred by week 1 and all were sustained throughout the study. Nocturnal frequency decreased significantly by week 4 and Incontinence Impact Questionnaire scores improved at week 12. Trospium was well tolerated. Trospium was found to have sustained effectiveness beginning at the end of week 1 in decreasing the number of voids, urge incontinent episodes, total daily micturitions and urge severity, and in increasing volume per void. It also improved symptoms of OAB and quality of life. [\hyperlink{Trospium Chloride}{PMID: 15126811}, Norman Zinner et al., 2004]

\hypertarget{pmid_15482001}{T}rospium chloride is an orally active, quaternary ammonium compound with antimuscarinic activity. It binds specifically and with high affinity to muscarinic receptors M(1), M(2) and M(3), but not nicotinic, cholinergic receptors. It is hydrophilic and does not cross the normal blood-brain barrier in significant amounts and, therefore, has minimal central anticholinergic activity. Peak plasma trospium chloride concentrations are attained approximately 5-6 hours after oral administration, which should occur before meals as concurrent food ingestion significantly reduces trospium bioavailability. Trospium chloride undergoes negligible metabolism by the hepatic cytochrome P450 system; few metabolic drug interactions are known. While trospium chloride dosage adjustments based on age or sex appear unwarranted, such adjustments may be needed in patients with severe renal impairment. Direct comparative studies in patients with overactive bladder indicate that trospium chloride is at least as effective as oxybutynin and tolterodine. Placebo-controlled studies have also confirmed the efficacy of trospium chloride in terms of improved urodynamic parameters; small-scale, noncomparative studies have documented significant trospium chloride-induced improvements in patients with reflex neurogenic bladder, postoperative bladder irritation and radiation-induced cystitis; and observational studies including >10,000 patients have also revealed favourable findings for trospium chloride, including a marked decrease in incontinence episodes and substantial improvement in health-related quality of life. Trospium chloride is generally well tolerated, and significantly more so than immediate-release oxybutynin. The most frequent adverse events, occurring in >1\% of trospium chloride-treated patients, are dry mouth, dyspepsia, constipation, abdominal pain and nausea. Available for many years in several countries outside North America, trospium chloride is likely to develop an important role in the management of overactive bladder following its approval in the US on 28 May 2004. [\hyperlink{Trospium Chloride}{PMID: 15482001}, Eric S Rovner et al., 2004]

\hypertarget{pmid_18360555}{T}rospium chloride is a quaternary ammonium compound, which is a competitive antagonist at muscarinic cholinergic receptors. Preclinical studies using porcine and human detrusor muscle strips demonstrated that trospium chloride was many-fold more potent than oxybutynin and tolterodine in inhibiting contractile responses to carbachol and electrical stimulation. The drug is poorly bioavailable orally (< 10\%) and food reduces absorption by 70\%- 80\%. It is predominantly eliminated renally as unchanged compound. Trospium chloride, dosed 20 mg twice daily, is significantly superior to placebo in improving cystometric parameters, reducing urinary frequency, reducing incontinence episodes, and increasing urine volume per micturition. In active-controlled trials, trospium chloride was at least equivalent to immediate-release formulations of oxybutynin and tolterodine in efficacy and tolerability. The most problematic adverse effects of trospium chloride are the anticholinergic effects of dry mouth and constipation. Comparative efficacy/tolerability data with long-acting formulations of oxybutynin and tolterodine as well as other anticholinergics such as solifenacin and darifenacin are not available. On the basis of available data, trospium chloride does not appear to be a substantial advance upon existing anticholinergics in the management of urge urinary incontinence. [\hyperlink{Trospium Chloride}{PMID: 18360555}, David Rp Guay et al., 2005]

\hypertarget{pmid_8494577}{T}he safety and tolerance of increasing single oral doses of 20, 40, 80, 120, 180, 240 and 360 mg trospium chloride (Spasmo-lyt, CAS 10405-0204) were investigated in 29 healthy male volunteers in a double-blind placebo-controlled study. Blood pressure, heart rate, ECG, pupillary diameter, salivary secretion, and subjective reports of tolerance revealed no essential differences between placebo and trospium chloride in doses up to 120 mg. Starting with single doses of 180 mg, anticholinergic effects were observed with increasing intensity, i.e., dilatation of the pupils, reduction of salivary flow, and increase of heart rate. While the highest administered dose of 360 mg trospium chloride did not cause any relevant changes of vital parameters (blood pressure, pulse, ECG), it was subjectively rated as quite unpleasant. The data show that trospium chloride is well tolerated in single oral doses well above the current therapeutic daily dose of up to 40 mg. [\hyperlink{Trospium Chloride}{PMID: 8494577}, H P Breuel et al., 1993]

\hypertarget{pmid_16461077}{T}o study the efficacy and safety of trospium chloride in treating overactive bladder. Trospium chloride is an anticholinergic agent with predominantly peripheral nonselective antimuscarinic activity and thus has potential therapeutic value in treating patients with overactive bladder. Patients with overactive bladder were randomized on a 1:1 basis to either placebo or trospium chloride 20 mg twice daily in this 12-week, multicenter, parallel, double-blind, placebo-controlled study. The primary endpoint was the change in the average number of toilet voids per 24 hours. The secondary efficacy variables were changes in the average void urgency severity, volume per toilet void, urge frequency, number of daily urge urinary incontinence episodes, and daytime sleepiness. A total of 658 patients were randomized at 52 sites. Trospium chloride significantly decreased the average number of daily toilet voids, average urgency severity, urge frequency, and urge urinary incontinence episodes and increased the average volume per void at weeks 1, 4, and 12. All effects occurred by the end of week 1 and all improved and were sustained throughout the 12-week study. Adverse events included dry mouth and constipation. Trospium chloride had significant and sustained effectiveness beginning at the end of week 1 and continuing through 12 weeks of treatment. Trospium chloride was also safe and generally well tolerated. [\hyperlink{Trospium Chloride}{PMID: 16461077}, Delbert Rudy et al., 2006]

\hypertarget{pmid_9622945}{T}he authors investigated in a group of six children with glaucoma persisting for a long time the possibility to use locally applied carbonic anhydrase inhibitor, 2\% dorsolamide hydrochloride in the form of eye drops (TRUSOPT, Merck Co.). In the submitted preliminary study they evaluate the effectiveness of the drug in glaucoma in children very favourably, previous essential treatment with oral acetazolamide could be discontinued in all patients without a deleterious effect. The authors did not encounter any undesirable effects of the drug nor manifestations of intolerance calling for discontinuation of TRUSOPT treatment. This is so far the first communication on TRUSOPT treatment of child glaucoma in the available literature. [\hyperlink{Trospium Chloride}{PMID: 9622945}, J Rehůrek et al., 1998]

\hypertarget{pmid_7900236}{H}igh doses of metoclopramide are contraindicated to prevent chemotherapy-induced emesis in pediatric patients, since the incidence of extrapyramidal reactions is increased in these patients. The aim of this small study was to evaluate the antiemetic activity and the safety of tropisetron (a new selective antagonist of 5-HT3 receptors) in children who suffered nausea and vomiting during previous chemotherapy courses, despite the administration of an anxiolytic agent (hydroxyzine hydrochloride). The children with a malignant neoplasm were treated for emesis with tropisetron (5 mg o.a.d. or b.i.d.) during a total of 20 cycles of chemotherapy with carboplatin combined with other antitumor agents. In 14 cycles (70\%), there was no vomiting. There were two or less episodes of vomiting in 2 cycles (10\%), 3-4 episodes in 2 cycles (10\%), and no inhibition of vomiting at all in 2 cycles (10\%). In 8 cycles there were no episodes of nausea (40\%), in 5 cycles (25\%) there were episodes of moderate nausea, and in 4 (20\%) there were episodes of severe nausea. One child had a mild headache during one cycle and moderate hypotension during another. The results suggest that tropisetron is both efficacious and safe for the treatment of pediatric patients. [\hyperlink{Trospium Chloride}{PMID: 7900236}, P Rosso et al., 1994]

\hypertarget{pmid_1398851}{W}e prospectively studied the pharmacokinetics of intravenous Chloramphenicol succinate (CS) in children (age 6 months-14 years) with culture proven typhoid fever (n = 30) and non typhoidal illnesses (n = 10). CS was administered in three different dosage regimens (50, 75 and 100 mg/kg/d-q 6 hourly). Liver function tests were monitored. Plasma trough and peak chloramphenicol concentrations were measured by HPLC analysis after 42 hrs. The 50 mg/kg/day dosage schedule was terminated midway through the study, as blood levels were consistently low and two patients with typhoid relapsed, children with typhoid had significantly lower clearance of CS in comparison with those with non-typhoidal illness (0.29 +/- 0.1 versus 0.5 +/- 0.37 1/kg/hr, P 0.05). There was no significant difference between mean peak and trough concentrations of chloramphenicol on 100 mg/kg/day and 75 mg/kg/day in children with typhoid. However, two children on 100 mg/kg/day dosage developed trough concentrations greater than 20 mcg/ml. No correlation was found between CS clearance and serum bilirubin, SGPT (alanine transaminase) and alkaline phosphatase. Our data show altered clearance of CS in children with typhoid and suggests that 75 mg/kg/day may be a safer dose in children with hepatic dysfunction in typhoid. [\hyperlink{Trospium Chloride}{PMID: 1398851}, Z A Bhutta et al., ]

\hypertarget{pmid_36625617}{T}o evaluate the efficiency of long-term use of trospium chloride (Spazmex) for the treatment of patients with neurogenic overactive bladder due to Parkinson's disease (PD) and to determine the influence of therapy on the cognitive status of patients. 60 patients with PD and neurogenic overactive bladder with stages 2.5, 3 and 4 according to Hoehn-Yahr scale were included in the main group. The mean age was 58.2+/-5.7 years. All patients were prescribed trospium chloride at entry into the study, with doses titrated gradually according to clinical efficacy (30 to 90 mg). The comparison group included 15 patients with PD and neurogenic overactive bladder at stages 2,5 and 3, who received tibial neuromodulation according to the standard technique with skin electrodes. The mean age of patients was 56.4+/-4.6 years. At baseline, both groups were comparable in terms of gender, age and cognitive status (p=0.801). All patients received treatment for 52 weeks. The efficiency of therapy was assessed according to bladder diaries, while safety outcomes included postvoid residual, side effects, cognitive status according to the MoCA scale and quality of life according to the SF-Qualiveen questionnaire. clinical efficacy and satisfaction were achieved in all patients who completed the study (47 patients in the main group and 15 patients in the comparison group). Good clinical efficacy was demonstrated in both groups, since there was a decrease in the number of urinations, episodes of urgency and urinary incontinence. In addition, there was an improvement in the quality of life according to the SF-Qualiveen scale. The cognitive status during the entire follow-up period remained without significant changes in both groups. Trospium chloride is an effective drug in patients with PD. It does not affect cognitive functions during long-term use. Trospium chloride should be considered as first-line drug in those with urologic manifestations of PD. [\hyperlink{Trospium Chloride}{PMID: 36625617}, E S Korshunova et al., 2022]

\hypertarget{pmid_17632131}{A}n extended release formulation of trospium chloride was recently developed for the once daily treatment of overactive bladder. We investigated the safety, efficacy and tolerability of 60 mg trospium chloride once daily. Subjects with overactive bladder were randomized 1:1 to receive 60 mg trospium chloride once daily or placebo in this 12-week multicenter, parallel, double-blind, placebo controlled trial. Primary end points were calculated changes in diary recorded daily urinary frequency and daily urgency urinary incontinence episodes. Secondary end points were urgency severity, volume voided per void and the number of urgency voids per day. Safety was assessed by clinical examination, adverse event monitoring, clinical laboratory values and resting electrocardiograms. Overall 601 subjects were prescribed trospium once daily (298) or placebo (303). Trospium once daily treatment resulted in significant improvements over placebo in all primary and key secondary efficacy outcomes at weeks 1 through 12. The most common adverse events were dry mouth (trospium 8.7\% vs placebo 3\%) and constipation (trospium 9.4\% vs placebo 1.3\%). Central nervous system adverse events were rare (headache with trospium 1.0\% vs placebo 2.6\%). No clinically meaningful changes in laboratory, physical examination or electrocardiogram parameters were noted. Trospium once daily provided significant improvements in overactive bladder symptoms (frequency, urgency urinary incontinence and urgency). Efficacy was similar to that seen previously with trospium chloride twice daily, while class effect anticholinergic adverse events occurred at comparatively low levels. Dry mouth was elicited at the lowest reported rate in the oral antimuscarinic drug class. [\hyperlink{Trospium Chloride}{PMID: 17632131}, David Staskin et al., 2007]

\hypertarget{pmid_34853794}{T}ris(1,3-dichloro-2-propyl) phosphate (TDCIPP) has been widely used as a flame retardant and is commonly detected in environmental samples. Biomonitoring studies relying on urinary metabolite levels (i.e. bis(1,3-dichloro-2-propyl) phosphate (BDCIPP)) demonstrate widespread exposure, but TDCIPP intake is unknown. Intake data area critical component of meaningful risk assessments and are needed to elucidate the potential health impacts of TDCIPP exposure. Using biomonitoring data, we estimated TDCIPP intake for infants. Infants aged 2-18 months were recruited from central, North Carolina (n=43, recruited 2014-2015), and spot urine samples were analyzed for BDCIPP. TDCIPP intake rates were estimated using daily urine excretion and the fraction of TDCIPP excreted as BDCIPP in urine. Daily TDCIPP intake estimates ranged from 0.01-15.03 μg/kg-day for children included in our assessment, with some variation depending on model assumptions. The U.S. Consumer Products Safety Commission (CPSC) previously established an acceptable daily intake of 5μg/kg-day for non-cancer health risks. Depending on modeling assumptions, we found that 2-9\% percent of infants had TDCIPP intake estimates above this threshold. Our results indicate that current TDCIPP exposure levels could pose health risks for highly exposed infants. [\hyperlink{Trospium Chloride}{PMID: 34853794}, Kate Hoffman et al., 2017]

\hypertarget{pmid_8439640}{T}rospium chloride is a muscarinergic antagonist acting on oesophageal smooth muscle and on ganglionic and/or myenteric neurons. The effect of this drug on oesophageal motility was tested in 16 healthy male subjects in a double-blind randomized cross-over examination of trospium chloride or placebo following phentolamine or placebo application. Each subject underwent two separate investigations at least one week apart. Trospium chloride was effective in the oesophagus to reduce contractile activity (amplitude and duration of peristalsis) in all parts of the oesophageal body, and this effect was not blocked by phentolamine. Its potent action and its minor side-effects appear to be promising for clinical use in patients with motility disorders such as the hypercontractile oesophagus. [\hyperlink{Trospium Chloride}{PMID: 8439640}, T Frieling et al., 1993]

\hypertarget{pmid_19813252}{T}ranscranial Doppler ultrasonography (TCD) is used to predict stroke risk in children with sickle cell anemia (SCA), but has not been adequately studied in children under age 2 years. TCD was performed on infants with SCA enrolled in the BABY HUG trial. Subjects were 7-17 months of age (mean 12.6 months). TCD examinations were successfully performed in 94\% of subjects (n = 192). No patient had an abnormal TCD as defined in the older child (time averaged maximum mean TAMM velocity > or =200 cm/sec) and only four subjects (2\%) had velocities in the conditional range (170-199 cm/sec). TCD velocities were inversely related to hemoglobin (Hb) concentration and directly related to increasing age. Determination of whether the TCD values in this very young cohort of infants with SCA can be used to predict stroke risk later in childhood will require analysis of exit TCD's and long-term follow-up, which is ongoing (ClinicalTrials.gov number, NCT00006400). [\hyperlink{Trospium Chloride}{PMID: 19813252}, Steven G Pavlakis et al., 2010]

\hypertarget{pmid_22120415}{T}o analyse whether the permeability of the blood-brain barrier to the antimuscarinic drug trospium chloride is altered with ageing. This is a relevant question for elderly patients with overactive bladder syndrome who are treated with trospium chloride as the occurrence of adverse effects on the central nervous system (CNS) highly depends on the absolute drug concentration in the brain. Trospium chloride at 1 mg/kg was intravenously administered to adult, middle-aged, and aged mice at 6, 12, and 24 months of age, respectively, and the absolute drug concentrations in the brain were analysed after 2 h. Furthermore, mRNA expression levels of relevant markers of blood-brain barrier integrity (occludin, claudin-5, and the drug efflux carrier P-glycoprotein) were analysed in brain samples from adult and aged mice. The absolute brain concentrations of the drug were identical in adult and middle-aged mice (13 ± 2 ng/g vs. 13 ± 2 ng/g) and were slightly, but significantly, lower in aged mice (8 ± 4 ng/g). The brain/plasma drug concentration ratios were not different between the age groups and demonstrated the generally low capability of trospium chloride in permeating the blood-brain barrier. Occludin, claudin-5, and P-glycoprotein showed identical mRNA expression levels in the brains of adult and aged mice. Based on our in vivo data in a mouse model, we conclude that trospium chloride permeation across the BBB is not increased in ageing per se, and therefore, the occurrence of adverse CNS drug effects is also not expected to increase with ageing. [\hyperlink{Trospium Chloride}{PMID: 22120415}, Jasmin Kranz et al., 2013]

\hypertarget{pmid_23024102}{W}e conducted this single blind randomized clinical trial to compare the efficacy and safety of oral chloral hydrate and intranasal midazolam for induction of sedation for computerized tomography scan of brain in children. Participants aged 1-10 years (n=60) were randomized to receive 100 mg/kg chloral hydrate orally with intra nasal normal saline OR intranasal midazolam 0.2 mg/kg with oral normal saline. Adequate sedation (Ramsay sedation score of four) was obtained and CT scan completed successfully in 76.7\% of chloral hydrate group and in 40\% of midazolam group (P=0.004). No significant difference was seen for side effects frequency between the two drugs (10\% in chloral hydrate, 3.3\% in midazolam group; P=0.34). We conclude that oral chloral hydrate can be considered as a safe and effective drug for sedation in children undergoing CT scan of brain. [\hyperlink{Trospium Chloride}{PMID: 23024102}, Razieh Fallah et al., 2013]

\hypertarget{pmid_18370455}{I}n a placebo-controlled, double-blind study the effects of depressing duodenal motility by administration of intravenous trospium chloride during gastroduodenoscopy were studied in 72 patients randomised to receive trospium chloride or saline (controls). Intravenous trospium chloride 1.2mg stopped the visible contractile activity of the duodenum as assessed by 3 independent observers during videoendoscopy within 76 seconds (median). During a 4-minute observation period of duodenal peristalsis, duodenal motor activity was found to stop in 18 of 36 patients after trospium chloride but in only 5 of 36 patients in the placebo group (p = 0.002). Adverse effects were dry mouth, micturition difficulties, sweat retention, accommodation disturbance and tachycardia. Trospium chloride was effective in reducing contractile activity in the duodenum. Its potent action and minor adverse effect profile appear to be promising for gastroduodenoscopy and especially for sphincter of Oddi motility in patients during routine endoscopic retrograde cholangiopancreatography. [\hyperlink{Trospium Chloride}{PMID: 18370455}, H Rohde et al., 1997]

\hypertarget{pmid_16780480}{M}olluscum contagiosum is a common viral infection of the skin that frequently affects children. Lesions take between 6 and 18 months to resolve spontaneously and are a source of great embarrassment to both caretakers and children, often affecting attendance at school and limiting social activity. Treatment options to date have been poorly tolerated by children but recent studies have suggested that potassium hydroxide may be beneficial. This double-blind, randomized, placebo-controlled study compared 10\% potassium hydroxide with placebo (normal saline). Twenty patients, aged 2 to 12 years, were recruited. Parents applied a solution twice daily to lesional skin until signs of inflammation appeared. Children were examined by the same observer on days 0, 15, 30, 60, and 90. Seventy percent of children receiving topical potassium hydroxide cleared, compared with 20\% in the placebo group. Further dosing studies are required to identify whether weaker concentrations of potassium hydroxide are as efficacious, with less irritancy. [\hyperlink{Trospium Chloride}{PMID: 16780480}, Katherine A Short et al., ]

\hypertarget{pmid_31292919}{T}riclofos sodium (TFS) has been used for many years in children as a sedative for painless medical procedures. It is physiologically and pharmacologically similar to chloral hydrate, which has been censured for use in children with neurocognitive disorders. The aim of this study was to investigate the safety and efficacy of TFS sedation in a pediatric population with a high rate of neurocognitive disability. The database of the neurodiagnostic institute of a tertiary academic pediatric medical center was retrospectively reviewed for all children who underwent sedation with TFS in 2014. Data were collected on demographics, comorbidities, neurologic symptoms, sedation-related variables, and outcome. The study population consisted of 869 children (58.2\% male) of median age 25 months (range 5-200 months); 364 (41.2\%) had neurocognitive diagnoses, mainly seizures/epilepsy, hypotonia, or developmental delay. TFS was used for routine electroencephalography in 486 (53.8\%) patients and audiometry in 401 (46.2\%). Mean (± SD) dose of TFS was 50.2 ± 4.9 mg/kg. Median time to sedation was 45 min (range 5-245), and median duration of sedation was 35 min (range 5-190). Adequate sedation depth was achieved in 769 cases (88.5\%). Rates of sedation-related adverse events were low: apnea, 0; desaturation ≤ 90\%, 0.2\% (two patients); and emesis, 0.35\% (three patients). None of the children had hemodynamic instability or signs of poor perfusion. There was no association between desaturations and the presence of hypotonia or developmental delay. TFS, when administered in a controlled and monitored environment, may be safe for use in children, including those with underlying neurocognitive disorders. [\hyperlink{Trospium Chloride}{PMID: 31292919}, Eytan Kaplan et al., 2019]

\hypertarget{pmid_18278305}{C}hloral hydrate and hydroxyzine are a drug combination frequently used by practitioners to sedate pediatric dental patients, but their effectiveness has not been compared to a negative control group in humans. The aim of this crossover, double-blinded study was to evaluate the effect of these drugs compared to a placebo, administered to young children for dental treatment. Thirty-five dental sedation sessions were carried out on 12 uncooperative ASA I children aged less than 5 years old. In each session patients were randomly assigned to groups P (placebo), CH (chloral hydrate 75 mg/kg) and CHH (chloral hydrate 50 mg/kg plus hydroxyzine 2.0 mg/kg). Vital signs and behavioral variables were evaluated every 15 min. Comparisons were statistically analyzed using Friedman and Wilcoxon tests. P, CH and CHH had no differences concerning vital signs, except for breathing rate. All vital signs were in the normal range. CH and CHH promoted more sleep in the first 30 min of treatment. Overall behavior was better in CH and CHH than in P. CH, CHH and P were effective in 62.5\%, 61.5\% and 11.1\% of the cases, respectively. Chloral hydrate was safe and relatively effective, causing more satisfactory behavioral and physiological outcomes than a placebo. [\hyperlink{Trospium Chloride}{PMID: 18278305}, Luciane Ribeiro de Rezende Sucasas da Costa et al., 2007]

\hypertarget{pmid_8169182}{T}here is evidence for the efficacy and safety of clonazepam (CZP) in adult anxiety disorders, but no formal studies to substantiate clinical reports of similar benefit in children with anxiety disorders. In this double-blind pilot study, 15 children, aged 7 to 13 years, entered a randomly assigned, double-blind crossover trial of 4 weeks of CZP (up to 2 mg/day) and 4 weeks of placebo. Twelve children completed the trial. All but 1 had a diagnosis of separation anxiety disorder, and all but 2 had comorbid diagnoses. Nine children appeared to have moderate to significant clinical improvement, but statistical comparisons on several ratings failed to confirm a trend in favor of CZP. Side effects of drowsiness, irritability, and/or oppositional behavior were notable in 10 children in the CZP phase compared with 5 in the placebo phase. Clonazepam was believed to have clinical benefit for some children, but this was not confirmed statistically in this small sample. Problematic side effects of drowsiness and disinhibition were common and possibly were due to rapid titration. [\hyperlink{Trospium Chloride}{PMID: 8169182}, F Graae et al., ]

\hypertarget{pmid_10759661}{T}o assess the efficacy and safety of trospium chloride (TCl, 20 mg twice daily) in the treatment of detrusor instability, compared with placebo. In all, 208 patients were allocated at random to either TCl or placebo in a double-blind clinical study; the patients were treated for 3 weeks. Urodynamic values were measured at the beginning and end of the treatment period. Adverse events were recorded on patient diary cards. A confirmatory adaptive procedure with one planned interim analysis was used to evaluate efficacy. Trospium chloride produced significant improvements in maximum cystometric bladder capacity (median treatment effect 22.0 mL, mean 37.3 mL, one-sided P = 0. 0054) and urinary volume at first unstable contraction (median treatment effect 45.0 mL, mean 63.6 mL, one-sided P = 0.0015). The patients' assessment of efficacy showed significantly greater clinical improvement in the TCl group than in the placebo group (two-sided P = 0.0047). Furthermore, TCl was well tolerated, with similar frequencies of adverse events reported in both groups (68\% in the TCl and 62\% in the placebo group). Trospium chloride (20 mg twice daily) is an effective and safe option for the treatment of detrusor instability. [\hyperlink{Trospium Chloride}{PMID: 10759661}, L Cardozo et al., 2000]

\hypertarget{pmid_3899048}{N}ineteen children (mean [+/- SD] age, 14.5 +/- 2.3 years) with severe, primary obsessive-compulsive disorder completed a ten-week, double-blind, controlled trial of clomipramine hydrochloride (mean dosage, 141 mg/day) or placebo, each of which was administered for five weeks. Half of the subjects had not responded to previous treatment with other tricyclic antidepressants. There was a significant improvement in observed and self-reported obsessions and compulsions that was independent of the presence of depressive symptoms at baseline. Improvement in obsessive-compulsive symptoms did not correlate significantly with plasma concentrations of the drug or its metabolites. Clomipramine appears to be effective in the treatment of children with obsessive-compulsive disorder and the treatment seems to be independent of an antidepressant effect. [\hyperlink{Trospium Chloride}{PMID: 3899048}, M F Flament et al., 1985]

\hypertarget{pmid_15951862}{D}iagnostic and therapeutic procedures in children are made easier using sedation. However, there is no consensus about which drug should be used to achieve this. Furthermore, none of the drugs used for sedation are risk free. The aim of this work is to study sedation indications, effectiveness, and safety at our center. A prospective observational study conducted at the Pediatric Day Care Unit, King Fahad National Guard Hospital, Riyadh, Saudi Arabia. The study covered 17.5 weeks in 2 periods: May 9th 1999 to June 13th 1999 and October 31st 2001 to February 11th 2002. Children <12 years were included. Collected data included demographics, indication, drug dosing and outcome. Data were reported as mean +/- SD. We included 148 patients, age 38 +/- 30 months. Adequate sedation was achieved in 79\% after initial chloral hydrate (CH) dose of 56.9 +/- 9.3 mg/kg, in 95\% after adding 18.5 +/- 6.4 mg/kg CH and in 96\% after adding second drug. Compared to nonrespondents, first CH dose respondents were younger and lower in weight. The CH side effects were few and mild. Chloral hydrate is a safe and effective agent for sedation in children with an age and weight dependent response. [\hyperlink{Trospium Chloride}{PMID: 15951862}, Omar M Hijazi et al., 2005]

\hypertarget{pmid_27367297}{A} severe clinical phenotype along with concern for ensuring normal growth and development has a major impact on treatment choices for children newly diagnosed with Crohn's disease (CD). We review the increasingly outdated concept of 'conventional' therapy of pediatric CD based on aminosalicylates, corticosteroids, and immunomodulators for patients at high risk of complicated disease. Key safety concerns with each treatment are reviewed. There are minimal data supporting the use of aminosalicylates in the treatment of pediatric CD. Corticosteroids are effective short-term for improving signs and symptoms of disease but are ineffective for maintenance therapy. Thiopurines decrease corticosteroid dependence but may not alter progression to complicated disease requiring surgery. Concerns for lymphoma as well as hemophagocytic lymphohistiocytosis with thiopurines are valid. Further data are required on the efficacy and safety of methotrexate as an alternative immunomodulator. Though generally well tolerated and efficacious in most patients, anti-TNF-α therapy can be associated with both mild as well as more serious complications. Current data do not support an increased risk for malignancy associated with anti-TNF therapy alone in children. Anti-adhesion therapy appears to have a favorable safety profile but the experience in children is extremely limited. [\hyperlink{Trospium Chloride}{PMID: 27367297}, Victoria Grossi et al., 2016]

\section*{Verelan}
\subsection*{Result}
\subsubsection*{Answer}

Unknown

\subsubsection*{{Explanation}}
\hypertarget{Verelan}
A review of the available abstracts reveals that none of them are about Verelan (verapamil). The abstracts provided discuss the safety and efficacy of various medications in children, including dabigatran, rivaroxaban, levetiracetam, pregabalin, sertraline, and others, but there is no mention of Verelan or verapamil in any of the abstracts. Therefore, based on the abstracts available, there is not enough data to determine whether Verelan is safe or unsafe for use in children in any age range.

\subsection*{Abstracts}
\hypertarget{pmid_30046763}{A}nticoagulant therapy for venous thromboembolism (VTE) in children is largely based on treatment recommendations for adults. However, differences in both physiology (ie, renal maturation and drug excretion) and developmental hemostasis must be considered when treating children, as such differences could affect dose appropriateness, safety and efficacy. To address these concerns, a study was designed to evaluate the safety of dabigatran etexilate in children requiring secondary thrombus prevention in whom an initial VTE was associated with an identified risk factor that persisted after the acute VTE treatment period. We report herein the rationale and design of the study. This phase 3, open-label, single-arm, multicenter, multinational, prospective cohort study will be conducted in ≥100 children aged 0 to <18 years at \textasciitilde{}100 specialist sites worldwide. Children will be treated with dabigatran etexilate for 12 months, or for a shorter duration if their identified VTE risk factor resolves, as per current American College of Chest Physicians recommendations. A nomogram will be used to determine starting doses for each patient. The primary outcomes of the study will be VTE recurrence, bleeding events, overall mortality, and VTE-related mortality. Secondary outcomes will include occurrence of post-thrombotic syndrome, the pharmacokinetics of dabigatran, and the need for dose adjustments during treatment. Data on adverse events during the study will also be collected. This study will evaluate the safety of dabigatran etexilate for the secondary prevention of VTE in children, in addition to providing further data to guide pediatric dosing with dabigatran. [\hyperlink{Verelan}{PMID: 30046763}, Matteo Luciani et al., 2018]

\hypertarget{pmid_31805182}{T}his open-label, single-arm, prospective cohort trial is the first phase 3 safety study to describe outcomes in children treated with dabigatran etexilate for secondary venous thromboembolism (VTE) prevention. Eligible children aged 12 to <18 years (age stratum 1), 2 to <12 years (stratum 2), and >3 months to <2 years (stratum 3) had an objectively confirmed diagnosis of VTE treated with standard of care (SOC) for ≥3 months, or had completed dabigatran or SOC treatment in the DIVERSITY trial (NCT01895777) and had an unresolved clinical thrombosis risk factor requiring further anticoagulation. Children received dabigatran for up to 12 months, or less if the identified VTE clinical risk factor resolved. Primary end points included VTE recurrence, bleeding events, and mortality at 6 and 12 months. Overall, 203 children received dabigatran, with median exposure being 36.3 weeks (range, 0-57 weeks); 171 of 203 (84.2\%) and 32 of 203 (15.8\%) took capsules and pellets, respectively. Overall, 2 of 203 children (1.0\%) experienced on-treatment VTE recurrence, and 3 of 203 (1.5\%) experienced major bleeding events, with 2 (1.0\%) reporting clinically relevant nonmajor bleeding events, and 37 (18.2\%) minor bleeding events. There were no on-treatment deaths. On-treatment postthrombotic syndrome was reported for 2 of 162 children (1.2\%) who had deep vein thrombosis or central-line thrombosis as their most recent VTE. Pharmacokinetic/pharmacodynamic relationships of dabigatran were similar to those in adult VTE patients. In summary, dabigatran showed a favorable safety profile for secondary VTE prevention in children aged from >3 months to <18 years with persistent VTE risk factor(s). This trial was registered at www.clinicaltrials.gov as \#NCT02197416. [\hyperlink{Verelan}{PMID: 31805182}, Leonardo R Brandão et al., 2020]

\hypertarget{pmid_21404143}{V}enous thrombosis and pulmonary embolism rarely occur in children but are associated with significant morbidity and mortality. Venous thromboembolism (VTE) mostly affects children with severe underlying conditions and multiple risk factors. Newborns and adolescents are at the highest risk. Standard and low molecular weight heparins and vitamin K antagonists are routinely used for the prevention and treatment of VTE. The new anticoagulants, both parenteral such as argatroban, bivalirudin and fondaparinux and oral such as dabigatran and rivaroxaban, have favourable pharmacological properties, all are approved for clinical use in adults and are currently being investigated in children. Argatroban is the only new anticoagulant licensed for use in children so far. The role of these new anticoagulants as alternative anticoagulants for children remains to be defined. This review focuses on the characteristics of VTE in children and reviews current knowledge on the use of the new thrombin and factor Xa inhibitors in this population. [\hyperlink{Verelan}{PMID: 21404143}, Werner Streif et al., 2011]

\hypertarget{pmid_36495716}{P}aediatric clinical practice for treatment of venous thromboembolism (VTE) is based on extrapolation from adult trials with minimal data on anticoagulation efficacy and safety in children. Based on EINSTEIN-Jr clinical trial data, rivaroxaban was approved to treat VTE and prevent its recurrence in children of all ages. To report the safety and efficacy of rivaroxaban use in paediatric VTE and to present real-world data, specifically about very young children. We conducted a retrospective observational study at Birmingham Children's Hospital. Data were collected from patients <16 years old who received rivaroxaban after its licensure in the period between March 2021 and June 2022. Rivaroxaban was used for treatment of acute VTE in 64 patients. Thrombosis was CVC-related in 26 patients, unprovoked in 3, while the rest had one or more risk factors for VTE. Safety and efficacy of rivaroxaban were assessed in 52 patients after excluding patients who were on current rivaroxaban treatment and those who were lost to follow up or stopped rivaroxaban due to intolerance. No bleeding events were reported, and recurrence of thrombosis occurred in only 3.6 \%. About 35 \% had normalised re-imaging, 40.3 \% improved, 9.6 \% were unchanged and 11.5 \% stopped rivaroxaban without re-imaging. Rivaroxaban was used for secondary VTE prophylaxis in 6 patients in our cohort with no recurrence of thrombosis or bleeding reports. Our real-world experience confirmed that rivaroxaban was well tolerated, effective and safe. Further real-world data and observational studies are essential to investigate the use of rivaroxaban among different risk groups. [\hyperlink{Verelan}{PMID: 36495716}, Eman Hassan et al., 2023]

\hypertarget{pmid_12706638}{C}entral venous lines (CVLs) are major risk factors for venous thromboembolism (VTE) in children. The objective of PROTEKT was to determine if a low molecular weight heparin (reviparin-sodium) safely prevents CVL-related VTE. This multi-center, open-label study, with blinded central outcome adjudication, randomized patients with new CVLs to twice-daily reviparin-sodium or standard care. The efficacy outcome was based on an exit venogram at Day 30 (+14 days), or earlier in case of CVL removal, or confirmed symptomatic VTE. The safety outcomes were major bleeding and death. Due to slow and restricted patient accrual, PROTEKT was closed prematurely. With reviparin-sodium, 14.1\% (11:78) of patients had VTE compared to 12.5\% (10:80) of control patients (odds ratio=1.15; 95\% confidence interval 0.42, 3.23); 2P=0.82). One patient had a major bleed and there were two deaths, all three events occurring in the standard care group. The use of reviparin-sodium for short-term prophylaxis of CVL-related VTE in children was safe but its efficacy remains unclear. Although underpowered, PROTEKT provided valuable information on event rates and predictors of CVL-related VTE. [\hyperlink{Verelan}{PMID: 12706638}, Patricia Massicotte et al., 2003]

\hypertarget{pmid_29202215}{V}enous thromboembolism (VTE) is more frequent in infants than in older children. Treatment guidelines in children are adapted from adult VTE data, but do not currently include direct oral anticoagulant use. Dabigatran etexilate (DE) use in the paediatric population with VTE therefore requires verification. We investigated the pharmacokinetic/pharmacodynamic (PK/PD) relationship, safety and tolerability of DE oral liquid formulation (OLF) in infants with VTE (aged < 12 months) who had completed standard anticoagulant treatment in an open-label, phase IIa study. Patients received a single-dose of DE OLF (based on an age- and weight-adjusted nomogram) yielding an exposure comparable to 150 mg in adults. The PK end point was plasma concentration of total dabigatran; PD end points included activated partial thromboplastin time (aPTT), ecarin clotting time (ECT) and diluted thrombin time (dTT). Safety end points included incidence of all bleeding and other adverse events (AEs). Ten patients were screened; eight entered the study (age range, 41-169 days). The geometric mean (gMean) total dabigatran plasma concentrations 2 hours post-dose (around peak concentrations) were 120 ng/mL with a geometric coefficient of variation (gCV) of 62.1\%. The gMean at 12 hours post-dosing was 60.4 ng/mL (gCV 30\%). PK/PD relationship was linear for ECT and dTT ( [\hyperlink{Verelan}{PMID: 29202215}, Jacqueline M L Halton et al., 2017] The safety of a novel 0.5\% ivermectin lotion (IVL) and potential for ivermectin absorption after application was investigated in an open-label study in young children, and a human repeat insult patch test (HRIPT) and cumulative irritation test (CIT) assessed any potential for cumulative dermal irritation and contact sensitization. In the pharmacokinetic and safety study, 30 head louse-infested children ages 6 months to 3 years received a 10-minute application of IVL on day 1. Blood was collected before application; 0.5, 1, and 6 hours after rinsing; and on days 2 and 8. Samples from 20 subjects were assayed for ivermectin (test sensitivity 0.05 ng/mL). Liver panel and complete blood counts were completed for all subjects. For the HRIPT/CIT, occlusive patches containing IVL or vehicle control lotion (CL) were repeatedly applied to 220 healthy adult subjects to assess contact sensitization; for cumulative dermal irritation testing, additional patches with normal saline and sodium dodecyl sulfate (SDS) were applied to 36 subjects. In the open-label study, all detected ivermectin plasma concentrations were <1 ng/mL. No safety signals emerged, and treatment was well tolerated. In the HRIPT/CIT, IVL was significantly less irritating than normal saline and SDS, with no evidence of dermal irritation or sensitization in human skin. IVL was safe when applied topically, absorption was de minimus, there was no evidence of irritation or sensitization from repeated exposures, and results support the safety of topical IVL use in children as young as 6 months. [\hyperlink{Verelan}{PMID: 29202215}, Lydie Hazan et al., ]

\hypertarget{pmid_27011634}{T}o report the effectiveness and safety of intravenous (IV) levetiracetam (LEV) in the treatment of critically ill children with acute repetitive seizures and status epilepticus (SE) in a children's hospital. We retrospectively analyzed data from children treated with IV LEV. The mean age of the 108 children was 69.39 ± 46.14 months (1-192 months). There were 58 (53.1\%) males and 50 (46.8\%) females. LEV load dose was 28.33 ± 4.60 mg/kg/dose (10-40 mg/kg). Out of these 108 patients, LEV terminated seizures in 79 (73.1\%). No serious adverse effects were observed but agitation and aggression were developed in two patients, and mild erythematous rash and urticaria developed in one patient. Antiepileptic treatment of critically ill children with IV LEV seems to be effective and safe. Further study is needed to elucidate the role of IV LEV in critically ill children. [\hyperlink{Verelan}{PMID: 27011634}, Faruk Incecik et al., ]

\hypertarget{pmid_20027345}{P}rophylactic efficiency and safety of anaferon (pediatric formulation) in children aging 1?month to 4 years, including sickly children, was proven. The use of the preparation in children reduced the incidence of acute respiratory infections, alleviated the course of the disease, and decreased the incidence of detection of viral antigens in nasal meatuses. [\hyperlink{Verelan}{PMID: 20027345}, E S Erman et al., 2009]

\hypertarget{pmid_36610740}{T}he aim of this article is to provide an overview of the current literature for direct-acting oral anticoagulant (DOAC) use in pediatric patients and summarize ongoing trials. In treatment of venous thromboembolism (VTE) in pediatric patients, evidence supports use of both dabigatran and rivaroxaban. Dabigatran has been shown to be noninferior to standard of care (SOC) in terms of efficacy, with similar bleeding rates. Similarly, treatment with rivaroxaban in children with acute VTE resulted in a low recurrence risk and reduced thrombotic burden, without increased risk of bleeding, compared to SOC. Treatment of pediatric cerebral venous thrombosis as well as central venous catheter-related VTE with rivaroxaban appeared to be both safe and efficacious and similar to that with SOC. Dabigatran also has a favorable safety profile for prevention of VTE, and rivaroxaban has a favorable safety profile for VTE prevention in children with congenital heart disease. Many studies with several different DOACs are ongoing to evaluate both safety and efficacy in unique patient populations, as well as VTE prevention. The literature regarding pediatric VTE treatment and prophylaxis is growing, but the need for evidence-based pediatric guidelines remains. Additional long-term, postauthorization studies are warranted to further elucidate safety and efficacy in clinical scenarios excluded in clinical trials. Additional data on safety, efficacy, and dosing strategies for reversal agents are also necessary, especially as the use of DOACs becomes more common in the pediatric population. [\hyperlink{Verelan}{PMID: 36610740}, Kimberly Mills et al., 2023]

\hypertarget{pmid_32935597}{A}nticoagulant therapy is in use for both prevention and treatment of venous and arterial thromboembolic disorders. Delivering safe and effective anticoagulation in the pediatric population is challenging, since the available standard therapy with parenteral UFH and LMWH is troublesome for most pediatric patients, and VKAs require frequent INR monitoring due to the unpredictable pharmacokinetics and numerous food and drug interactions. Rivaroxaban, a direct FXa inhibitor, offers the convenience of oral administration and predictable pharmacokinetics across a wide range of patients. Its safety and efficacy have been previously established in various adult indications. This review outlines pharmacologic and clinical aspects regarding rivaroxaban treatment in adults and children, and provides a broad appraisal of the The EINSTEIN-Jr program which evaluated the safety and efficacy of body-weight adjusted pediatric rivaroxaban regimens for the treatment of VTE in children. A review of the literature using the keywords rivaroxaban and pediatric venous thromboembolism was conducted within the National Center for Biotechnology (NCBI) and EMBASE databases. Rivaroxaban represents an appealing therapeutic alternative for VTE in children. Further research should explore additional indications for rivaroxaban in the pediatric population beyond that of VTE. [\hyperlink{Verelan}{PMID: 32935597}, Omri Cohen et al., 2020]

\hypertarget{pmid_12915341}{T}reatment of seizures in pediatric patients is complicated by the fact that the etiology of the disorder and the pharmacokinetics, efficacy, and safety of antiepileptic drugs (AEDs) may differ from that in adults. With few controlled clinical trials of AEDs in children, the selection of agents to treat pediatric patients must be made on the basis of information from small uncontrolled studies or the extrapolation of clinical trial results in adults. Data from a large number of children with a wide range of seizure disorders who were treated in small-scale prospective studies, or whose records were retrospectively evaluated, indicate that levetiracetam reduces the frequency of seizures in pediatric patients. Available data also indicate that levetiracetam is well tolerated in pediatric patients, with a safety profile similar to that in adults, a low potential for behavioral disturbances, and no reported idiosyncratic adverse reactions. As with other AEDs, children metabolize and clear levetiracetam more rapidly than adults, and somewhat higher doses (based on body weight) are needed to achieve desired plasma concentrations. Several ongoing studies will provide further information on the pharmacokinetics, efficacy, and safety of levetiracetam in this patient population. [\hyperlink{Verelan}{PMID: 12915341}, Tracy A Glauser et al., 2003]

\hypertarget{pmid_31859250}{C}hildren are physiologically protected against venous thromboembolism (VTE). Specific triggering events or contributing factors have been identified in the majority of reported cases, which differs from the adult pathology where 50\% of the thromboses are considered "idiopathic". This is a rare disease in children with an estimated frequency of less than 1/1000. The risk is highest in neonates, then decreases and increases again around 13 years to reach the same level as adults at 16 years. The risk of VTE is clearly higher in certain situations: significant trauma, prolonged immobilization, central venous catheter, stay in intensive care unit, inherited thrombophilia, cancer, obesity, oral contraceptives, etc. Thromboprophylaxis should not be used systematically, even in adolescents. Proper hydration and early mobilization form the basis of mechanical thromboprophylaxis. A prescription is only given after careful analysis of the child's risk factors and the orthopedic context. Thrombotic risk assessment scores - which are based on expert opinion and large VTE registers but have not been evaluated in clinical studies - are currently the most reliable method to evaluate the thrombotic risk in children and to prescribe thromboprophylaxis. Low-molecular weight heparin are the most commonly used thromboprophylaxis agents in children, with good tolerance and efficacy. [\hyperlink{Verelan}{PMID: 31859250}, Thierry Odent et al., 2020]

\hypertarget{pmid_7308338}{T}en asthmatic children received regular daily therapy with terbutaline aerosol for 50 weeks. No evidence was found for adverse effects of this drug on growth, bone marrow, liver function or the cardiovascular system. All children had improved respiratory function throughout the year. In acute experiments carried out in 12 symptom-free asthmatic children with 0.5 mg terbutaline, it was demonstrated that the improvement in respiratory function, i.e. increase in FEV1, MMEF25-75\%, FVC and PEF, was quick in onset, was maintained for at least 60 min and was not accompanied by effects on pulse rate. Thus, the bronchodilator aerosol, terbutaline, can be safely used as a regular daily therapy in asthmatic children aged 7 to 14 years. [\hyperlink{Verelan}{PMID: 7308338}, M I Blackhall et al., 1981]

\hypertarget{pmid_37885787}{T}he use of antihistamine therapy in children for the management of upper respiratory tract infections remains a topic of debate. In this study, we focused on evaluating the effectiveness of promethazine (Phenergan), a first-generation H1 receptor antagonist and sedative, in addressing preoperative and intra-operative sequelae in cleft surgeries. A single-centered, parallel, randomized, double-blinded controlled clinical trial was conducted on 128 children aged 2 to 4 years undergoing cleft palate surgery under general anesthesia. The case group received Phenergan syrup orally twice a day for three days, while the control group received a placebo. Primary outcomes measured preoperative anxiety levels using a children's fear scale, while secondary outcomes assessed preoperative sleep quality and cough rate through objective scales. Intraoperative heart rate was monitored using an ECG connected to a monitor. The results demonstrated that the administration of promethazine resulted in a 34\% reduction in anxiety levels, a 46\% reduction in cold and cough, a 38\% improvement in sleep score, and stable heart rates throughout the surgery compared to the control group. Based on these findings, promethazine is considered a safe premedication option for children undergoing cleft palate surgeries; given its benefits outweigh its adverse effects. [\hyperlink{Verelan}{PMID: 37885787}, Vedha Vivigdha A et al., 2023]

\hypertarget{pmid_28346967}{V}enous thromboembolism (VTE) is very uncommon in children and adolescents compared with older adults, though its incidence has significantly increased over the past two decades. Given the rarity of the condition, the data on pediatric VTE lag behind the adult experience and consequently the management of VTE in children is, in large part, modeled on the adult strategies. This approach has certain limitations, given that young children have developmental particularities of the hemostatic system and differences in the pharmacokinetics and pharmacodynamics of various anticoagulant agents. The most commonly used anticoagulants in children continue to be the heparins and the vitamin K antagonists. Direct intravenous thrombin inhibitors, argatroban, bivalirudin, have very limited pediatric use. The non-vitamin K antagonist oral anticoagulant drugs (novel oral anticoagulants) present potential advantages in terms of efficacy, safety, and convenience, though pediatric data are limited to preclinical and small phase I trials. There are several ongoing phase I, II, and III trials for dabigatran rivaroxaban, apixaban, and edoxaban, the results of which are likely to change the future management of pediatric thromboses. [\hyperlink{Verelan}{PMID: 28346967}, Vlad Calin Radulescu et al., 2017]

\hypertarget{pmid_18376287}{L}ow-molecular-weight heparins are increasingly used for treatment of pediatric venous thromboembolic disease (VTE). Pediatric data about therapeutic doses of nadroparin are not available. To evaluate pharmacodynamics and safety of therapeutic doses of nadroparin, consecutive patients (age 0 to 18 y) with objectively diagnosed VTE and treated with nadroparin were included in this single center study over a 12-year period. All patients started with 85.5 IU/kg of nadroparin twice daily. The target therapeutic range (TTR) was set at 0.5 to 1.0 anti-Xa IU/mL 4 hours postdose. Safety end points were major bleeding and therapy-related death. A total of 84 patients were enrolled, of whom 8 patients did not undergo measurement of anti-Xa levels. Fifty-four (71\%) of 76 patients achieved TTR. The maintenance dose (mean+/-SE) was 448+/-42 IU/kg/d in neonates (<2 mo, n=6), 253+/-22 IU/kg/d in infants (2 mo to 1 y, n=10), 214+/-8 IU/kg/d in children (2 to 11 y, n=13), and 183+/-5 IU/kg/d in adolescents (12 to 18 y, n=25). Neonates required significantly more dose adjustments and time to achieve TTR than adolescents. No major bleeding or therapy-related death occurred. In summary, an age-dependent response to nadroparin exists in pediatric patients. Nadroparin therapy seems to be safe for treatment of pediatric VTE. [\hyperlink{Verelan}{PMID: 18376287}, C Heleen van Ommen et al., 2008]

\hypertarget{pmid_22105561}{I}n contrast to drugs established to treat neonatal seizures, levetiracetam shows little neurotoxicity in experimental animal models and has good safety records in adults and children. Here, we present results from a survey on the off-label use of levetiracetam in newborn infants among neonatologists and pediatric neurologists in German university hospitals. [\hyperlink{Verelan}{PMID: 22105561}, A Koppelstäetter et al., 2011]

\hypertarget{pmid_32189338}{T}o evaluate the efficacy and safety of pregabalin as adjunctive treatment for children (aged 1 month-<4 years) with focal onset seizures (FOS) using video-electroencephalography (V-EEG). This randomized, placebo-controlled, international study included V-EEG seizure monitoring (48-72 hours) at baseline and over the last 3 days of 14-day (5-day dose escalation; 9-day fixed dose) double-blind pregabalin treatment (7 or 14 mg/kg/d in three divided doses). This was followed by a double-blind 1-week taper. The primary efficacy endpoint was log-transformed seizure rate (log Overall, 175 patients were randomized (mean age = 28.2 months; 59\% male, 69\% white, 30\% Asian) in a 2:1:2 ratio to pregabalin 7 or 14 mg/kg/d (n = 71 or n = 34, respectively), or placebo (n = 70). Pregabalin 14 mg/kg/d (n = 28) resulted in a statistically significant 35\% reduction of log Pregabalin 14 mg/kg/d (but not 7 mg/kg/d) significantly reduced seizure rate in children with FOS, when assessed using V-EEG, compared with placebo. Both pregabalin dosages were generally safe and well tolerated in children 1 month to <4 years of age with FOS. Safety and tolerability were consistent with the known profile of pregabalin in older children with epilepsy. [\hyperlink{Verelan}{PMID: 32189338}, Donald Mann et al., 2020]

\hypertarget{pmid_18496412}{T}o review the physiology and the published literature on the role of vasopressin in shock in children. We searched MEDLINE (1966-2007), EMBASE (1980-2007), and the Cochrane Library, using the terms vasopressin, terlipressin, and shock and synonyms or related terms for relevant studies in pediatrics. We searched the online ISRCTN-Current Controlled Trials registry for ongoing trials. We reviewed the reference lists of all identified studies and reviews as well as personal files to identify other published studies. Beneficial effects have been reported in vasodilatory shock and asystolic cardiac arrest in adults. Solid evidence for vasopressin use in children is scant. Observational studies have reported an improvement in blood pressure and rapid weaning off catecholamines during administration of low-dose vasopressin. Dosing in children is extrapolated from adult studies. Vasopressin offers promise in shock and cardiac arrest in children. However, in view of the limited experience with vasopressin, it should be used with caution. Results of a double-blind, randomized controlled trial in children with vasodilatory shock will be available soon. [\hyperlink{Verelan}{PMID: 18496412}, Karen Choong et al., 2008]

\hypertarget{pmid_28553374}{T}he aim of this study is to evaluate the efficacy and safety of levetiracetam (LEV) as first-line treatment of neonatal seizures. This study was conducted in patients of Neonatal Intensive Care Unit of Santo Bambino Hospital, University of Catania, Italy, from January to August 2016. A total of 16 neonates with convulsions not associated with major syndromes, which required anticonvulsant therapy, were included and underwent IV LEV at standard doses. All patients responded to treatment, with a variety range of seizure resolution period (from 24 h to 15 days; mean hours: 96 ± 110.95). No patient required a second anticonvulsant therapy. Regarding safety of LEV, no major side-effects were observed. To our knowledge, it is one of the few studies confirming the efficiency of LEV as first-line treatment in seizures of this age group. LEV was effective in resolving seizures and was safely administered in the current study. [\hyperlink{Verelan}{PMID: 28553374}, Raffaele Falsaperla et al., ]

\hypertarget{pmid_28130888}{M}ore than 50\% of children report apian during venepuncture or intravenous cannulation and using local anaesthetics before needle procedures can lead to different success rates. This study examined how many needle procedures were successful at the first attempt when children received either a warm lidocaine and tetracaine patch or an eutectic mixture of lidocaine and prilocaine (EMLA) cream. We conducted this multicentre randomised controlled trial at three tertiary-level children's hospitals in Italy in 2015. Children aged three to 10 years were enrolled in an emergency department, paediatric day hospital and paediatric ward and randomly allocated to receive a warm lidocaine and tetracaine patch or EMLA cream. The primary outcome was the success rate at the first attempt. The analysis included 172 children who received a warm lidocaine and tetracaine patch and 167 who received an EMLA cream. The needle procedure was successful at the first attempt in 158 children (92.4\%) who received the warm patch and in 142 children (85.0\%) who received the cream (p = 0.03). The pain scores were similar in both groups. This study showed that the first-time needle procedure success was 7.4\% higher in children receiving a warm lidocaine and tetracaine patch than EMLA cream. [\hyperlink{Verelan}{PMID: 28130888}, Giorgio Cozzi et al., 2017]

\hypertarget{pmid_27357738}{V}enous thromboembolism (VTE) incidence is increasing among children owing to many factors, including improved diagnosis of VTE. There is a need for alternative treatment options. Our objective was to investigate the safety, pharmacokinetics (PK) and pharmacodynamics (PD) of dabigatran etexilate in adolescents with VTE. Adolescents aged 12 to <18 years (n = 9) who successfully completed planned treatment for primary VTE were administered dabigatran etexilate twice daily for three days; initially 1.71 (± 10 \%) mg/kg (80 \% of a 150 mg/70 kg twice daily adult dose), followed by 2.14 (± 10 \%) mg/kg (target adult dose adjusted for patient's weight), if there were no safety concerns. No bleeding events, deaths or drug-related serious adverse events (AEs) were reported; three treatment-emergent AEs, all gastrointestinal-related, occurred in two patients. In these adolescent patients with normal renal function, presumed steady-state trough plasma concentrations of dabigatran were low (geometric mean dose-normalised total dabigatran plasma concentration: 0.493 ng/ml/mg at 72 hours). Total dabigatran concentrations were well predicted by the RE-LY® population PK model (94 \% of trough concentrations were within the 80 \% prediction interval). The relationship between total dabigatran plasma concentration, diluted thrombin time and ecarin clotting time (ECT) was linear; the relationship with activated partial thromboplastin time (aPTT) was non-linear. Adult population PK/PD models predicted the adolescent concentration-ECT and -aPTT relationships well. In conclusion, dabigatran etexilate was generally well tolerated, except for occurrence of dyspepsia in two patients, over the three-day treatment period. The dabigatran PK/PD relationship observed in adolescent patients was similar to that in adult patients. [\hyperlink{Verelan}{PMID: 27357738}, Jacqueline M L Halton et al., 2016]

\hypertarget{pmid_9846923}{E}MLA cream 5\% (a eutectic mixture of lidocaine and prilocaine) is a topical anaesthetic that has become widely used to minimize pain from venipuncture in children. It has not, however, been recommended in neonates owing to the potential risk of methaemoglobinaemia induced by prilocaine. The aim of this study was to establish the safety of 1 g EMLA cream 5\% used on intact skin in term neonates. Forty-seven neonates, aged 0-3 months, with a postconceptual age of > or = 37 weeks and a body weight between 2.8 and 5.7 kg, were included in a double-blind, randomized, placebo-controlled study. After baseline observations a total dose of 1.0 g EMLA/placebo was applied to two sites (0.5 g site(-1)) for 60-70 min. Venous methaemoglobin (metHb) levels were determined in each patient at baseline and at three randomly assigned times, 0.5-18 h after application. Following application of the cream, the mean metHb levels were 1.17\% (range 0.50-2.53) in the EMLA group and 0.96\% (range 0.50-1.53) in the placebo group. The metHb concentrations were significantly higher in the EMLA group in the intervals from 3.5 to 13 h after application than in the placebo group, but were well below potentially harmful levels. Based on these results, a 1-h application of 1 g EMLA cream is safe when used on the intact skin of term neonates below 3 months of age. [\hyperlink{Verelan}{PMID: 9846923}, M Brisman et al., 1998]

\hypertarget{pmid_11552629}{T}he aim of the study was to research the efficiency of sertraline (zoloft) in depressions, anxious states and obsessive-compulsive disorders. Diagnosis of the mental disorders was carried out according to ICD-10. 72 children (59 boys, 13 girls) aged 6-18 years were treated. There were 32 inpatients and 40 outpatients. Therapy with sertraline was performed during 8 weeks with a gradual increase (titration) and individual selection of the dose from 12.5 to 100 mg/day. During the therapy clinical observation was combined with the patients' examination using Hamilton Depression Scale and Hamilton Anxiety Scale (HAM-D and HAM-A), and a Clinical Global Impression Scale (CGI). It was established that sertraline was very effective and safe drug in children (it has no influence on cognitive functions, has neither myorelaxing or sedative effects). Activity of this drug is characterized by quick manifestation of thymoanaleptic and anxiolytic effects. It mild depressive states 50 mg/day is a significant dose; in more severe depressions and obsessive-compulsive disorders the dose in juveniles was to 100 mg, the duration of the therapy was more than 2 months. [\hyperlink{Verelan}{PMID: 11552629}, V M Voloshina et al., 2001]

\section*{Bylvay}
\subsection*{Result}
\subsubsection*{Answer}

Unknown

\subsubsection*{{Explanation}}
\hypertarget{Bylvay}
A review of the available abstracts reveals one that specifically discusses Bylvay (odevixibat):

- \hyperlink{pmid_34499340}{PMID: 34499340}, Emma D Deeks et al., 2021: This abstract states that odevixibat (Bylvay™) is approved in the EU for the treatment of progressive familial intrahepatic cholestasis (PFIC) in patients aged ≥6 months, and in the USA for the treatment of pruritus in patients aged ≥3 months with PFIC. The abstract notes that odevixibat is in clinical development for other cholestatic diseases in various countries. However, the abstract does not provide details of any targeted safety studies in children, nor does it affirm the safety of Bylvay in children through specific study results. It only summarizes regulatory approvals and ongoing development.

No other abstracts mention Bylvay or odevixibat.

Based on the information in the abstracts, there is no evidence of a targeted safety study in children that affirms the safety of Bylvay. Therefore, the safety of Bylvay in children is unknown according to the available abstracts.

\subsection*{Abstracts}
\hypertarget{pmid_27746226}{T}he rates of influenza illness and associated complications are high among children in Bangladesh. We assessed the clinical efficacy and safety of a Russian-backbone live attenuated influenza vaccine (LAIV) at two field sites in Bangladesh. Between Feb 27 and April 9, 2013, children aged 2-4 years in urban Kamalapur and rural Matlab, Bangladesh, were randomly assigned in a 2:1 ratio, according to a computer-generated schedule, to receive one intranasal dose of LAIV or placebo. After vaccination, we monitored children in weekly home visits until Dec 31, 2013, with study clinic surveillance for influenza illness. The primary outcome was symptomatic, laboratory-confirmed influenza illness due to vaccine-matched strains. Analysis was per protocol. The trial is registered with ClinicalTrials.gov, number NCT01797029. Of 1761 children enrolled, 1174 received LAIV and 587 received placebo. Laboratory-confirmed influenza illness due to vaccine-matched strains was seen in 93 (15·8\%) children in the placebo group and 79 (6·7\%) in the LAIV group. Vaccine efficacy of LAIV for vaccine-matched strains was 57·5\% (95\% CI 43·6-68·0). The vaccine was well tolerated, and adverse events were balanced between the groups. The most frequent adverse events were tachypnoea (n=86 in the LAIV group and n=54 in the placebo group), cough (n=73 and n=43), and runny nose (n=68 and n=39), most of which were mild. This single-dose Russian-backbone LAIV was safe and efficacious at preventing symptomatic laboratory-confirmed influenza illness due to vaccine-matched strains. LAIV programmes might reduce the burden of influenza illness in Bangladesh. The Bill \& Melinda Gates Foundation. [\hyperlink{Bylvay}{PMID: 27746226}, W Abdullah Brooks et al., 2016]

\hypertarget{pmid_27592696}{I}nfluenza causes a substantial burden in young children. Vaccine efficacy (VE) data are limited in this age group. We examined trivalent influenza vaccine (TIV) efficacy and safety in young children attending childcare. A double-blind, randomised controlled trial in children aged 6 to <48 months was conducted with recruitment from Sydney childcare centres in 2011. Children were randomised to receive two doses of TIV or control hepatitis A vaccine. Efficacy was evaluated against polymerase chain reaction-confirmed influenza using parent-collected nose/throat swabs during influenza-like-illness. Safety outcomes were assessed during 6 months of follow-up. Fifty-seven children were allocated to influenza vaccine and 67 to control; all completed the study. The influenza attack rate was 1.8 vs 13.4\% in the TIV and control groups, respectively; VE 87\% (95\%CI: 0-98\%). For children aged 24 to <48 months, 0 vs 8 (18.6\%) influenza infections occurred in the TIV and control groups respectively, giving a VE of 100\% (16-100\%). Efficacy was not shown in children 6 to <24 months, probably due to insufficient power. Injection site and systemic adverse events were mostly mild to moderate with no significant differences, apart from more mild diarrhoea following dose 2 in TIV recipients (11.8 vs 0\%). Influenza vaccine appeared efficacious in the subgroup of children aged 24 to <48 months, although caution is required due to the small number of participants. There were no serious adverse events and most parents would vaccinate again. Influenza vaccination in a childcare setting could be valuable and a larger confirmatory study would be helpful. [\hyperlink{Bylvay}{PMID: 27592696}, Jean P Li-Kim-Moy et al., 2017]

\hypertarget{pmid_34499340}{O}devixibat (Bylvay™) is a small molecule inhibitor of the ileal bile acid transporter being developed by Albireo Pharma, Inc. for the treatment of various cholestatic diseases, including progressive familial intrahepatic cholestasis (PFIC). In July 2021, odevixibat received its first approval in the EU for the treatment of PFIC in patients aged ≥ 6 months, followed shortly by its approval in the USA for the treatment of pruritus in patients aged ≥ 3 months with PFIC. Odevixibat is also in clinical development for the treatment of other cholestatic diseases, including Alagille syndrome and biliary atresia, in various countries. This article summarizes the milestones in the development of odevixibat leading to this first approval for PFIC. [\hyperlink{Bylvay}{PMID: 34499340}, Emma D Deeks et al., 2021]

\hypertarget{pmid_17663736}{T}hrombosis is not uncommon in children with serious medical conditions. Unfractionated heparin, the most commonly used anticoagulant in the acute management of thrombosis, has significant pharmacologic limitations, especially in infants. Newer anticoagulants have improved properties relative to heparin, and this may enhance the outcome in children. To determine dosing, and to assess the safety and efficacy of bivairudin for infants with thrombosis. Infants <6 months old were chosen for this pilot study as they may most benefit from a direct thrombin inhibitor because of their physiologically low antithrombin levels. This was an open label, dose-finding and safety study. Patients received one of three bolus doses and one of two initial infusion doses with subsequent dosing adjusted utilizing the activated partial thromboplastin time. Safety was assessed by specific bleeding endpoints. Efficacy was determined by reassessing the initial imaging study at 48-72 h and by measurement of molecular markers of thrombin generation. Sixteen patients completed the study. All three bolus doses resulted in therapeutic anticoagulation, as did both initial infusion doses. A dose-response effect was noted for the continuous infusion but not the bolus dosing. Two patients met the study criteria for major bleeding, both with gross hematuria, which resolved with a reduction in the bivalirudin infusion rate. In terms of efficacy, 37.5\% of patients had complete or partial resolution of their thrombosis by 48-72 h. There was a significant decrease in all three molecular markers of thrombin generation. This study demonstrates the potential utility of bivalirudin in the pediatric population. [\hyperlink{Bylvay}{PMID: 17663736}, G Young et al., 2007]

\hypertarget{pmid_15014289}{I}ncreasing use of influenza vaccine in children is expected as this important virus becomes more widely recognized as a major cause of morbidity in young children. Clinicians and third party payers must consider the implications of national vaccine use recommendations, with their current focus on young children, on their practices and on the community at large. Two influenza vaccines are available in the United States, an inactivated, trivalent intramuscular formulation (TIV) which is approved for use among children > or =6 months of age; and a live, attenuated intranasal trivalent preparation (LAIV) indicated for healthy persons 5 to 49 years of age. This review summarizes available data regarding the safety and efficacy of TIV, in comparison with LAIV, with particular attention to children <9 years of age, the population for whom two doses of vaccine are recommended for first time vaccination. It is apparent that relatively few data are available on the safety of TIV in young children, that important age-specific differences in TIV vaccine efficacy exist and that LAIV appears similar to TIV with regard to safety and efficacy in younger children, but no head-to-head comparison of these two licensed products is available. [\hyperlink{Bylvay}{PMID: 15014289}, Kenneth M Zangwill et al., 2004]

\hypertarget{pmid_25917680}{L}ive-attenuated influenza vaccines (LAIVs) have the potential to be affordable, effective, and logistically feasible for immunization of children in low-resource settings. We conducted a phase II, randomized, double-blind, parallel group, placebo-controlled trial on the safety of the Russian-backbone, seasonal trivalent LAIV among children aged 24 through 59 months in Dhaka, Bangladesh in 2012. After vaccination, we monitored participants for six months with weekly home visits and study clinic surveillance for solicited and unsolicited adverse events, protocol-defined wheezing illness (PDWI), and serious adverse events (SAEs), including all cause hospitalizations. Three hundred children were randomized and administered LAIV (n=150) or placebo (n=150). No immediate post-vaccination reactions occurred in either group. Solicited reactions were similar between vaccine and placebo groups during the first 7 days post-vaccination and throughout the entire trial. There were no statistically significant differences in participants experiencing PDWI between LAIV and placebo groups throughout the trial (n=13 vs. n=16, p=0.697). Of 131 children with a history of medical treatment or hospitalization for asthma or wheezing at study entry, 65 received LAIV and 66 received placebo. Among this subset, there was no statistical difference in PDWI occurring throughout the trial between the LAIV or placebo groups (7.7\% vs. 19.7\%, p=0.074). While there were no related SAEs, LAIV recipients had six unrelated SAEs and placebo recipients had none. These SAEs included three due to traumatic injury and bone fracture, and one each due to accidental overdose of paracetamol, abdominal pain, and acute gastroenteritis. None of the participants with SAEs had laboratory-confirmed influenza, wheezing illness, or other signs of acute respiratory illness at the time of their events. In this randomized, controlled trial among 300 children aged 24 through 59 months in urban Bangladesh, Russian-backbone LAIV was safe and well tolerated. Further evaluation of LAIV safety and efficacy in a larger cohort is warranted. [\hyperlink{Bylvay}{PMID: 25917680}, Justin R Ortiz et al., 2015]

\hypertarget{pmid_37815398}{A}ntimicrobial resistance increases infection morbidity in both adults and children, necessitating the development of new therapeutic options. Telavancin, an antibiotic approved in the United States for certain bacterial infections in adults, has not been examined in pediatric patients. The objectives of this study were to evaluate the short-term safety and pharmacokinetics (PK) of a single intravenous infusion of telavancin in pediatric patients. Single-dose safety and PK of 10 mg/kg telavancin was investigated in pediatric subjects >12 months to ≤17 years of age with known or suspected bacterial infection. Plasma was collected up to 24-h post-infusion and analyzed for concentrations of telavancin and its metabolite for noncompartmental PK analysis. Safety was monitored by physical exams, vital signs, laboratory values, and adverse events following telavancin administration. Twenty-two subjects were enrolled: 14 subjects in Cohort 1 (12-17 years), 7 subjects in Cohort 2 (6-11 years), and 1 subject in Cohort 3 (2-5 years). A single dose of telavancin was well-tolerated in all pediatric age cohorts without clinically significant effects. All age groups exhibited increased clearance of telavancin and reduced exposure to telavancin compared to adults, with mean peak plasma concentrations of 58.3 µg/mL (Cohort 1), 60.1 µg/mL (Cohort 2), and 53.1 µg/mL (Cohort 3). A 10 mg/kg dose of telavancin was well tolerated in pediatric subjects. Telavancin exposure was lower in pediatric subjects compared to adult subjects. Further studies are needed to determine the dose required in phase 3 clinical trials in pediatrics. [\hyperlink{Bylvay}{PMID: 37815398}, John S Bradley et al., 2023]

\hypertarget{pmid_18267372}{C}hildren and adolescents comprise a significant proportion of the hemophilia population, including those patients who have developed inhibitors to factor VIII or FIX. We examine the use of rFVIIa for the treatment of bleeding episodes and the prevention of bleeding in children and adolescents with hemophilia A and B with inhibitors, focusing on registry data and recent clinical trial results. Based on this review of the literature, we conclude that recombinant FVIIa is safe and effective for use in controlling bleeding in these patient populations. [\hyperlink{Bylvay}{PMID: 18267372}, Brahm Goldstein et al., 2008]

\hypertarget{pmid_19731330}{E}vidence of the laboratory benefits of hydroxyurea and its clinical efficacy in reducing acute vaso-occlusive events in adults and children with sickle cell anemia has accumulated for more than 15 years. A definitive clinical trial showing that hydroxyurea can also prevent organ damage might support widespread use of the drug at an early age. BABY HUG is a randomized, double-blind placebo-controlled trial to test whether treating young children ages 9-17 months at entry with a liquid preparation of hydroxyurea (20 mg/kg/day for 2 years) can decrease organ damage in the kidneys and spleen by at least 50\%. Creation of BABY HUG entailed unique challenges and opportunities. Although protection of brain function might be considered a more compelling endpoint, preservation of spleen and renal function has clinical relevance, and significant treatment effects might be discernable within the mandated sample size of 200. Concerns about unanticipated severe toxicity and burdensome testing and monitoring requirements were addressed in part by an internal Feasibility and Safety Pilot Study, the successful completion of which was required prior to enrolling a larger number of children on the protocol. Concerns over recruitment of potentially vulnerable subjects were allayed by inclusion of a research subject advocate, or ombudsman. Finally, maintenance of blinding of research personnel was aided by inclusion of an unblinded primary endpoint person, charged with transmitting endpoint data and monitoring blood work locally for toxicity (ClinicalTrials.gov number, NCT00006400). [\hyperlink{Bylvay}{PMID: 19731330}, Bruce W Thompson et al., 2010]

\hypertarget{pmid_27232114}{T}he relative safety and efficacy of recombinant activated coagulation factor VII (rFVIIa, NovoSeven® RT) across pediatric age cohorts is poorly defined. The objective of this analysis was to assess the safety and efficacy of rFVIIa in pediatric patients with congenital hemophilia with inhibitors (CHwI) in the clinical studies supporting the U.S. labeling. Pediatric data were derived from seven studies (five acute and two perioperative treatments) and pooled. All data were stratified by age (<2, 2 to <6, 6 to <12, and 12-16 years) and study category (acute treatment of bleeding episodes or surgery). The pediatric dataset included 172 patients; 144 received rFVIIa for the treatment of bleeding episodes and 28 for the control of bleeding perioperatively. Recombinant FVIIa was effective for 95.4\% (1,026/1,076) of the evaluable bleeding episodes and had similar treatment effectiveness across pediatric age groups (range, 94.1-97.2\%). The majority received doses of 90 mcg/kg. rFVIIa was effective in achieving perioperative hemostasis across pediatric age groups (range, 91-100\%), with greater efficacy observed with the recommended (90 mcg/kg) versus lower dose (35 mcg/kg). A total of 88 pediatric patients experienced a total of 285 adverse drug reactions, similar in type to those reported among adult patients. A total of seven thrombotic events were recorded in seven pediatric patients; only one was confirmed related to rFVIIa upon individual case review. rFVIIa is safe and effective in the treatment of bleeding episodes and prevention of periprocedure bleeding in CHwI with no apparent differences observed among pediatric age groups. [\hyperlink{Bylvay}{PMID: 27232114}, Stacy E Croteau et al., 2016] miniSTONE-2 (NCT03629184) was a global, phase 3, randomized, controlled study that investigated the safety and efficacy of single-dose baloxavir marboxil in otherwise healthy children 1-<12 years of age and showed a positive risk-benefit profile. This post hoc analysis evaluated the safety and efficacy of baloxavir versus oseltamivir in children 5-11 years old with influenza. Children received single-dose baloxavir or twice-daily oseltamivir for 5 days. Safety was the primary objective. Efficacy and virological outcomes included time to alleviation of symptoms, duration of fever and time to cessation of viral shedding by titer. Data were summarized descriptively. Ninety-four children 5-11 years old were included (61 baloxavir and 33 oseltamivir). Baseline characteristics were similar between the groups. The incidence of adverse events was balanced and low in both treatment groups, with the most common being vomiting (baloxavir 5\% vs. oseltamivir 18\%), diarrhea (5\% vs. 0\%) and otitis media (0\% vs. 5\%). No serious adverse events or deaths occurred. Median (95\% CI) time to alleviation of symptoms with baloxavir was 138.4 hours (116.7-163.4) versus 126.1 hours (95.9-165.7) for oseltamivir; duration of fever was comparable between groups [41.2 hours (23.5-51.4) vs. 51.3 hours (30.7-56.8), respectively]. Median time to cessation of viral shedding was shorter in the baloxavir group versus oseltamivir (1 vs. ≈3 days). Safety, efficacy and virological results in children 5-11 years were similar to those from the overall study population 1-<12 years of age. Single-dose baloxavir provides an additional treatment option for pediatric patients 5-11 years old with influenza. [\hyperlink{Bylvay}{PMID: 27232114}, Jeffery B Baker et al., 2023]

\hypertarget{pmid_18445016}{E}xperience of recombinant activated factor VII (rFVIIa, NovoSeven; Novo Nordisk A/S, Bagsvaerd, Denmark) to control haemorrhage in non-haemophilic children is limited. The object of this study was to examine the applicability and safety of rFVIIa amongst a group of non-haemophilic paediatric subjects. Details of all non-haemophilic children < or =16 years receiving rFVIIa whose data were recorded in the investigational, internet-based registry, haemostasis.com were analysed. A total of 265 children (mean age 7.7 years) were treated with rFVIIa; the median dose administered was 78.4 microg kg(-1) body weight (range 9.0-393.4) and the median total dose received 100.0 microg kg(-1) body weight (range 10.9-1341.2). Therapeutic areas included surgery (34.5\%), coagulopathy (including thrombocytopenia; 29.0\%), spontaneous bleeding (17.2\%), trauma (8.4\%) and intracranial haemorrhage (4.5\%). Two patients experienced thromboembolic events following administration of rFVIIa. Thirty-nine patients died on account of haemorrhage or complications relating to their underlying condition; neither the thromboembolic events nor the deaths were related to rFVIIa administration. Bleeding stopped in 118/237 (49.8\%), markedly decreased in 54/237 (22.8\%), decreased in 51/237 (21.5\%), remained unchanged in 13/237 (5.5\%) and increased in 1/237 (0.4\%) patients. These results suggest that rFVIIa is safe and widely applicable in children to control non-haemophilic haemorrhage. [\hyperlink{Bylvay}{PMID: 18445016}, M Herbertson et al., 2008]

\hypertarget{pmid_32311184}{T}he use of a peripheral intravenous cannula is a common clinical practice, and it is known to be a major source of pain and anxiety in children. The aim of this study was to examine the effect of the use of the Buzzy The research sample consisted of 60 children between the ages of 8 and 16. For children in the experimental group, external cold and vibration were applied by means of the Buzzy The results of the statistical analysis showed no statistically significant difference between the anxiety levels of the groups before and after the procedure (P > 0.05). The results of the statistical analysis also showed no statistically significant difference between the postprocedural mean pain scores of the children as reported by the children themselves and by the observer (P > 0.05). In this study, unlike most studies in the literature, the conclusion was reached that the use of the Buzzy [\hyperlink{Bylvay}{PMID: 32311184}, Dilek Yılmaz et al., 2020] To evaluate long-term safety and tolerability of adjunctive brivaracetam (BRV) in children with epilepsy. This was an interim analysis (cut-off March 15, 2017) of pooled data from two open-label, single-arm, multicentre trials. N01263 (NCT00422422) was a 3-week trial of BRV 0.8-4 mg/kg/day in patients (1 month-<16 years) with epilepsy. Patients who completed this trial could continue into a long-term follow-up trial (N01266, NCT01364597) which also directly enrolled patients (4-<17 years) with focal seizures. After dose-escalation, patients received BRV 1-5 mg/kg/day (maximum 200 mg/day) during long-term evaluation. Data are reported for patients aged 4 to <16 years with focal seizures. The safety set comprised 149 patients: 34 from the initial trial (26 entered long-term trial) and 115 directly enrolled into the long-term trial. At the cut-off, 90 patients were receiving BRV (total exposure: 299.4 patient-years). Treatment-emergent adverse events (TEAEs) were reported by 140/149 (94.0\%) patients, most commonly (≥20\%) nasopharyngitis (24.8\%), pharyngitis (22.1\%), convulsion (21.5\%), and pyrexia (20.1\%). TEAEs considered drug-related by the investigator were reported by 56/149 (37.6\%) patients, most commonly somnolence (6.0\%). Two patients died; neither death was considered related to BRV. Mean changes from baseline in child behaviour rating scales were small; most patients remained in their baseline category. In this pooled analysis of two open-label trials including long-term data, adjunctive BRV was generally well tolerated in children aged 4 to <16 years with focal seizures. These findings supported approval of BRV as a new therapy option for children aged ≥4 years with focal seizures. [\hyperlink{Bylvay}{PMID: 32311184}, Anup D Patel et al., 2020]

\hypertarget{pmid_24389586}{W}e evaluated efficacy and safety of recombinant activated factor VII (rFVIIa) in nonhemophilia children with life-threatening or severe bleeding. Using data from the SeveNBleeP registry, we analyzed demographic, clinical, laboratory, and treatment data for children who received rFVIIa to treat severe hemorrhage. The SeveNBleeP registry was international registry formed in 2005, to collect information on the use of rFVIIa in the off-label setting of severe bleeding in nonhemophilia patients. There were 191 patient records entered into this registry, of which 164 were validated. Of the 164 records, in 137 patient records, rFVIIa was used for treatment of bleeding episodes. Of these 137 treatment episodes, 42 were in neonates and infants under 1 year of age. Use of rFVIIa significantly improved laboratory parameters (prothrombin time, international normalized ratio, activated partial thromboplastin time, hematocrit), reduced estimated blood loss, and reduced requirements for blood products (packed red blood cells and fresh frozen plasma) in those more than 1 year of age. There was no significant reduction in requirements for blood products after rFVIIa administration in the neonates and infants, but there was a trend to lower frequency of FFP use after rFVIIa administration. There was one thromboembolic event in an infant that was related to administration of rFVIIa. No other serious adverse events were reported that were related to administration of rFVIIa. In nonhemophilia-associated bleeding in children, rFVIIa appears to be safe and efficacious in reducing estimated blood loss in children over 1 year of age, although its effectiveness in infants below 1 year of age was less clear.  [\hyperlink{Bylvay}{PMID: 24389586}, Jan Blatný et al., 2014] Three large studies, all including more than 60 children, on the efficacy on gamma-vinyl GABA (vigabatrin, GVG) in the treatment of childhood epilepsy have been reviewed. On the basis of these and other studies, we conclude that (a) GVG is a promising drug for the treatment of childhood epilepsy, (b) GVG is effective both in partial and generalized epilepsies, (c) GVG seems to be effective in the treatment of symptomatic infantile spasms, (d) no clear dose-response relationship has been demonstrated, (e) a dose of 40-80 mg/kg/day is recommended, and (f) although tolerance is excellent, hyperactivity seems to occur more frequently in children than in adults. [\hyperlink{Bylvay}{PMID: 24389586}, L Gram et al., 1992]

\hypertarget{pmid_30390435}{T}his multicenter, retrospective study aimed to evaluate the efficiency, retention, safety, and tolerability of brivaracetam (BRV) in children and adolescents with focal epilepsy. All patients aged ≤17 years with focal epilepsy who started BRV in 2016 and 2017 were analyzed. Thirty-four patients (mean age: 12.2 years, range: 3-17 years, 56\% female) were treated with BRV for 25 days to 24 months, with a total exposure time of 19.7 years. Overnight switch from levetiracetam (LEV) to BRV was performed in 20 patients at a median ratio of 10:1. Retention rate was 97\% at three months, with only one patient reporting a discontinuation of BRV treatment. Further dropouts were reported in one patient after seven months and in two patients after one year of treatment, respectively. The median length of exposure to BRV was 180 days. Efficacy at three months was 47\% (50\% responder rate), with 10 patients (29\%) reporting seizure freedom. A long-term 50\% responder rate was present in 12 patients [35\%; four patients seizure-free (12\%)] for more than six months and in seven patients (21\%; no seizure-free patients) for more than 12 months. Treatment-emergent adverse events were observed in 12\% of patients, with the most common being sedation, somnolence, loss or gain of appetite, and psychobehavioral adverse events. Use of BRV in children and adolescents seems to be safe and well-tolerated. The results with 50\% responder rate of 47\% are consistent with those from randomized controlled trials and postmarketing studies in adults. An immediate switch from LEV to BRV at a ratio of 10:1 is feasible. The occurrence of psychobehavioral adverse events seems less prominent than under LEV and a switch to BRV can be considered in patients with LEV-induced adverse events. [\hyperlink{Bylvay}{PMID: 30390435}, Susanne Schubert-Bast et al., 2018]

\hypertarget{pmid_23847058}{M}ismatch between circulating influenza B viruses (Yamagata and Victoria lineages) and vaccine strains occurs frequently. In a randomized controlled trial, immunogenicity and safety of an inactivated quadrivalent influenza vaccine candidate (QIV) versus trivalent inactivated influenza vaccine (TIV)-Victoria(Vic) and TIV-Yamagata(Yam) in children 3-17 years of age was evaluated. In an open-label study arm, QIV only was assessed in children 6-35 months of age. A total of 3094 children (932 QIV, 929 TIV-Vic, 932 TIV-Yam, and 301 QIV only) were vaccinated. QIV was noninferior to the TIVs for shared strains (A/H3N2 and A/H1N1) based on hemagglutination-inhibition (HI) antibodies 28 days after last vaccination, and superior for the unique B strains Victoria and Yamagata (geometric mean titer ratios 2.61, 3.78; seroconversion rate differences 33.96\%, 44.63\%). Among children in the randomized trial, adverse event rates were similar except for injection site pain (dose 1: 65.4\% QIV, 54.6\% TIV-Vic, 55.7\% TIV-Yam). QIV elicited superior HI responses to the added B strains compared to TIV controls, potentially improving its effectiveness against influenza B. HI responses were similar between QIV and TIV controls for the shared strains. QIV had an acceptable safety profile relative to TIVs. NCT01198756. [\hyperlink{Bylvay}{PMID: 23847058}, Joanne M Langley et al., 2013]

\hypertarget{pmid_14872180}{T}o determine the safety of cold-adapted trivalent intranasal influenza virus vaccine (CAIV) in children and adolescents. A randomized, double blind, placebo-controlled safety trial in healthy children age 12 months to 17 years given CAIV (FluMist; MedImmune Vaccines, Inc.) or placebo (randomization, 2:1). Children <9 years of age received a second dose of CAIV or placebo 28 to 42 days after the first dose. Enrolled children were then followed for 42 days after each vaccination for all medically attended events. Prespecified outcomes included 4 prespecified diagnostic groups and 170 observed individual diagnostic categories. The relative risk and the 2-sided 90\% confidence interval were calculated for each diagnostic group and individual category by clinical setting, dose and age. More than 1500 relative risk analyses were performed. A total of 9689 evaluable children were enrolled in the study. Of the 4 prespecified diagnostic categories (acute respiratory tract events, systemic bacterial infection, acute gastrointestinal tract events and rare events potentially associated with wild-type influenza), none was associated with vaccine. Of the biologically plausible individual diagnostic categories, 3, acute gastrointestinal events, acute respiratory events and abdominal pain, had different analyses that demonstrated increased and decreased relative risks, making their association with the vaccine unlikely. For reactive airway disease a significant increased relative risk was observed in children 18 to 35 months of age with a relative risk of 4.06 (90\% confidence interval, 1.29 to 17.86) in this age group. The individual diagnostic categories of upper respiratory infection, musculoskeletal pain, otitis media with effusion and adenitis/adenopathy had at least one analysis that achieved a significant increased risk ratio. All of these events were infrequent. CAIV was generally safe in children and adolescents. The observation of an increased risk of asthma/reactive airway disease in children <36 months of age is of potential concern. Further studies are planned to evaluate the risk of asthma/reactive airway disease after vaccine. [\hyperlink{Bylvay}{PMID: 14872180}, Randy Bergen et al., 2004]

\hypertarget{pmid_32302941}{T}he primary objective of the study was to analyze the efficacy of brivaracetam (BRV) in pediatric patients 12 months after starting treatment. The secondary objective was to establish safety 3, 6, and 12 months after starting treatment. This was an observational and retrospective study. Data were collected from the electronic medical record. Inclusion criteria were as follows: patients under 18 years of age, diagnosis of focal or generalized epilepsy, treatment as an added therapy, initiation of treatment with BRV between June and September 2017, and at least one unprovoked seizure in the year prior to the start of treatment. Forty-six patients were included. The response rate was 65\%, including 30\% seizure-free patients. The rate of adverse effects was 43.5\%, resulting in withdrawal in 16 patients (34.7\%). The most common adverse effects were drowsiness (17.3\%) and irritability (17.3\%). Brivaracetam is effective in very diverse childhood epilepsies, including some that present with primarily generalized seizures. Given the characteristics of the population studied, we have not been able to confirm a better tolerability of BRV compared with levetiracetam (LEV). [\hyperlink{Bylvay}{PMID: 32302941}, Núria Visa-Reñé et al., 2020]

\hypertarget{pmid_19033492}{S}nakebite envenomations occur in 45,000 patients in the USA annually and are associated with morbidity especially in children and the elderly. Crotalidae polyvalent immune fab (ovine; FabAV) is a polyvalent antivenom derived from sheep for crotalid envenomations. Limited clinical trials are available in paediatric patients. A literature search using MEDLINE (1950-February 2008), International Pharmaceutical Abstracts (1970-February 2008), EMBASE (1988-February 2008) and Cochrane Library (1996-June 2008) was conducted using key words including: antivenom OR snakebites OR children OR Crotalid OR envenomations. All English-language articles were identified from data sources. Pertinent studies pertaining to FabAV in children and adolescents with crotalid envenomations were included for analysis. Ten papers were included for review, representing 47 children. Initial doses ranging from 2 to 18 g were administered and initial control was achieved in most children. Maintenance dosing was continued in 63.8\% (30/47) of patients; 4.3\% (2/47) of patients had episodes of venom recurrence. Adverse events were noted in 8.5\% of children (4/47) when pooled for data analysis. FabAV appears to be a safe and effective agent for children with crotalid envenomations. Clinicians should adapt dosing recommendations used for adults until future large, well-designed trials can confirm the efficacy and safety from observation studies and case reports. [\hyperlink{Bylvay}{PMID: 19033492}, P N Johnson et al., 2008]

\hypertarget{pmid_26258888}{I}nfluenza vaccine safety is an ongoing issue. In 2010, inactivated trivalent influenza vaccines (TIVs), Fluvax(®) and Fluvax Junior(®) manufactured by CSL Biotherapies ('CSL'), Parkville, Australia, were associated with a marked increase in febrile seizures (FS) in children <5 years old. Extensive investigations initially failed to identify a root cause. The company's researchers recently published two papers outlining their latest findings. Cytokine responses to TIV were measured in paediatric whole blood assays (WBA); NF-κB activation was assessed using a HEK293 cell line reporter assay. CSL suggest that the combination of new influenza strains (H1N1 A/California/7/2009 and B/Brisbane/60/2008), increased complexes of viral RNA and lipid in the vaccine, and inherent sensitivities of some children <5 years old caused elevated inflammatory responses resulting in FS. Whilst the papers provide insight into pathogenesis, much remains unclear. The WBA were from only 10 'healthy' children, potentially affecting generalisability of the results and reliability of these in vitro tests in assessing future influenza vaccine safety. Increased fever rates (without FS) found in CSL TIV studies between 2005 and 2010 suggest a long-standing contribution to reactogenicity from the manufacturing process. More detailed comparisons with non-CSL vaccines would have helped elucidate the relative contribution of patient/strain factors and the manufacturing process. The focus remains on manufacturing process differences as the key causative factor of elevated febrile responses. Studies underway, of modified vaccines in young children, will determine whether reactogenicity issues have been successfully addressed and whether CSL TIV can be relicensed in children <5 years of age.  [\hyperlink{Bylvay}{PMID: 26258888}, Jean Li-Kim-Moy et al., 2016] Spinal anaesthesia has been used in children for over 100 years and in the last two decades its popularity for newborns and infants has increased, but there are still unanswered questions with the technique. We evaluated the characteristics of spinal block including ease of performance, efficacy, adverse effects and complications in 1132 children, aged 6 months to 14 years, undergoing surgery in the lower part of the body. Local ethical committee approved the protocol of this prospective study, and parents gave written informed consent and older children their assent. All patients were sedated with midazolam, thiopental or propofol intravenously during spontaneous ventilation. No inhalation anaesthetics were used. Spinal block was performed with 0.5\% hyperbaric bupivacaine at a dose of 0.2 mg x kg(-1). Efficacy, safety and ease of performance of the spinal block were shown to be satisfactory in most children. Only 27 of the 1132 children needed any supplementation. The incidence and severity of complications was low. Only nine of 942 children, < 10 years of age and eight of 190 children, 10 years or older, developed hypotension. The incidences of postdural puncture headache, in five of the 1132 children, and backache, in nine of the 1132, were low. No other neurological complications were reported. Spinal anaesthesia with hyperbaric bupivacaine is a feasible anaesthetic method in children for surgery in the lower part of the body. [\hyperlink{Bylvay}{PMID: 26258888}, Franco Puncuh et al., 2004]

\hypertarget{pmid_21906647}{Y}oung children are at increased risk for influenza infections and related complications. The protection offered to children by conventional trivalent inactivated influenza vaccines (TIV) is suboptimal, due to poor immunogenicity and a higher exposure to infection and complications in this age group, particularly from influenza B strains. In this dose-ranging, factorial design trial, we report the safety and immunogenicity of different combinations of adjuvanted (ATIV) and non-adjuvanted trivalent (TIV) and quadrivalent (QIV) influenza vaccines in 480 healthy children 6 to <36 months of age. The results show that the second B strain added to TIV was immunogenic and did not affect immunogenicity of the other strains. The addition of the MF59(®) adjuvant promoted robust immune responses with notable elevations in antibodies observed even after one dose. A dose-response relationship was observed between the antibody response and MF59 adjuvant. No patterns in safety and tolerability emerged with different adjuvant and antigen doses nor with the addition of a second B strain. MF59-adjuvanted QIV offers potential advantages to young children. [\hyperlink{Bylvay}{PMID: 21906647}, Giovanni Della Cioppa et al., 2011]

\hypertarget{pmid_37849499}{T}his study presents the results of a real-life, multicenter, prospective, post-approval safety evaluation of Clairyg® 50 mg/mL, a 5\% intravenous immunoglobulin (IVIg) liquid, in 59 children (aged < 12 years) with primary immunodeficiency diseases (PID) ( The primary objective of the study was to assess the safety and tolerability of Clairyg®, recording all serious and non-serious adverse events (AEs), whether related (rAEs) or not related to the product. Secondary objectives aimed at evaluating the administration of Clairyg® under routine conditions and the available efficacy data to better document the benefit/risk ratio in this pediatric population. An exploratory objective was added to evaluate the potential factors associated with the occurrence of rAEs. Patients received Clairyg® according to the approved dosage under normal conditions of prescriptions over a median follow-up period of 11.8 months. A total of 549 infusions (PID:  Clairyg® was well tolerated and allowed for control of serious bacterial infection in PID and serious bleeding in ITP, which are the main complications in these respective pediatric disorders. No new safety signal was detected in children less than 12 years-old in real-life conditions of use. [\hyperlink{Bylvay}{PMID: 37849499}, Nizar Mahlaoui et al., 2023]

\section*{Acyclovir}
\subsection*{Result}
\subsubsection*{Answer}

Yes (Neonates and infants, 0–3 months)
Yes (Children, 3 months–18 years)

\subsubsection*{{Explanation}}
\hypertarget{Acyclovir}
Based on the available abstracts, there are multiple targeted studies evaluating the safety of acyclovir in children across various age ranges. Here is a summary by age group:

Neonates and Infants (0–3 months):
- A study of infants with neonatal herpes simplex virus disease treated with high-dose intravenous acyclovir (mostly 60 mg/kg/d) found that adverse events (AEs) were common but usually not severe, and many AEs may be related to the underlying infection rather than acyclovir exposure. No infants developed renal failure requiring dialysis. The study concludes that high-dose acyclovir is generally safe in this population, though AEs should be monitored [\hyperlink{pmid_27977557}{PMID: 27977557}, Jessica E Ericson et al., 2017].
- Another study of nine infants (including premature infants) treated with intravenous acyclovir for herpes simplex or cytomegalovirus infections found no renal toxicity and all infants survived. Hematologic values improved during therapy [\hyperlink{pmid_6285712}{PMID: 6285712}, A S Yeager et al., 1982].
- A pharmacokinetic study in children younger than 2 years (including infants <1 month) found that oral acyclovir was well tolerated, with dosage adjustment by body weight recommended [\hyperlink{pmid_11120958}{PMID: 11120958}, M Tod et al., 2001].
- A study of oral acyclovir in children aged 3–24 months (including infants) for varicella found an excellent clinical and laboratory safety profile [\hyperlink{pmid_7629563}{PMID: 7629563}, F Chiodo et al., 1995].

Children (3 months–18 years):
- Multiple randomized controlled trials and cohort studies in otherwise healthy children (ages 2–18 years) with varicella or herpes simplex infections found that oral and intravenous acyclovir was safe and well tolerated, with no significant increase in adverse effects compared to placebo [\hyperlink{pmid_1944438}{PMID: 1944438}, L M Dunkle et al., 1991; \hyperlink{pmid_2156984}{PMID: 2156984}, H H Balfour et al., 1990; \hyperlink{pmid_8245899}{PMID: 8245899}, S Feldman et al., 1993; \hyperlink{pmid_8545564}{PMID: 8545564}, S Catania et al.; \hyperlink{pmid_8151150}{PMID: 8151150}, H Kamiya et al., 1994; \hyperlink{pmid_1638653}{PMID: 1638653}, D Bartosová et al., 1992].
- A study of acyclovir for empiric treatment of encephalitis in children aged 1 month to 18 years found no significant difference in renal injury or failure between standard- and high-dose regimens, suggesting safety in this age group [\hyperlink{pmid_24497110}{PMID: 24497110}, Jennifer G Kendrick et al., 2014].
- A systematic review of randomized controlled trials in otherwise healthy children with chickenpox found no clinically important differences in adverse effects between acyclovir and placebo [\hyperlink{pmid_12356336}{PMID: 12356336}, Terry P Klassen et al., 2002].
- A study of 183 children aged 0–14 years treated with oral acyclovir for chickenpox found the drug to be safe and well tolerated [\hyperlink{pmid_8545564}{PMID: 8545564}, S Catania et al.].
- A study of 126 children (mean age 81 months) found that nephrotoxicity was associated with concomitant use of other nephrotoxic drugs or impaired baseline renal function, but not with acyclovir dose, age, or sex within recommended dosing [\hyperlink{pmid_18345723}{PMID: 18345723}, Ruth Schreiber et al., 2008].
- Several case reports and small studies note rare adverse effects such as neurotoxicity or nephrotoxicity, but these are not common and are often associated with high doses, dehydration, or pre-existing risk factors [\hyperlink{pmid_9695630}{PMID: 9695630}, N Nakamoto et al., 1998; \hyperlink{pmid_20307148}{PMID: 20307148}, Gürkan Genc et al., 2010].

Summary:
- For neonates and infants (including premature infants), targeted studies affirm that acyclovir is generally safe when used as indicated, with monitoring for adverse events.
- For children from 3 months to 18 years, multiple randomized controlled trials and cohort studies affirm the safety of acyclovir for both oral and intravenous use, with rare and generally manageable adverse effects.
- For children with pre-existing renal impairment or those receiving other nephrotoxic drugs, caution and monitoring are advised.

There is no evidence from the abstracts reviewed that acyclovir is unsafe in children when used appropriately. The safety profile is well established in targeted pediatric studies.

\subsection*{Abstracts}
\hypertarget{pmid_32916823}{A}cyclovir is an antiviral drug used for the treatment of  [\hyperlink{Acyclovir}{PMID: 32916823}, Haniza Hassan et al., 2020] Acyclovir is used to treat herpes simplex virus disease in infants. Treatment with high-dose acyclovir, 60 mg/kg/d, is recommended; however, the safety of this dosage has not been assessed in the past 15 years, and this dosage is not currently approved for infants by the US Food and Drug Administration. We included infants with neonatal herpes simplex virus disease treated with ≥14 days of intravenous acyclovir starting in the first 120 days of life admitted to 1 of 42 neonatal intensive care units managed by the Pediatrix Medical Group between 2002 and 2012. We determined the frequency and proportion of infants with clinical and laboratory adverse events (AEs) as well as the number and proportion of infant days with laboratory AEs occurring during acyclovir exposure. We identified 89 infants during the study period with 1658 days of acyclovir exposure. Almost all received high-dose acyclovir therapy (79/89, 89\%). The most common clinical AEs were hypotension and seizure, both occurring in 9\% of infants. Thrombocytopenia was the most common laboratory AE occurring in 25\% of infants and on 9\% of infant-days. Elevated creatinine occurred in 2\% of infants and 0.2\% of infant-days and no infants developed renal failure requiring dialysis. Overall, 45\% of infants had ≥1 AE. In this cohort of infants treated during the high-dose acyclovir era, AEs were common but usually not severe. Many of the AEs reported in this cohort may be related to the underlying infection rather than due to acyclovir exposure. [\hyperlink{Acyclovir}{PMID: 32916823}, Jessica E Ericson et al., 2017]

\hypertarget{pmid_11120958}{A}cyclovir is approved for the treatment of herpes simplex virus (HSV) and varicella-zoster virus (VZV) infections in children by the intravenous and oral routes. However, its use by the oral route in children younger than 2 years of age is limited due to a lack of pharmacokinetic data. The objectives of the present study were to determine the typical pharmacokinetics of an oral suspension of acyclovir given to children younger than 2 years of age and the interindividual variabilities in the values of the pharmacokinetic parameters in order to support the proposed dosing regimen (24 mg/kg of body weight three times a day for patients younger than 1 month of age or four times a day otherwise). Children younger than age 2 years with HSV or VZV infections were enrolled in a multicenter study. Children were treated for at least 5 days with an acyclovir oral suspension. Plasma samples were obtained at steady state, before acyclovir administration, and at 2, 3, 5, and 8 h after acyclovir administration. Acyclovir concentrations were measured by radioimmunoassay. The data were analyzed by a population approach. Data for 79 children were considered in the pharmacokinetic study (212 samples, 1 to 5 samples per patient). Acyclovir clearance was related to the estimated glomerular filtration rate, body surface area, and serum creatinine level. The volume of distribution was related to body weight. The elimination half-life decreased sharply during the first month after birth, from 10 to 15 h to 2.5 h. Bioavailability was 0.12. The interindividual variability was less pronounced when the parameters were normalized with respect to body weight. Hence, dosage adjustment by body weight is recommended for this population. Simulations showed that the length of time that acyclovir remains above the 50\% inhibitory concentration during a 24-h period was more than 12 h for HSV but not for VZV. The proposed dosing regimen seems adequate for the treatment of HSV infections, while for the treatment of VZV infections, a twofold increase in the dose seems necessary for children older than age 3 months. [\hyperlink{Acyclovir}{PMID: 11120958}, M Tod et al., 2001]

\hypertarget{pmid_32988829}{A}cyclovir is an antiviral currently used for the prevention and treatment of herpes simplex virus (HSV) and varicella-zoster virus (VZV) infections. This study aimed to characterize the pharmacokinetics (PK) of acyclovir and its oral prodrug valacyclovir to optimize dosing in children. Children receiving acyclovir or valacyclovir were included in this study. PK were described using nonlinear mixed-effect modeling. Dosing simulations were used to obtain trough concentrations above a 50\% inhibitory concentration for HSV or VZV (0.56 mg/liter and 1.125 mg/liter, respectively) and maximal peak concentrations below 25 mg/liter. A total of 79 children (212 concentration-time observations) were included: 50 were taking intravenous (i.v.) acyclovir, 22 were taking oral acyclovir, and 7 were taking both i.v. and oral acyclovir, 57 for preventive and 22 for curative purposes. A one-compartment model with first-order elimination best described the data. An allometric model was used to describe body weight effect, and the estimated glomerular filtration rate (eGFR) was significantly associated with acyclovir elimination. To obtain target maximal and trough concentrations, the more suitable initial acyclovir i.v. dose was 10 mg/kg of body weight/6 h for children with normal renal function (eGFR ≤ 250 ml/min/1.73 m [\hyperlink{Acyclovir}{PMID: 32988829}, S Abdalla et al., 2020] Intravenous acyclovir is the treatment of choice for herpes simplex virus encephalitis. In 2006, the American Academy of Pediatrics updated its dosing recommendations for children aged 3 months to 12 years to receive high-dose acyclovir (60 mg/kg/day). The association between acyclovir dose and toxicity is unclear. The purpose of our study was to review our institution's experience with standard- and high-dose acyclovir for the empiric treatment of encephalitis. This retrospective cohort study included patients aged 1 month to 18 years who received acyclovir as empiric treatment for encephalitis between 2005 and 2009 at a tertiary care children's hospital. We excluded patients with baseline renal impairment and those without serum creatinine measurements prior to and during treatment. The main outcome measure of this study was to compare the occurrence of renal injury or failure between children who received the standard- versus high-dose regimen. Sixty-one patients were included (n = 32 standard-dose; n = 29 high-dose). There was no statistical difference in change in serum creatinine from baseline between children who received standard- versus high-dose acyclovir (0 vs. 5.1 \%; p = 0.79). One child in the standard-dose group and three children in the high-dose group developed renal injury or failure during treatment (3.1 vs. 10.3 \%; p = 0.34). Children with renal injury or failure were older, had a longer length of stay, and longer duration of therapy than children without. The incidence of renal injury or failure was similar between children who received standard-dose and high-dose acyclovir. [\hyperlink{Acyclovir}{PMID: 32988829}, Jennifer G Kendrick et al., 2014]

\hypertarget{pmid_8245899}{A}cyclovir has been approved in the United States and elsewhere as antiviral therapy for otherwise healthy children and adolescents with varicella. This development arose from multicentre placebo-controlled trials of acyclovir in normal patients, 2-18 years of age, which showed that the drug accelerated cutaneous healing, and reduced fever and related constitutional symptoms without harmful side effects. Acyclovir did not, however, decrease transmission of chickenpox within the household, nor was there any demonstrable effect of antiviral therapy on varicella complications. In this article, the background and rationale for the multicentre studies of acyclovir in normal paediatric patients with chickenpox is reviewed. The evidence for and against its routine administration within 24 hours of the eruption of skin rash is also discussed. [\hyperlink{Acyclovir}{PMID: 8245899}, S Feldman et al., 1993]

\hypertarget{pmid_8545564}{W}e evaluated safety and tolerance of acyclovir ACV per os in immunocompetent children affected by chicken-pox admitted to our department from January 1993 to December 1994. 183 subjects (102 males and 81 females) aged between 0 and 14 years were treated by ACV (80 mg/kg/daily in 4 divided doses): 88 children were treated within 24 hours and 95 subjects within 48 hours from the onset of symptoms. The control group consisted of 83 children (52 males and 31 females) aged between 0 to 14 years. In all patients routine blood-test were performed and in those with respiratory illness Chest-Rx was also done. We evaluated clinical course, degree of eruption, the appearance and kind of complications, duration of hospitalization, the compliance and the potential consequences on specific antibody response. Our results show a faster improvement of clinical symptoms in treated patients with respect to the control group with shortening of the period of the fever, itch and appearance of new vescicles. The percentage of complications was lower in treated than in untreated patients. 16 cases tested for specific antibody response showed protective titers six months after treatment. In conclusion, ACV administered per os within 48 hours from onset of exanthema causes reduction of the period and the degree of general symptoms and exanthema, a lower incidence of complications even if non statistically significant. The drug is safe and well-tolerated. [\hyperlink{Acyclovir}{PMID: 8545564}, S Catania et al., ]

\hypertarget{pmid_12353186}{A}n extensive clinical trial program combined with 5 years' postmarketing experience with valacyclovir provides evidence of favorable safety and efficacy in herpes simplex virus (HSV) management. Valacyclovir enhances acyclovir bioavailability compared with orally administered acyclovir. Long-term use of acyclovir for up to 10 years for HSV suppression is effective and well tolerated. Acyclovir is also approved for use in children, is available in some countries over the counter in cream formulation for herpes labialis, and has been monitored in over 1000 pregnancies. Safety monitoring data from clinical trials of valacyclovir, involving over 3000 immunocompetent and immunocompromised persons receiving long-term therapy for HSV suppression, were analyzed. Safety profiles of valacyclovir (</=1000 mg/day), acyclovir (800 mg/day), and placebo were similar. Extensive sensitivity monitoring of HSV isolates confirmed a very low rate of acyclovir resistance among immunocompetent subjects (<0.5\%). The incidence of resistance among immunocompromised patients remains low at about 5\%. [\hyperlink{Acyclovir}{PMID: 12353186}, Stephen K Tyring et al., 2002]

\hypertarget{pmid_12356336}{A}cyclovir has the potential to shorten the course of chickenpox which may result in reduced costs and morbidity. We conducted a systematic review of randomised controlled trials that evaluated acyclovir for the treatment of chickenpox in otherwise healthy children. MEDLINE, EMBASE, and the Cochrane Library were searched. The reference lists of relevant articles were examined and primary authors and Glaxo Wellcome were contacted to identify additional trials. Two reviewers independently screened studies for inclusion, assessed study quality using the Jadad scale and allocation concealment, and extracted data. Continuous data were converted to a weighted mean difference (WMD). Overall estimates were not calculated due to differences in the age groups studied. Three studies were included. Methodological quality was 3 (n = 2) and 4 (n = 1) on the Jadad scale. Acyclovir was associated with a significant reduction in the number of days with fever, from -1.0 (95\% CI -1.5,-0.5) to -1.3 (95\% CI -2.0,-0.6). Results were inconsistent with respect to the number of days to no new lesions, the maximum number of lesions and relief of pruritus. There were no clinically important differences between acyclovir and placebo with respect to complications or adverse effects. Acyclovir appears to be effective in reducing the number of days with fever among otherwise healthy children with chickenpox. The results were inconsistent with respect to the number of days to no new lesions, the maximum number of lesions and the relief of itchiness. The clinical importance of acyclovir treatment in otherwise healthy children remains controversial. [\hyperlink{Acyclovir}{PMID: 12356336}, Terry P Klassen et al., 2002]

\hypertarget{pmid_9695630}{W}e reported a 5-year-old boy with acute encephalitis due to suspected herpes simplex infection, who developed confusion, agitation and insomnia during intravenous administration of acyclovir. He recovered from these neuro-psychiatric symptoms two days after the cessation of acyclovir. The same symptoms recurred two days after its re-administration and resolved on the next day of the second cessation of the drug. Electroencephalogram (EEG) showed periodic lateralized epileptiform discharges (PLEDs) on hospital day 16, which disappeared on hospital day 27, suggesting that neurotoxicity of acyclovir may induce PLEDs. Although acyclovir is useful for the treatment of herpes simplex and varicella-zoster virus infections, we have to pay attention to its neurotoxicity. [\hyperlink{Acyclovir}{PMID: 9695630}, N Nakamoto et al., 1998]

\hypertarget{pmid_2829714}{E}ighteen children from 3 weeks to 6.9 years of age were given an oral acyclovir suspension for herpes simplex or varicella-zoster virus infections. Thirteen patients who were 6 months to 6.9 years old received 600 mg/m2 per dose, and three infants and two children less than 2 years old were given 300 mg/m2 per dose. The drug was given four times a day, except to one infant who was treated with three doses a day. Among the 13 children who received the 600-mg/m2 dose, the maximum concentration in plasma (Cmax) was 0.99 +/- 0.38 microgram/ml (mean +/- standard deviation), the time to maximum concentration (Tmax) was 3.0 +/- 0.86 h, the area under the curve (AUC) was 5.56 +/- 2.17 micrograms.h/ml, and the elimination half-life (t1/2) was 2.59 +/- 0.78 h. The three infants less than 2 months of age who received the 300-mg/m2 dose had a Cmax of 1.88 +/- 1.11 micrograms/ml, a Tmax of 4.10 +/- 0.48 h, an AUC of 6.54 +/- 4.32 micrograms.h/ml, and a t1/2 of 3.26 +/- 0.33 h. The acyclovir suspension was well tolerated by young children. No adverse effects requiring discontinuation of the drug occurred. [\hyperlink{Acyclovir}{PMID: 2829714}, W M Sullender et al., 1987]

\hypertarget{pmid_8151150}{T}he efficacy and safety of aciclovir granules (containing 40\% w/w aciclovir) were evaluated in the treatment of chickenpox in otherwise healthy children. Patients presenting with chickenpox received aciclovir granules at a dose of 20 mg/kg four times daily for five to seven days. Overall 51 children received treatment with aciclovir. A further 53 patients receiving conventional symptomatic therapy acted as a control. In the aciclovir group the overall efficacy rate was 92.2\%. There were reductions in the numbers of lesions, fever, itching and the duration of symptoms. No adverse experiences were reported. Overall this formulation of aciclovir appears to be a safe and effective treatment for chickenpox in this patient population. However the need for anti-viral therapy in otherwise healthy children is still the subject of debate and it might be appropriate to identify sub-groups for whom such therapy is justified. [\hyperlink{Acyclovir}{PMID: 8151150}, H Kamiya et al., 1994]

\hypertarget{pmid_18345723}{A}ciclovir is the drug of choice for severe systemic herpes virus infections. Nephrotoxicity is one of the clinically significant adverse effects of this drug, but studies examining nephrotoxicity in children are scarce. To identify risk factors for aciclovir-associated nephrotoxicity in the pediatric population. A retrospective review was conducted on all children (mean age 81 months; n = 126 [74 boys]) who were treated with aciclovir in a tertiary center between July 2005 and January 2006 and who met our inclusion criteria. Glomerular filtration rate (GFR) was calculated on the first day of treatment and at the peak measured creatinine level while on therapy, using Schwartz's method. Aciclovir therapy was associated with a significant increase in serum creatinine levels and a parallel decrease in GFR (n = 93; both p <or= 0.0001). Children with immunosuppression who received a variety of other nephrotoxic drugs exhibited more severe nephrotoxicity than those not receiving nephrotoxic drugs. In multiple regression analysis, the use of nephrotoxic drugs (p = 0.02) and impaired GFR at baseline (p = 0.04) were predictive for nephrotoxicity. Within the recommended age-dependent dosage schedule of aciclovir there was no effect of dose per kg, age, or sex on nephrotoxicity. The predictors of aciclovir nephrotoxicity were the concomitant use of nephrotoxic drugs and impaired GFR at baseline. [\hyperlink{Acyclovir}{PMID: 18345723}, Ruth Schreiber et al., 2008]

\hypertarget{pmid_17935955}{A}cyclovir is a synthetic nucleoside analogue active against herpes viruses. Exposure during human pregnancy and during the neonatal period seems safe. We report a case of early necrotizing enterocolitis in a full term infant treated with acyclovir as a prophylactic therapy. The mother had herpes genitalis with preterm, premature ruture of membranes at 32 weeks of gestational age and was treated with acyclovir until vaginal delivery. Acyclovir treatment in utero and after birth is discussed as a possible cause of necrotizing enterocolitis in the infant. Acyclovir should be used only if its benefit outweighs the potential risk to the baby. [\hyperlink{Acyclovir}{PMID: 17935955}, N Montjaux-Régis et al., 2007]

\hypertarget{pmid_20307148}{A}cyclovir is an effective, frequently used antiviral agent. Adverse effects of this drug are well known and are especially seen with high doses and/or dehydration. In this article, we report a 6-year-old boy with leukemia with nonoliguric acute renal failure in normal hydration status after using acyclovir treatment. He had no preexisting renal impairment, and there were no additional symptoms. Dimercaptosuccinic acid radionucleid scyntigraphy and other laboratory findings revealed impairment of proximal tubule function, in addition to distal tubule. We emphasize that renal functions should be monitored carefully during treatment with acyclovir, and asymptomatic nephrotoxicity must be kept in mind. [\hyperlink{Acyclovir}{PMID: 20307148}, Gürkan Genc et al., 2010]

\hypertarget{pmid_20014952}{V}alacyclovir provides enhanced acyclovir bioavailability in adults, but limited data are available in children. Children 1 month through 5 years of age with or at risk for herpesvirus infection received a single 25 mg/kg dose of extemporaneously compounded valacyclovir oral suspension (n = 57), whereas children 1 through 11 years of age received 10 mg/kg valacyclovir oral suspension twice daily for 3-5 days (herpes simplex virus infection) (n = 28) or 20 mg/kg 3 times daily for 5 days (varicella-zoster virus infection) (n = 27). Blood samples for pharmacokinetic analysis were collected during the 6 h after the first dose. Safety was monitored throughout the studies. Dose proportionality in the maximum observed concentration (C(max)) of acyclovir and the area under the concentration-time curve from time zero extrapolated to infinity (AUC(0-infinity)) existed across the 10 to 20 mg/kg valacyclovir dose range. For children 2 through 5 years of age, an increase in dose from 20 to 25 mg/kg resulted in near doubling of the C(max) and AUC(0-infinity). Among infants 1 through 2 months of age receiving 25 mg/kg, the mean AUC(0-infinity) and C(max) were higher ( approximately 60\% and 30\%, respectively) than those among older infants and children receiving the same dose. Valacyclovir oral suspension was well tolerated. No clinically significant trends were noted in clinical chemical, hematologic, or urinalysis values from screening to follow-up. Among children 3 months through 11 years of age, the 20 mg/kg dose of this formulation of valacyclovir oral suspension produces favorable acyclovir blood concentrations and is well tolerated. A dosing recommendation cannot be made for infants <3 months of age because of decreased clearance in this age group. Trial registration. ClinicalTrials.gov identifier: NCT00297206 . [\hyperlink{Acyclovir}{PMID: 20014952}, David W Kimberlin et al., 2010]

\hypertarget{pmid_1944438}{C}hickenpox, the primary infection caused by the varicella-zoster virus, affects more than 3 million children a year in the United States. Although usually self-limited, chickenpox can cause prolonged discomfort and is associated with infrequent but serious complications. To evaluate the effectiveness of acyclovir for the treatment of chickenpox, we conducted a multicenter, double-blind, placebo-controlled study involving 815 healthy children 2 to 12 years old who contracted chickenpox. Treatment with acyclovir was begun within the first 24 hours of rash and was administered by the oral route in a dose of 20 mg per kilogram of body weight four times daily for five days. The children treated with acyclovir had fewer varicella lesions than those given placebo (mean number, 294 vs 347; P less than 0.001), and a smaller proportion of them had more than 500 lesions (21 percent, as compared with 38 percent with placebo; P less than 0.001). In over 95 percent of the recipients of acyclovir no new lesions formed after day 3, whereas new lesions were forming in 20 percent of the placebo recipients on day 6 or later. The recipients of acyclovir also had accelerated progression to the crusted and healed stages, less itching, and fewer residual lesions after 28 days. In the children treated with acyclovir the duration of fever and constitutional symptoms was limited to three to four days, whereas in 20 percent of the children given placebo illness lasted more than four days. There was no significant difference between groups in the distribution of 11 disease complications (10 bacterial skin infections and 1 case of transient cerebellar ataxia). Acyclovir was well tolerated, and there was no significant difference between groups in the titers of antibodies against varicella-zoster virus. Acyclovir is a safe treatment that reduces the duration and severity of chickenpox in normal children when therapy is initiated during the first 24 hours of rash. Whether treatment with acyclovir can reduce the rare, serious complications of chickenpox remains uncertain. [\hyperlink{Acyclovir}{PMID: 1944438}, L M Dunkle et al., 1991]

\hypertarget{pmid_9412400}{A}moxyclav (amoxycillin/potassium clavulanate, A/PC) was used in the treatment of 55 children with acute bronchitis and pneumonia. The drug was administered in a dose of 20-40 mg/kg body weight a day in 3 portions. The treatment course was 4 to 10 days. The treatment was performed under careful clinicoroent-genologic control. The clinical picture of the disease in the children was characterized by a moderate process which made it possible to treat the children as outpatients. The clinical efficacy amounted to 90.5 per cent. The isolates of Streptococcus pneumoniae, Streptococcus pyogenes, Staphylococcus aureus and Haemophilus influenzae proved to be susceptible to A/PC. It may be used as the 1st class agent in the treatment of children with lower respiratory tract infection. [\hyperlink{Acyclovir}{PMID: 9412400}, B M Blokhin et al., 1997]

\hypertarget{pmid_7629563}{A}n open multicenter study has been carried out to evaluate efficacy and tolerability of oral acyclovir in the treatment of varicella in immunocompetent patients in the first two years of life. Fifty-three children aged 3-24 months received acyclovir at 80 mg/Kg/day in four divided doses for 4 to 6 days; 24 of them were treated in the first 24 hours following disease onset, while the remaining 29 patients were enrolled within 48 hours. The assessment of evolution of disease signs and symptoms showed a rapid resolution of fever, itching and other constitutional symptoms, with interruption of vesicle formation and acceleration of cutaneous healing processes. No statistically significant differences have been demonstrated as to disease progression between patients treated in the first 24 hours, when compared with subjects receiving acyclovir in the following 24 hours. Acyclovir confirmed its excellent clinical and laboratory safety profile. By acting favorably on both the duration and severity of disease signs and symptoms, acyclovir treatment should be recommended in young children and infants with varicella, since a higher incidence of severe and complicated disease has been observed in these patient groups. [\hyperlink{Acyclovir}{PMID: 7629563}, F Chiodo et al., 1995]

\hypertarget{pmid_1638653}{T}he authors submit information on the course and therapeutic experience with acyclovir (Zovirax Wellcome Co. and Herpesin Lachema Co.) in 67 children with eczema herpeticatum (EH) who were hospitalized at the Clinic of Infectious Child Diseases in Brno from January 1983 to January 1991. In all instances treatment led to rapid drying of the herpetis eruptions, a shorter period of new eruption and rapid improvement of the serious clinical condition. In none of the children visceral dissemination of the virus of herpes simplex (HSV) were occurred and in none of the children toxic side-effects were found. The authors confirmed the assumed identical course of EH after i. v. administration of acyclovir of foreign or local origin. After i.v. administration frequently dramatic improvement of the general and local finding was recorded, as compared with oral administration. There were no therapeutic differences in the clinical effects of tablets and suspension, the clinical effect being comparable. [\hyperlink{Acyclovir}{PMID: 1638653}, D Bartosová et al., 1992]

\hypertarget{pmid_1311067}{O}ral acyclovir was given prophylactically to 37 children in the early stages of three outbreaks of herpes simplex virus type 1 (HSV-1) infection and the results were compared with those in untreated control subjects in two other outbreaks. The rates of seroconversion to HSV were significantly reduced in children treated with acyclovir compared with control subjects (91\% vs 27\%, P less than .001). The incidence of symptomatic disease was also significantly reduced (82\% vs 0\%, P less than .001). In some children receiving prophylactic acyclovir, anti-HSV antibody titers did not rise despite the presence of replicative HSV on throat swabs just before the start of treatment. Restriction endonuclease analysis of isolated HSV-DNA confirmed that one strain was responsible for the five outbreaks. No resistance to acyclovir was detected during the study, and no adverse effects of treatment were noted. In conclusion, short-term prophylactic acyclovir may limit the spread and reduce clinical manifestations of HSV infections in closed communities, although this use should be restricted to communities where severe symptoms are observed. [\hyperlink{Acyclovir}{PMID: 1311067}, K Kuzushima et al., 1992]

\hypertarget{pmid_6750068}{A} randomized double-blind, placebo-controlled, multicenter investigation assessed the usefulness of acyclovir in the treatment of immunosuppressed children with chickenpox. Twelve patients received placebo and eight received acyclovir. If the event of clinical deterioration, patients could be removed from the study to receive acyclovir. Eighteen patients had skin lesions within 96 hours of admission to the study. Nineteen patients had malignancies. The two groups of patients were similar in age, in concomitant or preceding immunosuppressive therapy, in status of malignancy, and in presenting granulocyte and lymphocyte counts. Zoster immune globulin or plasma had been given to 50\% of the placebo group but to only 25\% of the acyclovir group. One patient in each group had pneumonitis at entry. Of the patients without pneumonitis at entry, five of the 11 placebo patients compared with none of the seven acyclovir patients developed pneumonitis during treatment (P = 0.054). No evidence of toxicity related to acyclovir was observed. [\hyperlink{Acyclovir}{PMID: 6750068}, C G Prober et al., 1982]

\hypertarget{pmid_2156984}{T}o determine whether acyclovir administered orally affects the duration and severity of varicella in otherwise normal children. Randomized, placebo-controlled, double-blind trial. Patients' residence and university hospital clinic. One hundred five children between 5 and 16 years of age with laboratory-confirmed varicella entered the study. Of the 102 who were included in the final analysis, 50 received acyclovir and 52 received placebo. Placebo or acyclovir was given orally four times daily, for 5 to 7 days. The acyclovir dose was adjusted as follows: 5 to 7 years of age, 20 mg/kg; 7 to 12 years, 15 mg/kg; and 12 to 16 years, 10 mg/kg. Acyclovir recipients, compared with the placebo group, defervesced sooner (median, 1 day vs 2 days; p = 0.001), experienced onset of cutaneous healing sooner, as reflected by a decrease in number of lesions (median, 3 days vs 2 days; p = 0.002), and had fewer skin lesions (median, 500 vs 336; p = 0.02). Acyclovir did not significantly change the rate of complications of varicella (10\% in the acyclovir group vs 13.5\% among placebo subjects). Adverse drug effects were not observed. Acyclovir recipients had lower geometric mean serum antibody titers to varicella-zoster virus than their placebo counterparts 4 weeks after the onset of illness, but antibody titers in both groups were similar 1 year later. These results provide evidence that acyclovir is useful and well tolerated for treatment of varicella in otherwise healthy children. [\hyperlink{Acyclovir}{PMID: 2156984}, H H Balfour et al., 1990]

\hypertarget{pmid_6285712}{N}ine infants with symptomatic infections caused by herpes simplex virus or cytomegalovirus were treated with acyclovir. At the onset of therapy, the infants ranged in weight from 880 to 4550 gm. Five were premature. Acyclovir was administered intravenously in a dosage of 5 to 15 mg/kg every eight hours for five to 10 days. The peak serum acyclovir levels ranged from 20 to 163 microM and the trough levels ranged from 1 to 129 microM. The variation in peak serum acyclovir levels in different infants receiving the same dosage on a weight basis was large but correlated with the expected renal maturity of the individual infant. Hematologic values improved during therapy. No renal toxicity was noted. All of the infants survived, including the five with herpes simplex infections. [\hyperlink{Acyclovir}{PMID: 6285712}, A S Yeager et al., 1982]

\hypertarget{pmid_3318610}{I}n the 5 years since its release for clinical use, acyclovir (9-[2-hydroxyethoxymethyl]guanine) has proved to be a safe and effective agent for therapy of herpes simplex and varicella-zoster infections. The drug's availability in topical, oral, and intravenous preparations has allowed its use in a range of clinical situations. Acyclovir must be phosphorylated by viral thymidine kinase in infected cells, where it then acts to inhibit viral DNA replication specifically. Epstein-Barr virus and human cytomegalovirus infections do not seem to respond to acyclovir therapy, although in-vitro effects on these viruses may be seen. Acyclovir is well absorbed and distributed, with cerebrospinal fluid levels 50\% that of plasma. Clearance is almost entirely by the renal route, with a half-life of 20 hours in the anuric patient. Acyclovir has an excellent safety profile, its major adverse effect being transient serum creatinine elevations during high-dose intravenous use. Major uses include treatment of primary and recurrent genital herpes and herpes encephalitis and prophyllaxis and therapy of mucocutaneous herpes and varicella-zoster infections in immunocompromised patients. Resistance to acyclovir in herpes simplex virus is rarely encountered and does not seem to be due to long-term chronic suppressive therapy. [\hyperlink{Acyclovir}{PMID: 3318610}, D I Dorsky et al., 1987]

\section*{Altace}
\subsection*{Result}
\subsubsection*{Answer}
Unknown
\subsubsection*{{Explanation}}
\hypertarget{Altace}
A review of the available abstracts reveals that none of them are about Altace (ramipril) or its safety in children. There are no targeted studies in the provided abstracts that evaluate the safety of Altace in pediatric populations, nor are there any studies that affirm or refute its safety in specific age ranges of children. Therefore, based on the abstracts available, the safety of Altace for use in children is unknown.

\subsection*{Abstracts}
\hypertarget{pmid_3998379}{A}lcide, a liquid sterilizer, was evaluated for teratogenic potential in rats and mice. Sodium chlorite and lactic acid, the active ingredients of this compound, form chlorine dioxide when mixed. Pregnant rats and mice were administered 1 ml and 0.1 ml, respectively, of Alcide liquid by gavage on days 6-15 of gestation. The general health of the dams was evaluated and the fetuses examined for external, visceral and skeletal malformations. There was no evidence of maternal toxicity among treated rats and mice. Fetal viability, weight, length and number of resorptions were comparable with control groups. Teratogenic toxicity was not detected in either species. There was some incidence of skeletal and visceral anomalies; however, these variances were not significantly different from control animals. [\hyperlink{Altace}{PMID: 3998379}, G A Skowronski et al., 1985]

\hypertarget{pmid_11445811}{W}e evaluated the efficacy and safety of alteplase to restore central venous line (CVL) patency in a consecutive cohort study. A uniform, weight-dependent protocol for alteplase administration was established prospectively. For children < or =10 kg, a dose of 0.5 mg was used; for children >10 kg, doses of 1 to 2 mg were used. The alteplase remained instilled for 1 to 4 hours or overnight. Retrospective data accrual found that 25 children received alteplase for a total of 34 courses; 29 (85\%) of the 34 courses of alteplase completely restored CVL patency. Alteplase appears to be a safe and effective thrombolytic agent for CVL patency restoration in children. [\hyperlink{Altace}{PMID: 11445811}, M Choi et al., 2001]

\hypertarget{pmid_36719881}{U}rsodeoxycholic acid (UDCA) is the main therapeutic drug for cholestasis, but its use in children is controversial. We conducted this study to evaluate the efficacy and safety of ursodeoxycholic acid in children with cholestasis. We searched Medline (Ovid), Embase (Ovid), Cochrane Central Register of Controlled Trials (CENTRAL), CNKI, WanFang Data and VIP from the establishment of databases to July 2022. Eligible studies included Chinese or English randomized controlled trials (RCTs) comparing the efficacy and safety of no UDCA (placebo or blank control) and UDCA in children with cholestasis. This study had been registered with PROSPERO (CRD42022354052). A total of 32 RCTs proved eligible, which included 2153 patients. The results of meta-analysis showed that UDCA could improve symptoms of children with cholestasis (risk ratio 1.24, 95\% CI 1.18 to 1.29; moderate quality of evidence), and serum levels of alanine aminotransferase, total bilirubin, direct bilirubin and total bile acid (low quality of evidence). For some children with specific cholestasis, UDCA could also effectively drop serum levels of aspartate aminotransferase (parenteral nutrition-associated cholestasis) and γ-glutamyl transferase (infantile hepatitis syndrome, parenteral nutrition-associated cholestasis). The most common adverse drug reactions (ADRs) of UDCA in children were gastrointestinal adverse reactions, with an incidence of 10.63\% (67/630). There was no significant difference in the incidence of ADRs between UDCA and placebo/blank control groups (risk difference 0.03, 95\%CI -0.01 to 0.06; moderate quality of evidence), and among children taking different UDCA doses (P = 0.27). The available short-term evidence showed that UDCA was effective and safe for children with cholestasis. Clinicians should use UDCA with caution (start with a low dose) until the long-term effect is further explored in future larger RCTs. [\hyperlink{Altace}{PMID: 36719881}, Liang Huang et al., 2023]

\hypertarget{pmid_22293541}{C}linical studies were conducted to investigate the pharmacokinetics of roxatidine acetate hydrochloride capsules (ALTAT(®) CAPSULES) in children. In a single-dose pharmacokinetic (PK) study in pediatric patients aged between 6 and 14 years with acid-related diseases, 37.5 mg or 75 mg roxatidine capsules were given orally, and blood samples were collected to determine the plasma roxatidine concentrations. Meanwhile, a single-dose PK study in healthy adult volunteers was newly conducted; subjects were given 37.5 mg, 75 mg or 150 mg roxatidine capsules. Differences were present between the PK parameters in pediatric patients and those in healthy adult volunteers. However, the CL/F and Vd/F adjusted by body surface area (BSA) or body weight (BW) were comparable. A close correlation of the C(max) and AUC(0-∞) to the dose per unit BSA (mg/m(2)) or BW (mg/kg) was also shown. In the multiple-dose study in pediatric patients, no roxatidine accumulation in plasma was observed, as was the case with a previous study in adults. These data show that the PK profile of roxatidine in pediatric patients is similar to the profile in healthy adult volunteers when adjusted by BSA or BW. [\hyperlink{Altace}{PMID: 22293541}, Hidefumi Nakamura et al., 2012]

\hypertarget{pmid_29490769}{T}he safety of a novel intranasal formulation of azelastine hydrochloride (AZE) and fluticasone propionate (FP) has been established in adults and adolescents with allergic rhinitis but not in children <12 years old. To evaluate the safety and tolerability of an intranasal formulation of AZE and FP in children ages 4-11 years with allergic rhinitis. The study was a randomized, 3-month, parallel-group, open-label design. Qualified patients were randomized in a 3:1 ratio to AZE/FP (n = 304) or fluticasone propionate (FP) (n = 101), one spray per nostril twice daily, and to one of three age groups: ≥4 to <6 years, ≥6 to <9 years, and ≥9 to <12 years. Safety was assessed by child- or caregiver-reported adverse events, nasal examinations, vital signs, and laboratory assessments. The incidence of treatment-related adverse events (TRAEs) was low in both the AZE/FP (16\%) and FP-only (12\%) groups after 90 days' continuous use. Epistaxis was the most frequently reported TRAE in both groups (AZE/FP, 9\%; FP, 9\%), followed by headache (AZE/FP, 3\%; FP, 1\%). All other TRAEs in the AZE/FP group were reported by ≤1\% of the children. The majority of TRAEs were of mild intensity and resolved spontaneously. Results of nasal examinations showed an improvement over time in both groups, with no cases of mucosal ulceration or nasal septal perforation. There were no unusual or unexpected changes in laboratory parameters or vital signs. The intranasal formulation of AZE and FP was safe and well tolerated after 3 months' continuous use in children with allergic rhinitis.The study was registered on <ext-link xmlns:xlink="http://www.w3.org/1999/xlink" ext-link-type="uri" xlink:href="http://ClinicalTrials.gov">ClinicalTrials.gov</ext-link> (NCT01794741). [\hyperlink{Altace}{PMID: 29490769}, William Berger et al., 2018]

\hypertarget{pmid_15510203}{T}he recombinant urate oxidase, rasburicase (Elitek, Sanofi-Synthelabo, Inc.), has recently received regulatory approval for the prevention and treatment of hyperuricemia in children with leukemia, lymphoma, and solid tumors. Prior to approval, 682 children and 387 adults in the US and Canada received rasburicase on compassionate-use basis. Uric acid concentration declined rapidly in both adult and pediatric patients after rasburicase treatment. Similar responses were observed in patients treated with subsequent courses. Possible drug-related adverse events, including allergic reactions, were uncommon. These data confirm that rasburicase is effective and safe for the treatment and prophylaxis of children and adults with malignancy-associated hyperuricemia. [\hyperlink{Altace}{PMID: 15510203}, S Jeha et al., 2005]

\hypertarget{pmid_995517}{A}mikacin, a new aminoglycoside antibiotic with a spectrum similar to that of gentamicin, has been used mainly in adults. This report summarizes the first use of this drug in children with urinary tract infection. Organisms were eradicated in all cases and recurrent infection occurred in one half after one week. No evidence of ototoxicity or nephrotoxicity was found. Four children developed transient elevation of serum glutamic oxaloacetic transaminase. Serum level (17 mug/ml) of the drug at one hour and its urinary excretion in six hours (60\% of the dose) was comparable to those of adults. This antibiotic is potentially valuable and has thus far not shown major toxicity when given for up to 11 days in patients with normal renal and liver functions. [\hyperlink{Altace}{PMID: 995517}, A J Khan et al., 1976]

\hypertarget{pmid_24175945}{T}he purpose of the study was to compare the safety of artemether-lumefantrine (AL) with other artemisinin-based combinations in children. A search of EMBASE (from 1974 to April 2013), MEDLINE (from 1946 to April 2013) and the Cochrane library of registered controlled trials for randomized controlled trials (RCTs) which compared AL with other artemisinin-based combinations was done. Only studies involving children ≤ 17 years old in which safety of AL was an outcome measure were included. Four thousand, seven hundred and twenty six adverse events (AEs) were recorded in 6,000 patients receiving AL. Common AEs (≥ 1/100 and <1/10) included: coryza, vomiting, anaemia, diarrhoea, vomiting and abdominal pain; while cough was the only very commonly reported AE (≥ 1/10). AL-treated children have a higher risk of body weakness (64.9\%) than those on artesunate-mefloquine (58.2\%) (p = 0.004, RR: 1.12 95\% CI: 1.04-1.21). The risk of vomiting was significantly lower in patients on AL (8.8\%) than artesunate-amodiaquine (10.6\%) (p = 0.002, RR: 0.76, 95\% CI: 0.63-0.90). Similarly, children on AL had a lower risk of vomiting (1.2\%) than chlorproguanil-dapsone-artesunate (ACD) treated children (5.2\%) (p = 0.002, RR: 0.63, 95\% CI: 0.47-0.85). The risk of serious adverse events was significantly lower for AL (1.3\%) than ACD (5.2\%) (p = 0.002, RR: 0.45, 95\% CI: 0.27-0.74). Artemether-lumefantrine combination is as safe as ASAQ and DP for use in children. Common adverse events are cough and gastrointestinal symptoms. More studies comparing AL with artesunate-mefloquine and artesunate-azithromycin are needed to determine the comparative safety of these drugs. [\hyperlink{Altace}{PMID: 24175945}, Oluwaseun Egunsola et al., 2013]

\hypertarget{pmid_19840089}{T}he primary objective of the present study was to determine the effectiveness of intranasal fentanyl analgesia in children aged 1-3 years with acute moderate to severe pain presenting to the ED. We also aimed to gather information on the safety and acceptability of intranasal fentanyl in this age group. Two paediatric ED enrolled children aged 1-3 years, with acute moderate or severe pain. Intranasal fentanyl was administered (1.5 microg/kg) via a mucosal atomiser device using a 50 microg/mL solution of fentanyl. Physiological parameters (heart rate, respiratory rate, oxygen saturations and level of consciousness) were measured at regular intervals. Objective pain assessment was completed using the Faces, Legs, Arms, Cry, Consolability (FLACC) score. Forty-six children presenting with acute moderate to severe pain were included. The median FLACC score before intranasal fentanyl administration was 8 (interquartile range [IQR] 5-10), decreasing to 2 (IQR 0-4) 10 min post fentanyl (P < 0.0001) and 0 (IQR 0-2) 30 min post fentanyl (P < 0.0001). A clinically significant decrease in FLACC scores was seen in 93\% of children 10 min post fentanyl administration and 98\% of children 30 min post fentanyl. Intranasal fentanyl delivery using a mucosal atomiser was well tolerated by all children. There were no adverse drug reactions or adverse events detected. Intranasal fentanyl is an effective, safe and well-tolerated mode of analgesia for children aged 1-3 years with moderate to severe pain. [\hyperlink{Altace}{PMID: 19840089}, Joanne Cole et al., 2009]

\hypertarget{pmid_20100745}{A}n apparent life-threatening event (ALTE) caused by ingestion of drugs or toxins has been reported rarely among infants. None of these agents was homeopathic medication. We report 11 infants who presented with an ALTE after ingestion of Gali-col Baby, a homeopathic agent indicated for "infantile colic." A retrospective case-control study was conducted. Charts of all infants who were younger than 1 year and were admitted with an ALTE from January 2005 through August 2008 to the pediatric division at the Barzilai Medical Center were reviewed. Age-matched infants who were admitted on the same dates for a reason other than ALTE served as a control group. Information on medications administered before admission was recorded. During the study period, 36 635 children visited the pediatric emergency department of the Barzilai Medical Center. There were 11 057 admissions to the pediatric division during this period, 115 of which were because of an ALTE. Eleven of these infants received Gali-col Baby before the event as opposed to none in the control group (P < .005). Three infants received a significant overdose, compared with the manufacturer's recommended dosage. After a thorough investigation, no other presumptive causes for ALTE were found among the 11 infants. Gali-col Baby is associated with an ALTE in some infants. There are no published controlled trials on the efficacy or safety of its use; therefore, better control and supervision of Gali-col Baby and probably other homeopathic medications are needed to prevent possible serious adverse effects. [\hyperlink{Altace}{PMID: 20100745}, Shraga Aviner et al., 2010]

\hypertarget{pmid_24107880}{A} randomized study on the efficacy and safety of the hopantenic acid preparation (pantocalcin) and its effect on cognitive functions in children with cerebral palsy (CCP) has been carried out. The positive effect of pantocalcin on the visual memory and attention concentration, activity and fatigability has been shown. At the same time, there was a decrease of anxiety in children and adolescents with CCP. No evidence for the effect of the drug on visual-motor skills has been found. The results of the study have demonstrated the high safety profile of pantocalcin when used in pediatric practice.  [\hyperlink{Altace}{PMID: 24107880}, T T Batysheva et al., 2013] Data on the safety of paternal use of 5-aminosalicylic acid (5-ASA) prior to conception are lacking, and the safety of maternal use of 5-ASA during pregnancy has not been examined in nationwide data. To examine offspring outcomes after paternal pre-conception use of 5-ASA, and after maternal use during pregnancy METHODS: This nationwide cohort study was based on Danish health registries. The study population included live born singletons of patients with ulcerative colitis (UC) or Crohn's disease (CD). Paternal exposure included 2168 children fathered by men treated with 5-ASA, and 7732 unexposed. Maternal exposure included 3618 children exposed in utero to 5-ASA, and 7128 unexposed. The outcomes were pre-term birth, small for gestational age (SGA), low Apgar score and major congenital abnormalities (CAs) according to EUROCAT guidelines. The vast majority of fathers and mothers used mesalazine. In children fathered by men with UC using 5-ASA, we found no increased risk of pre-term birth, SGA or low Apgar score. The hazard ratio (HR) of CAs was 1.30 (95\% CI 0.92-1.85). In children of fathers with CD, the odds ratio (OR) of SGA was 1.52 (95\% CI 0.65-3.55). After maternal 5-ASA exposure, the OR of SGA in children of women with UC was 1.46 (95\% CI: 0.93-2.30); for CAs in children of women with CD, HR was 1.44 (95\% CI 0.84-2.47). Paternal and maternal use of 5-ASA was safe across offspring outcomes; none of the findings reached statistical significance. The safety of 5-ASA formulations that are used infrequently cannot be settled here. [\hyperlink{Altace}{PMID: 24107880}, Bente Mertz Nørgård et al., 2022]

\hypertarget{pmid_18391694}{R}asburicase (Fasturtec), a recombinant urate oxidase, is highly effective in preventing and treating hyperuricemia in children with hematologic malignancies. We conducted a prospective, multicenter observational study in 174 patients at 8 pediatric hemato-oncology centers to establish whether the SFCE (Société Française de Lutte contre les Cancers et Leucémies de l'Enfant et de l'Adolescent) recommendations for the use of rasburicase in the management of pediatric patients at risk of tumor lysis syndrome (TLS) are valid in routine clinical practice. Patients were classified as being at high or low risk of TLS according to the Children's Oncology Group criteria and were treated in accordance with the SFCE recommendations. The primary end point was the number of patients requiring a higher dose of rasburicase or a longer duration of treatment than advised in the SFCE recommendations. Of the 135 patients at high risk of TLS, 27 patients received a higher dose and 35 patients received a longer duration of treatment. Some patients received treatment with rasburicase for less than the recommended duration (median 4 d for high-risk patients). One patient required hemodialysis. Only minor adjustments to the SFCE recommendations were required to ensure the optimal use of rasburicase in pediatric patients at risk of TLS. [\hyperlink{Altace}{PMID: 18391694}, Yves Bertrand et al., 2008]

\hypertarget{pmid_18066121}{E}chinacea purpurea (L.) Moench was mistakenly taken from North America to Germany in 1939 where it was cultivated and various extractions were prepared and subsequently used to treat upper respiratory tract infections. Parents often administer Echinacea to their children, but safety data on the use of Echinacea in Canadian children is lacking. A screening history, physical examination, and daily record of symptoms from an initial visit through to a the follow-up visit 13 days later were used to increase patient safety. Each subject was administered an aerial part Echinacea extract. The dose was based on age (2.5 mL three times per day for children aged 2-5 years, and 5 mL two times per day for children aged 6-12 years) and administered for 10 days in an open-label trial. A rating scale was used to measure tolerance to the treatment. We assessed the safety and compliance of use of the Echinacea extract by measuring the amount of extract returned at the end of the study, having the parents complete and return a daily symptom diary, and recording the subjects' use of other natural health products or medications during the trial. Clinical effectiveness of the Echinacea extract could not be accurately assessed because of the small trial size and because the extract had been administered when some of the subjects had an upper respiratory tract infection that had begun 1 or more days prior to the study; however, each subject's symptoms improved. No allergic or adverse reaction occurred and no safety issues arose. [\hyperlink{Altace}{PMID: 18066121}, Paul Richard Saunders et al., 2007]

\hypertarget{pmid_2511644}{T}he authors diagnosed lactose malabsorption by the breath hydrogen analysis in 11 premature and mature babies, in 16 infants and in 28 children between the ages of 3-18 years. All patients were treated with Galantase (beta-galactosidase). According to the results, Galantase is very effective in splitting of lactose of breast-milk, cow-milk and artificial formulas. Pathological hydrogen increase was not detected during the treatment. [\hyperlink{Altace}{PMID: 2511644}, K Horváth et al., 1989]

\hypertarget{pmid_8865979}{A}denosine has been approved for intravenous use for paroxysmal supraventricular tachycardias (SVT) in adults and children. However, effectiveness and safety of intravenous adenosine in preterm infants are not well established. Thirteen episodes of SVT in three preterm and two full-term neonates were treated with intravenous adenosine. All had narrow QRS tachycardia at 230 to 260 beats/min. Adenosine prepared as a sterile 1 mg/mL solution was given as an intravenous bolus starting at 0.05 mg/kg, and increased by 0.05 mg/kg until tachycardia was terminated. Termination of tachycardia was achieved within 12 to 25 seconds in all patients. In one, termination of SVT was followed by temporary suppression of the sinus node, followed by resumption of normal sinus rhythm. No other side effects were noted. Adenosine is a safe and effective agent for treating preterm infants with SVT. However, further investigation of adenosine in this group of patients is warranted. [\hyperlink{Altace}{PMID: 8865979}, G Paret et al., 1996]

\hypertarget{pmid_33832544}{I}n children, up to 30\% of viral respiratory tract infections (RTIs) develop into bacterial complications associated with pneumonia, sinusitis or otitis media to trigger a tremendous need for antibiotics. This study investigated the efficacy of Echinacea for the prevention of viral RTIs, for the prevention of secondary bacterial complications and for reducing rates of antibiotic prescriptions in children. Echinaforce® Junior tablets [400 mg freshly harvested Echinacea purpurea alcoholic extract] or vitamin C [50 mg] as control were given three times daily for prevention to children 4-12 years. Two × 2 months of prevention were separated by a 1-week treatment break. Parents assessed respiratory symptoms in children via e-diaries and collected nasopharyngeal secretions for screening of respiratory pathogens (Allplex® RT-PCR). Overall, 429 cold days occurred in N Our results support the use of Echinacea for the prevention of RTIs and reduction of associated antibiotic usage in children. Trial registration clinicaltrials.gov, NCT02971384, 23th Nov 2016. [\hyperlink{Altace}{PMID: 33832544}, Mercedes Ogal et al., 2021]

\hypertarget{pmid_20427940}{E}xperience with alteplase in pediatric patients is limited and recommendations are extrapolated from adult data. Comprehensive guidelines on the management of thromboembolic events in this group are lacking. We assessed the efficacy and safety of alteplase (recombinant tissue plasminogen activator) in the management of intracardiac and major cardiac vessel thrombosis in pediatric patients. All pediatric patients, 14 years of age and younger, with intracardiac or major cardiac vessel thrombus who were treated with alteplase from 1997 to 2004 at our tertiary care institute were identified through the pharmacy database. Patient data were retrospectively evaluated for the efficacy and safety of altepase. Five cases were eligible out of nineteen who received alteplase. Patient ages ranged from 40 days to 13 years. The initial dose of alteplase ranged from 0.3 to 0.6 mg/kg followed by a continuous infusion in three patients with a dosage range between 0.05 and 0.5 mg/kg/hr, while intermittent infusion was used in the other two patients. The duration of therapy ranged from 2 to 4 days. By the end of the treatment, two patients had complete resolution of thrombus and one had partial resolution. Two patients failed to respond and had "old" thrombus. Major bleeding events were reported in three patients. The rest had minor bleeding events. Alteplase may effectively dissolve intracardiac thrombi, particularly when freshly formed. Continuous infusion for a long duration appears to be associated with an increased risk of major bleeding. Optimal dose and duration of infusion are still unknown. [\hyperlink{Altace}{PMID: 20427940}, Abdulrazaq S Al-Jazairi et al., ]

\hypertarget{pmid_32223002}{P}ain control is a mandatory aspect in pediatric dentistry office through local anesthesia. To assess the safety and efficacy of 4\% articaine local anesthetic in young children below four years old. An equivalent randomized control trial with two parallel arms included 184 young children (92 per group) aged from 36 to 47 months seeking pulpotomy of mandibular primary molars which performed after buccal infiltration injection. The control group received lidocaine hydrochloride 2\% with epinephrine 1:100 000. The intervention was articaine hydrochloride 4\% with epinephrine 1:100 000. Children's behavior during injection and treatment have assessed using Faces, Legs, Activity, Cry, and Consolability (FLACC) and child's behavior using Frankl Behavior Rating Scale (FBRS). In addition, post-operative complications have been addressed. Both anesthetic agents were equivalent during the injection phase. During the treatment phase, the absolute risk difference (ARR) between the two groups was 0.120 (95\% CI: -0.003; 0.243). The maximum limit of 95\% CI surpassed the margin of equivalence, indicating that less pain has been expressed during pulpotomy among children delivered articaine when compared to their counterparts in the lidocaine group. Concerning post-operative complications, no statistically significant difference was detected between the two anesthetic drugs. The findings supported the efficient and secure use of articaine hydrochloride 4\% with epinephrine 1:100 000 to treat children between the ages of 3 and below 4 years old. [\hyperlink{Altace}{PMID: 32223002}, Ahmad Abdel Hamid Elheeny et al., 2020]

\hypertarget{pmid_22477803}{S}afety and efficacy issues regarding over-the-counter cough and cold (CAC) products for use in children have surfaced. Late in 2007 the FDA began reviewing CAC product status for use in children under 6 years old. In regards to CAC products for children < 6 years old, to determine pharmacists: 1) comfort level in recommending; 2) attitudes towards behind-the-counter (BTC) status; and 3) level of support for BTC status. An additional objective was to determine how frequently pharmacists were asked for CAC product recommendations for children Georgia Pharmacy Association members (2,045) were invited to anonymously participate in a self-administered online survey from January 3 - Feb 6, 2008. Topic areas included demographics, comfort in recommending CAC, and BTC status. Most responding pharmacists (99.1\%) feel pediatric CAC medicine safety problems are due to inappropriate use. More than 50\% of chain and independent pharmacists were asked to recommend CAC medicines for children during cold/flu season once a day or less, and 79\% reported counseling on less than 50\% of total CAC sales. The majority of pharmacists felt comfortable recommending CAC medications when thinking of both safety and efficacy. Most pharmacists supported a BTC condition of sale for children under two for decongestants, antihistamines, and antitussives, and for decongestants and antitussives for children between 2 and 5 years old. Most pharmacists indicate comfort in recommending CAC despite lack of evidence for safety or efficacy and support BTC status. Pharmacist education on this topic would be useful. [\hyperlink{Altace}{PMID: 22477803}, Sally A Huston et al., 2010]

\hypertarget{pmid_19818174}{I}nfants and children under five years of age are the most vulnerable to malaria with over 1,700 deaths per day from malaria in this group. However, until recently, there were no WHO-endorsed paediatric anti-malarial formulations available. Artemisinin-based combination therapy is the current standard of care for patients with uncomplicated falciparum malaria in Africa. Artemether/lumefantrine (AL) meets WHO pre-qualification criteria for efficacy, safety and quality. Coartem, a fixed dose combination of artemether and lumefantrine, has consistently achieved cure rates of >95\% in clinical trials. However, AL tablets are inconvenient for caregivers to administer as they need to be crushed and mixed with water or food for infants and young children. Further, in common with other anti-malarials, they have a bitter taste, which may result in children spitting the medicine out and not receiving the full therapeutic dose. There was a clear unmet medical need for a formulation of AL specifically designed for children. Ahead of a call from WHO for child-friendly medicines, Novartis, working in partnership with Medicines for Malaria Venture (MMV), started the development of a new formulation of AL for infants and young children: Coartem Dispersible. The excellent efficacy, safety and tolerability already demonstrated by AL tablets were confirmed with dispersible AL in a large trial comparing the crushed tablets with dispersible tablets in 899 African children with falciparum malaria. In the evaluable population, 28-day PCR-corrected cure rates of >96\% were achieved. Further, its sweet taste means that it is palatable for children, and the dispersible formulation makes it easier for caregivers to administer than bitter crushed tablets. Easing administration may foster compliance, hence improving therapeutic outcomes in infants and young children and helping to preserve the efficacy of ACT. [\hyperlink{Altace}{PMID: 19818174}, Salim Abdulla et al., 2009]

\hypertarget{pmid_23631461}{C}urrent American Academy of Pediatrics Guidelines recommended that statins should be considered as a first-line agent in children as early as 8 years of age. The aim of our work is to assess the safety of 3-hydroxy-3-methylglutaryl coenzyme A reductase inhibitors in children with hypercholesterolaemia. Controlled studies in children show that statin monotherapy is efficacious, well tolerated and safe in the short-time. Unfortunately, these studies have relatively short-term follow-up periods, and therefore, long-term safety remains unclear. [\hyperlink{Altace}{PMID: 23631461}, Norman Lamaida et al., 2013]

\hypertarget{pmid_20386439}{A}lbumin has been regarded as the gold standard for maintaining adequate colloid osmotic pressure in children, but increased cost, the lack of clear-cut benefits for survival, and fear of transmission of unknown viruses have contributed to its replacement by hydroxyethyl starch and gelatin preparations. Each of the synthetic colloids has unique physicochemical characteristics that determine their likely efficacy and adverse effect profile. This review will examine the advantages and disadvantages of the use of different colloid solutions in children with a particular focus on their safety profile. Dextrans are rarely used because of their negative effects on coagulation and potential for anaphylactic reactions. Gelatin and albumin have little effect on hemostasis, but the disadvantages of gelatin include its high anaphylactoid potential and limited beneficial volume effect. Tetrastarches have significantly fewer adverse effects on coagulation and renal function than the older hydroxyethyl starches and are now approved for children. Dissolving tetrastarches in a plasma-adapted, balanced solution rather than in saline further improves safety with regard to coagulation and acid-base balance. Tetrastarches offer the best currently available compromise between cost-effectiveness and safety profile in children with preexisting normal renal function and coagulation. [\hyperlink{Altace}{PMID: 20386439}, Sonja Saudan et al., 2010]

\hypertarget{pmid_33430841}{A}cute cough in children often causes discomfort to children and parents, reducing their quality of life. Despite the extensive utilization of over-the-counter remedies for cough, the efficacy of most of these treatments in children has not been confirmed. We conducted a randomized, double blind, placebo-controlled clinical trial of 106 children with acute cough to evaluate the efficacy and safety of KalobaTUSS®, a paediatric cough syrup based on acacia honey and on Malva sylvestris extract, Inula helenium extract, Plantago major extract, and Helichrysum stoechas extract by using a validated 6 points Likert scale. Children were orally treated with KalobaTUSS® or placebo for 8 days. Children receiving KalobaTUSS® showed an early and significant reduction in night-time and day-time cough scores measured using a specific scale and a shorter duration of cough than children treated with the placebo. KalobaTUSS® is well tolerated and produces positive effects by reducing the severity and shortening the duration of cough in children. Clinicaltrials.gov no. NCT04073251 . Retrospectively registered. [\hyperlink{Altace}{PMID: 33430841}, Ilaria Carnevali et al., 2021]

\hypertarget{pmid_7594705}{T}he safety, tolerability, and pharmacokinetics of zalcitabine (ddC) in a single oral dose (0.02 mg/kg) was evaluated in 23 mildly symptomatic human immunodeficiency virus-infected children (mean age, 4.2 years). After administration of ddC, blood samples were obtained at 0.5, 1, 1.5, 2, 4, 6, and 8 h for analysis. The drug was well tolerated and no side effects were noted. Plasma ddC levels were determined by ion spray liquid chromatography/tandem mass spectrometry. ddC was rapidly absorbed, with a mean maximum plasma concentration of 9.3 ng/mL (range, 3.2-14.1) attained within a mean of 1 h (range, 0.5-2.0). Mean elimination half-life was 1.4 h (range, 1.0-3.5), mean area under the plasma concentration-time curve was 25 ng.h/mL (range, 11-37), and mean total body clearance was 14.6 mL/min/kg (range, 8.9-30.6). Plasma concentrations were lower and the half-life shorter in these children than in adults given comparable doses, suggesting that ddC may be cleared more rapidly in children than adults. [\hyperlink{Altace}{PMID: 7594705}, E G Chadwick et al., 1995]

\section*{Alvimopan}
\subsection*{Result}
\subsubsection*{Answer}

Unknown

\subsubsection*{{Explanation}}
\hypertarget{Alvimopan}
A review of the available abstracts reveals the following regarding the safety of Alvimopan in children:

- \hyperlink{pmid_19574601}{PMID: 19574601} [Heather R Bream-Rouwenhorst et al., 2009] reviews the efficacy, safety, pharmacology, and administration of alvimopan for postoperative ileus. The abstract states that alvimopan is generally well tolerated, but all referenced studies and safety data pertain to adults. There is no mention of pediatric patients or targeted studies in children.

- \hyperlink{pmid_18778122}{PMID: 18778122} [Monique P Curran et al., 2008] summarizes phase III trials of alvimopan for postoperative ileus. All trials were conducted in adults undergoing bowel resection. No pediatric patients were included, and no safety data for children are presented.

- \hyperlink{pmid_27825721}{PMID: 27825721} [Abhijit Nair et al., 2016] discusses the use of alvimopan in the context of its FDA approval and logistical challenges. The abstract does not mention children or provide any pediatric safety data.

- \hyperlink{pmid_16626607}{PMID: 16626607} [Thomas J Herzog et al., 2006] reports on a randomized trial of alvimopan in women undergoing hysterectomy. All participants were adults; there is no mention of children.

- \hyperlink{pmid_19635772}{PMID: 19635772} [Timothy J Bell et al., 2009] analyzes the economic effect of alvimopan in phase III trials. All patients were 18 years or older.

Based on the abstracts reviewed, there are no targeted studies evaluating the safety of alvimopan in children (defined as individuals under 18 years of age). No abstract provides evidence of alvimopan being studied in pediatric populations, nor do any abstracts affirm its safety or indicate it is unsafe in children. Therefore, the safety of alvimopan in children is unknown.

\subsection*{Abstracts}
\hypertarget{pmid_24175945}{T}he purpose of the study was to compare the safety of artemether-lumefantrine (AL) with other artemisinin-based combinations in children. A search of EMBASE (from 1974 to April 2013), MEDLINE (from 1946 to April 2013) and the Cochrane library of registered controlled trials for randomized controlled trials (RCTs) which compared AL with other artemisinin-based combinations was done. Only studies involving children ≤ 17 years old in which safety of AL was an outcome measure were included. Four thousand, seven hundred and twenty six adverse events (AEs) were recorded in 6,000 patients receiving AL. Common AEs (≥ 1/100 and <1/10) included: coryza, vomiting, anaemia, diarrhoea, vomiting and abdominal pain; while cough was the only very commonly reported AE (≥ 1/10). AL-treated children have a higher risk of body weakness (64.9\%) than those on artesunate-mefloquine (58.2\%) (p = 0.004, RR: 1.12 95\% CI: 1.04-1.21). The risk of vomiting was significantly lower in patients on AL (8.8\%) than artesunate-amodiaquine (10.6\%) (p = 0.002, RR: 0.76, 95\% CI: 0.63-0.90). Similarly, children on AL had a lower risk of vomiting (1.2\%) than chlorproguanil-dapsone-artesunate (ACD) treated children (5.2\%) (p = 0.002, RR: 0.63, 95\% CI: 0.47-0.85). The risk of serious adverse events was significantly lower for AL (1.3\%) than ACD (5.2\%) (p = 0.002, RR: 0.45, 95\% CI: 0.27-0.74). Artemether-lumefantrine combination is as safe as ASAQ and DP for use in children. Common adverse events are cough and gastrointestinal symptoms. More studies comparing AL with artesunate-mefloquine and artesunate-azithromycin are needed to determine the comparative safety of these drugs. [\hyperlink{Alvimopan}{PMID: 24175945}, Oluwaseun Egunsola et al., 2013]

\hypertarget{pmid_19574601}{T}he efficacy, safety, pharmacology, pharmacokinetics, drug-drug interactions, and administration of alvimopan for postoperative ileus are reviewed. Alvimopan is a selective mu-opioid receptor antagonist with no central nervous system activity. When orally administered after partial small- or large-bowel resection in patients with primary anastomosis, alvimopan shortened the return of bowel function and time to discharge by approximately one day without compromising analgesia. Alvimopan was not shown to be beneficial on these same outcomes after hysterectomy and has not been studied in other surgical populations. Alvimopan is generally well tolerated, with the frequency of adverse events being similar to placebo when used postoperatively for one week or less. Long-term studies of alvimopan in opioid-induced bowel dysfunction have shown an association with adverse cardiovascular outcomes, neoplasms, and fractures. Because of these concerns, the Entereg Access Support and Education program was developed. The recommended dosage of alvimopan is 12 mg administered with a sip of water 30 minutes to five hours before surgery, followed by 12 mg twice daily beginning the day after surgery for a maximum of seven days, 15 total doses, or until discharge. There is a limited amount of pharmacoeconomic analysis concerning alvimopan. Alvimopan, a peripherally acting mu-opioid receptor antagonist, is a novel agent for the treatment of postoperative ileus. It appears to decrease the duration of postoperative ileus and hospitalization by approximately one day, theoretically offsetting its acquisition costs. Unresolved long-term safety issues, a limited indication, and its restricted-access program are likely to hinder its widespread use in the surgical population. [\hyperlink{Alvimopan}{PMID: 19574601}, Heather R Bream-Rouwenhorst et al., 2009]

\hypertarget{pmid_9570603}{R}egional nerve blocks are often used for the treatment of postoperative pain in children. Ammonium sulfate is a non-narcotic anesthetic agent, which has been reported to provide pain relief lasting days to weeks, with few reported side effects in adult studies. Prior to considering clinical use in children, the neurotoxicity of ammonium sulfate in 4-day and 3-week old rats was assessed and compared with that of bupivacaine. Each rat received a posterior tibial nerve intrafascicular injection (0.01 mL in 4-day-old and 0.02 mL in 3-week-old rats) using either 10\% ammonium sulfate (n = 24 per age group), 0.5\% bupivacaine (n = 18 per age group), 0.9\% saline (n = 18 per age group), or 5\% phenol (n = 18 per age group). A functional assessment by serial walking track analysis and a morphologic assessment by neurohistology were made. No abnormalities in serial walking track analysis and no structural nerve damage were detected after ammonium sulfate, bupivacaine, or saline injection. Bupivacaine caused mild focal changes in both age groups, which recovered by 8 weeks. Intrafascicular injection of ammonium sulfate was as safe as bupivacaine in this animal model. Further animal studies must be made before human trials are initiated. [\hyperlink{Alvimopan}{PMID: 9570603}, M C Hertl et al., ]

\hypertarget{pmid_18778122}{A}lvimopan, a trans-3,4-dimethyl-4-(3-hydroxy-phenyl) piperidine, is a selective, peripherally acting micro-opioid receptor antagonist that is available for short-term use in hospitalized patients who have undergone bowel resection. The efficacy of alvimopan in the management of postoperative ileus has been evaluated in five phase III trials; four conducted in North America and one conducted in Europe/Australasia. Patients who had undergone partial large or small bowel resection surgery with primary anastomosis were randomized to receive alvimopan 12 mg or placebo as a single oral pre-operative dose followed by twice-daily administration for up to 7 days postoperatively. In the five phase III trials, alvimopan was significantly more effective than placebo in reducing the time to recovery of upper and lower gastrointestinal (GI) function, as assessed using a two-component endpoint (GI2) comprising time to tolerance of solid food and first bowel movement. The mean time to reach the GI2 endpoint was 11-26 hours sooner with alvimopan than with placebo. In the phase III trials conducted in North America, the time to writing the hospital discharge order was 13-21 hours sooner with alvimopan than with placebo. Alvimopan did not reduce opioid-induced analgesia and/or increase the amount of opioids administered postoperatively. Short-term alvimopan was generally well tolerated in adults undergoing bowel resection. [\hyperlink{Alvimopan}{PMID: 18778122}, Monique P Curran et al., 2008]

\hypertarget{pmid_25145624}{O}lanzapine is frequently prescribed in young children for psychiatric conditions. It may be an option for chemotherapy-induced nausea and vomiting (CINV) control in children. The objective of this review was to describe the safety of olanzapine in children less than 13 years of age to determine if safety concerns would be a barrier to its use for CINV prevention. Electronic searches were performed in MEDLINE, EMBASE, Cochrane Central Register of Controlled Trials, Web of Science and Scopus. All studies in English reporting adverse effects associated with olanzapine use in children younger than 13 years or with a mean/median age less than 13 years were included. Adverse outcomes were synthesized for prospective studies. A total of 47 studies (17 prospective) involving 387 children aged 0.6-18 years were included; nine described olanzapine poisonings. Weight gain or sedation were reported in 78 \% [95 \% confidence interval (CI) 63-95] and 48 \% (95 \% CI 35-67), respectively. Extrapyramidal symptoms or electrocardiogram abnormalities were reported in 9 \% (95 \% CI 4-21) and 14 \% (95 \% CI 7-26), respectively. Elevation in liver function tests or blood glucose abnormalities were reported in 7 \% (95 \% CI 2-20) and 4 \% (95 \% CI 1-17), respectively. No deaths were attributed to olanzapine. No studies were identified with a primary focus on evaluating safety, and the adverse effects reported in the included studies were heterogeneous. Most adverse events associated with olanzapine use in children less than 13 years of age are of minor clinical significance. These findings support the exploration of olanzapine for the prevention of CINV in children in future trials. [\hyperlink{Alvimopan}{PMID: 25145624}, Jacqueline Flank et al., 2014]

\hypertarget{pmid_27825721}{A}lvimopan is an US-FDA approved, peripherally acting mu opioid receptor antagonist which when started pre-operatively has been shown to hasten intestinal motility and reduce the duration of post-operative ileus. However the logistics involved in procuring, storing and dispensing the drug and the cost of the drug for fifteen doses as approved by FDA prohibits the use of it on a regular basis. [\hyperlink{Alvimopan}{PMID: 27825721}, Abhijit Nair et al., 2016]

\hypertarget{pmid_15595588}{T}he aim of this study was to investigate the quality of intra- and postoperative analgesia obtained by alfentanil compared to that produced by peripheral blockade in children. During sevoflurane-nitrous oxide atracurium anaesthesia for minor abdominal or genito-urinary surgery, three groups of children aged 0-8 yr received 25 microg kg(-1) alfentanil intravenously (n = 28), or peripheral nerve blockade using 1 mLkg(-1) ropivacaine 0.475\% (n = 24), or 12.5 microg kg(-1) alfentanil intravenously with peripheral nerve blockade using 1 mL kg(-1) ropivacaine 0.475\% (n = 30). Changes in blood pressure and heart rate were measured during the procedures. Postoperative pain was assessed using the face, legs, activity, cry, consolability (FLACC) observational tool for quantifying pain behaviour and a numerical scale scored by nurses, doctors, parents and children. There was no significant difference in intra- or postoperative analgesic efficacy among the three groups. Patients who received alfentanil had significantly lower heart rates than those who received nerve blockade only (96.0+/-15.6 vs. 115.9+/-23.2 beats min(-1), P < 0.001). FLACC and numerical scale scores did not differ among the groups. There were no significant differences in incidence of vomiting or use of pain medications. It was concluded that a low-dose, intravenous bolus of alfentanil may be an efficient alternative to peripheral nerve blockade in controlling pain during and after minor abdominal and genito-urinary surgery. [\hyperlink{Alvimopan}{PMID: 15595588}, F Leoni et al., 2004]

\hypertarget{pmid_29388634}{P}ediatric data on the use of thrombopoietin receptor agonists are fairly limited. The recent approval of eltrombopag by the US Food and Drug Administration for children aged ≥1 year, based on data from two randomized, placebo-controlled clinical trials, may lead to the increased use of this drug in clinical practice, and therefore, it is important to have a basic understanding of the biology, pharmacokinetics, safety, and efficacy of the medication. [\hyperlink{Alvimopan}{PMID: 29388634}, Michele P Lambert et al., 2016]

\hypertarget{pmid_14707960}{B}rimonidine 0.2\% (Alphagan) is a topical alpha-2 agonist widely used as an antihypertensive. There have been occasional reports of systemic adverse effects in children including apparent central nervous system depression. There are few data available on the overall safety of brimonidine 0.2\% in children. Computerised pharmacy records were used to identify all children who had been prescribed brimonidine 0.2\% in our eye department between August 1999 and June 2001, and their notes were reviewed. In all, 23 patients were identified from pharmacy records and 22 sets of notes were recovered and reviewed. The mean age at commencement of treatment was 8 years (range 0-14 years). In all, 10 (46\%) were treated in one eye and 12 (54\%) in both. Brimonidine 0.2\% was taken for a mean 14 months (range 1 day-75 months). A total of 14 (64\%) patients were already taking a topical beta-blocker when brimonidine 0.2\% was commenced and a further four (18\%) were being treated with another topical hypotensive agent. Of the 22 patients, six (27\%) had to stop brimonidine 0.2\% because of adverse side effects (two because of local irritation/allergy, two because of tiredness, and two because of fainting attacks). Many topical hypotensive agents are not licensed for use in children and few safety data are available. In this study, 18\% of children had systemic adverse effects sufficient to necessitate stopping the drug. It is possible that educational impairment may have passed unnoticed in others. Larger studies are required to investigate this further. [\hyperlink{Alvimopan}{PMID: 14707960}, R J C Bowman et al., 2004]

\hypertarget{pmid_9123909}{I}n the course of treatment for tumors and recurrences, 86 children, aged 4-16 years, received polychemotherapy which induced excessive vomiting. Navoban (tropisetron) was administered to control vomiting. Total or partial control of nausea and vomiting was observed in 94.1\%. No side-effects were registered. [\hyperlink{Alvimopan}{PMID: 9123909}, S A Safonova et al., 1996]

\hypertarget{pmid_11045391}{A}mlodipine has potential advantages in children since it can be dissolved into a liquid preparation and has a long elimination half-life, allowing for once-daily administration. The objective of this study was to compare the efficacy and compliance of amlodipine with that of standard long-acting calcium channel blockers (felodipine or nifedipine) in hypertensive children. A randomized, prospective, crossover study of 11 hypertensive children (9-17 years of age, 10 renal transplant patients) was performed with electronic monitoring of compliance. Each treatment arm was 30 days. No significant differences were observed in mean systolic (SBP) and diastolic blood pressures (DBP) between amlodipine and the other calcium channel blockers. Using 24-h blood pressure monitoring there were no significant differences over each drug treatment period in both mean day-time and night-time SBP and DBP. Patient compliance was similar in both the amlodipine and the nifedipine/felodipine treatment periods. These data suggest that amlodipine is as effective in pediatric nephrology patients as nifedipine and felodipine. Amlodipine may be optimally suited for treatment of young children because at present it is the only calcium channel blocker which can be administered once daily as a liquid preparation. [\hyperlink{Alvimopan}{PMID: 11045391}, J W Rogan et al., 2000]

\hypertarget{pmid_19635772}{T}he economic effect of the use of alvimopan in four randomized, double-blind, placebo-controlled, Phase III, North American efficacy trials was analyzed. Patients were eligible for the study if they were 18 years or older, were undergoing laparotomy for partial small or large bowel resection with primary anastomosis, and were scheduled for postoperative pain management with opioid-based i.v. patient-controlled analgesia. Patients analyzed in the North American Phase III trials received placebo or alvimopan 12 mg orally before surgery. Doses were administered twice daily beginning the day after surgery until hospital discharge or for a maximum of 15 doses. Compared with placebo, alvimopan was associated with a significantly shorter mean time to gastrointestinal (GI) recovery and a significantly shorter mean time to a written discharge order. Alvimopan was also associated with a mean hospital length of stay (LOS) of one full day less than placebo. The mean cost of alvimopan based on a mean of 8.9 12-mg doses was \$558.00; the alvimopan cost at the upper limit of allowed dosing was \$937.50. Combining the alvimopan and hospital costs for each patient, total costs for the alvimopan group were estimated to be lower than for the placebo group. In a post hoc analysis, alvimopan was associated with significantly faster upper and lower GI recovery after bowel resection and a mean LOS reduction of one day compared with placebo. The mean estimated hospital cost was \$879-\$977 less for patients who received alvimopan compared with placebo. The base-case and sensitivity analyses suggest that, on average, the use of alvimopan compared with placebo may have a cost-saving effect in the hospital setting. [\hyperlink{Alvimopan}{PMID: 19635772}, Timothy J Bell et al., 2009]

\hypertarget{pmid_9279301}{I}n 1993, the nonbenzodiazepine sedative-hypnotic zolpidem tartrate (Ambien) was approved for use in the US. Zolpidem has an imidazopyridine structure and possesses a rapid onset of action and a short half-life. The toxic threshold and profile have not been well established in the pediatric population. All pediatric zolpidem exposures reported to a regional poison information center over 24 months were reviewed retrospectively from the American Association of Poison Control Centers Toxic Exposure Surveillance System data collection forms. Twelve pediatric zolpidem exposures were reported. Seven were unintentional (ages 20 mon-5 y) and five were intentional misuse/suicide (ages 12-16 y). The regional poison information center was contacted within 1 h in ten cases with onset of symptoms within 10 to 60 min (mean 31.6 min). One child had no effect with 2.5 mg. As little as 5 mg caused symptoms with minor outcome in six unintentional ingestions (5-30 mg). Minor to moderate symptoms were reported 1-4 h after intentional ingestions (12.5-150 mg). The duration of symptoms in the unintentional cases ranged from less than 60 min up to 4 h (mean 2.4 h) and 6-10 h (mean 7.5 h) in the intentional exposures. Treatment consisted of observation (4), syrup of ipecac (1), lavage and activated charcoal (1), activated charcoal alone (5), and unknown (1). Due to the very rapid onset of central nervous system symptoms in children, emesis is not a treatment option. Supportive care, activated charcoal in large ingestions, and observation until symptoms resolve may be sufficient in most pediatric cases. [\hyperlink{Alvimopan}{PMID: 9279301}, D L Kurta et al., 1997]

\hypertarget{pmid_37087633}{E}ltrombopag is clinically approved for use in immune thrombocytopenia (ITP), chronic hepatitis C-related thrombocytopenia, and aplastic anemia and suitable for children; however, data on its overall safety profile are scarce. This study aimed to explore the clinical features of adverse drug events (ADEs) associated with eltrombopag in different age groups using individual case safety reports (ICSRs) from the World Health Organization database VigiBase and the US Food and Drug Administration Adverse Event Reporting System database from 2008 to 2022 in combination with a meta-analysis of data from randomized clinical trials in the literature from inception to July 28, 2022. We conducted disproportionality analyses by grouping patients into the following age groups: 0-17 (0-23 months, 2-11 years, and 12-17 years), 18-64, and ≥ 65 years. The ADEs about hepatobiliary disorders, thrombosis, skin and subcutaneous tissue disorders, infections, and so on were observed more differently in each age group. Meta-analysis results showed differences in the four system organ classes between adults and children with ITP: infections and infestations, general disorders and administration site conditions, skin and subcutaneous tissue disorders, and investigations. The adverse drug reactions in the latest version of instructions were searched in the databases to analyze their postmarketing safety signal strength. We observed signals of elevated alanine aminotransferase, aspartate aminotransferase, and blood bilirubin levels in all age groups. For children, urinary tract infection and back pain showed signals. Due to the inherent limitations of pharmacovigilance studies, more experiments are needed to assess the risks of eltrombopag in different ages. [\hyperlink{Alvimopan}{PMID: 37087633}, Han Qu et al., 2023]

\hypertarget{pmid_16626607}{T}he purpose of this study was to investigate the safety and efficacy of alvimopan, a novel peripherally acting mu-opioid receptor antagonist, in patients who undergo simple total abdominal hysterectomy. Women (n = 519) were randomized (4:1) to receive alvimopan 12 mg (n = 413) or placebo (n = 106) > or = 2 hours before the operation then twice daily for 7 days (hospital and home). Adverse events were monitored up to 30 days after the last dose of study drug was administered. Efficacy was assessed for 7 postoperative days. Overall, the most common adverse events were nausea, vomiting, and constipation; < 5\% of patients discontinued use because of adverse events. Alvimopan significantly accelerated the time to first bowel movement (hazard ratio, 2.33; P <.001). Average time to first bowel movement was reduced by 22 hours, with more frequent bowel movement and better bowel movement quality found in the treatment cohort. Alvimopan has a safety profile that is similar to that of placebo and provides significantly improved lower gastrointestinal recovery in women who undergo simple total abdominal hysterectomy. [\hyperlink{Alvimopan}{PMID: 16626607}, Thomas J Herzog et al., 2006]

\hypertarget{pmid_36030422}{I}ntroduction: Biological therapy can be used in uveitis in children since 2016. With ophthalmological indication only adalimumab therapy can be started. Adalimumab is a monoclonal antibody that inhibits tumor necrosis factor alpha.Objective: To summarize our experience with patients receiving adalimumab for pediatric non-infectious uveitis.Patients and methods: We investigated our juvenile patients of non-infectious uveitis treated with adalimumab be-tween 2017 and 2021 in a retrospective case series at the Department of Ophthalmology, Szeged University. Results: Between 01 January, 2017 and 31 May, 2021, we examined 46 children with uveitis. The mean age of these 23 girls and 23 boys was 11 years. 21 of them had juvenile idiopathic arthritis, 14 had infectious uveitis, 3 had hae-matological disorders, 8 had idiopathic uveitis. Adalimumab was given to 11 patients because of severe, chronic uveitis. There were 3 boys and 8 girls, their mean age was 10 years. Adalimumab was given according to the licence of the European Medicines Agency. Indication was anterior uveitis at 6 children, panuveitis at 5 children. Adali-mumab can be given to children over 2 years, who have chronic, non-infectious, anterior uveitis. Children with panuveitis received the therapy by the help of a pediatric rheumatologist.Conclusion: The significance of pediatric uveitis and its therapy is emergent. Our aim was to preserve vision and de-crease the possibilities of side effects and to provide a better life for these uveitic children. Early diagnosis, adequate therapy and regular ophthalmological check-ups are important. Children treated with adalimumab have good visual acuity due to the effectiveness of the therapy. No new ocular side effect was detected at the children treated with adalimumab. [\hyperlink{Alvimopan}{PMID: 36030422}, Lilla Smeller et al., 2022]

\hypertarget{pmid_19535212}{A}mlodipine is a long-acting calcium channel blocker capable of producing hypotension and dysrhythmia in overdose. The toxic doses of amlodipine in children are unclear. The purposes of this study were to describe amlodipine poisoning in children and to determine whether a dose-response relationship could be detected in this population using standardized call data from United States (US) poison centers. 1251 amlodipine-only ingestions in children < 6 years of age were reviewed. Cases with doses coded as "Exact" or "Estimated" and with dose, age, and medical outcome were analyzed (n = 678). Ingestions reported as a "taste or lick" (n = 53) were included as a dose of 1/10 of the dosage form involved. A clinically important response was defined as bradycardia, hypotension, dysrhythmia, conduction disturbance, or hyperglycemia. The risk of such responses was examined over four dosage intervals (< 2.5 mg, 2.5-5 mg, 5.1-10 mg, and > 10 mg). The median estimated dose ingested was 5 mg (range 0.25-200 mg). Clinically important responses developed in 27 patients (3.98\%), and the prevalence of such response significantly increased from 0\% for the lowest to 11.1\% for the highest dose interval (p = 0.001). The smallest dose to produce a clinically important response was 2.5 mg (0.15 mg/kg). Children who ingested > 10 mg were 4.4 times more likely to develop clinically important responses than those ingesting < or = 5 mg. Hypotension may occur in children with amlodipine doses as low as 2.5 mg. The National Poison Data System might provide useful insights regarding dose-response. [\hyperlink{Alvimopan}{PMID: 19535212}, Blaine E Benson et al., 2010]

\hypertarget{pmid_25328089}{T}his retrospective review provides preliminary data regarding the safety and efficacy of olanzapine for chemotherapy-induced vomiting (CIV) control in children. Children <18 years old who received olanzapine for acute chemotherapy-induced nausea and vomiting (CINV) control from December 2010 to August 2013 at four institutions were identified. Patient characteristics, chemotherapy, antiemetic prophylaxis, olanzapine dosing, CIV control, liver function test results and adverse events were abstracted from the health record. Toxicity was graded using CTCAEv4.03. Sixty children (median age 13.2 years; range: 3.10-17.96) received olanzapine during 158 chemotherapy blocks. Olanzapine was most often (59\%) initiated due to a history of poorly controlled CINV. The mean initial olanzapine dose was 0.1 mg/kg/dose (range: 0.026-0.256). Most children who received olanzapine beginning on the first day of the chemotherapy block experienced complete CIV control throughout the acute phase (83/128; 65\%). There was no association between the olanzapine dose/kg and complete CIV control (OR 1.01; 95\% CI: 0.999-1.020; P = 0.091). Sedation was reported in 7\% of chemotherapy blocks and was significantly associated with increasing olanzapine dose (OR: 1.17; 95\% CI: 1.08-1.27; P = 0.0001). Of the 25 chemotherapy blocks where ALT and/or AST were reported more than once, grade 1-3 elevations were observed in five. The mean weight change in 31 children who received olanzapine during more than one chemotherapy block was 0\% (range: -22 to +18). Olanzapine may be an important option to improve CIV control in children. Prospective controlled evaluation of olanzapine for CINV prophylaxis in children is warranted. [\hyperlink{Alvimopan}{PMID: 25328089}, Jacqueline Flank et al., 2015]

\hypertarget{pmid_12690278}{T}o evaluate the effectiveness of oral amoxicillin/clavulanate (25 mg/kg every 12 h) for prevention of fever and/or infection in neutropenic children with cancer. Multicenter, prospective, randomized, double blind placebo-controlled trial. In the intention-to-treat analysis, amoxicillin/clavulanate had a 12\% benefit increase in terms of reduction in the incidence of febrile or infectious episodes, compared with placebo [44 of 83 (53\%) vs.55 of 84 (65\%); 95\% confidence interval, -28\% to +3\%; P = 0.101]. This benefit was also associated with a 30\% increase in the probability of failure-free survival at Day 15 (P = 0.138). A logistic regression analysis showed the effect of prophylaxis to be relevant, especially in patients with leukemia or lymphoma and in those not receiving hematopoietic growth factors, with 17 and 15\% absolute benefit increases (logistic P = 0.014 and 0.034, respectively). Compliance with oral drugs was good, with very few and nonsevere drug-related adverse events. In this study amoxicillin/clavulanate was associated with a detectable clinical effect in the reduction of fever and infection in neutropenic children with cancer, especially those with acute leukemia and not receiving growth factors; the study was not powered to demonstrate a statistically significant effect in the overall patient population. [\hyperlink{Alvimopan}{PMID: 12690278}, Elio Castagnola et al., 2003]

\hypertarget{pmid_20889882}{A}rtemether-lumefantrine (AL) and dihydroartemisinin-piperaquine (DP) are highly efficacious antimalarial therapies in Africa. However, there are limited data regarding the tolerability of these drugs in young children. We used data from a randomized control trial in rural Uganda to compare the risk of early vomiting (within one hour of dosing) for children 6-24 months of age randomized to receive DP (n = 240) or AL (n = 228) for treatment of uncomplicated malaria. Overall, DP was associated with a higher risk of early vomiting than AL (15.1\% versus 7.1\%; P = 0.007). The increased risk of early vomiting with DP was only present among breastfeeding children (relative risk [RR] = 3.35, P = 0.001) compared with children who were not breastfeeding (RR = 1.03, P = 0.94). Age less than 18 months was a risk factor for early vomiting independent of treatment (RR = 3.27, P = 0.02). Our findings indicate that AL may be better tolerated than DP among young breastfeeding children treated for uncomplicated malaria. [\hyperlink{Alvimopan}{PMID: 20889882}, Darren Creek et al., 2010]

\hypertarget{pmid_20401256}{A}cetaminophen has become the non-narcotic of choice for children because of concerns regarding the connection between acetylsalicylic acid exposure and Reye's syndrome. Ibuprofen, recently granted over-the-counter status for children over two years of age, offers another choice for treatment. The efficacy and safety of both drugs have been studied in numerous clinical trials. This paper reviews the published evidence about the efficacy and safety of acetaminophen and ibuprofen with regard to treating fever and mild to moderate pain in children. [\hyperlink{Alvimopan}{PMID: 20401256}, H N McCullough et al., 1998]

\hypertarget{pmid_25972500}{T}his systematic review aimed to assess the safety and efficacy of antiretroviral options for postexposure prophylaxis (PEP). Recognizing the limited data on the safety and efficacy of antiretroviral drugs for PEP in children, this review was extended to include consideration of data on the use of antiretroviral drugs for treatment of infants and children living with human immunodeficiency virus. The PEP literature was assessed to identify studies reporting safety and completion rates for children given PEP, and this information was complemented by safety and efficacy data for drugs used in antiretroviral therapy. The proportion of patients experiencing each outcome was calculated and data were pooled using random-effects meta-analysis. Three prospective cohort studies reported outcomes of children given zidovudine (ZDV) plus lamivudine (3TC) as a 2-drug PEP regimen. The proportion of children completing the full 28-day course of PEP was 64.0\% (95\% confidence interval [CI], 41.2\%-86.8\%), whereas the proportion discontinuing due to adverse events was 4.5\% (95\% CI, .4\%-8.6\%). One randomized trial compared abacavir (ABC) plus lamivudine (3TC) and ZDV+3TC as part of a dual or triple first-line antiretroviral therapy regimen; this study showed better efficacy in the ABC-containing combinations and no difference in the time to first serious adverse event. Three randomized trials compared lopinavir/ritonavir (LPV/r) to nevirapine (NVP) for antiretroviral therapy and showed a lower risk of treatment discontinuations associated with LPV/r vs NVP (hazard ratio, 0.56 [95\% CI, .41-.75]) but no difference in drug-related adverse events. The overall quality of the evidence was rated as very low. This review supports ZDV+3TC+LPV/r as the preferred 3-drug regimen for PEP in children. [\hyperlink{Alvimopan}{PMID: 25972500}, Martina Penazzato et al., 2015]

\hypertarget{pmid_26710331}{T}olvaptan, a vasopressin V2-receptor antagonist, has been reported to improve congestion in adult patients with heart failure. However, it has not been fully clarified whether tolvaptan is also effective and safe for pediatric patients as well as adult. This trial was a multicenter, retrospective, observational study, and was led by the Japanese Society of PEdiatric Circulation and Hemodynamics (J-SPECH). Thirty-four pediatric patients who received tolvaptan to treat congestive heart failure were enrolled in this study. An increment in the urinary volume and decrease in the body weight from baseline were significant at day 1 (+106.7 ± 241.5\%, p = 0.008 and -2.30 ± 4.17\%, p = 0.01), day 3 (+113.5 ± 261.9\%, p = 0.02 and -2.30 ± 4.17\%, p = 0.01), week 1 (+56.3 ± 163.5\%, p = 0.01 and -1.55 ± 4.09\%, p = 0.03) and month 1 (+91.1 ± 171.6\%, p = 0.01 and -2.95 ± 5.98, p = 0.03). The significant predictive factors in responders, who was defined as patients who achieved an increase in the urinary volume at day 1, were older age (p = 0.03), larger body weight before exacerbation (p = 0.04), higher weight at one day before the first administration of tolvaptan (p = 0.03), higher aspartate aminotransferase levels (p = 0.03) and higher urinary osmolality levels (p = 0.03). A logistic regression analysis showed that the urinary osmolality was the only significant predictive factor for responders to tolvaptan. Adverse drug reactions were observed in 7 patients (20.6\%). Six patients had thirst and a dry month, and 1 had a mild increase in the alanine aminotransferase and aspartate aminotransferase. Tolvaptan can be effectively and safely administered in pediatric patients. Because the kidneys in neonates and infants are resistant to arginine vasopressin, the efficacy of tolvaptan may be less effective compared to older children. [\hyperlink{Alvimopan}{PMID: 26710331}, Kouji Higashi et al., 2016]

\hypertarget{pmid_17561929}{T}here are more than 40 H(1)-antihistamines available worldwide. Most of these medications have never been optimally studied in prospective, randomized, double-masked, placebo-controlled trials in children. The aim was to perform a long-term study of levocetirizine safety in young atopic children. In the randomized, double-masked Early Prevention of Asthma in Atopic Children Study, 510 atopic children who were age 12-24 months at entry received either levocetirizine 0.125 mg/kg or placebo twice daily for 18 months. Safety was assessed by: reporting of adverse events, numbers of children discontinuing the study because of adverse events, height and body mass measurements, assessment of developmental milestones, and hematology and biochemistry tests. The population evaluated for safety consisted of 255 children given levocetirizine and 255 children given placebo. The treatment groups were similar demographically, and with regard to number of children with: one or more adverse events (levocetirizine, 96.9\%; placebo, 95.7\%); serious adverse events (levocetirizine, 12.2\%; placebo, 14.5\%); medication-attributed adverse events (levocetirizine, 5.1\%; placebo, 6.3\%); and adverse events that led to permanent discontinuation of study medication (levocetirizine, 2.0\%; placebo, 1.2\%). The most frequent adverse events related to: upper respiratory tract infections, transient gastroenteritis symptoms, or exacerbations of allergic diseases. There were no significant differences between the treatment groups in height, mass, attainment of developmental milestones, and hematology and biochemistry tests. The long-term safety of levocetirizine has been confirmed in young atopic children. [\hyperlink{Alvimopan}{PMID: 17561929}, F Estelle R Simons et al., 2007]

\hypertarget{pmid_23388069}{T}olvaptan, an oral vasopressin receptor antagonist, was administered to a 6-year-old boy with advanced congestive cardiac failure due to restrictive cardiomyopathy. Slow up-titration of the tolvaptan improved the loop diuretic-resistant congestive cardiac failure without hypernatraemia, deterioration of vital signs, and significant complications. Tolvaptan could be useful and safe for children with severe congestive cardiac failure. It is necessary to accumulate clinical data on tolvaptan administration in children in order to determine the optimal method of administration.  [\hyperlink{Alvimopan}{PMID: 23388069}, Yoko Horibata et al., 2014]

\section*{Cevimeline Hydrochloride}
\subsection*{Result}
\subsubsection*{Answer}

Unknown

\subsubsection*{{Explanation}}
\hypertarget{Cevimeline Hydrochloride}
After reviewing all available abstracts, only two are directly about Cevimeline Hydrochloride: [\hyperlink{pmid_12038948}{PMID: 12038948}, Rose S Fife et al., 2002] and [\hyperlink{pmid_17899308}{PMID: 17899308}, K C M Leung et al., 2008]. Both studies focus on adult populations with Sjögren's syndrome and do not include children or adolescents. There are no abstracts describing targeted safety studies of Cevimeline Hydrochloride in children of any age range. Therefore, based on the abstracts provided, the safety of Cevimeline Hydrochloride in children is unknown.

\subsection*{Abstracts}
\hypertarget{pmid_12038948}{C}evimeline hydrochloride is a cholinergic agent with muscarinic agonist activity prominently affecting the M1 and M3 receptors prevalent in exocrine glands. We evaluated the safety and efficacy of cevimeline in the treatment of xerostomia in patients with Sjögren syndrome. Seventy-five patients with Sjögren syndrome and associated salivary gland dysfunction were enrolled in a double-blind, randomized, placebo-controlled trial at 8 university- and office-based outpatient clinical facilities in the United States. Eligible study participants were randomized to receive 30 mg of cevimeline 3 times daily, 60 mg of cevimeline 3 times daily, or placebo for 6 weeks. Subjective responses were determined using global patient evaluation and visual analog scales. Salivary flow was measured objectively. Sixty-one participants completed the study. Patients in both cevimeline groups had significant improvements in dry mouth, as indicated by symptoms, salivary flow, and use of artificial saliva, compared with the placebo group. The drug was generally well tolerated, with expected adverse events resulting from the drug's muscarinic agonist action. Fourteen patients withdrew from the study because of adverse events, the most frequent being nausea. Therapy with cevimeline, 30 mg 3 times daily, seems to be well tolerated and to provide substantive relief of xerostomia symptoms. Although both dosages of cevimeline provided symptomatic improvement, 60 mg 3 times daily was associated with an increase in the occurrence of adverse events, particularly gastrointestinal tract disorders. Use of 30 mg of cevimeline provides a new option for the treatment of xerostomia in Sjögren syndrome. [\hyperlink{Cevimeline Hydrochloride}{PMID: 12038948}, Rose S Fife et al., 2002]

\hypertarget{pmid_28827252}{C}eftriaxone is widely used in children in the treatment of sepsis. However, concerns have been raised about the safety of ceftriaxone, especially in young children. The aim of this review is to systematically evaluate the safety of ceftriaxone in children of all age groups. MEDLINE, PubMed, Cochrane Central Register of Controlled Trials, EMBASE, CINAHL, International Pharmaceutical Abstracts and adverse drug reaction (ADR) monitoring systems will be systematically searched for randomised controlled trials (RCTs), cohort studies, case-control studies, cross-sectional studies, case series and case reports evaluating the safety of ceftriaxone in children. The Cochrane risk of bias tool, Newcastle-Ottawa and quality assessment tools developed by the National Institutes of Health will be used for quality assessment. Meta-analysis of the incidence of ADRs from RCTs and prospective studies will be done. Subgroup analyses will be performed for age and dosage regimen. Formal ethical approval is not required as no primary data are collected. This systematic review will be disseminated through a peer-reviewed publication and at conference meetings. CRD42017055428. [\hyperlink{Cevimeline Hydrochloride}{PMID: 28827252}, Linan Zeng et al., 2017]

\hypertarget{pmid_28741653}{C}hloral hydrate is commonly used to sedate children for painless procedures. Children may recover more quickly after sedation with dexmedetomidine, which has a shorter half-life. We randomly allocated 196 children to chloral hydrate syrup 50 mg.kg [\hyperlink{Cevimeline Hydrochloride}{PMID: 28741653}, V M Yuen et al., 2017] To examine whether three cycles of a low-intensity chemotherapy consisting of cyclophosphamide [500 mg/m(2) - day 1], vinblastine [6 mg/m(2) - days 1 and 8] and prednisolone [40 mg/m(2) - days 1-7] (CVP) is safe and therapeutically effective in children and adolescents with early stage nodular lymphocyte predominant Hodgkin lymphoma [nLPHL]. Fifty-five children and adolescents with early stage nLPHL [median age 13 years, range 4-17 years] diagnosed between June 2005 and October 2010 in the UK and France are the subjects of this report. Staging investigations included conventional cross sectional as well as 18 fluro-deoxyglucose [FDG] PET imaging. Histology was confirmed as nLPHL by an expert pathology panel. Of the 45 patients, who received CVP as first line treatment, 36 [80\%, 95\% Confidence Interval [CI]: (68; 92)] either achieved a complete remission [CR] or CR unconfirmed [CRu], the remaining nine patients achieved a partial response. All nine subsequently achieved CR with salvage chemotherapy [n=7] or radiotherapy [n=2]. Ten patients received CVP at relapse after primary treatment that consisted of surgery alone and all achieved CR. To date, only three patients have relapsed after CVP chemotherapy and all had received CVP as first line treatment at initial diagnosis. The 40-month freedom from treatment failure and overall survival for the entire cohort were 75.4\% (SE ± 6\%) and 100\%, respectively. No significant early toxicity was observed. Our results show that CVP is an effective chemotherapy regimen in children and adolescents with early stage nLPHL that is well tolerated with minimal acute toxicity. [\hyperlink{Cevimeline Hydrochloride}{PMID: 28741653}, Ananth Shankar et al., 2012]

\hypertarget{pmid_28275979}{S}edation is often required for children undergoing diagnostic procedures. Chloral hydrate has been one of the sedative drugs most used in children over the last 3 decades, with supporting evidence for its efficacy and safety. Recently, chloral hydrate was banned in Italy and France, in consideration of evidence of its carcinogenicity and genotoxicity. Dexmedetomidine is a sedative with unique properties that has been increasingly used for procedural sedation in children. Several studies demonstrated its efficacy and safety for sedation in non-painful diagnostic procedures. Dexmedetomidine's impact on respiratory drive and airway patency and tone is much less when compared to the majority of other sedative agents. Administration via the intranasal route allows satisfactory procedural success rates. Studies that specifically compared intranasal dexmedetomidine and chloral hydrate for children undergoing non-painful procedures showed that dexmedetomidine was as effective as and safer than chloral hydrate. For these reasons, we suggest that intranasal dexmedetomidine could be a suitable alternative to chloral hydrate. [\hyperlink{Cevimeline Hydrochloride}{PMID: 28275979}, Giorgio Cozzi et al., 2017]

\hypertarget{pmid_25246305}{T}he aim of this study was to compare the efficacy and safety of different oral chloral hydrate and dexmedetomidine doses used for sedation during electroencephalography (EEG) in children. One hundred sixty children aged 1 to 9 years with American Society of Anesthesiologists physical status I-II who were uncooperative during EEG recording or who were referred to our electrodiagnostic unit for sleep EEG were included to the study. The patients were randomly assigned into 4 groups. In groups D1 and D2, patients received oral dexmedetomidine doses of 2 and 3 µg/kg, respectively. In group C1 and C2, patients received oral chloral hydrate doses of 50 and 100 mg/kg, respectively. The induction time was significantly shorter in group C2 compared with other groups (P = .000). The rate of adverse effects was significantly higher in group C2 compared with the dexmedetomidine groups (D1 and D2; P = .004). In conclusion, dexmedetomidine can be used safely for sedation during EEG in children.  [\hyperlink{Cevimeline Hydrochloride}{PMID: 25246305}, Hakan Gumus et al., 2015] Chloral hydrate is the most commonly used sedative for paediatric diagnostic procedures in China with a success rate of around 80\%. Intranasal dexmedetomidine is used for rescue sedation in our centre. This prospective investigation evaluated 213 children aged one month to 10 years who were not adequately sedated following administration of chloral hydrate. Children were randomly assigned to receive rescue intranasal dexmedetomidine at 1 μg.kg(-1) (group 1), 1.5 μg.kg(-1) (group 2) or 2 μg.kg(-1) (group 3). The sedation level was assessed every 10 min using a modified observer's assessment of alertness/sedation scale. Successful rescue sedation in groups 1, 2 and 3 were 56 (83.6\%), 66 (89.2\%) and 51 (96.2\%), respectively. Increasing the rescue dose was associated with an increased success rate with an odds ratio of 4.12 (95\% CI 1.13-14.98), p = 0.032. We conclude that intranasal dexmedetomidine is effective for sedation in children who do not respond to chloral hydrate.  [\hyperlink{Cevimeline Hydrochloride}{PMID: 25246305}, B L Li et al., 2014] A clinical trial of ceftizoxime suppositories (CZX-S) was performed to evaluate the therapeutic effectiveness in children with bacterial infection. The subjects were 10 children comprising 4 with pneumonia, 3 with lacunar tonsillitis, 2 with pharyngitis, and 1 with UTI. They were given 1 suppository containing either 125 mg or 250 mg of CZX 2 to 4 times a day. The daily per kg body weight dose ranged from 17.1 to 60.0 mg. The result was "markedly effective" in 3, "effective" in 6, and "failure" was recorded in 1. Bacteriologically, successful eradication of causative organisms was confirmed in all the 4 children who underwent the test. No clinical side effects were observed. The only laboratory test abnormality recorded in a single patient was eosinophilia, which was not definitely ascribable to CZX-S. In conclusion, CZX-S have proved to be a clinically safe and effective antibiotic preparation in infantile infection, even in children whose treatment with conventional antibiotics is associated with difficulties. [\hyperlink{Cevimeline Hydrochloride}{PMID: 25246305}, T Hosoda et al., 1985]

\hypertarget{pmid_34834338}{C}efixime (CEF) is a cephalosporin included in the WHO Model List of Essential Medicines for Children. Liquid formulations are considered the best choice for pediatric use, due to their great ease of administration and dose-adaptability. Owing to its very low aqueous solubility and poor stability, CEF is only available as a powder for oral suspensions, which can lead to reduced compliance by children, due to its unpleasant texture and taste, and possible non-homogeneous dosage. The aim of this work was to develop an oral pediatric CEF solution endowed with good palatability, exploiting the solubilizing and taste-masking properties of cyclodextrins (CDs), joined to the use of amino acids as an auxiliary third component. Solubility studies indicated sulfobutylether-β-cyclodextrin (SBEβCD) and Histidine (His) as the most effective CD and amino acid, respectively, even though no synergistic effect on drug solubility improvement by their combined use was found. Molecular Dynamic and  [\hyperlink{Cevimeline Hydrochloride}{PMID: 34834338}, Marzia Cirri et al., 2021] Cefamandole, a new cephalosporin antibiotic, has greater activity against common pathogens, including Escherichia coli, Haemophilus influenzae, and Proteus (including indole-positive strains), than available cephalosporin drugs. We have evaluated the safety and pharmacokinetics of this drug in 30 infants and children. Blood levels and urinary excretion of the drug were similar to those previously found in adults. The only side effects were mild and transient elevation of serum glutamic oxalacetic transaminase in 12 patients and of blood urea nitrogen in 1 patient in whom serum creatinine remained normal and unchanged. [\hyperlink{Cevimeline Hydrochloride}{PMID: 34834338}, C T Chang et al., 1978]

\hypertarget{pmid_2041160}{P}harmacokinetics and clinical effects of cefpirome (CPR, HR 810) in children were studied. When 20 mg/kg and 40 mg/kg doses of CPR were administered to 4 children through 30 minutes' drip infusion, half-lives were 1.23 +/- 0.23 (mean +/- S.D.) hours and 1.37 +/- 0.35 (mean +/- S.D.) hours, respectively for the 2 dose levels, and recovery rates in urine in the first 6 hours after administration were 74.8\% and 56.1\%, respectively. CPR was administered to 15 cases (3 tonsillitis, 3 bronchitis, 5 bronchopneumonia, 1 acute cystitis, 1 coxoiliatitis, 1 otitis media, 1 otitis externa). The efficacy rate was 86.7\%. Seven strains of bacteria were isolated and identified 4 Haemophilus influenzae, 3 Staphylococcus aureus, 1 Pseudomonas sp. from these cases. These bacteria in children were followed after administration of CPR. Six strains were eradicated and one was reduced in number. No adverse effects of CPR were observed except in 2 cases, one of which showed transient eosinophilia and the other showed a transient increase of transaminase. These results suggest that CPR may be an effective and safe drug to use on children clinically. [\hyperlink{Cevimeline Hydrochloride}{PMID: 2041160}, T Ihara et al., 1991]

\hypertarget{pmid_6306289}{T}he present study was performed to evaluate the clinical effectiveness and safety of cefmenoxime (CMX), a new cephalosporin antibiotic for injection in the field of pediatrics. Thirty-one cases, including 2 cases with sepsis, 18 cases with respiratory tract infections and 7 cases with urinary tract infections, were given CMX at daily doses of 30 mg/kg to 125 mg/kg divided into 3 or 4 for 3 days to 13 days. Clinical responses were excellent in 16 cases, good in 9 cases and poor in 6 cases, the satisfactory response being 80.6\%. No side effects and no abnormal laboratory findings relating to the drug were observed. [\hyperlink{Cevimeline Hydrochloride}{PMID: 6306289}, M Takimoto et al., 1982]

\hypertarget{pmid_6655838}{C}efpiramide (CPM), a new broad-spectrum cephalosporin antibiotic with good antipseudomonas activities, was evaluated for its safety and efficacy in 20 children with bacterial infections. The diagnoses of the patients included pneumonia (10), acute bronchitis (1), streptococcal pharyngitis (1), purulent cervical lymphadenitis (1), urinary tract infections (2), acute enterocolitis (1), infections in agranulocytosis and acute leukemia (2), and acute purulent meningitis (2). Of the 20 patients, 17 were cured by the CPM therapy. The main etiologic pathogens were H. influenzae, P. aeruginosa, P. fluorescens, S. pneumoniae and E. coli. The serum half-life of CPM was 2.4 to 4.1 hours after an intravenous bolus injection. As an adverse reaction, diarrhea was encountered in 4 cases, and 1 of them experienced severe watery diarrhea with significant fecal colonization of K. oxytoca. The data suggest that CPM is an effective antibiotic when used in children with susceptible bacterial infections. Administrations divided in 2 to 3 dosages will be enough to maintain effective serum levels. [\hyperlink{Cevimeline Hydrochloride}{PMID: 6655838}, H Meguro et al., 1983]

\hypertarget{pmid_1880934}{L}aboratory and clinical studies on cefpirome (CPR, HR 810), a newly developed cephem antibiotic, were performed. The results obtained are summarized as follows: 1. Absorption and elimination of the drug were examined in a total of 7 children including 3 cases of administered with 20 mg/kg intravenous bolus injection (i.v.), 2 cases with 20 mg/kg drip infusion (d.i.v.) for 60 minutes and 2 cases with 40 mg/kg (d.i.v.) for 60 minutes. Maximum serum levels were attained immediately after i.v. or d.i.v. Cmax's were 233 +/- 7.6, 88.5 +/- 14.5, and 116 +/- 15 micrograms/ml, respectively for the above 3 modes of administration. These values were determined using a bioassay method with Bacillus subtilis ATCC 6633. T 1/2 (beta)'s were 1.18 +/- 0.17, 1.61 +/- 0.28 and 2.68 +/- 0.83 hours, respectively. Cumulative urinary recovery rates were 40.2-69.8\% in a period of 0-6 hours after admissions. 2. Clinical efficacies were evaluated in a total of 20 patients with ages ranging from 9 months to 11 years. The treated cases were 6 cases of acute pneumonia, 4 cases of acute bronchitis, 4 cases of acute purulent tonsillitis, 2 cases of acute urinary tract infections, 2 cases of cellulitis, 1 case of purulent lympadenitis and 1 case of acute otitis media. The clinical efficacy rate was 94.7\%. Adverse reactions occurred in no patients. Abnormal changes in laboratory test values involved only 1 case with elevated GOT and GPT. CPR was considered to be a safe and useful drug in treating various infectious diseases in children. [\hyperlink{Cevimeline Hydrochloride}{PMID: 1880934}, K Nagano et al., 1991]

\hypertarget{pmid_10496153}{A}n open-labeled and randomized trial was conducted to compare the efficacy and safety of once daily cefpodoxime proxetil suspension (10mg/kg/day) and thrice daily cefaclor (45mg/kg/day) in the treatment of acute otitis media in children. A total of 57 children aged from 6 months to 9 years were enrolled; 23 were treated with cefpodoxime and 34 with cefaclor. Satisfactory clinical outcome, either cure or improvement, was achieved at the end of treatment in 90\% of patients in the cefaclor group and 95\% of patients in the cefpodoxime group (p > 0.05). Clinical recurrence was identified at the follow-up visits in one case of the cefaclor group (3\%), and none in the cefpodoxime group (p > 0.05). These drugs were well tolerated by 14/21 (67\%) in the cefpodoxime-treated group and 27/32 (84\%) in the cefaclor-treated group. The incidence of adverse events was slightly higher in the cefpodoxime group than in the cefaclor group, however the difference did not reach statistical significance (p > 0.05). The daily cost of once-daily cefpodoxime was lower than that of thrice-daily cefaclor. We conclude that cefpodoxime administered once daily is as effective and safe as cefaclor administered thrice daily in the treatment of acute otitis media in children. The less dosing frequency and lower daily price of cefpodoxime provide additional benefits. [\hyperlink{Cevimeline Hydrochloride}{PMID: 10496153}, H Y Tsai et al., 1998]

\hypertarget{pmid_17611334}{T}o observe the effect of sevoflurane on the induction and maintenance of anaesthesia in children, and to evaluate its safety and effectiveness. Forty child patients who conformed to the selection standard were operated under anaesthesia with intubation.Without premedicant, all the patients inhaled 100\% oxygen(1L/min) and sevoflurane by mask, and escalated the concentration of sevoflurane (to the maximum concentration 7\%) until the lash reflex disappeared, and the maintenance concentration was controlled under 4\%. All the patients were intubated, together with vecuronium 0.1mg/kg. With little tract excretion, the achievement ratio of induction by sevoflurane was 100\%, and the children tolerated well. With stable hemodynajmics,1\% approximately 4.0\% maintenance concentration of sevoflurane during the operation showed effective anaesthesia, no decreased heart rate or blood pressure appeared, and all the patients' body temperature was normal. Sevoflurane for children induction can bring fewer stimuli in the respiratory tract,less cardiac vascular inhibition and palinesthesia time. Anaesthesia in children induced by sevoflurane is safe and effective. [\hyperlink{Cevimeline Hydrochloride}{PMID: 17611334}, Xi-ying Zhang et al., 2007]

\hypertarget{pmid_6330022}{T}he clinical efficacy and safety of ceftriaxone, a long half-life cephalosporin were evaluated in 48 children with a variety of serious bacterial infections. Clinical cure was achieved in 92\% (44 of 48) of patients. Peak serum bactericidal titres for Haemophilus influenzae type b, Streptococcus pneumoniae, Str. pyogenes and Escherichia coli were greater than or equal to 1:1024. Mean peak and trough ceftriaxone levels were 173 and 42 mg/l, respectively. Mild and transient diarrhoea was observed in 10\% of patients. Laboratory side effects encountered were eosinophilia, thrombocytosis and neutropenia in another 8\%. Ceftriaxone is a useful antibiotic for common childhood infections. Its prolonged half-life allows twice daily administration which reduces problems related to intravenous therapy as well as the cost and personnel time. [\hyperlink{Cevimeline Hydrochloride}{PMID: 6330022}, T Chonmaitree et al., 1984]

\hypertarget{pmid_18611612}{T}he safety and efficacy of cefetamet pivoxil, an oral cephalosporin of the third generation, have been studied in open, prospective, randomized comparative, clinical trials including 301 toddlers (children aged 1 to 2 years) with upper and lower respiratory tract infections, and urinary tract infections. Cefetamet pivoxil (CAT) syrup formulation was given to 177 toddlers either in the standard dose of 10 mg/kg b.i.d. [n = 116] or 20 mg/kg b.i.d. [n = 61] and 124 toddlers have been treated with comparator drugs [cefaclor, n = 98; phenoxymethylpenicillin, n = 18; amoxicillin plus clavulanic acid; n = 8]. The treatment period was 7 days mainly, except for pharyngotonsillitis for which the treatment duration was 7 or 10 days. The assessment of treatment was based on clinical signs and symptoms primarily in infections of lower respiratory tract and acute otitis media, whereas in patients with pharyngotonsillitis and acute urinary tract infections the bacteriological findings were the main evaluation criteria. The overall therapeutic outcome was successful in 148 (95.4\%) of the 155 toddlers to whom CAT was administered and in 87 (85.3\%) out of 102 toddlers receiving standard drugs. Adverse events of mild to moderate severity, mainly of gastro-intestinal type (vomiting or diarrhoea) occurred in 14.7\% in the patient group receiving CAT, 11.2\% in the toddlers receiving the standard dose of CAT, and in 12.9\% with the comparator drugs. From the data presented it is concluded that cefetamet pivoxil is efficient and safe in toddlers presenting with community-acquired respiratory and urinary infections mainly caused by S. pneumoniae, H. influenzae, Group A beta-haemolytic streptococci, M. catarrhalis, E. coli, Proteus spp. and K. pneumoniae. [\hyperlink{Cevimeline Hydrochloride}{PMID: 18611612}, A Chibante et al., 1994]

\hypertarget{pmid_501918}{I}n order to evaluate efficacy and safety, cefamandole, a new cephalosporin, was given intravenously to 12 children with respiratory tract infections (11 cases) and urinary tract infection (1 case), who ranged in age from 2 months to 5 years old. Cefamandole sodium was administered 74 approximately 112 mg/kg/day in three or four equally divided doses by one-shot injection. The overall efficacy rate was 83.3\% in 12 cases, i.e., good in 8, fairly good in 2, and poor in 2 cases. No adverse reaction was noted on any of our 12 cases. [\hyperlink{Cevimeline Hydrochloride}{PMID: 501918}, T Ichioka et al., 1979]

\hypertarget{pmid_17899308}{C}evimeline hydrochloride, a specific agonist of the M3 muscarinic receptor, is beneficial in the treatment of symptoms of xerostomia and xerophthalmia associated with Sjögren's syndrome (SS). Cevimeline has not been evaluated in southern Chinese patients. Furthermore, the effects of cevimeline on health-related quality of life and oral health status are not known. In this randomised, double-blind, placebo-controlled crossover study, patients received cevimeline 30 mg or matched placebo three times per day over 10 weeks followed by a 4-week washout period before treatment crossover. Participants self-completed the following questionnaires: Xerostomia Inventory (XI), the General Oral Health Assessment Index (GOHAI), the Ocular Surface Disease Index (OSDI) and the Medical Outcomes Short Form (SF-36). Clinical assessments included sialometry, examination of the oral cavity for the degree of xerostomia and dental complications of xerostomia. Fifty patients (22 primary SS and 28 secondary SS) were enrolled in the trial. Forty-four patients completed the study. There was a significant improvement in the XI and GOHAI scores as well as the objective rating of xerostomic signs of the oral cavity after treatment with cevimeline. However, there was no improvement in salivary flow rates and dry eye symptoms. SS patients had lower SF-36 scores, but these did not improve after treatment with cevimeline. [\hyperlink{Cevimeline Hydrochloride}{PMID: 17899308}, K C M Leung et al., 2008]

\hypertarget{pmid_2402648}{C}hloral hydrate has been used extensively to sedate children, but at Brooke Army Medical Center, other drug combinations were becoming increasingly popular due to a perception that chloral hydrate had a high rate of failure, especially with younger or neurologically impaired children. Therefore, 50 children were given the drug before a diagnostic study, and patient data and a sedation score were recorded on a worksheet. Of 50 children, 43 (86\%) were "successfully sedated" on the first attempt with no side effects. Children with neurologic disorders had a much greater (27\% vs 4\%) failure rate than "normal" children. The sedation rate did not significantly differ by age, sex, or initial drug dosage. The study suggest that chloral hydrate is a safe and effective oral sedative but that children with neurologic disorders may need alternative drugs for sedation. [\hyperlink{Cevimeline Hydrochloride}{PMID: 2402648}, P D Rumm et al., 1990]

\hypertarget{pmid_28414899}{P}ediatric ophthalmic examinations can be conducted under sedation either by chloral hydrate or by dexmedetomidine. The objective was to compare the success rates and quality of ophthalmic examination of children sedated by intranasal dexmedetomidine vs oral chloral hydrate. One hundred and forty-one children aged from 3 to 36 months (5-15 kg) scheduled to ophthalmic examinations were randomly sedated by either intranasal dexmedetomidine (2 μg·kg Sixty-one children were sedated by dexmedetomidine with a success rate of 85.9\%, which is significantly higher than that by chloral hydrate (64.3\%) [OR 3.39, 95\% CI: 1.48-7.76, P = 0.003]. Furthermore, children in the dexmedetomidine group displayed better eye position in anterior segment analysis than in chloral hydrate group median difference. All children displayed stable hemodynamics and none suffered hypoxemia in both groups. Oral chloral hydrate induced higher percentages of vomiting and altered bowel habit after discharge than dexmedetomidine. Intranasal dexmedetomidine provides more successful sedation and better quality of ophthalmic examinations than oral chloral hydrate for small children. [\hyperlink{Cevimeline Hydrochloride}{PMID: 28414899}, Qianzhong Cao et al., 2017]

\hypertarget{pmid_3908729}{A} clinical trial of ceftizoxime suppositories (CZX-S) was conducted in children whose chemotherapy was considered to be best performed in this dosage form at the physician's discretion. The subjects were 5 children with infection, consisting of 2 with pneumonia, 1 with tonsillitis, and 2 with UTI. The results were as follows. The clinical response to CZX-S was "markedly effective" in 3 and "effective" in 2, with the 100\% effectiveness rate. Neither adverse drug reactions nor abnormal laboratory tests were detected. No unwanted expulsion of the suppository occurred. The serum concentration of CZX 30 minutes after the first insertion ranged from 8.38 to 11.4 micrograms/ml, and the urinary concentration of CZX in the 6-hour urine collections, from 23.6 to 290 micrograms/ml. [\hyperlink{Cevimeline Hydrochloride}{PMID: 3908729}, S Furukawa et al., 1985]

\hypertarget{pmid_513298}{H}aving resistance to beta-lactamase-producing strains and showing resistance to not only cephalosporin resistant strains of E. coli and Klebsiella but also to Citrobacter, Proteus and Enterobacter, Cefuroxime (CXM) was used in pediatric field for both fundamental and clinical studies. CXM was found to be a useful antibiotic in views of high clinical efficacy rate obtained and no side effect noted. As for the dose, the single dose of 25 mg/kg achieved sufficient blood levels. Also in view of good clinical effect, the dose of 25 mg/kg three or four times daily seems appropriate for treatment of children. [\hyperlink{Cevimeline Hydrochloride}{PMID: 513298}, M Hotta et al., 1979]

\hypertarget{pmid_28382572}{T}he rate of antibiotic resistance in children continues to rise requiring the use of new antibiotics. Ceftaroline fosamil, a newer-generation cephalosporin, was recently approved for the treatment of acute bacterial skin and skin structure infections and community-acquired bacterial pneumonia in children aged >2 months. Ceftaroline provides coverage against staphylococcal and streptococcal infections, including methicillin-resistant Staphylococcus aureus and penicillin-resistant Streptococcus pneumoniae. Pediatric dosing differs from adult dosing, but it maintains a similar pharmacokinetic profile and offers similar efficacy in terms of time above the minimum inhibitory concentration as compared to the adult population. The clinical safety and efficacy of this antibiotic has been assessed in three pediatric clinical trials that led to its approval by the US Food and Drug Administration, and each trial is described within this review. This article will also discuss the ongoing trials assessing the possibility of expanding the indications of this antibiotic to late-onset sepsis, meningitis and osteomyelitis in the pediatric population. [\hyperlink{Cevimeline Hydrochloride}{PMID: 28382572}, Adam Corey et al., 2017]

\section*{Crenessity}
\subsection*{Result}
\subsubsection*{Answer}
Unknown
\subsubsection*{{Explanation}}
\hypertarget{Crenessity}
After reviewing all available abstracts, there is no mention of "Crenessity" in any of the abstracts provided. Therefore, there are no targeted studies evaluating the safety of Crenessity in children of any age group. As a result, the safety of Crenessity for use in children is unknown based on the abstracts available to me.

\subsection*{Abstracts}
\hypertarget{pmid_22031047}{C}hlorhexidine gluconate (CHG) is a widely used topical antiseptic that is recommended by the Centers for Disease Control and Prevention for skin cleansing before central venous catheter insertion in adults and children. Because of limited safety data, CHG is not recommended for use in children <2 months of age. CHG is, however, frequently used in Neonatal Intensive Care Units across the United States. Here, we will review the safety of CHG use in preterm infants. [\hyperlink{Crenessity}{PMID: 22031047}, A K Chapman et al., 2012]

\hypertarget{pmid_32864651}{A} child with cretinism poses multiple anaesthetic challenges due to the associated mental and physical disability, deranged metabolic and physiologic functions, difficult airway and propensity to perioperative cardiorespiratory complications. Spinal anaesthesia in children is associated with remarkable cardiorespiratory stability and provides complete surgical anaesthesia. Here, we report a case that describes the first successful anaesthetic management of a child who was an unevaluated case of cretinism under subarachnoid block. [\hyperlink{Crenessity}{PMID: 32864651}, Anju Gupta et al., 2020]

\hypertarget{pmid_685312}{P}eracetic acid is more and more used for the purpose of air desinfections in rooms. The examinations of DWORSCHAK and LINDE encouraged us to use peracetic acid  in the rooms of creches in presence of children systematically. The applied concentration of peracetic acid is 4.6 mg/m3. With these examinations it was intended to prove if it is possible to influence the morbidity of acute respiratory diseases. Under the choosed conditions no side effects are observed over a time of twenty weeks. Publications concerning cocarcinogetic activity of peroxy compounds induced us to make a preliminary stop in the application of peracetic acid in presence of children. The number of diseases specially of respiratory diseases in the time of the examinations was very small. That concerns the examination groups and those compared with these. But nevertheless it is to be seen that the morbidity in the groups with application of peracetic acid is 3.5\% and in the others 11.20\%. The differences are significant. If the results concerning cocarcinogetic activity of peroxy compounds will be shure it is to decide whether the room desinfections will be continued in presence of children or not. [\hyperlink{Crenessity}{PMID: 685312}, G Pickroth et al., 1978]

\hypertarget{pmid_37287398}{S}eizures are common in critically ill children and neonates, and these patients would benefit from intravenous (IV) antiseizure medications with few adverse effects. We aimed to assess the safety profile of IV lacosamide (LCM) among children and neonates. This retrospective multicenter cohort study examined the safety of IV LCM use in 686 children and 28 neonates who received care between January 2009 and February 2020. Adverse events (AEs) were attributed to LCM in only 1.5\% (10 of 686) of children, including rash (n = 3, .4\%), somnolence (n = 2, .3\%), and bradycardia, prolonged QT interval, pancreatitis, vomiting, and nystagmus (n = 1, .1\% each). There were no AEs attributed to LCM in the neonates. Across all 714 pediatric patients, treatment-emergent AEs occurring in >1\% of patients included rash, bradycardia, somnolence, tachycardia, vomiting, feeling agitated, cardiac arrest, tachyarrhythmia, low blood pressure, hypertension, decreased appetite, diarrhea, delirium, and gait disturbance. There were no reports of PR interval prolongation or severe cutaneous adverse reactions. When comparing children who received a recommended versus a higher than recommended initial dose of IV LCM, there was a twofold increase in the risk of rash in the higher dose cohort (adjusted incidence rate ratio = 2.11, 95\% confidence interval = 1.02-4.38). This large observational study provides novel evidence demonstrating the tolerability of IV LCM in children and neonates. [\hyperlink{Crenessity}{PMID: 37287398}, Susan L Fong et al., 2023]

\hypertarget{pmid_28275979}{S}edation is often required for children undergoing diagnostic procedures. Chloral hydrate has been one of the sedative drugs most used in children over the last 3 decades, with supporting evidence for its efficacy and safety. Recently, chloral hydrate was banned in Italy and France, in consideration of evidence of its carcinogenicity and genotoxicity. Dexmedetomidine is a sedative with unique properties that has been increasingly used for procedural sedation in children. Several studies demonstrated its efficacy and safety for sedation in non-painful diagnostic procedures. Dexmedetomidine's impact on respiratory drive and airway patency and tone is much less when compared to the majority of other sedative agents. Administration via the intranasal route allows satisfactory procedural success rates. Studies that specifically compared intranasal dexmedetomidine and chloral hydrate for children undergoing non-painful procedures showed that dexmedetomidine was as effective as and safer than chloral hydrate. For these reasons, we suggest that intranasal dexmedetomidine could be a suitable alternative to chloral hydrate. [\hyperlink{Crenessity}{PMID: 28275979}, Giorgio Cozzi et al., 2017]

\hypertarget{pmid_24413032}{D}espite the lack of safety data, chlorhexidine gluconate (CHG) is an antiseptic with broadspectrum coverage often used in neonatal intensive care units (NICUs). Adverse skin reactions, most commonly burns, have been reported after the use of CHG. Preserving skin integrity in preterm infants is vital in the prevention of sepsis, excessive water loss, hypothermia, and renal failure. This is a case report of two incidents of significant skin burning in extremely low birth weight (ELBW) infants who were treated with CHG for the purposes of umbilical cord sterilization prior to umbilical line placement. This case report of burns associated with CHG in one infant weighing 610 g at birth and a second infant weighing 600 g at birth. CHG does have a strong association with causing skin burns in the ELBW population; however, wiping the solution off of the skin seems to reduce injury. [\hyperlink{Crenessity}{PMID: 24413032}, Jennifer Kutsch et al., ]

\hypertarget{pmid_20112608}{C}hloral hydrate is generally considered a safe sedative-hypnotic drug, and is commonly used for sedation of infants and young children before diagnostic procedures. Even chloral hydrate administered within the recommended maximal dose limits can cause serious morbidity and mortality. Here the authors describe a four-month-old girl with a life-threatening central nervous system and respiratory depression after administration of a therapeutic dose of chloral hydrate. The patient gradually recovered with supportive treatment including ventilation therapy. [\hyperlink{Crenessity}{PMID: 20112608}, Emre Ceçen et al., ]

\hypertarget{pmid_24447296}{C}hloral hydrate is the most commonly used sedative for paediatric diagnostic procedures in China with a success rate of around 80\%. Intranasal dexmedetomidine is used for rescue sedation in our centre. This prospective investigation evaluated 213 children aged one month to 10 years who were not adequately sedated following administration of chloral hydrate. Children were randomly assigned to receive rescue intranasal dexmedetomidine at 1 μg.kg(-1) (group 1), 1.5 μg.kg(-1) (group 2) or 2 μg.kg(-1) (group 3). The sedation level was assessed every 10 min using a modified observer's assessment of alertness/sedation scale. Successful rescue sedation in groups 1, 2 and 3 were 56 (83.6\%), 66 (89.2\%) and 51 (96.2\%), respectively. Increasing the rescue dose was associated with an increased success rate with an odds ratio of 4.12 (95\% CI 1.13-14.98), p = 0.032. We conclude that intranasal dexmedetomidine is effective for sedation in children who do not respond to chloral hydrate.  [\hyperlink{Crenessity}{PMID: 24447296}, B L Li et al., 2014] An on-going study to determine the efficacy and safety of ketamine dissociative anesthesia in the emergency department for children less than 10 years of age is presented. Preliminary results of the first 30 cases indicate that this technique may be safe and convenient if used within the appropriate protocol. [\hyperlink{Crenessity}{PMID: 24447296}, R H Dailey et al., 1979]

\hypertarget{pmid_7382013}{W}e report 2 cases of Cushing's syndrome following intralesional triamcinolone acetonide injections of urethral strictures in children. The pharmacology of triamcinolone and its 2 parenteral forms, triamcinolone acetonide and triamcinolone diacetate, is discussed. For children we recommend the short-acting triamcinolone diacetate at 4-week intervals with dosage adjusted to age. In adults either type of triamcinolone may be used but triamcinolone acetonide should be given at 6-week intervals. [\hyperlink{Crenessity}{PMID: 7382013}, R R Augspurger et al., 1980]

\hypertarget{pmid_4008381}{S}erum chloramphenicol levels were evaluated in 52 children with severe infection treated intravenously with chloramphenicol succinate and orally with chloramphenicol palmitate, chloramphenicol monostearoylglycolate or chloramphenicol in capsules. Effective serum levels were recorded with all chloramphenicol preparations. The variability was largest with chloramphenicol monostearoylglycolate. In a case of neonatal Escherichia coli meningitis good serum levels of chloramphenicol were achieved with chloramphenicol palmitate orally, supporting the view that oral chloramphenicol palmitate can be used to treat serious infections in this age group. Our data and those in the literature show that monitoring of serum chloramphenicol levels in neonates is necessary. After the neonatal period monitoring of serum chloramphenicol levels is useful in avoiding too high concentrations. On the other hand, toxic effects of high concentrations can be recognized from reticulocyte and haemoglobin, neutrophil and platelet counts, which should be performed every three to four days. [\hyperlink{Crenessity}{PMID: 4008381}, H Ekblad et al., 1985]

\hypertarget{pmid_35049507}{E}very year, millions of children undergo medical procedures that require anesthesia. Fear and anxiety are common among young children undergoing such procedures and can interfere with the child's recovery and well-being. Relaxation, distraction, and education are methods that can be used to prepare children and help them cope with fear and anxiety, and serious games may be a suitable medium for these purposes. User-centered design emphasizes the involvement of end users during the development and testing of products, but involving young, preschool children may be challenging. One objective of this study was to describe the development and usability of a computer-based educational health game intended for preschool children to prepare them for upcoming anesthesia. A further objective was to describe the lessons learned from using a child-centered approach with the young target group. A formative mixed methods child (user)-centered study design was used to develop and test the usability of the game. Preschool children (4-6 years old) informed the game design through playful workshops (n=26), and usability testing was conducted through game-playing and interviews (n=16). Data were collected in Iceland and Finland with video-recorded direct observation and interviews, as well as children's drawings, and analyzed with content analysis and descriptive statistics. The children shared their knowledge and ideas about hospitals, different emotions, and their preferences concerning game elements. Testing revealed the high usability of the game and provided important information that was used to modify the game before publishing and that will be used in its further development. Preschool children can inform game design through playful workshops about health-related subjects that they are not necessarily familiar with but that are relevant for them. The game's usability was improved with the participation of the target group, and the game is now ready for clinical testing. [\hyperlink{Crenessity}{PMID: 35049507}, Brynja Ingadottir et al., 2022]

\hypertarget{pmid_19650857}{R}esearch has documented the drastic reduction of unintentional poisonings of children since the introduction of child resistant (CR) packaging. However, studies also indicate that consumers report difficulty using CR packages, in part because tests which determine the 'senior friendliness' of CR designs that are used throughout the world disallow people with 'overt or obvious' disabilities from being test subjects. Our review of drug package usability suggests that the current tests of CR packaging can and should be revised to correct this problem. We use US legislation, regulation and data to exemplify these points, but the conclusions are applicable to all protocols that include the exclusionary provision. [\hyperlink{Crenessity}{PMID: 19650857}, Laura Bix et al., 2009]

\hypertarget{pmid_3168097}{G}olytely solution is now commonly used in preoperative bowel preparation or in colonoscopy and barium enema in adults. Studies have demonstrated its effectiveness and good acceptance in regards to clinical as well as biological point of view. In children, it has been used more recently, but since 1984 several teams agree to find the method excellent. Our study aimed to confirm there was no electrolytic movement caused by golytely and that using it without reserve in children was possible, even in the very young ones. Children are generally very sensible to those movements and mainly as they have a general anaesthesia in the hours following the golytely administration for investigation or surgery. Up to now, 54 children from 4 months to 18 years aged have been studied. Besides the good quality of the preparation noted by the operator and the good clinical tolerance, no significant change of the sodium, potassium, chloride, creatinine and proteins has been noticed. Only urea has decreased very lightly but not out of norms. These results confirm that golytely is safe and effective in preparing the bowel in children. [\hyperlink{Crenessity}{PMID: 3168097}, A Bichet-Sicard et al., 1988]

\hypertarget{pmid_29064302}{T}he objective of this study was to investigate whether the 5-point harness or the impact shield child restraint system (CRS) or both have the potential to cause chest injuries to children. This is determined by examining whether the loading to the chest reaches the internal organ injury threshold for children. The chest injury risk to a child occupant in a CRS was investigated using Q3 dummy tests, finite element (FE) simulations (Q3 dummy and human models), and animal tests. The investigation was done for 2 types of CRSs (i.e., the impact shield CRS and 5-point harness CRS) based on the UN R44 dynamic test specifications. The tests using a Q3 dummy indicated that although the chest deflection of the dummy in the impact shield CRS was large, it was less than the injury threshold (40 mm). Computational biomechanics simulations (using finite element FE analysis) showed that the Q3 dummy's chest is loaded by the shield and deforms substantially under this load. To clarify whether chest injuries due to chest compression can occur with an impact shield or with the 5-point harness CRS, 7 experiments were performed using Tibetan miniature pigs with weights ranging from 9.7 to 13 kg. Severe chest and abdominal injuries (lung contusion, coronary artery laceration, liver laceration) were found in the tests using the impact shield CRS. No chest injuries were present when using the 5-point harness CRS. When using the impact shield CRS, the chest deformed substantially in dummy tests and FE simulations, and chest and abdominal injuries were observed in pig tests. It is possible that these chest injuries could also occur to child occupants sitting in the impact shield CRS. [\hyperlink{Crenessity}{PMID: 29064302}, Yong Han et al., 2018]

\hypertarget{pmid_1419759}{A}nxiety about the use of etretinate in children has been provoked by several reports describing skeletal abnormalities during long-term therapy. However, we have observed no evidence of skeletal toxicity in 42 children treated over an 11-year period. Radiological screening before and during treatment has failed to reveal abnormalities that would influence our decision to commence or to continue etretinate administration. We recommend that children who are to be treated with etretinate should have a baseline selective skeletal survey, with follow-up radiology restricted to those with pretreatment radiological abnormalities and those who develop musculo-skeletal symptoms. In addition we advise that dosage should not exceed 1 mg/kg/day. If these guidelines are followed, we believe that long-term therapy with etretinate can be given to children, with an acceptable margin of safety. [\hyperlink{Crenessity}{PMID: 1419759}, D G Paige et al., 1992]

\hypertarget{pmid_2402648}{C}hloral hydrate has been used extensively to sedate children, but at Brooke Army Medical Center, other drug combinations were becoming increasingly popular due to a perception that chloral hydrate had a high rate of failure, especially with younger or neurologically impaired children. Therefore, 50 children were given the drug before a diagnostic study, and patient data and a sedation score were recorded on a worksheet. Of 50 children, 43 (86\%) were "successfully sedated" on the first attempt with no side effects. Children with neurologic disorders had a much greater (27\% vs 4\%) failure rate than "normal" children. The sedation rate did not significantly differ by age, sex, or initial drug dosage. The study suggest that chloral hydrate is a safe and effective oral sedative but that children with neurologic disorders may need alternative drugs for sedation. [\hyperlink{Crenessity}{PMID: 2402648}, P D Rumm et al., 1990]

\hypertarget{pmid_30473868}{F}urosemide is a potent loop diuretic commonly and variably used by neonatologists to improve oxygenation and lung compliance in premature infants. There are several safety concerns with use of furosemide in premature infants, specifically the risk of sensorineural hearing loss (SNHL), and nephrocalcinosis/nephrolithiasis (NC/NL). We conducted a systematic review of all trials and observational studies examining the association between these outcomes with exposure to furosemide in premature infants. We searched MEDLINE, EMBASE, CINAHL, and clinicaltrials.gov. We included studies reporting either SNHL or NC/NL in premature infants (< 37 weeks completed gestational age) who received at least one dose of enteral or intravenous furosemide. Thirty-two studies met full inclusion criteria for the review, including 12 studies examining SNHL and 20 studies examining NC/NL. Only one randomized controlled trial was identified in this review. We found no evidence that furosemide exposure increases the risk of SNHL or NC/NL in premature infants, with varying quality of studies and found the strength of evidence for both outcomes to be low. The most common limitation in these studies was the lack of control for confounding factors. The evidence for the risk of SNHL and NC/NL in premature infants exposed to furosemide is low. Further randomized controlled trials of furosemide in premature infants are urgently needed to adequately assess the risk of SNHL and NC/NL, provide evidence for improved FDA labeling, and promote safer prescribing practices. [\hyperlink{Crenessity}{PMID: 30473868}, Wesley Jackson et al., 2018]

\hypertarget{pmid_29912720}{T}he review question is: How safe is parent/nurse controlled analgesia and what is its effectiveness on patient outcomes in the neonatal intensive care unit? [\hyperlink{Crenessity}{PMID: 29912720}, Renee Muirhead et al., 2018] Ceftriaxone is widely used in children in the treatment of sepsis. However, concerns have been raised about the safety of ceftriaxone, especially in young children. The aim of this review is to systematically evaluate the safety of ceftriaxone in children of all age groups. MEDLINE, PubMed, Cochrane Central Register of Controlled Trials, EMBASE, CINAHL, International Pharmaceutical Abstracts and adverse drug reaction (ADR) monitoring systems will be systematically searched for randomised controlled trials (RCTs), cohort studies, case-control studies, cross-sectional studies, case series and case reports evaluating the safety of ceftriaxone in children. The Cochrane risk of bias tool, Newcastle-Ottawa and quality assessment tools developed by the National Institutes of Health will be used for quality assessment. Meta-analysis of the incidence of ADRs from RCTs and prospective studies will be done. Subgroup analyses will be performed for age and dosage regimen. Formal ethical approval is not required as no primary data are collected. This systematic review will be disseminated through a peer-reviewed publication and at conference meetings. CRD42017055428. [\hyperlink{Crenessity}{PMID: 29912720}, Linan Zeng et al., 2017]

\hypertarget{pmid_31584401}{M}illions of children every year undergo seemingly safe general anesthetics for surgical procedures and imaging studies. Anesthetic agents have been shown to cause detrimental effects on brain cell survival and cognitive function in animals. As a result, the safety of general anesthetics in children is an active field of investigation. The objective of this review is to evaluate the human research on anesthesia neurotoxicity in the young child. Three databases were searched for studies on anesthesia exposure in infants and children. Positive clinical outcomes in several studies showed no difference in cognitive function between children exposed and unexposed to anesthesia. Research findings also demonstrated negative clinical outcomes following anesthesia exposure, including physical changes on magenetic resonance imaging such as lower gray matter density in the occipital cortex and cerebellum; lower scores on performance IQ, listening comprehension, and expressive language; overrepresentation in the lowest fifth percentile of academic achievement; and increased risk of learning disabilities. More studies are needed that simultaneously measure cognitive function, physical changes, and disability risk to learn how these factors interact in the human brain. [\hyperlink{Crenessity}{PMID: 31584401}, Audrey Rosenblatt et al., 2019]

\hypertarget{pmid_21510025}{S}afe and effective antiseptic use in neonatal intensive care units is mandatory. High efficacy and a low number of side-effects from chlorhexidine have permitted avoidance of the use of mercurials and iodine derivatives, but methanol use can be unsafe in extreme preterm newborns. We report two cases of chemical burn after skin cleansing, due to alcoholic chlorhexidine (0.5\%) use in extremely premature infants used for umbilical catheter insertion. Although this formulation is less concerning for use in full-term newborns, nonalcoholic preparations are preferable for use in preterm newborns. [\hyperlink{Crenessity}{PMID: 21510025}, Xavier Bringué Espuny et al., ]

\hypertarget{pmid_17608315}{S}eizures are a common occurrence in the neonatal intensive care unit, especially among low-birth-weight infants. The efficacy and safety of standard anticonvulsants have not been evaluated extensively in the neonate. In addition, there is concern for the adverse effects of phenobarbital on long-term development. Levetiracetam has been a commonly prescribed oral anticonvulsant for the use of adjunctive therapy for partial seizures in adults with favorable tolerability, and it has been recently approved for children older than age 4 years. There are no published studies regarding the safety and efficacy of this medication in the infant population. This report describes the initiation of levetiracetam in 3 infants, aged 2 days to 3 months, for refractory seizures or intolerance to other anticonvulsants. Each patient was without seizure on levetiracetam monotherapy, and there were no adverse effects. [\hyperlink{Crenessity}{PMID: 17608315}, Michael T Shoemaker et al., 2007]

\hypertarget{pmid_21531030}{C}hloral hydrate (CH) is an oral sedative widely used to sedate infants and young children during auditory brainstem response (ABR) testing. The aim of this study was to record effectiveness, complications and safety of CH as a sedative for ABR. From January of 2003 until December of 2007, 1903 children were tested for ABR, 568 of them being under the age of 6 months. CH (8\%) was used for sedation at a dose of 40 mg/kg with a repeat dose, if necessary, for an adequate sedation, in 20-30 min. We recorded the effectiveness of CH as a sedative for ABR examination, as well as all complications related to the use of CH such as vomiting, rash, hyperactivity, respiratory distress and apnea. The statistical method used was the absolute and percentage frequency distribution of the occurrences. Sedation with CH was necessary to perform testing in 1591 (83.6\%) of the examined children. However, in the population of the examined infants, only 341 (60\%) were sedated with CH, because the remaining 227 (40\%) fell asleep by themselves. Complications included hyperactivity in 152 children (8\%), minor respiratory distress in 10 children (0.4\%), vomiting in 217 children (11.4\%), apnea in 4 children (0.2\%) and rash in 10 children (0.4\%). The complications of hyperactivity, vomiting and rash resolved without any medical treatment. The apnea cases were managed effectively by supplying ventilation to the children via a mask in the presence of an anesthesiologist. The use of CH at a dose of 40 mg/kg up to 80 mg/kg is safe and effective when administered in a setting with adequate equipment and the presence of well trained personnel. [\hyperlink{Crenessity}{PMID: 21531030}, Eirini Avlonitou et al., 2011]

\hypertarget{pmid_8169182}{T}here is evidence for the efficacy and safety of clonazepam (CZP) in adult anxiety disorders, but no formal studies to substantiate clinical reports of similar benefit in children with anxiety disorders. In this double-blind pilot study, 15 children, aged 7 to 13 years, entered a randomly assigned, double-blind crossover trial of 4 weeks of CZP (up to 2 mg/day) and 4 weeks of placebo. Twelve children completed the trial. All but 1 had a diagnosis of separation anxiety disorder, and all but 2 had comorbid diagnoses. Nine children appeared to have moderate to significant clinical improvement, but statistical comparisons on several ratings failed to confirm a trend in favor of CZP. Side effects of drowsiness, irritability, and/or oppositional behavior were notable in 10 children in the CZP phase compared with 5 in the placebo phase. Clonazepam was believed to have clinical benefit for some children, but this was not confirmed statistically in this small sample. Problematic side effects of drowsiness and disinhibition were common and possibly were due to rapid titration. [\hyperlink{Crenessity}{PMID: 8169182}, F Graae et al., ]

\section*{Cuprimine}
\subsection*{Result}
\subsubsection*{Answer}

Unknown

\subsubsection*{{Explanation}}
\hypertarget{Cuprimine}
After reviewing all available abstracts, there are no abstracts specifically about Cuprimine (penicillamine) or its safety in children. None of the abstracts describe targeted studies evaluating the safety of Cuprimine in pediatric populations, nor do they provide evidence affirming or refuting its safety in any specific age range of children. Therefore, based on the abstracts provided, the safety of Cuprimine in children is unknown.

\subsection*{Abstracts}
\hypertarget{pmid_28827252}{C}eftriaxone is widely used in children in the treatment of sepsis. However, concerns have been raised about the safety of ceftriaxone, especially in young children. The aim of this review is to systematically evaluate the safety of ceftriaxone in children of all age groups. MEDLINE, PubMed, Cochrane Central Register of Controlled Trials, EMBASE, CINAHL, International Pharmaceutical Abstracts and adverse drug reaction (ADR) monitoring systems will be systematically searched for randomised controlled trials (RCTs), cohort studies, case-control studies, cross-sectional studies, case series and case reports evaluating the safety of ceftriaxone in children. The Cochrane risk of bias tool, Newcastle-Ottawa and quality assessment tools developed by the National Institutes of Health will be used for quality assessment. Meta-analysis of the incidence of ADRs from RCTs and prospective studies will be done. Subgroup analyses will be performed for age and dosage regimen. Formal ethical approval is not required as no primary data are collected. This systematic review will be disseminated through a peer-reviewed publication and at conference meetings. CRD42017055428. [\hyperlink{Cuprimine}{PMID: 28827252}, Linan Zeng et al., 2017]

\hypertarget{pmid_9818601}{T}o review the safety and efficacy of cyclosporine in the treatment of children with severe bilateral sight-threatening intermediate uveitis or panuveitis. A retrospective chart review was performed on all children younger than 18 years of age with chronic bilateral sight-threatening uveitis who were treated with cyclosporine. Assessment of the therapeutic efficacy and development of adverse effects of cyclosporine after 6 months, 2 years, and 4 years of therapy was performed. Between 1983 and 1992, 15 children and adolescents were treated with cyclosporine. After 6 months, visual acuity improved or stabilized in 82.1\% of eyes, while median vitreous inflammation decreased from 2.0 to 0.5. After 2 and 4 years, visual acuity improved or stabilized in 64\% and 75\% of eyes, respectively. Median vitreous inflammation remained 0.5 after 2 and 4 years of therapy. Mean creatinine clearance and hemoglobin values decreased and serum creatinine increased after 6 months. After 2 years, only mean hemoglobin values remained decreased. After 4 years, no significant differences were noted in any of the laboratory studies. The most frequently noted side effects included transient increases in serum creatinine in 53\%, gingival hyperplasia in 40\%, and hirsutism in 20\% of patients. The authors' results suggest that cyclosporine is a safe and effective therapy for the treatment of children with severe bilateral sight-threatening intermediate uveitis or panuveitis. [\hyperlink{Cuprimine}{PMID: 9818601}, R C Walton et al., 1998]

\hypertarget{pmid_22364032}{A}cute respiratory infections are the second leading cause of morbidity in children under 18 years. Several drugs have been used with variable efficacy and safety, trying to reduce the associated symptoms and improve quality of life. To evaluate the efficacy and safety of buphenine, aminophenazone and diphenylpyraline hydrochloride when compared with placebo for the control of symptoms associated with common cold in children 6-24 months of age. Randomized clinical trial, double blind, placebo controlled, in 100 children < 24 months of any gender, with symptoms associated to common cold. They received the drug under study vs. placebo for seven days. Both groups received acetaminophen. The change on common cold related symptoms were evaluated. Statistic analysis was made with STATA 11.0 for Mac. Fifty-three children were randomized to study drug and forty-seven to placebo. Age of children in each group was similar (12.2 +/- 5.8 months vs. 12.7 +/- 5.8 months, p NS). There were significant differences between groups in relation to rhinorrea and sneezing resolution, with better results in Flumil group and no adverse events observed. The results in this study indicates that Flumil is a safe and effective drug for control of symptoms present in the common cold in children aged 6-24 months. [\hyperlink{Cuprimine}{PMID: 22364032}, Ericka Montijo-Barrios et al., ]

\hypertarget{pmid_22246411}{T}he quality of MRI and CT depends largely on immobility of the patient during the procedure, which is often difficult to achieve without sedation in children below the age of 6 years. To assess the efficacy and safety of intravenous chlorpromazine sedation for repeated imaging in young children treated for cancer. From July 2003 to January 2007, information on children younger than 6 years of age having MRI or CT was prospectively collected. Forty-five minutes before the scan, a 10-min infusion of chlorpromazine 0.5 mg/kg was administered and managed by non-anesthetic staff. Patient monitoring included continuous measurement of pulse, respiration, oxygen saturation and arterial blood pressure. Procedure-related parameters and adverse events were documented. Sedation was considered successful when the procedure was completed and at least 95\% of images were usable. One-hundred-one procedures (82 MRI, 19 CT) were evaluated in 62 children, 3-74 months old. Adequate sedation was achieved in 96\% of cases, with mean induction time, 22 min; mean duration of sleep, 72 min, and mean duration of procedure, 33 min. Mean time spent in the radiology unit was 104 min. Ninety-six percent of imaging procedures were successfully completed. No cardiac, respiratory, neurological or allergic complication occurred. Intravenous chlorpromazine is safe and effective for procedural sedation in young children with cancer undergoing MRI and CT. [\hyperlink{Cuprimine}{PMID: 22246411}, C Heng Vong et al., 2012]

\hypertarget{pmid_17401268}{T}he present study aimed at verifying the safety and efficacy of rifampicin in ameliorating pruritus in cholestatic children. Twenty-three Egyptian children (14 boys and 9 girls), suffering from intractable pruritus of cholestasis, were included. Rifampicin was started at a dose of 10 mg/Kg/day in two divided doses and increased gradually to a maximum of 20 mg/Kg/day if there was no response. Liver function tests were followed up weekly. Seventeen patients (74\%) showed improvement of pruritus with rifampicin. None of the patients showed any deterioration in liver functions. Rifampicin in a dose of 10-20 mg/Kg/day is safe and effective in ameliorating uncontrollable pruritus in children with persistent cholestasis. [\hyperlink{Cuprimine}{PMID: 17401268}, Hanaa El-Karaksy et al., 2007]

\hypertarget{pmid_34562149}{T}here is no approved dosage and administration of inulin for children. Therefore, we measured inulin clearance (Cin) in pediatric patients with renal disease using the pediatric dosage and administration formulated by the Japanese Society for Pediatric Nephrology, and compared Cin with creatinine clearance (Ccr) measured at the same time. We examined to what degree Ccr overestimates Cin, using the clearance ratio (Ccr/Cin), and confirmed the safety of inulin in pediatric patients. Pediatric renal disease patients aged 18 years or younger were enrolled. Inulin (1.0 g/dL) was administered intravenously at a priming rate of 8 mL/kg/hr (max 300 mL/hr) for 30 min. Next, patients received inulin at a maintenance rate of 0.7 × eGFR mL/min/1.73 m Inulin was administered to 60 pediatric patients with renal disease; 1 patient was discontinued and 59 completed. The primary endpoint, Ccr/Cin, was 1.78 ± 0.52 (mean ± standard deviation). Regarding safety, five adverse events were observed in four patients (6.7\%); all were non-serious. No adverse reactions were observed in this study. The results in this study on the dosage and administration of inulin showed that inulin can safely and accurately determine GFR in pediatric patients with renal disease. CLINICALTRIALS. NCT03345316. [\hyperlink{Cuprimine}{PMID: 34562149}, Osamu Uemura et al., 2022]

\hypertarget{pmid_28562262}{T}he safety of cough and cold medication (CCM) use in children has been questioned. We describe the safety profile of CCMs in children <12 years of age from a multisystem surveillance program. Cases with adverse events (AEs) after ingestion of at least 1 index CCM ingredient (brompheniramine, chlorpheniramine, dextromethorphan, diphenhydramine, doxylamine, guaifenesin, phenylephrine, and pseudoephedrine) in children <12 years of age were collected from 5 data sources. An expert panel determined relatedness, dose, intent, and risk factors. Case characteristics and AEs are described. Of the 4202 cases reviewed, 3251 (77.4\%) were determined to be at least potentially related to a CCM, with accidental unsupervised ingestions (67.1\%) and medication errors (13.0\%) the most common exposure types. Liquid (67.3\%), pediatric (75.5\%), and single-ingredient (77.5\%) formulations were most commonly involved. AEs occurring in >20\% of all cases included tachycardia, somnolence, hallucinations, ataxia, mydriasis, and agitation. Twenty cases (0.6\%) resulted in death; most were in children <2 years of age (70.0\%) and none involved a therapeutic dose. The overall reported AE rate was 0.573 cases per 1 million units (ie, tablets, gelatin capsules, or liquid equivalent) sold (95\% confidence interval, 0.553-0.593) or 1 case per 1.75 million units. The rate of AEs associated with CCMs in children was low. Fatalities occurred even less frequently. No fatality involved a therapeutic dose. Accidental unsupervised ingestions were the most common exposure types and single-ingredient, pediatric liquid formulations were the most commonly reported products. These characteristics present an opportunity for targeted prevention efforts. [\hyperlink{Cuprimine}{PMID: 28562262}, Jody L Green et al., 2017]

\hypertarget{pmid_34184244}{A}s standard treatments are not licensed for use in the infantile population, the treatment of scabies in this age group can be challenging. We review the relevant evidence to determine the roles of topical permethrin and oral ivermectin in the management of infantile scabies. Demographic and clinical data were collected from relevant English articles published from January 2000 to December 2020. Complete resolution was observed in 100\% of infants younger than two months treated with permethrin, and 87.6\% of infants aged 12 months or less and/or children weighing under 15 kg treated with ivermectin. Adverse effects from permethrin use were limited to local eczematous reactions. Adverse effects from ivermectin use included mildly elevated creatine kinase levels, eczema flare-ups, diarrhoea, vomiting, irritability, pruritus and pustular skin reactions. Overall, both permethrin and ivermectin appear to have an acceptable safety profile in infants. Permethrin is highly effective as a first-line therapy for scabies in infants younger than two months. Ivermectin use is recommended when authorised topical treatment has failed, in crusted scabies, in cases where compliance with topical agents may be problematic, and in infants with severely inflamed or broken skin where prescription of topical therapies would likely cause cutaneous and systemic toxicity. Additional high-quality studies are needed to guide best practice in the management of infantile scabies. [\hyperlink{Cuprimine}{PMID: 34184244}, Yolanka Lobo et al., 2021]

\hypertarget{pmid_16674757}{T}he aim of this study was to assess the efficacy of pharmacological prophylactic treatments of migraine in children. Databases were searched from inception to June 2004 and references were checked. We selected controlled trials on the effects of pharmacological prophylactic treatments in children with migraine. We assessed trial quality using the Delphi list and extracted data. Analyses were carried out according to type of intervention. A total of 20 trials were included. Headache improvement was significantly higher for flunarizine compared with placebo (relative risk 4.00, 95\% confidence interval 1.60, 9.97). There is conflicting evidence for the use of propranolol. Nimodipine, clonidine, L-5HTP, trazodone and papaverine showed no effect when compared with placebo. All medications were well tolerated and adverse events showed no significant differences. Flunarizine may be effective as prophylactic treatment for migraine in children. Because of the small number of studies and the methodological shortcomings, conclusions regarding effectiveness have to be drawn with caution. [\hyperlink{Cuprimine}{PMID: 16674757}, L Damen et al., 2006]

\hypertarget{pmid_20944041}{I} frequently see children with scabies in my practice. A variety of medications are available to treat scabies. Permethrin is one of the most common medications used. Is permethrin a safe and effective option for children? Scabies is a common parasitic skin infection. It is highly prevalent in young children. Topical permethrin (5\% cream) is a safe and effective scabicide in children. It is recommended as a first-line therapy for patients older than 2 months of age. Because there are theoretical concerns regarding percutaneous absorption of permethrin in infants younger than 2 months of age, guidelines recommend 7\% sulfur preparation instead of permethrin. [\hyperlink{Cuprimine}{PMID: 20944041}, Lina Albakri et al., 2010]

\hypertarget{pmid_9085807}{T}he results of this study show that postoperative patient-controlled pain therapy in children with piritramide is - in a similar way as with adults - a safe method involving a low incidence of side effects. A special pump parameter setting is required with larger bolus dose sizes and longer lockout intervals, not very different from the experience gained with adults, and which is based on other values than those recommended up to now with morphine for paediatric PCA. Side effects were rarely observed. The fear of respiratory depression constitutes no rational reason to deny the younger patients this form of analgesia provided that monitoring is guaranteed. [\hyperlink{Cuprimine}{PMID: 9085807}, G Petrat et al., 1997]

\hypertarget{pmid_7843952}{A}n open, prospective study was undertaken to assess the efficacy and safety of subcutaneous sumatriptan in 17 children, ages 6 to 16 years, with severe, recurrent migraine. A 6-mg dose was used in 15 patients and relieved headache within 1 hour in six and by 2 hours in five others. Two smaller children received a 3-mg dose and both were headache-free within 2 hours. Most also reported marked improvement in associated symptoms such as nausea and photophobia. Four subjects had no clinical improvement after a 6-mg dose. Side effects, such as neck pressure, were brief and mild. These findings suggest that subcutaneous sumatriptan can be both effective and safe as an abortive agent in juvenile migraine, but the appropriate dose in smaller children will need further investigation. [\hyperlink{Cuprimine}{PMID: 7843952}, J T MacDonald et al., ]

\hypertarget{pmid_2041162}{T}his study describes the pharmacokinetic characteristics and clinical usefulness of cefpirome (CPR) in children. Mean half-lives of 20 mg/kg and 40 mg/kg of CPR injected intravenously in one shot were 1.18 and 1.34 hours, respectively, and their mean recovery rates into urine were 69.8 and 72.2\%, respectively. Minimum inhibitory concentrations of CPR against Staphylococcus aureus, Streptococcus pneumoniae, Klebsiella pneumoniae, Escherichia coli and Haemophilus influenzae were the same as or lower than those of ceftazidime. CPR was clinically effective in 14/15 of patients with bacterial infections; 8/9 of pneumonia, 2/2 of bronchitis, 1/1 of pharyngitis, 1/1 of tonsillitis, 1/1 of osteomyelitis, 1/1 of urinary tract infection. No clinically overt side effects of CPR were found, while an increase of eosinophils in blood was observed in 2 cases, and an increase of platelet in blood in 1 case and an elevation of serum GPT activity in 1 case were also observed. These findings indicate that CPR is useful for the treatment of bacterial infections in children. [\hyperlink{Cuprimine}{PMID: 2041162}, K Kida et al., 1991]

\hypertarget{pmid_9374563}{T}o test the hypothesis that short-term use of ibuprofen increases the risk of impaired renal function in children. Randomized, double-blind acetaminophen-controlled clinical trial. Children with a febrile illness were enrolled from outpatient pediatric and family medicine practices and randomly assigned to receive either acetaminophen suspension or one of two dosages of ibuprofen suspension (5 mg/kg or 10 mg/kg) for fever control. Mean blood urea nitrogen levels on admission among children admitted to hospital and assigned ibuprofen 5 mg/kg (n = 96), ibuprofen 10 mg/kg (n = 102), and acetaminophen 12 mg/kg (n = 87) were 4.1, 3.8, and 3.9 mmol/L, respectively. The corresponding creatinine levels were 43, 41, and 43 micromol/L, respectively. The prevalence of a creatinine level >62 micromol/L was 9.5\% overall and did not vary by antipyretic assignment. Among 83 children hospitalized with dehydration, the mean creatinine level was 44 micromol/L, and the prevalence of an elevated creatinine was 14\%; neither measure varied by antipyretic assignment. Although renal failure in children has been reported after ibuprofen use, these data suggest that for short-term use the risk of less severe renal impairment, as reflected by blood urea nitrogen and creatinine levels, is small and not significantly greater than that after acetaminophen use. [\hyperlink{Cuprimine}{PMID: 9374563}, S M Lesko et al., 1997]

\hypertarget{pmid_16638734}{B}uspirone is used to treat generalized anxiety disorder in children and may be useful in developmental disorders in which brain serotonin synthesis is altered. Autistic children (13 boys, 7 girls) were given a single oral dose of 2.5 mg (2-3 years) or 5.0 mg (4-6 years). Blood was collected for 8 hours, and plasma was assayed for buspirone and its metabolite 1-pyrimidinylpiperazine (1-PP). The peak concentration of buspirone averaged 1141 +/- 748 pg/mL with a time to maximum concentration of 0.8 hours. Half-life was 1.6 +/- 0.3 hours. Peak concentrations of 1-PP were 4.5-fold higher than for buspirone. Girls had higher peak concentrations (1876 vs 746 pg/mL) for buspirone and a lower peak 1-PP/buspirone concentration ratio. These results suggest that buspirone is rapidly absorbed and eliminated in young children with extensive metabolism to 1-PP. Plasma concentrations with 2.5- to 5.0-mg doses were similar to those observed in older children receiving 7.5- to 15-mg doses. [\hyperlink{Cuprimine}{PMID: 16638734}, David J Edwards et al., 2006]

\hypertarget{pmid_25135766}{R}ecently, an association between childhood growth stunting and aflatoxin (AF) exposure has been identified. In Ghana, homemade nutritional supplements often consist of AF-prone commodities. In this study, children were enrolled in a clinical intervention trial to determine the safety and efficacy of Uniform Particle Size NovaSil (UPSN), a refined calcium montmorillonite known to be safe in adults. Participants ingested 0.75 or 1.5 g UPSN or 1.5 g calcium carbonate placebo per day for 14 days. Hematological and serum biochemistry parameters in the UPSN groups were not significantly different from the placebo-controlled group. Importantly, there were no adverse events attributable to UPSN treatment. A significant reduction in urinary metabolite (AFM1) was observed in the high-dose group compared with placebo. Results indicate that UPSN is safe for children at doses up to 1.5 g/day for a period of 2 weeks and can reduce exposure to AFs, resulting in increased quality and efficacy of contaminated foods.  [\hyperlink{Cuprimine}{PMID: 25135766}, Nicole J Mitchell et al., 2014] In spite of the high occurrence of migraine headaches in school-age children, there are currently no approved and widely accepted pharmacologic agents for migraine prophylaxis in children. Our previous open-label study in children revealed the efficacy of cinnarizine, a calcium channel blocker, in migraine prophylaxis. A placebo-controlled trial was conducted to demonstrate the efficacy and safety of cinnarizine in the prophylaxis of migraine in children. A double-blind, placebo-controlled, parallel-group study conducted in a tertiary medical center in Tehran, Iran. Children (5-17 years) who experienced migraines with and without aura, as defined on the basis of 2004 International Headache Society criteria, were recruited into the study. Children were excluded if they had complicated migraine, epilepsy, or a history of use of migraine prophylactic agents. Each participant was randomly assigned to receive cinnarizine (a single 1.5 mg/kg/day dose in children weighing less than 30 kg and a single 50 mg dose in children weighing more than 30 kg, administered at bedtime) or placebo. The frequency, severity, and duration of headaches over the trial period were assessed and adverse effects were monitored. A total of 68 children (34 in each group) with migraine were enrolled and 62 participants completed the study. After 3 months of taking cinnarizine or placebo, children in both groups experienced significantly reduced frequency, severity, and duration of headaches compared with baseline measurements (P < 0.001). However, compared with 31.3\% of children in the placebo group, 60\% of children in the cinnarizine group reported more than 50\% reduction in monthly headache frequency (P = 0.023), suggesting that cinnarizine was significantly more effective than placebo in reducing the frequency of headaches. No serious adverse effects of the medications were observed in the treated children, including no abnormal weight gain or extrapyramidal signs. Our results indicate that the use of cinnarizine at doses administered in this study is effective and safe for prophylaxis of migraine headaches in children. [\hyperlink{Cuprimine}{PMID: 25135766}, Mahmoud Reza Ashrafi et al., 2014]

\hypertarget{pmid_20527137}{O}nly a few corticosteroids for topical use have proven safe and effective in pediatric populations down to 3 months of age. The authors report the results of a study designed to assess the efficacy and safety of hydrocortisone butyrate (HCB) 0.1\% in lipocream (LCr) vehicle in infants and children. A total of 264 boys and girls 3 months to less than 18 years old, with stable, mild to moderate atopic dermatitis affecting at least 10\% body surface area applied HCB 0.1\% in LCr or LCr alone twice daily for up to 1 month without occlusion. Primary end-points included: percent of patients who achieved treatment success based on physician global assessments. Secondary endpoint included: difference in pruritus and Eczema Area and Severity Index (EASI) at day 29. Treatment was significant (P < 0.001) for HCB 0.1\% LCr over vehicle. No serious nor significant adverse events were reported. Results are representative of a short duration treatment for a chronic disease. HCB 0.1\% in LCr is more effective than its vehicle in pediatric populations down to 3 months of age without significant adverse events when used twice a day for up to 1 month. [\hyperlink{Cuprimine}{PMID: 20527137}, William Abramovits et al., ]

\hypertarget{pmid_21144334}{C}yclosporine has been found to be effective and safe in many inflammatory skin disorders such as psoriasis and atopic dermatitis (AD), in adults and in children. Its use in paediatrics is still under scope. We present three patients who started cyclosporine but stopped due to complications. It is our aim to warn about potential side effects of cyclosporine and recommend cautious utilization. Two children, aged 4 and 13 years, with AD and one child, aged 2 years, with erythrodermic psoriasis, were treated with oral cyclosporine. developed secondary impetigo on the 6th day of treatment. Started topical corticosteroids and topical calcineurin inhibitors afterwards, with no relapses. developed herpetic infection, hepatic and renal impairment (eventual drug interaction) on the 4th day of treatment. THIRD CASE: Psoriasis and impetigo, treated with flucloxacillin, gentamicin. Generalized angioedema and urticariform lesions after 6 days of cyclosporine. Beta lactam hypersensitivity reaction under study. Eventual cyclosporine toxicity to consider. The data on cyclosporine use in children is still scarce. Use should be limited to cases with precise indication, after considering risks and benefits. [\hyperlink{Cuprimine}{PMID: 21144334}, João Antunes et al., ]

\hypertarget{pmid_20356918}{T}wo studies were conducted to characterize multiple-dose pharmacokinetics and potential drug interactions of ibuprofen and pseudoephedrine combined in a suspension and to evaluate safety of this combination in children with common cold, flu, or sinusitis. In the pharmacokinetic study, 24 healthy children aged 4-11 years were administered ibuprofen -pseudoephedrine suspension at 7.5 and 1.125 mg/kg, respectively, every 6 hours for 5 doses. Serial blood samples were drawn over 6 hours after final dose for assessment of steady-state pharmacokinetics. In the open-label, multicenter safety study, more than 100 children aged 2-11 years experiencing symptomatic rhinitis were enrolled. Ibuprofen -pseudoephedrine suspension was administered as needed at similar mg/kg doses every 6-8 hours for up to 3 days. Subjects enrolled in the pharmacokinetic study showed no accumulation of either drug; their weight-adjusted clearances were independent of age, and results were comparable with those from previous single-ingredient studies. For ibuprofen, oral clearance (Cl/F) was 77.5 + or - 16.4 mL/kg/h and volume of distribution (Vd/F) was 0.147 + or - 0.037 L/kg. For pseudoephedrine, Cl/F was 12.3 + or - 2.2 mL/kg/min and Vd/F was 2.52 + or - 0.47 L/kg. In the safety study, adverse events were reported for 18.4\% of subjects; most were mild to moderate intensity. There was little difference in incidence of adverse events among different age and weight groups. In conclusion, administration of combined ibuprofen and pseudoephedrine in children demonstrated similar pharmacokinetics when compared with reports of the pharmacokinetics for the single-ingredient products, consistent with no apparent drug interactions. The combination suspension was generally well tolerated. [\hyperlink{Cuprimine}{PMID: 20356918}, Cathy K Gelotte et al., 2010]

\hypertarget{pmid_16174597}{T}o evaluate the therapeutic effects and safety of cefepime (Maxipime) and sulbactam/cefoperazone (sulperazone) on moderate and severe respiratory infection in children. Totally 100 children hospitalized for pneumonia were randomized equally into 2 groups, namely group A with maxipime treatment at the dose of 50 mg/kg given intravenously twice daily, and group B with sulperazone treatment at 50-100 mg/kg given intravenously twice a day. The therapeutic effects and safety of both medications were observed. In maxipime group, 44 of the 50 cases were cured with complete elimination of the symptoms and signs, and obvious therapeutic effect was achieved in 5 cases with significant improvement or resolution of the majority of symptoms and signs. One case failed to respond favorably to the treatment, with the overall efficacy rate of 98\% without incidence of adverse effects. In sulperazone group, 40 of the 50 cases were cured, 6 showed significant improvement, and 4 failed to respond to the treatment, with the rate of 92\%. One patient complained of rashes during sulperazone treatment, which disappeared the next day without sulperazone withdrawal. Significant difference was not noted between the groups (chi(2)=2.43, P>0.05). Both maxipime and sulperazon are effective and safe for application in children with moderate and severe respiratory infections. [\hyperlink{Cuprimine}{PMID: 16174597}, Jian-jun Huang et al., 2005]

\hypertarget{pmid_11493818}{C}ontrolled intubation in the pediatric emergency department (ED) requires a paralytic agent that is safe, efficacious, and of rapid onset. The safety of succinylcholine has been challenged, leading some clinicians to use vecuronium as an alternative. Rocuronium's onset is similar to that of succinylcholine. To evaluate the safety and efficacy of rocuronium for controlled intubation with paralysis (CIP) in the pediatric ED. A retrospective, observational study reviewed the records of patients less than 15 years of age, who received controlled intubation with paralytics at two Dallas EDs. The patients received either vecuronium or rocuronium. The study included 84 patients (vecuronium 19, rocuronium 65). Complications were similar between the two groups. Rocuronium had a shorter time from administration to intubation when compared to vecuronium (P < 0.05). Rocuronium is as safe and efficacious as vecuronium for CIP in the pediatric ED. [\hyperlink{Cuprimine}{PMID: 11493818}, D R Mendez et al., 2001]

\hypertarget{pmid_28361405}{S}afety concerns regarding potential life-threatening adverse events associated with codeine have resulted in policy decisions to restrict its use in pediatrics. However, whether these drug safety communications have had an immediate and strong impact on codeine use remains in question. We aimed to investigate the impact of the two implemented safety-related regulations (label changes and reimbursement regulations) on the use of codeine for upper respiratory infection (URI) or cough. A quasi-experimental study was performed using Taiwan's National Health Insurance Research Database. Quarterly data of codeine prescription rates for URI/cough visits were reported, and an interrupted time series design was used to assess the impact of the safety regulations on the uses of codeine among children with URI/cough visits. Multivariable logistic regression models were used to explore patient and provider characteristics associated with the use of codeine. The safety-related regulations were associated with a significant reduction in codeine prescription rates of -4.24\% (95\% confidence interval [CI] -4.78 to -3.70), and the relative reduction compared with predicted rates based on preregulation projections was 60.4, 56.6, and 53.2\% in the first, second, and third year after the regulations began, respectively. In the postregulation period, physicians specializing in otolaryngology (odds ratio [OR] 1.47, 95\% CI 1.45-1.49), practicing in district hospitals (OR 6.84, 95\% CI 5.82-8.04) or clinics (OR 6.50, 95\% CI 5.54-7.62), and practicing in the least urbanized areas (OR 1.60, 95\% CI 1.55-1.64) were more likely to prescribe codeine to children than their counterparts. Our study provides a successful example of how to effectively reduce the codeine prescriptions in children in the 'real-world' settings, and highlights areas where future effort could be made to improve the safety use of codeine. Future research is warranted to explore whether there was a simultaneous decrease in the incidence rates of codeine-related adverse events following the safety-related regulations. [\hyperlink{Cuprimine}{PMID: 28361405}, Chih-Wan Lin et al., 2017]

\hypertarget{pmid_1429411}{T}he pharmacological management of anxiety in children primarily has used antidepressants, such as imipramine. Buspirone, an atypical anxiolytic, has been shown to be of benefit in both adults and children. It has relatively few side effects and is generally well tolerated. Two cases are reported here involving children treated for anxiety with buspirone who subsequently suffered a possible psychotic deterioration. [\hyperlink{Cuprimine}{PMID: 1429411}, P Soni et al., 1992]

\hypertarget{pmid_29392757}{C}hronic spontaneous urticaria (CSU) is not frequent in children. Management guidelines have been developed for adults and randomized controlled trials (RCTs) included teenagers aged 12-18, but data for children under age 12 are limited. We performed a systematic review to assess comorbidities in children <12 years old with CSU and the efficacy and safety of treatments. We searched for original articles of epidemiologic and treatment data in children <12 years old with CSU that were published from 2005 to July 2016 in MEDLINE, EMBASE, CENTRAL, and LILACS. Article selection and data extraction were performed in duplicate. Our systematic review included 9 reports on epidemiologic data (633 children). Five comorbidities and laboratory anomalies associated with CSU found were atopy (28.1\%), positive autologous serum skin test (36.8\%), thyroid biologic anomalies (6.4\%) and detectable antinuclear antigen (10.4\%), seroprevalence for Helicobacter pylori (21.1\%), low vitamin D level (69.1\%), and psychiatric disorders (70.4\%). Only one study allowed for comparison with a control group. Our review included 10 studies (322 children), describing 5 different drug families, mostly H1-antihistamines (n = 297). One randomized controlled study compared single-dose rupatadine with single-dose desloratadine and placebo. Cyclosporine was effective and had no adverse effects in 18 children. Omalizumab, montelukast, and cefuroxime were reported in very small series (5, 1, and 1 patients). H1-antihistamines are effective for CSU in children <12 years old, with reassuring safety data at licensed doses. Cyclosporine seems effective, but the level of evidence is low. [\hyperlink{Cuprimine}{PMID: 29392757}, Hélène Cornillier et al., 2018]

\section*{Cyclophosphamide}
\subsection*{Result}
\subsubsection*{Answer}

Yes (Infants and young children 0–4.9 years)
Yes (Children, general pediatric population)
Yes (Adolescents 12–21 years)

\subsubsection*{{Explanation}}
\hypertarget{Cyclophosphamide}
Based on the abstracts provided, several studies have specifically evaluated the safety of cyclophosphamide in children across different age ranges and disease contexts. Here is a summary of the relevant evidence:

1. **Infants and Young Children (0–4.9 years):**
   - A pharmacokinetic study in 171 patients aged 0.07–4.9 years with brain tumors found that cyclophosphamide exposure and toxicity were higher in young infants (<6 months), leading to a recommendation for dose reduction in this group to reduce toxicity. The study established an exposure-toxicity association and suggested that careful dosing can mitigate risks, but did not report unexpected or severe adverse events beyond what is expected for chemotherapy [\hyperlink{pmid_31796512}{PMID: 31796512}, Olivia Campagne et al., 2020].

2. **Children (General Pediatric Population):**
   - A meta-analysis of 38 studies involving 1,504 children with relapsing steroid-sensitive nephrotic syndrome found that cyclophosphamide (2–3 mg/kg for 8–12 weeks) is effective and generally safe, with a fatality rate of approximately 1\%. Leukopenia and infections were noted, but severe side effects were rare at recommended doses. Gonadal toxicity was a concern at higher cumulative doses, especially in males [\hyperlink{pmid_11322378}{PMID: 11322378}, K Latta et al., 2001].
   - A study of 43 children with minimal lesion nephrotic syndrome reported that cyclophosphamide was beneficial, with some cases of hemorrhagic cystitis and leukopenia, but overall affirmed its safety and efficacy in this context [\hyperlink{pmid_1589052}{PMID: 1589052}, R de Moor et al., 1992].
   - Another study in children with minimal-change nephropathy found transient immunosuppression but no long-term immune dysfunction attributable to cyclophosphamide [\hyperlink{pmid_6229699}{PMID: 6229699}, J Feehally et al., 1984].
   - In children with anti-NMDAR encephalitis (n=6, age not specified but pediatric), cyclophosphamide as second-line therapy was effective and no adverse reactions or abnormal lab results were observed during treatment [\hyperlink{pmid_28606234}{PMID: 28606234}, Wei-Wen Zhu et al., 2017].
   - In severe and refractory juvenile dermatomyositis, two studies (n=12 and a larger cohort) found cyclophosphamide to be efficacious with no serious short-term toxicity; reversible side effects included lymphopenia, herpes zoster, and alopecia. The risk of malignancy, infertility, and gonadal failure was considered low at the doses used, but longer-term safety data were limited [\hyperlink{pmid_14722349}{PMID: 14722349}, P Riley et al., 2004; \hyperlink{pmid_29342499}{PMID: 29342499}, Claire T Deakin et al., 2018].
   - In pediatric brain tumors, studies reported that high-dose cyclophosphamide regimens were associated with expected hematologic toxicity (neutropenia, need for transfusions, fever), but no deaths were attributed to infection or bleeding, and toxicity was considered acceptable for the context [\hyperlink{pmid_10894368}{PMID: 10894368}, K S Bottom et al., 2000; \hyperlink{pmid_7749762}{PMID: 7749762}, T G Abrahamsen et al., 1995].
   - A case report described a fatal syndrome of inappropriate antidiuretic hormone secretion in a 5-year-old after high-dose cyclophosphamide, indicating rare but serious potential adverse effects [\hyperlink{pmid_476599}{PMID: 476599}, P J Harlow et al., 1979].

3. **Adolescents (12–21 years):**
   - A quality improvement study in girls aged 12–21 years receiving intravenous cyclophosphamide for autoimmune disorders focused on pregnancy screening, not on drug safety per se, but did not report unexpected adverse events [\hyperlink{pmid_27940668}{PMID: 27940668}, Kristen Hayward et al., 2016].

**Summary:** Multiple targeted studies and meta-analyses in children (including infants, young children, and adolescents) affirm that cyclophosphamide, when used at recommended doses and with appropriate monitoring, is generally safe for pediatric use in several indications (e.g., nephrotic syndrome, autoimmune diseases, brain tumors). The main risks are hematologic toxicity, infections, and potential gonadal toxicity at higher cumulative doses. Rare but serious adverse events can occur. Long-term safety, especially regarding fertility and malignancy risk, requires further study, but short-term safety is supported by the available evidence.

\subsection*{Abstracts}
\hypertarget{pmid_15907638}{C}yclophosphamide is an alkylating agent widely used from cancer chemotherapy to immunotherapy purposes. In paediatrics oncology, oral cyclophosphamide prescribed at low dosages for a long time treatment is currently investigated. This treatment is a putative well tolerated regimen for children treated for a wide variety of recurrent solid tumours. For these purposes, new oral formulations more convenient for children than cyclophosphamide 50mg tablets are needed. Thus, we present a rapid method for the assay of cyclophosphamide in various pharmaceutical preparations using high-performance thin-layer chromatography (HPTLC) and derivatization with phosphomolybdic acid. This method is accurate and precise and allows quantitation of cyclophosphamide in aqueous solutions from 400 to 1200 microg/mL. It is suitable for quantitation and stability studies of cyclophosphamide in pharmaceutical products, i.e. capsules and infusion bags prepared in a hospital pharmacy. According to pharmaceutical guidelines, we demonstrated that low dose cyclophosphamide capsules, extemporaneously prepared for children, are stable at least for 70 days. [\hyperlink{Cyclophosphamide}{PMID: 15907638}, Jérôme Bouligand et al., 2005]

\hypertarget{pmid_1589052}{T}he effect of cyclophosphamide therapy was evaluated in the treatment of children with nephrotic syndrome due to minimal lesions. Most of the children, 37 out of 43, presented with frequent relapsing nephrotic syndrome. Cyclophosphamide was given in a dose of 3 mg/kg body weight/day for a period of 8 weeks. Two patients received two courses, one patient received three courses. Only one patient, who was steroid-resistant, did not respond to cyclophosphamide therapy (therapy was, however, stopped after 3 weeks because of haemorrhagic cystitis). 57\% of the patients were still in remission after 18 months (n = 37) and 50\% after 30 months (n = 34). A haemorrhagic cystitis developed in 3 patients and leucopenia in 2 patients. From this study, which confirms data reported in literature, it can be concluded that cyclophosphamide has a beneficial effect in children with minimal lesion nephrotic syndrome and steroid toxicity. [\hyperlink{Cyclophosphamide}{PMID: 1589052}, R de Moor et al., 1992]

\hypertarget{pmid_27940668}{C}yclophosphamide is a teratogenic medication used in the treatment of adolescents with autoimmune disorders. This adolescent population is sexually active, does not receive adequate contraceptive care, and is at risk for unintended pregnancy. We undertook a quality improvement initiative to improve rates of pregnancy screening before intravenous cyclophosphamide administration in our adolescent girl patients. Data were collected from the electronic medical record. The primary outcome was completion of a urine pregnancy test before intravenous cyclophosphamide infusion in girls aged 12 to 21 years between July 2011 and June 2015. Data were reviewed quarterly and an iterative quality improvement approach was used. Interventions included staff education, electronic order set updates, and a Maintenance of Certification project. Interrupted time series analysis and multivariable mixed effects logistic regression were used to evaluate trends over time and to adjust for potential confounders. Thirty girls received 153 cyclophosphamide infusions during the study. Pregnancy testing before medication administration increased from 25\% to 100\% by study completion. Infusions in the last time period were significantly more likely to be accompanied by a pregnancy test versus those in the first time period (odds ratio: 17.7; 95\% confidence interval [CI]: 3.1-101.6) after adjustment for patient age, managing service, infusion setting, and insurance type. Our institution achieved a significant increase in standard pregnancy screening in adolescent girls receiving intravenous cyclophosphamide. The interventions most valuable in increasing screening rates were updating electronic order sets, educating staff, and physician engagement in the Maintenance of Certification program. [\hyperlink{Cyclophosphamide}{PMID: 27940668}, Kristen Hayward et al., 2016]

\hypertarget{pmid_18927240}{C}yclophosphamide-based regimens are front-line treatment for numerous pediatric malignancies; however, current dosing methods result in considerable interpatient variability in tumor response and toxicity. In this pediatric population, the authors' objectives were (1) to quantify and explain the pharmacokinetic variability of cyclophosphamide and 2 of its metabolites, hydroxycyclophosphamide (HCY) and carboxyethylphosphoramide mustard (CEPM), and (2) to apply a population pharmacokinetic model to describe the disposition of cyclophosphamide and these metabolites. A total of 196 blood samples were obtained from 22 children with neuroblastoma receiving intravenous cyclophosphamide (400 mg/m2/d) and topotecan. Blood samples were quantitated for concentrations of cyclophosphamide, HCY, and CEPM using liquid chromatography-mass spectrometry and analyzed using nonlinear mixed-effects modeling with the NONMEM software system. After model building was complete, the area under the concentration-time curve (AUC) was computed using NONMEM. Cyclophosphamide elimination was described by noninducible and inducible routes, with the latter producing HCY. Glomerular filtration rate was a covariate for the fractional elimination of HCY and its conversion to CEPM. Considerable interpatient variability was observed in the AUC of cyclophosphamide, HCY, and CEPM. These results represent a critical first step in developing pharmacokinetic-linked pharmacodynamic studies in children receiving cyclophosphamide to determine the clinical relevance of the pharmacokinetic variability in cyclophosphamide and its metabolites. [\hyperlink{Cyclophosphamide}{PMID: 18927240}, Jeannine S McCune et al., 2009]

\hypertarget{pmid_34550448}{C}yclophosphamide is still clinically used in rheumatic diseases with severe disease courses. Cyclophosphamide has a pronounced gonadotoxic effect largely depending on the cumulative dose. The risk of amenorrhea is reported to be in the range of 12-54\% and is dependent on the age of the patient at initiation of treatment. Every patient of reproductive age should therefore be offered counseling on options for fertility protection. There are 3 options for fertility protection: oocyte harvesting and cryopreservation after a hormonal stimulation of 10-14 days, ovarian wedge resection and cryopreservation and administration of a gonadotropin-releasing hormone (GnRH) agonist. The decision whether and, if so, which treatment should be performed is made in close consultation between the patient, rheumatologists and reproductive physicians and depends on the available treatment time window, the age of the patient and the severity of the underlying disease. [\hyperlink{Cyclophosphamide}{PMID: 34550448}, Philippos Edimiris et al., 2021]

\hypertarget{pmid_6684023}{C}yclosphosphamide, dissolved in saline, was injected into the air sac of white Leghorn chick eggs in dose levels of 0.005, 0.007, 0.010, 0.012, 0.015, and 0.017 mg per egg. Eggs received a single injection of cyclophosphamide on Days 0, 1, 2, or 3 of incubation. Control eggs were injected with an equivalent volume of saline (0.1 ml per egg). In all 904 chicken eggs were used for this study. Surviving embryos were sacrificed when they reached 11 days of incubation. The LD50 values for Days 1, 2, and 3 were 0.017, 0.007, and 0.012 mg per egg, respectively. The overall incidence of abnormal embryos for Days 0, 1, 2, and 3 were 7, 6.3, 12, and 22\%, respectively. Abnormalities such as reduced body size, everted viscera, short and twisted limbs, eye defects, abnormal beak, and short and twisted neck were commonly seen in survivors no matter when exposed to cyclophosphamide. The teratogenicity of cyclophosphamide was noted to be the highest in the embryos treated on Day 3. The present study has demonstrated that cyclophosphamide is toxic and teratogenic during the period of early organogenesis in the chick embryos. [\hyperlink{Cyclophosphamide}{PMID: 6684023}, S H Gilani et al., 1983]

\hypertarget{pmid_19882369}{C}yclophosphamide (Cy) is an alkylating agent used over the past 40 years to halt rapidly progressive forms of multiple sclerosis (MS). High doses of Cy produce marked immunosuppression and an anti-inflammatory immune deviation. Cy is most effective in young patients, with very active MS (frequent relapses, rapid accumulation of disability, and gad+ lesions on brain MRI). Monthly intravenous pulses of Cy for 1 year, followed by bimonthly pulses for the second year are a well-tolerated protocol in MS. Most side effects (mild alopecia, nausea and vomiting, and cystitis) are transient, dose dependent, and reversible. Permanent amenorrhoea and bladder cancer have rarely been described. As second-line therapy, Cy can be used in non-responders to IFN-beta or glatiramer acetate. As induction therapy, a short course (6-12 months) of Cy can precede immunomodulatory drugs in selected patients with an aggressive MS onset. [\hyperlink{Cyclophosphamide}{PMID: 19882369}, Luciano Rinaldi et al., 2009]

\hypertarget{pmid_10363852}{R}esults of a phase II trial of cyclophosphamide (CPM) for children with progressive low-grade astrocytoma are reported. Fifteen patients with a median age of 39 months (range, 2 to 71) were included in this study. The tumors of 11 children were located in the optic pathway, hypothalamus, or thalamus. Four courses of intravenous CPM 1.2 g/m2 were administered every 3 weeks during the upfront window portion of this protocol. Subsequently, chemotherapy was to continue with CPM, vincristine, and carboplatin for 2 years. By study design, the first 14 patients were centrally reviewed after completion of the initial 4 CPM courses. Toxicity was primarily hematologic. One patients had a complete response, 8 had stable disease, and 5 had progressive disease (PD). The excessive number of children with PD prompted study closure. CPM as used in this protocol showed insufficient activity against astrocytoma to justify further patient accrual. [\hyperlink{Cyclophosphamide}{PMID: 10363852}, R P Kadota et al., ]

\hypertarget{pmid_392408}{T}here is good, controlled evidence which suggests that cyclophosphamide, and perhaps related drugs, have a definite role in the treatment of nephrotic children with the minimal change lesion. This role is one of secondary treatment, and the drugs should not be used as a first line of attack; they should be employed only when corticosteroid resistance or toxicity is a problem. In a few patients, azathioprine or 6-mercaptopurine may have a role in minimising corticosteroid toxicity, but the remission induced in relapsing children is no more durable than that after corticosteroids. Chlorambucil must be given in doses, and for periods long enough to run the risk of neoplasia, particularly leukaemia; there does not appear to be a place for its use in nephrotic children unless the duration of remission can be shown to be longer than that obtainable with cyclophosphamide. There is no evidence that any immunosuppressive agent has a place in the management of children with idiopathic glomerular disease showing structural alterations in the glomeruli. Children with systemic lupus erythematosus and nephritis may benefit from the addition of cytotoxic agents to their corticosteroid regime, although the indications for this are not clear, and controlled evidence is lacking. [\hyperlink{Cyclophosphamide}{PMID: 392408}, J S Cameron et al., 1979]

\hypertarget{pmid_34818796}{C}yclophosphamide (CP) is a broad-spectrum anticancer drug and has been frequently detected in aquatic environments due to its incomplete removal by wastewater treatment facilities and slow degradation in waters. Its toxicity in fish remains largely unknown. In this study, zebrafish eggs <4 h post fertilization (hpf) were exposed to CP at the concentrations from 0.5 to 50.0 μg/L until 168 hpf, and its toxicity was evaluated by biochemical, transcriptomic, and behavioral approaches. The results showed that malformation and mortality rates increased with CP concentrations. The 7-day malformation EC [\hyperlink{Cyclophosphamide}{PMID: 34818796}, Dan Li et al., 2022] Cyclophosphamide is commonly used in the treatment of children with malignant brain tumors. The purpose of this study was to develop a multicycle, high-dose intensity cyclophosphamide regimen with granulocyte-macrophage colony-stimulating factor (GM-CSF) and to assess its activity against malignant glioma and primitive neuroectodermal tumor (PNET). Twenty-three patients with brain tumors, including 15 with malignant glioma and six with PNET, were enrolled. Cyclophosphamide (1.8-2.25 g/m2/day for 2 days i.v.; total dose 3.6-4.5 g/m2) was administered and was followed by recombinant human GM-CSF (5 micrograms/kg/day s.c.) on days 3-11 or until the absolute granulocyte count reached 1.5 x 10(9)/L. With a total of 83 cycles administered, the mean dose intensity of cyclophosphamide ranged from 1.5 g/m2/week through cycle 2 (22 patients) to 0.8 g/m2/week through cycle 8 (two patients). No activity was seen against malignant glioma, and five of six patients with PNET had partial responses. The mean duration of a neutrophil count of < 0.5 x 10(9)/L was only 8 days; the platelet recovery was substantially longer. Fever during neutropenia occurred in 54 of 83 cycles. One patient died from transfusion-related graft-versus-host disease. A cyclophosphamide regimen equal to twice the dose intensity of that used in conventional therapy was administered. The regimen was active against PNET but inactive against malignant glioma. [\hyperlink{Cyclophosphamide}{PMID: 34818796}, T G Abrahamsen et al., 1995]

\hypertarget{pmid_34374211}{C}yclophosphamide (CYP) is a widely used antineoplastic and immunosuppressive drug, however, despite its efficacy, it has shown extensive multiple organ toxicities, including peripheral neuropathy which significantly affects the quality of life of cancer patients. This study elucidated the protective properties of Shorea roxburghii polyphenol extract (SLPE) in CYP-induced peripheral neuropathy. Rats were treated with SLPE (100 and 400 mg/kg) for five weeks plus CYP once a week from the second week of SLPE treatment. Using UHPLC-QTOF-MS, 54 polyphenolic compounds were identified in SLPE extract. After the treatment period the antinociceptive, anti-hyperalgesia and antiallodynic effects was evaluated using formalin paw edema, acetic acid abdominal writhing, hot plate, tail immersion and von Frey filament tests. While the locomotive and motor coordination effects were evaluated by open field and rotarod tests. The administration of CYP led to significant increases in mechanical and thermal hyperalgesia, in addition to hyper-nociceptive responses in the formalin and acetic acid writhing tests. CYP also significantly reduced locomotive activity and motor coordination. SLPE significantly protected against CYP-induced mechanical and thermal hyperalgesia. Furthermore, SLPE displayed robust antinociceptive effect by counteracting formalin and acetic acid induced hyper-nociception. In addition, SLPE increased the locomotive activity as well as the grip and motor coordination of the CYP treated rats. In conclusion, these results revealed the protective effects of SLPE against CYP-induced peripheral neuropathy and could be an effective therapeutic remedy for chemotherapy induced peripheral neuropathy. [\hyperlink{Cyclophosphamide}{PMID: 34374211}, Haili Wang et al., 2021]

\hypertarget{pmid_11322378}{F}or over 30 years cyclophosphamide (CYC) and chlorambucil (CHL) have been used to treat children with relapsing steroid-sensitive nephrotic syndrome (SSNS). A meta-analysis on treatment protocols, efficacy, and side effects of CYC and CHL was performed from the literature. Thirty-eight studies comprising 1,504 children and 1,573 courses of cytotoxic drug therapy were systematically evaluated. Relapse-free survival rates increased with the cumulative dosage of CHL and CYC and were higher in children with frequently relapsing than steroid-dependent NS. The fatality rate of the treatment was approximately 1\%. Leukopenia occurred in one-third of patients treated with either drug. Severe bacterial infections developed in 1.5\% of the patients under CYC and in 6.8\% under CHL. Seizures were observed in 3.6\% of children treated with CHL. Malignancies were observed in 14 children after high doses of either drug. Females rarely developed permanent gonadal damage. However, no safe threshold for a cumulative amount of CYC was found in males, but there was a marked increase in the risk of oligo- or azoospermia with higher cumulative doses. From this meta-analysis we recommend CYC 2-3 mg/kg body weight for 8-12 weeks as the standard scheme. CHL has higher rates of severe side effects and should be considered a second-line drug. [\hyperlink{Cyclophosphamide}{PMID: 11322378}, K Latta et al., 2001]

\hypertarget{pmid_6229699}{C}yclophosphamide is widely used to induce a remission of minimal-change nephropathy, but concerns have been raised about whether its effects on cellular immunity persist after treatment is discontinued. We studied functional and numerical measures of cellular immunity in children who had minimal-change nephropathy with frequent steroid-responsive relapses and were receiving cyclophosphamide (2.5 mg per kilogram of body weight per day for eight weeks). Sequential studies during such treatment showed that cyclophosphamide caused lymphopenia, particularly among T helper cells, resulting in a significant fall in the immunoregulatory (helper/suppressor) cell ratio. This change persisted 1 to 3 months after cyclophosphamide was discontinued, but measures of immune function reverted to normal after 6 to 12 months. Children with minimal-change nephropathy in long-term remission had no difference in T-cell subpopulations, lymphocyte responses to mitogens, or suppressor-cell function that could be attributed to the disease itself or to the previous use of cyclophosphamide. [\hyperlink{Cyclophosphamide}{PMID: 6229699}, J Feehally et al., 1984]

\hypertarget{pmid_14722349}{T}o assess the efficacy and safety of intravenous cyclophosphamide (CYP) used in severe and refractory juvenile dermatomyositis (JDM). Retrospective case note review of the outcome of 12 patients. Assessment at 6 months of therapy in 10 of the 12 patients showed a significant improvement in muscle function as assessed by the Childhood Myositis Assessment Scale (CMAS) (P = 0.012), muscle strength (P = 0.008), global extramuscular disease score (P = 0.008), skin disease severity (P = 0.015) and lactate dehydrogenase (P = 0.028). There were reductions in creatine kinase, alanine aminotransferase, prednisolone dose and ESR, but these did not reach statistical significance. Clinical improvement was maintained after CYP until the most recent follow-up (between 6 months and 7 yr) and no severe side-effects were seen. Reversible complications included lymphopenia, herpes zoster infections and alopecia. The median cumulative dose was 4.6 g/m(2) (range 3-9 g/m(2)). The available evidence suggests that, at the doses required, risks of malignancy, infertility and gonadal failure are low. Two patients with severe treatment-resistant disease died after one dose of CYP, both of whom were ventilated prior to commencement of CYP and were thought to have died as a result of their severe disease process, and too early for clinical benefit to be obtained from the drug. In this cohort of children with severe and refractory JDM, CYP appeared to have provided major clinical benefit with no evidence of serious toxicity in the short term. [\hyperlink{Cyclophosphamide}{PMID: 14722349}, P Riley et al., 2004]

\hypertarget{pmid_10894368}{C}yclophosphamide is an alkylating agent that has shown activity in the treatment of pediatric brain tumors, including high-grade gliomas. This study was designed to evaluate the response of patients with newly diagnosed glioblastoma multiforme to pre-radiotherapy cyclophosphamide. Fourteen patients with glioblastoma multiforme were treated with high-dose cyclophosphamide (2 g/m2/day for 2 doses every 28 days) followed by either sargramostim or filgrastin. Sargramostim was given 250 microg/m2 subcutaneously twice a day continuing through the leukocyte nadir until the absolute neutrophil count was more than 1000 cells/microl for 2 consecutive days. The filgrastin dose was 10 microg/kg given subcutaneously once daily until the post nadir absolute neutrophil count was > or = 10,000 cells/microl. A total of 46 courses was given. Four patients received a total of 3 courses, 7 patients completed 4 courses and 3 patients received 2 courses. Three patients demonstrated complete response; 3 stable disease; and 8 progressive disease. The most common toxicity was hematologic, requiring platelet and packed red blood cell transfusions, with 13 admissions for neutropenia with fever. There were no deaths related to infection or bleeding. These results suggest that high-dose cyclophosphamide has modest activity with acceptable toxicity against newly diagnosed glioblastoma multiforme. [\hyperlink{Cyclophosphamide}{PMID: 10894368}, K S Bottom et al., 2000]

\hypertarget{pmid_28606234}{T}o evaluate the efficacy and safety of cyclophosphamide as a second-line drug in the treatment of children with anti-N-methyl-D-aspartate receptor (NMDAR) encephalitis. Six children with anti-NMDAR encephalitis, who showed poor response to steroids and intravenous immunoglobulin, were given cyclophosphamide as a second-line immunotherapy. Follow-up was performed to evaluate the efficacy and safety of cyclophosphamide. After first-line immunotherapy for 1-4 weeks, the six patients had reduced psychiatric symptoms, seizures, and involuntary movements; three patients had an improved level of consciousness and were able to make simple conversations. However, all the patients still showed slow response, as well as cortical dysfunction symptoms such as aphasia, alexia, agraphia, acalculia, apraxia, and movement disorders. The six patients continued to receive cyclophosphamide as a sequential therapy. They were able to answer simple questions 7 days after treatment. Three school-aged patients were able to make simple calculation, had greatly improved reading and writing ability, and almost recovered self-care ability 2-3 weeks later. The cognitive function of the six patients was almost restored to the level before the onset of disease, and their living ability returned to normal 2-3 months later. During the treatment period, there were no adverse reactions or abnormal results of routine blood test and liver and kidney function tests. Children with anti-NMDAR encephalitis should be given appropriate immunotherapy as soon as possible. Cyclophosphamide as a sequential therapy has good efficacy and safety. [\hyperlink{Cyclophosphamide}{PMID: 28606234}, Wei-Wen Zhu et al., 2017]

\hypertarget{pmid_29342499}{I}n patients with severe or refractory juvenile dermatomyositis (DM), second-line treatments may be required. Cyclophosphamide (CYC) is used to treat some connective tissue diseases, but evidence of its efficacy in juvenile DM is limited. This study was undertaken to describe clinical improvement in juvenile DM patients treated with CYC and model the efficacy of CYC treatment compared to no CYC treatment. Clinical data on skin, global, and muscle disease for patients recruited to the Juvenile DM Cohort and Biomarker Study were analyzed. Clinical improvement following CYC treatment was described using unadjusted analysis. Marginal structural models (MSMs) were used to model treatment efficacy and adjust for confounding by indication. Compared to the start of CYC treatment, there were reductions at 6, 12, and 24 months in skin disease (P = 1.3 × 10 Our findings indicate that CYC is efficacious with no short-term side effects. Improvements in skin, global, and muscle disease were observed. Further studies are required to evaluate longer-term side effects. [\hyperlink{Cyclophosphamide}{PMID: 29342499}, Claire T Deakin et al., 2018]

\hypertarget{pmid_7622780}{C}yclophosphamide is an alkylating agent used to treat haematologic malignant diseases and multisystem diseases with progressive glomerulonephritis. It is rarely prescribed during pregnancy. We report a case of Henoch-Schönlein purpura discovered at the end of the first trimester of pregnancy. Despite steroid therapy, glomerulonephritis worsened and 100 mg/day cyclophosphamide per os was administered from 28th week till delivery. The infant, prematurely born, was normal and did not have any haematological disorder. Congenital malformations are often reported (5 out of 19 newborns exposed in utero to cyclophosphamide), but in all those cases, there was another potentially teratogenic agent: either radiotherapy or another antineoplastic drug. Therefore, if mother's life is in jeopardy, cyclophosphamide therapy should be given and not postponed. [\hyperlink{Cyclophosphamide}{PMID: 7622780}, R Nguyen Tan Lung et al., 1995]

\hypertarget{pmid_31796512}{T}o characterize the population pharmacokinetics of cyclophosphamide, active 4-hydroxy-cyclophosphamide (4OH-CTX), and inactive carboxyethylphosphoramide mustard (CEPM), and their associations with hematologic toxicities in infants and young children with brain tumors. To use this information to provide cyclophosphamide dosing recommendations in this population. Patients received four cycles of a 1-hour infusion of 1.5 g/m Data from 171 patients (0.07-4.9 years) were adequately fitted by a two-compartment (cyclophosphamide) and one-compartment model (metabolites). Young infants (<6 months) exhibited higher mean 4OH-CTX exposure than did young children (138.4 vs. 107.2 μmol/L·h,  A 4OH-CTX exposure-toxicity association was established, and a decreased cyclophosphamide dosage for young infants was suggested to reduce toxicity in this population. Bayesian modeling to predict 4OH-CTX exposure may reduce clinical processing-related costs and provide insights into further exposure-response associations. [\hyperlink{Cyclophosphamide}{PMID: 31796512}, Olivia Campagne et al., 2020] 1. Cyclophosphamide pharmacokinetics were measured in 38 children with cancer. 2. A high degree of inter-patient variation was seen in all pharmacokinetic parameters. Cyclophosphamide half-life varied between 1.1 and 16.8 h, clearance varied between 1.2 and 10.61 h-1 m-2 and volume of distribution varied between 0.26 and 1.48 1 kg-1. 3. The half-life of cyclophosphamide was prolonged at high dose levels (P = 0.008). 4. Children who had received prior treatment with dexamethasone showed a mean increase in clearance of 2.51 h-1 m-2 (P = 0.001) presumably as a result of CYP450 enzyme induction. 5. Treatment with allopurinol or chlorpromazine was associated with a significant increase in cyclophosphamide half-life (P < 0.001 in both cases). 6. Dose and concurrent treatment may influence cyclophosphamide metabolism in vivo and thus potentially alter the drugs therapeutic effect. [\hyperlink{Cyclophosphamide}{PMID: 31796512}, S M Yule et al., 1996]

\hypertarget{pmid_26262887}{C}yclophosphamide (CP) is an oxazaphosphorine nitrogen mustard alkylating drug used for the treatment of chronic and acute leukemias, lymphoma, myeloma, and cancers of the breast and ovary. It is known to cause severe cardiac toxicity. This study investigated the protective effect of N-Acetylcysteine (NAC) on CP-induced cardiotoxicity in rats. CP resulted in a significant increase in serum aminotransferases, creatine kinase (CK), lactate dehydrogenase(LDH) enzymes, asymmetric dimethylarginine and tumor necrosis factor-α and significant decrease in total nitrate/nitrite(NOx). In cardiac tissues, a single dose of CP (200mg/kg, i.p.) resulted in significant increase in malondialdehyde and NOx and a significant decrease in reduced glutathione content, glutathione peroxidase, catalase, and superoxide dismutase activities. Interestingly, Administration of NAC (200mg/kg, i.p.) for 5 days prior to CP attenuates all the biochemical changes induced by CP. These results revealed that NAC attenuates CP-induced cardiotoxicity by inhibiting oxidative and nitrosative stress and preserving the activity of antioxidant enzymes.  [\hyperlink{Cyclophosphamide}{PMID: 26262887}, Heba H Mansour et al., 2015] Cyclophosphamide (Cyc) is an alkylating agent used to treat malignancies and autoimmune diseases, such as lupus nephritis, rheumatoid arthritis and immune-mediated neuropathies. Over the past 40 years, Cyc has also been applied to treat multiple sclerosis (MS) and the effective stabilisation of rapidly progressive forms of MS has been demonstrated in several studies. Cyc has a dose-dependent bimodal effect on the immune system. High doses have been demonstrated to induce an anti-inflammatory immune deviation (i.e., suppression of T helper 1 and enhancement of T helper 2 activity), affect CD4CD25(high) regulatory T cells and establish a state of marked immunosuppression. Data from the literature suggest that Cyc is particularly indicated in the treatment of young MS patients, suffering from a very active inflammatory disease characterised by frequent relapses and rapid accumulation of disability and displaying gadolinium-enhancing lesions on brain magnetic resonance. The most common Cyc-based therapeutic protocol applied in MS consists of monthly intravenous pulses for 1 year followed by bimonthly pulses for the second year, with or without prior infusion of corticosteroids. This protocol is usually well tolerated by the patients. Indeed, most of the side effects (mild alopecia, nausea and vomiting, cystitis) are dose dependent, transient and completely reversible. Definitive amenorrhoea is observed only in older female patients (aged > 40 years). Cyc has a safety and efficacy profile similar to that of mitoxantrone and can be used in patients whose disease is not controlled by IFN-beta or glatiramer acetate. Short course (6-12 months) of Cyc therapy can precede the initiation of immunomodulatory treatment in selected patients with an aggressive MS onset. [\hyperlink{Cyclophosphamide}{PMID: 26262887}, Paola Perini et al., 2007]

\hypertarget{pmid_30357088}{C}yclophosphamide is a chemotherapy drug for the treatment of cancer. Chicken embryo amniotic fluid, vitamin C and coenzyme Q10 have anti-oxidant properties. Total of 70 adult female mice were selected and divided into seven groups. The first group that received 2 ml kg [\hyperlink{Cyclophosphamide}{PMID: 30357088}, Alireza Kabirian et al., 2018] Cyclophosphamide is used extensively to treat malignancies. A 5-year-old boy with stage IV neuroblastoma is described who developed a fatal syndrome of inappropriate antidiuretic hormone (ADH) secretion after high dose cyclophosphamide therapy. [\hyperlink{Cyclophosphamide}{PMID: 30357088}, P J Harlow et al., 1979]

\section*{Dibenzyline}
\subsection*{Result}
\subsubsection*{Answer}
Unknown
\subsubsection*{{Explanation}}
\hypertarget{Dibenzyline}
A review of the provided abstracts reveals that none of them are about Dibenzyline (phenoxybenzamine). The abstracts discuss a variety of other medications and compounds, including midazolam, diphenhydramine, dibucaine, diazepam, adefovir dipivoxil, and others, but there is no mention of Dibenzyline or studies specifically evaluating its safety in children. Therefore, based on the abstracts available, there is not enough data to determine whether Dibenzyline is safe or unsafe for use in children in any age range.

\subsection*{Abstracts}
\hypertarget{pmid_16719877}{T}he purpose of this study was to compare the safety and efficacy of oral midazolam and midazolam-diphenhydramine combination to sedate children undergoing magnetic resonance imaging (MRI). We performed a prospective randomized double-blind study in 96 children who were randomly allocated into two groups. Group D received oral diphenhydramine (1.25 mg x kg(-1)) with midazolam (0.5 mg x kg(-1)), and Group P received oral placebo with midazolam (0.5 mg x kg(-1)) alone. Sedation scores, onset and duration of sleep were evaluated. Adverse effects, including hypoxemia, failed sedation, and the return of baseline activity, were documented. Diphenhydramine facilitated an earlier onset of midazolam sedation (P < 0.01), and higher sedation scores (P < 0.01). In children who received midazolam alone, 20 (41\%) were inadequately sedated, compared with 9 (18\%) children who received midazolam and diphenhydramine combination (P < 0.01). Time to complete recovery was not significantly different between the two groups. Our study indicates that the combination of oral diphenhydramine with oral midazolam resulted in safe and effective sedation for children undergoing MRI. The use of this combination might be more advantageous compared with midazolam alone, resulting in less sedation failure during MRI. [\hyperlink{Dibenzyline}{PMID: 16719877}, Mustafa Cengiz et al., 2006]

\hypertarget{pmid_822269}{S}erum concentrations and therapeutic effects of DBED-Penicilline was checked in 22 resp. 26 children, 1-14 years of age. 15 min up to 2 hours and more following oral application of DBED-Penicilline a bactericide serum level was observed. The clinical effectiveness of orally administered DBED-Penicilline was proved in airway-infections of children. The flavor of the drug was well accepted. [\hyperlink{Dibenzyline}{PMID: 822269}, R Gädeke et al., 1976]

\hypertarget{pmid_20224995}{D}ibucaine is a potent, long-lasting local anesthetic (LA). Topical dibucaine ointments are marketed directly to consumers in the USA without prescription. Dibucaine ointment is intended to treat discomfort associated with sunburn, eczema, minor rashes, minor scratches, insect bites, and poison ivy and is used alone or in combination with other active ingredients to treat pain associated with hemorrhoids or other anorectal disorders. Oral dibucaine toxicosis has been reported in children and includes gastrointestinal upset and neurologic and cardiovascular dysfunction. An 18-month-old, female, Parson Russell terrier ingested approximately 23 g of 1\% dibucaine ointment (approximately 38 mg/kg dibucaine) recommended to the owner for the treatment of hemorrhoids. Onset and resolution of clinical signs were relatively rapid, 5 min and 60 min, respectively. Clinical signs included vomiting, ptyalism, whole-body muscle fasciculations, disorientation, and severe ataxia. Oral dibucaine toxicosis in dogs is similar to oral dibucaine toxicosis in children. Dibucaine ointment poses a real and potentially serious toxicological risk to pets and thus should be stored in a safe location. [\hyperlink{Dibenzyline}{PMID: 20224995}, Andrew S Hanzlicek et al., 2010]

\hypertarget{pmid_18276803}{T}here is a continued need for safe and effective treatments for children and adolescents with chronic hepatitis B. Adefovir dipivoxil (ADV) is a licensed treatment for chronic hepatitis B in adults. This study was designed to characterize the pharmacokinetic profile of adefovir following the administration of 0.14 mg/kg and 0.3 mg/kg of ADV (oral solution) in children aged 2 to 11 years and of ADV 10 mg in adolescents aged 12 to 17 years. Forty-five subjects were included in the pharmacokinetic and safety evaluations. Adefovir was rapidly absorbed. Adefovir levels rose rapidly in the first hour and then declined in a biphasic manner. Dose-proportional pharmacokinetics was observed in the 0.14-mg/kg and 0.3-mg/kg groups. The 0.3-mg/kg dose in children aged 2 to 6 and the 10-mg dose in adolescents resulted in exposures that were comparable to those seen previously in adults given ADV 10 mg. Adefovir dipivoxil was well tolerated at the doses evaluated in this study. Adverse events were generally mild and reported as being unrelated to study medication. There was 1 serious adverse event reported that was not related to study medication. No patient discontinued the study prematurely due to an adverse event related to the study drug. [\hyperlink{Dibenzyline}{PMID: 18276803}, Etienne M Sokal et al., 2008]

\hypertarget{pmid_6708895}{T}he antinausea drug combination, doxylamine/dicyclomine/pyridoxine (Debendox or Bendectin [US] ), has been withdrawn from the market because of litigation based upon charges that it causes congenital limb defects. To investigate this allegation, the pregnancy histories of mothers of 155 limb-deficient children, born between 1970 and 1981, have been compared with those of mothers of 273 matched normal controls. There was no significant difference between the pregnancy histories of mothers of case children and those of mothers of control children in respect of reported frequency of morning sickness, the use of doxylamine/dicyclomine/pyridoxine, the date of commencement of its intake, the duration of intake, or its dose. The relative risk of limb deficiency in children of mothers exposed to this drug is estimated to be 1.1 with confidence limits of 0.8-1.5. No risk of congenital limb defects was found to be associated with the use of this drug. [\hyperlink{Dibenzyline}{PMID: 6708895}, J McCredie et al., 1984]

\hypertarget{pmid_19834416}{D}ibucaine is considered one of the most potent and consequently toxic amide anesthetics available, and despite withdrawal from the US market as a spinal anesthetic, it remains accessible as an over-the-counter preparation in the United States. Dibucaine exposures in children are infrequently encountered, but to date, all reported consequential ingestions have resulted in death. We report the first case of a potentially fatal dibucaine-induced wide-complex arrhythmia in a child who survived her clinical course without sequelae. It is our hope that this report will highlight the toxicity of dibucaine and prompt a review of its over-the-counter status. The rationale and success of a new antidote, 20\% lipid emulsion, for the management of local anesthetic toxicity is discussed. [\hyperlink{Dibenzyline}{PMID: 19834416}, Jamie Nelsen et al., 2009]

\hypertarget{pmid_12665005}{D}iphenhydramine is an antihistamine available in numerous over-the-counter preparations. Often used for its sedative effects in adults, it can cause paradoxical central nervous system stimulation in children, with effects ranging from excitation to seizures and death. Reports of fatal intoxications in young children are rare. We present five cases of fatal intoxication in infants 6, 8, 9, 12, and 12 weeks old. Postmortem blood diphenhydramine levels in the cases were 1.6, 1.5, 1.6, 1.1 and 1.1 mg/L, respectively. Anatomic findings in each case were normal. In one case the child's father admitted giving the infant diphenhydramine in an attempt to induce the infant to sleep; in another case, a daycare provider admitted putting diphenhydramine in a baby bottle. Two cases remain unsolved; one case remains under investigation. The postmortem drug levels in these cases are lower than seen in adult fatalities. We review the literature on diphenhydramine toxicity, particularly as it pertains to small children, and discuss the rationale for treating these cases as fatal intoxications. [\hyperlink{Dibenzyline}{PMID: 12665005}, Andrew M Baker et al., 2003]

\hypertarget{pmid_8510706}{P}henobarbital, once widely prescribed to prevent febrile seizures, is now in disfavor because of its side effects and lack of efficacy. Diazepam, administered only during episodes of fever, may be a safe, effective agent to prevent the recurrence of febrile seizures. We conducted a randomized, double-blind, placebo-controlled trial among 406 children (mean age, 24 months) who had at least one febrile seizure. Diazepam (0.33 mg per kilogram of body weight) or placebo was administered orally every eight hours during all febrile illnesses. During a mean follow-up of 1.9 years (a period during which 90 percent of febrile seizures recur), our intention-to-treat analysis showed a reduction of 44 percent in the risk of febrile seizures per person-year with diazepam (relative risk = 0.56; 95 percent confidence interval, 0.38 to 0.81; P = 0.002). A survival analysis of the length of time to the first recurrent febrile seizure did not show a significant difference between the treatment groups (P = 0.064 by the log-rank test), but after adjustment for covariates, diazepam was found to have a benefit (P = 0.027 by Cox regression analysis). An analysis restricted to children who had seizures while actually receiving the study medication (7 in the diazepam group and 29 in the placebo group) showed an 82 percent reduction in the risk of febrile seizures with diazepam (relative risk = 0.18; 95 percent confidence interval, 0.09 to 0.37; P < 0.001). Of the 153 children who took at least one dose of diazepam, 39 percent had ataxia, lethargy, or irritability or at least one other moderate side effect that was reversed after a reduction in the dose. There were no severe side effects. Oral diazepam, given only when fever is present, is safe and reduces the risk of recurrent febrile seizures. [\hyperlink{Dibenzyline}{PMID: 8510706}, N P Rosman et al., 1993]

\hypertarget{pmid_9440805}{M}idazolam is a recently developed water-soluble benzodiazepine that shares anxiolytic, muscle relaxant, hypnotic and anticonvulsant actions with other members of this class. There are limited studies that midazolam can be used successfully to treat seizures in adults and children. In this study, 0.2 mg/kg intramuscular (i.m.) midazolam was administered to 11 children (eight boys and three girls), aged 3 days to 4 years (mean age 1.8 +/- 1.4 years), with seizures of various types. In all but one child, seizures stopped in 15 s-5 min after injection. No side effects were observed. These results suggest that i.m. administration of midazolam may be useful in a variety of seizures during childhood, especially in case of intravenous (i.v.) line problem. [\hyperlink{Dibenzyline}{PMID: 9440805}, C Yakinci et al., 1997]

\hypertarget{pmid_15123028}{M}idazolam, a water-soluble benzodiazepine, is usually given intravenously in status epilepticus. The aim of this study was to determine whether intranasal midazolam is as safe and effective as intravenous diazepam in the treatment of acute childhood seizures. Seventy children aged 2 months to 15 years with acute seizures (febrile or afebrile) admitted to the pediatric emergency department of a general hospital during a 14-month period were eligible for inclusion. Intranasal midazolam 0.2 mg/kg and intravenous diazepam 0.2 mg/kg were administered after intravenous lines were established. Intranasal midazolam and intravenous diazepam were equally effective. The mean time to control of seizures was 3.58 (SD 1.68) minutes in the midazolam group and 2.94 (SD 2.62) in the diazepam group, not counting the time required to insert the intravenous line. No significant side effects were observed in either group. Although intranasal midazolam was as safe and effective as diazepam, seizures were controlled more quickly with intravenous diazepam than with intranasal midazolam. Intranasal midazolam can possibly be used not only in medical centers, but also in general practice and at home after appropriate instructions are given to families of children with recurrent seizures. [\hyperlink{Dibenzyline}{PMID: 15123028}, T Mahmoudian et al., 2004]

\hypertarget{pmid_20101040}{T}o evaluate published evidence of efficacy and safety of pharmacologic treatments for childhood spasticity due to cerebral palsy. A multidisciplinary panel systematically reviewed relevant literature from 1966 to July 2008. For localized/segmental spasticity, botulinum toxin type A is established as an effective treatment to reduce spasticity in the upper and lower extremities. There is conflicting evidence regarding functional improvement. Botulinum toxin type A was found to be generally safe in children with cerebral palsy; however, the Food and Drug Administration is presently investigating isolated cases of generalized weakness resulting in poor outcomes. No studies that met criteria are available on the use of phenol, alcohol, or botulinum toxin type B injections. For generalized spasticity, diazepam is probably effective in reducing spasticity, but there are insufficient data on its effect on motor function and its side-effect profile. Tizanidine is possibly effective, but there are insufficient data on its effect on function and its side-effect profile. There were insufficient data on the use of dantrolene, oral baclofen, and intrathecal baclofen, and toxicity was frequently reported. For localized/segmental spasticity that warrants treatment, botulinum toxin type A should be offered as an effective and generally safe treatment (Level A). There are insufficient data to support or refute the use of phenol, alcohol, or botulinum toxin type B (Level U). For generalized spasticity that warrants treatment, diazepam should be considered for short-term treatment, with caution regarding toxicity (Level B), and tizanidine may be considered (Level C). There are insufficient data to support or refute use of dantrolene, oral baclofen, or continuous intrathecal baclofen (Level U). [\hyperlink{Dibenzyline}{PMID: 20101040},  et al., 2010]

\hypertarget{pmid_12410871}{T}o determine which is the most effective and safe treatment for controlling seizures in children out-of-hospital: diazepam or midazolam. A retrospective review of the medical records of children presenting to the Emergency Department of the Children's Hospital at Westmead (CHW-ED) with seizures requiring treatment in the field by paramedics was carried out over a 4-year period (April 1996 to March 2000). In New South Wales, children with seizures in the prehospital setting received 0.5 mg/kg per rectum (p.r.) or 0.1 mg/kg i.v. diazepam until March 1998 and from March 1997 onwards they received 0.15 mg/kg i.m. or 0.1 mg/kg i.v. midazolam. The main outcome measured was cessation of seizure in the prehospital setting. Secondary outcomes were time taken to initiate treatment and the frequency of cardiorespiratory compromise. Over the 4-year period, 2566 children presented to CHW-ED with a seizure; 107 children were eligible for entry into the present study. Of these 107 patients, 62 received diazepam and 45 received midazolam. Thirty-one (50.0\%) in the diazepam group and 15 (33.3\%) in the midazolam group were febrile seizures. Both groups were similar in terms of demographics and seizure type. A comparison of diazepam with midazolam showed that both drugs were effective in stopping seizures within 5 min of drug administration (37.1\% cf. 51.1\%). Fewer patients in the midazolam group suffered apnoea (20.0\% cf. 29.0\%; P < 0.05). Midazolam controls seizures as effectively as diazepam in the prehospital setting. Furthermore, midazolam potentially reduces respiratory depression and time to treatment. [\hyperlink{Dibenzyline}{PMID: 12410871}, J Rainbow et al., 2002]

\hypertarget{pmid_8984974}{W}e present our results of intermittent prophylaxis with oral diazepam in febrile seizures. We treated 82 patients aged between 3 months and 5 years. They have had simple or complex febrile seizures. Recurrence occurred in 22 patients (26\%), none had a long-lasting febrile convulsion. Transient side effects occurred in 21.95\% of the cases. We conclude that diazepam is a safe and effective drug for prophylaxis of febrile seizures when used as soon as any sign of illness appears. We suggest, however, that the administration of the drug should be indicated if the child presents at least one consistent predictor of risk of recurrent febrile seizures. [\hyperlink{Dibenzyline}{PMID: 8984974}, M Costa et al., 1996]

\hypertarget{pmid_3774436}{W}e report one case of Digoxin intoxication in a child treated with Fab Fragments of Digoxin-Specific antibodies (Fabad), although there was no evidence of early life threatening complications. The efficacy of this treatment, which prevents further complications as well as its safety, represent strong arguments to treat children at the early stage of the intoxication in order to avoid temporary cardiac pacing. [\hyperlink{Dibenzyline}{PMID: 3774436}, P S Jouk et al., ]

\hypertarget{pmid_26488029}{D}apsone (DDS-diamino diphenyl sulphone) is a sulfone antibiotic being used for a variety of clinical conditions. Poisoning in children by DDS is rarely reported. Poisoning in acute cases will be frequently unrecognized due to relative lack of severe signs and symptoms. Methemoglobinemia is the major life-threatening situation associated with poisoning of DDS. Hence, any delay for medical attention can lead to increased rate of mortality. In this case, we describe acute DDS poisoning in a 3-year-old child and the successful management using intravenous methylene blue.  [\hyperlink{Dibenzyline}{PMID: 26488029}, Menon Narayanankutty Sunilkumar et al., 2015] Dialkyl phthalates are plasticizers used in household products made from polyvinyl chloride (PVC). Diisononyl phthalate (DINP) is the principal phthalate in soft plastic toys. Because DINP is not tightly bound to PVC, it may be released when children mouth PVC products. The potential chronic health risks of phthalate exposure to infants have been under scrutiny by regulatory agencies in Europe, Canada, Japan, and the U.S. This report describes a risk assessment of DINP exposure from children's products, by the U.S. Consumer Product Safety Commission (CPSC) staff. This report includes the findings of a CPSC Chronic Hazard Advisory Panel (CHAP) which: (1) concluded that DINP is unlikely to present a human cancer hazard and (2) recommended an acceptable daily intake (ADI) level of 120 microg/kg-d, based on spongiosis hepatis in rats. The risk assessment incorporates new measurements of DINP migration rates from 24 toys and a new observational study of children's mouthing activities, with a detailed characterization of the objects mouthed. Probabilistic methods were used to estimate exposure. Mouthing behavior and, thus, exposure depend on the child's age. Approximately 42\% of tested soft plastic toys contained DINP. Estimated DINP exposures for soft plastic toys were greatest among children 12-23 months old. The mean exposure for this age group was 0.08 (95\% confidence interval 0.04-0.14) microg/kg-d, with a 99th percentile of 2.4 (1.3-3.2) microg/kg-d. The authors conclude that oral exposure to DINP from mouthing soft plastic toys is not likely to present a health hazard to children. The opinions expressed by the authors have not been reviewed or approved by, and do not necessarily reflect the views of, the U.S. Consumer Product Safety Commission. Because this material was prepared by the authors in their official capacity, it is in the public domain and may be freely copied or reprinted. [\hyperlink{Dibenzyline}{PMID: 26488029}, Michael A Babich et al., 2004]

\hypertarget{pmid_23236934}{T}he use of midazolam for children was approved in March, 2010. Since the efficacy and safety data of midazolam used in children, excluding low-birth-weight infants and newborns, for "sedation under artificial respiration in intensive care units" were quite limited, a post-marketing survey was carried out to confirm the validity of the established dosage and administration. A consecutive enrollment method was adopted. Based on the data of 153 patients collected from 8 institutes, efficacy and safety profiles were analyzed. Among the 149 patients included in the safety analysis set, 6 adverse reactions were reported in 6 patients. The incidence of adverse events was 4.0\% (6/149). Reported adverse reactions included depressed level of consciousness: 1 event, delirium: 1 event, psychomotor hyperactivity: 1 event, hypotension: 2 events, and blood pressure increase: 1 event. Serious adverse drug reaction (ADR) reported in this survey was depressed level of consciousness. This ADR resolved on the following day after the treatment with flumazenil. Paradoxical reactions were reported in 1 patient, and tolerance was reported in 2 patients. The efficacy rate was 96.5\% (138/143). No additional safety issues (status of adverse reactions, status of adverse events, status of serious adverse events, etc.) and efficacy issue were manifest in the patients treated with the dosage and administration method established at the approval of the drug. [\hyperlink{Dibenzyline}{PMID: 23236934}, Keizo Sogabe et al., 2012]

\hypertarget{pmid_26247686}{D}ioxin concentrations remain elevated in both the environment and in humans residing near former US Air Force bases in South Vietnam. This may potentially have adverse health effects, particularly on infant neurodevelopment. We followed 214 infants whose mothers resided in a dioxin-contaminated area in Da Nang, Vietnam, from birth until 1 year of age. Perinatal exposure to dioxins was estimated from toxic equivalent (TEQ) levels of polychlorinated dibenzodioxins and polychlorinated dibenzofurans (PCDDs/Fs-TEQ), and 2,3,7,8-tetrachlorodibenzo-p-dioxin (2,3,7,8-TetraCDD) concentrations in breast milk. In infants, daily dioxin intake (DDI) was used as an index of postnatal exposure through breastfeeding. Neurodevelopment of toddlers was assessed using the Bayley Scales of Infant and Toddler Development, Third Edition (Bayley-III). No significant differences in neurodevelopmental scores were exhibited for cognitive, language or motor functions between four exposure groups of PCDDs/Fs-TEQ or 2,3,7,8-TetraCDD. However, social-emotional scores were decreased in the high PCDDs/Fs-TEQ group and the high 2,3,7,8-TetraCDD group compared with those with mild exposure, after adjusting for confounding factors. Cognitive scores in the mild, moderate, and high DDI groups were significantly higher than those in low DDI group, but there were no differences in cognitive scores among the three higher DDI groups. These results suggest that perinatal exposure to dioxins may affect social-emotional development of 1-year-old toddlers, without diminishing global neurodevelopmental function. [\hyperlink{Dibenzyline}{PMID: 26247686}, Tai The Pham et al., 2015]

\hypertarget{pmid_24829888}{P}rocedural sedation in children continues to be a problem in the emergency department (ED). Midazolam is the first water-soluble benzodiazepine and it has been widely used for procedural sedation in pediatric patients. The aim of this study was evaluation of clinical safety and effectiveness of intramuscular Midazolam for pediatric sedation in the ED setting. We performed a self-controlled clinical trial on 30 children who referred to the Baqiyatallah Hospital ED between 2009 and 2010. They received intramuscular Midazolam 0.3 mg/kg for procedural sedation and then they were followed for sedative effectiveness and safety. Vital signs and O2 saturation were also observed. The findings were compared using SPSS ver. 16 software. The mean age was 5.50 ± 2.70 years, the mean weight was 19.50 ± 6.63 kilograms and 16 patients (53.3\%) were females. The most common adverse effect was euphoria (66.66\%) and vertigo (6.7\%); 27.7\% did not show any side effects. There was an overall complication rate of 72.3\%. The vital signs including heart rate, respiratory rate, systolic and diastolic blood pressure and O2 saturation decreased significantly during sedation (P value < 0.05). Midazolam is an effective and relatively safe sedative for pediatric patients in the ED. The patient should be observed closely and monitored for psychological and hemodynamic side effects. [\hyperlink{Dibenzyline}{PMID: 24829888}, Mohammad Reza Ghane et al., 2012]

\hypertarget{pmid_23738612}{T}reatment of neonatal seizures still relies primarily on phenobarbital, despite an estimated efficacy of less than 50\% and concern over neurodegenerative side effects. The objective of this study was to evaluate the efficacy and safety of lidocaine as second-line treatment of neonatal seizures in infants following benzodiazepine treatment but without previous treatment with phenobarbital. In a 10-year cohort, a retrospective chart review was conducted for all infants (gestational age ≥ 37 w, age ≤ 28 days) who had received lidocaine as second-line treatment of neonatal seizures prior to treatment with phenobarbital between January 2000 and June 2010. Infants were included if they had electroencephalographic seizures. Cessation of seizure activity was seen in 16 of 30 infants based on clinical and electroencephalographic features, and a probable response was seen in an additional 3 of 30 patients. Suspected adverse effects were seen in only one patient, who developed a transient bradycardia. Lidocaine has a moderate efficacy as second-line therapy following benzodiazepines for treating neonatal seizures and is not frequently associated with cardiovascular adverse effects. Lidocaine should therefore be considered in the treatment of seizures in the neonatal period to a higher extent than is the case today. [\hyperlink{Dibenzyline}{PMID: 23738612}, M Lundqvist et al., 2013]

\hypertarget{pmid_12055814}{T}here are no reports documenting toxicity or adverse effects after treatment of children aged < 24 months with benzimidazole derivatives and there is an urgent need to clarify this point in light of the potential detrimental effect that soil-transmitted helminthiasis has on this age-group. A total of 653 treatments (317 mebendazole 500 mg; 336 placebo) were administered in 1996/97 to 212 children aged < 24 months as part of a 1-year anthelmintic drug study conducted among preschool-age children in Tanzania. Data on fever, cough, diarrhoea, dysentery and acute respiratory illness were collected 1 week following the treatment. No differences between the occurrence of adverse effects in the 2 groups were observed. In light of the potential nutritional benefit achieved by regular deworming in this young age-group, the policy that excludes children aged < 24 months from treatment should be re-considered. [\hyperlink{Dibenzyline}{PMID: 12055814}, Antonio Montresor et al., ]

\hypertarget{pmid_20819318}{A}llergic rhinitis (AR) and chronic idiopathic urticaria (CIU) are common causes of substantial illness and disability in preschool children. Antihistamines are commonly used to treat preschool children with these conditions, but their use is based mostly on extrapolated efficacy from adult populations; it is thus important to characterize the safety of antihistamines in the pediatric population. This study was designed to assess the safety of levocetirizine dihydrochloride oral liquid drops in infants and children with AR or CIU. Two multicenter, double-blind, randomized, parallel-group studies randomized infants aged 6-11 months (study 1, n = 69) and children aged 1-5 years (study 2, n = 173) to levocetirizine, 1.25 mg (q.d. or b.i.d., respectively), or placebo for 2 weeks, using a 2:1 ratio. Safety evaluations included treatment-emergent adverse events (TEAEs), vital signs, electrocardiographic (ECG) assessments, and laboratory tests. The overall incidence of TEAEs was similar between levocetirizine and placebo in both studies. Most TEAEs were mild or moderate in intensity. TEAEs prompted discontinuation of therapy in three patients receiving levocetirizine in study 1. No clinically relevant changes from baseline in vital signs or laboratory parameters were apparent in either study; changes from baseline in these evaluations were similar between groups. No significant changes were observed in ECG parameters, including corrected QT interval. Levocetirizine, 1.25 and 2.5 mg/day, was well tolerated in infants aged 6-11 months and in children aged 1-5 years, respectively, with AR or CIU. [\hyperlink{Dibenzyline}{PMID: 20819318}, Frank Hampel et al., ]

\hypertarget{pmid_498689}{T}he safety and efficacy of diazoxide administered intravenously in the treatment of children with acute severe hypertension have been evaluated by a collaborative study. Observations of the response of blood pressure in 36 patients, ranging in age from two months to 18 years, during the initial episode of hospitalization reveal diazoxide treatment to be effective in lowering blood pressure in 94 per cent of the cases. No serious adverse circulatory, fluid and electrolyte, metabolic or hematologic effects were observed. Symptomatic and subjective reactions observed with diazoxide administered intravenously to children were identical with those described in adults. Reinstitution of other means of antihypertensive therapy is safe and effective when delayed until the transiently induced period of hypotension has passed. Repeated use of diazoxide for subsequent recurrence of severe hypertension was equally effective and safe in 93 per cent of the instances. The results lead us to recommend the use of intravenous diazoxide for treatment of children with severe symptomatic hypertension especially when it is refractory to control by other hypertensive agents. [\hyperlink{Dibenzyline}{PMID: 498689}, W W McCrory et al., 1979]

\hypertarget{pmid_23792085}{T}he dietary exposure of infants to polychlorinated dibenzo dioxins and furans (PCDD/Fs) and dioxin like polychlorinated biphenyls (dl-PCBs) is an issue of great social impact. We investigated for the first time the dietary intake of these compounds in infants living in Greece. We included in our study two age groups: 0-6 months, when infants are fed exclusively by human milk and/or formula milk, and 6 to 12 months, when solid food is introduced to nutrition. We took into consideration analytical results for PCDD/Fs and dl-PCBs concentrations in the most popular infant formulae in the Greek market, previous data for mother milk concentrations of PCDD/Fs and dl-PCBs from Greece, and finally analytical data for fat-containing food products from the Greek market. In the first study group, it was found than in infants exclusively fed by breast milk, the calculated sum of PCDD/Fs and dioxin-like PCBs (60.3-80.4 TEQ pg/kg body weight) was significantly higher than that of infants that consume a combination of human milk and formula (31.2-41.6 TEQ pg/kg body weight). In the second study group, separate daily intake estimations were performed for babies receiving human milk (estimated total daily intake 19.76-24.95 TEQ pg/kg body weight) and formula milk (estimated total daily intake 1.60-2.24 TEQ pg/kg body weight). The risks of this exposure should not be overestimated because nursing is restricted to a limited period of human life and besides, the potential consumption of higher levels of dioxin-like compounds is fully compensated by the significant benefits of breast-feeding. [\hyperlink{Dibenzyline}{PMID: 23792085}, Danae Costopoulou et al., 2013]

\hypertarget{pmid_16396300}{A}n analysis of the use of diacarb in treatment of 96 children with epilepsy and febrile seizures was carried out. Positive effects of the medication have been observed in these patients in terms of decreasing episodes of epilepsy seizures, protecting against recurrence of febrile seizures as well as positive electroencephalogram dynamics against background of the treatment. It enables the medication to be used as a preventive means of seizures in children with epilepsy and febrile seizures. [\hyperlink{Dibenzyline}{PMID: 16396300}, L N Tantsura et al., ]

\section*{Diclofenac Potassium}
\subsection*{Result}
\subsubsection*{Answer}

Unknown

\subsubsection*{{Explanation}}
\hypertarget{Diclofenac Potassium}
To determine if Diclofenac Potassium is safe for use in children, I reviewed all available abstracts for targeted studies on its safety in pediatric populations. The following summarizes the evidence:

- One abstract specifically mentions Diclofenac Potassium (not sodium) in a clinical trial for sprained ankles, but does not specify the age range of the patients. The study included 93 patients with mild to severe sprained ankles, but there is no explicit mention that children were included, nor is there a breakdown of age groups. The study reports no serious side effects and good tolerability, but without clear evidence that children were studied, this cannot be considered proof of safety in children [\hyperlink{pmid_2111251}{PMID: 2111251}, L A Bahamonde et al.].

- Another study assessed Diclofenac Potassium for menstrual pain and exercise performance in "young women," but does not specify inclusion of children or adolescents under 18. Therefore, this cannot be used to determine safety in children [\hyperlink{pmid_19038583}{PMID: 19038583}, Ingrid Chantler et al., 2009].

- Several other abstracts provide evidence for the safety of diclofenac sodium (various formulations) in children, including studies in children aged 1-16 years, but these do not specifically address Diclofenac Potassium.

- The pharmacokinetic and safety studies, as well as clinical trials in children, focus on diclofenac sodium or do not specify the salt form, or they are animal studies.

In summary, based on the abstracts available, there are no targeted studies that specifically evaluate the safety of Diclofenac Potassium in children. Therefore, the safety of Diclofenac Potassium in children is unknown.

\subsection*{Abstracts}
\hypertarget{pmid_31768103}{D}iclofenac sodium (DS), a potent inhibitor of cyclooxygenase, reduces the release of arachidonic acid and formation of prostaglandins. Being a nonsteroid drug that shows antiinflammatory action, the possible side effects of fetal DS administration gain importance in public and medical applications. Herein, the effects of DS administration (1 mg/kg) during gestational days 5-20 were investigated on the performance of Wistar rat pups in a variety of behavioral tasks. Four-week-old pups were subjected to sensory motor tests, a plus maze, an open field, the Morris water maze, and a radial arm maze. Fetal DS disrupted some sensory motor performances, such as visual placing and climbing in both females and males. In the open field, DS females had a higher level of anxiety and male DS pups habituated to the environment slowly compared to controls. The DS pups showed slower rates of learning, whereas no substantial between-group differences were found in the performance of spatial memory compared to both controls. Furthermore, working memory was negatively affected by fetal DS. In conclusion, it was indicated that DS administration during pregnancy had slight behavioral impacts with a delay in learning and a defect in the short-term memory in juvenile rats. [\hyperlink{Diclofenac Potassium}{PMID: 31768103}, Birsen Elibol et al., 2019]

\hypertarget{pmid_21276131}{D}iclofenac is an effective, opiate-sparing analgesic for acute pain in children, which is commonly used in pediatric surgical units. Recently, a Cochrane review concluded the major knowledge gap in diclofenac use is dosing information. A pharmacokinetic meta-analysis has been undertaken with the aim of recommending a dose for children aged 1-12 years. Studies containing diclofenac pharmacokinetic data were identified during a Cochrane systematic review, and authors were asked to provide raw data. A pooled population analysis was undertaken in NONMEM to define the pharmacokinetics of intravenous, oral, and rectal diclofenac in children. Simulations were performed to recommend a dose yielding an equivalent area under diclofenac concentration-time curve (AUC) to a 50-mg dispersible tablet in adults. Data from 111 children aged 1-14 years consisting of 375 samples following intravenous, oral suspension, and suppositories were used. Adult dispersible tablet and suspension data were added to provide a reference AUC and support the absorption modeling, respectively. A three-compartment model described disposition, a dual-absorption compartment model was used for suspension and dispersible tablet data, and single-absorption compartment model for suppositories. The estimate of clearance was 16.5 l·h(-1) ·70 kg(-1) and bioavailabilities were 0.36, 0.63, and 0.35 for suspension, suppository, and dispersible tablets, respectively. Single doses of 0.3 mg·kg(-1) for intravenous, 0.5 mg·kg(-1) for suppositories, and 1 mg·kg(-1) for oral diclofenac in children aged 1-12 years are recommended as they yield a similar AUC to 50 mg in adults. [\hyperlink{Diclofenac Potassium}{PMID: 21276131}, Joseph F Standing et al., 2011]

\hypertarget{pmid_24815417}{D}iclofenac dosing in children for analgesia is currently extrapolated from adult data. Oral diclofenac 1.0 mg·kg(-1) is recommended for children aged 1-12 years. Analgesic effect from combination diclofenac/acetaminophen is unknown. Children (n = 151) undergoing tonsillectomy (c. 1995) were randomized to receive acetaminophen elixir 40 mg·kg(-1) before surgery and 20 mg·kg(-1) rectally at the end of surgery with diclofenac suspension 0.1 mg·kg(-1) , 0.5 mg·kg(-1) , or 2.0 mg·kg(-1) before surgery or placebo. A further 93 children were randomized to receive diclofenac 0.1 mg·kg(-1) , 0.5 mg·kg(-1) , or 2.0 mg·kg(-1) only. Postoperative pain was assessed (visual analogue score, VAS 0-10) at half-hourly intervals from waking until discharge. Data were pooled with those from a further 222 children and 30 adults. One-compartment models with first-order absorption and elimination described the pharmacokinetics of both medicines. Combined drug effects were described using a modified EMAX model with an interaction term. An interval-censored model described the hazard of study dropout. Analgesia onset had an equilibration half-time of 0.496 h for acetaminophen and 0.23 h for diclofenac. The maximum effect (EMAX ) was 4.9. The concentration resulting in 50\% of EMAX (C50 ) was 1.23 mg·l(-1) for diclofenac and 13.3 mg·l(-1) for acetaminophen. A peak placebo effect of 6.8 occurred at 4 h. Drug effects were additive. The hazard of dropping out was related to pain (hazard ratio of 1.35 per unit change in pain). Diclofenac 1.0 mg·kg(-1) with acetaminophen 15 mg·kg(-1) achieves equivalent analgesia to acetaminophen 30 mg·kg(-1) . Combination therapy can be used to achieve similar analgesia with lower doses of both drugs. [\hyperlink{Diclofenac Potassium}{PMID: 24815417}, Jacqueline A Hannam et al., 2014]

\hypertarget{pmid_11952445}{N}ausea, vomiting and pain are common complications after strabismus surgery in children. Diclofenac, a non-steroid anti-inflammatory drug, is widely used to treat acute and chronic pain but there are few reports of its use given rectally in children undergoing strabismus surgery. This open randomised study was designed to investigate the analgesic and anti-emetic properties of rectally administered diclofenac compared with opioid (morphine) given i.v. in connection with strabismus surgery in children. After obtaining approval from the local ethics committee and written informed consent from the parents, 50 ASA class I-II children, 4-16 years of age, were randomised to receive either rectally administered diclofenac (Voltaren) 1 mg/kg or i.v. opioid (morphine) 0.05 mg/kg perioperatively. The children were consecutively operated upon from May 1999 to January 2001. Anaesthesia was induced with fentanyl and propofol and maintained with propofol. Nitrous oxide was omitted. The postoperative pain was assessed after arrival at the post anaesthesia care unit (PACU) by using the validated Wong and Baker scale (FACES) Pain Rating Scale. Postoperative nausea and vomiting (PONV) was assessed by measuring the frequency of vomiting and the degree of nausea. In the diclofenac group the incidence of PONV during the first 24 h was 12\% (of which one child had severe vomiting). The incidence of PONV was much higher, 72\% (P = 0.0000), in the morphine group, where 56\% of the children also had severe vomiting. There were no difference in pain score between the two groups. Recovery time at the PACU was longer (P < 0.002) and the postoperative analgesic requirement higher in the morphine group (10 vs. 5 children). No children needed overnight admission to the hospital. Diclofenac given rectally is an effective analgesic for this kind of surgery and gives less postoperative nausea than i.v. morphine. No serious adverse events were observed. [\hyperlink{Diclofenac Potassium}{PMID: 11952445}, B Wennström et al., 2002]

\hypertarget{pmid_2111251}{I}n a double-blind between-patient study the efficacy of diclofenac potassium, a non-steroidal anti-inflammatory drug, was assessed in 93 patients with mild to severe sprained ankles; patients with more severe sports injuries were excluded. Patients were randomly allocated to receive 50 mg diclofenac potassium three times daily, 20 mg/day piroxicam or placebo for 7 days. Diclofenac potassium was more effective than piroxicam or placebo in reducing pain at rest and on walking, but did not significantly reduce the degree of swelling when measured volumetrically by water displacement. No serious side-effects were reported. It is concluded that diclofenac potassium is useful in the treatment of moderately inflammatory processes with the advantage that it had a rapid onset of action with good overall tolerability. [\hyperlink{Diclofenac Potassium}{PMID: 2111251}, L A Bahamonde et al., ]

\hypertarget{pmid_3377147}{A} controlled investigation was conducted to compare the effectiveness of diclofenac and papaveretum in the prevention of pain and restlessness after tonsillectomy in children. Sixty children between 3 and 13 years of age were randomly allocated to receive rectal diclofenac 2 mg/kg, intramuscular papaveretum 0.2 mg/kg or no medication immediately after induction of anaesthesia. Pain and appearance were assessed 1, 3 and 6 hours postoperatively, and the following morning. The assessments were double-blind and performed by the same observer. No significant differences in postoperative pain were found between the groups at any time. The use of diclofenac was associated with a significantly more rapid return to calm wakefulness and had significantly less effect upon respiratory rate. Consumption of paracetamol on the day of operation was significantly less in the diclofenac group. Diclofenac may offer advantages compared to papaveretum with regard to safety and convenience for use in the treatment of pain after tonsillectomy in children. [\hyperlink{Diclofenac Potassium}{PMID: 3377147}, M E Bone et al., 1988]

\hypertarget{pmid_2340849}{D}iclofenac sodium 0.5 mg/kg i.v. was given preoperatively to small children (age 4-6 y). Vt and total plasma clearance were higher than in adults but the elimination half-life was similar. [\hyperlink{Diclofenac Potassium}{PMID: 2340849}, R Korpela et al., 1990]

\hypertarget{pmid_37070921}{T}he aim of the present study was to assess the safety and efficacy of Diclofenac sodium (DS) 140 mg medicated plaster vs. Diclofenac epolamine (DIEP) 180 mg medicated plaster and placebo plaster, for the treatment of painful disease due to traumatic events of the limbs. This was a multicenter, phase III study involving 214 patients, aged 18-65 years, affected by painful conditions due to soft tissue injuries. Patients were randomized to DS, DIEP or placebo arms and treated with once-daily application of the plaster for a total treatment period of 7 days. The primary objective was first to demonstrate the non-inferior efficacy of the DS treatment when compared to the reference DIEP treatment and second that both, test and reference treatments, were superior with respect to placebo. The secondary objectives included the evaluation of efficacy, adhesion, safety, and local tolerability of DS in comparison to both DIEP and placebo. The mean visual analog scale (VAS) score decrease for pain at rest was higher in the DS (-17.65 mm) and the DIEP group (-17.5 mm) than in the placebo (-11.3 mm). Both active formulation plasters were associated with a statistically significant pain reduction compared to placebo. No statistically significant differences were observed between DIEP and DS plasters efficacy in relieving pain. Secondary endpoint evaluations supported the primary efficacy results. No serious adverse events (SAEs) were registered, and the most commonly detected adverse events were skin reactions at the application site. The results showed that both the DS 140 mg plaster and the reference DIEP 180 mg plaster are effective in relieving pain and present a good safety profile. [\hyperlink{Diclofenac Potassium}{PMID: 37070921}, H Pabst et al., 2023]

\hypertarget{pmid_26984645}{D}iclofenac sodium (DS) is used primarily to treat fever and to alleviate pain and inflammation. We investigated the effects of DS exposure during gestation on the testes of rat pups to investigate the safety of its use during the prenatal period. Pregnant rats were separated into control, saline, low dose, medium dose and high dose groups. DS was given between weeks 15 and 21 of gestation. Total numbers of spermatogonia and Sertoli cells were counted in the testes of 7-day-old male rats using the physical disector method. By the end of the study, the total number of Sertoli cells was decreased significantly in a dose dependent manner in the medium and high dose groups compared to controls. No significant differences were found in the total number of spermatogonia in the control, saline and low dose DS groups. Medium and high dose DS administration reduced the total number of spermatogonia compared to other groups. We suggest that prenatal administration of DS can cause deleterious effects on the testis development, especially in high doses.  [\hyperlink{Diclofenac Potassium}{PMID: 26984645}, H Arslan et al., 2016] The purpose of the case-study was to evaluate the efficiency of non-steroid antiinflammatory drugs (NAD) for postoperative analgesia in children after small-scope surgical interventions. Diclofenac, 1 mg/kg per day administered as rectal suppositories or intramuscular injections after initial narcosis, was used for postoperative analgesia in children of the main group; postoperative analgesia made by analgin and promedol in the control group was compared with the former. Forty-seven children and 10 children with identical diseases like groin hernia, varicocele and dropsy of testicular membranes, were respectively in the main and control groups. Clinical examinations and registration of functional parameters were made in patients during certain time periods, i.e. before surgery (in the standing and lying postures) and after surgery (in 20 minutes, as well as in 1, 2, and 3 hours after surgical interventions). The efficiency of postoperative analgesia was evaluated by means of cardiointervalography according to Bayevsky method as well as by a state of central hemodynamics and by clinical examinations, including the visual-analogue 10-point scale and the 0-4 point verbal pain assessment scale. The postoperatively obtained data revealed a pronounced misbalance between the main and control groups, which is indicative of that the application of NAD for preventive and postoperative analgesia in children improves essentially the postoperative course and contributes to a fast rehabilitation of patients. A comparative analysis of the efficiency of postoperative analgesia by the discussed drugs showed that diclofenac possesses a sufficient analgetic activity and is free of any side-effects inherent in narcotic analgetics. [\hyperlink{Diclofenac Potassium}{PMID: 26984645}, D V Leont'ev et al., ]

\hypertarget{pmid_19694745}{T}he aim of this study was to investigate the type of common (occurring in >1\% of patients) adverse reactions caused by diclofenac when given to children for acute pain. A prospective observational study was undertaken on paediatric surgical patents aged < or =12 years at Great Ormond Street and University College London Hospitals. All adverse events were recorded, and causality assessment used to judge the likelihood of them being due to diclofenac. Prospective recruitment meant not all patients were prescribed diclofenac, allowing an analysis of utilization. Causality of all serious adverse events was reviewed by an expert panel. Children prescribed diclofenac were significantly older, and stayed in hospital for shorter periods than those who were not. Diclofenac was not avoided in asthmatic patients. Data on 380 children showed they suffer similar types of nonserious adverse reactions to adults. The incidence (95\% confidence interval) of rash was 0.8\% (0.016, 2.3); minor central nervous system disturbance 0.5\% (0.06, 1.9); rectal irritation with suppositories 0.3\% (0.009, 1.9); and diarrhoea 0.3\% (0.007, 1.5). No serious adverse event was judged to be caused by diclofenac, meaning the incidence of serious adverse reactions to diclofenac in children is <0.8\%. Children given diclofenac for acute pain appeared to suffer similar types of adverse reactions to adults; the incidence of serious adverse reaction is <0.8\%. [\hyperlink{Diclofenac Potassium}{PMID: 19694745}, Joseph F Standing et al., 2009]

\hypertarget{pmid_34826122}{A} topical formulation of diclofenac (FLECTOR diclofenac epolamine topical system (FDETS)) is approved in adults for the treatment of acute pain due to minor strains, sprains, and contusions; however, its safety and efficacy have not been investigated in a pediatric population. This study assessed the safety and efficacy of the FLECTOR (diclofenac epolamine) topical system in children. This was an open-label, single-arm, phase IV study at ten USA-based family medicine or pediatric practices in children aged 6-16 years with a clinically significant minor soft tissue injury sustained within the preceding 96 h and at least moderate spontaneous pain on the Wong-Baker FACES 104 patients were enrolled; 52 were 6-11 years old, and 52 were 12-16 years old (mean age 11.6 years). The maximum tolerability score experienced by any patient was 1 (faint redness). Fourteen adverse events (none serious) in nine patients (8.7\%) were considered possibly treatment-related. Reduction in pain during the study was somewhat greater for patients aged 6-11 versus 12-16 years (p < 0.011). The diclofenac plasma concentration tended to be higher in the younger age group compared with older patients: 1.83 versus 1.46 ng/mL at the first assessment and 2.49 versus 1.11 ng/mL at the last assessment (p = 0.002). The FLECTOR topical system safely and effectively provided pain relief for minor soft tissue injuries in the pediatric population, with minimal systemic nonsteroidal anti-inflammatory drug exposure and low potential risk of local or systemic adverse events. ClinicalTrials.gov identifier NCT02132247. [\hyperlink{Diclofenac Potassium}{PMID: 34826122}, Christopher A Jones et al., 2022]

\hypertarget{pmid_20804444}{D}iclofenac potassium liquid-filled soft gelatin capsule (DPSGC) is a rapidly absorbed formulation of diclofenac approved for the treatment of mild to moderate acute pain in adults (≥18 years of age). The objective of this study was to investigate the efficacy and safety of DPSGC 25 mg in a multicenter, randomized, double-blind, placebo-controlled study in patients experiencing pain following first metatarsal bunionectomy. Patients experiencing a requisite level of pain (≥4 based on an 11-point numeric pain rating scale [NPRS]; 0 = no pain, 10 = worst pain possible) on the day following surgery were randomized to receive DPSGC 25 mg or placebo. Patients received a second dose (remedication) on request or at 8 hours postdose followed by additional doses every 6 hours through the end of postsurgery Day 4. Rescue medication (hydrocodone/acetaminophen) was available as needed after the second dose. NCT00375934. The primary efficacy endpoint was the average NPRS score over the 48 hour inpatient multiple-dose period. DPSGC provided a significant improvement in mean 48 hour NPRS scores over placebo (3.29 vs 5.74, respectively; p < 0.0001), as well as for summed pain intensity difference (203.1 vs 86.6; p < 0.0001). Patients treated with DPSGC experienced a faster onset of meaningful pain relief compared with placebo (p = 0.0034). Rescue medication use on Day 1 and Day 2 was reduced in the DPSGC group compared with placebo (53.5\% vs 92.1\% on Day 1; 30.3\% vs 67.3\% on Day 2; p < 0.0001). DPSGC was well tolerated and no patients treated with DPSGC reported serious adverse events. As with any study, there are potential limitations including study design and patient population. These results indicate that DPSGC reduced pain in patients who underwent bunionectomy and this novel formulation of diclofenac potassium may be a practical option for treating mild to moderate acute pain. [\hyperlink{Diclofenac Potassium}{PMID: 20804444}, Stephen E Daniels et al., 2010]

\hypertarget{pmid_32777255}{D}iclofenac is a non-steroidal anti-inflammatory drug widely used by the general population and, although generally contraindicated during pregnancy, it is also used by some pregnant women. This study investigated endocrine, reproductive and behavioral effects of diclofenac in male and female offspring rats exposed in utero from gestational days 10-20. Pregnant rats were treated with diclofenac at doses of 0.2, 1 and 5 mg/kg/day via oral gavage. Anogenital distance (AGD), number of nipples, and developmental landmarks of puberty onset - vaginal opening (VO), first estrus (FE) and preputial separation (PPS) - were evaluated in the offspring. At adulthood, behavioral and reproductive parameters were assessed. Male and female rats were tested in the elevated plus maze test to assess locomotor activity and anxiety-like behaviors, while male rats were also evaluated in the partner preference test. No significant effects were observed on AGD and number of nipples in both males and females. Diclofenac treatment induced an overall delay in developmental landmarks of puberty onset in male and female offspring, which reached statistical significance for PPS at the lowest diclofenac dose. Prenatal exposure to all tested doses abolished the preference of male rats for an estrous female, suggesting an impairment of brain masculinization. No changes were observed on male or female reproductive parameters at adulthood. Overall, our results indicate that prenatal exposure to therapeutically relevant doses of diclofenac may have an impact in the pubertal development of rats and negatively affect male partner preference behavior. [\hyperlink{Diclofenac Potassium}{PMID: 32777255}, Daniele Cristine Krebs Ribeiro et al., 2020]

\hypertarget{pmid_11765589}{D}iclofenac (CAS 15307-86-5) is a non-steroidal anti-inflammatory drug largely used, mainly to relief pain of various origin. Diclofenac is present on the market as free acid, as sodium salt (CAS 15307-79-6) and as potassium salt (CAS 15307-81-0). The last salification form has shown a prompter absorption rate and a faster onset of analgesic activity than the acid form and sodium salt. This paper extensively reviews three trials carried out on healthy volunteers, where potassium salt of diclofenac present in three fast-acting formulations, namely sachets (Trial 1), tablets (Trial 2) and oral drops (Trial 3), were compared to reference tablet formulations from the market. A very fast absorption rate was encountered with the three test formulations, with the peak reached in one case 5 min and in most cases within 10-15 min after dosing. The quick absorption rate of test formulations was attributed to the special combination of the salt of diclofenac with a dynamic buffering agent, namely bicarbonate, present in the test formulations and covered by an international patent. The prompt absorption of diclofenac from the new fast-acting formulations was accompanied by the presence of only one peak, whereas the reference formulations produced in most cases two peaks, as widely described in literature. This finding suggested the hypothesis that the absorption of test formulations should occur in a shorter tract of the gut. The faster absorption of diclofenac from the three fast-acting formulations is expected to produce a faster onset of analgesic action, which highlights these new formulations as particularly indicated to relief pain of any origin. [\hyperlink{Diclofenac Potassium}{PMID: 11765589}, V Reiner et al., 2001]

\hypertarget{pmid_2235663}{W}e refer the results of an open non-comparative study aimed to evaluate the clinical efficacy of Diclofenac sodium in the treatment of Polyarticular Juvenile Chronic Arthritis. We decided to use this drug to investigate if it exerts also in younger patients the anti-inflammatory and analgesic effects known in adults. We treated 26 patients (14 girls and 12 boys) aged 2-16 years; the disease duration ranged between 3 months and 14 years. Treatment was started only if previous anti-inflammatory drugs had been considered ineffective after a prolonged use (3-12 months). None was on basic therapy. No wash-out period was used for ethical reasons. During the trial period other additional symptomatic or anti-inflammatory drugs were not used. Diclofenac sodium was given by tablets and/or suppositories at the mean daily dosage of 2.4 mg/kg (min. 0.3-max. 5, according to disease activity) for a period of 2-52 months. Diclofenac sodium was particularly effective on joint pain and morning stiffness but also on joint swelling, and functional capacity. We noticed a tendency of JRA to improve during the trial period. The drug was well tolerated; one patient stopped because of headache, another continued the treatment only per os because of intolerance of rectal administration. [\hyperlink{Diclofenac Potassium}{PMID: 2235663}, G Minisola et al., ]

\hypertarget{pmid_11299404}{T}o compare the analgesic efficacy of diclofenac sodium and paracetamol on post adenotonsillectomy postoperative pain and oral intake. Between January 1999 and July 2000, 80 children aged 3-14 years, underwent tonsillectomy and adenoidectomy for either recurrent tonsillitis or adenotonsillar hypertrophy in Prince Zeid Ben Al-Hussein Hospital and Prince Rashid Ben Al-Hussein Hospital. Forty-one children received diclofenac sodium suppositories (1-3mg/kg) postoperatively, whereas 39 children received only paracetamol syrup (10-15 mg/kg) in 4 divided doses. All children were observed for postoperative pain, oral intake, vomiting, temperature and complications. Children who received diclofenac sodium had significantly less pain, less elevation of temperature, more oral intake, and started drinking significantly sooner than the paracetamol group. Two children in the diclofenac group experienced nausea and vomiting compared to 12 children in the paracetamol group in the first day. The time to first solid intake was significantly earlier in the diclofenac sodium group (p < 0.0001). With regard to complications, one patient in each group developed secondary hemorrhage, one child developed otitis media in the 2nd group. Each group had one readmission, and 2 children from the paracetamol group had an emergency department visit for pain and dehydration. Diclofenac sodium has a significant effect on decreasing the pain associated with swallowing postoperatively and on the general condition of the patient. Improved oral intake resulted in a lower incidence of nausea and vomiting and allowed safer and earlier hospital discharge. [\hyperlink{Diclofenac Potassium}{PMID: 11299404}, M I Tawalbeh et al., 2001]

\hypertarget{pmid_6361986}{D}iclofenac sodium was investigated in the treatment of juvenile rheumatoid arthritis (JRA). The pharmacokinetics of diclofenac in children aged 2-7 was assessed. Seven patients were included in a single-dose trial to determine plasma levels and renal elimination of diclofenac sodium. Venous blood samples were taken at 0, 0.5, 1, 2, 4 and 6 hours after administration of a 25 mg enteric-coated Voltaren tablet. Urine was collected before and 0-6 and 6-12 hours after tablet ingestion. Maximum concentrations ranged from 0.79 to 4.25 micrograms/ml, and were found between 0.5 and 2 hours. Renal elimination of total diclofenac ranged from 5.4 to 10.2\% of the oral dose in 6 of the 7 patients. The youngest patient (2 years) had a lower elimination rate (2.25\%) during the 12 hours observed. The values for children over 2 years corresponded to the range measured in adults. The pharmacokinetic study was followed by a placebo-controlled study with diclofenac sodium and acetylsalicylic acid (ASA) for 2 weeks in 45 hospitalized patients aged 3-15 years. The patients were randomly assigned to either: DS 2-3 mg/kg/day, microcrystallized ASA 50-100 mg/kg/day, or placebo matching diclofenac. Global evaluation of therapeutic efficacy showed improvement in 73\% of the patients in the diclofenac group, in 50\% of the ASA group and in 27\% of the placebo group. A statistically significant difference between these groups was found (p less than 0.05). The sum of grades of joint tenderness decreased during the 2 weeks in 67\% of patients in the diclofenac group, in 56\% of the ASA group and in 36\% of the placebo group.(ABSTRACT TRUNCATED AT 250 WORDS) [\hyperlink{Diclofenac Potassium}{PMID: 6361986}, J Haapasaari et al., 1983] We investigated the analgesic effect of intra-operative intravenous diclofenac in a randomized, double blind placebo-controlled paralled group study after adenoidectomy in 150 children aged 1-7 years. A standard anaesthetic method was used and all children received oral diazepam as premedication. Anaesthesia was induced with thiopentone and maintained with halothane and nitrous oxide in oxygen with controlled ventilation. Children in the diclofenac group received 1 mg/kg i.v. after induction of anaesthesia followed by an infusion of diclofenac 1 mg/kg over 2 hours. Children in the placebo group received 0.9\% saline. At the end of procedure the children were transferred to the recovery room for continuous monitoring of vital signs and assessment of pain. Standard deviation, means, ranges and students' t-test statistics were used for data analysis. Worst pain observed in the recovery room was lower in the diclofenac group both at rest and during swallowing. It was therefore concluded that intravenous diclofenac given intra-operatively has analgesic effect in the immediate post-operative period and it is recommended for small children during adenoidectomy. [\hyperlink{Diclofenac Potassium}{PMID: 6361986}, P U N Nze et al., 2006]

\hypertarget{pmid_17897274}{T}onsillectomy is a common pediatric surgical procedure resulting in significant postoperative pain. There is ongoing controversy as to the most satisfactory analgesic regimen. Nonsteroidal antiinflammatory drugs (NSAIDs) are an alternative to opioids in this setting. NSAID use in tonsillectomy has been shown to be opioid sparing in the recovery period and to have similar analgesic effects to opioids in pediatric patients. Because of their nonspecific action on the enzyme cyclo-oxygenase there is potential for increased bleeding which has led many practitioners to avoid NSAIDs completely in this patient population potentially resulting in suboptimal pain control. Our aim in this study was to assess the effect of preoperatively administered diclofenac on the blood clot strength in children undergoing (adeno-) tonsillectomy. Twenty patients undergoing (adeno-) tonsillectomy were recruited into this prospective observational study. All patients received 2 mg.kg(-1) of diclofenac rectally immediately preoperatively. Blood was taken for thromboelastograph analysis pre-diclofenac and 1 and 4 h post-diclofenac administration. There was a statistically significant increase in maximal clot strength (MA) at 1 and 4 h after diclofenac. Similarly there was a statistically significant reduction in time to initial fibrin formation (R time) post-diclofenac. There was no primary or secondary hemorrhage. Diclofenac when given preoperatively does not adversely affect clot strength in the immediate postoperative period when the risk of primary hemorrhage is greatest. [\hyperlink{Diclofenac Potassium}{PMID: 17897274}, Mairead Heaney et al., 2007]

\hypertarget{pmid_25260983}{T}o investigate the possible effect of intraoperative analgesia, namely diclofenac sodium compared to acetaminophen on post-recovery pain perception in children undergoing painful dental procedures under general anaesthesia. A double-blind randomised clinical trial. A sample of 180 consecutive cases of children undergoing full dental rehabilitation under general anaesthesia in a private hospital in Saudi Arabia during 2013 was divided into three groups (60 children each) according to the analgesic used prior to extubation. Group A, children had diclofenac sodium suppository. Group B, children received acetaminophen suppository and Group C, the control group. Using an authenticated Arabic version of the Wong and Baker faces Pain assessment Scale, patients were asked to choose the face that suits best the pain he/she is suffering. Data were collected and recorded for statistical analysis. Student's t test was used for comparison of sample means. A preliminary F test to compare sample variances was carried out to determine the appropriate t test variant to be used. A "p" value less than 0.05 was considered significant. More than 93\% of children had post-operative pain in varying degrees. High statistical significance was observed between children in groups A and B compared to control group C with the later scoring high pain perception. Diclofenac showed higher potency in multiple painful procedures, while the statistical difference was not significant in children with three or less painful dental procedures. Diclophenac sodium is more potent than acetaminophen, especially for multiple pain-provoking or traumatic procedures. A timely use of NSAID analgesia just before extubation helps provide adequate coverage during recovery. Peri-operative analgesia is to be recommended as an essential treatment adjunct for child dental rehabilitation under general anaesthesia. [\hyperlink{Diclofenac Potassium}{PMID: 25260983}, H Y El Batawi et al., 2015]

\hypertarget{pmid_17868656}{D}iclofenac sodium (DS) is commonly used as a non-steroidal anti-inflammatory drug. Although several adverse effects are clearly established, it is still unknown whether prenatal exposure to DS has an effect on the development of the cerebellum. In this study, we investigated the total number of Purkinje cells of the cerebellum in a control group and in a DS-treated group of male rats using a stereological method. The DS in a dose of 1 mg/kg daily was intraperitoneally injected to the drug-treated group of pregnant rats beginning from the 5th day after mating for a period of 15 days during pregnancy. Physiological serum at 1 ml dose was intraperitoneally injected to the control group of pregnant rats at the same period. After delivery, male offspring were obtained and each main group was divided into two subgroups that were 4-week-old (4W-old) and 20-week-old (20W-old). Our results showed that the total number of Purkinje cells in offspring of drug-treated rats was significantly lower than in the offspring of control animals. These results suggest that the Purkinje cells of a developing cerebellum may be affected by administration of DS during the prenatal period. [\hyperlink{Diclofenac Potassium}{PMID: 17868656}, Murat Cetin Ragbetli et al., 2007]

\hypertarget{pmid_26889398}{F}ever is the most common complaint in pediatric medicine and its treatment is recommended in some situations. Paracetamol is the most common antipyretic drug, which has serious side effects such as toxicity along with its positive effects. Diclofenac is one of the strongest non-steroidal anti-inflammatory (NSAID) drugs, which has received little attention as an antipyretic drug. This study was designed to compare the antipyretic effectiveness of the rectal form of Paracetamol and Diclofenac. This double-blind controlled clinical trial was conducted on 80 children aged six months to six years old. One group was treated with rectal Paracetamol suppositories at 15 mg/kg dose and the other group received Diclofenac at 1 mg/kg by rectal administration (n = 40). Rectal temperature was measured before and one hour after the intervention. Temperature changes in the two groups were compared. The average rectal temperature in the Paracetamol group was 39.6 ± 1.13°C, and 39.82 ± 1.07°C in the Diclofenac group (P = 0.37). The average rectal temperature, one hour after the intervention, in the Paracetamol and the Diclofenac group was 38.39 ± 0.89°C and 38.95 ± 1.09°C, respectively (P = 0.02). Average temperature changes were 0.65 ± 0.17°C in the Paracetamol group and 1.73 ± 0.69°C in the Diclofenac group (P < 0.001). In the first one hour, Diclofenac suppository is able to control the fever more efficient than Paracetamol suppositories. [\hyperlink{Diclofenac Potassium}{PMID: 26889398}, Mohammad Reza Sharif et al., 2016]

\hypertarget{pmid_19038583}{W}e assessed the efficacy of diclofenac potassium, a nonsteroidal anti-inflammatory drug, in alleviating menstrual pain and restoring exercise performance to that measured in the late-follicular phase of the menstrual cycle. Twelve healthy young women with a history of primary dysmenorrhea completed, in a random order, laboratory exercise-testing sessions when they were in the late-follicular (no menstruation, no pain) phase of the menstrual cycle and when they were experiencing dysmenorrhea and receiving, in a double-blinded fashion, either 100 mg of diclofenac potassium or placebo. We assessed the women's leg strength (1-repetition maximum test), aerobic capacity (treadmill walking test), and ability to perform a functional test (task-specific test). Compared with placebo, diclofenac potassium significantly decreased dysmenorrhea on the day of administration (Visual Analog Scale, P < .001 at all times). When receiving placebo for menstrual pain, the women's performance in the tests was decreased significantly, compared with when they were receiving diclofenac potassium for menstrual pain (P < .05) and compared with when they were in the late-follicular phase of the menstrual cycle (P < .05 for treadmill test, P < .01 for task-specific test and 1-repetition maximum test). Administration of diclofenac potassium for menstrual pain restored exercise performance to a level not different from that achieved in the late-follicular phase of the cycle. In women with primary dysmenorrhea, menstrual pain, if untreated, decreases laboratory-assessed exercise performance. A recommended daily dose of a readily available nonsteroidal anti-inflammatory drug, diclofenac potassium, is effective in relieving menstrual pain and restoring physical performance to levels achieved when the women were in the late-follicular (no menstruation, no pain) phase of the menstrual cycle. [\hyperlink{Diclofenac Potassium}{PMID: 19038583}, Ingrid Chantler et al., 2009]

\hypertarget{pmid_28346003}{O}BJECTIVE To determine the plasma pharmacokinetics and safety of 1\% diclofenac sodium cream applied topically to neonatal foals every 12 hours for 7 days. ANIMALS Twelve 2- to 14-day old healthy Arabian and Arabian-pony cross neonatal foals. PROCEDURES A 1.27-cm strip of cream containing 7.3 mg of diclofenac sodium (n = 6 foals) or an equivalent amount of placebo cream (6 foals) was applied topically to a 5-cm square of shaved skin over the anterolateral aspect of the left tarsometatarsal region every 12 hours for 7 days. Physical examination, CBC, serum biochemistry, urinalysis, gastric endoscopy, and ultrasonographic examination of the kidneys and right dorsal colon were performed before and after cream application. Venous blood samples were collected at predefined intervals following application of the diclofenac cream, and plasma diclofenac concentrations were determined by liquid chromatography-mass spectrometry. RESULTS No foal developed any adverse effects attributed to diclofenac application, and no significant differences in values of evaluated variables were identified between treatment groups. Plasma diclofenac concentrations peaked rapidly following application of the diclofenac cream, reaching a maximum of < 1 ng/mL within 2 hours, and declined rapidly after application ceased. CONCLUSIONS AND CLINICAL RELEVANCE Topical application of the 1\% diclofenac sodium cream to foals as described appeared safe, and low plasma concentrations of diclofenac suggested minimal systemic absorption. Practitioners may consider use of this medication to treat focal areas of pain and inflammation in neonatal foals. [\hyperlink{Diclofenac Potassium}{PMID: 28346003}, Susan E Barnett et al., 2017]

\section*{Doxepin Hydrochloride}
\subsection*{Result}
\subsubsection*{Answer}

Unknown (ages 2-17)
Unknown (under 12, especially for topical use)

\subsubsection*{{Explanation}}
\hypertarget{Doxepin Hydrochloride}
A review of the available abstracts reveals several that mention Doxepin Hydrochloride in pediatric populations:

1. One abstract describes a case of a 5-year-old girl who experienced significant central nervous system depression after excessive topical application of doxepin hydrochloride 5\% cream. The authors note that "the safety and efficacy of doxepin cream has not been established in children younger than 12 years, it should be used with caution in this population" [\hyperlink{pmid_10917379}{PMID: 10917379}, M Zell-Kanter et al., 2000]. This is a case report, not a targeted safety study, and it highlights a lack of established safety data for children under 12.

2. Another abstract presents a case of chronic doxepin toxicity in a 10-year-old boy, with symptoms including confusion, ataxia, and seizures, attributed to supratherapeutic dosing, pharmacogenomic variability, and drug-drug interactions. The authors conclude that "caution must be exercised when prescribing doxepin" and that "chronic toxicity should be contemplated in patients taking doxepin without acute overdose who present with persistent neurologic abnormalities including seizure" [\hyperlink{pmid_37682427}{PMID: 37682427}, James D Whitledge et al., 2023]. This is also a case report, not a targeted safety study.

3. A retrospective chart review evaluated the efficacy and tolerability of doxepin in 29 children and adolescents (ages 2-17) with insomnia refractory to behavioral intervention and melatonin. The study found that low-dose doxepin was "both effective and well tolerated in pediatric patients with insomnia," with only 2 patients (6.9\%) experiencing adverse effects (aggression and enuresis), and 4 (13.8\%) discontinuing due to lack of efficacy or side effects [\hyperlink{pmid_32029069}{PMID: 32029069}, Yash D Shah et al., 2020]. However, this was a retrospective review, not a prospective, controlled safety study, and the sample size was small.

4. Other abstracts discuss doxepin use in adults or do not address pediatric safety.

In summary, there is one small retrospective study suggesting tolerability in children aged 2-17 with insomnia, but no large, prospective, controlled studies specifically designed to assess the safety of doxepin hydrochloride in children. The other pediatric data are limited to case reports of toxicity. For children under 12, especially for topical use, the safety and efficacy have not been established. Therefore, based on the abstracts available, the safety of doxepin hydrochloride in children is unknown.

\subsection*{Abstracts}
\hypertarget{pmid_10917379}{T}o describe a case of a child with altered mental status following the topical administration of doxepin. A five-year-old Hispanic girl was brought to the emergency department because she was difficult to arouse at school. She had recently developed a generalized eczematous rash for which she was prescribed doxepin hydrochloride 5\% cream. An entire tube (30 g) of doxepin cream was applied in the 24 hours prior to presentation. The patient was responsive only to noxious stimuli, with no focal neurologic abnormalities. She was decontaminated and observed in a pediatric intensive care unit. By 18 hours after presentation, she had fully recovered and was discharged. Topical doxepin, available as a 5\% cream, is indicated for the treatment of pruritus secondary to eczematous dermatoses in adults. Diminished skin integrity and the application of a massive quantity of doxepin 5\% cream to a large body surface area contributed to the toxicity in this child. Since the safety and efficacy of doxepin cream has not been established in children younger than 12 years, it should be used with caution in this population. [\hyperlink{Doxepin Hydrochloride}{PMID: 10917379}, M Zell-Kanter et al., 2000]

\hypertarget{pmid_17614751}{D}oxapram hydrochloride, a respiratory stimulant, has several undesirable side effects during high-dose administration, including second-degree atrioventricular (AV) block and QT prolongation. In Japan, this drug is contraindicated for newborn infants. Recent studies, however, have demonstrated the efficacy and safety of doxapram therapy for apnea of prematurity (AOP) using lower doses than those previously tested. As a result, approximately 60\% of Japanese neonatologists continue to use this drug. This study used surface ECG recordings to assess the cardiac safety of low-dose doxapram hydrochloride (0.2 mg/kg/h) in fifteen premature very-low-birth-weight infants with idiopathic AOP. Cardiac intervals and number of apnea episodes were compared before and after drug administration. Low-dose doxapram hydrochloride resulted in approximately 90\% reduction in the frequency of apnea without side effects. None of the infants developed QT or PR prolongation, arrhythmia, or other conduction disorders. In addition, there was no change in the slope of QT/RR before versus after administration of doxapram hydrochloride. We conclude that low-dose administration of doxapram hydrochloride did not have any undesirable effects on myocardial depolarization and repolarization. [\hyperlink{Doxepin Hydrochloride}{PMID: 17614751}, Masafumi Miyata et al., 2007]

\hypertarget{pmid_3782654}{D}oxepin hydrochloride, a tricyclic antidepressant, was evaluated in a double-blind, placebo-controlled crossover trial for the treatment of chronic idiopathic urticaria in 16 adults. Efficacy was evaluated by symptom scores, concomitant antihistamine use, and suppression of histamine- and codeine-induced wheal response. Doxepin-treated subjects experienced fewer lesions (p less than 0.001), less waking hours with lesions (p less than 0.01), lesser degree of itch and/or discomfort (p less than 0.001), and less swelling or angioedema (p less than 0.001) as compared to placebo-treated subjects. Doxepin-treated subjects required less daily concomitant antihistamine use (mean 0.13 tablets versus 1.48 tablets, p less than 0.05). Doxepin also significantly suppressed histamine- and codeine-induced cutaneous wheal response as compared to placebo. Lethargy was commonly observed but diminished with continued use. Dry mouth and constipation were also commonly observed. We conclude that doxepin is an effective agent for the treatment of chronic idiopathic urticaria. [\hyperlink{Doxepin Hydrochloride}{PMID: 3782654}, A B Goldsobel et al., 1986]

\hypertarget{pmid_2522789}{T}he neuromuscular and cardiovascular effects of doxacurium chloride (BW A938U) were evaluated in 27 children (2-12 yr) anaesthetized with 1\% halothane and nitrous oxide in oxygen. In nine children the incremental technique was used to establish a cumulative dose-response curve by train-of-four stimulation. The remaining children received either 30 or 50 micrograms kg-1 of the drug as a single bolus. The median ED50 and ED95 of doxacurium in children were 19 and 32 micrograms kg-1, respectively. No clinically significant change in heart rate or arterial pressure occurred. Following doxacurium 30 micrograms kg-1 and 50 micrograms kg-1, recovery to 25\% of control occurred in 25 (SEM 6) and 44 (3) min, respectively. The recovery index (25-75\% of control) was 27 (2) min. The duration of action of doxacurium is similar to that of tubocurarine and dimethyl-tubocurarine in children. Compared with adults, children seem to require more doxacurium (microgram kg-1) to achieve a comparable degree of neuromuscular depression, and they recover more rapidly. [\hyperlink{Doxepin Hydrochloride}{PMID: 2522789}, N G Goudsouzian et al., 1989]

\hypertarget{pmid_11847958}{I}nformation regarding the treatment of anthrax infection is scarce in adults and is even more limited in children. Children, however, may be at a greater risk for developing an infection and systemic disease if exposed to anthrax than adults. The Centers for Disease Control and Prevention (CDC) recommends the use of doxycycline or ciprofloxacin for prophylaxis and treatment in children. Doxycycline currently is not indicated for use in children < 8 years old, due to staining of teeth and inhibition of bone growth associated with tetracyclines. Doxycycline, however, may have less adverse effect on teeth than its precursors. Ciprofloxacin has a pediatric indication only when a child is potentially exposed to inhaled anthrax. Ciprofloxacin is contraindicated in pediatric patients because fluoroquinolones were shown to cause cartilage toxicity in immature animals. Although children of various ages have received ciprofloxacin, there are few reports of cartilage toxicity. Because anthrax is a potentially fatal infection, the benefits to using these antibiotics greatly outweigh the risks. Therefore, the use of these antibiotics in children can be recommended, despite the lack of adequate efficacy and safety studies in pediatric patients with anthrax. [\hyperlink{Doxepin Hydrochloride}{PMID: 11847958}, Sandra Benavides et al., 2002]

\hypertarget{pmid_37682427}{C}hronic tricyclic antidepressant toxicity is rarely described in children. Symptoms include confusion, ataxia, and seizures. Toxicity may result from dosing error, CYP2C19 and CYP2D6 genetic variability, and drug-drug interactions. Chronic doxepin toxicity has not been previously reported in children. Doxepin is prescribed for insomnia and depression, with a maximum off-label dose of 3 mg/kg in children. We present a case of chronic doxepin toxicity mimicking epilepsy in a child attributable to three potential factors: supratherapeutic dosing, pharmacogenomic variability, and drug-drug interactions. A 10-year-old boy with insomnia, diagnosed with epilepsy 6 months prior, presented to an emergency department with confusion, ataxia, and increasing seizure frequency. He was prescribed doxepin for insomnia and four antiepileptics for seizures. After admission, he had two seizures and remained confused. EKGs showed QRS prolongation, suggesting doxepin toxicity. Doxepin-nordoxepin combined serum concentration was 1419 ng/mL (therapeutic 100-300 ng/mL), confirming doxepin toxicity. Outpatient records showed onset of confusion and seizures as doxepin dose was gradually uptitrated to 300 mg nightly (4.41 mg/kg). Symptoms worsened following addition of clobazam (CYP2D6 inhibitor) and topiramate (CYP2C19 inhibitor). Following doxepin discontinuation, all symptoms resolved. CYP2D6 testing showed intermediate metabolizer phenotype (CYP2D6*1/*4; activity score = 1.0; copy number = 2.0). No seizures have occurred in more than one year since doxepin discontinuation. Caution must be exercised when prescribing doxepin. Pharmacogenomics, dose, drug-drug interactions, and age should be considered. Chronic toxicity should be contemplated in patients taking doxepin without acute overdose who present with persistent neurologic abnormalities including seizure. [\hyperlink{Doxepin Hydrochloride}{PMID: 37682427}, James D Whitledge et al., 2023]

\hypertarget{pmid_17542008}{T}here is growing evidence to support the use of early central cholinergic enhancement to improve cognitive functioning in individuals with Down syndrome (DS). This report summarizes preliminary safety and cognitive efficacy data for seven children (8-13 years) with DS who participated in a 22-week, open-label trial of donepezil hydrochloride. Donepezil was dosed once daily at 2.5 mg and, based on tolerability, increased to 5 mg/day. Safety assessments were conducted at Week 1 (baseline), Week 8 (2.5 mg donepezil), Week 16 (5 mg) and Week 22 (after the donepezil had been discontinued). Measures of cognitive function were administered at each visit, encompassing the following domains: memory; attention; mood; and adaptive functioning. Donepezil was well tolerated at the 2.5 and 5 mg doses. The side effects were mild, transient, and consistent with the adverse events noted with cholinesterase inhibitors. Some children showed improvement on measures of memory (NEPSY Memory for Names and Narrative Memory) and sustained attention to tasks (Conners' Parent Rating Scales), although increased irritability and/or assertiveness were noted in some patients. Overall, this clinical report series adds to our initial findings of language gains in children with DS treated with donepezil. It also supports the need for larger, double-blind studies of the safety and efficacy of donepezil and other cholinesterase inhibitors for children with DS. [\hyperlink{Doxepin Hydrochloride}{PMID: 17542008}, Gail A Spiridigliozzi et al., 2007]

\hypertarget{pmid_8703459}{T}o evaluate neuromuscular potency of doxacurium during balanced anesthesia in pediatric patients. Prospective, consecutive sample trial. Operating room at a university hospital. 15 infants (1 to 11 months), 20 children (3 to 10 years), and 20 adolescents (13 to 19 years). Anesthesia was induced and maintained with thiopental, alfentanil, and nitrous oxide in oxygen. No volatile drugs were used at any time during the study. The neuromuscular function was recorded as adductor pollicis electromyography evoked by a train-of-four stimulation at 20-second intervals. A cumulative log-dose probit-response curve of doxacurium was established for every patient. ED50 and ED95 doses of doxacurium (14 micrograms/kg and 25 micrograms/kg in infants, 26 micrograms/kg and 53 micrograms/kg in children, and 20 micrograms/kg and 41 micrograms/kg in adolescents, respectively) were smallest in infants and greatest in children (p < 0.05 between each pair of groups by analysis of variance and Scheffe's F-test). Potency of doxacurium was greatest in infants and least in children. We suggest that doxacurium can be administered safely in infants, and with dosages close to those reported in adults. Children's dose requirement was almost 50\% greater than that of infants. [\hyperlink{Doxepin Hydrochloride}{PMID: 8703459}, T R Taivainen et al., 1996]

\hypertarget{pmid_19740527}{E}noxaparin, a low molecular weight heparin (LMWH), is frequently used for the prevention and treatment of thromboembolic complications in infants and children (Sutor et al., 2004 [1]). Injection pain and the fear and anxiety associated with needle phobia in the pediatric population are well documented. Best practice pediatric pain management standards of care recommend mitigating the child's pain experience whenever possible. The use of topical anesthetics such as liposomal-lidocaine 4\% results in a rapid onset of anesthesia, minimal blanching, without vasoconstriction (Koh et al., 2004 [2]) or risk of methemoglobinemia. Topical lidocaine has been used to reduce the injection pain of enoxaparin, but there is no data available examining whether it will interfere with the absorption of LMWH. To determine if the topical lidocaine, Maxilene, interferes with enoxaparin absorption as measured by peak anti-Xa levels. Infants and children clinically prescribed enoxaparin were eligible for this study. Children in group 1 were pre-treated with Maxilene prior to enoxaparin injection on day 1 with no Maxilene pre-treatment on day 2. For group 2, the order was reversed. Peak anti-Xa levels were measured following each enoxaparin dose and were compared between the groups. 26 children of ages 14d-16 y (median 6.7 months) were enrolled. Anti-Xa levels following topical lidocaine administration were 0.070 U/mL (95\%CI 0.025; 0.114) lower than without prior topical lidocaine administration. Anti-Xa levels on the second day were on average 0.013 U/mL (95\%CI -0.066; 0.040) higher compared to day one regardless of the order of topical lidocaine administration. There were no reported incidences of local reactions such as redness, hives or blanching. Topical lidocaine (Maxilene) administration before enoxaparin injection results in a small, clinically non-significant, reduction in anti-Xa levels. [\hyperlink{Doxepin Hydrochloride}{PMID: 19740527}, S M Duncan et al., 2010]

\hypertarget{pmid_32029069}{P}ediatric insomnia is a widespread problem and especially difficult to manage in children with neurodevelopmental disorders. There are currently no US Food and Drug Administration-approved medications to use once first-line therapy fails. The objective of this study was to evaluate the efficacy and tolerability of doxepin in pediatric patients. This is a retrospective single-center chart review of children and adolescents (2-17 years of age) whose sleep failed to improve with behavioral intervention and melatonin who were then trialed on doxepin. Treatment was initiated at a median starting dose of 2 mg and slowly escalated to a median maintenance dose of 10 mg. Improvement in sleep was recorded using a 4-point Likert scale reported by parents on follow-up visits. A total of 29 patients were included in the analysis. Mean follow-up duration was 6.5 ± 3.5 months. Of 29 patients, 4 (13.8\%) patients discontinued doxepin because of lack of efficacy or side effects. Eight (27.6\%) patients showed significant improvement of their insomnia, 8 (27.6\%) showed moderate improvement, 10 (34.5\%) showed mild improvement, and 3 (10.3\%) showed minimal to no improvement on treatment with doxepin (P < .05) Only 2 patients (6.9\%) experienced adverse effects in the form of behavioral side effects (aggression) and enuresis. Results of our studies suggest that low-dose doxepin is both effective and well tolerated in pediatric patients with insomnia. [\hyperlink{Doxepin Hydrochloride}{PMID: 32029069}, Yash D Shah et al., 2020]

\hypertarget{pmid_17941284}{T}he safety of fexofenadine has been examined extensively in adults and school-age children. However, the safety of fexofenadine in children younger than 6 years has not been reported to date. To compare the safety and tolerability of twice-daily fexofenadine hydrochloride, 30 mg, and placebo in preschool children aged 2 to 5 years with allergic rhinitis. This was a multicenter, double-blind, randomized, placebo-controlled, parallel-group study, conducted between February 29, 2000, and June 14, 2001. Participants were randomized to either fexofenadine hydrochloride, 30 mg, or placebo twice daily for a 2-week period. To facilitate dosing, capsule content was mixed with applesauce (approximately 10 mL). Safety assessments depended on date of entry into the study because of an amendment to the protocol. Before the amendment, assessments included physical examination, vital signs reporting (oral temperature, heart rate, and respiratory rate), and adverse event (AE) reporting. After the amendment, safety assessments included laboratory testing (blood chemistry and hematology profiles), physical examination, 12-lead electrocardiography, and vital signs (oral temperature, blood pressure, heart rate, and respiratory rate) and AE reporting. Treatment-emergent AEs were observed in 116 of 231 participants receiving placebo and 111 of 222 receiving fexofenadine. These AEs were possibly related to study medication in 19 (8.2\%) and 21 (9.5\%) of the participants receiving placebo and fexofenadine, respectively, and most frequently involved the digestive system. No clinically relevant differences in laboratory measures, vital signs, and physical examinations were observed. The findings show that fexofenadine hydrochloride, 30 mg, is well tolerated and has a good safety profile in children aged 2 to 5 years with allergic rhinitis. [\hyperlink{Doxepin Hydrochloride}{PMID: 17941284}, Henry Milgrom et al., 2007]

\hypertarget{pmid_26499007}{T}o evaluate the efficacy and safety of Drotaverine hydrochroride in children with recurrent abdominal pain. Double blind, randomized placebo-controlled trial. Pediatric Gastroenterology clinic of a teaching hospital. 132 children (age 4-12 y) with recurrent abdominal pain (Apley Criteria) randomized to receivedrotaverine (n=66) or placebo (n=66) orally. Children between 4-6 years of age received 10 mL syrup orally (20 mg drotaverine hydrochloride or placebo) thrice daily for 4 weeks while children >6 years of age received one tablet orally (40 mg drotaverine hydrochloride or placebo) thrice daily for 4 weeks. Primary: Number of episodes of pain during 4 weeks of use of drug/placebo and number of pain-free days. Secondary: Number of school days missed during the study period, parental satisfaction (on a Likert scale), and occurrence of solicited adverse effects. Reduction in number of episodes of abdominal pain [mean (SD) number of episodes 10.3 (14) vs 21.6 (32.4); P=0.01] and lesser school absence [mean (SD) number of school days missed 0.25 (0.85) vs 0.71 (1.59); P=0.05] was noticed in children receiving drotaverine in comparison to those who received placebo. The number of pain-free days, were comparable in two groups [17.4 (8.2) vs 15.6 (8.7); P=0.23]. Significant improvement in parental satisfaction score was noticed on Likert scale by estimation of mood, activity, alertness, comfort and fluid intake. Frequency of adverse events during follow-up period was comparable between children receiving drotaverine or placebo (46.9\% vs 46.7\%; P=0.98). Drotaverine hydrochloride is an effective and safe pharmaceutical agent in the management of recurrent abdominal pain in children. [\hyperlink{Doxepin Hydrochloride}{PMID: 26499007}, Manish Narang et al., 2015]

\hypertarget{pmid_17685877}{L}ow-dose doxepin hydrochloride (1, 3 and 6 mg) is a tricyclic antidepressant currently being investigated for the treatment of primary insomnia in adult and geriatric patients. Although it has been used at much higher doses to treat depression effectively for a number of decades, it offers a unique potency and selectivity for antagonizing the H1 (histamine) receptor at low doses. This mechanism of action may prove to be advantageous compared with other medications currently approved for the treatment of insomnia. This article reviews previous clinical studies using doxepin for insomnia and the recent clinical trial data, and briefly discusses other potential roles of this compound in clinical practice. [\hyperlink{Doxepin Hydrochloride}{PMID: 17685877}, Haramandeep Singh et al., 2007]

\hypertarget{pmid_37936265}{H}exaxim® is fully liquid, hexavalent, combination vaccine that provides immunization against diphtheria, tetanus, pertussis (whooping cough), polio, hepatitis B, and invasive diseases caused by  Safety and immunogenicity data were reviewed from >25 clinical trials involving approximately 7200 infants/toddlers, identified using PubMed searches to April 2023. These trials have evaluated a diverse range of primary series and booster schedules, including antibody persistence, co-administration of Hexaxim with other routine pediatric vaccines, and specific populations (born to Tdap-vaccinated women, preterm, and immunocompromised infants). Lastly, post-marketing surveillance and real-world effectiveness data were assessed. An extensive program of clinical development prior to licensure demonstrated favorable vaccine safety and good immunogenicity of each antigen, and Hexaxim was first approved for use in 2012. In the 10 years since licensure, Hexaxim has been adopted widely, with more than 180 million doses distributed worldwide. The widespread use of this hexavalent vaccine is a crucial tool in the ongoing and future control of six pediatric infectious diseases globally. [\hyperlink{Doxepin Hydrochloride}{PMID: 37936265}, Florence Boisnard et al., ]

\hypertarget{pmid_28741653}{C}hloral hydrate is commonly used to sedate children for painless procedures. Children may recover more quickly after sedation with dexmedetomidine, which has a shorter half-life. We randomly allocated 196 children to chloral hydrate syrup 50 mg.kg [\hyperlink{Doxepin Hydrochloride}{PMID: 28741653}, V M Yuen et al., 2017] Duloxetine hydrochloride is a dual reuptake inhibitor of both serotonin and norepinephrine. In the present open-label study, the safety of duloxetine at a fixed-dose of 60 mg twice daily (BID) for up to 52 weeks was evaluated and compared to routine care in the therapy of patients diagnosed with diabetic peripheral neuropathic pain (DPNP). Patients who completed a 13-week, double-blind, duloxetine and placebo acute therapy period were rerandomly assigned in a 2:1 ratio to therapy with duloxetine 60 mg BID (N=161) or routine care (N=76) for an additional 52 weeks. Routine care consisted primarily of gabapentin, amitriptyline, and venlafaxine. The study included male or female outpatients 18 years of age or older with a diagnosis of DPNP caused by type 1 or type 2 diabetes. A higher percentage of routine care-treated patients experienced 1 or more serious adverse events. No statistically significant therapy-group difference was observed in the overall incidence of treatment-emergent adverse events (TEAEs). The TEAEs reported by 10\% or more of duloxetine 60 mg BID-treated patients were nausea, and by the routine care-treated patients were peripheral edema, pain in the extremity, somnolence, and dizziness. Duloxetine did not appear to adversely affect glycemic control, lipid profiles, nerve function, or the course of DPNP. There were no statistically significant therapy-group differences observed in the 36-item Short-Form Health Survey subscales or in the EuroQol 5-Dimension Questionnaire. In this study, duloxetine was safe and well tolerated compared to routine care in the long-term management of patients with DPNP. [\hyperlink{Doxepin Hydrochloride}{PMID: 28741653}, Joel Raskin et al., 2006]

\hypertarget{pmid_23211689}{T}here is growing concern regarding the long-term negative side effects of chemotherapy in childhood cancer survivors. Doxorubicin (DOX) is commonly used in the treatment of childhood cancers and has been shown to be both cardiotoxic and osteotoxic. It is unclear whether exercise can attenuate the negative skeletal effects of this chemotherapy. Rat pups were treated with saline or DOX. Animals remained sedentary or voluntarily exercised. After 10 weeks, femoral bone mineral content and bone mineral density were measured using dual-energy x-ray absorptiometry. Cortical and cancellous bone architecture was then evaluated by microcomputed tomography. DOX had a profound negative effect on all measures of bone mass and cortical and cancellous bone architecture. Treatment with DOX resulted in shorter femora and lower femoral bone mineral content and bone mineral density, lower cross-sectional volume, cortical volume, marrow volume, cortical thickness, and principal (IMAX, IMIN) and polar (IPOLAR) moments of inertia in the femur diaphysis, and lower cancellous bone volume/tissue volume, trabecular number, and trabecular thickness in the distal femur metaphysis. Exercise failed to protect bones from the damaging effects of DOX. Other modalities may be necessary to mitigate the deleterious skeletal effects that occur in juveniles undergoing treatment with anthracyclines. [\hyperlink{Doxepin Hydrochloride}{PMID: 23211689}, Reid Hayward et al., 2013]

\hypertarget{pmid_36174614}{S}urvivors of childhood cancer are at risk of anthracycline-induced cardiotoxicity, which might be prevented by dexrazoxane. However, concerns exist about the safety of dexrazoxane, and little guidance is available on its use in children. To facilitate global consensus, a working group within the International Late Effects of Childhood Cancer Guideline Harmonization Group reviewed the existing literature and used evidence-based methodology to develop a guideline for dexrazoxane administration in children with cancer who are expected to receive anthracyclines. Recommendations were made in consideration of evidence supporting the balance of potential benefits and harms, and clinical judgement by the expert panel. Given the dose-dependent risk of anthracycline-induced cardiotoxicity, we concluded that the benefits of dexrazoxane probably outweigh the risk of subsequent neoplasms when the cumulative doxorubicin or equivalent dose is at least 250 mg/m [\hyperlink{Doxepin Hydrochloride}{PMID: 36174614}, Esmée C de Baat et al., 2022] Doxylamine is a first-generation antihistamine similar in structure to diphenhydramine. Unlike diphenhydramine, however, there is a paucity of data regarding the risk of toxicity following unintentional exposures in pediatric patients. We performed an observational case series with data collected retrospectively from a poison system database for all single-substance pediatric (5 years-old and younger) doxylamine ingestions for the period of 1997-2012. Data collected included age, gender, weight, reason for exposure, exact or estimated maximum dose, clinical effects and medical interventions. A total of 140 cases were identified; 74 (53\%) involved males. Ages ranged 6 months to 5 years. In 30 cases (21\%), the exact amount ingested was documented and ranged from 6.25-50 mg with a maximum weight-based dose of 6.2 mg/kg. In 76 cases, the estimated maximum dose ranged from 12.5 to 375 mg with a maximum weight-based dose of 37 mg/kg. All symptoms were mild and self-limiting. The only documented intervention was the administration of activated charcoal in 13 cases. Unintentional isolated pediatric doxylamine ingestions did not result in significant toxicity in our 140 cases. Reported doses of up to 6.2 mg/kg resulted in only transient drowsiness and tachycardia. [\hyperlink{Doxepin Hydrochloride}{PMID: 36174614}, F Lee Cantrell et al., 2015]

\hypertarget{pmid_20819318}{A}llergic rhinitis (AR) and chronic idiopathic urticaria (CIU) are common causes of substantial illness and disability in preschool children. Antihistamines are commonly used to treat preschool children with these conditions, but their use is based mostly on extrapolated efficacy from adult populations; it is thus important to characterize the safety of antihistamines in the pediatric population. This study was designed to assess the safety of levocetirizine dihydrochloride oral liquid drops in infants and children with AR or CIU. Two multicenter, double-blind, randomized, parallel-group studies randomized infants aged 6-11 months (study 1, n = 69) and children aged 1-5 years (study 2, n = 173) to levocetirizine, 1.25 mg (q.d. or b.i.d., respectively), or placebo for 2 weeks, using a 2:1 ratio. Safety evaluations included treatment-emergent adverse events (TEAEs), vital signs, electrocardiographic (ECG) assessments, and laboratory tests. The overall incidence of TEAEs was similar between levocetirizine and placebo in both studies. Most TEAEs were mild or moderate in intensity. TEAEs prompted discontinuation of therapy in three patients receiving levocetirizine in study 1. No clinically relevant changes from baseline in vital signs or laboratory parameters were apparent in either study; changes from baseline in these evaluations were similar between groups. No significant changes were observed in ECG parameters, including corrected QT interval. Levocetirizine, 1.25 and 2.5 mg/day, was well tolerated in infants aged 6-11 months and in children aged 1-5 years, respectively, with AR or CIU. [\hyperlink{Doxepin Hydrochloride}{PMID: 20819318}, Frank Hampel et al., ]

\hypertarget{pmid_20386439}{A}lbumin has been regarded as the gold standard for maintaining adequate colloid osmotic pressure in children, but increased cost, the lack of clear-cut benefits for survival, and fear of transmission of unknown viruses have contributed to its replacement by hydroxyethyl starch and gelatin preparations. Each of the synthetic colloids has unique physicochemical characteristics that determine their likely efficacy and adverse effect profile. This review will examine the advantages and disadvantages of the use of different colloid solutions in children with a particular focus on their safety profile. Dextrans are rarely used because of their negative effects on coagulation and potential for anaphylactic reactions. Gelatin and albumin have little effect on hemostasis, but the disadvantages of gelatin include its high anaphylactoid potential and limited beneficial volume effect. Tetrastarches have significantly fewer adverse effects on coagulation and renal function than the older hydroxyethyl starches and are now approved for children. Dissolving tetrastarches in a plasma-adapted, balanced solution rather than in saline further improves safety with regard to coagulation and acid-base balance. Tetrastarches offer the best currently available compromise between cost-effectiveness and safety profile in children with preexisting normal renal function and coagulation. [\hyperlink{Doxepin Hydrochloride}{PMID: 20386439}, Sonja Saudan et al., 2010]

\hypertarget{pmid_20040824}{T}o assess the long-term safety and tolerability of atomoxetine hydrochloride in children and adolescents with attention-deficit/hyperactivity disorder treated for > or = 3 years. Data from 13 double-blind, placebo-controlled trials and 3 open-label extension studies were pooled. Outcome measures were patient-reported treatment-emergent adverse events (AEs); discontinuations due to AEs, serious AEs, and changes in body weight, height, vital signs, electrocardiogram, and hepatic function tests. In total, 714 patients were treated with atomoxetine for > or = 3 years (mean follow-up 4.8 years [SD 1.1 years]), including a subset of 508 treated for > or = 4 years (mean follow-up 5.3 years [SD 0.8 years]). Most subjects were younger than 12 years at entry (73.8\%), male (78.4\%), and white (88.9\%). The mean final daily dose of atomoxetine was 1.35 mg/kg (SD 0.37 mg/kg). No new or unexpected AEs were observed compared with acute-phase treatment. Less than 6\% of patients exhibited aggressive/hostile behaviors, and less than 1.6\% reported suicidal ideation/behavior. No clinically significant effects were seen on growth rate, vital signs, or electrocardiographic parameters, and < or = 2\% of patients showed potentially clinically significant hepatic changes. Atomoxetine was safe and well tolerated for children and adolescents with > or = 3 and/or > or = 4 years of treatment. [\hyperlink{Doxepin Hydrochloride}{PMID: 20040824}, Craig Donnelly et al., 2009]

\hypertarget{pmid_18702885}{A}llergic rhinitis (AR) is a common chronic condition in children and may impact a child's quality of life. Increasing treatment compliance may improve quality of life. An oral suspension of fexofenadine hydrochloride (HCl) has been developed to ease administration to children and may, therefore, improve treatment compliance. The purpose of this study was to assess the pharmacokinetic behavior, safety, and tolerability of a single dose of fexofenadine HCl oral suspension administered to children aged 2-5 years with allergic rhinitis. Children (aged 2-5 years) with AR were recruited in a multicenter, open-label, single-dose study. Fexofenadine HCl (30 mg) was administered as a 6-mg/mL suspension (5 mL). Plasma samples were collected up to 24 hours postdose. Adverse events (AEs); electrocardiograms (ECGs); vital signs; and clinical laboratory tests for hematology, blood chemistry, and urinalysis were analyzed to evaluate safety and tolerability. Fifty subjects completed the study. Mean maximum plasma concentration of fexofenadine was 224 ng/mL, and mean area under the plasma concentration curve was 898 ng . hour/mL. Treatment-emergent AEs were mild in intensity and reported in a total of seven subjects. No trends or clinically meaningful changes in mean ECG, vital sign, or clinical laboratory test data occurred during the study. In children aged 2-5 years, the exposure after a 30-mg dose of fexofenadine HCl suspension was similar to the exposures previously seen after a 30- and 60-mg dose of fexofenadine HCl in children aged 6-11 years and in adults, respectively. The suspension was also well tolerated. [\hyperlink{Doxepin Hydrochloride}{PMID: 18702885}, Nathan Segall et al., ]

\hypertarget{pmid_2682552}{D}opamine hydrochloride is widely used to increase blood pressure, cardiac output, urine output, and peripheral perfusion in neonates, infants, and older children with shock and cardiac failure. Its pharmacologic effects are dose dependent, and at low, intermediate, and high dosages include dilation of renal, mesenteric, and cerebral vasculature; inotropic response in the myocardium; and increases in peripheral and renal vascular resistance, respectively. The inotropic response is diminished in neonates compared with older children and adults due to maturational differences in norepinephrine stores. The clearance of dopamine varies widely in the pediatric population, depending on age. Its elimination half-life is approximately 2 minutes in full-term neonates and older children, and may be as long as 4-5 minutes in preterm infants. Due to immaturity of the autonomic nervous system, the drug may produce some adverse respiratory responses at high dose in neonates, the most common being tachycardia and cardiac arrhythmias. Dobutamine resembles dopamine chemically and is an analog of isoproterenol. It is relatively cardioselective at dosages used in clinical practice, with its main action being on beta 1-adrenergic receptors. Unlike dopamine, it does not have any effect on specific dopaminergic receptors. Dobutamine is used to increase cardiac output in infants and children with circulatory failure. Its elimination half-life is about 2 minutes in adults and older children. No information is available about its pharmacokinetics in neonates and infants. Adverse effects such as an increase in heart rate usually occur at high dosages. [\hyperlink{Doxepin Hydrochloride}{PMID: 2682552}, V Bhatt-Mehta et al., 1989]

\hypertarget{pmid_29747541}{D}exrazoxane can prevent anthracycline-associated cardiotoxicity. However, in 2011, its use in children was contraindicated by the EMA over concerns of increased risk of infection, myelosuppression and second primary malignancies, and because its efficacy in children had not then been established. We review here the evidence published since 2011, which confirms that dexrazoxane is an effective cardioprotectant in children and adolescents, is not associated with an increased risk of second primary malignancies or excess early or late mortality and does not impair chemotherapy efficacy. Based on this evidence, the contraindication for children and adolescents requiring high doses of anthracyclines and at risk for cardiotoxicity was removed from the European labeling for dexrazoxane. [\hyperlink{Doxepin Hydrochloride}{PMID: 29747541}, Peter Reichardt et al., 2018]

\section*{Dyrenium}
\subsection*{Result}
\subsubsection*{Answer}

Unknown

\subsubsection*{{Explanation}}
\hypertarget{Dyrenium}
A review of the provided abstracts reveals that none of them are about Dyrenium (triamterene), nor do any of them address the safety of Dyrenium in children. There are no targeted studies in the abstracts that evaluate the safety of Dyrenium in any pediatric age group. Therefore, based on the available abstracts, the safety of Dyrenium for use in children is unknown.

\subsection*{Abstracts}
\hypertarget{pmid_9428981}{T}he efficacy and tolerability of dimethindene maleate (CAS 3614-69-5, DMM, Fenistil) as drops in the treatment of pruritus in children suffering from chicken-pox were investigated in a study with two different doses of dimethindene maleate and placebo. 128 children, 1 to 6 years of age, were included in a double blind, randomized, placebo controlled, multi-center clinical trial. Patients received either a dosage of DMM of 0.1 mg/kg x d, or 0.05 mg/kg x d, or placebo, respectively. All patients received a commercially available astringent lotion for topical treatment of skin lesions. The primary efficacy criterion which was the change in the itching severity score from baseline to the end of the treatment assessed as area under the baseline (AUB) showed for both treatments with DMM a statistically significant superiority versus placebo in reducing the severity of itching. There was no statistically proven difference between the two verum groups. [\hyperlink{Dyrenium}{PMID: 9428981}, W Englisch et al., 1997]

\hypertarget{pmid_16719877}{T}he purpose of this study was to compare the safety and efficacy of oral midazolam and midazolam-diphenhydramine combination to sedate children undergoing magnetic resonance imaging (MRI). We performed a prospective randomized double-blind study in 96 children who were randomly allocated into two groups. Group D received oral diphenhydramine (1.25 mg x kg(-1)) with midazolam (0.5 mg x kg(-1)), and Group P received oral placebo with midazolam (0.5 mg x kg(-1)) alone. Sedation scores, onset and duration of sleep were evaluated. Adverse effects, including hypoxemia, failed sedation, and the return of baseline activity, were documented. Diphenhydramine facilitated an earlier onset of midazolam sedation (P < 0.01), and higher sedation scores (P < 0.01). In children who received midazolam alone, 20 (41\%) were inadequately sedated, compared with 9 (18\%) children who received midazolam and diphenhydramine combination (P < 0.01). Time to complete recovery was not significantly different between the two groups. Our study indicates that the combination of oral diphenhydramine with oral midazolam resulted in safe and effective sedation for children undergoing MRI. The use of this combination might be more advantageous compared with midazolam alone, resulting in less sedation failure during MRI. [\hyperlink{Dyrenium}{PMID: 16719877}, Mustafa Cengiz et al., 2006]

\hypertarget{pmid_11493818}{C}ontrolled intubation in the pediatric emergency department (ED) requires a paralytic agent that is safe, efficacious, and of rapid onset. The safety of succinylcholine has been challenged, leading some clinicians to use vecuronium as an alternative. Rocuronium's onset is similar to that of succinylcholine. To evaluate the safety and efficacy of rocuronium for controlled intubation with paralysis (CIP) in the pediatric ED. A retrospective, observational study reviewed the records of patients less than 15 years of age, who received controlled intubation with paralytics at two Dallas EDs. The patients received either vecuronium or rocuronium. The study included 84 patients (vecuronium 19, rocuronium 65). Complications were similar between the two groups. Rocuronium had a shorter time from administration to intubation when compared to vecuronium (P < 0.05). Rocuronium is as safe and efficacious as vecuronium for CIP in the pediatric ED. [\hyperlink{Dyrenium}{PMID: 11493818}, D R Mendez et al., 2001]

\hypertarget{pmid_9174878}{I}n a prospective, double-blind, controlled study the efficacy of clonidine was assessed in children, with respect to sedation, intubation response, and recovery. Fifty children, aged 4-12 years, undergoing general anesthesia were studied. Twenty-five children (group I) received oral diazepam) 0.2 mg/kg and another 25 children (group II) received oral clonidine 3 micrograms/kg, 90-120 minutes before induction of anesthesia. The level of sedation, hemodynamic changes to laryngoscopy and intubation, the recovery from anesthesia were noted and compared between the groups. Clonidine 3 micrograms/kg produced sedation comparable to diazepam 0.2 mg/kg (p > 0.1). There was significant (p > 0.01) attenuation of hemodynamic intubation response with clonidine. The recovery with clonidine was not delayed (p < 0.01). Clinically significant hypotension and bradycardia were not observed in any of the patients. We conclude that clonidine 3 micrograms/kg produces sedation comparable to diazepam 0.2 mg/kg and also attenuates the intubation response without increasing the incidence of complications. [\hyperlink{Dyrenium}{PMID: 9174878}, V J Ramesh et al., 1997]

\hypertarget{pmid_10828633}{D}iaper dermatitis is a common childhood affliction. Aiming to help reduce the prevalence of this problem, we have developed a novel diaper to deliver to the skin dermatological formulations intended to help protect the skin from overhydration and irritation. To determine the clinical benefits of a novel disposable diaper designed to deliver a petrolatum-based formulation continuously to the skin during use. Two independent, blinded, randomized clinical trials were conducted, involving an aggregate total of 391 children, 8-24 months of age. All comparisons were done versus a control diaper, identical to the test product except for the absence of the petrolatum formulation. The studies determined the effects of the novel diaper on skin erythema and diaper rash. Use of the formulation-treated diaper was associated with significant reductions in severity of erythema and diaper rash compared to the control product. The results demonstrated the clinical benefits associated with continuous topical administration of a petrolatum-based formulation by this novel diaper. We anticipate that this advance in diaper design will contribute significantly to further reduce the prevalence and severity of irritant contact dermatitis in the diaper area. [\hyperlink{Dyrenium}{PMID: 10828633}, M R Odio et al., 2000]

\hypertarget{pmid_33712253}{E}mergence Delirium (ED), particularly in children, is characterized by mental confusion, irritability, disorientation, and inconsolable crying. ED prolongs the time required in the Post-Anesthesia Care Unit (PACU) and increases concern and anxiety in parents. The present study aimed to determine the effectiveness and safety of low-dose clonidine in preventing ED in children receiving sevoflurane anesthesia for tonsillectomy/adenotonsillectomy. A randomized, double-blind clinical trial was conducted between November 2013 and January 2014. Sixty-two children aged 2-12 years, scheduled to undergo tonsillectomy/adenotonsillectomy, and classified as American Society of Anesthesiologists (ASA) physical status I/II were included, with 29 being randomized to receive 1 μg.kg The frequency of ED was significantly decreased in the group of children who received clonidine (17.2\% vs. 57.6\%; RR = 0.30; 95\% CI 0.13-0.70; p =  0.001). There was no difference between groups with respect to the frequency of postoperative self-harm (falls and bruises), dislodged catheters, and for most of the other adverse events evaluated. The use of 1 μg.kg NCT02181543. [\hyperlink{Dyrenium}{PMID: 33712253}, Fernando A Sousa-Júnior et al., ]

\hypertarget{pmid_26226440}{T}hree multicenter, randomized, controlled studies evaluated doripenem in children 3 months to <18 years of age, with complicated intra-abdominal or urinary tract infections and bacterial pneumonia.In the 66 patients treated with doripenem before early termination of the studies for nonsafety reasons, doripenem was safe and generally well tolerated. Low enrollment limited ability to assess benefits and risks of doripenem in children.  [\hyperlink{Dyrenium}{PMID: 26226440}, Christopher R Cannavino et al., 2015] This prospective self-controlled study was designed to evaluate the safety and efficacy of Desmopressin (DDAVP) in the treatment of childhood primary nocturnal enuresis (PNE) and the diurnal variation of ADH secretion and urine excretion/concentration in these children. Twenty-three children (15M,8F), aged 5-16 years, who wet their beds at least 3 nights per week were enrolled in the study. After a four-week observation period, they were hospitalized for one day to monitor intake, output, renal sonography, plasma ADH, urine and serum osmolality. Intranasal DDAVP treatment at a dose of 15-30 micrograms at bedtime was started with a 4-week titration period followed by a 3 to 6-month full dose treatment period. Subsequently the dose was tapered off for one to two months, and the patients were followed for at least two months to observe any recurrence. The results showed no diurnal difference of ADH level in these children (p > 0.05); serum osmolality decreased slightly during sleep (p < 0.01); urine production decreased, and urine osmolality increased, during sleep. Seventeen children (81\%) responded with a more than 50\% reduction in frequency of enuresis: 11 were excellent responders, 6 were partial responders, while 4 failed. After completion of therapy, four (19\%) remained dry and were considered cured; the rest had much less frequent recurrence. There were no subjective complaints other than mild local discomfort; laboratory test results remained normal. It was concluded that intranasal DDAVP is a safe and effective treatment for PNE which usually works promptly. Given the spontaneous annual remission rate of 14\%, the cure rate of 19\% in this study was not satisfactory. [\hyperlink{Dyrenium}{PMID: 26226440}, S G Shu et al., 1993]

\hypertarget{pmid_11720074}{D}iaper dermatitis is a common childhood affliction. Aiming to help reduce the prevalence of this problem, we have advanced in our development of a novel diaper that delivers dermatological formulations to help protect the skin from over-hydration and irritation. To determine the clinical benefits of a novel disposable diaper designed to deliver a zinc oxide and petrolatum-based formulation continuously to the skin during use. All studies were independent, blinded, randomized clinical trials. Study A was conducted to confirm transfer of the zinc oxide/petrolatum (ZnO/Pet) formulation from the diaper to the child's skin during use. Children wore a single diaper for 3 h or multiple diapers for 24 h. After the use period, stratum corneum samples were taken from each child and analysed for ZnO/Pet. Study B evaluated the prevention of skin irritation and barrier damage from a standard skin irritant (SLS) in an adult arm model. Study C evaluated skin erythema and diaper rash in 268 infants over a 4-week usage period. One half of the infants used the ZnO/Pet diaper, while the other half used a control diaper that was identical except for the absence of the ZnO/Pet formulation. The ointment formulation and ZnO transferred effectively from the diaper to the child's skin during product use. Transfer of ZnO increased from 4.2 microg/cm2 at 3 h to > 8 microg/cm2 at 24 h. Exposure to the formulations directly on adult skin prior to an irritant challenge was associated with up to a 3.5 reduction in skin barrier damage and skin erythema. Greatest reductions were seen for the ZnO containing formulations. Wearing of the formulation treated diaper was also associated with a significant reduction in skin erythema and diaper rash compared to the control product. The results demonstrated the clinical benefits associated with continuous topical administration of a zinc oxide/petrolatum-based formulation by this novel diaper. [\hyperlink{Dyrenium}{PMID: 11720074}, S Baldwin et al., 2001]

\hypertarget{pmid_28470100}{O}bjective To evaluate the safety and efficacy of dexmedetomidine (Dex) to prevent emergence agitation (EA) and delirium (ED) in children undergoing laparoscopic hernia repair under general anesthesia. Methods 100 children (1-5 years, 10-25 kg) were randomized into four groups: controls (saline) and intravenous Dex at 0.25, 0.5, and 1.0 µg/kg (D1, D2, D3, respectively). Dex/saline infusion was started following anesthesia. EA and ED were evaluated on a 5-point scale. Results For the C, D1, D2, and D3 groups, respectively, EA frequencies were 45.8\%, 30.4\%, 12\%, 4\%; ED frequencies 29.1\%, 13\%, 4\%, 4\%; CHIPPS scores 8, 6, 3, 3; sevoflurane doses from 13.2 ± 3.4 (controls) to 9.4 ± 3.5 ml (D3). Intervals until mask removal/spontaneous eye opening were significantly longer for D2 and D3 than controls. PACU stay was longer for D3. Conclusions There was significantly less postoperative EA and pain, with less sevoflurane required, using Dex. [\hyperlink{Dyrenium}{PMID: 28470100}, Yingying Sun et al., 2017] (1) Respiratory distress and seizures developed in an 18-month-old boy following brief exposure to low-strength (17.6\%) N,N-diethyl-m-toluamide (DEET). A review of the literature revealed 17 reports of DEET-induced encephalopathy in children. The objective of this study was to test the hypothesis that the potential toxicity of DEET is high and that available repellents containing DEET, irrespective of their strength, are not safe when applied to children's skin. (2) Although this is a case report, we used the features of published reports of DEET-induced encephalopathy in children to support the diagnosis, since the evidence that the child's illness was caused by DEET was circumstantial. In the following case analysis, clinical reports of children < 16 years old have been reviewed and analyzed in an effort to relate direct DEET toxicity to various clinical, demographic, and toxic compound exposure factors (Fisher's exacttest and logistic regression analysis). (3) DEET-induced encephalopathy in children (56\% girls) followed not only ingestion or repeated and extensive application of repellents, but also a brief exposure to DEET (45\%). Of those who reported a dermal exposure, 33\% reported an exposure to a product containing DEET < 20\%. Seizures, the most prominent symptom (72\%), were significantly more frequent when DEET solutions were applied to the skin (P<0.01). Mortality (16.6\%) did not correlate significantly with the concentration of the DEET liquid used, duration of skin exposure, pattern of use, age, or sex. (4) Data of this case analysis suggest that repellents containing DEET are not safe when applied to children's skin and should be avoided in children. Additionally, since the potential toxicity of DEET is high, less toxic preparations should be probably substituted for DEET-containing repellents, whenever possible. [\hyperlink{Dyrenium}{PMID: 28470100}, G Briassoulis et al., 2001]

\hypertarget{pmid_17109137}{T}o determine the efficacy and safety of intraperitoneal administration of darbepoetin in children with renal anemia on peritoneal dialysis, we conducted a single-arm, retrospective, two-centre study in which children were treated with intraperitoneal darbepoetin at the end of nightly intermittent peritoneal dialysis. Controls were those children treated with intraperitoneal erythropoietin six months before conversion to darbepoetin. Children were converted with the conversion factor 200 units erythropoietin=1 microg darbepoetin. Children who started with darbepoetin, started with 0.45 microg/kg/week. Nineteen children entered the study. The mean age was 6.8 years. Eight children were converted from 201 U/kg/week intraperitoneal erythropoietin to 1.0 microg/kg/week intraperitoneal darbepoetin. They were treated for a median period of 31.5 months. Median darbepoetin dose for an adequate erythropoesis over this period was 0.79 microg/kg/week. All 19 children were treated with darbepoetin for a median period of 13.4 months. The median dose for an adequate erythropoesis over this period was 0.63 microg/kg/week. The peritonitis incidence during this study was once every 25.1 months. Three children developed hypertension; one child developed headache. These complications developed after a rapid increase of hemoglobin concentration. Intraperitoneal administration of darbepoetin is effective and safe for children on peritoneal dialysis. [\hyperlink{Dyrenium}{PMID: 17109137}, Yvonne Rijk et al., 2007]

\hypertarget{pmid_7699844}{A} new oral penem antibiotic, SY5555, was evaluated for its safety and efficacy in 35 children with various bacterial infections. SY5555 was effective in 100\% of scarlet fever, pharyngotonsillitis, pneumonia, otitis media, bacterial diarrhea, urinary tract infections and skin and soft tissue infections. The etiologic bacteria were eradicated except Salmonella sp. Side effects were observed in 3.5\% cases; one was diarrhea and Candida dermatitis, one was loose stool, and one was Candida dermatitis. From these data, SY5555 is thought to be a safe and effective antibiotic in the pediatric field. Regular dose of suspension preparation is 15 mg/kg/day in 3 divided dosages, and when needed the dose may be doubled. [\hyperlink{Dyrenium}{PMID: 7699844}, H Meguro et al., 1995]

\hypertarget{pmid_6349908}{T}he pharmacokinetic properties and dosage guidelines for digoxin in pediatric patients with congestive heart failure are reviewed. Interindividual variability in the pharmacokinetics of digoxin in pediatric patients has been reported. The bioavailability of digoxin elixir in newborns and infants is similar to adults; however, the apparent volume of distribution has been reported to be greater in infants than in adults. The total body clearance of digoxin is lowest in premature and full-term neonates and highest in infants aged one month to one year. The elimination half-life of digoxin has been reported to vary significantly among the different age groups of pediatric patients. The usefulness of monitoring digoxin serum concentrations in pediatric patients remains a controversial issue. Serum samples should be drawn under steady-state conditions to evaluate predicted daily maintenance doses. Although infants have been reported to be more tolerant than adults to elevated serum digoxin concentrations, infants experience a higher rate of digoxin toxicity than previously realized. Recent studies have shown appropriate therapeutic response in neonates and infants when low dosages of digoxin are administered. Low digoxin dosage regimens should be used initially for infants with congestive heart failure. If the clinical response is unsatisfactory or if toxicity is suspected, steady-state serum concentrations should be determined and the dosage adjusted. [\hyperlink{Dyrenium}{PMID: 6349908}, R Bendayan et al., ]

\hypertarget{pmid_30036731}{T}he World Health Organization recommends indoor residual spraying (IRS) of insecticides (including dichlorodiphenyltrichloroethane [DDT]) to fight malaria vectors in endemic countries. There is limited information on children's exposure to DDT in sprayed areas, and tools to estimate early-life exposure have not been thoroughly evaluated in this context. To document serum p,p'-DDT/E levels in 47 mothers and children participating in the Venda Health Examination of Mothers, Babies and their Environment (VHEMBE), a study conducted in an area where IRS insecticides are used annually, and to evaluate the precision and accuracy of a published pharmacokinetic model for the estimation of children's p,p'-DDT/E levels. p,p'-DDT/E levels were measured in maternal serum at delivery, and in children's serum at 12 and 24 months of age. A pharmacokinetic model of gestational and lactational exposure was used to estimate children's p,p'-DDT/E levels during pregnancy and the first two years of life, and estimated levels were compared to measured levels. The geometric means of children's serum p,p'-DDT/E levels at 12 and 24 months were higher than those of maternal serum levels. Regression models of measured children's p,p'-DDT/E levels vs. levels estimated with the pharmacokinetic model (which only accounted for children's exposure through placental transfer and breastfeeding) had coefficients of determination (R Results indicate that children living in a sprayed area have serum p,p'-DDT/E levels exceeding their mothers' during the first two years of life. The pharmacokinetic model may be useful to estimate children's levels in the VHEMBE population. [\hyperlink{Dyrenium}{PMID: 30036731}, Marc-André Verner et al., 2018]

\hypertarget{pmid_16928841}{T}his study sought to investigate the efficacy of dextromethorphan (DM), diphenhydramine (DPH), and placebo (PL) for symptoms attributed to upper respiratory infections as determined by children, and to evaluate the concordance of perception of nocturnal symptoms between children and parents. A total of 37 children age 6 to 18 years of age were randomized in a double-masked fashion to receive a single bedtime dose of DM, DPH, or PL. Children found no significant difference in the effect of DM, DPH, or PL for any study outcome, and responses by parents and children were significantly correlated. [\hyperlink{Dyrenium}{PMID: 16928841}, Katharine E Yoder et al., 2006]

\hypertarget{pmid_7699835}{C}linical studies on SY5555 dry syrup, a new oral penem antibiotic, were carried out in the field of pediatrics. The following results were obtained. 1. SY5555 was administered to 10 children with various bacterial infections (2 patients with acute tonsillitis, 2 with acute bronchitis, 1 with pharyngitis, 2 with scarlet fever, 1 with pertussis and 2 with urinary tract infections). The overall clinical efficacy rate was 90\%. 2. Side effects or abnormal laboratory test values were not observed except for loose stool in 1 and eosinophilia in 1. [\hyperlink{Dyrenium}{PMID: 7699835}, K Kuno et al., 1995]

\hypertarget{pmid_568870}{A} double-blind study of 18 children aged 6--12 years suffering from primary nocturnal enuresis without signs of underlying organic disease is reported. 20 microgram of DDAVP (desamino-D-arginine vasopressin, Minirin) was given intranasally at bedtime. The effect was prompt and satisfactory in 8 children and relatively good in another 8 children. No adverse effects were noted. DDAVP is advocated for temporary use in children with nocturnal enuresis needing immediate help. [\hyperlink{Dyrenium}{PMID: 568870}, T Tuvemo et al., 1978]

\hypertarget{pmid_28827252}{C}eftriaxone is widely used in children in the treatment of sepsis. However, concerns have been raised about the safety of ceftriaxone, especially in young children. The aim of this review is to systematically evaluate the safety of ceftriaxone in children of all age groups. MEDLINE, PubMed, Cochrane Central Register of Controlled Trials, EMBASE, CINAHL, International Pharmaceutical Abstracts and adverse drug reaction (ADR) monitoring systems will be systematically searched for randomised controlled trials (RCTs), cohort studies, case-control studies, cross-sectional studies, case series and case reports evaluating the safety of ceftriaxone in children. The Cochrane risk of bias tool, Newcastle-Ottawa and quality assessment tools developed by the National Institutes of Health will be used for quality assessment. Meta-analysis of the incidence of ADRs from RCTs and prospective studies will be done. Subgroup analyses will be performed for age and dosage regimen. Formal ethical approval is not required as no primary data are collected. This systematic review will be disseminated through a peer-reviewed publication and at conference meetings. CRD42017055428. [\hyperlink{Dyrenium}{PMID: 28827252}, Linan Zeng et al., 2017]

\hypertarget{pmid_21110437}{T}o investigate the efficacy and safety of ademetionine for treatment of cholestatic or mixed-type drug-induced liver disease (DILD) in children. The children with DILD were divided into the treated group and control group. Yinzhihuang Granule was orally administered and Compound Glycyrrhizin Injection intravenously given in patients of both groups. Those patients in the treated group were additionally treated with intravenous infusion of 250-1000 mg ademetionine for 28 d. The incidence of pruritus and adverse effects as well as biochemical parameters in all the patients and compared between the 2 groups. For statistical analysis, Chi2 test was used for between-group comparison and t test for processing the data. 1) Before treatment, severe pruritus was found in 17 and 16 children in the treated and control group, respectively. Two weeks after the treatment, the symptom was significantly relieved in 14 and 3 patients in the treated and control group, respectively (Chi2 = 4.52, P < 0.05). 2) As for comparisons between the 2 groups, a P value of 0.0014 for AST level was found 4 weeks, 0.045 and 0.007 for disappearance and recovery rate of jaundice, 0.0014 and 0.0006 for decrease in TBA level and 0.0003 for gammaGT level 2 and 4 weeks after the treatment. Intravenous administration of ademetionine is safe in children with DILD and it can effectively alleviate pruritus, promote the recovery of various biochemical parameters and fasten liver functional recovery in these children. Therefore, ademetionine can be widely used for treatment of intrahepatic cholestasis in children. [\hyperlink{Dyrenium}{PMID: 21110437}, Shi-Shu Zhu et al., 2010]

\hypertarget{pmid_25102462}{T}he need to manage children using safe, effective and inexpensive conscious sedation materials and techniques in paediatric dentistry is high. This study evaluated the safety and effectiveness of a combination of low dose ketamine (5 mg/kg) and diazepam (0.2 mg/kg) used for conscious sedation in healthy children undergoing paediatric dental procedures at a paediatric dental outpatient clinic over a 3-year period. All children who were scheduled for conscious sedation between 2009 and 2012 were included in the study. All children received ketamine 5 mg/kg body weight in combination with diazepam 0.2 mg/kg body weight in a single oral dose for use as conscious sedation. Patients were considered sedated when the Ramsey Score was 2 or 3. Time of onset and duration of surgical procedures were recorded. Side effects during and after discharge were recorded. Twenty five patients participated in the study. The effectiveness of the sedation was 84.0\%. The mean time of onset of action was 10.5 ± 7.2 minutes. All cases that needed additional sedation needed this after 35?36 minutes. Three cases (12.0\%) developed high temperature in the night of the day of the procedure. There was a case (4.0\%) of hallucination. Ketamine and diazepam as medication for conscious sedation was considered effective. The duration of effectiveness appears to be 35 minutes. The combination is considered safe for use for conscious sedation in healthy paediatric dental outpatients undergoing minor oral surgical procedures. [\hyperlink{Dyrenium}{PMID: 25102462}, M O Folayan et al., 2014]

\hypertarget{pmid_18462105}{W}e conducted a randomized, double-blind, placebo-controlled trial comparing supplementation with bovine lactoferrin versus placebo for the prevention of diarrhea in children. Comparison of overall diarrhea incidence and prevalence rates found no significant difference between the 2 groups. However, there was a lower prevalence of colonization with Giardia species and better growth among children in the lactoferrin group. [\hyperlink{Dyrenium}{PMID: 18462105}, Theresa J Ochoa et al., 2008]

\hypertarget{pmid_27077621}{D}ata on the use of deferiprone in young children with iron overload are limited. To study the safety profile of a liquid formulation of deferiprone in chelating young children with transfusion-induced iron overload. A daily dose of 50-100 mg/kg BW in three divided doses of oral deferiprone was given to young patients who had received at least ten packed red cell transfusions and achieved a serum ferritin level >1000 μg/L during a 12-month period from 2011 to 2012. Nine children (four males) diagnosed with various types of thalassaemia (n = 8) and hereditary spherocytosis (n = 1) were enrolled. Their mean (SD) age was 4.5 (1.9) years. The patients received 15-20 ml/kg BW of packed red cell transfusions every 4-8 weeks from a mean (SD) age of 2.1 (1.7) years to maintain a pre-transfusion haematocrit at 27\%. A mean (SD) total of packed red cells of 5132 (2725) ml were given within a mean (SD) duration of 2.4 (1.1) years before the study. During the 1-year study period, they received a mean (SD) total of packed red cells of 2194 (680) ml or 138 (50) ml/kg BW with a mean (SD) daily iron load of 0.29 (0.12) mg/kg BW. The pre-treatment geometric mean of serum ferritin of 1863.8 μg/L decreased to 1279.7 μg/L after 1 year of treatment (P = 0.05). All patients tolerated the liquid formulation well and did not experience any gastro-intestinal discomfort, nausea or vomiting. The liquid formulation of deferiprone is safe in young children with transfusion-induced iron overload. [\hyperlink{Dyrenium}{PMID: 27077621}, Ampaiwan Chuansumrit et al., 2016]

\hypertarget{pmid_26052612}{D}ata on the use of deferiprone in young children with iron overload are limited. To study the safety profile of a liquid formulation of deferiprone in chelating young children with transfusion-induced iron overload. A daily dose of 50-100 mg/kg BW in three divided doses of oral deferiprone was given to young patients who had received at least ten packed red cell transfusions and achieved a serum ferritin level >1000 μg/L during a 12-month period from 2011 to 2012. Nine children (four males) diagnosed with various types of thalassaemia (n = 8) and hereditary spherocytosis (n = 1) were enrolled. Their mean (SD) age was 4.5 (1.9) years. The patients received 15-20 ml/kg BW of packed red cell transfusions every 4-8 weeks from a mean (SD) age of 2.1 (1.7) years to maintain a pre-transfusion haematocrit at 27\%. A mean (SD) total of packed red cells of 5132 (2725) ml were given within a mean (SD) duration of 2.4 (1.1) years before the study. During the 1-year study period, they received a mean (SD) total of packed red cells of 2194 (680) ml or 138 (50) ml/kg BW with a mean (SD) daily iron load of 0.29 (0.12) mg/kg BW. The pre-treatment geometric mean of serum ferritin of 1863.8 μg/L decreased to 1279.7 μg/L after 1 year of treatment (P = 0.05). All patients tolerated the liquid formulation well and did not experience any gastro-intestinal discomfort, nausea or vomiting. The liquid formulation of deferiprone is safe in young children with transfusion-induced iron overload. [\hyperlink{Dyrenium}{PMID: 26052612}, Ampaiwan Chuansumrit et al., 2016]

\hypertarget{pmid_1610033}{T}o assess the clinical severity of envenomation by the Centruroides scorpion in young children; the potential benefit of antivenin administration with regard to time to resolution of symptoms, prevention of hospitalization, and prevention of invasive intervention; and the risk of acute and delayed hypersensitivity reaction. Retrospective chart review with follow-up contact. Referral toxicology management center. Children less than 10 years of age experiencing severe Centruroides scorpion envenomation. Administration of goat serum-derived anti-Centruroides antivenin. Stings in young children may be life threatening, resulting in extreme neuromuscular activity, tachycardia, and respiratory distress. Administration of anti-Centruroides antivenin resulted in rapid resolution of all symptoms without acute reaction in all 12 patients so treated; 58\% of patients treated with antivenin had a delayed onset of rash or symptoms of serum sickness. The benefit of rapid resolution of life-threatening symptoms and potential for outpatient management of severe envenomation in young children may justify the risk of acute and delayed reaction associated with antivenin use in this group. The use of antivenin for the less severe envenomation common in older children and adults may subject them to unjustified risk. [\hyperlink{Dyrenium}{PMID: 1610033}, G R Bond et al., 1992]

\section*{Etoposide}
\subsection*{Result}
\subsubsection*{Answer}

Infants (<1 year): Yes  
Young Children (1–5 years): Yes  
Children (6–12 years): Yes  
Adolescents (13–18 years): Yes  

\subsubsection*{{Explanation}}
\hypertarget{Etoposide}
A review of the available abstracts reveals multiple targeted studies evaluating the safety and toxicity of etoposide in children across various age ranges and disease contexts. Below is a summary of the evidence by age group:

Infants (<1 year):
- One pharmacokinetic study included children as young as 3 months and found no significant age-dependent differences in etoposide clearance or tolerance between infants and older children. The study concluded that special dose-calculation guidelines for infants are not substantiated by age-dependent pharmacokinetics or tolerance, suggesting that etoposide is tolerated in this age group [\hyperlink{pmid_8523060}{PMID: 8523060}, J Boos et al., 1995]. Another study included children 0.5-1.8 years old and found similar clearance rates to older children, with no specific safety concerns raised [\hyperlink{pmid_17001183}{PMID: 17001183}, Josefine Palle et al., 2006]. Additionally, a small study of six children with Langerhans cell histiocytosis, five of whom were less than 2 years old, reported complete response in five and no major toxicity [\hyperlink{pmid_1669481}{PMID: 1669481}, M B Viana et al., 1991].

Young Children (1–5 years):
- Several studies included children in this age range. The studies on Langerhans cell histiocytosis and recurrent brain tumors included children as young as 4 years and reported that etoposide was well tolerated, with no treatment-related deaths and only manageable, expected toxicities (e.g., alopecia, neutropenia, anemia) [\hyperlink{pmid_9152112}{PMID: 9152112}, M C Chamberlain et al., 1997; \hyperlink{pmid_1669481}{PMID: 1669481}, M B Viana et al., 1991; \hyperlink{pmid_1449115}{PMID: 1449115}, E Ishii et al., 1992]. 

Children (6–12 years):
- Multiple studies specifically included children in this age range. For example, studies of children with recurrent brain tumors (median age 9 years, range 4–18) and ependymoma (median age 8 years) found that oral etoposide was well tolerated, with modest and manageable toxicity, and no treatment-related deaths [\hyperlink{pmid_9152112}{PMID: 9152112}, M C Chamberlain et al., 1997; \hyperlink{pmid_11275460}{PMID: 11275460}, M C Chamberlain et al., 2001; \hyperlink{pmid_16189442}{PMID: 16189442}, Alessandro Sandri et al., 2005]. Another study of 18 children with recurrent Langerhans cell histiocytosis found no dose-limiting major toxicities [\hyperlink{pmid_3056605}{PMID: 3056605}, A Ceci et al., 1988].

Adolescents (13–18 years):
- Studies included adolescents up to 18 years old and found similar pharmacokinetics and toxicity profiles as in younger children, with no new safety concerns [\hyperlink{pmid_17001183}{PMID: 17001183}, Josefine Palle et al., 2006; \hyperlink{pmid_12654074}{PMID: 12654074}, Yasuhiro Kato et al., 2003]. A study of 28 children with recurrent brain and solid tumors (age range not specified, but pediatric) found toxicity was manageable and rarely required intervention [\hyperlink{pmid_9142202}{PMID: 9142202}, M N Needle et al., 1997].

General Pediatric Population (<18 years):
- Multiple studies and reviews affirm that etoposide is widely used in pediatric oncology, with predictable and manageable toxicity (mainly myelosuppression, alopecia, GI toxicity), and is considered well tolerated in children [\hyperlink{pmid_9038612}{PMID: 9038612}, S P Lowis et al., 1996; \hyperlink{pmid_6326063}{PMID: 6326063}, J A Sinkule et al.]. Hypersensitivity reactions can occur but are generally manageable, and desensitization protocols or switching to etoposide phosphate are available options [\hyperlink{pmid_32031209}{PMID: 32031209}, Nicole Martinez et al., 2020; \hyperlink{pmid_31315549}{PMID: 31315549}, Winifred M Stockton et al., 2020].

Summary:
- Across all pediatric age groups, including infants, young children, children, and adolescents, multiple targeted studies have evaluated the safety of etoposide. These studies consistently report that etoposide is well tolerated in children, with expected and manageable toxicities, and no evidence of unexpected or unacceptable safety concerns. Therefore, based on the available abstracts, etoposide is affirmed as safe for use in children when used with appropriate monitoring and dosing.

\subsection*{Abstracts}
\hypertarget{pmid_9152112}{A} long-term regimen of oral etoposide, a type of chemotherapy, is used in oncology and is effective in treating germ-cell tumors, lymphomas, Kaposi sarcoma, and primary brain tumors. To examine the toxic effects and efficacy of long-term salvage chemotherapy using oral etoposide. Fourteen children (8 boys and 6 girls) with recurrent supratentorial gliomas, ranging in age from 4 to 18 years (median age, 9 years), were treated with etoposide. Tumor histologic grades included Daumas-Duport grade 3 (10 children) and grade 4 astrocytomas (4 children). All children had been treated previously with radiotherapy (median dose, 60 Gy) and nitrosourea-based chemotherapy. Each cycle of therapy consisted of 21 days of etoposide (50 mg/m2 daily) followed by a 14-day period of rest and an additional 21 days of etoposide (50 mg/m2 daily). Measurements of complete blood cell counts were taken biweekly. A neurological examination and a magnetic resonance image of the brain with contrast medium were performed before each cycle of therapy. Treatment-related complications included the following partial alopecia (8 children); diarrhea (6 children); weight loss (4 children); anemia (4 children); neutropenia (4 children) and thrombocytopenia (4 children). Four children required transfusion (4 with packed red blood cells and 3 with platelets) and 2 children received antibiotic therapy for neutropenic fever. There were no treatment-related deaths. All children were examined for response. In 7 children (50\%), the results of magnetic resonance imaging indicated either a partial response (3 children) or stable disease (4 children), with a median duration of response of 8 months. Oral etoposide is a well tolerated and relatively nontoxic chemotherapeutic agent with demonstrated activity in children with recurrent supratentorial gliomas. [\hyperlink{Etoposide}{PMID: 9152112}, M C Chamberlain et al., 1997]

\hypertarget{pmid_17001183}{W}e studied the pharmacokinetics of etoposide in 45 children treated for newly diagnosed acute myeloid leukemia. Etoposide, 100 mg/m body surface area/24 h, was administered by 96-h continuous intravenous infusion. Concomitantly, the children received cytarabine 200 mg/m/24 h by intravenous infusion and 6-thioguanine 100 mg/m twice daily orally. Median total body clearance in children 0.5-1.8 (n=4) and 2.3-17.7 years old (n=36) without Down's syndrome was 17.1 and 17.6 ml/min/m, respectively (P=0.96). Five children with Down's syndrome had a median clearance of 13.6 ml/min/m (P=0.067 compared with non-Down's syndrome children). Eighteen of the children received a second identical treatment course 3-4 weeks later; there was a significant correlation between individual clearance values (rho=0.56; P=0.017). We found no significant correlation between etoposide pharmacokinetics and the remission rate or the relapse rate. In conclusion, our findings indicate that special dose-calculation guidelines for infants above 3 months old are not substantiated by age-dependent pharmacokinetics of etoposide. Down's syndrome children might be candidates for dose reduction if our data are confirmed in larger numbers of patients. Low course-to-course variability indicates that pharmacokinetically guided dosing of etoposide might be clinically relevant, if larger studies can demonstrate that this approach decreases toxicity or increases response rates. [\hyperlink{Etoposide}{PMID: 17001183}, Josefine Palle et al., 2006]

\hypertarget{pmid_10898537}{T}he pharmacokinetics of etoposide (VP-16), a semi-synthetic derivative of podophyllotoxin, were studied in 16 pediatric patients (median age 8.3 years; range 4 months to 22 years) including two girls with Down's syndrome (DS). The drug was administered as infusions (1-3 h) in a wide range of doses (9-322 mg, corresponding to 32-210 mg/m2). The area under the plasma concentration versus time curve (AUC), dose normalized by the body surface area, was independent of age, while AUC normalized by the dose in mg/kg increased with increasing age of the patients. The interpatient variability of AUC, normalized for the dose in mg/m2, was 23\% (CV) compared to 32\% (CV) normalized for the dose in mg/kg. The terminal half-life time was 4.1 h (median value; range 2.0-7.8 h). The pharmacokinetics of etoposide in children with DS and chromosomally normal children were very similar with regard to systemic drug exposure and plasma half-life time. From the pharmacokinetic point of view it was therefore not necessary to make any dose modifications in the two girls with DS. The two DS patients did not experience any enhanced degree of toxicity from their etoposide treatments. The results support that dosing of etoposide to children should be based on body surface area. [\hyperlink{Etoposide}{PMID: 10898537}, S Eksborg et al., 2000]

\hypertarget{pmid_1669481}{S}ix children received etoposide as the single agent for treatment of Langerhans cell histiocytosis (LCH; histiocytosis X). Five were less than 2 years old at diagnosis. All had multiorgan involvement; one had liver and pulmonary dysfunction. Two infants also had clinical signs of immune deficiency. Complete response was observed in five. There was no major toxicity. Although three of four evaluable patients relapsed, the drug was considered useful in moving the children from a symptomatic to an asymptomatic clinical status. Etoposide may become a "first-line" drug in the treatment of systemic LCH, especially when the side effects of steroid therapy are considered unacceptable. [\hyperlink{Etoposide}{PMID: 1669481}, M B Viana et al., 1991]

\hypertarget{pmid_15224403}{E}toposide is a podophyllotoxin semiderivative that is used in a variety of chemotherapy treatments, including therapy for children tumors. This drug promotes the formation of a ternary DNA-topoisomerase II-etoposide complex that triggers apoptosis. The purpose of this work was to analyze the occurrence of apoptosis in the seminiferous epithelium of prepubertal, pubertal, and adult rats treated with 10, 20, and 40 mg/Kg of etoposide during the prepubertal phase, as well as the role of apoptosis in etoposide-induced testicular damage. The rat testes were fixed in Bouin's liquid, and the apoptotic cells were quantified by means of the hematoxylin and eosin (H\&E) technique (all groups) and the terminal dUTP nick end labeling (TUNEL) method (prepubertal groups only). The results obtained from both the H\&E and TUNEL methods showed an increased frequency of apoptosis in the seminiferous epithelium of treated animals, except for the subgroup that received the 10-mg/Kg dose and was sacrificed 12 hr after the treatment and for the etoposide-treated pubertal group, that did not show cells suggesting apoptosis during H\&E analysis. The labeled cells were mainly primary spermatocytes and differentiated spermatogonia. The prepubertal rats showed an etoposide-dose-dependent diminution of differentiated spermatogonia. Etoposide treatment during the prepubertal phase increases the frequency of apoptosis in the seminiferous epithelium, and causes serious harm to male fertility. 2004. [\hyperlink{Etoposide}{PMID: 15224403}, Taiza Stumpp et al., 2004]

\hypertarget{pmid_11275460}{C}hronic oral VP-16 (etoposide) is a chemotherapy regimen with a wide application in oncology and documented efficacy against germ cell tumors, lymphomas, Kaposi's sarcoma, and primary brain tumors. This study was performed to assess the toxicity and activity of chronic oral etoposide in the management of children with recurrent intracranial nondisseminated ependymoma. Twelve children (median age of 8 years) with recurrent ependymoma who were refractory to surgery, radiotherapy, and chemotherapy (carboplatinum or the combination of procarbazine, lomustine, and vincristine) were treated with chronic oral etoposide (50 mg/m(2)/day). Treatment-related complications included the following: alopecia (10 children), diarrhea (6), weight loss (5), anemia (4), neutropenia (3), and thrombocytopenia (3). Three children required transfusion (two with packed red blood cells; two with platelets), and two children developed neutropenic fever. No treatment-related deaths occurred. Six children (50\%) demonstrated either a radiographic response (two children, both with partial response) or stable disease (four children) with a median duration of response or stable disease of 7 months. In this small cohort of children with recurrent intracranial ependymoma, oral etoposide was well tolerated, produced modest toxicity, and had apparent activity. [\hyperlink{Etoposide}{PMID: 11275460}, M C Chamberlain et al., 2001]

\hypertarget{pmid_3056605}{E}ighteen evaluable children with recurrent Langerhans' cell histiocytosis (LCH) which was resistant to standard therapy, were treated with etoposide (VP 16-213), 200 mg/m2/day for 3 days every 3 weeks, to study the efficacy and toxicity of this drug. Complete and partial responses were demonstrated in 15 patients (83.3\%). Only one of the 12 children achieving a complete remission has relapsed. No dose-limiting major toxicities were registered. Although etoposide might be an effective treatment in recurrent LCH which needs a chemotherapeutic approach, it is emphasized that this drug must be used carefully. [\hyperlink{Etoposide}{PMID: 3056605}, A Ceci et al., 1988]

\hypertarget{pmid_32031209}{T}he implementation of a pediatric desensitization protocol specific to etoposide in an adolescent with Hodgkin lymphoma is described. Etoposide is part of many chemotherapy regimens used to treat malignancies in children and adults, and it is also part of the backbone of many regimens used in clinical trials. Etoposide is known to produce hypersensitivity reactions during administration. Substitution with etoposide phosphate, which has less potential for hypersensitivity reactions, is used in place of etoposide after severe hypersensitivity reactions. Etoposide desensitization protocols (EDPs) have been reported in adult patients. The implementation of an etoposide desensitization protocol for pediatric patients is safe and helpful to prevent the elimination of etoposide from treatment protocols. The use of an EDP allowed the patient to remain on clinical trial and complete the prescribed treatment. [\hyperlink{Etoposide}{PMID: 32031209}, Nicole Martinez et al., 2020]

\hypertarget{pmid_8827052}{E}toposide is bound to plasma albumin (94\%). Previous studies have revealed altered protein binding of etoposide in cancer patients. This has clinical implications since only the free fraction is considered pharmacologically active. We have studied the etoposide protein binding in 11 children (eight acute lymphocytic leukemia, two malignant histiocytosis, and one oligodendroglioma; age 1-17 years) and 46 adult patients (28 acute myelocytic leukemia, eight lymphoma, one multiple myeloma, and nine small cell lung cancer; age 38-81 years). All patients were treated with etoposide 50-200 mg/m2 i.v. or orally. Plasma from ten healthy volunteers, 26-50 years of age, was spiked with etoposide, 10 micrograms/ml, and the protein binding was compared with that in patient samples. The free etoposide concentration was determined by high performance liquid chromatography (HPLC) after ultrafiltration at room temperature. The free etoposide fraction was lower, 2.5 +/- 0.6\% (mean +/- SD), in the children compared with 5.0 +/- 3.6\% in adult cancer patients. In plasma from healthy adults it was 3.2 +/- 0.3\%. It is concluded that children have significantly lower levels of free etoposide compared with adult patients (P = 0.03) as well as with healthy subjects (P = 0.001), which is likely to affect metabolism and renal clearance as well as cellular uptake of the drug. [\hyperlink{Etoposide}{PMID: 8827052}, E Liliemark et al., 1996]

\hypertarget{pmid_31315549}{E}toposide is critical in treating pediatric cancers, although hypersensitivity can be severe and treatment-limiting. Reported rates of hypersensitivity range from 2\% to 51\%. Hypersensitivity data for etoposide phosphate, a newer product, are lacking. The primary objective of this study was to assess etoposide and etoposide phosphate hypersensitivity incidence. Secondary objectives included evaluation of potential risk factors for hypersensitivity and strategies to prevent recurrence. This retrospective cohort study evaluated pediatric patients who received initial etoposide phosphate or etoposide dose between August 2012 and July 2017. The primary outcome was documentation of hypersensitivity within four months of initial dose. Potential risk factors evaluated included age, allergies, dose, infusion rate, infusion concentration, and premedication. Of 246 patients, hypersensitivity reactions occurred in five of 54 patients (9.3\%) who received etoposide phosphate and 52 of 192 patients (27.1\%) who received etoposide ( Etoposide was associated with more hypersensitivity than etoposide phosphate in pediatric patients. Etoposide hypersensitivity was associated with higher infusion rates, but not etoposide phosphate. Differences in hypersensitivity incidence and infusion rate influence indicate a formulation-effect. Etoposide hypersensitivity recurrence may be prevented by changing to etoposide phosphate formulation. During etoposide phosphate shortages, etoposide desensitization may prevent recurrent hypersensitivity. [\hyperlink{Etoposide}{PMID: 31315549}, Winifred M Stockton et al., 2020]

\hypertarget{pmid_9038612}{E}toposide is one of the most important drugs available for the treatment of paediatric malignancies. Although there is evidence of schedule dependency for etoposide therapy in adults with small-cell lung cancer, the relevance of this observation to childhood cancers is uncertain. Prolonged parenteral or oral etoposide therapy has not yet shown a clear-cut advantage over intermittent treatment, and there are still no data to show that the administration of etoposide as a short intravenous (i.v.) daily infusion for 5 days does not represent acceptable therapy for primary disease. The pharmacokinetic variability seen with etoposide argues strongly for the use of pharmacologically guided dosing, and the introduction of etoposide phosphate will simplify both parenteral etoposide administration and the future evaluation of alternative etoposide schedules. Although the impact of molecular and cellular pharmacological investigations on the clinical use of etoposide has yet to be felt, the tools to perform these studies are now available and prospective trials can be designed. Such studies, performed in the setting of a pharmacologically guided trial to ensure control over pharmacokinetic variability, should identify the best way of treating children with etoposide. [\hyperlink{Etoposide}{PMID: 9038612}, S P Lowis et al., 1996]

\hypertarget{pmid_21159109}{E}toposide (VP-16) is one of the most widely used antitumor agents in pediatric oncology as well as chemotherapeutic agents used in conditioning regimen prior to allo-HSCT for childhood ALL. This study included 21 children with ALL who underwent allo-HSCT after conditioning with FTBI and high-dose of VP-16 (60 mg/kg) given intravenously as single four-h infusion on day -3 (n=2) or day -4 (n=19) prior to allo-HSCT. Blood samples were collected at defined time intervals until 120 h elapsed from the end of infusion. VP-16 plasma concentrations were determined using validated HPLC method. Three-compartment model was assumed for assessing PK parameters of VP-16. The median value of VP-16 C(max) measured at the end of infusion was 188.0 μg/mL (range 148.0-407.0 μg/mL). Out of 21 studied children, VP-16 was still detectable in 17 patients 72 h (median concentration 0.31 μg/mL) and in eight patients 96 h (median concentration 0.31 μg/mL) after the end of infusion. VP-16 concentration 96 h after the end of infusion was positively correlated with VP-16 AUC and negatively correlated with VP-16 CL normalized to body weight. [\hyperlink{Etoposide}{PMID: 21159109}, Maria Chrzanowska et al., 2011]

\hypertarget{pmid_1449115}{T}en children with Langerhans cell histiocytosis (LCH) were treated with etoposide. For five patients, this was the initial diagnosis. The other five had failed to respond to previous therapies. Etoposide (100 mg/m2) was given intravenously twice a week for 4 weeks, followed by maintenance therapy every 2 to 4 weeks for 2 years. All 10 patients responded to etoposide, and 6 of them (60\%) have been in complete remission for 3 to 36 months without any side effects. One patient relapsed with diabetes insipidus, one with a soft tissue mass, and two others developed multiple bone lesions. Chemotherapy with etoposide appears to be effective and safe for the treatment of children with systemic LCH. [\hyperlink{Etoposide}{PMID: 1449115}, E Ishii et al., 1992]

\hypertarget{pmid_12654074}{P}harmacokinetics of etoposide in Japanese children and adolescents has not been investigated. The objectives of the present study were (i) to document the pharmacokinetics of etoposide in Japanese children; (ii) to determine the intra- and interpatient variability in systemic etoposide exposure and (iii) to obtain insights into the age-pharmacokinetic parameter relationship. Pharmacokinetic studies of etoposide, given at doses of 60-200 mg/m2 by intravenous (i.v.) route of administration, were conducted in 18 children and adolescents (aged <19 years) with malignant diseases. High performance liquid chromatography was used to measure the blood etoposide levels. Pharmacokinetic parameters (mean\textasciitilde{}SD) of the 14 patients (24 courses) who received etoposide 100 mg/m2 were as follows: peak serum concentration (Cmax), 18.5\textasciitilde{}6.4 microg/mL; trough serum concentration, 0.2\textasciitilde{}0.1 microg/mL; biological half-life (T1/2), 3.6\textasciitilde{}0.7 h; volume of distribution (Vd), 6.3\textasciitilde{}3.4 L/m2; area under the etoposide serum concentration-time curve (AUC), 129\textasciitilde{}38 hr x microg/mL; systemic clearance, 21.1\textasciitilde{}10.8 mL/min per m2. The T1/2, Vd, and AUC were not associated with age. An increase in etoposide dose per body surface area (BSA) was associated with increase in its Cmax and area under the time-concentration curve (AUC). Wide interpatient variability in these parameters was demonstrated. The present study demonstrated that: (i) Pharmacokinetics of etoposide in Japanese children and adolescents were similar to those in Caucasians. (ii) Increased exposure to etoposide was associated with the Cmax. A clear correlation between Cmax and AUC was also found. (iii) Selecting the dose of etoposide according to body surface area (BSA) might give an acceptable range of exposure for children more than 1 year of age. [\hyperlink{Etoposide}{PMID: 12654074}, Yasuhiro Kato et al., 2003]

\hypertarget{pmid_23065812}{E}toposide (VP-16) is a hydrophobic anticancer agent inhibiting Topoisomerase II, commonly used in pediatric brain chemotherapeutic schemes as mildly toxic. Unfortunately, despite its appropriate solubilization in vehicle solvents, its poor bioavailability and limited passage of the blood-brain barrier concur to disappointing results requiring the development of new delivery system forms. In this study, etoposide formulated as a parenteral injectable solution (Teva®) was loaded into all-biocompatible poly(lactide-co-glycolide) (PLGA) or PLGA/P188-blended nanoparticles (size 110-130 nm) using a fully biocompatible nanoprecipitation technique. The presence of coprecipitated P188 on encapsulation efficacies and in vitro drug release was investigated. Drug encapsulation was determined using HPLC. Inflammatory response was checked by FACS analysis on human monocytes. Cytotoxic activity of the various simple (Teva®) or double (Teva®-loaded NPs) formulations was studied on the murine C6 and F98 cell lines. Obtained results suggest that, although noninflammatory neither nontoxic by themselves, the use of PLGA and PLGA/P188 nanoencapsulations over pre-existing etoposide formulation could induce a greatly improved cytotoxic activity. This approach demonstrated a promising perspective for parenteral delivery of VP16 and potential development of a therapeutic entity. [\hyperlink{Etoposide}{PMID: 23065812}, Maïté Callewaert et al., 2013]

\hypertarget{pmid_34679474}{T}here is a sparsity of data on the use of ethiodized poppy seed oil (EPO) contrast agent (Lipiodol) in patients. We investigated the safety of EPO in children, adolescents, and some adults for diagnostic and therapeutic interventions. All patients who underwent procedures with EPO between 1995 and 2014 were retrospectively included. Demographic characteristics, diagnosis, dose, route of administration, preparation of EPO in combination with other agents, and complications were recorded. In 1422 procedures, EPO was used for diagnostic or treatment purposes performed in 683 patients. The mean patient age was 13.4 years (range: 2 months-50 years); 58\% of patients were female. Venous malformations ( The use of an ethiodized poppy seed oil contrast agent in children, adolescents, and adults for diagnostic or therapeutic purposes is safe. [\hyperlink{Etoposide}{PMID: 34679474}, Robert K Clemens et al., 2021]

\hypertarget{pmid_16189442}{I}n this study the authors retrospectively evaluated the feasibility and effectiveness of prolonged oral etoposide therapy in children with recurrent ependymoma. Twelve ependymoma patients with documented recurrent or persistent disease were treated between May 1998 and October 2003. All patients were treated monthly with oral VP-16 administered at a dose of 50 mg/m2/d for 21 days, with a 7-day interval between cycles, for a planned minimum number of six cycles. Response (complete plus partial) after two cycles occurred in 5 of the 12 patients (41.6\%). Response plus stable disease occurred in 10 of the 12 (83.3\%), with a median duration of response or stable disease of 7 months (range 4-30). The median survival was 7 months; the 2-year progression-free survival was 16.7\%. These results emphasize that oral etoposide is an attractive option for childhood recurrent ependymomas in terms of administration, tolerability, and neuroradiologic response. [\hyperlink{Etoposide}{PMID: 16189442}, Alessandro Sandri et al., 2005]

\hypertarget{pmid_7031350}{E}toposide is a semisynthetic podophyllotoxin derivative with a broad spectrum of antitumor activity and a relatively high therapeutic index. The synergism in animal with cis-platinum, cyclophosphamide, BCNU, and cytosinarabinoside is interesting for combination regimen. Mechanisms of action are inhibition of nucleoside transfer and of DNA and RNA synthesis, single stranded breaks, inhibition of protein synthesis and of microtubular assembly. While in lower concentrations etoposide is acting cell-cycle-dependent with accumulation of cells in the G2-phase it has, in high concentrations, also a cellcycle-phase-unspecific lethal effect. Most suitable is the oral and i.v. application of etoposide in fractionated doses of 80--120 mg/m2 on 3--5 consecutive days and repetition after 21 [14--28] days. Side effects are dose-limiting bone marrow toxicity, nausea, vomiting, fever, hypotension, phlebitis, mucositis, neuropathy, cardiotoxicity, alopecia. Etoposide is one of the most active single agents in small-cell bronchus carcinoma with a remission rate of 37\% (10\% CR), and is very active in NHL (36\%), testicular carcinoma (37\%), AMML (35\%), choriocarcinoma (35\%), and neuroblastoma (29\%). The role of etoposide in combination with other active drugs in these tumors is currently investigated in bronchus and testicular carcinoma and NHL, where etoposide will belong to the drugs of the first choice in the future. [\hyperlink{Etoposide}{PMID: 7031350}, H J Schmoll et al., 1981]

\hypertarget{pmid_7628187}{T}he objectives of this study were to determine etoposide pharmacokinetics during continuous low-dose oral administration to children with solid tumors and to evaluate the relationships between parameters of etoposide systemic exposure and toxicity. In this phase I study, children were administered oral etoposide (25 to 75 mg/m2/day) for 21 days as a diluted solution of the intravenous preparation, divided into three equal daily doses. Plasma pharmacokinetics were studied on day 1 of therapy in 18 children and again on day 21 in 14 of these children. Etoposide plasma concentration-time data were fitted to a first-order absorption, two-compartment model with use of bayesian estimation. Pharmacokinetic parameter estimates from day 1 were used to estimate steady-state etoposide systemic exposure in all children. Stepwise multivariate regression was used in an exploratory manner to determine patient, laboratory, or pharmacokinetic predictors of toxicity. Although there was substantial intrapatient variability, there was no difference in the area under the concentration-time curve [AUC(0-8hr)] measured at day 21 compared with the steady-state AUC(0-8hr) estimated from day 1 pharmacokinetic parameters (p = 0.64). Degree of neutropenia was best predicted by the estimated duration that steady-state plasma etoposide concentrations were maintained above 1 microgram/ml (t > 1 microgram/ml) rather than peak plasma concentrations, AUC(0-8hr), dosage, or other patient characteristics. Assuming a bioavailability of the oral solution of approximately 50\%, the median etoposide systemic clearance was 21.4 ml/min/m2, a value similar to clearance estimates after intravenous etoposide in pediatric populations. We conclude that a parameter reflective of etoposide systemic exposure (t > 1 microgram/ml) correlates more strongly with neutropenia than does dosage or other patient characteristics. [\hyperlink{Etoposide}{PMID: 7628187}, D S Sonnichsen et al., 1995]

\hypertarget{pmid_8162893}{E}toposide (VP 16-213), the epipodophyllotoxin derivative that is widely used in the treatment of cancer, forms complexes with DNA-topoisomerase type II alpha to exert its cytotoxicity. The drug was evaluated in vivo in Swiss albino mouse bone marrow cells for its ability to induce clastogenicity and sister chromatid exchanges (SCEs). Doses of 5, 10, 15, and 20 mg/kg body weight etoposide given intraperitoneally induced a dose-dependent significant increase of clastogenicity (Trend test, alpha < or = 0.05). The aberrations induced were predominantly chromatid types. The drug shows specificity for S-phase cells: cells harvested 6 and 12 hr posttreatment showed a significantly increased number of damaged cells and aberrations per cell. Doses of 0.5, 1.0, 2.5, 5.0, and 10.0 mg etoposide/kg body weight induced a dose-dependent significant induction of SCEs (Trend test, alpha < or = 0.05). The minimal effective concentration was 0.5 mg/kg body weight. Etoposide significantly prolonged the cell cycle time at all concentrations tested: 12-13 hr in treated animals vs. 11 hr in control. The results confirm in vivo cell cycle phase specificity of the drug and further designate etoposide as a potent clastogen and a genotoxic agent in mice. [\hyperlink{Etoposide}{PMID: 8162893}, K Agarwal et al., 1994]

\hypertarget{pmid_6326063}{E}toposide (VP 16) is a semi-synthetic derivative of 4'- demethylepipodophyllotoxin , a naturally occurring compound synthesized by the North American May apple (Podophyllum peltatum ) and the Indian species Podophyllum emodi Wallich . Although podophyllotoxins are classical spindle poisons causing inhibition of mitosis by blocking mitrotubular assembly, etoposide inhibits cell cycle progression at a premitotic phase (late S and G2), probably via inhibition of DNA synthesis. There appears to be a selective antileukemic dose response relationship when compared to normal hematopoietic elements. Etoposide is effective when administered orally at about twice the recommended parenteral dosage. Schedule dependency in both animal models and clinical trials has been observed; multiple dosing over three to five consecutive days is superior to weekly single dose administration. Etoposide's dose-limiting toxicity is myelosuppression (leukopenia), which is quite predictable; alopecia and Gl toxicity (nausea, vomiting, stomatitis) occur in about 20-30\% of patients given recommended dosages. Etoposide appears to be one of the most active drugs for small cell lung cancer, testicular carcinoma (the Food and Drug Administration approved indication), ANLL and malignant lymphoma. Etoposide also has demonstrated activity in refractory pediatric neoplasms, hepatocellular, esophageal, gastric and prostatic carcinoma, ovarian cancer, chronic and acute leukemias and non-small cell lung cancer, although additional single and combination drug studies are needed to substantiate these data. Its contribution in front-line combination chemotherapeutic regimens for these cancers will be better defined in the forthcoming years. Etoposide appears to have minimal activity in breast cancer and, based on current data, it is inactive against malignant melanoma, colorectal adenocarcinoma and cancer of the head and neck, although the dosage and schedules used in many of the Phase II studies may have been suboptimal. [\hyperlink{Etoposide}{PMID: 6326063}, J A Sinkule et al., ]

\hypertarget{pmid_30885040}{H}ypersensitivity reactions to etoposide have been reported and patients have been safely transitioned to etoposide phosphate for continued therapy. However, the safety and efficacy of substituting etoposide phosphate for etoposide has not been well established in pediatric orthopedic malignancies. The aim of this study is to determine whether etoposide phosphate can be substituted for etoposide in pediatric orthopedic malignancies. A chart review of pediatric patients who developed hypersensitivity reactions to etoposide while being treated for orthopedic malignancies was performed at a large academic medical center. Three patients were identified, two with Ewing sarcoma and one with an osteosarcoma. All three patients experienced hypersensitivity reactions to their first doses of etoposide and were switched to etoposide phosphate for further therapy. After premedication, all three patients tolerated full doses of etoposide phosphate without a graded dose challenge or desensitization. Two of the patients were premedicated with diphenhydramine alone, while the third received diphenhydramine and dexamethasone. Etoposide phosphate is a potentially safe alternative for pediatric patients with orthopedic malignancies who experience etoposide hypersensitivity. However, caution is needed as there are cases of etoposide phosphate hypersensitivity. [\hyperlink{Etoposide}{PMID: 30885040}, Joel P Brooks et al., 2020]

\hypertarget{pmid_9142202}{P}re-clinical data and adult experience suggests that topoisomerase targeted anti-cancer agents may be highly schedule dependent, and efficacy may improve with prolonged exposure. To investigate this hypothesis, 28 children with recurrent brain and solid tumors were enrolled in a phase II study of oral etoposide (ETP). Patients were prescribed ETP at 50 mg/m2/ day for 21 consecutive days. Courses were repeated every 28 days pending bone marrow recovery. Evaluation of response was initially performed after 8 weeks and then every 12 weeks either by CT or MRI. Three of 4 patients with PNET (primitive neuroectodermal tumor)/medulloblastora achieved a partial response (PR). Two of 5 with ependymoma responded, one with a complete response and one with a PR. Toxicity was manageable with only 1 admission for fever and neutropenia in 120 cycles of therapy. Five patients had grade 3 or 4 neutropenia. One had grade 4 thrombocytopenia and one grade 2 mucositis and withdrew as a result. One patient had grade 2 diarrhea. Two patients who achieved a PR had received ETP as part of prior combination chemotherapy regimens. Daily oral etoposide is active in recurrent PNET/medulloblastoma and ependymoma. Toxicity is manageable and rarely requires intervention. Daily oral etoposide in combination with crosslinking agents should be considered in future phase III trials. Determination of activity in glioma and solid tumors is not complete. [\hyperlink{Etoposide}{PMID: 9142202}, M N Needle et al., 1997]

\hypertarget{pmid_7551961}{E}toposide phosphate, a water soluble prodrug of etoposide, has several potential advantages including easier and more rapid administration, avoidance of large fluid loads, and elimination of hypersensitivity reactions and other problems related to the solubilizer. This randomized Phase II study was done to evaluate the efficacy and toxicity of etoposide phosphate and etoposide, when used in combination with cisplatin in the treatment of patients with small cell lung cancer. Previously untreated small cell lung cancer patients were randomized to receive cisplatin in combination with molar equivalent does of either etoposide or etoposide phosphate. The patients were evaluated with respect to response rate, time to progression, survival, and toxicity. Response rates with etoposide phosphate and etoposide were 61\% (95\% confidence interval 55-67\%) and 58\% (95\% confidence interval 52-64\%), respectively (P = 0.85). Median time to progression was 6.9 months for patients who received etoposide phosphate and 7.0 months for those with etoposide (P = 0.50). For extensive stage disease patients, median survival with etoposide phosphate was 9.5 months versus 10 months for etoposide (P = 0.93). The corresponding median survivals for patients with limited stage disease were > 16 months and 17 months, respectively (P = 0.62). Myelosuppression was the most common toxicity; Grade 3 and 4 leukopenia occurred in 63\% of patients receiving etoposide phosphate compared with 77\% receiving etoposide (P = 0.16). The combination of etoposide phosphate and cisplatin is effective in the treatment of small cell lung cancer, and can be administered with acceptable toxicity. This study was not designed to be a formal Phase III comparative trial, but the efficacy and toxicity observed with this regimen were found to be similar to a standard etoposide/cisplatin regimen, using molar equivalent etoposide doses. Etoposide phosphate is preferable to etoposide because it is easier to use. [\hyperlink{Etoposide}{PMID: 7551961}, F A Greco et al., 1995]

\hypertarget{pmid_8523060}{M}ost pediatric treatment protocols specify dose calculations for cytostatic drugs based on body-surface area (BSA). However, for children less than 1 year of age, calculation guidelines vary. Normally, reduced dosages are recommended with calculations based on body weight (BW). However, the optimal dose for infants should take age-dependent and drug-specific pharmacokinetic parameters into account. The current investigation focused on the effects of different dose-reduction rules on the steady-state levels (Css) of etoposide and related bone marrow toxicity. One hundred seventy three treatment courses in 78 children on a 96-hour continuous infusion schedule were monitored for Css (determined by high-performance liquid chromatography [HPLC]), and 100 courses were documented in detail with regard to dose calculation (125 mg/m2, 4.17/kg, or 2/3 x 4.17/kg) and toxicity. Dose calculation on the basis of BSA led to Css of 4.9 +/- 1.2 micrograms/mL, which on the basis of BW was 3.5 +/- 1.1 micrograms/mL and 1.95 +/- 0.6 micrograms/mL (2/3.kg-dose), respectively. However, conversion of the latter levels to those expected with the regular square-meter-dose, resulted in values of 4.7 +/- 1.4 micrograms/mL and 4.2 +/- 1.2 micrograms/mL (/125 mg/m2). Lower etoposide Css levels resulted in less pronounced thrombocyte and WBC nadirs in the respective groups. The dose-calculation rules for infants, therefore, decreased both dose-intensity and related toxicity. Etoposide clearance rates related to BW (0.8 +/- 0.3 mL/min/kg) or BW (19 +/- 6 mL/min/m2) did not show any differences between children and infants, even in the age range of 3 to 12 months. In the case of etoposide, special dose-calculation guidelines for infants are not substantiated by age-dependent pharmacokinetics or tolerance. [\hyperlink{Etoposide}{PMID: 8523060}, J Boos et al., 1995]

\section*{Fenoprofen Calcium}
\subsection*{Result}
\subsubsection*{Answer}

Unknown

\subsubsection*{{Explanation}}
\hypertarget{Fenoprofen Calcium}
A review of the available abstracts reveals only two that specifically mention Fenoprofen Calcium:

1. One study compared the effects of plain and enteric-coated fenoprofen calcium on gastrointestinal microbleeding in 32 normal male volunteers. This study was conducted in adults, not children, and does not provide any data on pediatric safety or efficacy [\hyperlink{pmid_3496186}{PMID: 3496186}, J R Ryan et al., 1987].

2. Another study compared fenoprofen calcium with acetylsalicylic acid in the treatment of 27 patients with definite or classic rheumatoid arthritis. However, the abstract does not specify the age range of the patients, and based on the context and typical study populations for rheumatoid arthritis, it is likely that these were adults. There is no explicit mention of children or pediatric patients in the abstract [\hyperlink{pmid_781232}{PMID: 781232}, J F Fries et al., 1976].

No abstracts were found that report targeted studies of fenoprofen calcium in children, nor any that affirm its safety or demonstrate it is unsafe in pediatric populations. Therefore, based on the abstracts available, the safety of fenoprofen calcium in children is unknown for all pediatric age ranges.

\subsection*{Abstracts}
\hypertarget{pmid_28169973}{I}n young children, the use of fecal calprotectin (fCP) as a biomarker is limited because reference values have not been widely accepted up to now. Moreover, reference values for fecal eosinophil-derived neurotoxin (fEDN) in children have not been established. The aim of the present study was to investigate fCP and fEDN levels in young healthy children to establish reference values. Stool samples were obtained from healthy children ages 0 to 12 years. fCP and fEDN levels were analyzed using the EliA Calprotectin 2 assay (Phadia AB) and a novel research assay (on the ImmunoCAP platform), respectively. In the 174 included children (87 boys), 95th Percentile values ranged from 1519 mg/kg at 0 months to 54.4 mg/kg at 144 months for fCP and from 9.9 mg/kg at 0 months to 0.2 mg/kg at 144 months for fEDN. There was a statistically significant association between age and fCP concentrations (P < 0.001) and age and fEDN concentrations (P < 0.001). We also found a statistically significant association between fEDN and fCP concentrations (rho = 0.52, P < 0.001). According to our results, we provide a nomogram and we suggest 3 different age groups for evaluation of fCP and fEDN concentrations, the 95th percentile being respectively 910.3 and 7.4 mg/kg for 0-12 months, 285.9 and 2.9 mg/kg for >1 to 4 years, and 54.4 and 0.2 mg/kg for >4 to 12 years. By using an improved analytical method, we have confirmed that young healthy children have higher fCP concentrations than healthy adults. We, for the first time, report reference values for fEDN concentrations in a pediatric population. The proposed nomograms and reference values for fCP and fEDN are aimed at facilitating the applicability of biomarkers for both neutrophil- and eosinophil-mediated intestinal inflammation in children in clinical practice. [\hyperlink{Fenoprofen Calcium}{PMID: 28169973}, María Roca et al., 2017]

\hypertarget{pmid_2507975}{T}he safety and clinical efficacy of calcium carbonate therapy in children with chronic renal failure were assessed in 68 patients (average age 8.38 years) during a mean follow-up period of 19.9 months (range 1.2-49.4). Forty-seven episodes of hypercalcaemia occurred in 29 children (3.5 episodes per 100 patient-months). There were no significant differences in mean GFR or biochemical parameters between these patients at the start of calcium carbonate therapy and the group of children who never experienced hypercalcaemia. Good control of secondary hyperparathyroidism and a significant reduction in serum aluminum were achieved. Two of 23 hypercalcaemic patients showed nephrocalcinosis on ultrasonography. 99Tc pyrophosphate scanning failed to detect any other ectopic calcification. The incidence of hypercalcaemia increased significantly when the GFR was less than 15 ml/min per 1.73 m2 and was most frequent in children receiving dialysis (48 episodes per 100 patient-months). The decrease in GFR during therapy was significantly more in the hypercalcaemic group compared to the normocalcaemic group (P less than 0.01), despite no irreversible acute effects of hypercalcaemia being observed on the rate of decline of GFR. We believe that the reduced renal homeostatic reserve is a major factor predisposing to hypercalcaemia. Consequently calcium carbonate is safe to use in children with severe chronic renal failure with close biochemical monitoring; the benefits over aluminium phosphate binders far outweigh the risks of hypercalcaemia and ectopic calcification. [\hyperlink{Fenoprofen Calcium}{PMID: 2507975}, A G Clark et al., 1989]

\hypertarget{pmid_31198713}{C}efotaxime is one of the third generation cephalosporins, which is used against many infections. This drug has a urinary excretion and potentially may have nephrotoxic effects. Hypercalciuria can cause important complications, including the formation of kidney stones. In the recent study, we decided to evaluate hypercalciuria in children receiving cefotaxime. This case-control study was conducted in Amirkabir hospital (Arak, Iran), where 30 children received intravenous cefotaxime were placed in the case group and 30 children without intravenous administration of cefotaxime were included in the control group. The ratio of calcium to creatinine was measured in both groups. Data were analyzed by SPSS software version 23. This study showed that the ratios of male and female children in both the groups were 19 (63.3\%) and 11 (36.7\%) respectively, the mean age of children in the case group was 2.36 years with a standard deviation of 0.71 and the mean age of the children in the control group was 5.18 years with a standard deviation of 3.31. The ratios of urine calcium to creatinine in the case and control groups were 0.90 with a standard deviation of 1.79 and 0.37 with a standard deviation of 0.44 ( According to the above results, it is concluded that receiving intravenous cefotaxime may increase calcium to creatinine ratio in children. [\hyperlink{Fenoprofen Calcium}{PMID: 31198713}, Zahra Kalantari et al., 2019]

\hypertarget{pmid_25135766}{R}ecently, an association between childhood growth stunting and aflatoxin (AF) exposure has been identified. In Ghana, homemade nutritional supplements often consist of AF-prone commodities. In this study, children were enrolled in a clinical intervention trial to determine the safety and efficacy of Uniform Particle Size NovaSil (UPSN), a refined calcium montmorillonite known to be safe in adults. Participants ingested 0.75 or 1.5 g UPSN or 1.5 g calcium carbonate placebo per day for 14 days. Hematological and serum biochemistry parameters in the UPSN groups were not significantly different from the placebo-controlled group. Importantly, there were no adverse events attributable to UPSN treatment. A significant reduction in urinary metabolite (AFM1) was observed in the high-dose group compared with placebo. Results indicate that UPSN is safe for children at doses up to 1.5 g/day for a period of 2 weeks and can reduce exposure to AFs, resulting in increased quality and efficacy of contaminated foods.  [\hyperlink{Fenoprofen Calcium}{PMID: 25135766}, Nicole J Mitchell et al., 2014] Xenon has minimal haemodynamic side effects when compared to volatile or intravenous anaesthetics. Moreover, in in vitro and in animal experiments, xenon has been demonstrated to convey cardio- and neuroprotective effects. Neuroprotection could be advantageous in paediatric anaesthesia as there is growing concern, based on both laboratory studies and retrospective human clinical studies, that anaesthetics may trigger an injury in the developing brain, resulting in long-lasting neurodevelopmental consequences. Furthermore, xenon-mediated neuroprotection could help to prevent emergence delirium/agitation. Altogether, the beneficial haemodynamic profile combined with its putative organ-protective properties could render xenon an attractive option for anaesthesia of children undergoing cardiac catheterization. In a phase-II, mono-centre, prospective, single-blind, randomised, controlled study, we will test the hypothesis that the administration of 50\% xenon as an adjuvant to general anaesthesia with sevoflurane in children undergoing elective cardiac catheterization is safe and feasible. Secondary aims include the evaluation of haemodynamic parameters during and after the procedure, emergence characteristics, and the analysis of peri-operative neuro-cognitive function. A total of 40 children ages 4 to 12 years will be recruited and randomised into two study groups, receiving either a combination of sevoflurane and xenon or sevoflurane alone. Children undergoing diagnostic or interventional cardiac catheterization are a vulnerable patient population, one particularly at risk for intra-procedural haemodynamic instability. Xenon provides remarkable haemodynamic stability and potentially has cardio- and neuroprotective properties. Unfortunately, evidence is scarce on the use of xenon in the paediatric population. Our pilot study will therefore deliver important data required for prospective future clinical trials. EudraCT: 2014-002510-23 (5 September 2014). [\hyperlink{Fenoprofen Calcium}{PMID: 25135766}, Sarah Devroe et al., 2015]

\hypertarget{pmid_17611334}{T}o observe the effect of sevoflurane on the induction and maintenance of anaesthesia in children, and to evaluate its safety and effectiveness. Forty child patients who conformed to the selection standard were operated under anaesthesia with intubation.Without premedicant, all the patients inhaled 100\% oxygen(1L/min) and sevoflurane by mask, and escalated the concentration of sevoflurane (to the maximum concentration 7\%) until the lash reflex disappeared, and the maintenance concentration was controlled under 4\%. All the patients were intubated, together with vecuronium 0.1mg/kg. With little tract excretion, the achievement ratio of induction by sevoflurane was 100\%, and the children tolerated well. With stable hemodynajmics,1\% approximately 4.0\% maintenance concentration of sevoflurane during the operation showed effective anaesthesia, no decreased heart rate or blood pressure appeared, and all the patients' body temperature was normal. Sevoflurane for children induction can bring fewer stimuli in the respiratory tract,less cardiac vascular inhibition and palinesthesia time. Anaesthesia in children induced by sevoflurane is safe and effective. [\hyperlink{Fenoprofen Calcium}{PMID: 17611334}, Xi-ying Zhang et al., 2007]

\hypertarget{pmid_28469850}{S}ubcutaneous fat necrosis (SFN) in infants producing severe hypercalcemia is a life-threatening emergency. Pathophysiology may include enhanced gastrointestinal calcium absorption and bone resorption. We treated an infant with SFN and serum calcium of 15 mg/dL with prednisolone and low-dose zoledronic acid. Serum calcium promptly normalized without rebound hypocalcemia, and redosing of zoledronic acid was not necessary. [\hyperlink{Fenoprofen Calcium}{PMID: 28469850}, Jeremy A Di Bari et al., 2017]

\hypertarget{pmid_36896687}{F}ecal calprotectin (FCP) is a biomarker of intestinal inflammation and has recently been proposed as a diagnostic biomarker of food allergy (FA) in children. The aim of this study was to compare FCP level in infants and children under 4 years old with 1) atopic dermatitis (AD) with food allergy (FA) and 2) children with AD and without FA with the results in healthy controls. In total, 46 infants and children (mean age 14 months ± 12) diagnosed with AD were divided into two groups: G1, children with atopic AD with FA (n=28) and G2, children with AD without FA (n=18). The control group (G3) was made up of healthy children of the same age (n=18). The median FCP was significantly higher in G1 compared with G2 (G1: median 154, IQR 416 µg/g vs G2: median 41.3, IQR 59 µg/g; P=0.0096). The median FCP in children with AD and FA was significantly higher before elimination diet in comparison with FCP after 3 months of elimination diet (median 154, IQR 416 µg/g vs median 35, IQR 23 µg/g; P=0.0039). The level of FCP was significantly positively correlated with the SCORAD score (r=0.5544, P=0.0022). Our study showed a significant difference in level of FCP in patients with AD without FA compared with patients with AD and FA. We also found a positive correlation of FCP with SCORAD score, a biomarker of AD severity. New studies are needed to investigate the role of FCP as a biomarker of FA in children with AD. [\hyperlink{Fenoprofen Calcium}{PMID: 36896687}, Alen Švigir et al., 2021]

\hypertarget{pmid_26468483}{N}ephrolithiasis is a common worldwide problem both in children and adults. Ceftriaxone as a widely used antibiotic can contribute to the formation of renal stones and hypercalciuria. To find the effect of ceftriaxone, a widely used antibiotic, on urinary calcium excretion rate in children. 84 infants and children over 3 months admitted to hospital for non-renal problems. They were all previously healthy children affected with a condition mandating hospitalisation. They were randomly divided into 2 groups; those who received ceftriaxone according to their physician decision as the case group and those who did not receive antibiotics as the control group. The patients urinary calcium excretion was determined as calcium to creatinine ratio in a random urine sample in the first and third day of their admission. All data was expressed by mean ± SD and analysed by t independent and chi-square tests by SPSS 16. P P value less than 0.05 was significant. Eighty-four cases were analysed. Calcium excretion in received and non-received ceftriaxone groups was 0.13 ± 0.06 and 0.14 ± 0.02 respectively at first day of admission ( P = 0.1). After 3 days, the urine calcium to creatinine ratio increased to 0.27 ± 0.2 and 0.26 ± 0.08 in received and non- received ceftriaxone groups ( P = 0.8). In children, urinary calcium excretion increases 2 times in average in a short time after admission because of gastroenteritis, and ceftriaxone is not different to other antibiotics for increase urinary calcium excretion in 3 days after admission. [\hyperlink{Fenoprofen Calcium}{PMID: 26468483}, Anoush Azarfar et al., 2015]

\hypertarget{pmid_37522100}{C}alcium carbonate (E 170) was re-evaluated in 2011 by the former EFSA Panel on Food Additives and Nutrient sources added to Food (ANS). As a follow-up to this assessment, the Panel on Food Additives and Flavourings (FAF) was requested to assess the safety of calcium carbonate (E 170) for its uses as a food additive in food for infants below 16 weeks of age belonging to food category 13.1.5.1 (Dietary foods for infants for special medical purposes and special formulae for infants) and as carry over in line with Annex III, Part 5 Section B to Regulation (EC) No 1333/2008. In addition, the FAF Panel was requested to address the issues already identified during the re-evaluation of the food additive when used in food for the general population. The process involved the publication of a call for data to allow the interested business operators (IBOs) to provide the requested information to complete the risk assessment. The Panel concluded that there is no need for a numerical acceptable daily intake (ADI) for calcium carbonate and that, in principle, there are no safety concern with respect to the exposure to calcium carbonate  [\hyperlink{Fenoprofen Calcium}{PMID: 37522100},  et al., 2023] Calprotectin is a protein abundant in neutrophils. Fecal calprotectin can be used as a marker of gastrointestinal inflammation, and an improved assay has recently been developed. The aim of this study was to establish reference values for fecal calprotectin in healthy children aged between 4 and 17 years. Fecal samples were obtained from 117 healthy children classified into four age groups: 4 to 6 years, 7 to 10 years, 11 to 14 years, and 15 to 17 years. A health questionnaire was used to ensure that these children fulfilled the inclusion criterion and did not have intercurrent disease, nasal or menstrual bleeding, or nonsteroidal anti-inflammatory drug medication before the sampling period. Calprotectin was analyzed using a quantitative enzyme-linked immunosorbent assay (Calprest, Eurospital SpA, Trieste, Italy). Children with fecal calprotectin values >50 microg/g were asked to deliver an additional sample. The overall median fecal calprotectin concentration was 13.6 microg/g (95\% confidence interval, 9.9-19.5 microg/g) in the 117 children. In the different age groups, 4 to 6 years, 7 to 10 years, 11 to 14 years, and 15 to 17 years, the median calprotectin concentrations were 28.2, 13.5, 9.9, and 14.6 microg/g, respectively. Of these children, 104 (89\%) had a concentration <50 microg/g. The remaining 13 children with a calprotectin concentration >50 microg/g delivered one additional fecal sample. All showed a lower concentration in the second sample except for one teenager who later proved to have proctitis. The suggested cutoff level for adults (<50 microg/g) can be used for children aged from 4 to 17 years regardless of sex. A fecal calprotectin concentration >50 microg/g warrants follow-up. [\hyperlink{Fenoprofen Calcium}{PMID: 37522100}, Ulrika Lorentzon Fagerberg et al., 2003]

\hypertarget{pmid_3701525}{O}rally administered calcium carbonate was evaluated as a phosphate binding agent in 15 children, ages 0.6 to 17.2 years, receiving maintenance dialysis. Changes in plasma aluminum concentration were assessed after discontinuation of treatment with aluminum-containing gels. The mean daily dose of calcium carbonate was 5.1 +/- 2.5 gm (384 +/- 315 mg/kg/day), and correlated inversely with body weight (r = 0.72, P less than 0.01) and age (r = 0.71, P less than 0.01). Mean serum calcium, phosphorus, and bicarbonate values were unchanged throughout the study. Plasma aluminum concentration fell from 90 +/- 51 to 34 +/- 22 micrograms/L (P less than 0.005). Dietary phosphorus intakes were 44 +/- 21 and 42 +/- 19 mg/kg/day during the control period and at the end of the study, respectively. Transitory hypercalcemia was the only side effect in 92\% of the patients. In none of the patients did uncontrolled hyperphosphatemia, metabolic alkalosis, diarrhea, or symptoms or signs of hypercalcemia develop. Our data indicate that calcium carbonate is an effective phosphate binding agent in children receiving dialysis, and should be used in lieu of aluminum-containing gels in young children with renal failure. [\hyperlink{Fenoprofen Calcium}{PMID: 3701525}, I B Salusky et al., 1986]

\hypertarget{pmid_23534952}{A} test dose is used to detect intravascular injection during neuraxial block in pediatrics. Accidental intravascular epidural local anesthetic injection might be unrecognized in anesthetized children leading to potential life-threatening complications. In children, sevoflurane anesthesia blunts the hemodynamic response when intravascular cathecolamines are administered. No studies have explored the hemodynamics and the criteria for a positive test dose result following phenylephrine in sevoflurane anesthetized children. Healthy children undergoing minor procedures were randomly assigned to receive intravenous placebo, or 5 μg∙kg(-1) phenylephrine (n = 11/group) during sevoflurane anesthesia. Hemodynamic response was assessed using electrocardiography, pulse oxymetry and non-invasive blood pressure monitoring for 5 min following drug administration in anesthetized patients. All patients receiving phenylephrine showed a decreased heart rate (HR) but not all of them met the positive criterion for test dose response. Overall, at 1 min, patients receiving phenylephrine showed a 25\% decrease in HR from the baseline while an increase in blood pressure was noticed in 54\% of patients receiving phenylephrine. Phenylephrine might be a future indicator of positive intravascular test dose. Further investigation is needed to find out the phenylephrine dose that elicits a reliable hemodynamic response and whether phenylephrine needs to be dose age-adjusted in order to appreciate relevant hemodynamic changes in children receiving neuraxial blocks undergoing general anesthesia. [\hyperlink{Fenoprofen Calcium}{PMID: 23534952}, Carlo Pancaro et al., 2013]

\hypertarget{pmid_318975}{T}he effect of calcium salt of fosfomycin in the treatment of 43 neonates suffering from acute gastroenterocolitis produced by enteropathogenic E. coli is evaluated. The minimal inhibitory concentration of these E. coli was, generally, lower than 128 mug/ml. Dosages of 150-200 mg/kg body weight/day were administered orally every 8 h. This treatment lasted for 4 days only. Clinical evolution was favorable in 38 (88\%) babies and bacteriological evolution in 30 (70\%). In eight cases a different flora to the initial was selected during the treatment with fosfomycin. None of the cases treated showed any toxic alteration attributed to the antibiotic. [\hyperlink{Fenoprofen Calcium}{PMID: 318975}, C G Taylor et al., 1977]

\hypertarget{pmid_1865281}{U}sing a stable isotopic technique in which 42Ca was administered via a bolus injection, we measured endogenous fecal calcium excretion, Vf, in five healthy children, aged 3-14 years. The Vf averaged 1.4 +/- 0.4 mg/kg/day, and was lower than urinary Ca excretion (Vu) in four of the five children. These results for Vf are consistent with previously reported results for Vf in healthy adults and much lower than those reported in premature infants. These results may be useful in understanding developmental changes in Ca metabolism and in interpreting dual tracer Ca isotope studies in children. [\hyperlink{Fenoprofen Calcium}{PMID: 1865281}, S A Abrams et al., 1991]

\hypertarget{pmid_33239728}{R}eference values of fecal calprotectin (fCP) have not been convincingly established in children. We aimed to investigate fCP concentrations in a larger population of healthy children aged 4-16 years to analyze more in depth the behavior of fCP in this age range and to determine if cut-off levels could be conclusively recommended. A prospective study was conducted to investigate fCP concentrations of healthy children aged 4-16 years. In 212 healthy children, the median and 95th percentile for fCP were 18.8 mg/kg and 104.5 mg/kg, respectively. We found a statistically significant association between the 95th percentile of fCP concentrations and age (p < 0.001). We propose a nomogram to facilitate the interpretation of fCP results in children aged 4-16 years. Further studies are required to validate the proposed values in clinical practice. [\hyperlink{Fenoprofen Calcium}{PMID: 33239728}, María Roca et al., 2020]

\hypertarget{pmid_10847237}{T}he safety of use of the calcium channel blocker nifedipine in pregnancy as it affects child development has not been well evaluated. We report the results, with regard to the safety for children of use of nifedipine in pregnancy, on children followed up at 18 months of age born from women recruited in a study comparing routine treatment with nifedipine compared with no treatment. [\hyperlink{Fenoprofen Calcium}{PMID: 10847237}, R Bortolus et al., 2000]

\hypertarget{pmid_10357743}{A}lthough additional dietary calcium is recommended frequently to reduce the risk of lead poisoning, its role in preventing lead absorption has not been evaluated clinically. The objective was to determine the safety and to estimate the size of the effect of calcium- and phosphorus-supplemented infant formula in preventing lead absorption. One hundred three infants aged 3.5-6 mo were randomly assigned to receive iron-fortified infant formula (465 mg Ca and 317 mg P/L) or the same formula with added calcium glycerophosphate (1800 mg Ca and 1390 mg P/L) for 9 mo. There was no significant difference between groups in the mean ratio of urinary calcium to creatinine, serum calcium and phosphorus, or change in iron status (serum ferritin, total iron binding capacity). At month 4, the median (+/-SD) increase from baseline in blood lead concentration for the supplemented group was 57\% of the increase for the control group (0.04 +/- 0.09 compared with 0.07 +/- 0.10 micromol/L; P = 0.039). This effect was attenuated during the latter half of the trial, with an overall median increase in blood lead concentration from baseline to month 9 of 0.12 +/- 0.13 micromol/L for the control group and 0.10 +/- 0.18 micromol/L for the supplemented group (P = 0.284). Supplementation did not have a measurable effect on urinary calcium excretion, calcium homeostasis, or iron status. The significant effect on blood lead concentrations during the first 4 mo was in the direction expected; however, because this was not sustained throughout the 9-mo period we cannot conclude that the calcium glycerophosphate supplement prevented lead absorption in this population. [\hyperlink{Fenoprofen Calcium}{PMID: 10357743}, J D Sargent et al., 1999]

\hypertarget{pmid_2127076}{T}he prevention of osteopenia and frequency of renal and intestinal side effects of mineral supplementation was studied in 24 preterm infants with birth weight under 1,500 g, prospectively (gestational age 26-34 weeks). Calcium intake varied from 2.5 vs. 3.75 vs. 5 mmol/kg/day, phosphate was offered in dose of 2.5 mmol/kg/day. At the expected birth date 40\% of infants with low calcium dose showed an activity of serum alkaline phosphatase greater than five times the maximum adult normal value which is defined as a reliable marker; for osteopenia no infant with medium or high calcium intake reached this critical value (p = 0.03). Medium and high calcium doses resulted in an increased risk for hypercalcuria (25 vs. 50\%) (p = 0.03). Half of these infants developed typical signs of nephrocalcinosis on ultrasound examination. No significant difference of fecal fat content was observed with increased calcium intake; but more episodes of abdominal distension occurred during the first days of high calcium supplementation (p = 0.03). We conclude, that a calcium intake of 3.75 mmol/kg/day in combination with phosphate 2.5 mmol/kg/day is sufficient for adequate bone mineralization on a low level of side effects. Calcium excretion in urine has to be observed for early diagnosis of nephrocalcinosis. [\hyperlink{Fenoprofen Calcium}{PMID: 2127076}, J Kreuder et al., 1990]

\hypertarget{pmid_3496186}{T}he effects of plain and enteric-coated fenoprofen calcium (Nalfon, Dista, Indianapolis, Ind.) on gastrointestinal microbleeding were studied in 32 normal male volunteers in a randomized, open-label, parallel trial at two inpatient research facilities. A 1-week placebo (baseline) period preceded 2 weeks of fenoprofen therapy (enteric coated or plain, 600 mg q.i.d.). Fecal blood loss was measured by 51Cr-tagged erythrocyte assay and averaged over days 4 to 7 (baseline) and 11 to 14 and 18 to 21 (active therapy). At one center gastrointestinal irritation was evaluated endoscopically before and after active therapy. Endoscopy showed both formulations to cause mucosal damage not evident by subject-reported symptoms. Four of the 16 subjects developed asymptomatic duodenal ulcers. Mean daily fecal blood loss was significantly lower (P = 0.03) with enteric-coated (mean +/- SD, 1.104 +/- 0.961 ml/day) than with plain fenoprofen calcium (mean +/- SD, 1.686 +/- 0.858 ml/day), suggesting that tolerance of fenoprofen can be improved with administration in an enteric-coated form. [\hyperlink{Fenoprofen Calcium}{PMID: 3496186}, J R Ryan et al., 1987]

\hypertarget{pmid_23668874}{H}ypocalcemia is a common, treatable cause of neonatal seizures. A term girl neonate with no apparent risk factors developed seizures on day 5 of life, consisting of rhythmic twitching of all extremities in a migrating pattern. Physical examination was normal except for jitteriness. Laboratory evaluation was unremarkable except for decreased total and ionized serum calcium levels and an elevated serum phosphorus level. The mother had ingested 3-6 g of calcium carbonate daily during the final 4 months of pregnancy to control morning sickness. The baby's electroencephalogram showed multifocal interictal sharp waves and intermittent electrographic seizures consisting of focal spikes in the left hemisphere accompanied by rhythmic jerking of the right arm and leg. Treatment with intravenous calcium gluconate over several days resulted in cessation of seizures and normalization of serum calcium. The child has remained seizure free and is normal developmentally at 9 years of age. Hypocalcemic seizures in this newborn were likely secondary to excessive maternal calcium ingestion, which led to transient neonatal hypoparathyroidism and hypocalcemia. Inquiry about perinatal maternal medication use should include a search for over-the-counter agents that might not be thought of as "drugs," as in this case, antacids. [\hyperlink{Fenoprofen Calcium}{PMID: 23668874}, Jenna F Borkenhagen et al., 2013]

\hypertarget{pmid_781232}{F}enoprofen calcium was compared with acetylsalicylic acid in the treatment of 27 patients with definite or classic rheumatoid arthritis, over a period of one year. Both drugs appeared efficacious, with a slight edge to fenoprofen in the doses employed. Fewer side effects were noted with fenoprofen. Effectiveness continued undiminished throughout the year, and mean values of most parameters continued to improve in both groups over the entire period. Three problems which influence extrapolation of results from this and similar studies to the general setting are discussed. (1) Individual patients show great variation from the mean and from one observation point to another. Thus, expectations developed from mean values will seldom be accurate in a particular patient. (2) The relative doses chosen for two drugs in the clinical trial can profoundly influence both efficacy and toxicity. The qualification "at the doses employed" is seldom emphasized in clinical reports. (3) Patient compliance in the general clinical setting is importantly different from that in a clinical trial, and this potential problem must be assessed by the physician choosing an appropriate medication for a particular patient. [\hyperlink{Fenoprofen Calcium}{PMID: 781232}, J F Fries et al., 1976]

\hypertarget{pmid_33106892}{P}ediatric patients with advanced chronic kidney disease (CKD) are often prescribed oral phosphate binders (PBs) for the management of hyperphosphatemia. However, available PBs have limitations, including unfavorable tolerability and safety. This phase 3, multicenter, randomized, open-label study investigated safety and efficacy of sucroferric oxyhydroxide (SFOH) in pediatric and adolescent subjects with CKD and hyperphosphatemia. Subjects were randomized to SFOH or calcium acetate (CaAc) for a 10-week dose titration (stage 1), followed by a 24-week safety extension (stage 2). Primary efficacy endpoint was change in serum phosphorus from baseline to the end of stage 1 in the SFOH group. Safety endpoints included treatment-emergent adverse events (TEAEs). Eighty-five subjects (2-18 years) were randomized and treated (SFOH, n = 66; CaAc, n = 19). Serum phosphorus reduction from baseline to the end of stage 1 in the overall SFOH group (least squares [LS] mean ± standard error [SE]) was - 0.488 ± 0.186 mg/dL; p = 0.011 (post hoc analysis). Significant reductions in serum phosphorus were observed in subjects aged ≥ 12 to ≤ 18 years (LS mean ± SE - 0.460 ± 0.195 mg/dL; p = 0.024) and subjects with serum phosphorus above age-related normal ranges at baseline (LS mean ± SE - 0.942 ± 0.246 mg/dL; p = 0.005). Similar proportions of subjects reported ≥ 1 TEAE in the SFOH (75.8\%) and CaAc (73.7\%) groups. Withdrawal due to TEAEs was more common with CaAc (31.6\%) than with SFOH (18.2\%). SFOH effectively managed serum phosphorus in pediatric patients with a low pill burden and a safety profile consistent with that reported in adult patients. [\hyperlink{Fenoprofen Calcium}{PMID: 33106892}, Larry A Greenbaum et al., 2021]

\hypertarget{pmid_21172879}{T}o evaluate the efficacy of low-dose chemotherapy in infants with nonmetastatic and unresectable neuroblastoma (NB) without MYCN amplification. Infants with localized NB and no MYCN amplification were eligible in the SIOPEN Infant Neuroblastoma European Study 99.1 study. Primary tumor was deemed unresectable according to imaging defined risk factors. Diagnostic procedures and staging were carried out according to International Staging System recommendations. Children without threatening symptoms received low-dose cyclophosphamide (5 mg/kg/d × 5 days) and vincristine (0.05 mg/kg at day 1; CyV), repeated once to three times every 2 weeks until surgical excision could be safely performed. Children with either one threatening symptom or insufficient response to CyV were given carboplatin and etoposide (CaE), sometimes followed by vincristine, cyclophosphamide, and doxorubicin. No postoperative treatment was to be administered. Between December 1999 and April 2004, 120 infants were included in the study. Eighty-eight had no threatening symptoms and 79 received CyV. CaE was given to 49 of them because of insufficient response. Thirty-two children had threatening symptoms, 30 of whom received CaE. Anthracyclines were given to 46 children. Surgery was attempted in 102 patients, leading to gross surgical excision in 93. Relapse occurred in 12 patients (nine local and three metastatic). Five-year overall and event-free survivals were 99\% ± 1\% and 90\% ± 3\%, respectively, with a median follow-up of 6.1 years (range, 1.6 to 9.1). Low-dose chemotherapy without anthracyclines is effective in 62\% of infants with an unresectable NB and no MYCN amplification, allowing excellent survival rates without jeopardizing their long-term outcome. [\hyperlink{Fenoprofen Calcium}{PMID: 21172879}, Hervé Rubie et al., 2011]

\hypertarget{pmid_36989008}{P}ediatric septic shock is a life-threatening condition with significant rates of morbidity and mortality. Standard management includes fluid resuscitation, timely antimicrobial administration, and epinephrine or norepinephrine if unresolved with initial management. Additional therapies are not well defined and include vasopressin, hydrocortisone, phenylephrine, levosimendan, dopamine, and others. Many of these agents modify cellular effects of calcium in the smooth muscle. The use of a calcium infusion may improve vasoactivity in the smooth muscle without the use of signaling pathways. Children are more susceptible to the effects of calcium, which may predispose them to enhanced vasoconstriction with the administration of intravenous calcium. We present a case in which a patient on chronic calcium channel blocker therapy presented with septic shock. She continued to remain hypotensive after fluid resuscitation, antibiotics, epinephrine, and norepinephrine. Her blood pressure improved with the initiation of a continuous calcium chloride infusion. Norepinephrine and epinephrine doses were decreased after the initiation of the calcium infusion. [\hyperlink{Fenoprofen Calcium}{PMID: 36989008}, Aviva J Whelan et al., 2022]

\section*{Ingrezza}
\subsection*{Result}
\subsubsection*{Answer}

Unknown

\subsubsection*{{Explanation}}
\hypertarget{Ingrezza}
A review of the available abstracts reveals that none of them are about Ingrezza (valbenazine) or its use in children. Therefore, there is not enough data available in the abstracts to determine whether Ingrezza is safe or unsafe for use in children of any age range. No targeted studies on the safety of Ingrezza in children are present in the provided abstracts.

\subsection*{Abstracts}
\hypertarget{pmid_37152518}{I}n children, the Impella [\hyperlink{Ingrezza}{PMID: 37152518}, John S Sullivan et al., ] Scientific literature data on the experience of use of Proteflazid® (drops) and Immunoflazid® (syrup) for the treatment of viral diseases in children of the first six years of life are analysed in the article. A systematic review was conducted on the basis of postmarketing comparative clinical trials and long-term follow-up (during the period of 2002 to 2016) that involved about 1500 children (the intent-to-treat population comprised more than 800 of them). The safety and efficacy of the Proteflazid® (drops) and Immunoflazid® (syrup) usage in children for the treatment of viral infections have been proven. [\hyperlink{Ingrezza}{PMID: 37152518}, Galina Beketova et al., 2018]

\hypertarget{pmid_23904337}{I}n Sub-Saharan Africa, intrarectal diazepam is the first-line anticonvulsant mostly used in children. We aimed to assess this standard care against sublingual lorazepam, a medication potentially as effective and safe, but easier to administer. A randomized controlled trial was conducted in the pediatric emergency departments of 9 hospitals. A total of 436 children aged 5 months to 10 years with convulsions persisting for more than 5 minutes were assigned to receive intrarectal diazepam (0.5 mg/kg, n = 202) or sublingual lorazepam (0.1 mg/kg, n = 234). Sublingual lorazepam stopped seizures within 10 minutes of administration in 56\% of children compared with intrarectal diazepam in 79\% (P < .001). The probability of treatment failure is higher in case of sublingual lorazepam use (OR = 2.95, 95\% CI = 1.91-4.55). Sublingual lorazepam is less efficacious in stopping pediatric seizures than intrarectal diazepam, and intrarectal diazepam should thus be preferred as a first-line medication in this setting.  [\hyperlink{Ingrezza}{PMID: 23904337}, Célestin Kaputu Kalala Malu et al., 2014] To establish the safety of an intranasal diamorphine (IND) spray in children. An open-label, single-dose pharmacovigilance trial. Emergency departments in eight UK hospitals. Children aged 2-16 years with a fracture or other trauma. Adverse events (AE) specifically related to nasal irritation, respiratory and central nervous system depression. 226 patients received 0.1 mg/kg IND. No serious or severe AEs occurred. The incidence of treatment-emergent AEs (TEAEs) was 26.5\% (95\% CI 20.9\% to 32.8\%), 93\% being mild. 89\% were related to treatment, all being known effects of the drug or route of administration except for three events in two patients. 20.4\% (95\% CI 15.3\% to 26.2\%) patients reported nasal irritation, all mild except one moderate and one 'unknown' severity. No respiratory depression was reported. Three AEs related to reduced Glasgow Coma Score (GCS) occurred, all mild. There were no safety concerns raised during the conduct of the study. In addition to expected side effects, IND can cause mild nasal irritation in a proportion of patients. 2009-014982-16. [\hyperlink{Ingrezza}{PMID: 23904337}, Jason Kendall et al., 2015]

\hypertarget{pmid_22228569}{H}izentra® is a 20\% liquid IgG product approved for subcutaneous administration in adults and children greater than 2 years of age. We report two infants less than 2 years in which administration of Hizentra® was safe and effective. [\hyperlink{Ingrezza}{PMID: 22228569}, Joel L Gallagher et al., 2012]

\hypertarget{pmid_24942239}{A}lthough the analgesic effect of sucrose on newborns is well established, little is known about whether these solutions are effective in reducing procedural pain in infants beyond the newborn period. The purpose of this study was to determine the effect of sucrose solution given orally on infant crying times and measure the distress in a 16-19-month age group. A total of 537 healthy, 16-19-month-old infants attending for their immunizations with intramuscular diphtheria, tetanus, and acellular pertussis (DTaP)/Haemophilus influenza type b/IPV (along with oral polio vaccination (OPV)), intramuscular pneumococcus and intramuscular hepatitis A were randomized to receive 2 mL of a 75 \% sucrose solution, a 25 \% sucrose solution or sterile water 2 min before injections. Infants receiving a 75 \% sucrose solution had significantly reduced total crying times and Children's Hospital of Eastern Ontario Pain Scale scores (CHEOPS) compared with infants in the control and 25 \% sucrose solution groups (p < 0.001). Sucrose solution reduces infant distress and is safe and clinically useful even for 16-19-month-old infants. [\hyperlink{Ingrezza}{PMID: 24942239}, Gonca Yilmaz et al., 2014]

\hypertarget{pmid_25754747}{I}ntravenous iron sucrose is not recommended by its manufacturers for use in children despite extensive safety and efficacy data in adults. We reviewed the experience of our department between January, 2011 and February, 2014 with the use of intravenous iron sucrose in children ≤14 years of age who failed in oral iron therapy for iron deficiency anemia (IDA). Twelve children (6 females) aged 1.2-14 years (median age 8.9 years) received at least one dose of intravenous iron sucrose. Ten patients had IDA inadequately treated or non-responsive to oral iron therapy. One patient received therapy for blood transfusion avoidance and one for presumed iron refractory iron deficiency anemia (IRIDA). Iron sucrose infusions were given on alternate days up to three times per week. The number of infusions per patient ranged from 2 to 6 (median, 3), the individual doses from 100 mg to 200 mg (median, 200 mg), and the total doses from 200 mg to 1200 mg (median, 400 mg). Iron sucrose was effective in raising the hemoglobin concentration to normal in all patients with IDA, i.e., from 7.6±2.38 g/dL to 12.4±0.64 g/dL, within 31-42 days after the first infusion. The single patient with IRIDA demonstrated a 1.8 g/dL rise. Injection site disorders in three cases and transient taste perversion in one case were the only side effects. Intravenous iron sucrose appears to be safe and very effective in children with IDA who do not respond or cannot tolerate oral iron therapy. [\hyperlink{Ingrezza}{PMID: 25754747}, Elpis Mantadakis et al., 2016]

\hypertarget{pmid_34562149}{T}here is no approved dosage and administration of inulin for children. Therefore, we measured inulin clearance (Cin) in pediatric patients with renal disease using the pediatric dosage and administration formulated by the Japanese Society for Pediatric Nephrology, and compared Cin with creatinine clearance (Ccr) measured at the same time. We examined to what degree Ccr overestimates Cin, using the clearance ratio (Ccr/Cin), and confirmed the safety of inulin in pediatric patients. Pediatric renal disease patients aged 18 years or younger were enrolled. Inulin (1.0 g/dL) was administered intravenously at a priming rate of 8 mL/kg/hr (max 300 mL/hr) for 30 min. Next, patients received inulin at a maintenance rate of 0.7 × eGFR mL/min/1.73 m Inulin was administered to 60 pediatric patients with renal disease; 1 patient was discontinued and 59 completed. The primary endpoint, Ccr/Cin, was 1.78 ± 0.52 (mean ± standard deviation). Regarding safety, five adverse events were observed in four patients (6.7\%); all were non-serious. No adverse reactions were observed in this study. The results in this study on the dosage and administration of inulin showed that inulin can safely and accurately determine GFR in pediatric patients with renal disease. CLINICALTRIALS. NCT03345316. [\hyperlink{Ingrezza}{PMID: 34562149}, Osamu Uemura et al., 2022]

\hypertarget{pmid_16698412}{I}n sub-Saharan Africa, rectal diazepam or intramuscular paraldehyde are commonly used as first-line anticonvulsant agents in the emergency treatment of seizures in children. These treatments can be expensive and sometimes toxic. We aimed to assess a drug and delivery system that is potentially more effective, safer, and easier to administer than those presently in use. We did an open randomised trial in a paediatric emergency department of a tertiary hospital in Malawi. 160 children aged over 2 months with seizures persisting for more than 5 min were randomly assigned to receive either intranasal lorazepam (100 microg/kg, n=80) or intramuscular paraldehyde (0.2 mL/kg, n=80). The primary outcome measure was whether the presenting seizure stopped with one dose of assigned anticonvulsant agent within 10 min of administration. The primary analysis was by intention-to-treat. This study is registered with ClinicalTrials.gov, number NCT00116064. Intranasal lorazepam stopped convulsions within 10 min in 60 (75\%) episodes treated (absolute risk 0.75, 95\% CI 0.64-0.84), and intramuscular paraldehyde in 49 (61.3\%; absolute risk 0.61, 95\% CI 0.49-0.72). No clinically important cardiorespiratory events were seen in either group (95\% binomial exact CI 0-4.5\%), and all children finished the trial. Intranasal lorazepam is effective, safe, and provides a less invasive alternative to intramuscular paraldehyde in children with protracted convulsions. The ease of use of this drug makes it an attractive and preferable prehospital treatment option. [\hyperlink{Ingrezza}{PMID: 16698412}, Shafique Ahmad et al., 2006]

\hypertarget{pmid_30917911}{M}anagement of neonatal pain is not only ethical but is also essential. Barriers to pain management in infants include lack of safe and effective medications and fear of adverse effects of conventional pain medications. Sweet solutions given intraorally have been shown to reduce pain behaviors and associated symptoms. Sucrose and other sweet solutions are being increasingly used at the NICUs and immunization clinics. Sucrose for mild invasive procedures is effective and safe for those procedures that need to be repeated multiple times during the day. Only few studies examine the efficacy of sucrose for the management of inflammatory pain during infancy. In this study, Complete Freund's Adjuvant (CFA) was used to induce inflammation in 5-day-old rat pups; CFA also produces inflammation that lasts for more than a day, thus can also be a model for chronic pain. Sucrose or ibuprofen was given to subset of pups shortly after CFA intraplantar injections. Thermal as well as mechanical pain sensitivity was assessed on subsequent days as well as during adolescence and early adulthood. Sucrose and ibuprofen were both effective in preventing hyperalgesia and allodynia produced by CFA. Interestingly, sucrose was even more effective than ibuprofen, and the analgesic effects continued further to adolescence and adult life of the rats. Thus, and according to the results of this study, sucrose seems to be just as effective for inflammatory pain as Ibuprofen. In addition, sucrose protects against later-in-life hypersensitivity consequences to neonatal pain. [\hyperlink{Ingrezza}{PMID: 30917911}, Khawla Q Nuseir et al., 2019]

\hypertarget{pmid_21723413}{E}LISA for filaria-specific IgG4 in urine (urine ELISA) was applied to children in 7 schools in Sri Lanka, before and after 5 rounds of annual mass drug administration (MDA). The pre-treatment IgG4 prevalence in 2002 was 3.20\%, which decreased to 0.91\% in 2003 after the first MDA (P<0.001), and finally to 0.36\% in 2007 after the 5th MDA. Among 5-10 year-old children, the prevalence decreased from 3.37\% in 2002 to 0.51\% in 2003 (P=0.009). A pattern of IgG4 titer distribution according to age and its yearly change could also provide useful information in drug efficacy analysis. In 2008, new samples from eleven 2006/07 urine ELISA-positive students and their family members (total n=56) were examined by ICT antigen test, microfilaria test, and urine ELISA. No infection was confirmed among them. Urine ELISA will be useful in monitoring elimination/resurgence in a post-MDA low endemic situation. [\hyperlink{Ingrezza}{PMID: 21723413}, Makoto Itoh et al., 2011]

\hypertarget{pmid_37640809}{D}eficiencies of citrulline and arginine have been associated with adverse outcomes in preterm-infants and data regarding enteral supplementation in preterm infants is limited. This randomized -trial [NCT03649932] included 42 preterm infants (gestational age ≤33 weeks) randomized to receive enteral L-citrulline in low (100 mg/kg/day), medium (200 mg/kg/day) and high-dose (300 mg/kg/day) groups for 7 days. Plasma citrulline and arginine levels were obtained pre-and-post supplementation and efficacy was determined by a significant increase in levels after supplementation. A p < 0.05 was considered significant. Safety monitoring included blood-pressure-monitoring as well as complications and death during hospitalization. A total of 40/42 (95\%) of the recruits completed the 7-day supplementation with no adverse events. Plasma-citrulline levels increased significantly in all three groups while plasma-arginine levels increased significantly in the high-dose group. Enteral L-citrulline supplementation in preterm infants is safe and effective in increasing plasma citrulline and arginine levels. NCT03649932 https://clinicaltrials.gov/ct2/show/NCT03649932 . [\hyperlink{Ingrezza}{PMID: 37640809}, Amna Qasim et al., 2023]

\hypertarget{pmid_15544588}{G}iven the morbidity and mortality of asthma and the recent dramatic increase in its prevalence, pharmacologic prophylaxis of this disease in children at risk would represent a major medical advance. The Preventia I Study was designed to evaluate the efficacy and long-term safety of loratadine in reducing the number of respiratory infections in children at 24 months. A secondary objective was to investigate the benefit of loratadine treatment in preventing the onset of respiratory exacerbations. Preventia I was a randomized placebo-controlled study involving 22 countries worldwide. The children were 12-30 months of age at enrollment and had experienced at least five episodes of ENT infections, and no more than two episodes of wheezing during the previous 12 months. Phase I was a 12-month double-blind period during which the children were treated with loratadine 5 mg/day (2.5 mg/day for children</=24 months of age) or placebo. Phase II was a double-blind follow-up period without study medication. Of the 412 children enrolled, 342 and 310 completed Phase I and Phase II, respectively. The results showed a significant decrease in the number of infections in the whole population of children. However, no difference was observed between the loratadine and placebo group. When considering secondary end-points, loratadine was shown to reduce the number of respiratory exacerbations during the treatment phase. None of the 204 children who received loratadine discontinued the study because of drug-related events. Loratadine treatment was not more sedative than placebo and was not associated with cardiovascular events. The strong decrease in the rate of infections in the children at risk of recurrent infections, while not being influenced by loratadine treatment, should encourage future reflection in terms of prophylactic management. This study also confirms the long-term safety of loratadine and its metabolites in young children. [\hyperlink{Ingrezza}{PMID: 15544588}, A Grimfeld et al., 2004]

\hypertarget{pmid_25023977}{I}n spite of the high occurrence of migraine headaches in school-age children, there are currently no approved and widely accepted pharmacologic agents for migraine prophylaxis in children. Our previous open-label study in children revealed the efficacy of cinnarizine, a calcium channel blocker, in migraine prophylaxis. A placebo-controlled trial was conducted to demonstrate the efficacy and safety of cinnarizine in the prophylaxis of migraine in children. A double-blind, placebo-controlled, parallel-group study conducted in a tertiary medical center in Tehran, Iran. Children (5-17 years) who experienced migraines with and without aura, as defined on the basis of 2004 International Headache Society criteria, were recruited into the study. Children were excluded if they had complicated migraine, epilepsy, or a history of use of migraine prophylactic agents. Each participant was randomly assigned to receive cinnarizine (a single 1.5 mg/kg/day dose in children weighing less than 30 kg and a single 50 mg dose in children weighing more than 30 kg, administered at bedtime) or placebo. The frequency, severity, and duration of headaches over the trial period were assessed and adverse effects were monitored. A total of 68 children (34 in each group) with migraine were enrolled and 62 participants completed the study. After 3 months of taking cinnarizine or placebo, children in both groups experienced significantly reduced frequency, severity, and duration of headaches compared with baseline measurements (P < 0.001). However, compared with 31.3\% of children in the placebo group, 60\% of children in the cinnarizine group reported more than 50\% reduction in monthly headache frequency (P = 0.023), suggesting that cinnarizine was significantly more effective than placebo in reducing the frequency of headaches. No serious adverse effects of the medications were observed in the treated children, including no abnormal weight gain or extrapyramidal signs. Our results indicate that the use of cinnarizine at doses administered in this study is effective and safe for prophylaxis of migraine headaches in children. [\hyperlink{Ingrezza}{PMID: 25023977}, Mahmoud Reza Ashrafi et al., 2014]

\hypertarget{pmid_22858745}{T}he objective of this study was to compare the efficacy of 3 doses of intranasal ketamine (INK) for sedation of children from 1 to 7 years old requiring laceration repair. This was a randomized, prospective, double-blind trial of children requiring sedation for laceration repair. Patients with simple lacerations were randomized by age to receive 3, 6, or 9 mg/kg INK. Adequacy and efficacy of sedation were measured with the Ramsay sedation score and the Observational Scale of Behavioral Distress-Revised. Serum ketamine and norketamine levels were drawn during the procedure. Sedation duration and adverse events were recorded. Of the 12 patients enrolled, 3 patients achieved adequate sedation, all at the 9-mg/kg dose. The study was suspended at that time as per predetermined criteria. Nine milligrams of INK per kilogram produced a significantly higher proportion of successful sedations than the 3- and 6-mg/kg doses. [\hyperlink{Ingrezza}{PMID: 22858745}, Daniel S Tsze et al., 2012]

\hypertarget{pmid_8603220}{T}o assess the effectiveness of sucrose as an analgesic agent during routine immunization injections for infants (age range, 2 weeks to 18 months). Double-blind, randomized control trial. Ambulatory care clinic of a large tertiary care center. A consecutive sample of 285 infants were randomly assigned to one of three treatment groups. Subjects received either no intervention or drank 2 mL of sterile water or 2 mL or a 12\% sucrose solution 2 minutes before administration of the immunization. Infants were videotaped during immunization for later interval recording of pain-induced vocalizations. Results were analyzed by using two-way repeated measure analyses of variance. Two-week-old infants who received either the sterile water or sucrose solution cried significantly less than infants who received no intervention (F=5.92,P<.005). For older infants, those who received water or sucrose cried significantly less only if they were administered one injection rather than two injections (F=3.36,P<.05). We found that when infants drank sucrose or sterile water, significantly fewer pain vocalizations were produced, but only for 2-week-old infants. For older infants, differences were found only when the number of injections was included in the analysis. We expand on previous findings by demonstrating that both the age of the child and the number of painful exposures can attenuate calming effects. In addition, the results suggest that in the absence of nonnutritive sucking, the actual analgesic effects of sucrose may be nonspecific. Further study is needed of the possible analgesic effects of sucrose. [\hyperlink{Ingrezza}{PMID: 8603220}, K D Allen et al., 1996]

\hypertarget{pmid_16613039}{T}he study has comparatively evaluated the effectiveness and safety of halothane, enflurane, and isoflurane in children during induction. Seventy hundred and eight patients aged 1-14 years who had ASA I-II anesthetic risks were examined. Gas induction was performed as monoanesthesia through the semi-open circuit with high gas flow (100\% O2 6 l/min) in combination with halothane (n = 236), enflurane (n = 236), or isoflurane (n = 236) without N2O. The authors have compared the following criteria: the speed and comfort of induction, the parameters of hemodynamics and external respiration, and the rate of adverse reactions and complications during induction. The studies have established that in terms of comfort, safety, and the rate clinical effect achievement, the drugs of choice for gas induction in children are enflurane and, to a lesser extent, halothane. Gas induction with isoflurane should not be performed in children since the agent rather frequently exerts an irritant action on the upper airways, which reduces the speed of initial narcosis and increases the likelihood of one or another adverse reactions; however; it has advantages as a less hemodynamic effect. [\hyperlink{Ingrezza}{PMID: 16613039}, V A Sidorov et al., ]

\hypertarget{pmid_22929601}{T}his study examined the effects of oral sucrose as an analgesic agent during routine immunization for infants at 2, 4, and 6 months of age. A sample of 113 healthy infants were recruited from three ambulatory clinics and randomly assigned to one of three treatment groups. Infants were given 2 mL orally of either 50\% sucrose, 75\% sucrose, or sterile water 2 minutes before administration of immunizations. No significant difference was found among the different age groups with the different treatments for pain as measured with the FLACC scores and crying time. Consolability factors are felt to have some influence. [\hyperlink{Ingrezza}{PMID: 22929601}, Donna Miles Curry et al., 2012]

\hypertarget{pmid_37962334}{I}savuconazole (ISA) is approved for treating invasive aspergillosis and mucormycosis in adults, but its use in children remains off-label. We report on the use of ISA in real-world pediatric practice with 15 patients receiving ISA for treatment of invasive fungal infections. Therapeutic drug monitoring (TDM) was performed in all patients, with 52/111 (46.8\%) C [\hyperlink{Ingrezza}{PMID: 37962334}, Berta Fernández Ledesma et al., 2023] An inhaled corticosteroid (ICS) or leukotriene receptor antagonist (LTRA) may prevent wheezing/asthma attacks in preschoolers with recurrent wheeze when added to short-acting β-agonist (SABA). The aim of this historical matched cohort study was to assess the effectiveness of these treatments for preventing wheezing/asthma attacks. Electronic medical records from the Optimum Patient Care Research Database were used to characterize a UK preschool population (1-5 years old) with two or more episodes of wheezing during 1 baseline year before first prescription (index date) of ICS or LTRA, or repeat prescription of SABA. Children initiating ICS or LTRA on the index date were matched 1:4 to those prescribed only SABA for age, sex, year of index prescription, mean baseline SABA dose, baseline attacks, baseline antibiotic prescriptions, and eczema diagnosis. Wheezing/asthma attacks (defined as asthma-related emergency attendance, hospital admission, or acute oral corticosteroid prescription) during 1 outcome year were compared using conditional logistic regression. Matched ICS and SABA cohorts included 990 and 3,960 children, respectively (61\% male; mean [SD] age 3.2 [1.3] years), and matched LTRA and SABA cohorts included 259 and 1,036 children, respectively (65\% male; mean [SD] age 2.6 [1.2] years). We observed no significant difference between matched cohorts in the odds of a wheezing/asthma attack: ICS vs SABA, OR (95\% CI) 1.01 (0.85-1.19) and LTRA vs SABA, OR (95\% CI) 1.28 (0.96-1.72). We found no evidence that initiation of ICS or LTRA therapy is associated with fewer attacks during 1 outcome year than SABA alone for a heterogeneous group of preschool children with recurrent wheeze in the real-life clinical setting. [\hyperlink{Ingrezza}{PMID: 37962334}, Jonathan Grigg et al., 2018]

\hypertarget{pmid_12075764}{W}e conducted the first trial to assess the safety and immunogenicity of an oral, killed enterotoxigenic Escherichia coli plus cholera toxin B-subunit vaccine in children <2 years old. Three doses of vaccine or killed E. coli K-12 control were given at 2-week intervals to 64 Egyptian infants, 6 to 18 months old, in a randomized, double blind manner. Adverse events were monitored for 3 days after each dose. Blood was collected before immunization and 7 to 10 days after each dose to assess vaccine-specific serologic responses. There was no statistically significant intergroup difference in the percentage of subjects reporting the primary safety endpoint (diarrhea or vomiting) after the first (31\%, vaccine; 30\%, control) or third (14\%, vaccine; 18\%, control) dose, whereas there was a trend toward greater reporting in the vaccine group after Dose 2 (36\%, vaccine; 18\%, control; P = 0.052). The percentage of children showing IgA seroconversion after any dose was higher in the vaccine than the control group for recombinant cholera toxin B-subunit (97\% vs. 46\%), colonization factor antigen I (61\% vs. 18\%) and coli surface antigen 4 (39\% vs. 4\%) (P < 0.001 for each comparison). IgG seroconversion rates in the vaccine and control groups were 97 and 21\% to recombinant cholera toxin B-subunit (P < 0.001), 64 and 29\% for colonization factor antigen I (P < 0.01), 53 and 21\% for coli surface antigen 2 (P < 0.05) and 58 and 4\% for coli surface antigen 4 (P < 0.001), respectively. The third vaccine dose was followed by augmented IgG antitoxin titers. The oral enterotoxigenic E. coli vaccine was safe and immunogenic in this setting in Egyptian infants. [\hyperlink{Ingrezza}{PMID: 12075764}, Stephen J Savarino et al., 2002]

\hypertarget{pmid_20413804}{I}n 2006, intravenous levetiracetam received US Food and Drug Administration (FDA) approval for adjunctive treatment of partial onset seizures in adults with epilepsy, 16 years or older. We have established the safety, tolerability, and dosage of intravenous levetiracetam in children. This prospective study included 30 children (6 months to <15 years of age). Patients were administered a single dose of intravenous levetiracetam (50 mg/kg, maximal dose 2500 mg) over 15 minutes. A blood level of levetiracetam was performed 10 minutes after the infusion. The treated children's average age was 6.3 years (range 0.5-14.8 years). The mean levetiracetam level was 83.3 microg/mL (range 47-128 microg/mL). There were no serious adverse reactions. Minor reactions included sleepiness, fatigue, and restlessness. An apparent decrease in seizure frequency across all seizure types was noted. The dose of 50 mg/kg was well tolerated by the patients and is a safe, appropriate loading dose. [\hyperlink{Ingrezza}{PMID: 20413804}, Yu-tze Ng et al., 2010]

\hypertarget{pmid_8301460}{T}he authors retrospectively review their experience in children with the latest addition to the postoperative analgesic armamentarium: interpleural analgesia (IPA). IPA was used in 14 children following thoracotomy. There were 9 boys and 5 girls. Patients varied in age from 2 months to 17 years 4 months (mean +/- SEM = 7.6 +/- 1.6 yr). Catheters were left in place from 10 to 72 hours (mean +/- SEM = 45.1 +/- 4.6 h). Four patients received intermittent bolus doses and 10 patients received a continuous infusion through the interpleural catheters. Adequate analgesia, as judged by both subjective responses (decreased irritability or complaints of pain) and by objective physiologic responses (decreased heart rate, respiratory rate, and systolic blood pressure), was achieved in 13 of 14 patients. Eight of the 14 children required no additional analgesic agents. One child received 2 doses of oral codeine and 4 patients received 2 to 3 doses of intravenous narcotic during IPA. IPA was not effective in one patient who required 6 doses of intravenous meperidine. Patients more than 10 years of age required significantly more (P < 0.05) intravenous narcotic supplementation than patients less than 10 years of age (1.60 +/- 0.50 v 0.14 +/- 0.11 mg meperidine/kg/d). No complications related to placement or subsequent use of IPA were identified in any of the patients. IPA provides effective postoperative analgesia following thoracotomy in children. [\hyperlink{Ingrezza}{PMID: 8301460}, J D Tobias et al., 1993]

\hypertarget{pmid_26336818}{H}izentra® (IGSC 20\%) is a 20\% liquid IgG product approved for subcutaneous administration in adults and children 2 years of age and older who have primary immunodeficiency disease (PIDD). There is limited information about the use of IGSC 20 \% in very young children including those less than 5 years of age. A retrospective chart review involved 88 PIDD infants and children less than 5 years of age who received Hizentra®. The mean age at the start of Hizentra® was 34 months (range 2 to 59 months). IGSC 20 \% was administered weekly to 86 infants (two additional infants received twice weekly and three times weekly infusions, respectively) and included an average of 63 infusions (range 6-182) for an observation period up to 45.5 months. Infusion by manual delivery occurred in 15 patients. The mean dose was 674 mg/kg/4 weeks. The mean IgG level was 942 mg/dL while on IGSC 20 \%, compared to a mean trough IgG level of 794 mg/dL (p < 0.0001) during intravenous or subcutaneous IgG administration prior to IGSC 20 \%. Average infusion time was 47 (range 5-120) minutes, and the median number of infusion sites was 2 (range 1-4). Local reactions were mostly mild and observed in 36/88 (41\%) children. No serious adverse events were reported. A significant increase in weight percentile (7 \% ± 19.2, p = 0.0012) among subjects was observed during IGSC 20\% administration. The rate of serious bacterial infections was 0.067 per patient-year while receiving IGSC 20\%, similar to previously reported efficacy studies. Hizentra® is effective in preventing infections, and is well tolerated in children less than age 5 years. [\hyperlink{Ingrezza}{PMID: 26336818}, Niraj C Patel et al., 2015]

\hypertarget{pmid_18326612}{M}ultiple studies have shown the benefits of zinc supplementation among young children in high-risk populations. However, the optimal dose and safe upper level of zinc have not been determined. The objectives of this study were to measure the effects of different doses of supplemental zinc on the plasma zinc concentration, morbidity, and growth of young children; to detect any adverse effects of 10 mg supplemental Zn on markers of copper or iron status; and to determine whether any adverse effects are alleviated by providing copper with zinc. This randomized, double-masked, community-based intervention trial was conducted in 631 Ecuadorian children who were 12-30 mo old at baseline and who had initial length-for-age z scores <-1.3. Children received 1 of 5 daily supplements for 6 mo: 3, 7, or 10 mg Zn as zinc sulfate, 10 mg Zn + 0.5 mg Cu as copper sulfate, or placebo. The change in plasma zinc concentration from baseline was positively related to the zinc dose (P < 0.001). Zinc supplementation, including doses as low as 3 mg/d, reduced the incidence of diarrhea by 21-42\% (P < 0.01). There were no other significant group-wise differences. Zinc supplementation with a dose as low as 3 mg/d increased plasma zinc concentrations and reduced diarrhea incidence in the study population. There were no observed adverse effects of 10 mg Zn/d on indicators of copper or iron status. The current tolerable upper level of zinc recommended by the Institute of Medicine should be reassessed for young children. [\hyperlink{Ingrezza}{PMID: 18326612}, Sara E Wuehler et al., 2008]

\section*{Itraconazole}
\subsection*{Result}
\subsubsection*{Answer}

Infants (0–1 year): Yes  
Children (1–12 years): Yes  
Adolescents (12–18 years): Yes  

\subsubsection*{{Explanation}}
\hypertarget{Itraconazole}
To determine if itraconazole is safe for use in children, I reviewed the available abstracts for targeted studies on pediatric populations, focusing on specific age ranges and direct safety outcomes.

Infants (0–1 year):
- Multiple studies specifically address infants. One review of 45 articles (13 on superficial, 32 on systemic infections) concluded that itraconazole is "safe and effective" at 5 mg/kg/day for superficial and 10 mg/kg/day for systemic infections in infants, with adverse event profiles similar to adults and older children [\hyperlink{pmid_27286691}{PMID: 27286691}, Shuang Chen et al., 2016].
- Case series and clinical trials for infantile hemangiomas (6 infants, 36 infants, and 2 infants in separate studies) report good efficacy and mild, limited side effects, with no serious adverse events [\hyperlink{pmid_25512128}{PMID: 25512128}, Yuping Ran et al., 2015; \hyperlink{pmid_31668109}{PMID: 31668109}, Hagar Bessar et al., 2022; \hyperlink{pmid_34754871}{PMID: 34754871}, Zhe Liu et al., 2021].

Children (1–12 years):
- Several studies specifically target this age group:
    - A 10-year retrospective review of 81 children (≤12 and >12 years) found that itraconazole was used for prophylaxis and treatment, with gastrointestinal symptoms (15.2\%) and hepatotoxicity (6.5\%) as the most frequent adverse events, neither associated with elevated drug levels. The study notes the need for higher empiric doses in children <12 years to achieve therapeutic levels, but does not report unexpected safety issues [\hyperlink{pmid_29601447}{PMID: 29601447}, Ying Hua Leong et al., 2019].
    - An open study of 103 neutropenic children aged 0–14 years (median 5) using 5 mg/kg/day oral solution found no unexpected safety or tolerability problems. The most common adverse events were vomiting, abnormal liver function, and abdominal pain, but no proven systemic fungal infections occurred [\hyperlink{pmid_10578159}{PMID: 10578159}, A B Foot et al., 1999].
    - A study of 163 children with tinea capitis (5 mg/kg/day) found both clinical and mycologic cure, with side effects in 6.7\% (diarrhea, cutaneous eruption, abdominal pain) [\hyperlink{pmid_15283801}{PMID: 15283801}, Gabriele Ginter-Hanselmayer et al.].
    - A review of systemic antifungals for onychomycosis in children (ages 1–17) found that safety profiles of itraconazole were similar to adults, with no serious adverse events reported [\hyperlink{pmid_23278514}{PMID: 23278514}, Aditya K Gupta et al.].
    - A study of 36 children/adolescents (4–17 years) with onychomycosis treated with itraconazole (200 mg daily for 12 weeks) reported no serious adverse events [\hyperlink{pmid_18194238}{PMID: 18194238}, G Ginter-Hanselmayer et al., 2008].
    - A study of SUBA-itraconazole in 19 children (median age 12) found it was well tolerated, with no cessations due to adverse effects [\hyperlink{pmid_32929460}{PMID: 32929460}, Joanne Abbotsford et al., 2021].
    - A study of 53 children (median age 8.6, range 0.4–18.3) undergoing hematopoietic stem cell transplantation found itraconazole prophylaxis feasible and safe, despite abnormal laboratory results [\hyperlink{pmid_17430480}{PMID: 17430480}, L Grigull et al., 2007].
    - A pharmacokinetic and safety study in neutropenic children stratified by age found effective plasma concentrations were reached and maintained, with no specific safety concerns reported [\hyperlink{pmid_11302829}{PMID: 11302829}, C Schmitt et al., 2001].

Adolescents (12–18 years):
- The above studies often include adolescents, and no age-specific safety concerns are reported. For example, the onychomycosis studies include children up to 17 years, and the hematopoietic stem cell transplantation study includes patients up to 18.3 years, with no serious adverse events attributed to itraconazole [\hyperlink{pmid_18194238}{PMID: 18194238}, G Ginter-Hanselmayer et al., 2008; \hyperlink{pmid_17430480}{PMID: 17430480}, L Grigull et al., 2007].

General pediatric population (0–18 years):
- Multiple reviews and studies affirm that the safety profile of itraconazole in children is similar to that in adults, with adverse events generally mild and transient [\hyperlink{pmid_12956205}{PMID: 12956205}, Aditya K Gupta et al., 2003; \hyperlink{pmid_23278514}{PMID: 23278514}, Aditya K Gupta et al.].
- However, one study highlights a significant drug interaction: itraconazole can enhance vincristine neurotoxicity in children, leading to severe and potentially fatal toxicities. The authors recommend avoiding this combination [\hyperlink{pmid_16012330}{PMID: 16012330}, Mar Bermúdez et al., 2005].

Summary:
- There are multiple targeted studies in infants, children, and adolescents affirming the safety of itraconazole for various indications, with adverse events similar to those seen in adults and generally mild. The exception is the risk of severe neurotoxicity when combined with vincristine.
- For all pediatric age groups, the evidence supports that itraconazole is safe for use, provided drug interactions are carefully managed.

\subsection*{Abstracts}
\hypertarget{pmid_29601447}{I}traconazole is a broad-spectrum antifungal agent used for prophylaxis and treatment of fungal infections in immunocompromised children. Achieving the recommended target serum itraconazole trough concentration of ≥0.5 mg/L is challenging in children because of variation in itraconazole pharmacokinetics with age. We studied itraconazole use and treatment outcomes in a tertiary children's hospital. We did a 10-year retrospective review of medical records of children at the Royal Children's Hospital Melbourne who received oral itraconazole and had therapeutic drug monitoring (TDM). Overall, 81 children received 92 courses of oral itraconazole and had TDM. Of 222 TDM samples, 183 (82.4\%) were taken at the appropriate time (trough level at steady state). Patients ≤12 and >12 years of age required median doses of 6.2 and 3.9 mg/kg/d, respectively, to attain target trough levels (P < 0.001). Of children ≤12 years of age, 71.4\% required doses above the recommended dose of 5 mg/kg/d to achieve therapeutic levels, compared with 17.4\% of those >12 years of age. At least 1 subtherapeutic trough concentration was reported in 63 (76.8\%) courses; in only 18 (28.6\%) of these was the dose adjusted. Gastrointestinal symptoms [14/92 (15.2\%) courses] and hepatotoxicity [6/92 (6.5\%)] were the most frequent adverse events. Neither was associated with elevated trough levels. The poor attainment of target levels with current recommended dosing in children <12 years of age suggests that higher empiric doses are needed in this age group. The poor compliance with TDM guidelines highlights the need for better education about appropriate timing of sampling and dose adjustment. [\hyperlink{Itraconazole}{PMID: 29601447}, Ying Hua Leong et al., 2019]

\hypertarget{pmid_27286691}{I}traconazole has been used to treat fungal infections, in particular invasive fungal infections in infants or neonates in many countries. Literature search was conducted through Ovid EMBASE, PubMed, ISI Web of Science, CNKI and Google scholarship using the following key words: "pediatric" or "infant" or "neonate" and "fungal infection" in combination with "itraconazole". Based on the literature and our clinical experience, we outline the administration of itraconazole in infants in order to develop evidence-based pharmacotherapy. Of 45 articles on the use of itraconazole in infancy, 13 are related to superficial fungal infections including tinea capitis, sporotrichosis, mucosal fungal infections and opportunistic infections. The other 32 articles are related to systemic fungal infections including candidiasis, aspergillosis, histoplasmosis, zygomycosis, trichosporonosis and opportunistic infections as caused by Myceliophthora thermophila. Itraconazole is safe and effective at a dose of 5 mg/kg per day in a short duration of therapy for superficial fungal infections and 10 mg/kg per day for systemic fungal infections in infants. With a good compliance, it is cost-effective in treating infantile fungal infections. The profiles of adverse events induced by itraconazole in infants are similar to those in adults and children. [\hyperlink{Itraconazole}{PMID: 27286691}, Shuang Chen et al., 2016]

\hypertarget{pmid_12956205}{C}urrent dosing regimens for itraconazole are effective, safe, and versatile for use in superficial fungal infections in children, particularly tinea capitis. Good efficacy rates have been noted in both Trichophyton and Microsporum tinea capitis infections. Itraconazole has a high affinity for keratin, and accumulates to high levels at the site of superficial fungal infections. A pulse regimen may be chosen over continuous dosing, because the accumulation persists after dosing of itraconazole has been stopped. An oral solution of itraconazole is available, and may be more convenient for children who cannot swallow capsules. The oral solution also produces good rates of efficacy, but may be associated with a somewhat higher potential for gastrointestinal adverse events than the capsules. The range of adverse events noted with itraconazole capsules or oral solution use in children is similar to the range in adults. Events are generally mild and transient. Attention must be taken to note any medications that the child is using, because itraconazole is associated with a range of potential drug interactions. This safety of use, in combination with itraconazole's wide antifungal spectrum and pharmacokinetic properties, which allow for shorter dosing regimens, may make itraconazole a suitable alternative to griseofulvin for pediatric superficial fungal infections. [\hyperlink{Itraconazole}{PMID: 12956205}, Aditya K Gupta et al., 2003]

\hypertarget{pmid_17430480}{T}his single-centre, retrospective, observational pilot study was performed to evaluate the safety and efficacy of intravenous and oral itraconazole prophylaxis in paediatric haematopoietic stem cell transplantation (HCT). Study end-points were proven invasive fungal infection (IFI), survival, adverse reactions and graft-vs.-host disease (GVHD); 53 children and one young adult (median age 8.6 yr; range 0.4-18.3) transplanted between November 2001 and August 2004 were included in this study. Itraconazole was given intravenously from day +3 after HCT until oral medication became possible and continued until day +100 after HCT. Two proven new IFI in the itraconazole group (candidiasis, n = 1; aspergillosis, n = 1) were observed. After a median follow-up of 1.6 yr (0.3-6.1), six deaths (8\%) were seen; 24 patients (45\%) developed GVHD degree I-II, three children (6\%) had GVHD degree III-IV. In 11 of 53 patients (21\%), itraconazole prophylaxis was discontinued prematurely, mostly because of fever of unknown origin (n = 7). In total, 21 of 53 (40\%) of the children had abnormal results of laboratory investigations during the prophylaxis. The results of this pilot study indicate that itraconazole prophylaxis during HCT in children is feasible and safe, despite abnormal laboratory results. The efficacy in terms of prevention of IFI, however, has to be addressed in a prospective large-scale study. [\hyperlink{Itraconazole}{PMID: 17430480}, L Grigull et al., 2007]

\hypertarget{pmid_25512128}{I}nfantile hemangiomas can present a therapeutic challenge to clinicians, especially when associated with severe pain and feeding difficulties. The standard therapeutic management includes corticosteroids and propranolol. However, the clinical response is not always satisfactory. We present six cases of infantile hemangiomas successfully treated with oral itraconazole approximately 5 mg/kg per day. In the first month, the red color of the lesions became a little lighter and the growth of the lesions was controlled in all cases. An obvious clinical improvement was noted in all cases during the 3-month period, with 80-100\% improvement in each patient at the end of the treatment, which was judged by both their parents and the dermatologists. Compliance with treatment instructions of oral itraconazole in infants was judged to be very good. Side-effects were mild and limited. Although itraconazole can inhibit angiogenesis and tumor growth in vitro and in vivo associated with some cancers, further research is required to understand the pathogenesis of infantile hemangiomas and the mechanism of itraconazole.  [\hyperlink{Itraconazole}{PMID: 25512128}, Yuping Ran et al., 2015] We investigated the pharmacokinetics and safety of an oral solution of itraconazole in two groups of neutropenic children stratified by age. Effective concentrations of itraconazole in plasma were reached quickly and maintained throughout treatment. The results indicate a trend toward higher concentrations of itraconazole in plasma in older children. [\hyperlink{Itraconazole}{PMID: 25512128}, C Schmitt et al., 2001]

\hypertarget{pmid_16012330}{I}traconazole is particularly attractive in fungal prophylaxis for cancer patients due to its broad spectrum, including Candida and Aspergillus. It is generally well tolerated. However, its efficacy in preventing invasive aspergillosis could not be demonstrated. A 3-year-old boy diagnosed with acute lymphoblastic leukemia received induction chemotherapy. On day 14, itraconazole solution at a dose of 5 mg/kg was begun. Ten days after itraconazole was started, he developed paralytic ileus, neurogenic bladder, mild left ptosis, and absence of deep reflexes, with severe paralysis of the lower extremities and mild weakness of the upper extremities. Itraconazole withdrawal was followed by rapid improvement, with neurologic examination returning to normal within 6 weeks. Nineteen cases of unusual enhanced vincristine neurotoxicity related to itraconazole have been reported in children. Although the manifestations are the same as those usually associated with the use of vincristine, in these cases the severity appears remarkable. The authors suggest that in the absence of any proven benefit of itraconazole prophylaxis, and given the interaction of this drug with vincristine leading to severe and even potentially fatal toxicities, the combination use of these drugs should be avoided. [\hyperlink{Itraconazole}{PMID: 16012330}, Mar Bermúdez et al., 2005]

\hypertarget{pmid_21628477}{I}traconazole has become the first choice for treatment of cutaneous sporotrichosis. However, this recommendation is based on case reports and small series. The safety and efficacy of itraconazole were evaluated in 645 patients who received a diagnosis on the basis of isolation of Sporothrix schenckii in Rio de Janeiro, Brazil. A standard regimen of itraconazole (100 mg/day orally) was used. Clinical and laboratory adverse events were assessed a grades 1-4. A multivariate Cox model was used to analyze the response to treatment. The median age was 43 years. Lymphocutaneous form occurred in 68.1\% and fixed form in 23.1\%. Six hundred ten patients (94.6\%) were cured with itraconazole (50-400 mg/day): 547 with 100 mg/day, 59 with 200-400 mg/day, and 4 children with 50 mg/day. Three patients switched to potassium iodide, 2 to terbinafine, and 4 to thermotherapy. Twenty-six were lost to follow-up. Clinical adverse events occurred in 18.1\% of patients using 100 mg/day and 21.9\% of those using 200-400 mg/day. The most frequent clinical adverse events were nausea and epigastric pain. Laboratory adverse events occurred in 24.1\%; the most common was hypercholesterolemia, followed by hypertriglyceridemia. Four hundred sixty-two patients (71.6\%) completed clinical follow-up, and all remained cured. Only 2 variables were significant in explaining the cure: patients with erythema nodosum healed faster, and lymphocutaneous form took longer to cure. In the current series, the therapeutic response was excellent with the minimum dose of itraconazole, and there was a low incidence of adverse events and treatment failure. [\hyperlink{Itraconazole}{PMID: 21628477}, Mônica Bastos de Lima Barros et al., 2011]

\hypertarget{pmid_31668109}{T}he initial recommendation propranolol usage in managing infantile hemangioma was in 2008 followed by various researches assessing the dosage, efficacy, and other parameters. Itraconazole is a world-wide tolerated antifungal but only a few studies have focused on its assessment in the treatment of infantile hemangiomas (IH). This study aimed to investigate the newly proposed antifungal drug ICZ and characterize different aspects of its usage as an antiangiogenic drug. This was an interventional clinical trial to assess the efficacy of ICZ versus propranolol in the treatment of infantile hemangioma with studying the change in serum angiopoietin 2 (Ang2). A total of 36 pediatric patients were divided into two equal groups: firstly treated with oral itraconazole and secondly treated by oral propranolol. Response to treatment was observed using a modified IH score. In itraconazole-treated infants, good response was observed in 44.4\% of the patients. This was slightly higher than the propranolol group which showed 22.2\% with good response. We observed a decrease in serum ang2 level after usage of ICZ and propranolol and the change in serum Ang2 level before and after treatment in each group was statistically significant ( Oral itraconazole can be an equivalent option for oral propranolol while safer and shorter treatment periods. [\hyperlink{Itraconazole}{PMID: 31668109}, Hagar Bessar et al., 2022]

\hypertarget{pmid_10523732}{O}ver the past 10 years, itraconazole has been used to treat more than 34 million patients worldwide. We present a review of the safety of various continuous itraconazole schedules used in the treatment of dermatomycosis and onychomycosis. Data from controlled clinical trials and extensive post-marketing surveillance show that itraconazole has an impressive safety profile at a dose of 50-200 mg/day for 1-4 weeks for dermatomycosis and 200 mg/day for 3 months for onychomycosis. In addition, itraconazole is safe to use in diabetic patients with dermatomycosis or onychomycosis. Short-term, intermittent itraconazole regimens, which may offer additional benefits in terms of safety and cost, have now been introduced. [\hyperlink{Itraconazole}{PMID: 10523732}, S K Nolting et al., ]

\hypertarget{pmid_32929460}{I}traconazole remains a first-line antifungal agent for certain fungal infections in children, including allergic bronchopulmonary aspergillosis (ABPA) and sporotrichosis, but poor attainment of therapeutic drug levels is frequently observed with available oral formulations. A formulation of 'SUper BioAvailability itraconazole' (SUBA-itraconazole; Lozanoc®) has been developed, with adult studies demonstrating rapid and reliable attainment of therapeutic levels, yet paediatric data are lacking. To assess the safety, efficacy and attainment of therapeutic drug levels of the SUBA-itraconazole formulation in children. A single-centre retrospective cohort study was conducted, including all patients prescribed SUBA-itraconazole from May 2018 to February 2020. The recommended initial treatment dose was 5 mg/kg twice daily (to a maximum of 400 mg/day) rounded to the nearest capsule size and 2.5 mg/kg/day for prophylaxis. Nineteen patients received SUBA-itraconazole and the median age was 12 years. The median dose was 8.5 mg/kg/day and the median duration was 6 weeks. Indications included ABPA (16 patients), sporotrichosis (1), cutaneous fungal infection (1) and prophylaxis (1). Of patients with serum levels measured, almost 60\% (10/17) achieved a therapeutic level, 3 with one dose adjustment and 7 following the initial dose. Adherence to dose-adjustment recommendations amongst the seven patients not achieving therapeutic levels was poor. Of patients with ABPA, 13/16 (81\%) demonstrated a therapeutic response in IgE level. SUBA-itraconazole was well tolerated with no cessations related to adverse effects. SUBA-itraconazole is well tolerated in children, with rapid attainment of therapeutic levels in the majority of patients, and may represent a superior formulation for children in whom itraconazole is indicated for treatment or prevention of fungal infection. [\hyperlink{Itraconazole}{PMID: 32929460}, Joanne Abbotsford et al., 2021]

\hypertarget{pmid_1655460}{A}n 11-year-old boy with chronic granulomatous disease caused by cytochrome b deficiency developed right upper lung lobe aspergillosis. Intracerebral lesions developed on maximum doses of flucytosine and amphotericin B. Treatment with 16 mg/kg oral itraconazole was followed by a dramatic clinical improvement and almost complete disappearance of the intracerebral lesions. Plasma itraconazole levels were between 40 and 3440 ng/ml depending on concomitant medication. Toxicity was restricted to transient elevation of alkaline phosphatase and gamma glutamyl transferase. We conclude that further trials with itraconazole are justified in high risk patients in whom conventional therapy has failed. [\hyperlink{Itraconazole}{PMID: 1655460}, S Kloss et al., 1991]

\hypertarget{pmid_18194238}{O}nychomycosis is a rare disease in children with an estimated prevalence ranging from 0\% to 2.6\%. Thus far, only limited experience with itraconazole and terbinafine treatment in children with onychomycosis is available in the literature. Evaluation of treatment experience with itraconazole or terbinafine in childhood onychomycosis. Thirty-six children and adolescents (aged 4-17 years, 18 males and 18 females) with clinical and mycologically proven onychomycosis were enrolled in the present study. METHODS AND OUTCOME: In 27 of 36 patients, the causative agent (Trichophyton rubrum in 26 cases and Trichophyton interdigitale in one patient) could be identified by means of cultivation. Nineteen patients were treated with itraconazole 200 mg once daily for 12 weeks, and 17 patients were treated with terbinafine for 12 weeks in a dosage according to their body weight, respectively. Clinical cure was achieved within 1 to 5 months after discontinuation in all patients treated with itraconazole and in all but two patients after cessation of terbinafine treatment. Neither in the itraconazole nor in the terbinafine group were serious adverse events reported. Clinical cure was achieved within 1 to 5 months after discontinuation in all patients treated with itraconazole and in all but two patients after cessation of terbinafine treatment. Neither in the itraconazole nor in the terbinafine group were serious adverse events reported. To our experience, a mycological and clinical cure appears in children in a shorter time after treatment discontinuation (average 2-5 months) compared with adults. Itraconazole and terbinafine seem to be safe and effective in childhood onychomycosis; therefore, these antifungals seem to be potential alternatives to griseofulvin. [\hyperlink{Itraconazole}{PMID: 18194238}, G Ginter-Hanselmayer et al., 2008]

\hypertarget{pmid_24173819}{O}ral antifungal prophylaxis with extended-spectra azoles is widely used in pediatric patients after allogeneic hematopoietic stem cell transplantation (HSCT), while controlled studies for oral antifungal prophylaxis after bone marrow transplantation in children are not available. This survey analyzed patients who had received either itraconazole, voriconazole, or posaconazole. We focused on the safety, feasibility, and initial data of efficacy in a cohort of pediatric patients and adolescents after high-dose chemotherapy and HSCT. Fifty consecutive pediatric patients received itraconazole, 50 received voriconazole, and 50 pediatric patients received posaconazole after HSCT as oral antifungal prophylaxis. The observation period lasted from the start of oral prophylactic treatment with itraconazole, voriconazole, or posaconazole until two weeks after terminating the oral antifungal prophylaxis. No incidences of proven or probable invasive mycosis were observed during itraconazole, voriconazole, or posaconazole treatment. A total of five possible invasive fungal infections occurred, two in the itraconazole group (4\%) and three in the voriconazole group (6\%). The percentage of patients with adverse events potentially related to clinical drugs were 14\% in the voriconazole group, 12\% in the itraconazole group, and 8\% in the posaconazole group. Itraconazole, voriconazole, and posaconazole showed comparable efficacy as antifungal prophylaxis in pediatric patients after allogeneic HSCT. [\hyperlink{Itraconazole}{PMID: 24173819}, M Döring et al., 2014]

\hypertarget{pmid_34754871}{I}nfantile hemangiomas (IHs) are the most common childhood benign tumors, showing distinctive progression characteristics and outcomes. Due to the high demand for aesthetics among parents of IH babies, early intervention is critical in some cases. β-Adrenergic blockers and corticosteroids are first-line medications for IHs, while itraconazole, an antifungal medicine, has shown positive results in recent years. In the present study, itraconazole was applied to treat two IH cases. The therapeutic course lasted 80-90 d, during which the visible lesion faded by more than 90\%. Moreover, no obvious side effects were reported, and the compliance of the baby and parents was desirable. Although these outcomes further support itraconazole as an effective therapeutic choice for IHs, large-scale clinical and basic studies are still warranted to improve further treatment. [\hyperlink{Itraconazole}{PMID: 34754871}, Zhe Liu et al., 2021]

\hypertarget{pmid_8387801}{I}traconazole is a new orally active antifungal agent shown to have in vitro and experimental activity against Aspergillus spp. This case report documents the successful eradication of biopsy-proven invasive pulmonary aspergillosis in a 17 year old boy with acute lymphocytic leukaemia. Cerebral involvement by the fungal infection was suspected clinically but was not biopsy proven. Although the patient subsequently died following bone marrow transplant and Escherichia coli septicaemia there was no evidence of residual Aspergillus at autopsy. [\hyperlink{Itraconazole}{PMID: 8387801}, L Moore et al., 1993]

\hypertarget{pmid_23278514}{B}ecause of the low prevalence of onychomycosis in children, little is known about the efficacy and safety of systemic antifungals in this population. PubMed and Embase databases and the references of related publications were searched in March 2012 for clinical trials (CTs), retrospective analyses (RAs), and case reports (CRs) on the use of systemic antifungals for onychomycosis in children (<18 years). Twenty-six studies (5 CTs, 3 RAs, and 18 CRs) were published between 1976 and 2011. Most of these studies reported the use of systemic terbinafine and itraconazole for the treatment of onychomycosis in children. Therapy with systemic antifungals alone in children ages 1 to 17 years resulted in a complete cure rate of 70.8\% (n = 151), whereas combined systemic and topical antifungal therapy in one infant and 19 children age 8 and older resulted in a complete cure rate of 80.0\% (n = 20). The efficacy and safety profiles of terbinafine, itraconazole, griseofulvin, and fluconazole in children were similar to those previously reported for adults. In conclusion, based on the little information available on onychomycosis in children, systemic antifungal therapies in children are safe and cure rates are similar to the rates achieved in adults. [\hyperlink{Itraconazole}{PMID: 23278514}, Aditya K Gupta et al., ]

\hypertarget{pmid_34002355}{T}riazoles represent an important class of antifungal drugs in the prophylaxis and treatment of invasive fungal disease in pediatric patients. Understanding the pharmacokinetics of triazoles in children is crucial to providing optimal care for this vulnerable population. While the pharmacokinetics is extensively studied in adult populations, knowledge on pharmacokinetics of triazoles in children is limited. New data are still emerging despite drugs already going off patent. This review aims to provide readers with the most current knowledge on the pharmacokinetics of the triazoles: fluconazole, itraconazole, voriconazole, posaconazole, and isavuconazole. In addition, factors that have to be taken into account to select the optimal dose are summarized and knowledge gaps are identified that require further research. We hope it will provide clinicians guidance to optimally deploy these drugs in the setting of a life-threatening disease in pediatric patients. [\hyperlink{Itraconazole}{PMID: 34002355}, Didi Bury et al., 2021]

\hypertarget{pmid_8381643}{I}traconazole is a new orally active triazole antifungal agent with enhanced activity against Candida species. In the clinical trial described in this paper, we compared the efficacy and safety of itraconazole capsules with those of clotrimazole vaginal tablets and placebo oral capsules for women with acute vulvovaginal candidiasis. Ninety-five patients were randomized in a 2:1:1 fashion to receive itraconazole (200 mg/day), clotrimazole (200 mg/day), or placebo (two capsules per day) for 3 consecutive days. Clinical success rates (cure and improvement) were similar for women who received itraconazole (96\%) and clotrimazole (100\%) 1 week posttreatment. These response rates were statistically superior to those obtained with placebo treatment (77\%, P < 0.05). Negative mycological cultures were found in 95, 73, and 32\% of the patients treated with clotrimazole, itraconazole, and placebo, respectively (P < 0.005) [active treatments versus placebo]). By 4 weeks posttreatment, the clinical failure rate for itraconazole was less than that observed for clotrimazole (17 versus 30\%), but this difference did not reach statistical significance (P > 0.05; beta = 0.81). Mycological response rates for itraconazole and clotrimazole were also similar. No patients enrolled in this study discontinued treatment because of an adverse event. Minor side effects were reported by 35, 4, and 41\% of patients who received itraconazole, clotrimazole, and placebo, respectively. The most common side effects associated with itraconazole therapy were nausea and headache. In summary, itraconazole was found to be as effective and safe as clotrimazole in women with acute candida vaginitis. Moreover, oral therapy was highly favored over intravaginal treatment in our survey of patients. [\hyperlink{Itraconazole}{PMID: 8381643}, G E Stein et al., 1993]

\hypertarget{pmid_10578159}{T}his was an open study of oral antifungal prophylaxis in 103 neutropenic children aged 0-14 (median 5) years. Most (90\%) were undergoing transplantation for haematological conditions (77\% allogeneic BMT, 7\% autologous BMT, 6\% PBSC transplants and 10\% chemotherapy alone). They received 5.0 mg/kg itraconazole/day (in 10 mg/ml cyclodextrin solution). Where possible, prophylaxis was started at least 7 days before the onset of neutropenia and continued until neutrophil recovery. Of the 103 who entered the study, 47 completed the course of prophylaxis, 27 withdrew because of poor compliance, 19 because of adverse events and 10 for other reasons. Two patients died during the study and another five died within the subsequent 30 days. No proven systemic fungal infections occurred, but 26 patients received i.v. amphotericin for antibiotic-unresponsive pyrexia. One patient received amphotericin for mycologically confirmed oesophageal candidosis. Three patients developed suspected oral candidosis but none was mycologically proven and no treatment was given. Serious adverse events (other than death) occurred in 21 patients, including convulsions (7), suspected drug interactions (6), abdominal pain (4) and constipation (4). The most common adverse events considered definitely or possibly related to itraconazole were vomiting (12), abnormal liver function (5) and abdominal pain (3). Tolerability of study medication at end-point was rated as good (55\%), moderate (11\%), poor (17\%) or unacceptable (17\%). Some patients had poor oral intakes due to mucositis. No unexpected problems of safety or tolerability were encountered. We conclude that itraconazole oral solution may be used as antifungal prophylaxis for neutropenic children. [\hyperlink{Itraconazole}{PMID: 10578159}, A B Foot et al., 1999]

\hypertarget{pmid_1319313}{I}traconazole is a lipophilic triazole with potent in vitro activity. It is also effective after topical, oral and parenteral administration. The antifungal activity of itraconazole has been evaluated against more than 6,500 different strains, belonging to more than 260 fungal species, using the serial decimal dilution test in fluid broth medium (brain-heart infusion broth). Candida spp., Torulopsis spp., Cryptococcus neoformans, Pityrosporum spp. (Dixon broth), various other yeasts, dermatophytes, Aspergillus spp., Penicillium spp., Sporothrix schenckii, dimorphic fungi (mycelium phase and yeast phase), Phaeohyphomycetes, Entomophthorales and various Hyalohyphomycetes are sensitive. Most strains of Fusarium and Zygomycetes are poorly sensitive. Itraconazole was administered orally and parenterally in normal and immunocompromised guinea-pigs infected with C. albicans, Cr. neoformans, Histoplasma duboisii, S. schenckii, P. marneffei and A. fumigatus. It was effective in terms of both survival of the animals and elimination of the fungi from the various tissues. Itraconazole was superior to fluconazole in candidosis, cryptococcosis, sporotrichosis and aspergillosis, and to amphotericin B and to flucytosine in candidosis, cryptococcosis and aspergillosis. No comparative studies have yet been undertaken for other deep mycoses. The results of combination therapy with itraconazole and fluconazole in cryptococcosis were indifferent; with flucytosine or amphotericin B, additive or synergistic effects were seen in systemic candidosis, cryptococcosis and aspergillosis. No drug-related side-effects were observed after oral or parenteral administration of itraconazole. [\hyperlink{Itraconazole}{PMID: 1319313}, J Van Cutsem et al., 1992]

\hypertarget{pmid_15283801}{M}ycotic scalp infection caused by Microsporum canis is one of the more recalcitrant disorders, with increasing incidence during the last decade. We report our experience with administration of itraconazole in 163 children (86 girls, 77 boys) with M. canis tinea capitis. Fifty-five patients had previous treatment with terbinafine without success. In all children, the dosage of itraconazole was adjusted according to body weight, with 5 mg/kg/day given in a continuous regimen either as a capsule (116 patients) or an oral suspension (47 patients). In all children, there was both clinical and mycologic cure after a mean treatment period of 39 +/- 12 days (range 10-77 days). Eleven children (6.7\%) had side effects: diarrhea in five children, cutaneous eruption in four, and abdominal pain in two. Itraconazole was effective and safe for the treatment of M. canis tinea capitis. [\hyperlink{Itraconazole}{PMID: 15283801}, Gabriele Ginter-Hanselmayer et al., ]

\hypertarget{pmid_9144703}{W}e report on two children affected by chronic mucocutaneous candidiasis involving the mouth and all the nails who were successfully treated with itraconazole at 200 mg/day for 2 months. This therapy produced a rapid cure of both candidal nail and mouth infections. The drug was very well tolerated, and routine laboratory monitoring during treatment did not reveal any abnormalities. [\hyperlink{Itraconazole}{PMID: 9144703}, A Tosti et al., ]

\hypertarget{pmid_25680318}{P}ediatric patients with hemato-oncological malignancies and neutropenia resulting from chemotherapy have a high risk of acquiring invasive fungal infections. Oral antifungal prophylaxis with azoles, such as fluconazole or itraconazole, is preferentially used in pediatric patients after chemotherapy. During this retrospective analysis, posaconazole was administered based on favorable results from studies in adult patients with neutropenia and after allogeneic hematopoietic stem cell transplantation. Retrospectively, safety, feasibility, and initial data on the efficacy of posaconazole were compared to fluconazole and itraconazole in pediatric and adolescent patients during neutropenia. Ninety-three pediatric patients with hemato-oncological malignancies with a median age of 12 years (range 9 months to 17.7 years) that had prolonged neutropenia (>5 days) after chemotherapy or due to their underlying disease, and who received fluconazole, itraconazole, or posaconazole as antifungal prophylaxis, were analyzed in this retrospective single-center survey. The incidence of invasive fungal infections in pediatric patients was low under each of the azoles. One case of proven aspergillosis occurred in each group. In addition, there were a few cases of possible invasive fungal infection under fluconazole (n = 1) and itraconazole (n = 2). However, no such cases were observed under posaconazole. The rates of potentially clinical drug-related adverse events were higher in the fluconazole (n = 4) and itraconazole (n = 5) groups compared to patients receiving posaconazole (n = 3). Posaconazole, fluconazole, and itraconazole are comparably effective in preventing invasive fungal infections in pediatric patients. Defining dose recommendations in these patients requires larger studies. [\hyperlink{Itraconazole}{PMID: 25680318}, M Döring et al., 2015]

\hypertarget{pmid_28744925}{I}traconazole is a first-generation triazole agent with an extended spectrum of activity; it is licensed in adults for superficial and systemic fungal infections; no recommendation has been yet established for use in children patients. Its variable and unpredictable oral bioavailability make it difficult to determine the optimal dosing regimen. Hence, therapeutic drug monitoring, highly available in clinical practice, may improve itraconazole treatment success and safety. The aim of the study was to describe in paediatrics the oral itraconazole pharmacokinetics, used for prophylaxis. Moreover, we evaluated the utility of its therapeutic drug monitoring in this cohort. A fully validated chromatographic method was used to quantify itraconazole concentration in plasma collected from paediatric patients, at the end of dosing interval. Associations between variables were tested using the Pearson test. Mann-Whitney U test has been used to probe the influence of categorical variables on continuous ones. Any predictive power of the considered variables was finally evaluated through univariate and multivariate linear and logistic regression analyses. A high inter-individual variability was shown; ethnicity (beta coefficient, β -0.161 and interval of confidence at 95\%, IC -395.035; -62.383) and gender (β 0.123 and IC 9.590; 349.395) remained in the final linear regression model with P value of .007 and .038, respectively. This study highlights that therapeutic drug monitoring is required to achieve an adequate target itraconazole serum exposure. [\hyperlink{Itraconazole}{PMID: 28744925}, Sarah Allegra et al., 2017]

\section*{Linzess}
\subsection*{Result}
\subsubsection*{Answer}

Unknown

\subsubsection*{{Explanation}}
\hypertarget{Linzess}
Based on the abstracts provided, there is no information available regarding Linzess (linaclotide) and its safety in children. All the relevant abstracts are about linezolid, an antibiotic, and not Linzess, which is a medication used for irritable bowel syndrome and chronic idiopathic constipation. Therefore, there are no targeted studies in the abstracts about the safety of Linzess in any pediatric age group. As such, the safety of Linzess for use in children is unknown based on the abstracts available.

\subsection*{Abstracts}
\hypertarget{pmid_11144380}{L}inezolid is an oxazolidinone antibiotic with excellent in vitro activity against a number of Gram-positive organisms including antibiotic-resistant isolates. The safety and pharmacokinetics of intravenously administered linezolid were evaluated in children and adolescents to examine the potential for developmental dependence on its disposition characteristics. Fifty-eight children (3 months to 16 years old) participated in this study; 44 received a single 1.5-mg/kg dose and 14 received a single 10-mg/kg dose of linezolid administered by intravenous infusion. Repeated blood samples (n = 10 in children > or = 12 months; n = 8 in children 3 to 12 months) were obtained during 24 h after drug administration, and linezolid was quantitated from plasma by high performance liquid chromatography with mass spectrometry detection. Plasma concentration vs. time data were evaluated with a model independent approach. Linezolid was well-tolerated by all subjects. The disposition of linezolid appears to be age-dependent. A significant although weak correlation between age and total body clearance was observed. The mean (+/- SD) values for elimination half-life, total clearance and apparent volume of distribution were 3.0 +/- 1.1 h, 0.34 +/- 0.15 liter/h/kg and 0.73 +/- 0.18 liter/kg, respectively. Estimates of total body clearance and volume of distribution were significantly greater in children than historical values of adult data. As such maximum achievable linezolid plasma concentrations were slightly lower in children, and concentrations 12 h after a single 10-mg/kg dose were below the MIC90 for selected pathogens with in vitro susceptibility to the drug. Based on these data a linezolid dose of 10 mg/kg given two to three times daily would appear appropriate for use in pediatric therapeutic clinical trials of this agent. [\hyperlink{Linzess}{PMID: 11144380}, G L Kearns et al., 2000]

\hypertarget{pmid_24706161}{L}inezolid is an oxazolidinone antibacterial agent, with activity against Gram-positive bacteria. This study aimed to evaluate the efficacy and safety of linezolid in children with infections caused by Gram-positive pathogens. A systematic search was conducted by two independent reviewers to identify published studies up to September 2013. The accumulated relevant literature was subsequently systematically reviewed, and a meta-analysis was conducted. Eligible studies were randomized controlled trials assessing the clinical efficacy and safety of linezolid in children versus other antimicrobial agents for infections caused by Gram-positive bacteria. The primary outcome was treatment success in patients who received at least one dose of study drug, had clinical evidence of disease, and had complete follow-up. Meta-analysis was conducted with random effects models because of heterogeneity across the trials. Two randomized controlled trials (RCTs), involving 815 patients, were included. Linezolid was slightly more effective than control antibiotic agents, but the difference was not statistically significant [odds ratio (OR) = 1.39, 95 \% confidence interval (CI) 0.98-1.98]. Treatment with linezolid was not associated with more adverse effects in general (OR = 0.61, 95 \% CI 0.25-1.48). Eradication efficiency did not differ between linezolid and control regimens, but the sample size for these comparisons was small. The use of linezolid cannot be steadily supported from the results of the current meta-analysis. It appears to be slightly more effective than control antibiotic agents, but the difference was not significant, and the serious limitations present in this study restrict its use. Further studies providing evidence for clinical and microbiological efficacy of linezolid will support its use. [\hyperlink{Linzess}{PMID: 24706161}, Maria Ioannidou et al., 2014]

\hypertarget{pmid_30642929}{L}inezolid is a synthetic antibiotic very effective in the treatment of infections caused by Gram-positive pathogens. Although the clinical application of linezolid in children has increased progressively, data on linezolid pharmacokinetics in pediatric patients are very limited. The aim of this study was to develop a population pharmacokinetic model for linezolid in children and optimize the dosing strategy in order to improve therapeutic efficacy. We performed a prospective pharmacokinetic study of pediatric patients aged 0 to 12 years. The population pharmacokinetic model was developed using the NONMEM program. Goodness-of-fit plots, nonparametric bootstrap analysis, normalized prediction distribution errors, and a visual predictive check were employed to evaluate the final model. The dosing regimen was optimized based on the final model. The pharmacokinetic data from 112 pediatric patients ages 0.03 to 11.9 years were analyzed. The pharmacokinetics could best be described by a one-compartment model with first-order elimination along with body weight and the estimated glomerular filtration rate as significant covariates. Simulations demonstrated that the currently approved dosage of 10 mg/kg of body weight every 8 h (q8h) would lead to a high risk of underdosing for children in the presence of bacteria with MICs of ≥2 mg/liter. To reach the pharmacokinetic target, an elevated dosage of 15 or 20 mg/kg q8h may be required for them. The population pharmacokinetics of linezolid were characterized in pediatric patients, and simulations provide an evidence-based approach for linezolid dosage individualization. [\hyperlink{Linzess}{PMID: 30642929}, Si-Chan Li et al., 2019]

\hypertarget{pmid_22460827}{L}inezolid, an oxazolidinone antibiotic, exhibits a broad spectrum of activity against Gram-positive bacteria. It has been licensed for adult use in Japan since 2006 for MRSA infections, and has also been used off-label for pediatric patients. At our university hospital, a total of 16 infants and children (including one non-Japanese Asian) were administered linezolid owing to infection with multidrug-resistant Gram-positive bacteria, after consent had been provided. All patients had severe underlying diseases or indications for surgery. Eighty-eight percent of the causal microorganisms were methicillin-resistant Staphylococcus aureus (MRSA) or methicillin-resistant coagulase-negative Staphylococcus and all were sensitive to linezolid. Linezolid was administered because the antecedent anti-MRSA medications were ineffective or contraindicated, or intravenous-to-oral switch therapy was requested owing to cardiac or orthopedic surgical-site infections. The median duration of administration was 13 days (range 3-31 days). The overall efficacy was 91 \% (10/11) in those for whom efficacy could be evaluated. Only two patients (both teen-aged) encountered linezolid-related adverse effects (13 \%, 2/16). One patient showed elevation of liver enzymes (aspartate aminotransferase [AST] and alanine aminotransferase [ALT]), requiring that administration be withdrawn, but enzyme levels returned to normal after the patient had been switched to vancomycin. The other patient showed transiently decreased platelet counts. Linezolid is considered generally safe and effective for children in Japan, especially for those who cannot use other anti-MRSA medications or those who require oral antibiotics for infections with multidrug-resistant Gram-positive bacteria. [\hyperlink{Linzess}{PMID: 22460827}, Masayoshi Shinjoh et al., 2012]

\hypertarget{pmid_17984803}{T}he excellent oral bioavailability and the Gram-positive antimicrobial spectrum make linezolid an attractive antibiotic for treatment of osteoarticular infections. The clinical efficacy of this drug has not been previously evaluated for Gram-positive osteoarticular infections in children. Between July 2003 and June 2006, 13 children who received a linezolid-containing regimen for osteoarticular infections were identified from a hospital pharmacy database. The medical records were reviewed and outcomes with regard to clinical efficacy and safety were analyzed. Eight (61.5\%) children were male. Ages ranged from 3 months to 14 years. Nine previously healthy children had acute hematogenous osteoarticular infections involving the pelvis (n = 1) or lower limbs (n = 8). The remaining 4 children had postoperative infections of sternal wounds (n = 2) and fractured lower limbs (n = 2). Causative pathogens included methicillin-resistant Staphylococcus aureus in 11 children, methicillin-susceptible S. aureus in one, and Enterococcus faecium and coagulase-negative staphylococci in one. Surgical debridement was attempted in 9 children and effective antistaphylococcal antibiotics were used in all 13 patients for a median duration of 23 days (range, 5-41 days) before the use of linezolid. Linezolid was administered orally to 10 children as step-down therapy and by the parenteral followed by oral route to 3 children who were intolerant of glycopeptide for a median duration of 20 days (range, 9-36 days). Eleven of the 13 children were cured after management. Two children developed anemia during linezolid therapy. There was no premature cessation of linezolid because of severe adverse effects. Linezolid appears to be useful and well tolerated in step-down therapy or compassionate use for pediatric Gram-positive orthopedic infections. A well-designed prospective comparative study is needed to confirm this observation. [\hyperlink{Linzess}{PMID: 17984803}, Chih-Jung Chen et al., 2007]

\hypertarget{pmid_32938279}{T}o describe safety and feasibility of long-term inhalative sedation (LTIS) in children with severe respiratory diseases compared to patients with normal lung function with respect to recent studies that showed beneficial effects in adult patients with acute respiratory distress syndrome (ARDS). Single-center retrospective study. 12-bed pediatric intensive care unit (PICU) in a tertiary-care academic medical center in Germany. All patients treated in our PICU with LTIS using the AnaConDa® device between July 2011 and July 2019. Thirty-seven courses of LTIS in 29 patients were analyzed. LTIS was feasible in both groups, but concomitant intravenous sedatives could be reduced more rapidly in children with lung diseases. Cardiocirculatory depression requiring vasopressors was observed in all patients. However, severe side effects only rarely occured. In this largest cohort of children treated with LTIS reported so far, LTIS was feasible even in children with severely impaired lung function. From our data, a prospective trial on the use of LTIS in children with ARDS seems justified. However, a thorough monitoring of cardiocirculatory side effects is mandatory. [\hyperlink{Linzess}{PMID: 32938279}, Jochen Meyburg et al., 2021]

\hypertarget{pmid_36183690}{T}he use of linezolid is relatively safe for all age categories, including premature infants. The case of an extremely premature infant with hyperglycemia and lactic acidosis associated with linezolid is reported. A 350-g male infant was born at 24 weeks by cesarean section. His Apgar scores were 1 and 1 at 1 and 5 min, respectively. On the day of life (DOL) 7, linezolid was started at a dose of 10 mg/kg/dose every 8 h for a catheter-related blood stream infection caused by methicillin-resistant coagulase-negative Staphylococci. After linezolid was given, serum lactate and glucose levels increased gradually. After discontinuation of linezolid on DOL 16, hyperglycemia and lactic acidosis improved immediately. In conclusion, a rare case of an extremely premature infant with hyperglycemia and lactic acidosis associated with linezolid was reported. It is crucial to monitor glucose levels along with lactate and pH levels during linezolid therapy. [\hyperlink{Linzess}{PMID: 36183690}, Takafumi Asai et al., 2022]

\hypertarget{pmid_11241029}{T}he objectives were to evaluate appropriate doses of zinc acetate and its efficacy for the maintenance management of Wilson's disease in pediatric cases. Pediatric patients of 1 to 5 years of age were given 25 mg of zinc twice daily; patients of 6 to 15 years of age, if under 125 pounds body weight, were given 25 mg of zinc three times daily; and patients 16 years of age or older were given 50 mg of zinc three times daily. Patients were followed for efficacy (or over-treatment) until their 19th birthday by measuring levels of urine and plasma copper, urine and plasma zinc and through liver function tests and quantitative speech and neurologic scores. Patients were followed for toxicity by measures of blood counts, blood biochemistries, urinalysis, and clinical follow-up. Thirty-four patients, ranging in ages from 3.2 to 17.7 years of age, were included in the study. All doses met efficacy objectives of copper control, zinc levels, neurologic improvement, and maintenance of liver function except for episodes of poor compliance. No instance of over-treatment was encountered. Four patients exhibited mild and transient gastric disturbance from the zinc. Zinc therapy in pediatric patients appears to have a mildly adverse effect on the high-density lipoprotein/total cholesterol ratio, contrary to results of an earlier large study of primarily adults. In conclusion, zinc is effective and safe for the maintenance management of pediatric cases of Wilson's disease. Our data are strongest in children above 10 years of age. More work needs to be done in very young children, and the cholesterol observations need to be studied further. [\hyperlink{Linzess}{PMID: 11241029}, G J Brewer et al., 2001]

\hypertarget{pmid_31593796}{L}inezolid (LNZ) has recently been listed by the World Health Organization (WHO) as a Group A agent for the treatment of multidrug-resistant tuberculosis (MDR-TB) and extensively drug-resistant tuberculosis (XDR-TB) in longer regimens (18-20 months). However, little is known about the safety of LNZ in longer TB treatment regimens in children. Here we report 31 children who received LNZ treatment for drug-resistant tuberculosis (DR-TB) and extensive tuberculosis in the Children's Hospital of Chongqing Medical University, China, during September 2016 to March 2019. The mean duration of LNZ treatment was 8.56 months (range, 1-24 months). Of the 31 patients, 13 (42\%) had suspected or confirmed adverse events (AEs) related to LNZ treatment, including digestive symptoms, haematological toxicity, neuropathy and lactic acidosis. Haematological toxicity was the most frequent AE, presenting as leukopenia (9/13) and anaemia (5/13). No hepatotoxicity or nephrotoxicity was observed. Two patients suffered from life-threatening lactic acidosis when the LNZ dose was increased to 1.2 g daily, however they recovered following LNZ withdrawal. A high rate of AEs of LNZ treatment was observed in children receiving a longer regimen, which might relate to the treatment course and dose. Haematological toxicity was the most frequent AE in children. It is necessary to regularly monitor the blood chemistry and lactic acid concentration during LNZ treatment. [\hyperlink{Linzess}{PMID: 31593796}, ZhenZhen Zhang et al., 2020]

\hypertarget{pmid_21521704}{T}he worldwide spread of multidrug-resistant organisms has required the development of new antimicrobials. Linezolid, the first oxazolidinone, has a broad spectrum of activity against Gram-positive bacteria, including resistant strains. Although approved by the Food and Drug Administration in 2002, the clinical experience with linezolid in the paediatric population is still limited, also given the fact that in most European countries the paediatric use of linezolid is off-label. In this paper we summarize the actual evidence on both licensed and off-label clinical uses of linezolid in children, including efficacy, safety and tolerability issues. Taking into account the potential bias in comparing heterogeneous clinical trials and reports, the available literature data suggest that linezolid is a safe and effective agent for the treatment of serious Gram-positive bacterial infections in neonates and children. At present, linezolid is reserved for those children who are intolerant to or fail conventional agents. A linezolid-containing regimen can be a valuable option for treating multidrug-resistant and extensively drug-resistant tuberculosis in children as well as disseminated non-tuberculous mycobacterial infections. Given the rare occurrence of serious side effects, careful monitoring of haematological parameters, possible drug interactions and neurological manifestations is recommended in linezolid-treated children, especially in case of prolonged treatments. Appropriate linezolid dosage and hospital infection control measures are essential to avoid the spread of linezolid resistance. Further studies are needed to establish novel paediatric indications for linezolid use and to assess the tolerability of long-term treatments. [\hyperlink{Linzess}{PMID: 21521704}, Silvia Garazzino et al., 2011]

\hypertarget{pmid_8169182}{T}here is evidence for the efficacy and safety of clonazepam (CZP) in adult anxiety disorders, but no formal studies to substantiate clinical reports of similar benefit in children with anxiety disorders. In this double-blind pilot study, 15 children, aged 7 to 13 years, entered a randomly assigned, double-blind crossover trial of 4 weeks of CZP (up to 2 mg/day) and 4 weeks of placebo. Twelve children completed the trial. All but 1 had a diagnosis of separation anxiety disorder, and all but 2 had comorbid diagnoses. Nine children appeared to have moderate to significant clinical improvement, but statistical comparisons on several ratings failed to confirm a trend in favor of CZP. Side effects of drowsiness, irritability, and/or oppositional behavior were notable in 10 children in the CZP phase compared with 5 in the placebo phase. Clonazepam was believed to have clinical benefit for some children, but this was not confirmed statistically in this small sample. Problematic side effects of drowsiness and disinhibition were common and possibly were due to rapid titration. [\hyperlink{Linzess}{PMID: 8169182}, F Graae et al., ]

\hypertarget{pmid_21030365}{T}o evaluate the safety and efficacy of a sedation protocol based on intranasal lidocaine spray and midazolam (INM) in children who are anxious and uncooperative when undergoing minor painful or diagnostic procedures, such as peripheral line insertion, venipuncture, intramuscular injection, echocardiogram, CT scan, audiometry testing and dental examination and extractions. 46 children, aged 5-50 months, received INM (0.5 mg/kg) via a mucosal atomiser device. To avoid any nasal discomfort a puff of lidocaine spray (10 mg/puff) was administered before INM. The child's degree of sedation was scored using a modified Ramsay sedation scale. A questionnaire was designed to evaluate the parents' and doctors' opinions on the efficacy of the sedation. Statistical analysis was used to compare sedation times with children's age and weight. The degree of sedation achieved by INM enabled all procedures to be completed without additional drugs. Premedication with lidocaine spray prevented any nasal discomfort related to the INM. The mean duration of sedation was 23.1 min. The depth of sedation was 1 on the modified Ramsay scale. The questionnaire revealed high levels of satisfaction by both doctors and parents. Sedation start and end times were significantly correlated with age only. No side effects were recorded in the cohort of children studied. This study has shown that the combined use of lidocaine spray and atomised INM appears to be a safe and effective method to achieve short-term sedation in children to facilitate medical care and procedures. [\hyperlink{Linzess}{PMID: 21030365}, Antonio Chiaretti et al., 2011]

\hypertarget{pmid_23904337}{I}n Sub-Saharan Africa, intrarectal diazepam is the first-line anticonvulsant mostly used in children. We aimed to assess this standard care against sublingual lorazepam, a medication potentially as effective and safe, but easier to administer. A randomized controlled trial was conducted in the pediatric emergency departments of 9 hospitals. A total of 436 children aged 5 months to 10 years with convulsions persisting for more than 5 minutes were assigned to receive intrarectal diazepam (0.5 mg/kg, n = 202) or sublingual lorazepam (0.1 mg/kg, n = 234). Sublingual lorazepam stopped seizures within 10 minutes of administration in 56\% of children compared with intrarectal diazepam in 79\% (P < .001). The probability of treatment failure is higher in case of sublingual lorazepam use (OR = 2.95, 95\% CI = 1.91-4.55). Sublingual lorazepam is less efficacious in stopping pediatric seizures than intrarectal diazepam, and intrarectal diazepam should thus be preferred as a first-line medication in this setting.  [\hyperlink{Linzess}{PMID: 23904337}, Célestin Kaputu Kalala Malu et al., 2014] Linezolid, an oxazolidinone antibacterial agent, is available for intravenous/oral administration, with activity against Gram-positive bacteria including methicillin-resistant Staphylococcus aureus (MRSA), vancomycin-resistant enterococci (VRE), and penicillin-resistant Streptococcus pneumoniae (PRSP). These pathogens are important causes of hospital- and community-associated infections in children. PubMed was searched for all English language articles on patients younger than 18 years of age treated with linezolid, and an analysis of these articles was performed. From the 133 articles retrieved, a total of 30 were studied (18 case reports, nine case series, and three clinical trials) based on the inclusion criteria preset for this review. In these articles, a total of 597 children received linezolid. MRSA was the most common pathogen, followed by VRE, PRSP, other bacteria and less common mycobacterial species. Linezolid was reported to be safe and effective for the treatment of pneumonia and endocarditis, as well as skin and soft tissue, central nervous system and osteoarticular infections. Linezolid is promising as a safe and efficacious agent for the treatment of infections due to mainly resistant Gram-positive organisms in children who are unable to tolerate conventional agents or after treatment failure. [\hyperlink{Linzess}{PMID: 23904337}, John Dotis et al., 2010]

\hypertarget{pmid_26901441}{L}inezolid serves as an important component for the treatment of drug-resistant tuberculosis although there is little published data about linezolid use in children, especially in childhood tuberculous meningitis (TBM). In this study, we retrospectively reviewed records of childhood TBM patients who started treatment between January 2012 and August 2014. A total of 86 childhood TBM patients younger than 15 years old were enrolled. Out of 86 children, 36 (41.9\%) received the regimen containing linezolid. Thirty-two (88.9\%) of 36 linezolid-treated cases had favorable outcomes, and 35 (70.0\%) cases were successfully treated in the control group. The frequency of favorable outcome of linezolid group was significantly higher than that of control group (P = 0.037). In addition, compared with cases with fever clearance time of <1 week, the control group had more cases with fever clearance time of 1-4 weeks (P = 0.010) and >4 weeks (P = 0.000) than linezolid group. Furthermore, there was no significant difference in the frequency of adverse events between the two regimens (P = 0.896). In addition, the patients with adverse events were more likely to have treatment failure, the P value of which was 0.008. Our data demonstrate that linezolid improves early outcome of childhood TBM. The low frequency of linezolid-associated adverse effects highlights the promising prospects of its use for treatment of childhood TBM. [\hyperlink{Linzess}{PMID: 26901441}, Huimin Li et al., 2016]

\hypertarget{pmid_20605418}{L}inezolid is an antibiotic of the oxazolidinone class that has bacteriostatic and bactericidal activity against a broad range of Gram-positive bacteria, including multiresistant pathogens. Owing to increasing resistance of Gram-positive pathogens to traditional antibiotics such as vancomycin, the oxazolidinones were introduced into therapy. The aim of this review was to summarise actual data on the pharmacokinetics, safety and clinical use of linezolid in preterm infants. The Medline and EMBASE databases were searched using the term 'linezolid' combined with 'newborn', 'neonate', 'preterm' and 'premature' for papers published between January 1987 and June 2009. Studies reporting on a population including preterm infants and other age groups as well as case reports on preterm infants only were acceptable for analysis. Five studies and eight case reports were identified evaluating linezolid in preterm infants. A dosage regimen of 10mg/kg body weight given either orally or intravenously every 8h in infants aged >or=1 week and the same dose given every 12h in infants <1 week was shown to be safe and effective with a mean treatment duration of 10-28 days. In summary, linezolid was shown to be a safe and effective alternative to vancomycin in the treatment of infections with multiresistant Gram-positive pathogens in preterm infants. [\hyperlink{Linzess}{PMID: 20605418}, S Kocher et al., 2010]

\hypertarget{pmid_17959327}{T}o evaluate the efficacy and safety of zonisamide (ZNS) as long-term adjunctive therapy in children with Lennox-Gastaut syndrome (LGS). We evaluated the seizure frequency, cognitive outcomes, and side effects of 62 LGS patients maintained on ZNS for at least 12 months in three tertiary centers. Of the 62 LGS patients maintained on ZNS, 3 (4.8\%) had 100\% seizure control; 14 (22.6\%) had >75\% to <100\% reduction in seizure frequency; 15 (24.2\%) had >50\% to <75\% reduction in seizure frequency; 6 (9.7\%) had >0\% to <50\% reduction in seizure frequency, and 24 (38.7\%) had no seizure reduction. Seizure outcomes were not related to seizure types or etiologies. Adverse events included somnolence and anorexia, but all were transient and successfully managed by careful follow-up. Our results indicate that adjunctive treatment with ZNS is safe and effective in pediatric LGS patients. [\hyperlink{Linzess}{PMID: 17959327}, Su Jeong You et al., 2008]

\hypertarget{pmid_25145624}{O}lanzapine is frequently prescribed in young children for psychiatric conditions. It may be an option for chemotherapy-induced nausea and vomiting (CINV) control in children. The objective of this review was to describe the safety of olanzapine in children less than 13 years of age to determine if safety concerns would be a barrier to its use for CINV prevention. Electronic searches were performed in MEDLINE, EMBASE, Cochrane Central Register of Controlled Trials, Web of Science and Scopus. All studies in English reporting adverse effects associated with olanzapine use in children younger than 13 years or with a mean/median age less than 13 years were included. Adverse outcomes were synthesized for prospective studies. A total of 47 studies (17 prospective) involving 387 children aged 0.6-18 years were included; nine described olanzapine poisonings. Weight gain or sedation were reported in 78 \% [95 \% confidence interval (CI) 63-95] and 48 \% (95 \% CI 35-67), respectively. Extrapyramidal symptoms or electrocardiogram abnormalities were reported in 9 \% (95 \% CI 4-21) and 14 \% (95 \% CI 7-26), respectively. Elevation in liver function tests or blood glucose abnormalities were reported in 7 \% (95 \% CI 2-20) and 4 \% (95 \% CI 1-17), respectively. No deaths were attributed to olanzapine. No studies were identified with a primary focus on evaluating safety, and the adverse effects reported in the included studies were heterogeneous. Most adverse events associated with olanzapine use in children less than 13 years of age are of minor clinical significance. These findings support the exploration of olanzapine for the prevention of CINV in children in future trials. [\hyperlink{Linzess}{PMID: 25145624}, Jacqueline Flank et al., 2014]

\hypertarget{pmid_37287398}{S}eizures are common in critically ill children and neonates, and these patients would benefit from intravenous (IV) antiseizure medications with few adverse effects. We aimed to assess the safety profile of IV lacosamide (LCM) among children and neonates. This retrospective multicenter cohort study examined the safety of IV LCM use in 686 children and 28 neonates who received care between January 2009 and February 2020. Adverse events (AEs) were attributed to LCM in only 1.5\% (10 of 686) of children, including rash (n = 3, .4\%), somnolence (n = 2, .3\%), and bradycardia, prolonged QT interval, pancreatitis, vomiting, and nystagmus (n = 1, .1\% each). There were no AEs attributed to LCM in the neonates. Across all 714 pediatric patients, treatment-emergent AEs occurring in >1\% of patients included rash, bradycardia, somnolence, tachycardia, vomiting, feeling agitated, cardiac arrest, tachyarrhythmia, low blood pressure, hypertension, decreased appetite, diarrhea, delirium, and gait disturbance. There were no reports of PR interval prolongation or severe cutaneous adverse reactions. When comparing children who received a recommended versus a higher than recommended initial dose of IV LCM, there was a twofold increase in the risk of rash in the higher dose cohort (adjusted incidence rate ratio = 2.11, 95\% confidence interval = 1.02-4.38). This large observational study provides novel evidence demonstrating the tolerability of IV LCM in children and neonates. [\hyperlink{Linzess}{PMID: 37287398}, Susan L Fong et al., 2023]

\hypertarget{pmid_31214452}{S}edation and analgesia using opioids and benzodiazepines is frequently required in critically ill children to minimize pain and anxiety. In some patients, difficult sedation occurs when tolerance or unacceptable side effects limit the efficacy of conventional analgo-sedative treatment. We describe seven patients (age range 1 to 17 yr) where difficult sedation was successfully managed with enteral levomepromazine (LMZ). LMZ is a neuroleptic antipsychotic agent that exhibits potent analgo-sedative properties without respiratory depression, through non-opioid and non-benzodiazepine pathways. We describe its use in our pediatric intensive care unit to control agitation in patients with known behavioral disorders who frequently pose a significant sedation challenge. We also illustrate its successful use in cases of withdrawal syndrome and delirium, and discuss the association of fever and its distinction from neuroleptic malignant syndrome in two patients. LMZ should be considered as a useful sedative in critically ill children where difficult sedation occurs and conventional agents are exhausted. [\hyperlink{Linzess}{PMID: 31214452}, Aarjan Snoek et al., 2014]

\hypertarget{pmid_1354907}{E}ven though acute poisonings with benzodiazepines are extremely common, less is known of the clinical toxicity of recent derivatives, particularly in children. 1,989 cases involving ethyle loflazepate, flunitrazepam, prazepam or triazolam recorded at the Lyons Poison Center and due to 1 compound and associated with clinical symptoms were selected for study. Children less than 16-y of age accounted for 482 cases. Sleepiness, agitation and ataxia were significantly more frequent in the children. Hypotonia was seldom observed but was indicative of severe poisoning. The dangerous toxic dose of these compounds in children is suggested to be 0.78-0.90 mg ethyle loflazepate/kg, 0.26-0.29 mg flunitrazepam/kg, 7.80-9.00 mg prazepam/kg and 0.06-0.07 mg triazolam/kg. These results are in keeping with the relatively low acute toxicity of the older benzodiazepines. [\hyperlink{Linzess}{PMID: 1354907}, C Pulce et al., 1992]

\hypertarget{pmid_37680615}{P}reoperative anxiety often prevails in children at higher levels than adults, which is a common impediment for surgeons and anesthesiologists. It is of great necessity to explore an appropriate medication to improve this situation. Remimazolam, a type of benzodiazepine drug, has been indicated for the induction and maintenance of procedural sedation in adults since 2020. To date, rare studies were reported to investigate the effect of remimazolam on children. In this study, we investigated the safety and efficacy of intranasal drops of remimazolam and tried to determine the 95\% effective dose (ED In this study, 114 children were enrolled who underwent laparoscopic high-level inguinal hernia ligation between January 2021 and December 2022 and were divided into an early childhood children group and a pre-school children group. The biased coin design (BCD) was used to determine the target doses. A positive response was defined as the effective relief of preoperative anxiety (modified Yale Preoperative Anxiety Scale, mYPAS < 30). The initial nasal dose of remimazolam was 0.5 mg·kg A total of 80 children completed the study, including 40 in the early childhood group and 40 in the pre-school children group. As statistical analysis indicated, the ED Remimazolam is an effective medication to relieve preoperative anxiety in children. Moreover, the ED [\hyperlink{Linzess}{PMID: 37680615}, Xiang Long et al., 2023] In sub-Saharan Africa, rectal diazepam or intramuscular paraldehyde are commonly used as first-line anticonvulsant agents in the emergency treatment of seizures in children. These treatments can be expensive and sometimes toxic. We aimed to assess a drug and delivery system that is potentially more effective, safer, and easier to administer than those presently in use. We did an open randomised trial in a paediatric emergency department of a tertiary hospital in Malawi. 160 children aged over 2 months with seizures persisting for more than 5 min were randomly assigned to receive either intranasal lorazepam (100 microg/kg, n=80) or intramuscular paraldehyde (0.2 mL/kg, n=80). The primary outcome measure was whether the presenting seizure stopped with one dose of assigned anticonvulsant agent within 10 min of administration. The primary analysis was by intention-to-treat. This study is registered with ClinicalTrials.gov, number NCT00116064. Intranasal lorazepam stopped convulsions within 10 min in 60 (75\%) episodes treated (absolute risk 0.75, 95\% CI 0.64-0.84), and intramuscular paraldehyde in 49 (61.3\%; absolute risk 0.61, 95\% CI 0.49-0.72). No clinically important cardiorespiratory events were seen in either group (95\% binomial exact CI 0-4.5\%), and all children finished the trial. Intranasal lorazepam is effective, safe, and provides a less invasive alternative to intramuscular paraldehyde in children with protracted convulsions. The ease of use of this drug makes it an attractive and preferable prehospital treatment option. [\hyperlink{Linzess}{PMID: 37680615}, Shafique Ahmad et al., 2006]

\hypertarget{pmid_29490769}{T}he safety of a novel intranasal formulation of azelastine hydrochloride (AZE) and fluticasone propionate (FP) has been established in adults and adolescents with allergic rhinitis but not in children <12 years old. To evaluate the safety and tolerability of an intranasal formulation of AZE and FP in children ages 4-11 years with allergic rhinitis. The study was a randomized, 3-month, parallel-group, open-label design. Qualified patients were randomized in a 3:1 ratio to AZE/FP (n = 304) or fluticasone propionate (FP) (n = 101), one spray per nostril twice daily, and to one of three age groups: ≥4 to <6 years, ≥6 to <9 years, and ≥9 to <12 years. Safety was assessed by child- or caregiver-reported adverse events, nasal examinations, vital signs, and laboratory assessments. The incidence of treatment-related adverse events (TRAEs) was low in both the AZE/FP (16\%) and FP-only (12\%) groups after 90 days' continuous use. Epistaxis was the most frequently reported TRAE in both groups (AZE/FP, 9\%; FP, 9\%), followed by headache (AZE/FP, 3\%; FP, 1\%). All other TRAEs in the AZE/FP group were reported by ≤1\% of the children. The majority of TRAEs were of mild intensity and resolved spontaneously. Results of nasal examinations showed an improvement over time in both groups, with no cases of mucosal ulceration or nasal septal perforation. There were no unusual or unexpected changes in laboratory parameters or vital signs. The intranasal formulation of AZE and FP was safe and well tolerated after 3 months' continuous use in children with allergic rhinitis.The study was registered on <ext-link xmlns:xlink="http://www.w3.org/1999/xlink" ext-link-type="uri" xlink:href="http://ClinicalTrials.gov">ClinicalTrials.gov</ext-link> (NCT01794741). [\hyperlink{Linzess}{PMID: 29490769}, William Berger et al., 2018]

\hypertarget{pmid_3369308}{T}he anticonvulsive effect of lidocaine was evaluated in 46 newborn infants with severe, recurrent seizures. Before the lidocaine all infants were being given phenobarbital, and 22 infants were also treated with diazepam. Different dosages of lidocaine were tested. A loading dose of 2 mg/kg followed by i.v. infusion of 6 mg/kg/hour was the most effective dosage and had an immediate anticonvulsive effect in 18 of 25 infants; within 30 min the same effect was attained in another five of the infants, with an overall seizure control in 92\% of the sample population. During the lidocaine treatment cerebral electrical activity was followed continuously with a cerebral function monitor (CFM), which also enabled evaluation of the treatment. No serious side effects on blood-pressure, heart-rate or cerebral electrical activity were registered. For newborn infants with severe recurrent seizures not responding to other drugs, lidocaine is an effective additional mode of treatment. [\hyperlink{Linzess}{PMID: 3369308}, L Hellström-Westas et al., 1988]

\section*{Litfulo}
\subsection*{Result}
\subsubsection*{Answer}

Unknown

\subsubsection*{{Explanation}}
\hypertarget{Litfulo}
A review of the available abstracts reveals that none of them are about Litfulo. Therefore, there is not enough data available in the abstracts provided to determine whether Litfulo is safe or unsafe for use in children of any age range. No targeted studies on the safety of Litfulo in children are present in the abstracts, and thus the safety of Litfulo in children is unknown based on the evidence at hand.

\subsection*{Abstracts}
\hypertarget{pmid_25891210}{T}he purpose of this study was to evaluate the efficacy and safety of LIVALO tablets (pitavastatin) in Japanese male children with heterozygous familial hypercholesterolemia (FH). A multicenter, randomized, double-blind, parallel study was conducted in 14 male children 10-15 years of age with heterozygous FH. Pitavastatin (1 mg/day or 2 mg/day) was administered orally for 52 weeks.The primary endpoint was the percent change in the LDL-cholesterol (LDL-C) concentrations from baseline to endpoint (repeated measures ANCOVA at Weeks 8 and 12). Secondary endpoints included the percentage of patients who achieved the target LDL-C concentration and percent changes in the levels of lipoprotein and lipid parameters at the visit performed at 52 weeks. The percent change in LDL-C from baseline (mean 258 mg/dL for all patients) to the endpoint was -27.3\% (95\%CI; -34.0, -20.5) and -34.3\% (95\%CI; -41.0, -27.5) in the patients receiving 1 mg and 2 mg of pitavastatin, respectively. Stable reductions in the total cholesterol (TC), non-HDL cholesterol (non-HDL-C), apolipoprotein B (Apo-B) and LDL-C levels and non-HDL-C/HDL-C and Apo-B/Apo-A1 ratios were observed up to 52 weeks in both groups. One patient in each dose group (14\%) reached the treatment target level of 130 mg/dL.Adverse events were observed in seven (100\%) patients receiving 1 mg and five (71\%) patients receiving 2 mg of pitavastatin, although none were considered related to the study treatment. One patient in the 1 mg group reported a musculoskeletal AE; however, it was attributed to recent excessive exercise. Pitavastatin significantly reduced the LDL-C levels and was well tolerated when administered at usual adult doses in 14 male children 10-15 years of age with heterozygous FH. Pitavastatin is a promising therapeutic agent for pediatric dyslipidemia with few safety concerns. [\hyperlink{Litfulo}{PMID: 25891210}, Mariko Harada-Shiba et al., 2016]

\hypertarget{pmid_22238470}{C}hildren have a lower response rate to antimonial drugs and higher elimination rate of antimony (Sb) than adults. Oral miltefosine has not been evaluated for pediatric cutaneous leishmaniasis. A randomized, noninferiority clinical trial with masked evaluation was conducted at 3 locations in Colombia where Leishmania panamensis and Leishmania guyanensis predominated. One hundred sixteen children aged 2-12 years with parasitologically confirmed cutaneous leishmaniasis were randomized to directly observed treatment with meglumine antimoniate (20 mg Sb/kg/d for 20 days; intramuscular) (n = 58) or miltefosine (1.8-2.5 mg/kg/d for 28 days; by mouth) (n = 58). Primary outcome was treatment failure at or before week 26 after initiation of treatment. Miltefosine was noninferior if the proportion of treatment failures was ≤15\% higher than achieved with meglumine antimoniate (1-sided test, α = .05). Ninety-five percent of children (111/116) completed follow-up evaluation. By intention-to-treat analysis, failure rate was 17.2\% (98\% confidence interval [CI], 5.7\%-28.7\%) for miltefosine and 31\% (98\% CI, 16.9\%-45.2\%) for meglumine antimoniate. The difference between treatment groups was 13.8\%, (98\% CI, -4.5\% to 32\%) (P = .04). Adverse events were mild for both treatments. Miltefosine is noninferior to meglumine antimoniate for treatment of pediatric cutaneous leishmaniasis caused by Leishmania (Viannia) species. Advantages of oral administration and low toxicity favor use of miltefosine in children. NCT00487253. [\hyperlink{Litfulo}{PMID: 22238470}, Luisa Consuelo Rubiano et al., 2012]

\hypertarget{pmid_16028153}{B}ecause of concerns about arthrotoxicity, fluoroquinolones are restricted for use in children. This study describes the safety and efficacy of gatifloxacin when used for treatment of children with recurrent acute otitis media (ROM) or acute otitis media (AOM) treatment failure (AOMTF). We performed an analysis of 867 children included in 4 clinical trials who had ROM and/or AOMTF and were treated with gatifloxacin (10 mg/kg once daily for 10 days). Gatifloxacin had adverse event rates that were similar overall to those of a comparator antibiotic (amoxicillin-clavulanate), except for increased diarrhea in children <2 years old receiving amoxicillin-clavulanate. There was no evidence of arthrotoxicity, hepatotoxicity, alteration of glucose homeostasis, or central nervous system toxicity acutely or during 1 year follow-up in any child. Regarding efficacy, in 2 noncomparative trials, the gatifloxacin cure rate of AOM was 89\% (95\% confidence interval [CI], 83\%-95\%) at the test of cure (TOC) visit, 3-10 days after completion of therapy. In 2 comparative trials of gatifloxacin versus amoxicillin-clavulanate, the efficacy of gatifloxacin was 88\% (95\% CI, 82\%-94\%). Gatifloxacin led to better clinical outcomes than amoxicillin-clavulanate for AOMTF (91\% vs. 81\%; P=.029), for AOMTF and age <2 years old (89\% vs. 69\%; P=.009), and for severe AOM in children <2 years old (90\% vs. 75\%; P=.012). Among children with AOMTF previously treated with amoxicillin-clavulanate or ceftriaxone injections, gatifloxacin cure rates were high (88\% and 75\%, respectively). Gatifloxacin appears to be safe for children, with no evidence of producing arthrotoxicity in 867 children exposed to the antibiotic when used as treatment for ROM and AOMTF. [\hyperlink{Litfulo}{PMID: 16028153}, Michael E Pichichero et al., 2005]

\hypertarget{pmid_14699453}{M}iltefosine has previously been shown to cure 97\% of cases of visceral leishmaniasis (VL) in Indian adults. Because approximately one-half of cases of VL occur in children, we evaluated use of the adult dosage of miltefosin (2.5 mg/kg per day for 28 days) in 80 Indian children (age, 2-11 years) with parasitologically confirmed infection in an open-label clinical trial. Clinical and parasitological parameters were reassessed at the end of treatment and 6 months later. One patient died of intercurrent pneumonia on day 6. The other 79 patients demonstrated no parasites after treatment, had marked clinical improvement, and were deemed initially cured. Three patients had relapse, and 1 patient was lost to follow-up. The final cure rate was 94\% for all enrolled patients and 95\% for evaluable patients. Side effects included mild-to-moderate vomiting or diarrhea (each in approximately 25\% of patients) and mild-to-moderate, transient elevations in the aspartate aminotransferase level during the early treatment phase (in 55\%). This trial indicates that miltefosine is as effective and well tolerated in Indian children with VL as in adults and that it can be recommended as the first choice for treatment of childhood VL in India. [\hyperlink{Litfulo}{PMID: 14699453}, Sujit K Bhattacharya et al., 2004]

\hypertarget{pmid_12792385}{M}iltefosine is the first oral drug with demonstrable success in treating visceral leishmaniasis in adults. Because approximately one-half of the visceral leishmaniasis patients worldwide are children, we performed a Phase I/II dose ranging study in the pediatric population in India. Thirty-nine (39) children (defined as < 12 years of age) with visceral leishmaniasis demonstrated by parasites in splenic aspirates, were treated with oral miltefosine daily for 28 days: 21 patients received 1.5 mg/kg/day (Group A); and 18 patients received 2.5 mg/kg/day (Group B). About one-half of these children had failed prior antileishmanial treatment. All patients were parasitologically negative and symptomatically improved by the end of therapy on Day 28 of therapy; the initial parasitologic cure rate was 100\%. Two patients in each treatment group relapsed with fever, splenomegaly and parasite-positive splenic aspirates by the end of the 6-month follow-up. The per protocol final clinical cure rate was 19 of 21 = 90\% in Group A and 15 of 17 = 88\% in Group B. Miltefosine was well-tolerated. As per the adult experience, gastrointestinal adverse events were seen: 33 and 39\% of children experienced vomiting and 5 and 17\% experienced diarrhea in Groups A and B, respectively, but all episodes were mild to moderate in severity and commonly lasted <1 day without symptomatic treatment. Oral miltefosine was safe and approximately 90\% effective in this initial clinical trial of childhood visceral leishmaniasis. [\hyperlink{Litfulo}{PMID: 12792385}, Shyam Sundar et al., 2003]

\hypertarget{pmid_11916787}{I}n this randomized, double-blinded, placebo-controlled study, we evaluated the safety and efficacy of lidocaine iontophoresis for the prevention of pain associated with venipuncture in 59 children aged 6-17 yr. Children received either lidocaine HCl 2\% with epinephrine 1:100,000 (Active) or the same formulation without lidocaine (Placebo) via a 20 mA/min iontophoretic treatment. Pain during venipuncture was assessed by the subject, parent, and nurse using a 100-mm visual analog scale. Median (interquartile range) visual analog scale scores were significantly lower in the Active versus Placebo groups: subject, 7.0 (2.0-20.8) versus 31.0 (12.0-48.0), P < 0.001; nurse, 5.0 (2.2-10.8) versus 24.0 (9.0-47.0), P < 0.001; and parent, 3.0 (0.8-7.2) versus 20.0 (4.5-43.0), P < 0.002, respectively. Similarly, higher median satisfaction scores were given to the Active versus Placebo group by the three evaluators. Of the 59 subjects completing the study, 10 subjects experienced a total of 12 adverse events that were all graded as mild. In conclusion, lidocaine iontophoresis is safe in children, reduces discomfort associated with venipuncture, and increases satisfaction when compared with the placebo. In this randomized, double-blinded, placebo-controlled study, we found that dermal anesthesia with lidocaine HCl 2\% combined with epinephrine 1:100,000 administered via iontophoresis in children is achieved in 8.8 +/- 2.1 min, reduces discomfort associated with venipuncture, is safe, and increases satisfaction when compared with the placebo. [\hyperlink{Litfulo}{PMID: 11916787}, John B Rose et al., 2002]

\hypertarget{pmid_15690910}{T}he efficacy of the fluoroquinolone levofloxacin in the treatment of 35 children with bronchopulmonary disease exacerbation was practically the same as that of amoxycillin/clavulanate, cefotaxime or ceftriaxone. The clinical and bacteriological results were favourable. The eradication of the pathogens responsible for the bronchopulmonary inflammations in 86\% of the patients was stated. There is no doubt that fluoroquinolones should not be widely used in pediatrics. They should be considered as reserve drugs for the treatment of severe cases when the routine agents fail. Their use is justified when the situation is risky and the data on the pathogen susceptibility to the drugs are available. Still, levofloxacin is the most safe fluoroquinolone with low hepatotoxicity and lower effect on the central nervous system. The episodes of its negative cardiovascular action are less frequent. Moreover, the most frequent side effects of fluoroquinolones such as nausea, diarrhea or vomiting are less frequent with the use of levofloxacin. No signs of arthropathy in the patients treated with levofloxacin were observed in our trial. [\hyperlink{Litfulo}{PMID: 15690910}, I K Volkov et al., 2004]

\hypertarget{pmid_8132376}{L}ike all fluoroquinolones, ciprofloxacin causes articular damage in juvenile animals. Consequently, this drug was not recommended for children or pregnant women. However, due to its antibacterial effectiveness and convenience of oral administration, ciprofloxacin is now increasingly used for the treatment of certain infectious conditions in children and adolescents aged less than 18 years. In this paper the published literature on this subject is reviewed. Up to now, data are available on more than 1,500 paediatric patients who were given ciprofloxacin, two-thirds of whom were suffering from acute infectious bronchopulmonary exacerbations of cystic fibrosis, mainly due to Pseudomonas aeruginosa. The effectiveness of oral ciprofloxacin for this indication compared well to that of standard intravenous combination regimens. The majority of the remaining published trials was conducted in children with multiresistant typhoid fever; the administration of ciprofloxacin was successful in up to 100\% of the cases. The safety profile of ciprofloxacin in children and adolescents was very similar to that observed in adult patients. Adverse events were noted in 5-15\%, with gastrointestinal, skin and central nervous system reactions being the most common. Reversible arthralgia occurred in 36 out of 1,113 patients with cystic fibrosis, and in no case could cartilage damage be demonstrated by radiographic procedures. Thus, publication data clearly suggest that the administration of ciprofloxacin to children is effective and safe, but there is a need for further prospective, well-controlled clinical trials. [\hyperlink{Litfulo}{PMID: 8132376}, R Kubin et al., ]

\hypertarget{pmid_25861908}{I}n 2012, tenofovir disoproxil fumarate (TDF) was approved for use in children over 2 years of age at a dose of 8 mg/kg/day, and is the WHO recommended first-line therapy for children over 10 years of age or 35 kg in weight, at 300 mg daily. Whilst postmarketing experience of paediatric TDF is limited, prior off-licence use has occurred at our centre due to its tolerability, efficacy and resistance profiles. In this article we describe a single-centre experience of TDF nephrotoxicity in children aged <16 years. We conducted a retrospective case-note audit of children with perinatally-acquired HIV who ever received TDF-based antiretroviral therapy. From 2001 to December 2013, 70 children [39 (56 \%) females] ever received TDF. Median age at the start of TDF treatment was 12 years (interquartile range 10-14). Seven (10 \%) children developed asymptomatic renal tubular leak with associated hypophosphataemia (3) and hypokalaemia (1), all resulting in TDF withdrawal and biochemical resolution. Comparison of the nephrotoxic group versus the rest of the cohort showed no significant differences for age, sex, antiretroviral regimen or CD4 count. Lower weight (p = 0.05) and initial dose of TDF received (p = 0.0048) were significantly associated with TDF-induced nephrotoxicity: median dose of TDF (7.8 mg/kg/day) compared with the remainder of the cohort (6.5 mg/kg/day). Concurrent use of protease inhibitors (PIs) with TDF may be a contributing factor to the development of nephrotoxicity (odds ratio 6; 95 \% CI 0.7-54; p = 0.111). Although all children with TDF-associated nephrotoxicity had biochemical resolution on drug withdrawal, renal monitoring of children receiving TDF is important, especially with the co-administration of PIs. Postmarketing surveillance is essential in the paediatric setting. [\hyperlink{Litfulo}{PMID: 25861908}, Yinru Lim et al., 2015]

\hypertarget{pmid_15728910}{G}atifloxacin is an 8-methoxy fluoroquinolone effective against a broad spectrum of pathogens common in pediatric infections. The safety and pharmacokinetics of a single dose of gatifloxacin were studied in pediatric patients from 6 months to 16 years of age. Seventy-six pediatric patients (average age, 6.7 +/- 5.0 years) were administered a single oral dose of gatifloxacin suspension (5, 10, or 15 mg/kg of body weight; 600-mg maximum) in a dose-escalating manner. Subjects were stratified by age into 4 groups. An additional 12 children, greater than 6 years of age, received gatifloxacin as the tablet formulation at a dose of approximately 10 mg/kg. Gatifloxacin's apparent clearance and half-life were 5.5 +/- 2.1 ml/min/kg and 5.1 +/- 1.4 h. The maximum concentration of drug in plasma and area under the concentration-time curve (AUC) increased in a manner approximately proportional to the dose. At the 10-mg/kg dose, the bioavailability was similar between the suspension and tablet formulation. The apparent oral clearance of gatifloxacin, normalized for body weight, exhibited a small but statistically significant decrease with increasing age. In all subjects receiving gatifloxacin at 10 mg/kg, the AUC exceeded 20 microg . h/ml (estimated free AUC/MIC ratio of > or =34 for MIC of < or =0.5 microg/ml). These data suggest that gatifloxacin at a dose of 10 mg/kg every 24 h will achieve therapeutic concentrations in plasma in infants and children. [\hyperlink{Litfulo}{PMID: 15728910}, Edmund V Capparelli et al., 2005]

\hypertarget{pmid_3261948}{T}he induction characteristics of propofol were studied and compared with thiopentone in children aged 3-14 years who received either no premedication or pethidine-atropine or trimeprazine. Anaesthesia was maintained with nitrous oxide in oxygen, and isoflurane. The induction doses of propofol and thiopentone were 2.9 mg/kg and 6.5-7.1 mg/kg respectively; premedication had no significant effect on the induction doses of either agent. Spontaneous movement and hypertonus occurred in about 20\% of children with both agents. The use of propofol was associated with a high incidence of pain on injection (injections were mostly in veins on the dorsum of the hand), but this was reduced by mixing lignocaine with propofol. Cardiovascular effects were not clinically significant with either agent. Apnoea occurred in 35\% of patients given propofol and in 50\% of those given thiopentone. Children anaesthetised with propofol awoke significantly earlier after cessation of all anaesthesia. It is concluded that the use of propofol is safe in children and may have advantages where early recovery from anaesthesia is desirable, but offers no advantage over thiopentone for routine induction of anaesthesia. [\hyperlink{Litfulo}{PMID: 3261948}, R K Mirakhur et al., 1988]

\hypertarget{pmid_33205828}{C}utaneous leishmaniasis poses a therapeutic challenge in the paediatric population. The aim of this study was to assess the efficacy and safety of miltefosine treatment for Old World cutaneous leishmaniasis in paediatric patients. A multicentre retrospective review of 10 children (≤ 18 years of age) with cutaneous leishmaniasis treated with miltefosine in Israel was performed. Mean ± standard deviation age at diagnosis was 9.1 ± 5.0 years. The Leishmania species diagnosed was L. tropica in 8 cases and Leishmania major in 2 cases. Mean ± standard deviation duration of treatment was 44.8 ± 20.6 days, with a mean follow-up period of 12.1 ± 17.1 months. Complete response was noted in 8 (80\%) patients. Treatment failure was noted in 2 (20\%) cases. Side-effects related to the medication were minimal. In conclusion, oral miltefosine may be an effective and safe treatment for Old World cutaneous leishmaniasis caused by Leishmania tropica or Leishmania major in children. However, further studies are warranted to draw a definite conclusion. [\hyperlink{Litfulo}{PMID: 33205828}, Ayelet Ollech et al., 2020]

\hypertarget{pmid_32160320}{E}valuate technical success, tolerability, and safety of lidocaine iontophoresis and tympanostomy tube placement for children in an office setting. Prospective individual cohort study. This prospective multicenter study evaluated in-office tube placement in children ages 6 months through 12 years of age. Anesthesia was achieved via lidocaine/epinephrine iontophoresis. Tube placement was conducted using an integrated and automated myringotomy and tube delivery system. Anxiolytics, sedation, and papoose board were not used. Technical success and safety were evaluated. Patients 5 to 12 years old self-reported tube placement pain using the Faces Pain Scale-Revised (FPS-R) instrument, which ranges from 0 (no pain) to 10 (very much pain). Children were enrolled into three cohorts with 68, 47, and 222 children in the Operating Room (OR) Lead-In, Office Lead-In, and Pivotal cohorts, respectively. In the Pivotal cohort, there were 120 and 102 children in the <5 and 5- to 12-year-old age groups, respectively, with a mean age of 2.3 and 7.6 years, respectively. Bilateral tube placement was indicated for 94.2\% of children <5 and 88.2\% of children 5 to 12 years old. Tubes were successfully placed in all indicated ears in 85.8\% (103/120) of children <5 and 89.2\% (91/102) of children 5 to 12 years old. Mean FPS-R score was 3.30 (standard deviation [SD] = 3.39) for tube placement and 1.69 (SD = 2.43) at 5 minutes postprocedure. There were no serious adverse events. Nonserious adverse events occurred at rates similar to standard tympanostomy procedures. In-office tube placement in selected patients can be successfully achieved without requiring sedatives, anxiolytics, or papoose restraints via lidocaine iontophoresis local anesthesia and an automated myringotomy and tube delivery system. 2b Laryngoscope, 130:S1-S9, 2020. [\hyperlink{Litfulo}{PMID: 32160320}, Lawrence R Lustig et al., 2020]

\hypertarget{pmid_10990583}{T}inea capitis is one of the most common infections of children. The standard treatment is griseofulvin. Itraconazole and terbinafine have in large part replaced griseofulvin in the treatment of onychomycosis and, in addition to fluconazole and ketoconazole, are evolving treatments for tinea capitis. The purpose of this review is to compare the efficacy, safety, and cost of oral antifungal agents for tinea capitis. Small, open-label studies of itraconazole, terbinafine, and fluconazole have reported encouraging results, suggesting that these drugs may be effective alternatives to griseofulvin; however, in large controlled studies griseofulvin continues to exhibit greater or equal efficacy. Ketoconazole appears to be the least efficacious. All five drugs appear relatively safe, however, only griseofulvin has a long track record of safety, is Food and Drug Administration (FDA) approved for the treatment of tinea capitis in children, and has the least known drug interactions. Fluconazole is FDA approved for use in children more than 6 months of age, yet not for the treatment of tinea capitis. Oral griseofulvin and terbinafine tablets are the least expensive of the antifungal agents; griseofulvin suspension is, however, more expensive than fluconazole suspension. For the combined reasons of efficacy, safety, and cost, and a long track record of use, we feel oral griseofulvin is still the present treatment of choice for tinea capitis. Newer antifungals are currently under investigation, and their role in treating tinea capitis in children is still being defined. [\hyperlink{Litfulo}{PMID: 10990583}, M L Bennett et al., ]

\hypertarget{pmid_10092716}{T}o assess the efficacy of dermal anesthesia by lidocaine iontophoresis in children undergoing peripheral intravenous (PIV) catheter placement in the emergency department. A double-blind, randomized, clinical trial was conducted at a tertiary children's hospital ED. Alert children 7 years or older requiring nonemergency PIV were eligible. Patients in the lidocaine group received 1 mL of 2\% lidocaine with 1:100,000 epinephrine over a potential PIV site by iontophoresis. The control group received 1 mL of.9\% saline solution with 1:100,000 epinephrine. After PIV placement, patients ranked the procedural pain using a visual analog scale. Complications were noted by visual inspection or telephone follow-up. During a 6-month period, 22 patients were assigned to the lidocaine group and 25 to the control group. There was no significant difference in age, sex, or ethnic background between the 2 study groups, and mean application time was 12.0 minutes. The median pain score was.5 in the lidocaine group compared with 4 in the control group (P =.0002; 95\% confidence interval [CI] 1 to 5). No significant immediate or delayed complications were observed. Lidocaine iontophoresis provides effective dermal anesthesia for children older than 7 years of undergoing nonemergency PIV placement in the ED. [\hyperlink{Litfulo}{PMID: 10092716}, M K Kim et al., 1999]

\hypertarget{pmid_26858095}{S}edation is increasingly used to facilitate procedures on children in emergency departments (EDs). This overview of systematic reviews (SRs) examines the safety and efficacy of sedative agents commonly used for procedural sedation in children in the ED or similar settings. We followed standard SR methods: comprehensive search; dual study selection, quality assessment, data extraction. We included SRs of children (1 month to 18 years) where the indication for sedation was procedure-related and performed in the ED. Fourteen SRs were included (210 primary studies). The most data were available for propofol (six reviews/50,472 sedations) followed by ketamine (7/8,238), nitrous oxide (5/8,220), and midazolam (4/4,978). Inconsistent conclusions for propofol were reported across six reviews. Half concluded that propofol was sufficiently safe; three reviews noted a higher occurrence of adverse events, particularly respiratory depression (upper estimate 1.1\%; 5.4\% for hypotension requiring intervention). Efficacy of propofol was considered in four reviews and found adequate in three. Five reviews found ketamine to be efficacious and seven reviews showed it to be safe. All five reviews of nitrous oxide concluded it is safe (0.1\% incidence of respiratory events); most found it effective in cooperative children. Four reviews of midazolam made varying recommendations. To be effective, midazolam should be combined with another agent that increases the risk of adverse events (upper estimate 9.1\% for desaturation, 0.1\% for hypotension requiring intervention). This comprehensive examination of an extensive body of literature shows consistent safety and efficacy for nitrous oxide and ketamine, with very rare significant adverse events for propofol. There was considerable heterogeneity in outcomes and reporting across studies and previous reviews. Standardized outcome sets and reporting should be encouraged to facilitate evidence-based recommendations for care. [\hyperlink{Litfulo}{PMID: 26858095}, Lisa Hartling et al., 2016]

\hypertarget{pmid_1626674}{I}n a single-blind study of 100 children aged 1 to 10 years, the minimum effective dose of lignocaine required to prevent injection pain due to propofol was 0.2 mg.kg-1 when veins on the dorsum of the hand were used. This is more than twice the adult value. We concluded that injection pain should not limit the use of propofol in children if an adequate amount of lignocaine is mixed immediately prior to injection. [\hyperlink{Litfulo}{PMID: 1626674}, E Cameron et al., 1992]

\hypertarget{pmid_12122400}{T}his study examined the safety, tolerability, and efficacy of iontophoresis with 30 mA of lidocaine for dermal anesthesia in children younger than 84 months in the emergency department and the usefulness of a modified version of the Pre-verbal, Early Verbal Pediatric Pain Scale (M-PEPPS). Three expert nurses completed the protocol for iontophoresis and inserted an intravenous catheter. Parents scored pain by using the 10-cm visual analogue scale, nurses used the M-PEPPS, and children, if able, self-reported pain during the procedure and at needle stick. Serum lidocaine levels were within the normal laboratory reference range. Adverse effects were minor and disappeared prior to discharge from the emergency department. Eighty-five percent of the children had M-PEPPS scores < or=6 during the iontophoresis procedure; 42\% had scores of < or =6 at needle stick. Eighty-two percent of the parents marked the vas as < or =30 during the procedure; 65\% indicated scores of < or =30 at needle stick. Four children self-reported "a lot of pain" at needle stick. Although low to moderate, M-PEPPS scores and parental pain ratings were significantly correlated at both points in time. Iontophoresis with lidocaine is safe for use in young children. It does not create any long-term untoward effects and is quite well tolerated. It is not clear if the higher pain scores at needle stick reflect anxiety and fear of a needle or a painful experience. [\hyperlink{Litfulo}{PMID: 12122400}, Alyce A Schultz et al., 2002]

\hypertarget{pmid_29788331}{S}afety concerns persist for long-term pediatric fluoroquinolone use. Seventy children (median age, 2.1 years) treated with levofloxacin 10-20 mg/kg once daily for multidrug-resistant tuberculosis (median observation time, 11.8 months) had few musculoskeletal events, no levofloxacin-attributed serious adverse events, and no Fridericia-corrected QT interval >450 ms. Long-term levofloxacin was safe and well tolerated. [\hyperlink{Litfulo}{PMID: 29788331}, Anthony J Garcia-Prats et al., 2018]

\hypertarget{pmid_16781498}{C}hildren frequently suffer infections accompanied by fever, which is commonly treated with acetaminophen (paracetamol), a use not devoid of risk. The effect of a complex homeopathic medicine (Viburcol, Heel Belgium, Gent, Belgium) was compared with that of acetaminophen in children with infectious fever. Non-randomized observational study. Thirty-eight Belgian centers practicing homeopathy and conventional medicine. Children <12 years old. Viburcol (drops) or acetaminophen (pills, capsules, or liquid form) for a maximum of 2 weeks. Fever, cramps, distress, disturbed sleep, crying, and difficulties with eating or drinking. Symptoms were graded by the practitioner on a scale from 0 to 3. Severity of infection was evaluated on a scale from 0 to 4. Data were captured on body temperature, subjective impression of health status, time to first improvement of symptoms, and global evaluation of treatment effects. Tolerability and compliance were monitored. Both treatment groups improved during the treatment period. Body temperature was reduced by 1.7 degrees C +/- 0.7 degrees C with Viburcol and by 1.9 degrees C +/- 0.9 degrees C with acetaminophen; fever score (scale from 0 to 3) from 1.7 +/- 0.6 to 0.1 +/- 0.2 with Viburcol and from 1.9 +/- 0.7 to 0.2 +/- 0.5 with acetaminophen (all values mean +/-SD). The overall severity of infection (scale from 0 to 4) decreased from 2.0 +/- 0.5 to 0.0 +/- 0.2 with Viburcol and from 2.2 +/- 0.7 to 0.2 +/- 0.6 with acetaminophen. There were no statistically significant differences between treatment groups in time to symptomatic improvement. Viburcol was noninferior to acetaminophen on all variables evaluated. Both treatments were very well tolerated, but the Viburcol group had a significantly greater number of patients with the highest tolerability score. In this patient population, Viburcol was an effective alternative to acetaminophen treatment and significantly better tolerated. [\hyperlink{Litfulo}{PMID: 16781498}, Mireille Derasse et al., 2005]

\hypertarget{pmid_24138461}{T}he aim of this study was to determine the safety and the efficacy of paediatrician-administered propofol in children undergoing different painful procedures. We conducted a retrospective study over a 12-year period in three Italian hospitals. A specific training protocol was developed in each institution to train paediatricians administering propofol for painful procedures. In this study, 36,516 procedural sedations were performed. Deep sedation was achieved in all patients. None of the children experienced severe side effects or prolonged hospitalisation. There were six calls to the emergency team (0.02\%): three for prolonged laryngospasm, one for bleeding, one for intestinal perforation and one during lumbar puncture. Nineteen patients (0.05\%) developed hypotension requiring saline solution administration, 128 children (0.4\%) needed O2 ventilation by face mask, mainly during upper endoscopy, 78 (0.2\%) patients experienced laryngospasm, and 15 (0.04\%) had bronchospasm. There were no differences in the incidence of major complications among the three hospitals, while minor complications were higher in children undergoing gastroscopy. This multicentre study demonstrates the safety and the efficacy of paediatrician-administered propofol for procedural sedation in children and highlights the importance of appropriate training for paediatricians to increase the safety of this procedure in children. [\hyperlink{Litfulo}{PMID: 24138461}, Antonio Chiaretti et al., 2014]

\hypertarget{pmid_20819318}{A}llergic rhinitis (AR) and chronic idiopathic urticaria (CIU) are common causes of substantial illness and disability in preschool children. Antihistamines are commonly used to treat preschool children with these conditions, but their use is based mostly on extrapolated efficacy from adult populations; it is thus important to characterize the safety of antihistamines in the pediatric population. This study was designed to assess the safety of levocetirizine dihydrochloride oral liquid drops in infants and children with AR or CIU. Two multicenter, double-blind, randomized, parallel-group studies randomized infants aged 6-11 months (study 1, n = 69) and children aged 1-5 years (study 2, n = 173) to levocetirizine, 1.25 mg (q.d. or b.i.d., respectively), or placebo for 2 weeks, using a 2:1 ratio. Safety evaluations included treatment-emergent adverse events (TEAEs), vital signs, electrocardiographic (ECG) assessments, and laboratory tests. The overall incidence of TEAEs was similar between levocetirizine and placebo in both studies. Most TEAEs were mild or moderate in intensity. TEAEs prompted discontinuation of therapy in three patients receiving levocetirizine in study 1. No clinically relevant changes from baseline in vital signs or laboratory parameters were apparent in either study; changes from baseline in these evaluations were similar between groups. No significant changes were observed in ECG parameters, including corrected QT interval. Levocetirizine, 1.25 and 2.5 mg/day, was well tolerated in infants aged 6-11 months and in children aged 1-5 years, respectively, with AR or CIU. [\hyperlink{Litfulo}{PMID: 20819318}, Frank Hampel et al., ]

\hypertarget{pmid_25746065}{S}ildenafil (Revatio®) and tadalafil (Adcirca®) are specific inhibitors of the phosphodiesterase-5 enzyme and produce pulmonary vasodilation by inhibiting the breakdown of cyclic guanosine monophosphate (cGMP) in the walls of pulmonary arterioles. We focus on the efficacy and safety of sildenafil and tadalafil in the treatment of pulmonary hypertension (PH) in children through a PubMed literature search. Although used since 1999 in the treatment of PH in children, it is only in the past few years that robust evidence for the use of sildenafil has emerged principally in the pivotal STARTS-1 study. The open-label extension of this study, STARTS-2, has revealed safety concerns substantiated by FDA post marketing surveillance leading to recommendations to use lower doses. More recently, tadalafil has been introduced allowing once daily dosing with apparently similar efficacy to sildenafil in children. Recently there have been suggestions that sildenafil and tadalafil may have a place in treating muscular dystrophy. [\hyperlink{Litfulo}{PMID: 25746065}, Alan G Magee et al., 2015]

\hypertarget{pmid_31292919}{T}riclofos sodium (TFS) has been used for many years in children as a sedative for painless medical procedures. It is physiologically and pharmacologically similar to chloral hydrate, which has been censured for use in children with neurocognitive disorders. The aim of this study was to investigate the safety and efficacy of TFS sedation in a pediatric population with a high rate of neurocognitive disability. The database of the neurodiagnostic institute of a tertiary academic pediatric medical center was retrospectively reviewed for all children who underwent sedation with TFS in 2014. Data were collected on demographics, comorbidities, neurologic symptoms, sedation-related variables, and outcome. The study population consisted of 869 children (58.2\% male) of median age 25 months (range 5-200 months); 364 (41.2\%) had neurocognitive diagnoses, mainly seizures/epilepsy, hypotonia, or developmental delay. TFS was used for routine electroencephalography in 486 (53.8\%) patients and audiometry in 401 (46.2\%). Mean (± SD) dose of TFS was 50.2 ± 4.9 mg/kg. Median time to sedation was 45 min (range 5-245), and median duration of sedation was 35 min (range 5-190). Adequate sedation depth was achieved in 769 cases (88.5\%). Rates of sedation-related adverse events were low: apnea, 0; desaturation ≤ 90\%, 0.2\% (two patients); and emesis, 0.35\% (three patients). None of the children had hemodynamic instability or signs of poor perfusion. There was no association between desaturations and the presence of hypotonia or developmental delay. TFS, when administered in a controlled and monitored environment, may be safe for use in children, including those with underlying neurocognitive disorders. [\hyperlink{Litfulo}{PMID: 31292919}, Eytan Kaplan et al., 2019]

\hypertarget{pmid_33323581}{O}ral Triclofos is widely used as a sedative agent in children. However, the role of Triclofos as a sedative agent in children undergoing ophthalmological procedures has not been adequately studied. The aim of this study was to determine the safety and efficacy of oral Triclofos in children suffering from pediatric glaucoma who were undergoing ocular examination. 80 children aged less than 5 years were assessed for eligibility for the trial after taking hospital ethical committee approval. The children were administered 80 mg/kg of oral Triclofos and Ramsay sedation score was measured every 5 min starting from 20 min after administration of the drug. If the child was not adequately sedated after 30 min, additional dose of 05 mg/kg was administered every 5 min till 60 min of drug administration. The procedure was considered a failure and general anesthesia (GA) administered if Ramsay sedation score was ≤4 after 60 min of initial drug administration. Heart rate and arterial oxygen saturation were measured throughout the period of sedation. The duration of sedation and incidence of side effects was also noted. A total of 73 patients underwent the study. The mean age of children was 23.4 months (SD - 14.72) and mean weight was 12 kg (SD - 3.84). The mean dose of Pedicloryl (Triclofos) used was 83.8 mg/kg and the median duration of onset was 25 min. 64 children completed examination successfully, 2 children had to be administered GA during the procedure. There were no major side effects. Administration of oral Triclofos in a dose of 80 mg/kg body weight was safe and effective in children less than 5 years of age undergoing ocular examination. [\hyperlink{Litfulo}{PMID: 33323581}, Sameer Sethi et al., 2021]

\section*{Lyrica}
\subsection*{Result}
\subsubsection*{Answer}

Unknown

\subsubsection*{{Explanation}}
\hypertarget{Lyrica}
A review of the provided abstracts reveals that none of them are about Lyrica (pregabalin). The abstracts focus on the safety and efficacy of various medications in children, including lidocaine, articaine, bilastine, hydrocortisone butyrate, lithium, 6-mercaptopurine, and others, but there is no mention of Lyrica or studies specifically evaluating its safety in pediatric populations. Therefore, based on the abstracts available, there is not enough data to determine whether Lyrica is safe or unsafe for use in children in any age range.

\subsection*{Abstracts}
\hypertarget{pmid_2245677}{T}o evaluate the safety of topical lidocaine anesthesia in children undergoing bronchoscopy, we determined SLC in 15 children aged 3 months to 9.5 years during flexible fiberoptic bronchoscopy. A total lidocaine dose of 3.2 to 8.5 (mean +/- SEM = 5.7 +/- 0.5) mg/kg was administered to nose, larynx and bronchial tree over 9 to 45 (mean +/- SEM = 20 +/- 2.7) minutes. No complication occurred during the procedure. Peak SLC were 1-3.5 (mean +/- SEM = 2.5 +/- 0.2) micrograms/ml. The Vd beta was 1.79 +/- 0.19 L/kg, the t1/2 beta was 109 +/- 12 minutes, and the total body clearance 12.2 +/- 1.1 ml/min/kg. Peak SLC correlated well with the dose expressed as mg/kg (r = 0.59, p less than 0.025), and even better when related to body surface area (r = 0.63, p less than 0.01). Lidocaine doses up to 8.5 mg/kg proved safe and resulted in therapeutic SLC in our patients. Lidocaine dose up to 7 mg/kg appears to be safe provided that it does not exceed an upper limit of 175 mg/m2 and is gradually administered over a minimum of 15 minutes. Doses of 7-8.5 mg/kg appear to be safe when administered over longer periods. [\hyperlink{Lyrica}{PMID: 2245677}, Y Amitai et al., 1990]

\hypertarget{pmid_19740527}{E}noxaparin, a low molecular weight heparin (LMWH), is frequently used for the prevention and treatment of thromboembolic complications in infants and children (Sutor et al., 2004 [1]). Injection pain and the fear and anxiety associated with needle phobia in the pediatric population are well documented. Best practice pediatric pain management standards of care recommend mitigating the child's pain experience whenever possible. The use of topical anesthetics such as liposomal-lidocaine 4\% results in a rapid onset of anesthesia, minimal blanching, without vasoconstriction (Koh et al., 2004 [2]) or risk of methemoglobinemia. Topical lidocaine has been used to reduce the injection pain of enoxaparin, but there is no data available examining whether it will interfere with the absorption of LMWH. To determine if the topical lidocaine, Maxilene, interferes with enoxaparin absorption as measured by peak anti-Xa levels. Infants and children clinically prescribed enoxaparin were eligible for this study. Children in group 1 were pre-treated with Maxilene prior to enoxaparin injection on day 1 with no Maxilene pre-treatment on day 2. For group 2, the order was reversed. Peak anti-Xa levels were measured following each enoxaparin dose and were compared between the groups. 26 children of ages 14d-16 y (median 6.7 months) were enrolled. Anti-Xa levels following topical lidocaine administration were 0.070 U/mL (95\%CI 0.025; 0.114) lower than without prior topical lidocaine administration. Anti-Xa levels on the second day were on average 0.013 U/mL (95\%CI -0.066; 0.040) higher compared to day one regardless of the order of topical lidocaine administration. There were no reported incidences of local reactions such as redness, hives or blanching. Topical lidocaine (Maxilene) administration before enoxaparin injection results in a small, clinically non-significant, reduction in anti-Xa levels. [\hyperlink{Lyrica}{PMID: 19740527}, S M Duncan et al., 2010]

\hypertarget{pmid_26918853}{R}egulations on medicinal products for paediatric use require that pharmacokinetics and safety be characterized specifically in the paediatric population. A previous study established that a 10-mg dose of bilastine in children aged 2 to <12 years provided an equivalent systemic exposure as 20 mg in adults. The current study assessed the safety and tolerability of bilastine 10 mg in children with allergic rhinoconjunctivitis and chronic urticaria. In this phase III, multicentre, double-blind study, children were randomized to once-daily treatment with bilastine 10-mg oral dispersible table (n = 260) or placebo (n = 249) for 12 weeks. Safety evaluations included treatment-emergent adverse events (TEAEs), laboratory tests, cardiac safety (ECG recordings) and somnolence/sedation using the Pediatric Sleep Questionnaire (PSQ). The primary hypothesis of non-inferiority between bilastine 10 mg and placebo was demonstrated on the basis of a near-equivalent proportion of children in each treatment arm without TEAEs during 12 weeks' treatment (31.5 vs. 32.5\%). No clinically relevant differences between bilastine 10 mg and placebo were observed from baseline to study end for TEAEs or related TEAEs, ECG parameters and PSQ scores. The majority of TEAEs were mild or moderate in intensity. TEAEs led to discontinuation of two patients treated with bilastine 10 mg and one patient treated with placebo. Bilastine 10 mg had a safety and tolerability profile similar to that of placebo in children aged 2 to <12 years with allergic rhinoconjunctivitis or chronic urticaria. [\hyperlink{Lyrica}{PMID: 26918853}, Zoltán Novák et al., 2016]

\hypertarget{pmid_30130637}{M}any clinicians are reluctant to use traditional mood-stabilizing agents, especially lithium, in children and adolescents. This review examined the evidence for lithium's safety and efficacy in this population. A systematic review was conducted on the use of lithium in children and adolescents with bipolar disorder (BD). Relevant papers published through June 30 30 articles met inclusion criteria, including 12 randomized controlled trials (RCTs). Findings from RCTs demonstrate efficacy for acute mania in up to 50\% of patients, and evidence of long-term maintenance efficacy. Lithium was generally safe, at least in the short term, with most common side effects being gastrointestinal, polyuria, or headache. Only a minority of patients experienced hypothyroidism. No cases of acute kidney injury or chronic kidney disease were reported. Though the available literature is mostly short-term, there is evidence that lithium monotherapy is reasonably safe and effective in children and adolescents, specifically for acute mania and for prevention of mood episodes. [\hyperlink{Lyrica}{PMID: 30130637}, A Amerio et al., 2018]

\hypertarget{pmid_15898975}{T}he age below 5 years is considered a prudential limit for immunotherapy in view of the possible severity of side-effects. Sublingual immunotherapy (SLIT) seems to be safe, but no study in very young children is available. We performed a safety post-marketing surveillance study in children below 5 years. Children aged 3-5 years with respiratory allergy receiving SLIT were followed-up for at least 2 years. A diary card for side-effects was filled by parents at each dose given. Local and systemic side-effects were graded as: mild (no intervention, no dose adjustment), moderate (medical treatment and/or dose reduction), severe (life-threatening/hospitalization/emergency care). The comparative safety of different allergens and regimens was also assessed. One hundred and twenty-six children (mean age 4.2 years, 67 male) were included. Seventy-six (60\%) had rhinitis with asthma, 34 (27\%) rhinitis only and 16 (13\%) only asthma. Immunotherapy was prescribed for mites (62\%), grasses (22.2\%), Parietaria (11.9\%), Alternaria (2.4\%) and olive (1.5\%). Eighteen children underwent an accelerated build-up. The total number of doses was about 39,000. Nine side-effects were reported in seven children (5.6\% patients and 0.2/1000 doses). Two episodes of oral itching and one of abdominal pain were mild. Six gastrointestinal side-effects were controlled by reducing the dose. All side-effects occurred during up-dosing phase. No difference in terms of safety among the allergens used was observed. SLIT is safe also in children under the age of 5 years. [\hyperlink{Lyrica}{PMID: 15898975}, V Di Rienzo et al., 2005]

\hypertarget{pmid_24992191}{T}he purpose of this meta-analysis was to determine the efficacy of lidocaine in preventing laryngospasm during general anaesthesia in children. An electronic search of six databases was conducted. The Preferred Reporting Items for Systematic Reviews and Meta-analyses (PRISMA) guidelines were adhered to. We included randomised controlled trials reporting the effects of intravenous and/or topical lidocaine on the incidence of laryngospasm during general anaesthesia. Nine studies including 787 patients were analysed. The combined results demonstrated that lidocaine is effective in preventing laryngospasm (risk ratio (RR) 0.39, 95\% CI 0.24-0.66; I(2)  = 0). Subgroup analysis revealed that both intravenous lidocaine (RR 0.34, 95\% CI 0.14-0.82) and topical lidocaine (RR 0.42, 95\% CI 0.22-0.80) lidocaine are effective in preventing laryngospasm. The results were not affected by studies with a high risk of bias. We conclude that, both topical and intravenous lidocaine are effective for preventing laryngospasm in children.  [\hyperlink{Lyrica}{PMID: 24992191}, T Mihara et al., 2014] Topical lidocaine has been found to result in overestimation of the severity of laryngomalacia in infants undergoing flexible bronchoscopy under conscious sedation with midazolam and nalbuphine. This effect has never been confirmed and may depend on the level of sedation and the drugs used. We assessed the effect of topical lidocaine on laryngomalacia in infants undergoing flexible bronchoscopy under general anesthesia with propofol. Thirteen infants with congenital stridor referred to diagnostic flexible video-bronchoscopy were studied under propofol anesthesia before and 3 minutes after topical lidocaine administration to the larynx at a dose of 3 mg/kg body weight. Laryngomalacia was scored using 60 seconds video recordings of the larynx before and after lidocaine administration in random order by 2 independent blinded observers using the previously described arytenoid score (AS), epiglottis score (ES), and the total score (TS=AS+ES). No significant differences in AS, ES, and laryngomalacia score were found between the prelidocaine and postlidocaine assessments by the 2 raters. The intraclass correlation coefficients were 0.995 (95\% confidence interval, 0.986-0.998) and 0.975 (0.930-0.991) and 0.989 (0.971-996) for AS, ES, and TS, respectively. The assessment of laryngomalacia is not affected by topical lidocaine under propofol anesthesia. The lidocaine effect on laryngomalacia may vary depending on the medication regime used and the depth of procedural sedation. [\hyperlink{Lyrica}{PMID: 24992191}, Britta S von Ungern-Sternberg et al., 2016]

\hypertarget{pmid_11916787}{I}n this randomized, double-blinded, placebo-controlled study, we evaluated the safety and efficacy of lidocaine iontophoresis for the prevention of pain associated with venipuncture in 59 children aged 6-17 yr. Children received either lidocaine HCl 2\% with epinephrine 1:100,000 (Active) or the same formulation without lidocaine (Placebo) via a 20 mA/min iontophoretic treatment. Pain during venipuncture was assessed by the subject, parent, and nurse using a 100-mm visual analog scale. Median (interquartile range) visual analog scale scores were significantly lower in the Active versus Placebo groups: subject, 7.0 (2.0-20.8) versus 31.0 (12.0-48.0), P < 0.001; nurse, 5.0 (2.2-10.8) versus 24.0 (9.0-47.0), P < 0.001; and parent, 3.0 (0.8-7.2) versus 20.0 (4.5-43.0), P < 0.002, respectively. Similarly, higher median satisfaction scores were given to the Active versus Placebo group by the three evaluators. Of the 59 subjects completing the study, 10 subjects experienced a total of 12 adverse events that were all graded as mild. In conclusion, lidocaine iontophoresis is safe in children, reduces discomfort associated with venipuncture, and increases satisfaction when compared with the placebo. In this randomized, double-blinded, placebo-controlled study, we found that dermal anesthesia with lidocaine HCl 2\% combined with epinephrine 1:100,000 administered via iontophoresis in children is achieved in 8.8 +/- 2.1 min, reduces discomfort associated with venipuncture, is safe, and increases satisfaction when compared with the placebo. [\hyperlink{Lyrica}{PMID: 11916787}, John B Rose et al., 2002]

\hypertarget{pmid_20527137}{O}nly a few corticosteroids for topical use have proven safe and effective in pediatric populations down to 3 months of age. The authors report the results of a study designed to assess the efficacy and safety of hydrocortisone butyrate (HCB) 0.1\% in lipocream (LCr) vehicle in infants and children. A total of 264 boys and girls 3 months to less than 18 years old, with stable, mild to moderate atopic dermatitis affecting at least 10\% body surface area applied HCB 0.1\% in LCr or LCr alone twice daily for up to 1 month without occlusion. Primary end-points included: percent of patients who achieved treatment success based on physician global assessments. Secondary endpoint included: difference in pruritus and Eczema Area and Severity Index (EASI) at day 29. Treatment was significant (P < 0.001) for HCB 0.1\% LCr over vehicle. No serious nor significant adverse events were reported. Results are representative of a short duration treatment for a chronic disease. HCB 0.1\% in LCr is more effective than its vehicle in pediatric populations down to 3 months of age without significant adverse events when used twice a day for up to 1 month. [\hyperlink{Lyrica}{PMID: 20527137}, William Abramovits et al., ]

\hypertarget{pmid_29520391}{L}ocal anesthetic agents such as bupivacaine and lidocaine are commonly used after surgery for pain control. The aim of this prospective study was to evaluate the safety of a mixture of bupivacaine and lidocaine in children who underwent urologic inguinal and scrotal surgery. Fifty-five patients who underwent pediatric urologic outpatient surgeries, were prospectively enrolled in this study. The patients were divided into three groups according to age (group I: under 2 years, group II: between 3-4 years, and group III: 5 years and above). Patients were further sub-divided into unilateral and bilateral groups. All patients were injected with a mixture of 0.5\% bupivacaine and 2\% lidocaine (2:1 volume ratio) at the surgical site, just before the surgery ended. Hemodynamic and electrocardiographic parameters were measured before local anesthesia, 30 minutes after administration of local anesthesia, and 60 minutes after administration. The patients' mean age was 40.5±39.9 months. All patients had normal hemodynamic and electrocardiographic parameters before local anesthesia, after 30 minutes, and after 60 minutes. Also, results of all intervals were within normal values, when analyzed by age and laterality. No mixture related adverse events (nausea, vomiting, pruritus, sedation, respiratory depression) or those related to electrocardiographic parameters (arrhythmias and asystole) were reported in any patients. A mixture of bupivacaine and lidocaine can be safely used in children undergoing urologic inguinal and scrotal surgery. An appropriate dose has no clinically significant hemodynamic or cardiac changes and adverse effects. [\hyperlink{Lyrica}{PMID: 29520391}, Kyoung Lee et al., 2018]

\hypertarget{pmid_15054144}{T}o evaluate the role of a new formulation of lidocaine (ELA-max) in local anesthesia in children and compare it with the eutectic mixture of local anesthetics (EMLA). Relevant literature was identified by a MEDLINE search (1966-November 2003) using the search terms ELA-max and EMLA. Bibliographies of selected articles were also examined to include all relevant investigations. The product manufacturer was contacted for inclusion of the most recent data available. Topical anesthesia in children is clinically challenging. ELA-max has been shown to be as effective as EMLA for venipuncture in children, but with faster onset. Adverse effects, such as transient blanching with redness and erythema, have been reported. Further investigation is needed to determine the effectiveness of ELA-max on other painful procedures in children, as well as its safety. [\hyperlink{Lyrica}{PMID: 15054144}, Ran D Goldman et al., 2004]

\hypertarget{pmid_8865055}{W}e undertook a randomized double-blind trial to compare the efficacy of 1.5 mg/kg body weight (low dose) and 3 mg/kg (moderate dose) lidocaine regional anesthesia for closed reductions of forearm fractures in childhood. Of the 283 children studied, 143 were randomized to the moderate-dose group and 140 to the low-dose group. The characteristics of the children and their injuries did not differ significantly. There were no complications due to lidocaine toxicity. In children with angulated and incompletely displaced fractures, satisfactory anesthesia was achieved in 94\% of those receiving the low dose and in 97\% of those receiving the moderate dose of lidocaine. In children with completely displaced fractures, satisfactory anesthesia was achieved in 93\% of those receiving the moderate dose but in only 67\% of those receiving the low dose of lidocaine. We conclude that the low-dose lidocaine protocol is suitable for children requiring closed reductions of angulated and incompletely displaced fractures of the forearm. In contrast, the moderate-dose lidocaine protocol is more reliable in children with displaced forearm fractures. Meticulous adherence to the protocols is essential to prevent systemic lidocaine toxicity from premature deflation of the tourniquet. This potential risk is further reduced by use of the low-dose protocol, which is applicable to approximately 70\% of the children with forearm fractures requiring closed reductions. [\hyperlink{Lyrica}{PMID: 8865055}, H D Bratt et al., ]

\hypertarget{pmid_12853831}{M}any children with urinary tract infections and vesicoureteral reflux require catheterization. Catheterization can be a painful experience, and a desire to avoid this experience may affect patient care. We evaluated the effectiveness of lubricant containing lidocaine for minimizing patient pain and distress during catheterization. We conducted a prospective, double-blind, placebo controlled trial. Twenty children (16 girls and 4 boys, mean age 7.7 years) had urethral lubricant with or without lidocaine placed within 10 minutes before urethral catheterization. In all children pre-procedure anxiety, and pain and distress during catheterization were recorded. Pre-procedure anxiety was measured using a visual analog scale, pain was measured with the Oucher Pain Scale and distress was recorded by 2 independent observers with a simple 7-point Likert-type scale. There were no significant group differences for age, number of previous catheterizations or pre-procedure anxiety. The group receiving lubricant with lidocaine had significantly lower pain (21 +/- 19.69 versus 65.5 +/- 26.29) and distress (2.65 +/- 1.97 versus 4.7 +/- 2.07) (p = 0.001 and 0.007, respectively). The use of lubricant with lidocaine significantly decreases pain with pediatric urethral catheterization and is recommended with pediatric catheterizations. [\hyperlink{Lyrica}{PMID: 12853831}, Lisa L Gerard et al., 2003]

\hypertarget{pmid_7899121}{P}hysicians who prescribe viscous lidocaine preparations should be aware of the adverse effects and the high risk for overdose in pediatric patients. Owing to altered pharmacokinetics (increased absorption, decreased clearance, and prolonged half-life), doses that are innocuous for adults may present a significant potential toxic hazard in children. Lidocaine should not be used to treat painful mouth lesions in children until further safety data are available. Benzocaine may be considered as a safe alternative to lidocaine. Its low incidence of side effects makes it a safer choice for infants and children. If no other choices are appropriate, then very specific instructions should be given to parents. The amount, frequency, maximum daily dose, and mode of administration should be clearly communicated (eg, cotton pledget to individual lesions, one-half dropper to each cheek every four hours, or 20 minutes before meals). They should never be prescribed on a "PRN" basis. [\hyperlink{Lyrica}{PMID: 7899121}, J Gonzalez del Rey et al., 1994]

\hypertarget{pmid_32223002}{P}ain control is a mandatory aspect in pediatric dentistry office through local anesthesia. To assess the safety and efficacy of 4\% articaine local anesthetic in young children below four years old. An equivalent randomized control trial with two parallel arms included 184 young children (92 per group) aged from 36 to 47 months seeking pulpotomy of mandibular primary molars which performed after buccal infiltration injection. The control group received lidocaine hydrochloride 2\% with epinephrine 1:100 000. The intervention was articaine hydrochloride 4\% with epinephrine 1:100 000. Children's behavior during injection and treatment have assessed using Faces, Legs, Activity, Cry, and Consolability (FLACC) and child's behavior using Frankl Behavior Rating Scale (FBRS). In addition, post-operative complications have been addressed. Both anesthetic agents were equivalent during the injection phase. During the treatment phase, the absolute risk difference (ARR) between the two groups was 0.120 (95\% CI: -0.003; 0.243). The maximum limit of 95\% CI surpassed the margin of equivalence, indicating that less pain has been expressed during pulpotomy among children delivered articaine when compared to their counterparts in the lidocaine group. Concerning post-operative complications, no statistically significant difference was detected between the two anesthetic drugs. The findings supported the efficient and secure use of articaine hydrochloride 4\% with epinephrine 1:100 000 to treat children between the ages of 3 and below 4 years old. [\hyperlink{Lyrica}{PMID: 32223002}, Ahmad Abdel Hamid Elheeny et al., 2020] 6-mercaptopurine (6-MP), a key drug for treatment of acute lymphoblastic leukemia (ALL), has until recently had no adequate formulation for pediatric patients. Several approaches have been taken but the only oral paraben-free 6-MP liquid formulation named Loulla was developed and evaluated in the target population. Preclinical and clinical evaluations were performed according to a Pediatric Investigation Plan, in order to apply for a Pediatric Use Marketing Authorization. The pre-clinical study assessed the maximum tolerated dosage-volume and evaluated local mucosal toxicity of 28 daily administrations in treated compared to controls gold hamsters. The multi-centre clinical study was single-dose, open-label, crossover trial, conducted in 15 ALL children during maintenance therapy. The bioavailability and palatability of a single 50mg fixed dose of Loulla compared to 50mg registered tablets were evaluated in a random order on two consecutive days. Seven blood samples over 9h were obtained each day to determine 6-MP pharmacokinetic parameters, including Tmax, Cmax, AUC0-9 and AUC0-∞. A questionnaire adapted to children testing Loulla palatability and preference for either Loulla or the usual 6-MP tablet was completed. Occurrence of adverse events was determined at study visits by vital sign measurements, patient's spontaneous reporting, investigator's questioning and clinical examination. The preclinical study in gold hamsters showed that dosage-volume of 75 mg/kg/day was well tolerated. The relative bioavailability of liquid Loulla formulation compared to the reference presentation is 76\% for AUC0-9 and AUC0-∞ and 80\% for Cmax. The taste of Loulla and the mouth feeling after ingestion compare favorably to the tablet. No adverse event occurred. Pharmacokinetic, palatability and safety data support the use of Loulla in children. [\hyperlink{Lyrica}{PMID: 32223002}, Adam de Beaumais Tiphaine et al., 2016]

\hypertarget{pmid_7989628}{T}he use of 5 percent EMLA (an eutectic mixture of local anesthetics comprised of a mixture of prilocaine and lidocaine) as an intraoral topical anesthetic in children has been assessed in a clinical investigation. In a split-mouth study in twenty children there was no difference in the efficacy of EMLA and 5 percent lidocaine ointment in alleviating the pain of maxillary buccal infiltration injections of local anesthetics. EMLA did not differ significantly from placebo in the changes in pulpal responses of maxillary primary teeth to electrical stimulation before and after application in a double-blind split-mouth study in twenty children. [\hyperlink{Lyrica}{PMID: 7989628}, J G Meechan et al., ]

\hypertarget{pmid_3687359}{T}he aim of the study was to measure the plasma levels of lidocaine and prilocaine after dermal application of EMLA in infants and to evaluate whether this procedure increases the levels of methaemoglobin (Met-Hb). Two groups of infants, 3-6 (n = 12) and 6-12 months (n = 10) of age, were studied. In total, 2 ml of EMLA was applied to 4 x 4 cm of skin surface for 4 h and blood samples for detection of Met-Hb and plasma levels of local anaesthetics were taken at 0, 2, 4 and 8 h after the application. After removal of the cream the infants were operated mainly for minor procedures under general anaesthesia. The plasma concentrations of lidocaine and prilocaine were in all cases below toxic levels and there were only minor increases in Met-Hb in a few infants. In conclusion, EMLA can be used safely in infants above 3 months of age provided that the recommendations with regard to dose, application area and time are followed. The use of EMLA in smaller infants and in children taking other Met-Hb-inducing drugs needs further evaluation. [\hyperlink{Lyrica}{PMID: 3687359}, G Engberg et al., 1987]

\hypertarget{pmid_9881990}{T}o determine whether application of topical aqueous lidocaine to a laceration attenuates the pain from the subsequent lidocaine injection in children. Prospective, double-blind study. A large, urban, tertiary care children's hospital emergency department. A convenience sample of 100 children, five to 16 years of age, presenting with simple lacerations over a six-month period. An unlabelled 3-ml solution of either 1\% lidocaine or placebo (saline) was used to soak a Telfa pad (Kendall, Mansfield, MA) and then placed onto the laceration for 10 minutes. The wound was then injected with 1\% lidocaine, irrigated, and sutured per standard emergency department protocol. Independent pain response was elicited from the patient and parent four times: before any intervention, after the soak, after the injection, and at the end of the procedure. Blood pressure and heart rates were recorded at the same intervals. Four patients were excluded. Of the 96 remaining patients, 46 received the placebo and 50 received lidocaine. Age, sex, race, and laceration length and location were similar between groups. Physiologic parameters did not differ between groups. For all four pain ratings, the independent variables of age, sex, race, and laceration length or location did not differ between groups. Topical lidocaine was ineffective in relieving pain from the injection. When groups were combined, a significant negative correlation was noted for age versus injection pain (P = .035), with older children reporting less pain from injection than younger children. For children, soaking a simple laceration with 1\% lidocaine does not decrease pain from the subsequent lidocaine injection. [\hyperlink{Lyrica}{PMID: 9881990}, G M Stewart et al., 1998]

\hypertarget{pmid_2336293}{O}ral chloral hydrate sedation in children is safely used prior to cross-sectional imaging. We report two children with Leigh's disease who developed acute respiratory insufficiency following high dose oral chloral hydrate sedation. [\hyperlink{Lyrica}{PMID: 2336293}, S B Greenberg et al., 1990]

\hypertarget{pmid_19104973}{S}ince the FDA held hearings in February 2004 on the safety of antidepressants in children, there has been a great deal of controversy regarding the use of antidepressants in children, culminating in the well publicized black box warnings about increased risk of suicidal behavior in children and young adults (up to age 25) caused by these medications. Using questions that a parent might ask, the current article attempts to summarize the efficacy and safety data on the use of antidepressants in children so that psychologists, with or without prescription privileges, may be able to inform parents of young patients about the science behind this treatment. This article is based on a presentation at the 2007 American Psychological Association conference by the author in acceptance of the 2006 APAHC Bud Orgel Award for Distinguished Achievement in Research. Much of the information described in this article is drawn from the recent APA Report of the Working Group on Psychoactive Medications for Children and Adolescents. (Brown et al. 2006; available at www.apa.org/pi/cyf/childmeds.pdf ) culminating in a book by the same authors (Brown et al., Childhood mental health disorders: Evidence base and contextual factors for psychosocial, psychopharmacological, and combined interventions 2007). [\hyperlink{Lyrica}{PMID: 19104973}, David Antonuccio et al., 2008]

\hypertarget{pmid_34357739}{T}he article is a review of modern literature and an analysis of the legal framework regarding the use of local anesthetics in children under 4 years of age. There is a discussion of the validity of the off-label principle. In the domestic and foreign literature, there are publications highlighting the use of drugs based on 4\% articaine in children under 4 years old, despite age-related contraindications according to the instructions. It is necessary to pay special attention to the regulatory status of the instructions for the medical use of the medicinal product. Summarizing the presented arguments, we can talk about the available spectrum of clinical studies, meta-analyzes and RCT data on the use of articaine in dental practice in children under 4 years of age, which indicates its effectiveness and is considered a safe alternative to lidocaine for use. in children of all ages. [\hyperlink{Lyrica}{PMID: 34357739}, O V Gulenko et al., 2021]

\hypertarget{pmid_24589808}{I}ngestion of viscous lidocaine in children can lead to potentially lethal neurologic and cardiac effects. We report the case of a 2-year-old boy who developed posterior reversible encephalopathy syndrome 2 days after unobserved ingestion of about 500 mg viscous lidocaine (40 mg/kg of bodyweight). Initially, the child presented with convulsive status epilepticus and subsequent cardiac arrest necessitating cardiopulmonary resuscitation for eight minutes. After 2 days of full recovery, the child presented with progressive disorientation, dizziness, and visual neglect. Lasting for 2 days, these symptoms finally disappeared completely. Combined with the findings on cerebral magnetic resonance imaging, this episode was interpreted as posterior reversible encephalopathy syndrome. Two weeks after the ingestion, no neurologic and visual abnormalities were found. Viscous lidocaine is prescribed routinely for dentition or other painful lesions in the oral cavity in children. Despite the potential hazardousness of the drug, packaging of viscous lidocaine is not childproof. Therefore, physicians have to instruct the parents carefully to minimize the risk of overuse or accidental ingestion. In general, the use of viscous lidocaine should be limited.  [\hyperlink{Lyrica}{PMID: 24589808}, Simon Kargl et al., 2014] Lidocaine is a well-accepted topical anaesthetic, also used in minors to treat painful conditions on mucosal membranes. This randomized, double-blind, placebo-controlled study (registered prospectively as EudraCT number 2011-005336-25) was designed to generate efficacy and safety data for a lidocaine gel (2\%) in younger children with painful conditions in the oral cavity. One hundred sixty-one children were included in two subgroups: 4-8 years, average age 6.4 years, treated with verum or placebo and 6 months-<4 years, average age 1.8 years, treated only with verum. Pain reduction was measured from the time prior to administration to 10 or 30 minutes after. In addition, adverse events and local tolerability were evaluated. In group I, pain was reduced significantly after treatment with verum compared to placebo at both time points. In group II, the individual pain rating shift showed statistically significant lower pain after treatment. Only seven out of 161 patients reported an adverse event but none were classified as being related to the study medication. The local tolerability was assessed as very good in over 97\% of cases. For painful sites in the oral cavity, a 2\% lidocaine gel is a meaningful tool for short-term treatment in the paediatric population.  [\hyperlink{Lyrica}{PMID: 24589808}, Dörte Wolf et al., 2015] Lidocaine is used to treat neonatal seizures refractory to other anticonvulsants. It is effective, but also associated with cardiac toxicity. Previous studies have reported on the pharmacokinetics of lidocaine in preterm and term neonates and proposed a dosing regimen for effective and safe lidocaine use. The objective of this study was to evaluate the previously developed pharmacokinetic models and dosing regimen. As a secondary objective, lidocaine effectiveness and safety were assessed. Data from preterm neonates and (near-)term neonates with and without therapeutic hypothermia receiving lidocaine were included. Pharmacokinetic analyses were performed using non-linear mixed effects modelling. Simulations were performed to evaluate the proposed dosing regimen. Lidocaine was considered effective if no additional anticonvulsant was required and safe if no cardiac adverse events occurred. Data were available for 159 neonates; 50 (31.4\%) preterm and 109 term neonates, of whom 49 (30.8\%) were treated with therapeutic hypothermia. Lidocaine clearance increased with postmenstrual age by 0.69\%/day (95\% confidence interval 0.54-0.84\%). During therapeutic hypothermia (33.5°C), lidocaine clearance was reduced by 21.8\% (7.26\%/°C, 95\% confidence interval 1.63-11.2\%) compared to normothermia (36.5°C). Simulations demonstrated that the proposed dosing regimen leads to adequate average lidocaine plasma concentrations. Effectiveness and safety were assessed in 92 neonates. Overall effectiveness was 53.3\% (49/92) and 56.5\% (13/23) for neonates receiving the proposed dosing regimen. No cardiac toxicity was observed. Lidocaine pharmacokinetics was adequately described across the entire neonatal age range. With the proposed dosing regimen, lidocaine can provide effective and safe treatment for neonatal seizures. [\hyperlink{Lyrica}{PMID: 24589808}, Laurent M A Favié et al., 2020]

\section*{Miglustat}
\subsection*{Result}
\subsubsection*{Answer}

Children with mucopolysaccharidosis type III (likely including under 10 years): Yes  
Children with other lysosomal storage diseases (10 years and older, based on case reports/series): Unknown  
Children under 10 years with other indications: Unknown  

\subsubsection*{{Explanation}}
\hypertarget{Miglustat}
To determine if miglustat is safe for use in children, I reviewed all available abstracts for targeted studies on pediatric safety. Here are the relevant findings:

1. **Mucopolysaccharidosis type III (Sanfilippo syndrome):**
   - A randomized, double-blind, placebo-controlled phase IIb/III study included children (age not specified, but this disease is pediatric-onset). Safety and tolerability were assessed throughout the study. The abstract states: "Miglustat has an acceptable safety profile" in this population, but does not specify adverse event rates or detail the age range. The study did not show efficacy, but did affirm an acceptable safety profile in the pediatric population studied [\hyperlink{pmid_21658716}{PMID: 21658716}, Nathalie Guffon et al., 2011].

2. **Juvenile GM1-gangliosidosis:**
   - A case series of three patients, two of whom started miglustat at ages 10 and 17 (juvenile form), reported gradual improvement and no mention of significant adverse effects. However, as a case series, this is not a controlled safety study, and the sample size is very small [\hyperlink{pmid_28577204}{PMID: 28577204}, Federica Deodato et al., 2017].

3. **Niemann-Pick disease type C (NP-C):**
   - One case report describes a 16-year-old boy who started miglustat at age 11 and was followed for 3 years. The report notes global improvement and no mention of significant adverse effects, but this is a single case and not a controlled safety study [\hyperlink{pmid_27599728}{PMID: 27599728}, Jean-Marie Cuisset et al., 2016].
   - Another review abstract states: "Miglustat (Zavesca®) is approved for the oral treatment of adult patients with mild to moderate type 1 Gaucher disease (GD1) for whom enzyme replacement therapy is unsuitable, and for the treatment of progressive neurological manifestations in adult and paediatric patients with Niemann-Pick disease type C (NP-C)." However, this is a review and does not present new, targeted pediatric safety data [\hyperlink{pmid_21779792}{PMID: 21779792}, Nadia Belmatoug et al., 2011].

4. **Other studies:**
   - Several studies on Gaucher disease type 1 and other indications either do not specify pediatric patients or are focused on adults.

**Summary by age range:**
- **Children under 10 years:** The only targeted safety data comes from the mucopolysaccharidosis type III study, which likely included children under 10, but the exact age range is not specified. The study affirms an acceptable safety profile in the pediatric population studied [\hyperlink{pmid_21658716}{PMID: 21658716}].
- **Children 10 years and older:** The GM1-gangliosidosis case series and the NP-C case report include patients aged 10 and above, with no significant safety concerns reported, but these are not controlled safety studies.

**Conclusion:** Only one abstract provides targeted, controlled safety data for miglustat in children (likely including those under 10), and it affirms an acceptable safety profile. For other pediatric age ranges, the evidence is limited to case reports or series, which are insufficient to definitively affirm safety. Therefore, based on the abstracts, miglustat can be considered to have an acceptable safety profile in children with mucopolysaccharidosis type III, but the safety in other pediatric populations or age ranges remains unknown.

\subsection*{Abstracts}
\hypertarget{pmid_21658716}{T}o evaluate the efficacy and safety of oral miglustat treatment in patients with mucopolysaccharidosis type III. The primary outcome was efficacy with improvement or stabilization in at least two domains of Vineland Adaptative Behavior Scales at 6 months. The secondary outcome measured the evolution of other cognitive tests at 12 months. The safety and tolerability were assessed throughout the study. This was a randomized, double-blind, placebo-controlled, monocenter, institutional, phase IIb to III study. In case of efficacy at 6 months, the study would go on for another 6 months on an open design with all patients receiving miglustat. In the absence of efficacy at 6 months, the trial had to be continued for 6 more months with the initial design. After 6 months, efficacy was not superior in patients with miglustat. The independent review board confirmed continuing the study until 12 months. Miglustat treatment was not associated with any improvement/stabilization in behavior problems in patients with mucopolysaccharidosis type III. Miglustat has an acceptable safety profile. However, the study has confirmed that miglustat is able to pass through the blood-brain barrier without significantly decreasing ganglioside levels. [\hyperlink{Miglustat}{PMID: 21658716}, Nathalie Guffon et al., 2011]

\hypertarget{pmid_28577204}{J}uvenile and adult GM1-gangliosidosis are invariably characterized by progressive neurological deterioration. To date only symptomatic therapies are available. We report for the first time the positive results of Miglustat (OGT 918, N-butyl-deoxynojirimycin) treatment on three Italian GM1-gangliosidosis patients. The first two patients had a juvenile form (enzyme activity ≤5\%, GLB1 genotype p.R201H/c.1068 + 1G > T; p.R201H/p.I51N), while the third patient had an adult form (enzyme activity about 7\%, p.T329A/p.R442Q). Treatment with Miglustat at the dose of 600 mg/day was started at the age of 10, 17 and 28 years; age at last evaluation was 21, 20 and 38 respectively. Response to treatment was evaluated using neurological examinations in all three patients every 4-6 months, the assessment of Movement Disorder-Childhood Rating Scale (MD-CRS) in the second patient, and the 6-Minute Walking Test (6-MWT) in the third patient. The baseline neurological status was severely impaired, with loss of autonomous ambulation and speech in the first two patients, and gait and language difficulties in the third patient. All three patients showed gradual improvement while being treated; both juvenile patients regained the ability to walk without assistance for few meters, and increased alertness and vocalization. The MD-CRS class score in the second patient decreased from 4 to 2. The third patient improved in movement and speech control, the distance covered during the 6-MWT increased from 338 to 475 m. These results suggest that Miglustat may help slow down or reverse the disease progression in juvenile/adult GM1-gangliosidosis. [\hyperlink{Miglustat}{PMID: 28577204}, Federica Deodato et al., 2017]

\hypertarget{pmid_34050973}{T}o investigate the efficacy and safety of home-treatment with oral piv-mecillinam or amoxicillin-clavulanate in children with acute pyelonephritis. Children aged over 6 months diagnosed with culture confirmed pyelonephritis at Danish Paediatric Departments were home-treated with piv-mecillinam (tablets) or amoxicillin-clavulanate (liquid or tablets). Follow-up was performed by phone (second treatment day) and clinical review of the patients in the hospital (day three). Four hundred eighteen children were included. In total, 333/418 (80\%) responded well to the initial oral antibiotic treatment. 85/418 (20\%) were changed to another treatment of these 47/418 (11\%) to a second-line oral antibiotic and 38/418 (9\%) to intravenous antibiotics due to insufficient clinical improvement or bacterial resistance. Bacterial resistance was similar for piv-mecillinam and amoxicillin-clavulanate: 4/74 (5\%) versus 33/333 (10\%) (p = 0.22). Insufficient clinical improvement, despite no resistance, primarily occurred in children treated with piv-mecillinam: 16/74 (22\%) versus 28/344 (8\%) (p < 0.001), and predominantly occurred in piv-mecillinam treated children <5 years: 7/20 (35\%) versus 9/54 (17\%) (p < 0.05), potentially because of problems with piv-mecillinam tablets. In the study population no cases of death or septicemia developed after start of initial oral treatment. A home-treatment regime for pyelonephritis in children >6 months is safe; however, during treatment, clinical re-evaluation is required as in 20\% of cases a change in treatment was necessary. [\hyperlink{Miglustat}{PMID: 34050973}, Line Thousig Sehested et al., 2021] (1) For patients with type 1 Gaucher's disease the standard treatment is imiglucerase enzyme replacement therapy, provided in fortnightly intravenous infusions. (2) Miglustat inhibits the synthesis of glucosyl-ceramide, the cerebroside that accumulates in Gaucher's disease. Miglustat is now licensed for oral therapy in patients with mild to moderate type 1 Gaucher's disease and who cannot take imiglucerase, regardless of the reason. (3) The evaluation data we managed to gather (see literature search) includes data from three trials involving a total of 82 patients. One of these trials compared miglustat with ongoing imiglucerase therapy. Miglustat slightly reduced the size of the liver and spleen, and slightly increased the haemoglobin level and platelet count after 18 months. The impact of these effects is unknown, especially on bone disorders. In patients with previous response to imiglucerase, miglustat has not been found to maintain clinical effects in the longer term. (4) Miglustat has many adverse effects, some of which occur very frequently, such as diarrhea (86\%), weight loss (64\%), peripheral neuropathies (19\%), tremor (29\%), and cognitive disorders. Animal studies suggest a risk of reproductive toxicity. (5) In practice, miglustat therapy offers minimal benefits for the few patients who cannot use imiglucerase. The potential advantages of miglustat therapy relative to purely symptomatic treatment must be carefully weighed in individual patients. [\hyperlink{Miglustat}{PMID: 34050973}, Miglustat: new drug. In type 1 Gaucher's disease : a slight benefit after imiglucerase therapy., 2005]

\hypertarget{pmid_22281182}{P}reclinical data suggest that miglustat could restore the function of the cystic fibrosis transmembrane conductance regulator gene in cystic fibrosis cells. Single-center, randomized, double-blind, placebo-controlled, crossover Phase II study in 11 patients (mean±SD age, 26.3±7.7 years) homozygous for the F508del mutation received oral miglustat 200 mgt.i.d. or placebo for two 8-day cycles separated by a 14-day washout period. The primary endpoint was the change in total chloride secretion (TCS) assessed by nasal potential difference. No statistically significant changes in TCS, sweat chloride values or FEV(1) were detected. Pharmacokinetic and safety were similar to those observed in patients with other diseases exposed to miglustat. There was no evidence of a treatment effect on any nasal potential difference variable. Further studies with miglustat need to adequately address criteria for assessment of nasal potential difference. [\hyperlink{Miglustat}{PMID: 22281182}, Anissa Leonard et al., 2012]

\hypertarget{pmid_15505381}{I}t has been shown that treatment with miglustat (Zavesca, N-butyldeoxynojirimycin, OGT 918) improves key clinical features of type I Gaucher disease after 1 year of treatment. This study reports longer-term efficacy and safety data. Patients who had completed 12 months of treatment with open-label miglustat (100-300 mg three times daily) were enrolled to continue with therapy in an extension study. Data are presented up to month 36. Liver and spleen volumes measured by CT or MRI were scheduled every 6 months. Biochemical and haematological parameters, including chitotriosidase activity (a sensitive marker of Gaucher disease activity) were monitored every 3 months. Safety data were also collected every 3 months. Eighteen of 22 eligible patients at four centres entered the extension phase and 14 of these completed 36 months of treatment with miglustat. After 36 months, there were statistically significant improvements in all major efficacy endpoints. Liver and spleen organ volumes were reduced by 18\% and 30\%, respectively. In patients whose haemoglobin value had been below 11.5 g/dl at baseline, mean haemoglobin increased progressively from baseline by 0.55 g/dl at month 12 (NS), 1.28 g/dl at month 24 (p =0.007), and 1.30 g/dl at month 36 (p =0.013). The mean platelet count at month 36 increased from baseline by 22 x 10(9)/L. No new cases of peripheral neuropathy occurred since previously reported. Diarrhoea and weight loss, which were frequently reported during the initial 12-month study, decreased in magnitude and prevalence during the second and third years. Patients treated with miglustat for 3 years show significant improvements in organ volumes and haematological parameters. In conclusion, miglustat was increasingly effective over time and showed acceptable tolerability in patients who continued with treatment for 3 years. [\hyperlink{Miglustat}{PMID: 15505381}, D Elstein et al., 2004]

\hypertarget{pmid_12641681}{T}he bitter taste of midazolam is more acceptable to children when the drug is mixed with fruit juice or syrup. We use a thick grape syrup (Syrpalta), and children are sedated in 10-15 min. A premixed cherry-flavoured midazolam solution (Roche), 2 mg.ml (-1), is currently available. It has been our impression that the premixed midazolam has a slower onset of action. Our aim was to evaluate the effects of the midazolam mixtures (midazolam 0.5 mg.kg (-1), 2 mg.ml (-1)) on children's anxiety, sedation, separation anxiety, mask acceptance, and recovery time. Seventy-six healthy children, 1-4 years of age, scheduled for elective placement of ear tubes, were enrolled. The trial was double-blinded and randomized. For premedication, one group received the premixed midazolam, and a second group received the midazolam/Syrpalta mixture. An independent blinded observer evaluated the children, using anxiety and sedation scales at baseline, at 5, 10 and 15 min and at parental separation. Mask acceptance and awakening time were evaluated. Children who received the midazolam/Syrpalta mixture had less anxiety at 15 min (P = 0.046) and at parental separation (P < 0.001) than those who received the premixed midazolam solution. Mask acceptance was not different. We concluded that the midazolam/Syrpalta mixture has a faster onset of action than the premixed midazolam solution. [\hyperlink{Miglustat}{PMID: 12641681}, Samia N Khalil et al., 2003]

\hypertarget{pmid_27836529}{W}e report data from a prospective, observational study (ZAGAL) evaluating miglustat 100mg three times daily orally. in treatment-naïve patients and patients with type 1 Gaucher Disease (GD1) switched from previous enzyme replacement therapy (ERT). Clinical evolution, changes in organ size, blood counts, disease biomarkers, bone marrow infiltration (S-MRI), bone mineral density by broadband ultrasound densitometry (BMD), safety and tolerability annual reports were analysed. Between May 2004 and April 2016, 63 patients received miglustat therapy; 20 (32\%) untreated and 43 (68\%) switched. At the time of this report 39 patients (14 [36\%] treatment-naïve; 25 [64\%] switch) remain on miglustat. With over 12-year follow-up, hematologic counts, liver and spleen volumes remained stable. In total, 80\% of patients achieved current GD1 therapeutic goals. Plasma chitotriosidase activity and CCL-18/PARC concentration showed a trend towards a slight increase. Reductions on S-MRI (p=0.042) with an increase in BMD (p<0.01) were registered. Gastrointestinal disturbances were reported in 25/63 (40\%), causing miglustat suspension in 11/63 (17.5\%) cases. Thirty-eight patients (60\%) experienced a fine hand tremor and two a reversible peripheral neuropathy. Overall, miglustat was effective as a long-term therapy in mild to moderate naïve and ERT stabilized patients. No unexpected safety signals were identified during 12-years follow-up. [\hyperlink{Miglustat}{PMID: 27836529}, Pilar Giraldo et al., 2018]

\hypertarget{pmid_22976762}{M}iglustat is an oral medication that has approved indication for type I Gaucher disease and Niemann pick disease type C. Usually treatment with Miglustat is associated with occurrence of gastrointestinal side effects similar to carbohydrate maldigestion symptoms. Here, we studied the direct influence of Miglustat on the enzymatic function of the major disaccharidases of the intestinal epithelium. Our findings show that an immediate effect of Miglustat is its interference with carbohydrate digestion in the intestinal lumen via reversible inhibition of disaccharidases that cleave α-glycosidically linked carbohydrates. Higher non physiological concentrations of Miglustat can partly affect lactase activity. We further show that the inhibition of the disaccharidases function by Miglustat is mainly competitive and does not occur via alteration of the enzyme folding. [\hyperlink{Miglustat}{PMID: 22976762}, Mahdi Amiri et al., 2012]

\hypertarget{pmid_10741880}{T}he aim of this article is to review data on the efficacy and safety of montelukast in the treatment of children with asthma. Available published literature, including published abstracts, is reviewed. In patients aged 6 to 14 years with asthma (n = 27), montelukast 5mg demonstrated a significant decrease in exercise-induced bronchoconstriction 20 to 24 hours postdose after 2 days of treatment. For children with chronic asthma, only one study of the regular use of a leukotriene receptor antagonist has been published. The efficacy and safety of montelukast in children aged 6 to 14 years with asthma (n = 336) were studied during an 8-week, double-blind, placebocontrolled trial. There was a significantly greater improvement in forced expiratory volume in 1 second (FEV1) from baseline for the montelukast group (8.23\%) compared with the placebo group (3.58\%). There was a significant decrease in the use of a 3-agonist for symptom relief, as well as in the percentage of days and percentage of patients with asthma exacerbations. An asthma specific quality-of-life (QOL) questionnaire revealed a significant overall improvement in QOL and a significant improvement in the QOL domains for symptoms, activity and emotions in montelukast recipients. There was no significant difference between montelukast and placebo recipients in the frequency of adverse events, with the exception of allergic rhinitis, which was more prevalent in the placebo group. An open label follow-up of patients from the above study was undertaken. The effect of montelukast on FEV1 was consistent for up to 1.4 years, with the increase in FEV1 being not significantly different from that in a small control group treated with inhaled beclomethasone dipropionate. QOL remained significantly improved during the open treatment period. Montelukast appears effective and safe for the treatment of children with asthma. [\hyperlink{Miglustat}{PMID: 10741880}, A Becker et al., 2000]

\hypertarget{pmid_17934951}{L}imited information exists on the toxicity of pediatric ingestions of the drug montelukast used in the treatment of chronic asthma. All ingestions of montelukast involving children age 0-5 yr reported to Texas poison control centers during 2000-2005 were retrieved. For a subset of cases where the final medical outcome and dose in milligrams or milligrams per kilogram were known, the pattern of exposures by final medical outcome and management site was evaluated. There was a total of 3698 cases. Of those cases with a known final medical outcome and dose, the mean dose in milligrams was 42.5 mg (range 0.4-536 mg) and the mean dose in milligrams per kilogram was 3.36 mg/kg (range 0.18-33.71 mg/kg). The final medical outcome was no observed effect in 95\% of the cases and minor effect in the remainder of the cases. The patient was managed on site in 80\% of the cases. The proportion of cases with a minor effect increased from 5\% for ingested dose of < or = 100 mg to 10\% for > 100 mg but was 5\% for dose < or = 5 mg/kg and > 5 mg/kg. The proportion of cases managed with health care facility involvement increased from 15\% for ingested dose of < or = 100 mg to 56\% for > 100 mg and rose from 10\% for dose < or = 5 mg/kg to 47\% for dose > 5 mg/kg. Pediatric montelukast ingestions of doses up to 536 mg or 33.71 mg/kg do not appear likely to result in serious adverse effects and usually can be managed at home. [\hyperlink{Miglustat}{PMID: 17934951}, Mathias B Forrester et al., 2007]

\hypertarget{pmid_15102869}{M}ontelukast is a cysteinyl leukotriene receptor antagonist approved for the treatment of asthma for those ages 1 year old to adult. The purpose of this study was to evaluate the pharmacokinetic comparability of a 4-mg dose of montelukast oral granules in patients > or = 6 to < 24 months old to the 10-mg approved dose in adults. This was an open-label study in 32 patients. Population pharmacokinetic parameters included estimates of AUC(pop), C(max), and t(max). Results were compared with estimates from adults (10-mg film-coated tablet [FCT]). Dose selection criteria were for the 95\% confidence interval (CI) for the AUC(pop) estimate ratio (pediatric/adult 10 mg FCT) to be within comparability bounds of (0.5, 2.00). The AUC(pop) ratio and the 95\% CI for children compared with adults were within the predefined comparability bounds. Observed plasma concentrations were also similar. Based on systemic exposure of montelukast, a 4-mg dose of montelukast appears appropriate for children as young as 6 months of age. [\hyperlink{Miglustat}{PMID: 15102869}, Elizabeth Migoya et al., 2004]

\hypertarget{pmid_10741881}{T}he tolerability of a medication, especially in children with asthma, is linked to a number of key factors. These include clinical effectiveness, adverse effects, frequency of drug regimen, ease and route of administration. and taste. Montelukast is unusual in that, in most countries, a licence for children aged > or =6 years was granted at the same time as the adult licence. This is related to a variety of evidence. which includes pharmacological and adult studies suggesting the specificity and safety of the drug at many times the licensed dose, and a tolerability profile similar to that with placebo or inhaled corticosteroids in both adult and paediatric studies. The most common adverse effects in paediatric studies were headache, asthma and upper respiratory tract infection at rates not statistically significantly different from those with placebo. Up to July 1999, more than 2 million patients worldwide have received montelukast, of whom nearly 220,000 have received the paediatric formulation. In the UK, one prescribing database suggests that, of children who commenced montelukast therapy, less than 25\% discontinued the drug. This implies that montelukast is effective and well tolerated in most children. Adverse effect monitoring by regulatory bodies has revealed little that would not be expected on the basis of the results of clinical trials. Montelukast has been associated with Churg-Strauss syndrome in a very small number of adults. In most. the syndrome was associated with corticosteroid withdrawal, which may have unmasked the condition. Churg-Strauss syndrome has not been reported in children. Its clinical effectiveness, lack of major adverse effects, oral route of administration, palatability and the once-daily regimen combine to make montelukast a generally well tolerated medication in children. [\hyperlink{Miglustat}{PMID: 10741881}, D Price et al., 2000]

\hypertarget{pmid_18548983}{B}ased on the outcome of several randomized controlled trials, the orally active leukotriene receptor antagonist montelukast (Singulair, Merck) has been licensed for treatment of asthma. The drug is favored for treating childhood asthma, where a therapeutic challenge has arisen due to poor compliance with inhalation therapy. To assess the efficiency of and satisfaction with Singulair in asthmatic children under real-life conditions. Montelukast was prescribed for 6 weeks to a cohort of 506 children aged 2 to 18 years with mild to moderate persistent asthma, who were enrolled by 200 primary care pediatricians countrywide. Four clinical correlates of childhood asthma--wheeze, cough, difficulty in breathing, night awakening--were evaluated from patients' diary cards. Due to under-treatment by their physicians, almost 60\% of the children were not receiving controller therapy at baseline. By the end of the study, which consisted of montelukast treatment, a significant improvement over baseline was noted in asthma symptoms and severity, as well as in treatment compliance. The participating pediatricians and parents were highly satisfied with the treatment. The results of this extensive study show that the use of montelukast as monotherapy in children presenting with persistent asthma resulted in a highly satisfactory outcome for themselves, their parents and their physicians. [\hyperlink{Miglustat}{PMID: 18548983}, Israel Amirav et al., 2008]

\hypertarget{pmid_15891924}{M}icturating cystourethrogram (MCUG) is an imaging technique indicated in the diagnosis and follow-up of many diseases. We investigated the reliability and the efficacy of midazolam and chloral hydrate in sedation and anxiolysis during micturating cystourethrogram. Fifty-three children of similar ages (39 girls, 14 boys, mean age of 5.8+/-3.5 years) were randomized to midazolam (n=17), chloral hydrate (n=18) and control groups (n=18). Oral midazolam 0.6 mg/kg or chloral hydrate 25 mg/kg or saline were administered to the study groups 15-30 min prior to the urinary catheterization. Brietkopf and Buttner, Frankl and Houpt scales and Spielberger's State Anxiety Inventory and parent's impressions were used to assess the level of sedation and anxiety. The Brietkopf and Buttner classification of emotional status and Houpt behavior rating scale demonstrated a significantly better emotional status and sedation in the midazolam group when compared to controls (P=0.01 and P=0.018, respectively). The catheterization was described as a more unpleasant and distressing event by the parents of the control and the chloral hydrate groups when compared to the parents of the midazolam group (P<0.05). Bladder capacity and frequency of detection of residual urine were not statistically different between the three study groups (P>0.05). Vital signs did not change significantly in any child. Sedation with midazolam does not have adverse effects on the results of micturating cystourethrogram, while it reduces the discomfort in children undergoing this radiological technique. [\hyperlink{Miglustat}{PMID: 15891924}, Ipek Akil et al., 2005]

\hypertarget{pmid_17609429}{E}nzyme replacement therapy (ERT) with imiglucerase reduces hepatosplenomegaly and improves hematologic parameters in Gaucher disease type 1 within 6-24 months. Miglustat reduces organomegaly, improves hematologic parameters, and reverses bone marrow infiltration. This trial evaluates miglustat in patients clinically stable on ERT. Tolerability of miglustat and imiglucerase, alone and in combination, pharmacokinetic profile, organ reduction, and chitotriosidase activity were assessed. Thirty-six patients stable on imiglucerase were randomized into this phase II, open-label trial. Statistically significant changes from baseline were assessed (paired t test) on primary objectives with secondary analyses on biochemical and safety parameters. Liver and spleen volume were unchanged in switched patients. No significant differences were seen between groups regarding mean change in hemoglobin. Mean change in platelet counts was only significant between miglustat and imiglucerase groups (P = .035). Chitotriosidase activity remained stable. In trial extension, clinical endpoints were generally maintained. Miglustat was well tolerated alone or in combination. Miglustat's safety profile was consistent with previous trials; moreover, no new cases of peripheral neuropathy were observed. Gaucher disease type 1 (GD1) parameters were stable in most switched patients. Combination therapy did not show benefit. Findings suggest miglustat could be an effective maintenance therapy in stabilized patients with GD1. [\hyperlink{Miglustat}{PMID: 17609429}, Deborah Elstein et al., 2007]

\hypertarget{pmid_21779792}{M}iglustat (Zavesca®) is approved for the oral treatment of adult patients with mild to moderate type 1 Gaucher disease (GD1) for whom enzyme replacement therapy is unsuitable, and for the treatment of progressive neurological manifestations in adult and paediatric patients with Niemann-Pick disease type C (NP-C). Gastrointestinal disturbances such as diarrhoea, flatulence and abdominal pain/discomfort have consistently been reported as the most frequent adverse events associated with miglustat during clinical trials and in real-world clinical practice settings. These adverse events are generally mild or moderate in severity, occurring mostly during the initial weeks of therapy. The mechanism underlying these gastrointestinal disturbances is the inhibition by miglustat of intestinal disaccharidase enzymes (mainly sucrase and maltase), leading to sub-optimal hydrolysis of carbohydrates and subsequent osmotic diarrhoea and altered colonic fermentation. Transient decreases in body weight, which are often observed during initial miglustat therapy, are considered likely due to gastrointestinal carbohydrate malabsorption and associated negative caloric balance. While most cases of diarrhoea resolve spontaneously during continued miglustat therapy, diarrhoea also responds well to anti-propulsive medications such as loperamide. Dietary modifications such as reduced consumption of dietary sucrose, maltose and lactose have been shown to improve the gastrointestinal tolerability of miglustat and reduce the magnitude of any changes in body weight, particularly if initiated at or before the start of therapy. Miglustat dose escalation at treatment initiation may also reduce gastrointestinal disturbances. This article discusses these aspects in detail, and provides practical recommendations on how to optimize the gastrointestinal tolerability of miglustat. [\hyperlink{Miglustat}{PMID: 21779792}, Nadia Belmatoug et al., 2011]

\hypertarget{pmid_27599728}{N}iemann-Pick disease type C is a rare inherited neurodegenerative disease involving impaired intracellular lipid trafficking and accumulation of glycolipids in various tissues, including the brain. Miglustat, a reversible inhibitor of glucosylceramide synthase, has been shown to be effective in the treatment of progressive neurological manifestations in pediatric and adult patients with Niemann-Pick disease type C, and has been used in that indication in Europe since 2010. We describe the case of a 16-year-old white French boy with late-infantile-onset Niemann-Pick disease type C who had the unusual presentation of early-onset behavioral disturbance and learning difficulties (aged 5) alongside epileptic seizures. Over time he developed characteristic, progressive vertical ophthalmoplegia, ataxic gait, and cerebellar syndrome; at age 10 he was diagnosed as having Niemann-Pick disease type C based on filipin staining and genetic analysis (heterozygous I1061T/R934X NPC1 mutations). He was commenced on miglustat therapy aged 11 and over the course of approximately 3 years he showed a global improvement as well as improved cognitive and ambulatory function. During this period he remained seizure free on antiepileptic therapy, using valproate and lamotrigine. Miglustat improved the neurological status of our patient, including seizure control. Based on our findings in this patient and previous published data, we discuss the importance of effective seizure control in neurological improvement in Niemann-Pick disease type C, and the relevance of cerebellar involvement as a possible link between these clinical phenomena. Thus the therapeutic efficacy of miglustat could be hypothesized as a substrate reduction effect on Purkinje cells. [\hyperlink{Miglustat}{PMID: 27599728}, Jean-Marie Cuisset et al., 2016]

\hypertarget{pmid_20885413}{C}onscious sedation for young children is a rapidly developing area of clinical activity. Many studies have shown positive results using oral midazolam on children. These case series investigated oral midazolam conscious sedation as an alternative to general anaesthesia in a clinical service setting. The purpose of this work was to determine the safety and efficacy of oral midazolam for conscious sedation in children undergoing dental treatment. Patients were selected by colleagues for treatment under oral sedation. The main general criteria were weight below 36 kilos and ASA I, II, or III. Midazolam 0.5 mg/kg was administered orally. A pulse oximeter was applied to a finger to monitor vital signs and the Houpt scale was used to assess behaviour. A total of 510 children aged between 13 months and 11 years were included. The behaviour of 379 (74\%) was excellent or very good. The pulse rate and peripheral oxygenation were within the normal range for all patients. The main adverse effects were diplopia and post-sedation dysphoria. Oral midazolam is a safe and effective method of sedation although some children were agitated and distressed either during or after treatment. Parents need to be warned about this. [\hyperlink{Miglustat}{PMID: 20885413}, L Lourenço-Matharu et al., 2010]

\hypertarget{pmid_24863482}{M}iglustat is an oral medication for treatment of lysosomal storage diseases such as Gaucher disease type I and Niemann Pick disease type C. In many cases application of Miglustat is associated with symptoms similar to those observed in intestinal carbohydrate malabsorption. Previously, we have demonstrated that intestinal disaccharidases are inhibited immediately by Miglustat in the intestinal lumen. Nevertheless, the multiple functions of Miglustat hypothesize long term effects of Miglustat on intracellular mechanisms, including glycosylation, maturation and trafficking of the intestinal disaccharidases. Our data show that a major long term effect of Miglustat is its interference with N-glycosylation of the proteins in the ER leading to a delay in the trafficking of sucrase-isomaltase. Also association with lipid rafts and plausibly apical targeting of this protein is partly affected in the presence of Miglustat. More drastic is the effect of Miglustat on lactase-phlorizin hydrolase which is partially blocked intracellularly. The de novo synthesized SI and LPH in the presence of Miglustat show reduced functional efficiencies according to altered posttranslational processing of these proteins. However, at physiological concentrations of Miglustat (≤50 μM) a major part of the activity of these disaccharidases is found to be still preserved, which puts the charge of the observed carbohydrate maldigestion mostly on the direct inhibition of disaccharidases in the intestinal lumen by Miglustat as the immediate side effect.  [\hyperlink{Miglustat}{PMID: 24863482}, Mahdi Amiri et al., 2014] The efficacy and safety of intravenous and sequential intravenous-oral clavulanate-potentiated amoxycillin therapy was evaluated in 71 hospitalized paediatric patients, one month to 16 years of age. The infections treated included peritonsillar abscess (2 patients), purulent tracheitis (1), acute epiglottitis (24), pneumonia (31), pansinusitis (4), mastoiditis (1), cellulitis (4), lymphadenitis (2) and pyelonephritis (2). The severity of disease was rated as moderate in 31 patients (44\%), and as severe in 40 (56\%). Bacterial pathogens could be cultured in 26 cases (37\%). The response to therapy was prompt and followed by clinical cure in each patient. Adverse drug effects included phlebitis (in 6\%), mild gastrointestinal complaints (6\%), rash (4\%) and transient neutropenia and elevation of transaminases (one case each). It is concluded that amoxycillin/clavulanate is effective and safe treatment for bacterial infections of the respiratory tract, urinary tract, skin or soft tissues in children. [\hyperlink{Miglustat}{PMID: 24863482}, U B Schaad et al., 1987]

\hypertarget{pmid_17766511}{A} recurring epidemic of asthma exacerbations in children occurs annually in September in North America when school resumes after summer vacation. Our goal was to determine whether montelukast, added to usual asthma therapy, would reduce days with worse asthma symptoms and unscheduled physician visits of children during the September epidemic. A total of 194 asthmatic children aged 2 to 14 years, stratified according to age group (2-5, 6-9, and 10-14 years) and gender, participated in a double-blind, randomized, placebo-controlled trial of the addition of montelukast to usual asthma therapy between September 1 and October 15, 2005. Children randomly assigned to receive montelukast experienced a 53\% reduction in days with worse asthma symptoms compared with placebo (3.9\% vs 8.3\%) and a 78\% reduction in unscheduled physician visits for asthma (4 [montelukast] vs 18 [placebo] visits). The benefit of montelukast was seen both in those using and not using regular inhaled corticosteroids and among those reporting and not reporting colds during the trial. There were differences in efficacy according to age and gender. Boys aged 2 to 5 years showed greater benefit from montelukast (0.4\% vs 8.8\% days with worse asthma symptoms) than did older boys, whereas among girls the treatment effect was most evident in 10- to 14-year-olds (4.6\% [montelukast] vs 17.0\% [placebo]), with nonsignificant effects in younger girls. Montelukast added to usual treatment reduced the risk of worsened asthma symptoms and unscheduled physician visits during the predictable annual September asthma epidemic. Treatment-effect differences observed between age and gender groups require additional investigation. [\hyperlink{Miglustat}{PMID: 17766511}, Neil W Johnston et al., 2007]

\hypertarget{pmid_11167954}{M}ontelukast is a leukotriene receptor antagonist administered orally once daily for treatment of chronic asthma in adults and children. A comprehensive analysis of safety data from double-blind, randomized, placebo-controlled trials with montelukast has not been previously reported. A pooled analysis of safety data from 11 multicentre, randomized, controlled montelukast Phase IIb and III trials and five long-term extension studies was performed. A total of 3386 adult patients (aged 15-85 years) and 336 paediatric patients (aged 6-14 years) were enrolled in the trials; 2031 adults received montelukast for up to 4.1 years, and 257 children received montelukast for up to 1.8 years. Summary statistics comparing incidences of adverse events among treatment groups were calculated. The overall incidence of clinical and laboratory adverse events among montelukast-treated patients, both adult and paediatric, was similar to that among patients receiving placebo. There were no clinically relevant differences in individual adverse events, including infectious upper respiratory conditions and transaminase elevations, between montelukast and placebo groups. Discontinuations due to adverse events occurred with similar frequencies during placebo, montelukast and inhaled beclomethasone therapy. No dose-related adverse effects of montelukast were observed in adults treated with dosages as high as 200 mg per day (20 times the recommended dose) for 5 months. This tolerability profile montelukast observed in clinical trials has been generally reflected in the post-marketing safety experience seen to date. These data indicate a tolerability profile for montelukast similar to placebo during both short-term and long-term administration, even at doses substantially higher than the recommended clinical dose of 10 mg once daily for adults and 5 mg once daily for children aged 6-14 years. [\hyperlink{Miglustat}{PMID: 11167954}, W Storms et al., 2001]

\hypertarget{pmid_38085143}{O}ral Montelukast is recommended as maintenance therapy for persistent asthma, but there is controversy regarding its effectiveness in controlling asthma attacks. The present study was conducted to investigate the clinical efficacy of oral Montelukast for asthma attacks in children. This study was conducted as a double-blind placebo-controlled clinical trial on 80 children aged 1-14 years with asthma who were admitted to the emergency department of Bahrami Children's Hospital (Tehran, Iran) during one year. Patients were randomly divided into case and control groups. In addition to the standard asthma attack treatment, Montelukast was prescribed in the case group and placebo in the control group for one week. Patients were evaluated in terms of asthma attack severity score and oxygen saturation percentage (SpO2) in room air as primary outcomes 1, 4, 8, 24 and 48 hours after admission. In the first 48 hours, there was no significant difference in the score of asthma attack severity and SpO2 between the case and control groups. There was no significant difference between the groups in terms of length of hospitalization or number of admissions to the intensive care unit. None of the patients were re-hospitalized after discharge. The results of this study showed that the use of Montelukast along with the standard treatment of asthma attacks in children has no added benefit. [\hyperlink{Miglustat}{PMID: 38085143}, Mohsen Jafari et al., 2023]

\hypertarget{pmid_7717236}{M}idazolam is a relatively short-acting water-soluble benzodiazepine that provides anxiolysis and anterograde amnesia and can be given orally with few adverse effects. We evaluated the benefit and safety of oral midazolam for sedation of young children during voiding cystourethrography or nuclear cystography. For 3.5 years, a highly selected group of 98 children, ages 23 months to 9 years (mean, 4 years), were given oral midazolam 0.6 mg/kg 20-30 min before cystourethrography or nuclear cystography. These children either had been frightened by a previous catheterization (39\%) or seemed particularly frightened during an examination of their genitals in the office (61\%). A control group of 25 children, similar in age to the study group, did not receive midazolam before cystourethrography. Parents were interviewed to assess their child's recollection of the procedure. Voiding dynamics were assessed by evaluating the postvoiding radiograph. Of the midazolam-treated patients, 60\% had no recollection of the study, and 31\% remembered part or all of the study but did not have a negative experience. No significant change in vital signs or oxygen saturation was observed in any child. In the control group, 24 (96\%) of 25 children remembered the cystographic examination (p < .01). Behavioral side effects occurred in 12\% of the children receiving midazolam and consisted primarily of combative behavior as the medication was wearing off. Ninety-five percent of the parents indicated that they would want their child to have midazolam again if the cystography needed to be repeated. Of the children receiving midazolam, 76\% had little or no residual urine after voiding, compared with 72\% of the control group (no significant difference). In children who have been or are likely to be excessively frightened during cystourethrography or nuclear cystography, midazolam usually provides satisfactory amnesia and anxiolysis with few side effects or adverse impact on voiding dynamics. [\hyperlink{Miglustat}{PMID: 7717236}, J S Elder et al., 1995]

\section*{Mycobutin}
\subsection*{Result}
\subsubsection*{Answer}

Unknown

\subsubsection*{{Explanation}}
\hypertarget{Mycobutin}
After reviewing all available abstracts, there is no evidence from any abstract that specifically addresses the safety of Mycobutin (also known as rifabutin) in children. None of the abstracts provided are about Mycobutin or its use in pediatric populations. Therefore, based on the abstracts available, there is not enough data to determine whether Mycobutin is safe or unsafe for use in children of any age range.

\subsection*{Abstracts}
\hypertarget{pmid_23958810}{M}icafungin is an echinocandin with proven efficacy against a broad range of fungal infections, including those caused by Candida spp. To evaluate the safety and pharmacokinetics of once-daily 3 mg/kg and 4.5 mg/kg micafungin in children with proven, probable or suspected invasive candidiasis. Micafungin safety and pharmacokinetics were assessed in 2 phase I, open-label, repeat-dose trials. In Study 2101, children aged 2-16 years were grouped by weight to receive 3 mg/kg (≥25 kg) or 4.5 mg/kg (<25 kg) intravenous micafungin for 10-14 days. In Study 2102, children aged 4 months to <2 years received 4.5 mg/kg micafungin. Study protocols were otherwise identical. Safety was analyzed in 78 and 9 children in Studies 2101 and 2102, respectively. Although adverse events (AEs) were experienced by most children (2101: n=62; 2102: n=9), micafungin-related AEs were less common (2101: n=28; 2102: n=1), and the number of patients discontinuing due to AEs was low (2101: n=4; 2102: n=1). The most common micafungin-related AEs were infusion-associated symptoms, pyrexia and hypomagnesemia (Study 2101), and liver function abnormalities (Study 2102). The micafungin pharmacokinetic profile was similar to that seen in other studies conducted in children, but different than that observed in adults. In this small cohort of children, once-daily doses of 3 mg/kg and 4.5 mg/kg micafungin were well tolerated. Pharmacokinetic data will be combined in a population pharmacokinetic analysis to support US dosing recommendations in children. [\hyperlink{Mycobutin}{PMID: 23958810}, Daniel K Benjamin et al., 2013]

\hypertarget{pmid_25929612}{L}imited data are available about the safety and efficacy of micafungin in children. A postmarketing surveillance study was conducted to assess the safety and effectiveness of micafungin, an echinocandin antifungal, in pediatric patients. A prospective multicenter postmarketing observational study was carried out between October 2006 and September 2008 in Japan. Pediatric patients under 16 years received an intravenous infusion of micafungin at a dose of 1 mg/kg for candidiasis and 1 to 3 mg/kg for aspergillosis, with the option of increasing the dose if required to 6 mg/kg once daily. All adverse events were recorded. A total of 201 pediatric patients were enrolled. There were 55 adverse drug reactions reported among 42 of 190 patients evaluated for safety (22.1\%); the most frequently reported adverse drug reaction was hepatobiliary disorders. No adverse drug reactions were reported in 18 neonates (aged below 4 wk). The overall clinical response rate in 91 patients evaluated for efficacy was 86.8\%. The response rate in neonates was 90.0\%, and there were no differences in the response rate by age. Micafungin was found to have sufficient safety and effectiveness for the treatment of fungal infections in pediatric patients with various backgrounds.  [\hyperlink{Mycobutin}{PMID: 25929612}, Chie Kobayashi et al., 2015] Micafungin is an echinocandin approved by the European Medicines Evaluation Agency for the treatment of invasive candidiasis in children, including premature infants born before 29 weeks of pregnancy, and as prophylaxis in children undergoing hematopoietic stem-cell transplantation or patients at risk of prolonged neutropenia. This drug has good activity in several Candida spp., including those resistant to fluconazole. Although micafungin is active against Aspergillus spp., it has been used mainly in combination therapy for invasive aspergillosis. There is ample information on the use of micafungin in children, including neonates, and this drug is the only echinocandin approved for use in infants aged less than 3 months. The efficacy, pharmacokinetics and safety of micafungin have been evaluated in phase II and III clinical trials in children, in which its efficacy and safety were demonstrated in comparison with liposomal amphotericin B and fluconazole. The pharmacokinetic profile of micafungin in children allows once daily intravenous administration, with greater clearance than in adults, and consequently pediatric doses are relatively higher. The most appropriate dose in children weighing less than 40 kg is 2 mg/kg/day in the treatment of invasive candidiasis and 1 mg/kg/day as prophylaxis in children undergoing hematopoietic stem-cell transplantation. Doses in neonates should be higher. In premature infants, the most appropriate doses to achieve levels in the brain parenchyma are 7 mg/kg/day and 10 mg/kg/day in those weighing more and less than 1,000 g, respectively. Micafungin has few drug-drug interactions and an acceptable safety profile. Withdrawal of this drug due to adverse effects is rare, although transaminase monitoring is recommended during treatment, as well as evaluation of the risk-benefit balance in patients with liver disease or concomitant administration of hepatotoxic drugs. [\hyperlink{Mycobutin}{PMID: 25929612}, José Tomás Ramos Amador et al., 2011]

\hypertarget{pmid_11880086}{T}o test the safety and efficacy of extended-release oxybutynin in children with bladder dysfunction. The efficacy of oxybutynin in children has been limited by side effects. A new extended-release formulation of oxybutynin has some benefits versus traditional oxybutynin but has never been evaluated in children. A retrospective study was performed on 25 children who had been treated with extended-release oxybutynin. Fourteen had neurogenic bladder dysfunction and 11 had urinary frequency and urgency and urge incontinence but no neurologic abnormalities. Patients and families were asked to semiquantitatively (0 to 10 grading with 10 = severe) assess the effects of the medication on efficacy, as well as side effects and compliance with medication schedules. All 25 patients had improvement in incontinence and/or voiding dysfunction on extended-release oxybutynin. Twelve (48\%) experienced no side effects. Of the 13 who did, 10 complained of dry mouth (grade 4.6 plus minus 0.5), 4 had constipation (grade 5.8 plus minus 1.8), 4 had heat intolerance (grade 5.1 plus minus 0.9), and 3 had drowsiness (grade 5.3 plus minus 2.4). Of patients previously treated with oxybutynin, the extended-release oxybutynin was equally or more efficacious and had the same or fewer side effects, especially less dry mouth. Families reported much better patient compliance with the medication regimen using extended-release oxybutynin compared with oxybutynin. Patient and family satisfaction was very high, and 21 of 25 have continued using the medication. Extended-release oxybutynin is safe and efficacious in children. In this preliminary evaluation, it had benefits over traditional, immediate-release oxybutynin. [\hyperlink{Mycobutin}{PMID: 11880086}, Katrin Youdim et al., 2002]

\hypertarget{pmid_22814964}{W}e report 6 pediatric cases of tuberculosis caused by Mycobacterium tuberculosis and treated them with levofloxacin or moxifloxacin in the mother-child unit of a university hospital in France between 2005 and 2011. We assess the clinical efficacy and safety of fluoroquinolones and the benefit-risk ratio for their use as second-line antituberculosis drugs in children and adolescents. [\hyperlink{Mycobutin}{PMID: 22814964}, Jean-Vannak Chauny et al., 2012]

\hypertarget{pmid_18611612}{T}he safety and efficacy of cefetamet pivoxil, an oral cephalosporin of the third generation, have been studied in open, prospective, randomized comparative, clinical trials including 301 toddlers (children aged 1 to 2 years) with upper and lower respiratory tract infections, and urinary tract infections. Cefetamet pivoxil (CAT) syrup formulation was given to 177 toddlers either in the standard dose of 10 mg/kg b.i.d. [n = 116] or 20 mg/kg b.i.d. [n = 61] and 124 toddlers have been treated with comparator drugs [cefaclor, n = 98; phenoxymethylpenicillin, n = 18; amoxicillin plus clavulanic acid; n = 8]. The treatment period was 7 days mainly, except for pharyngotonsillitis for which the treatment duration was 7 or 10 days. The assessment of treatment was based on clinical signs and symptoms primarily in infections of lower respiratory tract and acute otitis media, whereas in patients with pharyngotonsillitis and acute urinary tract infections the bacteriological findings were the main evaluation criteria. The overall therapeutic outcome was successful in 148 (95.4\%) of the 155 toddlers to whom CAT was administered and in 87 (85.3\%) out of 102 toddlers receiving standard drugs. Adverse events of mild to moderate severity, mainly of gastro-intestinal type (vomiting or diarrhoea) occurred in 14.7\% in the patient group receiving CAT, 11.2\% in the toddlers receiving the standard dose of CAT, and in 12.9\% with the comparator drugs. From the data presented it is concluded that cefetamet pivoxil is efficient and safe in toddlers presenting with community-acquired respiratory and urinary infections mainly caused by S. pneumoniae, H. influenzae, Group A beta-haemolytic streptococci, M. catarrhalis, E. coli, Proteus spp. and K. pneumoniae. [\hyperlink{Mycobutin}{PMID: 18611612}, A Chibante et al., 1994]

\hypertarget{pmid_19890251}{G}iven the risk of central nervous system infection, relatively high weight-based echinocandin dosages may be required for the successful treatment of invasive candidiasis and candidemia in young infants. This open-label study assessed the safety and pharmacokinetics (PK) of micafungin in 13 young infants (>48 h and <120 days of life) with suspected candidemia or invasive candidiasis. Infants of body weight > or =1,000 and <1,000 g received 7 and 10 mg/kg/day, respectively, for a minimum of 4-5 days. In the 7-mg/kg/day group, the mean baseline weight and gestational age were 2,101 g and 30 weeks, respectively; in the 10-mg/kg/day group, they were 688 g and 25 weeks, respectively. The median pharmacokinetic values for the 7- and 10-mg/kg/day groups, respectively, were as follows: area under the concentration-time curve from 0 to 24 h (AUC(0-24)), 258.1 and 291.2 microg x h/ml; clearance at steady state adjusted for body weight, 0.45 and 0.57 ml/min/kg; maximum plasma concentration, 23.3 and 24.9 micro g/ml; and volume of distribution at steady state adjusted for body weight, 341.4 and 542.8 ml/kg. No deaths or discontinuations from treatment occurred. These data suggest that micafungin dosages of 7 and 10 mg/kg/day are well tolerated and provide exposure levels that have been shown (in animal models) to be adequate for central nervous system coverage. [\hyperlink{Mycobutin}{PMID: 19890251}, D K Benjamin et al., 2010]

\hypertarget{pmid_11350411}{T}o evaluate the incidence of side-effects of oral and intravesical oxybutynin chloride in children with meningomyelocele (MMC) and a neurogenic bladder. The study comprised 225 children with a neurogenic bladder from MMC who were evaluated with urodynamic testing and voiding cysto-urethrography to identify those at high risk of upper tract damage. In all, 101 children (mean age 4.2 years, range 0.25-10) had unco-ordinated detrusor-sphincter function and low compliance; they were treated with either oral or intravesical oxybutynin and clean intermittent catheterization. Of the 101 patients, 67 were treated with oral oxybutynin; in 11 the treatment was discontinued because of the side-effects. The other 34 patients used both clean intermittent catheterization and intravesical oxybutynin. In this group there were side-effects in six patients, including drowsiness, hallucinations and cognitive changes. Oral and intravesical oxybutynin is effective for managing neurogenic bladder dysfunction, but intravesical administration is safer and better tolerated than oral oxybutynin in the treatment of children with MMC. However, adverse effects such as cognitive impairment can also occur in children treated with intravesical oxybutynin and these patients must be closely monitored because these effects may differ from those with oral administration. [\hyperlink{Mycobutin}{PMID: 11350411}, P Ferrara et al., 2001]

\hypertarget{pmid_8649608}{P}henytoin is widely used for the prevention and treatment of acute seizures in children. Although it has the advantage of being available in parenteral form, it cannot be given through the i.m. route. Furthermore, problems with venous accessibility and maintenance may complicate i.v. administration of phenytoin in newborns and very sick infants. Fosphenytoin, a new phenytoin prodrug, can be safely administered through the i.m. route, and, because of the physical characteristics of its formulation, it offers advantages over phenytoin for i.v. administration. Clinical studies with i.v. and i.m. fosphenytoin demonstrate that the efficacy, safety, and pharmacokinetics of this drug are similar in 5- to 18-year-old children and in young adults. The safety and pharmacokinetic profile of i.v. and i.m. fosphenytoin in younger children and infants is currently being investigated. [\hyperlink{Mycobutin}{PMID: 8649608}, J M Pellock et al., 1996]

\hypertarget{pmid_18818954}{A} randomized, open, coordinated multi-center trial compared the bacteriological and clinical efficacy and safety of orally administered ceftibuten and trimethoprim-sulfamethoxazole (TMP-SMX) in children with febrile urinary tract infection (UTI). Children aged 1 month to 12 years presenting with presumptive first-time febrile UTI were eligible for enrollment. A 2:1 assignment to treatment with ceftibuten 9 mg/kg once daily (n = 368) or TMP-SMX (3 mg + 15 mg)/kg twice daily (n = 179) for 10 days was performed. Escherichia coli was recovered in 96\% of the cases. Among the E. coli isolates, 14\% were resistant to TMP-SMX but none to ceftibuten. In the modified intention-to-treat population, the bacteriological elimination rates at follow-up did not differ significantly between patients treated with ceftibuten and those treated with TMP-SMX [91 vs. 95\%, with a 95\% confidence interval (CI) for difference of -9.7 to 1.0]. However, the clinical cure rate was significantly higher among those treated with ceftibuten (93 vs. 83\%, with a 95\% CI for difference of 2.4 to 17.0). Adverse events were similar for both regimens and consisted mainly of gastrointestinal disturbances. In conclusion, ceftibuten is a safe and effective drug for the empirical treatment of febrile UTI in young children. [\hyperlink{Mycobutin}{PMID: 18818954}, Staffan Mårild et al., 2009]

\hypertarget{pmid_31977308}{W}e sought to assess and compare safety and efficacy of fesoterodine and oxybutynin extended-release in the treatment of pediatric overactive bladder (OAB). We conducted a non-inferiority, randomized, double-blind, crossover trial comparing fesoterodine 4-8 mg and oxybutynin 10-20 mg once daily (QD) in children with OAB aged 5-14 years (2015-2018). Every child received the first medication for eight weeks, followed by crossover to the second antimuscarinic after a three-days washout. Dose up-titration was possible at mid-course. Patients could enter a fesoterodine 12-month extension. Endpoints were assessed through changes on voiding diaries, Patient's Perception of Bladder Condition score (PPBC), adverse events, vital signs, electrocardiogram, post-void residual, urinalysis, and blood tests. The Wilcoxon rank sum and Wilcoxon signed rank tests were used for statistical analysis. A total of 62 patients were randomized (two early dropouts). Expected class effects (dry mouth/constipation) were present but no significant difference was observed. There was a 10.1 beats/minute increase in heart rate with fesoterodine (p<0.01) (oxybutynin-1.9 beats/min; p=non-significant [ns]). No life-threatening or serious adverse events occurred. Efficacy was similar for both drugs. Bladder capacity improved over the 16 months of the study; baseline capacity of 125 mL (44.5\% expected bladder capacity for age [\%EBC]) to 171 mL (53.9 \%EBC) at the end of the extension phase. No clinical or statistical difference was shown between efficacy measures for fesoterodine or oxybutynin. The use of fesoterodine or oxybutynin appear safe and effective for the treatment of OAB in children. Based on our study, long-term treatment to achieve the ultimate goal of urinary continence is needed in this population. [\hyperlink{Mycobutin}{PMID: 31977308}, Sophie Ramsay et al., 2020]

\hypertarget{pmid_33423814}{I}mmunosuppressants are prescribed for pediatric uveitis in cases of severe involvement affecting the prognosis for vision or life, in cases of recurrent or chronic uveitis to achieve corticosteroid sparing, or in cases of corticosteroid resistance. Immunosuppressants used in children include antimetabolites (methotrexate, mycophenolate mofetil, azathioprine), cyclosporine, tacrolimus, and biologics, including infliximab, adalimumab, anakinra, canakinumab, and tocilizumab. The mechanisms of action and indications of the various immunosuppressants are described in this review. [\hyperlink{Mycobutin}{PMID: 33423814}, N Stolowy et al., 2021]

\hypertarget{pmid_29596219}{E}chinocandins are recommended for the treatment of suspected or confirmed invasive candidiasis (IC) in adults. Less is known about the use of echinocandins for the management of IC in children. The aim of this study was to investigate the overall efficacy and safety of echinocandin class in neonatal and pediatric patients with IC. PubMed, Cochrane Central, Scopus and Clinical trial registries were searched up to July 27, 2017. Eligible studies were randomized controlled trials that evaluated the efficacy and safety of any echinocandin versus agents of other antifungal classes for the treatment of IC in pediatric patients. The primary outcome was treatment success with resolution of symptoms and signs, and absence of IC. In the meta-analysis a random effects model was used, and the odds ratio (OR) and 95\% confidence intervals (CIs) were calculated. Four randomized clinical trials (324 patients), 2 confirmed IC (micafungin vs. liposomal amphotericin B (L-AmB) and caspofungin vs. L-AmB) and 2 empirical therapy trials (caspofungin vs. deoxycholate amphotericin B and caspofungin vs. L-AmB) were included. There was no significant difference between echinocandins and comparator in terms of treatment success (OR = 1.61, 95\% CI: 0.74-3.50) and incidence of treatment-related adverse events (OR = 0.70, 95\% CI: 0.39-1.26). However, fewer children treated with echinocandins discontinued treatment because of adverse events than amphotericin B formulations (OR = 0.26, 95\% CI: 0.08-0.82, P = 0.02). In the treatment of IC in children, echinocandins show non-inferior efficacy compared with amphotericin B formulations with fewer discontinuations than in comparator arm. [\hyperlink{Mycobutin}{PMID: 29596219}, Magdalini Tsekoura et al., 2019]

\hypertarget{pmid_3288951}{T}he effectiveness of oxybutynin in the treatment of primary enuresis was evaluated in a double-blind study. A total of 30 children (25 boys, five girls), at least 6 years of age, with primary enuresis and no daytime incontinence or history of other urinary tract problems were selected at random from an enuresis clinic population. The study was explained to the families and they were told how to keep records of nocturnal bed-wetting episodes and side effects. The patients were treated with a 10 mg of oxybutynin at suppertime for 28 days. Before or after the treatment period, all children received an identical placebo for 4 weeks. Two-sided paired t tests were used to compare frequency of nocturnal enuresis. Frequency during the drug regimen did not differ significantly from that during the placebo study. There were no differences in findings between boys and girls or between children who had previously taken imipramine and those who had not. The study showed no evidence that oxybutynin is effective in treating primary enuresis. [\hyperlink{Mycobutin}{PMID: 3288951}, J S Lovering et al., 1988]

\hypertarget{pmid_19740527}{E}noxaparin, a low molecular weight heparin (LMWH), is frequently used for the prevention and treatment of thromboembolic complications in infants and children (Sutor et al., 2004 [1]). Injection pain and the fear and anxiety associated with needle phobia in the pediatric population are well documented. Best practice pediatric pain management standards of care recommend mitigating the child's pain experience whenever possible. The use of topical anesthetics such as liposomal-lidocaine 4\% results in a rapid onset of anesthesia, minimal blanching, without vasoconstriction (Koh et al., 2004 [2]) or risk of methemoglobinemia. Topical lidocaine has been used to reduce the injection pain of enoxaparin, but there is no data available examining whether it will interfere with the absorption of LMWH. To determine if the topical lidocaine, Maxilene, interferes with enoxaparin absorption as measured by peak anti-Xa levels. Infants and children clinically prescribed enoxaparin were eligible for this study. Children in group 1 were pre-treated with Maxilene prior to enoxaparin injection on day 1 with no Maxilene pre-treatment on day 2. For group 2, the order was reversed. Peak anti-Xa levels were measured following each enoxaparin dose and were compared between the groups. 26 children of ages 14d-16 y (median 6.7 months) were enrolled. Anti-Xa levels following topical lidocaine administration were 0.070 U/mL (95\%CI 0.025; 0.114) lower than without prior topical lidocaine administration. Anti-Xa levels on the second day were on average 0.013 U/mL (95\%CI -0.066; 0.040) higher compared to day one regardless of the order of topical lidocaine administration. There were no reported incidences of local reactions such as redness, hives or blanching. Topical lidocaine (Maxilene) administration before enoxaparin injection results in a small, clinically non-significant, reduction in anti-Xa levels. [\hyperlink{Mycobutin}{PMID: 19740527}, S M Duncan et al., 2010]

\hypertarget{pmid_36302965}{E}arly supports to enhance social development in children with autism are widely promoted. While oxytocin has a crucial role in mammalian social development, its potential role as a medication to enhance social development in humans remains unclear. We investigated the efficacy, tolerability, and safety of intranasal oxytocin in young children with autism using a double-blind, randomized, placebo-controlled, clinical trial, following a placebo lead-in phase. A total of 87 children (aged between 3 and 12 years) with autism received 16 International Units (IU) of oxytocin (n = 45) or placebo (n = 42) nasal spray, morning and night (32 IU per day) for twelve weeks, following a 3-week placebo lead-in phase. Overall, there was no effect of oxytocin treatment over time on the caregiver-rated Social Responsiveness Scale (SRS-2) (p = 0.686). However, a significant interaction with age (p = 0.028) showed that for younger children, aged 3-5 years, there was some indication of a treatment effect. Younger children who received oxytocin showed improvement on caregiver-rated social responsiveness ( SRS-2). There was no other evidence of benefit in the sample as a whole, or in the younger age group, on the clinician-rated Clinical Global Improvement Scale (CGI-S), or any secondary measure. Importantly, placebo effects in the lead-in phase were evident and there was support for washout of the placebo response in the randomised phase. Oxytocin was well tolerated, with more adverse side effects reported in the placebo group. This study suggests the need for further clinical trials to test the benefits of oxytocin treatment in younger populations with autism.Trial registration www.anzctr.org.au (ACTRN12617000441314). [\hyperlink{Mycobutin}{PMID: 36302965}, Adam J Guastella et al., 2023]

\hypertarget{pmid_36043350}{W}e investigated the efficacy and safety of fluoxetine, a selective serotonin reuptake inhibitor, for treating refractory primary monosymptomatic nocturnal enuresis in children. Children 8-18 years old with severe primary monosymptomatic nocturnal enuresis unresponsive to alarm therapy, desmopressin, and anticholinergics were screened for eligibility. After excluding children with daytime urinary symptoms, constipation, underlying urological, neuropsychiatric, endocrinological, or cardiac conditions, patients were randomly and equally assigned to 10 mg fluoxetine once daily or placebo for 12 weeks. The primary outcome was treatment response according to the International Children's Continence Society terminology. Treatment-related adverse effects and nighttime arousal were secondary outcomes. A total of 150 children were enrolled, of whom 110 (56 in fluoxetine group and 54 in placebo group) with a mean age of 11.8 (SD 2.46) years were finally analyzed. After 4 weeks, 7.1\% and 66.1\% of the fluoxetine group achieved complete response and partial response (defined as 50\%-99\% reduction of the number of wet nights), respectively, versus 0\% and 16.7\% of the placebo group ( Fluoxetine is safe treatment for refractory primary monosymptomatic nocturnal enuresis in children with good initial response that declines at 12 weeks. [\hyperlink{Mycobutin}{PMID: 36043350}, Mohamed Hussiny et al., 2022]

\hypertarget{pmid_32160320}{E}valuate technical success, tolerability, and safety of lidocaine iontophoresis and tympanostomy tube placement for children in an office setting. Prospective individual cohort study. This prospective multicenter study evaluated in-office tube placement in children ages 6 months through 12 years of age. Anesthesia was achieved via lidocaine/epinephrine iontophoresis. Tube placement was conducted using an integrated and automated myringotomy and tube delivery system. Anxiolytics, sedation, and papoose board were not used. Technical success and safety were evaluated. Patients 5 to 12 years old self-reported tube placement pain using the Faces Pain Scale-Revised (FPS-R) instrument, which ranges from 0 (no pain) to 10 (very much pain). Children were enrolled into three cohorts with 68, 47, and 222 children in the Operating Room (OR) Lead-In, Office Lead-In, and Pivotal cohorts, respectively. In the Pivotal cohort, there were 120 and 102 children in the <5 and 5- to 12-year-old age groups, respectively, with a mean age of 2.3 and 7.6 years, respectively. Bilateral tube placement was indicated for 94.2\% of children <5 and 88.2\% of children 5 to 12 years old. Tubes were successfully placed in all indicated ears in 85.8\% (103/120) of children <5 and 89.2\% (91/102) of children 5 to 12 years old. Mean FPS-R score was 3.30 (standard deviation [SD] = 3.39) for tube placement and 1.69 (SD = 2.43) at 5 minutes postprocedure. There were no serious adverse events. Nonserious adverse events occurred at rates similar to standard tympanostomy procedures. In-office tube placement in selected patients can be successfully achieved without requiring sedatives, anxiolytics, or papoose restraints via lidocaine iontophoresis local anesthesia and an automated myringotomy and tube delivery system. 2b Laryngoscope, 130:S1-S9, 2020. [\hyperlink{Mycobutin}{PMID: 32160320}, Lawrence R Lustig et al., 2020]

\hypertarget{pmid_16094064}{W}e prospectively evaluated the efficacy of a combination of desmopressin and oxybutynin for treating children with nocturnal enuresis, compared to the single drugs imipramine and desmopressin. We enrolled 158 patients from 2003 to 2004. Children were randomly assigned to 1 of 3 groups and treated with desmopressin, imipramine or a combination of desmopressin plus oxybutynin. Of these patients 145 (100 boys and 45 girls, mean age 7.8 +/- 2.5 years, range 5 to 15) were followed for more than 6 months. Efficacy was measured at 1, 3 and 6 months in terms of average enuretic frequency, 5-scale response based on change in nocturnal enuretic frequency after treatment and posttreatment enuretic frequency as a percentage of pretreatment baseline frequency. The latter efficacy was classified according to daytime voiding symptoms. Statistical evaluation was performed using chi-square tests and ANOVA. Of the 145 children followed 48 received combination therapy, 49 received desmopressin and 48 received imipramine. A total of 68 patients (47\%) had monosymptomatic enuresis and 77 (53\%) had polysymptomatic enuresis. Combination therapy produced the best and most rapid results regardless of whether the children had monosymptomatic or polysymptomatic enuresis. Combination therapy with desmopressin plus oxybutynin for the treatment of pediatric nocturnal enuresis was well tolerated, and gave significantly faster and more cost-effective results than single drug therapy using either desmopressin or imipramine. [\hyperlink{Mycobutin}{PMID: 16094064}, Tack Lee et al., 2005]

\hypertarget{pmid_12943481}{I}n the US, 6\% sulfur in petrolatum has been the most frequently administered treatment for infantile scabies. It appears to be safe but there is no literature containing a large series of patients on which to base that determination. In the UK, benzyl benzoate is the approved product. Benzyl benzoate is rarely used in the US at the present time. 5\% Permethrin is an excellent substitute and has many advantages. It appears to be quite safe in infants, although it is more expensive than other products. It remains present on the skin for several days, therefore protecting against reinfestation. Ivermectin is a systemic drug which is assumed to be safe in infants, although it requires repeated doses and does not protect against reinfestation. In the opinion of the author, 5\% permethrin is the best treatment for scabies in infants and young children. [\hyperlink{Mycobutin}{PMID: 12943481}, Mervyn L Elgart et al., 2003]

\hypertarget{pmid_18368837}{T}his study has evaluated the efficacy and safety of different doses of Esmerone in children under fluorothane anesthesia. It enrolled 85 children from a senior age group (7-14 years). According to the myorelaxant used, all the patients were divided into 2 groups: S and R. In the S group (n=25), myoplegia was carried out administering succinylcholine in a dose of 1 mg/kg. In the R group, myorelaxation was achieved using rocuronium bromide (Esmerone). According to the dose used, the group was divided into 3 subgroups, each comprising 20 patients. The initial dose of Esmemrone was 0.3, 0.6, and 0.9 mg/kg, respectively. The efficacy of the agent was evaluated by accelerographic and clinical data. The study has demonstrated that Esmerone was safer than succinylcholine. Increasing the dose (0.3-0.6-0.9 mg/kg) of Esmerone advantages in the onset of NMB and shows an increase in the duration of its action. [\hyperlink{Mycobutin}{PMID: 18368837}, A K Shaginian et al., ]

\hypertarget{pmid_18947791}{W}e have previously reported that intravesical oxybutynin chloride with hydroxypropylcellulose (modified intravesical oxybutynin) is an effective therapeutic agent for patients with detrusor overactivity. In this study, we report on the efficacy, safety and side effects of modified intravesical oxybutynin administration in children with neurogenic bladder. Modified intravesical oxybutynin (1.25mg/5 mL, twice a day) was administered to four children (three males and one female) with neurogenic bladder (detrusor overactivity and/or low compliance bladder), who were previously unresponsive to or experienced intolerable side effects from oral medications. A cystometrogram was obtained before, 1 week after, and 1 year after the first intravesical instillation of modified oxybutynin. We also carefully observed anticholinergic side effects, occurrence of urinary tract infection and degree of incontinence during this treatment. After 1 week, both cystometric bladder capacity and compliance were improved in all patients, and detrusor overactivity was undetectable in three of four patients. At 1 year, there was further improvement in bladder compliance in three patients, and detrusor overactivity was not observed in two patients. Significant improvement in the degree of incontinence was achieved. No systemic anticholinergic side effects were observed in any of the patients. One patient with vesicoureteral reflux discontinued the therapy after 2 months due to upper urinary tract infections. Modified intravesical oxybutynin is an effective and relatively safe therapeutic option for children with neurogenic bladders. [\hyperlink{Mycobutin}{PMID: 18947791}, Atsushi Hayashi et al., 2007]

\hypertarget{pmid_23627681}{T}he effects of oxybutynin for treating hyperhidrosis in children are still unknown. Therefore the aim of this study was to investigate the effects of oxybutynin on improving symptoms of hyperhidrosis and quality of life (QOL) in children with palmar hyperhidrosis (PH). Forty-five children ages 7-14 years with PH were evaluated 6 weeks after protocol treatment with oxybutynin. QOL was evaluated before and after treatment using a validated clinical questionnaire. More than 85\% of the children with PH treated with oxybutynin experienced moderate or greater improvement in the level of sweating and 80\% experienced improvement in QOL. Children who initially presented with very poor QOL were those who benefited most from oxybutynin therapy. Side effects occurred in 25 children (55.5\%) and were mainly dry mouth. Only one patient had neurologic symptoms, which was reported as drowsiness. Oxybutynin is an effective treatment option for children with PH because it improves clinical symptoms and QOL. Further studies are required to determine the long-term outcomes of treatment with oxybutynin. [\hyperlink{Mycobutin}{PMID: 23627681}, Nelson Wolosker et al., ]

\hypertarget{pmid_8808907}{W}e evaluated the clinical use of long-term intravesical oxybutynin chloride in the treatment of neurogenic bladder dysfunction in children with myelodysplasia who could not tolerate oral anticholinergics. We retrospectively reviewed the records of all patients recommended for intravesical oxybutynin chloride therapy. A total of 12 girls and 18 boys 1 to 17 years old was recruited for study. Oxybutynin chloride (5 mg.) was instilled 2 times daily and pretreatment cystograms were compared to followup urodynamic studies. Duration of therapy was 2 to 26 months (mean 13, median 12). Mean total capacity plus or minus standard deviation increased from 209 +/- 103 to 282 +/- 148 ml. (p < 0.01), mean safe capacity increased from 157 +/- 105 to 234 +/- 147 ml. (p < 0.01) and mean age adjusted safe capacity increased from 76 +/- 36 to 115 +/- 62\%. Of the 29 patients who were incontinent 3 (10\%) achieved continence and 19 (65\%) reported a decreased use of sanitary pads. None of the patients had systemic side effects related to intravesical treatment. We believe that intravesical oxybutynin chloride is a viable treatment option for patients with myelodysplasia in whom oral therapy fails. [\hyperlink{Mycobutin}{PMID: 8808907}, K A Painter et al., 1996]

\hypertarget{pmid_7961355}{T}he objective of this open study was to determine the efficacy and safety of fluoxetine for the treatment of children and adolescents with anxiety disorders. Twenty-one patients with overanxious disorders, social phobia, or separation anxiety disorder, who were unresponsive to previous psychopharmacological and psychotherapeutic interventions, were treated openly with fluoxetine for up to 10 months. Patients with lifetime histories of obsessive-compulsive disorder (OCD) or panic disorder, or with current major depression, were excluded. Beneficial and adverse effects of fluoxetine were ascertained using the improvement and severity subscales of the Clinical Global Impression Scale (CGIS) in two ways: (1) independent chart reviews by two child psychiatrists and (2) prospective assessments by the treating nurses and the patients' mothers. Eighty-one percent (n = 17) of patients showed moderate to marked improvement in anxiety symptoms. The severity of anxiety as measured by the CGIS was also significantly reduced from marked to mild (effect size: 2.3). There were no significant side effects. These results suggest that fluoxetine may be an effective and safe treatment for nondepressed children and adolescents with anxiety disorders other than OCD and panic disorder. Future investigations using double-blind, placebo-controlled methodologies are warranted. [\hyperlink{Mycobutin}{PMID: 7961355}, B Birmaher et al., 1994]

\section*{Ninlaro}
\subsection*{Result}
\subsubsection*{Answer}

Unknown

\subsubsection*{{Explanation}}
\hypertarget{Ninlaro}
After reviewing all available abstracts, there is no evidence of a targeted safety study of Ninlaro (ixazomib) in children. None of the abstracts reference Ninlaro or provide data on its use, safety, or efficacy in pediatric populations of any age range. Therefore, based on the abstracts provided, the safety of Ninlaro in children is unknown.

\subsection*{Abstracts}
\hypertarget{pmid_21054515}{T}he aims of the study were to assess the long-term safety and compare neurodevelopmental outcomes in school-age children born prematurely who received inhaled nitric oxide or placebo during the first week of life in a randomized, double-blinded study. Children treated with inhaled nitric oxide had previously been shown to have decreased intraventricular haemorrhage and periventricular leukomalacia as newborns and decreased cognitive impairment at 2 years (L.W. Doyle and P.J. Anderson. (2005) Arch Dis Child Fetal Neonatal Ed, 90, F484-F8). It is follow-up study of medical outcomes, neurodevelopmental assessment and school readiness in 135 of 167 (81\%) surviving premature infants seen at 5.7±1.0 years. Compared to placebo-treated children (n=65), iNO-treated children (n=70) demonstrated no difference in growth parameters, school readiness or need for subsequent hospitalization. However, iNO-treated children were less likely to have multiple chronic morbidities or technology dependence (p=0.05). They also had less functional disability (p=0.05). These results demonstrate the long-term safety of iNO in premature infants. Furthermore, iNO treatment may improve health status by decreasing the incidence of severe ongoing morbidities and technology dependence and may also decrease the incidence of educational and community functional disability of premature infants at early school age. [\hyperlink{Ninlaro}{PMID: 21054515}, Athena I Patrianakos-Hoobler et al., 2011]

\hypertarget{pmid_5550546}{T}he anticonvulsant activity of nitrazepam (Mogadon) was studied in 31 children with various seizure patterns. Dosage ranged from 0.3 to 2.2 mg. per kg. body weight daily.Eleven of 15 children with minor motor seizures showed improvement and six obtained complete relief. Nine of 16 with miscellaneous seizures were improved, but only one was completely relieved and the other eight responded to a variable extent. In cases with more than one type of seizure, the myoclonic elements were those most often diminished, but sometimes this effect was only temporary. Side effects were transient and usually mild, consisting of drowsiness, ataxia, slurred speech and excessive secretion of mucus and saliva. However, three cases of aspiration pneumonia were encountered and may have been at least partly due to the side effects. No hematological or biochemical abnormalities were observed.The results indicate that nitrazepam is a relatively safe and effective drug in the treatment of minor motor seizures, particularly infantile spasms, and is even more useful than ACTH in this serious form of epilepsy. In older children its value is chiefly for myoclonic seizures, but the degree and duration of its effectiveness appear to be more limited. [\hyperlink{Ninlaro}{PMID: 5550546}, J E Jan et al., 1971]

\hypertarget{pmid_2391756}{W}e administered norfloxacin (NFLX) to 16 children aged 3 to 14 year-old at the dose of 5.2 to 17.2 mg/kg/day. We evaluated the efficacy and safety of NFLX in 6 children with respiratory tract infections, 8 urinary tract infections, and 2 gastrointestinal tract infections. Efficacy rate of NFLX was 93.8\% and eradicated rate was 92.9\%. Any adverse effects were not observed. These results suggested that NFLX could be used safely to the children. [\hyperlink{Ninlaro}{PMID: 2391756}, T Ihara et al., 1990]

\hypertarget{pmid_19740527}{E}noxaparin, a low molecular weight heparin (LMWH), is frequently used for the prevention and treatment of thromboembolic complications in infants and children (Sutor et al., 2004 [1]). Injection pain and the fear and anxiety associated with needle phobia in the pediatric population are well documented. Best practice pediatric pain management standards of care recommend mitigating the child's pain experience whenever possible. The use of topical anesthetics such as liposomal-lidocaine 4\% results in a rapid onset of anesthesia, minimal blanching, without vasoconstriction (Koh et al., 2004 [2]) or risk of methemoglobinemia. Topical lidocaine has been used to reduce the injection pain of enoxaparin, but there is no data available examining whether it will interfere with the absorption of LMWH. To determine if the topical lidocaine, Maxilene, interferes with enoxaparin absorption as measured by peak anti-Xa levels. Infants and children clinically prescribed enoxaparin were eligible for this study. Children in group 1 were pre-treated with Maxilene prior to enoxaparin injection on day 1 with no Maxilene pre-treatment on day 2. For group 2, the order was reversed. Peak anti-Xa levels were measured following each enoxaparin dose and were compared between the groups. 26 children of ages 14d-16 y (median 6.7 months) were enrolled. Anti-Xa levels following topical lidocaine administration were 0.070 U/mL (95\%CI 0.025; 0.114) lower than without prior topical lidocaine administration. Anti-Xa levels on the second day were on average 0.013 U/mL (95\%CI -0.066; 0.040) higher compared to day one regardless of the order of topical lidocaine administration. There were no reported incidences of local reactions such as redness, hives or blanching. Topical lidocaine (Maxilene) administration before enoxaparin injection results in a small, clinically non-significant, reduction in anti-Xa levels. [\hyperlink{Ninlaro}{PMID: 19740527}, S M Duncan et al., 2010]

\hypertarget{pmid_10472395}{W}e report our experience of the utilization of the 50\% oxygen-nitrous oxide mixture (nitrous oxide 50\%) in our general pediatric ward after one year of use. Between 1st April 1997 and 31st March 1998, children who had to undergo a painful procedure were proposed to inhale 50\% nitrous oxide before the procedure. We evaluate pain, restlessness and adverse effects. The procedures (127 of them) were carried out in 90 children (61 boys). They were aged from 5 months to 15 years (mean: 5.7 years; median: 4.1 years). Indications were: lumbar puncture (n = 45), burning dressing (n = 29), venous cannulation (n = 12), minor surgery (n = 27), and miscellaneous (n = 14). Inhalation time was between 2 to 70 min (mean: 14.4 min; median: 11 min). Pain was absent or low in 106 cases (83.4\%). Restlessness was absent or low in 100 cases (78.8\%). Averse events were observed 12 times, but they were always minor and quickly reversible. Nitrous oxide (50\%) can be used successfully in a general pediatric ward. Other studies are necessary to define the best conditions. [\hyperlink{Ninlaro}{PMID: 10472395}, P Vic et al., 1999]

\hypertarget{pmid_22134227}{N}itrous oxide is an effective sedative/analgesic for mildly to moderately painful pediatric procedures. This study evaluated the safety of nitrous oxide administered at high concentration (up to 70\%) for procedural sedation. This prospective, observational study included all patients younger than 18 years who received nitrous oxide for diagnostic or therapeutic procedures at a metropolitan children's facility. Patients' age, highest concentration and total duration of nitrous oxide administration, and adverse events were recorded. Nitrous oxide was administered on 7802 occasions to 5779 patients ranging in age from 33 days to 18 years (median, 5.0 years) during the 5.5-year study period. No adverse events were recorded for 95.7\% of cases. Minor adverse events included nausea (1.6\%), vomiting (2.2\%), and diaphoresis (0.4\%). Nine patients had potentially serious events, all of which resolved without incident. There was no difference in adverse event rates between nitrous oxide less than or equal to 50\% and greater than 50\% (P = 0.18). Patients aged 1 to 4 years had the lowest adverse event rate (P < 0.001), with no difference between groups younger than 1 year, 5 to 10 years, and 11 to 18 years. Compared with patients with less than 15 minutes of nitrous oxide administration, patients with 15 to 30 minutes or more than 30 minutes of nitrous oxide administration were 4.2 (95\% confidence interval, 3.2-5.4) or 4.9 (95\% confidence interval, 2.6-9.3) times more likely to have adverse events. Nitrous oxide can be safely administered at up to 70\% concentration by nasal mask for pediatric procedural sedation, particularly for short (<15 minutes) procedures. Nitrous oxide seems safe for children of all ages. [\hyperlink{Ninlaro}{PMID: 22134227}, Judith L Zier et al., 2011]

\hypertarget{pmid_9002122}{Q}uinolone-induced cartilage toxicity has been observed in experimental juvenile animal studies and is species- and dose-specific. Accordingly these findings have led to the contraindication of fluoroquinolones in children. Previous data in 634 adolescents and children treated with compassionate use ciprofloxacin demonstrated low rates of reversible arthralgia and a safety profile similar to that for adult patients. This report describes the safety findings in 1795 children who received 2030 treatment courses of intravenous or oral ciprofloxacin. The average doses of intravenous and oral ciprofloxacin in the study population were 8 and 25 mg/kg/day, respectively. Treatment-associated events were reported in 10.9\% of children receiving oral ciprofloxacin compared with 18.9\% among intravenous recipients. Overall arthralgia occurred during 31 ciprofloxacin treatment courses (1.5\%) and the majority of events were of mild to moderate severity and resolved without intervention. More than 60\% of arthralgia episodes were in children with cystic fibrosis. The adverse event pattern in children receiving ciprofloxacin in this analysis was similar to that observed in adults. Rates of reversible arthralgia were low and unchanged from previously published surveillance data in children. [\hyperlink{Ninlaro}{PMID: 9002122}, B Hampel et al., 1997]

\hypertarget{pmid_11966554}{W}e studied 411 children aged 3-10 years who were referred for dental treatment. They were randomly allocated to have inhalation conscious sedation with either sevoflurane/nitrous oxide mixture or nitrous oxide alone. Dental treatment was satisfactorily completed in 215/241 children who were given sevoflurane/nitrous oxide mixture (89\%) compared with 89/170 who were given nitrous oxide alone (52\%) (Chi square 70.3, p < 0.0001). All children remained conscious and responsive to verbal contact throughout the treatment and in the recovery room. No adverse side-effects were recorded in either group and there were no significant differences in oxygen saturation, heart rate, recovery profile, or time to discharge home between the groups. The study concluded that, for every 100 children treated with sevoflurane/nitrous oxide mixture, 37 children would be saved a general anaesthetic if given combined sevoflurane and nitrous oxide mixture rather than nitrous oxide alone. The use of sevoflurane in low concentrations 0.1-0.3\% to supplement nitrous oxide and oxygen for inhalation conscious sedation is safe, practical, and significantly more effective than nitrous oxide alone in children having dental treatment. [\hyperlink{Ninlaro}{PMID: 11966554}, G Y Lahoud et al., 2002]

\hypertarget{pmid_19769238}{C}ongenital Nasolacrimal Duct Obstruction (CNLDO) is one of the commonest causes of Childhood epiphora. This study was carried outat Sagarmatha Chaudhary Eye Hospital (SCEH), Lahan to determine the success rate of probing and syringing in children below 18 months of age with CNLDO. A hospital based prospective interventional study of 106 children with age 4 to 18 months (Mean 7.67, SD 4.37) who underwent probing and syringing under topical anesthesia in minor procedure room of OPD. The children were divided into 3 age groups, Group A (age d" 6 months), Group B (age = 7-12 months) and Group C (age = 13-18 months). Success of probing was defined as complete relief of signs and symptoms on follow up at 3-6 weeks. Out of 106 children with CNLDO below 18 months of age, 97 (91.5\%) children had better outcome with first attempt of probing. First attempt of probing resulted in resolution in 92.3\% (108 out of 117) eyes; 95.6\%, 92.7\%, 87.1\% in the age group less than 6 months, 7-12 and 13-18 months respectively. 9 eyes underwent a repeat procedure of which 6 eyes were cured. Probing and syringing below 6 months of life has shown to be very effective with almost 100\% success. The overall success of probing and syringing among children less than 18 months was 97.4\%. [\hyperlink{Ninlaro}{PMID: 19769238}, J B Shrestha et al., 2009]

\hypertarget{pmid_21930540}{I}nhaled nitric oxide (iNO) is an effective therapy for pulmonary hypertension and hypoxic respiratory failure in term infants. Fourteen randomized controlled trials (n = 3430 infants) have been conducted on preterm infants at risk for chronic lung disease (CLD). The study results seem contradictory. Individual-patient data meta-analysis included randomized controlled trials of preterm infants (<37 weeks' gestation). Outcomes were adjusted for trial differences and correlation between siblings. Data from 3298 infants in 12 trials (96\%) were analyzed. There was no statistically significant effect of iNO on death or CLD (59\% vs 61\%: relative risk [RR]: 0.96 [95\% confidence interval (CI): 0.92-1.01]; P = .11) or severe neurologic events on imaging (25\% vs 23\%: RR: 1.12 [95\% CI: 0.98-1.28]; P = .09). There were no statistically significant differences in iNO effect according to any of the patient-level characteristics tested. In trials that used a starting iNO dose of >5 vs ≤ 5 ppm there was evidence of improved outcome (interaction P = .02); however, these differences were not observed at other levels of exposure to iNO. This result was driven primarily by 1 trial, which also differed according to overall dose, duration, timing, and indication for treatment; a significant reduction in death or CLD (RR: 0.85 [95\% CI: 0.74-0.98]) was found. Routine use of iNO for treatment of respiratory failure in preterm infants cannot be recommended. The use of a higher starting dose might be associated with improved outcome, but because there were differences in the designs of these trials, it requires further examination. [\hyperlink{Ninlaro}{PMID: 21930540}, Lisa M Askie et al., 2011]

\hypertarget{pmid_22003294}{E}xcessive drooling may complicate the care of children with chronic neurological conditions by socially isolating both patients and families and by causing secondary dermatitis and infection. Normal control of saliva requires normal integrity of oral structures, normal oropharyngeal sensation, and motor functioning, as well as normal cognitive awareness and rate of salivary production. Glycopyrrolate is an anticholinergic medication with a quaternary structure that recently received Food and Drug Administration approval to treat sialorrhea due to neurological problems in children ages 3-16 years. This review summarizes the few published studies of safety and efficacy of glycopyrrolate for drooling in children with chronic neurological conditions. [\hyperlink{Ninlaro}{PMID: 22003294}, Marian L Evatt et al., 2011]

\hypertarget{pmid_24567811}{S}evoflurane anesthesia commonly causes emergence agitation (EA) in children. One previous study has reported that the use of nitrous oxide (N2O) during the washout of sevoflurane may reduce EA by decreasing the residual sevoflurane concentration, while many animal studies suggest that N2O poses a potential risk to children. The present study was designed to compare EA in children assigned to receive sevoflurane with N2O (group N) or sevoflurane alone (group S). We enrolled 80 children aged 3-10 years. Anesthesia was induced with 5 mg/kg thiopental sodium, 0.6 mg/kg rocuronium and 0.5 mg/kg ketorolac, and was maintained with 50\% N2O and sevoflurane in group N or with sevoflurane alone in group S. The sevoflurane concentration was adjusted with a bispectral index (BIS) of 40-60. After completion of the surgery, N2O and sevoflurane were simultaneously discontinued and replaced with oxygen (O2) at 6 L/min. End-tidal sevoflurane concentration (Et Sevo) (\%), BIS at the end of surgery, Et Sevo at recovery of self-respiration and emergence profiles were recorded. EA occurrence, pain score and rescue fentanyl consumption were assessed in the postanesthesia care unit. Et Sevo was significantly lower in group N (1.9\%) than in group S (2.3\%) at the end of surgery. However, there were no differences in Et Sevo at recovery of self-respiration, emergence times, the incidence of EA, pain score or dose of rescue fentanyl between the groups. In children undergoing adenotonsillectomy with preemptive ketorolac, anesthetic maintenance using sevoflurane alone does not affect the incidence of EA or emergence profiles compared to anesthetic maintenance using sevoflurane with N2O. [\hyperlink{Ninlaro}{PMID: 24567811}, Ji Hye Park et al., 2014]

\hypertarget{pmid_16148661}{W}e compare the efficacy and safety profile of oral midazolam and continuous flow 50\% nitrous oxide (N(2)O) for alleviating anxiety and pain during voiding cystourethrography (VCU) in children. This prospective, randomized clinical trial study was conducted in the radiology unit of a tertiary care center. Children older than 3 years scheduled for VCU were given either 0.5 mg/kg midazolam orally or continuous flow 50\% N(2)O. Main outcomes were degree of anxiety and pain as assessed by the attending nurse and radiologist performing the test using a behavioral anxiety score, a distress score and an overall satisfaction score, side effects and recovery profile. The study included 47 children (89\% girls) with a mean age of 6 years (range 3 to 15). There were 24 subjects in the midazolam group and 23 in the N(2)O group. Midazolam and N(2)O provided adequate anxiety and pain relief to perform the examination, yet children given N(2)O required less restraining and experienced a significantly shorter recovery time (29 +/- 10 vs 63 +/- 25 minutes, p <0.001). Continuous flow 50\% nitrous oxide and oral midazolam are comparably safe and effective in reducing anxiety and distress during VCU in children older than 3 years. However, N(2)O provides a more rapid onset of sedating effect and has a shorter recovery time. [\hyperlink{Ninlaro}{PMID: 16148661}, Ilan Keidan et al., 2005]

\hypertarget{pmid_36053397}{N}on-steroidal anti-inflammatory drugs (NSAIDs) are commonly used in infants, children, and adolescents worldwide; however, despite sufficient evidence of the beneficial effects of NSAIDs in children and adolescents, there is a lack of comprehensive data in infants. The present review summarizes the current knowledge on the safety and efficacy of various NSAIDs used in infants for which data are available, and includes ibuprofen, dexibuprofen, ketoprofen, flurbiprofen, naproxen, diclofenac, ketorolac, indomethacin, niflumic acid, meloxicam, celecoxib, parecoxib, rofecoxib, acetylsalicylic acid, and nimesulide. The efficacy of NSAIDs has been documented for a variety of conditions, such as fever and pain. NSAIDs are also the main pillars of anti-inflammatory treatment, such as in pediatric inflammatory rheumatic diseases. Limited data are available on the safety of most NSAIDs in infants. Adverse drug reactions may be renal, gastrointestinal, hematological, or immunologic. Since NSAIDs are among the most frequently used drugs in the pediatric population, safety and efficacy studies can be performed as part of normal clinical routine, even in young infants. Available data sources, such as (electronic) medical records, should be used for safety and efficacy analyses. On a larger scale, existing data sources, e.g. adverse drug reaction programs/networks, spontaneous national reporting systems, and electronic medical records should be assessed with child-specific methods in order to detect safety signals pertinent to certain pediatric age groups or disease entities. To improve the safety of NSAIDs in infants, treatment needs to be initiated with the lowest age-appropriate or weight-based dose. Duration of treatment and amount of drug used should be regularly evaluated and maximum dose limits and other recommendations by the manufacturer or expert committees should be followed. Treatment for non-chronic conditions such as fever and acute (postoperative) pain should be kept as short as possible. Patients with chronic conditions should be regularly monitored for possible adverse effects of NSAIDs. [\hyperlink{Ninlaro}{PMID: 36053397}, Victoria C Ziesenitz et al., 2022]

\hypertarget{pmid_11430720}{I}nhaled nitric oxide (iNO) is used to treat preterm infants with hypoxaemic respiratory failure. In this study we describe the long-term survival and neurodevelopmental status of high-risk preterm infants enrolled into a randomized controlled trial of iNO therapy. Information regarding long-term outcome was available for all 25 children enrolled in the original trial who survived until discharge from hospital. Formal, blinded, developmental assessment and neurological examinations were performed in 21 out of 22 children still alive at 30 mo of age, corrected for prematurity. No significant differences were found in long-term mortality (12/20 vs 8/22, RR 1.65, 95\% CI 0.87-3.3), neurodevelopmental delay (4/7 vs 9/14, RR 0.89, 95\% CI 0.37-1.75), severe neurodisability (0/7 vs 5/14, p = 0.12) or cerebral palsy (0/7 vs 2/14, p = 0.53) between iNO-treated and control infants. In this study there was no evidence of a significant effect on either survival or long-term neurodevelopmental status in infants treated with iNO. [\hyperlink{Ninlaro}{PMID: 11430720}, A J Bennett et al., 2001]

\hypertarget{pmid_10646327}{E}fficacy, tolerability and safety of Dipyrone (Novalgin) in the management of pain and fever in children. Open, non-comparative study in Ganga Ram Hospital, Lahore. Children (of both sexes) aged 3 months to 12 years with oral temperature of 38.5 degrees C or more/complaining of pain due to various reasons. Sixty-two (66.7\%) out of 93 who had fever showed good response, 24 (25.8\%) showed satisfactory response and 7 (7.5\%) showed unsatisfactory response to Dipyrone (Novalgin). Dipyrone (Novalgin) in a dose of 10-15 mg/kg/dose every 6-8 hrs. is effective and safe in the treatment of pain and fever in children. [\hyperlink{Ninlaro}{PMID: 10646327}, T Izhar et al., 1999]

\hypertarget{pmid_24750360}{I}t has been shown that early placement of an intravenous line in children administered sevoflurane anesthesia increased the incidence of laryngospasm and movement. However, the optimal time for safe cannulation after the loss of the eyelash reflex during the administration of sevoflurane and nitrous oxide is not known. The aim of the study was to determine the optimum time for intravenous cannulation after the induction of anesthesia with sevoflurane and nitrous oxide in children premedicated with oral midazolam. We performed a prospective, observer-blinded, up-down sequential, allocation study, and children, aged 2-6 years, ASA physical status I, scheduled for an elective procedure undergoing inhalational induction were included in the study. Anesthesia was induced with sevoflurane and nitrous oxide after premedication with oral midazolam. For the first child, 4 min after the loss of the eyelash reflex, the intravenous cannulation was attempted by an experienced anesthesiologist. The time for intravenous cannulation was considered adequate if movement, coughing, or laryngospasm did not occur. The time for cannulation was increased by 15 s if the time was inadequate in the previous patient, and conversely, the time for cannulation was decreased by 15 s if the time was adequate in the previous patient. The probit test was used in the analysis of up-down sequences. A total of 32 children were enrolled sequentially during the study period. The adequate time for effective intravenous cannulation after induction with sevoflurane and nitrous oxide in 50\% and 95\% of patients were 1.29 min (95\% confidence interval, 0.96-1.54 min) and 1.86 min (95\% confidence interval 1.58-4.35 min), respectively. We recommend waiting 2 min for attempting intravenous placement following the loss of the eyelash reflex in children sedated with midazolam and receiving an inhalation induction with sevoflurane and nitrous oxide. [\hyperlink{Ninlaro}{PMID: 24750360}, Alper Kilicaslan et al., 2014]

\hypertarget{pmid_11552629}{T}he aim of the study was to research the efficiency of sertraline (zoloft) in depressions, anxious states and obsessive-compulsive disorders. Diagnosis of the mental disorders was carried out according to ICD-10. 72 children (59 boys, 13 girls) aged 6-18 years were treated. There were 32 inpatients and 40 outpatients. Therapy with sertraline was performed during 8 weeks with a gradual increase (titration) and individual selection of the dose from 12.5 to 100 mg/day. During the therapy clinical observation was combined with the patients' examination using Hamilton Depression Scale and Hamilton Anxiety Scale (HAM-D and HAM-A), and a Clinical Global Impression Scale (CGI). It was established that sertraline was very effective and safe drug in children (it has no influence on cognitive functions, has neither myorelaxing or sedative effects). Activity of this drug is characterized by quick manifestation of thymoanaleptic and anxiolytic effects. It mild depressive states 50 mg/day is a significant dose; in more severe depressions and obsessive-compulsive disorders the dose in juveniles was to 100 mg, the duration of the therapy was more than 2 months. [\hyperlink{Ninlaro}{PMID: 11552629}, V M Voloshina et al., 2001]

\hypertarget{pmid_36403819}{T}o evaluate a heterologous vaccination scheme in children 3-18 years old (y/o) combining two SARS-CoV-2r- receptor binding domain (RBD)protein vaccines. A phase I/II open-label, adaptive, and multicenter trial evaluated the safety and immunogenicity of two doses of FINLAY-FR-2 (subsequently called SOBERANA 02) and the third heterologous dose of FINLAY-FR-1A (subsequently called SOBERANA Plus) in 350 children 3-18 y/o in Havana Cuba. Primary outcomes were safety (phase I) and safety/immunogenicity (phase II) measured by anti-RBD immunoglobulin (Ig)G enzyme-linked immunoassay (ELISA), molecular and live-virus neutralization titers, and specific T-cells response. A comparison with adult immunogenicity and predictions of efficacy were made based on immunological results. Local pain was the unique adverse event with frequency >10\%, and none was serious neither severe. Two doses of FINLAY-FR-2 elicited a humoral immune response similar to natural infection; the third dose with FINLAY-FR-1A increased the response in all children, similar to that achieved in vaccinated young adults. The geometric mean (GMT) neutralizing titer was 173.8 (95\% confidence interval [CI] 131.7; 229.5) vs Alpha, 142 (95\% CI 101.3; 198.9) vs Delta, 24.8 (95\% CI 16.8; 36.6) vs Beta and 99.2 (95\% CI 67.8; 145.4) vs Omicron. The heterologous scheme was safe and immunogenic in children 3-18 y/o. https://rpcec.sld.cu/trials/RPCEC00000374. [\hyperlink{Ninlaro}{PMID: 36403819}, Rinaldo Puga-Gómez et al., 2023]

\hypertarget{pmid_9249106}{W}e compared the efficacy and tolerance of pediatric inductions with immediate 8\% sevoflurane in 70\% nitrous oxide with either incremental sevoflurane or incremental halothane in 70\% nitrous oxide. Forty-six unpremedicated children had anesthesia induced by immediate 8\% sevoflurane (high sevoflurane [HS]; circuit primed with 70\% N2O and 8\% sevoflurane before application of the face mask), gradual sevoflurane (GS; primed with 70\% N2O with increments of sevoflurane), and gradual halothane (HAL; 70\% N2O with incremental halothane). Blind video recordings were made, and each child's distress was rated prior to mask application, during mask application, and every 10 s thereafter using a behavioral rating scale. There were no complications. Of those subjects not quiet and cooperative throughout, times to complete quiet were significantly different (P = 0.001): HS 19.8 +/- 8 s (range 9-34); GS 52 +/- 17 s (range 8-73); HAL 43 +/- 22 s (range 13-73). Times to eye closure were also significantly different (P < 0.001): HS 37 +/- 10 s (range 15-56); GS 70 +/- 18 s (range 35-114); HAL 81 +/- 34 s (range 55-140). Distress scale scores showed more rapid decrement with HS than with GS or HAL. We conclude that 1) immediate 8\% sevoflurane/N2O results in a significantly faster induction than GS or HAL;2) in children, HS in N2O will not result in a single-breath induction under the conditions of this study; 3) in this small group, HS was extremely well tolerated in ASA class I and II patients. [\hyperlink{Ninlaro}{PMID: 9249106}, V C Baum et al., 1997]

\hypertarget{pmid_18977585}{T}o prospectively study the efficacy and safety of intraparotid gland injection of Botulinum neurotoxin serotype A (Dysport) for the treatment of sialorrhea (drooling) in children with cerebral palsy (CP). Twenty-four children, ages 21 months to 7 years, were recruited and randomized to receive either treatment with 100U Botulinum toxin or placebo. Rating scales for the frequency and severity of drooling were performed at the time of injection, at 1 month, and at baseline prior to the second injection. A second set of injections of either 140U of drug or placebo was given 4 months later, and the same rating scales were used. Eight patients declined the second injection. Due to high dropouts in the placebo group in second set of injections, statistical analysis was performed for the results of the initial injection only. Scores of the median frequency (p=0.034) and severity (p=0.026) of drooling were reduced in the treatment group. Median total score also declined in the treatment group (p=0.027). After the second injection, five out of nine patients injected with the drug showed a decline in the total score; including three patients who did not respond to the first injection. Only two patients experienced transient increase in drooling after the treatment with the drug. Botulinum toxin is an effective and safe treatment option for drooling in children with CP. [\hyperlink{Ninlaro}{PMID: 18977585}, Ali H Alrefai et al., 2009]

\hypertarget{pmid_12954676}{T}o describe the experience of using high concentration nitrous oxide (N(2)O) relative analgesia administered by nursing staff in children undergoing minor procedures in the emergency department (ED) and to demonstrate its safety. Data were collected over a 12 month period for all procedures in the ED performed under nurse administered N(2)O sedation. All children greater than 12 months of age requiring a minor procedure who had no contraindication to the use of N(2)O were considered for sedation by this method. The primary outcome measure was the incidence of a major complication namely respiratory distress or hypoxia during the procedure. Secondary outcome measures were minor complications and the maximum concentration of N(2)O used. Data were collected for a total of 224 episodes of nurse administered N(2)O sedation over a 12 month period. In 73.2\% of children no complications were recorded. One major complication was recorded (respiratory distress) and the most common minor complication was mask intolerance in 17\%. The mean maximum concentration of N(2)O used was 60.2\%. N(2)O is a safe analgesic in children over the age of 1 year undergoing painful or stressful procedures in the ED. It may safely be administered in concentrations of up to 70\% by nursing staff after appropriate training. [\hyperlink{Ninlaro}{PMID: 12954676}, A Frampton et al., 2003]

\hypertarget{pmid_8534466}{T}o compare the analgesic and anxiolytic effects of nitrous oxide (N2O) when inhaled by face mask with those of a cutaneous application of a eutectic mixture of local anesthetics (EMLA) cream with lidocaine and prilocaine during pre-operative venous cannulation in children. Prospective, randomized study. Outpatient presurgical area and operating rooms of a freestanding children's hospital. 50 unpremedicated ASA status I and II outpatients, aged 6 to 12 years, undergoing an elective surgical procedure. Each patient received either 70\% N2O in 30\% oxygen (O2) administered by face mask for 120 seconds or an application of 2.5 g of EMLA cream under an occlusive dressing for a minimum of 60 minutes. All patients then underwent a single attempt at venous cannulation in the dorsum of the hand with a 22-gauge intravenous catheter. A visual analog scale (VAS) pain score (0 to 100) was generated by the investigator and subsequently obtained from each patient immediately after the venous cannulation was completed. The pain scores generated by the investigator were significantly lower in the N2O group than the EMLA cream group (p = 0.001). When compared with the patients in the EMLA cream group, the patients in the N2O group also self-reported significantly lower VAS pain scores (p = 0.006). N2O administered by face mask appears to provide greater anxiolysis and attendant superior analgesia for pediatric venous cannulation than a cutaneous application of EMLA cream. [\hyperlink{Ninlaro}{PMID: 8534466}, T R Vetter et al., 1995]

\hypertarget{pmid_18368837}{T}his study has evaluated the efficacy and safety of different doses of Esmerone in children under fluorothane anesthesia. It enrolled 85 children from a senior age group (7-14 years). According to the myorelaxant used, all the patients were divided into 2 groups: S and R. In the S group (n=25), myoplegia was carried out administering succinylcholine in a dose of 1 mg/kg. In the R group, myorelaxation was achieved using rocuronium bromide (Esmerone). According to the dose used, the group was divided into 3 subgroups, each comprising 20 patients. The initial dose of Esmemrone was 0.3, 0.6, and 0.9 mg/kg, respectively. The efficacy of the agent was evaluated by accelerographic and clinical data. The study has demonstrated that Esmerone was safer than succinylcholine. Increasing the dose (0.3-0.6-0.9 mg/kg) of Esmerone advantages in the onset of NMB and shows an increase in the duration of its action. [\hyperlink{Ninlaro}{PMID: 18368837}, A K Shaginian et al., ]

\hypertarget{pmid_22348510}{I}nhaled nitric oxide (iNO) has been used not only for pulmonary vasodilation in term neonates with hypoxemic respiratory failure, but also in preterm ones at risk of chronic lung disease (CLD) with variable results in prevention and treatment of CLD and/or brain injury. However, meta analysis of clinical trials does not support that iNO should be used routinely in preterm infants with hypoxic respiratory failure as it has no convincing long-term follow-up data to show its advantages in neurodevelopment. Investigation of extra-pulmonary effects of iNO through nitrosothiol hemoglobin-associated hypoxic vasodilation, as well as its intra- and extra-pulmonary anti-inflammation effect, would have biological and physiological potential in the management of the lung and brain injury of prematurity. The eligibility and safety of iNO in these premature infants at high risk of neurodevelopmental disability require more clinical and follow-up effort to test its pharmacological benefit over harm. [\hyperlink{Ninlaro}{PMID: 22348510}, Bo Sun et al., 2012]

\section*{Oxycodone Hydrochloride}
\subsection*{Result}
\subsubsection*{Answer}

0–6 months: Unknown  
>6 months to 7 years: Unknown  
Older children (no specific upper age limit): Unknown  

\subsubsection*{{Explanation}}
\hypertarget{Oxycodone Hydrochloride}
Based on the abstracts available, the safety of Oxycodone Hydrochloride in children has been specifically studied in several age groups. Here is a summary of the relevant evidence, organized by age range:

Neonates and Infants (0–6 months):
- One study investigated the pharmacokinetics of intravenous oxycodone hydrochloride in 22 infants aged 0–6 months, divided into three groups: <1 week, 1 week–2 months, and 2–6 months. The study found significant variability in clearance and half-life, especially in the youngest groups. Importantly, the authors concluded: "Routine dosing of oxycodone in young infants may be dangerous. The dose of oxycodone must be titrated individually" [\hyperlink{pmid_15960639}{PMID: 15960639}, Marja-Leena Pokela et al., 2005]. While no severe respiratory depression was observed in spontaneously breathing infants, the high variability and the authors' caution indicate that safety is not established for routine use in this age group.

Children (post-infant, specific ages not always clear):
- A study of 18 children (age not specified, but after ophthalmic surgery) given intravenous oxycodone hydrochloride (0.1 mg/kg) found that oxycodone caused greater ventilatory depression than comparable doses of other opioids [\hyperlink{pmid_7605420}{PMID: 7605420}, K T Olkkola et al., 1994]. This suggests a potential safety concern, but the abstract does not specify age ranges or affirm safety.
- Another study used population pharmacokinetic modeling to describe oxycodone in children aged 6 months to 7 years, concluding that a weight-based dose without age adjustment is appropriate for this range [\hyperlink{pmid_16554451}{PMID: 16554451}, Ahmed El-Tahtawy et al., 2006]. However, this study focused on pharmacokinetics, not safety outcomes.
- A recent study (abstract incomplete) mentions oxycodone as a commonly used oral opioid in children for postoperative pain, but does not provide safety data [\hyperlink{pmid_36618804}{PMID: 36618804}, Soroush Merchant et al., 2022].
- A review article states that "data on their efficacy and safety are limited in children" for oxycodone and other opioids, and that oral morphine is the opiate with the most evidence of safety and efficacy in children [\hyperlink{pmid_33747308}{PMID: 33747308}, Michael J Rieder et al.]. Another similar review reiterates the limited data on safety and efficacy of oxycodone in children [\hyperlink{pmid_33747307}{PMID: 33747307}, Michael J Rieder et al.].

Other studies:
- One in vitro study examined the effect of oxycodone on neural stem cells and found that high doses reduced cell survival and proliferation, but low doses had no significant effect. The authors suggest that "short term exposure to oxycodone in low dose might be allowed for developing brain" [\hyperlink{pmid_30290204}{PMID: 30290204}, Gang Wu et al., 2018]. However, this is not a clinical safety study in children.

Summary:
- For neonates and infants (0–6 months), there is evidence of high pharmacokinetic variability and a warning that routine dosing may be dangerous. No study affirms safety in this group.
- For children older than 6 months, there are pharmacokinetic data and some clinical use, but no targeted safety studies affirming safety. Some evidence suggests a risk of ventilatory depression.
- For all pediatric age groups, review articles emphasize that data on safety and efficacy are limited, and do not affirm safety.

Therefore, based on the abstracts, there is no targeted study that affirms the safety of Oxycodone Hydrochloride in children for any specific age group. In neonates and young infants, there is evidence suggesting potential danger with routine use.

\subsection*{Abstracts}
\hypertarget{pmid_15960639}{T}he pharmacokinetics of oxycodone (13-hydroxy-7,8-dihydrocodeinone) has been studied in adults and in children who are older than 6 months but there is no information on the disposition of oxycodone in neonates and young infants. The aim of this study was to study the pharmacokinetics of oxycodone in infants varying in age from 0 to 6 months. Twenty-two infants undergoing surgery were given postoperatively an intravenous bolus of 0.1 mg.kg(-1) of oxycodone hydrochloride. Ten of the patients were younger than 1 week (group 1), six from 1 week to 2 months (group 2) and six from 2 to 6 months (group 3). Plasma samples were collected for the analysis of oxycodone concentrations up to 24 h. Pharmacokinetics were characterized by noncompartmental methods. The median (range) values for the clearance (Cl) were 9.9 (2.3-17.2), 20.1 (3.7-40.4) and 15.4 (14.8-80.2) ml.min(-1).kg(-1) in the above three groups. The values for volume of distribution at steady-state were 3.3 (1.9-4.7), 5.6 (1.3-8.5) and 3.2 (1.8-6.0) l.kg(-1) and for elimination half-life (t(1/2)) 4.4 (2.4-14.1), 3.6 (1.6-11.6) and 2.0 (0.8-3.9) h, respectively. Both Cl (r = 0.46) and half-life (r = -0.46) were correlated to the age of the patient (P < 0.05). There were 13 patients who were on mechanical ventilation at the time of oxycodone administration. None of the spontaneously breathing infants had hypoventilation which required assistance during the study. The values for Cl and t(1/2) varied greatly between the subjects. This variability was most pronounced in the two youngest groups. Routine dosing of oxycodone in young infants may be dangerous. The dose of oxycodone must be titrated individually. [\hyperlink{Oxycodone Hydrochloride}{PMID: 15960639}, Marja-Leena Pokela et al., 2005] 1. Oxycodone hydrochloride (0.1 mg kg-1) was given by intravenous bolus to 18 children after ophthalmic surgery. Plasma was sampled for up to 8 h. Blood pressure, heart rate, peripheral arteriolar oxygen saturation, end-tidal carbon dioxide and halothane concentrations and ventilatory rate were also recorded. 2. Mean (+/- s.d.) values of drug clearance and volume of distribution (Vss) were 15.2 +/- 4.2 ml min-1 kg-1 and 2.1 +/- 0.8 l kg-1. Maximum mean end-tidal carbon dioxide concentration and minimum mean ventilatory rate occurred 8 min after administration of oxycodone but the minimum mean peripheral arteriolar oxygen saturation occurred at 4 min. 3. Oxycodone (0.1 mg kg-1) appears to cause greater ventilatory depression than comparable analgesic doses of other opioids. [\hyperlink{Oxycodone Hydrochloride}{PMID: 15960639}, K T Olkkola et al., 1994]

\hypertarget{pmid_17803435}{T}he aim of this study was to evaluate the safety of olopatadine hydrochloride ophthalmic solution 0.2\% in children and adolescents 3-17 years of age. In this 6-week, randomized, double-masked safety evaluation, eligible subjects with asymptomatic eyes underwent in-office visits at weeks 1, 3, and 6 and were contacted by telephone at weeks 2, 4, and 5. Qualified subjects were assigned randomly in a 2:1 ratio of olopatadine 0.2\% to vehicle (identical formation without the active ingredient) for dosing on a once-daily schedule. Safety parameters assessed included adverse events, visual acuity, ocular signs (slit-lamp assessments), dilated fundus examinations, intraocular pressure (IOP), pulse, and blood pressure. An evaluation of 126 subjects (age range, 3-17) revealed no clinically relevant treatment-related changes in visual acuity, IOP, slit-lamp assessments, fundus examinations, or cardiovascular parameters. All adverse events reported were mild or moderate. Olopatadine 0.2\% administered once-daily for 6 weeks is safe and well tolerated in children and adolescent patients. [\hyperlink{Oxycodone Hydrochloride}{PMID: 17803435}, Steven J Lichtenstein et al., 2007]

\hypertarget{pmid_36618804}{O}xycodone is a commonly used oral opioid in children for treating postoperative pain. Highly polymorphic gene  Patients who underwent Nuss procedure and spine fusion with  Of 193 subjects (age 15.9±0.25 years, 28.5\% female, 93.78\% White; 101 NM, 76 IM, 10 PM and 6 UM), 77.72\% underwent pectus surgery.  Our findings suggest  [\hyperlink{Oxycodone Hydrochloride}{PMID: 36618804}, Soroush Merchant et al., 2022] Anticholinergics are a key element in treating neurogenic detrusor overactivity, but only limited data are available in the pediatric population, thus limiting the application to children even for oxybutynin chloride (OC), a prototype drug. This retrospective study was designed to provide data regarding the efficacy, tolerability, and safety of OC in the pediatric population (0-15 years old) with spinal dysraphism (SD). Records relevant to OC use for neurogenic bladder were gathered and scrutinized from four specialized clinics for pediatric urology. The primary efficacy outcomes were maximal cystometric capacity (MCC) and end filling pressure (EFP). Data on tolerability, compliance, and adverse events (AEs) were also analyzed. Of the 121 patient records analyzed, 41 patients (34\%) received OC at less than 5 years of age. The range of prescribed doses varied from 3 to 24 mg/d. The median treatment duration was 19 months (range, 0.3-111 months). Significant improvement of both primary efficacy outcomes was noted following OC treatment. MCC increased about 8\% even after adjustment for age-related increases in MCC. Likewise, mean EFP was reduced from 33 to 21 cm H2O. More than 80\% of patients showed compliance above 70\%, and approximately 50\% of patients used OC for more than 1 year. No serious AEs were reported; constipation and facial flushing consisted of the major AEs. OC is safe and efficacious in treating pediatric neurogenic bladder associated with SD. The drug is also tolerable and the safety profile suggests that adjustment of dosage for age may not be strictly observed. [\hyperlink{Oxycodone Hydrochloride}{PMID: 36618804}, Jung Hoon Lee et al., 2014]

\hypertarget{pmid_29026333}{O}xycodone is poorly studied as an adjuvant to central blockades. The aim of this pilot study was to assess the efficacy and safety of oxycodone hydrochloride in epidural blockade among patients undergoing total hip arthroplasty (THA). In 11 patients (American Society of Anesthesiologists physical status classification system II/III, age range: 59-82 years), THA was conducted with an epidural blockade using 15 mL 0.25\% bupivacaine (37.5 mg) with 5 mg oxycodone hydrochloride and sedation with propofol infusion at a dose of 3-5 mg/kg/h. After the surgery, patients received ketoprofen at a dose of 100 mg twice daily. In the first 24 hours postoperative period, pain was assessed by numerical rating scale at rest and on movement; adverse effects (AEs) were recorded; and plasma concentrations of oxycodone, noroxycodone, and bupivacaine were measured. The administration of epidural oxycodone at a dose of 5 mg in patients undergoing THA provided analgesia for a mean time of 10.3±4.89 h. In one patient, mild pruritus was observed. Oxycodone did not evoke other AEs. Plasma concentrations of oxycodone while preserving analgesia were >2.9 ng/mL. Noroxycodone concentrations in plasma did not guarantee analgesic effect. The administration of epidural oxycodone at a dose of 5 mg prolongs the analgesia period to \textasciitilde{}10 hours in patients after THA. Oxycodone may evoke pruritus. A 5 mg dose of oxycodone hydrochloride used in an epidural blockade seems to be a safe drug in patients after THA. [\hyperlink{Oxycodone Hydrochloride}{PMID: 29026333}, Bogumił Olczak et al., 2017]

\hypertarget{pmid_36719881}{U}rsodeoxycholic acid (UDCA) is the main therapeutic drug for cholestasis, but its use in children is controversial. We conducted this study to evaluate the efficacy and safety of ursodeoxycholic acid in children with cholestasis. We searched Medline (Ovid), Embase (Ovid), Cochrane Central Register of Controlled Trials (CENTRAL), CNKI, WanFang Data and VIP from the establishment of databases to July 2022. Eligible studies included Chinese or English randomized controlled trials (RCTs) comparing the efficacy and safety of no UDCA (placebo or blank control) and UDCA in children with cholestasis. This study had been registered with PROSPERO (CRD42022354052). A total of 32 RCTs proved eligible, which included 2153 patients. The results of meta-analysis showed that UDCA could improve symptoms of children with cholestasis (risk ratio 1.24, 95\% CI 1.18 to 1.29; moderate quality of evidence), and serum levels of alanine aminotransferase, total bilirubin, direct bilirubin and total bile acid (low quality of evidence). For some children with specific cholestasis, UDCA could also effectively drop serum levels of aspartate aminotransferase (parenteral nutrition-associated cholestasis) and γ-glutamyl transferase (infantile hepatitis syndrome, parenteral nutrition-associated cholestasis). The most common adverse drug reactions (ADRs) of UDCA in children were gastrointestinal adverse reactions, with an incidence of 10.63\% (67/630). There was no significant difference in the incidence of ADRs between UDCA and placebo/blank control groups (risk difference 0.03, 95\%CI -0.01 to 0.06; moderate quality of evidence), and among children taking different UDCA doses (P = 0.27). The available short-term evidence showed that UDCA was effective and safe for children with cholestasis. Clinicians should use UDCA with caution (start with a low dose) until the long-term effect is further explored in future larger RCTs. [\hyperlink{Oxycodone Hydrochloride}{PMID: 36719881}, Liang Huang et al., 2023]

\hypertarget{pmid_28741653}{C}hloral hydrate is commonly used to sedate children for painless procedures. Children may recover more quickly after sedation with dexmedetomidine, which has a shorter half-life. We randomly allocated 196 children to chloral hydrate syrup 50 mg.kg [\hyperlink{Oxycodone Hydrochloride}{PMID: 28741653}, V M Yuen et al., 2017] To assess the pharmacokinetics and safety of hydrochloride ophthalmic solution 0.77\% olopatadine from 2 independent (Phase I and Phase III, respectively) clinical studies in healthy subjects. The Phase I, multicenter, randomized (2:1), vehicle-controlled study was conducted in subjects ≥18 years old (N=36) to assess the systemic pharmacokinetics of olopatadine 0.77\% following single- and multiple-dose exposures. The Phase III, multicenter, randomized (2:1), vehicle-controlled study was conducted in subjects ≥2 years old (N=499) to evaluate long-term ocular safety of olopatadine 0.77\%. Subjects received olopatadine 0.77\% or vehicle once daily bilaterally for 7 days in the pharmacokinetic study and 6 weeks in the safety study. In the pharmacokinetic study, olopatadine 0.77\% was absorbed slowly and reached a peak plasma concentration (C Olopatadine 0.77\% had minimal systemic exposure or accumulation in healthy subjects and was well tolerated in both adult and pediatric subjects. [\hyperlink{Oxycodone Hydrochloride}{PMID: 28741653}, Edward Meier et al., 2017]

\hypertarget{pmid_30124097}{P}ediatric patients present changing physiological features. Because of the lack of land suitable for commercial management, pediatric specialties very often need to prepare extemporaneous formulations to improve the dosage and administration of drugs for children. Oral liquid formulations are the most suitable for pediatric patients. Clonidine is widely used in the pediatric population for opioid withdrawal, hypertensive crisis, attention deficit disorders and hyperactivity syndrome, and as an analgesic in neuropathic cancer pain. The objective was to study the physicochemical and microbiological stability and determine the shelf life of an oral solution containing 20 µg/mL clonidine hydrochloride in different storage conditions (5 ± 3 °C, 25 ± 3 °C, and 40 ± 2 °C). Using raw material with excipients safe for all pediatric age groups, two oral liquid formulations of clonidine hydrochloride were designed (with and without preservatives). Solutions stored at 5 ± 3 °C (with and without preservatives) were physically and microbiologically stable for at least 90 days in closed containers and for 42 days after opening. Two oral solutions of clonidine hydrochloride 20 µg/mL were developed for pediatric use from raw materials that are readily available and easy to process, containing safe excipients that are stable over a long period of time. [\hyperlink{Oxycodone Hydrochloride}{PMID: 30124097}, V Merino-Bohórquez et al., 2019]

\hypertarget{pmid_2295577}{F}luoxetine hydrochloride is the first selective serotonin uptake inhibitor introduced commercially in the United States. This report describes preliminary clinical experience with fluoxetine in 10 children and adolescents, aged 8 to 15 years, with primary obsessive compulsive disorder (OCD) or Tourette's syndrome (TS) plus OCD. In general, fluoxetine, which was administered from 4 to 20 weeks at a dosage of 10 or 40 mg per day, was well tolerated. Adverse effects included behavioral agitation/activation in four patients and mild gastrointestinal symptoms in two patients. No abnormalities were noted in the seven children who had follow-up EKGs. Five of the 10 patients (50\%) were considered responders; their obsessive-compulsive symptoms decreased substantially during treatment with fluoxetine. Responder rates were similar in the primary OCD (two of four, 50\%) and TS + OCD (three of six, 50\%) groups. In conclusion, short-term fluoxetine administration appears to be safe in children and adolescents. Placebo-controlled trials are needed to further assess the efficacy of fluoxetine. [\hyperlink{Oxycodone Hydrochloride}{PMID: 2295577}, M A Riddle et al., 1990]

\hypertarget{pmid_29203293}{O}rganochlorine pesticides (OCPs) are environmental contaminants that persist in the environment and bioaccumulate through the food chain in humans and animals. Although previous studies have shown an association between prenatal OCP exposure and subsequent neurodevelopment, the levels of OCPs included in these studies were inconsistent. A hospital-based prospective birth cohort study was conducted to examine the associations between prenatal exposure to relatively low levels of OCPs and neurodevelopment in infants at 6 (n=164) and 18 (n=115)months of age. Blood samples were analyzed using gas chromatography/mass spectrometry techniques to quantify 29 OCPs. The Bayley Scales of Infant Development 2nd edition (BSID-II) was used to assess the Mental and Psychomotor Developmental Index. After controlling for confounders, we found an inverse association between prenatal exposure to cis-heptachlor epoxide and the Mental Developmental Index at 18 months of age. Furthermore, infants born to mothers with prenatal concentrations of cis-heptachlor epoxide in the highest quartile had Mental Developmental Index scores -9.8 (95\% confidence interval: -16.4, -3.1) lower than that recorded for infants born to mothers with concentrations of cis-heptachlor epoxide in the first quartile (p for trend <0.01). These results support the hypothesis that prenatal exposure to OCPs, especially cis-heptachlor epoxide, may have an adverse effect on the neurodevelopment of infants at specific ages, even at low levels. [\hyperlink{Oxycodone Hydrochloride}{PMID: 29203293}, Keiko Yamazaki et al., 2018]

\hypertarget{pmid_30290204}{A}s one of several opioids, oxycodone has been widely used, particularly in postoperative analgesia in children and cesarean section. However, the effect of oxycodone on developing brain still remains to be seen. Since there is a link between anesthetics exposure and long-term behavioral or cognitive dysfunction in young children, in the current study, the direct effect of oxycodone on neural stem cells (NSCs) biological behaviors was investigated. After exposed to a high dose of oxycodone (10 μg/mL) for 48 h, NSCs survival and proliferation were significantly reduced, while NSCs apoptosis and differentiation were enhanced. These effects were significantly weaker than that when exposed to same dose of morphine. No significant difference was observed regarding to above biological behaviors when exposed to lower doses (0.1 μg/mL and 1.0 μg/mL) of oxycodone. The antagonist of opioid receptor, nalmefene, successfully reversed the influence of oxycodone. Taken together, our results indicated that short term exposure to oxycodone in low dose might be allowed for developing brain. [\hyperlink{Oxycodone Hydrochloride}{PMID: 30290204}, Gang Wu et al., 2018]

\hypertarget{pmid_9322726}{O}ral pharmacotherapy has been commonly used as an adjunct to clean intermittent catheterization (CIC) in the treatment of neurogenic bladder in order to achieve continence, but may be associated with unacceptable side effects. The authors' experience with sterile intravesical preparations of oxybutynin hydrochloride and ephedrine in children is reported here. Patients requiring CIC for neurogenic bladder but with incontinence that was unresponsive to standard oral therapy or that was associated with severe systemic side effects were studied over a 1-year period. Clinical, radiological and urodynamic assessments were made prior to commencing treatment with intravesical oxybutynin hydrochloride. Patients who remained incontinent with poor internal sphincter muscle tone had intravesical ephedrine added. Seven patients were involved in the study over a 1-year period. Two patients became continent and one patient had an improvement in upper tract dilatation. One patient had a limited improvement with oxybutynin alone but became continent with the addition of ephedrine. Three patients had no response to treatment. There were few side effects. Intravesical agents have a role in the management of paediatric neurogenic bladder for those children with significant adverse sequelae from oral pharmacotherapy who would otherwise require surgical intervention. Intravesical therapy is a safe technique in children with sterile preparations. Further investigation of this modality should be pursued. [\hyperlink{Oxycodone Hydrochloride}{PMID: 9322726}, A J Holland et al., 1997]

\hypertarget{pmid_15721878}{E}xposure to organochlorine compounds (OCs) occurs both in utero and through breastfeeding. Levels of hexachlorobenzene (HCB) in the cord serum of newborns from a population located in the vicinity of an electrochemical factory in Spain are among the highest ever reported. We aimed to assess the degree of breast milk contamination in this population and the subsequent exposure of children to these chemicals through breastfeeding. A birth cohort including 92 mother-infant pairs (84\% of all births in the study area) was recruited between 1997 and 1999 in five neighboring villages. OCs were measured in cord serum, colostrum, breast milk, and children's serum at 13 months of age. Concentrations of OCs were detected and quantified in all colostrum and milk samples. The concentrations in mature milk were lower than those encountered in colostrum. At 13 months of age the highest concentration of OC was found for dichlorodiphenyl dichloroethane (p,p'-DDE), in contrast to what these children presented at birth, where HCB was the highest compound. Those infants who were breastfed had higher concentrations at the age of 1 than those who were formula fed (2.13 ng/mL of HCB among formula feeders vs 4.26 among breast feeders, and 1.95 of p,p'-DDE vs 6.00 (P<0.05)). Long-term breastfeeding leads to a dose-response increase of the concentrations in children's serum during the first year of life. [\hyperlink{Oxycodone Hydrochloride}{PMID: 15721878}, Núria Ribas-Fitó et al., 2005]

\hypertarget{pmid_33747308}{L}a douleur est un problème courant chez les enfants. Des mesures pharmacologiques et non pharmacologiques sont utilisées pour la prendre en charge. Depuis quelques décennies, les opioïdes par voie orale sont populaires pour soulager la douleur modérée à grave. La codéine a longtemps été l'opioïde par voie orale le plus connu pour les enfants. Pour des raisons de sécurité, elle est désormais nettement moins accessible et moins employée. Divers autres opioïdes la remplacent, mais les données sur leur efficacité et leur sécurité sont limitées chez les enfants. L'oxycodone par voie orale emprunte les mêmes voies métaboliques que la codéine, mais sa pharmacocinétique est très variable. Les données sur la sécurité et l'efficacité de l'hydromorphone et du tramadol par voie orale chez les enfants sont également limitées. Lorsqu'on y recourt au lieu de la codéine, la morphine par voie orale est l'opiacé dont la sécurité et l'efficacité sont les mieux démontrées chez les enfants. Des recherches devront être réalisées pour explorer d'autres approches relatives aux médicaments opioïdes et non opioïdes, afin d'orienter les traitements analgésiques fondés sur des données probantes qui soulageront la douleur modérée à grave chez les enfants. [\hyperlink{Oxycodone Hydrochloride}{PMID: 33747308}, Michael J Rieder et al., ]

\hypertarget{pmid_16554451}{Y}oung children are often undertreated for pain. One barrier to effective pain treatment is understanding the pharmacokinetic behavior of analgesics in this age group. Oxycodone is a commonly prescribed opioid for severe pain, yet little is known about its pharmacokinetics in young children. This article used population pharmacokinetic modeling to synthesize pharmacokinetic data from several studies into a model. A single population model that described the observed pharmacokinetics was developed. The combined data were best described with a 2-compartment linear model with different first-order absorption rates depending on route of administration. Weight was found to significantly influence both clearance (CL) and volume of distribution (Vd). The following model adequately describes the population pharmacokinetic profile of oxycodone where absolute bioavailability (F) is estimated for each administration route: CL/F=55x(body weight/70)0.87; V/F=86x(body weight/70)1.16. The interindividual coefficients of variation in CL and Vd were 20.2 and 19.7\%, respectively. This finding confirms that the allometric scaling using the above model explained most of the variability in exposure observed among children. This model confirms using a weight-based dose for oxycodone without adjustment for age between 6 months and 7 years and is valuable for evaluating dosing schedules and dosing routes. [\hyperlink{Oxycodone Hydrochloride}{PMID: 16554451}, Ahmed El-Tahtawy et al., 2006]

\hypertarget{pmid_18682543}{T}o review the role of oxandrolone in pediatric patients with severe thermal burn injury. MEDLINE (1950-April 2008) and Science Citation Index (1900-April 2008) searches were performed using the key terms oxandrolone, burn, and children. All English-language articles that evaluated the efficacy and safety of oxandrolone in pediatric patients with severe thermal burn injury were included in this review. Oxandrolone stimulates protein synthesis by binding to androgen receptors. The efficacy and safety of adjunct oxandrolone therapy in pediatric patients (<or=18 y old) with severe thermal burn injury (total body surface area burn >20\%) were evaluated in 8 clinical studies. Oral oxandrolone 0.1 mg/kg twice daily increased protein synthesis, lean body mass accretion, and muscle strength; improved serum visceral protein concentrations; promoted weight gain; and increased bone mineral content. During the postburn rehabilitation period, oxandrolone 0.1 mg/kg/day improved muscle strength, especially when combined with exercise. Based on clinical studies, oxandrolone 0.1 mg/kg twice daily is safe when given for up to 12 months. However, mild increases in serum liver transaminase concentrations and reversible sexual changes were observed during therapy. Although data on the efficacy and safety of oxandrolone in severely burned children are supported by prospective, randomized, controlled studies, limitations of available data are that they originated from a single study center and that wound healing measurement is lacking in children with severe thermal burns. The benefits of adjunct oxandrolone therapy in severely burned pediatric patients have been demonstrated in the acute postburn injury and long-term postburn rehabilitation periods. Close monitoring of liver function, sexual development, and growth pattern is recommended during oxandrolone treatment. [\hyperlink{Oxycodone Hydrochloride}{PMID: 18682543}, James T Miller et al., 2008]

\hypertarget{pmid_17550483}{O}xycodone has become popular for post-Caesarean section (CS) analgesia yet it is not currently recommended for use in breast-feeding mothers because of limited information on its excretion into breast milk. To investigate the relationship between maternal ingestion of oxycodone after CS and the resultant maternal plasma, breast milk and neonatal plasma drug levels up to 72-h post-partum. Fifty breast-feeding mothers taking oxycodone had blood and breast milk samples analysed for oxycodone levels at 24 h intervals after CS. Forty-one neonates had blood samples taken at 48 h. Oxycodone was detected in the milk of mothers who had taken any dose in a 24-h period, with significant correlation between maternal plasma and milk levels (R(2) = 0.81). The median milk:plasma (M:P) ratio for the same period was 3.2:1. Over the subsequent 48 h, the relationship between plasma and milk levels was less strong (R(2) = 0.59) and there was a larger range of M:P levels with evidence of persistence of oxycodone in the breast milk of some mothers. Oxycodone levels up to 168 ng/mL were detected in breast milk (20\% > 100 ng/mL). Oxycodone was detected in the plasma of one infant. Oxycodone is concentrated in human breast milk up to 72-h post-partum. Breastfed infants may receive > 10\% of a therapeutic infant dose. However, maternal oxycodone intake up to 72-h post-CS poses only minimal risk to the breast-feeding infant as low volumes of breast milk are ingested during this period. [\hyperlink{Oxycodone Hydrochloride}{PMID: 17550483}, Suzette Seaton et al., 2007]

\hypertarget{pmid_2402648}{C}hloral hydrate has been used extensively to sedate children, but at Brooke Army Medical Center, other drug combinations were becoming increasingly popular due to a perception that chloral hydrate had a high rate of failure, especially with younger or neurologically impaired children. Therefore, 50 children were given the drug before a diagnostic study, and patient data and a sedation score were recorded on a worksheet. Of 50 children, 43 (86\%) were "successfully sedated" on the first attempt with no side effects. Children with neurologic disorders had a much greater (27\% vs 4\%) failure rate than "normal" children. The sedation rate did not significantly differ by age, sex, or initial drug dosage. The study suggest that chloral hydrate is a safe and effective oral sedative but that children with neurologic disorders may need alternative drugs for sedation. [\hyperlink{Oxycodone Hydrochloride}{PMID: 2402648}, P D Rumm et al., 1990]

\hypertarget{pmid_493246}{W}ithin the scope of mass examinations 949 children from kindergartens in Basle were submitted to the scotch tape test of Oxyures. The findings were positive in 64 cases (6,7\%). The parents of 50 children accepted the suggestion that their children be treated twice with 100 mg of ciclobendazole each, at a weekly interval. This therapy resulted in complete cure of the oxyuriasis in all treated cases. Despite the impressive and safe action against this disease of ciclobendazole it is recommended in any case to take a second tablet, since the risk of reinfection is high. [\hyperlink{Oxycodone Hydrochloride}{PMID: 493246}, A Bächlin et al., 1979]

\hypertarget{pmid_28275979}{S}edation is often required for children undergoing diagnostic procedures. Chloral hydrate has been one of the sedative drugs most used in children over the last 3 decades, with supporting evidence for its efficacy and safety. Recently, chloral hydrate was banned in Italy and France, in consideration of evidence of its carcinogenicity and genotoxicity. Dexmedetomidine is a sedative with unique properties that has been increasingly used for procedural sedation in children. Several studies demonstrated its efficacy and safety for sedation in non-painful diagnostic procedures. Dexmedetomidine's impact on respiratory drive and airway patency and tone is much less when compared to the majority of other sedative agents. Administration via the intranasal route allows satisfactory procedural success rates. Studies that specifically compared intranasal dexmedetomidine and chloral hydrate for children undergoing non-painful procedures showed that dexmedetomidine was as effective as and safer than chloral hydrate. For these reasons, we suggest that intranasal dexmedetomidine could be a suitable alternative to chloral hydrate. [\hyperlink{Oxycodone Hydrochloride}{PMID: 28275979}, Giorgio Cozzi et al., 2017]

\hypertarget{pmid_8010205}{T}he purpose of this prospective study was to evaluate the safety and efficacy of thioridazine as an adjunct to chloral hydrate sedation when children undergoing MR imaging are difficult to sedate. All 87 children in the study either could not be sedated with chloral hydrate alone or were mentally retarded. Thioridazine (2-4 mg/kg) was administered orally 2 hr before and chloral hydrate (50-100 mg/kg) was administered orally 30 min before the 104 MR examinations. All children were monitored by continuous pulse oximetry. All images were individually evaluated by pediatric radiologists and were graded acceptable if they contained only minimal motion artifact or no motion artifact. Studies were considered successful only when 95\% or more of the images were acceptable. MR imaging was successful in 93 (89\%) of 104 examinations. The success rate for children entered into the study because of prior failure of chloral hydrate sedation was not significantly different from the success rate for children with mental retardation. A tendency for increasing failure rate with age was not significant. No serious complications occurred during the study. The most common adverse reaction, transient reduced oxygen saturation, was seen in five children. Other adverse effects encountered were vomiting in four children, hyperactivity in two children, transient tachycardia in one child, and prolonged sedation in one child. No child required hospitalization because of an adverse reaction to sedation. The study indicates that thioridazine is a safe and effective adjunct to chloral hydrate when a child undergoing MR imaging is difficult to sedate. [\hyperlink{Oxycodone Hydrochloride}{PMID: 8010205}, S B Greenberg et al., 1994]

\hypertarget{pmid_16520840}{C}hloral hydrate is generally considered to be a safe hypnotic drug, and is commonly used for short-term sedation before diagnostic procedures. Its irritant actions to the mucous membranes are usually limited. We report a rare complication of chloral hydrate overdose in an infant. An 8-month-old male infant became unconscious and required ventilation support after an overdose of chloral hydrate was administered to provide sedation for an ophthalmologic examination. White plaques and sloughing of the oropharyngeal mucosa were observed on the next day. Esophagogastroscopy revealed severe corrosive lesions on the whole esophagus. The child recovered after supportive treatment and his oral intake remained well without dysphagia after 1 year. This report illustrates the potential corrosive effect of chloral hydrate. Strict attention should be paid to the dosing and administration protocol of chloral hydrate in infants. The condition of the oropharyngeal mucosa should be carefully monitored after chloral hydrate administration. [\hyperlink{Oxycodone Hydrochloride}{PMID: 16520840}, Yu-Cheng Lin et al., 2006]

\hypertarget{pmid_33747307}{P}ain is a common problem for children, and pain management comprises both pharmacologic and nonpharmacologic measures. For moderate to severe pain, oral opioids have been a popular choice for the last few decades. Codeine has historically been the best-known oral opioid for use in children. However, availability and use of codeine have sharply declined due to safety concerns. A variety of other opioids have been used in place of codeine, but data are limited regarding their efficacy and safety in children. While the same pathways metabolize oral oxycodone as codeine, oxycodone's pharmacokinetics varies widely. There are also limited data on the safety and efficacy of oral hydromorphone and tramadol use for children. Oral morphine is the opiate alternative to codeine for which there is the most evidence of safety and efficacy in children. Research is needed to investigate both other opioids and non-opioid approaches to guide evidence-based analgesic therapy and treatment for moderate-to-severe pain in children. [\hyperlink{Oxycodone Hydrochloride}{PMID: 33747307}, Michael J Rieder et al., ]

\section*{Phentermine Hydrochloride}
\subsection*{Result}
\subsubsection*{Answer}

Unknown

\subsubsection*{{Explanation}}
\hypertarget{Phentermine Hydrochloride}
A review of the available abstracts reveals the following regarding the safety of Phentermine Hydrochloride in children:

- One abstract specifically discusses Phentermine Hydrochloride, describing it as a noradrenergic sympathetic amine approved for short-term treatment of obesity in adults. The abstract summarizes its use, dosing, and adverse events in the context of adult clinical trials and combination therapy with topiramate. However, it does not mention any studies or data regarding the safety or efficacy of Phentermine Hydrochloride in children or adolescents. There is no indication that a targeted safety study in a pediatric population has been conducted or reported in this abstract [\hyperlink{pmid_22348915}{PMID: 22348915}, H E Bays et al., 2011].

- No other abstracts mention Phentermine Hydrochloride in the context of pediatric use or safety studies.

Therefore, based on the abstracts available, there is no evidence from targeted studies affirming the safety or unsafety of Phentermine Hydrochloride in children of any age range. The safety of Phentermine Hydrochloride in children is therefore unknown.

\subsection*{Abstracts}
\hypertarget{pmid_1172955}{P}hencyclidine hydrochloride is a dangerous drug. Its incidence as the causative agent in childhood poisoning is increasing. A pressor effect of phencyclidine has been noted in studies both in man and in experimental animals. We summarize seven cases of poisoning with this drug, including one in which death occurred following a hypertensive crisis. Patients who have ingested this drug should have continuous monitoring of blood pressure in an intensive care unit. [\hyperlink{Phentermine Hydrochloride}{PMID: 1172955}, J W Eastman et al., 1975]

\hypertarget{pmid_18702885}{A}llergic rhinitis (AR) is a common chronic condition in children and may impact a child's quality of life. Increasing treatment compliance may improve quality of life. An oral suspension of fexofenadine hydrochloride (HCl) has been developed to ease administration to children and may, therefore, improve treatment compliance. The purpose of this study was to assess the pharmacokinetic behavior, safety, and tolerability of a single dose of fexofenadine HCl oral suspension administered to children aged 2-5 years with allergic rhinitis. Children (aged 2-5 years) with AR were recruited in a multicenter, open-label, single-dose study. Fexofenadine HCl (30 mg) was administered as a 6-mg/mL suspension (5 mL). Plasma samples were collected up to 24 hours postdose. Adverse events (AEs); electrocardiograms (ECGs); vital signs; and clinical laboratory tests for hematology, blood chemistry, and urinalysis were analyzed to evaluate safety and tolerability. Fifty subjects completed the study. Mean maximum plasma concentration of fexofenadine was 224 ng/mL, and mean area under the plasma concentration curve was 898 ng . hour/mL. Treatment-emergent AEs were mild in intensity and reported in a total of seven subjects. No trends or clinically meaningful changes in mean ECG, vital sign, or clinical laboratory test data occurred during the study. In children aged 2-5 years, the exposure after a 30-mg dose of fexofenadine HCl suspension was similar to the exposures previously seen after a 30- and 60-mg dose of fexofenadine HCl in children aged 6-11 years and in adults, respectively. The suspension was also well tolerated. [\hyperlink{Phentermine Hydrochloride}{PMID: 18702885}, Nathan Segall et al., ]

\hypertarget{pmid_942230}{K}etamine hydrochloride 2 mg/kg, together with atropine 0.2 mg, has been given intravenously on 100 occasions on a general paediatric ward. No serious side effects occurred. Dreams followed in 4 children but did not reduce acceptability of the drug. In our hands it has greatly reduced the pain and distress of children undergoing many routine medical procedures, particularly the dread which builds up when these have to be repeated in the same child. It has also produced close to ideal conditions for the operator, and probably increased his efficiency by reducing the emotional strain which occurs when doing painful things to a frightened patient. [\hyperlink{Phentermine Hydrochloride}{PMID: 942230}, E Elliott et al., 1976]

\hypertarget{pmid_17941284}{T}he safety of fexofenadine has been examined extensively in adults and school-age children. However, the safety of fexofenadine in children younger than 6 years has not been reported to date. To compare the safety and tolerability of twice-daily fexofenadine hydrochloride, 30 mg, and placebo in preschool children aged 2 to 5 years with allergic rhinitis. This was a multicenter, double-blind, randomized, placebo-controlled, parallel-group study, conducted between February 29, 2000, and June 14, 2001. Participants were randomized to either fexofenadine hydrochloride, 30 mg, or placebo twice daily for a 2-week period. To facilitate dosing, capsule content was mixed with applesauce (approximately 10 mL). Safety assessments depended on date of entry into the study because of an amendment to the protocol. Before the amendment, assessments included physical examination, vital signs reporting (oral temperature, heart rate, and respiratory rate), and adverse event (AE) reporting. After the amendment, safety assessments included laboratory testing (blood chemistry and hematology profiles), physical examination, 12-lead electrocardiography, and vital signs (oral temperature, blood pressure, heart rate, and respiratory rate) and AE reporting. Treatment-emergent AEs were observed in 116 of 231 participants receiving placebo and 111 of 222 receiving fexofenadine. These AEs were possibly related to study medication in 19 (8.2\%) and 21 (9.5\%) of the participants receiving placebo and fexofenadine, respectively, and most frequently involved the digestive system. No clinically relevant differences in laboratory measures, vital signs, and physical examinations were observed. The findings show that fexofenadine hydrochloride, 30 mg, is well tolerated and has a good safety profile in children aged 2 to 5 years with allergic rhinitis. [\hyperlink{Phentermine Hydrochloride}{PMID: 17941284}, Henry Milgrom et al., 2007]

\hypertarget{pmid_22348915}{P}hentermine hydrochloride is a noradrenergic sympathetic amine approved for decades by the U.S. Food and Drug Administration (FDA) at doses as high as 37.5 mg/day for the short-term treatment of obesity. Topiramate is a sulfamate-substituted monosaccharide marketed since 1996, and approved by the FDA for seizure disorders at doses up to 400 mg/day and for the prevention of migraine headaches at doses up to 100 mg/day. Clinical trial data suggest topiramate promotes weight loss. The prescribing information of neither agent describes adverse drug interactions with the other. The controlled-release formulation of phentermine and topiramate at low, medium and full doses (with full dose containing 15 mg of phentermine hydrochloride and 92 mg of topiramate) promotes weight reduction, with clinical trial data supporting improvement in adiposopathic consequences leading to metabolic diseases. Reported adverse events with this combination agent are as expected, based upon knowledge of the individual components. [\hyperlink{Phentermine Hydrochloride}{PMID: 22348915}, H E Bays et al., 2011]

\hypertarget{pmid_20527137}{O}nly a few corticosteroids for topical use have proven safe and effective in pediatric populations down to 3 months of age. The authors report the results of a study designed to assess the efficacy and safety of hydrocortisone butyrate (HCB) 0.1\% in lipocream (LCr) vehicle in infants and children. A total of 264 boys and girls 3 months to less than 18 years old, with stable, mild to moderate atopic dermatitis affecting at least 10\% body surface area applied HCB 0.1\% in LCr or LCr alone twice daily for up to 1 month without occlusion. Primary end-points included: percent of patients who achieved treatment success based on physician global assessments. Secondary endpoint included: difference in pruritus and Eczema Area and Severity Index (EASI) at day 29. Treatment was significant (P < 0.001) for HCB 0.1\% LCr over vehicle. No serious nor significant adverse events were reported. Results are representative of a short duration treatment for a chronic disease. HCB 0.1\% in LCr is more effective than its vehicle in pediatric populations down to 3 months of age without significant adverse events when used twice a day for up to 1 month. [\hyperlink{Phentermine Hydrochloride}{PMID: 20527137}, William Abramovits et al., ]

\hypertarget{pmid_28741653}{C}hloral hydrate is commonly used to sedate children for painless procedures. Children may recover more quickly after sedation with dexmedetomidine, which has a shorter half-life. We randomly allocated 196 children to chloral hydrate syrup 50 mg.kg [\hyperlink{Phentermine Hydrochloride}{PMID: 28741653}, V M Yuen et al., 2017] Recently there has been a resurgence in the utilization of ketamine, a unique anaesthetic, for emergency procedures requiring sedation. The purpose of the present study was to examine the safety and efficacy of ketamine for sedation in the treatment of children's fractures in the small clinic setup of rural Nepal. One hundred and fourteen children (average age, 5.3 years; range, twelve months to ten years and ten months) who underwent closed reduction of an isolated fracture or dislocation in the Orthopaedic \& Trauma Clinic at Janakpurdham were prospectively evaluated. Ketamine hydrochloride was administered intravenously (at a dose of less than two milligrams per kilogram of body weight) in ninety-nine of the patients and intramuscularly (at a dose of four milligrams per kilogram of body weight) in the other fifteen. Adequate fracture reduction was obtained in 111 of the children. Ninety-nine percent (sixty-eight) of the sixty-nine parents present during the reduction were pleased with the sedation and would allow it to be used again in a similar situation. Minor side effects included nausea (thirteen patients), emesis (eight of the thirteen patients with nausea), clumsiness (evident as ataxic movements in ten patients), and dysphonic reaction (one patient). No long-term sequelae were noted, and no patients had hallucinations or nightmares. Ketamine reliably, safely, and quickly provided adequate sedation to effectively facilitate the reduction of children's fractures at our institution. Therefore, it was ideal for small clinic in our setup. [\hyperlink{Phentermine Hydrochloride}{PMID: 28741653}, Ram Kewal Shah et al., 2003]

\hypertarget{pmid_18219837}{A}ntihistamines are an established first-line treatment for allergic rhinitis and are widely prescribed in infants for allergic symptoms. To establish the safety and tolerability of fexofenadine hydrochloride in children aged 6 months to 2 years in 2 studies (T/3001 and T/3002). Both studies had a multicenter, randomized, placebo-controlled design. Mean treatment duration was 8 days. Subjects were randomized (T/3001, n = 174; and T/3002, n = 219) to twice-daily fexofenadine hydrochloride, 15 or 30 mg, or placebo mixed with a standard vehicle. In the combined population, the incidence of treatment-emergent adverse events (TEAEs) was comparable between groups (placebo, 48.2\% [96/199]; fexofenadine hydrochloride, 15 mg, 40.0\% [34/85]; and fexofenadine hydrochloride, 30 mg, 35.2\% [38/108]). Vomiting was the most common TEAE (placebo, 13.6\%; fexofenadine hydrochloride, 15 mg, 14.1\%; and fexofenadine hydrochloride, 30 mg, 5.6\%). Most TEAEs were unrelated to study medication, as evaluated by investigators; those possibly related to study medication were mild or moderate in intensity. No clinical differences were seen between fexofenadine and placebo for vital signs, electrocardiographic results, or physical examination results. Fexofenadine hydrochloride, 15 or 30 mg, given for a mean duration of 8 days is well tolerated, with a good safety profile, in children aged 6 months to 2 years. [\hyperlink{Phentermine Hydrochloride}{PMID: 18219837}, Frank C Hampel et al., 2007]

\hypertarget{pmid_30463814}{D}exmendetomidine hydrochloride (DEX) is a new common adrenergic receptor agonist, which not only keeps children calm but also has analgesic effect. Dexmedetomidine hydrochloride will enable children to maintain the natural non-REM sleep, which can be stimulated sedation or language arousal. The aim of this study is to observe the sedative effect and adverse drug reactions of dexmedetomidine hydrochloride injection and propofol injection in MRI examination. In this study, no children in the experimental group were required to add sedative drugs, and 2 cases in the control group were treated with sedative drugs. In experimental group, it used dexmedetomidine hydrochloride as (1.64±0.91) g/kg; in control group, dosage of narcotic drugs as (5.26±1.82) g/kg, and the total complication rate of the children in the experimental group was lower than that of the control group (P<0.05). After returning to the ward, the doses of phenobarbital sedation were dexmedetomidine group (4.28±1.53) mg/kg and propofol group (6.40±1.71) mg/kg. There was significant difference between the two groups. The total complication rate in the experimental group was lower than that in the control group (P<0.05). The quality of MRI in the test group was significantly higher than that in the control group, which showed that dexmedetomidine hydrochloride could provide a satisfactory sedative effect in the MRI examination of children. To sum up, dexmedetomidine hydrochloride is a wide range of clinical applications. It is an effective drug for the maintenance of sedation in clinical disease treatment. It is flexible in the way of administration and with less adverse reactions. It is suitable for popularization and application in clinical practice. [\hyperlink{Phentermine Hydrochloride}{PMID: 30463814}, Zhendong Yang et al., 2018]

\hypertarget{pmid_2058184}{D}iphenhydramine hydrochloride is an antihistamine with anticholinergic properties that is frequently used both orally and topically for the temporary relief of pruritus. Significant systemic absorption may occur following topical administration of diphenhydramine in patients with varicella-zoster lesions. We describe three children with varicella-zoster infection (VZI) who developed bizarre behavior as well as visual and auditory hallucinations following topical applications of large amounts of diphenhydramine to the majority of skin surfaces. In two cases, oral diphenhydramine was also administered. Serum diphenhydramine concentrations approximated or exceeded those previously reported. In each case, a complete resolution of mental status abnormalities occurred within 24 hours after discontinuation of all diphenhydramine-containing products. Pharmacists and other health professionals should be aware of the potential toxicity of topical diphenhydramine in patients with VZI. [\hyperlink{Phentermine Hydrochloride}{PMID: 2058184}, C Y Chan et al., 1991]

\hypertarget{pmid_28292340}{T}he purpose of this study was to evaluate, using a randomized, double-blind methodology: (1) the safety of phentolamine mesylate (Oraverse) in accelerating the recovery of soft tissue anesthesia following the injection of two percent lidocaine plus 1:100,000 epinephrine in two- to five-year-olds; and (2) efficacy in four- to five-year-olds only. One hundred fifty pediatric dental patients underwent routine dental restorative procedures with two percent lidocaine plus 1:100,000 epinephrine with doses based on body weight. Phentolamine mesylate or a sham injection (two to one ratio) was then administered. Subjects were monitored for safety and, in four- to five-year-olds, for efficacy during the two-hour evaluation period. There were no significant differences in adverse events between the phentolamine and sham injections. Compared to sham, phentolamine was not associated with nerve injury, increased analgesic use, or abnormalities of the oral cavity. Phentolamine was associated with transient decreased blood pressure in some children. In four- and five-year-olds, phentolamine induced more rapid recovery of lip anesthesia by 48 minutes (P<0.0001). Phentolamine was well tolerated and safe in three- to five-year-olds; in four- to five-year-olds, a statistically significant more rapid recovery of lip sensation compared to sham injections was determined. [\hyperlink{Phentermine Hydrochloride}{PMID: 28292340}, Elliot V Hersh et al., 2017]

\hypertarget{pmid_6378158}{O}ne hundred fifty-two children were enrolled in a randomized, controlled clinical trial of the efficacy of phenylephrine hydrochloride nose drops or nasal spray in hastening the resolution of middle ear effusion. Children with persistent effusion were recruited for the study during a return visit two weeks after an episode of acute otitis media. Forty-six patients (30\%) dropped out of the study, many because they failed to tolerate the medication, especially the nose drops. Another 27 (18\%) had to be excluded because of intercurrent illness or systemic drug therapy. Among those children completing the study, rates of clinical and tympanometric cure during the following four weeks were similar in the drug and placebo groups. In view of the absence of documented clinical efficacy and the practical difficulties inherent in their administration, topical decongestants appear to have a limited role, if any, in treating children with persistent effusion. [\hyperlink{Phentermine Hydrochloride}{PMID: 6378158}, G F Hayden et al., 1984]

\hypertarget{pmid_16777373}{P}enequine hydrochloride, a novel anticholinergic agent, was developed as an effective treatment for organophosphorus intoxication (e.g., soman poisoning). The current study was performed to assess the potential pre- and post-natal toxicity of penequine hydrochloride in mice. Approximately 120 timed-pregnant mice were assigned to four dose groups (n=30 per group). Dams were exposed orally to 0, 2.5, 12.5, 62.5 mg/L penequine hydrochloride in drinking water from gestation day 6 to lactation day 21. The F1 generation mice, which were not exposed directly to penequine hydrochloride as pups or as adults, were bred to produce F2 generation fetuses for the fertility test of the F1 population. Various pre- and post-natal measurements, including neurobehavioral tests, were performed with the F0 and F1 mice. Among the significant findings were decreases in water consumption, viability, organ weights and delay of physical landmarks in 62.5 mg/L groups. With the exception of treatment-unrelated abnormality in surface righting reflex in the F1 generation, penequine hydrochloride did not produce any adverse effects at doses up to and including 12.5 mg/L (equal to 2.5 mg/kg/day in mice) that were at least 75 times of human therapeutic dosage. [\hyperlink{Phentermine Hydrochloride}{PMID: 16777373}, Zibo Zhang et al., 2006]

\hypertarget{pmid_15951862}{D}iagnostic and therapeutic procedures in children are made easier using sedation. However, there is no consensus about which drug should be used to achieve this. Furthermore, none of the drugs used for sedation are risk free. The aim of this work is to study sedation indications, effectiveness, and safety at our center. A prospective observational study conducted at the Pediatric Day Care Unit, King Fahad National Guard Hospital, Riyadh, Saudi Arabia. The study covered 17.5 weeks in 2 periods: May 9th 1999 to June 13th 1999 and October 31st 2001 to February 11th 2002. Children <12 years were included. Collected data included demographics, indication, drug dosing and outcome. Data were reported as mean +/- SD. We included 148 patients, age 38 +/- 30 months. Adequate sedation was achieved in 79\% after initial chloral hydrate (CH) dose of 56.9 +/- 9.3 mg/kg, in 95\% after adding 18.5 +/- 6.4 mg/kg CH and in 96\% after adding second drug. Compared to nonrespondents, first CH dose respondents were younger and lower in weight. The CH side effects were few and mild. Chloral hydrate is a safe and effective agent for sedation in children with an age and weight dependent response. [\hyperlink{Phentermine Hydrochloride}{PMID: 15951862}, Omar M Hijazi et al., 2005]

\hypertarget{pmid_28275979}{S}edation is often required for children undergoing diagnostic procedures. Chloral hydrate has been one of the sedative drugs most used in children over the last 3 decades, with supporting evidence for its efficacy and safety. Recently, chloral hydrate was banned in Italy and France, in consideration of evidence of its carcinogenicity and genotoxicity. Dexmedetomidine is a sedative with unique properties that has been increasingly used for procedural sedation in children. Several studies demonstrated its efficacy and safety for sedation in non-painful diagnostic procedures. Dexmedetomidine's impact on respiratory drive and airway patency and tone is much less when compared to the majority of other sedative agents. Administration via the intranasal route allows satisfactory procedural success rates. Studies that specifically compared intranasal dexmedetomidine and chloral hydrate for children undergoing non-painful procedures showed that dexmedetomidine was as effective as and safer than chloral hydrate. For these reasons, we suggest that intranasal dexmedetomidine could be a suitable alternative to chloral hydrate. [\hyperlink{Phentermine Hydrochloride}{PMID: 28275979}, Giorgio Cozzi et al., 2017]

\hypertarget{pmid_18254579}{P}heochromocytoma in children shows much worse complications than that in the adult patients. An 11-year-old girl was transferred to our emergency room after suffering from headache, dizziness, cold sweating and palpitation for 3 days. Severe hypertension, remarkable blood pressure fluctuation between 260/160 and 65/50 mmHg, decrease of cardiac contractility, as well as abnormal electrocardiogram findings including ST-T segment elevation and QT interval prolongation were noted soon after admission. Later, a 4x4.5x2.5 cm tumor in the right adrenal gland area was found by computed axial tomogram study. Assessment of the urine catecholamine metabolites showed high levels of vanillylmandelic acid, normetanephrine and norepinephrine indicating an active adrenal pheochromocytoma produced mainly norepinephrine. Although several antihypertensive drugs were used, ventricular tachycardia and Torsade de pointe still occurred on her for 3 times, each was preceded by a period of blood pressure fluctuation and burst out concomitantly at the peak of a hypertension crisis. From this case, we found that when the specific alpha-blocker like phenoxybenzamine or phentolamine was not available to us, labetalol by continuous intravenous infusion was the only effective drug to protect the patient from attacks of hypertensive crisis and ventricular tachycardia. Her right adrenal gland was resected smoothly when BP was well under control. Histological examination showed the adrenal medulla was full of pheochromocytoma cells. [\hyperlink{Phentermine Hydrochloride}{PMID: 18254579}, Yu-Chih Huang et al., ]

\hypertarget{pmid_37655364}{F}exofenadine hydrochloride (HCl) is a second-generation, nonsedating, histamine H1-receptor antagonist used to manage seasonal allergic rhinitis and chronic idiopathic urticaria. A new oral pediatric suspension of fexofenadine HCl has been developed, with the preservative potassium sorbate replacing parabens. The objective of this phase 1 single-center, open-label, randomized, 2-treatment, full-replicated, 4-period, 2-sequence crossover study in healthy adult volunteers was to assess the bioequivalence of 30 mg of the new oral suspension of fexofenadine HCl (test) versus 30 mg of the marketed pediatric oral suspension of fexofenadine HCl (reference). The replicate design was based on the high intra-individual variability of fexofenadine (>30\% on C [\hyperlink{Phentermine Hydrochloride}{PMID: 37655364}, Clemence Rauch et al., 2023] The aim of this technical note is to show that ketamine hydrochloride anesthesia, owing to the preservation of muscle tone, enables one to safely and comfortably carry out gaseous encephalography on a simple radiological table in the toddler or child. However the authors very strictly select the indications for this investigation. Whenever the child's clinical condition leads one to suspect intracranial hypertension and/or a cerebral tumor, they think it more prudent, owing to the vasopressor effect of ketamine hydrochloride and a possible elevation in the cerebrospinal fluid pressure, to transfer the child to a specialized neuroradiological center, where the investigations would be carried out under the best technical conditions, and close to a neurosurgical unit which is capable of intervening rapidly in case of complications. The authors voluntarily limit their indications to children suffering from psychomotor retardation, epilepsy or neurological disorders which make one suspect a congenital malformation. [\hyperlink{Phentermine Hydrochloride}{PMID: 37655364}, C Fauré et al., 1975]

\hypertarget{pmid_2295577}{F}luoxetine hydrochloride is the first selective serotonin uptake inhibitor introduced commercially in the United States. This report describes preliminary clinical experience with fluoxetine in 10 children and adolescents, aged 8 to 15 years, with primary obsessive compulsive disorder (OCD) or Tourette's syndrome (TS) plus OCD. In general, fluoxetine, which was administered from 4 to 20 weeks at a dosage of 10 or 40 mg per day, was well tolerated. Adverse effects included behavioral agitation/activation in four patients and mild gastrointestinal symptoms in two patients. No abnormalities were noted in the seven children who had follow-up EKGs. Five of the 10 patients (50\%) were considered responders; their obsessive-compulsive symptoms decreased substantially during treatment with fluoxetine. Responder rates were similar in the primary OCD (two of four, 50\%) and TS + OCD (three of six, 50\%) groups. In conclusion, short-term fluoxetine administration appears to be safe in children and adolescents. Placebo-controlled trials are needed to further assess the efficacy of fluoxetine. [\hyperlink{Phentermine Hydrochloride}{PMID: 2295577}, M A Riddle et al., 1990]

\hypertarget{pmid_24627951}{T}o determine the safety and efficacy of high-dose oral chloral hydrate for pediatric ophthalmic procedures. This study is a retrospective review of a quality audit of pediatric sedation for ophthalmic evaluation and imaging performed at King Khaled Eye Specialist Hospital between January 1 and December 31, 2011, in children aged 1 month to 6 years. Three hundred fifty-eight of 380 (94.2\%) sedation procedures were successful after a single dose of chloral hydrate, with 356 of 380 (93.7\%) children sedated within 45 minutes of the first dose. The total success rate of the sedation procedure increased to 97.9\% (372 of 380) when a second dose was administered. Children adequately sedated after a single dose of chloral hydrate were on average younger and weighed less than children who required additional doses. No major adverse events were documented. The use of chloral hydrate sedation for ophthalmic evaluation and imaging was safe and effective in this patient population with a high rate of procedure completion. [\hyperlink{Phentermine Hydrochloride}{PMID: 24627951}, Michelle E Wilson et al., ]

\hypertarget{pmid_23332206}{T}o determine whether the adrenoceptor agonist, ephedrine hydrochloride, is an effective treatment for resistant non-neurogenic daytime urinary incontinence in children. From 2000 to 2010, eighteen children with resistant non-neurogenic daytime urinary incontinence were treated with oral ephedrine hydrochloride at our institution. Sixteen were female and two were male. Median age at treatment was 12 years (range 5-15 years). Two children had spina bifida occulta. There were no other co-morbidities. Multiple anticholinergics were prescribed and dose maximized to support a bladder and bowel training programme, without achieving continence in this resistant group of children. Pre-treatment urodynamics were normal in 10, but revealed an open bladder neck in 8 patients. None showed detrusor over-activity. Oral ephedrine hydrochloride was started at 7.5 mg or 15 mg twice daily and titrated up to a maximum of 30 mg four times daily according to response. Median follow-up was 7 years (range 6-8 years). Seventeen children (94\%) reported improvement in symptoms and six (33\%) achieved complete urinary continence. All patients maintained compliant bladders on post-treatment urodynamics. Seven of the 8 previously open bladder necks were closed. No patients reported any significant side effects. Patients with open bladder necks on pre-treatment urodynamics were more likely to show a full response to ephedrine (odds ratio 15; 95\% CI 1.2-185.2). Oral ephedrine hydrochloride is an effective treatment for carefully selected children with resistant non-neurogenic daytime urinary incontinence. [\hyperlink{Phentermine Hydrochloride}{PMID: 23332206}, Neil Featherstone et al., 2013]

\hypertarget{pmid_6660444}{T}he article reports on the paediatric-anaesthesiological treatment of 6 phaeochromocytomas in 5 children who were 8 to 16 years of age. Therapeutic recommendations for the perioperative medication of infantile phaeochromocytoma patients are involved. The therapeutic aim of this study was the management of the effects of phaeochromocytoma before and after extirpation of the tumour, the effect of the phaeochromocytoma being of an alpha-adrenergic and beta-adrenergic cardiovascular nature and transmitted by catecholamines. Preoperative stabilization of blood pressure by means of the alpha-blocker phenoxybenzamine and a subsequent intraoperative, controlled reduction of blood pressure by means of sodium nitroprusside were found to be an effective, safe and easily appreciated therapeutic concept for the perioperative care of paediatric phaeochromocytoma patients. Considerable individual differences in dose an duration of the necessary preoperative phenoxybenzamine administration rendered ward control of therapy recommendable. The risk of complete alpha-sympathicolysis by additive drug effects during premedication and induction of anaesthesia, had to be taken into consideration for conducting phenoxybenzamine therapy. Additional administration of the beta-blocker pindolol successfully controlled the intraoperatively manifested tachycardial heart rhythm phases without provoking any complicating arrhythmias. During the entire perioperative treatment of the patients it is mandatory to ensure sufficient substitution of intravascular volume to prevent hypotensive complications. Our patients did not need any cardiac and sympathicomimetic drugs as postoperative administration. None of the patients had any perioperative complications worth mentioning. [\hyperlink{Phentermine Hydrochloride}{PMID: 6660444}, M Abel et al., 1983]

\hypertarget{pmid_15517550}{F}exofenadine hydrochloride is a non-sedating antihistamine that is used in the treatment of symptoms associated with seasonal allergic rhinitis and chronic idiopathic urticaria. A pooled analysis of pharmacokinetic data from children 6 months to 12 years of age and adults was conducted to identify the dose(s) in children that produce exposures comparable to those in adults for the treatment of seasonal allergic rhinitis. The pharmacokinetic parameter database included peak and overall exposure data from 269 treatment exposures from 136 adult subjects, and 90 treatment exposures from 77 pediatric allergic rhinitis patients. The data were pooled and analysed using NONMEM software, version 5.0. A covariate model based on body weight and age and a power function model based on body weight were identified as appropriate models to describe the variability in fexofenadine oral clearance and peak concentration, respectively. Individual oral clearance estimates were on average 44\%, 36\% and 61\% lower in children 6 to 12 years (n=14), 2 to 5 years (n=21), and 6 months to 2 years (n=42), respectively, compared with adults. Trial simulations (n=100) were carried out based on the final pharmacostatistical models and parameter estimates to identify the appropriate dose(s) in children relative to the marketed dose of 60 mg fexofenadine hydrochloride in adults. The trials were designed as crossover studies in 18 subjects comprising various potential dosing regimens with and without weight stratification. Pharmacokinetic parameter variability was assumed to have a log-normal distribution. Individual weights and ages were simulated using mean (SD) estimates derived from the studies used in this analysis and proportional measurement/model mis-specification errors derived from the analysis were incorporated into the simulation. The results indicated that a 30 mg dose of fexofenadine hydrochloride administered to children 1 to 12 years of age and weighing >10.5 kg and a 15 mg dose administered to children 6 months and older and weighing <or=10.5 kg produces exposures similar to those seen with the 60 mg dose in adults. [\hyperlink{Phentermine Hydrochloride}{PMID: 15517550}, Rajesh Krishna et al., 2004]

\hypertarget{pmid_439264}{K}etamine hydrochloride was used successfully to produce detumescence in 7 children undergoing hypospadias repair. There have been no complications or side effects. We recommend ketamine hydrochloride as a useful adjunct in managing corporeal erections during urologic procedures. [\hyperlink{Phentermine Hydrochloride}{PMID: 439264}, J R Pietras et al., 1979]

\section*{Retevmo}
\subsection*{Result}
\subsubsection*{Answer}
Unknown
\subsubsection*{{Explanation}}
\hypertarget{Retevmo}
A review of the available abstracts reveals that none of them are about Retevmo (selpercatinib). The abstracts provided discuss the safety and efficacy of other medications in pediatric populations, such as letermovir, levetiracetam, levamisole, ketamine, remimazolam, rituximab, and talimogene laherparepvec, but there is no mention of Retevmo or studies specifically evaluating its safety in children. Therefore, based on the abstracts available, there is not enough data to determine whether Retevmo is safe or unsafe for use in children in any age range.

\subsection*{Abstracts}
\hypertarget{pmid_36244677}{L}etermovir is an attractive cytomegalovirus (CMV) prophylactic agent, but published data in children are scarce. This retrospective chart review aimed to describe our experience using letermovir as CMV prophylaxis in pediatric hematopoietic cell transplantation (HCT) recipients. Pediatric patients (age <20 years) undergoing allogeneic HCT and receiving letermovir prophylaxis in the Mayo Clinic Pediatric Bone Marrow Transplant Program were eligible for inclusion in this retrospective chart review. Medical records were reviewed to evaluate letermovir dosing, CMV levels, laboratory values, and reports of adverse effects. Between October 2020 and April 2022, 9 patients age 4 to 19 years undergoing allogeneic HCT in the Pediatric Bone Marrow Transplant Program received letermovir prophylaxis, either 240 mg or 480 mg daily at a mean and median dose of 10 mg/kg/day. Letermovir was crushed and administered via nasogastric tube in 4 of 9 patients. Two patients received letermovir for secondary CMV prophylaxis after initial treatment with ganciclovir/valganciclovir, and the remaining 7 received letermovir for primary prophylaxis. One patient, a 20-kg 6-year-old female receiving 240 mg (12 mg/kg), experienced low-level CMV viremia while on letermovir. No other patients experienced CMV reactivation while on letermovir prophylaxis. In 2 patients, transient mild transaminitis was noted within the first weeks of letermovir therapy, which resolved without intervention, and its relationship to letermovir could not be clearly established. Letermovir administration was feasible and well tolerated as CMV prophylaxis in our small cohort of pediatric patients undergoing HCT. Larger, prospective studies are warranted to confirm the safety and efficacy of letermovir in children. © 2022 American Society for Transplantation and Cellular Therapy. Published by Elsevier Inc. [\hyperlink{Retevmo}{PMID: 36244677}, Alexis Kuhn et al., 2023]

\hypertarget{pmid_27011634}{T}o report the effectiveness and safety of intravenous (IV) levetiracetam (LEV) in the treatment of critically ill children with acute repetitive seizures and status epilepticus (SE) in a children's hospital. We retrospectively analyzed data from children treated with IV LEV. The mean age of the 108 children was 69.39 ± 46.14 months (1-192 months). There were 58 (53.1\%) males and 50 (46.8\%) females. LEV load dose was 28.33 ± 4.60 mg/kg/dose (10-40 mg/kg). Out of these 108 patients, LEV terminated seizures in 79 (73.1\%). No serious adverse effects were observed but agitation and aggression were developed in two patients, and mild erythematous rash and urticaria developed in one patient. Antiepileptic treatment of critically ill children with IV LEV seems to be effective and safe. Further study is needed to elucidate the role of IV LEV in critically ill children. [\hyperlink{Retevmo}{PMID: 27011634}, Faruk Incecik et al., ]

\hypertarget{pmid_25593242}{R}ituximab (RTX) has been used to treat many pediatric autoimmune conditions. We investigated the safety and efficacy of RTX in a variety of pediatric autoimmune diseases, especially systemic lupus erythematosus (SLE). Retrospective study of children treated with RTX. Effectiveness data was recorded for patients with at least 12 months of followup; safety data was recorded for all subjects. The study included 104 children; 50 had SLE. Improvements in corticosteroid dosage, physician's global assessment of disease activity, and SLE-associated markers of disease activity were seen. The incidence of hospitalized infections was similar to previous studies of patients with childhood-onset SLE. RTX can be safely administered to children and appears to contribute to decreased disease activity and steroid burden. [\hyperlink{Retevmo}{PMID: 25593242}, Ajay Tambralli et al., 2015]

\hypertarget{pmid_12370722}{T}o evaluate the effectiveness of levamisole in maintaining remission in children with steroid-sensitive nephrotic syndrome (SSNS) who had a frequent relapsing or steroid-dependent course. All children with SSNS who had a frequent relapsing or steroid-dependent course and were treated with levamisole between 1997 and 2001 at King Abdul-Aziz University Hospital, Jeddah, Kingdom of Saudi Arabia were reviewed. All patients were treated by the same steroid protocol used in our unit. Levamisole was considered effective if the patient successfully remained in remission on Prednisolone 0.5 mg/kg/48 hours or less. Nine children were treated with levamisole (3 mg/kg/48 hours) with median (range) age of 6 (3.5-10) years. Seven received levamisole for more than 6 (6-24) months and 2 were excluded because they did not adhere to treatment. Levamisole was effective in 4 patients (57\%) with remarkable reduction in the number of relapses and the steroid maintenance dose. Renal biopsy was performed in 4 patients: 2 responders with biopsy findings of minimal change disease (MCD) and mesangioproliferative glomerulonephritis and another 2 non responders with biopsy findings of MCD and focal segmental glomerulosclerosis. No significant side effect was observed. Levamisole is effective in maintaining remission in steroid SSNS in Arab children and has few side effects. [\hyperlink{Retevmo}{PMID: 12370722}, Hammad O Alshaya et al., 2002]

\hypertarget{pmid_17258474}{L}evetiracetam (LEV) is a novel antiepileptic drug (AED) that has recently obtained marketing authorisation for use in children. The purpose of this study was to assess the efficacy, tolerability and retention rate of LEV in children with refractory epilepsies. It is a retrospective multicentre observational study reporting the use of LEV in 200 children, aged 0.3-19 years (median 9-years-old) over a 4-year period. All of the patients included in the study had refractory epilepsy with a median age of onset of epilepsy of 3 years (range 0-13 years). The 38\% had failed and withdrawn 3 or more AEDs previously and 24\% were taking at least 2 other AEDs in addition to LEV. The 47\% had focal, and 58\% had symptomatic epilepsies. The LEV dose ranged from 8 to 100 mg/kg/day (mean 39 mg/kg). The study comprised 215 person years of LEV exposure. LEV was well tolerated with a retention rate of 49\% at 1 year. No serious adverse events were reported with possibly related adverse events reported in only 24\% of patients (mainly emotional or behavioural changes). At more than 2, 6 and 12 months, worthwhile improvement (>50\% seizure reduction) was noted in 60, 40 and 32\%, including seizure freedom in 14, 14 and 5\%, respectively. Our results confirm the efficacy and tolerability of LEV in children with refractory epilepsies and demonstrate good response and retention rates at 12 months. It represents the largest cohort of paediatric patients published so far on LEV with a 1-year follow-up. [\hyperlink{Retevmo}{PMID: 17258474}, D Peake et al., 2007]

\hypertarget{pmid_17006857}{L}evetiracetam (LEV) is the latest drug approved in the European Union for use in polytherapy in children over 4 years of age with partial epileptic seizures that are resistant to other antiepileptic drugs. AIM. To report our experience of associating LEV in children with medication resistant epileptic seizures. We conducted an open, observational, respective study involving 133 children with refractory epilepsies: 106 with focal seizures and 27 with other types of seizures. LEV was associated over a period of more than 6 months and we evaluated its repercussion on the frequency of the seizures and the side effects related to the drug. With average doses of LEV of 1,192 +/- 749 mg/day the frequency of the seizures was reduced by over 50\% in 58.6\% of cases and seizures were quelled in 15.8\% of patients. Side effects were produced in 27.8\% of cases, and were usually transient or tolerable; these effects led to withdrawal of LEV in only eight cases (6.02\%). In 37 children (27.8\%), their relatives noted an improvement in their social behaviour and cognitive abilities. a) LEV is an effective drug that is well tolerated in children with refractory epilepsy; b) Its effectiveness in different types of seizures indicates a broad therapeutic spectrum; and c) LEV can even condition favourable secondary effects, a circumstance that has been reported only exceptionally in the case of other antiepileptic drugs. [\hyperlink{Retevmo}{PMID: 17006857}, J L Herranz et al., ]

\hypertarget{pmid_34729069}{K}etamine has been a safe and effective sedative agent commonly used for painful pediatric procedures in the emergency department (ED). This study aimed to compare the effect of dexmedetomidine (Dex) and propofol when used as co-administration with ketamine on recovery agitation in children who underwent procedural sedation. In this prospective, randomized, and double-blind clinical trial, 93 children aged between 3 and 17 years with American Society of Anesthesiologists Class I and II undergoing short procedures in the ED were enrolled and assigned into three equal groups to receive either ketadex (Dex 0.7 μg/kg and ketamine 1 mg/kg), ketofol (propofol 0.5 mg/kg and ketamine 0.5 mg/kg), or ketamine alone (ketamine1 mg/kg) intravenously. Incidence and severity of recovery agitation were evaluated using the Richmond Agitation-Sedation Scale and compared between the groups. There was no statistically significant difference between the three groups with respect to age, gender, and weight ( The co-administering of Dex to ketamine could significantly reduce the incidence and severity of recovery agitation in children sedated in the ED. [\hyperlink{Retevmo}{PMID: 34729069}, Reza Azizkhani et al., 2021]

\hypertarget{pmid_19325512}{I}ntravenous (IV) levetiracetam (LEV) is approved for use in patients older than 16 years and may be useful in critically ill children, although there is little data available regarding pharmacokinetics. We aim to investigate the safety, an appropriate dosing, and efficacy of IV LEV in critically ill children. We describe a cohort of critically ill children who received IV LEV for status epilepticus, including refractory or nonconvulsive status, or acute repetitive seizures. There were no acute adverse effects noted. Children had temporary cessation of ongoing refractory status epilepticus, termination of ongoing nonconvulsive status epilepticus, cessation of acute repetitive seizures, or reduction in epileptiform discharges with clinical correlate. IV LEV was effective in terminating status epilepticus or acute repetitive seizures and well tolerated in critically ill children. Further study is needed to elucidate the role of IV LEV in critically ill children. [\hyperlink{Retevmo}{PMID: 19325512}, Nicholas S Abend et al., 2009]

\hypertarget{pmid_37292376}{T}he survival rates for pediatric patients with relapsed and refractory tumors are poor. Successful treatment strategies are currently lacking and there remains an unmet need for novel therapies for these patients. We report here the results of a phase 1 study of talimogene laherparepvec (T-VEC) and explore the safety of this oncolytic immunotherapy for the treatment of pediatric patients with advanced non-central nervous system tumors. T-VEC was delivered by intralesional injection at 10 Fifteen patients were enrolled into two cohorts based on age: cohort A1 ( T-VEC was tolerable as assessed by the observation of no DLTs. The safety data were consistent with the patients' underlying cancer and the known safety profile of T-VEC from studies in the adult population. No objective responses were observed. ClinicalTrials.gov: NCT02756845. https://clinicaltrials.gov/ct2/show/NCT02756845. [\hyperlink{Retevmo}{PMID: 37292376}, Lucas Moreno et al., 2023]

\hypertarget{pmid_29054532}{L}evamisole has been considered the least toxic and least expensive steroid-sparing drug for preventing relapses of steroid-sensitive idiopathic nephrotic syndrome (SSINS). However, evidence for this is limited as previous randomized clinical trials were found to have methodological limitations. Therefore, we conducted an international multicenter, placebo-controlled, double-blind, randomized clinical trial to reassess its usefulness in prevention of relapses in children with SSINS. The efficacy and safety of one year of levamisole treatment in children with SSINS and frequent relapses were evaluated. The primary analysis cohort consisted of 99 patients from 6 countries. Between 100 days and 12 months after the start of study medication, the time to relapse (primary endpoint) was significantly increased in the levamisole compared to the placebo group (hazard ratio 0.22 [95\% confidence interval 0.11-0.43]). Significantly, after 12 months of treatment, six percent of placebo patients versus 26 percent of levamisole patients were still in remission. During this period, the most frequent serious adverse event (four of 50 patients) possibly related to levamisole was asymptomatic moderate neutropenia, which was reversible spontaneously or after treatment discontinuation. Thus, in children with SSINS and frequent relapses, levamisole prolonged the time to relapse and also prevented recurrence during one year of treatment compared to prednisone alone. However, regular blood controls are necessary for safety issues. [\hyperlink{Retevmo}{PMID: 29054532}, Mariken P Gruppen et al., 2018]

\hypertarget{pmid_37762878}{R}emimazolam, an ultra-short-acting benzodiazepine sedative, was first approved in 2020 in Japan as a general anesthetic for adults. However, its utilization in pediatric settings remains unexplored and, to date, is confined to isolated case reports due to a lack of specific pediatric labeling. The primary objective of our study was to evaluate the safety profile of remimazolam when used for procedural sedation in children following dosages established in adult protocols. Additional parameters, including dosage per kg of body weight, duration of the procedure, efficacy (measured as successful completion of the procedure), the necessity for supplemental medications, and changes in physiological parameters, such as the heart rate (HR) and mean arterial blood pressure (MAP), were assessed. Our study encompassed 48 children with an average age of 7.0 years. The objective Tracking and Reporting Outcomes of Procedural Sedation tool indicated no adverse events. In our cohort, propofol and ketamine were used as adjunctive treatments in 8 and 39 patients, respectively, with successful completion of all procedures. Notable hemodynamic variability was observed, with 88.4\% of patients experiencing a ≥20\% change (increase or decrease) and 62.8\% experiencing a ≥30\% change in MAP. Additionally, a ≥20\% change in HR was observed in 54.3\% of patients, and a ≥30\% change was observed in 34.8\% of patients. Nevertheless, none of the patients required pharmacological intervention to manage these hemodynamic fluctuations. Our findings suggest that remimazolam, when supplemented with propofol or ketamine, could offer a safe and effective pathway for administering procedural sedation in pediatric populations. [\hyperlink{Retevmo}{PMID: 37762878}, Tatsuya Hirano et al., 2023]

\hypertarget{pmid_19546014}{T}he goals of this study are to evaluate the efficacy and tolerability of levetiracetam (LEV) as add-on therapy in children with refractory epilepsies and to determine the value of LEV blood level monitoring in this population. Sixty-nine children (39 males and 30 females) treated with LEV between 2006 and 2007 were selected. Their medical files were reviewed for LEV efficacy and tolerability. In a subgroup of children currently taking LEV, plasma concentrations were determined by high performance liquid chromatography by ultraviolet detection (HPLC-UV) method and correlated with the given dose per kilo as well as clinical response. Fifty-one patients (74\%) had a more than 50\% reduction in seizure frequency with 16 patients (23\%) becoming seizure free on LEV. Eighteen (26\%) patients had a less than 50\% reduction in seizure frequency. Adverse events due to LEV ranged from mild to moderate in only 18 patients (26\%). The most frequently observed were drowsiness, behavioral difficulties, increase in seizure frequency and headaches. The majority (60.5\%) of the responders received doses between 10 and 50mg/kg/day and had a plasma concentration (PC) between 5 and 40microg/ml. However, we found no clear correlation between PC and efficacy. Levetiracetam given twice a day in children with refractory epilepsy reduces seizure frequency in all types of epilepsy. In children, LEV is a broad spectrum anticonvulsant with a favourable safety profile. [\hyperlink{Retevmo}{PMID: 19546014}, Patricia C Giroux et al., 2009]

\hypertarget{pmid_19903593}{P}rocedural sedation and analgesia for children is widely practiced. Since 2005 to 2007, we evaluated the safety and efficacy of ketamine to control pain induced by diagnostic procedures in pediatric oncology patients. Eight hundred fifty procedures were carried out in 125 patients aged 2 to 16 years. We associated EMNO (inhaled equimolar mixture of nitrous oxide and oxygen), atropin (oral or rectal), midazolam (oral or rectal) and ketamin (intravenous). An anesthesiologist injected ketamin. Average dose of ketamine was 0.33 to 2 mg/kg depending on number and invasiveness of procedures. This method requires careful monitoring and proper precautions. With these conditions, no complication was observed. All patients were effectively sedated. These results indicate that ketamine - in association with EMNO, atropine and midazolam - is safe and effective in pain management induced by diagnostic procedures in pediatric oncology patients. The sedative regimen of intravenous ketamine has greatly reduced patient, family and practitioners anxiety for diagnostic and therapeutic procedures. [\hyperlink{Retevmo}{PMID: 19903593}, C Ricard et al., 2009]

\hypertarget{pmid_9862587}{T}o determine the safety of i.v. ketamine when administered by emergency physicians (EPs) for pediatric procedures, and to contrast the sedation characteristics of the i.v. and i.m. routes. The study was a retrospective consecutive case series of children aged < or =15 years given i.v. ketamine in the EDs of a university medical center and an affiliated county hospital over a 9-year period. A protocol for ketamine was used by treating physicians. Records were reviewed for adverse effects, indication, dosing, adjunctive drugs, inadequate sedation, and time to release. Results were contrasted with previously reported data for the i.m. route. During the study period i.v. ketamine was administered 156 times, primarily for laceration repair and fracture reduction. Transient apnea and respiratory depression occurred in one patient each; both were quickly identified and were without sequelae. Laryngospasm or aspiration was not noted in any children. There were 6 children with emesis and 2 with mild agitation during recovery. The median time from initial dose to ED release was 103 minutes (25th to 75th percentiles 76 to 146 minutes). The i.v. and i.m. routes were comparable in terms of adverse effects, inadequate sedation, and time to release. I.v. ketamine can be administered safely by EPs to facilitate pediatric procedures when used in a defined protocol. The sedation characteristics of the i.v. and i.m. routes appear comparable. [\hyperlink{Retevmo}{PMID: 9862587}, S M Green et al., 1998]

\hypertarget{pmid_17430408}{T}o review our experience of the efficacy and tolerability of levetiracetam (LEV) in children younger than 4 years. We used retrospective chart review to identify 122 children with seizures who were younger than 4 years and followed for >or=6 months. Efficacy was evaluated on the basis of the occurrence and durability of seizure remission. Tolerability was based on parent- and patient-reported side effects. Seventy (57\%) subjects achieved seizure remission, and 52 (43\%) did not. In univariate analysis, those achieving seizure remission were more likely to have partial epilepsy, require lower maintenance doses of LEV, and have fewer than two seizures per month at initiation of the medication. Only seizure frequency at initiation of LEV remained significant in multivariate analysis. The median duration of seizure freedom (8.9 month) was not influenced by age, epilepsy type, gender, or pretreatment seizure frequency. The dose of LEV was the only significant predictor of the duration of seizure remission, with longer duration of seizure remission seen in those taking <30 mg/kg/day compared with those taking > 30 mg/kg/day (median, 12.8 months vs. 3 months; p<0.0001). Side effects of LEV occurred in 34\% of subjects but required discontinuation in only 16\%, most commonly because of behavioral disturbances. LEV is an effective medication in children younger than 4 years and at doses lower than previously reported. It also well tolerated, suggesting that it represents an important option for the treatment of epilepsy in this age group. [\hyperlink{Retevmo}{PMID: 17430408}, M Scott Perry et al., 2007] 70 children with chronically relapsing mild-to-severe upper-respiratory-tract infections during autumn and winter participated in a six month double-blind placebo-controlled trial: 38 of them received about 1-25 mg/kg of levamisole twice daily for two consecutive days every week, the others were given placebo. During each of the three trial periods, levamisole proved superior to placebo in that it significantly reduced the number, the duration, and the severity of the infections. Moreover, in the group treated with the higher dose (i.e. greater than 2-5 mg/kg/day), the superiority of levamisole to placebo was much more clear-cut. No drug-related side-effects were reported. [\hyperlink{Retevmo}{PMID: 17430408}, M Van Eygen et al., 1976]

\hypertarget{pmid_25062293}{I}ntravenous levetiracetam (LEV) has been shown to be effective and safe in treating adults with refractory status epilepticus (SE). We sought to investigate the efficacy and safety of intravenous LEV for pediatric patients with refractory SE. We performed a retrospective medical-record review of pediatric patients who were treated with intravenous LEV for refractory SE. Clinical information regarding age, sex, seizure type, and underlying neurological status was collected. We evaluated other anticonvulsants that were used prior to administration of intravenous LEV and assessed loading dose, response to treatment, and any adverse events from intravenous LEV administration. Fourteen patients (8 boys and 6 girls) received intravenous LEV for the treatment of refractory SE. The mean age of the patients was 4.4 ± 5.5 years (range, 4 days to 14.6 years). Ten of the patients were neurologically healthy prior to the refractory SE, and the other 4 had been previously diagnosed with epilepsy. The mean loading dose of intravenous LEV was 26 ± 4.6 mg/kg (range, 20-30 mg/kg). Seizure termination occurred in 6 (43\%) of the 14 patients. In particular, 4 (57\%) of the 7 patients younger than 2 years showed seizure termination. No immediate adverse events occurred during or after infusions. The current study demonstrated that the adjunctive use of intravenous LEV was effective and well tolerated in pediatric patients with refractory SE, even in patients younger than 2 years. Intravenous LEV should be considered as an effective and safe treatment option for refractory SE in pediatric patients. [\hyperlink{Retevmo}{PMID: 25062293}, Jon Soo Kim et al., 2014]

\hypertarget{pmid_17368928}{T}he aim of this multicentric, retrospective, and uncontrolled study was to evaluate the efficacy and safety of levetiracetam (LEV) in 81 children younger than 4 years with refractory epilepsy. At an average follow-up period of 9 months, LEV administration was found to be effective in 30\% of patients (responders showing more than a 50\% decrease in seizure frequency) of whom 10 (12\%) became seizure free. This efficacy was observed for focal (46\%) as well as for generalized seizures (42\%). In addition, in a group of 48 patients, we compared the initial efficacy (evaluated at an average of 3 months of follow-up) and the retention at a mean of 12 months of LEV, with regard to loss of efficacy (defined as the return to the baseline seizure frequency). Twenty-two patients (46\%) were initial responders. After a minimum of 12 months of follow-up, 9 of 48 patients (19\%) maintained the improvement, 4 (8\%) of whom remained seizure free. A loss of efficacy was observed in 13 of the initial responders (59\%). Maintained LEV efficacy was noted in patients with focal epilepsy and West syndrome. LEV was well tolerated. Adverse events were seen in 18 (34\%) patients. The main side effects were drowsiness and nervousness. Adverse events were either tolerable or resolved in time with dosage reduction or discontinuation of the drug. We conclude that LEV is safe and effective for a wide range of epileptic seizures and epilepsy syndromes and, therefore, represents a valid therapeutic option in infants and young children affected by epilepsy. [\hyperlink{Retevmo}{PMID: 17368928}, S Grosso et al., 2007]

\hypertarget{pmid_33840407}{T}o systematically evaluate the efficacy and safety of levetiracetam (LEV) versus phenytoin (PHT) as second-line drugs for the treatment of convulsive status epilepticus (CSE) in children. English and Chinese electronic databases were searched for the randomized controlled trials comparing the efficacy and safety of LEV and PHT as second-line drugs for the treatment of childhood CSE. RevMan 5.3 software was used for data analysis. Seven studies with 1 434 children were included. The Meta analysis showed that compared with the PHT group, the LEV group achieved a significantly higher control rate of CSE ( LEV has a better clinical effect than PHT in the treatment of children with CSE and does not increase the incidence rate of adverse events. [\hyperlink{Retevmo}{PMID: 33840407}, Rui Shi et al., 2021]

\hypertarget{pmid_26189786}{T}he purpose of this study was to report on the efficacy and safety of intravenous ketamine (KE) in refractory convulsive status epilepticus (RCSE) in children and highlight its advantages with particular reference to avoiding endotracheal intubation. Since November 2009, we have used a protocol to treat RCSE including intravenous KE in all patients referred to the Neurology Unit of the Meyer Children's Hospital. From November 2009 to February 2015, 13 children (7 females; age: 2 months-11 years and 5 months) received KE. Eight patients were treated once, two were treated twice, and the remaining three were treated 3 times during different RCSE episodes, for a total of 19 treatments. Most of the RCSE episodes were generalized (14/19). A malformation of cortical development was the most frequent etiology (4/13 children). Ketamine was administered from a minimum of 22 h to a maximum of 17 days, at doses ranging from 7 to 60 mcg/kg/min, obtaining a resolution of the RCSE in 14/19 episodes. Five patients received KE in lieu of conventional anesthetics, thus, avoiding endotracheal intubation. Ketamine was effective in 4 of them. Suppression-burst pattern was observed after the initial bolus of 3mg/kg in the majority of the responder RCSE episodes (10/14). Ketamine is effective in treating RCSE and represents a practical alternative to conventional anesthetics for the treatment of RCSE. Its use avoids the pitfalls and dangers of endotracheal intubation, which is known to worsen RCSE prognosis. This article is part of a Special Issue entitled "Status Epilepticus". [\hyperlink{Retevmo}{PMID: 26189786}, Lucrezia Ilvento et al., 2015]

\hypertarget{pmid_36960520}{P}resent evidence regarding the efficacy and safety of levamisole in childhood nephrotic syndrome (NS), particularly the steroid-sensitive NS (SSNS), is limited. We searched relevant databases such as PubMed/MEDLINE, Embase, Google Scholar, and Cochrane CENTRAL till June 30, 2020. We included 12 studies for evidence synthesis (5 were clinical trials that included 326 children). The proportion of children without relapses at 6-12 months was higher in the levamisole group as compared to steroids (relative risk [RR]: 5.9 [95\% Confidence interval (CI): 0.13-264.8], I [\hyperlink{Retevmo}{PMID: 36960520}, Girish Chandra Bhatt et al., ] The purpose of this study was to evaluate the safety and efficacy of single-isomer (R)-albuterol (levalbuterol, LEV) in children aged 2-5 years. Children aged 2-5 years (n = 211) participated in this multicenter, randomized, double-blind study of 21 days of t.i.d. LEV (0.31 mg or 0.63 mg without regard to weight), racemic albuterol (RAC, 1.25 mg for children <33 pounds (lb); 2.5 mg for children >/=33 lb), or placebo (PBO). Endpoints included adverse-event (AE) reporting, safety parameters, peak expiratory flow (PEF), the Pediatric Asthma Questionnaire(c) (PAQ), and the Pediatric Asthma Caregiver's Quality of Life Questionnaire (PACQLQ). Baseline disease severity was generally mild in all groups, as defined by PAQ scores that ranged from 6.3-7.3 on a scale of 0-27 and 1.5 days/week of uncontrolled asthma. After treatment, the PAQ decreased in all groups (P = NS). In the subset of subjects able to perform PEF (51.7\%), all active treatments improved in-clinic PEF after the first dose (mean +/- SD: PBO, 1.4 +/- 20.8; LEV 0.31 mg, 12.4 +/- 12; LEV 0.63 mg, 16.7 +/- 15.4; RAC, 18.0 +/- 16.5 l/min; P < 0.01). PACQLQ measurements improved more than the minimally important difference only in the LEV-treated groups, and were significant in children <33 lb (P < 0.05). Asthma exacerbations occurred primarily in children >/=33 lb, and one serious asthma exacerbation occurred in the 2.5-mg RAC group. RAC and LEV 0.63 mg, but not LEV 0.31 mg or placebo, led to significant increases in ventricular heart rate. In this study of levalbuterol in children aged 2-5 years with asthma, LEV was generally well-tolerated, and in children able to perform PEF, led to significant bronchodilation compared with placebo. [\hyperlink{Retevmo}{PMID: 36960520}, David P Skoner et al., 2005]

\hypertarget{pmid_20416214}{T}o evaluate of the efficacy and safety of adjunctive levetiracetam (LEV) in children younger than 4 years with refractory epilepsy. One hundred and twelve children at age of 4 months to 4 years with refractory epilepsy received LEV as adjunctive therapy. LEV was administered in two equal daily doses of 10 mg/kg. The dose was increased by 10 mg/kg every week up to the target dose (20-40 mg/kg). The efficacy and tolerability were evaluated. At an average follow-up period of 13 months (6-22 months), LEV administration was found to be effective in 43 children (38.4\%) (responders showing more than a 50\% decrease in seizure frequency) and 14 children (12.5\%) became seizure-free. Fifty-three children (47.3\%) did not respond to the treatment and 2 children (1.8\%) worsened. The therapy-related adverse events were mild, including restlessness, reduction in sleep time, night terrors, debility, somnolence, nausea and vomiting. The adverse events were either tolerable or resolved in time with dosage reduction in most of children, and only 3 cases required discontinuation. LEV as adjunctive therapy is effective and well-tolerated in children younger than 4 years with refractory epilepsy, suggesting that it represents a valid option for the treatment of refractory epilepsy in this age group. [\hyperlink{Retevmo}{PMID: 20416214}, Yan Hu et al., 2010]

\hypertarget{pmid_25645999}{R}ituximab is considered to be a promising drug for treating childhood refractory nephrotic syndrome. However, the efficacy and safety of rituximab in treating childhood refractory nephrotic syndrome remain inconclusive. This meta-analysis aimed to investigate the efficacy and safety of rituximab treatment compared with other immunosuppressive agents in children with refractory nephrotic syndrome. Three randomized controlled trials and two comparative control studies were included in our analysis. The included studies were of moderately high quality. Compared with other immunotherapies, rituximab therapy significantly improved relapse-free survival (hazard ratio = 0.49, 95\% confidence interval [CI], 0.26-0.92, P = 0.03). Rituximab also achieved a higher rate of complete remission (risk ratio,1.62; 95\% CI, 0.92 to 2.84, P = 0.09) and reduced the occurrence of proteinuria (mean difference = -0.25, 95\% CI = -0.29 to -0.21, P < 0.00001); however, a more targeted rituximab treatment did not significantly increase serum albumin levels and did not significantly reduce adverse events. Rituximab might be a promising treatment for childhood refractory nephrotic syndrome; however, the long-term effects and cost-effectiveness of rituximab treatment were not fully assessed, and there were limited studies that evaluated the clinical benefits of a concurrent infusion of rituximab plus a steroid compared with an infusion of rituximab only. Additional studies are required to address these issues.  [\hyperlink{Retevmo}{PMID: 25645999}, Zhihong Zhao et al., 2015] To assess the efficacy, tolerability and safety of Levetiracetam (LEV) therapy, we identified 21 (15 male; 6 female) patients with a history of benign epilepsy with centrotemporal spikes (BECTS), with and without secondarily generalization in children and adolescents aged between 5.0 and 12.1 years. LEV was administered as a first drug (number of patients=9) or converted after previous treatment with other AEDs (number of patients=12). The patients were subdivided into two groups: "newly diagnosed" patients and "converted" patients. Patients were followed up for 12 months and all patients were able to continue on LEV treatment. At the end of follow-up (12 months), all patients were seizure free or showed a reduction of seizures >50\%. LEV dosage ranged from 1000 to 2500mg/daily. Overall, 100\% of patients completed the 12 months study, without any important side effect. Somnolence and irritability occurred in two (9.5\%) patients. Our results support findings that LEV monotherapy is effective and well tolerated in children with BECTS. Prospective, large, long-term double-blind studies are needed to confirm these findings. [\hyperlink{Retevmo}{PMID: 25645999}, A Verrotti et al., 2007]

\section*{Selegiline Hydrochloride}
\subsection*{Result}
\subsubsection*{Answer}

Unknown

\subsubsection*{{Explanation}}
\hypertarget{Selegiline Hydrochloride}
A review of the available abstracts reveals the following information regarding the safety of Selegiline Hydrochloride in children:

1. \hyperlink{pmid_12921918}{PMID: 12921918}, Shahin Akhondzadeh et al., 2003: This is a double-blind, controlled clinical trial specifically evaluating the efficacy and tolerability of selegiline in children with ADHD. The study included 28 children, with dosing adjusted for age (5 mg/day for under 5 years, 10 mg/day for over 5 years) over a 4-week period. The study reports that selegiline may be beneficial in the treatment of ADHD and that a tolerable side effect profile may be considered as one of the advantages of selegiline in the treatment of ADHD. However, the authors also state that the results must be considered preliminary.

2. No other abstracts provide targeted safety data for Selegiline Hydrochloride in children. The remaining abstracts discussing selegiline focus on adult populations (primarily Parkinson's disease) or do not specify pediatric use.

Summary by age range:
- Children under 5 years: The cited study included children under 5 years, but the sample size was small, the duration was short (4 weeks), and the authors themselves describe the results as preliminary.
- Children over 5 years: The same study included children over 5 years, with similar limitations.

Conclusion:
There is one small, short-term, controlled study suggesting selegiline may be tolerated in children with ADHD, but the evidence is preliminary and insufficient to definitively affirm safety. No studies were found that demonstrate selegiline is unsafe in children. Therefore, based on the abstracts, the safety of Selegiline Hydrochloride in children is unknown.

\subsection*{Abstracts}
\hypertarget{pmid_28741653}{C}hloral hydrate is commonly used to sedate children for painless procedures. Children may recover more quickly after sedation with dexmedetomidine, which has a shorter half-life. We randomly allocated 196 children to chloral hydrate syrup 50 mg.kg [\hyperlink{Selegiline Hydrochloride}{PMID: 28741653}, V M Yuen et al., 2017] Chloral hydrate has been used extensively to sedate children, but at Brooke Army Medical Center, other drug combinations were becoming increasingly popular due to a perception that chloral hydrate had a high rate of failure, especially with younger or neurologically impaired children. Therefore, 50 children were given the drug before a diagnostic study, and patient data and a sedation score were recorded on a worksheet. Of 50 children, 43 (86\%) were "successfully sedated" on the first attempt with no side effects. Children with neurologic disorders had a much greater (27\% vs 4\%) failure rate than "normal" children. The sedation rate did not significantly differ by age, sex, or initial drug dosage. The study suggest that chloral hydrate is a safe and effective oral sedative but that children with neurologic disorders may need alternative drugs for sedation. [\hyperlink{Selegiline Hydrochloride}{PMID: 28741653}, P D Rumm et al., 1990]

\hypertarget{pmid_15951862}{D}iagnostic and therapeutic procedures in children are made easier using sedation. However, there is no consensus about which drug should be used to achieve this. Furthermore, none of the drugs used for sedation are risk free. The aim of this work is to study sedation indications, effectiveness, and safety at our center. A prospective observational study conducted at the Pediatric Day Care Unit, King Fahad National Guard Hospital, Riyadh, Saudi Arabia. The study covered 17.5 weeks in 2 periods: May 9th 1999 to June 13th 1999 and October 31st 2001 to February 11th 2002. Children <12 years were included. Collected data included demographics, indication, drug dosing and outcome. Data were reported as mean +/- SD. We included 148 patients, age 38 +/- 30 months. Adequate sedation was achieved in 79\% after initial chloral hydrate (CH) dose of 56.9 +/- 9.3 mg/kg, in 95\% after adding 18.5 +/- 6.4 mg/kg CH and in 96\% after adding second drug. Compared to nonrespondents, first CH dose respondents were younger and lower in weight. The CH side effects were few and mild. Chloral hydrate is a safe and effective agent for sedation in children with an age and weight dependent response. [\hyperlink{Selegiline Hydrochloride}{PMID: 15951862}, Omar M Hijazi et al., 2005]

\hypertarget{pmid_2515719}{S}elegiline hydrochloride (deprenyl) is a safe, useful adjuvant therapy in patients with Parkinson's disease treated with L-dopa. The optimum time for its introduction into the treatment regimen of a patient remains controversial. A multicentre long-term study being conducted by the Parkinson's Disease Research Group of the United Kingdom to attempt to answer whether selegiline improves the natural history of Parkinson's disease is discussed. In a separate study we have been unable to demonstrate that higher doses of selegiline (up to 40 mg a day) produce additional therapeutic benefit above the conventional dose of 10 mg a day in levodopa-treated patients with motor fluctuations. Preliminary data from a neuropsychological study is also presented which suggests that selegiline may have beneficial effects on the speed of psychomotor responses supporting the anecdotal clinical observations of increased mental energy and alacrity. [\hyperlink{Selegiline Hydrochloride}{PMID: 2515719}, A J Lees et al., 1989]

\hypertarget{pmid_28275979}{S}edation is often required for children undergoing diagnostic procedures. Chloral hydrate has been one of the sedative drugs most used in children over the last 3 decades, with supporting evidence for its efficacy and safety. Recently, chloral hydrate was banned in Italy and France, in consideration of evidence of its carcinogenicity and genotoxicity. Dexmedetomidine is a sedative with unique properties that has been increasingly used for procedural sedation in children. Several studies demonstrated its efficacy and safety for sedation in non-painful diagnostic procedures. Dexmedetomidine's impact on respiratory drive and airway patency and tone is much less when compared to the majority of other sedative agents. Administration via the intranasal route allows satisfactory procedural success rates. Studies that specifically compared intranasal dexmedetomidine and chloral hydrate for children undergoing non-painful procedures showed that dexmedetomidine was as effective as and safer than chloral hydrate. For these reasons, we suggest that intranasal dexmedetomidine could be a suitable alternative to chloral hydrate. [\hyperlink{Selegiline Hydrochloride}{PMID: 28275979}, Giorgio Cozzi et al., 2017]

\hypertarget{pmid_12921918}{A}ttention deficit hyperactivity disorder (ADHD) is a common disorder of childhood that affects 3\% to 6\% of school-age children. Conventional stimulant medications are recognized by both specialists and parents as useful symptomatic treatment. Nevertheless, approximately 30\% of ADHD children treated with them do not respond adequately or cannot tolerate the associated adverse effects. Such difficulties highlight the need for alternative safe and effective medications in the treatment of this disorder. Selegiline is a type B monoamine oxidase inhibitor (MAOI) that is metabolized to amphetamine and methamphetamine stimulant compounds that may be useful in the treatment of ADHD. The authors undertook this study to further evaluate, under double-blind and controlled conditions, the efficacy of selegiline for ADHD in children. A total of 28 children with ADHD as defined by DSM IV were randomized to selegiline or methylphenidate dosed on an age and weight-adjusted basis at selegiline 5 mg/day (under 5 years) and 10 mg/day (over 5 years) (Group 1) and methylphenidate 1 mg/kg/day (Group 2) for a 4-week double-blind clinical trial. The principal measure of the outcome was the Teacher and Parent ADHD Rating Scale. Patients were assessed by a child psychiatrist at baseline, 14 and 28 days after the medication started. No significant differences were observed between the two protocols on the Parent and Teacher Rating Scale scores. Although the number of dropouts in the methylphenidate group was higher than in the selegiline group, there was no significant difference between the two protocols in terms of the dropouts. Decreased appetite, difficulty falling asleep and headaches were observed more in the methylphenidate group. The results of this study must be considered preliminary, but they do suggest that selegiline may be beneficial in the treatment of ADHD. In addition, a tolerable side effect profile may be considered as one of the advantages of selegiline in the treatment of ADHD. [\hyperlink{Selegiline Hydrochloride}{PMID: 12921918}, Shahin Akhondzadeh et al., 2003]

\hypertarget{pmid_24627951}{T}o determine the safety and efficacy of high-dose oral chloral hydrate for pediatric ophthalmic procedures. This study is a retrospective review of a quality audit of pediatric sedation for ophthalmic evaluation and imaging performed at King Khaled Eye Specialist Hospital between January 1 and December 31, 2011, in children aged 1 month to 6 years. Three hundred fifty-eight of 380 (94.2\%) sedation procedures were successful after a single dose of chloral hydrate, with 356 of 380 (93.7\%) children sedated within 45 minutes of the first dose. The total success rate of the sedation procedure increased to 97.9\% (372 of 380) when a second dose was administered. Children adequately sedated after a single dose of chloral hydrate were on average younger and weighed less than children who required additional doses. No major adverse events were documented. The use of chloral hydrate sedation for ophthalmic evaluation and imaging was safe and effective in this patient population with a high rate of procedure completion. [\hyperlink{Selegiline Hydrochloride}{PMID: 24627951}, Michelle E Wilson et al., ]

\hypertarget{pmid_23129068}{H}ydroxyurea (HU) is highly effective treatment for sickle cell disease (SCD). While pediatric use of HU is accepted clinical practice, barriers to use may impede its potential benefit. A survey of parents of children ages 5-17 years with SCD was performed across five institutions to assess factors associated with HU use. Of the 173 parent responses, 65 (38\%) had children currently taking HU. Among parents of children not taking HU, the most commonly cited reasons were that their hematology provider had not offered it, their child was not sufficiently symptomatic and concerns about potential side effects. Even parents of HU users reported widespread concern about effectiveness, long-term safety, and off-label use. In bivariate analyses, children's ages, parental demographics such as education level, or travel time to their hematology provider were not correlated with HU use. Bivariate analysis and multivariate logistic regression revealed three significant factors associated with current HU use: better parental knowledge about its major therapeutic effects (P < 0.001), sickle genotype (P = 0.005), and institution of clinical care (P = 0.04). Pervasive concerns about HU safety exist, even among parents of current users. Varying knowledge among parents appears to be independent of their demographics, and is associated with HU use. Inter-institutional variability in parental knowledge and drug uptake highlights potentially potent site-specific influences on likelihood of HU use. Overall, these survey data underscore the need for strategies to bolster parental understanding about benefits of HU and address concerns about its safety. [\hyperlink{Selegiline Hydrochloride}{PMID: 23129068}, Suzette O Oyeku et al., 2013]

\hypertarget{pmid_2026812}{C}hloral hydrate is commonly used to sedate children before CT. However, no prospective study has been published of the safety and efficacy of chloral hydrate at high dose levels for children undergoing CT. We define high dose levels of oral chloral hydrate to be 80-100 mg/kg, with a maximum total dose of 2 g. High dose chloral hydrate sedation was administered orally to 295 children for 326 CT examinations. Adverse reactions occurred in 7\% of the children, with vomiting being the most common (4.3\% of children). Hyperactivity and respiratory symptoms each occurred in less than 2\% of children. Prolonged sedation ( greater than 2 h) was not encountered in our series. Sedation was successful in producing motion free CT examinations, so that in 303 (93\%) of the cases, no repeat CT scans were needed. We conclude that high dose oral chloral hydrate provides safe and effective sedation for children undergoing CT. [\hyperlink{Selegiline Hydrochloride}{PMID: 2026812}, S B Greenberg et al., ]

\hypertarget{pmid_10778622}{S}elegiline hydrochloride, a selective MAO-B inhibitor is known to improve motor functions in Parkinson's disease (PD). The present study was undertaken to study the effect of selegiline on memory and intelligence of PD patients. Thirty two patients of PD were divided in two groups: selegiline group (n = 17) received 10 mg selegiline per day and control group (n = 15) did not receive selegiline. Patients receiving trihexyphenidyl and selegiline were excluded. All other treatment remained unchanged. All patients were examined at baseline and after three months for change in UPDRS score, WAIS score, memory test and P300. Patients in selegiline group had less severe disease (UPDRS score 24.11 +/- 14.07) as compared to controls (UPDRS score 40.53 +/- 18.52). There was significant improvement in UPDRS score (p < 0.05), WAIS (p < 0.001) and memory (p < 0.001) in selegiline group. In the control group there was a significant prolongation of P300 latency (p < 0.05). The study suggests that selegiline improves memory functions and intelligence in PD patients in addition to motor functions. It also prevents prolongation of P300 latency which is a marker of cognitive function. [\hyperlink{Selegiline Hydrochloride}{PMID: 10778622}, S N Dixit et al., 1999]

\hypertarget{pmid_15604217}{H}ydroxyurea (HU) is considered to be the most successful drug therapy for severe sickle cell disease (SCD). Nevertheless, questions remain regarding its benefits in very young children and its role in the prevention of cerebrovascular events. There were 127 SCD patients treated with no attempt to reach maximal tolerated doses who entered the Belgian Registry: 109 for standard criteria and 18 who were at risk of stroke only. During 426 patient-years of follow-up for patients with standard criteria, 3.3 acute chest syndromes, 1.3 cerebrovascular events, and 1.1 osteonecrosis per 100 patient-years were observed. A subgroup of 32 patients followed for 6 years experienced significant benefit over this period. In each subgroup of children (younger than 2 years, 2-5, 6-9, and 10-19 years) followed for 2 years, clinical and biologic changes were similar, except for children younger than 2 years who had no total hemoglobin increase and remained at risk of severe anemia. In 72 patients evaluated by transcranial Doppler studies (TCD), 34 patients were at risk of primary stroke and only 1 had a cerebrovascular event after a follow-up of 96 patient-years. These results confirm the benefit of HU, even in very young children, and its possible role in primary stroke prevention. [\hyperlink{Selegiline Hydrochloride}{PMID: 15604217}, Béatrice Gulbis et al., 2005]

\hypertarget{pmid_9060867}{C}hloral hydrate (CH) is used to sedate children unable to cooperate during investigations such as EEG requiring the patient to be still. It is not known if CH or its metabolites modify the EEG and our aim was to answer this question. Recordings of the EEG before, during and after rectal administration of CH (50-77 mg/kg) in 13 children aged 1.5-13.5 years with severe epilepsy and additional neurological impairments were made. All children had frequent spike-wave activity before CH. In 9 children CH had no effect on the EEG. In 3 children there was a significant reduction in epileptic activity after 20-50 min and in one a significant increase. Cardiovascular parameters were stable throughout. At sedative doses, CH can generally be used before an EEG recording without loss of information but in 4 out of 13 children there were changes which could alter interpretation. [\hyperlink{Selegiline Hydrochloride}{PMID: 9060867}, M Thoresen et al., 1997]

\hypertarget{pmid_3218706}{A} female infant with seizures refractory to conventional therapeutic agents was presented. Mexiletine hydrochloride, administered orally, was effective in controlling her seizures. Her sleep structure and psychomotor development seemed to improve after reduction of the fits. [\hyperlink{Selegiline Hydrochloride}{PMID: 3218706}, J Kohyama et al., 1988]

\hypertarget{pmid_21531030}{C}hloral hydrate (CH) is an oral sedative widely used to sedate infants and young children during auditory brainstem response (ABR) testing. The aim of this study was to record effectiveness, complications and safety of CH as a sedative for ABR. From January of 2003 until December of 2007, 1903 children were tested for ABR, 568 of them being under the age of 6 months. CH (8\%) was used for sedation at a dose of 40 mg/kg with a repeat dose, if necessary, for an adequate sedation, in 20-30 min. We recorded the effectiveness of CH as a sedative for ABR examination, as well as all complications related to the use of CH such as vomiting, rash, hyperactivity, respiratory distress and apnea. The statistical method used was the absolute and percentage frequency distribution of the occurrences. Sedation with CH was necessary to perform testing in 1591 (83.6\%) of the examined children. However, in the population of the examined infants, only 341 (60\%) were sedated with CH, because the remaining 227 (40\%) fell asleep by themselves. Complications included hyperactivity in 152 children (8\%), minor respiratory distress in 10 children (0.4\%), vomiting in 217 children (11.4\%), apnea in 4 children (0.2\%) and rash in 10 children (0.4\%). The complications of hyperactivity, vomiting and rash resolved without any medical treatment. The apnea cases were managed effectively by supplying ventilation to the children via a mask in the presence of an anesthesiologist. The use of CH at a dose of 40 mg/kg up to 80 mg/kg is safe and effective when administered in a setting with adequate equipment and the presence of well trained personnel. [\hyperlink{Selegiline Hydrochloride}{PMID: 21531030}, Eirini Avlonitou et al., 2011]

\hypertarget{pmid_33706380}{H}ydroxyurea (HU) is used in children with sickle cell disease (SCD) to increase fetal hemoglobin (HF), contributing to a decrease in physical symptoms and potential protection against cerebral microvasculopathy. There has been minimal investigation into the association between HU use and cognition in this population. This study examined the relationship between HU status and cognition in children with SCD. Thirty-seven children with SCD HbSS or HbS/β0 thalassaemia (sickle cell anemia; SCA) ages 4:0-11 years with no history of overt stroke or chronic transfusion completed a neuropsychological test battery. Other medical, laboratory, and demographic data were obtained. Neuropsychological function across 3 domains (verbal, nonverbal, and attention/executive) was compared for children on HU (n = 9) to those not taking HU (n = 28). Children on HU performed significantly better than children not taking HU on standardized measures of attention/executive functioning and nonverbal skills. Performance on verbal measures was similar between groups. These results suggest that treatment with HU may not only reduce physical symptoms, but may also provide potential benefit to cognition in children with SCA, particularly in regard to attention/executive functioning and nonverbal skills. Replication with larger samples and longitudinal studies are warranted. [\hyperlink{Selegiline Hydrochloride}{PMID: 33706380}, Reem A Tarazi et al., 2021]

\hypertarget{pmid_9317198}{C}hildren with sickle cell anemia provide the best opportunity to assess the efficacy of hydroxyurea (HU) in preventing complications and progressive organ damage. The possibility of treating infants with sickle cell disease (SCD) to inhibit the development of organ dysfunction may be the most important future use of HU. The possibility even exists that instituting HU in the neonate may stop the fetal-to-adult globin chain switch and thus markedly change the clinical phenotype of SCD. Recent data suggest HU may also be especially beneficial in children not only by increasing hemoglobin F (HbF), but also by altering the adhesive receptors expressed on red blood cells and vascular endothelium, further increasing the possibility that vasculopathy can be prevented. Six pediatric trials that included small numbers of severely ill patients have been reported recently. All patients received relatively standard HU doses. All studies reported a significant improvement in HbF and mean corpuscular volume and a mild to marked increase in hemoglobin. The clinical response to HU in children with SCD seems to be consistent. The National Institutes of Health pediatric multicenter trial should help answer the question of short-term HU toxicity; however, questions remain concerning long-term risks, such as carcinogenesis, gametogenesis, marrow toxicity, growth retardation, and chromosomal damage. Long-term studies are needed to answer these questions. The future treatment of most children with SCD with HU alone or in combination with other agents looks promising, and long-term trials are warranted. [\hyperlink{Selegiline Hydrochloride}{PMID: 9317198}, E P Vichinsky et al., 1997]

\hypertarget{pmid_31369972}{C}hloral hydrate is a sedative that has been used for many years in clinical practice and, under proper conditions, gives a deep and long enough sleep to allow performance of objective hearing tests in young children. The reluctance to use this substance stems from side effects reported over time that can vary, depending on dose, procedure settings and immediate life supporting intervention when needed. Our study adds to those that have appeared in recent years, showing that chloral hydrate is an effective and safe substance when is used in proper conditions. The study included 322 children who needed sedation for objective hearing tests, from April 2014 to March 2018. Parents were instructed to bring the child tired and fasted for at least 2 h before sedation. The sedative was administered by trained staff in the hospital, and the child was monitored until awaking. In our study group, over half of the children were in the age 1-4 years group, and only 15\% were older than 4 years. The dose of chloral hydrate ranged between 50 and 83 mg/kg body weight, with an average of 75 mg. Successful sedation occurred in 94.1\% of children; 0.9\% of children awoke during testing and required supplemental sedation or rescheduling of the testing. The most common side effects were vomiting, agitation, prolonged sleep, and failure to fall asleep. Comparing the side effects of chloral hydrate in our study with those from other studies, ours were similar to those described in the literature. In our study chloral hydrate was effective and had only limited adverse effects. The use of chloral hydrate under hospital conditions with proper monitoring could be a practical and safe solution for outpatients or those with short-term hospitalisation. [\hyperlink{Selegiline Hydrochloride}{PMID: 31369972}, Violeta Necula et al., 2019]

\hypertarget{pmid_31534313}{C}hloral hydrate (CH), as a sedation agent, is widely used in children for diagnostic or therapeutic procedures. However, it has not come into the market and is currently only used as hospital preparation in China. This review aims to systematically evaluate the efficacy of CH in children of all age groups for sedation before medical procedures. Seven electronic databases and three clinical trial registry platforms were searched and the deadline was September 2018. Randomized controlled trials (RCTs) evaluating the efficacy of CH for sedation in children were included by two reviewers. The extracted information included success rate of sedation, sedation latency and sedation duration. The Cochrane risk of bias tool was applied to assess the risk of bias. The outcomes were analyzed by Review Manager 5.3 software and expressed as relative risks (RR) or Mean Difference (MD) with 95\% confidence interval (CI). Heterogeneity was assessed with I-squared (I A total of 24 RCTs involving 3564 children of CH for sedation were included in the meta-analysis. Compared to placebo group, CH group had a significant increase in success rate of sedation when used for painless and painful procedure (RR=4.15, 95\% CI [1.21, 14.24], P=0.02; RR=1.28, 95\% CI [1.17, 1.40], P<0.01), which included 22 and 455 children for this analysis, respectively. Compared to midazolam group, CH group had a significant increase in success rate of sedation (RR=1.63, 95\% CI [1.48, 1.79], I From the extrapolation of the existing literature, CH oral solution is an appropriate effective alternative for sedation in pediatrics. [\hyperlink{Selegiline Hydrochloride}{PMID: 31534313}, Zhe Chen et al., 2019]

\hypertarget{pmid_25246305}{T}he aim of this study was to compare the efficacy and safety of different oral chloral hydrate and dexmedetomidine doses used for sedation during electroencephalography (EEG) in children. One hundred sixty children aged 1 to 9 years with American Society of Anesthesiologists physical status I-II who were uncooperative during EEG recording or who were referred to our electrodiagnostic unit for sleep EEG were included to the study. The patients were randomly assigned into 4 groups. In groups D1 and D2, patients received oral dexmedetomidine doses of 2 and 3 µg/kg, respectively. In group C1 and C2, patients received oral chloral hydrate doses of 50 and 100 mg/kg, respectively. The induction time was significantly shorter in group C2 compared with other groups (P = .000). The rate of adverse effects was significantly higher in group C2 compared with the dexmedetomidine groups (D1 and D2; P = .004). In conclusion, dexmedetomidine can be used safely for sedation during EEG in children.  [\hyperlink{Selegiline Hydrochloride}{PMID: 25246305}, Hakan Gumus et al., 2015] Hydroxyurea (HU) increases fetal hemoglobin (HgbF) and ameliorates sickle cell disease (SCD) symptoms. Studies have demonstrated the safety and efficacy of HU in infants and children. Initiation of HU in infancy for children with SCD needs to be implemented in community practice. Starting in 2011, the Pediatric Sickle Cell Program of Northern Virginia initiated HU in infants with SCD. A prospective longitudinal database tracked the clinical course and outcomes. Twenty-four children with HgbSS who started HU by age 1 were continuously followed for a total of 95 person-years. Age at the time of analysis ranged from 2 to 7 years. Average hemoglobin at 6-month intervals ranged from 9.5 + 1.9 to 10.7 + 0.8 g/dL, and average HgbF ranged from 27.8 + 5.0\% to 34.1 + 6.6\%. Twenty-seven hospitalizations occurred (0.28/person-year), all before age 3, including 19 (70\%) for fever or infection, five (19\%) for splenic sequestration, and one (4\%) for pain in an infant prior to starting HU. The treat-and-release emergency department visits totaled 68 (0.72/person-year), including 62 visits (91\%) for fever, infection, or viral illness, and two visits (3\%) for pain/dactylitis in infants before HU initiation. Splenic sequestration accounted for all five transfusions. No pain episodes requiring medical attention were documented after HU initiation. No complicated acute chest syndrome, no abnormal or conditional transcranial Doppler ultrasound, and no overt strokes occurred. Implementation of HU in infancy for patients with SCD in community practice is feasible and is highly effective in preventing disease complications. [\hyperlink{Selegiline Hydrochloride}{PMID: 25246305}, Ronay Thomas et al., 2019]

\hypertarget{pmid_22246409}{C}hloral hydrate (CH) is safe and effective for sedation of suitable children. The purpose of this study was to assess whether adequate sedation is achieved with reduced CH doses. We retrospectively recorded outpatient CH sedations over 1 year. We defined standard doses of CH as 50 mg/kg (infants) and 75 mg/kg (children >1 year). A reduced dose was defined as at least 20\% lower than the standard dose. In total, 653 children received CH sedation (age, 1 month-3 years 10 months), 42\% were given a reduced initial dose. Augmentation dose was required in 10.9\% of all children, and in a higher proportion of children >1 year (15.7\%) compared to infants (5.7\%; P < 0.001). Sedation was successful in 96.7\%, and more frequently successful in infants (98.3\%) than children >1 year (95.3\%; P = 0.03). A reduced initial dose had no negative effect on outcome (P = 0.19) or time to sedation. No significant complications were seen. We advocate sedation with reduced CH doses (40 mg/kg for infants; 60 mg/kg for children >1 year of age) for outpatient imaging procedures when the child is judged to be quiet or sleepy on arrival. [\hyperlink{Selegiline Hydrochloride}{PMID: 22246409}, Jennifer Bracken et al., 2012]

\hypertarget{pmid_2672786}{T}his study assessed the safety and efficacy of methylphenidate in children with seizures and attention-deficit disorder. Ten children, aged 6 years 10 months to 10 years 10 months, without seizures while receiving a single antiepileptic drug, were evaluated in a double-blind medication-placebo crossover study with methylphenidate hydrochloride was administered at 0.3 mg/kg per dose and given at 8 AM and 12 PM on school days only. The use of methylphenidate was associated with statistically significant improvements on the Conners' Teacher Rating Scale and on the Finger Tapping Task and with trends toward improvement on the Matching Familiar Figures Test and Discriminant Reaction Time tests. No child had seizures during the study period nor subsequently for those who continued receiving psychostimulants. There were no significant changes of epileptiform features or back-ground activity on electroencephalograms and no alterations in antiepileptic drug levels. Methylphenidate may be a safe and effective treatment for certain children with seizures and concurrent attention-deficit disorder. [\hyperlink{Selegiline Hydrochloride}{PMID: 2672786}, H Feldman et al., 1989]

\hypertarget{pmid_19047254}{H}ydroxyurea is the only approved medication for the treatment of sickle cell disease in adults; there are no approved drugs for children. Our goal was to synthesize the published literature on the efficacy, effectiveness, and toxicity of hydroxyurea in children with sickle cell disease. Medline, Embase, TOXLine, and the Cumulative Index to Nursing and Allied Health Literature through June 2007 were used as data sources. We selected randomized trials, observational studies, and case reports (English language only) that evaluated the efficacy and toxicity of hydroxyurea in children with sickle cell disease. Two reviewers abstracted data sequentially on study design, patient characteristics, and outcomes and assessed study quality independently. We included 26 articles describing 1 randomized, controlled trial, 22 observational studies (11 with overlapping participants), and 3 case reports. Almost all study participants had sickle cell anemia. Fetal hemoglobin levels increased from 5\%-10\% to 15\%-20\% on hydroxyurea. Hemoglobin concentration increased modestly (approximately 1 g/L) but significantly across studies. The rate of hospitalization decreased in the single randomized, controlled trial and 5 observational studies by 56\% to 87\%, whereas the frequency of pain crisis decreased in 3 of 4 pediatric studies. New and recurrent neurologic events were decreased in 3 observational studies of hydroxyurea compared with historical controls. Common adverse events were reversible mild-to-moderate neutropenia, mild thrombocytopenia, severe anemia, rash or nail changes (10\%), and headache (5\%). Severe adverse events were rare and not clearly attributable to hydroxyurea. Hydroxyurea reduces hospitalization and increases total and fetal hemoglobin levels in children with severe sickle cell anemia. There was inadequate evidence to assess the efficacy of hydroxyurea in other groups. The small number of children in long-term studies limits conclusions about late toxicities. [\hyperlink{Selegiline Hydrochloride}{PMID: 19047254}, John J Strouse et al., 2008]

\hypertarget{pmid_18540545}{I}n the absence of a general anaesthetic facility for MRI scanning in children, we introduced a sedation protocol using chloral hydrate. Our aim was to evaluate the success and safety of our protocol. This was a retrospective study enrolling 36 children over a 7 month period. The overall success rate was 86\% with no child experiencing respiratory complications. In those less than one year, the success rate was 100\%, aged 1-5 years 91\%, with 50\% successful at 80 mg/kg and 50\% at 100 mg/kg dose. For children greater than 5 years of age the success rate was 70\%. 92\% of developmentally normal children and 83\% of developmentally delayed children were successfully sedated. Success rates were poorer in children older than 5 years and in those with developmental delay. Our findings suggest that this protocol could be safely used in units where general anaesthetic facilities are unavailable for MRI and for other radiological investigations. [\hyperlink{Selegiline Hydrochloride}{PMID: 18540545}, E Low et al., 2008]

\hypertarget{pmid_942230}{K}etamine hydrochloride 2 mg/kg, together with atropine 0.2 mg, has been given intravenously on 100 occasions on a general paediatric ward. No serious side effects occurred. Dreams followed in 4 children but did not reduce acceptability of the drug. In our hands it has greatly reduced the pain and distress of children undergoing many routine medical procedures, particularly the dread which builds up when these have to be repeated in the same child. It has also produced close to ideal conditions for the operator, and probably increased his efficiency by reducing the emotional strain which occurs when doing painful things to a frightened patient. [\hyperlink{Selegiline Hydrochloride}{PMID: 942230}, E Elliott et al., 1976]

\section*{Tolmetin Sodium}
\subsection*{Result}
\subsubsection*{Answer}

Unknown

\subsubsection*{{Explanation}}
\hypertarget{Tolmetin Sodium}
A review of the available abstracts reveals the following:

- None of the abstracts provided report targeted studies evaluating the safety of Tolmetin Sodium specifically in children. 
- The studies that do address safety and efficacy of Tolmetin Sodium are conducted in adults, geriatric populations, or animal models (e.g., rabbits, rats, dogs), but not in pediatric human populations [\hyperlink{pmid_6350377}{PMID: 6350377}, W M O'Brien et al., 1983; \hyperlink{pmid_6886030}{PMID: 6886030}, G E Ehrlich et al., 1983; \hyperlink{pmid_61224}{PMID: 61224}, L J Cordrey et al., 1976; \hyperlink{pmid_7893639}{PMID: 7893639}, D M Wiseman et al.; \hyperlink{pmid_535831}{PMID: 535831}, H Nakamura et al., 1979; \hyperlink{pmid_7248666}{PMID: 7248666}, H Nakamura et al., 1981].
- There are no abstracts describing clinical trials, observational studies, or case series that specifically assess the safety of Tolmetin Sodium in children of any age group.
- Therefore, based on the abstracts available, the safety of Tolmetin Sodium for use in children is unknown, as no targeted pediatric safety studies are reported.

\subsection*{Abstracts}
\hypertarget{pmid_7893639}{T}olmetin is a nonsteroidal anti-inflammatory drug (NSAID) that reduces adhesion formation in several animal models after a single intraperitoneal (i.p.) dose delivered at the time of surgery. We set out to determine the period during which tolmetin could prevent adhesions. Adhesions were induced in New Zealand White rabbits (2-3 kg) by abrading the uterine horns and removing their mesouterine vasculature. Tolmetin sodium (1 mg/5 ml saline) was given at various times relative to the start of surgery as a single dose i.p. One week later adhesions were assessed using a standard scoring system (0 = no adhesions; 1 = light adhesions involving both uterine horns; 2 = more tenacious adhesions to bowel or bladder; 3 = tenacious adhesions to bowel and bladder partly immobilizing the uterus; 4 = completely fixed horns adherent to bowel and bladder). Scores were arranged in ascending rank order. Mean rank positions were calculated for each group and compared against controls (Dunnett's multiple comparison). Tolmetin sodium was most effective when administered within 1 hour of surgery. Mild effects could still be observed after 4 hours and the effect diminished after 24 hours. When these effects are compared to the temporal biochemical and cellular effects of tolmetin obtained in related studies, the data support the hypothesis that tolmetin reduces adhesions at least in part by modulating fibrinolytic activity of resident macrophages and macrophages present in the early postsurgical period. [\hyperlink{Tolmetin Sodium}{PMID: 7893639}, D M Wiseman et al., ]

\hypertarget{pmid_6350377}{I}n order to evaluate the effectiveness and safety of tolmetin sodium in the treatment of both rheumatoid arthritis (RA) and osteoarthritis in geriatric patients, a retrospective study was made of patients 65 years and older who participated in long-term, controlled, double-blind and open trials during both the investigational period and since marketing of the drug. Standard entrance criteria, methods of evaluating disease activity, and statistical methods were used in the study of both arthritic diseases. A total of 847 geriatric patients were studied for periods of up to one year; 171 had RA, while 676 had osteoarthritis of large or small joints. Average daily dose of tolmetin sodium was 1141 mg for patients with RA and 953 mg for patients with osteoarthritis. The results of this retrospective study of both RA and osteoarthritis patients show that tolmetin was as effective in geriatric patients as in nongeriatric patients. Symptoms responded rapidly to treatment with tolmetin, and both the inflammatory symptoms of RA and the joint pain and functional parameters of osteoarthritis showed improvement that was both statistically and clinically significant throughout the major course of therapy. Tolmetin was also found to be safe and well tolerated by the elderly patient population. The major complaints were gastrointestinal, but serious or limiting side effects occurred in few patients. The dropout rates due to adverse effects during the entire year of therapy were 15.8 per cent in the RA population and 15.4 per cent in osteoarthritis patients. This retrospective evaluation of tolmetin therapy shows significant relief of the symptoms of both RA and osteoarthritis in a geriatric population and fails to reveal any unusual or serious conditions which would contraindicate its use in the elderly patient. Tolmetin, which is an antiinflammatory agent with a short half-life, can provide adequate, safe therapy in the geriatric population. [\hyperlink{Tolmetin Sodium}{PMID: 6350377}, W M O'Brien et al., 1983]

\hypertarget{pmid_11176516}{T}olterodine was recently approved for the treatment of incontinence and overactive bladder in adults, and has fewer side effects than oxybutynin. We evaluated the safety and efficacy of tolterodine in children with dysfunctional voiding. We retrospectively reviewed our experience with 30 pediatric patients treated with tolterodine for a primary diagnosis of dysfunctional voiding. Patients were treated with adult doses of tolterodine and behavioral modifications. Standard definitions determined by the International Children's Continence Society were adapted to designate final treatment outcomes on medication as cured-greater than 90\% reduction in wetting episodes, improved-greater than 50\% reduction or failed-less than 50\% reduction. The children were 4 to 17 years old (mean age 8.7) and were treated with tolterodine for an average of 5.2 months. The final dose was 1 mg. twice daily in 1, 2 mg. twice daily in 27 and 4 mg. twice daily in 2 patients. Wetting episodes were cured in 10 (33\%), improved in 12 (40\%), and failed to show improvement in 8 (27\%) cases. Four patients (13.3\%) reported side effects and only 1 discontinued the medication due to diarrhea. There were no reports of hyperpyrexia, flushing or intolerance to sunshine and outdoor temperature. Our results demonstrate that tolterodine at adult doses without titration can be used safely to decrease wetting episodes in children with dysfunctional voiding. Controlled clinical trials should be completed to evaluate further efficacy and safety in children. [\hyperlink{Tolmetin Sodium}{PMID: 11176516}, M Munding et al., 2001]

\hypertarget{pmid_6886030}{D}ata from over 1000 patients with rheumatoid arthritis who received tolmetin sodium in double-blind and open studies have been pooled to assess long-term efficacy and safety. Duration of the studies was 12 weeks to 48 months. Mean age of patients was 54 years; ratio of males to females was 1:3. The results showed that tolmetin provided rapid onset of action and continuous progressive decrease in symptoms in all measurements of inflammation. Mean number of painful joints was reduced from 22 at baseline to 16 at one month, to 9 at one year, and to 6 at two years. Duration of morning stiffness was 155 minutes at baseline, 123 minutes at one month, 74 minutes at one year, and 78 minutes at two years. The final global evaluation by the investigators showed that 61 per cent of patients had a marked or moderate response. Mean erythrocyte sedimentation rates did not increase during therapy with tolmetin. Initial dose of tolmetin in the patients pooled for this analysis was generally 600 to 800 mg/day, and the mean dose throughout the study was 1256 mg/day. The drug was well tolerated overall. As anticipated, gastrointestinal symptoms were the most frequently reported; nausea was experienced by 13 per cent of the patients at some time during therapy, and gastrointestinal distress, dyspepsia, or abdominal pain was reported by approximately 8.6 per cent each. Only 12.7 per cent of patients discontinued tolmetin because of untoward reactions; 15.9 per cent of patients discontinued because of insufficient therapeutic response. The results of these long-term studies of patients with rheumatoid arthritis demonstrated that tolmetin is an effective antiinflammatory agent with an acceptable record of safety. [\hyperlink{Tolmetin Sodium}{PMID: 6886030}, G E Ehrlich et al., 1983]

\hypertarget{pmid_25724485}{T}he aims are to evaluate the efficacy and safety of sodium valproate for children with Tourette׳s syndrome (TS). We searched PubMed, EMBASE, the Cochrane library, Cochrane Central, CBM, CNKI, VIP, WANG FANG database and relevant reference lists. Five RCTs (N=247) and five case series (N=163) studies were included. Only one RCT (93 patients) evaluated total YGTSS scores and there was significant difference in the reduction of total YGTSS scores between sodium valproate and the control group (3.50±4.59 vs 7.86±7.03, P<0.01). One RCT (30 patients) evaluated motor and vocal tics, and there was significant difference in the reduction of motor and vocal tics scores between sodium valproate and haloperidol (10.45±4.15 vs 14.92±3.01, P<0.01). Meta-analysis of three RCTs (N=124) showed there was no significant difference in the reduction of the number of tics between sodium valproate and the positive control group [Relative Risk (RR)=1.09, 95\%CI (0.92, 1.30), P=0.30]. The pooled proportion in five case series studies which used tics symptom improvement self-defined by authors was 80.7\% (95\% CI: 73.7-86.2, I(2)=0). No fatal side effects were reported. In conclusion, based on the limited evidence, the routine use of sodium valproate for treatment of TS in children is not recommended. Further well-conducted trials that examine long-term outcomes are required.  [\hyperlink{Tolmetin Sodium}{PMID: 25724485}, Chun-Song Yang et al., 2015] The effectiveness of tolmetin sodium in the treatment of rheumatoid arthritis was evaluated by: 1) a 12-week, double-blind study with a dosage range of 800-1600 mg daily; and 2) an open 2-year study with a dosage range of 400-2400 mg daily. The double-blind study involved 14 patients (7 tolmetin sodium, 7 placebo), and the long-term study involved 24 patients. At frequent intervals, evaluations were made of joint pain, swelling, stiffness and inflammation; grip strength; walking time; and subjective well-being. Various laboratory tests were also performed. In the double-blind study, tolmetin sodium proved superior to placebo and produced moderate improvement. In the long-term study, 5 patients improved markedly, 14 moderately, and 3 minimally. Severe side effefts were notably absent. Some mild side effects occurred but they were transient and did not interfere with therapy. Tolmetin sodium seems effective and safe in the management of rheumatoid arthritis. [\hyperlink{Tolmetin Sodium}{PMID: 25724485}, L J Cordrey et al., 1976]

\hypertarget{pmid_27028950}{U}sing fluid restriction to treat the syndrome of inappropriate antidiuretic hormone secretion (SIADH) in infants is potentially hazardous, as fluid intake and caloric intake are connected. Antagonists for the type 2 vasopressin receptor have demonstrated efficacy in adult patients with SIADH, but evidence in children is lacking. We reviewed our experience from two cases in the UK. This was a retrospective review of the clinical data on two patients diagnosed with SIADH in infancy and treated with tolvaptan, an oral vasopressin receptor antagonist. Persistent hyponatraemia was noted in both patients in the first month of life and eventually led to SIADH diagnoses. Initial salt supplementation in one patient resulted in severe hypertension, treated with four antihypertensive drugs. Tolvaptan was commenced at two and four months of age, respectively, and was associated with normalisation of plasma sodium values and blood pressure without the need for antihypertensive treatment. There was transient hypernatraemia in one patient, which was normalised with a dose reduction. Tolvaptan was administered by crushing the tablet and mixing it with water. Tolvaptan provided effective treatment for SIADH in both infants and could be administered orally. [\hyperlink{Tolmetin Sodium}{PMID: 27028950}, Daniela Marx-Berger et al., 2016]

\hypertarget{pmid_6350376}{T}he efficacy and safety of tolmetin sodium in the management of ankylosing spondylitis are presented in a review of published and unpublished data. In both open and controlled clinical studies, tolmetin was superior to placebo and equal to indomethacin in its capacity to relieve pain, inflammation, and other symptoms of ankylosing spondylitis (AS). Objective and subjective assessments showed that both tolmetin sodium and indomethacin provided significant therapeutic benefits to patients with AS. In AS, the two drugs showed similar adverse reaction profiles. Adverse reactions with both drugs were minimal and predominantly affected the gastrointestinal tract; in most cases these symptoms cleared spontaneously without discontinuing the drugs. [\hyperlink{Tolmetin Sodium}{PMID: 6350376}, A Calin et al., 1983]

\hypertarget{pmid_19087826}{W}e intended to ascertain the effectiveness and safety of oral solutions of magnesium and vitamin B(6) in alleviating the symptoms emerged during clinical exacerbations in children aged 7-14 years suffering from Tourette syndrome (TS). We also aimed to determine the mean and the standard deviation of such an improvement in order to estimate sample sizes in future assays with a control group. The treatment under investigation was administered to children diagnosed with TS, in accordance with Diagnostic and Statistical Manual of Mental Disorders, fourth edition -IV, under conditions of clinical exacerbation. The effects were scored on the Yale Global Tics Severity Scale (YGTSS) at 0, 15, 30, 60 and 90 days. The total tics score decreased from 26.7 (t0) to 12.9 (t4) and the total effect on the YGTSS was a reduction from 58.1 to 18.8. Both results were statistically significant. With respect to the application of conventional treatment or otherwise, no significant differences were observed. No side effects were seen. The treatment assayed is safe and effective in reducing the harmful effects of TS in children. Further studies are needed, with a control group, and evaluation of different doses of the drugs. [\hyperlink{Tolmetin Sodium}{PMID: 19087826}, Rafael García-López et al., 2008]

\hypertarget{pmid_535831}{E}ffect of tolmetin sodium on the pain-like responses caused by various nociceptive stimuli was examined in experimental animals. Tolmetin sodium showed a potent inhibitory activity on the acetic acid-induced writhing in mice and rats, and its potency, (ED50 = 23.4 and 3.01 mg/kg, p.o.) was about 2.4--10.3 times that of ibuprofen and aspirin. The hypertension induced by intraarterial injection of bradykinin toward the spleen of dogs was inhibited by tolmetin sodium (ED50 = 80 mg/kg, i.v.), but the hypertension by a simultaneous injection of bradykinin and PGE1 was not inhibited by tolmetin sodium and sulpyrine, though pentazocine inhibited both hypertensions. The pain-like response caused by pressing mechanically the inflamed paws or joints of rats induced by kaolin-carrageenin or adjuvant was inhibited by tolmetin sodium (30--100 or 20--40 mg/kg, p.o., respectively), and the potency was approximately equal that of ibuprofen and phenylbutazone. Tolmetin sodium produced a significant inhibition of the pain-like response induced by electrical stimulation of tooth pulp of dogs, but showed no effect when the methods of Haffner and D'Amour-Smith were applied to mice. Anti-writhing action of tolmetin sodium was not antagonized by naloxone. From these results, it was concluded that tolmetin sodium has a potent inhibitory activity on the pain-like responses induced by the chemical nociceptive stimuli and by the mechanical pressure stimulus of the inflamed tissue, especially on the writhing. The analgesic activity probably involves a peripheral mechanism. [\hyperlink{Tolmetin Sodium}{PMID: 535831}, H Nakamura et al., 1979]

\hypertarget{pmid_10940537}{T}he safety of the avermectin, selamectin, was evaluated for topical use on the skin of cats of age six weeks and above, including reproducing cats and cats infected with adult heartworms. All studies used healthy cats. Acute safety was evaluated in domestic cross-bred cats. Margin of safety was evaluated in domestic-shorthaired cats, starting at six weeks of age. Reproductive, heartworm-infected, and oral safety studies were conducted in adult, domestic-shorthaired cats. Studies were designed to measure the safety of selamectin at the recommended dosage range of 6-12mgkg(-1) of body weight. Assessments included clinical, biochemical, pathologic, and reproductive indices. Selected variables in the margin of safety study and the reproductive studies were subjected to statistical analyses by using a mixed linear model. Cats received large doses of selamectin at the beginning of the margin of safety study when they were six weeks of age and at their lowest body weight, yet displayed no clinical or pathologic evidence of toxicosis. Similarly, selamectin had no adverse effect on reproduction in adult male and female cats. There were no adverse effects in heartworm-infected cats. Oral administration of the topical formulation, which might occur accidentally, caused mild, intermittent, self-limiting salivation and vomiting. Selamectin is a broad-spectrum avermectin endectocide that is safe for use in cats starting at six weeks of age, including heartworm-infected cats and cats of reproducing age, when administered topically to the skin monthly at the recommended dosage to deliver at least 6mgkg(-1). [\hyperlink{Tolmetin Sodium}{PMID: 10940537}, M J Krautmann et al., 2000]

\hypertarget{pmid_33730099}{O}ral ivermectin is a safe broad spectrum anthelminthic used for treating several neglected tropical diseases (NTDs). Currently, ivermectin use is contraindicated in children weighing less than 15 kg, restricting access to this drug for the treatment of NTDs. Here we provide an updated systematic review of the literature and we conducted an individual-level patient data (IPD) meta-analysis describing the safety of ivermectin in children weighing less than 15 kg. A systematic review was conducted using the Preferred Reporting Items for Systematic Reviews and Meta-Analyses (PRISMA) for IPD guidelines by searching MEDLINE via PubMed, Web of Science, Ovid Embase, LILACS, Cochrane Database of Systematic Reviews, TOXLINE for all clinical trials, case series, case reports, and database entries for reports on the use of ivermectin in children weighing less than 15 kg that were published between 1 January 1980 to 25 October 2019. The protocol was registered in the International Prospective Register of Systematic Reviews (PROSPERO): CRD42017056515. A total of 3,730 publications were identified, 97 were selected for potential inclusion, but only 17 sources describing 15 studies met the minimum criteria which consisted of known weights of children less than 15 kg linked to possible adverse events, and provided comprehensive IPD. A total of 1,088 children weighing less than 15 kg were administered oral ivermectin for one of the following indications: scabies, mass drug administration for scabies control, crusted scabies, cutaneous larva migrans, myiasis, pthiriasis, strongyloidiasis, trichuriasis, and parasitic disease of unknown origin. Overall a total of 1.4\% (15/1,088) of children experienced 18 adverse events all of which were mild and self-limiting. No serious adverse events were reported. Existing limited data suggest that oral ivermectin in children weighing less than 15 kilograms is safe. Data from well-designed clinical trials are needed to provide further assurance. [\hyperlink{Tolmetin Sodium}{PMID: 33730099}, Podjanee Jittamala et al., 2021]

\hypertarget{pmid_10940536}{S}elamectin is a broad-spectrum avermectin endectocide for treatment and control of canine parasites. The objective of these studies was to evaluate the clinical safety of selamectin for topical use in dogs 6 weeks of age and older, including breeding animals, avermectin-sensitive Collies, and heartworm-positive animals. The margin of safety was evaluated in Beagles, which were 6 weeks old at study initiation. Reproductive, heartworm-positive, and oral safety studies were conducted in mature Beagles. Safety in Collies was evaluated in avermectin-sensitive, adult rough-coated Collies. Studies were designed to measure the safety of selamectin at the recommended dosage range of 6-12mgkg(-1) of body weight. Endpoints included clinical examinations, clinical pathology, gross and microscopic pathology, and reproductive indices. Selected variables in the margin of safety and reproductive safety studies were subjected to statistical analyses. Pups received large doses of selamectin at the beginning of the margin of safety study when they were 6 weeks of age and at their lowest body weight, yet displayed no clinical or pathologic evidence of toxicosis. Similarly, selamectin had no adverse effects on reproduction in adult male and female dogs. There were no adverse effects in avermectin-sensitive Collies or in heartworm-positive dogs. Oral administration of the topical formulation caused no adverse effects. Selamectin is safe for topical use on dogs at the recommended minimum dosage of 6mgkg(-1) (6-12mgkg(-1)) monthly starting at 6 weeks of age, and including dogs of reproducing age, avermectin-sensitive Collies, and heartworm-positive dogs. [\hyperlink{Tolmetin Sodium}{PMID: 10940536}, M J Novotny et al., 2000]

\hypertarget{pmid_3321483}{T}he prevention of postoperative pain in children who had undergone tonsillectomy was investigated in a double-blind trial. Ketamine (Ketalar; Parke-Davis) 0.5 mg/kg was given intravenously before the operation to 20 children and saline to a control group of 20 children. Premedication consisted of oral trimeprazine 4 mg/kg given 2 hours pre-operatively. The anaesthetic technique was standardised. There were no significant differences between the groups pre-or intra-operatively. Postoperatively there were significant differences in the measurement of pain but not in that of sedation. No hallucinations were encountered in those receiving ketamine. It is concluded that analgesic doses of ketamine are safe and effective. [\hyperlink{Tolmetin Sodium}{PMID: 3321483}, W B Murray et al., 1987]

\hypertarget{pmid_30450703}{S}edation is often required for young children during transthoracic echocardiography. Dexmedetomidine and ketamine are two sedatives that are commonly used in children for procedural sedation, but they have some disadvantages when they are used alone. The aim of this retrospective study was to analyze the effects and safety of intranasal sedation with a combination of dexmedetomidine and ketamine during transthoracic echocardiography in young children and to analyze risk factors for sedation failure. After IRB approval, we retrospectively evaluated data on patients who underwent echocardiography between May 2016 and August 2017 utilizing a combination of dexmedetomidine 2 μg/kg and ketamine 1 mg/kg. We collected information including heart rate, pulse oxygen saturation, sedation onset time, exam time, recovery time, and adverse reactions. Stepwise logistic regression analyses were performed to analyze the risk factors for sedation failure. Sedation was successful in 2212 patients (96\%) and took effect in 15.7 (IQR: 10-23) min, while sedation failed in 92 patients. Cyanotic heart disease, history of sedation failure, history of congenital heart disease surgery, and fever were independent risk factors for sedation failure. Most of the patients in this study had an American Society of Anesthesiologists (ASA) grade of II to III, but no severe adverse reactions were observed. Intranasal sedation with a combination of dexmedetomidine and ketamine is effective and appears to have an acceptable safety profile for young children during transthoracic echocardiography. [\hyperlink{Tolmetin Sodium}{PMID: 30450703}, Jianxia Liu et al., 2019]

\hypertarget{pmid_12603422}{T}o assess the safety and efficacy of tolterodine tartrate prescribed to children who previously failed to tolerate oxybutynin chloride. We reviewed 34 children, followed for>1 year, who were prospectively crossed-over from oxybutynin to tolterodine because of side-effects. The initial diagnosis was dysfunctional voiding in 31 patients. All patients were placed on a behavioural modification protocol. When their symptoms did not improve after 6 months, treatment with an anticholinergic agent was considered. Urodynamic studies were conducted in 20 patients, confirming uninhibited contractions in 19. The remaining 14 patients were empirically started on antimuscarinic or anticholinergic agents. The 34 patients were treated with oxybutynin for a median (range) of 6 (2-84) months. When significant side-effects were reported, they were crossed over to tolterodine. The efficacy of tolterodine was assessed as defined by the International Children's Continence Society, with tolerability assessed and side-effects documented using a questionnaire. The mean age at the first dose of tolterodine was 8.9 years; the dose was 1 mg twice daily for 12 patients and 2 mg twice daily for 22. The median treatment with tolterodine was 11.5 months, with 20 (59\%) patients reporting no side-effects; six described the same but tolerable side-effects as with oxybutynin. Eight patients discontinued tolterodine because of side-effects after a median (range) of 5 (1-11) months. The efficacy of tolterodine was comparable with that of oxybutynin, as reported by the questionnaire and voiding diaries. The reduction in wetting episodes at 1 year was> 90\% in 23 (68\%), more than half in five and less than half (or failure) in six patients. Tolterodine is tolerated well in children. In this subgroup of patients who could not tolerate oxybutynin, 77\% were able to continue tolterodine treatment with no significant side-effects. [\hyperlink{Tolmetin Sodium}{PMID: 12603422}, S Bolduc et al., 2003]

\hypertarget{pmid_9088998}{T}he purpose of this study was to evaluate the antipyretic action of tolfenamic acid, as well as its possible adverse reactions, especially in children with severe or partial form of glucose-6-phosphate dehydrogenase (G6PD) deficiency. In the study 55 febrile children were included, whose mean age was +/- SD 3.5 +/- 3.3 years, range 0.5-15. Ten of them had severe or partial form of G6PD deficiency. Fifty-three of the patients responded with a decrease of temperature which lasted at least 6 hours, though in 2 of them the temperature decrease lasted less than 6 hours. The tolerance of the drug was good and no side-effects were noted. None of the patients with or without G6PD deficiency showed symptoms, signs, or laboratory findings indicating hemolysis before administration of the drug and 4 days thereafter. In conclusion, tolfenamic acid is a strong antipyretic agent with excellent tolerance and high safety in children. [\hyperlink{Tolmetin Sodium}{PMID: 9088998}, F A Haliotis et al., 1997]

\hypertarget{pmid_9831007}{T}opiramate is a sulfamate-substituted monosaccharide that has demonstrated efficacy as an antiepileptic drug in adults with partial onset seizures. Experience in children has been limited, but early reports have supported its safety and effectiveness in children as young as 2 years of age. In two infants ages 12 and 9 months, respectively, with partial seizures, the authors report excellent efficacy with good tolerability at doses up to 7.7 mg/kg. Although long-term safety and possible adverse sequelae have not been fully established in children, topiramate may represent an option for infants with high seizure frequency unresponsive to standard antiepileptic drugs. [\hyperlink{Tolmetin Sodium}{PMID: 9831007}, S L Kugler et al., 1998] 1 The site of the analgesic action of tolmetin sodium was investigated by use of the acetic acid writhing test in rats. 2 Tolmetin sodium was administered to the rat between 15 and 60 min after intraperitoneal injection of 1 ml of a 1\% acetic acid aqueous solution. Number of writhing was counted for 20 min beginning from 60 min after acetic acid injection. 3 When the rat was given tolmetin sodium 5 mg/kg orally, a relatively large quantity of tolmetin was found in the peritoneal exudate and there was a rough correlation between anti-writhing activity and the exudate tolmetin content. 4 Anti-writhing ED50 of tolmetin sodium was 1.42 (0.82-2.91) and 92.0 (57.0-140) microgram/kg when given intraperitoneally and intravenously, respectively, and the potency ratio of intraperitoneal to intravenous tolmetin sodium was 40.0 (18.5-80.2). This potency ratio for salicylic acid and morphine hydrochloride was 19.4 and 1.0, respectively. 5 When equipotent doses ( 5 microgram/kg i.p.; 200 microgram/kg i.v.) of tolmetin sodium were administered to the rat, the plasma tolmetin level after the intraperitoneal administration was less than one-fortieth that after the intravenous administration during the counting time of 20 min, while both the peritoneal exudate contents of tolmetin were nearly equal. 6 From these results, it is concluded that the site of anti-writhing action of tolmetin sodium is in the peritoneum and that tolmetin sodium produces its anti-writhing action mainly by a peripheral mechanism in the rat. [\hyperlink{Tolmetin Sodium}{PMID: 9831007}, H Nakamura et al., 1981]

\hypertarget{pmid_32896942}{O}n May 16, 2019, the U.S. Food and Drug Administration (FDA) approved dalteparin sodium for the treatment of symptomatic venous thromboembolism (VTE) to reduce the risk of recurrence in pediatric patients 1 month of age and older. Approval was primarily based on FDA review of a single-arm trial evaluating dalteparin administered subcutaneous twice daily in 38 pediatric patients with symptomatic VTE. Efficacy was based on the achievement of therapeutic plasma anti-Xa levels. The FDA concluded that dalteparin has efficacy and acceptable safety for pediatric patients. [\hyperlink{Tolmetin Sodium}{PMID: 32896942}, Margret Merino et al., 2020]

\hypertarget{pmid_17095898}{T}his was a prospective open study to establish the efficacy, tolerability, and problems associated with the use of topiramate as first-choice drug in children with infantile spasms. Open-label follow-up study, ranging from 24 to 36 months, of the cases of 54 patients with infantile spasms treated initially with topiramate as first-choice drug. Thirty-one patients (57.4\%) were seizure free for more than 24 months; 9 patients were treated with topiramate alone and 22 patients with topiramate plus nitrazepam and/or valproate. In 44 cases (81.4\%), the reduction of seizure frequency from baseline was greater than 30\%, whereas in 10 cases (18.6\%), there was poor or no response. The average dosage applied was 5.2 mg/kg per day (maximum dosage, 26 mg/kg per day; minimum dosage, 1.56 mg/kg per day). Adverse events occurred in 14 patients (26\%). They included poor appetite leading to anorexia, absence of sweating, and sleeplessness. Topiramate proves to be an effective and safe first-choice drug not only as adjunctive but also as monotherapy of infantile spasms in children younger than 2 years. [\hyperlink{Tolmetin Sodium}{PMID: 17095898}, Li-Ping Zou et al., ]

\hypertarget{pmid_11298060}{T}o determine the safety, efficacy and pharmacokinetics of tolterodine in children with an overactive bladder. Thirty-three children (20 boys and 13 girls, aged 5-10 years) with an overactive bladder and symptoms of urgency, frequency and/or urge incontinence were enrolled in an open, dose-escalation study. Patients were treated with oral tolterodine 0.5 mg (n = 11), 1 mg (n = 10) or 2 mg (n = 12) twice daily for 14 days. The primary safety endpoint was the change in residual urinary volume, as determined by ultrasonography. In addition, voiding diary variables (frequency and incontinence episodes) and pharmacokinetics were evaluated. Other safety endpoints included laboratory variables, electrocardiogram recordings and reported adverse events. There were no safety concerns in terms of the change in residual urinary volume for any of the three dosage groups; values were comparable with baseline after 2 weeks of treatment for all three dosages. Adverse events were reported by 20 patients (six on 0.5 mg, five on 1 mg, and nine on 2 mg). Most adverse events were not considered to be drug-related; of the 13 possibly related events, 10 occurred in those taking 2 mg. Headache was the most commonly reported adverse event. No serious adverse events were reported and there were no general safety concerns. There was an improvement in voiding diary variables in all treatment groups after 2 weeks of treatment, although the efficacy was greatest in those taking 1 mg and 2 mg. Pharmacokinetic findings were consistent with dose linearity over the range 0.5-2 mg. The results support the use of 1 mg twice daily as the optimal dose of tolterodine for treating children aged 5-10 years with an overactive bladder. [\hyperlink{Tolmetin Sodium}{PMID: 11298060}, K Hjälmås et al., 2001]

\hypertarget{pmid_24968572}{T}o examine the efficacy, safety and tolerability of tolterodine in children with overactive bladder in comparison with standard treatment i.e. oxybutynin as demonstrated in randomized clinical trials and other studies. A systematic search was done to screen the studies evaluating the effect of tolterodine in children with non-neurogenic overactive bladder. Results of studies were pooled and compared. Efficacy was determined from micturition diaries and dysfunctional voiding symptoms score. Safety and tolerability were assessed from the reported treatment emergent adverse events. A total of six randomized clinical trials and 11 other studies of tolterodine in children with urinary incontinence were included in the present systematic review. The dose of tolterodine used in different settings ranged from '0.5 to 8 mg/day' instead of '0.5 to 8 mg/kg per day' and the duration of studies ranged from 2 weeks to 12 months. Both extended and immediate release preparations of tolterodine were shown to have comparable efficacy and tolterodine proved to have comparable efficacy with better tolerability than oxybutynin in these studies. It can be concluded that tolterodine is efficacious in treatment of urinary incontinence in children. Moreover, its efficacy is comparable to oxybutynin, the most commonly prescribed anticholinergic in this condition, while having better tolerability. Hence, it can be considered as first line therapy for the treatmentof urinary incontinence in children. [\hyperlink{Tolmetin Sodium}{PMID: 24968572}, B Medhi et al., ]

\hypertarget{pmid_30303407}{T}olmetin sodium (TS) is a nonsteroidal anti-inflammatory drug (NSAID) indicated for treatment of musculoskeletal issues. As other NSAID, TS displays a marked side effects on the gastro-intestinal (GI) tract after oral administration. Traditional solid suppositories can cause pain and discomfort for patients, may reach the end of the colon; consequently, the drug can undergo the first-pass effect. TS liquid suppository (TS- [\hyperlink{Tolmetin Sodium}{PMID: 30303407}, Mohamed A Akl et al., 2019] The pharmacokinetics of tolmetin sodium were studied in five patients with rheumatoid arthritis (RA) and five normal volunteers to determine whether data derived from normals could be applied to RA patients. In addition, prostaglandin E (PGE) levels in synovial fluid were compared with tolmetin levels in serum and synovial fluid. Both groups received 400 mg tolmetin every 6 hours for seven days. During a 24-hour washout period after the dose of tolmetin (400 mg) on day 8, blood and urine samples were obtained from all study participants, and synovial fluid samples from the RA patients only. The patients continued into a second 24-hour drug-free period, after which they received a single 400-mg dose of tolmetin. Blood and urine samples were again collected. No clinically or statistically significant differences in tolmetin kinetics between normal volunteers and RA patients were found. A comparison of multiple-dose and single-dose results in the patient group showed an 11 per cent increase in the tolmetin serum concentration after multiple dosing. Total PGE levels in synovial fluid remained significantly depressed in the patient group for 24 hours after the 400-mg test dose of tolmetin on day 8. These findings suggest that tolmetin serum kinetics may not be an appropriate indicator of the duration of biologic activity of tolmetin. [\hyperlink{Tolmetin Sodium}{PMID: 30303407}, D E Furst et al., 1983]

\section*{Turalio}
\subsection*{Result}
\subsubsection*{Answer}

Unknown

\subsubsection*{{Explanation}}
\hypertarget{Turalio}
A review of the available abstracts reveals only one abstract specifically about Turalio (pexidartinib):

- [\hyperlink{pmid_31602563}{PMID: 31602563}, Yvette N Lamb et al., 2019] summarizes the development and approval of pexidartinib (Turalio™) for the treatment of adult patients with symptomatic tenosynovial giant cell tumor (TGCT). The abstract states that the US FDA approved pexidartinib for adults based on the phase III ENLIVEN trial and notes that pexidartinib is being investigated in various malignancies. However, the abstract does not mention any studies or data regarding the safety or efficacy of Turalio in children or any specific pediatric age group.

No other abstracts mention Turalio (pexidartinib) or provide data on its use in children. There are no targeted studies in the abstracts that evaluate the safety of Turalio in children or affirm its safety or lack thereof in any pediatric age range.

Therefore, based on the abstracts available, the safety of Turalio in children is unknown.

\subsection*{Abstracts}
\hypertarget{pmid_34798685}{T}opical tacrolimus is used off-label in young children, but data are limited on its use in children under 2 years of age and for long-term treatment. To compare safety differences between topical tacrolimus (0.03\% and 0.1\% ointments) and topical corticosteroids (mild and moderate potency) in young children with atopic dermatitis (AD). We conducted a 36-month follow-up study with 152 young children aged 1-3 years with moderate to severe AD. The children were followed up prospectively, and data were collected on infections, disease severity, growth parameters, vaccination responses and other relevant laboratory tests were gathered. There were no significant differences between the treatment groups for skin-related infections (SRIs) (P = 0.20), non-SRIs (P = 0.20), growth parameters height (P = 0.60), body weight (P = 0.81), Eczema Area and Severity Index (EASI) (P = 0.19), vaccination responses (P = 0.62), serum cortisone levels (P = 0.23) or serum levels of interleukin (IL)-4, IL-10, IL-12, IL-31 and interferon-γ. EASI decreased significantly in both groups (P < 0.001). In the tacrolimus group, nine patients (11.68\%) had detectable tacrolimus blood concentrations at the 1-week visit. There were no malignancies or severe infections during the study, and blood eosinophil counts were similar in both groups. Topical tacrolimus (0.03\% and 0.1\%) and topical corticosteroids (mild and moderate potency) are safe to use in young children with moderate to severe AD, and have comparable efficacy and safety profiles. [\hyperlink{Turalio}{PMID: 34798685}, A Salava et al., 2022]

\hypertarget{pmid_26861518}{I}nfantile hemangioma is the most common benign vascular tumor of childhood that has a tendency for spontaneous involution. The aim of this study was to evaluate the efficacy of topical timolol maleate in the treatment of superficial infantile hemangioma and associated side effects during the course of treatment. Four boys and five girls with a median age of 5 months were reviewed at 2-week intervals for a period of 16 weeks. A decrease in size, color, and consistency were noted. Adverse effects caused by timolol maleate were noted and managed. Of nine cases, two patients showed excellent response, five showed good response, one showed partial response, and one had poor response. Topical timolol maleate is safe and effective in the treatment of infantile hemangioma.  [\hyperlink{Turalio}{PMID: 26861518}, Abhijeet Kumar Jha et al., ] To date the efficacy and safety of topical timolol in the treatment of infantile hemangioma has not been reviewed and analysed systematically. We collated all published data on the efficacy and safety of topical timolol in the treatment of infantile hemangioma. A total of 31 studies with 691 patients were included. The fixed effects pooled estimate of the response rate defined as any improvement from baseline of infantile hemangioma after treatment with topical timolol was significant (RR = 8.96; 95\% CI 5.07-15.47; heterogeneity test p = 0.99), and the treatment was overall well tolerated. However, the quality of evidence was low to moderate. Topical timolol is an effective treatment for small infantile hemangioma, with no significant adverse effects noted. However, there is still a need for adequately powered randomised controlled trials. [\hyperlink{Turalio}{PMID: 26861518}, Maham Khan et al., 2017]

\hypertarget{pmid_31454858}{T}acrolimus is effective for refractory ulcerative colitis in adults, while data for children is sparse. We aimed to evaluate the effectiveness and safety of tacrolimus for induction and maintenance therapy in Japanese children with ulcerative colitis. We retrospectively reviewed the multicenter survey data of 67 patients with ulcerative colitis aged < 17 years treated with tacrolimus between 2000 and 2012. Patients' characteristics, disease activity, Pediatric Ulcerative Colitis Activity Index (PUCAI) score, initial oral tacrolimus dose, short-term (2-week) and long-term (1-year) outcomes, steroid-sparing effects, and adverse events were evaluated. Clinical remission was defined as a PUCAI score < 10; treatment response was defined as a PUCAI score reduction of ≥ 20 points compared with baseline. Patients included 35 boys and 32 girls (median [interquartile range] at admission: 13 [11-15] years). Thirty-nine patients were steroid-dependent and 26 were steroidrefractory; 20 had severe colitis and 43 had moderate colitis. The initial tacrolimus dose was 0.09 mg/kg/day (range, 0.05-0.12 mg/kg/day). The short-term clinical remission rate was 47.8\%, and the clinical response rate was 37.3\%. The mean prednisolone dose was reduced from 19.2 mg/day at tacrolimus initiation to 5.7 mg/day at week 8 (P< 0.001). The adverse event rate was 53.7\%; 6 patients required discontinuation of tacrolimus therapy. Tacrolimus was a safe and effective second-line induction therapy for steroid-dependent and steroid-refractory ulcerative colitis in Japanese children. [\hyperlink{Turalio}{PMID: 31454858}, Tadahiro Yanagi et al., 2019]

\hypertarget{pmid_25047057}{I}nfant hemangioma (IH) is the most common tumor in infants, which affects 5-10\% of white children. It is a tumor of vascular origin that appears in the first months of life. The indication for the treatment of the IH is not approved in the datasheet of the product, however it has been used in the infant hemangioma by topical administration as an alternative to oral propranolol, avoiding the main problems of the oral route (bradycardia and hypotension). The objective of this work is to study the physical and chemical (HPLC stability indicating method) stability of a 0.5\% timolol gel for topical application during 60 days (considering the stability limit as 90\% of initial concentration of timolol maleate). The gel was prepared with a polyacrylic acid derivative and the physical stability of the system was studied by visual control, rheological and mechanical characterization. The studied formulation guarantees the correct dose administering and stability after 60 days stored at 25 ± 2 °C and light protected (tube of aluminum). We have developed an easy topical gel for the treatment of infant hemangioma with physical and chemical stability higher than those provided by the majority of hospitals.  [\hyperlink{Turalio}{PMID: 25047057}, V Merino-Bohórquez et al., 2015] Children with tularemia are, irrespective of severity of disease, usually subjected to parenteral treatment with aminoglycosides. Based on available susceptibility testing, quinolones might be effective oral alternatives of parenteral therapy. These drugs cause arthropathy in immature animals, but this risk is currently regarded to be low in humans. In 12 patients (median age, 4 years; range, 1 to 10) with ulceroglandular tularemia, a 10- to 14-day course of oral ciprofloxacin, 15 to 20 mg/kg daily in 2 divided doses, was prescribed. Microbiologic investigations included identification of the infectious agent by PCR and culture of wound specimens, as well as determination of antibiotic susceptibility of isolates of Francisella tularensis. Defervescence occurred within 4 days of institution of oral ciprofloxacin in all patients. After a median period of 4.5 days (range, 2 to 24), the patients were capable of outdoor activities. In 2 cases, treatment was withdrawn after 3 and 7 days because of rash. In both cases a second episode of fever occurred. All children recovered without complications. In 7 cases F. tularensis was successfully cultured from ulcer specimens and tested for susceptibility to ciprofloxacin. MIC values for all isolates were 0.03 mg/l. In our sample of 12 patients ciprofloxacin was satisfactory for outpatient treatment of tularemia in children. [\hyperlink{Turalio}{PMID: 25047057}, A Johansson et al., 2000]

\hypertarget{pmid_26840644}{T}opical use of timolol for infantile hemangiomas has recently emerged with promising results. It is unknown whether topical β-blockers act locally or if their effect is partly due to systemic absorption. This study investigates whether topically applied timolol is absorbed and reports on the efficacy of this treatment. We treated 40 infants with small proliferating hemangiomas with topical timolol gel 0.5\% twice daily and assessed urinary excretion and serum levels in a proportion of patients. Clinical response was evaluated on a visual analog scale of standardized photographs after 1, 2, 3, and 5 months. Forty infants with a median age of 18 weeks (range 2-35 wks) were included; 23 (58\%) had superficial and 17 (42\%) mixed-type hemangiomas. The median size was 3 cm(2) (range 0.1-15 cm(2) ) and nine hemangiomas were ulcerated. The hemangiomas improved significantly during treatment, with a median increase in visual analog scale of 7 points after 5 months (p < 0.001). Urinalysis for timolol was performed in 24 patients and was positive in 20 patients (83\%). In three infants, serum levels of timolol were also measured and were all positive (median 0.16 ng/mL [range 0.1-0.18 ng/mL]). No significant side effects were recorded. Topical therapy with timolol is effective for infantile hemangiomas, but systemic absorption occurs. Serum levels in our patients were low, suggesting that using timolol for small hemangiomas is safe, but caution is advised when treating ulcerated or large hemangiomas, very young infants, or concomitantly using systemic propranolol. [\hyperlink{Turalio}{PMID: 26840644}, Lisa Weibel et al., ]

\hypertarget{pmid_15550128}{C}hildhood vitiligo is a common disorder of pigmentation in India. Considering the lack of uniformly effective and safe treatment modalities for children with vitiligo, search for newer therapeutic agents continues. This study was designed to evaluate the role of topical tacrolimus in the treatment of childhood vitiligo. Twenty-five children with vitiligo were treated with topical 0.03\% tacrolimus ointment applied twice daily for 12 weeks. Response was noted as marked to complete (> 75\% repigmentation), moderate (50-75\% repigmentation) and mild (< 50\% repigmentation). Twenty-two children (9 boys and 13 girls) of mean age 7.2 +/- 1.4 years completed the study. Twelve (54.5\%) children had vitiligo vulgaris, nine (40.9\%) had focal vitiligo and one (4.5\%) had segmental vitiligo. The mean duration of disease was 8 +/- 3 months. Nineteen (86.4\%) children showed some repigmentation at the end of 3 months and other three had no response. Of these 19 children, repigmentation was marked to complete in 11 (57.9\%), moderate in five (26.3\%) and mild in three (15.7\%) children. Side effects were minimal, such as the pruritus and burning noted in only three patients. Topical tacrolimus is an effective and well-tolerated treatment modality in Asian children with vitiligo. [\hyperlink{Turalio}{PMID: 15550128}, A J Kanwar et al., 2004]

\hypertarget{pmid_25041277}{T}he European Medicine Agency recommendations limiting codeine use in children have created a void in managing moderate pain. We review the evidence on the pharmacokinetic, pharmacodynamic and safety profile of tramadol, a possible substitute for codeine. Tramadol appears to be safe in both paediatric inpatients and outpatients. It may be appropriate to limit the current use of tramadol to monitored settings in children with risk factors for respiratory depression, subject to further safety evidence. [\hyperlink{Turalio}{PMID: 25041277}, Pierluigi Marzuillo et al., 2014]

\hypertarget{pmid_10961786}{T}he objective of this study was to determine the safety and tolerability of the immunomodulatory agent thalidomide as adjunct therapy in children with tuberculous meningitis. Children with stage 2 tuberculous meningitis received oral thalidomide for 28 days in a dose-escalating study, in addition to standard four-drug antituberculosis therapy, corticosteroids, and specific treatment of complications such as raised intracranial pressure. Clinical and laboratory evaluations were carried out. Fifteen patients (median age, 34 months) were enrolled. Thalidomide was administered via nasogastric tube in a dosage of 6 mg/kg/day, 12 mg/kg/day, or 24 mg/kg/day. The only adverse events possibly related to the study drug were transient skin rashes in two patients. Levels of tumor necrosis factor-alpha in the cerebrospinal fluid decreased markedly during thalidomide therapy. Clinical outcome and neurologic imaging showed greater improvement than that experienced with historical controls. Thalidomide appeared safe and well tolerated in children with stage 2 tuberculous meningitis and could have important anti-inflammatory effects. These promising results have led us to embark on a randomized, double-blind, placebo-controlled trial of the efficacy of thalidomide in tuberculous meningitis. [\hyperlink{Turalio}{PMID: 10961786}, J F Schoeman et al., 2000]

\hypertarget{pmid_20510770}{A}lthough clinical trials for new drugs are often limited in children because of safety concerns or restrictions, new therapies or novel strategies with old drugs have recently expanded dermatologic armamentarium for pediatric patients. Oral propranolol is currently the first choice in the treatment of alarming infantile hemangiomas. In atopic dermatitis, proactive strategy with topical calcineurin inhibitors can safely prevent disease exacerbation. Tacrolimus, in particular, is also useful for the treatment of vitiligo occurring in sensitive areas such as the eyelids. Among biologic drugs, use of etanercept is safe and efficient in children and adolescents with moderate-to-severe plaque psoriasis. Engineered tissues with special antimicrobial properties (silver-coated fabrics or engineered silk) are now used to treat eczema and fungal diseases in children. In athlete's foot, the use of 5-finger socks can also be helpful. [\hyperlink{Turalio}{PMID: 20510770}, Carlo Gelmetti et al., 2010]

\hypertarget{pmid_23412986}{T}o compare efficacy and safety of topiramate (TPM) and propranolol for migraine prophylaxis in children. In a parallel single-blinded randomized clinical trial, 5-15 y-old referred migraineurs to Pediatric Neurology Clinic of Shahid Sadoughi Medical Sciences University, Yazd, Iran from May through October 2011, were evaluated. Patients were distributed into two groups, 50 of whom were treated with 3 mg/kg/d of topiramate (TPM) and another group of 50, were treated with 1 mg/kg of propranolol for 3 mo. Primary endpoints were efficacy in reduction of monthly frequency, severity, duration and headache related disability. Secondary outcome was clinical side effects. Fifty two girls and 48 boys with mean age of 10.34 ± 2.31 y were evaluated. Monthly frequency, severity and duration of headache decreased with TPM, from 13.88 ± 8.4 to 4.13 ± 2.26 attacks, from 6.32 ± 1.93 to 2.8 ± 2.12, and from 2.36 ± 1.72 to 0.56 ± 0.5 h, respectively. Monthly frequency, severity and duration of headache also decreased with propranolol from 16.2 ± 6.74 to 8.8 ± 4.55 attacks, from 6.1 ± 1.54 to 4.8 ± 1.6 and from 2.26 ± 1.26 to 1.35 ± 1.08 h, respectively. Pediatric Migraine Disability Assessment score reduced from 31.88 ± 9.72 to 9.26 ± 7.21 with TPM and from 33.08 ± 8.98 to 23.64 ± 9.88 with propranolol. Transient mild side effects were seen in 18 \% of TPM and in 10 \% of propranolol (P = 0.249) groups. Topiramate is more effective than propranolol for pediatric migraine prophylaxis. [\hyperlink{Turalio}{PMID: 23412986}, Razieh Fallah et al., 2013]

\hypertarget{pmid_31602563}{P}exidartinib (TURALIO™) is an orally administered small molecule tyrosine kinase inhibitor with selective activity against the colony-stimulating factor 1 (CSF1) receptor, KIT proto-oncogene receptor tyrosine kinase (KIT) and FMS-like tyrosine kinase 3 harboring an internal tandem duplication mutation (FLT3-ITD). In August 2019, the US FDA approved pexidartinib capsules for the treatment of adult patients with symptomatic tenosynovial giant cell tumor (TGCT) associated with severe morbidity or functional limitations and not amenable to improvement with surgery. This approval was based on positive results from the phase III ENLIVEN trial. Pexidartinib is being investigated in various malignancies as monotherapy or combination therapy. This article summarizes the milestones in the development of pexidartinib leading to its first approval for TGCT. [\hyperlink{Turalio}{PMID: 31602563}, Yvette N Lamb et al., 2019]

\hypertarget{pmid_36138604}{B}ackground: Sirolimus, a mammalian target of rapamycin inhibitor, has been widely used in pediatric patients, but the safety of sirolimus in pediatric patients has not been well determined. Objective: The objective of this study was to systematically evaluate prospective studies reporting the safety of sirolimus in the treatment of childhood diseases. Methods: The following data were extracted in a standardized manner: study design, demographic characteristics, intervention, and safety outcomes. Results: In total, 9 studies were included, encompassing 575 patients who received oral sirolimus for at least 6 months. Various adverse events occurred. The most common adverse event was oral mucositis (8.2\%, 95\% CI: 0.054 to 0.110). Through comparative analysis of the subgroups based on the targeted concentration range, we discovered that many adverse events were significantly higher in the high concentration group (≥10 ng/mL) than in the low concentration group (<10 ng/mL) (p < 0.01). More interestingly, we found that oral mucositis was more frequently reported in children with vascular anomalies than tuberous sclerosis complex. Conclusions: This study shows that oral sirolimus in the treatment of childhood diseases is safe and reliable. However, sirolimus treatment in the pediatric population should be strictly monitored to reduce the occurrence of serious or fatal adverse events. [\hyperlink{Turalio}{PMID: 36138604}, Zixin Zhang et al., 2022]

\hypertarget{pmid_28840010}{T}he new drugs delamanid and bedaquiline are increasingly used to treat multidrug-resistant (MDR-) and extensively drug-resistant tuberculosis (XDR-TB). As evidence is lacking, the World Health Organization recommends their use under specific conditions in adults, delamanid only being recommended in children ≥6 years of age. No systematic review has yet evaluated the efficacy, safety and tolerability of the new drugs in children. A search of peer-reviewed, scientific evidence was performed, to evaluate the efficacy/effectiveness, safety, and tolerability of delamanid or bedaquiline-containing regimens in children with confirmed M/XDR-TB. We used PubMed and Embase to identify any relevant manuscripts in English until 31 December 2016, excluding editorials and reviews. Three out of 96 manuscripts retrieved satisfied the inclusion criteria, while 93 were excluded because dealing exclusively with adults (12: 4 on delamanid and 8 on bedaquiline), being recommendations or guidelines (8 manuscripts), reviews (17 papers) or other studies (56 papers). One of the studies retrieved reported evidence on 19 M/XDR-TB children, 16 of them treated under compassionate use with delamanid (13 achieving consistent bacteriological conversion) and 3 candidates for the drug. Two studies reported details on the first paediatric case treated (and cured) with a delamanid-containing regimen. Eight trials including children were also retrieved (clinicaltrials.gov). Although the methodology used in the study was rigorous, the results are limited by the paucity of the studies available in the literature on the use of new anti-TB drugs in children. In conclusion, more evidence is needed on the use of delamanid and bedaquiline in paediatric patients. [\hyperlink{Turalio}{PMID: 28840010}, Lia D'Ambrosio et al., 2017]

\hypertarget{pmid_24716805}{M}ethicillin-resistant Staphylococcus aureus (MRSA) remains a significant cause of morbidity in hospitalized infants. Over the past 15 years, several drugs have been approved for the treatment of S. aureus infections in adults (linezolid, quinupristin/dalfopristin, daptomycin, telavancin, tigecycline and ceftaroline). The use of the majority of these drugs has extended into the treatment of MRSA infections in infants, frequently with minimal safety or dosing information. Only linezolid is approved for use in infants, and pharmacokinetic data in infants are limited to linezolid and daptomycin. Pediatric trials are underway for ceftaroline, telavancin, and daptomycin; however, none of these studies includes infants. Here, we review current pharmacokinetic, safety and efficacy data of these drugs with a specific focus in infants.  [\hyperlink{Turalio}{PMID: 24716805}, Martyn Gostelow et al., 2014] Propranolol has become the treatment of choice of large and complicated infantile hemangiomas. There is a controversy concerning the safety of systemic propranolol. Here we show that topical use of the beta-blocker timolol can also inhibit the growth and promote regression of infantile hemangiomas. In this case series we treated 11 infantile hemangiomas in nine children including six preterm babies with the nonselective betablocker timolol. A timolol containing gel was manufactured from an ophthalmic formulation of timolol 0.5\% eyedrops. This gel was applied using a standardized occlusive dressing (Finn-Chambers) containing approximately 0.25 mg of timolol. In all infants topical timolol was associated with growth arrest, a reduction in redness and thickness within the first 2 weeks. Seven hemangiomas showed almost complete resolution, and four became much paler and thinner. No data are available on the transdermal absorption of timolol. Even supposing complete absorption of timolol from the occlusive dressing, a maximum dose of 0.25 mg of timolol would result per day and hemangioma. Regression of infantile hemangiomas treated using 0.5\% timolol gel in this case series occurred earlier than spontaneous regression which is generally not observed before the age of 9-12 months. The promising results need to be verified in prospective randomized trials on topical beta blocker administration for infantile hemangiomas which should address dose, duration, and mode of application. [\hyperlink{Turalio}{PMID: 24716805}, Matthias Moehrle et al., ]

\hypertarget{pmid_28719370}{T}he discovery of beta-adrenergic blocker effects on infantile hemangiomas has affected the choice of treatment in recent years. Oral propranolol is effective in treating infantile hemangiomas, but the risk of systemic side effects remains a concern. Data from literature review reported positive clinical outcomes with no major adverse effects observed in children using topical beta-blocker such as timolol maleate. Topical application of timolol eye drop has been mentioned in few studies, most of which reported that the solution cannot stay on the site of application due to its fluidity. Adding hydroxyethyl cellulose into a timolol solution increased its viscosity by forming a hydrogel and thus changed the rheology property. The compounded timolol gel exhibited thixotropy property allowing better and longer contact at sites of application. Experimental data from literature review showed desirable characteristics and measurable flux of timolol across human stratum corneum. Gel dosage form allows easy and precise application and maximizes timolol's therapeutic efficacy at the sites of application. [\hyperlink{Turalio}{PMID: 28719370}, Winnie Choo et al., ]

\hypertarget{pmid_15247700}{M}any children with urological disease require long-term treatment with antibiotics. In many cases the choice of medical instead of surgical management hinges on the implied safety of certain drugs. Recently some groups have advocated subureteral injection procedures to avoid long-term antibiotics for low grade reflux. We present a concise and relevant review on the use and adverse reactions of nitrofurantoin, trimethoprim and sulfamethoxazole in children. We reviewed the literature regarding the safety and toxicity of these drugs. Information regarding absorption, excretion and dosing was also gathered to explain better the mechanisms of toxicity. Adverse reactions in children reported in the literature related to nitrofurantoin are gastrointestinal disturbance (4.4/100 person-years at risk), cutaneous reactions (2\% to 3\%), pulmonary toxicity (9 patients), hepatoxicity (12 patients and 3 deaths), hematological toxicity (12 patients), neurotoxicity and an increased rate of sister chromatid exchanges. Adverse reactions in children related to trimethoprim/sulfamethoxazole are almost exclusively due to the sulfamethoxazole component, including cutaneous reactions (1.4 to 7.4 events per 100 person-years at risk), hematological toxicity (0\% to 72\% of patients) and hepatotoxicity (5 patients). The majority of adverse reactions were found in children on full dose therapy and not prophylaxis. The use of nitrofurantoin, trimethoprim and sulfamethoxazole is safe in children for long-term prophylactic therapy. The antibiotic safety issue should not be misconstrued as an argument for surgical therapy, whether minimally invasive or not. Adverse reactions exist to these medicines but they are less common than seen in adults, presumably because of the lower dose used for therapy, and the lack of significant comorbidities and drug interactions in children. Serious side effects are extremely rare and most are reversible by discontinuing therapy. The extremely low potential for significant adverse reactions should be discussed with parents. [\hyperlink{Turalio}{PMID: 15247700}, Edward Karpman et al., 2004]

\hypertarget{pmid_1451925}{A} total of 101 children (47 males, 54 females; age range, 3 months-16 years) with mild to moderate upper or lower respiratory tract infections, or skin and soft tissue infections entered a clinical study conducted at two centres in Izmir, Turkey. The children received a mean daily dose of 25 mg/kg sultamicillin oral suspension administered as two equal doses approximately 12 h apart. In total, 100 children met all requirements for evaluability and were included in the clinical efficacy assessment, and 49 children were evaluated for bacteriological efficacy. Clinical cure was reported by the investigators in 93 patients, improvement in six and failure in only one. The bacteriological eradication rate of isolated pathogens was 100\%. Of the 101 patients evaluated for drug safety, four experienced adverse drug-related or possibly drug-related reactions. All side-effects were gastro-intestinal and diarrhoea was reported in three patients. No discontinuation of therapy was reported, nor were any significant laboratory abnormalities recorded. [\hyperlink{Turalio}{PMID: 1451925}, P Raillard et al., 1992]

\hypertarget{pmid_17176789}{S}everal trials have indicated that topical tacrolimus is safe and effective for several immunologic-based skin disorders. We report a child <2 months old, who was admitted to our pediatric department because of generalized erythroderma. We started tacrolimus ointment 0.03\% therapy by applying the ointment to the skin once daily, on a limited surface of 10 cm(2), changing the skin areas only after the disappearance of inflammation. After only one application we observed a marked improvement of the treated area and after another application the localized erythroderma disappeared. We changed the site of application and continued this treatment regime for 4 weeks. After 2 weeks we reduced treatment application once every other day and after an additional 2 weeks we discontinued therapy because of complete resolution of erythroderma. No adverse clinical effect was recorded and tolerance to the treatment was good. We conclude that 0.03\% tacrolimus ointment is efficacious and safe even in an infant <2 months of age. [\hyperlink{Turalio}{PMID: 17176789}, Salvatore Leonardi et al., ]

\hypertarget{pmid_385400}{I}n a double-blind placebo controlled study of levamisole in the treatment of children with recurrent upper respiratory tract infection (URI) eighty-six patients took part. Medication was given once a week, in a single body-weight adjusted dose. The children treated with levamisole had a statistically significantly reduced incidence of episodes of infection which were severe, less prolonged and required less antibiotics. No side-effects were reported. [\hyperlink{Turalio}{PMID: 385400}, F Dils et al., 1979]

\hypertarget{pmid_27588127}{T}he aim of the present study was to assess the efficacy and safety of topical timolol maleate combined with oral propranolol for parotid infantile hemangiomas. Between October 2012 and April 2014, propranolol was administered orally at a dose of 1.0-1.5 mg/kg/day to 22 infants with proliferating hemangiomas in the Department of Oral and Maxillofacial Surgery (Hospital of Stomatology, China Medical University, Shenyang, Liaoning, China). A small amount of 0.5\% timolol maleate eye drop solution was topically applied with medical cotton swabs to the area of the lesion twice a day, every 12 h. The study group consisted of 9 males and 13 females, aged 2-9 months, with a median age of 4.7 months. The lesions were all located in the parotid region, and measured between 3.5×4×0.5 and 7×8×3 cm in volume. The planned duration of therapy was 6-8 months, or the two drugs were stopped when complete regression of the lesions was obtained. The therapeutic outcomes and safety were assessed by the change in the size and color of the tumor, and the presence of adverse effects throughout the course of treatment. The mean duration of therapy was 21.1 weeks and ranged from 3 to 8 months. Of the 22 patients, 16 demonstrated an excellent response, 6 showed a good response and 2 displayed a moderate response. No major collateral effects were observed. Overall, oral propranolol combined with topical timolol maleate may be used as the first-line therapeutic choice in the treatment of infantile parotid mixed hemangioma. [\hyperlink{Turalio}{PMID: 27588127}, Shuang Tong et al., 2016]

\hypertarget{pmid_37969618}{T}his study aimed to assess the efficacy and safety of topical timolol in treating facial angiofibromas (FAs) in pediatric patients with tuberous sclerosis complex (TSC). A prospective clinical trial was conducted involving 15 children diagnosed with TSC and presenting with FAs. The participants were administered topical timolol gel 0.5\% twice daily. Prior to the intervention, the severity of FAs in each patient was evaluated using the FA severity index (FASI), which assessed erythema, size, and extent of lesions. Clinical response was assessed at weeks 2 and 4 during the intervention period as well as 1 month after discontinuation of treatment. Four weeks after discontinuing topical timolol 0.5\%, statistically significant reductions were observed in the mean FASI score, erythema, size, and extent of lesions ( Topical timolol 0.5\% represents a cost-effective and readily available treatment option for pediatric patients with FAs associated with tuberous sclerosis. [\hyperlink{Turalio}{PMID: 37969618}, Mohammadreza Ghazavi et al., ]

\hypertarget{pmid_32943036}{D}rooling is common in children with neurological disorders, but its management is very challenging, Scopolamine transdermal patch (STP) appears to be useful in controlling drooling, although it is not approved for this indication and there are limited clinical studies about its effectiveness. This study aimed (1) to assess the impact of STP use on the severity of drooling and on the frequency of emergency department (ED) and hospital readmission (RA) visits related to drooling, and (2) to determine the level of family satisfaction with STP when used in children with neurological disorders. This is a retrospective cohort study of all pediatric patients aged 3-14 years, with non-progressive neurodevelopmental disability, who used STP for more than one year during the period between April 2015 and July 2018 (n = 44). Data on demographics, clinical status, comorbidities, STP dose and duration, other medications, ED and RA visits were collected. Follow-up phone-call interviews with parents/caregivers were performed using a parent-reported frequency and severity rating scale of sialorrhea. Absolute and relative risk reductions were calculated to assess the impact of STP on ED and RA visits. Significance was considered at p-value of ≤ 0.05. STP use showed significant reduction in severity of drooling (p < 0.001), wiping of the child's mouth (p < 0.001), bibs or clothing changes (p < 0.001), choking and aspiration of saliva (p = 0.001). The Relative Risk Reduction of the drooling-related ED and RA visits were 86\% and 67\% respectively. Nearly two-thirds (60\%) of caregivers were satisfied with using STP. This is the first study of its kind done in Saudi Arabia demonstrating favorable impact of STP use by children on the consequences associated with drooling and with the frequency of ER and RA visits due to drooling. Development of a medication use protocol is recommended to standardize STP treatment in order to optimize its effectiveness. This study serves as baseline information for future prospective interventional studies. [\hyperlink{Turalio}{PMID: 32943036}, Majed Al Jeraisy et al., 2020]

\section*{Vitrakvi}
\subsection*{Result}
\subsubsection*{Answer}

Unknown

\subsubsection*{{Explanation}}
\hypertarget{Vitrakvi}
A review of the provided abstracts reveals that none of them are about Vitrakvi (larotrectinib). The abstracts discuss the safety and efficacy of various other medications in children, such as levetiracetam, vigabatrin, mivacurium, rivaroxaban, lidocaine, tetracaine, and others, but there is no mention of Vitrakvi or studies specifically evaluating its safety in pediatric populations. Therefore, based solely on the abstracts available, there is not enough data to determine whether Vitrakvi is safe or unsafe for use in children in any age range.

\subsection*{Abstracts}
\hypertarget{pmid_31154809}{F}ew drugs are available for migraine prophylaxis in children. Levetiracetam is a broad-spectrum anti-seizure drug that has been suggested to be effective in reducing adult migraine episodes. We assessed the safety and efficacy of levetiracetam in the prevention of pediatric migraine. A randomized double-blind placebo-controlled trial was performed. Eligible participants were aged 4-17 years old with at least four migrainous episodes monthly or had severe disabling or intolerable episodes. Primary endpoints were the mean changes in monthly frequency and intensity of headaches from the baseline phase to the last month of the double-blind phase. Safety endpoint was the adverse effects reported. Sixty-one participants (31 taking levetiracetam and 30 taking placebo) completed the study. All had a significant reduction in frequency and intensity of episodes that was significantly greater in the levetiracetam arm. Sixty eight percent of individuals in the treatment group reported more than 50\% reduction of episodes at the end of the trial compared with 30\% in the placebo group ( Levetiracetam may be useful in migraine prevention and may decrease migraine episodes and severity. The study is prospectively registered with Iranian Registry of Clinical Trials; IRCT.ir, number IRCT2017021632603N1. [\hyperlink{Vitrakvi}{PMID: 31154809}, Hadi Montazerlotfelahi et al., 2019]

\hypertarget{pmid_10561962}{T}o evaluate the efficacy of vigabatrin in the treatment of infantile spasms in Thai children. From March 1996 to May 1998, patients aged under 2 years presenting with infantile spasms at Ramathibodi Hospital were initiated with vigabatrin 35-50 mg/kg/day in two-divided doses. The dosage was escalated by 25 mg/kg weekly until spasms ceased or the maximum dose of 130 mg/kg was reached. There were 20 patients enrolled. The ages ranged from 3 to 23 months (mean 7.6 months). They were categorized as 4 cryptogenic and 16 symptomatic. Infantile spasms were completely controlled in 12 patients (60\%). Six patients (30\%) had at least 50 per cent reduction of seizure frequency. There were 2 patients whose seizure frequencies and severity were not altered. Only one patient whose infantile spasms partially responded to vigabatrin developed orofacial dyskinesis which disappeared after discontinuation of vigabatrin. Five patients had their vision evaluated which was unremarkable. Based on parental global evaluation, there was an increase in alertness, cheerfulness and interaction to the environment and stimulation in 8 out of 15 patients who were still taking vigabatrin and responded to treatment. Vigabatrin is effective for infantile spasms. A long-term follow-up of these patients is necessary to evaluate its efficacy and side-effects. [\hyperlink{Vitrakvi}{PMID: 10561962}, A Visudtibhan et al., 1999]

\hypertarget{pmid_10183381}{I}n children with infantile spasms, vigabatrin monotherapy has been assessed in three published comparative trials.  The small numbers of patients make it impossible to draw precise conclusions on effectiveness.  However, a few days' treatment with a dose of about 100 mg/kg/day clears infantile spasms in a larger proportion of cases than a placebo or steroids.  Vigabatrin seems to be more effective in Bourneville disease.  The effect is sometimes transient: despite continued treatment, spasms or other types of epilepsy occur in approximately 50\% of patients who are initially improved.  In a trial versus ACTH, the lesser initial efficacy of vigabatrin was partly offset by a lower incidence of relapse and other types of seizures.  Vigabatrin is effective in some children who are resistant to ACTH or steroids.  As with steroids and ACTH, there is no proof that vigabatrin improves the long-term psychomotor development of these children.  In comparative trials the incidence of adverse events was statistically lower on vigabatrin than on steroids.  Most of the events were relatively mild neuropsychological effects, but a question mark still hangs over the possible neurotoxicity or oculotoxicity of vigabatrin during long-term administration. [\hyperlink{Vitrakvi}{PMID: 10183381}, Vigabatrin: new indication. An advance in infantile spasms., 1998]

\hypertarget{pmid_11673582}{I}nfantile spasms are a rare but devastating pediatric epilepsy that, outside the United States, is often treated with vigabatrin. The authors evaluated the efficacy and safety of vigabatrin in children with recent-onset infantile spasms. This 2-week, randomized, single-masked, multicenter study with a 3- year, open-label, dose-ranging follow-up study included patients who were younger than 2 years of age, had a diagnosed duration of infantile spasms of no more than 3 months, and had not previously been treated with adrenocorticotropic hormone, prednisone, or valproic acid. Patients were randomly assigned to receive low-dose (18-36 mg/kg/day) or high-dose (100-148 mg/kg/day) vigabatrin. Treatment responders were those who were free of infantile spasm for 7 consecutive days beginning within the first 14 days of vigabatrin therapy. Time to response to therapy was evaluated during the first 3 months, and safety was evaluated for the entire study period. Overall, 32 of 142 patients who were able to be evaluated for efficacy were treatment responders (8/75 receiving low-dose vigabatrin vs 24/67 receiving high doses, p < 0.001). Response increased dramatically after approximately 2 weeks of vigabatrin therapy and continued to increase over the 3-month follow-up period. Time to response was shorter in those receiving high-dose versus low-dose vigabatrin (p = 0.04) and in those with tuberous sclerosis versus other etiologies (p < 0.001). Vigabatrin was well tolerated and safe; only nine patients discontinued therapy because of adverse events. These results confirm previous reports of the efficacy and safety of vigabatrin in patients with infantile spasms, particularly among those with spasms secondary to tuberous sclerosis. [\hyperlink{Vitrakvi}{PMID: 11673582}, R D Elterman et al., 2001]

\hypertarget{pmid_15941649}{V}igabatrin (VGB) is an important treatment option for infantile spasms. Vigabatrin-induced visual field defects are at present the most important safety issue in the use of the drug. The knowledge concerning VGB-associated visual dysfunction in pediatric patients, particularly in those who have been exposed to VGB in utero is limited. We explored ophthalmic and neurologic findings in two children who have been exposed prenatally to VGB. [\hyperlink{Vitrakvi}{PMID: 15941649}, Iiris Sorri et al., 2005]

\hypertarget{pmid_26901576}{V}i polysaccharide typhoid vaccines cannot be used in children <2 years owing to poor immunogenic and T cell independent properties. Conjugate vaccine prepared by binding Vi to tetanus toxoids (Vi-TT) induces protective levels even in children <2 years. We evaluated efficacy and safety following vaccination with a Vi-TT vaccine in children 6 months to 12 years of age. Overall, 1765 subjects were recruited from two registered municipal urban slums of southern Kolkata. Most of the children of the slum dwellers attended the schools in the locality which was selected with permission from the school authority. Schools were randomly divided into vaccinated (Test group) and unvaccinated group (Control group). Children and their siblings of test group received 2-doses of PedaTyph™ vaccine at 6 weeks interval. Control group received vaccines as per national guidelines. Adverse events (AEs) were examined after 30 minutes, 1 month and clinical events were observed till 12 months post-vaccination. Incidence of culture positive typhoid fever in the control group was 1.27\% vis-a-vis none in vaccine group during 12 months. In subgroup evaluated for immunogenicity, an antibody titer value of 1.8 EU/ml (95\% CI: 1.5 EU/ml, 2.2 EU/ml), 32 EU/ml (95\% CI: 27.0 EU/ml, 39.0 EU/ml) and 14 EU/ml (95\% CI: 12.0 EU/ml, 17.0 EU/ml) at baseline, 6 weeks and 12 months, respectively was observed. Sero-conversion among the sub-group was 100\% after 6 weeks of post-vaccination and 83\% after 12 months considering 4-fold rise from baseline. The efficacy of vaccine was 100 \% (95\% CI: 97.6\%, 100\%) in the first year of follow-up with minimal AEs post vaccination. Vi conjugate typhoid vaccine conferred 100\% protection against typhoid fever in 1765 children 6 months to 12 years of age with high immunogenicity in a subgroup from the vaccine arm. [\hyperlink{Vitrakvi}{PMID: 26901576}, Monjori Mitra et al., 2016]

\hypertarget{pmid_9095401}{V}igabatrin has been shown to be efficient in infants with infantile spasms and tuberous sclerosis, in open studies. In order to compare vigabatrin to oral steroids, a prospective randomized multicenter study was implemented using both drugs as monotherapy in newly diagnosed patients with infantile spasms and tuberous sclerosis. Eleven infants received vigabatrin (150 mg/kg per day) and 11 hydrocortisone (15 mg/kg per day) for 1 month. Spasm free patients continued vigabatrin or progressively stopped hydrocortisone in 1 month, non-responders were crossed to the other drug for a new 2 month-period. All vigabatrin patients (11/11) were spasm-free versus 5/11 hydrocortisone infants (P < 0.01). Seven patients were crossed to vigabatrin (six for inefficacy, one for adverse events) and became also totally controlled. Mean time to disappearance of infantile spasms was 3.5 days on vigabatrin versus 13 days on hydrocortisone (P < 0.01). Five patients exhibited side effects on vigabatrin but nine on hydrocortisone (P = 0.006). Vigabatrin should therefore be considered as the first choice treatment for infantile spasms due to tuberous sclerosis. [\hyperlink{Vitrakvi}{PMID: 9095401}, C Chiron et al., 1997]

\hypertarget{pmid_36495716}{P}aediatric clinical practice for treatment of venous thromboembolism (VTE) is based on extrapolation from adult trials with minimal data on anticoagulation efficacy and safety in children. Based on EINSTEIN-Jr clinical trial data, rivaroxaban was approved to treat VTE and prevent its recurrence in children of all ages. To report the safety and efficacy of rivaroxaban use in paediatric VTE and to present real-world data, specifically about very young children. We conducted a retrospective observational study at Birmingham Children's Hospital. Data were collected from patients <16 years old who received rivaroxaban after its licensure in the period between March 2021 and June 2022. Rivaroxaban was used for treatment of acute VTE in 64 patients. Thrombosis was CVC-related in 26 patients, unprovoked in 3, while the rest had one or more risk factors for VTE. Safety and efficacy of rivaroxaban were assessed in 52 patients after excluding patients who were on current rivaroxaban treatment and those who were lost to follow up or stopped rivaroxaban due to intolerance. No bleeding events were reported, and recurrence of thrombosis occurred in only 3.6 \%. About 35 \% had normalised re-imaging, 40.3 \% improved, 9.6 \% were unchanged and 11.5 \% stopped rivaroxaban without re-imaging. Rivaroxaban was used for secondary VTE prophylaxis in 6 patients in our cohort with no recurrence of thrombosis or bleeding reports. Our real-world experience confirmed that rivaroxaban was well tolerated, effective and safe. Further real-world data and observational studies are essential to investigate the use of rivaroxaban among different risk groups. [\hyperlink{Vitrakvi}{PMID: 36495716}, Eman Hassan et al., 2023]

\hypertarget{pmid_9212476}{M}ivacurium is considered a relaxant suitable for tracheal intubation in children due to its rapid onset. We compared the neuromuscular effects of mivacurium, with and without priming, in children undergoing elective surgery during halothane anesthesia. Forty pediatric patients (2-10 yr, ASA class I) were randomly into 2 groups and studied under halothane anesthesia. The non-priming group (n = 20) received mivacurium 0.25 mg/kg, and the priming group (n = 20) received a priming dose of mivacurium 0.025 mg/kg, followed by an intubating dose of 0.225 mg/kg 3 min later. Thenar Electromyogram responsive to supramaximal train-of-four stimulation of the ulnar nerve at 12 s intervals was used as neuromuscular monitoring. The onset time in the priming group was significantly faster than in the non-priming group (1.04 min vs. 1.7 min). The mean time from injection of intubating dose to spontaneous recovery to 25\%, 50\% and 75\% twitch were not influenced by priming technique. Side effects, such as cutaneous flushing and hypotension, were unremarkable at this dose in children. Priming technique can significantly accelerates the onset of mivacurium in the pediatric patients under halothane anesthesia. [\hyperlink{Vitrakvi}{PMID: 9212476}, Y C Chu et al., 1997]

\hypertarget{pmid_11916787}{I}n this randomized, double-blinded, placebo-controlled study, we evaluated the safety and efficacy of lidocaine iontophoresis for the prevention of pain associated with venipuncture in 59 children aged 6-17 yr. Children received either lidocaine HCl 2\% with epinephrine 1:100,000 (Active) or the same formulation without lidocaine (Placebo) via a 20 mA/min iontophoretic treatment. Pain during venipuncture was assessed by the subject, parent, and nurse using a 100-mm visual analog scale. Median (interquartile range) visual analog scale scores were significantly lower in the Active versus Placebo groups: subject, 7.0 (2.0-20.8) versus 31.0 (12.0-48.0), P < 0.001; nurse, 5.0 (2.2-10.8) versus 24.0 (9.0-47.0), P < 0.001; and parent, 3.0 (0.8-7.2) versus 20.0 (4.5-43.0), P < 0.002, respectively. Similarly, higher median satisfaction scores were given to the Active versus Placebo group by the three evaluators. Of the 59 subjects completing the study, 10 subjects experienced a total of 12 adverse events that were all graded as mild. In conclusion, lidocaine iontophoresis is safe in children, reduces discomfort associated with venipuncture, and increases satisfaction when compared with the placebo. In this randomized, double-blinded, placebo-controlled study, we found that dermal anesthesia with lidocaine HCl 2\% combined with epinephrine 1:100,000 administered via iontophoresis in children is achieved in 8.8 +/- 2.1 min, reduces discomfort associated with venipuncture, is safe, and increases satisfaction when compared with the placebo. [\hyperlink{Vitrakvi}{PMID: 11916787}, John B Rose et al., 2002]

\hypertarget{pmid_11118795}{I}n many countries, vigabatrin is now recommended as the first choice of treatment for infantile spasms instead of steroids. The aim of this study was to review the efficacy and side effects of the two drugs, steroids and vigabatrin, by using data from published series. Results suggest that vigabatrin certainly is efficacious in the treatment of the disorder but, on the whole, it does not seem to be any more effective than steroids, especially corticotrophin, even in children with tuberous sclerosis. The possible benefits of vigabatrin do not justify the risks of the possible irreversible visual changes associated with vigabatrin. [\hyperlink{Vitrakvi}{PMID: 11118795}, R S Riikonen et al., 2000]

\hypertarget{pmid_28827252}{C}eftriaxone is widely used in children in the treatment of sepsis. However, concerns have been raised about the safety of ceftriaxone, especially in young children. The aim of this review is to systematically evaluate the safety of ceftriaxone in children of all age groups. MEDLINE, PubMed, Cochrane Central Register of Controlled Trials, EMBASE, CINAHL, International Pharmaceutical Abstracts and adverse drug reaction (ADR) monitoring systems will be systematically searched for randomised controlled trials (RCTs), cohort studies, case-control studies, cross-sectional studies, case series and case reports evaluating the safety of ceftriaxone in children. The Cochrane risk of bias tool, Newcastle-Ottawa and quality assessment tools developed by the National Institutes of Health will be used for quality assessment. Meta-analysis of the incidence of ADRs from RCTs and prospective studies will be done. Subgroup analyses will be performed for age and dosage regimen. Formal ethical approval is not required as no primary data are collected. This systematic review will be disseminated through a peer-reviewed publication and at conference meetings. CRD42017055428. [\hyperlink{Vitrakvi}{PMID: 28827252}, Linan Zeng et al., 2017]

\hypertarget{pmid_8552215}{I}n an retrospective uncontrolled long-term study in 30 children with intractable epilepsy, it was found that treatment with vigabatrin resulted in a seizure reduction of more than 50\% at 1-year follow-up in 40\% of the children. The responders were all children with partial seizures. Side effects were mild and did not lead to discontinuation of the drug. Increased numbers of seizures were seen in three cases. A moderate weight increase was seen in 27\% of the children. At 5-year follow-up 7 children (23\%) still maintained a seizure reduction of more than 50\%. Trials of monotherapy in three seizure-free patients were unsuccessful. No further side effects were observed. A study of evoked potentials in 12 children showed no alteration in latency and amplitudes of VEP following treatment with vigabatrin. Our results show that in children vigabatrin seems to have a stable effect even though a few children may experience a breakthrough of seizures. The presented results together with previous reports on MRI-scans seem to indicate that even in children with a still maturing CNS vigabatrin is a safe drug. [\hyperlink{Vitrakvi}{PMID: 8552215}, P Uldall et al., 1995]

\hypertarget{pmid_16781498}{C}hildren frequently suffer infections accompanied by fever, which is commonly treated with acetaminophen (paracetamol), a use not devoid of risk. The effect of a complex homeopathic medicine (Viburcol, Heel Belgium, Gent, Belgium) was compared with that of acetaminophen in children with infectious fever. Non-randomized observational study. Thirty-eight Belgian centers practicing homeopathy and conventional medicine. Children <12 years old. Viburcol (drops) or acetaminophen (pills, capsules, or liquid form) for a maximum of 2 weeks. Fever, cramps, distress, disturbed sleep, crying, and difficulties with eating or drinking. Symptoms were graded by the practitioner on a scale from 0 to 3. Severity of infection was evaluated on a scale from 0 to 4. Data were captured on body temperature, subjective impression of health status, time to first improvement of symptoms, and global evaluation of treatment effects. Tolerability and compliance were monitored. Both treatment groups improved during the treatment period. Body temperature was reduced by 1.7 degrees C +/- 0.7 degrees C with Viburcol and by 1.9 degrees C +/- 0.9 degrees C with acetaminophen; fever score (scale from 0 to 3) from 1.7 +/- 0.6 to 0.1 +/- 0.2 with Viburcol and from 1.9 +/- 0.7 to 0.2 +/- 0.5 with acetaminophen (all values mean +/-SD). The overall severity of infection (scale from 0 to 4) decreased from 2.0 +/- 0.5 to 0.0 +/- 0.2 with Viburcol and from 2.2 +/- 0.7 to 0.2 +/- 0.6 with acetaminophen. There were no statistically significant differences between treatment groups in time to symptomatic improvement. Viburcol was noninferior to acetaminophen on all variables evaluated. Both treatments were very well tolerated, but the Viburcol group had a significantly greater number of patients with the highest tolerability score. In this patient population, Viburcol was an effective alternative to acetaminophen treatment and significantly better tolerated. [\hyperlink{Vitrakvi}{PMID: 16781498}, Mireille Derasse et al., 2005]

\hypertarget{pmid_19215279}{T}he use of vigabatrin (VGB) as an antiepileptic drug (AED) has been limited by evidence showing that it causes vigabatrin-attributed visual field loss (VAVFL) in at least 20-40\% of patients exposed at school age or later. VGB is an effective drug for infantile spasms, but there are no reports on later visual field testing after such treatment. Our aim was to investigate the risk of VAVFL in school-age children who had received VGB in infancy. Visual fields of 16 children treated with VGB for infantile spasms were examined by Goldmann kinetic perimetry at age 6-12 years. Normal fields were defined as the temporal meridian extending to more than 70 degrees , and mild VAVFL between 50 and 70 degrees . Abnormal findings were always confirmed by repeating the test. Exposure data were collected from hospital charts. Vigabatrin was started at a mean age of 7.6 (range, 3.2-20.3) months. The mean duration of therapy was 21.0 (9.3-29.8) months and cumulative dose 655 g (209-1,109 g). Eight children were never treated with other AEDs, five received only adrenocorticotropic hormone (ACTH) in addition to VGB, and three children had been treated with other AEDs. Fifteen children had normal visual fields. Mild VAVFL was observed in one child (6\%) who had been treated with VGB for 19 months and who received a cumulative dose of 572 g. The risk of VAVFL may be lower in children who are treated with VGB in infancy compared to patients who receive VGB at a later age. [\hyperlink{Vitrakvi}{PMID: 19215279}, Eija Gaily et al., 2009]

\hypertarget{pmid_32935597}{A}nticoagulant therapy is in use for both prevention and treatment of venous and arterial thromboembolic disorders. Delivering safe and effective anticoagulation in the pediatric population is challenging, since the available standard therapy with parenteral UFH and LMWH is troublesome for most pediatric patients, and VKAs require frequent INR monitoring due to the unpredictable pharmacokinetics and numerous food and drug interactions. Rivaroxaban, a direct FXa inhibitor, offers the convenience of oral administration and predictable pharmacokinetics across a wide range of patients. Its safety and efficacy have been previously established in various adult indications. This review outlines pharmacologic and clinical aspects regarding rivaroxaban treatment in adults and children, and provides a broad appraisal of the The EINSTEIN-Jr program which evaluated the safety and efficacy of body-weight adjusted pediatric rivaroxaban regimens for the treatment of VTE in children. A review of the literature using the keywords rivaroxaban and pediatric venous thromboembolism was conducted within the National Center for Biotechnology (NCBI) and EMBASE databases. Rivaroxaban represents an appealing therapeutic alternative for VTE in children. Further research should explore additional indications for rivaroxaban in the pediatric population beyond that of VTE. [\hyperlink{Vitrakvi}{PMID: 32935597}, Omri Cohen et al., 2020]

\hypertarget{pmid_8694155}{W}e assessed neuromuscular blocking effects and tracheal intubation conditions following mivacurium in 121 anaesthetised children aged 1-10 years. The study was conducted in three parts. Parts 1 and 2 were undertaken during thiopentone-alfentanil-nitrous oxide anaesthesia; neuromuscular blockade was evaluated by recording the force of contraction of the adductor pollicis in response to train-of-four stimulation at 0.1 Hz. In part 1 the potency of mivacurium was determined in 15 children using a single dose-response technique; in part 2 onset and recovery times were determined in six children following a dose of mivacurium 0.2 mg.kg-1. In part 3 of the study, clinical intubation conditions were assessed in two groups of 50 children whose tracheas were intubated 60 or 90 s after injection of mivacurium 0.2 mg.kg-1 during thiopentone-nitrous oxide anaesthesia. The ED50 and ED95 of mivacurium were 54 and 105 micrograms.kg-1 respectively. The times to 90\% and 100\% depression of control twitch were 1.3 (0.3) and 1.9 (0.5) min; times to 5\%, 25\%, 75\% and 90\% recovery were 6.4 (1.0), 8.4 (1.1), 12.5 (1.1) and 14.4 (1.9) min, respectively. Intubation conditions were rated satisfactory in 33/50 children (0.66, 95\% confidence interval 0.51-0.79) at 60 s and in 49/50 children (0.98, 95\% confidence interval 0.89-1.0) at 90 s (p = 0.0001). Intubation conditions 90 s after mivacurium 0.2 mg.kg-1 were significantly better than those obtained in 10 patients given anaesthetic drugs alone (p = 0.002). [\hyperlink{Vitrakvi}{PMID: 8694155}, A McCluskey et al., 1996]

\hypertarget{pmid_11242620}{T}he use of adrenocorticotrophic hormone (ACTH) and prednisolone in the management of infantile spasms has been well established, but is associated with significant morbidity and cannot be used as long-term medication. Since the introduction of vigabatrin, results have been promising with suggestions currently that it should be used as first-line management of infantile spasms. The aim of this study was to establish the efficacy, tolerability and problems associated with the use of vigabatrin, in Asian children with infantile spasms. Eighteen local Asian patients with infantile spasms were given vigabatrin, in 8 as first-line monotherapy and in 10 patients as add-on therapy to pre-existing anti-convulsants. Thirty-nine per cent (7/18) showed 100\% suppression of seizures at 2 weeks and 50\% (9/18) showing complete suppression of seizures at 4 weeks after starting therapy. There was however, a high relapse rate (56\%) in complete responders within the first 6 months of therapy. This was probably due to a lower maintenance dose in the patients, as those who relapsed were on an average dose of 59 mg/kg/day at the time of relapse and responded to a stepped up dose of 83 mg/kg/day. Vigabatrin was well tolerated and only 2 patients developed somnolence and irritability. A dose of at least 70 mg/kg/day may be necessary to achieve adequate control and yet avoid the common adverse affects of vigabatrin. [\hyperlink{Vitrakvi}{PMID: 11242620}, S K Tay et al., 2001]

\hypertarget{pmid_17288611}{P}rocedural pain relief is sub-optimal in neonates. Topical tetracaine provides pain relief in children. Evidence of its efficacy and safety in neonates is limited. The objective of this study was to assess the efficacy and safety of topical tetracaine on the pain response of neonates during a venipuncture. Medically stable infants greater than or equal to 24 weeks gestation, requiring a venipuncture, were included. Following randomization and double blinding, 1.1 g of tetracaine or placebo was applied to the skin for 30 minutes. Participants received oral sucrose if they met local eligibility criteria. The venipuncture was performed according to a standard protocol. A medium effect size in the pain score (corresponding to about 2 point difference in the PIPP score) was considered clinically significant, leading to a sample size of 142 infants, with 80\% statistical power. Local skin reactions and immediate adverse cardiorespiratory events were noted. The primary outcome, PIPP score at 1 minute, was analysed using an independent Student's t-test. One hundred and forty two infants were included, 33 +/- 4 weeks gestation, 2100 +/- 900 grams and 6 +/- 3 days of age. There was almost no difference in PIPP scores at 1 minute between groups (mean difference -0.09; 95\% confidence interval [CI]: -1.68 to 1.50; P = . 91). Similarly, there were no differences in PIPP scores during the 2nd, 3rd and 4th minute. Duration of cry did not differ between the groups (median difference, 0; 95\% CI, -3 to 0; P = . 84). The majority of infants in both groups received sucrose 24\%. Sucrose had a significant effect on the PIPP score, as assessed by an ANOVA model (p = 0.0026). Local skin erythema was observed transiently in 11 infants (7 in the tetracaine and 4 in the placebo group). No serious side effect was observed. Tetracaine did not significantly decrease procedural pain in infants undergoing a venipuncture, when used in combination with routine sucrose administration. [\hyperlink{Vitrakvi}{PMID: 17288611}, Brigitte Lemyre et al., 2007]

\hypertarget{pmid_30192381}{V}igabatrin is an antiepileptic drug indicated as monotherapy in infantile spasms. However, the pharmacokinetic profile of this compound in infants and young children is still poorly understood, as is the minimal effective dose, critical information given the risk of exposure-related retinal toxicity with vigabatrin. A reasonable approach to determining this minimal dose would be to identify the lowest dose providing a low risk of exposure overlap with the 36-mg/kg dose, which is the highest dose associated with an increased risk for treatment failure, based on randomized dose-ranging data. A population pharmacokinetic model was consequently developed from 28 children (aged 0.4-5.7 years) for the active S(+)-enantiomer, using Monolix software. In parallel, a population model was developed from published adult data and scaled to children using theoretical allometry and maturation of the renal function. A one-compartment model with zero-order absorption and first-order elimination described the pediatric data. Mean population estimates (percentage interindividual variability) for the apparent clearance, apparent distribution volume, and absorption duration were 2.36 L/h (24.5\%), 17 L (38\%), and 0.682 hours, respectively. Apparent clearance and apparent distribution volume were related to body weight by empirical allometric equations. Monte Carlo simulations evidenced that a daily dose of 80 mg/kg should minimize exposure overlap with the 36-mg/kg dose. Similar results were obtained for the adult model scaled to children. Consequently, a minimal effective dose of 80 mg/kg/day could be considered for patients with infantile spasms. [\hyperlink{Vitrakvi}{PMID: 30192381}, Marwa Ounissi et al., 2019]

\hypertarget{pmid_9713224}{T}he aim of this study was to evaluate the clinical use of mivacurium, a short-acting, non-depolarising muscle relaxant, in the paediatric population in Singapore. Twenty children between the ages of 2 and 12 years were given mivacurium to maintain neuromuscular blockade during nitrous oxide-halothane anaesthesia. Reversal from neuromuscular blockade was spontaneous. The onset, ease of intubation after different doses of mivacurium, and the ease of reversal were evaluated. Different intubating doses of mivacurium did not result in significantly different times of onset. The mean recovery index (25\% to 75\% recovery) was 4.1 minutes. There were no adverse reactions. Mivacurium provided rapid and efficacious onset of neuromuscular blockade in the local paediatric population. Rapid spontaneous recovery obviated the need for reversal agents. [\hyperlink{Vitrakvi}{PMID: 9713224}, H L Chee et al., 1998]

\hypertarget{pmid_24910743}{A}pproximately one-third of all children with epilepsy do not achieve complete seizure improvement. This study evaluated the efficacy of Vigabatrin in children with intractable epilepsy. From November 2011 to October 2012, 73 children with refractory epilepsy (failure of seizure control with the use of two or more anticonvulsant drugs) who were referred to the Children's Medical Center and Mofid Children's Hospital were included in the study. The patients were treated with Vigabatrin in addition to their previous medication, and followed-up after three to four weeks to determine the daily frequency, severity, and duration of seizures in addition to any reported side effects. Of the 67 children, 41 (61.2\%) were males and 26 (38.8\%) females, their age ranging from three months to 13 years with an average of 3.1 [standard deviation (SD), 2.6] years. The mean daily frequency of seizures at baseline was 6.61 (SD, 5.9) seizures per day. Vigabatrin reduced the seizure frequency ≤2.9 (SD, 5.2) (56\% decline) and 3.0 (SD, 5.3) (54.5\% decline) per day after three and six months of treatment, respectively. A significant difference was observed between seizure frequencies at three (P<0.001) and six months (P<0.001) after Vigabatrin initiation compared with the baseline. Somnolence [3 (4.5\%)], horse laugh [1 (1.5\%)], urinary stones [1 (1.5\%)], increased appetite [1 (1.5\%)], and abnormal electroretinographic pattern [3 (4.5\%)] were the most common side effects in our patients. This study confirms the short-term efficacy and safety of Vigabatrin in children with refractory epilepsies. [\hyperlink{Vitrakvi}{PMID: 24910743}, Mohammad-Mahdi Taghdiri et al., 2013]

\hypertarget{pmid_27011634}{T}o report the effectiveness and safety of intravenous (IV) levetiracetam (LEV) in the treatment of critically ill children with acute repetitive seizures and status epilepticus (SE) in a children's hospital. We retrospectively analyzed data from children treated with IV LEV. The mean age of the 108 children was 69.39 ± 46.14 months (1-192 months). There were 58 (53.1\%) males and 50 (46.8\%) females. LEV load dose was 28.33 ± 4.60 mg/kg/dose (10-40 mg/kg). Out of these 108 patients, LEV terminated seizures in 79 (73.1\%). No serious adverse effects were observed but agitation and aggression were developed in two patients, and mild erythematous rash and urticaria developed in one patient. Antiepileptic treatment of critically ill children with IV LEV seems to be effective and safe. Further study is needed to elucidate the role of IV LEV in critically ill children. [\hyperlink{Vitrakvi}{PMID: 27011634}, Faruk Incecik et al., ]

\hypertarget{pmid_7925171}{T}he antiepileptic effect of vigabatrin (gamma-vinyl GABA, VGB) in children has been demonstrated in controlled and open studies. According to the literature, results were good to excellent in partial seizures (with and without becoming secondarily generalized) and promising in infantile spasms (IS). In patients with myoclonic epilepsies of early childhood and especially those with Lennox-Gastaut syndrome (LGS), the effect of VGB has been investigated only to a limited extent and the pattern of response was variable. The present open, add-on, dose-ranging study was initiated to assess the long-term effect and safety of VGB in a cohort of 20 children with LGS who were not responding sufficiently to first-line drug monotherapy with valproate (VPA) instead of adding classical second-line antiepileptic drugs [AEDs: benzodiazepines (BZD), phenobarbital (PB), primidone (PRM)], which usually are associated with rapid diminution of their antiepileptic properties and a high frequency of side effects. Eighty-five percent of children experienced a 50-100\% reduction in seizure frequency, even after dose reduction of VPA. No serious side effects occurred except in 1 patient who experienced dyskinesia. Mood changes, sedation, ataxia, and hypersalivation, well-known complications of other AEDs, were not observed. [\hyperlink{Vitrakvi}{PMID: 7925171}, M Feucht et al., ]

\hypertarget{pmid_16257311}{F}ive independent, multicentered, double-masked, parallel, controlled studies were conducted to determine the safety of moxifloxacin ophthalmic solution 0.5\% (VIGAMOX) in pediatric and nonpediatric patients with bacterial conjunctivitis. Patients were randomized into one of two treatment groups in each study and received either moxifloxacin ophthalmic solution 0.5\% b.i.d. or t.i.d. or a comparator. A total of 1,978 patients (918 pediatric and 1,060 nonpediatric) was evaluable for safety. The most frequent adverse event in the overall safety population was transient ocular discomfort, occurring at an incidence of 2.8\%, which was similar to that observed with the vehicle. No treatment-related changes in ocular signs or visual acuity were observed with moxifloxacin ophthalmic solution 0.5\%, except for one clinically relevant change in visual acuity. Thus, based upon a review of adverse events and an assessment of ocular parameters, moxifloxacin ophthalmic solution 0.5\% formulated without the preservative, benzalkonium chloride, is safe and well tolerated in pediatric (3 days-17 years of age) and nonpediatric (18-93 years) patients with bacterial conjunctivitis. [\hyperlink{Vitrakvi}{PMID: 16257311}, Lewis H Silver et al., 2005]

\section*{Vyndaqel}
\subsection*{Result}
\subsubsection*{Answer}

Unknown

\subsubsection*{{Explanation}}
\hypertarget{Vyndaqel}
After reviewing all available abstracts, there is no evidence from any abstract that specifically addresses the safety of Vyndaqel (tafamidis) in children. None of the abstracts mention Vyndaqel, tafamidis, or studies evaluating its safety or efficacy in pediatric populations of any age range. Therefore, based on the abstracts provided, the safety of Vyndaqel in children is unknown, as no targeted pediatric safety studies are available.

\subsection*{Abstracts}
\hypertarget{pmid_20434100}{T}hiamylal is widely used for procedural sedation in emergency departments (ED); however, there are limited safety data for doses of thiamylal > 5 mg/kg in children. We investigated whether intravenous thiamylal in combination with local anesthetics is safe and effective for pediatric procedural sedation in the ED and to identify the association between increasing doses thiamylal and adverse events. Between July 2004 and June 2008, 227 children who underwent procedural sedation met the inclusion criteria, including 105 males (46.3\%) and 122 females (53.7\%). Facial laceration was the most common indication for procedural sedation. All children received an intravenous injection of thiamylal, with a loading dose of 5 mg/kg. Eighty-one children (35.7\%) received a supplemental dose of 2.5 mg/kg thiamylal because of inadequate sedation. Of these, 27 (11.9\%) received a second supplemental dose of 2.5 mg/kg because of inadequate sedation. Sixty-six patients (29.1\%) experienced 75 mild and self-resolving adverse events, and most of which (15/75; 20\%) were drowsiness. Four (1.8\%) patients experienced oxygen saturation below 96\%, which was related to the supplemental dose of thiamylal (p = 0.002). No children suffered from any lasting or potentially serious complications. Our results indicate that intravenous thiamylal in combination with local anesthetic infiltration is a well tolerated for therapeutic procedures in the ED. Thiamylal offers rapid onset of sedation without compromising the patient's cardiorespiratory function during pediatric procedural sedation. [\hyperlink{Vyndaqel}{PMID: 20434100}, Ching-Kuo Lin et al., 2010]

\hypertarget{pmid_16781498}{C}hildren frequently suffer infections accompanied by fever, which is commonly treated with acetaminophen (paracetamol), a use not devoid of risk. The effect of a complex homeopathic medicine (Viburcol, Heel Belgium, Gent, Belgium) was compared with that of acetaminophen in children with infectious fever. Non-randomized observational study. Thirty-eight Belgian centers practicing homeopathy and conventional medicine. Children <12 years old. Viburcol (drops) or acetaminophen (pills, capsules, or liquid form) for a maximum of 2 weeks. Fever, cramps, distress, disturbed sleep, crying, and difficulties with eating or drinking. Symptoms were graded by the practitioner on a scale from 0 to 3. Severity of infection was evaluated on a scale from 0 to 4. Data were captured on body temperature, subjective impression of health status, time to first improvement of symptoms, and global evaluation of treatment effects. Tolerability and compliance were monitored. Both treatment groups improved during the treatment period. Body temperature was reduced by 1.7 degrees C +/- 0.7 degrees C with Viburcol and by 1.9 degrees C +/- 0.9 degrees C with acetaminophen; fever score (scale from 0 to 3) from 1.7 +/- 0.6 to 0.1 +/- 0.2 with Viburcol and from 1.9 +/- 0.7 to 0.2 +/- 0.5 with acetaminophen (all values mean +/-SD). The overall severity of infection (scale from 0 to 4) decreased from 2.0 +/- 0.5 to 0.0 +/- 0.2 with Viburcol and from 2.2 +/- 0.7 to 0.2 +/- 0.6 with acetaminophen. There were no statistically significant differences between treatment groups in time to symptomatic improvement. Viburcol was noninferior to acetaminophen on all variables evaluated. Both treatments were very well tolerated, but the Viburcol group had a significantly greater number of patients with the highest tolerability score. In this patient population, Viburcol was an effective alternative to acetaminophen treatment and significantly better tolerated. [\hyperlink{Vyndaqel}{PMID: 16781498}, Mireille Derasse et al., 2005]

\hypertarget{pmid_16148661}{W}e compare the efficacy and safety profile of oral midazolam and continuous flow 50\% nitrous oxide (N(2)O) for alleviating anxiety and pain during voiding cystourethrography (VCU) in children. This prospective, randomized clinical trial study was conducted in the radiology unit of a tertiary care center. Children older than 3 years scheduled for VCU were given either 0.5 mg/kg midazolam orally or continuous flow 50\% N(2)O. Main outcomes were degree of anxiety and pain as assessed by the attending nurse and radiologist performing the test using a behavioral anxiety score, a distress score and an overall satisfaction score, side effects and recovery profile. The study included 47 children (89\% girls) with a mean age of 6 years (range 3 to 15). There were 24 subjects in the midazolam group and 23 in the N(2)O group. Midazolam and N(2)O provided adequate anxiety and pain relief to perform the examination, yet children given N(2)O required less restraining and experienced a significantly shorter recovery time (29 +/- 10 vs 63 +/- 25 minutes, p <0.001). Continuous flow 50\% nitrous oxide and oral midazolam are comparably safe and effective in reducing anxiety and distress during VCU in children older than 3 years. However, N(2)O provides a more rapid onset of sedating effect and has a shorter recovery time. [\hyperlink{Vyndaqel}{PMID: 16148661}, Ilan Keidan et al., 2005]

\hypertarget{pmid_9239339}{T}wo hundred and seven children undergoing either intravenous pyelography (i.v.p.) or voiding cystourethrography (VCUG) were examined. Under the age of 5 years, children received intra-rectal midazolam (0.5 mg/kg) with a maximum of 5 mg. Children over 5 years, self-inhaled an equimolar mixture of oxygen and nitrous oxide. Pain and stress were evaluated in children under 5, by the pediatric radiologist according to the 4 non verbal items of the Le Baron-Zeltzer scale and in children over 5 by the child himself with a visual analogic scale. Under 5 years of age, midazolam significantly reduced pain and stress during i.v.p. (p < 0.0001), VCUG both in boys (p < 0.0001) and girls (p < 0.0001). In children over 5, nitrous oxide inhalation reduced pain during i.v.p. (p = 0.0004), during VCUG in girls (p = 0.0025), but not in boys ((p > 0.05). Pediatric radiologists should evaluate pain and stress in their patients as they can be easily and safely limited. [\hyperlink{Vyndaqel}{PMID: 9239339}, P Schmit et al., 1997]

\hypertarget{pmid_11893636}{I}odixanol (Visipaque) is a dimeric, non-ionic iodinated contrast medium that is isotonic with blood at all clinically relevant concentrations. Iodixanol was compared in a randomized, double blind, parallel group, phase III multicentre trial with a monomeric, non-ionic contrast medium, iohexol (Omnipaque), at two concentrations assessing safety, tolerability and radiographic efficacy during contrast enhanced gastrointestinal radiography examinations of children. 154 children entered the trial; 152 formed the safety population and 147 the efficacy population. All examinations were performed following standard departmental practice. Children were assigned into either a high or low concentration group (iodixanol, 150 mgI ml(-1) and 320 mgI ml(-1) vs iohexol, 140 mgI ml(-1) and 300 mgI ml(-1)). The primary outcome measure for efficacy was the overall quality of visualization, which was assessed using a 100 mm visual analogue scale (VAS). The secondary efficacy variables assessed were quality of contrast opacification, mucosal coating and overall quality of diagnostic information. Safety evaluation involved patient follow-up for at least 48 h. Taste acceptance was also assessed. There was no statistically significant difference between the two contrast media with regard to the primary and secondary efficacy variables assessed, although higher ratings were observed for iodixanol. The 100 mm VAS score overall was 86 mm for iodixanol and 82 mm for iohexol (95\% confidence interval -2.56, 10.42). The frequency of adverse events was lower for patients receiving iodixanol. Adverse events, mainly diarrhoea, occurred in 12 patients (16.2\%) in the iodixanol group and 28 patients (35.9\%) in the iohexol group. This reached statistical significance (p=0.006). Overall, iodixanol is well suited for examinations of the gastrointestinal tract, giving good efficacy results and fewer adverse events than iohexol. [\hyperlink{Vyndaqel}{PMID: 11893636}, N B Wright et al., 2002]

\hypertarget{pmid_30662387}{T}o evaluate the efficacy and safety of fentanyl for sedation therapy in mechanically ventilated children. This was a double-blind, randomized controlled trial of mechanically ventilated patients between 2 months and 18 years of age. Patients were randomly divided into two groups; the control group with midazolam alone, and the combination group with both fentanyl and midazolam. The sedation level was evaluated using the Comfort Behavior Scale (CBS), and the infusion rates were adjusted according to the difference between the measured and the target CBS score. Forty-four patients were recruited and randomly allocated, with 22 patients in both groups. The time ratio of cumulative hours with a difference in CBS score (measured CBS-target CBS) of ≥ 4 points (i.e., under-sedation) was lower in the combination group (median, 0.06; interquartile range [IQR], 0-0.2) than in the control group (median, 0.15; IQR, 0.04-0.29) ( Fentanyl combined with midazolam is safe and more effective than midazolam alone for sedation therapy in mechanically ventilated children. ClinicalTrials.gov Identifier: NCT02172014. [\hyperlink{Vyndaqel}{PMID: 30662387}, Bongjin Lee et al., 2019]

\hypertarget{pmid_20470336}{P}ain, anxiety and fear of needles make intravenous cannulation extremely difficult in children. We assessed the efficacy and safety of oral midazolam and a low-dose combination of midazolam and ketamine to reduce the stress and anxiety during intravenous cannulation in children undergoing computed tomography (CT) imaging when compared to placebo. Ninety-two ASA I or II children (1-5 years) scheduled for CT imaging under sedation were studied. Children were randomized to one of the three groups. Group M received 0.5 mg x kg(-1) midazolam in 5 ml of honey, group MK received 0.25 mg x kg(-1) midazolam mixed with 1 mg x kg(-1) ketamine in 5-ml honey and group P received 5-ml honey alone, orally. In 20-30 min after premedication, venipuncture was attempted at the site of eutectic mixture of local anesthetics cream. Sedation scores and venipuncture scores were recorded. Primary outcome of the study was incidence of children crying at venipuncture (venipuncture score of 4). Significantly more children cried during venipuncture in placebo group compared to the other two groups (19/32 (59\%) in group P vs 1 each in groups M and MK, (P < 0.001) (RR 2.37, 95\% CI 1.55-3.63). In 20-30 min after premedication, group P had more children in sedation score 1 or 2 (crying or anxious) compared to the other two groups (P < 0.05). At this time, group MK showed more children in calm and awake compared to group M (P = 0.02). At venipuncture, group P had more children in venipuncture score 3 or 4 (crying or withdrawing) compared to group M or MK (P < 0.05), while groups M and MK were comparable. A low-dose combination of oral midazolam and ketamine or oral midazolam alone effectively reduces the stress during intravenous cannulation in children undergoing CT imaging without any adverse effects. However, the combination provides more children in calm and quiet state when compared to midazolam alone at venipuncture. [\hyperlink{Vyndaqel}{PMID: 20470336}, Kajal Jain et al., 2010]

\hypertarget{pmid_17639417}{P}ain is the most common discomfort experienced by children undergoing major operations. It is most often not adequately treated because of inexperience and unfounded fears related to the use of opioid drugs. In adults, patient-controlled analgesia (PCA) is widely administered, while in children, its use with opioid drugs is still under evaluation for safety and efficacy. The objective of the study is to evaluate the safety and efficacy of an opioid drug (fentanil) administered by PCA associated with a sedative-adjuvant drug (midazolam) administered by continuous infusion in children having undergone major neurosurgical procedures. Sixteen children with moderate to severe postoperative pain were treated with fentanil by PCA (booster doses of 1 microg/kg) plus continuous infusion of midazolam (2 microg/kg per min) by an intravenous route. To evaluate safety and efficacy of this analgesic protocol, different subjective and objective parameters were monitored at 4-h intervals. In addition, patients' satisfaction was assessed by a questionnaire at the end of the treatment. All children experienced a good degree of analgesia and did not require any other analgesic drug during the treatment. Both subjective and objective parameters improved after starting pain-relieving treatment, and no major side effects occurred. The analysis of the answers of the questionnaire administered to the children showed a high grade of satisfaction. PCA with fentanil plus continuous infusion of midazolam is a safe and efficacious method for analgesia in children with moderate to severe postoperative neurosurgical pain. The association of midazolam to fentanil also contributes to control anxiety and stress in this subset of patients and does not show any important side effects. [\hyperlink{Vyndaqel}{PMID: 17639417}, Antonio Chiaretti et al., 2008]

\hypertarget{pmid_25535540}{P}roviding a safe and efficient dental treatment for a young patient is a challenge for the dentist and the child. The purpose of this study was to investigate the effectiveness, safety and acceptability of buccal midazolam in dental pediatric patients and to compare it with oral Midazolam. Eighteen uncooperative healthy children aged 2.5-6 years were randomized to each of buccal midazolam (0.3mg/kg) or oral midazolam (0.5mg/kg) at the first visit, the alternative has been used at the second visit in a cross-over manner. The study took place at pediatric dentistry clinic of Shahed University, Tehran, from November 2011 to June 2012. The patients' vital signs and behavioral scores were recorded. The patient, the operator and the observer were blinded to the applied medication. Post operatively, patients' and parents' satisfaction were assessed by Visual Analogue Score and a questionnaire respectively. The P-value was set at 0.05 for significance level. There were no significant differences in physiologic factors in the medication groups at time 0, 10, 20, 30 minutes and discharge. There was also no significant difference between the two groups in behavioral parameters. The majority of parents rated both sedative agents as "effective" or "very effective" and their children mostly were without anxiety or with minor anxiety. Buccal midazolam may be safely and efficiently used in sedation of pediatric dental patients. [\hyperlink{Vyndaqel}{PMID: 25535540}, Sara Tavassoli-Hojjati et al., 2014]

\hypertarget{pmid_20180109}{V}oiding cystourethrogram (VCUG) is a common procedure at pediatric tertiary care centres that can be painful as it involves a urinary catheter. Currently there are no widely utilized protocols for non-topical medications to decrease pain that children feel during catheterization. To determine if intranasal (IN) fentanyl is effective at decreasing pain that children feel during catheterization of VCUG when compared with sterile water. We performed a double-blind randomized controlled trial, using IN fentanyl (2 microg/kg) compared to placebo (sterile water,) in children 4-8 years of age scheduled for elective VCUG in one urban pediatric tertiary center. Using the Face Pain Score-Revised, children receiving IN fentanyl scored 2.58 (1.93-3.25 95\% CI) while those receiving sterile water scored 2.86 (2.20-3.51 95\% CI) showing no statistically significant difference. There were no adverse events. Although we were unable to show a statistically significant difference between our study and control groups, we believe that this may be due to technique (positioning, delivery device) and timing of administration of IN fentanyl as well as multi-factorial causes of distress during VCUG. Future studies investigating alternative delivery techniques of IN fentanyl for analgesia during VCUG may yield more promising results. [\hyperlink{Vyndaqel}{PMID: 20180109}, Seen Chung et al., 2010]

\hypertarget{pmid_18611612}{T}he safety and efficacy of cefetamet pivoxil, an oral cephalosporin of the third generation, have been studied in open, prospective, randomized comparative, clinical trials including 301 toddlers (children aged 1 to 2 years) with upper and lower respiratory tract infections, and urinary tract infections. Cefetamet pivoxil (CAT) syrup formulation was given to 177 toddlers either in the standard dose of 10 mg/kg b.i.d. [n = 116] or 20 mg/kg b.i.d. [n = 61] and 124 toddlers have been treated with comparator drugs [cefaclor, n = 98; phenoxymethylpenicillin, n = 18; amoxicillin plus clavulanic acid; n = 8]. The treatment period was 7 days mainly, except for pharyngotonsillitis for which the treatment duration was 7 or 10 days. The assessment of treatment was based on clinical signs and symptoms primarily in infections of lower respiratory tract and acute otitis media, whereas in patients with pharyngotonsillitis and acute urinary tract infections the bacteriological findings were the main evaluation criteria. The overall therapeutic outcome was successful in 148 (95.4\%) of the 155 toddlers to whom CAT was administered and in 87 (85.3\%) out of 102 toddlers receiving standard drugs. Adverse events of mild to moderate severity, mainly of gastro-intestinal type (vomiting or diarrhoea) occurred in 14.7\% in the patient group receiving CAT, 11.2\% in the toddlers receiving the standard dose of CAT, and in 12.9\% with the comparator drugs. From the data presented it is concluded that cefetamet pivoxil is efficient and safe in toddlers presenting with community-acquired respiratory and urinary infections mainly caused by S. pneumoniae, H. influenzae, Group A beta-haemolytic streptococci, M. catarrhalis, E. coli, Proteus spp. and K. pneumoniae. [\hyperlink{Vyndaqel}{PMID: 18611612}, A Chibante et al., 1994]

\hypertarget{pmid_11916787}{I}n this randomized, double-blinded, placebo-controlled study, we evaluated the safety and efficacy of lidocaine iontophoresis for the prevention of pain associated with venipuncture in 59 children aged 6-17 yr. Children received either lidocaine HCl 2\% with epinephrine 1:100,000 (Active) or the same formulation without lidocaine (Placebo) via a 20 mA/min iontophoretic treatment. Pain during venipuncture was assessed by the subject, parent, and nurse using a 100-mm visual analog scale. Median (interquartile range) visual analog scale scores were significantly lower in the Active versus Placebo groups: subject, 7.0 (2.0-20.8) versus 31.0 (12.0-48.0), P < 0.001; nurse, 5.0 (2.2-10.8) versus 24.0 (9.0-47.0), P < 0.001; and parent, 3.0 (0.8-7.2) versus 20.0 (4.5-43.0), P < 0.002, respectively. Similarly, higher median satisfaction scores were given to the Active versus Placebo group by the three evaluators. Of the 59 subjects completing the study, 10 subjects experienced a total of 12 adverse events that were all graded as mild. In conclusion, lidocaine iontophoresis is safe in children, reduces discomfort associated with venipuncture, and increases satisfaction when compared with the placebo. In this randomized, double-blinded, placebo-controlled study, we found that dermal anesthesia with lidocaine HCl 2\% combined with epinephrine 1:100,000 administered via iontophoresis in children is achieved in 8.8 +/- 2.1 min, reduces discomfort associated with venipuncture, is safe, and increases satisfaction when compared with the placebo. [\hyperlink{Vyndaqel}{PMID: 11916787}, John B Rose et al., 2002]

\hypertarget{pmid_1557241}{P}remedication for painful procedures in children with cancer is not routinely used. Many medications used are only intermittently effective or require special equipment or anesthesia support. In a randomized, double-blind, crossover study, the safety and efficacy of midazolam, a short-acting benzodiazepine, were compared with the safety and efficacy of fentanyl, a short-acting narcotic analgesic. In 25 children studied, 100\% of children and their parents preferred study drugs to any previous premedication. Seventy-two percent preferred midazolam to fentanyl. Preprocedural anxiety, adverse behavioral symptoms, and visual analog scales all improved and side effects were minimal. It is concluded that premedication for painful procedures should be used routinely in children with cancer. With proper monitoring, fentanyl and midazolam can be used safely in the outpatient clinic setting. Midazolam was found to be the drug of preference for the majority of patients. [\hyperlink{Vyndaqel}{PMID: 1557241}, E S Sandler et al., 1992]

\hypertarget{pmid_12803268}{A}lthough midazolam is commonly given orally to infants and small children for premedication, the taste is sometimes unacceptable even when mixed with syrup. We tested the efficacy and safety of oral fentanyl compared with oral midazolam in a randomized open-label study. Fifty-one children, aged 12-107 months and weighing 10-25 kg, were randomly assigned to fentanyl or midazolam treatment groups. Midazolam (5 mg) or fentanyl (0.1 mg) was given orally from a small bottle with a small orifice 30 min before transfer to the preoperative holding room. The excitation-sedation conditions of the patients were assessed before and after general anaesthesia. The preoperative scores did not differ significantly between the two groups. No major complications were observed in either group. Postoperative vomiting occurred in 5 of 27 (18.5\%) patients treated with oral fentanyl and in none of 24 of those treated with midazolam. Oral administration of fentanyl 30 min before entrance to the holding room for an operation from a bottle with a small orifice is a premedication option for children between 1 and 8 yr of age. [\hyperlink{Vyndaqel}{PMID: 12803268}, M Tamura et al., 2003]

\hypertarget{pmid_7590052}{T}o assess the safety and efficacy of intravenous sedation in pediatric upper endoscopy, all elective outpatient procedures performed during a 2-year period (January 1, 1991 through December 31, 1992) were retrospectively reviewed. Of 614 children, 553 received intravenous meperidine and midazolam; 61 received fentanyl and midazolam. The mean dose of meperidine was 1.5 +/- 0.7 mg/kg and of fentanyl 0.0031 +/- 0.0014 mg/kg. Less midazolam was needed for children receiving fentanyl than for those receiving meperidine (0.05 +/- 0.03 mg/kg versus 0.08 +/- 0.05 mg/kg, p < 002). Recovery time (minutes) was shorter for those receiving fentanyl (74.7 +/- 22.8 versus 95.1 +/- 23.0, p < .003). Side effects occurred in 117 patients (19.1\%), of which the majority were mild (83\%); all were transient with no residual sequelae. Inability to complete the procedure occurred in fewer than 1\%. We conclude that both combinations of medication are safe and effective for children of all ages. The use of fentanyl/midazolam results in a shorter recovery time and a lower dose of midazolam. [\hyperlink{Vyndaqel}{PMID: 7590052}, E Chuang et al., 1995]

\hypertarget{pmid_26464453}{S}ildenafil has strong cardiac preconditioning properties in animal studies and has a safe side-effect profile in children. Therefore, we evaluated the application of Sildenafil preconditioning to reduce myocardial ischaemia/reperfusion injury in children undergoing surgical ventricular septal defect (VSD) closure. This is a randomized, double-blind study. Children (1-17 years) undergoing VSD closure were randomized into three groups: placebo (Control group), preconditioning with 0.06 mg/kg (Sild-L group) and 0.6 mg/kg Sildenafil (Sild-H group). troponin release. CK-MB, Troponin I, inflammatory response (IL-6 and TNF-α), bypass and ventilation weaning times, inotropy score and echocardiographic function were assessed. Data expressed as median (range), and a value of P < 0.05 was considered significant. Thirty-nine patients were studied (13/group). Aortic cross-clamp time was similar [27 (18-85) and 27 (12-39) min] in the Control and Sild-L groups, respectively, but significantly longer [39 (20-96) min] in the Sild-H group when compared with the Control group. Area under the curve of CK-MB release was 1105 (620-1855) h ng/ml in the Control group, 1672 (564-2767) h ng/ml in the Sild-L group and was significantly higher in the Sild-H group [1695 (1252-3377) h ng/ml] when compared with the Control group. There were no significant differences in inflammatory response markers, cardiopulmonary bypass and ventilation weaning times, inotropy scores and echocardiographic function between the groups. In this small study, Sildenafil failed to reduce myocardial injury in children undergoing cardiac surgery, nor does it alter cardiac function, inotropic needs or postoperative course. A subclinical increase in cardiac enzyme release after Sildenafil preconditioning cannot be excluded. CTRI/2014/03/004468. [\hyperlink{Vyndaqel}{PMID: 26464453}, Varsha Walavalkar et al., 2016]

\hypertarget{pmid_21981332}{V}oiding cystourethrography (VCUG) is commonly performed to screen for vesicoureteric reflux or other urological anomalies but has a potential to provoke distress in infants and children. We performed a systematic review of randomized controlled trials of interventions to reduce distress, pain or anxiety during VCUG. Eight trials (591 participants) met the inclusion criteria. Conscious sedation with midazolam effectively alleviates the distress of VCUG in children older than 1 year of age. Psychological preparation and warmed contrast medium may also be effective. Nitrous oxide 50\% may be an alternative to midazolam, but further evidence is needed. [\hyperlink{Vyndaqel}{PMID: 21981332}, Jia Rao et al., 2012]

\hypertarget{pmid_37390311}{T}he direct oral anticoagulants (DOACs), rivaroxaban and dabigatran are newly licensed for the treatment and prevention of venous thromboembolism (VTE) in children and mark a renaissance in pediatric anticoagulation management. They provide a convenient option over standard-of-care anticoagulants (heparins, fondaparinux and vitamin K antagonists) due to their oral route of administration, child-friendly formulations, and significant reduction in monitoring. However, limitations related to therapeutic monitoring when needed and the lack of approved reversal agents for DOACs in children raise some safety concerns. There is accumulating experience of safety and efficacy of DOACs in the adults for a broad scope of indications, however the cumulative experience of using DOACs in pediatrics, specifically for those with coexisting chronic illnesses is sparse. Consequently, clinicians must often rely on their experience in treating VTE and extrapolation from adult data while using DOACs in these children. In this edition of "How I treat" the authors' share their experience of managing 4 scenarios that hematologists are likely to encounter in their day-to-day practice. Topics addressed include (1) appropriateness of indication; (2) use in special populations of children; (3) considerations for laboratory monitoring; (4) transition between anticoagulants; (5) major drug interactions; (6) perioperative management; and (7) anticoagulation reversal. [\hyperlink{Vyndaqel}{PMID: 37390311}, Rukhmi Bhat et al., 2023]

\hypertarget{pmid_11847958}{I}nformation regarding the treatment of anthrax infection is scarce in adults and is even more limited in children. Children, however, may be at a greater risk for developing an infection and systemic disease if exposed to anthrax than adults. The Centers for Disease Control and Prevention (CDC) recommends the use of doxycycline or ciprofloxacin for prophylaxis and treatment in children. Doxycycline currently is not indicated for use in children < 8 years old, due to staining of teeth and inhibition of bone growth associated with tetracyclines. Doxycycline, however, may have less adverse effect on teeth than its precursors. Ciprofloxacin has a pediatric indication only when a child is potentially exposed to inhaled anthrax. Ciprofloxacin is contraindicated in pediatric patients because fluoroquinolones were shown to cause cartilage toxicity in immature animals. Although children of various ages have received ciprofloxacin, there are few reports of cartilage toxicity. Because anthrax is a potentially fatal infection, the benefits to using these antibiotics greatly outweigh the risks. Therefore, the use of these antibiotics in children can be recommended, despite the lack of adequate efficacy and safety studies in pediatric patients with anthrax. [\hyperlink{Vyndaqel}{PMID: 11847958}, Sandra Benavides et al., 2002]

\hypertarget{pmid_22345942}{T}o compare oral midazolam (0.5 mg/kg) with oral butorphanol (0.2 mg/kg) as a premedication in 60 pediatric patients with regards to sedation, anxiolysis, rescue analgesic requirement, and recovery profile. In a double blinded study design, 60 pediatric patients belonging to ASA class I and II between the age group of 2-12 years scheduled for elective surgery were randomized to receive either oral midazolam (group I) or oral butorphanol (group II) 30 min before induction of anesthesia. The children were evaluated for levels of sedation and anxiety at the time of separation from the parents, venepuncture, and at the time of facemask application for induction of anesthesia. Rescue analgesic requirement, postoperative recovery, and complications were also recorded. Butorphanol had better sedation potential than oral midazolam with comparable anxiolysis at the time of separation of children from their parents. Midazolam proved to be a better anxiolytic during venepuncture and facemask application. Butorphanol reduced need for supplemental analgesics perioperatively without an increase in side effects such as nausea, vomiting, or unpleasant postoperative recovery. Oral butorphanol is a better premedication than midazolam in children in view of its excellent sedative and analgesic properties. It does not increase side effects significantly. [\hyperlink{Vyndaqel}{PMID: 22345942}, Chandni Sinha et al., 2012]

\hypertarget{pmid_21968355}{V}oriconazole pharmacokinetics are not well characterized in children despite prior studies. To assess the appropriate pediatric dosing, a study was conducted in 40 immunocompromised children aged 2 to <12 years to evaluate the pharmacokinetics and safety of voriconazole following intravenous (IV)-to-oral (PO) switch regimens based on a previous population pharmacokinetic modeling: 7 mg/kg IV every 12 h (q12h) and 200 mg PO q12h. Area under the curve over the 12-h dosing interval (AUC(0-12)) was calculated using the noncompartmental method and compared to that for adults receiving approved dosing regimens (6 → 4 mg/kg IV q12h, 200 mg PO q12h). On average, the AUC(0-12) in children receiving 7 mg/kg IV q12h on day 1 and at IV steady state were 7.85 and 21.4 μg · h/ml, respectively, and approximately 44\% and 40\% lower, respectively, than those for adults at 6 → 4 mg/kg IV q12h. Large intersubject variability was observed. At steady state during oral treatment (200 mg q12h), children had higher average exposure than adults, with much larger intersubject variability. The exposure achieved with oral dosing in children tended to decrease as weight and age increased. The most common treatment-related adverse events were transient elevated liver function tests. No clear threshold of voriconazole exposure was identified that would predict the occurrence of treatment-related hepatic events. Overall, voriconazole IV doses higher than 7 mg/kg are needed in children to closely match adult exposures, and a weight-based oral dose may be more appropriate for children than a fixed dose. Safety of voriconazole in children was consistent with the known safety profile of voriconazole. [\hyperlink{Vyndaqel}{PMID: 21968355}, Timothy A Driscoll et al., 2011]

\hypertarget{pmid_17114560}{V}oiding cystourethrography (VCU) is a distressing procedure for children. Conscious sedation using oral midazolam may reduce this distress, but its use may also alter the ability of the VCU to show vesicoureteric reflux (VUR). The objectives of our study were to assess the effectiveness of conscious sedation using oral midazolam when administered routinely in children undergoing VCU and to ensure that conscious sedation using oral midazolam does not alter the ability of VCU to show VUR. Our study was a randomized double-blind controlled trial performed at a university teaching hospital; our study group consisted of children over the age of 1 year who been referred for their first VCU examination from July 2001 to July 2003. Participants were randomized to receive a placebo or midazolam syrup (0.5 mg/kg) before the examination. The primary outcome measures were the Groningen Distress Rating Scale (GDRS) and grading of VUR, as defined by the international grading system established by the International Reflux Study Group. There were no serious adverse events. One hundred thirty-nine children were randomized in the study, and 117 underwent complete assessment. Eight who underwent VCU after the study day were included in a "complete case" intention-to-treat analysis. In the placebo group, 34 children (61\%) experienced serious distress or severe distress (GDRS score, 3 or 4). In the midazolam group, 16 children (26\%) experienced the same degree of distress. There was a significant difference between the GDRS scores (nonlinear mixed-model analysis, p < 0.001) of the two study groups. The number needed to treat to reduce serious or severe distress in one child was 2.9 (95\% CI, 1.9-5.5). VUR was identified in 16\% of all children. There was no difference in VUR grading between the groups (nonlinear mixed-model analysis, p = 0.31). Routine use of oral midazolam (0.5 mg/kg) for conscious sedation of children undergoing VCU reduces distress and does not alter the ability of VCU to show VUR well enough to allow diagnosis. [\hyperlink{Vyndaqel}{PMID: 17114560}, David W Herd et al., 2006]

\hypertarget{pmid_16257311}{F}ive independent, multicentered, double-masked, parallel, controlled studies were conducted to determine the safety of moxifloxacin ophthalmic solution 0.5\% (VIGAMOX) in pediatric and nonpediatric patients with bacterial conjunctivitis. Patients were randomized into one of two treatment groups in each study and received either moxifloxacin ophthalmic solution 0.5\% b.i.d. or t.i.d. or a comparator. A total of 1,978 patients (918 pediatric and 1,060 nonpediatric) was evaluable for safety. The most frequent adverse event in the overall safety population was transient ocular discomfort, occurring at an incidence of 2.8\%, which was similar to that observed with the vehicle. No treatment-related changes in ocular signs or visual acuity were observed with moxifloxacin ophthalmic solution 0.5\%, except for one clinically relevant change in visual acuity. Thus, based upon a review of adverse events and an assessment of ocular parameters, moxifloxacin ophthalmic solution 0.5\% formulated without the preservative, benzalkonium chloride, is safe and well tolerated in pediatric (3 days-17 years of age) and nonpediatric (18-93 years) patients with bacterial conjunctivitis. [\hyperlink{Vyndaqel}{PMID: 16257311}, Lewis H Silver et al., 2005]

\hypertarget{pmid_7717236}{M}idazolam is a relatively short-acting water-soluble benzodiazepine that provides anxiolysis and anterograde amnesia and can be given orally with few adverse effects. We evaluated the benefit and safety of oral midazolam for sedation of young children during voiding cystourethrography or nuclear cystography. For 3.5 years, a highly selected group of 98 children, ages 23 months to 9 years (mean, 4 years), were given oral midazolam 0.6 mg/kg 20-30 min before cystourethrography or nuclear cystography. These children either had been frightened by a previous catheterization (39\%) or seemed particularly frightened during an examination of their genitals in the office (61\%). A control group of 25 children, similar in age to the study group, did not receive midazolam before cystourethrography. Parents were interviewed to assess their child's recollection of the procedure. Voiding dynamics were assessed by evaluating the postvoiding radiograph. Of the midazolam-treated patients, 60\% had no recollection of the study, and 31\% remembered part or all of the study but did not have a negative experience. No significant change in vital signs or oxygen saturation was observed in any child. In the control group, 24 (96\%) of 25 children remembered the cystographic examination (p < .01). Behavioral side effects occurred in 12\% of the children receiving midazolam and consisted primarily of combative behavior as the medication was wearing off. Ninety-five percent of the parents indicated that they would want their child to have midazolam again if the cystography needed to be repeated. Of the children receiving midazolam, 76\% had little or no residual urine after voiding, compared with 72\% of the control group (no significant difference). In children who have been or are likely to be excessively frightened during cystourethrography or nuclear cystography, midazolam usually provides satisfactory amnesia and anxiolysis with few side effects or adverse impact on voiding dynamics. [\hyperlink{Vyndaqel}{PMID: 7717236}, J S Elder et al., 1995]

\hypertarget{pmid_36610740}{T}he aim of this article is to provide an overview of the current literature for direct-acting oral anticoagulant (DOAC) use in pediatric patients and summarize ongoing trials. In treatment of venous thromboembolism (VTE) in pediatric patients, evidence supports use of both dabigatran and rivaroxaban. Dabigatran has been shown to be noninferior to standard of care (SOC) in terms of efficacy, with similar bleeding rates. Similarly, treatment with rivaroxaban in children with acute VTE resulted in a low recurrence risk and reduced thrombotic burden, without increased risk of bleeding, compared to SOC. Treatment of pediatric cerebral venous thrombosis as well as central venous catheter-related VTE with rivaroxaban appeared to be both safe and efficacious and similar to that with SOC. Dabigatran also has a favorable safety profile for prevention of VTE, and rivaroxaban has a favorable safety profile for VTE prevention in children with congenital heart disease. Many studies with several different DOACs are ongoing to evaluate both safety and efficacy in unique patient populations, as well as VTE prevention. The literature regarding pediatric VTE treatment and prophylaxis is growing, but the need for evidence-based pediatric guidelines remains. Additional long-term, postauthorization studies are warranted to further elucidate safety and efficacy in clinical scenarios excluded in clinical trials. Additional data on safety, efficacy, and dosing strategies for reversal agents are also necessary, especially as the use of DOACs becomes more common in the pediatric population. [\hyperlink{Vyndaqel}{PMID: 36610740}, Kimberly Mills et al., 2023]

\section*{Xenical}
\subsection*{Result}
\subsubsection*{Answer}

Unknown

\subsubsection*{{Explanation}}
\hypertarget{Xenical}
A review of the available abstracts reveals that none of them are about Xenical (orlistat), nor do any of them address the safety of Xenical in children. The abstracts provided discuss other medications and interventions in pediatric populations, but there is no targeted study or evidence regarding the safety or unsafety of Xenical in children of any age range. Therefore, based on the abstracts available, the safety of Xenical for use in children is unknown.

\subsection*{Abstracts}
\hypertarget{pmid_25886748}{X}enon has minimal haemodynamic side effects when compared to volatile or intravenous anaesthetics. Moreover, in in vitro and in animal experiments, xenon has been demonstrated to convey cardio- and neuroprotective effects. Neuroprotection could be advantageous in paediatric anaesthesia as there is growing concern, based on both laboratory studies and retrospective human clinical studies, that anaesthetics may trigger an injury in the developing brain, resulting in long-lasting neurodevelopmental consequences. Furthermore, xenon-mediated neuroprotection could help to prevent emergence delirium/agitation. Altogether, the beneficial haemodynamic profile combined with its putative organ-protective properties could render xenon an attractive option for anaesthesia of children undergoing cardiac catheterization. In a phase-II, mono-centre, prospective, single-blind, randomised, controlled study, we will test the hypothesis that the administration of 50\% xenon as an adjuvant to general anaesthesia with sevoflurane in children undergoing elective cardiac catheterization is safe and feasible. Secondary aims include the evaluation of haemodynamic parameters during and after the procedure, emergence characteristics, and the analysis of peri-operative neuro-cognitive function. A total of 40 children ages 4 to 12 years will be recruited and randomised into two study groups, receiving either a combination of sevoflurane and xenon or sevoflurane alone. Children undergoing diagnostic or interventional cardiac catheterization are a vulnerable patient population, one particularly at risk for intra-procedural haemodynamic instability. Xenon provides remarkable haemodynamic stability and potentially has cardio- and neuroprotective properties. Unfortunately, evidence is scarce on the use of xenon in the paediatric population. Our pilot study will therefore deliver important data required for prospective future clinical trials. EudraCT: 2014-002510-23 (5 September 2014). [\hyperlink{Xenical}{PMID: 25886748}, Sarah Devroe et al., 2015]

\hypertarget{pmid_19740527}{E}noxaparin, a low molecular weight heparin (LMWH), is frequently used for the prevention and treatment of thromboembolic complications in infants and children (Sutor et al., 2004 [1]). Injection pain and the fear and anxiety associated with needle phobia in the pediatric population are well documented. Best practice pediatric pain management standards of care recommend mitigating the child's pain experience whenever possible. The use of topical anesthetics such as liposomal-lidocaine 4\% results in a rapid onset of anesthesia, minimal blanching, without vasoconstriction (Koh et al., 2004 [2]) or risk of methemoglobinemia. Topical lidocaine has been used to reduce the injection pain of enoxaparin, but there is no data available examining whether it will interfere with the absorption of LMWH. To determine if the topical lidocaine, Maxilene, interferes with enoxaparin absorption as measured by peak anti-Xa levels. Infants and children clinically prescribed enoxaparin were eligible for this study. Children in group 1 were pre-treated with Maxilene prior to enoxaparin injection on day 1 with no Maxilene pre-treatment on day 2. For group 2, the order was reversed. Peak anti-Xa levels were measured following each enoxaparin dose and were compared between the groups. 26 children of ages 14d-16 y (median 6.7 months) were enrolled. Anti-Xa levels following topical lidocaine administration were 0.070 U/mL (95\%CI 0.025; 0.114) lower than without prior topical lidocaine administration. Anti-Xa levels on the second day were on average 0.013 U/mL (95\%CI -0.066; 0.040) higher compared to day one regardless of the order of topical lidocaine administration. There were no reported incidences of local reactions such as redness, hives or blanching. Topical lidocaine (Maxilene) administration before enoxaparin injection results in a small, clinically non-significant, reduction in anti-Xa levels. [\hyperlink{Xenical}{PMID: 19740527}, S M Duncan et al., 2010]

\hypertarget{pmid_15675936}{P}atient-controlled analgesia (PCA) using intravenous opioids is increasing in popularity for children aged 5 years and over. To our knowledge there are no reports on the use of PCA in children with remifentanil in the postoperative period. We report successful use of remifentanil for intravenous (IV) PCA in a child scheduled for suprasellar arachnoid cystectomy with Axenfeld-Rieger syndrome who needed good postoperative analgesia because of accompanying serious problems. [\hyperlink{Xenical}{PMID: 15675936}, Oya Yalcin Cok et al., 2005]

\hypertarget{pmid_23572341}{X}enon, a monoatomic gas with very high tissue solubility, is a non-competitive inhibitor of N-methyl-D-aspartate (NMDA) glutamate receptor, has antiapoptotic effects and is neuroprotective following hypoxic ischaemic injury in animals. Xenon may be expected to have anticonvulsant effects through glutamate receptor blockade, but this has not previously been demonstrated clinically. We examined seizure activity on the real time and amplitude integrated EEG records of 14 full-term infants with perinatal asphyxial encephalopathy treated within 12 h of birth with 30\% inhaled xenon for 24 h combined with 72 h of moderate systemic hypothermia. Seizures were identified on 5 of 14 infants. Seizures stopped during xenon therapy but recurred within a few minutes of withdrawing xenon and stopped again after xenon was restarted. Our data show that subanaesthetic levels of xenon may have an anticonvulsant effect. Inhaled xenon may be a valuable new therapy in this hard-to-treat population. [\hyperlink{Xenical}{PMID: 23572341}, Denis Azzopardi et al., 2013]

\hypertarget{pmid_28476033}{S}everal studies have reported the use of dexmedetomidine (DEX) plus opioids for flexible bronchoscopy in both adults and children. To determine whether DEX plus sufentanil (SF) is safe for children, 142 children undergoing flexible bronchoscopy were assigned to one of three groups, each of which received the same SF loading dose and similar DEX and SF maintenance doses, but different loading doses of DEX: DS1 (DEX 0.5 μg·kg-1), DS2 (DEX 1.0 μg·kg-1), and DS3 (DEX 1.5 μg·kg-1). The Ramsay sedation scale was maintained at 3 in all groups. Results showed that anesthesia onset time was shorter, and the perioperative hemodynamic profile was more stable, in the DS3 group. The number of intraoperative movements was also lowest in the DS3 group. The time to first dose of rescue midazolam and lidocaine was significantly longer, but the total corresponding accumulated doses were lower in the DS3 group. Although the time to recovery prior to discharge from the post anesthesia care unit was longer, the overall incidence of tachycardia was lower in the DS3 group, and it received the highest bronchoscopist satisfaction score among the three groups. We therefore conclude that high-dose DEX plus SF can be safely and efficaciously used in children undergoing flexible bronchoscopy. [\hyperlink{Xenical}{PMID: 28476033}, Xiujing Dang et al., 2017]

\hypertarget{pmid_951627}{O}ral oxamniquine 800 mg/m2 body surface area/day in divided doses for 2 days, was given to 57 children with schistosomal infections. Haematological and biochemical tests suggest that this drug is safe, even in the presence of significant liver enlargement. Parasitological investigations in 32 children who were followed up for 1 month indicate that oxamniquine is effective in the treatment of S. mansoni infection, but that it has little effect on S. haematobium infection. [\hyperlink{Xenical}{PMID: 951627}, J H Axton et al., 1976]

\hypertarget{pmid_32145737}{I}ntranasal dexmedetomidine (DEX), as a novel sedation method, has been used in many clinical examinations of infants and children. However, the safety and efficacy of this method for electroencephalography (EEG) in children is limited. In this study, we performed a large-scale clinical case analysis of patients who received this sedation method. The purpose of this study was to evaluate the safety and efficacy of intranasal DEX for sedation in children during EEG. This was a retrospective study. The inclusion criteria were children who underwent EEG from October 2016 to October 2018 at the Children's Hospital affiliated with Chongqing Medical University. All the children received 2.5 μg·kg A total of 3475 cases were collected and analysed in this study. The success rate of the initial dose was 87.0\% (3024/3475 cases), and the success rate of intranasal sedation rescue was 60.8\% (274/451 cases). The median sedation onset time was 19 mins (IQR: 17-22 min), and the sedation recovery time was 41 mins (IQR: 36-47 min). The total incidence of adverse events was 0.95\% (33/3475 cases), and no serious adverse events occurred. Intranasal DEX (2.5 μg·kg [\hyperlink{Xenical}{PMID: 32145737}, Hang Chen et al., 2020] Xylitol is a safe dental caries preventive when incorporated into chewing gum or confections used habitually. The goal of this paper is to identify and assess the work on xylitol and other polyols and dental caries since 2008. Xylitol is effective when used by the mother prenatally or after delivery to prevent mutans transmission and subsequent dental caries in the offspring. One new completed trial confirmed that children of mothers who used xylitol lozenges after delivery had less dental caries than a comparison group. A similar study confirmed that the use of xylitol gum by the mother either prevented or postponed MS transmission to the offspring. Xylitol use among schoolchildren delivered via a gummy bear confection reduced S. mutans levels, but a once per day use of xylitol-containing toothpaste did not. Randomized trials, with caries outcomes, assessing xylitol-containing lozenges in adults and xylitol-containing gummy bears in children will release results in the coming year. Other studies are ongoing but are not systematic and will fail to answer important questions about how xylitol, or other polyols, can address the global dental caries problem. [\hyperlink{Xenical}{PMID: 32145737}, P Milgrom et al., 2012]

\hypertarget{pmid_23879257}{T}his demonstration programme tested topical use of xylitol as a possible oral health promoting regimen in infants at a Finnish Public Health Centre in 2002-2011. Parents (usually mothers) began once- or twice-daily administration of a 45\% solution of xylitol (2.96 m) onto all available deciduous teeth of their children at the age of approximately 6-8 months. The treatment (xylitol swabbing), which continued till the age of approximately 36 months (total duration 26-28 months), was carried out using cotton swabs or a children's toothbrush; the approximate daily xylitol usage was 13.5 mg per each deciduous tooth. At the age of 7 years, caries data on the deciduous dentition of 80 children were compared with those obtained from similar, untreated children (n = 90). Xylitol swabbing resulted in a significant (P < 0.001) reduction in the incidence of enamel and dentine caries compared with the comparison subjects (relative risk 2.1 and 4.0, respectively; 95\% confidence intervals 1.42-3.09 and 2.01-7.98, respectively). Similar findings were obtained when the children were 5 or 6 years old. The treatment reduced the need of tooth filling relative risk and 95\% confidence intervals at 7 years: 11.86 and 6.36-22.10, respectively; P < 0.001). Compared with untreated subjects, the oral counts of mutans streptococci were reduced significantly (P < 0.001). Considerable improvement in dental health was accomplished in infants participating in a topical at-home xylitol administration experiment, which was offered to families in the area by the Public Health Centre as a supplement to standard oral health care. Caregiver assessment of the programme was mostly rated as high or satisfactory. [\hyperlink{Xenical}{PMID: 23879257}, Kauko K Mäkinen et al., 2013]

\hypertarget{pmid_28275979}{S}edation is often required for children undergoing diagnostic procedures. Chloral hydrate has been one of the sedative drugs most used in children over the last 3 decades, with supporting evidence for its efficacy and safety. Recently, chloral hydrate was banned in Italy and France, in consideration of evidence of its carcinogenicity and genotoxicity. Dexmedetomidine is a sedative with unique properties that has been increasingly used for procedural sedation in children. Several studies demonstrated its efficacy and safety for sedation in non-painful diagnostic procedures. Dexmedetomidine's impact on respiratory drive and airway patency and tone is much less when compared to the majority of other sedative agents. Administration via the intranasal route allows satisfactory procedural success rates. Studies that specifically compared intranasal dexmedetomidine and chloral hydrate for children undergoing non-painful procedures showed that dexmedetomidine was as effective as and safer than chloral hydrate. For these reasons, we suggest that intranasal dexmedetomidine could be a suitable alternative to chloral hydrate. [\hyperlink{Xenical}{PMID: 28275979}, Giorgio Cozzi et al., 2017]

\hypertarget{pmid_28441266}{J}apanese encephalitis remains a serious health concern in Asian countries and has sporadically affected pediatric travelers. In the present study, we monitored the safety profile of the Japanese encephalitis virus vaccine IXIARO (Valneva Austria GmbH, Vienna, Austria) in a pediatric population. We randomized 1869 children between 2 months and 17 years of age in an age-stratified manner to vaccination with IXIARO or one of the control vaccines, Prevnar (formerly Wyeth Pharmaceuticals Inc., now Pfizer Inc., Kent, United Kingdom) and HAVRIX 720 (GlaxoSmithKline Biologicals, Rixensart, Belgium). Adverse events (AEs) (unsolicited and solicited local and systemic AEs), serious AEs and medically attended AEs were assessed up to day 56 and month 7 after the first dose. Incidences of AEs, serious AEs or medically attended AEs did not differ significantly between the groups in any age stratum. AEs were most frequent in children <1 year of age and decreased with age. AEs of special interest, predefined as AEs associated with potential hypersensitivity/allergy or neurologic disorders up to day 56, were reported in 4.6\% (IXIARO) versus 6.3\% (Prevnar) in the ≥2 months to <1 year age group and 3.4\% (IXIARO) versus 3.3\% (HAVRIX) in the ≥1 to <18 years age group. Fever, the most frequent systemic reaction in 23.7\% of infants to 3.8\% of adolescents, decreased with age and did not differ between groups. The safety profile of IXIARO was comparable to the control vaccines in terms of overall AE rates, serious AEs and medically attended AEs. [\hyperlink{Xenical}{PMID: 28441266}, Katrin L Dubischar et al., 2017]

\hypertarget{pmid_18163237}{C}ase series of ingestion in preschool children may include patients without significant exposure if the substance is not measured. In order to evaluate the unproven ingestion bias, we conducted, between January 2000 and June 2004, a retrospective analysis of a poison control center-based series of children <6 years old with a history of toxic methanol or ethylene glycol ingestion. Over the 54 month period, 115 children were referred to obtain a level. Of these, 102 children, aged 25 +/- 10 months, actually had a level analyzed. Only 21 patients had positive levels measured a median of 90 minutes post-ingestion. Our findings suggest that a significant fraction of purported cases were not confirmed. When a study aims at determining the toxicity of the substance, measurements of the xenobiotic should be required in any case series involving preschool aged children in order to decrease the unproven ingestion bias. [\hyperlink{Xenical}{PMID: 18163237}, Arielle Lévy et al., 2007]

\hypertarget{pmid_19681413}{T}his randomised controlled study evaluated the effects of fentanyl and dexmedetomidine on emergence characteristics of children having adenoidectomy and anaesthetised with sevoflurane. Ninety children, two to seven years of age and ASA physical status I, were studied. Children were randomly assigned to one of three groups of 30 children, with the study intervention injection given intravenously after intubation. Children in Group F received fentanyl 2.5 microg x kg(-1), children in Group D received dexmedetomidine 0.5 microG x kg(-1) and children in Group C received saline solution. Anaesthesia was induced with 50\% N2O and 8\% sevoflurane in O2 by mask and atracurium 0.6 mg x kg(-1) was administered for tracheal intubation. All children received paracetamol 40 mg/kg rectally one hour preoperatively and dexamethasone 0.5 mg x kg(-1) intravenously. The time to extubation was shorter in Group D than Group F. The eye-opening time was longer in Group F (16.1 +/- 5.3 minutes) than in Groups C (12.0 +/- 4.2 minutes) and D (12.7 +/- 3.2 minutes). The proportion of pain-free children in early recovery was significantly higher in Groups D (47\%) and F (43\%) than Group C (13\%) (P < 0.05). The proportion of children with agitation scores > 3 was lower in Groups D 17\% (5/30) and F 13\% (4/30) than in Group C 47\% (14/30) (P < 0.05). Fentanyl 2.5 microg x kg(-1) and dexmedetomidine 0.5 microg x kg(-1) had similar haemodynamic effects and emergence characteristics. Fentanyl has been safely used in children for many years. Further studies of dexmedetomidine safety and its interaction with other anaesthetic agents are required before recommending its routine use during general anaesthesia in children. [\hyperlink{Xenical}{PMID: 19681413}, F Erdil et al., 2009]

\hypertarget{pmid_29541305}{A}xillary block is an easy and recommended technique in children. Its use in children with acute hepatitis A is not risk free especially when associated with sedation using remifentanil and propofol. Similarly, the presence of a single hydatid cyst allows general anesthesia with mono-pulmonary ventilation. [\hyperlink{Xenical}{PMID: 29541305}, Anouar Jarraya et al., 2017]

\hypertarget{pmid_27468971}{H}erbal medicinal products are indispensable in children, e. g., in functional gastrointestinal diseases and coughs and colds, especially when available in liquid dosing forms for which dosing can be adapted ideally to different age groups. Despite being generally accepted as safe, the ethanol content of many of these products, necessary for Galenic reasons, has raised questions regarding their safety. Therefore, safety data from more than 50,000 children in noninterventional pediatric studies with these products, as well as data from routine clinical use in several million children, were assessed. No evidence of the involvement of the ethanol content in any adverse drug reactions was found. This allows us to conclude that these herbal medicinal products are safe in the age groups for which they are authorized or registered and that the present labeling is adequate to allow for their safe use in the pediatric population. [\hyperlink{Xenical}{PMID: 27468971}, Olaf Kelber et al., 2017]

\hypertarget{pmid_32258344}{O}zenoxacin is a topical antibiotic approved in the United States for treatment of impetigo in adults and children age ≥2 months. This analysis evaluated the efficacy and safety of ozenoxacin in specific pediatric age groups. Data for children aged 2 months to <18 years recruited from eight countries who had participated in phase 1 and 3 trials of ozenoxacin were extracted and analyzed by age range. Across studies, 644 pediatric patients with impetigo received ozenoxacin 1\% cream (n = 287) or vehicle (n = 247). One study included retapamulin 1\% ointment as the internal validity control (n = 110). The clinical success rate at the end of treatment and bacterial eradication rates after 3 to 4 days of treatment and at the end of treatment were significantly higher with ozenoxacin than vehicle (all  The results of this analysis suggest that ozenoxacin 1\% cream is an effective and safe treatment for impetigo in pediatric patients aged 2 months to <18 years. [\hyperlink{Xenical}{PMID: 32258344}, Adelaide A Hebert et al., 2020]

\hypertarget{pmid_28741653}{C}hloral hydrate is commonly used to sedate children for painless procedures. Children may recover more quickly after sedation with dexmedetomidine, which has a shorter half-life. We randomly allocated 196 children to chloral hydrate syrup 50 mg.kg [\hyperlink{Xenical}{PMID: 28741653}, V M Yuen et al., 2017] There were examined two groups of children, whose mother were treated with the uterine relaxant Hexoprenalin. 21 children were examined when they were three years of age, 28 children during their first week of live (internistic and neurological findings, EKG and blood-analyses). There were no signes of pathological development; all the findings were within the norm. [\hyperlink{Xenical}{PMID: 28741653}, F Wilk et al., ]

\hypertarget{pmid_25246305}{T}he aim of this study was to compare the efficacy and safety of different oral chloral hydrate and dexmedetomidine doses used for sedation during electroencephalography (EEG) in children. One hundred sixty children aged 1 to 9 years with American Society of Anesthesiologists physical status I-II who were uncooperative during EEG recording or who were referred to our electrodiagnostic unit for sleep EEG were included to the study. The patients were randomly assigned into 4 groups. In groups D1 and D2, patients received oral dexmedetomidine doses of 2 and 3 µg/kg, respectively. In group C1 and C2, patients received oral chloral hydrate doses of 50 and 100 mg/kg, respectively. The induction time was significantly shorter in group C2 compared with other groups (P = .000). The rate of adverse effects was significantly higher in group C2 compared with the dexmedetomidine groups (D1 and D2; P = .004). In conclusion, dexmedetomidine can be used safely for sedation during EEG in children.  [\hyperlink{Xenical}{PMID: 25246305}, Hakan Gumus et al., 2015] This paper reports on 350 pediatric patients who were studied over a 17-month period to determine the efficacy and safety of oral and intramuscular sedation techniques. The protocol using oral chloral hydrate, 50 mgm/kg, for infants under 1 year of age or intramuscular pentobarbital, 5 mgm/kg, for children over 1 year was found to be an effective, safe and fairly simple approach to pediatric sedation. Of the 350 sedated patients, 343 (98 percent) had satisfactory scans on the same day the examination was scheduled after a single dose or an initial dose and supplementary sedation. [\hyperlink{Xenical}{PMID: 25246305}, J B Temme et al., ]

\hypertarget{pmid_31958794}{O}zenoxacin is a topical antibiotic approved in Europe to treat non-bullous impetigo in adults and children aged ≥6 months. This analysis evaluated the efficacy and safety of ozenoxacin in paediatric patients by age group. Pooled data for patients aged 6 months to <18 years who had participated in a phase I or in two phase III clinical trials of ozenoxacin 1\% cream were analysed by age group: 0.5-<2, 2-<6, 6-<12, and 12-<18 years. The combined population comprised 529 patients with non-bullous impetigo treated with ozenoxacin (n = 239), vehicle (n = 201), or retapamulin as internal validation control (n = 89). Studies were well matched for extent and severity of impetigo and therapeutic schedule (twice daily application for 5 days). The clinical success rate after 5 days' treatment (day 6-7, end of therapy), and microbiological success rates after 3-4 days' treatment and at the end of therapy, were significantly higher with ozenoxacin than vehicle (p < 0.0001 for all comparisons). Clinical and bacterial eradication rates were higher with ozenoxacin than vehicle in each age group. No safety concerns were identified with ozenoxacin. One (0.3\%) of 327 plasma samples exceeded the lower limit of quantification for ozenoxacin, but the low concentration indicated negligible systemic absorption. This combined analysis supports the efficacy and safety of ozenoxacin administered twice daily for 5 days. Ozenoxacin 1\% cream is a new option to consider for treatment of non-bullous impetigo in children aged 6 months to <18 years. [\hyperlink{Xenical}{PMID: 31958794}, Antonio Torrelo et al., 2020]

\hypertarget{pmid_30463814}{D}exmendetomidine hydrochloride (DEX) is a new common adrenergic receptor agonist, which not only keeps children calm but also has analgesic effect. Dexmedetomidine hydrochloride will enable children to maintain the natural non-REM sleep, which can be stimulated sedation or language arousal. The aim of this study is to observe the sedative effect and adverse drug reactions of dexmedetomidine hydrochloride injection and propofol injection in MRI examination. In this study, no children in the experimental group were required to add sedative drugs, and 2 cases in the control group were treated with sedative drugs. In experimental group, it used dexmedetomidine hydrochloride as (1.64±0.91) g/kg; in control group, dosage of narcotic drugs as (5.26±1.82) g/kg, and the total complication rate of the children in the experimental group was lower than that of the control group (P<0.05). After returning to the ward, the doses of phenobarbital sedation were dexmedetomidine group (4.28±1.53) mg/kg and propofol group (6.40±1.71) mg/kg. There was significant difference between the two groups. The total complication rate in the experimental group was lower than that in the control group (P<0.05). The quality of MRI in the test group was significantly higher than that in the control group, which showed that dexmedetomidine hydrochloride could provide a satisfactory sedative effect in the MRI examination of children. To sum up, dexmedetomidine hydrochloride is a wide range of clinical applications. It is an effective drug for the maintenance of sedation in clinical disease treatment. It is flexible in the way of administration and with less adverse reactions. It is suitable for popularization and application in clinical practice. [\hyperlink{Xenical}{PMID: 30463814}, Zhendong Yang et al., 2018]

\hypertarget{pmid_30450703}{S}edation is often required for young children during transthoracic echocardiography. Dexmedetomidine and ketamine are two sedatives that are commonly used in children for procedural sedation, but they have some disadvantages when they are used alone. The aim of this retrospective study was to analyze the effects and safety of intranasal sedation with a combination of dexmedetomidine and ketamine during transthoracic echocardiography in young children and to analyze risk factors for sedation failure. After IRB approval, we retrospectively evaluated data on patients who underwent echocardiography between May 2016 and August 2017 utilizing a combination of dexmedetomidine 2 μg/kg and ketamine 1 mg/kg. We collected information including heart rate, pulse oxygen saturation, sedation onset time, exam time, recovery time, and adverse reactions. Stepwise logistic regression analyses were performed to analyze the risk factors for sedation failure. Sedation was successful in 2212 patients (96\%) and took effect in 15.7 (IQR: 10-23) min, while sedation failed in 92 patients. Cyanotic heart disease, history of sedation failure, history of congenital heart disease surgery, and fever were independent risk factors for sedation failure. Most of the patients in this study had an American Society of Anesthesiologists (ASA) grade of II to III, but no severe adverse reactions were observed. Intranasal sedation with a combination of dexmedetomidine and ketamine is effective and appears to have an acceptable safety profile for young children during transthoracic echocardiography. [\hyperlink{Xenical}{PMID: 30450703}, Jianxia Liu et al., 2019]

\hypertarget{pmid_717517}{E}ight children with corneal xerophthalmia (x2 or x3A) received standard high-protein diets and massive systemic vitamin A therapy. Retinoic acid, 0.1\% in oil, was applied daily to one eye, and oil alone to the other. Topical retinoic acid proved safe and effective in speeding corneal healing, especially during the first critical days. [\hyperlink{Xenical}{PMID: 717517}, A Sommer et al., 1978]

\hypertarget{pmid_6229197}{T}he safety and efficacy of fentanyl-oxygen (50 and 75 micrograms/kg) and sufentanil-oxygen (5 and 10 micrograms/kg) were studied in 40 infants undergoing repair of complex heart defects. When fentanyl or sufentanil was given simultaneously with pancuronium, induction of anesthesia was rapid and smooth with only mild and clinically insignificant hemodynamic changes. Hemodynamic responses to tracheal intubation were completely blocked, whereas hemodynamic responses to surgical incision and sternotomy were partially and variably blocked. Except for somewhat more effective blocking of responses to surgical stimulation by sufentanil, the effects of both narcotics were similar. No significant differences in effects were found between the two dose levels of either drug. Transcutaneous oxygen tensions increased with induction, intubation, and surgical stimulation with both fentanyl and sufentanil, even in cyanotic patients with right to left shunts. Fentanyl- and sufentanil-oxygen-pancuronium anesthesia were both safe and effective for cardiac surgery in infants. This study raises the question of possible beneficial effects of high dose fentanyl and sufentanil in blunting stress responses in the pulmonary circulation, a critical aspect of anesthesia and intensive care in the infant and neonate. [\hyperlink{Xenical}{PMID: 6229197}, P R Hickey et al., 1984]

\section*{Xolremdi}
\subsection*{Result}
\subsubsection*{Answer}

Unknown

\subsubsection*{{Explanation}}
\hypertarget{Xolremdi}
After reviewing all available abstracts, there is no evidence from any abstract that specifically addresses the safety of Xolremdi in children. None of the abstracts mention Xolremdi, nor do they present targeted studies evaluating its safety or efficacy in pediatric populations of any age range. Therefore, based on the abstracts provided, there is not enough data to determine whether Xolremdi is safe or unsafe for use in children.

\subsection*{Abstracts}
\hypertarget{pmid_28741653}{C}hloral hydrate is commonly used to sedate children for painless procedures. Children may recover more quickly after sedation with dexmedetomidine, which has a shorter half-life. We randomly allocated 196 children to chloral hydrate syrup 50 mg.kg [\hyperlink{Xolremdi}{PMID: 28741653}, V M Yuen et al., 2017] To assess the safety and efficacy of tolterodine tartrate prescribed to children who previously failed to tolerate oxybutynin chloride. We reviewed 34 children, followed for>1 year, who were prospectively crossed-over from oxybutynin to tolterodine because of side-effects. The initial diagnosis was dysfunctional voiding in 31 patients. All patients were placed on a behavioural modification protocol. When their symptoms did not improve after 6 months, treatment with an anticholinergic agent was considered. Urodynamic studies were conducted in 20 patients, confirming uninhibited contractions in 19. The remaining 14 patients were empirically started on antimuscarinic or anticholinergic agents. The 34 patients were treated with oxybutynin for a median (range) of 6 (2-84) months. When significant side-effects were reported, they were crossed over to tolterodine. The efficacy of tolterodine was assessed as defined by the International Children's Continence Society, with tolerability assessed and side-effects documented using a questionnaire. The mean age at the first dose of tolterodine was 8.9 years; the dose was 1 mg twice daily for 12 patients and 2 mg twice daily for 22. The median treatment with tolterodine was 11.5 months, with 20 (59\%) patients reporting no side-effects; six described the same but tolerable side-effects as with oxybutynin. Eight patients discontinued tolterodine because of side-effects after a median (range) of 5 (1-11) months. The efficacy of tolterodine was comparable with that of oxybutynin, as reported by the questionnaire and voiding diaries. The reduction in wetting episodes at 1 year was> 90\% in 23 (68\%), more than half in five and less than half (or failure) in six patients. Tolterodine is tolerated well in children. In this subgroup of patients who could not tolerate oxybutynin, 77\% were able to continue tolterodine treatment with no significant side-effects. [\hyperlink{Xolremdi}{PMID: 28741653}, S Bolduc et al., 2003]

\hypertarget{pmid_36188573}{S}irolimus is used to treat pediatric patients with  [\hyperlink{Xolremdi}{PMID: 36188573}, Xiao Chen et al., 2022] Background. Xyloglucan, a film-forming agent, improves intestinal mucosa resistance to pathologic damage. The efficacy, safety, and time of onset of the antidiarrheal effect of xyloglucan were assessed in children with acute gastroenteritis receiving oral rehydration solution (ORS). Methods. This randomized, controlled, open-label, parallel-group, multicenter, clinical trial included children (3 months-12 years) with acute gastroenteritis of infectious origin. Children were randomized to xyloglucan and ORS, or ORS only, for 5 days. Diarrheal symptoms, including stool number/characteristics, and safety were assessed at baseline and after 2 and 5 days and by fulfillment of a parent diary card. Results. Thirty-six patients (58.33\% girls) were included (n = 18/group). Patients receiving xyloglucan and ORS had better symptom evolution than ORS-only recipients, with a faster onset of action. At 6 hours, xyloglucan produced a significantly greater decrease in the number of type 7 stools (0.11 versus 0.44; P = 0.027). At days 3 and 5, xyloglucan also produced a significantly greater reduction in types 6 and 7 stools compared with ORS alone. Xyloglucan plus ORS was safe and well tolerated. Conclusions. Xyloglucan is an efficacious and safe option for the treatment of acute gastroenteritis in children, with a rapid onset of action in reducing diarrheal symptoms. This study is registered with ISRCTN number 65893282.  [\hyperlink{Xolremdi}{PMID: 36188573}, Cătălin Pleșea Condratovici et al., 2016] Sedation is often required for children undergoing diagnostic procedures. Chloral hydrate has been one of the sedative drugs most used in children over the last 3 decades, with supporting evidence for its efficacy and safety. Recently, chloral hydrate was banned in Italy and France, in consideration of evidence of its carcinogenicity and genotoxicity. Dexmedetomidine is a sedative with unique properties that has been increasingly used for procedural sedation in children. Several studies demonstrated its efficacy and safety for sedation in non-painful diagnostic procedures. Dexmedetomidine's impact on respiratory drive and airway patency and tone is much less when compared to the majority of other sedative agents. Administration via the intranasal route allows satisfactory procedural success rates. Studies that specifically compared intranasal dexmedetomidine and chloral hydrate for children undergoing non-painful procedures showed that dexmedetomidine was as effective as and safer than chloral hydrate. For these reasons, we suggest that intranasal dexmedetomidine could be a suitable alternative to chloral hydrate. [\hyperlink{Xolremdi}{PMID: 36188573}, Giorgio Cozzi et al., 2017]

\hypertarget{pmid_35017312}{A}lthough the recent approval of selumetinib is expected to transform the management of children with neurofibromatosis type 1 (NF1), particularly those with symptomatic and inoperable plexiform neurofibromas, no systematic review has summarized its efficacy and safety based on the latest studies. This study was conducted to systematically evaluate the efficacy and safety of selumetinib in children with NF1. Original articles reporting the efficacy and safety of selumetinib in patients with NF1 were identified in PubMed and EMBASE up to January 28, 2021. The pooled objective response rates (ORRs) and disease control rates (DCRs) were calculated using the DerSimonian-Laird method based on random-effects modeling. The pooled proportion of adverse events (AEs) was also calculated. The quality of the evidence was assessed using the Grading of Recommendations, Assessment, Development and Evaluation system. Five studies involving 126 patients were included in our analysis. The studies had a very low to moderate quality of the evidence. The pooled ORR was 73.8\% (95\% CI 57.3\%-85.5\%) and the DCR was 92.5\% (95\% CI 66.5\%-98.7\%). The 2 most common AEs were diarrhea, which had a pooled rate of 63.8\% (95\% CI 52.9\%-73.4\%), and an increase in creatine kinase levels, which had a pooled rate of 63.3\% (95\% CI 35.6\%-84.3\%). Our results indicate that selumetinib is an effective and safe treatment for pediatric patients with symptomatic, inoperable plexiform neurofibromas. Further larger-scale randomized controlled studies are needed to confirm the long-term outcome of patients treated with this drug. [\hyperlink{Xolremdi}{PMID: 35017312}, Jisun Hwang et al., 2022]

\hypertarget{pmid_2661789}{T}wenty-six children aged 4 to 15 years who were to receive cancer chemotherapy were enrolled in a double-blind, randomized, crossover trial that compared the antiemetic efficacy of a four-drug regimen (the MBDL regimen: metoclopramide, 8 mg/kg; benztropine, 0.04 mg/kg; dexamethasone, 0.7 mg/kg; lorazepam, 0.1 mg/kg), given over 24 hours, with the efficacy of chlorpromazine, 3.3 mg/kg, given in four doses over 24 hours. The MBDL regimen was more effective than chlorpromazine in both objective and subjective measures of antiemetic control. Of 26 children, 23 (89\%) had less vomiting on the MBDL regimen, and 20 (77\%) of 26 patients or parents preferred this regimen (p less than 0.01). The MBDL regimen reduced the number of vomiting episodes by a mean of 4.0 (p less than 0.01) and reduced the duration of vomiting by a mean of 3.7 hours (p less than 0.01). A moderate level of sedation was documented at some stage in the 24-hour period of observation in 27\% on the MBDL regimen and in 35\% receiving chlorpromazine. Dystonia was seen in 1 (4\%) of 26 children. We conclude that the MBDL regimen is safe in children and more effective than chlorpromazine. [\hyperlink{Xolremdi}{PMID: 2661789}, G Marshall et al., 1989]

\hypertarget{pmid_17401268}{T}he present study aimed at verifying the safety and efficacy of rifampicin in ameliorating pruritus in cholestatic children. Twenty-three Egyptian children (14 boys and 9 girls), suffering from intractable pruritus of cholestasis, were included. Rifampicin was started at a dose of 10 mg/Kg/day in two divided doses and increased gradually to a maximum of 20 mg/Kg/day if there was no response. Liver function tests were followed up weekly. Seventeen patients (74\%) showed improvement of pruritus with rifampicin. None of the patients showed any deterioration in liver functions. Rifampicin in a dose of 10-20 mg/Kg/day is safe and effective in ameliorating uncontrollable pruritus in children with persistent cholestasis. [\hyperlink{Xolremdi}{PMID: 17401268}, Hanaa El-Karaksy et al., 2007]

\hypertarget{pmid_24968572}{T}o examine the efficacy, safety and tolerability of tolterodine in children with overactive bladder in comparison with standard treatment i.e. oxybutynin as demonstrated in randomized clinical trials and other studies. A systematic search was done to screen the studies evaluating the effect of tolterodine in children with non-neurogenic overactive bladder. Results of studies were pooled and compared. Efficacy was determined from micturition diaries and dysfunctional voiding symptoms score. Safety and tolerability were assessed from the reported treatment emergent adverse events. A total of six randomized clinical trials and 11 other studies of tolterodine in children with urinary incontinence were included in the present systematic review. The dose of tolterodine used in different settings ranged from '0.5 to 8 mg/day' instead of '0.5 to 8 mg/kg per day' and the duration of studies ranged from 2 weeks to 12 months. Both extended and immediate release preparations of tolterodine were shown to have comparable efficacy and tolterodine proved to have comparable efficacy with better tolerability than oxybutynin in these studies. It can be concluded that tolterodine is efficacious in treatment of urinary incontinence in children. Moreover, its efficacy is comparable to oxybutynin, the most commonly prescribed anticholinergic in this condition, while having better tolerability. Hence, it can be considered as first line therapy for the treatmentof urinary incontinence in children. [\hyperlink{Xolremdi}{PMID: 24968572}, B Medhi et al., ]

\hypertarget{pmid_34374425}{W}e demonstrated in a randomized placebo-controlled trial that WRSS1, a live oral Shigella sonnei vaccine candidate, is safe in Bangladeshi adults and children, and elicits antigen-specific antibodies. Here, we describe functional antibody and innate immune responses to WRSS1. Adults (18-39 years) and children (5-9 years) received 3 doses of 3 × 105 or 3 × 106 colony forming units (CFU) of WRSS1 or placebo, 4 weeks apart; children additionally received 3 × 104 CFU. Blood and stool were collected at baseline and 7 days after each dose. Functional antibodies were measured using serum bactericidal antibody (SBA) assay. Cytokine/chemokine concentrations were measured in lymphocyte cultures. Host defense peptides LL-37, HBD-1, and HD-5 were analyzed in plasma and stool. Children showed increased SBA titers over baseline after the third dose of 3 × 106 CFU (P = .048). Significant increases of Th-17 and proinflammatory cytokines (TNF-α, G-CSF, MIP-1β), and reduction of anti-inflammatory and Th2 cytokines (IL-10, IL-13, GM-CSF) were observed in children. Plasma HBD-1 and LL-37 decreased in children after vaccination but were increased/unchanged in adults. Functional antibodies and Th1/Th17 cytokine responses in children may serve as important indicators of immunogenicity and protective potential of WRSS1. Clinical Trials Registration: NCT01813071. [\hyperlink{Xolremdi}{PMID: 34374425}, Protim Sarker et al., 2021]

\hypertarget{pmid_20128231}{T}he efficacy and safety of monomeric allergoid (Lofarma, Milan) have been demonstrated in adults but very few studies have examined it in children. This study therefore investigated the efficacy and safety of this sublingual immunotherapy (SLIT) at the dosage of 1000 AU five times a week without any up-dosing. Forty allergic children (17 M and 23 F, mean age 7 years, range 4-16 years), 16 with rhinitis and 24 with rhinitis and asthma, were randomized to SLIT or drug therapy. All the patients were sensitized to grass; some were also sensitized, though to a lesser extent, to Parietaria, Olea and Betulaceae. The patients were treated pre-/co-seasonally for two years. A visual analogue scale (VAS) was used at baseline and at the end of the first and second pollen seasons to rate the patients' well-being. The VAS score was significantly higher after both the first and the second year of treatment in the SLIT group than in the controls (p<0.05). It improved in comparison to baseline only in the active group. All 40 children tolerated the therapy very well. The monomeric allergoid at the dosage of 5000 AU/week thus appears to have a good efficacy and safety profile in children. [\hyperlink{Xolremdi}{PMID: 20128231}, F Agostinis et al., 2009]

\hypertarget{pmid_11176516}{T}olterodine was recently approved for the treatment of incontinence and overactive bladder in adults, and has fewer side effects than oxybutynin. We evaluated the safety and efficacy of tolterodine in children with dysfunctional voiding. We retrospectively reviewed our experience with 30 pediatric patients treated with tolterodine for a primary diagnosis of dysfunctional voiding. Patients were treated with adult doses of tolterodine and behavioral modifications. Standard definitions determined by the International Children's Continence Society were adapted to designate final treatment outcomes on medication as cured-greater than 90\% reduction in wetting episodes, improved-greater than 50\% reduction or failed-less than 50\% reduction. The children were 4 to 17 years old (mean age 8.7) and were treated with tolterodine for an average of 5.2 months. The final dose was 1 mg. twice daily in 1, 2 mg. twice daily in 27 and 4 mg. twice daily in 2 patients. Wetting episodes were cured in 10 (33\%), improved in 12 (40\%), and failed to show improvement in 8 (27\%) cases. Four patients (13.3\%) reported side effects and only 1 discontinued the medication due to diarrhea. There were no reports of hyperpyrexia, flushing or intolerance to sunshine and outdoor temperature. Our results demonstrate that tolterodine at adult doses without titration can be used safely to decrease wetting episodes in children with dysfunctional voiding. Controlled clinical trials should be completed to evaluate further efficacy and safety in children. [\hyperlink{Xolremdi}{PMID: 11176516}, M Munding et al., 2001]

\hypertarget{pmid_3894338}{T}reatment-emergent symptoms and behavioral changes were assessed during an 8-week double-blind study comparing molindone and thioridazine in 31 aggressive hospitalized children (ages 6-11). Molindone was found to be as effective as thioridazine in this sample. Adverse effects differed (nonsignificantly) for the two groups, with more sedation in the thioridazine-treated children. Clinical records from 6 adolescent inpatients treated with molindone were reviewed as a comparison group. Molindone is a relatively safe neuroleptic for child and adolescent inpatients because of its short half-life and minimal prolonged tissue accumulation. Additional studies on different child populations are necessary before the proper indications for molindone usage in the pediatric group can be established. [\hyperlink{Xolremdi}{PMID: 3894338}, L L Greenhill et al., 1985]

\hypertarget{pmid_8010205}{T}he purpose of this prospective study was to evaluate the safety and efficacy of thioridazine as an adjunct to chloral hydrate sedation when children undergoing MR imaging are difficult to sedate. All 87 children in the study either could not be sedated with chloral hydrate alone or were mentally retarded. Thioridazine (2-4 mg/kg) was administered orally 2 hr before and chloral hydrate (50-100 mg/kg) was administered orally 30 min before the 104 MR examinations. All children were monitored by continuous pulse oximetry. All images were individually evaluated by pediatric radiologists and were graded acceptable if they contained only minimal motion artifact or no motion artifact. Studies were considered successful only when 95\% or more of the images were acceptable. MR imaging was successful in 93 (89\%) of 104 examinations. The success rate for children entered into the study because of prior failure of chloral hydrate sedation was not significantly different from the success rate for children with mental retardation. A tendency for increasing failure rate with age was not significant. No serious complications occurred during the study. The most common adverse reaction, transient reduced oxygen saturation, was seen in five children. Other adverse effects encountered were vomiting in four children, hyperactivity in two children, transient tachycardia in one child, and prolonged sedation in one child. No child required hospitalization because of an adverse reaction to sedation. The study indicates that thioridazine is a safe and effective adjunct to chloral hydrate when a child undergoing MR imaging is difficult to sedate. [\hyperlink{Xolremdi}{PMID: 8010205}, S B Greenberg et al., 1994]

\hypertarget{pmid_33730099}{O}ral ivermectin is a safe broad spectrum anthelminthic used for treating several neglected tropical diseases (NTDs). Currently, ivermectin use is contraindicated in children weighing less than 15 kg, restricting access to this drug for the treatment of NTDs. Here we provide an updated systematic review of the literature and we conducted an individual-level patient data (IPD) meta-analysis describing the safety of ivermectin in children weighing less than 15 kg. A systematic review was conducted using the Preferred Reporting Items for Systematic Reviews and Meta-Analyses (PRISMA) for IPD guidelines by searching MEDLINE via PubMed, Web of Science, Ovid Embase, LILACS, Cochrane Database of Systematic Reviews, TOXLINE for all clinical trials, case series, case reports, and database entries for reports on the use of ivermectin in children weighing less than 15 kg that were published between 1 January 1980 to 25 October 2019. The protocol was registered in the International Prospective Register of Systematic Reviews (PROSPERO): CRD42017056515. A total of 3,730 publications were identified, 97 were selected for potential inclusion, but only 17 sources describing 15 studies met the minimum criteria which consisted of known weights of children less than 15 kg linked to possible adverse events, and provided comprehensive IPD. A total of 1,088 children weighing less than 15 kg were administered oral ivermectin for one of the following indications: scabies, mass drug administration for scabies control, crusted scabies, cutaneous larva migrans, myiasis, pthiriasis, strongyloidiasis, trichuriasis, and parasitic disease of unknown origin. Overall a total of 1.4\% (15/1,088) of children experienced 18 adverse events all of which were mild and self-limiting. No serious adverse events were reported. Existing limited data suggest that oral ivermectin in children weighing less than 15 kilograms is safe. Data from well-designed clinical trials are needed to provide further assurance. [\hyperlink{Xolremdi}{PMID: 33730099}, Podjanee Jittamala et al., 2021]

\hypertarget{pmid_30463814}{D}exmendetomidine hydrochloride (DEX) is a new common adrenergic receptor agonist, which not only keeps children calm but also has analgesic effect. Dexmedetomidine hydrochloride will enable children to maintain the natural non-REM sleep, which can be stimulated sedation or language arousal. The aim of this study is to observe the sedative effect and adverse drug reactions of dexmedetomidine hydrochloride injection and propofol injection in MRI examination. In this study, no children in the experimental group were required to add sedative drugs, and 2 cases in the control group were treated with sedative drugs. In experimental group, it used dexmedetomidine hydrochloride as (1.64±0.91) g/kg; in control group, dosage of narcotic drugs as (5.26±1.82) g/kg, and the total complication rate of the children in the experimental group was lower than that of the control group (P<0.05). After returning to the ward, the doses of phenobarbital sedation were dexmedetomidine group (4.28±1.53) mg/kg and propofol group (6.40±1.71) mg/kg. There was significant difference between the two groups. The total complication rate in the experimental group was lower than that in the control group (P<0.05). The quality of MRI in the test group was significantly higher than that in the control group, which showed that dexmedetomidine hydrochloride could provide a satisfactory sedative effect in the MRI examination of children. To sum up, dexmedetomidine hydrochloride is a wide range of clinical applications. It is an effective drug for the maintenance of sedation in clinical disease treatment. It is flexible in the way of administration and with less adverse reactions. It is suitable for popularization and application in clinical practice. [\hyperlink{Xolremdi}{PMID: 30463814}, Zhendong Yang et al., 2018]

\hypertarget{pmid_8929382}{T}he safety and efficacy of intravenous gadodiamide injection, 0.1 mmol/kg body weight, have been evaluated in an open label, non-comparative as to drug, phase III clinical trial in 50 children from 6 months to 13 years of age, referred for MRI requiring the injection of a contrast medium. The central nervous system and other body areas were examined with T1 sequences before and after intravenous injection of the contrast medium. Overall safety was very good and no clinically relevant changes were evident as regards heart rate and venous blood oxygen saturation after injection. No adverse event or discomfort was experienced by conscious patients that could with certainty be related to the contrast medium, but slight movements were observed in two sedated patients that could be related to the injection. Comparing pre- and post-injection images, additional diagnostic information could be obtained from the latter in 41 patients (82\%). In these images, the number of lesions detected increased and they were generally better delineated and their size more easily estimated. The results of this trial indicate that gadodiamide injection is safe and effective for MRI examinations in children. [\hyperlink{Xolremdi}{PMID: 8929382}, S Hanquinet et al., 1996]

\hypertarget{pmid_15740542}{C}o-treatment with Saccharomyces boulardii appears to lower the risk of antibiotic-associated diarrhoea in adults receiving broad-spectrum antibiotics. To determine whether S. boulardii prevents antibiotic-associated diarrhoea in children. A total of 269 children (aged 6 months to 14 years) with otitis media and/or respiratory tract infections were enrolled in a double-blind, randomized placebo-controlled trial in which they received standard antibiotic treatment plus 250 mg of S. boulardii (experimental group, n = 132) or a placebo (control group, n = 137) orally twice daily for the duration of antibiotic treatment. Analyses were based on allocated treatment and included data from 246 children. Patients receiving S. boulardii had a lower prevalence of diarrhoea (> or =3 loose or watery stools/day for > or =48 h occurring during or up to 2 weeks after the antibiotic therapy) than those receiving placebo [nine of 119 (8\%) vs. 29 of 127 (23\%), relative risk: 0.3, 95\% confidence interval: 0.2-0.7]. S. boulardii also reduced the risk of antibiotic-associated diarrhoea (diarrhoea caused by Clostridium difficile or otherwise unexplained diarrhoea) compared with placebo [four of 119 (3.4\%) vs. 22 of 127 (17.3\%), relative risk: 0.2; 95\% confidence interval: 0.07-0.5]. No adverse events were observed. This is the first randomized-controlled trial evidence that S. boulardii effectively reduces the risk of antibiotic-associated diarrhoea in children. [\hyperlink{Xolremdi}{PMID: 15740542}, M Kotowska et al., 2005]

\hypertarget{pmid_24627951}{T}o determine the safety and efficacy of high-dose oral chloral hydrate for pediatric ophthalmic procedures. This study is a retrospective review of a quality audit of pediatric sedation for ophthalmic evaluation and imaging performed at King Khaled Eye Specialist Hospital between January 1 and December 31, 2011, in children aged 1 month to 6 years. Three hundred fifty-eight of 380 (94.2\%) sedation procedures were successful after a single dose of chloral hydrate, with 356 of 380 (93.7\%) children sedated within 45 minutes of the first dose. The total success rate of the sedation procedure increased to 97.9\% (372 of 380) when a second dose was administered. Children adequately sedated after a single dose of chloral hydrate were on average younger and weighed less than children who required additional doses. No major adverse events were documented. The use of chloral hydrate sedation for ophthalmic evaluation and imaging was safe and effective in this patient population with a high rate of procedure completion. [\hyperlink{Xolremdi}{PMID: 24627951}, Michelle E Wilson et al., ]

\hypertarget{pmid_24447296}{C}hloral hydrate is the most commonly used sedative for paediatric diagnostic procedures in China with a success rate of around 80\%. Intranasal dexmedetomidine is used for rescue sedation in our centre. This prospective investigation evaluated 213 children aged one month to 10 years who were not adequately sedated following administration of chloral hydrate. Children were randomly assigned to receive rescue intranasal dexmedetomidine at 1 μg.kg(-1) (group 1), 1.5 μg.kg(-1) (group 2) or 2 μg.kg(-1) (group 3). The sedation level was assessed every 10 min using a modified observer's assessment of alertness/sedation scale. Successful rescue sedation in groups 1, 2 and 3 were 56 (83.6\%), 66 (89.2\%) and 51 (96.2\%), respectively. Increasing the rescue dose was associated with an increased success rate with an odds ratio of 4.12 (95\% CI 1.13-14.98), p = 0.032. We conclude that intranasal dexmedetomidine is effective for sedation in children who do not respond to chloral hydrate.  [\hyperlink{Xolremdi}{PMID: 24447296}, B L Li et al., 2014] Topical tacrolimus is used off-label in young children, but data are limited on its use in children under 2 years of age and for long-term treatment. To compare safety differences between topical tacrolimus (0.03\% and 0.1\% ointments) and topical corticosteroids (mild and moderate potency) in young children with atopic dermatitis (AD). We conducted a 36-month follow-up study with 152 young children aged 1-3 years with moderate to severe AD. The children were followed up prospectively, and data were collected on infections, disease severity, growth parameters, vaccination responses and other relevant laboratory tests were gathered. There were no significant differences between the treatment groups for skin-related infections (SRIs) (P = 0.20), non-SRIs (P = 0.20), growth parameters height (P = 0.60), body weight (P = 0.81), Eczema Area and Severity Index (EASI) (P = 0.19), vaccination responses (P = 0.62), serum cortisone levels (P = 0.23) or serum levels of interleukin (IL)-4, IL-10, IL-12, IL-31 and interferon-γ. EASI decreased significantly in both groups (P < 0.001). In the tacrolimus group, nine patients (11.68\%) had detectable tacrolimus blood concentrations at the 1-week visit. There were no malignancies or severe infections during the study, and blood eosinophil counts were similar in both groups. Topical tacrolimus (0.03\% and 0.1\%) and topical corticosteroids (mild and moderate potency) are safe to use in young children with moderate to severe AD, and have comparable efficacy and safety profiles. [\hyperlink{Xolremdi}{PMID: 24447296}, A Salava et al., 2022]

\hypertarget{pmid_11298060}{T}o determine the safety, efficacy and pharmacokinetics of tolterodine in children with an overactive bladder. Thirty-three children (20 boys and 13 girls, aged 5-10 years) with an overactive bladder and symptoms of urgency, frequency and/or urge incontinence were enrolled in an open, dose-escalation study. Patients were treated with oral tolterodine 0.5 mg (n = 11), 1 mg (n = 10) or 2 mg (n = 12) twice daily for 14 days. The primary safety endpoint was the change in residual urinary volume, as determined by ultrasonography. In addition, voiding diary variables (frequency and incontinence episodes) and pharmacokinetics were evaluated. Other safety endpoints included laboratory variables, electrocardiogram recordings and reported adverse events. There were no safety concerns in terms of the change in residual urinary volume for any of the three dosage groups; values were comparable with baseline after 2 weeks of treatment for all three dosages. Adverse events were reported by 20 patients (six on 0.5 mg, five on 1 mg, and nine on 2 mg). Most adverse events were not considered to be drug-related; of the 13 possibly related events, 10 occurred in those taking 2 mg. Headache was the most commonly reported adverse event. No serious adverse events were reported and there were no general safety concerns. There was an improvement in voiding diary variables in all treatment groups after 2 weeks of treatment, although the efficacy was greatest in those taking 1 mg and 2 mg. Pharmacokinetic findings were consistent with dose linearity over the range 0.5-2 mg. The results support the use of 1 mg twice daily as the optimal dose of tolterodine for treating children aged 5-10 years with an overactive bladder. [\hyperlink{Xolremdi}{PMID: 11298060}, K Hjälmås et al., 2001]

\hypertarget{pmid_15074060}{O}veractive bladder (OB) is one of the no-neurogenic voiding dysfunctions whose prevalence has been precisely defined among the general population but not so among the paediatric population. Its clinical manifestations are various, and its association with other pathologies like enuresis, vesico-ureteral reflux (VUR) and recurrent infections is particularly significant in children. OB is basically managed with anticholinergic drugs. The efficacy of oxybutynin chloride has been sufficiently proved; however its dosage and side effects, although scarce in children, usually cause treatment discontinuation. Tolterodine has been successfully used as an alternative therapy of OB in adults, however its use has not been sufficiently evaluated in children. Our objective is to determine tolterodine's efficacy and tolerability in the paediatric population suffering from OB. A retrospective study of 72 children who were diagnosed no-neurogenic OB and who received no previous treatment. A concomitant urological pathology diagnostic protocol was applied to all cases, as well as a urodynamic test (UDT) and a neurological examination. Post-treatment UDT was performed to one group of patients. The mean age was 10.9 years and the children were assessed between 4 and 31 months after treatment initiation. Healing was proved through cistomanometry in 67\% of the cases, there was improvement in 14\% and 19\% of the patients showed no changes in the UDT. Following the criteria of the International Children's Continence Society (ICCS) applied to those children with no post-treatment UDT, 51\% were healed, 27\% improved and 22\% experienced no changes. None of the patients had to discontinue the treatment due to side effects. Tolterodine's tolerability and efficacy are good within the paediatric population, which turns it into an alternative to the traditional anticholinergics for the treatment of OB. [\hyperlink{Xolremdi}{PMID: 15074060}, J M Garat Barredo et al., 2004]

\hypertarget{pmid_16532329}{C}hylothorax is a rare but life-threatening condition in children. To date, there is no commonly accepted treatment protocol. Somatostatin and octreotide have recently been used for treating chylothorax in children. We set out to summarise the evidence on the efficacy and safety of somatostatin and octreotide in treating young children with chylothorax. Systematic review: literature search (Cochrane Library, EMBASE and PubMed databases) and literature hand search of peer reviewed articles on the use of somatostatin and octreotide in childhood chylothorax. Thirty-five children treated for primary or secondary chylothorax (10/somatostatin, 25/octreotide) were found. Ten of the 35 children had been given somatostatin, as i.v. infusion at a median dose of 204 microg/kg/day, for a median duration of 9.5 days. The remaining 25 children had received octreotide, either as an i.v. infusion at a median dose of 68 microg/kg/day over a median 7 days, or s.c. at a median dose of 40 microg/kg/day and a median duration of 17 days. Side effects such as cutaneous flush, nausea, loose stools, transient hypothyroidism, elevated liver function tests and strangulation-ileus (in a child with asplenia syndrome) were reported for somatostatin; transient abdominal distension, temporary hyperglycaemia and necrotising enterocolitis (in a child with aortic coarctation) for octreotide. A positive treatment effect was evident for both somatostatin and octreotide in the majority of reports. Minor side effects have been reported, however caution should be exercised in patients with an increased risk of vascular compromise as to avoid serious side effects. Systematic clinical research is needed to establish treatment efficacy and to develop a safe treatment protocol. [\hyperlink{Xolremdi}{PMID: 16532329}, Charles C Roehr et al., 2006]

\hypertarget{pmid_36283870}{S}edation techniques and drugs are increasingly used in children undergoing imaging procedures. In this systematic review and meta-analysis, we present an overview of literature concerning sedation of children aged 0-8 yr for magnetic resonance imaging (MRI) procedures using needle-free pharmacological techniques. Embase, MEDLINE, Web of Science, and Cochrane databases were systematically searched for studies on the use of needle-free pharmacological sedation techniques for MRI procedures in children aged 0-8 yr. Studies using i.v. or i.m. medication or advanced airway devices were excluded. We performed a meta-analysis on sedation success rate. Secondary outcomes were onset time, duration, recovery, and adverse events. Sixty-seven studies were included, with 22 380 participants. The pooled success rate for oral chloral hydrate was 94\% (95\% confidence interval [CI]: 0.91-0.96); for oral chloral hydrate and intranasal dexmedetomidine 95\% (95\% CI: 0.92-0.97); for rectal, oral, or intranasal midazolam 36\% (95\% CI: 0.14-0.65); for oral pentobarbital 99\% (95\% CI: 0.90-1.00); for rectal thiopental 92\% (95\% CI: 0.85-0.96); for oral melatonin 75\% (95\% CI: 0.54-0.89); for intranasal dexmedetomidine 62\% (95\% CI: 0.38-0.82); for intranasal dexmedetomidine and midazolam 94\% (95\% CI: 0.78-0.99); and for inhaled sevoflurane 98\% (95\% CI: 0.97-0.99). We found a large variation in medication, dosage, and route of administration for needle-free sedation. Success rates for sedation techniques varied between 36\% and 98\%. [\hyperlink{Xolremdi}{PMID: 36283870}, Ingeborg de Rover et al., 2023]

\section*{Zaleplon}
\subsection*{Result}
\subsubsection*{Answer}

Unknown

\subsubsection*{{Explanation}}
\hypertarget{Zaleplon}
A review of the available abstracts reveals the following regarding the safety of Zaleplon in children:

- Most abstracts focus on adult or elderly populations, with multiple randomized controlled trials and safety studies in adults and the elderly, but not in children.
- One abstract [\hyperlink{pmid_17724894}{PMID: 17724894}, Jacek Sein Anand et al., 2007] describes a single case report of a 15-year-old female who ingested a large overdose of Zaleplon (60 mg, 1.2 mg/kg) in a suicide attempt. The clinical course was mild, requiring only symptomatic treatment, and the patient was discharged after 36 hours. However, this is a case report of acute overdose, not a targeted safety study of therapeutic use in children.
- Another abstract [\hyperlink{pmid_11117652}{PMID: 11117652}, K W Weitzel et al., 2000] specifically states: "Zaleplon has not been studied in pregnant women or children."
- No abstracts report a targeted, controlled study of Zaleplon safety in children (defined as a study designed to assess safety in a pediatric population, with therapeutic dosing and safety outcomes).
- No abstracts report Zaleplon being studied in children under 18 for insomnia or any other indication, nor do they affirm its safety in this population.

Therefore, based on the abstracts available, there is no evidence from targeted safety studies to affirm that Zaleplon is safe or unsafe for use in children of any age range. The safety of Zaleplon in children is therefore unknown.

\subsection*{Abstracts}
\hypertarget{pmid_33320969}{Z}aleplon (ZP) is a sedative and hypnotic drug used for the treatment of insomnia. Despite its potent anticonvulsant activity, ZP is not commonly used for the treatment of convulsion since ZP is characterized by its low oral bioavailability as a result of poor solubility and extensive liver metabolism. The following study aimed to formulate specifically controlled release nano-vehicles for oral and parenteral delivery of ZP to enhance its oral bioavailability and biological activity. A modified single emulsification-solvent evaporation method of sonication force was adopted to optimize the inclusion of ZP into biodegradable nanoparticles (NPs) using poly (dl-lactic-co-glycolic acid) (PLGA). The impacts of various formulation variables on the physicochemical characteristics of the ZP-PLGA-NPs and drug release profiles were investigated. Pharmacokinetics and pharmacological activity of ZP-PLGA-NPs were studied using experimental animals and were compared with generic ZP tablets. Assessment of gamma-aminobutyric acid (GABA) level in plasma after oral administration was conducted using enzyme-linked immunosorbent assay. The maximal electroshock-induced seizures model evaluated anticonvulsant activity after the parenteral administration of ZP-loaded NPs. The prepared ZP-PLGA NPs were negatively charged spherical particles with an average size of 120-300 nm. Optimized ZP-PLGA NPs showed higher plasma GABA levels, longer sedative, hypnotic effects, and a 3.42-fold augmentation in oral drug bioavailability in comparison to ZP-marketed products. Moreover, parenteral administration of ZP-NPs showed higher anticonvulsant activity compared to free drug. Oral administration of ZP-PLGA NPs achieved a significant improvement in the drug bioavailability, and parenteral administration showed a pronounced anticonvulsant activity. [\hyperlink{Zaleplon}{PMID: 33320969}, Yusuf A Haggag et al., 2021]

\hypertarget{pmid_10485636}{Z}aleplon is a short-acting pyrazolopyrimidine hypnotic with a rapid onset of action. This multicenter study compared the efficacy and safety of 3 doses of zaleplon with those of placebo in outpatients with DSM-III-R insomnia. Zolpidem, 10 mg, was used as an active comparator. After a 7-night placebo (baseline) period, 615 adult patients were randomly assigned to receive, in double-blind fashion, I of 5 treatments (zaleplon, 5, 10, or 20 mg; zolpidem, 10 mg; or placebo) for 28 nights, followed by placebo treatment for 3 nights. Sleep latency, sleep maintenance, and sleep quality were determined from sleep questionnaires that patients completed each morning. The occurrence of rebound insomnia and withdrawal effects on discontinuation of treatment was also assessed. All levels of significance were p < or = .05. Median sleep latency was significantly lower with zaleplon, 10 and 20 mg, than with placebo during all 4 weeks of treatment and with zaleplon, 5 mg, for the first 3 weeks. Zaleplon, 20 mg, also significantly increased sleep duration compared with placebo in all but week 3 of the study. There was no evidence of rebound insomnia or withdrawal symptoms after discontinuation of 4 weeks of zaleplon treatment. Zolpidem, 10 mg, significantly decreased sleep latency, increased sleep duration, and improved sleep quality at most timepoints compared with placebo; however, after discontinuation of zolpidem treatment, the incidence of withdrawal symptoms was significantly greater than that with placebo and there was an indication of significant rebound insomnia for some patients in the zolpidem group compared with those in the placebo group. The frequency of adverse events in the active treatment groups did not differ significantly from that in the placebo group. Zaleplon is effective in the treatment of insomnia. In addition, zaleplon appears to provide a favorable safety profile, as indicated by the absence of rebound insomnia and withdrawal symptoms once treatment was discontinued. [\hyperlink{Zaleplon}{PMID: 10485636}, R Elie et al., 1999]

\hypertarget{pmid_29189154}{Z}aleplon is a pyrazolopyrimidin derivative hypnotic drug indicated for the short-term management of insomnia. Zaleplon belongs to Class II drugs, according to the biopharmaceutical classification system (BCS), showing poor solubility and high permeability. It undergoes extensive first-pass hepatic metabolism after oral absorption, with only 30\% of Zaleplon being systemically available. It is available in tablet form which is unable to overcome the previous problems. The aim of this study is to enhance solubility and bioavailability via utilizing nanotechnology in the formulation of intranasal Zaleplon nano-emulsion (ZP-NE) to bypass the barriers and deliver an effective therapy to the brain. Screening studies were carried out wherein the solubility of zaleplon in various oils, surfactants( S) and co-surfactants(CoS) were estimated. Pseudo-ternary phase diagrams were constructed and various nano-emulsion formulations were prepared. These formulations were subjected to thermodynamic stability, in-vitro characterization, histopathological studies and assessment of the gamma aminobutyric acid (GABA) level in plasma and brain in rabbits compared to the market product (Sleep aid®). Stable NEs were successfully developed with a particle size range of 44.6±3.4 to 136.9±1.6 nm. A NE composed of 10\% Miglyol® 812, 40\% Cremophor® RH40 40\%Transcutol® HP and 10\% water successfully enhanced the bioavailability and brain targeting in the rabbits, showing a three to four folds increase than the marketed product. [\hyperlink{Zaleplon}{PMID: 29189154}, Eman Abd-Elrasheed et al., 2018]

\hypertarget{pmid_19874656}{Z}aleplon (Sonata) is a sedative hypnotic prescription medication used for the short-term treatment of insomnia. Although Zaleplon was approved by the FDA in 1999, there has been limited postmortem information about the drug cited in the toxicology literature. Zaleplon was separated from postmortem biological specimens utilizing liquid-liquid extraction coupled with a solid-phase extraction technique, and detection was accomplished by a gas chromatography-electron capture detector. The method was linear from 5.0 to 150 ng/mL with the limit of quantitation and detection determined to be 3.0 and 0.50 ng/mL, respectively. The postmortem tissue distribution of zaleplon in seven cases was as follows: 6.1-1490 ng/mL central blood (seven cases), < 3.0-503 ng/mL femoral blood (five cases), 108 ng/mL harvest blood (one case), 343-679 ng/g liver (four cases), 950 ng/g spleen (one case), < 3.0-85 ng/mL bile (three cases), 3.8-106 ng/mL urine (four cases), < 3.0-486 ng/mL vitreous humor (five cases), and 0.005-3.4 mg total gastric contents (four cases). A validated method for the analysis of zaleplon and postmortem concentrations of autopsy specimens are reported to aid the forensic toxicologist with interpretation of future casework. [\hyperlink{Zaleplon}{PMID: 19874656}, Daniel T Anderson et al., 2009]

\hypertarget{pmid_23616704}{Z}aleplon is a pyrazolopyrimidine hypnotic used for the treatment of insomnia. Zaleplon binds preferentially at the α1β2γ2 subunit of gamma aminobutyric acid type A (GABAA) receptors in the central nervous system, and has a half-life of about one hour. Efficacy studies show that zaleplon is a suitable hypnotic for sleep initiation purposes. However, because of its short half-life, zaleplon is less effective in sleep maintenance when compared with other hypnotics. Nevertheless, zaleplon does increase total sleep time. No rebound effects are observed after treatment discontinuation. The use of zaleplon is relatively safe. Adverse effects are mild and of short duration. No important interactions have been reported, and there is no evidence of abuse potential. Relative to benzodiazepine hypnotics, the biggest advantage of zaleplon is that current evidence suggests it does not produce residual next-day effects. As early as four hours after intake of zaleplon, no effects on cognitive, memory, psychomotor performance, and the ability to drive a car have been reported. Future studies should confirm these findings, and comparisons with new nonbenzodiazepine hypnotics should determine the importance of zaleplon in the future treatment of insomnia. [\hyperlink{Zaleplon}{PMID: 23616704}, Marieke M Ebbens et al., 2010]

\hypertarget{pmid_10960882}{I}nsomnia is a frequent complaint in the elderly population. Hypnotic agents, including benzodiazepines, with longer pharmacological half-lives have been associated with side effects, including residual sedation, memory impairment, and discontinuation effects. Zaleplon is a short-acting (elimination half-life of 1 hour), non-benzodiazepine hypnotic that acts on the benzodiazepine type 1 site of the gamma-aminobutyric acid type A (GABA(A)) receptor complex. The pharmacology and pharmacokinetics of Zaleplon suggest a safety profile that is improved over other hypnotics. The objective of this placebo-controlled study was to evaluate the efficacy and safety of Zaleplon (5 and 10 mg) in elderly (> or =65 years) outpatients with primary insomnia. This was a multicenter, double-blind, randomised, placebo-controlled 2-week outpatient study. Postsleep questionnaires were used to record subjective sleep variables: sleep latency, sleep duration, number of awakenings, and sleep quality. Zaleplon significantly reduced subjective sleep latency during both weeks of the study with both 5- and 10-mg doses. Subjective sleep quality was improved for significantly more patients treated with zaleplon 10 mg than those treated with placebo during both weeks of treatment. There was a weak indication of rebound insomnia after discontinuation of treatment with the 10-mg dose, but no significant difference in common treatment-emergent adverse events across treatment groups. Zaleplon is an effective and safe hypnotic for the treatment of insomnia in the elderly. [\hyperlink{Zaleplon}{PMID: 10960882}, J Hedner et al., 2000]

\hypertarget{pmid_10831020}{T}wenty-four healthy male and female subjects, who participated in this randomized, double-blind, crossover study, received single nighttime doses of zaleplon 10 mg (therapeutic dose), zaleplon 20 mg, zolpidem 10 mg (therapeutic dose), zolpidem 20 mg, triazolam 0.25 mg (positive control), and placebo. Subjective behavioral ratings and psychomotor tests were completed before and 1.25 and 8.25 hours after administration of the study drug. The Immediate and Delayed Word Recall tests and the Digit Span Test were used to assess memory. The Digit-Symbol Substitution Test, Paired Associates Learning Test, and Divided Attention Test were used to assess other cognitive skills. Zaleplon 10 mg did not produce any significant changes in memory or learning compared with placebo. All other active treatments, including zolpidem 10 mg, caused psychomotor impairment at the 1.25-hour test battery. Zolpidem 20 mg (twice the therapeutic dose) produced more psychomotor impairment at the 1.25-hour assessment than did any of the other active treatments, including zaleplon 20 mg. At the 8.25-hour time point, test scores for subjects who received zaleplon 10 mg and 20 mg did not differ from the test scores for those who received placebo. However, cognitive impairment persisted up to the 8.25-hour observation for subjects who were administered triazolam 0.25 mg and zolpidem 20 mg. Adverse events associated with the use of zaleplon were transient and mild-to-moderate in severity. Overall, this study shows that zaleplon is a safe hypnotic that does not affect memory, learning, or psychomotor skills associated with vigilance. [\hyperlink{Zaleplon}{PMID: 10831020}, S M Troy et al., 2000]

\hypertarget{pmid_17724894}{T}here has been little data in the medical literature about intoxication with a new hypnotic agent zaleplon. The zaleplon, chemically N-[3-(3-cyanopyrazolo[1,5-a]pyrimidin-7-yl)phenyl]-N-ethylacetamid, is a selective agonist of the benzodiazepine omega 1 receptor subtype. The case of a 15-year-old female who eat 60 mg of zaleplon (1.2 mg/kg) because of suicidal attempt was described. At the admission to the hospital the somnolence, blurred speech, slowdown, ataxia, tachycardia and hypokalaemia were observed. The child was treated symptomatically, and discharged from the hospital for further psychologic treatment after 36 hours. Acute intoxication with zaleplon had mild clinical course. The signs of intoxications were drowsiness, blurred speech, ataxia, tachycardia, dizziness, confusion and vomiting. The described case required only symptomatic treatment. [\hyperlink{Zaleplon}{PMID: 17724894}, Jacek Sein Anand et al., 2007]

\hypertarget{pmid_11219331}{Z}aleplon is a non-benzodiazepine sleep medication that shows efficacy as a sleep inducer comparable to that of other hypnotics but with significantly fewer residual effects. In addition, evaluations of psychomotor or memory function at zaleplon peak plasma levels show much less impairment than noted with other hypnotics, suggesting an improved benefit-to-risk profile for zaleplon compared with older available agents. Thus, zaleplon can be used to treat symptoms of insomnia when they occur without the concern of next-day psychomotor or memory impairment, whether administered at bedtime or later during the night. Such an approach permits physicians to reformulate their strategies for safe and effective management of sleeplessness. [\hyperlink{Zaleplon}{PMID: 11219331}, R M Mangano et al., 2001]

\hypertarget{pmid_15014684}{B}ACKGROUND: Insomnia is a very common symptom, particularly in the elderly. Thus, all hypnotic medications should be carefully evaluated in the elderly population. Zaleplon, a new nonbenzodiazepine hypnotic with a short elimination half-life (approximately 1 hour), was evaluated in the current study. METHOD: This multicenter, randomized, placebo-controlled outpatient study evaluated the efficacy and safety of zaleplon, 5 and 10 mg, in elderly patients with insomnia (as defined by DSM-IV); zolpidem, 5 mg, was the active comparator. Sleep was assessed in 549 elderly patients (>/= 65 years old) by using morning questionnaires completed after each of 7 baseline nights during which placebo was given, 14 nights of double-blind treatment, and 7 nights of placebo after discontinuation of active treatment. RESULTS: Zaleplon, 10 mg, and zolpidem, 5 mg, significantly reduced sleep latency during both weeks of the study. Zaleplon, 5 mg, reduced sleep latency only during week 2. Sleep duration was increased with zolpidem, 5 mg, during weeks 1 and 2 and with zaleplon, 10 mg, during week 1. No clinically significant rebound insomnia was observed after discontinuation of treatment with zaleplon, whereas evidence of rebound effects was seen with zolpidem. There was no significant difference between either zaleplon dose and placebo in the frequency of any central nervous system adverse events. CONCLUSION: Zaleplon is effective in reducing latency to sleep without evidence of undesired effects in elderly patients with insomnia. [\hyperlink{Zaleplon}{PMID: 15014684}, Sonia Ancoli-Israel et al., 1999]

\hypertarget{pmid_10870872}{T}he efficacy and safety of three doses of zaleplon, a novel non-benzodiazepine hypnotic, were compared with those of placebo in outpatients with insomnia in this 4-week study, using zolpidem 10 mg as active comparator. Postsleep questionnaires were used to determine treatment effects on the patient's perception of sleep, as well as any development of pharmacological tolerance during therapy or rebound insomnia or withdrawal symptoms upon discontinuation of therapy. During week 1, sleep latency was significantly shorter with zaleplon 5, 10, and 20 mg compared to placebo. The significant decrease in sleep latency persisted through week 4 with zaleplon 20 mg, and was again evident with zaleplon 10 mg at week 3. Zaleplon 20 mg also had significant effects on sleep duration, number of awakenings, and sleep quality compared to placebo. No pharmacological tolerance developed during treatment with zaleplon and there were no indications of rebound insomnia or withdrawal symptoms after treatment discontinuation. Zolpidem 10 mg had significant effects on sleep latency, sleep duration, and sleep quality compared to placebo. However, a significantly greater incidence of withdrawal symptoms and a suggestion of sleep difficulty after treatment discontinuation (rebound insomnia) for all sleep measures was seen with zolpidem compared to placebo. There was no significant difference in the frequency of adverse events with active treatment compared to placebo. These results show that zaleplon provides effective treatment of insomnia with a favourable safety profile. [\hyperlink{Zaleplon}{PMID: 10870872}, J Fry et al., 2000]

\hypertarget{pmid_11192136}{T}his study compared the pharmacokinetics, pharmacodynamics, and pharmacokinetic/pharmacodynamic (PK/PD) profile of zaleplon, a new pyrazolopyrimidine hypnotic, with those of zolpidem and placebo. This was a double-blind, 5-period crossover study in which healthy volunteers with no history of sleeping disorder were randomized to 10- or 20-mg oral doses of zaleplon, 10- or 20-mg oral doses of zolpidem, or placebo. The pharmacokinetic characteristics of the active drugs were estimated using a noncompartmental method and NONMEM. Pharmacodynamic characteristics were determined using psychophysical tests, including measures of sedation, mood, mental and motor speed, and recent and remote recall. Results of these tests were used to compare the drugs' relative PK/PD profiles. Ten healthy male and female volunteers, aged 23 to 31 years, took part in the study. The apparent elimination half-life of zaleplon (60.1+/-8.9 min) was significantly shorter than that of zolpidem (124.5+/-37.9 min) (P < 0.001). Zaleplon produced less sedation than zolpidem at the 2 doses studied (P < 0.001). The sedation scores of the zaleplon groups returned to baseline in less time than those of the zolpidem groups (4 vs 8 hours; P < 0.05). Zaleplon had no effect on recent or remote recall, whereas zolpidem had a significant effect on both measures (P < 0.05). In this study in 10 young, healthy volunteers, zaleplon was eliminated more rapidly, produced no memory loss, and caused less sedation than zolpidem at the same doses. [\hyperlink{Zaleplon}{PMID: 11192136}, D Drover et al., 2000]

\hypertarget{pmid_11117652}{I}nsomnia is the subjective complaint of poor sleep or an inadequate amount of sleep that adversely affects daily functioning. For the past 4 decades, treatment of insomnia has shifted away from the use of barbiturates toward the use of hypnotic agents of the benzodiazepine class. However, problems associated with the latter (eg, next-day sedation, rebound insomnia, dependence, and tolerance) have prompted development of other agents. This review describes the recently approved nonbenzodiazepine agent, zaleplon. Studies of zaleplon were identified through a search of English-language articles listed in MEDLINE and International Pharmaceutical Abstracts, with no limitation on year. These were supplemented by educational materials from conferences. The efficacy and tolerability of zaleplon have been documented in the literature. Zaleplon has been shown to improve sleep variables in comparison with placebo. Like most hypnotic agents, zaleplon can be used for problems of sleep initiation at the beginning of the night, but its short duration of clinical effect may also allow patients to take it later in the night without residual effects the next morning. Zaleplon can be taken < or = 2 hours before awakening without "hangover" effects. It is generally well tolerated, with headache being the most commonly reported adverse event in clinical trials (15\%-18\%). Compared with flurazepam, a long-acting benzodiazepine sedative-hypnotic agent, zaleplon causes significantly less psychomotor and cognitive impairment (P < 0.001). Zaleplon has not been studied in pregnant women or children. The dose of zaleplon should be individualized; the recommended daily dose for most adults is 10 mg. Insomnia has a substantial impact on daily functioning. If pharmacologic treatment is indicated for insomnia, the choice of an agent should be guided by individual patient characteristics. [\hyperlink{Zaleplon}{PMID: 11117652}, K W Weitzel et al., 2000]

\hypertarget{pmid_23257756}{E}fficacy and safety of sertraline (zoloft) have been assessed in 26 patients, aged from 7 to 15 years, with depressive states of different severity and psychopathological structure which are combined with obsessive-compulsive symptoms. Changes in patient's status are analyzed using psychometric scales during 6 weeks of treatment. It has been concluded that Zoloft is effective and safe drug for the treatment of mild to moderate depressive disorders concomitant to obsessive-compulsive disorders in children age. [\hyperlink{Zaleplon}{PMID: 23257756}, A V Goriunov et al., 2012]

\hypertarget{pmid_9579287}{S}ertraline (Zoloft) is a selective serotonin reuptake inhibitor that is commonly used in adults in the treatment of mood and anxiety disorders. Whereas it also is used to treat these illnesses in children, it is not currently approved by the Food and Drug Administration for use in this population. Sertraline use has been increasing secondary to its efficacy and its more tolerable side effect profile than the tricyclic antidepressants. It is also much safer in overdose than the tricyclic antidepressants. Although there have been numerous reports of sertraline overdose in adults, reports in the pediatric population are much less common. We review the literature regarding sertraline overdose in children, describe a case of sertraline ingestion in a 22-month-old infant, and discuss the treatment of such an overdose. [\hyperlink{Zaleplon}{PMID: 9579287}, G Catalano et al., ]

\hypertarget{pmid_14736130}{Z}aleplon appears to be a prime candidate for assisting individuals in obtaining sleep in situations not conducive to rest (i.e., a short period during the day). However, should an early unexpected awakening and return to duty be required, the effect on performance is not known. Zaleplon (10 mg) would negatively affect human performance for some duration, compared with placebo, after a sudden awakening from a short period (1 h) of daytime sleep. There were 16 participants, 8 men and 8 women, who volunteered to participate in this study. The study was conducted using a counterbalanced, double-blind, repeated measures design. At 1 h prior to drug administration, and at each of 7 h postdrug, performance measures (cognition, memory, balance, and strength) and subjective symptom reports were recorded. Zaleplon had a statistically significant (p < 0.05) negative impact on balance through the first 2 h postdose when compared with placebo. In addition, symptoms related to "drowsiness" were statistically more prevalent under zaleplon than under placebo through the first 3 h postdrug. Of the eight measures of cognitive performance, six were significantly negatively impacted in the zaleplon condition through 2 h postdose when compared with placebo, with one remaining significantly degraded through 3 h postdose. Zaleplon also had a significantly negative impact on memory at 1 h and 4 h postdose. Zaleplon (10 mg), when used as a daytime sleep aid, causes drowsiness (and related symptoms) up to 3 h postdose, and may impact task performance, especially more complex tasks, for at least 2-3 h postdose. [\hyperlink{Zaleplon}{PMID: 14736130}, Jeffrey N Whitmore et al., 2004]

\hypertarget{pmid_15716214}{I}nsomnia is a common problem that increases with age and can last months to years. While substantial data establish the efficacy and safety of short-acting hypnotic therapy for the management of short-term insomnia using benzodiazepines receptor agonists (BzRAs), there are few studies on the continued efficacy and safety of these drugs when used for sustained periods. This paper reports the results of a 1-year open-label extension phases of two randomized, double-blind trials of zaleplon. In the open-label phase, older patients self-administered zaleplon nightly from 6 to 12 months and were then followed through a 7-day single-blind placebo-controlled run-out period. The safety profile in this population of older adults was similar to that observed in a short-term trial of an equivalent population. The data also suggested that long-term therapy produced and maintained statistically significant improvement in time to persistent sleep onset, duration of sleep and number of nocturnal awakenings (P<0.001 for each variable) for treatment durations of up to 12 months. Discontinuation was not associated with rebound insomnia. The open-label trial of long-term hypnotic therapy with zaleplon 5 and 10 mg suggests that they are safe and effective for the treatment of insomnia in older patients. Placebo-controlled, double-blind trials are needed in zaleplon and other BzRAs to confirm these results. [\hyperlink{Zaleplon}{PMID: 15716214}, Sonia Ancoli-Israel et al., 2005]

\hypertarget{pmid_11552629}{T}he aim of the study was to research the efficiency of sertraline (zoloft) in depressions, anxious states and obsessive-compulsive disorders. Diagnosis of the mental disorders was carried out according to ICD-10. 72 children (59 boys, 13 girls) aged 6-18 years were treated. There were 32 inpatients and 40 outpatients. Therapy with sertraline was performed during 8 weeks with a gradual increase (titration) and individual selection of the dose from 12.5 to 100 mg/day. During the therapy clinical observation was combined with the patients' examination using Hamilton Depression Scale and Hamilton Anxiety Scale (HAM-D and HAM-A), and a Clinical Global Impression Scale (CGI). It was established that sertraline was very effective and safe drug in children (it has no influence on cognitive functions, has neither myorelaxing or sedative effects). Activity of this drug is characterized by quick manifestation of thymoanaleptic and anxiolytic effects. It mild depressive states 50 mg/day is a significant dose; in more severe depressions and obsessive-compulsive disorders the dose in juveniles was to 100 mg, the duration of the therapy was more than 2 months. [\hyperlink{Zaleplon}{PMID: 11552629}, V M Voloshina et al., 2001]

\hypertarget{pmid_10392321}{F}ive lactating mothers were administered the therapeutic dose of zaleplon (10 mg) orally in an open-label, single-dose, pharmacokinetic study. Plasma and breast milk were sampled through 8 hours after dose administration for subsequent determinations of zaleplon and its major, though inactive, plasma metabolite 5-oxo-zaleplon. Zaleplon concentrations peaked in plasma and milk approximately 1 hour after dosing and then disappeared rapidly. The mean terminal half-life was slightly greater than 1 hour. Milk concentrations "mirrored" plasma concentrations closely with no discernible delay between peak times. The average milk-to-plasma (M/P) concentration ratio for zaleplon was approximately 0.50 over the time course. 5-oxo-zaleplon was undetectable in all but one milk sample. The maximum exposure of an infant to zaleplon during a feeding at peak milk concentrations was estimated to range from 1.28 micrograms to 1.66 micrograms, corresponding to 0.013\% to 0.017\% of the maternal dose or 0.320 microgram/kg to 0.415 microgram/kg for a 4 kg infant. The results indicate that zaleplon taken by a nursing mother is transferred through breast milk to her infant in very small quantities that are unlikely to be clinically important. [\hyperlink{Zaleplon}{PMID: 10392321}, M Darwish et al., 1999]

\hypertarget{pmid_2816015}{Z}aditen combined with local therapy and a rational hygienic and dietetic regimen has been administered to 158 children suffering from neurodermatitis (n = 142), eczema (n = 13), and strophulus (n = 3). The treatment has been effective in 148 (93.7\%) children. Clinical remission has been achieved in 35 (22.2\%) children, a considerable improvement in 87 (55.0\%), and an improvement in 26 (16.5\%). In 10 (6.3\%) children no improvement has been observed. The improvement has been developing within the first fortnight of the treatment, but the drug has been continued for another week or two to stabilize the therapeutic effect. A follow-up of 102 children, effectively treated with zaditen, for 1-12 mos has shown relapses in 72 (70.6\%) children, of these 7 relapses of medium severity and 1 severe relapse. Zaditen has been well tolerated by the children. [\hyperlink{Zaleplon}{PMID: 2816015}, M E Lipets et al., 1989]

\hypertarget{pmid_10548901}{Z}aleplon (Sonata) is an original hypnotic derived from the pyrazolopyrimidine with a full agonistic activity on central benzodiazepine receptors B21 type. Zaleplon is characterized by an extremely short half-life (about 1 hour). At the 10 mg dose, it is an affective sleep inducer with limited risks of disturbances in morning performance. It is particularly suitable for the treatment of initial insomnia when the prescription of an hypnotic is justified. [\hyperlink{Zaleplon}{PMID: 10548901}, M Ansseau et al., 1999]

\hypertarget{pmid_29184807}{Z}oledronic acid (ZA), a highly potent intravenous bisphosphonate (BP), has been increasingly used in children with primary and secondary osteoporosis due to its convenience of shorter infusion time and less frequent dosing compared to pamidronate. Many studies have also demonstrated beneficial effects of ZA in other conditions such as hypercalcemia of malignancy, fibrous dysplasia (FD), chemotherapy-related osteonecrosis (ON) and metastatic bone disease. This review summarizes pharmacologic properties, mechanism of action, dosing regimen, and therapeutic outcomes of ZA in a variety of metabolic bone disorders in children. Several potential novel uses of ZA are also discussed. Safety concerns and adverse effects are also highlighted. [\hyperlink{Zaleplon}{PMID: 29184807}, Sasigarn A Bowden et al., 2017]

\hypertarget{pmid_10510148}{T}o compare the duration of the residual hypnotic and sedative effects of zaleplon with those of zolpidem and placebo following nocturnal administration at various times before morning awakening. Zaleplon 10 mg, zolpidem 10 mg, or placebo was administered double-blind to 36 healthy subjects under standardized conditions in a six-period, incomplete-block, crossover study. Subjects were gently awakened and given medication at predetermined times 5, 4, 3, or 2 h before morning awakening, which occurred 8 h after bedtime. When the subjects awoke in the morning, a battery of subjective and objective assessments of residual effects of hypnotics was administered. No residual effects were demonstrated after zaleplon 10 mg, when administered as little as 2 h before waking, on either subjective or objective assessments, whereas zolpidem 10 mg showed significant residual effects on DSST and memory (immediate and delayed free recall) after administration up to 5 h before waking and choice reaction time, critical flicker fusion threshold and Sternberg memory scanning after administration up to 4 h before waking. Residual effects of zolpidem were apparent in all objective and subjective measurements when the drug was administered later in the night. The present results demonstrate that zaleplon at the dose of 10 mg is free of residual hypnotic or sedative effects when administered nocturnally as little as 2 h before waking in normal subjects. In contrast, residual effects of zolpidem are still apparent on objective assessments up to 5 h after nocturnal administration, longer than has been reported from studies involving daytime administration. [\hyperlink{Zaleplon}{PMID: 10510148}, P Danjou et al., 1999]

\hypertarget{pmid_11965211}{I}nsomnia is the most frequently reported sleep symptom, severely affecting up to 15\% of the US population. The need to effectively treat this disorder is underscored by the significant adverse consequences on the productivity, safety, overall health, and quality of life of the affected individual. Pharmacologic intervention has traditionally involved the use of benzodiazepine receptor agonists (BzRAs), for which efficacy and general safety have been established. The purpose of this paper is to examine the potentially unique role of zaleplon in the treatment of insomnia. The clinical experience of the authors was critically applied to peer-reviewed published papers or abstracts regarding zaleplon, which were identified via MEDLINE (1995-September 2000). Adverse effects, usually related to residual sedation, impose limits on the use of older BzRAs and have prompted the development of new sleep medications with advantageous adverse event profiles. Zaleplon demonstrates a very rapid onset and offset of effect that permits symptomatic rather than prophylactic administration, resulting in comparable efficacy and reduced risk of the adverse effects associated with longer half-life agents. The characteristics of zaleplon may translate into distinct and significant clinical advances in the treatment of insomnia. [\hyperlink{Zaleplon}{PMID: 11965211}, Gary S Richardson et al., 2002]

\hypertarget{pmid_15014616}{B}ACKGROUND: Insomnia is a prevalent medical disorder that has significant effects on occupational performance, health, and quality of life. Insomnia places an enormous burden on society through increased visits to physicians, loss of productivity in the workplace, and an increased rate of accidents. An estimated sum of \$100 million is spent each year on direct treatment of unresolved insomnia. Physicians need to initiate early effective treatment to prevent development of chronic insomnia and its associated morbidity. Institution of good sleep hygiene practices may be useful in some patients but may not be adequate for resolution of all sleep problems. Behavioral treatments, while effective and durable, are time consuming and not widely utilized in clinical practice. Pharmacotherapy includes benzodiazepine hypnotics, but concerns regarding adverse effects (e.g., residual sedation) prompted the search for safer options. DATA SOURCES: Published and presented studies containing clinical data on zaleplon, a new nonbenzodiazepine sleep medication, were identified via MEDLINE, Current Contents (ISI database), bibliographic reviews, and consultation with sleep specialists. RESULTS: Zaleplon effectively shortens sleep onset time and improves the quality of sleep in patients with insomnia. Whether administered at bedtime or later at night, zaleplon is devoid of residual sedative effects that impair next-day functioning. Follow-up studies evaluating the long-term efficacy and safety of zaleplon showed that decreased time to sleep onset was maintained during therapy lasting up to 52 weeks, without a withdrawal syndrome after discontinuation. CONCLUSION: Insomnia is recurrent and unpredictable in nature. Despite the long-term morbidity of this sleep disorder, research evidence and practice guidelines have not explored long-term use of hypnotics. Many patients could benefit from long-term drug therapy with a sleep medication that is devoid of residual effects and can be taken at bedtime or later as symptoms occur, rather than nightly in anticipation of a sleep problem. [\hyperlink{Zaleplon}{PMID: 15014616}, Gary K. Zammit et al., 2001]

\section*{Sertraline Hydrochloride}
\subsection*{Result}
\subsubsection*{Answer}

Ages 6–12: Yes  
Ages 13–18: Yes  
Ages under 6: Unknown  

\subsubsection*{{Explanation}}
\hypertarget{Sertraline Hydrochloride}
A review of the available abstracts reveals several targeted studies evaluating the safety of sertraline hydrochloride in children and adolescents for various psychiatric conditions. Below is a summary by age range and indication:

Ages 6–12 (Children) and 13–18 (Adolescents):

- Major Depressive Disorder (MDD) and Obsessive-Compulsive Disorder (OCD): 
  - Multiple studies, including open-label and extension studies, specifically evaluated the safety and tolerability of sertraline in children (6–12) and adolescents (13–18) with MDD and/or OCD. These studies found sertraline to be generally well tolerated, with adverse events similar to those seen in adults, and no significant safety concerns over periods up to 24 or 52 weeks. No discontinuations due to changes in vital signs, laboratory values, or ECGs were reported. [\hyperlink{pmid_9549959}{PMID: 9549959}, J Alderman et al., 1998], [\hyperlink{pmid_11589530}{PMID: 11589530}, E H Cook et al., 2001], [\hyperlink{pmid_16553532}{PMID: 16553532}, Moira Rynn et al., ], [\hyperlink{pmid_16553533}{PMID: 16553533}, Jeffrey Alderman et al., ]
  - One study in 72 children (6–18 years) with depression, anxiety, and OCD also reported sertraline as "very effective and safe" with no influence on cognitive functions and no significant adverse effects. [\hyperlink{pmid_11552629}{PMID: 11552629}, V M Voloshina et al., 2001]

- Anxiety Disorders (including Social Anxiety Disorder):
  - An open-label study in children with social anxiety disorder (mean age not specified, but within the pediatric range) found sertraline to be generally well tolerated over 8 weeks, with significant improvement in symptoms and no major safety concerns. [\hyperlink{pmid_11349701}{PMID: 11349701}, S N Compton et al., 2001]
  - Another study in children and adolescents (8–17 years) with various anxiety disorders found no negative effects on attention and only minor cognitive side effects after 6–12 weeks of sertraline treatment. [\hyperlink{pmid_16190792}{PMID: 16190792}, Thomas Günther et al., 2005]

- Posttraumatic Stress Disorder (PTSD):
  - A 24-week double-blind, placebo-controlled study in burned children aged 6–20 found sertraline to be safe, with no significant safety issues reported. [\hyperlink{pmid_22040192}{PMID: 22040192}, Frederick J Stoddard et al., 2011]
  - Another randomized, double-blind, placebo-controlled trial in children and adolescents (6–17 years) with PTSD found sertraline to be "generally safe," though not more effective than placebo. Discontinuation due to adverse events was 7.5\% for sertraline vs. 3.2\% for placebo. [\hyperlink{pmid_21186964}{PMID: 21186964}, Adelaide S Robb et al., 2010]

Ages below 6:
- There is no evidence from the abstracts of targeted safety studies of sertraline hydrochloride in children under 6 years of age. One case report describes sertraline ingestion in a 22-month-old, but this is not a safety or efficacy study. [\hyperlink{pmid_9579287}{PMID: 9579287}, G Catalano et al., ]

Summary:
- For children and adolescents aged 6–18, multiple targeted studies affirm that sertraline hydrochloride is generally safe for use in this population for MDD, OCD, and certain anxiety disorders, with safety profiles similar to those in adults and no major safety concerns reported in the studies.
- For children under 6 years, there is no targeted safety data available in the abstracts reviewed, so safety is unknown for this age group.

\subsection*{Abstracts}
\hypertarget{pmid_11552629}{T}he aim of the study was to research the efficiency of sertraline (zoloft) in depressions, anxious states and obsessive-compulsive disorders. Diagnosis of the mental disorders was carried out according to ICD-10. 72 children (59 boys, 13 girls) aged 6-18 years were treated. There were 32 inpatients and 40 outpatients. Therapy with sertraline was performed during 8 weeks with a gradual increase (titration) and individual selection of the dose from 12.5 to 100 mg/day. During the therapy clinical observation was combined with the patients' examination using Hamilton Depression Scale and Hamilton Anxiety Scale (HAM-D and HAM-A), and a Clinical Global Impression Scale (CGI). It was established that sertraline was very effective and safe drug in children (it has no influence on cognitive functions, has neither myorelaxing or sedative effects). Activity of this drug is characterized by quick manifestation of thymoanaleptic and anxiolytic effects. It mild depressive states 50 mg/day is a significant dose; in more severe depressions and obsessive-compulsive disorders the dose in juveniles was to 100 mg, the duration of the therapy was more than 2 months. [\hyperlink{Sertraline Hydrochloride}{PMID: 11552629}, V M Voloshina et al., 2001]

\hypertarget{pmid_9579287}{S}ertraline (Zoloft) is a selective serotonin reuptake inhibitor that is commonly used in adults in the treatment of mood and anxiety disorders. Whereas it also is used to treat these illnesses in children, it is not currently approved by the Food and Drug Administration for use in this population. Sertraline use has been increasing secondary to its efficacy and its more tolerable side effect profile than the tricyclic antidepressants. It is also much safer in overdose than the tricyclic antidepressants. Although there have been numerous reports of sertraline overdose in adults, reports in the pediatric population are much less common. We review the literature regarding sertraline overdose in children, describe a case of sertraline ingestion in a 22-month-old infant, and discuss the treatment of such an overdose. [\hyperlink{Sertraline Hydrochloride}{PMID: 9579287}, G Catalano et al., ]

\hypertarget{pmid_1949975}{S}ertraline hydrochloride is a new naphthylamino compound that specifically blocks neuronal reuptake of serotonin. It is currently available in the United Kingdom and under review in the US. Sertraline follows first-order kinetics, with a plasma elimination half-life of 24-26 hours. It is highly bound to plasma proteins and has a large volume of distribution. Multicenter studies conducted by the manufacturer have shown sertraline to be efficacious in the treatment of depression and obsessive-compulsive disorder. The daily dose will range from 50 to 200 mg/d for the treatment of depression. The adverse-effect profile differs greatly from the tricyclic antidepressants, but is similar to that of fluoxetine. The most prominent adverse effects are gastrointestinal (nausea, diarrhea/loose stools, dyspepsia). [\hyperlink{Sertraline Hydrochloride}{PMID: 1949975}, S K Guthrie et al., 1991]

\hypertarget{pmid_9819070}{T}he serotonin selective reuptake inhibitors are increasingly being used for the treatment of panic disorder. We examined the efficacy and safety of the serotonin selective reuptake inhibitor sertraline hydrochloride in patients with panic disorder. One hundred seventy-six nondepressed outpatients with panic disorder, with or without agoraphobia, from 10 sites followed identical protocols that used a flexible-dose design. After 2 weeks of single-blind placebo, patients were randomly assigned to 10 weeks of double-blind, flexible-dose treatment with either sertraline hydrochloride (50-200 mg/d) or placebo. Sertraline-treated patients exhibited significantly greater improvement (P=.01) at end point than did patients treated with placebo for the primary outcome variable, panic attack frequency. Significant differences between groups were also evident for clinician and patient assessments of improvement as measured by the Clinical Global Impression Improvement (P=.01) and Severity (P=.009) Scales, Panic Disorder Severity Scale ratings (P=.03), high end-state function assessment (P=.03), Patient Global Evaluation rating (P=.01), and quality of life scores (P=.003). Adverse events, generally characterized as either mild or moderate, were not significantly different in overall incidence between the sertraline and placebo groups. Results support the safety and efficacy of sertraline for the short-term treatment of patients with panic disorder. [\hyperlink{Sertraline Hydrochloride}{PMID: 9819070}, M H Pollack et al., 1998]

\hypertarget{pmid_11349701}{T}he aim of this open-label study was to assess the therapeutic benefits, response pattern, and safety of sertraline in children with social anxiety disorder. Fourteen outpatient subjects with a primary Axis I diagnosis of social anxiety disorder were treated in an 8-week open trial of sertraline. Diagnostic and primary outcome measures included the Anxiety Disorders Interview Schedule for Children, Clinical Global Impressions scale (CGI), Social Phobia and Anxiety Inventory for Children, and a standardized behavioral avoidance test. As measured by the CGI (Improvement subscale), 36\% (5/14) of subjects were classified as treatment responders and 29\% (4/14) as partial responders by the end of the 8-week trial. A significant clinical response appeared by week 6. Self-report and behavioral measures showed significant clinical improvement into normal range across all domains measured. The mean dose of sertraline was 123.21+/-37.29 mg per day. Sertraline was generally well tolerated. In open treatment, sertraline resulted in significant improvement in symptoms of childhood social anxiety disorder. Absolute response rates varied depending on rating scales used. Findings from this study are sufficiently strong to warrant a future multisite, randomized, double-blind, placebo-controlled trial of sertraline for treatment of childhood social anxiety disorder. [\hyperlink{Sertraline Hydrochloride}{PMID: 11349701}, S N Compton et al., 2001]

\hypertarget{pmid_22040192}{T}his study evaluated the potential benefits of a centrally acting selective serotonin reuptake inhibitor, sertraline, versus placebo for prevention of symptoms of posttraumatic stress disorder (PTSD) and depression in burned children. This is the first controlled investigation based on our review of the early use of a medication to prevent PTSD in children. Twenty-six children aged 6-20 were assessed in a 24-week double-blind placebo-controlled design. Each child received either flexibly dosed sertraline between 25-150 mg/day or placebo. At each reassessment, information was collected in compliance with the study medication, parental assessment of the child's symptomatology and functioning, and the child's self-report of symptomatology. The protocol was approved by the Human Studies Committees of Massachusetts General Hospital and Shriners Hospitals for Children. The final sample was 17 subjects who received sertraline versus 9 placebo control subjects matched for age, severity of injury, and type of hospitalization. There was no significant difference in change from baseline with child-reported symptoms; however, the sertraline group demonstrated a greater decrease in parent-reported symptoms over 8 weeks (-4.1 vs. -0.5, p=0.005), over 12 weeks (-4.4 vs. -1.2, p=.008), and over 24 weeks (-4.0 vs. -0.2, p=0.017). Sertraline was a safe drug, and it was somewhat more effective in preventing PTSD symptoms than placebo according to parent report but not child report. Based on this study, sertraline may prevent the emergence of PTSD symptoms in children. [\hyperlink{Sertraline Hydrochloride}{PMID: 22040192}, Frederick J Stoddard et al., 2011]

\hypertarget{pmid_24627951}{T}o determine the safety and efficacy of high-dose oral chloral hydrate for pediatric ophthalmic procedures. This study is a retrospective review of a quality audit of pediatric sedation for ophthalmic evaluation and imaging performed at King Khaled Eye Specialist Hospital between January 1 and December 31, 2011, in children aged 1 month to 6 years. Three hundred fifty-eight of 380 (94.2\%) sedation procedures were successful after a single dose of chloral hydrate, with 356 of 380 (93.7\%) children sedated within 45 minutes of the first dose. The total success rate of the sedation procedure increased to 97.9\% (372 of 380) when a second dose was administered. Children adequately sedated after a single dose of chloral hydrate were on average younger and weighed less than children who required additional doses. No major adverse events were documented. The use of chloral hydrate sedation for ophthalmic evaluation and imaging was safe and effective in this patient population with a high rate of procedure completion. [\hyperlink{Sertraline Hydrochloride}{PMID: 24627951}, Michelle E Wilson et al., ]

\hypertarget{pmid_2402648}{C}hloral hydrate has been used extensively to sedate children, but at Brooke Army Medical Center, other drug combinations were becoming increasingly popular due to a perception that chloral hydrate had a high rate of failure, especially with younger or neurologically impaired children. Therefore, 50 children were given the drug before a diagnostic study, and patient data and a sedation score were recorded on a worksheet. Of 50 children, 43 (86\%) were "successfully sedated" on the first attempt with no side effects. Children with neurologic disorders had a much greater (27\% vs 4\%) failure rate than "normal" children. The sedation rate did not significantly differ by age, sex, or initial drug dosage. The study suggest that chloral hydrate is a safe and effective oral sedative but that children with neurologic disorders may need alternative drugs for sedation. [\hyperlink{Sertraline Hydrochloride}{PMID: 2402648}, P D Rumm et al., 1990]

\hypertarget{pmid_15951862}{D}iagnostic and therapeutic procedures in children are made easier using sedation. However, there is no consensus about which drug should be used to achieve this. Furthermore, none of the drugs used for sedation are risk free. The aim of this work is to study sedation indications, effectiveness, and safety at our center. A prospective observational study conducted at the Pediatric Day Care Unit, King Fahad National Guard Hospital, Riyadh, Saudi Arabia. The study covered 17.5 weeks in 2 periods: May 9th 1999 to June 13th 1999 and October 31st 2001 to February 11th 2002. Children <12 years were included. Collected data included demographics, indication, drug dosing and outcome. Data were reported as mean +/- SD. We included 148 patients, age 38 +/- 30 months. Adequate sedation was achieved in 79\% after initial chloral hydrate (CH) dose of 56.9 +/- 9.3 mg/kg, in 95\% after adding 18.5 +/- 6.4 mg/kg CH and in 96\% after adding second drug. Compared to nonrespondents, first CH dose respondents were younger and lower in weight. The CH side effects were few and mild. Chloral hydrate is a safe and effective agent for sedation in children with an age and weight dependent response. [\hyperlink{Sertraline Hydrochloride}{PMID: 15951862}, Omar M Hijazi et al., 2005]

\hypertarget{pmid_28741653}{C}hloral hydrate is commonly used to sedate children for painless procedures. Children may recover more quickly after sedation with dexmedetomidine, which has a shorter half-life. We randomly allocated 196 children to chloral hydrate syrup 50 mg.kg [\hyperlink{Sertraline Hydrochloride}{PMID: 28741653}, V M Yuen et al., 2017] Sertraline hydrochloride is a selective serotonin reuptake inhibitor (SSRI) widely prescribed to patients suffering from psychiatric disorders. Pharmaceutical products such as sertraline have been identified in environmental waters. This study describes the evaluation of sertraline using a battery of freshwater species representing four trophic levels. The species most sensitive to sertraline were Daphnia magna 21 d reproduction test, Pseudokirchneriella subcapitata 72 h growth inhibition, and Oncorhynchus mykiss 96 h mortality, with the Microtox assay being the least sensitive assay. The D. magna 21 d reproduction test was approximately two orders of magnitude more sensitive than the other bioassays. These results show the advantages of having a tiered approach within a test battery. The presented results indicate that sertraline hydrochloride adversely affects aquatic organisms at levels several orders of magnitude higher than that reported in municipal effluent concentrations, however adverse effects may result from lower concentration exposures, further research into chronic toxicity is therefore advocated. [\hyperlink{Sertraline Hydrochloride}{PMID: 28741653}, Elaine Minagh et al., 2009]

\hypertarget{pmid_9549959}{T}o evaluate the pharmacokinetics, safety, and efficacy of sertraline in children (6 to 12 years old) and adolescents (13 to 17 years old). Children (n = 29) and adolescents (n = 32) with major depression, obsessive-compulsive disorder (OCD), or both received a single dose of 50 mg of sertraline followed, 1 week later, by 35 days of sertraline treatment as follows: (1) either a starting dose of 25 mg/day titrated to 200 mg/day in 25-mg increments or (2) a starting dose of 50 mg/day titrated to 200 mg/day in 50-mg increments. Sertraline and desmethylsertraline pharmacokinetics were determined approximately weekly, and efficacy measures were assessed before drug administration and at the end of treatment. Mean area under the plasma concentration-time curve (AUC), peak plasma concentration (Cmax), and elimination half-life (t1/2) for sertraline and desmethylsertraline were similar to previously reported adult values. No titration-dependent pharmacokinetic or safety differences were seen. While Cmax and AUC0-24 were greater for children versus adolescents, these differences disappeared after parameters were normalized for body weight. Sertraline was well tolerated in both children and adolescents, with adverse experiences similar to those previously reported by adult patients. Efficacy measurements indicated improvement (p < .001) in depression and OCD symptomatology. Sertraline can be safely administered to pediatric patients using the currently recommended adult titration schedule. [\hyperlink{Sertraline Hydrochloride}{PMID: 9549959}, J Alderman et al., 1998]

\hypertarget{pmid_11589530}{T}o evaluate the safety and effectiveness of sertraline in the long-term treatment of pediatric obsessive-compulsive disorder (OCD). Children (6-12 years; n= 72) and adolescents (13-18 years; n = 65) with DSM-III-R-defined OCD who had completed a 12-week, double-blind, placebo-controlled sertraline study were given open-label sertraline 50 to 200 mg/day in this 52-week extension study. Concomitant psychotherapy was allowed during the extension study Outcome was evaluated by the Children's Yale-Brown Obsessive Compulsive Scale (CY-BOCS), National Institute of Mental Health Global Obsessive-Compulsive Scale, and Clinical Global Impression Severity (CGI-S) and Improvement (CGI-I) scores. Significant improvement (p < .0001) was demonstrated on all four outcome parameters on an intent-to-treat analysis for the overall study population (n = 132), as well as the child and the adolescent samples. At endpoint, 72\% of children and 61\% of adolescents met response criteria (>25\% decrease in CY-BOCS and a CGI-I score of 1 or 2). Significant (p < .05) improvements were also demonstrated from the extension study baseline to endpoint on all outcome parameters in those patients who received sertraline during the 12-week, double-blind acute study. Long-term sertraline treatment was well tolerated, and there were no discontinuations due to changes in vital signs, laboratory values, or electrocardiograms. Sertraline (50-200 mg/day) was effective and generally well tolerated in the treatment of childhood and adolescent OCD for up to 52 weeks. Improvement was seen with continued treatment. [\hyperlink{Sertraline Hydrochloride}{PMID: 11589530}, E H Cook et al., 2001]

\hypertarget{pmid_28275979}{S}edation is often required for children undergoing diagnostic procedures. Chloral hydrate has been one of the sedative drugs most used in children over the last 3 decades, with supporting evidence for its efficacy and safety. Recently, chloral hydrate was banned in Italy and France, in consideration of evidence of its carcinogenicity and genotoxicity. Dexmedetomidine is a sedative with unique properties that has been increasingly used for procedural sedation in children. Several studies demonstrated its efficacy and safety for sedation in non-painful diagnostic procedures. Dexmedetomidine's impact on respiratory drive and airway patency and tone is much less when compared to the majority of other sedative agents. Administration via the intranasal route allows satisfactory procedural success rates. Studies that specifically compared intranasal dexmedetomidine and chloral hydrate for children undergoing non-painful procedures showed that dexmedetomidine was as effective as and safer than chloral hydrate. For these reasons, we suggest that intranasal dexmedetomidine could be a suitable alternative to chloral hydrate. [\hyperlink{Sertraline Hydrochloride}{PMID: 28275979}, Giorgio Cozzi et al., 2017]

\hypertarget{pmid_22246409}{C}hloral hydrate (CH) is safe and effective for sedation of suitable children. The purpose of this study was to assess whether adequate sedation is achieved with reduced CH doses. We retrospectively recorded outpatient CH sedations over 1 year. We defined standard doses of CH as 50 mg/kg (infants) and 75 mg/kg (children >1 year). A reduced dose was defined as at least 20\% lower than the standard dose. In total, 653 children received CH sedation (age, 1 month-3 years 10 months), 42\% were given a reduced initial dose. Augmentation dose was required in 10.9\% of all children, and in a higher proportion of children >1 year (15.7\%) compared to infants (5.7\%; P < 0.001). Sedation was successful in 96.7\%, and more frequently successful in infants (98.3\%) than children >1 year (95.3\%; P = 0.03). A reduced initial dose had no negative effect on outcome (P = 0.19) or time to sedation. No significant complications were seen. We advocate sedation with reduced CH doses (40 mg/kg for infants; 60 mg/kg for children >1 year of age) for outpatient imaging procedures when the child is judged to be quiet or sleepy on arrival. [\hyperlink{Sertraline Hydrochloride}{PMID: 22246409}, Jennifer Bracken et al., 2012]

\hypertarget{pmid_28827252}{C}eftriaxone is widely used in children in the treatment of sepsis. However, concerns have been raised about the safety of ceftriaxone, especially in young children. The aim of this review is to systematically evaluate the safety of ceftriaxone in children of all age groups. MEDLINE, PubMed, Cochrane Central Register of Controlled Trials, EMBASE, CINAHL, International Pharmaceutical Abstracts and adverse drug reaction (ADR) monitoring systems will be systematically searched for randomised controlled trials (RCTs), cohort studies, case-control studies, cross-sectional studies, case series and case reports evaluating the safety of ceftriaxone in children. The Cochrane risk of bias tool, Newcastle-Ottawa and quality assessment tools developed by the National Institutes of Health will be used for quality assessment. Meta-analysis of the incidence of ADRs from RCTs and prospective studies will be done. Subgroup analyses will be performed for age and dosage regimen. Formal ethical approval is not required as no primary data are collected. This systematic review will be disseminated through a peer-reviewed publication and at conference meetings. CRD42017055428. [\hyperlink{Sertraline Hydrochloride}{PMID: 28827252}, Linan Zeng et al., 2017]

\hypertarget{pmid_2026812}{C}hloral hydrate is commonly used to sedate children before CT. However, no prospective study has been published of the safety and efficacy of chloral hydrate at high dose levels for children undergoing CT. We define high dose levels of oral chloral hydrate to be 80-100 mg/kg, with a maximum total dose of 2 g. High dose chloral hydrate sedation was administered orally to 295 children for 326 CT examinations. Adverse reactions occurred in 7\% of the children, with vomiting being the most common (4.3\% of children). Hyperactivity and respiratory symptoms each occurred in less than 2\% of children. Prolonged sedation ( greater than 2 h) was not encountered in our series. Sedation was successful in producing motion free CT examinations, so that in 303 (93\%) of the cases, no repeat CT scans were needed. We conclude that high dose oral chloral hydrate provides safe and effective sedation for children undergoing CT. [\hyperlink{Sertraline Hydrochloride}{PMID: 2026812}, S B Greenberg et al., ]

\hypertarget{pmid_31369972}{C}hloral hydrate is a sedative that has been used for many years in clinical practice and, under proper conditions, gives a deep and long enough sleep to allow performance of objective hearing tests in young children. The reluctance to use this substance stems from side effects reported over time that can vary, depending on dose, procedure settings and immediate life supporting intervention when needed. Our study adds to those that have appeared in recent years, showing that chloral hydrate is an effective and safe substance when is used in proper conditions. The study included 322 children who needed sedation for objective hearing tests, from April 2014 to March 2018. Parents were instructed to bring the child tired and fasted for at least 2 h before sedation. The sedative was administered by trained staff in the hospital, and the child was monitored until awaking. In our study group, over half of the children were in the age 1-4 years group, and only 15\% were older than 4 years. The dose of chloral hydrate ranged between 50 and 83 mg/kg body weight, with an average of 75 mg. Successful sedation occurred in 94.1\% of children; 0.9\% of children awoke during testing and required supplemental sedation or rescheduling of the testing. The most common side effects were vomiting, agitation, prolonged sleep, and failure to fall asleep. Comparing the side effects of chloral hydrate in our study with those from other studies, ours were similar to those described in the literature. In our study chloral hydrate was effective and had only limited adverse effects. The use of chloral hydrate under hospital conditions with proper monitoring could be a practical and safe solution for outpatients or those with short-term hospitalisation. [\hyperlink{Sertraline Hydrochloride}{PMID: 31369972}, Violeta Necula et al., 2019]

\hypertarget{pmid_21531030}{C}hloral hydrate (CH) is an oral sedative widely used to sedate infants and young children during auditory brainstem response (ABR) testing. The aim of this study was to record effectiveness, complications and safety of CH as a sedative for ABR. From January of 2003 until December of 2007, 1903 children were tested for ABR, 568 of them being under the age of 6 months. CH (8\%) was used for sedation at a dose of 40 mg/kg with a repeat dose, if necessary, for an adequate sedation, in 20-30 min. We recorded the effectiveness of CH as a sedative for ABR examination, as well as all complications related to the use of CH such as vomiting, rash, hyperactivity, respiratory distress and apnea. The statistical method used was the absolute and percentage frequency distribution of the occurrences. Sedation with CH was necessary to perform testing in 1591 (83.6\%) of the examined children. However, in the population of the examined infants, only 341 (60\%) were sedated with CH, because the remaining 227 (40\%) fell asleep by themselves. Complications included hyperactivity in 152 children (8\%), minor respiratory distress in 10 children (0.4\%), vomiting in 217 children (11.4\%), apnea in 4 children (0.2\%) and rash in 10 children (0.4\%). The complications of hyperactivity, vomiting and rash resolved without any medical treatment. The apnea cases were managed effectively by supplying ventilation to the children via a mask in the presence of an anesthesiologist. The use of CH at a dose of 40 mg/kg up to 80 mg/kg is safe and effective when administered in a setting with adequate equipment and the presence of well trained personnel. [\hyperlink{Sertraline Hydrochloride}{PMID: 21531030}, Eirini Avlonitou et al., 2011]

\hypertarget{pmid_16190792}{T}his study investigated the cognitive side effects of a 6-week course of sertraline treatment on verbal memory and attention in children and adolescents. Children with various anxiety disorders (social phobia, generalized and separation anxiety disorder; n = 28), between 8 and 17 years of age, received a standardized, computerized neuropsychological assessment before treatment and another 6 weeks after treatment onset with sertraline (daily dose range between 25 and 100 mg). The patient group was compared to healthy controls (n = 28), who were matched for age and IQ and were also tested twice over a 6-week period. Sertraline did not have any negative effects on attentional performance (p > 0.05) but did increase response speed in a divided attention paradigm (p = 0.02). By contrast, performance of the interference part of a verbal memory task decreased (p = 0.05). The described results also remained stable over a 12-week period after treatment onset. Thus, the cognitive side effects of sertraline seemed to differ slightly between pediatric patients and those described in adult patient groups, should, therefore, be carefully assessed. [\hyperlink{Sertraline Hydrochloride}{PMID: 16190792}, Thomas Günther et al., 2005]

\hypertarget{pmid_16553532}{T}he aim of this study was to assess the long-term safety, tolerability, and efficacy of sertraline 50-200 mg once-daily in children (6-11 year olds) and adolescents (12-18 year olds) with a Diagnostic and Statistical Manual of Mental Disorders, 4th edition (DSM-IV) diagnosis of major depressive disorder (MDD). This study consisted of a 24-week open-label observational study of children and adolescents who had completed either of two 10-week double-blind, placebo-controlled trials. The Children's Depression Rating Scale-Revised (CDRS-R) was the primary measure of efficacy. Two hundred ninety nine (299) patients completed the acute studies and were eligible for the extension study. Of these, 226 enrolled, but 5 did not receive treatment. Of 221 patients (107 children and 114 adolescents), 62.4\% completed the study. The endpoint mean daily dose was 109.9 mg/day. The mean decrease in CDRS-R score from double-blind baseline was 34.8 points (p < 0.001), with patients showing continued improvement in CDRS-R scores regardless of which treatment they received in the double-blind studies. At endpoint, 86\% of patients met CDRS-R responder and 58\% CDRS-R remitter criteria. Sertraline appears to be well tolerated and safe over 24 weeks of treatment in children and adolescents with MDD. Children and adolescents treated with sertraline appear to have increased improvement over that seen in the first 10 weeks of treatment. These findings need confirmation in placebo-controlled studies. [\hyperlink{Sertraline Hydrochloride}{PMID: 16553532}, Moira Rynn et al., ]

\hypertarget{pmid_25637819}{S}ertraline is one of the serotonin-specific reuptake inhibitors that is effective in treating several disorders such as major depression, obsessive-compulsive disorder, panic disorder, and social phobia. It is marketed in the form of its hydrochloride salt, which exhibits better solubility in water than its free base form. However, the absorption of sertraline through biological membranes could be improved by enhancing the solubility of its base because it is more hydrophobic than sertraline hydrochloride. To clarify the mechanism for the interaction of sertraline base with β-CD, it is important to study the basic interaction between the β-CD ring and sertraline base. Therefore, in this study, the currently used hydrochloride salt form was converted into the free base and β-CD was used as a model for β-CD derivatives to evaluate the interaction between β-CD and the sertraline base. The solid-state physicochemical characteristics of the sertraline-β-CD complex were investigated by the phase solubility method, differential scanning calorimetry, Fourier transform IR spectroscopy, FT-Raman spectroscopy, powder X-ray diffraction, and (13)C cross-polarization magic-angle spinning NMR measurements. The results showed that sertraline base and β-CD form an inclusion complex, and the stoichiometric ratio of the solid-state sertraline base-β-CD complex is 1:1, which was estimated by the (1)H NMR measurements of the complex dissolved in DMSO-d6.  [\hyperlink{Sertraline Hydrochloride}{PMID: 25637819}, Noriko Ogawa et al., 2015] The aim of this study was to evaluate the long-term pharmacokinetics, safety, and efficacy of sertraline in children and adolescents with obsessive-compulsive disorder (OCD) or major depressive disorder (MDD). After 42-day initial treatment and 9-day withdrawal phases, children (6-12 years, n = 16) and adolescents (13-18 years, n = 27) entered a 24-week open-label phase, with sertraline titrated to 200 mg/day. Blood samples for plasma sertraline and N-desmethylsertraline levels were taken at the beginning of the 24-week phase and at weeks 1, 4, 8, 12, and 24. Efficacy and safety data were also collected. Mean maximum daily dose at endpoint was 157 +/- 49 mg. For female and male children, mean sertraline/N-desmethylsertraline concentrations normalized to a 200-mg dose were 85.0/160 ng/mL (n = 8) and 79.3/134 ng/mL (n = 8), respectively, and for female and male adolescents, 70.5/109 ng/mL (n = 16) and 76.3/120 ng/mL (n = 8). No significant age or gender effects or age-by-gender interactions were observed in sertraline values. Mean sertraline plasma concentrations normalized for dose and body weight did not differ significantly by age or gender. Three (3) patients (7\%) discontinued owing to adverse events. In patients with OCD (n = 10), improvements were observed in Children's Yale-Brown Obsessive Compulsive Scale (CY-BOCS) (p = 0.029) and National Institute of Mental Health (NIMH) Global Obsessive Compulsive Scale (OCS) (p = 0.01). In MDD patients (n = 32), Clinical Global Impression (CGI) Severity (p = 0.002) and Improvement (p = 0.011) improved. Long-term treatment of MDD and OCD with sertraline up to 200 mg/day in children and adolescents results in dose-normalized plasma concentrations comparable to those seen in adults. Sertraline was generally well tolerated, and patients demonstrated clinical improvement over 24 weeks of treatment. [\hyperlink{Sertraline Hydrochloride}{PMID: 25637819}, Jeffrey Alderman et al., ]

\hypertarget{pmid_28242616}{A}lthough chloral hydrate (CH) has been used as a sedative for decades, it is not widely accepted as a valid choice for ophthalmic examinations in uncooperative children. This study aimed to systematically review the literature on the drug's safety and efficacy. We searched PubMed, EMBASE, ISI Web of Science, Scopus, CENTRAL, Google Scholar and Trip database to 1 October 2015, using the keywords 'chloral hydrate', 'paediatric' and 'procedural sedation OR diagnostic sedation'. A meta-analysis of randomised controlled trials (RCTs) was performed. A total of 6961 articles were screened and 104 were included in the review. Thirteen of these concerned paediatric ophthalmic examination, while 13 others were RCTs and were meta-analysed. CH was reported to have been administered in a total of 24 265 sedation episodes in children aged from <1 month to 18 years. The meta-analysis showed CH had a higher OR (2.95, 95\% CI 1.09 to 7.99) for successful sedation compared to other sedatives, but significant limitations apply. The commonest reported adverse events (AE) were not serious (eg, paradoxical reaction or transient vomiting) and required no intervention. Severe AE, including two deaths, were related to comorbidity, overdose or aspiration. Despite the paucity of high quality evidence, the existing literature suggests that the use of CH for procedural sedation in children appears to be an effective alternative to general anaesthesia, and it can be safe when administered in the hospital setting with appropriate monitoring and vigilance for intervention. [\hyperlink{Sertraline Hydrochloride}{PMID: 28242616}, Asimina Mataftsi et al., 2017]

\hypertarget{pmid_21186964}{T}he aim of this study was to evaluate the safety and efficacy of sertraline in children and adolescents who met Diagnostic and Statistical Manual of Mental Disorders, 4th edition (DSM-IV) criteria for posttraumatic stress disorder (PTSD). Children and adolescents (6-17 years old) meeting DSM-IV criteria for PTSD were randomized to 10 weeks of double-blind treatment with sertraline (50-200  mg/day) or placebo. The primary efficacy measure was the University of California, Los Angeles Post-Traumatic Stress Disorder Index for DSM-IV (UCLA PTSD-I). A total of 131 patients met entry criteria and were randomized to sertraline (n = 67; female, 59.7\%; mean age, 10.8; mean UCLA PTSD-I score, 43.8 ± 8.5) or placebo (n = 62; female, 61.3\%; mean age, 11.2; mean UCLA PTSD-I score, 42.1 ± 8.8). There was no difference between sertraline and placebo in least squares (LS) mean change in the UCLA PTSD-I score, either on a completer analysis (-20.4 ± 2.1 vs. -22.8 ± 2.1; p = 0.373) or on an last observation carried forward (LOCF) end point analysis (-17.7 ± 1.9 vs. -20.8 ± 2.1; p = 0.201). Attrition was higher on sertraline (29.9\%) compared to placebo (17.7\%). Discontinuation due to adverse events occurred in a 7.5\% treated with sertraline and 3.2\% treated with placebo. Sertraline was a generally safe treatment in children and adolescents with PTSD, but did not demonstrate efficacy when compared to placebo during 10 weeks of treatment. ClinicalTrials.gov Identifier: NCT00150306. [\hyperlink{Sertraline Hydrochloride}{PMID: 21186964}, Adelaide S Robb et al., 2010]

\section*{Elimite}
\subsection*{Result}
\subsubsection*{Answer}

2 months to 5 years: Yes
Younger than 2 months: Unknown
Older than 5 years: Unknown

\subsubsection*{{Explanation}}
\hypertarget{Elimite}
A review of the available abstracts reveals one targeted study specifically evaluating the safety of Elimite (permethrin 5\% cream) in children:

- In a double-blinded, randomized study, permethrin 5\% cream (Elimite) was compared with crotamiton 10\% cream (Eurax) for the treatment of scabies in children aged 2 months to 5 years. The study found that Elimite was significantly more effective than Eurax and stated: "Elimite offers a safe, efficacious, and cosmetically elegant alternative to Eurax in the treatment of scabies in children" [\hyperlink{pmid_2188239}{PMID: 2188239}, D Taplin et al., 1990]. No significant safety concerns were reported in this age group.

No other abstracts specifically address the safety of Elimite in children outside the 2 months to 5 years age range. Therefore, based on the abstracts available, Elimite has been studied and affirmed as safe for use in children aged 2 months to 5 years. The safety for children outside this age range (e.g., older than 5 years or younger than 2 months) is not addressed in the abstracts, so the safety in those groups is unknown.

\subsection*{Abstracts}
\hypertarget{pmid_2188239}{P}ermethrin 5\% cream (Elimite) was approved as a treatment for scabies by the U.S. Food and Drug Administration in September 1989. In a double-blinded, randomized study, it was compared with crotamiton 10\% cream (Eurax) for the treatment of scabies in children 2 months to 5 years of age. Two weeks after a single overnight treatment, 14 (30\%) of 47 children were cured with permethrin 5\% cream, in contrast to only 6 of 47 (13\%) of subjects treated with Eurax. Four weeks after treatment the figures were 89\% and 60\% cured for the two agents, respectively. In 10 of the 19 patients whose treatment failed, the condition became worse after therapy. The difference in efficacy in favor of permethrin was significant (P = 0.002). That agent also demonstrated greater effectiveness in reducing pruritus and secondary bacterial infections. Elimite offers a safe, efficacious, and cosmetically elegant alternative to Eurax in the treatment of scabies in children. [\hyperlink{Elimite}{PMID: 2188239}, D Taplin et al., 1990]

\hypertarget{pmid_16087357}{T}his study has been conducted to assess the efficacy and safety of topiramate in refractory epilepsies in infants and young children. A prospective clinical trial was performed in three tertiary care hospitals, on 47 children aged 6-60 months with refractory epilepsy. Topiramate was added to at least two baseline anti-epileptic drugs. The efficacy was rated according to seizure type, frequency and duration. Children with refractory epilepsy were classified according to their clinical, neuro-imaging, and neurophysiological profile into infantile spasms (IS) (9 cases, 19\%), Lennox-Gastaut syndrome (LGS) (25 cases, 53\%) and other epilepsies (13 cases, 28\%). Children were also classified into cryptogenic and symptomatic epilepsy. Topiramate was introduced as add-on therapy in a daily dose of 1 mg/kg/day for 2 weeks, followed by increments of 1-3 mg/kg/day at 2-week intervals, up to a maximum of 10 mg/kg/day. After a minimum treatment period of 6 months, 28 (60\%) of the children had a satisfactory response (completely seizure free, or more than a 50\% seizure reduction). The remaining 19 children (40\%) had an unsatisfactory response (50\% or less reduction in seizure frequency, no change or increased seizure frequency). Topiramate appeared to be equally effective in infantile spasms, Lennox-Gastaut syndrome and children with other types of epilepsy, with no significant difference between those with a satisfactory and an unsatisfactory response (p=0.089). There was also no significant difference in response between patients with cryptogenic and symptomatic epilepsy (p=0.360). Mild to moderate adverse effects, mainly somnolence, anorexia and nervousness, were present in 25 (53\%) of children. One of the children developed hypothyroidism. Although the long term safety and possible adverse effects of topiramate have not been fully established in infants and young children, this study has shown that it is a useful option for children with frequent seizures unresponsive to standard anti-epileptic drugs. [\hyperlink{Elimite}{PMID: 16087357}, S Al Ajlouni et al., 2005]

\hypertarget{pmid_16551447}{T}opiramate is a new antiepileptic drug with a broad spectrum of efficacy. Reports on the use of topiramate for treatment of infantile spasms are limited. We prospectively followed 15 children with recently diagnosed infantile spasms treated with topiramate for efficacy and tolerability. Twelve patients had symptomatic infantile spasms, and two patients had cryptogenic infantile spasms. Topiramate was started at a dose of 3 mg/kg/day and titrated up to a dose of 27 mg/kg/day in 2 to 3 weeks. The primary efficacy measure was comparison of the seizure rate during the 2-week baseline with the median seizure rate during the first 2 months of treatment with topiramate. We also compared baseline electroencephalograms (EEGs) with post-treatment EEGs. The median seizure rate reduction during the first 2 months of treatment was 41\% (P = .002). Three patients became spasm free (20\%), five had > 50\% reduction, and three had at least 25\% reduction. Four patients did not respond. Three of 15 patients had clearing of hypsarrhythmia. Topiramate was generally well tolerated, with irritability being the most common side effect. Topiramate was efficacious and well tolerated; one patient discontinued the medication because of adverse effects. (J Child Neurol 2006;21:17-19). [\hyperlink{Elimite}{PMID: 16551447}, Syed A Hosain et al., 2006]

\hypertarget{pmid_21206445}{T}o determine Topiramate efficacy on treatment of infantile spasms and ancillary seizures, and whether there were any improvements on EEG. A retrospective study of 18 patients with infantile spasms recruited from the Pediatric Unit at King Fahd Hospital of the University, Dammam University, Saudi Arabia was carried out between January 2004 and December 2008. Topiramate was used as treatment in 7 males and 11 females aged 2-14 months. The maximum dose was 12 mg/kg/day. The etiology in 9 (50\%) patients was cryptogenic, 6 (33\%) symptomatic, and 3 (17\%) idiopathic. After Topiramate treatment 6 (33\%) were spasm free, 8 (44\%) had ≥50\% reduction, 2 (11\%) had no change, and one (6\%) had worsening of their spasms. Eight patients had ancillary seizures, 2 (25\%) were seizure free, 2 (25\%) had ≥50\% seizure reduction, and 4 (50\%) had no change in the ancillary seizure. The EEG showed hypsarrhythmia in 14 (78\%). Post Topiramate, the EEG was normal in one (5\%), improved in 3 (17\%), showed persistent hypsarrhythmia in 8 (44\%), and evolved to other features in 3 (17\%). Three patients developed side effects such as weight loss and irritability, for which 2 patients stopped the medication. Topiramate has a good effect on the clinical features of West syndrome, but not on the EEG. It was tolerated with minimal side effects. [\hyperlink{Elimite}{PMID: 21206445}, Raidah S Al-Baradie et al., 2011]

\hypertarget{pmid_9831007}{T}opiramate is a sulfamate-substituted monosaccharide that has demonstrated efficacy as an antiepileptic drug in adults with partial onset seizures. Experience in children has been limited, but early reports have supported its safety and effectiveness in children as young as 2 years of age. In two infants ages 12 and 9 months, respectively, with partial seizures, the authors report excellent efficacy with good tolerability at doses up to 7.7 mg/kg. Although long-term safety and possible adverse sequelae have not been fully established in children, topiramate may represent an option for infants with high seizure frequency unresponsive to standard antiepileptic drugs. [\hyperlink{Elimite}{PMID: 9831007}, S L Kugler et al., 1998]

\hypertarget{pmid_23131185}{T}he safety of a novel 0.5\% ivermectin lotion (IVL) and potential for ivermectin absorption after application was investigated in an open-label study in young children, and a human repeat insult patch test (HRIPT) and cumulative irritation test (CIT) assessed any potential for cumulative dermal irritation and contact sensitization. In the pharmacokinetic and safety study, 30 head louse-infested children ages 6 months to 3 years received a 10-minute application of IVL on day 1. Blood was collected before application; 0.5, 1, and 6 hours after rinsing; and on days 2 and 8. Samples from 20 subjects were assayed for ivermectin (test sensitivity 0.05 ng/mL). Liver panel and complete blood counts were completed for all subjects. For the HRIPT/CIT, occlusive patches containing IVL or vehicle control lotion (CL) were repeatedly applied to 220 healthy adult subjects to assess contact sensitization; for cumulative dermal irritation testing, additional patches with normal saline and sodium dodecyl sulfate (SDS) were applied to 36 subjects. In the open-label study, all detected ivermectin plasma concentrations were <1 ng/mL. No safety signals emerged, and treatment was well tolerated. In the HRIPT/CIT, IVL was significantly less irritating than normal saline and SDS, with no evidence of dermal irritation or sensitization in human skin. IVL was safe when applied topically, absorption was de minimus, there was no evidence of irritation or sensitization from repeated exposures, and results support the safety of topical IVL use in children as young as 6 months. [\hyperlink{Elimite}{PMID: 23131185}, Lydie Hazan et al., ]

\hypertarget{pmid_11218055}{T}opiramate has been shown to be safe and effective in refractory partial epilepsy in children. Pharmacokinetic studies show that the clearance of topiramate is greater in children than in adults; therefore, higher doses may be needed in children than adults. It is generally well tolerated, except for cognitive dysfunction. Weight loss and the risk of renal stones can be significant in some cases. However, when compared with other anticonvulsant medications, topiramate has few serious idiosyncratic reactions such as rash, hematologic reactions, and hepatotoxicity. [\hyperlink{Elimite}{PMID: 11218055}, K D Holland et al., 2000]

\hypertarget{pmid_17095898}{T}his was a prospective open study to establish the efficacy, tolerability, and problems associated with the use of topiramate as first-choice drug in children with infantile spasms. Open-label follow-up study, ranging from 24 to 36 months, of the cases of 54 patients with infantile spasms treated initially with topiramate as first-choice drug. Thirty-one patients (57.4\%) were seizure free for more than 24 months; 9 patients were treated with topiramate alone and 22 patients with topiramate plus nitrazepam and/or valproate. In 44 cases (81.4\%), the reduction of seizure frequency from baseline was greater than 30\%, whereas in 10 cases (18.6\%), there was poor or no response. The average dosage applied was 5.2 mg/kg per day (maximum dosage, 26 mg/kg per day; minimum dosage, 1.56 mg/kg per day). Adverse events occurred in 14 patients (26\%). They included poor appetite leading to anorexia, absence of sweating, and sleeplessness. Topiramate proves to be an effective and safe first-choice drug not only as adjunctive but also as monotherapy of infantile spasms in children younger than 2 years. [\hyperlink{Elimite}{PMID: 17095898}, Li-Ping Zou et al., ]

\hypertarget{pmid_18637038}{T}o review our experience of the efficacy and tolerability of felbamate in children younger than 4 years. We used a retrospective chart review to identify 53 children with seizures who were younger than 4 years. Efficacy was evaluated based on the occurrence of responsiveness, defined as seizure frequency reduction of more than 50\% for a minimum period of 4 months. Tolerability was based on parent-reported side effects. Twenty-two (41\%) patients resulted to be responders and 31 (59\%) did not. By univariate analysis, those achieving seizure remission were probably much older, to have a shorter history of epilepsy and a lower frequency of seizures before felbamate therapy. The number of antiepileptic drugs (AEDs) used before felbamate therapy was the only significant predictor of the duration of response to felbamate, with a longer responsiveness to the drug seen in those who were placed under fewer than three AEDs before felbamate compared with those who had taken more than three (median, 16 months vs. 7 months; P < 0.0084). Side effects occurred in 30\% of the subjects, but these did not require discontinuation of the drug. Felbamate is an effective medication for a wide range of epilepsy syndromes in children younger than 4 years. Although caution is necessary when the drug is used in children, felbamate might represent a possible option for the treatment of epilepsy in this age group. [\hyperlink{Elimite}{PMID: 18637038}, S Grosso et al., 2008]

\hypertarget{pmid_37128832}{E}ltrombopag (ELT) is effective and safe in adult persistent/chronic immune thrombocytopenia (p/cITP); a proportion could achieve a sustained response off treatment (SRoT); however, data on children are lacking. We attempted to analyse SRoT of ELT in children with p/cITP in this study. A multicentre retrospective observational study was performed in November 2022 for children with p/cITP who used ELT alone for >2 months between January 2017 and November 2021. Clinical data of pre-, during and post-ELT were collected. SRoT was defined as maintaining a platelet count of ≥30 × 10 [\hyperlink{Elimite}{PMID: 37128832}, Zhifa Wang et al., 2023] The efficacy and safety of monomeric allergoid (Lofarma, Milan) have been demonstrated in adults but very few studies have examined it in children. This study therefore investigated the efficacy and safety of this sublingual immunotherapy (SLIT) at the dosage of 1000 AU five times a week without any up-dosing. Forty allergic children (17 M and 23 F, mean age 7 years, range 4-16 years), 16 with rhinitis and 24 with rhinitis and asthma, were randomized to SLIT or drug therapy. All the patients were sensitized to grass; some were also sensitized, though to a lesser extent, to Parietaria, Olea and Betulaceae. The patients were treated pre-/co-seasonally for two years. A visual analogue scale (VAS) was used at baseline and at the end of the first and second pollen seasons to rate the patients' well-being. The VAS score was significantly higher after both the first and the second year of treatment in the SLIT group than in the controls (p<0.05). It improved in comparison to baseline only in the active group. All 40 children tolerated the therapy very well. The monomeric allergoid at the dosage of 5000 AU/week thus appears to have a good efficacy and safety profile in children. [\hyperlink{Elimite}{PMID: 37128832}, F Agostinis et al., 2009]

\hypertarget{pmid_10701102}{T}opiramate is a new anti-epileptic drug with proven efficacy against partial seizures in adults. A retrospective assessment of the use of topiramate in drug-resistant childhood epilepsy was undertaken. Thirty-four children (median age of 10 years; range 2-18 years) were treated for a median of 9 months (range 6-18 months). The starting dose was 0.25-2.0 mg/kg/day increasing to a maximum of 13 mg/kg/day. Generalized seizures occurred in 27 patients, partial seizures in 15 and infantile spasms in two. Epilepsies were localization-related in 15 patients and generalized in 18. One patient had severe myoclonic epilepsy in infancy. Two patients had Lennox-Gastaut syndrome, five (two currently and three previously) had West syndrome and one had epilepsy with myoclonic absences. Twenty patients had a substantial (> 50\%) reduction in seizure frequency; two of whom became seizure-free. Two-patients had an increase in seizures. Efficacy was seen against simple and complex partial seizures, generalized tonic-clonic seizures (primarily generalized), atonic and tonic seizures, myoclonic seizures and infantile spasms. There was no response in the one patient with myoclonic absence seizures. Adverse effects were reported in nine patients; appetite suppression occurred in five patients, behaviour disturbances in three, somnolence in two and poor concentration in one patient. Topiramate is efficacious in a wide spectrum of childhood epilepsies and is well tolerated. [\hyperlink{Elimite}{PMID: 10701102}, S Yeung et al., 2000]

\hypertarget{pmid_30978265}{C}hemotherapy-induced nausea and vomiting (CINV) is a distressing treatment side-effect that could negatively affect children's quality of life (QoL). Different scoring systems for CINV were applied and different antiemetic drugs were used; however, few studies have been performed in children undergoing chemotherapy with Aprepitant. Herein, we report a pediatric experience on efficacy and safety of Aprepitant as part of triple antiemetic prophylaxis, in a cohort of thirty-two children and adolescents with Hodgkin Lymphoma (HL), treated with moderate/highly emetogenic chemotherapy (MEC/HEC) regimens in a single Hemato-Oncology Institution. The triple therapy was compared to standard antiemetic therapy in a cohort of twenty-three HL patients (control group). Aprepitant therapy was associated to a significant decrease of chemotherapy-induced vomiting (p = 0.0001), while no impact on the reduction of nausea was observed; these observations were also confirmed by multivariate analysis (p = 0.0040). Aprepitant was well tolerated and the most commonly reported adverse events were neutropenia and hypertransaminasemia. No significant differences on the toxicity were observed between the two compared groups. Our experience on Aprepitant efficacy and safety, associated with feasibility of orally administration, suggests a possible widespread use of the drug to prevent pediatric CINV. [\hyperlink{Elimite}{PMID: 30978265}, Giovanna Giagnuolo et al., 2019]

\hypertarget{pmid_33842124}{T}opiramate (TOPAMAX®) is an anti-epileptic drug for which acute toxicity is infrequently reported. We present the case report of a five-year-old, otherwise healthy boy who presented to the emergency department (ED) for symptoms of acute encephalopathy. He was lethargic, having slurred speech, hallucinating, intermittently agitated, and had multiple episodes of urinating on himself. Computed tomography (CT) of the head, lumbar puncture, electroencephalography, and magnetic resonance imaging (MRI) were all normal. The urine drug screen was also negative. Two days after admission, a saliva toxicology screen was significant for a topiramate level of 3487.8 ng/ml, which he was not taking and which his mother admitted taking for weight loss. The patient was observed for two days, over which time his symptoms completely resolved, and he was back to baseline. The following is the take-away for physicians: Careful history-taking should bedone to identify potential drug exposures in children presenting with acute encephalopathy. Especially, given the emerging off-label use of drugs, like in this case, topiramate, which was used by the mother for weight loss. We postulated a possible idiosyncratic reaction vs true drug toxicity, which correlates with findings in a previous case reportout of Boston Children's Hospital by Taub et al.; and in this case, serum level was about one-third the reported level in this case report [\hyperlink{Elimite}{PMID: 33842124}, Mohammad Baidoun et al., 2021] Based on the initial successful use of felbamate for infantile spasms in an infant with tuberous sclerosis, three additional infants with infantile spasms of different etiologies who had failed conventional therapies were treated with felbamate. Three of the four patients have shown complete resolution of infantile spasms. All responding patients did so within 1 week of starting felbamate. The one treatment failure had an initial reduction of seizure frequency and severity but has not maintained that response long term. Controlled studies are needed to firmly establish that felbamate is both safe and effective for the treatment of infantile spasms. As these cases document, felbamate is currently available for use in infantile spasms, and the frequent conversion of infantile spasms to Lennox-Gastaut syndrome, for which felbamate is approved, makes its use in infantile spasms logical. [\hyperlink{Elimite}{PMID: 33842124}, D L Hurst et al., 1995]

\hypertarget{pmid_21909500}{F}luoride is the most important factor in the decline of caries in children and adolescents. The aim of this observational study, begun in 2000, was to assess the effect of semiannual topical fluoride application in schoolchildren. Due to limited resources, only 334 of all first and second grade schoolchildren (6 to 8 years of age, 0.32 ± 1.02 decayed/missing/filled surface [DMFS], schools randomly selected) in Greifswald received a semiannual application of elmex fluid, while the remaining 442 children served as the control group (0.36 ± 1.15 DMFS). In 2002 and 2004, 230 and 349 of these children were re-examined according to WHO criteria by one calibrated examiner (DMFT/S). The parents filled out questionnaires on additional fluoride use, which was summarized as fluoride scores. In the dropout analysis, a selection bias among the dropout, fluoride, and control group regarding age, baseline caries prevalence, additional fluoride use, and sealants was excluded. During the entire study, no adverse effects were recorded with the use of elmex fluid. The caries increment was almost identical in the intervention and control groups (0.81 ± 1.74 and 0.78 ± 1.81 DMFS) with 72\% and 69\% of the children, respectively, showing no caries increment. The effect of only two applications of elmex fluid might have been overridden by the high background fluoride use. The participants had high mean values of the fluoride scores, reflecting the regular use of fluoride toothpaste and additional fluoride sources, without a polarization within the sample (intervention, 1.40 ± 0.60; control, 1.33 ± 0.60). Further studies should examine the effect of semiannual topical fluoride applications after caries decline. [\hyperlink{Elimite}{PMID: 21909500}, Christian H Splieth et al., 2011]

\hypertarget{pmid_36203978}{C}oncerns regarding felbamate adverse effects restrict its widespread use in children with drug-resistant epilepsy. We aimed to examine the efficacy and safety of felbamate in those children and identify the ones who may benefit most from its use. We retrospectively reviewed the medical files of all patients who were treated with felbamate in a tertiary pediatric epilepsy clinic between 2009-2021. Drug efficacy was determined at the first 3 months of treatment and thereafter. Therapeutic response and adverse reactions were monitored throughout the course of treatment. Our study included 75 children (age 8.9 ± 3.7 years), of whom 53 were treated with felbamate for seizures, 16 for electrical status epilepticus during sleep and 6 for both. The median follow-up time was 16 months (range 1-129 months). The most common cause for epilepsy was genetic (29\%). The median number of previous anti-seizure medications was six [4-8]. A therapeutic response ≥50\% was documented in 37 (51\%) patients, and a complete response in 9 (12\%). Nineteen patients (25\%) sustained adverse reactions, including three cases of elevated liver enzymes and one case of neutropenia with normal bone marrow aspiration. In all cases, treatment could be continued. All children with intractable epilepsy following herpes encephalitis showed a response to felbamate. Felbamate is an efficacious and safe anti-seizure medication in the pediatric population. [\hyperlink{Elimite}{PMID: 36203978}, Shira Rabinowicz et al., 2022]

\hypertarget{pmid_15797353}{S}tudies of the efficacy of topiramate (TPM) in infants and young children are few. Here we report an open, prospective, and pragmatic study of effectiveness of TPM in terms of epilepsy syndromes, in children aged less than 2 years. The median follow-up period was 11 months. We enrolled 59 children in the study: 22 affected by localization-related epilepsy (LRE), 23 by generalized epilepsy, six by Dravet's syndrome, and eight with unclassifiable epilepsy. TPM was effective (responders showed a decrease of more than 50\% in seizure frequency) in 47\% of patients, including 13\% who were seizure-free at the last visit. TPM was more effective in localization-related epilepsy (48\% of responders) than in generalized epilepsy (32\% of responders). In the latter group, 19 patients suffered from infantile spasms. Four of six patients with cryptogenic infantile spasms became seizure-free. Of the 13 patients with symptomatic infantile spasms, only one was seizure-free. Results were poor for patients with Dravet's syndrome. In general, TPM was well tolerated. The most frequently reported adverse effects were drowsiness, irritability, hyperthermia, and anorexia. The present study concludes that TPM is effective for a broad range of seizures in infants and young children and represents a valid therapeutic option in this population. [\hyperlink{Elimite}{PMID: 15797353}, S Grosso et al., 2005]

\hypertarget{pmid_23648886}{T}opiramate (TPM) is a new antiepileptic drug, which has a wide spectrum of activities suggesting a potentially valuable therapeutic profile. Our objective is to report our experience in treating children with intractable epilepsy. Prospective, open label, add on trial of TPM in treating consecutive children with intractable epilepsy (defined as recurrent seizures after at least 3 antiepileptic medication trials) seen between May 1, 1999 and April 28, 2002 at King Faisal Specialist Hospital and Research Centre and King Abdulaziz University Hospital in Jeddah, Kingdom of Saudi Arabia. Follow up by 2 pediatric neurologists was performed. Therapeutic response was recorded as complete (no seizures), good (>50\% seizure reduction), fair (<50\% seizure reduction), or none. Sixty-two children (36 males and 26 females) aged between 2 months and 16 years (mean 6 years) were treated with TPM and followed for up to 3 years (mean 15 months). Most children (55\%) had daily seizures and were tried on multiple antiepileptic drugs (mean 4.6). Nineteen (31\%) children had Lennox-Gastaut syndrome. After the introduction of TPM, 21 (34\%) became completely seizure free and 24 (39\%) had >50\% seizure reduction. Children with daily seizures were reduced from 55\% before TPM to 13\% on TPM (p=0.0007). Side effects were reported in 21 (34\%) children in the form of decreased appetite, weight loss, and sedation. The majority was transient; however, TPM had to be withdrawn in 7 (11\%) children because of progressive weight loss or seizure worsening. Follow up renal ultrasound was performed on 34 (55\%) children and was always normal. Topiramate is a very effective antiepileptic drug with a broad spectrum of antiepileptic activities. Most side effects were transient, however, careful monitoring of body weight is recommended. [\hyperlink{Elimite}{PMID: 23648886}, Abeer A Hassan et al., 2003]

\hypertarget{pmid_6879388}{F}ollowing the preliminary study in rats, etomidate (Hypnomidate; Janssen) 1,25\% in sterile water (pH 3,5) was administered rectally to 40 children aged between 6 months and 5 years to induce general anaesthesia for scheduled minor surgical procedures. The doses given ranged from 3,0 mg/kg to 6,5 mg/kg. The time taken for hypnosis to occur and the incidence of muscle movements were recorded. Cardiovascular and respiratory parameters were monitored. Anaesthesia was maintained with nitrous oxide/oxygen and halothane as needed. The time taken for recovery from anaesthesia was recorded and the children were observed for 24 hours postoperatively. The lowest hypnotic dose was 4,5 mg/kg, when 2 out of 5 children fell asleep. In all children given 6,5 mg/kg hypnosis occurred within 4 minutes. Cardiovascular and respiratory parameters were stable. There did not appear to be any delay in the recovery of the children from inhalational anaesthesia. There was no clinical evidence of irritation of the rectal mucosa by the etomidate solution. This study shows that rectal administration of etomidate 1,25\% in sterile water at a dose of approximately 20 times the recommended intravenous dose in children produces a rapid, predictable onset of hypnosis within 4 minutes and allows rapid recovery. This is highly suitable for outpatient anaesthesia in unpremedicated children who will accept neither inhalational nor intravenous induction. [\hyperlink{Elimite}{PMID: 6879388}, D M Linton et al., 1983]

\hypertarget{pmid_31850517}{T}o analyse the effects of felbamate in refractory infantile spasms/West syndrome. We conducted a 10-year retrospective study of infants (including all infants younger than 18mo) treated with felbamate for electroencephalography-recorded epileptic spasms persisting after first-line treatment. In total, 29 infants (17 males, 12 females) were included in the study. Felbamate was initiated at a mean age of 13.8 months (range 4.5-66mo) after sequential administration or combination of vigabatrin and oral steroids; a ketogenic diet was implemented in 23 infants. Eight infants became spasm-free at a mean dose of 34.6mg/kg/day felbamate (range 26-45mg/kg/day). Mean duration of felbamate use was 19 months (range 1-67mo) for the 19 infants whose treatment was terminated. No severe side effects were observed. Reversible neutropenia led to withdrawal of felbamate in six patients. One spasm-free patient demonstrated recurrence when felbamate was withdrawn. N-methyl-d-aspartate receptors with felbamate controlled epileptic spasms in eight infants resistant to first-line treatment should be targeted. [\hyperlink{Elimite}{PMID: 31850517}, Blandine Dozières-Puyravel et al., 2020]

\hypertarget{pmid_34002933}{E}arly introduction oral immunotherapy (E-OIT) in the first year of life can be a safe treatment for infants with cow's milk allergy (CMA). Once the protocol is completed, doubts remain whether children achieve tolerance or remain desensitized. According to current guidelines, this is determined by an avoidance period followed by a re-exposure to the food allergen during an in-hospital oral food challenge (OFC). In real life, this approach can be complicated, time-consuming, and anxiety-provoking for parents. We assessed the long-term safety of E-OIT for CMA in a cohort of children who switched to an unrestricted diet without testing the achievement of tolerance at the end of the OIT protocol. We performed a descriptive analysis of the clinical follow-up of a cohort of children diagnosed with IgE-mediated CMA and undergoing E-OIT protocol in their first year of life. In a previous publication, the same cohort of patients had been studied to assess the feasibility of E-OIT for CMA. In the present study, we reported the results of a telephone survey, carried out through a questionnaire to their families enquiring about milk consumption and other ongoing atopic conditions of children. After an average of 4 years from the start of E-OIT, 62/73 patients (85\% of the historical cohort) participated in the survey. Among them, all 56 patients who had previously successfully completed the protocol reported an unrestricted cow's milk intake. Ninety-three percent of these children did not experience any further allergic reactions, while the remaining 7\% described only mild and transitory reactions until the 6-month period after the end of the protocol. This study confirmed the long-term safety of E-OIT for CMA and challenged the paradigm of the need for allergen food withdrawal to discern between desensitization and tolerance. It could be a starting point for planning future trials on this issue. [\hyperlink{Elimite}{PMID: 34002933}, Laura Badina et al., 2021]

\hypertarget{pmid_20845767}{T}o evaluate the efficacy of topiramate (TPM) for the treatment of children with epilepsies, we introduced TPM to 45 patients whose epilepsy began in childhood and whose ages ranged from 4 months to 30 years old (mean age: 11 years 7 months). Thirteen of these patients had been diagnosed with generalized epilepsy (GE) (1 cryptogenic, 12 symptomatic), 30 with localization-related epilepsy (LRE) (7 idiopathic, 23 symptomatic), and 2 with unclassified epilepsy [1 case of severe myoclonic epilepsy in infancy (SMEI), 1 case of epilepsy with continuous spikes and waves during slow sleep (CSWS)]. The initial dose of TPM was 1.97 +/- 0.45 mg/kg/day, followed by a slow titration to the maximum dose of 7.32 +/- 1.32 mg/kg/day. After a mean treatment period of 13.5 months (range 4-20 months), the rate of reduction in seizure frequency by more than 50\% [50\% responder rate (50\% RR)] and the rate of complete remission (seizure-free) were 53.8\% and 23.1\%, respectively, in patients with GE, and 73.3\% and 23.3\%, respectively, in patients with LRE. TPM was significantly effective against many seizure types including tonic, clonic, complex partial, myoclonic, and atypical absence seizures. Adverse effects included sleepiness in 13 cases (28.9\%), weight loss in 6 cases (13.3\%), and metabolic acidosis in 2 cases (4.4\%); all of these effects were both mild and transient. In conclusion, TPM is effective and safe for the treatment of pediatric epilepsies. [\hyperlink{Elimite}{PMID: 20845767}, Masao Adachi et al., 2010]

\hypertarget{pmid_10371378}{T}he pharmacokinetic and safety profile of topiramate as adjunctive therapy was assessed in pediatric patients with epilepsy in an open-label, 4-week, single-center study. Six children from each of the following age groups were enrolled: 4-7 years, 8-11 years, and 12-17 years. Patients received topiramate 1 mg/kg/day for 1 week, with subsequent progressive weekly increases in dosage to 3, 6, and then 9 mg/kg/day or 800 mg/day, whichever was less. Topiramate oral plasma clearance (CI/F) was independent of dose, and steady-state plasma concentrations increased in proportion to dose. Weight-normalized topiramate CL/F was higher (P = 0.003) in pediatric patients receiving enzyme-inducing concomitant antiepileptic drugs (AEDs) (mean = 70.1 ml/minute/70 kg) than in those not receiving enzyme-inducing AEDs (mean = 33.1 mL/ minute/kg). Topiramate CL/F in children was approximately 50\% greater than that observed in adults regardless of the type of concomitant AED therapy. Thus steady-state plasma topiramate concentrations for the same mg/kg dose will be approximately 33\% lower in pediatric patients than in adult patients. The most frequently reported treatment-emergent adverse events considered related to topiramate therapy included anorexia, fatigue, and nervousness, and no patient discontinued therapy. This study indicates that, in children 4-17 years of age, topiramate has linear pharmacokinetics, 50\% higher clearance than in adults, and is generally well tolerated. [\hyperlink{Elimite}{PMID: 10371378}, W E Rosenfeld et al., 1999]

\hypertarget{pmid_24912730}{S}tudies on the efficacy and tolerability of rufinamide in infants and young children are scarce. Here we report on an open, retrospective, and pragmatic study about safety and efficacy of rufinamide in children aged less than four years, in terms of seizures types and epilepsy syndromes. Forty children (mean age 39.5 months; range 22-48) were enrolled in the study. The mean follow-up period was 12.2 months (range 5-21). Rufinamide was initiated at a mean age of 26.7 months (range 12-42). Final rufinamide mean dosage was 31.5 mg/kg/day if associated with valproic acid and 44.2 mg/kg/day if not. The highest seizure reduction rate was observed in the epileptic spasms (46\%) and drop attacks (42\%) groups. Seizure reduction was also observed in tonic seizures (35\%) and in the focal seizure (30\%) groups. In terms of epilepsy syndrome, rufinamide was effective in Lennox-Gastaut syndrome. Results were very poor for those affected by Dravet's syndrome. Globally, responder rate was 27.5\%, including two (5\%) patients seizure-free. Adverse reactions occurred in 37.5\% of children and were mainly represented by vomiting, drowsiness, irritability, and anorexia. Discontinuation rate due to treatment-emergent adverse events was 15\%. The present study concludes that rufinamide may be a safe and effective drug for a broad range of seizures and epilepsy syndromes in infants and young children and represents a valid therapeutic option in this population. [\hyperlink{Elimite}{PMID: 24912730}, Salvatore Grosso et al., 2014]

\section*{Luzu}
\subsection*{Result}
\subsubsection*{Answer}

Unknown

\subsubsection*{{Explanation}}
\hypertarget{Luzu}
A review of all available abstracts reveals that none of them are about "Luzu." The abstracts discuss a variety of medications and interventions in children, but there is no mention of Luzu, its safety, or any targeted pediatric safety studies. Therefore, based on the abstracts provided, there is not enough data available to determine if Luzu is safe or unsafe for use in children in any age range.

\subsection*{Abstracts}
\hypertarget{pmid_26415808}{I}n this paper, we focus on the characteristics of LUS in diagnosing childhood pulmonary disease. LUS is convenient, noninvasive, and free of radiation. It helps in the differentiation of lung diseases. Therefore, LUS has the potential to become a reference instrument for bedside dynamic respiratory monitoring. We hope that this review will help clinicians become acquainted with LUS and will accelerate the extensive application of LUS in children. [\hyperlink{Luzu}{PMID: 26415808}, Shui-Wen Chen et al., 2015]

\hypertarget{pmid_30622716}{P}neumonia is the leading infectious cause of death among children under 5 years of age worldwide. However, pneumonia is challenging to diagnose. Lung ultrasound (LUS) is a promising diagnostic technology. Further evidence is needed to better understand the role of LUS as a tool for the diagnosis of childhood pneumonia in low-resource settings. This study aims to pilot LUS in Mozambique and Pakistan and to generate evidence regarding the use of LUS as a diagnostic tool for childhood pneumonia. Children with cough <14 days with chest indrawing (n=230) and without chest indrawing (n=40) are enrolled. World Health Organization Integrated Management of Childhood Illness assessment is performed at enrolment, along with a chest radiograph and LUS examination. Respiratory and blood specimens are collected for viral and bacterial testing and biomarker assessment. Enrolled children are followed for 14 days (in person) and 30 days (phone call) post-enrolment with LUS examinations performed on Days 2, 6 and 14. Qualitative and quantitative data are also collected to assess feasibility, usability and acceptability of LUS among healthcare providers and caregivers. The primary outcome is LUS findings at enrolment with secondary outcomes including patient outcomes, repeat LUS findings, viral and bacterial test results, and patient status after 14 and 30 days of follow-up. This trial was approved by the Western Institutional Review Board as well as local ethics review committees at each site. We plan to disseminate study results in peer-reviewed journals and international conferences. NCT03187067. [\hyperlink{Luzu}{PMID: 30622716}, Jennifer L Lenahan et al., 2018]

\hypertarget{pmid_37492920}{C}hronic spontaneous urticaria (CSU), a long-lasting disease in children, impacts their quality of life. We report the results of a phase 2b dose-finding trial of ligelizumab (NCT03437278) and a high-affinity humanized monoclonal anti-IgE antibody, in adolescents with CSU, supported by modeling and simulation analyses, mitigating challenges in pediatric drug development. This multicenter, double-blind, placebo-controlled trial, randomized H1-antihistamine-refractory adolescent CSU patients (12-18 years) 2:1:1 to ligelizumab 24 mg, 120 mg, or placebo every 4 weeks for 24 weeks. Patients on placebo transitioned to ligelizumab 120 mg at week 12. Integrating data from the previous adult and present adolescent trial of ligelizumab, a nonlinear mixed-effects modeling described the longitudinal changes in ligelizumab pharmacokinetics, and its effect on weekly Urticaria Activity Score (UAS7). Baseline UAS7 (mean ± SD) was 30.5 ± 7.3 (n = 24), 29.3 ± 7.7 (n = 13), and 32.5 ± 9.0 (n = 12) for patients (median age, 15 years) on ligelizumab 24 mg, 120 mg, and placebo, respectively. Change from baseline in UAS7 at week 12 with ligelizumab 24 mg, 120 mg, and placebo was -15.7 ± 10.9, -18.4 ± 12.3, and -13.0 ± 13.0, respectively. Ligelizumab was well-tolerated. The modeling analysis showed that body weight, but not age, affected ligelizumab's apparent clearance. No significant differences between adolescents and adults were detected on the model-estimated maximum effect and potency. Ligelizumab exhibited efficacy and safety in adolescent CSU patients, consistent with that in adults. The PK and potency of ligelizumab were not impacted by age, and the same dose of ligelizumab can be used for treating adolescents and adults with CSU. Our study shows how modeling and simulation can complement pediatric drug development. [\hyperlink{Luzu}{PMID: 37492920}, Petra Staubach et al., 2023]

\hypertarget{pmid_20819318}{A}llergic rhinitis (AR) and chronic idiopathic urticaria (CIU) are common causes of substantial illness and disability in preschool children. Antihistamines are commonly used to treat preschool children with these conditions, but their use is based mostly on extrapolated efficacy from adult populations; it is thus important to characterize the safety of antihistamines in the pediatric population. This study was designed to assess the safety of levocetirizine dihydrochloride oral liquid drops in infants and children with AR or CIU. Two multicenter, double-blind, randomized, parallel-group studies randomized infants aged 6-11 months (study 1, n = 69) and children aged 1-5 years (study 2, n = 173) to levocetirizine, 1.25 mg (q.d. or b.i.d., respectively), or placebo for 2 weeks, using a 2:1 ratio. Safety evaluations included treatment-emergent adverse events (TEAEs), vital signs, electrocardiographic (ECG) assessments, and laboratory tests. The overall incidence of TEAEs was similar between levocetirizine and placebo in both studies. Most TEAEs were mild or moderate in intensity. TEAEs prompted discontinuation of therapy in three patients receiving levocetirizine in study 1. No clinically relevant changes from baseline in vital signs or laboratory parameters were apparent in either study; changes from baseline in these evaluations were similar between groups. No significant changes were observed in ECG parameters, including corrected QT interval. Levocetirizine, 1.25 and 2.5 mg/day, was well tolerated in infants aged 6-11 months and in children aged 1-5 years, respectively, with AR or CIU. [\hyperlink{Luzu}{PMID: 20819318}, Frank Hampel et al., ]

\hypertarget{pmid_23904337}{I}n Sub-Saharan Africa, intrarectal diazepam is the first-line anticonvulsant mostly used in children. We aimed to assess this standard care against sublingual lorazepam, a medication potentially as effective and safe, but easier to administer. A randomized controlled trial was conducted in the pediatric emergency departments of 9 hospitals. A total of 436 children aged 5 months to 10 years with convulsions persisting for more than 5 minutes were assigned to receive intrarectal diazepam (0.5 mg/kg, n = 202) or sublingual lorazepam (0.1 mg/kg, n = 234). Sublingual lorazepam stopped seizures within 10 minutes of administration in 56\% of children compared with intrarectal diazepam in 79\% (P < .001). The probability of treatment failure is higher in case of sublingual lorazepam use (OR = 2.95, 95\% CI = 1.91-4.55). Sublingual lorazepam is less efficacious in stopping pediatric seizures than intrarectal diazepam, and intrarectal diazepam should thus be preferred as a first-line medication in this setting.  [\hyperlink{Luzu}{PMID: 23904337}, Célestin Kaputu Kalala Malu et al., 2014] Chest radiography (CXR) is the test of choice for diagnosing pneumonia. Lung ultrasonography (LUS) has been shown to be accurate for diagnosing pneumonia in children and may be an alternative to CXR. Our objective was to determine the feasibility and safety of substituting LUS for CXR when evaluating children suspected of having pneumonia. We conducted a randomized control trial comparing LUS with CXR in 191 children from birth to 21 years of age suspected of having pneumonia in an ED. Patients in the investigational arm underwent LUS. If there was clinical uncertainty after ultrasonography, physicians had the option to perform CXR. Patients in the control arm underwent sequential imaging with CXR followed by LUS. The primary outcome was the rate of CXR reduction; secondary outcomes were missed pneumonia, subsequent unscheduled health-care visits, and adverse events between the investigational and control arms. There was a 38.8\% reduction (95\% CI, 30.0\%-48.9\%) in CXR among investigational subjects compared with no reduction (95\% CI, 0.0\%-3.6\%) in the control group. Novice and experienced physician-sonologists achieved 30.0\% and 60.6\% reduction in CXR use, respectively. There were no cases of missed pneumonia among all study participants (investigational arm, 0.0\%: 95\% CI, 0.0\%-2.9\%; control arm, 0.0\%: 95\% CI, 0.0\%-3.0\%), or differences in adverse events, or subsequent unscheduled health-care visits between arms. It may be feasible and safe to substitute LUS for CXR when evaluating children suspected of having pneumonia with no missed cases of pneumonia or increase in rates of adverse events. ClinicalTrials.gov; No.: NCT01654887; URL: www.clinicaltrials.gov. [\hyperlink{Luzu}{PMID: 23904337}, Brittany Pardue Jones et al., 2016]

\hypertarget{pmid_27917769}{T}his descriptive study provides the first information on an association between the use of sedation and a reduction in the prevalence of unsuccessful lumbar puncture (LP) in African children of all races. Our hypothesis was that children who do not receive any procedural sedation are more likely to have unsuccessful LPs. A cross-sectional observational study examined LPs performed from February to April 2013, including details of the procedure, sedation or analgesia used, and techniques. The setting was the Medical Emergency Unit at Red Cross War Memorial Children's Hospital, Cape Town, South Africa, and the participants all children aged 0 - 13 years who had an LP in the unit during the time period. Of 350 children, 62.9\% were \&lt;12 months of age, the median age being 4.8 months (interquartile range 1.5 - 21.7). The prevalence of unsuccessful (traumatic or dry) LP was 32.3\% (113/350). Sedation was used in 107 children (30.6\%) and was associated with a reduction in the likelihood of unsuccessful LP (p=0.002; risk ratio (RR) 0.5 (95\% confidence interval (CI) 0.34 - 0.78)) except in those \&lt;3 months of age, where sedation did not significantly reduce the likelihood (p=0.56; RR 1.20 (95\% CI 0.66 - 2.18)). Unsuccessful LP was common. Sedation was not routinely used, but the results suggest that it may be associated with a reduction in the rate of unsuccessful LP. Unsuccessful LP may lead to diagnostic uncertainty, prolonged hospitalisation and unnecessary antibiotic use. Whether a procedural sedation protocol would reduce the rate of unsuccessful LP requires further study. [\hyperlink{Luzu}{PMID: 27917769}, C Procter et al., 2016]

\hypertarget{pmid_9264438}{I}n this study, the first study performed in pediatric patients, we assessed the safety and efficacy of a steroid-eluting active fixation ventricular electrode in 18 children. Our study shows that steroid-eluting active fixation leads are safe and effective in children, and suggests that these leads with their easy implantation and low chronic threshold values may be considered as the first choice in this age group. [\hyperlink{Luzu}{PMID: 9264438}, A Celiker et al., 1997]

\hypertarget{pmid_17611334}{T}o observe the effect of sevoflurane on the induction and maintenance of anaesthesia in children, and to evaluate its safety and effectiveness. Forty child patients who conformed to the selection standard were operated under anaesthesia with intubation.Without premedicant, all the patients inhaled 100\% oxygen(1L/min) and sevoflurane by mask, and escalated the concentration of sevoflurane (to the maximum concentration 7\%) until the lash reflex disappeared, and the maintenance concentration was controlled under 4\%. All the patients were intubated, together with vecuronium 0.1mg/kg. With little tract excretion, the achievement ratio of induction by sevoflurane was 100\%, and the children tolerated well. With stable hemodynajmics,1\% approximately 4.0\% maintenance concentration of sevoflurane during the operation showed effective anaesthesia, no decreased heart rate or blood pressure appeared, and all the patients' body temperature was normal. Sevoflurane for children induction can bring fewer stimuli in the respiratory tract,less cardiac vascular inhibition and palinesthesia time. Anaesthesia in children induced by sevoflurane is safe and effective. [\hyperlink{Luzu}{PMID: 17611334}, Xi-ying Zhang et al., 2007]

\hypertarget{pmid_23131185}{T}he safety of a novel 0.5\% ivermectin lotion (IVL) and potential for ivermectin absorption after application was investigated in an open-label study in young children, and a human repeat insult patch test (HRIPT) and cumulative irritation test (CIT) assessed any potential for cumulative dermal irritation and contact sensitization. In the pharmacokinetic and safety study, 30 head louse-infested children ages 6 months to 3 years received a 10-minute application of IVL on day 1. Blood was collected before application; 0.5, 1, and 6 hours after rinsing; and on days 2 and 8. Samples from 20 subjects were assayed for ivermectin (test sensitivity 0.05 ng/mL). Liver panel and complete blood counts were completed for all subjects. For the HRIPT/CIT, occlusive patches containing IVL or vehicle control lotion (CL) were repeatedly applied to 220 healthy adult subjects to assess contact sensitization; for cumulative dermal irritation testing, additional patches with normal saline and sodium dodecyl sulfate (SDS) were applied to 36 subjects. In the open-label study, all detected ivermectin plasma concentrations were <1 ng/mL. No safety signals emerged, and treatment was well tolerated. In the HRIPT/CIT, IVL was significantly less irritating than normal saline and SDS, with no evidence of dermal irritation or sensitization in human skin. IVL was safe when applied topically, absorption was de minimus, there was no evidence of irritation or sensitization from repeated exposures, and results support the safety of topical IVL use in children as young as 6 months. [\hyperlink{Luzu}{PMID: 23131185}, Lydie Hazan et al., ]

\hypertarget{pmid_16900731}{H}ead louse infestations are prevalent worldwide. Over the past 20-25 years, 15-20\% of all children in Israel between 4 and 13 years of age have been infested with head lice. This is mainly due to the existence of ineffective pediculicides on the market. To examine the pediculicidal efficacy and safety of Prioderm Cream Shampoo in an open clinical study. The active ingredient of Prioderm Cream Shampoo is malathion (1\%) and it was applied to the hair of infested children three times for 10 minutes at 5-day intervals. Of 419 children examined, aged 6-14 years, from four schools in Jerusalem, 109 (26.0\%) were infested with lice and eggs, while 110 (26.3\%) were infested only with nits. All 109 louse-infested children were included in the study while 86 (78.8\%) complied with the instructions for use. Treatment was successful in 79 children (91.8\%), while failure was observed in 3 children after the first treatment and in 4 children after the third treatment. Although 21.5\% of the children disliked the odor of the product, there were no significant side effects related to this formulation. The product was very effective in controlling louse infestations under clinical conditions and caused no serious side effects. [\hyperlink{Luzu}{PMID: 16900731}, Kosta Y Mumcuoglu et al., 2006]

\hypertarget{pmid_29570940}{P}neumonia is still a leading cause of illness and death in infants worldwide. Lung ultrasound (LUS) is emerging as an extremely valuable non-ionizing radiation diagnostic tool in diagnosis and follow up of multiple paediatric pulmonary diseases. To assess the applicability of LUS in diagnosis and follow up of community acquired pneumonia (CAP) in Egyptian infants younger than 1-year old. LUS and chest X-ray (CXR) were performed in 50 infants presented with clinical symptoms and signs suggestive of CAP within the first 6 hours of admission in our inpatient department, then follow up LUS was performed 5 days after admission. This study showed that LUS was superior to CXR in initial diagnosis of CAP in infants. LUS detected pneumonia in 49 (98\%) compared to 36 (72\%) infants diagnosed by CXR (P < .05). On follow up, 5 days later, consolidation patch disappeared in 13 (26.5\%) infants, diminished in size in 27 (55.1\%) infants, remained at the same size in 2 (4.1\%) infants and increased in size in 7 (14.3\%) infants. This study showed that LUS is superior to CXR in diagnosing infants with CAP who are younger than 1-year old. It also serves as a safe follow up tool and could support the decision of hospital discharge in this category of patients. Further studies with larger sample size and longer follow up duration are recommended to confirm the results of the present study. [\hyperlink{Luzu}{PMID: 29570940}, Ahmed Omran et al., 2018]

\hypertarget{pmid_29931473}{P}neumonia is the third leading cause of death in children under 5 years of age worldwide. In pediatrics, both the accuracy and safety of diagnostic tools are important. Lung ultrasound (LUS) could be a safe diagnostic tool for this reason. We searched in the literature for diagnostic studies about LUS to predict pneumonia in pediatric patients using systematic review and meta-analysis. The Medline, CINAHL, Cochrane Library, Embase, SPORTDiscus, ScienceDirect, and Web of Science databases from inception to September 2017 were searched. All studies that evaluated the diagnostic accuracy of LUS in determining the presence of pneumonia in patients under 18 years of age were included. 1042 articles were found by systematic search. 76 articles were assessed for eligibility. Seventeen studies were included in the systematic review. We included 2612 pooled cases. The age of the pooled sample population ranged from 0 to about 21 years old. Summary sensitivity, specificity, and AUC were 0.94 (IQR: 0.89-0.97), 0.93 (IQR: 0.86-0.98), and 0.98 (IQR: 0.94-0.99), respectively. No agreement on reference standard was detected: nine studies used chest X-rays, while four studies considered the clinical diagnosis. Only one study used computed tomography. LUS seems to be a promise tool for diagnosing pneumonia in children. However, the high heterogeneity found across the individual studies, and the absence of a reliable reference standard, make the finding questionable. More methodologically rigorous studies are needed. [\hyperlink{Luzu}{PMID: 29931473}, Daniele Orso et al., 2018]

\hypertarget{pmid_24979880}{L}umbar puncture (LP) is usually associated with anxiety and apprehension in children and their parents. This study was performed for controlling children's anxiety before and during LP and increasing the success of LP due to relaxation of the child following the use of sedative drugs and to compare the efficacy and side effects of oral midazolam and oral chloral hydrate. This prospective randomized controlled clinical trial included 160 children aged 2-7 years, candidates for LP. They were divided into two randomized groups of 80 children each: group I received 80 mg/kg oral chloral hydrate and group II received 0.5 mg/kg oral midazolam before LP. The results indicated that the mean sedation grade was 3.8 +/- 1.0 in chloral hydrate group and 2.3 +/- 0.9 in midazolam group (P < 0.001). The mean onset of sedative effect was 30.9 +/- 8.8 min in midazolam group and 16.5 +/- 5.8 min in chloral hydrate group (P < 0.001). Prolonged sedation was the most common side effect in oral midazolam group (94.4\%) versus 22.2\% in chloral hydrate group. Based on the level of sedation, side effects, time to onset of sedation and recovery time from sedation, oral chloral hydrate is a better sedative medication than oral midazolam. [\hyperlink{Luzu}{PMID: 24979880}, Hojjat Derakhshanfar et al., 2013]

\hypertarget{pmid_12656908}{T}o assess the efficacy and safety of laparoscopically assisted ureterocystoplasty (LAU) in children. From 1999 to 2001, five patients (mean age 7 years, range 3.5-13) from four centres underwent LAU with laparoscopic mobilization of the small kidney and upper ureter combined with ureterocystoplasty, with exposure of the bladder through a Pfannenstiel incision. The details and outcomes are reviewed. The LAU was successful in all five patients; there were no complications. A large midline incision was avoided and the LAU carried out through the better tolerated and less painful Pfannenstiel incision. LAU is an appealing technique that is safe with the added benefit of a reduced abdominal incision and acceptable operative duration. This represents the first published report of LAU. [\hyperlink{Luzu}{PMID: 12656908}, B G Cilento et al., 2003]

\hypertarget{pmid_24665291}{L}umbar puncture (LP) essentially is a painful and stressful procedure, however indicated for diagnosis and therapeutic purposes. One way to reduce the anxiety is to administer an oral premedication. The aim of this study is to compare clinical effects of oral midazolam and oral promethazine in LP. This prospective randomized controlled clinical trial study was performed on 80 children aged 2-7 years that were candidate for LP. They were divided into two randomized equal groups. First group received oral midazolam syrup 0.5 mg/kg and the other group received oral promethazine syrup 1mg/kg. Level of sedation, hemodynamic changes and any other complications were monitored every 5 minutes from 30 minutes before the start of the procedure. Midazolam group and promethazine group were similar in age, gender and weight. Midazolam had significantly shorter onset of sedation and also shorter duration to maximal sedation. The two groups were similar with respect to sedative effect at all time. The only complication that was significantly more in midazolam group was nausea and vomiting. Midazolam syrup and promethazine syrup have same sedative effect in children. Both of these medications are easy to use in preschool children and none of them appeared to be superior to another. [\hyperlink{Luzu}{PMID: 24665291}, Hojjat Derakhshanfar et al., 2013]

\hypertarget{pmid_28475231}{N}umerous reports describe the successful use of nitrous oxide for analgesia in children undergoing painful procedures. Although shown to be safe, effective, and economical, nitrous oxide use is not yet common in pediatric oncology clinics and few reports detail its effectiveness for children undergoing repeated lumbar punctures. We developed a nitrous oxide clinic, and undertook a review of pediatric oncology lumbar puncture records for those patients receiving nitrous oxide in 2011. No major complications were noted. Minor complications were noted in 2\% of the procedures. We offer guidelines for establishing such a clinic. [\hyperlink{Luzu}{PMID: 28475231}, Mylynda Livingston et al., 2017]

\hypertarget{pmid_25593242}{R}ituximab (RTX) has been used to treat many pediatric autoimmune conditions. We investigated the safety and efficacy of RTX in a variety of pediatric autoimmune diseases, especially systemic lupus erythematosus (SLE). Retrospective study of children treated with RTX. Effectiveness data was recorded for patients with at least 12 months of followup; safety data was recorded for all subjects. The study included 104 children; 50 had SLE. Improvements in corticosteroid dosage, physician's global assessment of disease activity, and SLE-associated markers of disease activity were seen. The incidence of hospitalized infections was similar to previous studies of patients with childhood-onset SLE. RTX can be safely administered to children and appears to contribute to decreased disease activity and steroid burden. [\hyperlink{Luzu}{PMID: 25593242}, Ajay Tambralli et al., 2015]

\hypertarget{pmid_31384086}{F}aecal disimpaction is very important for successful management of the constipation in children. Lactulose is cheap and widely available medicine compared to other polyethylene glycol (PEG) preparations. From our experience, lactulose is effective and safe medicine for both disimpaction and maintenance therapy in constipated children. The purpose of the present study was to evaluate the safety and efficacy of lactulose in faecal impaction management in children with constipation. We conducted a prospective controlled trial in children with functional constipation, who presented with faecal impaction to Queen Rania Hospital for Children from April 15, 2018 until October 15, 2018. Two randomised matched groups; group A included 33 constipated children treated for disimpaction with higher dose lactulose (10 g/15 ml) 4-6 ml/kg/day (max. 120 ml/day) and group B included 32 children treated for disimpaction with macrogol (PEG 4000) 1-1.5 g/kg (max. 30 g/day). Both groups received treatment until resolution or up to 6 days. Patients were followed over 1 week and success of disimpaction was observed. Moreover, any adverse events were recorded. All the patients in both groups achieved successful disimpaction by seventh day of the therapy, group B showed significant faster response. Both therapies were tolerated and no significant adverse events were reported. Both agents were safe, effective and well tolerated. Lactulose may be a good alternative to PEG in the treatment of faecal impaction in constipated children. [\hyperlink{Luzu}{PMID: 31384086}, Mohammad Salem Shatnawi et al., 2019]

\hypertarget{pmid_28553374}{T}he aim of this study is to evaluate the efficacy and safety of levetiracetam (LEV) as first-line treatment of neonatal seizures. This study was conducted in patients of Neonatal Intensive Care Unit of Santo Bambino Hospital, University of Catania, Italy, from January to August 2016. A total of 16 neonates with convulsions not associated with major syndromes, which required anticonvulsant therapy, were included and underwent IV LEV at standard doses. All patients responded to treatment, with a variety range of seizure resolution period (from 24 h to 15 days; mean hours: 96 ± 110.95). No patient required a second anticonvulsant therapy. Regarding safety of LEV, no major side-effects were observed. To our knowledge, it is one of the few studies confirming the efficiency of LEV as first-line treatment in seizures of this age group. LEV was effective in resolving seizures and was safely administered in the current study. [\hyperlink{Luzu}{PMID: 28553374}, Raffaele Falsaperla et al., ]

\hypertarget{pmid_31978279}{L}ung ultrasound (LUS) has been increasingly used in diagnosing and monitoring of various pulmonary diseases in children. The aim of the current study was to evaluate its usefulness in children with persistent tachypnea of infancy (PTI). This was a controlled, prospective, cross-sectional study that included children with PTI and healthy subjects. In patients with PTI, LUS was performed at baseline and then after 6 and 12 months of follow-up. Baseline results of LUS were compared to (a) baseline high-resolution computed tomography (HRCT) images, (b) LUS examinations in control group, and (c) follow-up LUS examinations. Twenty children with PTI were enrolled. B-lines were found in all children with PTI and in 11 (55\%) control subjects (P < .001). The total number of B-lines, the maximal number of B lines in any intercostal space, the distance between B-lines, and pleural thickness were significantly increased in children with PTI compared to controls. An irregularity of the pleural line was found in all patients with PTI and in none of the healthy children. There were no significant changes in LUS findings in patients with PTI during the study period. The comparison of HRCT indices and LUS findings revealed significant correlations between the mean lung attenuation, skewness, kurtosis and fraction of interstitial pulmonary involvement, and the number of B-lines as well as the pleural line thickness. LUS seems to be a promising diagnostic tool in children with PTI. Its inclusion in the diagnostic work-up may enable to reduce the number of costly, hazardous, and ionizing radiation-based imaging procedures. [\hyperlink{Luzu}{PMID: 31978279}, Emilia Urbankowska et al., 2020]

\hypertarget{pmid_33569493}{I}mproved pneumonia diagnostics are needed, particularly in resource-constrained settings. Lung ultrasound (LUS) is a promising point-of-care imaging technology for diagnosing pneumonia. The objective was to explore LUS patterns associated with paediatric pneumonia. We conducted a prospective, observational study among children aged 2 to 23 months with World Health Organization Integrated Management of Childhood Illness chest-indrawing pneumonia and among children without fast breathing, chest indrawing or fever (no pneumonia cohort) at two district hospitals in Mozambique and Pakistan. We assessed LUS and chest radiograph (CXR) examinations, and viral and bacterial nasopharyngeal carriage, and performed a secondary analysis of LUS patterns. LUS demonstrated a range of distinctive patterns that differed between children with and without pneumonia and between children in Mozambique  Pattern recognition was discordant between LUS and CXR imaging modalities. Further research is needed to define and standardise LUS patterns associated with paediatric pneumonia and to evaluate the potential value of LUS as a reference standard. [\hyperlink{Luzu}{PMID: 33569493}, Amy Sarah Ginsburg et al., 2021]

\hypertarget{pmid_33788390}{A}lthough effectiveness of hydroxyurea (HU) in sickle cell disease is well established, unanswered questions persist about its use in African children. We determined real-life issues of acceptability, availability, and monitoring of HU use in Nigeria. A retrospective longitudinal review of laboratory data of patients on HU was done from case files, followed by a cross-sectional survey that captured families' perception of medication and clinic adherence, laboratory tests, benefits, side effects, and acceptability. One hundred sixteen patients (1.2-17 years) received HU (mean ± SD = 18.5 ± 4.3 mg/kg/day) in 33 months. Eighty-nine had laboratory analysis. Dose escalation was the initial goal, but only 80\% of patients had some form of it. Parents reported improvement in general well-being and reduction in bone pain episodes, hospital admissions, and blood transfusion. While most parents (89.5\%) reported satisfaction with HU, 61\% reported dissatisfaction with daily drug use, and the frequency and cost of monitoring. Sixteen percent voluntarily stopped therapy. Adherence to daily HU was 88.8\%, doctor's appointments 24.5\%, hematology tests 18.9\%, and organ function tests 37.4\%. There were no significant toxicities. Significant increases in hemoglobin, hemoglobin F and mean corpuscular volume, and reduction in absolute neutrophil count occurred despite inconsistent dose escalation. HU (10-15 mg/kg/day starting dose) is safe and seems effective and acceptable to parents. Parental commitment to therapy, pre-HU education (that continues during therapy), provision of affordable HU, and subsidized laboratory tests are important considerations for initiating therapy. Special HU clinics may facilitate dose escalation and reduce frequency of monitoring. Studies are needed on feasibility of maximum tolerable dose HU protocols in sub-Saharan Africa without compromising safety. [\hyperlink{Luzu}{PMID: 33788390}, Uche Nnebe-Agumadu et al., 2021]

\hypertarget{pmid_32938279}{T}o describe safety and feasibility of long-term inhalative sedation (LTIS) in children with severe respiratory diseases compared to patients with normal lung function with respect to recent studies that showed beneficial effects in adult patients with acute respiratory distress syndrome (ARDS). Single-center retrospective study. 12-bed pediatric intensive care unit (PICU) in a tertiary-care academic medical center in Germany. All patients treated in our PICU with LTIS using the AnaConDa® device between July 2011 and July 2019. Thirty-seven courses of LTIS in 29 patients were analyzed. LTIS was feasible in both groups, but concomitant intravenous sedatives could be reduced more rapidly in children with lung diseases. Cardiocirculatory depression requiring vasopressors was observed in all patients. However, severe side effects only rarely occured. In this largest cohort of children treated with LTIS reported so far, LTIS was feasible even in children with severely impaired lung function. From our data, a prospective trial on the use of LTIS in children with ARDS seems justified. However, a thorough monitoring of cardiocirculatory side effects is mandatory. [\hyperlink{Luzu}{PMID: 32938279}, Jochen Meyburg et al., 2021]

\hypertarget{pmid_25338496}{T}o define the efficacy and safety of low-dose rasburicase in children from south India with hematologic malignancies. This study is a retrospective analysis of data on 41 children with hematologic malignacies with laboratory evidence of tumor lysis syndrome (TLS) or clinical features indicating high risk for developing TLS. Patients were treated with rasburicase in doses of 0.1-0.15 mg/kg dose, repeated when necessary. Male : Female ratio was 32:9. Thirty-six children had laboratory evidence of TLS and 5 were at risk for TLS. Diagnoses were T-cell acute lymphoblastic leukemia (ALL), 19; Pre-B ALL, 17; B-non-Hodgkin lymphoma (NHL), 2; T-NHL, 2; and acute myeloid leukemia (AML), 1. Initial plasma uric acid (PUA): median, 8.5 mg/dl (range, 4.3 to 45.5). Six had creatinine levels of >2 mg/dl on admission; and 10 had peak PO4 levels of >10 mg/dl. Dose of rasburicase used: median, 0.12 mg/kg (range, 0.08-0.24). Median reduction of PUA at 6 h: 80 \% (range 40 to 98 \%). Twenty-seven needed only one dose; 12 needed 2 or 3 doses; and two needed 5 doses each. One child required dialysis. None of the children developed anaphylaxis or hemolysis and there were no deaths from TLS. Low-dose rasburicase (0.1-0.15 mg/kg) is safe and effective in reducing PUA in Indian children with lymphoid malignancies, and thus it may reduce the risk of renal failure from TLS. [\hyperlink{Luzu}{PMID: 25338496}, Somasundaram Jayabose et al., 2015]

\section*{Nystatin and Triamcinolone Acetonide}
\subsection*{Result}
\subsubsection*{Answer}

Unknown

\subsubsection*{{Explanation}}
\hypertarget{Nystatin and Triamcinolone Acetonide}
Based on the abstracts available, there are no studies specifically evaluating the safety of the combination product Nystatin and Triamcinolone Acetonide in children. However, there are several studies on the individual components:

**Nystatin:**
- Multiple studies have evaluated nystatin in children, including neonates, infants, and children up to adolescence. These studies generally report that nystatin is safe and well-tolerated in these populations, with no significant adverse events reported. For example, a randomized trial in infants with diaper dermatitis found nystatin to be safe and well-tolerated [\hyperlink{pmid_20553355}{PMID: 20553355}, P H Hoeger et al., 2010]. Another study in children undergoing chemotherapy found nystatin to be as safe as fluconazole [\hyperlink{pmid_9462438}{PMID: 9462438}, A H Groll et al., 1997]. In neonates, nystatin prophylaxis reduced invasive candidiasis without significant safety concerns [\hyperlink{pmid_17022766}{PMID: 17022766}, Mehmet Adnan Ozturk et al., 2006]. However, one study in very low birthweight infants reported a higher mortality rate in the nystatin group, though deaths were not directly attributed to nystatin [\hyperlink{pmid_19504425}{PMID: 19504425}, Kimon Violaris et al., 2010].

**Triamcinolone Acetonide:**
- Several studies have evaluated triamcinolone acetonide (TA) in children, particularly as a nasal spray for allergic rhinitis and as intraarticular or subtenon injections for other conditions. For intranasal use, studies in children aged 2–5 years [\hyperlink{pmid_19441606}{PMID: 19441606}, Steven Weinstein et al., 2009], 4–12 years [\hyperlink{pmid_1958002}{PMID: 1958002}, M J Welch et al., 1991], and 6–11 years [\hyperlink{pmid_8733987}{PMID: 8733987}, C H Banov et al.] found TA to be safe and well-tolerated, with no significant adverse effects on growth or cortisol levels. A 1–2 year follow-up study in children aged 6–14 years found no significant effect on statural growth [\hyperlink{pmid_18939734}{PMID: 18939734}, David P Skoner et al., 2008]. Intraarticular and subtenon injections in children (including those as young as 7 years) were generally well-tolerated, though rare cases of Cushing's syndrome have been reported after intralesional injection [\hyperlink{pmid_7382013}{PMID: 7382013}, R R Augspurger et al., 1980].

**Combination Product (Nystatin and Triamcinolone Acetonide):**
- There are no abstracts describing targeted safety studies of the combination of nystatin and triamcinolone acetonide in children of any age group. Therefore, based on the available abstracts, the safety of the combination product in children is unknown.

\subsection*{Abstracts}
\hypertarget{pmid_36610724}{T}he indications for nystatin as prophylaxis or treatment are limited. In the PASOAP (Pediatric Antifungal Stewardship Optimizing Antifungal Prescription) study, high use of nystatin in hospitalized children beyond the neonatal age was observed. In this report, we present the data on nystatin use in infants and children ≥ 3 months who participated in the PASOAP study. Nystatin was prescribed mainly for prophylaxis. Congenital heart disease, cystic fibrosis, and chronic renal disease were the most commonly reported conditions in children receiving prophylactic nystatin. There is sparse evidence supporting the use of nystatin prophylaxis beyond neonates; trials in specific pediatric patient groups are required. [\hyperlink{Nystatin and Triamcinolone Acetonide}{PMID: 36610724}, Harshani Jayawardena-Thabrew et al., 2022]

\hypertarget{pmid_19346579}{T}hirteen children with juvenile idiopathic arthritis (JIA) were treated with intraarticular steroid injection of triamcilone acetonide as a day care procedure. More than half (53.4\%) the children were free of pain, limp and NSAID's use, with improvement in functional score at 12 weeks. No side effects were reported during the period of the study. [\hyperlink{Nystatin and Triamcinolone Acetonide}{PMID: 19346579}, Sumit Verma et al., 2009]

\hypertarget{pmid_35179713}{S}ubtenon triamcinolone acetonide (Kenalog®; Bristol Myers Squibb) (STA) injections are commonly used in the treatment of adults in an outpatient setting. However, publications on detailing its outpatient use, safety, and efficacy in the pediatric population are scarce. We reviewed STA injections performed in children in the outpatient clinics at two tertiary centers from 2014 to 2020. All children were aged ≤ 18 years and had a diagnosis of non-infectious uveitis. STA injections were done using 0.5 cc (20 mg) triamcinolone injected superotemporally with only topical anesthesia. Data on the efficacy and safety of STA in treating inflammation and compiled data on visual acuity improvement and incidence of ocular complications were evaluated. Forty-eight eyes in 30 patients were included. The mean age of patients was 13.1 (range 7-18) years. There were no immediate complications observed in all injections performed. At the 3-month follow-up, inflammation had improved in 85.4\% of eyes, macular edema had resolved in 77.8\% of eyes, and there was significant vision improvement after STA. At 6 months after STA, the incidence of ocular hypertension was 12.5\% and no new cataracts had developed. STA injection with topical anesthesia was a well-tolerated, reasonable alternative for short-term treatment of uveitis among this pediatric population. [\hyperlink{Nystatin and Triamcinolone Acetonide}{PMID: 35179713}, Jennifer L Jung et al., 2022]

\hypertarget{pmid_7382013}{W}e report 2 cases of Cushing's syndrome following intralesional triamcinolone acetonide injections of urethral strictures in children. The pharmacology of triamcinolone and its 2 parenteral forms, triamcinolone acetonide and triamcinolone diacetate, is discussed. For children we recommend the short-acting triamcinolone diacetate at 4-week intervals with dosage adjusted to age. In adults either type of triamcinolone may be used but triamcinolone acetonide should be given at 6-week intervals. [\hyperlink{Nystatin and Triamcinolone Acetonide}{PMID: 7382013}, R R Augspurger et al., 1980]

\hypertarget{pmid_15844000}{N}oncompliance is frequent in children and adolescents with nephrotic syndrome. Once suspected, noncompliance is difficult to confirm and often impossible to avoid. The standard oral glucocorticoid treatment for children has been shown to be efficient and safe. However, a small number of children/parents are noncompliant to the steroid treatment, resulting in multiple relapses. For these patients the use of steroids with prolonged half-life such as triamcinolone acetonide (TA) can be helpful. We studied seven children (six boys, one girl; median age at diagnosis 8.6 years, range 1.8-10.7) receiving conventional steroid treatment for a median of 30 months (8-74) before starting intramuscular (IM) TA treatment. The standard prednisone treatment was replaced by 1 monthly IM injection of TA (1 mg/kg per day oral prednisone replaced by 1 mg/kg per month IM TA). The treatment was tapered off by a reduction of 10-20\% of the initial dose per month over 6-8 months. After a mean observation period of 14 months (3-36) the results were evaluated in terms of number of relapses and treatment tolerance. Four children showed a clear decrease in number of relapses (1.8 to 0 per year); in the other three the number of relapses remained stable. Tolerance was excellent (no cataract, no arterial hypertension), and the cushingoid syndrome did not exceed the level experienced under conventional oral steroid therapy. However, growth velocity decreased during the TA treatment and returned to normal after discontinuation of TA. These preliminary results demonstrate that TA may be used in patients of suspected noncompliance in steroid-sensitive patients who respond with a complete remission during TA treatment over the observation period. Patients who do not benefit from the TA can be classified as very probably steroid-dependent. TA seems to be a useful therapeutic strategy in those patients for whom noncompliance is strongly suspected. [\hyperlink{Nystatin and Triamcinolone Acetonide}{PMID: 15844000}, Tim Ulinski et al., 2005]

\hypertarget{pmid_20386439}{A}lbumin has been regarded as the gold standard for maintaining adequate colloid osmotic pressure in children, but increased cost, the lack of clear-cut benefits for survival, and fear of transmission of unknown viruses have contributed to its replacement by hydroxyethyl starch and gelatin preparations. Each of the synthetic colloids has unique physicochemical characteristics that determine their likely efficacy and adverse effect profile. This review will examine the advantages and disadvantages of the use of different colloid solutions in children with a particular focus on their safety profile. Dextrans are rarely used because of their negative effects on coagulation and potential for anaphylactic reactions. Gelatin and albumin have little effect on hemostasis, but the disadvantages of gelatin include its high anaphylactoid potential and limited beneficial volume effect. Tetrastarches have significantly fewer adverse effects on coagulation and renal function than the older hydroxyethyl starches and are now approved for children. Dissolving tetrastarches in a plasma-adapted, balanced solution rather than in saline further improves safety with regard to coagulation and acid-base balance. Tetrastarches offer the best currently available compromise between cost-effectiveness and safety profile in children with preexisting normal renal function and coagulation. [\hyperlink{Nystatin and Triamcinolone Acetonide}{PMID: 20386439}, Sonja Saudan et al., 2010]

\hypertarget{pmid_19504425}{W}e compared the efficacy and safety of fluconazole and nystatin oral suspensions for the prevention of systemic fungal infection (SFI) in very low birthweight infants. A prospective, randomized clinical trial was conducted over a 15-month period, from May 1997 through September 1998, in 80 preterm infants with birthweights <1500 g. The infants were randomly assigned to receive oral fluconazole or nystatin, beginning within the first week of life. Prophylaxis was continued until full oral feedings were attained. Blood and urine cultures were obtained at enrollment and then weekly thereafter. Thirty-eight infants were randomly assigned to receive oral fluconazole (group I), and 42 infants were assigned to receive nystatin (group II). Birthweight, gestational age, and risk factors for fungal colonization and SFI at the time of randomization and during the hospital course were similar in both groups. SFI developed in two infants (5.3\%) in group I and six infants (14.3\%) in group II. The difference between these two rates was not statistically significant (relative risk, 0.37; 95\% confidence interval, 0.08 to 1.72). There were no deaths in group I and six deaths in group II (P = 0.03). Two infants died of neonatal sepsis, and four deaths were related to necrotizing enterocolitis and/or spontaneous intestinal perforation. No deaths were due to SFI. Enrollment was halted before completion and the study did not attain adequate power to detect a hypothesized drop in SFI rate from 15 to 5\%. Although the results cannot justify any conclusion about the relative efficacy of fluconazole versus nystatin in prevention of SFI, the significantly higher mortality rate in the nystatin group raises questions about the relative safety of this medication. [\hyperlink{Nystatin and Triamcinolone Acetonide}{PMID: 19504425}, Kimon Violaris et al., 2010]

\hypertarget{pmid_1958002}{T}riamcinolone acetonide aerosol (TAA), a topical corticosteroid, now available for intranasal use, has been shown to be highly effective in the treatment of both seasonal and perennial allergic rhinitis (PAR) in adults. To evaluate the efficacy and safety of TAA in children, 210 patients (ages 4 to 12 years) with PAR were randomly assigned to one of three treatment groups (placebo, TAA 82.5 micrograms/day, or TAA 165 micrograms/day). Medication was given tid over 12 weeks in a double-blind fashion. Response to medication was evaluated using symptom scoring, physician evaluation, and, in 44 patients, nasal airflow determinations by anterior rhinomanometry. The higher dose of TAA (165 micrograms/day) significantly improved rhinitis symptoms relative to placebo: the total nasal symptom score and most individual symptom scores (eg, nasal stuffiness, itch, sneezing) were significantly better, duration of rhinitis symptoms (hours per day) was significantly reduced, and nasal airflow in a subset of patients showed significant improvement. The lower dose of TAA (82.5 micrograms/day) was superior to placebo by the same parameters as the higher dose, but this improvement was not as consistently significant as the higher dose. There were no clinically significant adverse events; nasal irritation and epistaxis were rare with a similar incidence among treatment groups. In conclusion, TAA at 165 micrograms/day was effective in controlling the symptoms of PAR and in improving nasal airflow in pediatric patients; the lower dose (82.5 micrograms/day) was marginally effective. Both doses were safe and well-tolerated in the children studied. [\hyperlink{Nystatin and Triamcinolone Acetonide}{PMID: 1958002}, M J Welch et al., 1991]

\hypertarget{pmid_23452680}{T}o examine the efficacy of N-acetylcysteine (NAC) for the treatment of pediatric trichotillomania (TTM) in a double-blind, placebo-controlled, add-on study. A total of 39 children and adolescents aged 8 to 17 years with pediatric trichotillomania were randomly assigned to receive NAC or matching placebo for 12 weeks. Our primary outcome was change in severity of hairpulling as measured by the Massachusetts General Hospital-Hairpulling Scale (MGH-HPS). Secondary measures assessed hairpulling severity, automatic versus focused pulling, clinician-rated improvement, and comorbid anxiety and depression. Outcomes were examined using linear mixed models to test the treatment×time interaction in an intention-to-treat population. No significant difference between N-acetylcysteine and placebo was found on any of the primary or secondary outcome measures. On several measures of hairpulling, subjects significantly improved with time regardless of treatment assignment. In the NAC group, 25\% of subjects were judged as treatment responders, compared to 21\% in the placebo group. We observed no benefit of NAC for the treatment of children with trichotillomania. Our findings stand in contrast to a previous, similarly designed trial in adults with TTM, which demonstrated a very large, statistically significant benefit of NAC. Based on the differing results of NAC in pediatric and adult TTM populations, the assumption that pharmacological interventions demonstrated to be effective in adults with TTM will be as effective in children, may be inaccurate. This trial highlights the importance of referring children with TTM to appropriate behavioral therapy before initiating pharmacological interventions, as behavioral therapy has demonstrated efficacy in both children and adults with trichotillomania. [\hyperlink{Nystatin and Triamcinolone Acetonide}{PMID: 23452680}, Michael H Bloch et al., 2013]

\hypertarget{pmid_29392100}{N}-acetylcysteine (NAC) is a well-known antidote for acetaminophen toxicity and is easily available over the counter. It has antioxidant and anti-inflammatory properties and an established tolerance and safety profile. Owing to its neuroprotective effects, its clinical use has recently expanded to include the treatment of different psychiatric and non-psychiatric disorders. Although a number of randomized controlled trials have documented the clinical evidence for NAC, there are no reviews that summarize the evidence. The present scoping review summarizes the study designs, the patient characteristics, the evidence and the limitations in randomized controlled trials designed to explore the efficacy of NAC for psychiatric conditions in the pediatric population. [\hyperlink{Nystatin and Triamcinolone Acetonide}{PMID: 29392100}, Sadiq Naveed et al., 2017]

\hypertarget{pmid_19441606}{I}ntranasal corticosteroids (INSs) are the most effective treatment for allergic rhinitis (AR). However, available INS safety and efficacy data in children younger than 6 years are limited. To report the first well-controlled study assessing the safety and efficacy of an INS in children aged 2 to 5 years with perennial AR. In a 4-week, multicenter, double-blind, parallel-group study, patients were randomized to receive triamcinolone acetonide aqueous nasal spray (TAA AQ), 110 microg once daily, or placebo. A subset of children continued into a 6-month, open-label phase. Efficacy end points included total nasal symptom scores. Safety measures included reports of adverse events, morning serum cortisol levels before and after cosyntropin infusion, and growth as measured using office stadiometry. A total of 474 patients were randomized to receive TAA AQ (n = 236) or placebo (n = 238); 436 entered the open-label extension phase. Adjusted mean (SE) changes from baseline during the double-blind period in instantaneous and reflective total nasal symptom scores were -2.28 (0.16) and -2.31 (0.15), respectively, in the TAA AQ group (P = .09) vs -1.92 (0.16) and -1.87 (0.15) in the placebo group (P = .03). Adverse event rates were comparable between treatment groups. There was no significant change from baseline in serum cortisol levels after cosyntropin infusion at study end. The distribution of children by stature-for-age percentile remained stable during the study. Use of TAA AQ, 110 microg once daily, for up to 6 months offers a favorable efficacy to safety ratio in children aged 2 to 5 years with perennial AR. [\hyperlink{Nystatin and Triamcinolone Acetonide}{PMID: 19441606}, Steven Weinstein et al., 2009]

\hypertarget{pmid_1546817}{A} 4-week, double-blind, parallel group study compared the safety and efficacy of once-a-day intranasal administration of triamcinolone acetonide (Nasacort) versus placebo in 304 patients (155 adult and 149 adolescent) with seasonal allergic rhinitis. Patients were randomized to receive triamcinolone acetonide (110, 220, or 440 microgram) or placebo once daily each morning. Daily rhinitis symptoms scores, weekly patient and physician global assessments, and weekly nasal eosinophil smears were obtained. In each triamcinolone acetonide group, significant (P less than .05) improvement over placebo was noted in the nasal index (sum of ratings for stuffiness, discharge, and sneezing) by week 1, the first point of analysis, and maintained throughout the study. Triamcinolone acetonide groups also demonstrated significant (P less than .05) improvement over placebo in all individual rhinitis symptoms evaluated. The greatest improvement in symptoms was observed at the 440 microgram dose. A significant decrease in eosinophil counts paralleled clinical improvement in all triamcinolone acetonide groups. Physicians and patients rated triamcinolone acetonide significantly (P less than .05) more effective than placebo. Responses of adult and adolescent patients were comparable. Adverse experiences, clinical laboratory values, and results of physical examinations were unremarkable and comparable between the triamcinolone acetonide and placebo groups. We conclude that triamcinolone acetonide is safe, well tolerated, and superior to placebo as a once-a-day treatment for seasonal allergic rhinitis. [\hyperlink{Nystatin and Triamcinolone Acetonide}{PMID: 1546817}, S Findlay et al., 1992]

\hypertarget{pmid_12506950}{O}ral thrush is a common condition in young infants. Nystatin treatment is associated with frequent recurrences and difficulty in administration. Fluconazole was compared with nystatin for the treatment of oral candidiasis in infants. Thirty-four infants were randomized to either nystatin oral suspension four times a day for 10 days or fluconazole suspension 3 mg/kg in a single daily dose for 7 days. Clinical cures for nystatin were 6 of 19 (32\%), and those for fluconazole were 15 of 15 (100\%), P < 0.0001. In this small pilot study fluconazole was shown to be superior to nystatin suspension for the treatment of oral thrush in otherwise healthy infants. [\hyperlink{Nystatin and Triamcinolone Acetonide}{PMID: 12506950}, R Alan Goins et al., 2002]

\hypertarget{pmid_16679073}{T}he non-steroidal anti-inflammatory drugs (NSAIDs) and acetaminophen (paracetamol) are the most common analgesic drugs used in neonates and infants despite limited pharmacodynamic data. Both drugs act through inhibition of cyclooxygenase enzymes. Neonatal acetaminophen clearance is reduced in premature neonates (0.7 L h(-1) x 70 kg(-1)) and increases to 5 L h(-1) x 70 kg(-1) at term (40\% adult rates); adult rates are reached within the first year of life; NSAID clearance follows similar trends. Volume of distribution is increased in the neonatal period. Dosing of both drug groups is tempered by concerns about toxicity. Acetaminophen hepatotoxicity is less common in neonates than in older children and adults, possibly due to reduced oxidative enzyme activity (e.g. CYP 2E1). Data concerning NSAID adverse effects in the neonatal period are few. Renal function is reduced 20\% after NSAID use for patent ductus arteriosus closure in premature neonates and there is no increased frequency of intraventricular haemorrhage. No significant difference in the change in cerebral blood volume, change in cerebral blood flow, or tissue oxygenation index was found between administration of ibuprofen or placebo in neonates. Future studies should define concentration-response relationships for these drugs that are age and pathology specific. [\hyperlink{Nystatin and Triamcinolone Acetonide}{PMID: 16679073}, Evelyne Jacqz-Aigrain et al., 2006]

\hypertarget{pmid_25754598}{T}his review evaluates the recent progress in clinical trials on oral triptans for acute migraine in children and adolescents. Randomized controlled trials (RCT) on the treatment of migraine in pediatric patients were rare and difficult to design. In particular, high placebo response in many of the trials made it difficult to prove efficacy of triptans. Using a "novel study design" for RCT, a study successfully proved the efficacy of an oral rizatriptan. This trial enrolled patients with unsatisfactory response to nonsteroidal anti-inflammatory or acetaminophen and with migraine lasting longer than 3 h. Rizatriptan was approved by Food and Drug Administration (FDA) (USA) for children and adolescents of 6-17 years. The triptan-NSAID combination drug for pediatric patients also showed efficacy. [\hyperlink{Nystatin and Triamcinolone Acetonide}{PMID: 25754598}, Fumihiko Sakai et al., 2015]

\hypertarget{pmid_28063133}{T}he antipyretic analgesics, paracetamol, and non-steroidal anti-inflammatory agents NSAIDs are one of the most widely used classes of medications in children. The aim of this review is to determine if there are any clinically relevant differences in safety between ibuprofen and paracetamol that may recommend one agent over the other in the management of fever and discomfort in children older than 3 months of age. [\hyperlink{Nystatin and Triamcinolone Acetonide}{PMID: 28063133}, Dipak J Kanabar et al., 2017]

\hypertarget{pmid_17022766}{T}he use of oral nystatin to prevent fungal colonisation and infection in neonates in the Neonatal Intensive Care Unit (NICU) is still an open question and not yet recommended as a standard of care. To determine whether prophylactic oral nystatin results in a decreased incidence of invasive candidiasis in the newborn infants, a total of 3991 infants were divided randomly into two groups. Group A infants (n = 1995), only those neonates who were identified as yeast carriers (oral moniliasis) were treated with oral nystatin. Group B infants, all neonates who were admitted to the unit received oral nystatin, was routinely administered three times a day. Group A was divided into groups A1 and A2 (who were treated only if identified as yeast carriers). Urine and rectal cultures were taken on admission and then weekly thereafter. There were 215 (14.2\%), 27 (5.6\%) and 36 (1.8\%) patients positive for invasive candidiasis in groups A1, A2 and B respectively. Oral nystatin prophylaxis significantly reduced the invasive candidiasis (P = 0.004) in extremely low-birth weight (ELBW) and very low-birth weight (VLBW) infants. Prophylactic administration of oral nystatine to the ELBW and VLBW infants results in a decreased risk of invasive candidiasis. [\hyperlink{Nystatin and Triamcinolone Acetonide}{PMID: 17022766}, Mehmet Adnan Ozturk et al., 2006]

\hypertarget{pmid_18939734}{G}uidelines recommend treatment with intranasal corticosteroids for patients with allergic rhinitis (AR), but concerns remain about possible adverse effects. To present the 1- and 2-year growth results for children with AR treated with triamcinolone acetonide aqueous nasal spray. Thirty-nine children (aged 6.1-14.3 years at study entry) were treated with triamcinolone acetonide aqueous for 1 year, and a subset of 30 children completed a second year of treatment. The dose was physician titered to achieve control over AR symptoms. For each child, statural heights at baseline and at the 1- and 2-year (where available) visits, together with growth rates, were measured and were compared with predicted values. There were no significant differences between measured and predicted heights at the 1- and 2-year visits. The mean (SD) measured--predicted difference was 0.3 (2.2) cm (95\% confidence interval, -0.4 to 1.0 cm) at the 1-year visit and 0.5 (3.0) cm (95\% confidence interval, -0.6 to 1.6 cm) at the 2-year visit. Mean differences in measured and predicted growth rates were nonsignificant at the 1- and 2-year visits. Triamcinolone acetonide aqueous titered to control AR symptoms and given for 1 or 2 years had no significant effect on statural growth in children with AR. [\hyperlink{Nystatin and Triamcinolone Acetonide}{PMID: 18939734}, David P Skoner et al., 2008]

\hypertarget{pmid_20553355}{D}iaper dermatitis (DD) is the most common type of irritative dermatitis in infancy. It is frequently complicated by Candida superinfection. Comparison of efficacy and safety of two antifungal pastes (Imazol = 1\% clotrimazole; Multilind = 100,000 IU nystatin/g + 20\% zinc oxide) in infants with DD. A total of 96 infants were included in this multi-centre, controlled, randomized, evaluator-blinded phase IV trial and treated with pastes containing either clotrimazole (n = 45) or nystatin (n = 46) twice daily for 14 days. In all, 91 children (age 12.1 +/- 5.3 months; 48 females) with DD were evaluable. Total symptom score after 7 days (TSS7) was assessed as primary parameter. Secondary efficacy parameters were TSS at 14 days (TSS14), clinical and microbiological cure rates and global assessment (GA) of clinical response. TSS improved markedly with both pastes. Decreases in symptom score were 4.5 +/- 2.1 (day 7) and 6.1 +/- 1.9 (day 14) with clotrimazole compared with 4.2 +/- 2.3 and 5.4 +/- 2.4 with nystatin (P < 0.0001). With respect to TSS14, clotrimazole was superior to nystatin (P = 0.0434). Clinical cure rate was higher with clotrimazole [36.2\% (day 7) and 68.1\% (day 14)] compared with 28.6\% and 46.9\% (nystatin). GA was very good in 26 (55.3\%) clotrimazole-treated children (nystatin: 16 [32.7\%], P = 0.0257). Frequency of adverse events was comparable in both treatment groups. Clotrimazole was superior to nystatin with respect to reduction in symptom score and GA. Microbiological cure rate was 100\% for both agents. Both treatments were safe and well-tolerated. [\hyperlink{Nystatin and Triamcinolone Acetonide}{PMID: 20553355}, P H Hoeger et al., 2010]

\hypertarget{pmid_36302965}{E}arly supports to enhance social development in children with autism are widely promoted. While oxytocin has a crucial role in mammalian social development, its potential role as a medication to enhance social development in humans remains unclear. We investigated the efficacy, tolerability, and safety of intranasal oxytocin in young children with autism using a double-blind, randomized, placebo-controlled, clinical trial, following a placebo lead-in phase. A total of 87 children (aged between 3 and 12 years) with autism received 16 International Units (IU) of oxytocin (n = 45) or placebo (n = 42) nasal spray, morning and night (32 IU per day) for twelve weeks, following a 3-week placebo lead-in phase. Overall, there was no effect of oxytocin treatment over time on the caregiver-rated Social Responsiveness Scale (SRS-2) (p = 0.686). However, a significant interaction with age (p = 0.028) showed that for younger children, aged 3-5 years, there was some indication of a treatment effect. Younger children who received oxytocin showed improvement on caregiver-rated social responsiveness ( SRS-2). There was no other evidence of benefit in the sample as a whole, or in the younger age group, on the clinician-rated Clinical Global Improvement Scale (CGI-S), or any secondary measure. Importantly, placebo effects in the lead-in phase were evident and there was support for washout of the placebo response in the randomised phase. Oxytocin was well tolerated, with more adverse side effects reported in the placebo group. This study suggests the need for further clinical trials to test the benefits of oxytocin treatment in younger populations with autism.Trial registration www.anzctr.org.au (ACTRN12617000441314). [\hyperlink{Nystatin and Triamcinolone Acetonide}{PMID: 36302965}, Adam J Guastella et al., 2023]

\hypertarget{pmid_8733987}{T}riamcinolone acetonide (TAA) aerosol nasal inhaler has been shown to effectively relieve the symptoms of seasonal allergic rhinitis in adults and adolescents. We conducted a study to evaluate the efficacy and safety of once-daily administration of TAA aerosol nasal inhaler in pediatric patients aged 6 to 11 years with grass seasonal allergic rhinitis. This multicenter, randomized, double-blind, placebo-controlled, parallel-group study enrolled 116 children who were treated with either TAA aerosol nasal inhaler (220 micrograms/d) or placebo once daily for 2 weeks. Patients evaluated the severity of rhinitis symptoms (nasal stuffiness, discharge, sneezing, and itching) daily according to a four-point scale (0 = absent, 1 = mild, 2 = moderate, and 3 = severe). Patients' and physicians' global evaluations of overall treatment efficacy were assessed at the end of the 2-week treatment period. Patients treated with TAA aerosol nasal inhaler had significantly greater reductions in all nasal symptom scores overall and in virtually all symptoms at the end of week 1 and week 2 compared with those in the placebo group. Both patients' and physicians' global evaluations of efficacy favored TAA aerosol nasal inhaler over placebo. This study demonstrated that once-daily administration of 220 micrograms of TAA aerosol nasal inhaler was well tolerated and effectively reduced the symptoms of seasonal allergic rhinitis in pediatric patients. [\hyperlink{Nystatin and Triamcinolone Acetonide}{PMID: 8733987}, C H Banov et al., ]

\hypertarget{pmid_9462438}{A}n open, prospective, randomized pilot study was performed to assess the efficacy and safety of oral fluconazole 3 mg/kg once daily compared with oral nystatin 50,000 units/kg/day in four divided doses in preventing candida infections in 50 children undergoing remission induction or consolidation therapy for cancer. In 21 of 25 fluconazole-treated and 20 of 25 nystatin-treated patients the overall outcome of prophylaxis was clearly successful. Mild and transient oropharyngeal candidosis was observed in two and three patients in the fluconazole and nystatin groups respectively. One patient randomized to fluconazole and two patients randomized to nystatin required empirical treatment with amphotericin B and one patient assigned to fluconazole developed tissue-proven candida colitis. Initially non-colonized patients remained yeast-free throughout treatment with no differences between the two study arms. Initially colonized patients stayed colonized throughout treatment although at the end of the study, more patients randomized to nystatin were still harbouring yeasts (P = 0.05). Almost exclusively, Candida albicans (95\%) was isolated. A change in species was observed in one patient in each arm of the study. Candida krusei or Candida glabrata were not encountered. Transient elevations of hepatic transaminases were more common in the fluconazole group, although not statistically significant (28\% vs 12\%, P = 0.15). Reversible grade I gastrointestinal and skin symptoms were observed in four patients randomized to fluconazole (16 vs 0\%, P < 0.05). Fluconazole was as safe and effective as nystatin in controlling yeast colonization and in preventing superficial and invasive candida infections and the empirical use of amphotericin B in children and adolescents undergoing intensive chemotherapy for cancer. [\hyperlink{Nystatin and Triamcinolone Acetonide}{PMID: 9462438}, A H Groll et al., 1997]

\hypertarget{pmid_6134564}{A}ll children aged under 15 years admitted to hospital in Newcastle upon Tyne between 1974 and 1981 with a diagnosis of poisoning were studied. After the introduction in 1976 of child resistant containers for salicylates and paracetamol, salicylate poisonings fell dramatically. The other most important medicines to cause poisoning in young children were tricyclic antidepressants, benzodiazapines, Lomotil (diphenoxylate and atropine), and iron preparations; these should also be packaged in child resistant containers by regulation. Few children had symptoms after poisoning with household products, but bleach, turpentine, and paraffin might also be packaged in child resistant containers. The numbers of adolescent girls admitted after deliberate self poisoning and of teenage boys admitted after ingestion of alcohol increased over the study period. [\hyperlink{Nystatin and Triamcinolone Acetonide}{PMID: 6134564}, G R Lawson et al., 1983]

\hypertarget{pmid_23078168}{P}aracetamol (acetaminophen) and ibuprofen are the most frequently purchased over-the-counter (OTC) medicines for children. Parents purchase these medicines for the treatment of fever and pain. In some countries other NSAIDs such as aspirin (acetylsalicylic acid) and dipyrone are available. We aimed to perform a narrative review of the efficacy and toxicity of OTC analgesic medicines for children in order to give guidance to health professionals and parents regarding the treatment of pain in a child. Neither aspirin nor dipyrone are recommended for OTC use because of the association with Reye's syndrome for the former and the risk of agranulocytosis for the latter. Both paracetamol and ibuprofen are effective for the treatment of mild pain in children. Adverse effects with both medicines are infrequent. Ibuprofen is an NSAID and therefore there is a greater risk of gastrointestinal adverse effects and hypersensitivity. Aspirin and dipyrone should be avoided. Paracetamol is the drug of first choice for mild pain in children because of its favourable safety profile. For the treatment of significant musculoskeletal pain, ibuprofen is the drug of first choice. [\hyperlink{Nystatin and Triamcinolone Acetonide}{PMID: 23078168}, Zeina Bárzaga Arencibia et al., 2012]

\hypertarget{pmid_23886027}{T}his study examined the efficacy and safety of N-acetylcysteine (NAC) augmentation for treating irritability in children and adolescents with autism spectrum disorders (ASD). Forty children and adolescents met diagnostic criteria for ASD according to DSM-IV. They were randomly allocated into one of the two groups of NAC (1200 mg/day)+risperidone or placebo+risperidone. NAC and placebo were administered in the form of effervescent and in two divided doses for 8 weeks. Irritability subscale score of Aberrant Behavior Checklist (ABC) was considered as the main outcome measure. Adverse effects were also checked. The mean score of irritability in the NAC+risperidone and placebo+risperidone groups at baseline was 13.2(5.3) and 16.7(7.8), respectively. The scores after 8 weeks were 9.7(4.1) and 15.1(7.8), respectively. Repeated measures of ANOVA showed that there was a significant difference between the two groups after 8 weeks. The most common adverse effects in the NAC+risperidone group were constipation (16.1\%), increased appetite (16.1\%), fatigue (12.9\%), nervousness (12.9\%), and daytime drowsiness (12.9\%). There was no fatal adverse effect. Risperidone plus NAC more than risperidone plus placebo decreased irritability in children and adolescents with ASD. Meanwhile, it did not change the core symptoms of autism. Adverse effects were not common and NAC was generally tolerated well. This trial was registered at http://www.irct.ir. The registration number of this trial was IRCT201106103930N6. [\hyperlink{Nystatin and Triamcinolone Acetonide}{PMID: 23886027}, Ahmad Ghanizadeh et al., 2013]

\section*{Trihexyphenidyl Hydrochloride}
\subsection*{Result}
\subsubsection*{Answer}

Ages 1-18 years (with dystonia): Yes  
Under 1 year: Unknown  
Children without dystonia: Unknown  

\subsubsection*{{Explanation}}
\hypertarget{Trihexyphenidyl Hydrochloride}
A review of the available abstracts reveals several studies specifically investigating the safety and tolerability of Trihexyphenidyl Hydrochloride in children, primarily for the treatment of dystonia associated with cerebral palsy.

1. Carranza-del Rio et al. (2011) [\hyperlink{pmid_21310336}{PMID: 21310336}] conducted a retrospective chart review of 101 children (ages 1-18 years, mean age 7 years 10 months) with cerebral palsy treated with trihexyphenidyl for dystonia or sialorrhea. The study found that 91\% tolerated the medication well, with a mean treatment duration of 3 years and 7 months. Side effects occurred in 69\% of subjects, mostly in those aged ≥7 years, but were generally manageable. 64\% continued treatment at study end, and 97 patients reported benefits. The study concludes that most children tolerated trihexyphenidyl well with gradual dose increases.

2. Sanger et al. (2007) [\hyperlink{pmid_17690057}{PMID: 17690057}] performed a prospective, open-label, multicenter pilot trial in 23 children aged 4 to 15 years with cerebral palsy and secondary dystonia. Trihexyphenidyl was titrated up to 0.75 mg/kg/day over 9 weeks, then tapered. Three children withdrew due to nonserious adverse events, and three required dose reduction. The study found significant improvement in arm function at 15 weeks, with close monitoring of safety and tolerability. The authors state that trihexyphenidyl "may be a safe and effective treatment for arm dystonia in some children with cerebral palsy," but note that children with hyperkinetic dystonia may worsen and call for larger, stratified trials.

3. Ben-Pazi et al. (2011) [\hyperlink{pmid_21498790}{PMID: 21498790}] retrospectively analyzed 31 children (mean age 8.2 years) with dystonia treated with high-dose trihexyphenidyl (>0.5 mg/kg/day). Most caregivers reported improvement, and side effects were generally transient, with one exception (hyperopia). The study concludes that trihexyphenidyl is effective, particularly in children without spasticity and with higher cognitive abilities.

4. Hoon et al. (2001) [\hyperlink{pmid_11483397}{PMID: 11483397}] retrospectively surveyed 22 children with extrapyramidal cerebral palsy treated with trihexyphenidyl. Improvements were reported in upper extremity function, expressive language, and drooling, with younger children more likely to respond. The authors recommend a prospective masked study for confirmation.

5. Pidcock et al. (1999) [\hyperlink{pmid_10207932}{PMID: 10207932}] reported a case study of an 8-year-old girl with dystonia treated with trihexyphenidyl, noting improvements in motor and oral skills and no adverse side effects.

Summary:
- The available evidence comes from retrospective reviews, a prospective pilot trial, and case reports/series, all focused on children with dystonia (primarily due to cerebral palsy), with age ranges from 1 to 18 years.
- Across these studies, trihexyphenidyl was generally well tolerated, with side effects mostly transient or manageable, and no reports of life-threatening adverse events.
- There is no evidence from these abstracts of targeted safety studies in children outside the context of dystonia/cerebral palsy, nor in infants under 1 year.
- The studies are not large randomized controlled trials, but they do provide targeted pediatric safety data for children aged 1-18 years with dystonia.

Therefore, based on the abstracts, trihexyphenidyl hydrochloride appears to be safe for use in children aged 1-18 years with dystonia, when used with gradual dose titration and monitoring. Safety in infants under 1 year, or in children without dystonia, is not established in these abstracts.

\subsection*{Abstracts}
\hypertarget{pmid_21310336}{A} paucity of information exists regarding medications to treat dystonia in children with cerebral palsy. This study sought to review the benefits and tolerability of trihexyphenidyl in children with cerebral palsy, treated for dystonia or sialorrhea or both in a pediatric tertiary care hospital, through a retrospective chart review. In total, 101 patients (61 boys and 40 girls) were evaluated. The mean age at drug initiation was 7 years and 10 months (range, 1-18 years). The mean initial dose was 0.095 mg/kg/day. The dose was increased by 10-20\% no sooner than every 2 weeks. The mean maximum dose reached was 0.55 mg/kg/day. Ninety-three patients (91\%) tolerated the medication well, with a mean duration of treatment of 3 years and 7 months. Side effects occurred in 69\% of subjects, the majority in patients aged ≥7 years, and soon after treatment initiation. Sixty-four percent continued the treatment at study end. Ninety-seven patients reported benefits, including reduction of dystonia in upper (59.4\%) and lower (37.6\%) extremities, sialorrhea (60.4\%), and speech issues (24.7\%). The majority of patients tolerated trihexyphenidyl well on a schedule of gradual dose increases, and almost all demonstrated improvements in dystonia or sialorrhea or both. [\hyperlink{Trihexyphenidyl Hydrochloride}{PMID: 21310336}, Jorge Carranza-del Rio et al., 2011]

\hypertarget{pmid_17690057}{A}lthough trihexyphenidyl is used clinically to treat both primary and secondary dystonia in children, limited evidence exists to support its effectiveness, particularly in dystonia secondary to disorders such as cerebral palsy. A prospective, open-label, multicenter pilot trial of high-dose trihexyphenidyl was conducted in 23 children aged 4 to 15 years with cerebral palsy judged to have secondary dystonia impairing function in the dominant upper extremity. All children were given trihexyphenidyl at increasing doses over a 9-week period up to a maximum of 0.75 mg/kg/d. Trihexyphenidyl was subsequently tapered off over the next 5 weeks. Objective motor assessments were performed at baseline, 9 weeks, and 15 weeks. The primary outcome measure was the Melbourne Assessment of Unilateral Upper Limb Function, tested in the dominant arm. Tolerability and safety were monitored closely throughout the trial. Of the 31 children who agreed to participate in the study, 5 failed to meet entry criteria and 3 withdrew due to nonserious adverse events (chorea, drug rash, and hyperactivity). Three children required a dosage reduction because of nonserious adverse events but continued to participate. The 23 children who completed the study showed a significant improvement in arm function at 15 weeks (P = .045) but not at 9 weeks (P = .985). Post hoc analysis showed that a subgroup (n = 10) with hyperkinetic dystonia (excess involuntary movements) worsened at 9 weeks (P = .04) but subsequently returned to baseline following taper of the medicine. The authors conclude that scientific evidence for the clinical use of trihexyphenidyl in cerebral palsy remains equivocal. Trihexyphenidyl may be a safe and effective for treatment for arm dystonia in some children with cerebral palsy if given sufficient time to respond to the medication. Post hoc analyses based on the type of movement disorder suggested that children with hyperkinetic forms of dystonia may worsen. A larger, randomized prospective trial stratified by the presence or absence of hyperkinetic movements is needed to confirm these results. [\hyperlink{Trihexyphenidyl Hydrochloride}{PMID: 17690057}, Terence D Sanger et al., 2007]

\hypertarget{pmid_18219837}{A}ntihistamines are an established first-line treatment for allergic rhinitis and are widely prescribed in infants for allergic symptoms. To establish the safety and tolerability of fexofenadine hydrochloride in children aged 6 months to 2 years in 2 studies (T/3001 and T/3002). Both studies had a multicenter, randomized, placebo-controlled design. Mean treatment duration was 8 days. Subjects were randomized (T/3001, n = 174; and T/3002, n = 219) to twice-daily fexofenadine hydrochloride, 15 or 30 mg, or placebo mixed with a standard vehicle. In the combined population, the incidence of treatment-emergent adverse events (TEAEs) was comparable between groups (placebo, 48.2\% [96/199]; fexofenadine hydrochloride, 15 mg, 40.0\% [34/85]; and fexofenadine hydrochloride, 30 mg, 35.2\% [38/108]). Vomiting was the most common TEAE (placebo, 13.6\%; fexofenadine hydrochloride, 15 mg, 14.1\%; and fexofenadine hydrochloride, 30 mg, 5.6\%). Most TEAEs were unrelated to study medication, as evaluated by investigators; those possibly related to study medication were mild or moderate in intensity. No clinical differences were seen between fexofenadine and placebo for vital signs, electrocardiographic results, or physical examination results. Fexofenadine hydrochloride, 15 or 30 mg, given for a mean duration of 8 days is well tolerated, with a good safety profile, in children aged 6 months to 2 years. [\hyperlink{Trihexyphenidyl Hydrochloride}{PMID: 18219837}, Frank C Hampel et al., 2007]

\hypertarget{pmid_18702885}{A}llergic rhinitis (AR) is a common chronic condition in children and may impact a child's quality of life. Increasing treatment compliance may improve quality of life. An oral suspension of fexofenadine hydrochloride (HCl) has been developed to ease administration to children and may, therefore, improve treatment compliance. The purpose of this study was to assess the pharmacokinetic behavior, safety, and tolerability of a single dose of fexofenadine HCl oral suspension administered to children aged 2-5 years with allergic rhinitis. Children (aged 2-5 years) with AR were recruited in a multicenter, open-label, single-dose study. Fexofenadine HCl (30 mg) was administered as a 6-mg/mL suspension (5 mL). Plasma samples were collected up to 24 hours postdose. Adverse events (AEs); electrocardiograms (ECGs); vital signs; and clinical laboratory tests for hematology, blood chemistry, and urinalysis were analyzed to evaluate safety and tolerability. Fifty subjects completed the study. Mean maximum plasma concentration of fexofenadine was 224 ng/mL, and mean area under the plasma concentration curve was 898 ng . hour/mL. Treatment-emergent AEs were mild in intensity and reported in a total of seven subjects. No trends or clinically meaningful changes in mean ECG, vital sign, or clinical laboratory test data occurred during the study. In children aged 2-5 years, the exposure after a 30-mg dose of fexofenadine HCl suspension was similar to the exposures previously seen after a 30- and 60-mg dose of fexofenadine HCl in children aged 6-11 years and in adults, respectively. The suspension was also well tolerated. [\hyperlink{Trihexyphenidyl Hydrochloride}{PMID: 18702885}, Nathan Segall et al., ]

\hypertarget{pmid_21498790}{T}here are conflicting reports regarding the efficacy of trihexyphenidyl, an anticholinergic drug, for treatment of dystonia in cerebral palsy. The author hypothesized that trihexyphenidyl may be more effective in specific subgroups and performed a retrospective analysis of 31 children (8.2 ± 5.8 years) with dystonia following treatment with high-dose trihexyphenidyl (>0.5 mg/kg/day). Main outcome measure was extent of motor improvement calculated according to the body areas affected. Most (21/31) caregivers reported improvement in 1 or more areas, mainly arm, hand, and oromotor function. Improvement was greater in children without spasticity (P = .02) and in those with higher cognitive function (P = .02). While a third of caregivers (10/31) reported tone reduction, and half (15/31) noted overall functional improvement. Side effects were transient, with the exception of hyperopia (n = 1), and occurred less frequently in children with a history of prematurity (P = .02). In summary, trihexyphenidyl is effective particularly in absence of spasticity and in children with higher cognitive abilities. [\hyperlink{Trihexyphenidyl Hydrochloride}{PMID: 21498790}, Hilla Ben-Pazi et al., 2011]

\hypertarget{pmid_17941284}{T}he safety of fexofenadine has been examined extensively in adults and school-age children. However, the safety of fexofenadine in children younger than 6 years has not been reported to date. To compare the safety and tolerability of twice-daily fexofenadine hydrochloride, 30 mg, and placebo in preschool children aged 2 to 5 years with allergic rhinitis. This was a multicenter, double-blind, randomized, placebo-controlled, parallel-group study, conducted between February 29, 2000, and June 14, 2001. Participants were randomized to either fexofenadine hydrochloride, 30 mg, or placebo twice daily for a 2-week period. To facilitate dosing, capsule content was mixed with applesauce (approximately 10 mL). Safety assessments depended on date of entry into the study because of an amendment to the protocol. Before the amendment, assessments included physical examination, vital signs reporting (oral temperature, heart rate, and respiratory rate), and adverse event (AE) reporting. After the amendment, safety assessments included laboratory testing (blood chemistry and hematology profiles), physical examination, 12-lead electrocardiography, and vital signs (oral temperature, blood pressure, heart rate, and respiratory rate) and AE reporting. Treatment-emergent AEs were observed in 116 of 231 participants receiving placebo and 111 of 222 receiving fexofenadine. These AEs were possibly related to study medication in 19 (8.2\%) and 21 (9.5\%) of the participants receiving placebo and fexofenadine, respectively, and most frequently involved the digestive system. No clinically relevant differences in laboratory measures, vital signs, and physical examinations were observed. The findings show that fexofenadine hydrochloride, 30 mg, is well tolerated and has a good safety profile in children aged 2 to 5 years with allergic rhinitis. [\hyperlink{Trihexyphenidyl Hydrochloride}{PMID: 17941284}, Henry Milgrom et al., 2007]

\hypertarget{pmid_21540483}{T}richlorophenols (TCPs) are organochlorine compounds which are ubiquitous in the environment and well known for their carcinogenic effects. However, little is known about their neurotoxicity in humans. Our goal was to examine the association between body burden of TCPs (ie, 2,4,5-TCP and 2,4,6-TCP) and attention deficit hyperactivity disorder (ADHD). We calculated ORs and 95\% CIs from logistic regression analyses using data from the 1999-2004 National Health and Nutrition Examination Survey (NHANES) to evaluate the association between urinary TCPs and parent-reported ADHD among 2546 children aged 6-15 years. Children with low levels (<3.58 μg/g) and high levels (≥3.58 μg/g) of urinary 2,4,6-TCP had a higher risk of parent-reported ADHD compared to children with levels below the limit of detection (OR 1.54, 95\% CI 0.97 to 2.43 and OR 1.77, 95\% CI 1.18 to 2.66, respectively; p for trend=0.006) after adjusting for covariates. No association was found between urinary 2,4,5-TCP and parent-reported ADHD. Exposure to TCP may increase the risk of behavioural impairment in children. The potential neurotoxicity of these chemicals should be considered in public health efforts to reduce environmental exposures/contamination, especially in countries where organochlorine pesticides are still commonly used. [\hyperlink{Trihexyphenidyl Hydrochloride}{PMID: 21540483}, Xiaohui Xu et al., 2011]

\hypertarget{pmid_2672786}{T}his study assessed the safety and efficacy of methylphenidate in children with seizures and attention-deficit disorder. Ten children, aged 6 years 10 months to 10 years 10 months, without seizures while receiving a single antiepileptic drug, were evaluated in a double-blind medication-placebo crossover study with methylphenidate hydrochloride was administered at 0.3 mg/kg per dose and given at 8 AM and 12 PM on school days only. The use of methylphenidate was associated with statistically significant improvements on the Conners' Teacher Rating Scale and on the Finger Tapping Task and with trends toward improvement on the Matching Familiar Figures Test and Discriminant Reaction Time tests. No child had seizures during the study period nor subsequently for those who continued receiving psychostimulants. There were no significant changes of epileptiform features or back-ground activity on electroencephalograms and no alterations in antiepileptic drug levels. Methylphenidate may be a safe and effective treatment for certain children with seizures and concurrent attention-deficit disorder. [\hyperlink{Trihexyphenidyl Hydrochloride}{PMID: 2672786}, H Feldman et al., 1989]

\hypertarget{pmid_23129068}{H}ydroxyurea (HU) is highly effective treatment for sickle cell disease (SCD). While pediatric use of HU is accepted clinical practice, barriers to use may impede its potential benefit. A survey of parents of children ages 5-17 years with SCD was performed across five institutions to assess factors associated with HU use. Of the 173 parent responses, 65 (38\%) had children currently taking HU. Among parents of children not taking HU, the most commonly cited reasons were that their hematology provider had not offered it, their child was not sufficiently symptomatic and concerns about potential side effects. Even parents of HU users reported widespread concern about effectiveness, long-term safety, and off-label use. In bivariate analyses, children's ages, parental demographics such as education level, or travel time to their hematology provider were not correlated with HU use. Bivariate analysis and multivariate logistic regression revealed three significant factors associated with current HU use: better parental knowledge about its major therapeutic effects (P < 0.001), sickle genotype (P = 0.005), and institution of clinical care (P = 0.04). Pervasive concerns about HU safety exist, even among parents of current users. Varying knowledge among parents appears to be independent of their demographics, and is associated with HU use. Inter-institutional variability in parental knowledge and drug uptake highlights potentially potent site-specific influences on likelihood of HU use. Overall, these survey data underscore the need for strategies to bolster parental understanding about benefits of HU and address concerns about its safety. [\hyperlink{Trihexyphenidyl Hydrochloride}{PMID: 23129068}, Suzette O Oyeku et al., 2013]

\hypertarget{pmid_31292919}{T}riclofos sodium (TFS) has been used for many years in children as a sedative for painless medical procedures. It is physiologically and pharmacologically similar to chloral hydrate, which has been censured for use in children with neurocognitive disorders. The aim of this study was to investigate the safety and efficacy of TFS sedation in a pediatric population with a high rate of neurocognitive disability. The database of the neurodiagnostic institute of a tertiary academic pediatric medical center was retrospectively reviewed for all children who underwent sedation with TFS in 2014. Data were collected on demographics, comorbidities, neurologic symptoms, sedation-related variables, and outcome. The study population consisted of 869 children (58.2\% male) of median age 25 months (range 5-200 months); 364 (41.2\%) had neurocognitive diagnoses, mainly seizures/epilepsy, hypotonia, or developmental delay. TFS was used for routine electroencephalography in 486 (53.8\%) patients and audiometry in 401 (46.2\%). Mean (± SD) dose of TFS was 50.2 ± 4.9 mg/kg. Median time to sedation was 45 min (range 5-245), and median duration of sedation was 35 min (range 5-190). Adequate sedation depth was achieved in 769 cases (88.5\%). Rates of sedation-related adverse events were low: apnea, 0; desaturation ≤ 90\%, 0.2\% (two patients); and emesis, 0.35\% (three patients). None of the children had hemodynamic instability or signs of poor perfusion. There was no association between desaturations and the presence of hypotonia or developmental delay. TFS, when administered in a controlled and monitored environment, may be safe for use in children, including those with underlying neurocognitive disorders. [\hyperlink{Trihexyphenidyl Hydrochloride}{PMID: 31292919}, Eytan Kaplan et al., 2019]

\hypertarget{pmid_10477679}{P}revious studies have determined the short-term toxicity profile, laboratory changes, and clinical efficacy associated with hydroxyurea (HU) therapy in adults with sickle cell anemia. The safety and efficacy of this agent in pediatric patients with sickle cell anemia has not been determined. Children with sickle cell anemia, age 5 to 15 years, were eligible for this multicenter Phase I/II trial. HU was started at 15 mg/kg/d and escalated to 30 mg/kg/d unless the patient experienced laboratory toxicity. Patients were monitored by 2-week visits to assess compliance, toxicity, clinical adverse events, growth parameters, and laboratory efficacy associated with HU treatment. Eighty-four children were enrolled between December 1994 and March 1996. Sixty-eight children reached maximum tolerated dose (MTD) and 52 were treated at MTD for 1 year. Significant hematologic changes included increases in hemoglobin concentration, mean corpuscular volume, mean corpuscular hemoglobin, and fetal hemoglobin parameters, and decreases in white blood cell, neutrophil, platelet, and reticulocyte counts. Laboratory toxicities typically were mild, transient, and were reversible upon temporary discontinuation of HU. No life-threatening clinical adverse events occurred and no child experienced growth failure. This Phase I/II trial shows that HU therapy is safe for children with sickle cell anemia when treatment was directed by a pediatric hematologist. HU in children induces similar laboratory changes as in adults. Phase III trials to determine if HU can prevent chronic organ damage in children with sickle cell anemia are warranted. [\hyperlink{Trihexyphenidyl Hydrochloride}{PMID: 10477679}, T R Kinney et al., 1999]

\hypertarget{pmid_17063023}{E}vidence on the caries-preventive effect of chlorhexidine (CHX) among high-risk children is inconclusive, possibly because obscured by fluoride exposure. We investigated the effect of CHX among initially 3-year-old subjects whose baseline d(3)ft was = 0 and whose only regular fluoride exposure came from toothpaste. The subjects were assigned to three groups: high-risk test (HRT, n = 70), high-risk control (HRC, n = 71), and low-risk control (LRC, n = 70). Risk classification was based on salivary mutans streptococcal levels (MS, </>or=1.0 x 10(5) cfu/ml). Basic measures (oral hygiene, dietary counselling every 4 months) were given to all groups. HRT also underwent CHX gel applications for 3 consecutive days at 3-month intervals for 15 months. Eighteen months after baseline d(3)ft increments and proportions of children with d(3)ft increment >or=1 (\%d(3)ft increment >or=1) among all groups were assessed. Anti-MS effect on high-risk children and caries-preventive effect on all children were statistically analysed by residual change analysis (MS), non-parametric tests and logistic regression analysis (caries). No differences were found between the groups in basic programme compliance. CHX significantly reduced MS levels. \%d(3)ft increment >or=1 and mean d(3)ft increments were 34.3\%, 0.56 (HRT), 32.4\%, 0.54 (HRC) and 11.4\%, 0.11 (LRC), with HRT/HRC values statistically significantly higher than LRC values and no significant difference between HRT and HRC. HRT children were not less likely to show new lesions than HRC children (OR = 1.09; 95\% confidence interval 0.54-2.19), while high-risk children were 4 times more likely to show new lesions than low-risk children (OR = 3.71; 95\% confidence interval 1.53-9.03). CHX gel applications showed moderate anti-MS effect but negligible caries-preventive effect. [\hyperlink{Trihexyphenidyl Hydrochloride}{PMID: 17063023}, S Petti et al., 2006]

\hypertarget{pmid_7771914}{T}he findings from case reports and patient questionnaire surveys have been interpreted as indicating that administration of stimulants is ill-advised for the treatment of attention-deficit hyperactivity disorder in children with tic disorder. Thirty-four prepubertal children with attention-deficit hyperactivity disorder and tic disorder received placebo and three dosages of methylphenidate hydrochloride (0.1, 0.3, and 0.5 mg/kg) twice daily for 2 weeks each, under double-blind conditions. Treatment effects were assessed using direct observations of child behavior in a simulated (clinic-based) classroom and using rating scales completed by the parents, teachers, and physician. Methylphenidate effectively suppressed hyperactive, disruptive, and aggressive behavior. There was no evidence that methylphenidate altered the severity of tic disorder, but it may have a weak effect on the frequency of motor (increase) and vocal (decrease) tics. Methylphenidate appears to be a safe and effective treatment for attention-deficit hyperactivity disorder in the majority of children with comorbid tic disorder. [\hyperlink{Trihexyphenidyl Hydrochloride}{PMID: 7771914}, K D Gadow et al., 1995]

\hypertarget{pmid_19747907}{C}hloral hydrate is used worldwide as a first-line agent for procedural sedation in paediatric patients undergoing painless diagnostic investigations. Chloral hydrate overdoses in children and adults have been reported to cause various toxicities, including central nervous system, respiratory and cardiac depression with sometimes fatal outcome. A 3-month-old girl was admitted after an unintentional administration of a 10-fold dose of chloral hydrate (667 mg/kg). She showed respiratory insufficiency in need of intubation and ventilation. Gastric endoscopy revealed esophagitis and gastric ulcerations. To assess the need for hemodialysis, serum trichloroethanol (TCE) was determined using a mass spectrometric quantification after a methyl tertiary butyl ether extraction using an external standard method. The serum TCE level 6 h after administration was 89 mg/L and declined to 20 mg/L within 24 h. The child could be extubated the next day; her further course was uneventful. The repeated determination of serum TCE levels prevented a technically difficult and risky hemodialysis in this very young patient. [\hyperlink{Trihexyphenidyl Hydrochloride}{PMID: 19747907}, Sultan Dogan-Duyar et al., 2010]

\hypertarget{pmid_10207932}{T}rihexyphenidyl has been found to be an effective treatment for dystonic movement disorders, improving gross motor function in patients with axial and torsional dystonia, tremors, and myoclonus. In this report, improvements in fine motor control, language, and oral motor skills are described with trihexyphenidyl in an 8-year-old female who developed dystonia after spontaneous bilateral putamenal hemorrhages. No adverse side effects occurred. The mechanism of action of trihexyphenidyl is believed to be in the basal ganglia where it inhibits muscarinic cholinergic receptors and increases the turnover of dopamine. [\hyperlink{Trihexyphenidyl Hydrochloride}{PMID: 10207932}, F S Pidcock et al., 1999]

\hypertarget{pmid_11483397}{T}rihexyphenidyl (Artane) is a centrally active muscarinic antagonist commonly used to treat patients with generalized dystonia. In a retrospective survey of 22 consecutive children with extrapyramidal cerebral palsy, we evaluated trihexyphenidyl on upper extremity and lower extremity function, expressive language, and drooling. Functional changes were assessed using a parental questionnaire (rating scale 1-5: from 1 = little or no change to 5 = tremendous change, with scores in either a positive or negative direction). Improvements of +4 or +5 were reported in eight children for upper extremity function, in eight children for verbal expressive language, in five for drooling, and in none for lower extremity function. Using bivariate linear regression modeling to investigate variables associated with treatment effects, there was a significant inverse relationship between age at initiation of medication and therapeutic response. Furthermore, beneficial responses were specific to upper-extremity function and expressive language. These results suggest that younger children are more likely to respond to trihexyphenidyl and that primary functional benefits include improved fine motor abilities and expressive language. A prospective masked study with a standardized clinical instrument is needed to confirm these findings. [\hyperlink{Trihexyphenidyl Hydrochloride}{PMID: 11483397}, A H Hoon et al., 2001]

\hypertarget{pmid_4938432}{O}ne hundred and three children with proved typhoid fever were treated with trimethoprim-sulphamethoxazole, and the results compared with those of a further 40 children treated with chloramphenicol. The bacteriological response to trimethoprim-sulphamethoxazole was unsatisfactory. From this study it seems that at present chloramphenicol is still the treatment of choice for typhoid fever. In view of the haematological changes occurring during therapy with trimethoprim-sulphamethoxazole caution is necessary and monitoring of the blood picture advisable, even at the recommended dose. [\hyperlink{Trihexyphenidyl Hydrochloride}{PMID: 4938432}, J N Scragg et al., 1971]

\hypertarget{pmid_31157521}{H}ydroxyurea (HU) increases fetal hemoglobin (HgbF) and ameliorates sickle cell disease (SCD) symptoms. Studies have demonstrated the safety and efficacy of HU in infants and children. Initiation of HU in infancy for children with SCD needs to be implemented in community practice. Starting in 2011, the Pediatric Sickle Cell Program of Northern Virginia initiated HU in infants with SCD. A prospective longitudinal database tracked the clinical course and outcomes. Twenty-four children with HgbSS who started HU by age 1 were continuously followed for a total of 95 person-years. Age at the time of analysis ranged from 2 to 7 years. Average hemoglobin at 6-month intervals ranged from 9.5 + 1.9 to 10.7 + 0.8 g/dL, and average HgbF ranged from 27.8 + 5.0\% to 34.1 + 6.6\%. Twenty-seven hospitalizations occurred (0.28/person-year), all before age 3, including 19 (70\%) for fever or infection, five (19\%) for splenic sequestration, and one (4\%) for pain in an infant prior to starting HU. The treat-and-release emergency department visits totaled 68 (0.72/person-year), including 62 visits (91\%) for fever, infection, or viral illness, and two visits (3\%) for pain/dactylitis in infants before HU initiation. Splenic sequestration accounted for all five transfusions. No pain episodes requiring medical attention were documented after HU initiation. No complicated acute chest syndrome, no abnormal or conditional transcranial Doppler ultrasound, and no overt strokes occurred. Implementation of HU in infancy for patients with SCD in community practice is feasible and is highly effective in preventing disease complications. [\hyperlink{Trihexyphenidyl Hydrochloride}{PMID: 31157521}, Ronay Thomas et al., 2019]

\hypertarget{pmid_15247700}{M}any children with urological disease require long-term treatment with antibiotics. In many cases the choice of medical instead of surgical management hinges on the implied safety of certain drugs. Recently some groups have advocated subureteral injection procedures to avoid long-term antibiotics for low grade reflux. We present a concise and relevant review on the use and adverse reactions of nitrofurantoin, trimethoprim and sulfamethoxazole in children. We reviewed the literature regarding the safety and toxicity of these drugs. Information regarding absorption, excretion and dosing was also gathered to explain better the mechanisms of toxicity. Adverse reactions in children reported in the literature related to nitrofurantoin are gastrointestinal disturbance (4.4/100 person-years at risk), cutaneous reactions (2\% to 3\%), pulmonary toxicity (9 patients), hepatoxicity (12 patients and 3 deaths), hematological toxicity (12 patients), neurotoxicity and an increased rate of sister chromatid exchanges. Adverse reactions in children related to trimethoprim/sulfamethoxazole are almost exclusively due to the sulfamethoxazole component, including cutaneous reactions (1.4 to 7.4 events per 100 person-years at risk), hematological toxicity (0\% to 72\% of patients) and hepatotoxicity (5 patients). The majority of adverse reactions were found in children on full dose therapy and not prophylaxis. The use of nitrofurantoin, trimethoprim and sulfamethoxazole is safe in children for long-term prophylactic therapy. The antibiotic safety issue should not be misconstrued as an argument for surgical therapy, whether minimally invasive or not. Adverse reactions exist to these medicines but they are less common than seen in adults, presumably because of the lower dose used for therapy, and the lack of significant comorbidities and drug interactions in children. Serious side effects are extremely rare and most are reversible by discontinuing therapy. The extremely low potential for significant adverse reactions should be discussed with parents. [\hyperlink{Trihexyphenidyl Hydrochloride}{PMID: 15247700}, Edward Karpman et al., 2004]

\hypertarget{pmid_33706380}{H}ydroxyurea (HU) is used in children with sickle cell disease (SCD) to increase fetal hemoglobin (HF), contributing to a decrease in physical symptoms and potential protection against cerebral microvasculopathy. There has been minimal investigation into the association between HU use and cognition in this population. This study examined the relationship between HU status and cognition in children with SCD. Thirty-seven children with SCD HbSS or HbS/β0 thalassaemia (sickle cell anemia; SCA) ages 4:0-11 years with no history of overt stroke or chronic transfusion completed a neuropsychological test battery. Other medical, laboratory, and demographic data were obtained. Neuropsychological function across 3 domains (verbal, nonverbal, and attention/executive) was compared for children on HU (n = 9) to those not taking HU (n = 28). Children on HU performed significantly better than children not taking HU on standardized measures of attention/executive functioning and nonverbal skills. Performance on verbal measures was similar between groups. These results suggest that treatment with HU may not only reduce physical symptoms, but may also provide potential benefit to cognition in children with SCA, particularly in regard to attention/executive functioning and nonverbal skills. Replication with larger samples and longitudinal studies are warranted. [\hyperlink{Trihexyphenidyl Hydrochloride}{PMID: 33706380}, Reem A Tarazi et al., 2021]

\hypertarget{pmid_36226857}{S}ickle cell disease (SCD) is a disease of abnormal hemoglobin associated with severe clinical phenotype and recurrent complications. Hydroxyurea (HU) is one of the US-FDA approved and commonly used drug for the treatment of adult SCD patients with clinical -severity. However, its use in the pediatric groups remains atypical. Despite a high prevalence of the disease in the state Chhattisgarh, there is a lack of evidence supporting its use in pediatric patients. This study aimed to evaluate the pharmacological and clinical efficacy and safety of HU in a large pediatric cohort with SCD from Central India. The study cohort consisted of 164 SCD (138 Hb SS and 26 Hb S beta-thalassemia) children (≤14 years of age) on HU therapy, who were monitored for toxicity, hematological and clinical efficacy at baseline (Pre-HU) and after 24 months (Post-HU). The results highlight the beneficial effects of HU at a mean dose of 18.7 ± 7.0 mg/kg/day. A significant improvement was observed, not only in physical and clinical parameters but also in hematological parameters which include fetal hemoglobin (Hb F), total hemoglobin, hematocrit, mean corpuscular volume (MCV) and mean corpuscular hemoglobin (MCH) levels, when evaluated against the baseline. We did not observe any significant adverse effects during the treatment period. Similar results were obtained on independent analysis of Hb SS and Hb Sβ patients. These findings strengthen the beneficial effect of hydroxyurea in pediatric population also without any serious adverse effects and builds up ground for expanding its use under regular monitoring. [\hyperlink{Trihexyphenidyl Hydrochloride}{PMID: 36226857}, Harsha Lad et al., 2023]

\hypertarget{pmid_19047254}{H}ydroxyurea is the only approved medication for the treatment of sickle cell disease in adults; there are no approved drugs for children. Our goal was to synthesize the published literature on the efficacy, effectiveness, and toxicity of hydroxyurea in children with sickle cell disease. Medline, Embase, TOXLine, and the Cumulative Index to Nursing and Allied Health Literature through June 2007 were used as data sources. We selected randomized trials, observational studies, and case reports (English language only) that evaluated the efficacy and toxicity of hydroxyurea in children with sickle cell disease. Two reviewers abstracted data sequentially on study design, patient characteristics, and outcomes and assessed study quality independently. We included 26 articles describing 1 randomized, controlled trial, 22 observational studies (11 with overlapping participants), and 3 case reports. Almost all study participants had sickle cell anemia. Fetal hemoglobin levels increased from 5\%-10\% to 15\%-20\% on hydroxyurea. Hemoglobin concentration increased modestly (approximately 1 g/L) but significantly across studies. The rate of hospitalization decreased in the single randomized, controlled trial and 5 observational studies by 56\% to 87\%, whereas the frequency of pain crisis decreased in 3 of 4 pediatric studies. New and recurrent neurologic events were decreased in 3 observational studies of hydroxyurea compared with historical controls. Common adverse events were reversible mild-to-moderate neutropenia, mild thrombocytopenia, severe anemia, rash or nail changes (10\%), and headache (5\%). Severe adverse events were rare and not clearly attributable to hydroxyurea. Hydroxyurea reduces hospitalization and increases total and fetal hemoglobin levels in children with severe sickle cell anemia. There was inadequate evidence to assess the efficacy of hydroxyurea in other groups. The small number of children in long-term studies limits conclusions about late toxicities. [\hyperlink{Trihexyphenidyl Hydrochloride}{PMID: 19047254}, John J Strouse et al., 2008]

\hypertarget{pmid_7767425}{T}o describe the epidemiologic findings associated with the use of methylphenidate hydrochloride among children aged 0 to 19 years in Michigan. A population-based data set of all prescriptions filed with the Michigan Triplicate Prescription Program during February and March 1992 was analyzed, maintaining complete anonymity. State of Michigan. All patients receiving a prescription for methylphenidate who are residents of Michigan, and all physicians prescribing methylphenidate. None. Descriptive data. Eleven of 1000 Michigan residents between the ages of 0 and 19 years received a prescription for methylphenidate during the study period. Eighty-four percent were boys. Boys aged 10 or 11 years received more prescriptions for methylphenidate than any other age group--43 per 1000. The number of children receiving prescriptions for methylphenidate ranged from 2.5 to 28 per 1000. The range for boys aged 10 or 11 years was from 9.6 to 117 per 1000. Primary care physicians wrote 84\% of all prescriptions; pediatricians wrote 59\% of the prescriptions for patients younger than 20 years old. Half of the prescriptions written by pediatricians were written by 5\% of the pediatricians in the state. Michigan has been among the states with the highest per capita consumption of methylphenidate for the past 10 years. The major use of methylphenidate is for treatment of attention deficit hyperactivity disorder. The number of boys in Michigan aged 10 or 11 years who were treated with methylphenidate was similar to the national prevalence of the disorder, 3\% to 5\%. A tenfold variation was noted in the percentage of children medicated when the data were analyzed by county. Relatively few pediatricians account for the largest proportion of prescriptions. Future studies are needed to link the use of methylphenidate with diagnostic and treatment considerations in attention deficit hyperactivity disorder. [\hyperlink{Trihexyphenidyl Hydrochloride}{PMID: 7767425}, M D Rappley et al., 1995]

\hypertarget{pmid_1056551}{T}he concentration of hexachlorophene was determined in serial blood samples taken from seven premature infants washed with pHisoHex. Results indicated that after a single wash with 5ml of pHisoHex, blood concentrations reached a maximum of 0.75-1.20mug/ml two to four days after application. The results obtained in this study confirm that the dermal absorption of hexachlorophene is greater in premature than in full-term infants. Use of hexachlorophene for infants. Use of hexachlorophene for infants of birth weight less than 2kg has now been discontinued at Queen Mary Hospital and the amount of pHisoHex used in the initial toileting of infants with birth weight more than 2kg has been limited to a single application of 5ml. [\hyperlink{Trihexyphenidyl Hydrochloride}{PMID: 1056551}, S J Greaves et al., 1975]

\hypertarget{pmid_23639119}{T}he Test of Everyday Attention for Children (TEA-Ch) is a reliable neuropsychological assessment of attention control in children. Methylphenidate (MPH) is an effective treatment to improve attentional difficulties in children with attention deficit/hyperactivity disorder (ADHD). Previous studies investigating the effects of MPH on attention performance of children with ADHD have produced mixed results and prior MPH usage may have confounded these results. No previous study has tested the effects of MPH on the entire TEA-Ch battery. This study investigated the effects of MPH on attention performance using the entire TEA-Ch in 51 medication-naïve children with ADHD compared with 35 nonmedicated typically developing children. All children were tested at baseline and after 6 weeks: The children with ADHD were medication-naïve at baseline, received MPH for 6 weeks and were tested whilst on medication at the second testing session. A beneficial effect of MPH administration was found on at least one subtest of each of the three forms of attention (selective, sustained, and attentional control) assessed by the TEA-Ch, independent of practice effects. MPH aided performance on the TEA-Ch tasks that were inherently nonarousing and that might require top-down control of attention. It is recommended that the TEA-Ch measures--Sky Search Count (selective attention),Score! (sustained attention), Creature Counting Time Taken for older children (attentional control), and Same Worlds (attentional control) be prioritized for use in future pharmacological studies using MPH. [\hyperlink{Trihexyphenidyl Hydrochloride}{PMID: 23639119}, Kate Paton et al., 2014]

\section*{Cleviprex}
\subsection*{Result}
\subsubsection*{Answer}

Unknown

\subsubsection*{{Explanation}}
\hypertarget{Cleviprex}
After reviewing all available abstracts, only one abstract mentions Cleviprex (clevidipine), and it is a pharmacokinetic and protein-binding study in vitro using human serum albumin. This study does not involve children, nor does it include any clinical safety or efficacy data in pediatric populations. No other abstracts provide targeted studies of Cleviprex safety in children of any age range. Therefore, based on the abstracts provided, there is no evidence from targeted pediatric safety studies to determine whether Cleviprex is safe or unsafe for use in children. The safety of Cleviprex in children is therefore unknown according to the available data [\hyperlink{pmid_28892755}{PMID: 28892755}, Xin Wang et al., 2017].

\subsection*{Abstracts}
\hypertarget{pmid_28892755}{C}leviprex is a short-acting dihydropyridine calcium channel antagonist used as an antihypertensive drug. In this work, the binding characterization of cleviprex to human serum albumin (HSA) and the competitive binding to HSA between cleviprex and two flavonoids, baicalin and rutin, were studied using multi-spectroscopic techniques and molecular docking method. The fluorescence quenching of HSA by cleviprex was initiated by the formation of HSA-cleviprex complex, which was confirmed by UV-vis spectra measurements. The results of thermodynamic analysis and molecular docking revealed that the hydrophobic interactions and hydrogen bonding were the major acting forces in stabilizing HSA-cleviprex complex. The results of substitution experiments and molecular docking demonstrated that cleviprex was mainly situated within the site I of HSA. Baicalin and rutin could reduce the values of binding constant and enhance the values of binding distance of cleviprex binding to HSA because they bind to the same binding site. The results of synchronous fluorescence and CD spectra suggested that the binding reaction of cleviprex to HSA could give rise to the changes of protein conformation and the combined actions of cleviprex and flavonoids could cause further changes of HSA conformation. Consequently, the intakes of flavonoid-rich foods and beverages should be lessened under the treatment of cleviprex to avoid food-drug interactions. [\hyperlink{Cleviprex}{PMID: 28892755}, Xin Wang et al., 2017]

\hypertarget{pmid_27011634}{T}o report the effectiveness and safety of intravenous (IV) levetiracetam (LEV) in the treatment of critically ill children with acute repetitive seizures and status epilepticus (SE) in a children's hospital. We retrospectively analyzed data from children treated with IV LEV. The mean age of the 108 children was 69.39 ± 46.14 months (1-192 months). There were 58 (53.1\%) males and 50 (46.8\%) females. LEV load dose was 28.33 ± 4.60 mg/kg/dose (10-40 mg/kg). Out of these 108 patients, LEV terminated seizures in 79 (73.1\%). No serious adverse effects were observed but agitation and aggression were developed in two patients, and mild erythematous rash and urticaria developed in one patient. Antiepileptic treatment of critically ill children with IV LEV seems to be effective and safe. Further study is needed to elucidate the role of IV LEV in critically ill children. [\hyperlink{Cleviprex}{PMID: 27011634}, Faruk Incecik et al., ]

\hypertarget{pmid_11879369}{T}he pharmacokinetics of the novel antiepileptic drug (AED) levetiracetam and its major metabolite, ucb L057, were studied in children with partial seizures in a multicenter, open-label, single-dose study. Twenty-four children (15 boys, nine girls), 6 to 12 years old, received a single dose of levetiracetam (20 mg/kg) as an adjunct to their stable regimen of a single concomitant AED, followed by a 24-h pharmacokinetic evaluation. In children, the half-lives of levetiracetam and its metabolite ucb L057 were 6.0 +/- 1.1 and 8.1 +/-2.7 hours, respectively. The Cmax and area under the curve (AUC) of levetiracetam equated for a 1-mg/kg dose were lower in children (Cmax, norm=1.33 plus minus 0.35 microg/ml; AUCnorm=12.4 +/- 3.5 microg/h/ml) than in adults (Cmax, norm=1.38 +/- 0.05 microg/ml; AUCnorm=11.48 +/- 0.63 microg/h/ml), whereas the renal clearance was higher. The apparent body clearance (1.43 +/- 0.36 ml/min/kg) was approximately 30-40\% higher in children than in adults. Levetiracetam was generally well tolerated. On the basis of these data, a daily maintenance dose equivalent to 130-140\% of the usual daily adult maintenance dosage (1,000-3,000 mg/day) in two divided doses, on a weight-normalized level (mg/kg/day) is initially recommended. Clinical efficacy trials in children are ongoing with dosages of 20 to 60 mg/kg/day. [\hyperlink{Cleviprex}{PMID: 11879369}, J M Pellock et al., 2001]

\hypertarget{pmid_18818954}{A} randomized, open, coordinated multi-center trial compared the bacteriological and clinical efficacy and safety of orally administered ceftibuten and trimethoprim-sulfamethoxazole (TMP-SMX) in children with febrile urinary tract infection (UTI). Children aged 1 month to 12 years presenting with presumptive first-time febrile UTI were eligible for enrollment. A 2:1 assignment to treatment with ceftibuten 9 mg/kg once daily (n = 368) or TMP-SMX (3 mg + 15 mg)/kg twice daily (n = 179) for 10 days was performed. Escherichia coli was recovered in 96\% of the cases. Among the E. coli isolates, 14\% were resistant to TMP-SMX but none to ceftibuten. In the modified intention-to-treat population, the bacteriological elimination rates at follow-up did not differ significantly between patients treated with ceftibuten and those treated with TMP-SMX [91 vs. 95\%, with a 95\% confidence interval (CI) for difference of -9.7 to 1.0]. However, the clinical cure rate was significantly higher among those treated with ceftibuten (93 vs. 83\%, with a 95\% CI for difference of 2.4 to 17.0). Adverse events were similar for both regimens and consisted mainly of gastrointestinal disturbances. In conclusion, ceftibuten is a safe and effective drug for the empirical treatment of febrile UTI in young children. [\hyperlink{Cleviprex}{PMID: 18818954}, Staffan Mårild et al., 2009]

\hypertarget{pmid_17561929}{T}here are more than 40 H(1)-antihistamines available worldwide. Most of these medications have never been optimally studied in prospective, randomized, double-masked, placebo-controlled trials in children. The aim was to perform a long-term study of levocetirizine safety in young atopic children. In the randomized, double-masked Early Prevention of Asthma in Atopic Children Study, 510 atopic children who were age 12-24 months at entry received either levocetirizine 0.125 mg/kg or placebo twice daily for 18 months. Safety was assessed by: reporting of adverse events, numbers of children discontinuing the study because of adverse events, height and body mass measurements, assessment of developmental milestones, and hematology and biochemistry tests. The population evaluated for safety consisted of 255 children given levocetirizine and 255 children given placebo. The treatment groups were similar demographically, and with regard to number of children with: one or more adverse events (levocetirizine, 96.9\%; placebo, 95.7\%); serious adverse events (levocetirizine, 12.2\%; placebo, 14.5\%); medication-attributed adverse events (levocetirizine, 5.1\%; placebo, 6.3\%); and adverse events that led to permanent discontinuation of study medication (levocetirizine, 2.0\%; placebo, 1.2\%). The most frequent adverse events related to: upper respiratory tract infections, transient gastroenteritis symptoms, or exacerbations of allergic diseases. There were no significant differences between the treatment groups in height, mass, attainment of developmental milestones, and hematology and biochemistry tests. The long-term safety of levocetirizine has been confirmed in young atopic children. [\hyperlink{Cleviprex}{PMID: 17561929}, F Estelle R Simons et al., 2007]

\hypertarget{pmid_8545564}{W}e evaluated safety and tolerance of acyclovir ACV per os in immunocompetent children affected by chicken-pox admitted to our department from January 1993 to December 1994. 183 subjects (102 males and 81 females) aged between 0 and 14 years were treated by ACV (80 mg/kg/daily in 4 divided doses): 88 children were treated within 24 hours and 95 subjects within 48 hours from the onset of symptoms. The control group consisted of 83 children (52 males and 31 females) aged between 0 to 14 years. In all patients routine blood-test were performed and in those with respiratory illness Chest-Rx was also done. We evaluated clinical course, degree of eruption, the appearance and kind of complications, duration of hospitalization, the compliance and the potential consequences on specific antibody response. Our results show a faster improvement of clinical symptoms in treated patients with respect to the control group with shortening of the period of the fever, itch and appearance of new vescicles. The percentage of complications was lower in treated than in untreated patients. 16 cases tested for specific antibody response showed protective titers six months after treatment. In conclusion, ACV administered per os within 48 hours from onset of exanthema causes reduction of the period and the degree of general symptoms and exanthema, a lower incidence of complications even if non statistically significant. The drug is safe and well-tolerated. [\hyperlink{Cleviprex}{PMID: 8545564}, S Catania et al., ]

\hypertarget{pmid_19325512}{I}ntravenous (IV) levetiracetam (LEV) is approved for use in patients older than 16 years and may be useful in critically ill children, although there is little data available regarding pharmacokinetics. We aim to investigate the safety, an appropriate dosing, and efficacy of IV LEV in critically ill children. We describe a cohort of critically ill children who received IV LEV for status epilepticus, including refractory or nonconvulsive status, or acute repetitive seizures. There were no acute adverse effects noted. Children had temporary cessation of ongoing refractory status epilepticus, termination of ongoing nonconvulsive status epilepticus, cessation of acute repetitive seizures, or reduction in epileptiform discharges with clinical correlate. IV LEV was effective in terminating status epilepticus or acute repetitive seizures and well tolerated in critically ill children. Further study is needed to elucidate the role of IV LEV in critically ill children. [\hyperlink{Cleviprex}{PMID: 19325512}, Nicholas S Abend et al., 2009]

\hypertarget{pmid_31154809}{F}ew drugs are available for migraine prophylaxis in children. Levetiracetam is a broad-spectrum anti-seizure drug that has been suggested to be effective in reducing adult migraine episodes. We assessed the safety and efficacy of levetiracetam in the prevention of pediatric migraine. A randomized double-blind placebo-controlled trial was performed. Eligible participants were aged 4-17 years old with at least four migrainous episodes monthly or had severe disabling or intolerable episodes. Primary endpoints were the mean changes in monthly frequency and intensity of headaches from the baseline phase to the last month of the double-blind phase. Safety endpoint was the adverse effects reported. Sixty-one participants (31 taking levetiracetam and 30 taking placebo) completed the study. All had a significant reduction in frequency and intensity of episodes that was significantly greater in the levetiracetam arm. Sixty eight percent of individuals in the treatment group reported more than 50\% reduction of episodes at the end of the trial compared with 30\% in the placebo group ( Levetiracetam may be useful in migraine prevention and may decrease migraine episodes and severity. The study is prospectively registered with Iranian Registry of Clinical Trials; IRCT.ir, number IRCT2017021632603N1. [\hyperlink{Cleviprex}{PMID: 31154809}, Hadi Montazerlotfelahi et al., 2019]

\hypertarget{pmid_1617039}{I}n two randomized clinical trials in children with otitis media, the efficacy and safety of cefprozil are compared to those of amoxicillin/clavulanate (n = 530) and of cefaclor and cefixime (n = 394). The rate of clinical cure or improvement was similar among patients receiving each drug regimen, ranging from 78\% for amoxicillin/clavulanate to 89\% for cefaclor; for cefprozil, this rate was 84\% and 85\% in the two studies, respectively. In the first study, cefprozil was superior to amoxicillin/clavulanate in the satisfactory clinical response rate for Streptococcus pneumoniae (P = .049), but response rates were similar for Haemophilus influenzae and Moraxella catarrhalis. Significantly more patients treated with amoxicillin/clavulanate (P less than .001) in the first study or cefixime (P less than .01) in the second study developed diarrhea than did those treated with cefprozil. We conclude that cefprozil therapy for otitis media in children produces clinical and bacteriologic response rates similar to those seen with amoxicillin/clavulanate, cefixime, or cefaclor. Furthermore, diarrhea was significantly less common with cefprozil than with cefixime or amoxicillin/clavulanate. [\hyperlink{Cleviprex}{PMID: 1617039}, H R Stutman et al., 1992]

\hypertarget{pmid_3897610}{I}n a total of 13 children with infections, ranging in age from 1 month to 6 years, cefminox (CMNX, MT-141), a new antibiotic of cephem group, was administered 14 times and its absorption, excretion, clinical results and safety were determined. Following intravenous drip infusion of CMNX, high blood level was achieved, with half-life of about 1 hour (0.77 to 1.13 hours). The urinary recovery rate was approximately 80\% within the first 6 hours after completion of administration. Clinical results were rated as effective in 8 out of 12 assessable cases (66.7\%). In any of the cases treated no side effects developed nor any abnormal changes in laboratory finding were observed. From these results, CMNX is considered to be a new antibiotic useful and safe for use in the field of pediatrics. [\hyperlink{Cleviprex}{PMID: 3897610}, K Tomimasu et al., 1985]

\hypertarget{pmid_15898964}{T}his prospective 6-month open trial examined the effectiveness and safety of divalproex sodium (DVPX) in pediatric mixed mania. Thirty-four subjects with a mean age of 12.3 (SD = 3.7) years, DSM-IV diagnosis of a current mixed episode and a baseline Young Mania Rating Scale (YMRS) score >20 were treated with DVPX monotherapy. The primary outcome measures were the YMRS and the Child Depression Rating Scale-Revised. Secondary measures were the Clinical Global Impression Scale for Bipolar Disorder (CGI-BP) and the Children's Global Assessment of Functioning Scale (C-GAS). Measures of safety and tolerability were also administered. Effect size (Cohen's d) based on change scores from baseline was 2.9 for the YMRS and 1.23 for the CDRS-R. Response rate (> or =50\% change from baseline YMRS score and < or =40 score on CDRS-R at the end of study) was 73.5\%. The remission rate (> or =50\% change from baseline on YMRS, < or =40 on CDRS-R, CGI-BP-Improvement subscale of < or =2, and > or =51 CGAS score) was 52.9\%. Significant improvements (p < 0.001) from baseline were seen for mean scores on all outcome measures (i.e., YMRS, CGI-BP, CDRS-R, and C-GAS). DVPX was safe and well tolerated with no serious adverse events during the 6-month trial. This study provides evidence for the effectiveness and safety of DVPX in the treatment of pediatric mixed mania over a 6-month period. Placebo-controlled, randomized trials involving larger samples will ultimately shed light on the efficacy of this agent. [\hyperlink{Cleviprex}{PMID: 15898964}, Mani N Pavuluri et al., 2005]

\hypertarget{pmid_12915341}{T}reatment of seizures in pediatric patients is complicated by the fact that the etiology of the disorder and the pharmacokinetics, efficacy, and safety of antiepileptic drugs (AEDs) may differ from that in adults. With few controlled clinical trials of AEDs in children, the selection of agents to treat pediatric patients must be made on the basis of information from small uncontrolled studies or the extrapolation of clinical trial results in adults. Data from a large number of children with a wide range of seizure disorders who were treated in small-scale prospective studies, or whose records were retrospectively evaluated, indicate that levetiracetam reduces the frequency of seizures in pediatric patients. Available data also indicate that levetiracetam is well tolerated in pediatric patients, with a safety profile similar to that in adults, a low potential for behavioral disturbances, and no reported idiosyncratic adverse reactions. As with other AEDs, children metabolize and clear levetiracetam more rapidly than adults, and somewhat higher doses (based on body weight) are needed to achieve desired plasma concentrations. Several ongoing studies will provide further information on the pharmacokinetics, efficacy, and safety of levetiracetam in this patient population. [\hyperlink{Cleviprex}{PMID: 12915341}, Tracy A Glauser et al., 2003]

\hypertarget{pmid_3928938}{C}efminox (CMNX, MT-141) was evaluated for its safety and efficacy in children. Fifteen cases of bacterial infections were treated with intravenous bolus injections of 30 to 100 mg/kg/day of CMNX. Each 5 cases of acute respiratory tract, urinary tract, and gastrointestinal infections were included. All the cases were cured after the CMNX therapy. No adverse reactions were encountered with the therapy. The serum half-life was approximately 1.5 to 2 hours after intravenous bolus injection in children. The data suggest that CMNX is a safe and effective antibiotic when used in children with susceptible bacterial infections. [\hyperlink{Cleviprex}{PMID: 3928938}, S Ohnari et al., 1985]

\hypertarget{pmid_17006857}{L}evetiracetam (LEV) is the latest drug approved in the European Union for use in polytherapy in children over 4 years of age with partial epileptic seizures that are resistant to other antiepileptic drugs. AIM. To report our experience of associating LEV in children with medication resistant epileptic seizures. We conducted an open, observational, respective study involving 133 children with refractory epilepsies: 106 with focal seizures and 27 with other types of seizures. LEV was associated over a period of more than 6 months and we evaluated its repercussion on the frequency of the seizures and the side effects related to the drug. With average doses of LEV of 1,192 +/- 749 mg/day the frequency of the seizures was reduced by over 50\% in 58.6\% of cases and seizures were quelled in 15.8\% of patients. Side effects were produced in 27.8\% of cases, and were usually transient or tolerable; these effects led to withdrawal of LEV in only eight cases (6.02\%). In 37 children (27.8\%), their relatives noted an improvement in their social behaviour and cognitive abilities. a) LEV is an effective drug that is well tolerated in children with refractory epilepsy; b) Its effectiveness in different types of seizures indicates a broad therapeutic spectrum; and c) LEV can even condition favourable secondary effects, a circumstance that has been reported only exceptionally in the case of other antiepileptic drugs. [\hyperlink{Cleviprex}{PMID: 17006857}, J L Herranz et al., ]

\hypertarget{pmid_24370319}{L}imited data are available for the effectiveness of the antiepileptic drugs in children in daily clinical practice. The aim of this study was to investigate the efficacy and tolerability of the first prescribed old and new antiepileptic drugs in children with newly diagnosed idiopathic epilepsy during a 12-month period. A total of 289 children (141 females and 148 males) who received phenobarbital (n=33), valproate (n=142), carbamazepine (n=42), oxcarbazepine (n=38), or levetiracetam (n=34) as the first-line treatment, were enrolled in the study. Seizure control and the occurrence of adverse events were assessed during a treatment period of 12 months. Overall, 245 (84.8\%) patients remained seizure-free during the study period. The rate of seizure control did not differ significantly between the drug groups (p=0.099). Forty-four (15.2\%) patients including 1 (3.0\%) treated with phenobarbital, 22 (15.5\%) with valproate, 7 (16.7\%) with carbamazepine, 10 (26.3\%) with oxcarbazepine, and 4 (11.8\%) with levetiracetam had treatment failure. There was no significant difference between seizure-free and failure groups in terms of age, gender, seizure type, and drugs used. Overall, 80 (27.7\%) patients had adverse events, of those the most common ones were behavioral problems, nausea and/or vomiting, weight gain, and learning difficulties. The reasons for treatment failures were lack of seizure control in 29 (10.0\%) patients and intolerable adverse events in 15 (5.2\%) patients. It appears that old (phenobarbital, valproate and carbamazepine) and new antiepileptic drugs (oxcarbazepine and levetiracetam) have similar efficacy and tolerability profiles. Institutional ethic number is 28.3.2013/14. [\hyperlink{Cleviprex}{PMID: 24370319}, Unsal Yılmaz et al., 2014]

\hypertarget{pmid_25819874}{A} clinical scenario of a young female on 800 mg of sodium valproate (VPA) who has recently failed lamotrigine (LTG) and levetiracetam (LEV) and who is currently planning a pregnancy is presented. Currently available data pertaining to the longer-term development of children exposed to antiepileptic drugs (AEDs) are reviewed along with considerations around the methodology and interpretation of such research. There is an accumulation of data highlighting significant risks associated with prenatal exposed to VPA, with the level of risk being mediated by dose. The majority of published evidence does not find a significant risk associated with carbamazepine (CBZ) exposure in utero for global cognitive abilities however the evidence for more specific cognitive skills are unclear. Limited data indicate that LTG may be a preferred treatment to VPA in terms of foetal outcome but further evidence is required. Too little data pertaining to LEV exposure is available and a lack of evidence regarding risk of this and other new AEDs should not be interpreted as evidence of safety. [\hyperlink{Cleviprex}{PMID: 25819874}, D McCorry et al., 2015]

\hypertarget{pmid_1494233}{C}efprozil (CFPZ, BMY-28100) was evaluated for its efficacy, safety and pharmacokinetics in children. CFPZ was effective against streptococcal pharyngitis, pneumococcal lower respiratory tract infections, staphylococcal skin infections and Escherichia coli urinary tract infections, but was less effective against lower respiratory tract infections and otitis media due to Haemophilus influenzae. No adverse reactions were encountered in 46 cases treated with CFPZ. With a premeal administration of 7.5 mg/kg, the Cmax was approximately 3.2 micrograms/ml and the T 1/2 beta was 1.4 hours. From the present study, CFPZ appears to be safe and effective against community-acquired childhood infections. [\hyperlink{Cleviprex}{PMID: 1494233}, H Meguro et al., 1992]

\hypertarget{pmid_17204435}{T}o assess the efficacy, tolerability and safety of Levetiracetam (LEV) therapy, we identified 21 (15 male; 6 female) patients with a history of benign epilepsy with centrotemporal spikes (BECTS), with and without secondarily generalization in children and adolescents aged between 5.0 and 12.1 years. LEV was administered as a first drug (number of patients=9) or converted after previous treatment with other AEDs (number of patients=12). The patients were subdivided into two groups: "newly diagnosed" patients and "converted" patients. Patients were followed up for 12 months and all patients were able to continue on LEV treatment. At the end of follow-up (12 months), all patients were seizure free or showed a reduction of seizures >50\%. LEV dosage ranged from 1000 to 2500mg/daily. Overall, 100\% of patients completed the 12 months study, without any important side effect. Somnolence and irritability occurred in two (9.5\%) patients. Our results support findings that LEV monotherapy is effective and well tolerated in children with BECTS. Prospective, large, long-term double-blind studies are needed to confirm these findings. [\hyperlink{Cleviprex}{PMID: 17204435}, A Verrotti et al., 2007]

\hypertarget{pmid_6820499}{A}ll children (6 months to 5 years of age) presenting with febrile seizures have been followed for 3-5 years. Analysis of the data of 123 children gave the following results: -in 20 cases the duration of the convulsions was longer than 20 minutes; -males were prevalent (60,2\%) and characterised by a longer duration of convulsive manifestations (20,2\% of cases); -valproic acid (20-30 mg/kg/die) or phenobarbital (3-5 mg/kg/die) prophylaxis for a minimum of I year protect against recurrence of convulsions (26,6\% of new episodes in non treated vs 10,2\% in treated patients); -irrespective of drug prophylaxis, recurrence is minimal and approaches zero, when the first episode occurs in children older than 3 years; -only one case (0,8\%) had recurrence not associated with fever, during valproic acid treatment two years after the first episode. [\hyperlink{Cleviprex}{PMID: 6820499}, G De Zan et al., ]

\hypertarget{pmid_22105561}{I}n contrast to drugs established to treat neonatal seizures, levetiracetam shows little neurotoxicity in experimental animal models and has good safety records in adults and children. Here, we present results from a survey on the off-label use of levetiracetam in newborn infants among neonatologists and pediatric neurologists in German university hospitals. [\hyperlink{Cleviprex}{PMID: 22105561}, A Koppelstäetter et al., 2011]

\hypertarget{pmid_12706638}{C}entral venous lines (CVLs) are major risk factors for venous thromboembolism (VTE) in children. The objective of PROTEKT was to determine if a low molecular weight heparin (reviparin-sodium) safely prevents CVL-related VTE. This multi-center, open-label study, with blinded central outcome adjudication, randomized patients with new CVLs to twice-daily reviparin-sodium or standard care. The efficacy outcome was based on an exit venogram at Day 30 (+14 days), or earlier in case of CVL removal, or confirmed symptomatic VTE. The safety outcomes were major bleeding and death. Due to slow and restricted patient accrual, PROTEKT was closed prematurely. With reviparin-sodium, 14.1\% (11:78) of patients had VTE compared to 12.5\% (10:80) of control patients (odds ratio=1.15; 95\% confidence interval 0.42, 3.23); 2P=0.82). One patient had a major bleed and there were two deaths, all three events occurring in the standard care group. The use of reviparin-sodium for short-term prophylaxis of CVL-related VTE in children was safe but its efficacy remains unclear. Although underpowered, PROTEKT provided valuable information on event rates and predictors of CVL-related VTE. [\hyperlink{Cleviprex}{PMID: 12706638}, Patricia Massicotte et al., 2003]

\hypertarget{pmid_10775074}{V}aricella, or chickenpox, is very communicable and has been shown to be transmitted to nearly 90\% of household contacts. Severe varicella infections with fatal complications have been noted in children receiving corticosteroids despite the administration of varicella-zoster immune globulin (VZIG). The use of post-exposure acyclovir prophylaxis in immunocompetent children exposed to a household contact with varicella has been shown to decrease the transmission rate of varicella significantly. We studied the safety and efficacy of acyclovir prophylaxis as an adjunctive preventive measure in 8 children (10 separate exposures) receiving corticosteroids for renal disease. Four children (6 separate exposures) served as controls. No adverse reactions were reported with the acyclovir prophylaxis. The maximum change between pre- and study serum creatinine levels was 0.1 mg/dl. None of the 8 patients who received acyclovir prophylaxis developed chickenpox. One of these 8 patients developed humoral immunity to varicella despite the absence of clinical infection. One of 4 patients who received VZIG prophylaxis alone developed chickenpox. These data support the use of acyclovir prophylaxis as an adjunctive measure to VZIG for the prevention of potentially serious varicella infection in children receiving steroids. [\hyperlink{Cleviprex}{PMID: 10775074}, S L Goldstein et al., 2000]

\hypertarget{pmid_30070726}{C}urrently, there are no interferon-free treatments available for hepatitis C virus (HCV)-infected patients younger than 12 years. We evaluated the safety and effectiveness of the all-oral regimen ledipasvir-sofosbuvir ± ribavirin in HCV-infected children aged 6 to <12 years. In an open-label study, patients aged 6 to <12 years received ledipasvir 45 mg-sofosbuvir 200 mg as two fixed-dose combination tablets 22.5/100 mg once daily, with or without ribavirin, for 12 or 24 weeks, depending on HCV genotype and cirrhosis status. The primary efficacy endpoint was sustained virologic response 12 weeks after therapy (SVR12). Twelve patients underwent intensive pharmacokinetic sampling to confirm the appropriateness of the ledipasvir and sofosbuvir dosages. Ninety-two patients were enrolled (88 genotype 1, 2 genotype 3, and 2 genotype 4), with a median age of 9 years (range, 6-11). Most were perinatally infected (97\%) and treatment-naive (78\%). Two were confirmed to have cirrhosis, while the degree of fibrosis was unknown in 55 patients. The overall SVR12 rate was 99\% (91/92; 95\% confidence interval, 94\%-100\%). The single patient not reaching SVR relapsed 4 weeks after completing 12 weeks of treatment. The most common adverse events were headache and pyrexia. One patient had three serious adverse events, which were considered to be not related to study treatment: tooth abscess, abdominal pain, and gastroenteritis. The area under the concentration-time curve and maximum concentration values for sofosbuvir, its primary metabolite GS-331007, and ledipasvir were within predefined pharmacokinetic equivalence boundaries (50\%-200\%) compared to values in adults in phase 2/3 of the ledipasvir and sofosbuvir studies. Conclusion: Ledipasvir-sofosbuvir was well tolerated and highly effective in children 6 to <12 years old with chronic HCV. [\hyperlink{Cleviprex}{PMID: 30070726}, Karen F Murray et al., 2018]

\hypertarget{pmid_8552215}{I}n an retrospective uncontrolled long-term study in 30 children with intractable epilepsy, it was found that treatment with vigabatrin resulted in a seizure reduction of more than 50\% at 1-year follow-up in 40\% of the children. The responders were all children with partial seizures. Side effects were mild and did not lead to discontinuation of the drug. Increased numbers of seizures were seen in three cases. A moderate weight increase was seen in 27\% of the children. At 5-year follow-up 7 children (23\%) still maintained a seizure reduction of more than 50\%. Trials of monotherapy in three seizure-free patients were unsuccessful. No further side effects were observed. A study of evoked potentials in 12 children showed no alteration in latency and amplitudes of VEP following treatment with vigabatrin. Our results show that in children vigabatrin seems to have a stable effect even though a few children may experience a breakthrough of seizures. The presented results together with previous reports on MRI-scans seem to indicate that even in children with a still maturing CNS vigabatrin is a safe drug. [\hyperlink{Cleviprex}{PMID: 8552215}, P Uldall et al., 1995]

\hypertarget{pmid_2041160}{P}harmacokinetics and clinical effects of cefpirome (CPR, HR 810) in children were studied. When 20 mg/kg and 40 mg/kg doses of CPR were administered to 4 children through 30 minutes' drip infusion, half-lives were 1.23 +/- 0.23 (mean +/- S.D.) hours and 1.37 +/- 0.35 (mean +/- S.D.) hours, respectively for the 2 dose levels, and recovery rates in urine in the first 6 hours after administration were 74.8\% and 56.1\%, respectively. CPR was administered to 15 cases (3 tonsillitis, 3 bronchitis, 5 bronchopneumonia, 1 acute cystitis, 1 coxoiliatitis, 1 otitis media, 1 otitis externa). The efficacy rate was 86.7\%. Seven strains of bacteria were isolated and identified 4 Haemophilus influenzae, 3 Staphylococcus aureus, 1 Pseudomonas sp. from these cases. These bacteria in children were followed after administration of CPR. Six strains were eradicated and one was reduced in number. No adverse effects of CPR were observed except in 2 cases, one of which showed transient eosinophilia and the other showed a transient increase of transaminase. These results suggest that CPR may be an effective and safe drug to use on children clinically. [\hyperlink{Cleviprex}{PMID: 2041160}, T Ihara et al., 1991]

\section*{Difluprednate}
\subsection*{Result}
\subsubsection*{Answer}

Yes (0-3 years, for short-term use after cataract surgery)
Unknown (older children, other indications)

\subsubsection*{{Explanation}}
\hypertarget{Difluprednate}
Based on the available abstracts, there is evidence from targeted studies evaluating the safety of difluprednate in children, specifically in the following age ranges:

1. **Ages 0-3 years**: A phase 3B, multicenter, randomized, double-masked, active-controlled study specifically evaluated the safety and efficacy of difluprednate 0.05\% ophthalmic emulsion in children aged 0-3 years who underwent uncomplicated cataract surgery. The study found that difluprednate had a safety and efficacy profile similar to prednisolone acetate 1\%. Adverse drug reactions included corneal edema (1 case) and increased intraocular pressure or ocular hypertension (2 cases), with mean intraocular pressure values 2-3 mm Hg higher than the comparator during treatment, but similar after cessation. The study concluded that difluprednate showed a similar safety profile to prednisolone acetate in this age group [\hyperlink{pmid_27367745}{PMID: 27367745}, M E Wilson et al., 2016].

2. **Ages (not precisely defined, but includes children up to at least 14 years)**: A retrospective, observational case series evaluated 14 pediatric patients (26 eyes) with noninfectious uveitis treated with topical difluprednate. While difluprednate was effective in controlling inflammation, a high rate of steroid-induced intraocular pressure (IOP) elevation (50\% of eyes) and cataract formation (39\% of eyes) was observed, with some requiring surgery. The authors recommend close monitoring due to these risks [\hyperlink{pmid_22265149}{PMID: 22265149}, Mark A Slabaugh et al., 2012].

3. **Case report in a child (age not specified, but pediatric)**: A case report described a child with pars planitis treated with difluprednate for one year, who developed delayed, sudden-onset ocular hypertension and posterior subcapsular cataract, necessitating alternative treatment. The report suggests difluprednate may be useful short-term but highlights the risk of delayed serious complications [\hyperlink{pmid_21034305}{PMID: 21034305}, Paul A Kurz et al., 2011].

Other studies on difluprednate either do not specify pediatric populations or are not targeted safety studies in children.

**Summary by age range:**
- For children aged 0-3 years, a targeted, controlled study supports that difluprednate is as safe as prednisolone acetate for short-term use after cataract surgery, with similar adverse event rates and no new safety concerns.
- For older children (up to at least 14 years), evidence from case series and reports indicates effectiveness but also a high risk of steroid-induced IOP elevation and cataract formation, requiring close monitoring. These are not controlled safety studies but do provide pediatric-specific safety data.
- For children outside these age ranges or for other indications, safety is not definitively established by targeted studies.

Therefore, difluprednate can be considered safe for short-term use in children aged 0-3 years after cataract surgery, based on a targeted safety study. For older children and other indications, the safety profile is less clear and associated with significant risks, so safety is not definitively established.

\subsection*{Abstracts}
\hypertarget{pmid_27367745}{P}urposeTo evaluate safety and efficacy of difluprednate 0.05\% ophthalmic emulsion for treatment of postoperative inflammation after cataract surgery in pediatric patients.MethodsThis was a phase 3B, multicentre, randomized, double-masked, active-controlled study of patients aged 0-3 years who underwent uncomplicated cataract surgery in one eye, with/without intraocular lens implantation. Patients were randomized to receive difluprednate 0.05\% four times daily or prednisolone acetate 1\% for 14 days post surgery, followed by tapering for 14 days. Safety included evaluation of adverse events. Primary efficacy was the proportion of patients with an anterior cell grade of 0 (no cells) at day 14; secondary efficacy was a global inflammation score.ResultsForty patients were randomized to each treatment group. Adverse drug reactions included corneal oedema (difluprednate 0.5\%, n=1; prednisolone acetate 1\%, n=0) and increased intraocular pressure or ocular hypertension (n=2/group). Mean intraocular pressure values during treatment were 2-3 mm Hg higher with difluprednate 0.05\% compared with prednisolone acetate 1\%; mean values were similar between groups by the first week after treatment cessation. At 2 weeks post surgery, the incidence of complete clearing of anterior chamber cells was similar between groups (difluprednate 0.05\%, n=30 (78.9\%); prednisolone acetate 1\%, n=31 (77.5\%). Compared with prednisolone acetate 1\%, approximately twice as many difluprednate 0.05\%-treated patients had a global inflammation assessment score indicating no inflammation on day 1 (n=12 (30.8\%) vs n=7 (17.5\%) and day 8 (n=18 (48.7\%) vs n=10 (25.0\%).ConclusionsDifluprednate 0.05\% four times daily showed safety and efficacy profiles similar to prednisolone acetate 1\% four times daily in children 0-3 years undergoing cataract surgery.  [\hyperlink{Difluprednate}{PMID: 27367745}, M E Wilson et al., 2016] The aim of this study was to evaluate the efficacy and safety of difluprednate ophthalmic solution 0.05\% (Durezol; Alcon Laboratories, Fort Worth, TX) compared with prednisolone acetate ophthalmic suspension 1\% (Pred Forte; Allergan, Inc., Irvine, CA) for endogenous anterior uveitis. In this phase 3, multicenter, randomized, noninferiority trial, 90 patients with endogenous anterior uveitis [>10 anterior chamber (AC) cells and an AC flare score of ≥2 in at least 1 eye] received either difluprednate 4x /day (QID) (n=50) or prednisolone 8x/day (n=40) for 14 days, followed by a 2-week tapering regimen. The main outcome measure was change from baseline in AC cell grade on day 14. At day 14, mean AC cell grade improvement for difluprednate-treated patients was similar to prednisolone-treated patients (2.1 vs. 1.9, respectively), proving noninferiority. At day 14, 68.8\% of difluprednate patients had AC cell clearing (grade 0:≥ 1cell) compared with 61.5\% of prednisolone patients. In the prednisolone-treated group, 12.5\% of patients were withdrawn because of investigator-determined lack of efficacy; no difluprednate-treated patients were withdrawn for this reason (P=0.01). Clinically significant intraocular pressure elevation occurred in 3 difluprednate-treated patients (6.0\%) and 2 prednisolone-treated patients (5.0\%). Difluprednate administered QID is at least as effective as prednisolone administered 8x/day in resolving the inflammation and pain associated with endogenous anterior uveitis. Difluprednate provides effective treatment for anterior uveitis and requires less frequent dosing than prednisolone acetate. Trial NCT00501579 was registered at the National Institutes of Health Registry in July 2007 ( http://clinicaltrials.gov/ct2/show/NCT00501579?term=sirion\&rank=4 ). [\hyperlink{Difluprednate}{PMID: 27367745}, C Stephen Foster et al., 2010]

\hypertarget{pmid_22265149}{T}o evaluate the clinical effect of topical difluprednate in pediatric patients for treatment of noninfectious uveitis. Retrospective, observational case series. Twenty-six eyes of 14 pediatric patients with noninfectious uveitis who were treated with topical difluprednate were evaluated. Anterior and posterior cell grade, visual acuity, intraocular pressure (IOP), and cystoid macular edema (CME) were recorded at each visit. Main outcome measures were changes in anterior segment cell, CME, visual acuity, and IOP and development of a visually significant cataract. A significant (≥ 2-grade decrease or decrease to 0 in anterior segment cell) reduction in anterior segment inflammation was observed during treatment with topical difluprednate in 88\% of eyes (22/25) when used as an adjuvant to systemic immunomodulatory therapy. In addition, improvement in CME associated with uveitis was seen in response to topical therapy with difluprednate in 78\% of eyes with CME (7/9). A significant IOP response (IOP increase of ≥ 10 mm Hg from baseline and IOP ≥ 24 mm Hg) was seen in 50\% of eyes (13/26) and in 50\% of patients (7/14); 3 eyes of 2 patients required glaucoma surgery. Cataract formation or progression was observed in 39\% of eyes (10/26) and in 43\% of patients (6/14); 5 eyes of 3 patients required cataract surgery. Difluprednate is an effective agent for both control of anterior segment inflammation and reduction of CME in pediatric patients with uveitis when used as an adjuvant to systemic immunomodulatory therapy. A high rate of steroid-induced IOP elevation and cataract formation is seen in this population. Close monitoring of pediatric patients receiving difluprednate is recommended. [\hyperlink{Difluprednate}{PMID: 22265149}, Mark A Slabaugh et al., 2012]

\hypertarget{pmid_22364032}{A}cute respiratory infections are the second leading cause of morbidity in children under 18 years. Several drugs have been used with variable efficacy and safety, trying to reduce the associated symptoms and improve quality of life. To evaluate the efficacy and safety of buphenine, aminophenazone and diphenylpyraline hydrochloride when compared with placebo for the control of symptoms associated with common cold in children 6-24 months of age. Randomized clinical trial, double blind, placebo controlled, in 100 children < 24 months of any gender, with symptoms associated to common cold. They received the drug under study vs. placebo for seven days. Both groups received acetaminophen. The change on common cold related symptoms were evaluated. Statistic analysis was made with STATA 11.0 for Mac. Fifty-three children were randomized to study drug and forty-seven to placebo. Age of children in each group was similar (12.2 +/- 5.8 months vs. 12.7 +/- 5.8 months, p NS). There were significant differences between groups in relation to rhinorrea and sneezing resolution, with better results in Flumil group and no adverse events observed. The results in this study indicates that Flumil is a safe and effective drug for control of symptoms present in the common cold in children aged 6-24 months. [\hyperlink{Difluprednate}{PMID: 22364032}, Ericka Montijo-Barrios et al., ]

\hypertarget{pmid_19101421}{T}o assess the efficacy and safety of difluprednate ophthalmic emulsion 0.05\% (Durezol) 2 or 4 times a day compared with those of a placebo in the treatment of inflammation and pain associated with ocular surgery. Twenty-six clinics in the United States. One day after unilateral ocular surgery, patients who had an anterior chamber cell grade of 2 or higher (>10 cells) were treated with 1 drop of difluprednate 2 times or 4 times a day or with a placebo (vehicle) 2 times or 4 times a day in the study eye for 14 days. This was followed by a 14-day tapering period and a 7-day safety evaluation. Outcome measures included cleared anterior chamber inflammation (grade 0, <or=1 cell), absence of pain, and analysis of ocular adverse events. Of the 438 patients, 111 received difluprednate 2 times a day, 107 received difluprednate 4 times a day, and 220 received a placebo 2 or 4 times a day. Both difluprednate dosage regimens reduced postoperative ocular inflammation and pain safely and effectively compared with the placebo. A greater proportion of difluprednate-treated patients had a reduction in inflammation and pain at 8 days and 15 days. Three percent of patients in both difluprednate groups had a clinically significant IOP rise (>or=10 mm Hg and >or=21 mm Hg from baseline, respectively) versus 1\% in the placebo group. Difluprednate given 2 or 4 times a day cleared postoperative inflammation and reduced pain rapidly and effectively. There were no serious ocular adverse events. Fewer adverse events were reported in the difluprednate-treated groups than in the placebo group. [\hyperlink{Difluprednate}{PMID: 19101421}, Michael S Korenfeld et al., 2009]

\hypertarget{pmid_2275328}{T}he efficacy of flunitrazepam (0.04 mg.kg-1) as a premedicant was evaluated in 40 young children of less than 5 years of age in a double-blind, placebo-controlled study. Flunitrazepam was given by the rectal route 15 min prior to an inhalational mask induction with halothane. Sedation score, mask acceptance and induction score were significantly better in premedicated children than in the placebo group. There were no hypoxic episodes, prolonged sedation or other complications in either group. This suggests that flunitrazepam administered rectally in a low dose is an acceptable premedication in young children. [\hyperlink{Difluprednate}{PMID: 2275328}, C Estève et al., 1990]

\hypertarget{pmid_34547279}{T}o describe the effectiveness and side effect profile of difluprednate therapy in a series of patients with anterior scleritis. Retrospective, interventional case series. Data collected from all patients with anterior scleritis who used difluprednate as a single treatment agent from January 1, 2018, to January 1, 2020, including demographics, scleritis type, presence of nodules or necrosis, changes in scleritis activity, intraocular pressure (IOP), number of difluprednate drops used, best-corrected visual acuity (BCVA), and lens status. The primary outcome was clinical resolution of scleritis. Secondary outcomes included BCVA loss ≥2 lines, change in lens status or cataract surgery, and IOP ≥24 mm Hg. Twenty-five patients (35 eyes) were analyzed. The median age was 60 years (range 13-78); 60\% were female; 64\% were White. Forty percent had bilateral disease, and 44\% of patients had an associated systemic disease. The majority of eyes (66\%) had diffuse anterior scleritis. Eighty-three percent of eyes achieved resolution of scleritis, with a median time of resolution of 6 weeks. Eyes treated with an initial dose of ≥4 times daily were more likely to achieve disease resolution (hazard ratio [HR] = 3.43, 95\% confidence interval [CI] 1.19, 9.88, P = .02). Nine eyes had IOP elevation. Four eyes lost ≥2 lines of BCVA, and 1 due to cataract progression. One eye underwent cataract surgery. Difluprednate alone may effectively treat non-infectious anterior scleritis with a tolerable side effect profile. [\hyperlink{Difluprednate}{PMID: 34547279}, Paulina Liberman et al., 2022]

\hypertarget{pmid_21034305}{T}o report the effects of twice-daily difluprednate in a child with pars planitis (PP). Case report. PP was controlled with topical difluprednate for 1 year. Then an atypical pattern of steroid response--delayed, relatively sudden onset of recalcitrant ocular hypertension (OHT)--and posterior subcapsular cataract (PSC) formation necessitated alternative treatment. Although not a standard treatment, in select cases of PP topical difluprednate therapy could be a useful short-term treatment option while alternative treatments are considered or immunosuppressive agents build to therapeutic levels. Ophthalmologists must be aware of the potential for delayed onset of serious complications when using difluprednate. [\hyperlink{Difluprednate}{PMID: 21034305}, Paul A Kurz et al., 2011]

\hypertarget{pmid_8675429}{D}ivalproex sodium is an effective drug for the treatment of migraine. Most adverse drug events are transient and not of great clinical concern. Although rare, well-documented examples of liver toxicity have been reported in children under 2 years of age on polypharmacy. Additional cases occur in children under 10 who are receiving polypharmacy, particularly those who have intractable seizures and degenerative central nervous system disease. Clinicians who treat migraineurs with divalproex sodium do not need to be overly preoccupied with monitoring of drug levels and liver function tests. The most valuable test is clinical observation of the patient. [\hyperlink{Difluprednate}{PMID: 8675429}, S D Silberstein et al., 1996]

\hypertarget{pmid_21111374}{D}ifluprednate ophthalmic emulsion 0.05\% (Durezol™, Alcon, Fort Worth, Texas) is a topical difluorinated derivative of prednisolone with potent anti-inflammatory activity. Difluprednate 0.05\% has a reported associated increase in intraocular pressure (IOP) in 3\% of patients. Although the occurrence may be low, the possible elevation in IOP may be substantially higher than commonly encountered with other topical steroids. A 49-year-old black man presented with a traumatic anterior uveitis recalcitrant to traditional prednisolone acetate 1\% treatment. The patient was switched to difluprednate 0.05\% in an attempt to better control the ocular inflammation. Although the patient did not exhibit an IOP response after 4 weeks of treatment with prednisolone acetate 1\%, he did experience a pressure response within 2 weeks of initiating difluprednate treatment, resulting in an IOP increase from 9 mmHg to 48 mmHg with subsequent microcystic edema. A 44-year-old black woman presented with recurrent scleritis resistant to topical prednisolone acetate, loteprednol etabonate, and oral nonsteroidal anti-inflammatory therapy. Topical loteprednol 0.5\% was replaced by difluprednate 0.05\%. All evidence of ocular inflammation was eradicated with the changed therapy. IOP rose in the difluprednate-treated eye from 18 mmHg to 34 mmHg over the course of 18 days. In both cases, the IOP elevation was managed rapidly with the discontinuation of difluprednate and temporary use of IOP-reducing agents with no lasting adverse effects. Difluprednate 0.05\% is a new topical therapeutic option indicated for the treatment of inflammation and pain management associated with ocular surgery with an off-label potential for treatment of other anterior segment inflammatory conditions. However, clinicians need to be aware of the potential risk for significant and potentially rapid onset of IOP increase with this medication and manage patients accordingly. [\hyperlink{Difluprednate}{PMID: 21111374}, Kelly Meehan et al., 2010]

\hypertarget{pmid_21490354}{D}imenhydrinate is an over-the-counter drug that is commonly used for the treatment of nausea and vomiting. Many of my adult patients use it, but is it safe and useful in the pediatric population? Dimenhydrinate appears to be safe for use in the pediatric population. While little literature has been published about adverse effects of this medication, family physicians need to identify the cause of the vomiting before considering if the drug will be effective and need to ensure that patients safely use the medication and avoid potential interaction of the drug with other products. [\hyperlink{Difluprednate}{PMID: 21490354}, Paul Enarson et al., 2011]

\hypertarget{pmid_7560629}{T}his study compared the safety and efficacy of digoxin and flecainide in the prophylaxis of supraventricular tachycardia in infants. Recurrence of supraventricular tachycardia in infants is common. Digoxin is the conventional drug of first choice for prophylaxis, but its efficacy has not been tested in a controlled clinical trial, and there is no consensus on the drug of choice when digoxin is ineffective. We reviewed retrospectively the records of all infants with supraventricular tachycardia due to atrioventricular (AV) reentry admitted to our hospital between January 1986 and December 1993. Thirty-nine infants presented with sustained AV reentrant tachycardia at age 1 to 330 days (median 12). Intravenous flecainide was required to maintain immediate control in six patients who were then treated with oral flecainide. The other 33 patients were treated with oral digoxin. There was no recurrence of tachycardia in 14 (42\%) of the 33 patients (95\% confidence interval [CI] 25\% to 61\%). In the other 19 patients (58\%) (95\% CI 39\% to 75\%), digoxin was replaced by oral flecainide because of multiple recurrence of tachycardia. Full control was achieved in all 19 of these patients (100\%) (95\% CI 82\% to 100\%) and in 5 of the 6 patients treated with both intravenous and oral flecainide. Thus, overall, flecainide was effective in 24 (96\%) of 25 patients (95\% CI 80\% to 100\%). Comparison with previous natural history studies suggests that digoxin is ineffective in the prophylaxis of supraventricular tachycardia. Oral flecainide was effective in a small number of infants, with no adverse effects (95\% CI 0\% to 12\%), and may now be preferred as the primary prophylactic agent. [\hyperlink{Difluprednate}{PMID: 7560629}, J J O'Sullivan et al., 1995]

\hypertarget{pmid_19752076}{V}omiting is a common symptom in children with infectious gastroenteritis. It contributes to fluid loss and is a limiting factor for oral rehydration therapy. Dimenhydrinate has traditionally been used for children with gastroenteritis in countries such as Canada and Germany. We investigated the efficacy and safety of dimenhydrinate in children with acute gastroenteritis. We performed a prospective, randomized, placebo-controlled, multicenter trial. We randomly assigned 243 children with presumed gastroenteritis and vomiting to rectal dimenhydrinate or placebo. Children with no or mild dehydration were included. All children received oral rehydration therapy. Primary outcome was defined as weight gain within 18 to 24 hours after randomization. Secondary outcomes were number of vomiting episodes, fluid intake, parents' assessment of well-being, number of diarrheal episodes, and admission rate to hospital. We recorded potential adverse effects. Change of weight did not differ between children who received dimenhydrinate or placebo. The mean number of vomiting episodes between randomization and follow-up visit was 0.64 in the dimenhydrinate group and 1.36 in the placebo group. In total, 69.6\% of the children in the dimenhydrinate group versus 47.4\% in the placebo group were free of vomiting between randomization and the follow-up visit. Hospital admission rate, fluid intake, general well-being of the children, and potential adverse effects, including the number of diarrhea episodes, were similar in both groups. Dimenhydrinate reduces the frequency of vomiting in children with mild dehydration; however, the overall benefit is low, because it does not improve oral rehydration and clinical outcome. [\hyperlink{Difluprednate}{PMID: 19752076}, Ulrike Uhlig et al., 2009]

\hypertarget{pmid_23236934}{T}he use of midazolam for children was approved in March, 2010. Since the efficacy and safety data of midazolam used in children, excluding low-birth-weight infants and newborns, for "sedation under artificial respiration in intensive care units" were quite limited, a post-marketing survey was carried out to confirm the validity of the established dosage and administration. A consecutive enrollment method was adopted. Based on the data of 153 patients collected from 8 institutes, efficacy and safety profiles were analyzed. Among the 149 patients included in the safety analysis set, 6 adverse reactions were reported in 6 patients. The incidence of adverse events was 4.0\% (6/149). Reported adverse reactions included depressed level of consciousness: 1 event, delirium: 1 event, psychomotor hyperactivity: 1 event, hypotension: 2 events, and blood pressure increase: 1 event. Serious adverse drug reaction (ADR) reported in this survey was depressed level of consciousness. This ADR resolved on the following day after the treatment with flumazenil. Paradoxical reactions were reported in 1 patient, and tolerance was reported in 2 patients. The efficacy rate was 96.5\% (138/143). No additional safety issues (status of adverse reactions, status of adverse events, status of serious adverse events, etc.) and efficacy issue were manifest in the patients treated with the dosage and administration method established at the approval of the drug. [\hyperlink{Difluprednate}{PMID: 23236934}, Keizo Sogabe et al., 2012]

\hypertarget{pmid_9890791}{I}n 47 children followed for 1 year after the first "simple" febrile convulsion, dipropylacetate (Depakine, 20 mg/kg) was as effective in preventing new febrile convulsions (a single recurrence in 4\% of 47 children) as was phenobarbital (5 mg/kg) or primidone (25 mg/kg) (a single recurrence in 4\% of 47 children), and there were no side effects. Of 47 untreated children followed for 1 year, 55\% had 1 to 4 new febrile convulsions. All medications were given in divided doses morning and evening. [\hyperlink{Difluprednate}{PMID: 9890791}, G B Cavazzuti et al., 1975]

\hypertarget{pmid_24579280}{T}he aim of this video-based study was to examine the taste acceptance of children between the ages of 2 and 5 years regarding highly concentrated fluoride preparations in kindergarten-based preventive programs. The fluoride preparation Duraphat was applied to 16 children, Elmex fluid to 15 children, and Fluoridin N5 to 14 children. The procedure was conducted according to a standardized protocol and videotaped Three raters evaluated the children's nonverbal behavior as a measure of taste acceptance on the Frankl Behavior Rating Scale. The interrater reliability (intraclass correlation coefficient; ICC) was .86. In an interview, children indicated the taste of the fluoride preparations on a three-point "smiley" rating scale. The interviewer used a hand puppet during the survey to establish confidence between the children and examiners. Children's nonverbal behavior was significantly more positive after Fluoridin N5 and Duraphat were applied compared to the application of Elmex fluid. The same trend was found during the smiley assessment. The response of children who displayed cooperative positive behavior before the application of fluoride preparations was significantly more positive than those who displayed uncooperative negative behavior. To achieve a high acceptance of the application of fluoride preparations among preschool children, flavorful preparations should be used. [\hyperlink{Difluprednate}{PMID: 24579280}, Anne-Kathrin Kolb et al., 2013]

\hypertarget{pmid_8249087}{E}flornithine (difluoromethylornithine, DFMO) has recently been approved for the treatment of Trypanosoma brucei gambiense trypanosomiasis. Treatment failures have been infrequent but have occurred among patients treated with oral DFMO only, and among children. To investigate the higher frequency of failures observed in young patients, DFMO trough concentrations in serum and cerebrospinal fluid (CSF) were measured at the end of treatment in 13 children and 50 adults who had received 200 mg/kg intravenously every 12 h for 14 d. Mean DFMO concentration in CSF was significantly lower among children aged less than 12 years when compared to older patients (25.1 vs 68.9 nmol/mL, P < 0.001). Mean serum concentration was also lower in children (49.2 vs 87.5 nmol/mL, P = 0.03). Among patients who received DFMO as initial therapy for sleeping sickness, the mean CSF/serum ratio was lower in children (0.41 vs 0.91, P < 0.005). The 3 patients who failed DFMO treatment had CSF trough concentrations around or below 50 nmol/mL. Convulsions and anaemia were associated with higher drug levels and previous therapy with melarsoprol. The lower CSF drug concentrations observed in children could result from higher renal clearance and different CSF pharmacokinetics of DFMO in that age group. To avoid treatment failures, a 6-hourly regimen as well as higher DFMO dosage based on body surface area rather than on weight are recommended for children. [\hyperlink{Difluprednate}{PMID: 8249087}, F Milord et al., ]

\hypertarget{pmid_21276131}{D}iclofenac is an effective, opiate-sparing analgesic for acute pain in children, which is commonly used in pediatric surgical units. Recently, a Cochrane review concluded the major knowledge gap in diclofenac use is dosing information. A pharmacokinetic meta-analysis has been undertaken with the aim of recommending a dose for children aged 1-12 years. Studies containing diclofenac pharmacokinetic data were identified during a Cochrane systematic review, and authors were asked to provide raw data. A pooled population analysis was undertaken in NONMEM to define the pharmacokinetics of intravenous, oral, and rectal diclofenac in children. Simulations were performed to recommend a dose yielding an equivalent area under diclofenac concentration-time curve (AUC) to a 50-mg dispersible tablet in adults. Data from 111 children aged 1-14 years consisting of 375 samples following intravenous, oral suspension, and suppositories were used. Adult dispersible tablet and suspension data were added to provide a reference AUC and support the absorption modeling, respectively. A three-compartment model described disposition, a dual-absorption compartment model was used for suspension and dispersible tablet data, and single-absorption compartment model for suppositories. The estimate of clearance was 16.5 l·h(-1) ·70 kg(-1) and bioavailabilities were 0.36, 0.63, and 0.35 for suspension, suppository, and dispersible tablets, respectively. Single doses of 0.3 mg·kg(-1) for intravenous, 0.5 mg·kg(-1) for suppositories, and 1 mg·kg(-1) for oral diclofenac in children aged 1-12 years are recommended as they yield a similar AUC to 50 mg in adults. [\hyperlink{Difluprednate}{PMID: 21276131}, Joseph F Standing et al., 2011]

\hypertarget{pmid_16199414}{R}emifentanil is recommended for use in procedures with painful intraoperative stimuli but minimal postoperative pain. However, bradycardia and hypotension are known side-effects. We evaluated haemodynamic effects of i.v. glycopyrrolate during remifentanil-sevoflurane anaesthesia for cardiac catheterization of children with congenital heart disease. Forty-five children undergoing general anaesthesia with remifentanil and sevoflurane were randomly allocated to receive either saline, glycopyrrolate 6 microg kg(-1) or glycopyrrolate 12 microg kg(-1). After induction of anaesthesia with sevoflurane, i.v. placebo or glycopyrrolate was administered. An infusion of remifentanil at the rate of 0.15 microg kg(-1) min(-1) was started, sevoflurane continued at 0.6 MAC and cisatracurium 0.2 mg kg(-1) was given. Heart rate (HR) and non-invasive arterial pressures were monitored and noted every minute for the first 10 min and then every 2.5 min for subsequent maximum of 45 min. Baseline HR [mean (SD)] of 117 (20) beats min(-1) decreased significantly from 12.5 min onwards after starting the remifentanil infusion in the control group [106 (18) at 12.5 min and 99 (16) beats min(-1) at 45 min]. In the groups receiving glycopyrrolate, no significant decrease in HR was noticed. Glycopyrrolate at 12 microg kg(-1) induced tachycardia between 5 and 9 min after administration. Systolic and diastolic arterial pressures decreased gradually, but there were no significant differences in the pressures between groups. I.V. glycopyrrolate 6 microg kg(-1) prevents bradycardia during general anaesthesia with remifentanil and sevoflurane for cardiac catheterization in children with congenital heart disease. Administering 12 microg kg(-1) of glycopyrrolate temporarily induces tachycardia and offers no additional advantage. [\hyperlink{Difluprednate}{PMID: 16199414}, K Reyntjens et al., 2005]

\hypertarget{pmid_34826122}{A} topical formulation of diclofenac (FLECTOR diclofenac epolamine topical system (FDETS)) is approved in adults for the treatment of acute pain due to minor strains, sprains, and contusions; however, its safety and efficacy have not been investigated in a pediatric population. This study assessed the safety and efficacy of the FLECTOR (diclofenac epolamine) topical system in children. This was an open-label, single-arm, phase IV study at ten USA-based family medicine or pediatric practices in children aged 6-16 years with a clinically significant minor soft tissue injury sustained within the preceding 96 h and at least moderate spontaneous pain on the Wong-Baker FACES 104 patients were enrolled; 52 were 6-11 years old, and 52 were 12-16 years old (mean age 11.6 years). The maximum tolerability score experienced by any patient was 1 (faint redness). Fourteen adverse events (none serious) in nine patients (8.7\%) were considered possibly treatment-related. Reduction in pain during the study was somewhat greater for patients aged 6-11 versus 12-16 years (p < 0.011). The diclofenac plasma concentration tended to be higher in the younger age group compared with older patients: 1.83 versus 1.46 ng/mL at the first assessment and 2.49 versus 1.11 ng/mL at the last assessment (p = 0.002). The FLECTOR topical system safely and effectively provided pain relief for minor soft tissue injuries in the pediatric population, with minimal systemic nonsteroidal anti-inflammatory drug exposure and low potential risk of local or systemic adverse events. ClinicalTrials.gov identifier NCT02132247. [\hyperlink{Difluprednate}{PMID: 34826122}, Christopher A Jones et al., 2022]

\hypertarget{pmid_9132194}{T}o evaluate the safety and efficacy of intranasal diamorphine as an analgesic for use in children in accident and emergency (A\&E). A prospective, randomised clinical trial with consecutive recruitment of patients aged between 3 and 16 years with clinically suspected limb fractures. One group received 0.1 mg/kg intranasal diamorphine, and the other group received 0.2 mg/kg intramuscular morphine. At 0, 5, 10, 20, and 30 minutes pain scores, Glasgow coma score, and peripheral oxygen saturations were recorded; parental acceptability was assessed at 30 minutes. 58 children were recruited, with complete data collection in 51 (88\%); the median summed decrease in pain score was better for intranasal diamorphine than intramuscular morphine (9 v 8), though this was not significant (P = 0.4, Mann-Whitney U test). The episode was recorded as "acceptable" in all parents whose child received intranasal diamorphine, compared with only 55\% of parents in the intramuscular morphine group (P < 0.0001, Fisher's exact test). There was no incidence of decreased peripheral oxygen saturation or depression in the level of consciousness in any patient. Intranasal diamorphine is an effective, safe, and acceptable method of analgesia for children requiring opiates in the A \& E department. [\hyperlink{Difluprednate}{PMID: 9132194}, J A Wilson et al., 1997]

\hypertarget{pmid_10971664}{T}o determine the beneficial use of divalproex sodium as a prophylactic treatment for migraine in children. Previous studies for treatment of migraine in adults have shown a greater than 50\% reduction in migraine attack frequencies. Few data exist, however, regarding the efficacy and safety of divalproex sodium use in children with migraine. We studied the incidence of headache relief in our patients with migraine aged 16 years and younger treated with divalproex sodium prophylactically at our institution from July 1996 to December 1998 to determine medication dosage used, concomitant headache medications, and possible adverse effects. A total of 42 patients, ranging in age from 7 to 16 years (mean age, 11.3 years), were treated with divalproex sodium for headache. All had a history of migraine with or without aura. Baseline headache frequency during a minimum 6-month period was one to four headaches per month. Divalproex sodium dosage ranged from 15 mg/kg/day to 45 mg/kg/day. Of the 42 patients, 34 (80.9\%) successfully discontinued their abortive medications. After 4 months' treatment, 50\% headache reduction was seen in 78.5\% of patients, 75\% reduction in 14.2\% of patients, and 9. 5\% of patients became headache-free. These results indicate divalproex sodium to be an effective and well-tolerated treatment for the prophylaxis of migraine in children. [\hyperlink{Difluprednate}{PMID: 10971664}, J M Caruso et al., 2000]

\hypertarget{pmid_26646324}{B}isphosphonates are used in the treatment of vitamin D intoxication (VDI) after failure of conventional therapy including prednisolone. Safety concerns restrict the use of bisphosphonates from being used as first-line therapy for VDI in children. The aim of this study was to evaluate the efficacy and safety of pamidronate in comparison with prednisolone in children with VDI. We reviewed the hospital records of children consecutively diagnosed with VDI at two medical centers in a 15 year period. The subjects consisted of 21 children (age, 0.3-4.2 years) who were treated with prednisolone and/or bisphosphonates. Pamidronate (n = 18) or alendronate (n = 3) was used in six patients after unsuccessful prednisolone treatment, and in 15 patients from baseline. Initial serum calcium and 25-hydroxyvitamin D were 16.1 ± 1.9 mg/dL and 493 ± 219 ng/mL, respectively. The median time to reach normocalcemia in the pamidronate, alendronate and prednisolone groups was 3 days (range, 2-12 days), 4 days (range, 3-6 days) and 17 days (range, 12-26 days), respectively (P = 0.013). The pamidronate group had a fivefold shorter hospital stay than the prednisolone group. Three patients initially treated with prednisolone developed nephrocalcinosis but this did not occur in any patient treated with bisphosphonates from baseline. Apart from transient fever and moderate hypophosphatemia, no side-effect of bisphosphonate treatment was observed. Pamidronate is efficient and safe for the treatment of VDI in children. Pamidronate use significantly shortens the duration of treatment, and thereby may prevent the development of nephrocalcinosis. Instead of prednisolone, pamidronate should be used together with hydration and furosemide as the first-line therapy for VDI. [\hyperlink{Difluprednate}{PMID: 26646324}, Cengiz Kara et al., 2016]

\hypertarget{pmid_21750615}{T}o review the most recent published data regarding the novel potent steroid, difluprednate ophthalmic emulsion, 0.05\%. A comprehensive search of recent published literature including difluprednate was performed. Clinical studies relevant to the characteristics and clinical efficacy of difluprednate in controlling postoperative inflammation were included, and a synopsis of each study was developed. Several recent publications were identified in which difluprednate was shown to be efficacious in the treatment of postoperative inflammation in different clinical settings, including a novel perioperative regimen. Additional literature retrieved from this search included data on the relative potency of difluprednate, potential utility in the posterior segment, as well as the advantages of the emulsion formulation. Difluprednate has been studied extensively and shown in recent literature to be a safe and effective topical anti-inflammatory drug. The proven strength and unique formulation of difluprednate, along with its potent efficacy in treating and preventing inflammation, provides clinicians with a beneficial treatment option. [\hyperlink{Difluprednate}{PMID: 21750615}, Eric D Donnenfeld et al., 2011]

\hypertarget{pmid_20856594}{T}o evaluate the efficacy and safety of twice-daily difluprednate ophthalmic emulsion 0.05\% (Durezol(®)) versus placebo administered before surgery for managing inflammation and pain following cataract extraction. Eligible subjects (N = 121) were randomized 2:1 to topical treatment with 1 drop difluprednate or placebo administered twice daily for 16 days, followed by a 14-day tapering period. Dosing was initiated 24 hours before unilateral ocular surgery. Clinical signs of inflammation (anterior chamber [AC] cell and flare grade, bulbar conjunctival injection, ciliary injection, corneal edema, and chemosis), ocular pain/discomfort, intraocular pressure (IOP), and adverse events were assessed. Clearing of inflammation on day 14 (primary endpoint), defined as an AC cell grade of 0 (≤5 cells) and a flare grade of 0 (complete absence), was achieved in a significantly greater percentage of subjects treated with difluprednate, compared with placebo (74.7\% vs 42.5\%; P = 0.0006). A significantly greater percentage of difluprednate-treated subjects were free of ocular pain/discomfort on day 14 than placebo-treated subjects (64.6\% vs 30.0\%; P = 0.0004). Three subjects (3.7\%) in the difluprednate group had a clinically significant IOP rise (defined as ≥21 mmHg and a change from baseline ≥10 mmHg at same visit). Difluprednate, administered 2 times daily starting 24 hours before cataract surgery, was highly effective for managing ocular inflammation and relieving pain and discomfort postoperatively. Difluprednate was well tolerated and provides a convenient twice-daily option for managing postoperative ocular inflammation. [\hyperlink{Difluprednate}{PMID: 20856594}, Stephen Smith et al., 2010]

\section*{Belbuca}
\subsection*{Result}
\subsubsection*{Answer}

Unknown

\subsubsection*{{Explanation}}
\hypertarget{Belbuca}
After reviewing all available abstracts, there is no evidence from any abstract that specifically addresses the safety of Belbuca (buccal buprenorphine) in children. None of the abstracts mention Belbuca, nor do they present targeted studies evaluating its safety in any pediatric age group. Therefore, based on the abstracts provided, the safety of Belbuca in children is unknown for all age ranges.

\subsection*{Abstracts}
\hypertarget{pmid_33430841}{A}cute cough in children often causes discomfort to children and parents, reducing their quality of life. Despite the extensive utilization of over-the-counter remedies for cough, the efficacy of most of these treatments in children has not been confirmed. We conducted a randomized, double blind, placebo-controlled clinical trial of 106 children with acute cough to evaluate the efficacy and safety of KalobaTUSS®, a paediatric cough syrup based on acacia honey and on Malva sylvestris extract, Inula helenium extract, Plantago major extract, and Helichrysum stoechas extract by using a validated 6 points Likert scale. Children were orally treated with KalobaTUSS® or placebo for 8 days. Children receiving KalobaTUSS® showed an early and significant reduction in night-time and day-time cough scores measured using a specific scale and a shorter duration of cough than children treated with the placebo. KalobaTUSS® is well tolerated and produces positive effects by reducing the severity and shortening the duration of cough in children. Clinicaltrials.gov no. NCT04073251 . Retrospectively registered. [\hyperlink{Belbuca}{PMID: 33430841}, Ilaria Carnevali et al., 2021]

\hypertarget{pmid_10741880}{T}he aim of this article is to review data on the efficacy and safety of montelukast in the treatment of children with asthma. Available published literature, including published abstracts, is reviewed. In patients aged 6 to 14 years with asthma (n = 27), montelukast 5mg demonstrated a significant decrease in exercise-induced bronchoconstriction 20 to 24 hours postdose after 2 days of treatment. For children with chronic asthma, only one study of the regular use of a leukotriene receptor antagonist has been published. The efficacy and safety of montelukast in children aged 6 to 14 years with asthma (n = 336) were studied during an 8-week, double-blind, placebocontrolled trial. There was a significantly greater improvement in forced expiratory volume in 1 second (FEV1) from baseline for the montelukast group (8.23\%) compared with the placebo group (3.58\%). There was a significant decrease in the use of a 3-agonist for symptom relief, as well as in the percentage of days and percentage of patients with asthma exacerbations. An asthma specific quality-of-life (QOL) questionnaire revealed a significant overall improvement in QOL and a significant improvement in the QOL domains for symptoms, activity and emotions in montelukast recipients. There was no significant difference between montelukast and placebo recipients in the frequency of adverse events, with the exception of allergic rhinitis, which was more prevalent in the placebo group. An open label follow-up of patients from the above study was undertaken. The effect of montelukast on FEV1 was consistent for up to 1.4 years, with the increase in FEV1 being not significantly different from that in a small control group treated with inhaled beclomethasone dipropionate. QOL remained significantly improved during the open treatment period. Montelukast appears effective and safe for the treatment of children with asthma. [\hyperlink{Belbuca}{PMID: 10741880}, A Becker et al., 2000]

\hypertarget{pmid_38085143}{O}ral Montelukast is recommended as maintenance therapy for persistent asthma, but there is controversy regarding its effectiveness in controlling asthma attacks. The present study was conducted to investigate the clinical efficacy of oral Montelukast for asthma attacks in children. This study was conducted as a double-blind placebo-controlled clinical trial on 80 children aged 1-14 years with asthma who were admitted to the emergency department of Bahrami Children's Hospital (Tehran, Iran) during one year. Patients were randomly divided into case and control groups. In addition to the standard asthma attack treatment, Montelukast was prescribed in the case group and placebo in the control group for one week. Patients were evaluated in terms of asthma attack severity score and oxygen saturation percentage (SpO2) in room air as primary outcomes 1, 4, 8, 24 and 48 hours after admission. In the first 48 hours, there was no significant difference in the score of asthma attack severity and SpO2 between the case and control groups. There was no significant difference between the groups in terms of length of hospitalization or number of admissions to the intensive care unit. None of the patients were re-hospitalized after discharge. The results of this study showed that the use of Montelukast along with the standard treatment of asthma attacks in children has no added benefit. [\hyperlink{Belbuca}{PMID: 38085143}, Mohsen Jafari et al., 2023]

\hypertarget{pmid_11167954}{M}ontelukast is a leukotriene receptor antagonist administered orally once daily for treatment of chronic asthma in adults and children. A comprehensive analysis of safety data from double-blind, randomized, placebo-controlled trials with montelukast has not been previously reported. A pooled analysis of safety data from 11 multicentre, randomized, controlled montelukast Phase IIb and III trials and five long-term extension studies was performed. A total of 3386 adult patients (aged 15-85 years) and 336 paediatric patients (aged 6-14 years) were enrolled in the trials; 2031 adults received montelukast for up to 4.1 years, and 257 children received montelukast for up to 1.8 years. Summary statistics comparing incidences of adverse events among treatment groups were calculated. The overall incidence of clinical and laboratory adverse events among montelukast-treated patients, both adult and paediatric, was similar to that among patients receiving placebo. There were no clinically relevant differences in individual adverse events, including infectious upper respiratory conditions and transaminase elevations, between montelukast and placebo groups. Discontinuations due to adverse events occurred with similar frequencies during placebo, montelukast and inhaled beclomethasone therapy. No dose-related adverse effects of montelukast were observed in adults treated with dosages as high as 200 mg per day (20 times the recommended dose) for 5 months. This tolerability profile montelukast observed in clinical trials has been generally reflected in the post-marketing safety experience seen to date. These data indicate a tolerability profile for montelukast similar to placebo during both short-term and long-term administration, even at doses substantially higher than the recommended clinical dose of 10 mg once daily for adults and 5 mg once daily for children aged 6-14 years. [\hyperlink{Belbuca}{PMID: 11167954}, W Storms et al., 2001]

\hypertarget{pmid_27249433}{T}he aim of the study is to evaluate the effect of Montelukast - leukotriene inhibitor in children population with risk of bronchial asthma. The research was conducted at LTD. Kutaisi Children primary care unit \#3. The data were collected from January 2013 till January 2016. 104 patients (5-18 year, 43 girl, 61 boy), with potential risk of bronchial asthma were involved into the research, 47 (45\%) patients out of 104 were considered as a real risk for asthma, based on Peak Expiratory Flow (PEF) and spirometry results. Patients with risk of asthma were grouped according to the method of treatment (monotherapy with inhaled glycocorticoid and inhaled glycocorticoid combined with leukotriene inhibitor). Descriptive statistics methods were used to characterize each variable. Our results indicate on positive influence of montelukast - selective leukotrien inhibitor in treatment of children with various forms of asthma.  [\hyperlink{Belbuca}{PMID: 27249433}, I Pkhakadze et al., 2016] Infant botulism is the most common form of human botulism in Argentina and the United States. BabyBIG (botulism immune globulin intravenous [human]) is the antitoxin of choice for specific treatment of infant botulism in the United States. However, its high cost limits its use in many countries. We report here the effectiveness and safety of equine botulinum antitoxin (EqBA) as an alternative treatment. We conducted an analytical, observational, retrospective, and longitudinal study on cases of infant botulism registered in Mendoza, Argentina, from 1993 to 2007. We analyzed 92 medical records of laboratory-confirmed cases and evaluated the safety and efficacy of treatment with EqBA. Forty-nine laboratory-confirmed cases of infant botulism demanding admission in intensive care units and mechanical ventilation included 31 treated with EqBA within the 5 days after the onset of signs and 18 untreated with EqBA. EqBA-treated patients had a reduction in the mean length of hospital stay of 23.9 days (P = 0.0007). For infants treated with EqBA, the intensive care unit stay was shortened by 11.2 days (P = 0.0036), mechanical ventilation was reduced by 11.1 days (P = 0.0155), and tube feeding was reduced by 24.4 days (P = 0.0001). The incidence of sepsis in EqBA-treated patients was 47.3\% lower (P = 0.0017) than in the untreated ones. Neither sequelae nor adverse effects attributable to EqBA were noticed, except for one infant who developed a transient erythematous rash. These results suggest that prompt treatment of infant botulism with EqBA is safe and effective and that EqBA could be considered an alternative specific treatment for infant botulism when BabyBIG is not available. [\hyperlink{Belbuca}{PMID: 27249433}, Elida E Vanella de Cuetos et al., 2011]

\hypertarget{pmid_10922144}{T}o date, only one study of chronic use of a leukotriene receptor antagonist in children has been published. The efficacy and safety of montelukast in children 6-14 years of age with asthma (n = 336) was studied during an 8-week, double-blind, placebo-controlled trial. There was significantly greater improvement in forced expired volume in 1 sec (FEV(1)) from baseline for the montelukast group (8. 23\%) compared to the placebo group (3.58\%). There was a significant decrease in use of beta agonists for symptom relief and a significant decrease in the percentage of days and percentage of patients with asthma exacerbations. An asthma-specific quality of life questionnaire revealed significant overall improvement in quality of life and significant improvement in the quality of life domains for symptoms, activity, and emotions. Adverse effects were not significantly different for montelukast than for placebo, with the exception of allergic rhinitis which was more prevalent in the placebo group. A 6-month open follow-up of patients from the above study was undertaken. Effects of montelukast on FEV(1) were consistent over the 6 months, with the increase in FEV(1) not significantly different from a small control group treated with beclomethasone. Quality of life remained significantly improved throughout the open treatment period. In conclusion, leukotriene receptor antagonists are of value for the treatment of children with asthma. [\hyperlink{Belbuca}{PMID: 10922144}, A Becker et al., 2000]

\hypertarget{pmid_33844890}{I}n chronic asthma treatment, leukotriene receptor antagonists have been recommended, but it is not clear whether montelukast can be used in acute recurrent wheezing attacks in children. To investigate the safety and effectiveness of oral montelukast in addition to standard treatment in hospitalized children aged between 6 and 72 months with acute recurrent wheezing attacks. One hundred patients aged between 6 and 72 months who had wheezing attacks with clinical asthma scores (CAS) ≥3 and were hospitalized were included in this randomized, double-blind, placebo-controlled, parallel-group clinical trial. All the patients included in the study were given 0.15 mg/kg (maximum 5 mg) nebulized salbutamol (8 L/min and with 100\% O Total hospital length of stay (LOS) was not different between the montelukast and placebo groups (p = 0.981). There was no statistically significant difference between the two treatment groups in terms of discharge time, CAS, and oxygen saturation (p ≥ 0.05). Adding montelukast to standard treatment in patients hospitalized for moderate-to-severe wheezing attacks did not affect hospital LOS and CAS. [\hyperlink{Belbuca}{PMID: 33844890}, Emine Demet Akbaş et al., 2021]

\hypertarget{pmid_33600687}{T}o compare the efficacy and safety of Exubera 121 children were randomized to receive EXU or SC insulin, plus intermediate/ long-acting insulin for 12 weeks. Change in HbA Decreases from baseline HbA The efficacy and safety profiles shown in this study are the foundation for further investigation of EXU in this population. [\hyperlink{Belbuca}{PMID: 33600687}, Neil H White et al., 2020]

\hypertarget{pmid_18717242}{T}o compare the efficacy and safety of Exubera (EXU) with subcutaneous (SC) insulin in children, ages 6-11 years, with type 1 diabetes mellitus. 121 children were randomized to receive EXU or SC insulin, plus intermediate/long-acting insulin for 12 weeks. Change in HbA1c was the primary efficacy endpoint. Decreases from baseline HbA1c were comparable between treatment groups (difference between adjusted mean decrease from baseline [EXU-SC insulin], -0.23 [95\% CI, -0.49, 0.03]). Differences between groups on pulmonary function tests were small and not significant. Mild to moderate cough occurred in 24.6\% of EXU versus 6.8\% of SC insulin patients. The risk for hypoglycemia was comparable between EXU and SC insulin (relative risk 0.88 [95\% CI, 0.71, 1.11]). Increased insulin antibodies with EXU were not associated with clinical findings. The efficacy and safety profiles shown in this study are the foundation for further investigation of EXU in this population. [\hyperlink{Belbuca}{PMID: 18717242}, Neil H White et al., 2008]

\hypertarget{pmid_11759189}{T}his 6-month, open-label extension study of a previously described base study compared oral montelukast with inhaled beclomethasone in terms of safety, forced expiratory volume in one second (FEV1) measurements, parent and patient satisfaction with treatment, asthma-related medical resource utilization, school absenteeism, and parental work loss in children with asthma. A total of 124 of 266 asthmatic children, 6 to 11 years of age, who enrolled in the base study entered a 6-month open-label extension study (74 boys, 50 girls) and were re-randomized (2:1 ratio) to receive once-daily oral montelukast (n = 83) or inhaled beclomethasone 100 mcg three times daily (n = 41). Children were evaluated in the clinic prior to re-randomization (Month 0) and at regular visits at 1, 3, and 6 months. Children and their parents showed a significantly higher overall satisfaction for montelukast at 6 months than for inhaled beclomethasone (p = 0.001 and p < 0.05, respectively). According to parents, montelukast was more convenient (p < 0.001), less difficult to use (p = 0.005), and was used as instructed more of the time (p = 0.006) compared with beclomethasone. Oral corticosteroid use was similar in the montelukast (13\% of patients) and beclomethasone (17\%) treatment groups. The montelukast treatment group was more adherent with their regimen than the inhaled beclomethasone treatment group; almost twice as many children on montelukast compared with inhaled beclomethasone were highly compliant (82\% versus 45\%). The two study groups were similar with respect to overall safety, change in FEV1, asthma-related medical resource utilization, school absenteeism, and parental work loss. Montelukast represents a safe and effective asthma treatment regimen to which children with asthma are more likely to adhere. [\hyperlink{Belbuca}{PMID: 11759189}, J F Maspero et al., 2001]

\hypertarget{pmid_37815398}{A}ntimicrobial resistance increases infection morbidity in both adults and children, necessitating the development of new therapeutic options. Telavancin, an antibiotic approved in the United States for certain bacterial infections in adults, has not been examined in pediatric patients. The objectives of this study were to evaluate the short-term safety and pharmacokinetics (PK) of a single intravenous infusion of telavancin in pediatric patients. Single-dose safety and PK of 10 mg/kg telavancin was investigated in pediatric subjects >12 months to ≤17 years of age with known or suspected bacterial infection. Plasma was collected up to 24-h post-infusion and analyzed for concentrations of telavancin and its metabolite for noncompartmental PK analysis. Safety was monitored by physical exams, vital signs, laboratory values, and adverse events following telavancin administration. Twenty-two subjects were enrolled: 14 subjects in Cohort 1 (12-17 years), 7 subjects in Cohort 2 (6-11 years), and 1 subject in Cohort 3 (2-5 years). A single dose of telavancin was well-tolerated in all pediatric age cohorts without clinically significant effects. All age groups exhibited increased clearance of telavancin and reduced exposure to telavancin compared to adults, with mean peak plasma concentrations of 58.3 µg/mL (Cohort 1), 60.1 µg/mL (Cohort 2), and 53.1 µg/mL (Cohort 3). A 10 mg/kg dose of telavancin was well tolerated in pediatric subjects. Telavancin exposure was lower in pediatric subjects compared to adult subjects. Further studies are needed to determine the dose required in phase 3 clinical trials in pediatrics. [\hyperlink{Belbuca}{PMID: 37815398}, John S Bradley et al., 2023]

\hypertarget{pmid_28971612}{M}ontelukast, a selective leukotriene receptor antagonist, is recommended in guidelines for the treatment of asthma in both children and adults. However, its effectiveness is debated, and recent studies have reported several adverse events such as neuropsychiatric disorders and allergic granulomatous angiitis. This study aims to obtain more insight into the safety profile of montelukast and to provide prescribing physicians with an overview of relevant adverse drug reactions in both children and adults. We retrospectively studied all adverse drug reactions on montelukast in children and adults reported to the Netherlands Pharmacovigilance Center Lareb and the WHO Global database, VigiBase [\hyperlink{Belbuca}{PMID: 28971612}, Meindina G Haarman et al., 2017] Using the Chick Embrotoxicity Screening Test (CHEST), two samples of bilirubin of different commercial origin were tested on 2, 3 and 4- day old chick embryos. Water soluble Bilirubin Lachema (containing 20 mg albumin per 1 ml) had no teratogenic effect. On the opposite, Bilirubin Merck (containing 8 mg albumin per 1 ml) manifested an apparent teratogenic potential when single doses 0.2 and 0.6 micrograms were administered intraamniotically on day 4. Dose-dependent malformations of brain and eyes, cleft beak and reduction deformities of limbs were observed. No such effects could be produced by administration of Bilirubin Merck on either day 2 and 3. A tentative explanation of the difference between teratogenic properties of Merck and Lachema bilirubin preparations may be sougth in the different proportion of the free and albumin bound fractions. [\hyperlink{Belbuca}{PMID: 28971612}, M Peterka et al., 1994]

\hypertarget{pmid_16802767}{A}ntileukotrienes and inhaled corticosteroids are asthma controller agents widely used in the treatment of pediatric asthma. To evaluate the effects of montelukast and beclomethasone on linear growth in prepubertal asthmatic children for 1 year. This was a 30-center study of boys (6.4-9.4 years old) and girls (6.4-8.4 years old) at Tanner stage I with mild, persistent asthma. After a placebo run-in period, 360 patients were randomized in equal ratios to double-blind, double-dummy treatment with 5 mg of montelukast, 200 microg of beclomethasone twice daily (positive control), or placebo for 56 weeks; 90\% of the patients completed the study. The primary end point was linear growth velocity, measured using a stadiometer. Linear growth rates were similar between the montelukast and placebo groups; the mean difference for the year was 0.03 cm. The mean growth rate with beclomethasone was significantly less than with placebo (-0.78 cm) or montelukast (0.81 cm) (P < .001 for both). Median percentage of days with beta-agonist use was greater with placebo (14.58\%) vs montelukast (10.55\%) or beclomethasone (6.65\%) (P < .05 for all). More patients used oral corticosteroid rescue with placebo (34.7\%) than with montelukast (25.0\%) or beclomethasone (23.5\%). An imbalance in bone marker levels was seen with beclomethasone but not with montelukast. In prepubertal asthmatic children, montelukast did not affect linear growth, whereas the growth rate with beclomethasone was significantly decreased during 1 year of treatment. [\hyperlink{Belbuca}{PMID: 16802767}, Allan B Becker et al., 2006] 6-mercaptopurine (6-MP), a key drug for treatment of acute lymphoblastic leukemia (ALL), has until recently had no adequate formulation for pediatric patients. Several approaches have been taken but the only oral paraben-free 6-MP liquid formulation named Loulla was developed and evaluated in the target population. Preclinical and clinical evaluations were performed according to a Pediatric Investigation Plan, in order to apply for a Pediatric Use Marketing Authorization. The pre-clinical study assessed the maximum tolerated dosage-volume and evaluated local mucosal toxicity of 28 daily administrations in treated compared to controls gold hamsters. The multi-centre clinical study was single-dose, open-label, crossover trial, conducted in 15 ALL children during maintenance therapy. The bioavailability and palatability of a single 50mg fixed dose of Loulla compared to 50mg registered tablets were evaluated in a random order on two consecutive days. Seven blood samples over 9h were obtained each day to determine 6-MP pharmacokinetic parameters, including Tmax, Cmax, AUC0-9 and AUC0-∞. A questionnaire adapted to children testing Loulla palatability and preference for either Loulla or the usual 6-MP tablet was completed. Occurrence of adverse events was determined at study visits by vital sign measurements, patient's spontaneous reporting, investigator's questioning and clinical examination. The preclinical study in gold hamsters showed that dosage-volume of 75 mg/kg/day was well tolerated. The relative bioavailability of liquid Loulla formulation compared to the reference presentation is 76\% for AUC0-9 and AUC0-∞ and 80\% for Cmax. The taste of Loulla and the mouth feeling after ingestion compare favorably to the tablet. No adverse event occurred. Pharmacokinetic, palatability and safety data support the use of Loulla in children. [\hyperlink{Belbuca}{PMID: 16802767}, Adam de Beaumais Tiphaine et al., 2016]

\hypertarget{pmid_17934951}{L}imited information exists on the toxicity of pediatric ingestions of the drug montelukast used in the treatment of chronic asthma. All ingestions of montelukast involving children age 0-5 yr reported to Texas poison control centers during 2000-2005 were retrieved. For a subset of cases where the final medical outcome and dose in milligrams or milligrams per kilogram were known, the pattern of exposures by final medical outcome and management site was evaluated. There was a total of 3698 cases. Of those cases with a known final medical outcome and dose, the mean dose in milligrams was 42.5 mg (range 0.4-536 mg) and the mean dose in milligrams per kilogram was 3.36 mg/kg (range 0.18-33.71 mg/kg). The final medical outcome was no observed effect in 95\% of the cases and minor effect in the remainder of the cases. The patient was managed on site in 80\% of the cases. The proportion of cases with a minor effect increased from 5\% for ingested dose of < or = 100 mg to 10\% for > 100 mg but was 5\% for dose < or = 5 mg/kg and > 5 mg/kg. The proportion of cases managed with health care facility involvement increased from 15\% for ingested dose of < or = 100 mg to 56\% for > 100 mg and rose from 10\% for dose < or = 5 mg/kg to 47\% for dose > 5 mg/kg. Pediatric montelukast ingestions of doses up to 536 mg or 33.71 mg/kg do not appear likely to result in serious adverse effects and usually can be managed at home. [\hyperlink{Belbuca}{PMID: 17934951}, Mathias B Forrester et al., 2007]

\hypertarget{pmid_17766511}{A} recurring epidemic of asthma exacerbations in children occurs annually in September in North America when school resumes after summer vacation. Our goal was to determine whether montelukast, added to usual asthma therapy, would reduce days with worse asthma symptoms and unscheduled physician visits of children during the September epidemic. A total of 194 asthmatic children aged 2 to 14 years, stratified according to age group (2-5, 6-9, and 10-14 years) and gender, participated in a double-blind, randomized, placebo-controlled trial of the addition of montelukast to usual asthma therapy between September 1 and October 15, 2005. Children randomly assigned to receive montelukast experienced a 53\% reduction in days with worse asthma symptoms compared with placebo (3.9\% vs 8.3\%) and a 78\% reduction in unscheduled physician visits for asthma (4 [montelukast] vs 18 [placebo] visits). The benefit of montelukast was seen both in those using and not using regular inhaled corticosteroids and among those reporting and not reporting colds during the trial. There were differences in efficacy according to age and gender. Boys aged 2 to 5 years showed greater benefit from montelukast (0.4\% vs 8.8\% days with worse asthma symptoms) than did older boys, whereas among girls the treatment effect was most evident in 10- to 14-year-olds (4.6\% [montelukast] vs 17.0\% [placebo]), with nonsignificant effects in younger girls. Montelukast added to usual treatment reduced the risk of worsened asthma symptoms and unscheduled physician visits during the predictable annual September asthma epidemic. Treatment-effect differences observed between age and gender groups require additional investigation. [\hyperlink{Belbuca}{PMID: 17766511}, Neil W Johnston et al., 2007]

\hypertarget{pmid_31920289}{M}ontelukast, a potent oral selective leukotriene-receptor antagonist, inhibits the action of cysteinyl-leukotriene in patients with asthma. Although pharmacokinetic studies of montelukast have been reported in Caucasian adults and children, and showed large inter-individual variability on pharmacokinetics, none of these studies has been explored in Chinese children. Given the potential inter-ethnic difference, the purpose of the present study was to evaluate the effects of developmental factors and pharmacogenetics of CYP2C8 and SLCO2B1 on montelukast clearance in Chinese pediatric patients. After the administration of montelukast, blood samples were collected from children and plasma concentrations were determined using an adapted micro high-performance liquid chromatography coupled with the fluorescence detection (HPLC-FLD) method. A previously published pharmacokinetic model was validated using the opportunistic pharmacokinetic samples, and individual patient's clearance was calculated using the validated model. Population pharmacokinetic analysis was performed using a nonlinear mixed-effects model approach (NONMEM V 7.2.0) and variants of CYP2C8 and SLCO2B1 were genotyped. Fifty patients (age range: 0.7-10.0 years) with asthma were enrolled in this study. The clearance of montelukast was significantly higher in children with the SLCO2B1 c.935GA and c.935AA genotypes compared with that of children with the SLCO2B1 c.935GG genotype (0.94 ± 0.26 versus 0.77 ± 0.21, p = 0.020). The patient's weight was also found to be significantly corrected with montelukast clearance (p <0.0001). The developmental pharmacology of montelukast in Chinese children was evaluated. Weight and SLCO2B1 genotype were found to have independent significant impacts on the clearance of montelukast. [\hyperlink{Belbuca}{PMID: 31920289}, Qian Li et al., 2019]

\hypertarget{pmid_2019938}{T}o test whether nebulized salbutamol (albuterol) is safe and efficacious for the treatment of young children with acute bronchiolitis, we enrolled 83 children (median age 6 months, range 1 to 21 months) in a randomized, double-blind clinical trial. Participants received two treatments at 30-minute intervals of either nebulized salbutamol (0.10 mg/kg in 2 ml 0.9\% saline solution) or a similar volume of 0.9\% saline solution placebo. Outcome measures were the respiratory rate, pulse oximetry, and a clinical score based on the degree of wheezing and retractions. Patients in the salbutamol arm had significantly greater improvement in clinical scores after the initial treatment (p = 0.04). There was no difference between the groups in oxygen saturation (p = 0.74); patients treated with salbutamol had a small increase in heart rate after two treatments (159 +/- 16 vs 151 +/- 16; p = 0.03). We conclude that salbutamol is safe and effective for the initial treatment of young children with acute bronchiolitis. [\hyperlink{Belbuca}{PMID: 2019938}, T P Klassen et al., 1991]

\hypertarget{pmid_37099340}{W}e studied the safety and efficacy of blinatumomab, a bispecific T-cell engager molecule targeting CD19, in infants with  The median follow-up was 26.3 months (range, 3.9 to 48.2). All 30 patients received the full course of blinatumomab. No toxic effects meeting the definition of the primary end point occurred. Ten serious adverse events were reported (fever [4 events], infection [4], hypertension [1], and vomiting [1]). The toxic-effects profile was consistent with that reported in older patients. A total of 28 patients (93\%) either were MRD-negative (16 patients) or had low levels of MRD (<5×10 Blinatumomab added to Interfant-06 chemotherapy appeared to be safe and had a high level of efficacy in infants with newly diagnosed  [\hyperlink{Belbuca}{PMID: 37099340}, Inge M van der Sluis et al., 2023] The tolerability of a medication, especially in children with asthma, is linked to a number of key factors. These include clinical effectiveness, adverse effects, frequency of drug regimen, ease and route of administration. and taste. Montelukast is unusual in that, in most countries, a licence for children aged > or =6 years was granted at the same time as the adult licence. This is related to a variety of evidence. which includes pharmacological and adult studies suggesting the specificity and safety of the drug at many times the licensed dose, and a tolerability profile similar to that with placebo or inhaled corticosteroids in both adult and paediatric studies. The most common adverse effects in paediatric studies were headache, asthma and upper respiratory tract infection at rates not statistically significantly different from those with placebo. Up to July 1999, more than 2 million patients worldwide have received montelukast, of whom nearly 220,000 have received the paediatric formulation. In the UK, one prescribing database suggests that, of children who commenced montelukast therapy, less than 25\% discontinued the drug. This implies that montelukast is effective and well tolerated in most children. Adverse effect monitoring by regulatory bodies has revealed little that would not be expected on the basis of the results of clinical trials. Montelukast has been associated with Churg-Strauss syndrome in a very small number of adults. In most. the syndrome was associated with corticosteroid withdrawal, which may have unmasked the condition. Churg-Strauss syndrome has not been reported in children. Its clinical effectiveness, lack of major adverse effects, oral route of administration, palatability and the once-daily regimen combine to make montelukast a generally well tolerated medication in children. [\hyperlink{Belbuca}{PMID: 37099340}, D Price et al., 2000]

\hypertarget{pmid_2292542}{T}he Cystic Fibrosis Clinic at the Royal Belfast Hospital for Sick Children has treated 31 children with ciprofloxacin, for serious pseudomonas infection in cystic fibrosis, and carefully monitored the safety and acceptability of the drug. Initially, eight very ill children were treated on a named-patient basis, with an encouraging clinical response and few adverse effects. Children aged 10-18 years were included in a study of four consecutive exacerbations of respiratory disease, comparing (i) oral ciprofloxacin in each episode with (ii) ciprofloxacin alternating with intravenous azlocillin and tobramycin. Other children with cystic fibrosis were subsequently treated with ciprofloxacin, as the need arose. In all the groups very few adverse reactions were found; in particular only one child developed arthralgia. A total of 202 children in the UK have been treated with ciprofloxacin on a named-patient basis, and their clinicians have reported 46 adverse events that may have been drug-related. Overall ciprofloxacin appears to be safe and effective in children but concern about the possible occurrence of arthropathy remains and long term follow-up of these children may be necessary. [\hyperlink{Belbuca}{PMID: 2292542}, A Black et al., 1990]

\hypertarget{pmid_19449366}{M}ontelukast is a potent leukotriene-receptor antagonist administered once daily that provides clinical benefit in the treatment of asthma and allergic rhinitis in children and adults. Because of its wide use as a pediatric controller, there is a need for a further review of the safety and tolerability of montelukast in children. To summarize safety and tolerability data for montelukast from previously reported as well as from unpublished placebo-controlled, double-blind, pediatric studies and their active-controlled open-label extension/extended studies. These studies evaluated 2,751 pediatric patients 6 months to 14 years of age with persistent asthma, intermittent asthma associated with upper respiratory infection, or allergic rhinitis. These patients were enrolled in seven randomized, placebo-controlled, double-blind registration and post-registration studies and three active-controlled open-label extension/extended studies conducted by Merck Research Laboratories between 1995 and 2004. Montelukast was well tolerated in all studies. Clinical and laboratory adverse experiences for patients treated with montelukast were generally mild and transient. The most frequent clinical adverse events for all treatments (placebo, montelukast, active control/usual care) in virtually all studies were upper respiratory infection, worsening asthma, pharyngitis, and fever. The clinical and laboratory safety profile for montelukast was similar to that observed for placebo or active control/usual care therapies. The safety profile of montelukast did not change with long-term use. [\hyperlink{Belbuca}{PMID: 19449366}, Hans Bisgaard et al., 2009]

\hypertarget{pmid_10825791}{T}o study of antileukotrienes with participation in the genesis of the asthma appeared in the last years. 17 children with a mean age of 10.29 years, male seven and ten girls. They were tried with montelukast 5 mg/day the 12-year-old smellers and with 10 mg/day the such age greats, during a period middle to 5.7 months. Ten of they had a concomitant treatment fluticasona, budesonide, salmeterol or formoterol. Was valued the clinic improvement, the saving in medication and the sound effects. a) Clinic improvement in 82.3\%; b) elimination of medication associated in 90\% and c) sound effects: in a case drowsiness and in other abdominal pain. Montelukast seems be a very useful product in the treatment of the bronchial asthma in children since generates a saving of corticoids and longtime beta 2 in the mild or moderate asthma, associate or not with the inmunoterapia. [\hyperlink{Belbuca}{PMID: 10825791}, O Pérez et al., ]

\section*{Clindamycin Palmitate Hydrochloride}
\subsection*{Result}
\subsubsection*{Answer}

Yes (children, including those under 25 kg and with osteomyelitis, likely ages 1 year and up)
Unknown (infants and neonates, and children under 1 year for the palmitate form specifically)

\subsubsection*{{Explanation}}
\hypertarget{Clindamycin Palmitate Hydrochloride}
A review of the available abstracts reveals several studies that specifically address the use of clindamycin palmitate hydrochloride in children, as well as studies on clindamycin in general pediatric populations. Below is a summary of the relevant evidence, organized by age range:

1. Children with Upper Respiratory Illnesses (No specific lower age limit, but includes children under 25 kg):
- A randomized study compared clindamycin palmitate to potassium phenoxymethyl penicillin in 103 children with upper respiratory illnesses and pharyngeal group A streptococci. Clindamycin palmitate was administered at 75 mg (for <25 kg) or 150 mg (for ≥25 kg) for 10 days. The study reported that clindamycin palmitate was as effective as penicillin in eradicating group A streptococci. Possible drug-related rashes were observed in 8 of 52 clindamycin palmitate-treated patients, but no serious adverse events were reported. The authors concluded that clindamycin palmitate should not be preferred to penicillin in non-allergic patients due to rash tendency and higher cost, but did not raise safety concerns that would preclude its use in children [\hyperlink{pmid_4208902}{PMID: 4208902}, M Stillerman et al., 1973].

2. Children with Osteomyelitis (No specific lower age limit stated):
- In a study of 29 children with osteomyelitis (25 acute, 4 chronic), clindamycin phosphate was given intravenously for \textasciitilde{}3 weeks, followed by oral clindamycin palmitate at 30 mg/kg/day for an additional 6 weeks. The study reported good to excellent clinical and bacteriologic results, with no diarrhea or enterocolitis observed despite high doses and prolonged therapy. Long-term evaluation showed satisfactory progress in all patients [\hyperlink{pmid_910760}{PMID: 910760}, W Rodriguez et al., 1977].

3. Infants and Neonates:
- Several studies address clindamycin (not specifically the palmitate form) in infants and neonates, including pharmacokinetic and safety studies. For example, a retrospective cohort study of 4089 infants (receiving clindamycin before postnatal day 121) found that clindamycin exposure was not associated with increased odds of death or serious adverse events, though a marginal increase in necrotizing enterocolitis was noted at the highest exposures [\hyperlink{pmid_31725114}{PMID: 31725114}, Rachel G Greenberg et al., 2020]. Another study of intravenous clindamycin phosphate in 40 children less than 1 year of age provided dosing recommendations based on age and weight, with no major safety concerns reported [\hyperlink{pmid_6470871}{PMID: 6470871}, M J Bell et al., 1984]. However, these studies do not specifically address clindamycin palmitate hydrochloride.

4. Children with Malaria (Ages 1–14 years):
- A study of 51 pediatric outpatients aged 1 to 14 years used a combination of fosmidomycin and clindamycin (not specified as palmitate) for malaria. The combination was generally well tolerated, but relatively high rates of neutropenia (16\%) and significant hemoglobin drops (14\%) were observed. Efficacy was lower in children aged 1–2 years, and the authors called for further studies in younger children [\hyperlink{pmid_16870763}{PMID: 16870763}, Steffen Borrmann et al., 2006].

Summary:
- There are targeted studies of clindamycin palmitate hydrochloride in children (including those under 25 kg and those with osteomyelitis), which affirm its safety in these populations, with rash being the most notable adverse event. No serious or life-threatening adverse events were reported in these studies. For infants and neonates, while clindamycin in general has been studied, there is no direct evidence for the palmitate form. For children aged 1–14 years, clindamycin (not specifically palmitate) was generally well tolerated, but some hematologic adverse events were noted in combination therapy.

Therefore, based on the abstracts, clindamycin palmitate hydrochloride is affirmed as safe for use in children (including those under 25 kg and those with osteomyelitis), with rash as a possible side effect. Safety in infants and neonates for the palmitate form is unknown, as is safety in children under 1 year for the palmitate form specifically.

\subsection*{Abstracts}
\hypertarget{pmid_4208902}{C}lindamycin palmitate and potassium phenoxymethyl penicillin were evaluated in 103 children with upper respiratory illnesses and pharyngeal group A streptococci, from November 1970 to July 1971. The children were assigned randomly by weight to one of the antibiotic regimens given orally for 10 days. Clindamycin palmitate and potassium phenoxymethyl penicillin dosages were 75 and 125 mg, respectively, in 5 ml tid for children weighing less than 25 kg, and 150 and 250 mg, respectively, in 10 ml bid for children weighing 25 kg or more. Recurrences of the original streptococcal group A, M, and T types within 3 weeks after the end of treatment were classified as failures. The failure rates were: clindamycin palmitate, 10\% (5 of 52), and potassium phenoxymethyl penicillin, 18\% (9 of 51). Possible drug-related rashes were observed in 8 of 52 clindamycin palmitate-treated patients. The geometric mean minimal inhibitory concentrations of clindamycin and penicillin against 103 isolates of group A streptococci were 0.033 and 0.007 mug/ml, respectively. The serum concentrations about 70 min after ingesting 150 mg of clindamycin palmitate averaged 3.8 mug/ml and after 250 mg of potassium phenoxymethyl penicillin averaged 0.9 mug/ml. Clindamycin palmitate was as effective as potassium phenoxymethyl penicillin in eradicating group A streptococci from the pharynx in tid and bid regimens. Nevertheless, because of its rash-producing tendency in some patients and higher cost, clindamycin palmitate should not be preferred to penicillin for treatment of streptococcal sore throat in the non-penicillin-allergic patient. [\hyperlink{Clindamycin Palmitate Hydrochloride}{PMID: 4208902}, M Stillerman et al., 1973]

\hypertarget{pmid_910760}{C}lindamycin phosphate was used in the treatment of 29 children with osteomyelitis of whom 25 had an acute and four a chronic type of infection. The usual dose was 50 mg/kg/day intravenously for approximately three weeks followed by oral clindamycin palmitate at home in a dose of 30 mg/kg/day for an additional six weeks. Staphylococcus aureus was isolated in 22 of 29 cases: 96\% of strains were penicillin resistant. The clinical and bacteriologic results in the present series were good to excellent. There was prompt clinical and bacteriologic response shortly after initiation of clindamycin therapy. Good bone penetration of the drug was observed. Long-term evaluation revealed satisfactory clinical and roentgenographic progress in all patients. No diarrhea or manifestations of enterocolitis appeared in any patient in spite of high doses of the drug for intervals up to nine weeks. [\hyperlink{Clindamycin Palmitate Hydrochloride}{PMID: 910760}, W Rodriguez et al., 1977]

\hypertarget{pmid_31725114}{D}espite the absence of adequate safety or efficacy data, clindamycin is widely prescribed in the neonatal intensive care unit. We evaluated the association between clindamycin exposure and adverse events, as well as antibiotic effectiveness in infants. This was a retrospective cohort study of infants receiving clindamycin before postnatal day 121 who were discharged from a Pediatrix Medical Group neonatal intensive care unit (1997-2015). Using a previously developed pharmacokinetic model, we performed simulations to predict clindamycin exposure based on available dosing data. We used multivariable logistic regression to evaluate the association between clindamycin exposure and safety outcomes during and after clindamycin therapy. We reported the proportion of infants with methicillin-resistant Staphylococcus aureus (MRSA) bacteremia and clearance of MRSA bacteremia. A total of 4089 infants received clindamycin at a median (25th-75th percentile) dose of 15 mg/kg/d (12-16). Clearance increased with older gestational age. Infants with the highest total clindamycin exposure had marginally increased odds of necrotizing enterocolitis within 7 days (adjusted odds ratio = 1.95 [1.04-3.63]), but exposure was not associated with death, sepsis, seizures, intestinal perforation or intestinal strictures. Of 25 infants who had MRSA bacteremia, 19 (76\%) cleared the infection by the end of the clindamycin course. Higher clindamycin exposure was not associated with increased odds of death or nonlaboratory adverse events. The use of pharmacokinetic models combined with available electronic health record data offers a valuable, cost-effective approach to analyzing the safety and effectiveness of drugs in infants when large-scale trials are not feasible. [\hyperlink{Clindamycin Palmitate Hydrochloride}{PMID: 31725114}, Rachel G Greenberg et al., 2020]

\hypertarget{pmid_26926644}{C}lindamycin may be active against methicillin-resistant Staphylococcus aureus, a common pathogen causing sepsis in infants, but optimal dosing in this population is unknown. We performed a multicenter, prospective pharmacokinetic (PK) and safety study of clindamycin in infants. We analyzed the data using a population PK analysis approach and included samples from two additional pediatric trials. Intravenous data were collected from 62 infants (135 plasma PK samples) with postnatal ages of <121 days (median [range] gestational age of 28 weeks [23 to 42] and postnatal age of 17 days [1 to 115]). In addition to body weight, postmenstrual age (PMA) and plasma protein concentrations (albumin and alpha-1 acid glycoprotein) were found to be significantly associated with clearance and volume of distribution, respectively. Clearance reached 50\% of the adult value at PMA of 39.5 weeks. Simulated PMA-based intravenous dosing regimens administered every 8 h (≤32 weeks PMA, 5 mg/kg; 32 to 40 weeks PMA, 7 mg/kg; >40 to 60 weeks PMA, 9 mg/kg) resulted in an unbound, steady-state concentration at half the dosing interval greater than a MIC for S. aureus of 0.12 μg/ml in >90\% of infants. There were no adverse events related to clindamycin use. (This study has been registered at ClinicalTrials.gov under registration no. NCT01728363.). [\hyperlink{Clindamycin Palmitate Hydrochloride}{PMID: 26926644}, Daniel Gonzalez et al., 2016]

\hypertarget{pmid_24447296}{C}hloral hydrate is the most commonly used sedative for paediatric diagnostic procedures in China with a success rate of around 80\%. Intranasal dexmedetomidine is used for rescue sedation in our centre. This prospective investigation evaluated 213 children aged one month to 10 years who were not adequately sedated following administration of chloral hydrate. Children were randomly assigned to receive rescue intranasal dexmedetomidine at 1 μg.kg(-1) (group 1), 1.5 μg.kg(-1) (group 2) or 2 μg.kg(-1) (group 3). The sedation level was assessed every 10 min using a modified observer's assessment of alertness/sedation scale. Successful rescue sedation in groups 1, 2 and 3 were 56 (83.6\%), 66 (89.2\%) and 51 (96.2\%), respectively. Increasing the rescue dose was associated with an increased success rate with an odds ratio of 4.12 (95\% CI 1.13-14.98), p = 0.032. We conclude that intranasal dexmedetomidine is effective for sedation in children who do not respond to chloral hydrate.  [\hyperlink{Clindamycin Palmitate Hydrochloride}{PMID: 24447296}, B L Li et al., 2014] Chloral hydrate is commonly used to sedate children for painless procedures. Children may recover more quickly after sedation with dexmedetomidine, which has a shorter half-life. We randomly allocated 196 children to chloral hydrate syrup 50 mg.kg [\hyperlink{Clindamycin Palmitate Hydrochloride}{PMID: 24447296}, V M Yuen et al., 2017] Sedation is often required for children undergoing diagnostic procedures. Chloral hydrate has been one of the sedative drugs most used in children over the last 3 decades, with supporting evidence for its efficacy and safety. Recently, chloral hydrate was banned in Italy and France, in consideration of evidence of its carcinogenicity and genotoxicity. Dexmedetomidine is a sedative with unique properties that has been increasingly used for procedural sedation in children. Several studies demonstrated its efficacy and safety for sedation in non-painful diagnostic procedures. Dexmedetomidine's impact on respiratory drive and airway patency and tone is much less when compared to the majority of other sedative agents. Administration via the intranasal route allows satisfactory procedural success rates. Studies that specifically compared intranasal dexmedetomidine and chloral hydrate for children undergoing non-painful procedures showed that dexmedetomidine was as effective as and safer than chloral hydrate. For these reasons, we suggest that intranasal dexmedetomidine could be a suitable alternative to chloral hydrate. [\hyperlink{Clindamycin Palmitate Hydrochloride}{PMID: 24447296}, Giorgio Cozzi et al., 2017]

\hypertarget{pmid_24949994}{C}lindamycin is commonly prescribed to treat children with skin and skin-structure infections (including those caused by community-acquired methicillin-resistant Staphylococcus aureus (CA-MRSA)), yet little is known about its pharmacokinetics (PK) across pediatric age groups. A population PK analysis was performed in NONMEM using samples collected in an opportunistic study from children receiving i.v. clindamycin per standard of care. The final model was used to optimize pediatric dosing to match adult exposure proven effective against CA-MRSA. A total of 194 plasma PK samples collected from 125 children were included in the analysis. A one-compartment model described the data well. The final model included body weight and a sigmoidal maturation relationship between postmenstrual age (PMA) and clearance (CL): CL (l/h) = 13.7 × (weight/70)(0.75) × (PMA(3.1)/(43.6(3.1) + PMA(3.1))); V (l) = 61.8 × (weight/70). Maturation reached 50\% of adult CL values at \textasciitilde{}44 weeks PMA. Our findings support age-based dosing.  [\hyperlink{Clindamycin Palmitate Hydrochloride}{PMID: 24949994}, D Gonzalez et al., 2014] To examine whether three cycles of a low-intensity chemotherapy consisting of cyclophosphamide [500 mg/m(2) - day 1], vinblastine [6 mg/m(2) - days 1 and 8] and prednisolone [40 mg/m(2) - days 1-7] (CVP) is safe and therapeutically effective in children and adolescents with early stage nodular lymphocyte predominant Hodgkin lymphoma [nLPHL]. Fifty-five children and adolescents with early stage nLPHL [median age 13 years, range 4-17 years] diagnosed between June 2005 and October 2010 in the UK and France are the subjects of this report. Staging investigations included conventional cross sectional as well as 18 fluro-deoxyglucose [FDG] PET imaging. Histology was confirmed as nLPHL by an expert pathology panel. Of the 45 patients, who received CVP as first line treatment, 36 [80\%, 95\% Confidence Interval [CI]: (68; 92)] either achieved a complete remission [CR] or CR unconfirmed [CRu], the remaining nine patients achieved a partial response. All nine subsequently achieved CR with salvage chemotherapy [n=7] or radiotherapy [n=2]. Ten patients received CVP at relapse after primary treatment that consisted of surgery alone and all achieved CR. To date, only three patients have relapsed after CVP chemotherapy and all had received CVP as first line treatment at initial diagnosis. The 40-month freedom from treatment failure and overall survival for the entire cohort were 75.4\% (SE ± 6\%) and 100\%, respectively. No significant early toxicity was observed. Our results show that CVP is an effective chemotherapy regimen in children and adolescents with early stage nLPHL that is well tolerated with minimal acute toxicity. [\hyperlink{Clindamycin Palmitate Hydrochloride}{PMID: 24949994}, Ananth Shankar et al., 2012]

\hypertarget{pmid_32373914}{T}o evaluate the practice and attitude of pediatrics nephrologists about cinacalcet use in children. An electronic structured questionnaire was answered by pediatric nephrologists practicing in the Kingdom of Saudi Arabia (KSA) and Gulf Council countries (GCC). A total of 42  pediatric nephrologists responded, of them, 42\% used cinacalcet for young children ≤5 years of age and 79\% used for children. There were wide variations in the method of administration (examples: crushed, divided, whole tablets), monitoring, doses and response definition, and follow-up. No serious complications after starting cinacalcet was observed in 50\%, while 40\% reported various complications, mainly hypocalcemia (70\%). Cinacalcet was stopped without achieving the target parathyroid hormone in more than half (55\%) of children because of intractable adverse effects (40\%), poor response (30\%), non-adherence (25\%), or high cost (5\%). Cinacalcet is used by the majority of pediatric nephrologists in KSA and GCC. A standard clinical guideline is needed to be followed by all users. [\hyperlink{Clindamycin Palmitate Hydrochloride}{PMID: 32373914}, Rafif A Al-Ahmad et al., 2020]

\hypertarget{pmid_10724028}{C}hildren infected with Chlamydia pneumoniae sometimes experience lower respiratory tract infections such as pneumonia and bronchitis. Although numerous anti-microbial compounds have been reported to be active against the organism, most of them have not been in a clinical trial in infants and children with C. pneumoniae infection. Clarithromycin has been shown to express anti-chlamydial effects in vitro. In this study, we evaluated the clinical anti-C. pneumoniae properties of clarithromycin in children with mainly lower respiratory tract infection. We administered clarithromycin orally to 21 infants and children at a dose of 10-15 mg/kg/day divided into two or three doses for 4-21 days. Clinical symptoms, roentgenographic and laboratory abnormal findings improved. The overall clinical efficacy rate was 85.7\% (18 of 21 cases). Administration of clarithromycin was considered to be a suitable treatment for improving lower respiratory infections in infants and children caused by C. pneumoniae. [\hyperlink{Clindamycin Palmitate Hydrochloride}{PMID: 10724028}, K Numazaki et al., 2000]

\hypertarget{pmid_27463797}{C}lindamycin hydrochloride (CLH) is a clinically important oral antibiotic with wide spectrum of antimicrobial activity that includes gram-positive aerobes (staphylococci, streptococci etc.), most anaerobic bacteria, Chlamydia and certain protozoa. The current study was focused to develop a stabilised clindamycin encapsulated poly lactic acid (PLA)/poly (D,L-lactide-co-glycolide) (PLGA) nano-formulation with better drug bioavailability at molecular level. Various nanoparticle (NPs) formulations of PLA and PLGA loaded with CLH were prepared by solvent evaporation method varying drug: polymer concentration (1:20, 1:10 and 1:5) and characterised (size, encapsulation efficiency, drug loading, scanning electron microscope, differential scanning calorimetry [DSC] and Fourier transform infrared [FTIR] studies). The ratio 1:10 was found to be optimal for a monodispersed and stable nano formulation for both the polymers. NP formulations demonstrated a significant controlled release profile extended up to 144 h (both CLH-PLA and CLH-PLGA). The thermal behaviour (DSC) studies confirmed the molecular dispersion of the drug within the system. The FTIR studies revealed the intactness as well as unaltered structure of drug. The CLH-PLA NPs showed enhanced antimicrobial activity against two pathogenic bacteria Streptococcus faecalis and Bacillus cereus. The results notably suggest that encapsulation of CLH into PLA/PLGA significantly increases the bioavailability of the drug and due to this enhanced drug activity; it can be widely applied for number of therapies.  [\hyperlink{Clindamycin Palmitate Hydrochloride}{PMID: 27463797}, Pradipta Ranjan Rauta et al., 2016] Chloral hydrate has been used extensively to sedate children, but at Brooke Army Medical Center, other drug combinations were becoming increasingly popular due to a perception that chloral hydrate had a high rate of failure, especially with younger or neurologically impaired children. Therefore, 50 children were given the drug before a diagnostic study, and patient data and a sedation score were recorded on a worksheet. Of 50 children, 43 (86\%) were "successfully sedated" on the first attempt with no side effects. Children with neurologic disorders had a much greater (27\% vs 4\%) failure rate than "normal" children. The sedation rate did not significantly differ by age, sex, or initial drug dosage. The study suggest that chloral hydrate is a safe and effective oral sedative but that children with neurologic disorders may need alternative drugs for sedation. [\hyperlink{Clindamycin Palmitate Hydrochloride}{PMID: 27463797}, P D Rumm et al., 1990]

\hypertarget{pmid_7247166}{C}lindamycin hydrochloride hydrate (Cleocin), a semisynthetic antibiotic shown experimentally to be effective in ocular toxoplasmosis in the rabbit, was used in the treatment of four patients with active retinochoroiditis secondary to toxoplasmosis. The drug was administered subjunctivally on alternate days for 30 days. Both subjective and objective evidence indicated beneficial results in these patients during the first 30 days. One of the four did not respond during the first 30 days but did respond during an extended period. One of those who responded initially had exacerbations when the drug was stopped and required treatment. [\hyperlink{Clindamycin Palmitate Hydrochloride}{PMID: 7247166}, J G Ferguson et al., 1981]

\hypertarget{pmid_35049570}{C}lindamycin hydrochloride is a widely used antibiotic for topical use, but its main disadvantage is poor skin penetration. Therefore, new approaches in the development of clindamycin topical formulations are of great importance. We aimed to investigate the effects of the type of gelling agent (carbomer and sodium carmellose), and the type and concentration of bile acids as penetration enhancers (0.1\% and 0.5\% of cholic and deoxycholic acid), on clindamycin release rate and permeation in a cellulose membrane in vitro model. Eight clindamycin hydrogel formulations were prepared using a 2 [\hyperlink{Clindamycin Palmitate Hydrochloride}{PMID: 35049570}, Nebojša Pavlović et al., 2022] Chloral hydrate (CH) is an oral sedative widely used to sedate infants and young children during auditory brainstem response (ABR) testing. The aim of this study was to record effectiveness, complications and safety of CH as a sedative for ABR. From January of 2003 until December of 2007, 1903 children were tested for ABR, 568 of them being under the age of 6 months. CH (8\%) was used for sedation at a dose of 40 mg/kg with a repeat dose, if necessary, for an adequate sedation, in 20-30 min. We recorded the effectiveness of CH as a sedative for ABR examination, as well as all complications related to the use of CH such as vomiting, rash, hyperactivity, respiratory distress and apnea. The statistical method used was the absolute and percentage frequency distribution of the occurrences. Sedation with CH was necessary to perform testing in 1591 (83.6\%) of the examined children. However, in the population of the examined infants, only 341 (60\%) were sedated with CH, because the remaining 227 (40\%) fell asleep by themselves. Complications included hyperactivity in 152 children (8\%), minor respiratory distress in 10 children (0.4\%), vomiting in 217 children (11.4\%), apnea in 4 children (0.2\%) and rash in 10 children (0.4\%). The complications of hyperactivity, vomiting and rash resolved without any medical treatment. The apnea cases were managed effectively by supplying ventilation to the children via a mask in the presence of an anesthesiologist. The use of CH at a dose of 40 mg/kg up to 80 mg/kg is safe and effective when administered in a setting with adequate equipment and the presence of well trained personnel. [\hyperlink{Clindamycin Palmitate Hydrochloride}{PMID: 35049570}, Eirini Avlonitou et al., 2011]

\hypertarget{pmid_16870763}{F}osmidomycin plus clindamycin was shown to be efficacious in the treatment of uncomplicated Plasmodium falciparum malaria in a small cohort of pediatric patients aged 7 to 14 years, but more data, including data on younger children with less antiparasitic immunity, are needed to determine the potential value of this new antimalarial combination. We conducted a single-arm study to improve the precision of efficacy estimates for an oral 3-day fixed-ratio combination of fosmidomycin and clindamycin at 30 and 10 mg/kg of body weight, respectively, every 12 hours for the treatment of uncomplicated P. falciparum malaria in 51 pediatric outpatients aged 1 to 14 years. Fosmidomycin plus clindamycin was generally well tolerated, but relatively high rates of treatment-associated neutropenia (8/51 [16\%]) and falls of hemoglobin concentrations of > or =2 g/dl (7/51 [14\%]) are of concern. Asexual parasites and fever were cleared within median periods of 42 h and 38 h, respectively. All patients who could be evaluated were parasitologically and clinically cured by day 14 (49/49; 95\% confidence interval [CI], 93 to 100\%). The per-protocol, PCR-adjusted day 28 cure rate was 89\% (42/47; 95\% CI, 77 to 96\%). Efficacy appeared to be significantly reduced in children aged 1 to 2 years, with a day 28 cure rate of only 62\% for this small subgroup (5/8). The inadequate efficacy in children of <3 years highlights the need for continued systematic studies of the current dosing regimen, which should include randomized trial designs. [\hyperlink{Clindamycin Palmitate Hydrochloride}{PMID: 16870763}, Steffen Borrmann et al., 2006]

\hypertarget{pmid_25044425}{C}lindamycin hydrochloride belongs to the antibiotic family of lincomycin. It has the same antibacterial spectrum as lincomycin, but the antibacterial activity is four to eight times stronger than that of lincomycin. There have been some adverse reactions in clinical use of clindamycin hydrochloride and its finished drug products. The impurities in drugs are directly related to their safety. In this study, two unknown impurities were isolated from the raw material of clindamycin hydrochloride through various chromatographic methods. Their structures were identified as clindamycin isomer (impurity 1) and dehydroclindamycin (impurity 2) by mass spectrometry and NMR spectroscopy. Both of them were found for the first time. The two impurities exhibit a similar but lower antibacterial activity compared with clindamycin hydrochloride.  [\hyperlink{Clindamycin Palmitate Hydrochloride}{PMID: 25044425}, Qiushi Sun et al., 2014] Although chloral hydrate (CH) has been used as a sedative for decades, it is not widely accepted as a valid choice for ophthalmic examinations in uncooperative children. This study aimed to systematically review the literature on the drug's safety and efficacy. We searched PubMed, EMBASE, ISI Web of Science, Scopus, CENTRAL, Google Scholar and Trip database to 1 October 2015, using the keywords 'chloral hydrate', 'paediatric' and 'procedural sedation OR diagnostic sedation'. A meta-analysis of randomised controlled trials (RCTs) was performed. A total of 6961 articles were screened and 104 were included in the review. Thirteen of these concerned paediatric ophthalmic examination, while 13 others were RCTs and were meta-analysed. CH was reported to have been administered in a total of 24 265 sedation episodes in children aged from <1 month to 18 years. The meta-analysis showed CH had a higher OR (2.95, 95\% CI 1.09 to 7.99) for successful sedation compared to other sedatives, but significant limitations apply. The commonest reported adverse events (AE) were not serious (eg, paradoxical reaction or transient vomiting) and required no intervention. Severe AE, including two deaths, were related to comorbidity, overdose or aspiration. Despite the paucity of high quality evidence, the existing literature suggests that the use of CH for procedural sedation in children appears to be an effective alternative to general anaesthesia, and it can be safe when administered in the hospital setting with appropriate monitoring and vigilance for intervention. [\hyperlink{Clindamycin Palmitate Hydrochloride}{PMID: 25044425}, Asimina Mataftsi et al., 2017]

\hypertarget{pmid_18805603}{W}e sought to evaluate efficacy, safety, and tolerability of a combination of clindamycin phosphate 1.2\% and benzoyl peroxide 2.5\% (clindamycin-BPO 2.5\%) aqueous gel in moderate to severe acne vulgaris. A total of 2813 patients, aged 12 years or older, were randomized to receive clindamycin-BPO 2.5\%, individual active ingredients, or vehicle in two identical, double-blind, controlled 12-week, 4-arm studies evaluating safety and efficacy (inflammatory and noninflammatory lesion counts) using Evaluator Global Severity Score and subject self-assessment. Clindamycin-BPO 2.5\% demonstrated statistical superiority to individual active ingredients and vehicle in reducing both inflammatory and noninflammatory lesions and acne severity. Visibly greater improvement was observed by patients with clindamycin-BPO 2.5\% as early as week 2. No substantive differences were seen in cutaneous tolerability among treatment groups and less than 1\% of patients discontinued treatment because of adverse events. Data from controlled studies may differ from clinical practice. Clindamycin-BPO 2.5\% provides statistically significant greater efficacy than individual active ingredients and vehicle with a highly favorable safety and tolerability profile. [\hyperlink{Clindamycin Palmitate Hydrochloride}{PMID: 18805603}, Diane Thiboutot et al., 2008]

\hypertarget{pmid_2026812}{C}hloral hydrate is commonly used to sedate children before CT. However, no prospective study has been published of the safety and efficacy of chloral hydrate at high dose levels for children undergoing CT. We define high dose levels of oral chloral hydrate to be 80-100 mg/kg, with a maximum total dose of 2 g. High dose chloral hydrate sedation was administered orally to 295 children for 326 CT examinations. Adverse reactions occurred in 7\% of the children, with vomiting being the most common (4.3\% of children). Hyperactivity and respiratory symptoms each occurred in less than 2\% of children. Prolonged sedation ( greater than 2 h) was not encountered in our series. Sedation was successful in producing motion free CT examinations, so that in 303 (93\%) of the cases, no repeat CT scans were needed. We conclude that high dose oral chloral hydrate provides safe and effective sedation for children undergoing CT. [\hyperlink{Clindamycin Palmitate Hydrochloride}{PMID: 2026812}, S B Greenberg et al., ]

\hypertarget{pmid_16553852}{T}he effects of subinhibitory concentrations of clindamycin on the morphological, biochemical and genetic characteristics of species of the Bacteroides fragilis group isolated from children with diarrhea were determined. The minimal inhibitory and subinhibitory concentrations for clindamycin were determined. Minimal inhibitory concentration values ranging from 0.25 to 512 microg mL(-1) were observed. Cultures grown with clindamycin were used to determine the macroscopic morphological characteristics, cellular viability, ultrastructural characteristics and DNA integrity. Clindamycin did not alter colonial morphology, but after 6 h elongated cells were observed. Also, extracellular vesicles and electron-lucent areas inside the cytoplasm were observed. Bacteria treated with clindamycin also showed fragmentation of DNA as determined by electrophoresis. The alterations produced by clindamycin might be indicative of a possible modification of the structures involved in bacterial pathogenesis. [\hyperlink{Clindamycin Palmitate Hydrochloride}{PMID: 16553852}, Elessandra Maria Silvestro et al., 2006]

\hypertarget{pmid_30141180}{C}alcimimetics, shown to control biochemical parameters of secondary hyperparathyroidism (SHPT), have well-established safety and pharmacokinetic profiles in adult end-stage renal disease subjects treated with dialysis; however, such studies are limited in pediatric subjects. In this study, the safety, tolerability, pharmacokinetics (PK), and pharmacodynamics (PD) of cinacalcet were evaluated in children with chronic kidney disease (CKD) and SHPT receiving dialysis. Twelve subjects received a single dose of cinacalcet (0.25 mg/kg) orally or by nasogastric or gastric tube. Subjects were randomized to one of two parathyroid hormone (PTH) and serum calcium sampling sequences: [(1) 2, 8, 48 h; or (2) 2, 12, 48 h] and assessed for 72 h after dosing. Median plasma cinacalcet t In conclusion, a single 0.25 mg/kg dose of cinacalcet was evaluated to be a safe starting dose in these children aged < 6 years. [\hyperlink{Clindamycin Palmitate Hydrochloride}{PMID: 30141180}, Winnie Y Sohn et al., 2019]

\hypertarget{pmid_6470871}{T}he pharmacokinetics of intravenously administered clindamycin phosphate was studied in 40 children less than 1 year of age. Mean peak serum concentrations were 10.92 micrograms/ml in premature infants less than 4 weeks of age, 10.45 micrograms/ml in term infants greater than 4 weeks, and 12.69 micrograms/ml in term infants less than 4 weeks of age. Mean trough concentrations were 5.52, 2.8, and 3.03 micrograms/ml, respectively, in the same groups. Serum half-life was significantly longer (8.68 vs 3.60 hours) in premature compared with term infants less than 4 weeks of age. Both premature and term infants less than 4 weeks had significantly decreased clearance when compared with infants greater than 4 weeks (0.294 and 0.678, respectively, vs 1.58 L/hr). Clearance was significantly greater (1.919 vs 0.310 L/hr) and serum half-life less (1.75 vs 7.57 hours) in infants with body weight greater than 3.5 kg. On the basis of these data it is recommended that in infants greater than 4 weeks or greater than 3.5 kg, intravenous clindamycin dosage be 20 mg/kg/day in four divided doses. In premature neonates less than 4 weeks, the dose should be reduced to 15 mg/kg/day in three divided doses. Term infants greater than 1 week of age may also receive 20 mg/kg/day in four doses. [\hyperlink{Clindamycin Palmitate Hydrochloride}{PMID: 6470871}, M J Bell et al., 1984]

\hypertarget{pmid_15951862}{D}iagnostic and therapeutic procedures in children are made easier using sedation. However, there is no consensus about which drug should be used to achieve this. Furthermore, none of the drugs used for sedation are risk free. The aim of this work is to study sedation indications, effectiveness, and safety at our center. A prospective observational study conducted at the Pediatric Day Care Unit, King Fahad National Guard Hospital, Riyadh, Saudi Arabia. The study covered 17.5 weeks in 2 periods: May 9th 1999 to June 13th 1999 and October 31st 2001 to February 11th 2002. Children <12 years were included. Collected data included demographics, indication, drug dosing and outcome. Data were reported as mean +/- SD. We included 148 patients, age 38 +/- 30 months. Adequate sedation was achieved in 79\% after initial chloral hydrate (CH) dose of 56.9 +/- 9.3 mg/kg, in 95\% after adding 18.5 +/- 6.4 mg/kg CH and in 96\% after adding second drug. Compared to nonrespondents, first CH dose respondents were younger and lower in weight. The CH side effects were few and mild. Chloral hydrate is a safe and effective agent for sedation in children with an age and weight dependent response. [\hyperlink{Clindamycin Palmitate Hydrochloride}{PMID: 15951862}, Omar M Hijazi et al., 2005]

\section*{Sabril}
\subsection*{Result}
\subsubsection*{Answer}

Yes (Ages 1 month to 2 years, for infantile spasms, based on FDA approval and targeted studies)  
Unknown (Children older than 2 years or for indications other than infantile spasms, due to lack of definitive targeted safety studies and evidence of significant risk of visual field loss)

\subsubsection*{{Explanation}}
\hypertarget{Sabril}
A review of the available abstracts reveals several that specifically address the safety of Sabril (vigabatrin) in children:

1. One abstract states that Sabril (vigabatrin) oral solution was approved by the FDA on August 21st, 2009 for the treatment of infantile spasms in children ages 1 month to 2 years, indicating that targeted studies were performed in this age group and that the FDA found it safe for use in this population [\hyperlink{pmid_22778839}{PMID: 22778839}, Corey R Hopkins et al., 2010].

2. Another abstract reviews clinical experience and surveillance data for vigabatrin, noting that it is marketed in some countries for the treatment of infantile spasms and that, in an estimated 350,000 patient-years of exposure, no definite case of a specific adverse effect (intramyelinic edema) was identified in humans. However, this abstract does not specify age ranges or directly address overall safety in children beyond this adverse effect [\hyperlink{pmid_10691111}{PMID: 10691111}, J A Cohen et al., 2000].

3. A further abstract discusses the approval of vigabatrin in the USA for the adjunctive treatment of refractory complex partial seizures and as treatment of infantile spasms, but does not provide specific safety data for children outside the infantile spasms indication [\hyperlink{pmid_19954276}{PMID: 19954276}, Justin A Tolman et al., 2009].

4. Importantly, one abstract reports a targeted study of 14 children on vigabatrin at a hospital, specifically investigating visual field loss. Ten of the 14 children had constriction of their visual fields attributed to vigabatrin, and the authors caution that the drug should be used with great care, as the risk of irreversible visual field damage exists and monitoring is difficult in children. This suggests a significant safety concern for children, particularly regarding vision [\hyperlink{pmid_11026995}{PMID: 11026995}, I M Russell-Eggitt et al., 2000].

In summary:
- For children ages 1 month to 2 years, there is evidence from targeted studies supporting the safety of Sabril for infantile spasms, as reflected in FDA approval.
- For children outside this age range, and for other indications, the safety is less clear. There is evidence of a significant risk of irreversible visual field constriction in children, based on a targeted study, which raises concerns about its safety in broader pediatric use.

\subsection*{Abstracts}
\hypertarget{pmid_22778839}{S}abril (vigabatrin) oral solution was approved by the FDA on August 21st, 2009 for treatment of infantile spasms in children ages 1 month to 2 years and complex partial seizures in adults (tablets). [\hyperlink{Sabril}{PMID: 22778839}, Corey R Hopkins et al., 2010]

\hypertarget{pmid_7270984}{A}lbuterol (salbutamol) syrup was studied in 14 asthmatic children (three to six years of age) in a four-week, double-blind, crossover (with placebo) trial to determine efficacy, safety and tolerance. Albuterol was found to be more effective as evaluated by measurements of symptom scores (p less than .01) daily WPF meter (p less than .01) and need for additional medications. Albuterol provided a significant (p less than .01) increase in FEV1 and FEF 25\%-75\% over three hours. Clinically unimportant effects on heart rate, personality and tremors were noted in most subjects. The authors conclude that albuterol syrup is effective and safe in the young asthmatic. [\hyperlink{Sabril}{PMID: 7270984}, G S Rachelefsky et al., 1981]

\hypertarget{pmid_10691111}{V}igabatrin (Sabril, Hoechst Marion Roussel) is an antiepilepsy drug (AED) presently marketed in 64 countries for the treatment of partial and secondarily generalized seizures. Vigabatrin (VGB) is marketed in a subset of these countries for the treatment of infantile spasms. Clinical experience in humans has shown that VGB provides effective seizure control with a wide margin of safety. However, animal toxicity studies raised concern when prolonged administration of VGB was shown to induce intramyelinic edema (IME) in some laboratory animal species. Animal and human data were reviewed with respect to the potential for VGB-induced IME. Surveillance of patients receiving VGB in clinical trials or by prescription has been conducted for >15 years to identify patients developing clinical abnormalities that might be IME related. The histologic lesions of VGB-induced IME in animals are reliably reproduced and correlate with changes in multimodality evoked potentials (EPs) and magnetic resonance imaging (MRI). Numerous studies of the effects of VGB on EP and MRI in epilepsy patients have demonstrated no clear-cut IME-related changes in these modalities. Additionally, autopsy and surgical brain samples from VGB-treated patients have been scrutinized for potential IME histopathology. In an estimated 350,000 patient-years of VGB exposure (approximately 175,000 patients exposed for 2 years at an average dose of 2 g/day), no definite case of VGB-induced IME has been identified. Comprehensive review of a variety of sources of data failed to identify any definite case of IME in humans treated with VGB. [\hyperlink{Sabril}{PMID: 10691111}, J A Cohen et al., 2000]

\hypertarget{pmid_19954276}{V}igabatrin (Sabril) was approved in the USA in mid-2009 for the adjunctive treatment of refractory complex partial seizures and as treatment of infantile spasms. Vigabatrin's more than 30-year history of research and development is condensed into a clinically relevant review pertaining to this 2009 approval. A review of the scientific literature was conducted with special focus given to the drug molecule, its mechanism of action, its effects on living systems (e.g., pharmacokinetic, pharmacologic and toxicologic), and its anticipated role among antiepileptic drugs in the USA. The recent approval of vigabatrin makes a significant addition to antiepileptic drug options. The FDA implemented a Risk Evaluation and Mitigation Strategy to control for the possibility of severe adverse drug events. [\hyperlink{Sabril}{PMID: 19954276}, Justin A Tolman et al., 2009]

\hypertarget{pmid_2019938}{T}o test whether nebulized salbutamol (albuterol) is safe and efficacious for the treatment of young children with acute bronchiolitis, we enrolled 83 children (median age 6 months, range 1 to 21 months) in a randomized, double-blind clinical trial. Participants received two treatments at 30-minute intervals of either nebulized salbutamol (0.10 mg/kg in 2 ml 0.9\% saline solution) or a similar volume of 0.9\% saline solution placebo. Outcome measures were the respiratory rate, pulse oximetry, and a clinical score based on the degree of wheezing and retractions. Patients in the salbutamol arm had significantly greater improvement in clinical scores after the initial treatment (p = 0.04). There was no difference between the groups in oxygen saturation (p = 0.74); patients treated with salbutamol had a small increase in heart rate after two treatments (159 +/- 16 vs 151 +/- 16; p = 0.03). We conclude that salbutamol is safe and effective for the initial treatment of young children with acute bronchiolitis. [\hyperlink{Sabril}{PMID: 2019938}, T P Klassen et al., 1991]

\hypertarget{pmid_11026995}{V}igabatrin (Sabril), a drug that blocks GABA transaminase, has been used in the treatment of epilepsy since 1989. There have been reports of irreversible constriction of the visual field in adult patients related to vigabatrin (VGB) therapy, resulting in reduced VGB usage in adults. Although used as a second or third line agent in adults, in children it is often considered as a first line treatment for several subgroups of seizures in spite of there being no way, in the majority of cases, to monitor visual fields. Some of these children have a pre-existing visual field defect as part of their primary disorder. We aimed to identify whether visual field loss due to VGB was occurring in our hospital. We have studied the results of ophthalmic examination in 14 children on VGB at Great Ormond Street Hospital who were able to perform Goldmann visual fields. Ten of the 14 patients had constriction of their visual fields attributed to VGB. In addition there were 2 patients with suspicious visual field defects thought to be due to VGB. There was pre-existing visual pathway damage in 4 cases and in 2 of these optic disc pallor increased in association with constricted visual fields. However, the optic discs were normal in 7 patients in spite of visual field constriction. Visual acuity was generally normal in spite of gross visual field constriction. We believe that VGB should be used with great caution where there is pre-existing visual pathway damage. In other cases the benefits should be considered in relation to the risks, which include irreversible visual field damage. At present visual fields can only be monitored by perimetry, which is often not possible in children with epilepsy. [\hyperlink{Sabril}{PMID: 11026995}, I M Russell-Eggitt et al., 2000]

\hypertarget{pmid_26719728}{S}ildenafil is a phosphodiesterase type-5 inhibitor approved for treatment of pulmonary arterial hypertension (PAH) in adults. Data from pediatric trials demonstrate a similar acute safety profile to the adult population but have raised concerns regarding the safety of long-term use in children. Interpretation of these trials remains controversial with major regulatory agencies differing in their recommendations - the US Food and Drug Administration recommends against the use of sildenafil for treatment of PAH in children, while the European Medicines Agency supports its use at "low doses". Here, we review the available pediatric data regarding dosing, acute, and long-term safety and efficacy of sildenafil for the treatment of PAH in children.  [\hyperlink{Sabril}{PMID: 26719728}, Andrew L Dodgen et al., 2015] In a single-blind study, 60 children in two age groups (30 patients: 6 months to 3 years; 30 patients: 3 years to 12 years), were orally treated with either alpha-methyl-4-(2-thienyl-carbonyl)phenylacetic acid (suprofen, Suprol), syrup 10 mg/ml or metamizole drops 50\% for a maximum period of 4 days, up to 4 times a day. The children presented with high fever due to bacterial or virus infections. Body temperature, pulse rate, and respiratory rate were evaluated at the beginning and then 30 min, 1, 1 1/2, 2, 3, 4, 5, and 6 h after the first administration of the respective drug. Significant differences between the drugs were found for all variables; this demonstrated that with suprofen the antipyretic effect set in more rapidly than with the reference drug. No side-effects were observed in children treated with suprofen syrup. Two patients showed adverse effects, i.e. sweating and hypotension, during the treatment with metamizole. Due to its good antipyretic effect and good tolerability, suprofen appears to be particularly useful for symptomatic treatment of pediatric patients with fever caused by bacterial or virus infections. [\hyperlink{Sabril}{PMID: 26719728}, M Giovannini et al., 1986]

\hypertarget{pmid_25611962}{T}adalafil is a selective Phosphodiesterase-5 inhibitor that has been reported to have vasodilatory and antiproliferative effects on the pulmonary artery. In this study we evaluated the safety and efficacy of oral tadalafil in children with pulmonary arterial hypertension (PAH). This open label study, prospective and interventional was carried out in 25 known patients aged 2 month-5 years in 3 medical centers in Iran, between March 2013-Jun 2014. Tadalafil suspension was administrated at 1 mg/kg daily for all patients. Hemodynamic and safety parameters were assessed at baseline and then monthly for a total of 4 visits. 19 patients received tadalafil as initial therapy, in all visits significant improvements in mean pulmonary arterial pressure were observed (p<0.01). Of the 25 patients, 6 (24\%) had been on sildenafil for longer than 6 months. After transition from sildenafil to tadalafil clinical improvement was noted (p<0.05). Administration of tadalafil suspension was generally safe and well tolerated. Nausea was the most frequently reported adverse events which occurred in 3 patients during treatment. Oral tadalafil was administered easily and tolerated well and improved mean pulmonary artery pressure (MPAP) in children with PAH, which suggests that oral tadalafil may be more effective and safer than sildenafil in the treatment of PAH. [\hyperlink{Sabril}{PMID: 25611962}, A Shiva et al., 2016]

\hypertarget{pmid_32848722}{C}hildren are more exposed to inappropriate medicine use and its consequent harms. Spontaneous reporting of suspected Serious Adverse Drug Reactions (SADR) increases knowledge and prevention of pharmacotherapy risk. Disproportionality measures are useful to quantify unexpected safety issues associated with a given drug-event pair (signals of disproportionality). This cross-sectional study aimed to assess SADR reporting and safety signals for Brazilian children from 0-12 years old, notified between January 2008 and December 2013 from the Brazilian Surveillance Agency (Notivisa). Information from serious reports (gender and age of the patient, event description, suspected drug) was included. Disproportionality analysis based on Reporting Odds Ratios with a confidence interval of 95\% was conducted to identify possible signals of disproportionate reporting (SDR). Almost 30\% of 1,977 suspected SADR was related to babies (0-1-year-old). 69\% of reports happened with intravenous dosage forms, and 35\% of suspected SADR involved off label use according to age. Laronidase, miglustat, imipenem/cilastatin, and clofarabine were involved in six or more suspected deaths among 75 deaths reported. There were 107 SDRs, of which 16 events (15\%) were not described in the product labels. There was a relatively higher number of SADRs in Brazilian children compared with studies from other countries. SDRs found, (especially drug-event pairs 'imipenen/cilastatin-pneumonia' and 'laronidase-respiratory insufficiency') should be investigated more. The reports of SADR with IV dosage forms and OL drug use suggest the need for drug research and the use of better dosage forms for children in Brazil. [\hyperlink{Sabril}{PMID: 32848722}, Jean Mendes de Lucena Vieira et al., 2020]

\hypertarget{pmid_11253489}{T}iagabine (Gabitril, Sanofi Synlhelabo) new antiepileptic drug was used in add-on therapy in 25 children with resistant partial complex and secondary generalized seizures. Treatment was carried out in children aged 4-17 years with low dose escalation from 5 to 45 mg/day, in three doses until good clinical effects were obtained. In 3 patients aged 4 years, in 11 children aged 5-12 years and in 11 children aged above 17 years Gabitril was used. Follow up period was 8-10 months. Frequency of epileptic seizures before implementation of Gabitril treatment, even during polytherapy with 2 or more antiepileptic drugs was several to hundred per day (status epilepticus was observed in 2 children with Rasmussen syndrome). During the observation 5 children became seizure free, in 11 patients reduction in seizures frequency above 50\% was observed and in 9 children effects of treatment were not good enough. Gabitril was well tolerated, and any adverse events were observed in add-on therapy. Preliminary observation and good results of add-on therapy with Gabitril are positive. Drug is safe and generally well-tolerated with good effects at add-on therapy in 64\% children with resistant partial complex and secondary generalized seizures. [\hyperlink{Sabril}{PMID: 11253489}, T Kmieć et al., 2000]

\hypertarget{pmid_16630637}{S}ucrose acetate isobutyrate (SAIB) is a water insoluble, biodegradable gel used for controlled-release oral and subcutaneous drug delivery. We investigated SAIB compatibility in the rat central nervous system (CNS) by implanting solutions of SAIB in adult and in neonatal brains. 10-15 microL solutions of SAIB gels in 0-30\% ethanol were injected into the cerebral cortex of adult Fischer 344 rats. Control animals were implanted with a 10 mg biodegradable poly anhydride copolymer of poly [bis (p-carboxyphenoxy) propane] anhydride and sebacic acid (PCPP:SA). Adult rats were evaluated for signs of pain and distress, including changes in posture, facial signs, and grooming behavior. 1-2 microL solutions of SAIB gels in 15\% ethanol were injected into brains of 12-24 h-old rats. Neonatal rats were evaluated for survival. Adult and neonatal brains were examined by histopathology 3-48 days after implant. Gel implants produced elliptical compression of cortical tissue, cell loss, and inflammation. Cell loss appeared to be confined to the implantation wound and associated neuronal fields. In adult rats, neurophil compression, inflammation, and cell loss appeared similar with the 10-mg PCPP:SA implants and the 10-mg SAIB implants. There was no clinical evidence of pain or distress from SAIB implants. 1-2 microL implants of SAIB-15\% ethanol had no effect on survival of neonatal animals. Brain implants of SAIB induce a mild to moderate inflammatory response and associated neuronal cell damage. The implants appeared to be biocompatible in adult and neonatal animals. These results suggest that further studies of SAIB as an injectable drug-delivery scaffold for CNS therapeutic agents are warranted. [\hyperlink{Sabril}{PMID: 16630637}, James Lee et al., 2006]

\hypertarget{pmid_12943481}{I}n the US, 6\% sulfur in petrolatum has been the most frequently administered treatment for infantile scabies. It appears to be safe but there is no literature containing a large series of patients on which to base that determination. In the UK, benzyl benzoate is the approved product. Benzyl benzoate is rarely used in the US at the present time. 5\% Permethrin is an excellent substitute and has many advantages. It appears to be quite safe in infants, although it is more expensive than other products. It remains present on the skin for several days, therefore protecting against reinfestation. Ivermectin is a systemic drug which is assumed to be safe in infants, although it requires repeated doses and does not protect against reinfestation. In the opinion of the author, 5\% permethrin is the best treatment for scabies in infants and young children. [\hyperlink{Sabril}{PMID: 12943481}, Mervyn L Elgart et al., 2003]

\hypertarget{pmid_38094112}{S}acubitril/valsartan is an angiotensin receptor neprilysin antagonist (ARNI) approved for adult heart failure (HF). Its safety and efficacy in pediatric HF patients with cardiomyopathy or congenital heart disease are poorly understood. A pilot study was conducted to assess the clinical response, efficacy and safety of sacubitril/valsartan in this population at a tertiary care hospital in China. Clinical parameters of patients who received sacubitril/valsartan from January 2019 to March 2023 were retrospectively collected and analyzed. Children over 1 month with a left ventricular ejection fraction (LVEF) <45\% were included. Clinical efficacy was evaluated by echocardiographic LVEF, N-terminal pro-brain natriuretic peptide (NT-proBNP), New York Heart Association (NYHA) HF classification, HF re-admission, and death or transplantation. The initial dose was either 0.2 mg/kg bid or 0.4 mg/kg bid, with a target dose of 2.3 mg/kg bid or 3.1 mg/kg bid. Forty-five patients (60\% male) with a median age of 7.86 years were enrolled. Among them, 23 had congenital heart disease and 22 had cardiomyopathies. The median maintenance dose was 0.76 mg/kg. The primary endpoint of LVEF up to 45\% was reached by 24 patients (53.3\%). The median NT-proBNP was significantly decreased from 5,501.5 pg/ml to 2,241.5 pg/ml ( Sacubitril/valsartan may be effective in children with HF, but its safety and outcomes may differ depending on the etiology and anatomy of HF. Early post-operative congenital heart disease patients had less tolerance, more hypotension but better recovery and outcomes, while mid- and late- post-operative congenital heart disease patients and cardiomyopathy patients had less side effects but poorer clinical outcomes. [\hyperlink{Sabril}{PMID: 38094112}, Yahe Xu et al., 2023]

\hypertarget{pmid_27826711}{E}nalapril is used to treat hypertension and congestive heart failure in infants. However, enalapril is not labeled for neonates, and safety data in infants are sparse. To evaluate the safety of enalapril in young infants, we conducted a retrospective cohort study of infants who were exposed to enalapril in the first 120 days of life and were cared for in 348 neonatal intensive care units from 1997 to 2012. We determined the proportion of exposed infants who developed adverse events, including death, hypotension requiring pressors, hyperkalemia, and elevated serum creatinine. Using multivariable logistic regression, we examined risk factors for adverse events, including postnatal age at first exposure, exposure duration, gestational age group, small for gestational age status, race, sex, 5-min Apgar score, and inborn status. Of a cohort of 887,910 infants, 662 infants (0.07\%) were exposed to enalapril. Among exposed infants, 142 infants (21\%) suffered an adverse event. The most common adverse event was hyperkalemia (13\%), followed by elevated serum creatinine (5\%), hypotension (4\%), and death (0.5\%). Significant risk factors for adverse events included postnatal age <30 days at first exposure and longer exposure duration. This study is the largest to date examining the safety of enalapril in young term and preterm infants without significant structural cardiac disease. [\hyperlink{Sabril}{PMID: 27826711}, Lawrence C Ku et al., 2017]

\hypertarget{pmid_20944041}{I} frequently see children with scabies in my practice. A variety of medications are available to treat scabies. Permethrin is one of the most common medications used. Is permethrin a safe and effective option for children? Scabies is a common parasitic skin infection. It is highly prevalent in young children. Topical permethrin (5\% cream) is a safe and effective scabicide in children. It is recommended as a first-line therapy for patients older than 2 months of age. Because there are theoretical concerns regarding percutaneous absorption of permethrin in infants younger than 2 months of age, guidelines recommend 7\% sulfur preparation instead of permethrin. [\hyperlink{Sabril}{PMID: 20944041}, Lina Albakri et al., 2010]

\hypertarget{pmid_614329}{T}wenty-one young asthmatics, 2-6 years of age (mean 4 years), were given an open trial of salbutamol syrup to assess its safety. Each patient was given 1 mg, then 2 mg, q8h for two weeks. Only one patient experienced side-effects and this was at the 2 mg dose. It is concluded that salbutamol syrup is safe at a dose of 1 to 2 mg q8h for the asthmatic children in this age group. [\hyperlink{Sabril}{PMID: 614329}, M T Lin et al., 1977]

\hypertarget{pmid_21982407}{G}abapentin (GAB) is a newer second-line antiepileptic drug (AED) used in children. This is a multi-centre retrospective observational study of the efficacy, tolerability and retention rate in 105 children, aged 0-17.5 years (mean 10.1) over a 14 year period. The median age of epilepsy onset was 2.5 years (range 0-14.6). 72\% started GAB as at least the 3rd AED, with 43\% having been withdrawn from at least 2 AEDs. 77\% had focal and 52\% symptomatic epilepsies. The maintenance doses for GAB ranged 6.0-87.3 mg/kg/day (mean 43.7). The study comprised 157 person-treatment years for GAB. GAB was well tolerated with 55\% remaining on treatment beyond 1 year. No serious adverse events were reported whilst on GAB, but 39\% reported possibly and probably related adverse events. Seizure improvement (<50\% seizure frequency compared to baseline) at more than 12 months of treatment, was reported in 35\% of patients starting GAB, including 6\% who remained seizure free. The results demonstrated the efficacy and tolerability of GAB in children with difficult to treat epilepsies, and a good response to treatment beyond 12 months, in both focal and generalised epilepsies. [\hyperlink{Sabril}{PMID: 21982407}, J K A Mills et al., 2012]

\hypertarget{pmid_22022012}{E}nalapril is an angiotensin converting enzyme inhibitor widely used in children for treatment of hypertension and congestive cardiac failure. We report a 5-year-old boy who developed severe hyponatremia and altered sensorium on enalapril therapy. The serum sodium gradually became normal within 3 days. The patient's sensorium improved significantly on correction of hyponatremia. Through this case, we highlight the importance of monitoring serum sodium in patients on enalapril therapy. [\hyperlink{Sabril}{PMID: 22022012}, Syed Ahmed Zaki et al., 2011]

\hypertarget{pmid_6937455}{H}aloperidol is safe and effective in children for relieving psychotic symptoms associated with childhood autism, schizophrenia and mental retardation. It is the drug of choice for Tourette's syndrome, and may be useful in nonpsychotic hyperactive or aggressive children to control acute episodes, or when the stimulants normally useful in hyperactive children are ineffective. Such children taking haloperidol not only become calmer, but are often better able to respond to other modalities of therapy and to school instruction. Dosage, initially low, is increased gradually to minimize drowsiness and extrapyramidal symptoms, the most common side effects. Haloperidol in children is usually well-tolerated. [\hyperlink{Sabril}{PMID: 6937455}, A C Serrano et al., 1981]

\hypertarget{pmid_9805178}{T}he aim of this study was to investigate the efficacy and toxicity of sulphasalazine (SASP) in the treatment of children with chronic arthritis. The medical records of 36 children (25 boys, 11 girls) who received SASP for the treatment of chronic arthritis were reviewed. Twenty-one patients had juvenile spondyloarthropathies (JSA) (eight juvenile ankylosing spondylitis (JAS), 13 undifferentiated JSA (uJSA) and 15 had juvenile rheumatoid arthritis (JRA). The patients received SASP therapy for a mean of 2.5 years (range 3 weeks to 8.1 years). Clinical and laboratory data were reviewed retrospectively to determine the effects of treatment. A clinically significant response occurred in 23 (64\%) children: remission in 14 (39\%) (JRA 5, JSA 9) and improvement (25\% reduction in joint count) in nine (25\%) (JRA 4, JSA 5). There was no difference in response rate between JRA and JSA patients (p = 0.11), but the time to remission was shorter in JSA patients (mean 5 months) than in JRA patients (mean 25 months) (p = 0.024). Twelve of the 36 patients discontinued non-steroidal anti-inflammatory drugs, and six of eight patients discontinued prednisolone. A significant fall in erythrocyte sedimentation rate and rise in haemoglobin occurred in SASP-treated patients (p < 0.005) comparing most recent results with pretreatment levels. Side-effects occurred in four of 36 patients (11\%); only one patient who had persisting severe diarrhoea required discontinuation of SASP. It was concluded that SASP appears to be effective and safe in the treatment of JRA and JSA patients. As a second-line agent, SASP is the drug of first choice for patients with JSA; for JRA patients SASP may be a useful, possibly less toxic alternative to methotrexate. [\hyperlink{Sabril}{PMID: 9805178}, J L Huang et al., 1998]

\hypertarget{pmid_36053397}{N}on-steroidal anti-inflammatory drugs (NSAIDs) are commonly used in infants, children, and adolescents worldwide; however, despite sufficient evidence of the beneficial effects of NSAIDs in children and adolescents, there is a lack of comprehensive data in infants. The present review summarizes the current knowledge on the safety and efficacy of various NSAIDs used in infants for which data are available, and includes ibuprofen, dexibuprofen, ketoprofen, flurbiprofen, naproxen, diclofenac, ketorolac, indomethacin, niflumic acid, meloxicam, celecoxib, parecoxib, rofecoxib, acetylsalicylic acid, and nimesulide. The efficacy of NSAIDs has been documented for a variety of conditions, such as fever and pain. NSAIDs are also the main pillars of anti-inflammatory treatment, such as in pediatric inflammatory rheumatic diseases. Limited data are available on the safety of most NSAIDs in infants. Adverse drug reactions may be renal, gastrointestinal, hematological, or immunologic. Since NSAIDs are among the most frequently used drugs in the pediatric population, safety and efficacy studies can be performed as part of normal clinical routine, even in young infants. Available data sources, such as (electronic) medical records, should be used for safety and efficacy analyses. On a larger scale, existing data sources, e.g. adverse drug reaction programs/networks, spontaneous national reporting systems, and electronic medical records should be assessed with child-specific methods in order to detect safety signals pertinent to certain pediatric age groups or disease entities. To improve the safety of NSAIDs in infants, treatment needs to be initiated with the lowest age-appropriate or weight-based dose. Duration of treatment and amount of drug used should be regularly evaluated and maximum dose limits and other recommendations by the manufacturer or expert committees should be followed. Treatment for non-chronic conditions such as fever and acute (postoperative) pain should be kept as short as possible. Patients with chronic conditions should be regularly monitored for possible adverse effects of NSAIDs. [\hyperlink{Sabril}{PMID: 36053397}, Victoria C Ziesenitz et al., 2022]

\hypertarget{pmid_33847760}{O}ral ivermectin can be used to treat scabies. Evidence for safe and effective use in young children in individual treatment situations has been developed and published. In order to also ensure a body weight-adapted dosage for children, an ivermectin-containing syrup was developed as an extemporaneous preparation. Since ivermectin is not available as a pure substance for the formulation, tablets containing active ingredient were used as a basic material for development. The formulation was designed according to pharmaceutical, regulatory and use-oriented criteria. An HPLC (high-pressure liquid chromatography) method was developed and validated to demonstrate chemical stability. In order to facilitate the practical implementation, information on suitable packaging material and application aids was also developed and the formulation was evaluated. It has been demonstrated that the final formulation produced in the pharmacy was stable and can be stored for 3 weeks. No concerns were raised regarding the tolerability of the syrup formulation. The physicochemical properties and the taste of the formulation allow the intended use as a well-dosed syrup for children. The developed formulation meets the requirements of the Apothekenbetriebsordnung (Pharmacy Work Rules; Section 7 ApBetrO) and enables an exact, body weight-adapted dosage of oral ivermectin in young children. Studies on human pharmacokinetics or clinical studies to demonstrate tolerability and/or efficacy are not available for the formulation. [\hyperlink{Sabril}{PMID: 33847760}, Johannes Wohlrab et al., 2021]

\hypertarget{pmid_2326439}{T}his paper reports on 350 pediatric patients who were studied over a 17-month period to determine the efficacy and safety of oral and intramuscular sedation techniques. The protocol using oral chloral hydrate, 50 mgm/kg, for infants under 1 year of age or intramuscular pentobarbital, 5 mgm/kg, for children over 1 year was found to be an effective, safe and fairly simple approach to pediatric sedation. Of the 350 sedated patients, 343 (98 percent) had satisfactory scans on the same day the examination was scheduled after a single dose or an initial dose and supplementary sedation. [\hyperlink{Sabril}{PMID: 2326439}, J B Temme et al., ]

\hypertarget{pmid_327944}{S}albutamol was given by the intramuscular route to children aged 3 to 16 years with an attack of asthma. Results with a dose of 8 microgram/kg were favourable but suggested that a higher dose might be more so. 16 children thereafter were treated with 20 microgram/kg which produced a greater mean increase in peak expiratory flow rates (PEFR) without increase in side effects. This dose was then used in a double-blind crossover trial of salbutamol against a saline placebo. Half of 36 children treated with 20 microgram/kg showed rapid clinical improvement, the maximum rise in PEFR occurring within the first 5 minutes. A rise in pulse rate and occasionally a tremor were the only side effects noted. We conclude that intramuscular salbutamol 20 microgram/kg is a safe and useful initial medication in the management of the asthmatic child suffering an acute exacerbation. [\hyperlink{Sabril}{PMID: 327944}, P Thompson et al., 1977]

\section*{Pantoprazole Sodium}
\subsection*{Result}
\subsubsection*{Answer}

Neonates and preterm infants: Yes  
Infants (1-11 months): Yes  
Children (2-16 years): Yes  
Adolescents (12-16 years): Yes  

\subsubsection*{{Explanation}}
\hypertarget{Pantoprazole Sodium}
Based on the abstracts available, several targeted studies have evaluated the safety of Pantoprazole Sodium in children across different age ranges:

1. **Ages 2 to 16 years**: A pharmacokinetic and safety study in children aged 2 to 16 years found that both oral and intravenous pantoprazole were safe and well tolerated. All adverse events were mild and considered unrelated to the study drug [\hyperlink{pmid_18664620}{PMID: 18664620}, Gregory L Kearns et al., 2008].

2. **Ages 6 to 16 years**: Two studies specifically examined children with GERD:
   - One study in children aged 6 to 16 years found that pantoprazole delayed-release tablets provided systemic exposure similar to adults, with no evidence of accumulation or serious drug-associated adverse events [\hyperlink{pmid_20852004}{PMID: 20852004}, Robert M Ward et al., 2011].
   - Another study in adolescents aged 12 to 16 years reported that pantoprazole was safe, well tolerated, and effective in reducing GERD symptoms [\hyperlink{pmid_16968960}{PMID: 16968960}, V Marc Tsou et al., 2006].
   - A study in children aged 6 to 13 years (n=15) with reflux esophagitis found pantoprazole 20 mg daily was safe and well tolerated over 28 days [\hyperlink{pmid_12548064}{PMID: 12548064}, Armando Madrazo-de la Garza et al., 2003].
   - A study in children aged 5 to 11 years with GERD found adverse events were similar among treatment groups, indicating safety [\hyperlink{pmid_16641576}{PMID: 16641576}, Vasundhara Tolia et al., 2006].

3. **Ages 1 through 11 months (Infants)**: A randomized, double-blind, placebo-controlled study in infants aged 1 through 11 months with GERD found that pantoprazole significantly improved GERD symptom scores and was well tolerated. No between-group differences in adverse event frequency were noted, and serious adverse events were considered unrelated to treatment [\hyperlink{pmid_20400912}{PMID: 20400912}, Harland Winter et al., 2010].

4. **Neonates and Preterm Infants**: A multicenter, randomized, open-label trial in neonates and preterm infants with GERD found that pantoprazole granules were generally well tolerated, with no safety-related discontinuations [\hyperlink{pmid_20306184}{PMID: 20306184}, Robert M Ward et al., 2010].

5. **Birth to 16 years (Population PK study)**: A population pharmacokinetics study including patients from birth to 16 years found that pantoprazole pharmacokinetic parameters were similar to adults when scaled for body size, and no safety concerns were raised [\hyperlink{pmid_20484619}{PMID: 20484619}, W Knebel et al., 2011].

6. **Ages 1 to 17 years (PBPK modeling)**: A physiologically based pharmacokinetic modeling study predicted pantoprazole clearance in children 1-17 years of age, but did not directly assess safety outcomes [\hyperlink{pmid_30520273}{PMID: 30520273}, Peng Duan et al., 2019].

7. **Formulation studies**: Some studies addressed the stability and compounding of pediatric pantoprazole formulations but did not provide clinical safety data [\hyperlink{pmid_36986577}{PMID: 36986577}, Nemanja Todorović et al., 2023].

**Summary by Age Range:**
- **Neonates and preterm infants:** One targeted study supports that pantoprazole is generally well tolerated in this group.
- **Infants (1-11 months):** One targeted, controlled study supports safety and tolerability.
- **Children (2-16 years):** Multiple targeted studies affirm safety and tolerability.
- **Children (5-16 years):** Additional studies confirm safety in this range.
- **Adolescents (12-16 years):** Targeted study affirms safety.

No studies in the abstracts indicate that pantoprazole sodium is unsafe in children. The available evidence from targeted studies supports that pantoprazole sodium is safe for use in children from neonates through 16 years of age, when used as studied.

\subsection*{Abstracts}
\hypertarget{pmid_18664620}{T}he primary objective was to determine the pharmacokinetics of single oral and intravenous doses of pantoprazole in children 2 to 16 years of age. The secondary objective was to assess the safety and tolerability of these doses. Male and female hospitalized and nonhospitalized patients from ages 5 to 16 years received single oral doses (20 mg or 40 mg), and those from ages 2 to 16 years received single intravenous doses (0.8 mg/kg or 1.6 mg/kg) of pantoprazole. The plasma concentration-time data for each patient were analyzed using noncompartmental methods. Routine safety and tolerability assessments were also obtained. The mean values for peak plasma concentration and total area under the plasma concentration-time curve increased with increasing dose. Pharmacokinetic values were similar in patients from ages 2 to 16 years and to those previously obtained in adults. Statistically significant differences were observed for dose-normalized pantoprazole area under the plasma concentration-time curve when compared between CYP2C19 extensive metabolizers with 1 versus 2 functional alleles. All adverse events were mild in severity and considered to be unrelated to study drug. The pharmacokinetic profile of oral and intravenous pantoprazole was similar in children ages 2 to 16 years. The doses used here were safe and well tolerated in this population. [\hyperlink{Pantoprazole Sodium}{PMID: 18664620}, Gregory L Kearns et al., 2008]

\hypertarget{pmid_24138461}{T}he aim of this study was to determine the safety and the efficacy of paediatrician-administered propofol in children undergoing different painful procedures. We conducted a retrospective study over a 12-year period in three Italian hospitals. A specific training protocol was developed in each institution to train paediatricians administering propofol for painful procedures. In this study, 36,516 procedural sedations were performed. Deep sedation was achieved in all patients. None of the children experienced severe side effects or prolonged hospitalisation. There were six calls to the emergency team (0.02\%): three for prolonged laryngospasm, one for bleeding, one for intestinal perforation and one during lumbar puncture. Nineteen patients (0.05\%) developed hypotension requiring saline solution administration, 128 children (0.4\%) needed O2 ventilation by face mask, mainly during upper endoscopy, 78 (0.2\%) patients experienced laryngospasm, and 15 (0.04\%) had bronchospasm. There were no differences in the incidence of major complications among the three hospitals, while minor complications were higher in children undergoing gastroscopy. This multicentre study demonstrates the safety and the efficacy of paediatrician-administered propofol for procedural sedation in children and highlights the importance of appropriate training for paediatricians to increase the safety of this procedure in children. [\hyperlink{Pantoprazole Sodium}{PMID: 24138461}, Antonio Chiaretti et al., 2014]

\hypertarget{pmid_20400912}{T}he objective of this study was to assess the efficacy of pantoprazole in infants with gastroesophageal reflux disease (GERD). Infants ages 1 through 11 months with GERD symptoms after 2 weeks of conservative treatment received open-label (OL) pantoprazole 1.2 mg x kg(-1) x day(-1) for 4 weeks followed by a 4-week randomized, double-blind (DB), placebo-controlled, withdrawal phase. The primary endpoint was withdrawal due to lack of efficacy in the DB phase. Mean weekly GERD symptom scores (WGSSs) were calculated from daily assessments of 5 GERD symptoms. Safety was assessed. One hundred twenty-eight patients entered OL treatment, and 106 made up the DB modified intent-to-treat population. Mean age was 5.1 months (82\% full-term, 64\% male). One third of patients had a GERD diagnostic test before OL study entry. WGSSs at week 4 were similar between groups. WGSSs decreased significantly from baseline during OL therapy (P < 0.001), when all patients received pantoprazole. The decrease in WGSSs was maintained during the DB phase in both treatment groups. There was no difference in withdrawal rates due to lack of efficacy (pantoprazole 6/52; placebo 6/54) or time to withdrawal during the DB phase. The greatest between-group difference in WGSS was slightly worse with placebo at week 5 (P = 0.09), mainly due to episodes of arching back (P = 0.028). No between-group differences in adverse event frequency were noted. Serious adverse events in 8 patients were considered unrelated to treatment. Pantoprazole significantly improved GERD symptom scores and was well tolerated. However, during the DB treatment phase, there were no significant differences noted between pantoprazole and placebo in withdrawal rates due to lack of efficacy. [\hyperlink{Pantoprazole Sodium}{PMID: 20400912}, Harland Winter et al., 2010]

\hypertarget{pmid_12973370}{P}antoprazole sodium is a substituted benzimidazole derivative which controls acid secretion by inhibition of gastric H(+)/K(+)-ATPase. The prodrug pantoprazole accumulates in the acidic space of the parietal cell where it is converted to the pharmacologically active principle, a thiophilic cyclic sulfenamide. The pH-dependent activation profile, i.e., activation at pH 1 versus activation at pH 4-6, is more favorable for pantoprazole than for the other proton pump inhibitors (PPIs) currently available. In vitro, pantoprazole interferes less potently than omeprazole with biological targets not related to gastric acid secretion. The gastric target sites for the pantoprazole sulfenamide are the cysteines 813 and 822 of the catalytic subunit of the H(+)/K(+)-ATPase. In contrast to omeprazole, the two binding sites are located right at the proton channel. In rats, dogs and humans, pantoprazole produces marked and prolonged inhibition of both basal and stimulated acid secretion. Overall, its antisecretory potency is equal to that of omeprazole. Antiulcer activity has been demonstrated for pantoprazole in two rat models. As seen with H(2)-receptor antagonists and other PPIs, pantoprazole causes an increase in serum gastrin concentration which reflects the degree of gastric acid inhibition. Pantoprazole is mainly metabolized by CYP3A4 and 2C19, but displays a lower affinity for these phase I cytochrome P450 enzymes than omeprazole. In contrast to the latter, pantoprazole is further conjugated with sulfate by the hepatic phase II metabolism. These two differences may explain why pantoprazole does not interfere with the metabolism of any other drug thus far tested in humans. [\hyperlink{Pantoprazole Sodium}{PMID: 12973370}, W Beil et al., 1999]

\hypertarget{pmid_36986577}{P}antoprazole is a model substance that requires dosage form adjustments to meet the needs of all patients. Pediatric pantoprazole formulations in Serbia are mostly compounded as capsules (divided powders), while in Western Europe liquid formulations are more common. The aim of this work was to examine and compare the characteristics of compounded liquid and solid dosage forms of pantoprazole. Three syrup bases were used: a sugar-free vehicle for oral solution (according to USP43-NF38), a vehicle with glucose and hydroxypropyl cellulose (according to the DAC/NRF2018) and a commercially available SyrSpend Alka base. Lactose monohydrate, microcrystalline cellulose and a commercially available capsule filler (excipient II, composition: pregelatinized corn starch, magnesium stearate, micronized silicon dioxide, micronized talc) were used as diluents in the capsule formulations. Pantoprazole concentration was determined by the usage of the HPLC method. Pharmaceutical technological procedures and microbiological stability measurements were performed according to the recommendations of the EP10. Although dose appropriate compounding with pantoprazole is suitable using both liquid vehicles as well as solid formulations, chemical stability is enhanced in solid formulation. Nevertheless, according to our results, if liquid formulation is a pH adjusted syrup, it could be safely kept in a refrigerator for up to 4 weeks. Additionally, liquid formulations could be readily applied, while solid formulation should be mixed with appropriate vehicles with higher pH values. [\hyperlink{Pantoprazole Sodium}{PMID: 36986577}, Nemanja Todorović et al., 2023]

\hypertarget{pmid_15960715}{T}his prospective, clinical trial evaluated the effects of short-term propofol administration on triglyceride levels and serum pancreatic enzymes in children undergoing sedation for magnetic resonance imaging. Laboratory parameters of 40 children, mean age (SD; range) 67 (66; 4-178) months undergoing short-term sedation were assessed before and 4 h after having received propofol. Mean (SD) propofol loading dose was 2.2 (1.1) mg.kg(-1) followed by continuous propofol infusion of 6.9 (0.9) mg.kg(-1).h(-1). Serum lipase levels (p = 0.035) and serum triglyceride levels (p = 0.003) were raised significantly after propofol administration but remained within normal limits. No significant changes in serum pancreatic-amylase levels were seen (p = 0.127). In two (5\%) children, pancreatic enzymes and in four (10\%) children triglyceride levels were raised above normal limits; however, no child showed clinical symptoms of pancreatitis. We conclude that even short-term propofol administration with standard doses of propofol may have a significant effect on serum triglyceride and pancreatic enzyme levels in children. [\hyperlink{Pantoprazole Sodium}{PMID: 15960715}, S Gottschling et al., 2005]

\hypertarget{pmid_31297294}{P}antoprazole sodium, a substituted benzimidazole derivative, is an irreversible proton pump inhibitor which is primarily used for the treatment of duodenal ulcers, gastric ulcers, and gastroesophageal reflux disease (GERD). The monographs of European Pharmacopoeia (Ph. Eur.) and United States Pharmacopoeia (USP) specify six impurities,  [\hyperlink{Pantoprazole Sodium}{PMID: 31297294}, Arun Kumar Awasthi et al., 2019] The objective of this study was to develop pediatric physiologically based pharmacokinetic (PBPK) models for pantoprazole and esomeprazole. Pediatric PBPK models were developed by Simcyp version 15 by incorporating cytochrome P450 (CYP)2C19 maturation and auto-inhibition. The predicted-to-observed pantoprazole clearance (CL) ratio ranged from 0.96-1.35 in children 1-17 years of age and 0.43-0.70 in term infants. The predicted-to-observed esomeprazole CL ratio ranged from 1.08-1.50 for children 6-17 years of age, and 0.15-0.33 for infants. The prediction was markedly improved by assuming no auto-inhibition of esomeprazole in infants in the PBPK model. Our results suggested that the CYP2C19 auto-inhibition model was appropriate for esomeprazole in adults and older children but could not be directly extended to infants. A better understanding of the complex interplay of enzyme maturation, inhibition, and compensatory mechanisms for CYP2C19 is necessary for PBPK modeling in infants. [\hyperlink{Pantoprazole Sodium}{PMID: 31297294}, Peng Duan et al., 2019]

\hypertarget{pmid_26858095}{S}edation is increasingly used to facilitate procedures on children in emergency departments (EDs). This overview of systematic reviews (SRs) examines the safety and efficacy of sedative agents commonly used for procedural sedation in children in the ED or similar settings. We followed standard SR methods: comprehensive search; dual study selection, quality assessment, data extraction. We included SRs of children (1 month to 18 years) where the indication for sedation was procedure-related and performed in the ED. Fourteen SRs were included (210 primary studies). The most data were available for propofol (six reviews/50,472 sedations) followed by ketamine (7/8,238), nitrous oxide (5/8,220), and midazolam (4/4,978). Inconsistent conclusions for propofol were reported across six reviews. Half concluded that propofol was sufficiently safe; three reviews noted a higher occurrence of adverse events, particularly respiratory depression (upper estimate 1.1\%; 5.4\% for hypotension requiring intervention). Efficacy of propofol was considered in four reviews and found adequate in three. Five reviews found ketamine to be efficacious and seven reviews showed it to be safe. All five reviews of nitrous oxide concluded it is safe (0.1\% incidence of respiratory events); most found it effective in cooperative children. Four reviews of midazolam made varying recommendations. To be effective, midazolam should be combined with another agent that increases the risk of adverse events (upper estimate 9.1\% for desaturation, 0.1\% for hypotension requiring intervention). This comprehensive examination of an extensive body of literature shows consistent safety and efficacy for nitrous oxide and ketamine, with very rare significant adverse events for propofol. There was considerable heterogeneity in outcomes and reporting across studies and previous reviews. Standardized outcome sets and reporting should be encouraged to facilitate evidence-based recommendations for care. [\hyperlink{Pantoprazole Sodium}{PMID: 26858095}, Lisa Hartling et al., 2016]

\hypertarget{pmid_31156811}{S}otalol hydrochloride (SOT) is an antiarrhythmic β-blocker which is highly effective for the treatment of supraventricular tachycardia in children. However, a licensed paediatric dosage form with sotalol is not currently available in Europe. The aim of this work was to formulate paediatric oral solutions with SOT 5 mg/mL for extemporaneous preparation in a hospital pharmacy with the lowest possible amount of excipients and to determine their stability. Three aqueous solutions were formulated. One preparation without any additives for neonates and two preparations for children from 1 month of age were compounded using citric acid to stabilise the pH value, potassium sorbate 0.1\% w/v as a preservative, and simple syrup or sodium saccharin as a sweetener. The samples were stored at room temperature and in a refrigerator, respectively, and the content of SOT and potassium sorbate was determined simultaneously using a validated high performance liquid chromatography method at different time points over 180 days. At least 95\% of the initial sotalol concentration remained throughout the 180-day study period in all three preparations at both temperatures. The content of potassium sorbate decreased by 17\% with sodium saccharin stored at room temperature. The three proposed oral aqueous solutions of SOT for neonates and infants were stable for 180 days. Storage in a refrigerator is preferred, particularly with sodium saccharin. The additive-free solution of SOT can be autoclaved to ensure microbiological stability and used particularly for neonates and in emergency situations. [\hyperlink{Pantoprazole Sodium}{PMID: 31156811}, Sylva Klovrzová et al., 2016]

\hypertarget{pmid_31110954}{V}arious publications on the use of sedation and anesthesia for diagnostic procedures in children have demonstrated that no ideal agent is available. Although propofol has been widely used for sedation during esophagogastroduodenoscopy in children, adverse events including hypoxia and hypotension, are concerns in propofol-based sedation. Propofol is used in combination with other sedatives in order to reduce potential complications. We aimed to analyze whether the administration of midazolam would improve the safety and efficacy of propofol-based sedation in diagnostic esophagogastroduodenoscopies in children. We retrospectively reviewed the hospital records of children who underwent diagnostic esophagogastroduodenoscopies during a 30-month period. Demographic characteristics, vital signs, medication dosages, induction times, sedation times, recovery times, and any complications observed, were examined. Baseline characteristics did not differ between the midazolam-propofol and propofol alone groups. No differences were observed between the two groups in terms of induction times, sedation times, recovery times, or the proportion of satisfactory endoscopist responses. No major procedural complications, such as cardiac arrest, apnea, or laryngospasm, occurred in any case. However, minor complications developed in 22 patients (10.7\%), 17 (16.2\%) in the midazolam-propofol group and five (5.0\%) in the propofol alone group ( The sedation protocol with propofol was safe and efficient. The administration of midazolam provided no additional benefit in propofol-based sedation. [\hyperlink{Pantoprazole Sodium}{PMID: 31110954}, Ulas Emre Akbulut et al., 2019]

\hypertarget{pmid_20852004}{C}hildren with gastroesophageal reflux disease (GERD) may benefit from gastric acid suppression with proton pump inhibitors such as pantoprazole. Effective treatment with pantoprazole requires correct dosing and understanding of the drug's kinetic profile in children. The aim of these studies was to characterize the pharmacokinetic (PK) profile of single and multiple doses of pantoprazole delayed-release tablets in pediatric patients with GERD aged 6 to 11 years (study 1) and 12 to 16 years (study 2). Patients were randomly assigned to receive pantoprazole 20 or 40 mg once daily. Plasma pantoprazole concentrations were obtained at intervals through 12 hours after the single dose and at 2 and 4 hours after multiple doses for PK evaluation. PK parameters were derived by standard noncompartmental methods and examined as a function of both drug dose and patient age. Safety was also monitored. Pantoprazole PK was dose independent (when dose normalized) and similar to PK reported from adult studies. There was no evidence of accumulation with multiple dosing or reports of serious drug-associated adverse events. In children aged 6 to 16 years with GERD, currently available pantoprazole delayed-release tablets can be used to provide systemic exposure similar to that in adults. [\hyperlink{Pantoprazole Sodium}{PMID: 20852004}, Robert M Ward et al., 2011]

\hypertarget{pmid_21767419}{P}ropofol is the sedative of choice in our hospital for all procedural sedations in children older than 3 months. Data were collected from all patients who underwent PSA with propofol in the period from November 2007 to December 2009. The procedure was performed by a paediatrician experienced in airway management, sedation and paediatric IC, and a specialized nurse. Patient characteristics, American Society of Anesthesiologists (ASA) classification, vital parameters and propofol dosage were registered on specially designed forms. Patient data were analyzed and compared with data from a non-matched historical cohort of patients who in the past had undergone PSA with chloral hydrate. 204 procedural sedations with intravenous propofol were performed in 196 patients. The mean cumulative induction dose was 3.39 mg/kg (SD: 1.34) and the mean maintenance dose was 4.05 mg/kg/h (SD: 2.23). The success rate was 99.5\%, compared to 88.6\% in the cohort that had received PSA with chloral hydrate. 1 procedure was aborted because of desaturation due to an obstructed airway, for which a jaw thrust was performed. No complications were observed in 199 procedures (97.5\%). In 4 procedures a mild and transient desaturation (85-89\%) occurred. The results suggest that propofol can be used safely and is effective for procedural sedation in selected children, provided that PSA is performed by experienced and trained staff. [\hyperlink{Pantoprazole Sodium}{PMID: 21767419}, Christine J P Bruijnen et al., 2011]

\hypertarget{pmid_20306184}{T}he pharmacokinetic profile of pantoprazole granules was assessed in neonates and preterm infants with gastroesophageal reflux disease (GERD) in a multicenter, randomized, open-label trial. Patients were randomly assigned to either the pantoprazole 1.25 mg (approx. 0.6 mg/kg) or 2.5 mg (approx. 1.2-mg/kg) group and treated for > or =5 consecutive days. Blood was sampled either at 0, 2, 8, and 18 h postdose or at 0, 1, 4, and 12 h postdose on day 1 and at 3 and 6 h postdose after > or =5 consecutive doses. Cytochrome P450 2C19 (CYP2C19) and CYP3A4 genotypes were determined. Safety was monitored. Population pharmacokinetics (popPK) analyses were conducted using nonlinear mixed-effects modeling. The popPK modeling of the pantoprazole 1.25 mg and 2.5 mg groups obtained mean (+/-standard deviation) estimates for the area under the plasma concentration versus time curve (AUC) of 3.54 (+/-2.82) and 7.27 (+/-5.30) microg h/mL, respectively, and mean estimates for half-life of 3.1 (+/-1.5) and 2.7 (+/-1.1) h, respectively. Pantoprazole did not accumulate following multiple-dose administration. The two patients with the CYP2C19 poor metabolizer genotype had a substantially higher AUC than extensive metabolizers. No safety-related discontinuations occurred. In preterm infants and neonates, pantoprazole granules were generally well tolerated, mean exposures with pantoprazole 2.5 mg were slightly higher than that in adults who received 40 mg. While the half-life was longer, accumulation did not occur. [\hyperlink{Pantoprazole Sodium}{PMID: 20306184}, Robert M Ward et al., 2010]

\hypertarget{pmid_12548064}{T}o investigate the efficacy and safety of oral pantoprazole, 20 mg (0.5 to 1.0 mg/kg/day) once daily for 28 days, in pediatric patients with reflux esophagitis. Patients in this study (n = 15; 6 to 13 years old, 9 boys) had reflux esophagitis grade Ic or II (Vandenplas classification). The efficacy of pantoprazole to reduce esophageal acid exposure time (pH < 4), reduce the number and duration of reflux episodes, and to increase the percentage of time with gastric pH > 3 was assessed by continuous 24-hour pH monitoring. The intensity of 5 common symptoms of esophagitis was scored before and after treatment on a 4-point scale. Esophagitis was assessed at baseline and after treatment by visual inspection and by the histology of biopsies from the distal third of the esophagus. Before treatment, the median percentage of time with intra-esophageal pH <4 was 9.3\%. After 28 days of therapy with pantoprazole, this value decreased to 2.7\% (P = 0.0006). The median percentage of time with intragastric pH > 3 increased from 21\% at baseline to 39\% on day 28 of therapy (P = 0.005). After 28 days of treatment, all patients experienced at least partial relief from reflux symptoms. Endoscopically confirmed healing of esophagitis was seen in 47\% of children (Savary-Miller classification). Histologic evidence of healing was not observed. Median serum gastrin levels were slightly elevated over baseline levels (from 74 pg/ml to 93 pg/ml). In one patient there was a transient elevation of serum GOT and GPT during treatment. Oral pantoprazole 20 mg daily provided gastric acid control in 15 pediatric patients with reflux esophagitis with partial clinical improvement of symptoms after 28 days of treatment. Pantoprazole was safe and well tolerated. [\hyperlink{Pantoprazole Sodium}{PMID: 12548064}, Armando Madrazo-de la Garza et al., 2003]

\hypertarget{pmid_11240876}{T}o document the safety and efficacy of an anaesthetic technique in paediatric patients undergoing transoesophageal echocardiography (TOE). Prospective descriptive study performed in a children's hospital with all patients undergoing TOE. Topical analgesia of the pharynx was achieved with lidocaine. Anaesthesia was induced with midazolam (25 microg.kg-1), fentanyl (1 microg.kg-1), and propofol (0.5-1 mg.kg-1), followed by a continuous infusion of propofol (5-10 mg.kg-1.h-1). Thirty patients are reported. The mean age was 11.4 +/- 5.1 years (range 1-22) and weight 40.5 +/- 22.1 kg (range 10-110). All the patients tolerated the procedure well. Two patients experienced brief oxygen desaturations during induction, 10 patients coughed during the procedure, and six patients had significant muscle activity requiring supplemental doses of propofol. None of the patients experienced nausea or vomiting. We conclude that our anaesthetic technique in spontaneously breathing paediatric patients during TOE is effective and appears to be safe in children with heart disease. [\hyperlink{Pantoprazole Sodium}{PMID: 11240876}, C M Heard et al., 2001]

\hypertarget{pmid_32682946}{P}osaconazole is approved for use in adults as an intravenous (IV) solution and two different oral formulations (a suspension and an improved bioavailability tablet). Data on the pharmacokinetics (PK), dosing and safety of posaconazole in children are limited. A novel powder for oral suspension (PFS) offers the bioavailability of the tablet formulated for weight-based dosing in children. A non-randomised, open-label, sequential dose-escalation, phase 1b trial evaluated the PK and safety of posaconazole IV and PFS in children aged 2 to 17 years with documented or expected neutropenia (ClinicalTrials.gov, NCT02452034; MSD protocol number, MK-5592-P097). Participants received posaconazole IV 3.5, 4.5 or 6.0 mg/kg/d for ≥10 days, with an option to switch to posaconazole PFS at the identical dose for ≤18 days. The target exposure was a mean within-dose cohort average steady-state plasma concentration (C [\hyperlink{Pantoprazole Sodium}{PMID: 32682946}, Andreas H Groll et al., 2020] To evaluate symptom improvement in 53 children (aged 5-11 years) with endoscopically proven gastroesophageal reflux disease (GERD) treated with pantoprazole (10, 20 and 40 mg) using the GERD Assessment of Symptoms in Pediatrics Questionnaire. The GERD Assessment of Symptoms in Pediatrics Questionnaire was used to measure the frequency and severity over the previous 7 days of abdominal/belly pain, chest pain/heartburn, difficulty swallowing, nausea, vomiting/regurgitation, burping/belching, choking when eating and pain after eating. Individual symptom scores were based on the product of the frequency and usual severity of each symptom. The sum of the individual symptom score values made up the composite symptom score (CSS). The primary end point was the change in the mean CSS from baseline to week 8. Mean frequency and severity of each symptom significantly decreased (from P < 0.006 to P < 0.001) over time. Similar significant decreases in CSS at week 8 versus baseline (P < 0.001) were seen in all groups. Significant decreases from baseline in CSS were noted from weeks 1 to 8 in the 20-mg (P < 0.003) and 40-mg (P < 0.001) groups. The 20- and 40-mg doses were significantly (P < 0.05) more effective than the 10-mg dose in improving GERD symptoms at week 1. Adverse events were similar among the treatment groups. Pantoprazole (20 and 40 mg) is effective in reducing endoscopically proven GERD symptoms in children. Both 20 and 40 mg pantoprazole significantly reduced symptoms as early as 1 week. [\hyperlink{Pantoprazole Sodium}{PMID: 32682946}, Vasundhara Tolia et al., 2006]

\hypertarget{pmid_31001938}{D}espite well-known advantages, propofol remains off-label in many countries for general anesthesia in children under 3 years of age due to insufficient evidence regarding its use in this population. This study aimed to evaluate the efficacy and safety of propofol compared with other general anesthetics in children under 3 years of age undergoing surgery through a systematic review and meta-analysis of existing randomized clinical trials. A comprehensive literature search was conducted of MEDLINE, Embase, and the Cochrane Central Register of Controlled Trials to find all randomized clinical trials comparing propofol with another general anesthetic that included children under 3 years of age. The relative risk or arcsine-transformed risk difference for dichotomous outcomes and the weighted or standardized mean difference for continuous outcomes were estimated using a random-effects model. A total of 249 young children from 6 publications were included. The children who received propofol had statistically significantly lower systolic and diastolic blood pressures, but hypotension was not observed in the propofol groups. The heart rate, stroke volume index, and cardiac index were not significantly different between the propofol and control groups. The propofol groups showed slightly shorter recovery times and a lower incidence of emergence agitation than the control groups, while no difference was observed for the incidence of hypotension, desaturation, and apnea. This systematic review and meta-analysis indicates that propofol use for general anesthesia in young healthy children undergoing surgery does not increase complications and that propofol could be at least comparable to other anesthetic agents. [\hyperlink{Pantoprazole Sodium}{PMID: 31001938}, Hyunsook Hong et al., 2019]

\hypertarget{pmid_16968960}{A}n age-appropriate questionnaire (GASP-Q) was used to assess the frequency and severity of the gastroesophageal reflux disease (GERD) symptoms: abdominal/belly pain, chest pain/heartburn, pain after eating, nausea, burping/belching, vomiting/regurgitation, choking when eating, and difficulty swallowing, in adolescents age 12 to 16 years. The primary objective was to compare the mean composite symptom score (CSS) at week 8 with baseline after treatment with 20 or 40 mg of pantoprazole. Statistically significant (p < 0.001) improvement in CSS occurred in both groups. Safety was comparable between the 2 groups. Pantoprazole was safe, well tolerated, and effective in reducing symptoms of GERD in adolescents. [\hyperlink{Pantoprazole Sodium}{PMID: 16968960}, V Marc Tsou et al., 2006]

\hypertarget{pmid_20484619}{T}he population pharmacokinetics of pantoprazole was characterized in pediatric patients from birth to 16 years using NONMEM and evaluated via bootstrap and predictive check. Data were described using a 2-compartment model with a typical parameterized in terms of clearance (CL) (95\% CI) of 1.93 L per hour (1.53, 2.61), given the reference covariates (female, full term, extensive/unknown CYP2C19 metabolizer status, non-African American, 10 kg weight, intravenous or tablet administration). Pantoprazole pharmacokinetic parameters appear to be similar in pediatric patients compared to adults when allometrically scaled. The effect of age on allometrically scaled CL was best described by a sigmoid Emax model with the age effect reaching an asymptote approximately equal to the adult CL by 1 year. CYP2C19 poor metabolizers exhibited reduced CL with the point estimate and 95\% CI more than 70\% lower than the typical value. Simulations from the final model indicated that the 1.2-mg/kg dose provides the best comparison to adults. [\hyperlink{Pantoprazole Sodium}{PMID: 20484619}, W Knebel et al., 2011]

\hypertarget{pmid_26719728}{S}ildenafil is a phosphodiesterase type-5 inhibitor approved for treatment of pulmonary arterial hypertension (PAH) in adults. Data from pediatric trials demonstrate a similar acute safety profile to the adult population but have raised concerns regarding the safety of long-term use in children. Interpretation of these trials remains controversial with major regulatory agencies differing in their recommendations - the US Food and Drug Administration recommends against the use of sildenafil for treatment of PAH in children, while the European Medicines Agency supports its use at "low doses". Here, we review the available pediatric data regarding dosing, acute, and long-term safety and efficacy of sildenafil for the treatment of PAH in children.  [\hyperlink{Pantoprazole Sodium}{PMID: 26719728}, Andrew L Dodgen et al., 2015] Use of propofol in pediatric age group has been marred by reports of its adverse effects like hypertriglyceridemia and acute pancreatitis, although a causal relation has not yet been established. This prospective, clinical trial was carried out to evaluate the effects of short-term propofol administration on serum lipid profile and serum pancreatic enzymes in children of ASA physical status I and II aged between 1 month and 36 months. Anesthesia was induced with Propofol (1\%) in the dose of 3 mg·kg(-1) intravenously and was maintained by propofol infusion (0.5\%) at the rate of 12 mg·kg(-1·) h(-1) for the first 20 min and at 8 mg·kg(-1·) h(-1) thereafter. The mean dose of propofol administered was 12.02 ± 2.75 mg·kg(-1) (fat load of 120.2 ± 27.5 mg·kg(-1) ). Lipid profile, serum amylase, and lipase were measured before induction of anesthesia, at 90 min, 4 h, and finally 24 h after induction. Serum lipase levels (P < 0.05), serum triglyceride levels (P < 0.05), and serum very low-density lipoproteins VLDL levels (P < 0.05) were raised significantly after propofol administration from baseline although remained within normal limits. Serum cholesterol levels and serum low-density lipoproteins LDL levels showed a statistically significant fall over 24 h. No significant changes in serum pancreatic amylase levels were seen (P > 0.05). None of the patients developed any clinical features of pancreatitis in the postoperative period. We conclude that despite a small, transient increase in serum triglycerides and pancreatic enzymes, short-term propofol administration in recommended dosages in children of ASA status I and II aged between 1 month and 36 months does not produce any clinically significant effect on serum lipids and pancreatic enzymes. [\hyperlink{Pantoprazole Sodium}{PMID: 26719728}, Munish Chauhan et al., 2013]

\hypertarget{pmid_11737739}{T}here is limited experience on sotalol use in the management of childhood arrhythmias. This study reviews the results of our experience with oral sotalol for treatment and prevention of tachyarrhythmias in children. The records of 62 patients (27 female, 35 male, mean age: 8.5+/-5.3 years) treated with sotalol for supraventricular or ventricular arrhythmias from 1994 to 1999 at our institution were reviewed. Demographic, clinical, echocardiographic, electrocardiographic (ECG), ambulatory ECG and electrophysiologic variables were collected. Forty-two (63.6\%) patients had re-entrant supraventricular tachycardia, eight patients (12.9\%) had atrial tachycardia, one patient (1.6\%) had junctional ectopic tachycardia, four patients (6.5\%) had ventricular tachycardia, and seven patients (11.3\%) had complex ventricular arrhythmias, as evidenced by surface or ambulatory ECG records; or revealed during the electrophysiological study. The mean sotalol dose was 3.9+/-1.2 mg/kg per day. In 15.5+/-13.9 months of sotalol use 50\% (n=31) had complete relief of symptoms and/or arrhythmia and 29\% (n=18) had partial relief. Sotalol was ineffective in 20\% (n=13). Sotalol was more effective in re-entrant type supraventricular tachycardias (P=0.012). Sotalol was the first choice in 35.5\% of patients. The sotalol therapy was initiated in inpatient settings in 40.3\% (25 patients). Complications due to sotalol were seen in six patients (five patients developed bradycardia/pauses, and one patient had torsades de pointes) for which the sotalol dose was modified. In patients with sick sinus syndrome, a pacemaker was implanted and in another patient sotalol was stopped. Sotalol, being an effective and safe drug particularly in children, is a good therapeutic alternative for the preventive treatment of childhood tachyarrhythmias. [\hyperlink{Pantoprazole Sodium}{PMID: 11737739}, A Celiker et al., 2001]

\hypertarget{pmid_25611962}{T}adalafil is a selective Phosphodiesterase-5 inhibitor that has been reported to have vasodilatory and antiproliferative effects on the pulmonary artery. In this study we evaluated the safety and efficacy of oral tadalafil in children with pulmonary arterial hypertension (PAH). This open label study, prospective and interventional was carried out in 25 known patients aged 2 month-5 years in 3 medical centers in Iran, between March 2013-Jun 2014. Tadalafil suspension was administrated at 1 mg/kg daily for all patients. Hemodynamic and safety parameters were assessed at baseline and then monthly for a total of 4 visits. 19 patients received tadalafil as initial therapy, in all visits significant improvements in mean pulmonary arterial pressure were observed (p<0.01). Of the 25 patients, 6 (24\%) had been on sildenafil for longer than 6 months. After transition from sildenafil to tadalafil clinical improvement was noted (p<0.05). Administration of tadalafil suspension was generally safe and well tolerated. Nausea was the most frequently reported adverse events which occurred in 3 patients during treatment. Oral tadalafil was administered easily and tolerated well and improved mean pulmonary artery pressure (MPAP) in children with PAH, which suggests that oral tadalafil may be more effective and safer than sildenafil in the treatment of PAH. [\hyperlink{Pantoprazole Sodium}{PMID: 25611962}, A Shiva et al., 2016]

\section*{Colesevelam Hydrochloride}
\subsection*{Result}
\subsubsection*{Answer}

Yes (10-17 years)
Unknown (<10 years)

\subsubsection*{{Explanation}}
\hypertarget{Colesevelam Hydrochloride}
Based on the abstracts available, several studies have specifically evaluated the safety of Colesevelam Hydrochloride in children, particularly those aged 10 to 17 years with heterozygous familial hypercholesterolemia (HeFH):

1. A 32-week clinical trial reviewed the efficacy and safety of colesevelam in pediatric/adolescent patients aged 10-17 years with HeFH. The study found significant benefits in lowering LDL cholesterol and reported that adverse drug reactions were mild in severity. The authors concluded that colesevelam has a mild side effect profile and should be considered in treatment regimens for this population, though they noted that further research would be beneficial for long-term effects [\hyperlink{pmid_24600261}{PMID: 24600261}, Travis Sonnett et al., 2010].

2. A randomized, double-blind, 41-site study evaluated colesevelam in 194 children aged 10 to 17 years with HeFH. The study found significant reductions in LDL cholesterol and other lipid parameters. Adverse events were evaluated, and the treatment was generally well tolerated [\hyperlink{pmid_19879596}{PMID: 19879596}, Evan A Stein et al., 2010].

3. Another review summarized that colesevelam is approved in the US for the treatment of pediatric patients aged 10-17 years with HeFH, either as monotherapy or in combination with a statin. The referenced 32-week trial found colesevelam to be effective and generally well tolerated for up to 26 weeks in this age group [\hyperlink{pmid_20218749}{PMID: 20218749}, Caroline M Perry et al., 2010].

4. A review article also stated that colesevelam is indicated for use in boys and postmenarchal girls aged 10-17 years with heterozygous FH, and that available data demonstrate it is efficacious and well tolerated in this population, with gastrointestinal disorders being the most common adverse events, rarely leading to withdrawal [\hyperlink{pmid_23916045}{PMID: 23916045}, Michael Davidson et al., 2013].

No abstracts were found that evaluated the safety of colesevelam in children younger than 10 years old. The available studies and reviews focus exclusively on the 10-17 year age range, and there is no evidence from these abstracts regarding safety in children under 10.

In summary:
- For children aged 10-17 years with HeFH, targeted studies affirm the safety of colesevelam hydrochloride, with mild adverse events and good tolerability.
- For children under 10 years, there is no data available in the abstracts to determine safety.

\subsection*{Abstracts}
\hypertarget{pmid_24600261}{C}olesevelam hydrochloride is a synthetic, nonsystemically absorbed polymer that functions as a bile acid sequestrant for the treatment of hypercholesterolemia. Recently, colesevelam was investigated for the treatment of heterozygous familial hypercholesterolemia (HeFH) in the pediatric/adolescent population aged 10-17 years. The purpose of this article is to review the disease state of HeFH in children and adolescents, review the pharmacologic mechanism of action, kinetics, and safety profile of colesevelam, analyze the results of a recent clinical trial of colesevelam in the pediatric/adolescent HeFH population, and discuss the role of colesevelam as a viable treatment option for HeFH. A literature search using Medline (1966-03 May 2010), PubMed (1950-03 May 2010), Science Direct (1994-03 May 2010), and International Pharmaceutical Abstracts (2004-2010) was performed using the search term colesevelam. English language, original research, and review articles were examined, and citations from these articles were also assessed. The manufacturer's prescribing information and the Food and Drug Administration review of the new drug application for the powder formulation were also examined. A 32-week trial was performed investigating the efficacy of colesevelam as monotherapy or combination therapy with a stable statin regimen. Upon completion of the trial, significant benefits were found in regard to the treatment of HeFH and the lowering of low-density lipoprotein cholesterol, total cholesterol, and other secondary measures. Safety and tolerability were also examined throughout the duration of the clinical trial, with adverse drug reactions considered mild in severity. Colesevelam has been shown to reduce low-density lipoprotein cholesterol levels significantly in pediatric/adolescent patients with HeFH, while maintaining a mild side effect profile. Although further research would be beneficial for long-term effects in this population, colesevelam should be considered when developing a treatment regimen for HeFH in the pediatric/adolescent population. [\hyperlink{Colesevelam Hydrochloride}{PMID: 24600261}, Travis Sonnett et al., 2010]

\hypertarget{pmid_19879596}{E}valuate the efficacy and safety of colesevelam hydrochloride in children with heterozygous familial hypercholesterolemia (heFH). This was a randomized, double-blind, 41-site study in 194 children aged 10 to 17 years (inclusive) with heFH (statin-naïve or on a stable statin regimen). After a 4-week stabilization period (period I), subjects were randomized 1:1:1 to placebo, colesevelam 1.875 g/d, or colesevelam 3.75 g/d for 8 weeks (period II). All then received open-label colesevelam 3.75 g/d for 18 weeks (period III), with follow-up 2 weeks later. The primary endpoint was percent change in low-density lipoprotein (LDL)-cholesterol from baseline to week 8. Secondary endpoints included percent change in other lipoprotein variables, including non-high-density lipoprotein (non-HDL)-cholesterol. Adverse events were also evaluated. At week 8, a significant difference from baseline in LDL-cholesterol was reported with colesevelam 1.875 g/d (-6.3\%; P = .031) and colesevelam 3.75 g/d (-12.5\%; P < .001) compared with placebo. Significant treatment effects were also reported for total cholesterol (-7.4\%), non-HDL-cholesterol (-10.9\%), HDL-cholesterol (+6.1\%), apolipoprotein A-I (+6.9\%), and apolipoprotein B (-8.3\%) and a nonsignificant effect for triglycerides (+5.1\%) with colesevelam 3.75 g/d compared with placebo at week 8. These treatment effects were maintained during period III. Colesevelam significantly lowered LDL-cholesterol levels in children with heFH. [\hyperlink{Colesevelam Hydrochloride}{PMID: 19879596}, Evan A Stein et al., 2010]

\hypertarget{pmid_20218749}{C}olesevelam hydrochloride (colesevelam), a non-absorbed, synthetic, lipid-lowering polymer, is a bile acid sequestrant. Colesevelam binds with high affinity to bile acids within the gastrointestinal tract, thereby inhibiting the reabsorption of bile acids, resulting in decreases in serum low-density lipoprotein cholesterol (LDL-C) levels. Colesevelam is available as tablets and as powder for oral suspension. At dosages of 3.75 g once daily or 1.875 g twice daily, colesevelam is approved in the US for the treatment of pediatric patients aged 10-17 years with heterozygous familial hypercholesterolemia. Colesevelam may be administered as monotherapy or in combination with an HMG-CoA reductase inhibitor (statin). A 32-week trial was conducted and consisted of a stablilization period ( approximately 4 weeks), a randomized period (8 weeks), an open-label period (18 weeks), and a 2-week follow-up period. In the 8-week, randomized, double-blind, placebo-controlled period of the trial, colesevelam (tablets), as monotherapy or with a statin, was an effective treatment for pediatric patients with heterozygous familial hypercholesterolemia. At week 8, recipients of colesevelam 3.75 g/day had significant percentage reductions from baseline in mean LDL-C levels (primary endpoint) compared with placebo recipients. Significant beneficial treatment effects for colesevelam 3.75 g/day versus placebo were also reported for several other lipid/lipoprotein parameters at week 8 of the study. The reported treatment effects on lipid/lipoprotein parameters were maintained over a subsequent 18-week, open-label, noncomparative period, when all patients received colesevelam 3.75 g/day. Colesevelam 3.75 g/day was generally well tolerated for up to 26 weeks by pediatric patients with heterozygous familial hypercholesterolemia. [\hyperlink{Colesevelam Hydrochloride}{PMID: 20218749}, Caroline M Perry et al., 2010]

\hypertarget{pmid_19862667}{T}he bile acid sequestrant, colesevelam hydrochloride, is approved for glycemic control in adults with type 2 diabetes. In three double-masked, placebo-controlled studies, colesevelam hydrochloride 3.75 g/day demonstrated its glycemic-lowering properties when added to existing metformin-, insulin-, or sulfonylurea-based therapy in adults with inadequately controlled type 2 diabetes. This was a 52-week open-label extension study conducted at 63 sites in the United States and one site in Mexico to further evaluate the safety and tolerability of colesevelam hydrochloride in subjects with type 2 diabetes. All subjects who completed the three double-masked, placebo-controlled studies were eligible to enroll in this open-label extension. In total, 509 subjects enrolled and received open-label colesevelam hydrochloride 3.75 g/day for 52 weeks. Safety and tolerability of colesevelam hydrochloride was evaluated by the incidence and severity of adverse events. In total, 360 subjects (70.7\%) completed the extension. Of the safety population, 361 subjects (70.9\%) experienced an adverse event, most (88.1\%) being mild or moderate in severity. Fifty-six adverse events (11.0\%) were drug-related; the most frequent drug-related adverse events were constipation and dyspepsia. Thirty-five subjects (6.9\%) discontinued due to an adverse event. Fifty-four subjects (10.6\%) experienced a serious adverse event; only one was considered drug-related (diverticulitis). Seventeen subjects (3.3\%) experienced hypoglycemia; most episodes were mild or moderate in severity. Glycemic improvements with colesevelam hydrochloride were seen without change in weight over 52 weeks (0.2 kg mean reduction from baseline). Colesevelam hydrochloride was safe and well-tolerated as long-term therapy for patients with type 2 diabetes. [\hyperlink{Colesevelam Hydrochloride}{PMID: 19862667}, A B Goldfine et al., 2010]

\hypertarget{pmid_11583720}{C}olesevelam hydrochloride is a novel, potent, non-absorbed lipid-lowering agent previously shown to reduce low density lipoprotein (LDL) cholesterol. To examine the efficacy and safety of coadministration of colesevelam and atorvastatin, administration of these agents alone or in combination was examined in a double-blind study of 94 hypercholesterolemic men and women (baseline LDL cholesterol > or =160 mg/dl). After 4 weeks on the American Heart Association Step I diet, patients were randomized among five groups: placebo; colesevelam 3.8 g/day; atorvastatin 10 mg/day; coadminstered colesevelam 3.8 g/day plus atorvastatin 10 mg/day; or atorvastatin 80 mg/day. Fasting lipids were measured at screening, baseline and 2 and 4 weeks of treatment. LDL cholesterol decreased by 12-53\% in all active treatment groups (P<0.01). LDL cholesterol reductions with combination therapy (48\%) were statistically superior to colesevelam (12\%) or low-dose atorvastatin (38\%) alone (P<0.01), but similar to those achieved with atorvastatin 80 mg/day (53\%). Total cholesterol decreased 6-39\% in all active treatment groups (P<0.05). High density lipoprotein cholesterol increased significantly for all groups including placebo (P<0.05). Triglycerides decreased in patients taking atorvastatin alone (P<0.05), but were unaffected by colesevelam alone or in combination. The frequency of side effects did not differ among groups. At recommended starting doses of each agent, coadministration of colesevelam and atorvastatin was well tolerated, efficacious and produced additive LDL cholesterol reductions comparable to those observed with the maximum atorvastatin dose. [\hyperlink{Colesevelam Hydrochloride}{PMID: 11583720}, D Hunninghake et al., 2001]

\hypertarget{pmid_12040732}{T}he pharmacology, pharmacodynamics, clinical efficacy, drug interactions, adverse effects, and dosage and administration of colesevelam hydrochloride are reviewed. Colesevelam hydrochloride is a nonabsorbed lipid-lowering agent approved for use alone or in combination with hydroxymethylglutaryl-coenzyme A (HMG-CoA) reductase inhibitors for the reduction of low-density-lipoprotein (LDL) cholesterol in patients with primary hypercholesterolemia. Colesevelam forms nonabsorbable complexes with bile acids in the gastrointestinal (GI) tract, resulting in changes in plasma lipid levels, including total, LDL, and high-density-lipoprotein cholesterol and triglycerides. Colesevelam has been reported to be four to six times as potent as traditional bile acid sequestrants (BASs), perhaps because of its greater binding affinity for glycocholic acid. Unlike cholestyramine and colestipol, colesevelam appears to reduce LDL cholesterol in a dose-dependent manner. In clinical trials, colesevelam demonstrated efficacy either alone or in combination with HMG-CoA reductase inhibitors in the treatment of primary hypercholesterolemia. Combination therapy appeared to be more effective than monotherapy. Although infection, headache, and GI adverse effects have been reported for colesevelam, the rates do not differ significantly from those occurring with placebo. The constipation that typically hinders compliance with traditional BASs is minimal. In one study, the rate of compliance with colesevelam was 93\%. There is little evidence of clinically significant interactions involving colesevelam. The maintenance dosage is three 625-mg tablets twice daily or six tablets once daily, taken with meals. Colesevelam provides an effective alternative to cholestyramine and colestipol while offering the potential for fewer adverse effects and better compliance. Studies are needed to directly compare colesevelam with traditional BASs. [\hyperlink{Colesevelam Hydrochloride}{PMID: 12040732}, Karen L Steinmetz et al., 2002]

\hypertarget{pmid_23805883}{C}olesevelam hydrochloride is used as an adjunct to diet and exercise to reduce elevated low-density lipoprotein (LDL) cholesterol in patients with primary hyperlipidemia as well as to improve glycemic control in patients with type 2 diabetes. This is likely to result in submission of abbreviated new drug applications (ANDA). This study was conducted to compare the efficacy of two tablet products of colesevelam hydrochloride based on the in vitro binding of bile acid sodium salts of glycocholic acid (GC), glycochenodeoxycholic acid (GCDA) and taurodeoxycholic acid (TDCA). Kinetic binding study was carried out with constant initial bile salt concentrations as a function of time. Equilibrium binding studies were conducted under conditions of constant incubation time and varying initial concentrations of bile acid sodium salts. The unbound concentration of bile salts was determined in the samples of these studies. Langmuir equation was utilized to calculate the binding constants k1 and k2. The amount of the three bile salts bound to both the products reached equilibrium at 3 h. The similarity factor (f2) was 99.5 based on the binding profile of total bile salts to the test and reference colesevelam tablets as a function of time. The 90\% confidence interval for the test to reference ratio of k2 values were 96.06-112.07 which is within the acceptance criteria of 80-120\%. It is concluded from the results that the test and reference tablets of colesevelam hydrochloride showed a similar in vitro binding profile and capacity to bile salts. [\hyperlink{Colesevelam Hydrochloride}{PMID: 23805883}, Yellela S R Krishnaiah et al., 2014]

\hypertarget{pmid_28741653}{C}hloral hydrate is commonly used to sedate children for painless procedures. Children may recover more quickly after sedation with dexmedetomidine, which has a shorter half-life. We randomly allocated 196 children to chloral hydrate syrup 50 mg.kg [\hyperlink{Colesevelam Hydrochloride}{PMID: 28741653}, V M Yuen et al., 2017] Colesevelam hydrochloride (HCl) was approved in January 2008 as an adjunct therapy for improving glycemic control in patients with type 2 diabetes mellitus (T2DM). Colesevelam HCl is a bile acid sequestrant that has been shown to significantly improve both glycemic control and the lipid profile in patients with T2DM when added to metformin-, sulfonylurea-, or insulin-based therapy. In addition, colesevelam HCl may be useful for reducing glucose and low-density lipoprotein cholesterol levels in patients with prediabetes (defined as fasting plasma glucose levels of 100-125 mg/dL or 2-hour poststimulation glucose levels of 140-199 mg/dL), who have an increased cardiovascular risk. As colesevelam HCl is a unique agent-with both significant glycemic and lipid benefits-it has the potential to play an important role in the management of T2DM. This article reviews the place of colesevelam HCl in therapy (both for T2DM and prediabetes), the benefits of early, intensive treatment of T2DM, and the importance of safe glycemic control later in the disease process. [\hyperlink{Colesevelam Hydrochloride}{PMID: 28741653}, Yehuda Handelsman et al., 2009]

\hypertarget{pmid_18458145}{H}yperglycemia is a risk factor for microvascular complications and may increase the risk of cardiovascular disease in patients with type 2 diabetes. This study tested the LDL cholesterol-lowering agent colesevelam HCl (colesevelam) as a potential novel treatment for improving glycemic control in patients with type 2 diabetes on sulfonylurea-based therapy. A 26-week, randomized, double-blind, placebo-controlled, parallel-group, multicenter study was carried out between August 2004 and August 2006 to evaluate the efficacy and safety of colesevelam for reducing A1C in adults with type 2 diabetes whose glycemic control was inadequate (A1C 7.5-9.5\%) with existing sulfonylurea monotherapy or sulfonylurea in combination with additional oral antidiabetes agents. In total, 461 patients were randomized (230 given colesevelam 3.75 g/day and 231 given placebo). The primary efficacy measurement was mean placebo-corrected change in A1C from baseline to week 26 in the intent-to-treat population (last observation carried forward). The least squares (LS) mean change in A1C from baseline to week 26 was -0.32\% in the colesevelam group and +0.23\% in the placebo group, resulting in a treatment difference of -0.54\% (P < 0.001). The LS mean percent change in LDL cholesterol from baseline to week 26 was -16.1\% in the colesevelam group and +0.6\% in the placebo group, resulting in a treatment difference of -16.7\% (P < 0.001). Furthermore, significant reductions in fasting plasma glucose, fructosamine, total cholesterol, non-HDL cholesterol, and apolipoprotein B were demonstrated in the colesevelam relative to placebo group at week 26. Colesevelam improved glycemic control and reduced LDL cholesterol levels in patients with type 2 diabetes receiving sulfonylurea-based therapy. [\hyperlink{Colesevelam Hydrochloride}{PMID: 18458145}, Vivian A Fonseca et al., 2008]

\hypertarget{pmid_11895050}{T}o assess whether colesevelam hydrochloride is absorbed in healthy volunteers. A single-center, open-label, radiolabeled study was performed with 16 healthy volunteers. Subjects were administered non-radiolabeled colesevelam hydrochloride 1.9 g twice daily for 4 weeks, followed by a single dose of [14C]-colesevelam 2.4 g (480 pCi). These subjects continued to receive non-radioactive colesevelam 1.9 g twice daily for 4 days after administration of the radiolabeled dose. Blood, urine, and feces were collected immediately prior to administration of [14C]-colesevelam and at specified intervals after administration. The whole-blood equivalent concentration of colesevelam was calculated using data collected throughout the 96 hours following radiolabeled drug administration. The proportion of [14C]-colesevelam excreted through urine or feces was calculated based on the amount of radioactivity recovered up to 216 hours after the radiolabeled dose. The mean cumulative total recovery of [14C]-colesevelam in urine and feces was 0.05\% and 74\%, respectively. Excluding 2 subjects for whom cumulative recovery was <25\%, the mean cumulative fecal recovery was 82\%. The mean maximum whole-blood equivalent concentration of colesevelam was 0.165+/-0.10 microg equiv/g 72 hours after administration of [14C]-colesevelam, which was estimated to be 0.04\% of the administered dose. All blood samples contained <4 times the number of background counts (dpm). The cumulative recovery data in urine and feces are consistent with the conclusion that colesevelam is not absorbed and is excreted entirely through the gastrointestinal system. [\hyperlink{Colesevelam Hydrochloride}{PMID: 11895050}, Dennis P Heller et al., 2002]

\hypertarget{pmid_11605698}{T}o evaluate the efficacy, tolerability, and safety of colesevelam hydrochloride, a new nonsystemic lipid-lowering agent. In this double-blind, placebo-controlled trial performed in 1998, 494 patients with primary hypercholesterolemia (low-density lipoprotein [LDL] cholesterol level > or = 130 mg/dL and < or = 220 mg/dL) were randomized to receive placebo or colesevelam (2.3 g/d, 3.0 g/d, 3.8 g/d, or 4.5 g/d) for 24 weeks. Fasting serum lipid profiles were measured to assess efficacy. Adverse events were monitored, and discontinuation rates and compliance rates were analyzed. The primary outcome measure was the mean absolute change of LDL cholesterol from baseline to the end of the 24-week treatment period. Colesevelam lowered mean LDL cholesterol levels 9\% to 18\% in a dose-dependent manner (P<.001), with a median LDL cholesterol reduction of 20\% at 4.5 g/d. The reduction in LDL cholesterol levels was maximal after 2 weeks and sustained throughout the study. Mean total cholesterol levels decreased 4\% to 10\% (P<.001), while median high-density lipoprotein cholesterol levels increased 3\% to 4\% (P<.001). Median triglyceride levels increased by 5\% to 10\% in placebo and colesevelam treatment groups relative to baseline (P<.05), but none of these differences were significantly different from placebo. Mean apolipoprotein B levels decreased 6\% to 12\% in an apparent dose-dependent manner (P<.001). No significant differences occurred in adverse events or discontinuation rates between groups, and compliance rates were between 88\% and 92\% for all groups. Colesevelam was efficacious, decreasing mean LDL cholesterol levels by up to 18\%, and well tolerated without serious adverse events. [\hyperlink{Colesevelam Hydrochloride}{PMID: 11605698}, W Insull et al., 2001]

\hypertarget{pmid_23916045}{F}amilial hypercholesterolemia (FH) is a common autosomal co-dominant genetic disorder that results in severely increased levels of LDL-C. Patients with FH are at an increased risk for premature coronary artery disease. Expert panels therefore recommend initiation of lipid-lowering therapy in childhood to reduce the very high lifetime risk of coronary artery disease. The bile acid sequestrant colesevelam is indicated to reduce elevated LDL-C levels in adults with primary hyperlipidemia and in boys and postmenarchal girls (aged 10-17 years) with heterozygous FH. The purpose of this article was to review currently available data on the use of colesevelam in the treatment of heterozygous FH. PubMed and Google Scholar were searched to identify clinical trials evaluating colesevelam in patients with heterozygous FH. The search returned 2 results (both multicenter, multinational studies): 1 study conducted in adults and the other in pediatric patients. In the study in adults with refractory FH, the addition of colesevelam to a maximally tolerated regimen of a statin plus ezetimibe provided a significantly greater reduction from baseline in LDL-C levels compared with placebo. Significantly greater reductions from baseline in LDL-C were also seen in pediatric patients with heterozygous FH receiving colesevelam (alone or in combination with statins) compared with placebo. Colesevelam was generally well tolerated in studies in patients with FH; consistent with other colesevelam studies, gastrointestinal disorders were the most common drug-related adverse events, but these events rarely led to study withdrawal. Currently available data demonstrate that colesevelam, alone or in combination therapy, is efficacious and well tolerated in the treatment of heterozygous FH in adults and pediatric patients, supporting its use as a treatment option in both of these patient populations. [\hyperlink{Colesevelam Hydrochloride}{PMID: 23916045}, Michael Davidson et al., 2013]

\hypertarget{pmid_20527137}{O}nly a few corticosteroids for topical use have proven safe and effective in pediatric populations down to 3 months of age. The authors report the results of a study designed to assess the efficacy and safety of hydrocortisone butyrate (HCB) 0.1\% in lipocream (LCr) vehicle in infants and children. A total of 264 boys and girls 3 months to less than 18 years old, with stable, mild to moderate atopic dermatitis affecting at least 10\% body surface area applied HCB 0.1\% in LCr or LCr alone twice daily for up to 1 month without occlusion. Primary end-points included: percent of patients who achieved treatment success based on physician global assessments. Secondary endpoint included: difference in pruritus and Eczema Area and Severity Index (EASI) at day 29. Treatment was significant (P < 0.001) for HCB 0.1\% LCr over vehicle. No serious nor significant adverse events were reported. Results are representative of a short duration treatment for a chronic disease. HCB 0.1\% in LCr is more effective than its vehicle in pediatric populations down to 3 months of age without significant adverse events when used twice a day for up to 1 month. [\hyperlink{Colesevelam Hydrochloride}{PMID: 20527137}, William Abramovits et al., ]

\hypertarget{pmid_25246305}{T}he aim of this study was to compare the efficacy and safety of different oral chloral hydrate and dexmedetomidine doses used for sedation during electroencephalography (EEG) in children. One hundred sixty children aged 1 to 9 years with American Society of Anesthesiologists physical status I-II who were uncooperative during EEG recording or who were referred to our electrodiagnostic unit for sleep EEG were included to the study. The patients were randomly assigned into 4 groups. In groups D1 and D2, patients received oral dexmedetomidine doses of 2 and 3 µg/kg, respectively. In group C1 and C2, patients received oral chloral hydrate doses of 50 and 100 mg/kg, respectively. The induction time was significantly shorter in group C2 compared with other groups (P = .000). The rate of adverse effects was significantly higher in group C2 compared with the dexmedetomidine groups (D1 and D2; P = .004). In conclusion, dexmedetomidine can be used safely for sedation during EEG in children.  [\hyperlink{Colesevelam Hydrochloride}{PMID: 25246305}, Hakan Gumus et al., 2015] The complications of type 2 diabetes mellitus (DM) can begin early in the progression from impaired glucose tolerance to type 2 DM. Metformin is recommended as initial drug therapy for managing hyperglycemia in type 2 DM. The bile acid sequestrant colesevelam hydrochloride (HCl) is approved in the United States for glycemic control in adults with type 2 DM. Colesevelam HCl improves glycemic control and reduces low-density lipoprotein-cholesterol in patients inadequately controlled on metformin-, sulfonylurea-, or insulin-based therapy. This trial is designed to evaluate whether initial therapy with metformin + colesevelam HCl provides greater glucose control and additional lipid and lipoprotein benefits, as compared to metformin alone in drug-naïve patients with type 2 DM, and whether treatment with colesevelam HCl has a beneficial effect on lipid and glucose levels in drug-naïve patients with impaired glucose tolerance and/or impaired fasting glucose (prediabetes). In this multicenter, randomized, double-blind, placebo-controlled, parallel-group trial, drug-naïve patients with type 2 DM will be randomized 1 : 1 to metformin + colesevelam HCl or metformin + matching placebo, while those with prediabetes will be randomized 1 : 1 to colesevelam HCl or placebo, for 16 weeks of treatment. The primary efficacy endpoint will be change in glycosylated hemoglobin (HbA(1c)) in patients with type 2 DM and change in low-density lipoprotein-cholesterol levels in patients with prediabetes. A potential limitation is that there is no direct comparator for the dual glucose- and lipid-lowering effect of colesevelam HCl in the prediabetes cohort. However, results of this trial will help to define the extent to which colesevelam HCl can help improve cardiometabolic risk factors for complications of type 2 DM in the first-line environment, and will also indicate the extent to which early intervention with colesevelam HCl can help to correct metabolic abnormalities associated with prediabetes. [\hyperlink{Colesevelam Hydrochloride}{PMID: 25246305}, Michael R Jones et al., 2009]

\hypertarget{pmid_33655976}{C}hildren evaluated in the emergency department for head trauma often undergo computed tomography (CT), with some uncooperative children requiring pharmacological sedation. Chloral hydrate (CH) is a sedative that has been widely used, but its rectal use for child sedation after head trauma has rarely been studied. The objective of this study was to document the safety and efficacy of rectal CH sedation for cranial CT in young children.We retrospectively studied all the children with head trauma who received rectal CH sedation for CT in the emergency department from 2016 to 2019. CH was administered rectally at a dose of 50 mg/kg body weight. When sedation was achieved, CT scanning was performed, and the children were monitored until recovery. The sedative safety and efficacy were analyzed.A total of 135 children were enrolled in the study group, and the mean age was 16.05 months. The mean onset time was 16.41 minutes. Successful sedation occurred in 97.0\% of children. The mean recovery time was 71.59 minutes. All of the vital signs were within normal limits after sedation, except 1 (0.7\%) with transient hypoxia. There was no drug-related vomiting reaction in the study group. Adverse effects occurred in 11 patients (8.1\%), but all recovered completely. Compared with oral CH sedation, rectal CH sedation was associated with quicker onset (P < .01), higher success rate (P < .01), and lower adverse event rate (P < .01).Rectal CH sedation can be a safe and effective method for CT imaging of young children with head trauma in the emergency department. [\hyperlink{Colesevelam Hydrochloride}{PMID: 33655976}, Quanmin Nie et al., 2021]

\hypertarget{pmid_24627951}{T}o determine the safety and efficacy of high-dose oral chloral hydrate for pediatric ophthalmic procedures. This study is a retrospective review of a quality audit of pediatric sedation for ophthalmic evaluation and imaging performed at King Khaled Eye Specialist Hospital between January 1 and December 31, 2011, in children aged 1 month to 6 years. Three hundred fifty-eight of 380 (94.2\%) sedation procedures were successful after a single dose of chloral hydrate, with 356 of 380 (93.7\%) children sedated within 45 minutes of the first dose. The total success rate of the sedation procedure increased to 97.9\% (372 of 380) when a second dose was administered. Children adequately sedated after a single dose of chloral hydrate were on average younger and weighed less than children who required additional doses. No major adverse events were documented. The use of chloral hydrate sedation for ophthalmic evaluation and imaging was safe and effective in this patient population with a high rate of procedure completion. [\hyperlink{Colesevelam Hydrochloride}{PMID: 24627951}, Michelle E Wilson et al., ]

\hypertarget{pmid_28275979}{S}edation is often required for children undergoing diagnostic procedures. Chloral hydrate has been one of the sedative drugs most used in children over the last 3 decades, with supporting evidence for its efficacy and safety. Recently, chloral hydrate was banned in Italy and France, in consideration of evidence of its carcinogenicity and genotoxicity. Dexmedetomidine is a sedative with unique properties that has been increasingly used for procedural sedation in children. Several studies demonstrated its efficacy and safety for sedation in non-painful diagnostic procedures. Dexmedetomidine's impact on respiratory drive and airway patency and tone is much less when compared to the majority of other sedative agents. Administration via the intranasal route allows satisfactory procedural success rates. Studies that specifically compared intranasal dexmedetomidine and chloral hydrate for children undergoing non-painful procedures showed that dexmedetomidine was as effective as and safer than chloral hydrate. For these reasons, we suggest that intranasal dexmedetomidine could be a suitable alternative to chloral hydrate. [\hyperlink{Colesevelam Hydrochloride}{PMID: 28275979}, Giorgio Cozzi et al., 2017]

\hypertarget{pmid_24953836}{B}ile acid malabsorption (BAM)-associated diarrhea is an important clinical issue in patients with Crohn's disease (CD). We analyzed the efficacy and safety of the bile acid sequestrant colesevelam for treatment of BAM-associated diarrhea in CD patients in a randomized, double-blind, placebo-controlled study. The primary endpoint was the proportion of patients with >30\% reduction of liquid stools/day from baseline to termination visit at week 4. Secondary endpoints were reduction of the number of liquid stools/day, improvement of stool consistency and quality of life. 26 patients were analyzed in the intention-to-treat (ITT) analysis. The primary endpoint was reached by 10 patients (69.7\%) in the colesevelam group compared to 3 patients (27.3\%) in the placebo group (risk difference RD=.394, 95\%CI:[-0.012; 0.706]; P=.0566). In the per-protocol analysis (n=22), the risk difference was statistically significant (RD=.470, 95\%CI:[0.018; 0.788], P(H0: RD=0)=0.0364; 95\% CI:[1.3;54.7]). Regarding secondary endpoints, in the ITT population colesevelam-treated patients had a significant reduction of liquid stools/day at week 4 (median 5.0 to 2.0; P=0.01), while patients treated with placebo had no significant reduction (median 4.0 to 3.0; P=0.42). Significantly more patients in the colesevelam group had improvement of stool consistency of at least one level in the Bristol stool chart, as compared to the placebo group (P=0.003). We found significant differences in favor for colesevelam treatment compared to placebo treatment for CD patients with BAM regarding the reduction of the number of liquid stools/day and stool consistency. ClinicalTrials.gov number: NCT01203254. [\hyperlink{Colesevelam Hydrochloride}{PMID: 24953836}, Florian Beigel et al., 2014]

\hypertarget{pmid_2402648}{C}hloral hydrate has been used extensively to sedate children, but at Brooke Army Medical Center, other drug combinations were becoming increasingly popular due to a perception that chloral hydrate had a high rate of failure, especially with younger or neurologically impaired children. Therefore, 50 children were given the drug before a diagnostic study, and patient data and a sedation score were recorded on a worksheet. Of 50 children, 43 (86\%) were "successfully sedated" on the first attempt with no side effects. Children with neurologic disorders had a much greater (27\% vs 4\%) failure rate than "normal" children. The sedation rate did not significantly differ by age, sex, or initial drug dosage. The study suggest that chloral hydrate is a safe and effective oral sedative but that children with neurologic disorders may need alternative drugs for sedation. [\hyperlink{Colesevelam Hydrochloride}{PMID: 2402648}, P D Rumm et al., 1990]

\hypertarget{pmid_23024102}{W}e conducted this single blind randomized clinical trial to compare the efficacy and safety of oral chloral hydrate and intranasal midazolam for induction of sedation for computerized tomography scan of brain in children. Participants aged 1-10 years (n=60) were randomized to receive 100 mg/kg chloral hydrate orally with intra nasal normal saline OR intranasal midazolam 0.2 mg/kg with oral normal saline. Adequate sedation (Ramsay sedation score of four) was obtained and CT scan completed successfully in 76.7\% of chloral hydrate group and in 40\% of midazolam group (P=0.004). No significant difference was seen for side effects frequency between the two drugs (10\% in chloral hydrate, 3.3\% in midazolam group; P=0.34). We conclude that oral chloral hydrate can be considered as a safe and effective drug for sedation in children undergoing CT scan of brain. [\hyperlink{Colesevelam Hydrochloride}{PMID: 23024102}, Razieh Fallah et al., 2013]

\hypertarget{pmid_23776789}{H}yperglycemia and hyperlipidemia are both risk factors for the development of various complications in patients with type 2 diabetes mellitus. Colesevelam hydrochloride is a novel agent that can improve both hypercholesterolemia and hyperglycemia in such patients. It is an orally administered bile acid sequestrant with high capacity for binding bile acids. This drug can offer potential new diabetes treatment along with other drugs. [\hyperlink{Colesevelam Hydrochloride}{PMID: 23776789}, Kavita Sekhri et al., 2011]

\hypertarget{pmid_31534313}{C}hloral hydrate (CH), as a sedation agent, is widely used in children for diagnostic or therapeutic procedures. However, it has not come into the market and is currently only used as hospital preparation in China. This review aims to systematically evaluate the efficacy of CH in children of all age groups for sedation before medical procedures. Seven electronic databases and three clinical trial registry platforms were searched and the deadline was September 2018. Randomized controlled trials (RCTs) evaluating the efficacy of CH for sedation in children were included by two reviewers. The extracted information included success rate of sedation, sedation latency and sedation duration. The Cochrane risk of bias tool was applied to assess the risk of bias. The outcomes were analyzed by Review Manager 5.3 software and expressed as relative risks (RR) or Mean Difference (MD) with 95\% confidence interval (CI). Heterogeneity was assessed with I-squared (I A total of 24 RCTs involving 3564 children of CH for sedation were included in the meta-analysis. Compared to placebo group, CH group had a significant increase in success rate of sedation when used for painless and painful procedure (RR=4.15, 95\% CI [1.21, 14.24], P=0.02; RR=1.28, 95\% CI [1.17, 1.40], P<0.01), which included 22 and 455 children for this analysis, respectively. Compared to midazolam group, CH group had a significant increase in success rate of sedation (RR=1.63, 95\% CI [1.48, 1.79], I From the extrapolation of the existing literature, CH oral solution is an appropriate effective alternative for sedation in pediatrics. [\hyperlink{Colesevelam Hydrochloride}{PMID: 31534313}, Zhe Chen et al., 2019]

\hypertarget{pmid_15951862}{D}iagnostic and therapeutic procedures in children are made easier using sedation. However, there is no consensus about which drug should be used to achieve this. Furthermore, none of the drugs used for sedation are risk free. The aim of this work is to study sedation indications, effectiveness, and safety at our center. A prospective observational study conducted at the Pediatric Day Care Unit, King Fahad National Guard Hospital, Riyadh, Saudi Arabia. The study covered 17.5 weeks in 2 periods: May 9th 1999 to June 13th 1999 and October 31st 2001 to February 11th 2002. Children <12 years were included. Collected data included demographics, indication, drug dosing and outcome. Data were reported as mean +/- SD. We included 148 patients, age 38 +/- 30 months. Adequate sedation was achieved in 79\% after initial chloral hydrate (CH) dose of 56.9 +/- 9.3 mg/kg, in 95\% after adding 18.5 +/- 6.4 mg/kg CH and in 96\% after adding second drug. Compared to nonrespondents, first CH dose respondents were younger and lower in weight. The CH side effects were few and mild. Chloral hydrate is a safe and effective agent for sedation in children with an age and weight dependent response. [\hyperlink{Colesevelam Hydrochloride}{PMID: 15951862}, Omar M Hijazi et al., 2005]

\section*{Famotidine}
\subsection*{Result}
\subsubsection*{Answer}

Neonates: Yes
Infants (1.3–10.5 months): Yes
Children (>1 year): Yes
Children 4–18 years: Yes

\subsubsection*{{Explanation}}
\hypertarget{Famotidine}
Based on the available abstracts, several targeted studies have evaluated the safety of famotidine in children across different age ranges:

1. **Neonates (newborns):**
   - A study of 54 neonates with stress ulcers treated with intravenous famotidine (0.5 mg/kg every 12 hours) reported that famotidine was effective and did not lead to abnormal respiration, heart rate, or blood pressure. No gastrointestinal or dermatological side effects were observed, and no augmented side effects from other drugs were noted. The authors concluded that famotidine is effective and safe for the treatment of stress ulcers in neonates [\hyperlink{pmid_18947477}{PMID: 18947477}, Yan-Yong Wu et al., 2008].

2. **Infants (1.3–10.5 months):**
   - In a randomized, placebo-controlled trial of 35 infants with gastro-oesophageal reflux disease, famotidine (0.5–1.0 mg/kg) was associated with some non-serious, possibly drug-related adverse events (agitation, irritability, somnolence, anorexia, headache, vomiting, hiccups, candidiasis), but no serious adverse events were reported. The study suggests that famotidine may be efficacious and generally safe in this age group, though agitation and headache may occur. The authors recommend further large, placebo-controlled studies [\hyperlink{pmid_12752346}{PMID: 12752346}, S R Orenstein et al., 2003].

3. **Children (over 1 year):**
   - Multiple studies, including pharmacokinetic and clinical trials, have shown that famotidine's pharmacokinetics and pharmacodynamics in children over 1 year are similar to adults. Studies involving over 150 children support the safety and efficacy of famotidine for peptic ulcer disease and esophagitis, with recommended dosing and no significant safety concerns reported [\hyperlink{pmid_8853932}{PMID: 8853932}, L P James et al., 1996; \hyperlink{pmid_7846741}{PMID: 7846741}, A Nagita et al., 1994].
   - In a study of 17 severely handicapped children (age not specified, but described as "children"), famotidine was effective for reflux esophagitis, and no side effects were observed [\hyperlink{pmid_2889529}{PMID: 2889529}, S Miyake et al., 1987].
   - A randomized controlled trial in children aged 2–17 years found that preoperative famotidine increased gastric pH without significant adverse effects [\hyperlink{pmid_1887750}{PMID: 1887750}, J S Jahr et al., 1991].
   - A study of a midazolam-famotidine suppository in children aged 5 months to 7 years found it effective as a premedicant, with no safety concerns reported [\hyperlink{pmid_7739102}{PMID: 7739102}, A Iwamoto et al., 1995].

4. **Children and Adolescents (4–18 years):**
   - Systematic reviews and randomized controlled trials in children aged 4–18 years with functional gastrointestinal disorders or functional dyspepsia found that famotidine may provide symptom improvement, with no serious adverse effects reported. However, the overall quality of evidence was low, and the authors call for more high-quality trials [\hyperlink{pmid_25449223}{PMID: 25449223}, Judith J Korterink et al., 2015; \hyperlink{pmid_30360666}{PMID: 30360666}, Pamela D Browne et al., 2018].

5. **Children with Autism Spectrum Disorders (3.8–8.1 years):**
   - A small pilot study of 9 boys found no significant safety concerns with famotidine at 2 mg/kg/day, though the study was not primarily designed to assess safety [\hyperlink{pmid_11459079}{PMID: 11459079}, L A Linday et al., 2001].

**Summary by Age Range:**
- **Neonates:** Targeted study affirms safety for short-term use in stress ulcers.
- **Infants (1.3–10.5 months):** Targeted study suggests general safety, with some non-serious adverse events; no serious adverse events reported.
- **Children (>1 year, including up to adolescence):** Multiple targeted studies affirm safety for various indications (ulcers, esophagitis, premedication, reflux), with no significant safety concerns.
- **Children 4–18 years:** Systematic reviews and RCTs report no serious adverse effects, but evidence quality is low and more studies are needed.

**Conclusion:** There is evidence from targeted studies affirming the safety of famotidine in neonates, infants, and children over 1 year, with no studies indicating it is unsafe in these populations. For children 4–18 years, while no serious adverse effects are reported, the evidence is of low quality and more research is needed.

\subsection*{Abstracts}
\hypertarget{pmid_8853932}{F}amotidine, an H2 receptor antagonist, has several potential advantages over cimetidine and ranitidine. These advantages include its potency, relatively longer elimination half-life, and lack of interaction with the cytochrome P450 isoforms. Eight studies addressing the use of famotidine in paediatric patients have been published. Data from these studies demonstrate that the pharmacokinetics and pharmacodynamics of intravenous famotidine appear to be similar in both children over the age of 1 year and adults. These data support a starting paediatric dosage for intravenous famotidine of 0.5 mg/kg every 8 to 12 hours. In addition, the safety and efficacy of famotidine in the treatment of peptic ulcer disease and esophagitis in paediatric patients is supported by these studies involving over 150 children. Future studies with famotidine in paediatrics should address its disposition in children under the age of 1 year and in children with compromised renal function, as well as the bioavailability of the oral formulation. [\hyperlink{Famotidine}{PMID: 8853932}, L P James et al., 1996]

\hypertarget{pmid_16028153}{B}ecause of concerns about arthrotoxicity, fluoroquinolones are restricted for use in children. This study describes the safety and efficacy of gatifloxacin when used for treatment of children with recurrent acute otitis media (ROM) or acute otitis media (AOM) treatment failure (AOMTF). We performed an analysis of 867 children included in 4 clinical trials who had ROM and/or AOMTF and were treated with gatifloxacin (10 mg/kg once daily for 10 days). Gatifloxacin had adverse event rates that were similar overall to those of a comparator antibiotic (amoxicillin-clavulanate), except for increased diarrhea in children <2 years old receiving amoxicillin-clavulanate. There was no evidence of arthrotoxicity, hepatotoxicity, alteration of glucose homeostasis, or central nervous system toxicity acutely or during 1 year follow-up in any child. Regarding efficacy, in 2 noncomparative trials, the gatifloxacin cure rate of AOM was 89\% (95\% confidence interval [CI], 83\%-95\%) at the test of cure (TOC) visit, 3-10 days after completion of therapy. In 2 comparative trials of gatifloxacin versus amoxicillin-clavulanate, the efficacy of gatifloxacin was 88\% (95\% CI, 82\%-94\%). Gatifloxacin led to better clinical outcomes than amoxicillin-clavulanate for AOMTF (91\% vs. 81\%; P=.029), for AOMTF and age <2 years old (89\% vs. 69\%; P=.009), and for severe AOM in children <2 years old (90\% vs. 75\%; P=.012). Among children with AOMTF previously treated with amoxicillin-clavulanate or ceftriaxone injections, gatifloxacin cure rates were high (88\% and 75\%, respectively). Gatifloxacin appears to be safe for children, with no evidence of producing arthrotoxicity in 867 children exposed to the antibiotic when used as treatment for ROM and AOMTF. [\hyperlink{Famotidine}{PMID: 16028153}, Michael E Pichichero et al., 2005]

\hypertarget{pmid_25449223}{T}o systematically review literature assessing efficacy and safety of pharmacologic treatments in children with abdominal pain-related functional gastrointestinal disorders (AP-FGIDs). MEDLINE and Cochrane Database were searched for systematic reviews and randomized controlled trials investigating efficacy and safety of pharmacologic agents in children aged 4-18 years with AP-FGIDs. Quality of evidence was assessed using Grades of Recommendation, Assessment, Development and Evaluation approach. We included 6 studies with 275 children (aged 4.5-18 years) evaluating antispasmodic, antidepressant, antireflux, antihistaminic, and laxative agents. Overall quality of evidence was very low. Compared with placebo, some evidence was found for peppermint oil in improving symptoms (OR 3.3 (95\% CI 0.9-12.0) and for cyproheptadine in reducing pain frequency (relative risk [RR] 2.43, 95\% CI 1.17-5.04) and pain intensity (RR 3.03, 95\% CI 1.29-7.11). Compared with placebo, amitriptyline showed 15\% improvement in overall quality of life score (P = .007) and famotidine only provides benefit in global symptom improvement (OR 11.0; 95\% CI 1.6-75.5; P = .02). Polyethylene glycol with tegaserod significantly decreased pain intensity compared with polyethylene glycol only (RR 3.60, 95\% CI 1.54-8.40). No serious adverse effects were reported. No studies were found concerning antidiarrheal agents, antibiotics, pain medication, anti-emetics, or antimigraine agents. Because of the lack of high-quality, placebo-controlled trials of pharmacologic treatment for pediatric AP-FGIDs, there is no evidence to support routine use of any pharmacologic therapy. Peppermint oil, cyproheptadine, and famotidine might be potential interventions, but well-designed randomized controlled trials are needed. [\hyperlink{Famotidine}{PMID: 25449223}, Judith J Korterink et al., 2015]

\hypertarget{pmid_8055765}{F}amotidine has been used for the treatment of peptic ulcers and Zollinger Ellison syndrome and is also useful in reflux and erosive oesophagitis. To evaluate the effects of Famotidine 20 mg given twice daily in the symptomatic relief of gastro-oesophageal reflux disease with normal oesophagus or mild endoscopic oesophagitis, patients were followed over a period of six weeks. 70\% of the patients had complete day-time heartburn relief during the study and 75\% had complete night-time heartburn relief during the study. Famotidine was found to be safe and there were no serious clinical or laboratory adverse experiences. [\hyperlink{Famotidine}{PMID: 8055765}, F A Okoth et al., 1994]

\hypertarget{pmid_7846741}{W}e treated 14 boys, six with gastric ulcers and eight with duodenal ulcers, to determine famotidine pharmacokinetics and its inhibition of gastric acid secretion (pharmacodynamics). Famotidine (1 mg/kg/day) was administered either intravenously or orally at a dose of 0.5 mg/kg twice a day (maximum: 40 mg/day). Blood samples were collected from all subjects and the intragastric pH monitored in eight. Pharmacokinetic parameters were calculated assuming a one-compartment model. Volume of distribution, elimination half-life, and area under the serum concentration-time curve were 1.52 +/- 0.37 l/kg, 2.29 +/- 0.38 h, and 1.14 +/- 0.32 ng.h/ml, respectively. The mean oral bioavailability of famotidine was 50.6\%. Both intravenously and orally administered famotidine neutralized gastric acidity during sleep but failed to continuously maintain the intragastric pH > 5.0. All subjects' ulcers healed within 8 weeks. There were no side effects noted during famotidine treatment. Twice daily administration of 0.5 mg/kg famotidine for 8 weeks appears to be a tolerated and effective treatment of children with gastroduodenal ulcers. [\hyperlink{Famotidine}{PMID: 7846741}, A Nagita et al., 1994]

\hypertarget{pmid_1887750}{A}spiration pneumonitis is a severe complication of anesthesia. The objectives of this study were to determine if preoperative famotidine, a new histamine2-receptor antagonist, given by mouth either the evening before or the morning of elective surgery, reduced gastric residual volume and increased gastric pH in pediatric patients. Either famotidine or placebo (or both) were orally administered to 58 children (aged 2-17 years). The patients were randomly assigned to four groups: Famotidine-Famotidine, Placebo-Placebo, Placebo-Famotidine, and Famotidine-Placebo; subjects in the Famotidine-Famotidine group received two doses of famotidine (0.5 mg.kg-1 per dose), those in the Placebo-Placebo group, two doses of placebo, those in the Placebo-Famotidine and Famotidine-Placebo group, one dose of each by mouth. The Famotidine-Famotidine group received one dose of famotidine at 22:00 the evening before surgery and a second dose 60-90 min before the scheduled time of surgery. The Placebo-Placebo group received two doses of placebo at the same times as the Famotidine-Famotidine group. The Placebo-Famotidine group received a dose of placebo the night before surgery and a dose of famotidine the morning of surgery; the Famotidine-Placebo group received famotidine the night before surgery and placebo the morning of surgery. The administration of famotidine on the morning of surgery significantly increased gastric pH (4.8 vs. 1.3) in comparison with placebo, as did two doses of famotidine (6.6). Famotidine failed to reduce gastric residual volume significantly in any group. The administration of famotidine significantly reduced the number of pediatric patients considered at higher risk for aspiration pneumonitis, despite not decreasing gastric residual volume. [\hyperlink{Famotidine}{PMID: 1887750}, J S Jahr et al., 1991]

\hypertarget{pmid_18947477}{T}o investigate the efficacy and safety of famotidine treatment for stress ulcers in neonates. Fifty-four neonates with stress ulcers from 2001 to 2006 were enrolled. Seven cases were confirmed with stress ulcers by gastroscopy. Famotidine was administered intravenously at a dosage of 0.5 mg/kg every other 12 hrs. After cessation of hematemesis and vomiting, famotidine was administered once a day for two days. Primary diseases and complications were concurrently treated. Clinical symptoms and gastric pH were assessed before and after famotidine treatment. Possible adverse effects of famotidine treatmentdouble ended arrowrelated were observed. After 24 hrs of famotidine treatment, hematemesis and vomiting ceased in 52 patients (96.3\%). Clinical symptoms disappeared in all of the 54 patients 48 hrs after famotidine treatment. Gastric pH value increased 6, 12, 24, 36 and 48 hrs after famotidine treatment from 2.07+/-0.22 (before treatment) to 5.01-5.15 (P<0.01). All of the 54 patients were successfully treated. Famotidine treatment did not lead to abnormal respiration, heart rate and blood pressure. Loss of appetite, nausea, vomiting, diarrhea, constipation and rashes were not seen after famotidine treatment. There were significant differences in white cell count, platelet count and hepatic enzyme levels before and after famotidine treatment. An augmented side effect of the other drugs concurrently used due to famotidine treatment was not noted. Famotidide is effective and safe for the treatment of stress ulcers in neonates. [\hyperlink{Famotidine}{PMID: 18947477}, Yan-Yong Wu et al., 2008]

\hypertarget{pmid_30360666}{C}hronic idiopathic nausea (CIN) and functional dyspepsia (FD) cause considerable strain on many children's lives and their families. Areas covered: This study aims to systematically assess the evidence on efficacy and safety of pharmacological treatments for CIN or FD in children. CENTRAL, EMBASE, and Medline were searched for Randomized Controlled Trials (RCTs) investigating pharmacological treatments of CIN and FD in children (4-18 years). Cochrane risk of bias tool was used to assess methodological quality of the included articles. Expert commentary: Three RCTs (256 children with FD, 2-16 years) were included. No studies were found for CIN. All studies showed considerable risk of bias, therefore results should be interpreted with caution. Compared with baseline, successful relief of dyspeptic symptoms was found for omeprazole (53.8\%), famotidine (44.4\%), ranitidine (43.2\%) and cimetidine (21.6\%) (p = 0.024). Compared with placebo, famotidine showed benefit in global symptom improvement (OR 11.0; 95\% CI 1.6-75.5; p = 0.02). Compared with baseline, mosapride versus pantoprazole reduced global symptoms (p = 0.011; p = 0.009). One study reported no occurrence of adverse events. This systematic review found no evidence to support the use of pharmacological drugs to treat CIN or FD in children. More high-quality clinical trials are needed. AP-FGID: Abdominal Pain Related Functional Gastrointestinal Disorders; BART: Biofeedback-Assisted Relaxation Training; CIN: Chronic Idiopathic Nausea; COS: Core Outcomes Sets; EPS: Epigastric Pain Syndrome; ESPGHAN: European Society for Pediatric Gastroenterology Hepatology and Nutrition; FAP: Functional Abdominal Pain; FD: Functional Dyspepsia; GERD: Gastroesophageal Reflux Disease; GES: Gastric Electrical Stimulation; H [\hyperlink{Famotidine}{PMID: 30360666}, Pamela D Browne et al., 2018] Using single subject research design, we performed pilot research to evaluate the safety and efficacy of famotidine for the treatment of children with autistic spectrum disorders. We studied 9 Caucasian boys, 3.8-8.1 years old, with a DSM-IV diagnosis of a pervasive developmental disorder, living with their families, receiving no chronic medications, and without significant gastrointestinal symptoms. The dose of oral famotidine was 2 mg/kg/day (given in two divided doses); the maximum total daily dose was 100 mg. Using single-subject research analysis and medication given in a randomized, double-blind, placebo-controlled, cross-over design, 4 of 9 children randomized (44\%) had evidence of behavioral improvement. Primary efficacy was based on data kept by primary caregivers, including a daily diary; daily visual analogue scales of affection, reciting, or aspects of social interaction; Aberrant Behavior Checklists (ABC, Aman); and Clinical Global Improvement scales. Children with marked stereotypy (meaningless, repetitive behaviors) did not respond. Our subjects did not have prominent gastrointestinal symptoms and endoscopy was not part of our protocol; thus, we cannot exclude the possibility that our subjects improved due to the effective treatment of asymptomatic esophagitis. The use of famotidine for the treatment of children with autistic spectrum disorders warrants further investigation. [\hyperlink{Famotidine}{PMID: 30360666}, L A Linday et al., 2001]

\hypertarget{pmid_31050796}{F}requently, infants and children require sedation to facilitate noninvasive procedures and imaging studies. Propofol and dexmedetomidine are used to achieve deep procedural sedation in children. The objective of this study was to compare the clinical safety and efficacy of propofol versus dexmedetomidine in pediatric patients undergoing sedation in a pediatric sedation unit. A retrospective analysis of patients sedated with either propofol or dexmedetomidine in a pediatric sedation unit by pediatric emergency physicians was performed. Both medications were dosed per protocol with propofol 2 mg/kg induction and 150 μg · kg A total of 2432 children were included- 1503 who received propofol and 929 who received dexmedetomidine. Propofol and dexmedetomidine resulted in successful completion of the study in 98.8\% and 99.7\%, respectively ( Propofol use led to significantly shorter recovery times, with an increased need for airway management, but rates of bag-mask ventilation (2.3\%), airway obstruction (1.1\%), and desaturation (1.6\%) were low. No patients required intubation. Propofol is a reasonable alternative to dexmedetomidine, with a clinically acceptable safety profile. [\hyperlink{Famotidine}{PMID: 31050796}, Nicole M Schacherer et al., 2019]

\hypertarget{pmid_8649608}{P}henytoin is widely used for the prevention and treatment of acute seizures in children. Although it has the advantage of being available in parenteral form, it cannot be given through the i.m. route. Furthermore, problems with venous accessibility and maintenance may complicate i.v. administration of phenytoin in newborns and very sick infants. Fosphenytoin, a new phenytoin prodrug, can be safely administered through the i.m. route, and, because of the physical characteristics of its formulation, it offers advantages over phenytoin for i.v. administration. Clinical studies with i.v. and i.m. fosphenytoin demonstrate that the efficacy, safety, and pharmacokinetics of this drug are similar in 5- to 18-year-old children and in young adults. The safety and pharmacokinetic profile of i.v. and i.m. fosphenytoin in younger children and infants is currently being investigated. [\hyperlink{Famotidine}{PMID: 8649608}, J M Pellock et al., 1996]

\hypertarget{pmid_22477789}{D}exmedetomidine was approved by the Food and Drug Administration in 1999 for the sedation of adults receiving mechanical ventilation in an intensive care setting. It provides sedation with minimal effects on respiratory function and may be used prior to, during, and following extubation. Based on its efficacy in adults, dexmedetomidine is now being explored as an alternative or adjunct to benzodiazepines and opioids in the pediatric intensive care setting. This review describes the studies evaluating the safety and efficacy dexmedetomidine in infants and children and provides recommendations on dosing and monitoring. The MEDLINE (1950-November 2009) database was searched for pertinent abstracts, using the key term dexmedetomidine. Additional references were obtained from the bibliographies of the articles reviewed and the manufacturer. All available English-language case reports, clinical trials, retrospective studies, and review articles were evaluated. Over two dozen case series and clinical studies have documented the utility of dexmedetomidine as a sedative in children requiring mechanical ventilation or procedural sedation. In several papers, dexmedetomidine use resulted in a reduction in the dose or discontinuation of other sedative agents. It may be of particular benefit in children with neurologic impairment or in those who do not tolerate benzodiazepines. The most frequent adverse effects reported with dexmedetomidine have been hypotension and bradycardia, in 10\% to 20\% of patients. These effects typically resolve with dose reduction. Dexmedetomidine offers an additional choice for the sedation of children receiving mechanical ventilation in the intensive care setting or requiring procedural sedation. While dexmedetomidine is well tolerated when used at recommended doses, it has the potential to cause hypotension and bradycardia and requires close monitoring. In addition to clinical trials currently underway, larger controlled studies are needed to further define the role of dexmedetomidine in pediatric intensive care. [\hyperlink{Famotidine}{PMID: 22477789}, Marcia L Buck et al., 2010]

\hypertarget{pmid_2866138}{E}xtensive preclinical safety studies with famotidine were performed or sponsored by Yamanouchi Phamaceutical Co, Ltd, Tokyo, Japan, and Merck, Sharp \& Dohme Research Laboratories, West Point, Pennsylvania, USA. These studies were performed in dogs, rats, mice and rabbits, receiving oral and intravenous administration of the compound. Minimal toxicologic effects (after acute, subacute, or chronic administration) have been observed even at extremely high dosage levels (4,000 mg/kg/day) and for extended periods of administration (2,000 mg/kg/day for 105 weeks). No evidence of teratogenic, mutagenic, or carcinogenic effects or alterations of reproductive function have been seen. Based on these data, there are no contraindications for administration of this compound to humans. [\hyperlink{Famotidine}{PMID: 2866138}, J D Burek et al., 1985]

\hypertarget{pmid_2877577}{F}amotidine is a potent histamine (H2)-receptor antagonist that binds to the H2 receptor in a competitive reversible manner as shown by in vivo, in vitro, and clinical studies. Famotidine has shown no evidence of carcinogenicity, mutagenicity, or teratogenicity in extensive and adequately designed safety assessment studies. The drug produces neither prolonged anacidity nor doses its use result in significant elevations of serum gastrin levels beyond those seen with other available H2-receptor antagonists when used as recommended for the treatment of ulcer disease. Taken together, these data demonstrate no undue or disproportionate risk to the use of famotidine. [\hyperlink{Famotidine}{PMID: 2877577}, R G Berlin et al., 1986]

\hypertarget{pmid_8851452}{F}amotidine is a specific, long-acting histamine2-receptor antagonist. It is indicated for the treatment of duodenal ulcer, gastric ulcer, gastroesophageal reflux disease, and Zollinger-Ellison syndrome. Since its introduction for the treatment of acid-related disorders in 1985, an estimated 18.8 million patients worldwide have been treated with famotidine. We present a comprehensive safety profile of oral famotidine, incorporating data from investigational trials, postmarketing studies, and reports of marketed use. The excellent tolerability profile of famotidine observed during investigational trials has remained substantially unchanged during postmarketing experience. Famotidine does not notably bind to cytochrome P-450 or gastric alcohol dehydrogenase and therefore has not been associated with clinically significant drug interactions. It is generally well tolerated in patients with cardiovascular, renal, or hepatic dysfunction or with Zollinger-Ellison syndrome who have tolerated doses up to 800 mg daily. [\hyperlink{Famotidine}{PMID: 8851452}, C W Howden et al., ]

\hypertarget{pmid_19681413}{T}his randomised controlled study evaluated the effects of fentanyl and dexmedetomidine on emergence characteristics of children having adenoidectomy and anaesthetised with sevoflurane. Ninety children, two to seven years of age and ASA physical status I, were studied. Children were randomly assigned to one of three groups of 30 children, with the study intervention injection given intravenously after intubation. Children in Group F received fentanyl 2.5 microg x kg(-1), children in Group D received dexmedetomidine 0.5 microG x kg(-1) and children in Group C received saline solution. Anaesthesia was induced with 50\% N2O and 8\% sevoflurane in O2 by mask and atracurium 0.6 mg x kg(-1) was administered for tracheal intubation. All children received paracetamol 40 mg/kg rectally one hour preoperatively and dexamethasone 0.5 mg x kg(-1) intravenously. The time to extubation was shorter in Group D than Group F. The eye-opening time was longer in Group F (16.1 +/- 5.3 minutes) than in Groups C (12.0 +/- 4.2 minutes) and D (12.7 +/- 3.2 minutes). The proportion of pain-free children in early recovery was significantly higher in Groups D (47\%) and F (43\%) than Group C (13\%) (P < 0.05). The proportion of children with agitation scores > 3 was lower in Groups D 17\% (5/30) and F 13\% (4/30) than in Group C 47\% (14/30) (P < 0.05). Fentanyl 2.5 microg x kg(-1) and dexmedetomidine 0.5 microg x kg(-1) had similar haemodynamic effects and emergence characteristics. Fentanyl has been safely used in children for many years. Further studies of dexmedetomidine safety and its interaction with other anaesthetic agents are required before recommending its routine use during general anaesthesia in children. [\hyperlink{Famotidine}{PMID: 19681413}, F Erdil et al., 2009]

\hypertarget{pmid_34853785}{A}sthma is the most common chronic disease in children, many of whom are managed solely with a short-acting β The aim of this study is to determine the efficacy and safety of as-needed budesonide-formoterol therapy compared with as-needed salbutamol in children aged 5 to 15 years with mild asthma, who only use a SABA. A 52-week, open-label, parallel group, phase III RCT will recruit 380 children aged 5 to 15 years with mild asthma. Participants will be randomised 1:1 to either budesonide-formoterol (Symbicort Rapihaler This is the first RCT to assess the safety and efficacy of as-needed budesonide-formoterol in children with mild asthma. The results will provide a much-needed evidence base for the treatment of mild asthma in children. [\hyperlink{Famotidine}{PMID: 34853785}, Lee Hatter et al., 2021]

\hypertarget{pmid_12752346}{G}astro-oesophageal reflux afflicts up to 7\% of all infants. Histamine-2 receptor antagonists are the most commonly prescribed medications for this disorder, but few controlled studies support this practice. To evaluate the safety and efficacy of famotidine for infant gastro-oesophageal reflux disease. Thirty-five infants, 1.3-10.5 months of age, entered an 8-week, multi-centre, randomized, placebo-controlled, two-phase trial: first 4 weeks, observer-blind comparison of famotidine 0.5 mg/kg and famotidine 1.0 mg/kg; second 4 weeks, double-blind withdrawal comparison (safety and efficacy) of each dose with placebo. No serious adverse events were reported. Eleven patients had 16 non-serious, possibly drug-related adverse experiences: 6 patients with agitation or irritability (manifested as head-rubbing in two), 3 patients with somnolence, 2 patients with anorexia, 2 with headache, 1 patient with vomiting, 1 patient with hiccups, and 1 patient with candidiasis. Of the 35 infants, 27 completed Part I. There were significant score improvements for famotidine 0.5 mg/kg in regurgitation frequency (P = 0.04), and for famotidine 1.0 mg/kg in crying time (P = 0.027) and regurgitation frequency (P = 0.004) and volume (P = 0.01). Eight infants completed Part II on double-blind treatment, which was insufficient for meaningful comparisons. Histamine-2 receptor antagonists may cause agitation and headache in infants. A possibly efficacious famotidine dose for infants is 0.5 mg/kg (frequency adjusted for age). As 1.0 mg/kg may be more efficacious in some, the dosage may require individualization based on response. Further sizeable placebo-controlled evaluations of histamine-2 receptor antagonists in infants with gastro-oesophageal reflux disease are warranted. [\hyperlink{Famotidine}{PMID: 12752346}, S R Orenstein et al., 2003]

\hypertarget{pmid_7739102}{W}e evaluated the efficacy of midazolam-famotidine suppository (M-F suppository) for premedication in children. After obtaining informed parental consent, we studied children aged 5m-7y, ASA I status, scheduled for minor elective surgery. The suppository group (n = 26) was given suppository of both midazolam 0.5 mg.kg-1 and famotidine 2 mg.kg-1, and the intramuscular injection group (n = 19) was given hydroxyzine 1 mg.kg-1. In the suppository group, gastric pH (3.90 +/- 0.34) was significantly higher, and gastric volume (1.88 +/- 0.54 ml) was significantly less than in the intramuscular injection group (1.82 +/- 0.15, 8.42 +/- 1.76 ml). The M-F suppository may offer similar sedative effect as an intramuscular injection of hydroxyzine. We concluded that the M-F suppository is an effective premedicant for pediatric patients. [\hyperlink{Famotidine}{PMID: 7739102}, A Iwamoto et al., 1995]

\hypertarget{pmid_32022483}{A}sthma affects over 6 million children in the United States alone. This study investigated the efficacy and long-term safety of mometasone furoate-formoterol (MF/F) and MF monotherapy in children with asthma. This phase 3, multicenter, randomized controlled trial evaluated metered-dose inhaler twice daily (BID) dosing with MF/F 100/10 µg or MF 100 µg in children, aged 5 to 11 years, with a history of asthma for greater than or equal to 6 months and confirmed bronchodilator reversibility, who were adequately controlled on inhaled corticosteroid/long-acting beta-agonist combination therapy for greater than or equal to 4 weeks. After a 2-week run-in on MF 100 µg BID, eligible patients received 24 weeks of double-blind treatment and were followed for safety up to 26 weeks. The primary efficacy endpoint was the change from baseline in AM postdose 60-minute AUC \%predicted FEV1\% across 12 weeks of treatment. A total of 181 participants received at least one dose of MF/F (n = 91) or MF (n = 90). MF/F was superior to MF across the 12-week evaluation period, with a treatment advantage of 5.21 percentage points (P < .001). Superior onset of action with MF/F over MF was achieved as early as 5 minutes postdose on day 1. Overall, approximately 50\% of participants experienced one or more treatment-emergent adverse events, with fewer occurring in the MF/F group. In children 5 to 11 years of age with persistent asthma, the addition of F to MF was well tolerated and provided significant, rapid, and sustained improvement in lung function compared with MF alone. [\hyperlink{Famotidine}{PMID: 32022483}, Cindy L J Weinstein et al., 2020]

\hypertarget{pmid_15690910}{T}he efficacy of the fluoroquinolone levofloxacin in the treatment of 35 children with bronchopulmonary disease exacerbation was practically the same as that of amoxycillin/clavulanate, cefotaxime or ceftriaxone. The clinical and bacteriological results were favourable. The eradication of the pathogens responsible for the bronchopulmonary inflammations in 86\% of the patients was stated. There is no doubt that fluoroquinolones should not be widely used in pediatrics. They should be considered as reserve drugs for the treatment of severe cases when the routine agents fail. Their use is justified when the situation is risky and the data on the pathogen susceptibility to the drugs are available. Still, levofloxacin is the most safe fluoroquinolone with low hepatotoxicity and lower effect on the central nervous system. The episodes of its negative cardiovascular action are less frequent. Moreover, the most frequent side effects of fluoroquinolones such as nausea, diarrhea or vomiting are less frequent with the use of levofloxacin. No signs of arthropathy in the patients treated with levofloxacin were observed in our trial. [\hyperlink{Famotidine}{PMID: 15690910}, I K Volkov et al., 2004]

\hypertarget{pmid_7961355}{T}he objective of this open study was to determine the efficacy and safety of fluoxetine for the treatment of children and adolescents with anxiety disorders. Twenty-one patients with overanxious disorders, social phobia, or separation anxiety disorder, who were unresponsive to previous psychopharmacological and psychotherapeutic interventions, were treated openly with fluoxetine for up to 10 months. Patients with lifetime histories of obsessive-compulsive disorder (OCD) or panic disorder, or with current major depression, were excluded. Beneficial and adverse effects of fluoxetine were ascertained using the improvement and severity subscales of the Clinical Global Impression Scale (CGIS) in two ways: (1) independent chart reviews by two child psychiatrists and (2) prospective assessments by the treating nurses and the patients' mothers. Eighty-one percent (n = 17) of patients showed moderate to marked improvement in anxiety symptoms. The severity of anxiety as measured by the CGIS was also significantly reduced from marked to mild (effect size: 2.3). There were no significant side effects. These results suggest that fluoxetine may be an effective and safe treatment for nondepressed children and adolescents with anxiety disorders other than OCD and panic disorder. Future investigations using double-blind, placebo-controlled methodologies are warranted. [\hyperlink{Famotidine}{PMID: 7961355}, B Birmaher et al., 1994]

\hypertarget{pmid_14567252}{T}he clinical efficacy and safety of clarithromycin (CAM) and cefdinir (CFDN) were evaluated in 65 pediatric outpatients with group A beta-hemolytic streptococcal tonsillopharyngitis. Treatment was "effective" or better in 26 (78.8\%) children receiving CAM and in 27 (87.1\%) receiving CFDN based on antigen clearance and the "Criteria for Evaluation in Clinical Trials of Antibacterial Agents in Children" proposed by Japan Society of Chemotherapy (p = NS). The causative organisms were eradicated in 94.7\% and 93.8\% of subjects in the CAM and CFDN groups, respectively (p = NS). Adverse drug reactions were limited to moderate diarrhea in one patient in each group, and subsided during treatment. Causative organisms exhibited good susceptibility to CAM and CFDN. These results suggest excellent efficacy, safety and usefulness of CAM and CFDN in the treatment of group A beta-hemolytic streptococcal tonsillopharyngitsis in children. [\hyperlink{Famotidine}{PMID: 14567252}, Tadafumi Nishimura et al., 2003]

\hypertarget{pmid_2889529}{V}omiting, hematemesis, and esophagitis resulting from gastroesophageal reflux or hiatal hernia are frequently observed in severely handicapped children. This study was conducted to determine whether the use of a new H2-antagonist, famotidine, could prevent recurrence of reflux esophagitis among such children. Seventeen severely handicapped, bedridden children admitted to a children's medical center between April 1985 and September 1986 were studied. All had vomiting or hematemesis as a main symptom, and the cause of esophagitis was suggested to be gastroesophageal reflux in 13 cases and hiatal hernia in four. Six had been previously treated with cimetidine or other drugs or a combination thereof without relief. Famotidine was administered at about 1 to 2 mg/kg/day, two times daily to patients weighing more than 10 kg and three times daily to those weighing less than 10 kg. In 13 cases, famotidine was administered intravenously for between seven and ten days and then given orally, while the rest were given the drug orally from the outset. The following results were obtained: (1) improvement was seen within seven days after start of famotidine treatment, and reduction of vomiting or hematemesis or both was reached within two weeks in 70\% of cases and within three weeks in 94\%; (2) famotidine was markedly effective in 29\% and moderately effective in 41\%; in no case was the drug ineffective; (3) no side effects were observed; five patients had transient, mild elevation of SGOT . SGPT, but this was not attributable to the drug. [\hyperlink{Famotidine}{PMID: 2889529}, S Miyake et al., 1987]

\hypertarget{pmid_31977308}{W}e sought to assess and compare safety and efficacy of fesoterodine and oxybutynin extended-release in the treatment of pediatric overactive bladder (OAB). We conducted a non-inferiority, randomized, double-blind, crossover trial comparing fesoterodine 4-8 mg and oxybutynin 10-20 mg once daily (QD) in children with OAB aged 5-14 years (2015-2018). Every child received the first medication for eight weeks, followed by crossover to the second antimuscarinic after a three-days washout. Dose up-titration was possible at mid-course. Patients could enter a fesoterodine 12-month extension. Endpoints were assessed through changes on voiding diaries, Patient's Perception of Bladder Condition score (PPBC), adverse events, vital signs, electrocardiogram, post-void residual, urinalysis, and blood tests. The Wilcoxon rank sum and Wilcoxon signed rank tests were used for statistical analysis. A total of 62 patients were randomized (two early dropouts). Expected class effects (dry mouth/constipation) were present but no significant difference was observed. There was a 10.1 beats/minute increase in heart rate with fesoterodine (p<0.01) (oxybutynin-1.9 beats/min; p=non-significant [ns]). No life-threatening or serious adverse events occurred. Efficacy was similar for both drugs. Bladder capacity improved over the 16 months of the study; baseline capacity of 125 mL (44.5\% expected bladder capacity for age [\%EBC]) to 171 mL (53.9 \%EBC) at the end of the extension phase. No clinical or statistical difference was shown between efficacy measures for fesoterodine or oxybutynin. The use of fesoterodine or oxybutynin appear safe and effective for the treatment of OAB in children. Based on our study, long-term treatment to achieve the ultimate goal of urinary continence is needed in this population. [\hyperlink{Famotidine}{PMID: 31977308}, Sophie Ramsay et al., 2020]

\section*{Vfend}
\subsection*{Result}
\subsubsection*{Answer}

Unknown

\subsubsection*{{Explanation}}
\hypertarget{Vfend}
To determine if Vfend (voriconazole) is safe for use in children, I reviewed the available abstracts for targeted pediatric safety studies.

One relevant abstract is [\hyperlink{pmid_22714745}{PMID: 22714745}, Masaaki Mori et al., 2013], which reports on voriconazole pharmacokinetics in 24 Japanese pediatric and adolescent patients. The study aimed to provide information about voriconazole pharmacokinetics and to explore relationships between drug levels, administered dose, and laboratory test results. The results showed that median plasma voriconazole concentrations increased with dose, but "no systematic trend was observed between [trough plasma concentration] and laboratory test results." However, the abstract does not specifically mention any targeted safety outcomes, adverse events, or affirm safety in children. It is primarily a pharmacokinetic study, not a safety study.

No other abstracts specifically address the safety of voriconazole (Vfend) in children, nor do any provide evidence from a targeted safety study affirming or refuting its safety in pediatric populations.

Therefore, based on the abstracts available, there is no definitive evidence from a targeted safety study to affirm that Vfend is safe or unsafe for use in children of any age range. The safety of Vfend in children is therefore unknown based on the available data.

\subsection*{Abstracts}
\hypertarget{pmid_19215279}{T}he use of vigabatrin (VGB) as an antiepileptic drug (AED) has been limited by evidence showing that it causes vigabatrin-attributed visual field loss (VAVFL) in at least 20-40\% of patients exposed at school age or later. VGB is an effective drug for infantile spasms, but there are no reports on later visual field testing after such treatment. Our aim was to investigate the risk of VAVFL in school-age children who had received VGB in infancy. Visual fields of 16 children treated with VGB for infantile spasms were examined by Goldmann kinetic perimetry at age 6-12 years. Normal fields were defined as the temporal meridian extending to more than 70 degrees , and mild VAVFL between 50 and 70 degrees . Abnormal findings were always confirmed by repeating the test. Exposure data were collected from hospital charts. Vigabatrin was started at a mean age of 7.6 (range, 3.2-20.3) months. The mean duration of therapy was 21.0 (9.3-29.8) months and cumulative dose 655 g (209-1,109 g). Eight children were never treated with other AEDs, five received only adrenocorticotropic hormone (ACTH) in addition to VGB, and three children had been treated with other AEDs. Fifteen children had normal visual fields. Mild VAVFL was observed in one child (6\%) who had been treated with VGB for 19 months and who received a cumulative dose of 572 g. The risk of VAVFL may be lower in children who are treated with VGB in infancy compared to patients who receive VGB at a later age. [\hyperlink{Vfend}{PMID: 19215279}, Eija Gaily et al., 2009]

\hypertarget{pmid_32549104}{T}o assess the effectiveness and safety of fast-track cardiac anesthesia using the short-acting opioid sufentanil in children undergoing intraoperative device closure of ventricular septal defect (VSD). This retrospective clinical study included 65 children who underwent intraoperative device closure of VSD between January 2017 and June 2017. Patients were diagnosed with isolated perimembranous VSD by transthoracic echocardiography. Then, they were divided into two groups, group F (n=30), whose patients were given sufentanil-based fast-track cardiac anesthesia, and group C (n=35), whose patients were given conventional cardiac anesthesia. Perioperative clinical data were analyzed. No significant differences were found between the preoperative clinical parameters and intraoperative hemodynamic indices between the two groups. In group C, compared with group F, the postoperative duration of mechanical ventilation, the length of stay in the intensive care unit, the length of hospital stay, and the hospital costs were significantly increased. In this retrospective study at a single center, sufentanil-based fast-track cardiac anesthesia was shown to be a safe and effective technique for minimally-invasive intraoperative device closure of VSD in children, which was performed with reduced in-hospital costs. [\hyperlink{Vfend}{PMID: 32549104}, Zeng-Chun Wang et al., 2020]

\hypertarget{pmid_15993727}{T}his post-market, observational study is intended to evaluate reported uses of pediatric pads that reduce the energy delivered by some adult automated external defibrillators (AEDs) so that they may be used with pediatric patients. Users of the pediatric pads were asked to report any use of the pads, even if no shock was delivered and to provide detailed information about the event, caregiver and the patient. Reports of the use of pediatric pads have been received and confirmed for 27 patients, age range 0 days to 23 years, median 2 years. Ventricular fibrillation (VF) was reported in eight cases, age range 4.5 months to 10 years, median 3 years. Shocks were delivered to all VF patients, the average shock number was 1.9, range 1-4. All patients had termination of VF, were admitted to the hospital and five survived to hospital discharge. Non-shockable rhythms were reported in 16 patients, and the AED appropriately did not advise a shock. Eleven of these patients had asystole or PEA as their initial rhythm and did not survive to hospital discharge. One report contained no additional information other than that the patient did not survive, and in two other reports, the pads were not applied to patients. Voluntary reports of the use of attenuated pediatric defibrillation pads indicate the devices performed appropriately. All eight VF patients had termination of VF and five survived to hospital discharge. These data support the rapid deployment of AEDs for young children as well as adolescents and adults. Since the pediatric pads are available and deliver an appropriate dose for children, their use should be strongly encouraged. [\hyperlink{Vfend}{PMID: 15993727}, Dianne L Atkins et al., 2005]

\hypertarget{pmid_17941284}{T}he safety of fexofenadine has been examined extensively in adults and school-age children. However, the safety of fexofenadine in children younger than 6 years has not been reported to date. To compare the safety and tolerability of twice-daily fexofenadine hydrochloride, 30 mg, and placebo in preschool children aged 2 to 5 years with allergic rhinitis. This was a multicenter, double-blind, randomized, placebo-controlled, parallel-group study, conducted between February 29, 2000, and June 14, 2001. Participants were randomized to either fexofenadine hydrochloride, 30 mg, or placebo twice daily for a 2-week period. To facilitate dosing, capsule content was mixed with applesauce (approximately 10 mL). Safety assessments depended on date of entry into the study because of an amendment to the protocol. Before the amendment, assessments included physical examination, vital signs reporting (oral temperature, heart rate, and respiratory rate), and adverse event (AE) reporting. After the amendment, safety assessments included laboratory testing (blood chemistry and hematology profiles), physical examination, 12-lead electrocardiography, and vital signs (oral temperature, blood pressure, heart rate, and respiratory rate) and AE reporting. Treatment-emergent AEs were observed in 116 of 231 participants receiving placebo and 111 of 222 receiving fexofenadine. These AEs were possibly related to study medication in 19 (8.2\%) and 21 (9.5\%) of the participants receiving placebo and fexofenadine, respectively, and most frequently involved the digestive system. No clinically relevant differences in laboratory measures, vital signs, and physical examinations were observed. The findings show that fexofenadine hydrochloride, 30 mg, is well tolerated and has a good safety profile in children aged 2 to 5 years with allergic rhinitis. [\hyperlink{Vfend}{PMID: 17941284}, Henry Milgrom et al., 2007]

\hypertarget{pmid_32607039}{F}ormulation V (VONCENTO PK investigations were performed following one dose of Formulation V at Day 1 and 180. Nonsurgical bleeds were analyzed, while hemostatic efficacy was graded as excellent/good/moderate/none. Safety assessments included adverse events, and presence of VWF and/or FVIII inhibitors. Formulation V was administered as on-demand (N=13) or prophylaxis therapy (N=4) for 12 months (<6 years, N=9; 6 to <12 years, N=8). PK parameters for VWF markers were generally comparable to adults but showed lower VWF:ristocetin cofactor (RCo) exposure. Incidence of major bleeds was lower for prophylaxis (3.3\%) than on-demand therapy (27.1\%); joint bleeds were also lower (3.3\% vs 11.5\%, respectively). Investigator-reported excellent/good hemostatic efficacy against nonsurgical bleeds was 100\%. No clinically relevant differences in PK, hemostatic efficacy, or safety were observed between age-groups (<6 years and 6 to <12 years). Formulation V was well tolerated. Adverse events were mild-moderate and consistent with the adult safety profile. No cases of anaphylactic reactions or angioedema, development of FVIII/VWF inhibitors, thromboembolic events, or viral infections were reported. This study provides evidence for use of Formulation V to treat and prevent bleeding in pediatric patients with severe VWD, and led to the European approval of Formulation V in children. [\hyperlink{Vfend}{PMID: 32607039}, Guenter Auerswald et al., 2020]

\hypertarget{pmid_25145415}{T}he aim of this study was to examine whether vigabatrin treatment had caused visual field defects (VFDs) in children of school age who had received the drug in infancy. In total, 35 children (14 males, 21 females; median age 11y, SD 3.4y, range 8-23y) were examined by static Humphrey perimetry, Goldmann kinetic perimetry, or Octopus perimetry. The aetiologies of infantile spasms identified were tuberous sclerosis (n=10), other symptomatic causes (n=3), or cryptogenic (n=22). Typical vigabatrin-attributed VFDs were found in 11 out of 32 (34\%) children: in one out of 11 children (9\%) who received vigabatrin for <1 year (group 1), in three out of 10 children (30\%) who received vigabatrin for 12 to 24 months (group 2), and in seven out of 11 children (63\%) who received vigabatrin treatment for longer than 2 years (group 3). VFDs were mild in five and severe in six children. Patients with tuberous sclerosis were at higher risk of VFDs (six out of 10 children). The mean cumulative doses of vigabatrin were 140.5, 758.8, and 2712g in group 1, 2, and 3, respectively. VFDs were found in 34\% of the cohort of children in this study. The rate of VFD increased from 9\% to 63\% as duration of treatment increased. The results of this study showed that the risk-benefit ratio should always be considered when using vigabatrin. [\hyperlink{Vfend}{PMID: 25145415}, Raili Riikonen et al., 2015]

\hypertarget{pmid_32430761}{T}his study evaluated the efficacy and safety of transcatheter radiofrequency ablation (RFCA) in treating ventricular premature contractions (PVCs) in children, summarized the countermeasures during intraoperative ventricular fibrillation (VF), and improved the safety of ventricular premature treatment. A retrospective analysis was conducted on 75 children with PVCs who received RFCA in the Second Affiliated Hospital of Wenzhou Medical University from January 2010 to April 2019. Data including age, sex, body weight, ejection fraction, left ventricular end diastolic diameter, burden and number of PVCs/24 h, origin of PVCs, and its complications were collected. Paired t test was used to compare changes in cardiac function before and after surgery. Among the 75 cases treated with RFCA, 68 were successfully ablated, giving a success rate of 90.67\%. After ablation, the left ventricular ejection fraction (LVEF) of the children was 69.13 ± 3.81\%, which was significantly higher than that before surgery (69.13 ± 3.81\% vs. 66.21 ± 3.22\%, P = 0.012). One of the patients experienced VF during the operation, with no other complications. The initial locus of origin was the anterior septum of the right ventricular outflow tract, but VF occurred during the ablation process. Mean follow-up time was 39 ± 33 months, with two recurrent cases (2.94\%). Performing RFCA in children is safe and effective, with a low recurrence rate and few complications. VF is not an indication to cease surgery; the key to eliminating complications is repositioning the catheter and finding a more accurate origin point. [\hyperlink{Vfend}{PMID: 32430761}, Yue-E He et al., 2021]

\hypertarget{pmid_22714745}{V}oriconazole (VFEND(®)) is a triazole antifungal agent which inhibits the biosynthesis of ergosterol, a fungal cell membrane component. In Japan, voriconazole has become a commonly used antimicrobial in off-label use for pediatric patients. The aims of this report were to provide information about voriconazole pharmacokinetics (PK) in Japanese pediatric and adolescent patients, and to explore relationships between the PK, administered dose, and laboratory test results. In total, data from 24 pediatric or adolescent patients (18 males and 6 females) were used for the analysis. For the measurement of plasma voriconazole concentrations, 103 blood samples were collected from the 24 patients. As a whole, median plasma voriconazole concentrations following intravenous and oral administrations were comparable, and the trough plasma concentrations at steady state (C                 (12,ss)) increased with increasing voriconazole doses (mg/kg). However, no systematic trend was observed between C                 (12,ss) and laboratory test results. [\hyperlink{Vfend}{PMID: 22714745}, Masaaki Mori et al., 2013]

\hypertarget{pmid_22189521}{T}here has been reluctance to use fluoroquinolones in children because of arthropathy in animal models; experience in pediatric fever and neutropenia (FN) has been limited. Our primary objective was to describe the effectiveness and safety of fluoroquinolones as empiric therapy for children with FN. We conducted electronic searches of Ovid Medline, EMBASE, the Cochrane Central Register of Controlled Trials, and limited studies to prospective pediatric trials in which any type of fluoroquinolone was administered as empiric therapy for FN. Of the 7281 reviewed articles, 10 were included in the meta-analysis that encompassed 740 episodes of FN. All studies consisted of low-risk FN episodes. The risk of treatment failure was 17\% among those given ciprofloxacin monotherapy (n = 5 studies), 17\% among those given nonciprofloxacin fluoroquinolone monotherapy (n = 2 studies), and 24\% among those given fluoroquinolone combination therapy (n = 3 studies; P = 0.80). There were no cases of infectious deaths reported. Rates of sepsis and adverse events were very low. Experience with fluoroquinolones demonstrates excellent outcomes and short-term safety, although reported studies have been restricted to low-risk patients. Fluoroquinolones can be comfortably adopted for low-risk FN, although experience in high-risk FN is uncertain in pediatrics. [\hyperlink{Vfend}{PMID: 22189521}, Lillian Sung et al., 2012]

\hypertarget{pmid_21859365}{A}s part of the Voluntary Children's Chemical Evaluation Program (VCCEP) program, a risk assessment was performed to evaluate the risks to children from environmental benzene exposures. This paper summarizes this risk assessment. Risk was characterized using two distinct methods: USEPA's default type of risk assessment, which used the Reference Dose (RfD) and Cancer Slope Factor (CSF) to characterize non-cancer and cancer risks, as well as a Margin of Safety (MOS) approach that utilized a point of departure (POD). The exposures for most scenarios evaluated in this VCCEP risk assessment are lower than both the cancer and non-cancer PODs by several orders of magnitude, indicating a large MOS and corresponding low potential for toxicity at these exposures. The highest benzene exposures likely experienced by children, associated with the lowest MOS, are from cigarette smoke. In addition, the potential for age-related differences in the sensitivity towards benzene-induced toxicity was investigated. In general, this risk assessment does not indicate that children are likely to be at a elevated risk of AML or hematopoietic toxicity associated with environmental exposures to benzene. [\hyperlink{Vfend}{PMID: 21859365}, David W Pyatt et al., 2012]

\hypertarget{pmid_22941776}{W}e examined the influence of age and type of concomitant antiepileptic drugs (AEDs) on the pharmacokinetics of rufinamide (RUF) as well as its efficacy and safety in 51 children with refractory epilepsy. In a retrospective noninterventional survey, dose-to-concentration ratios of RUF and concomitant AEDs were calculated: the weight-normalized dose (mg/kg/d) divided by the steady-state trough plasma drug level, which was used as a measure of clearance. During treatment with RUF concomitantly with valproic acid (VPA) young children, aged 0 to 4.9 years, had a low clearance of RUF, which did not differ from older children. If not on VPA but on enzyme inducers, young children had a threefold higher clearance of RUF than the older ones. In young children not on VPA, those on enzyme inducers had 1.7-fold higher clearance than those on nonenzyme inducers. In children neither on VPA nor on enzyme inducers, RUF clearance was age-dependent with higher clearance in younger children. Adding RUF did not change the pharmacokinetics of concomitantly used AEDs. Seizure response after 2 to 3 months on RUF treatment was found in 12 of 51 children (23.5\%), at mean plasma level of 36.9 ± 22.0 µmol/L. Adverse events were reported in 41\% of the patients of which fatigue was most frequent (24\%). [\hyperlink{Vfend}{PMID: 22941776}, Maria G Dahlin et al., 2012]

\hypertarget{pmid_33118730}{T}he aim of this study was to evaluate whether sufentanil can reduce emergence delirium in children undergoing transthoracic device closure of ventricular septal defect (VSD) after sevoflurane-based cardiac anesthesia. From February 2019 to May 2019, 68 children who underwent transthoracic device closure of VSD at our center were retrospectively analyzed. All patients were divided into two groups: 36 patients in group S, who were given sufentanil and sevoflurane-based cardiac anesthesia, and 32 patients in group F, who were given fentanyl and sevoflurane-based cardiac anesthesia. The following clinical data were recorded: age, sex, body weight, operation time, and bispectral index (BIS). After the children were sent to the intensive care unit (ICU), pediatric anesthesia emergence delirium (PAED) and face, legs, activity, cry, consolability (FLACC) scale scores were also assessed. The incidence of adverse reactions, such as nausea, vomiting, drowsiness and dizziness, was recorded. There was no significant difference in age, sex, body weight, operation time or BIS value between the two groups. Extubation time (min), PEAD score and FLACC scale score in group S were significantly better than those in group F (P<0.05). No serious anesthesia or drug-related side effects occurred. Sufentanil can be safely used in sevoflurane-based fast-track cardiac anesthesia for transthoracic device closure of VSD in children. Compared to fentanyl, sufentanil is more effective in reducing postoperative emergence delirium, with lower analgesia scores and greater comfort. [\hyperlink{Vfend}{PMID: 33118730}, Ning Xu et al., 2020]

\hypertarget{pmid_33410181}{T}his study aimed to evaluate the analgesic and sedative effects of remifentanil-based fast-track cardiac anesthesia in children undergoing transthoracic device closure of ventricular septal defects (VSDs). A retrospective analysis was conducted on 62 children who underwent transthoracic device closure of VSDs from May 2019 to August 2019. The patients were divided into two groups based on the anesthesia methods: group F was given remifentanil-based fast-track cardiac anesthesia, and Group C was given conventional anesthesia. Patient-related clinical data, postoperative analgesia scores, and sedation scores were collected and analyzed. There was no significant difference in intraoperative hemodynamic changes, bispectral index values, postoperative analgesia scores, sedation scores, or the incidence of adverse events between the two groups. Compared with Group C, the duration of mechanical ventilation and the length of intensive care unit (ICU) and hospital stay in group F were significantly lower. Remifentanil-based fast-track anesthesia can be safely applied in children undergoing transthoracic device closure of VSDs, with acceptable postoperative analgesia and sedation effects and shorter mechanical ventilation times and ICU and hospital stays compared with conventional anesthesia. [\hyperlink{Vfend}{PMID: 33410181}, Ning Xu et al., 2021]

\hypertarget{pmid_16891821}{W}e studied the prevalence, type and severity of vigabatrin (VGB)-attributed visual field defects (VFDs), and used these data to assess the associated risk factors in pediatric patients. Medical records were retrospectively reviewed for 67 pediatric patients who received VGB alone or in combination with other antiepileptic drugs, and who had undergone visual field examinations using a Humphrey visual field analyzer. Of the 67 patients, 15 had VGB-attributed VFDs: 13 had nasal arcuate type, 1 had nasal and temporal constricted type and 1 had nasal constricted type. In terms of severity, 7 patients had Grade I VGB-attributed VFDs, 5 had Grade II, 2 had Grade III, and 1 had Grade IV. Although there were no significant differences between the VFD and non-VFD groups with regards to all tested parameters, there were no cases of VGB-attributed VFDs in patients with total treatment durations <2 yr and cumulative doses <10 g/kg. In conclusion, the prevalence of VGB-attributed VFDs in VGB-treated pediatric epilepsy patients was 22\%. The high frequency of VGB-attributed VFDs indicates that physicians should inform all patients of this risk prior to VGB treatment and perform periodic visual field examinations. [\hyperlink{Vfend}{PMID: 16891821}, Su Jeong You et al., 2006]

\hypertarget{pmid_37203829}{T}ranscatheter closure of medium and large ventricular septal defects (VSDs) in young children is limited due to the use of over-sized devices that can cause hemodynamic instability and arrhythmia. In this study, we aimed to retrospectively evaluate the safety and efficacy of the device in the mid-term in children weighing less than 10 kg whose transcatheter VSD was closed only with the Konar-MFO device. Among 70 children whose transcatheter VSD was closed between January 2018 and January 2023, 23 patients weighing less than 10 kg were included in the study. Retrospectively, the medical records of all patients were reviewed. The mean age of the patients was 7.3 (4.5-26) months. 17 of the patients were females, 6 of them were males, F/M: 2.83. The average weight was 6.1 (3.7-9.9) kg. The mean the pulmonary blood flow/ systemic blood flow (Qp/Qs) was 3.3 (1.7-5.5). The mean defect diameter was 7.8 mm (5.7-11) for the left ventricle (LV) side, and 5.7 mm (3-9.3) for the right ventricle (RV) side. Based on the utilized device dimensions, the measurements on the LV side were recorded as 8.6 mm (range 6-12), while those on the RV side were recorded as 6.6 mm (range 4-10). Antegrade technique was applied to 15 (65.2\%) patients and retrograde technique was applied to 8 (34.8\%) patients in the closure procedure. The procedure success rate was 100\%. The incidence of death, device embolization, hemolysis, or infective endocarditis was zero. Perimembranous and muscular VSDs in children under 10 kg can be successfully closed under the management of an experienced operator with the Lifetech Konar-MFO device. This is the first study in the literature to evaluate the efficacy and safety of the device in children under 10 kg in whom only Konar-MFO VSD occluder device is used for transcatheter VSD closure. [\hyperlink{Vfend}{PMID: 37203829}, K Yildiz et al., 2023]

\hypertarget{pmid_14658963}{V}elocardiofacial syndrome (VCFS) is a common microdeletion syndrome associated with psychiatric morbidity and developmental disabilities. Although attention-deficit/hyperactivity disorder (ADHD) is the most common psychiatric problem associated with VCFS, there are no reports on methylphenidate treatment in this patient population. Indeed, clinicians have commonly avoided the use of methylphenidate in children with VCFS because of concerns about ineffectiveness or psychotic exacerbation. Forty subjects of mean +/- SD age 11.0 +/- 5.0 years with VCFS were assessed for DSM-IV diagnoses using the Schedule for Affective Disorders and Schizophrenia for School-Aged Children, Present and Lifetime Version, and its extended ADHD module (K-SADS-P-ADHD). Those found to have comorbid ADHD were treated with methylphenidate, 0.3 mg/kg once daily. Treatment efficacy was evaluated after 4 weeks with the K-SADS-P-ADHD, the Conners' Abbreviated Teacher Questionnaire, and the Conners' Continuous Performance Test. Side effects were evaluated with a modified version of the Barkley Side Effects Rating Scale. Of the 18 subjects (45\%) diagnosed with ADHD, 12 agreed to participate in the study. Their ADHD symptoms, both behavioral and cognitive, improved significantly with treatment. None of the patients showed clinical worsening or psychotic symptoms. Side effects were usually mild and did not warrant discontinuation of methylphenidate. The most common side effects were anorexia and depressive-like symptoms. This open-label study indicates that methylphenidate is effective and safe in patients with VCFS. Therefore, its current limited use in this population seems to be unjustified. Larger, controlled clinical and pharmacogenetic studies are needed to confirm these findings. [\hyperlink{Vfend}{PMID: 14658963}, Doron Gothelf et al., 2003]

\hypertarget{pmid_34826122}{A} topical formulation of diclofenac (FLECTOR diclofenac epolamine topical system (FDETS)) is approved in adults for the treatment of acute pain due to minor strains, sprains, and contusions; however, its safety and efficacy have not been investigated in a pediatric population. This study assessed the safety and efficacy of the FLECTOR (diclofenac epolamine) topical system in children. This was an open-label, single-arm, phase IV study at ten USA-based family medicine or pediatric practices in children aged 6-16 years with a clinically significant minor soft tissue injury sustained within the preceding 96 h and at least moderate spontaneous pain on the Wong-Baker FACES 104 patients were enrolled; 52 were 6-11 years old, and 52 were 12-16 years old (mean age 11.6 years). The maximum tolerability score experienced by any patient was 1 (faint redness). Fourteen adverse events (none serious) in nine patients (8.7\%) were considered possibly treatment-related. Reduction in pain during the study was somewhat greater for patients aged 6-11 versus 12-16 years (p < 0.011). The diclofenac plasma concentration tended to be higher in the younger age group compared with older patients: 1.83 versus 1.46 ng/mL at the first assessment and 2.49 versus 1.11 ng/mL at the last assessment (p = 0.002). The FLECTOR topical system safely and effectively provided pain relief for minor soft tissue injuries in the pediatric population, with minimal systemic nonsteroidal anti-inflammatory drug exposure and low potential risk of local or systemic adverse events. ClinicalTrials.gov identifier NCT02132247. [\hyperlink{Vfend}{PMID: 34826122}, Christopher A Jones et al., 2022]

\hypertarget{pmid_18219837}{A}ntihistamines are an established first-line treatment for allergic rhinitis and are widely prescribed in infants for allergic symptoms. To establish the safety and tolerability of fexofenadine hydrochloride in children aged 6 months to 2 years in 2 studies (T/3001 and T/3002). Both studies had a multicenter, randomized, placebo-controlled design. Mean treatment duration was 8 days. Subjects were randomized (T/3001, n = 174; and T/3002, n = 219) to twice-daily fexofenadine hydrochloride, 15 or 30 mg, or placebo mixed with a standard vehicle. In the combined population, the incidence of treatment-emergent adverse events (TEAEs) was comparable between groups (placebo, 48.2\% [96/199]; fexofenadine hydrochloride, 15 mg, 40.0\% [34/85]; and fexofenadine hydrochloride, 30 mg, 35.2\% [38/108]). Vomiting was the most common TEAE (placebo, 13.6\%; fexofenadine hydrochloride, 15 mg, 14.1\%; and fexofenadine hydrochloride, 30 mg, 5.6\%). Most TEAEs were unrelated to study medication, as evaluated by investigators; those possibly related to study medication were mild or moderate in intensity. No clinical differences were seen between fexofenadine and placebo for vital signs, electrocardiographic results, or physical examination results. Fexofenadine hydrochloride, 15 or 30 mg, given for a mean duration of 8 days is well tolerated, with a good safety profile, in children aged 6 months to 2 years. [\hyperlink{Vfend}{PMID: 18219837}, Frank C Hampel et al., 2007]

\hypertarget{pmid_33892771}{T}o compare the safety and efficacy of dexmedetomidine and remifentanil with sufentanil-based general anesthesia for the transthoracic device closure of ventricular septal defects (VSDs) in pediatric patients. A retrospective analysis was performed on 60 children undergoing the transthoracic device closure of VSDs from January 2019 to June 2020. The patients were divided into two groups based on different anesthesia strategies, including 30 cases in group R (dexmedetomidine- and remifentanil-based general anesthesia) and 30 cases in group S (sufentanil-based general anesthesia). There was no significant difference in preoperative clinical information, hemodynamics before induction and after extubation, postoperative pain scores, or length of hospital stay between the two groups. However, the hemodynamic data of group R were significantly lower than those of group S at the time points of anesthesia induction, skin incision, thoracotomy, incision closure, and extubation. The amount of intravenous patient-controlled analgesia (PCA), the duration of mechanical ventilation, and the length of the intensive care unit (ICU) stay in group R were significantly less than those in group S. Dexmedetomidine combined with remifentanil-based general anesthesia for the transthoracic device closure of VSDs in pediatric patients is safe and effective. [\hyperlink{Vfend}{PMID: 33892771}, Ling-Shan Yu et al., 2021]

\hypertarget{pmid_32092134}{A}lthough febrile neutropenia (FN) is a frequent complication in children with cancer receiving chemotherapy, there remains significant variability in selection of route (intravenous [IV] vs oral) and length of therapy. We implemented a guideline with a goal to change practice from using IV antibiotics after hospital discharge to the use of step-down oral therapy with levofloxacin for most children with FN until absolute neutrophil count > 500. The objectives of this study were to determine the impact of this guideline on home IV antibiotic use, and to evaluate the safety of implementation of this guideline. We performed a quasi-experimental, pre-post study of discharge FN treatment at a stand-alone children's hospital in patients without bacteremia discharged between January 2013 and October 2018. In January 2015, a multidisciplinary team created a guideline to switch most children with FN to oral levofloxacin, which was formally implemented as of September 2017. Discharges during the postintervention period (after September 2017) were compared to discharges in the preintervention period (between January 2013 and December 2014). In adjusted multivariable regression analyses, the postimplementation period was associated with a decrease in home IV antibiotics (adjusted risk ratio [aRR], 0.07 [95\% confidence interval \{CI\}, .03-.13]) and fewer IV antibiotic initiations within 24 hours of a new healthcare encounter up to 7 days after discharge (aRR, 0.39 [95\% CI, .17-.93]) compared to the preintervention time period. Step-down oral levofloxacin for children with FN who are afebrile with an ANC ≤ 500 at discharge is feasible and resulted in similar clinical outcomes compared to home IV antibiotics. [\hyperlink{Vfend}{PMID: 32092134}, Jared Olson et al., 2021]

\hypertarget{pmid_14661902}{V}igabatrin (GVG) is an effective antiepileptic drug used for treating partial seizures in adults and children. Over the last years, an increasing number of articles have been published reporting visual field defects (VFD) associated with GVG therapy in adults. To date, however, only an small number of pediatric patients have been reported. This paper is a retrospective review of clinical review to evaluate the prevalence and features of VFD in pediatric patients on GVG monotherapy. Fifteen children, on GVG therapy in the Department of Child Neurology, underwent visual field examination by static threshold automated perimetry using the Humphrey Field Analyzer Program 30-2. The age of these patients ranged from 6 to 18 years (12.4 +/- 3.6 years), 10 of them being male and five female. Three patients (20\%) on GVG monotherapy showed VFD. These consisted in localised, bilateral, and relatively symmetrical, nasal field loss, with relative preservation of the temporal field within the central 30 degree area. [\hyperlink{Vfend}{PMID: 14661902}, Francisco J Ascaso et al., 2003]

\hypertarget{pmid_8201490}{T}here are few objective studies of the benefit/risk ratio of H1-receptor antagonists in children. We hypothesized that terfenadine would provide as effective peripheral H1 blockade as chlorpheniramine in young patients, but would cause less central nervous system dysfunction. We tested this hypothesis with epicutaneous histamine tests to monitor peripheral H1 blockade, P300-event-related potentials as a measure of cognitive processing, and a visual analog scale for somnolence, in a double-blind, single-dose, placebo-controlled, three-way crossover study in 15 children with allergic rhinitis (mean age, 8.5 +/- 1.4 years). On 3 different days the children received terfenadine, 60 mg, chlorpheniramine, 4 mg, or placebo; the tests were performed before and 2 to 2 1/2 hours after dosing. Both terfenadine and chlorpheniramine suppressed the histamine-induced wheal and flare compared with baseline and with placebo; terfenadine was significantly more effective (p < 0.05). Terfenadine did not increase the latency of P300-event-related potentials at the parietal (Pz) or frontal (Fz) scalp electrodes compared with baseline, in contrast to chlorpheniramine and placebo, which did increase P300 latency. Terfenadine and placebo did not increase somnolence compared with baseline, but chlorpheniramine did. In children, as previously documented in adults, terfenadine has a more favorable benefit/risk ratio than chlorpheniramine, as shown by production of significantly greater peripheral histamine blockade and significantly less central nervous system dysfunction. [\hyperlink{Vfend}{PMID: 8201490}, F E Simons et al., 1994]

\hypertarget{pmid_11476456}{T}he incidence of allergic rhinitis in children is increasing. To evaluate the safety of fexofenadine HCI in children ages 6 through 11 years for treatment of seasonal allergic rhinitis. Two large, double-blind, randomized, placebo-controlled, parallel studies with identical protocols included patients with a positive skin test to fall allergen(s) and allergic rhinitis symptoms. Patients were randomized to receive fexofenadine 15, 30, or 60 mg or placebo twice daily for 2 weeks after a 1-week placebo lead-in. Safety was evaluated through adverse event reporting, electrocardiograms, and pre- and posttreatment laboratory panels and physical examinations. A total of 875 patients from both studies were eligible for safety analyses. Ten patients (5 on placebo, 5 on fexofenadine) discontinued because of an adverse event; no event that resulted in discontinuation was judged to be caused by study medication. Incidence of adverse events was similar in active and placebo groups, and did not increase with increasing fexofenadine dose: 36.2\% (83 of 229) in the placebo group versus 35.3\% (79 of 224), 36.8\% (77 of 209), and 34.7\% (74 of 213) in the 15, 30, and 60 mg twice-daily fexofenadine groups, respectively. Headache was the most commonly reported adverse event (6.6\%, 8.0\%, 7.2\%, and 9.4\% in the placebo, 15, 30, 60 mg twice-daily fexofenadine groups, respectively). Clinical, vital sign, electrocardiogram, and laboratory measures were similar in active and placebo groups. There was no statistically significant mean change from baseline in any electrocardiogram parameter after fexofenadine treatment. Fexofenadine, 15, 30, and 60 mg twice daily, was safe and well tolerated in this large pediatric patient population. [\hyperlink{Vfend}{PMID: 11476456}, D F Graft et al., 2001]

\hypertarget{pmid_7590052}{T}o assess the safety and efficacy of intravenous sedation in pediatric upper endoscopy, all elective outpatient procedures performed during a 2-year period (January 1, 1991 through December 31, 1992) were retrospectively reviewed. Of 614 children, 553 received intravenous meperidine and midazolam; 61 received fentanyl and midazolam. The mean dose of meperidine was 1.5 +/- 0.7 mg/kg and of fentanyl 0.0031 +/- 0.0014 mg/kg. Less midazolam was needed for children receiving fentanyl than for those receiving meperidine (0.05 +/- 0.03 mg/kg versus 0.08 +/- 0.05 mg/kg, p < 002). Recovery time (minutes) was shorter for those receiving fentanyl (74.7 +/- 22.8 versus 95.1 +/- 23.0, p < .003). Side effects occurred in 117 patients (19.1\%), of which the majority were mild (83\%); all were transient with no residual sequelae. Inability to complete the procedure occurred in fewer than 1\%. We conclude that both combinations of medication are safe and effective for children of all ages. The use of fentanyl/midazolam results in a shorter recovery time and a lower dose of midazolam. [\hyperlink{Vfend}{PMID: 7590052}, E Chuang et al., 1995]

\hypertarget{pmid_3118743}{T}he pharmacokinetics of alfentanil were studied in 18 children (3 months to 14 yr of age) undergoing surgery. Alfentanil was administered as a 30-s infusion of either 50 or 120 micrograms.kg-1. Pharmacokinetic values were independent of dose. There was no correlation between Vdss (Volume of distribution at a steady state, measured as l.kg-1) and age or weight, and there was a weak correlation between clearance and age (r = 0.51, P less than 0.05). Vdss and elimination t1/2 were not different in infants less than 1 yr of age when compared to older children. The mean value of Vdss was 0.419 (SE .028) l.kg-1) for the whole group, and elimination t1/2 was 76.3 (SE 6.5) min. The clearance rate [TBC = 7.9 (SE 0.41) ml.kg-1.min-1] was within the range of values previously determined in adult studies. From these data, it would appear that, although there may be differences in the disposition kinetics between children aged 3 months to 14 yr and those measured in adults in some studies by other investigators, age-related differences within this group were not demonstrable. [\hyperlink{Vfend}{PMID: 3118743}, G V Goresky et al., 1987]

\section*{Xarelto}
\subsection*{Result}
\subsubsection*{Answer}

Unknown

\subsubsection*{{Explanation}}
\hypertarget{Xarelto}
After reviewing all available abstracts, there are no abstracts that specifically address the safety of Xarelto (rivaroxaban) in children. None of the abstracts describe a targeted study evaluating the safety of Xarelto in pediatric populations, nor do they provide evidence affirming or refuting its safety in any specific age range of children. Therefore, based on the abstracts provided, the safety of Xarelto in children is unknown.

\subsection*{Abstracts}
\hypertarget{pmid_31319883}{L}ertal®, an oral nutraceutical, contains extract of Perilla, quercetin, and Vitamin D3. The current polycentric, randomized, parallel-group, controlled study aimed in the Phase II to evaluate the efficacy and safety of Lertal® in preventing allergic rhinitis (AR) exacerbations in children after the end of the pharmacological treatment phase. One hundred twenty-eight children completed Phase II. Sixty-four children continued Lertal® treatment (Lertal® Group: LG) and 64 ones did not assume any medication (Observation Group: OG) for 4-12 weeks. The study endpoints were the number, intensity, and duration of AR exacerbations, and the length of symptom-free time. Children of LG halved the risk (HR = 0.54) of having AR exacerbation. Children of LG had significantly (p = 0.039) less AR exacerbations than OG children. In children with AR exacerbations, the total number of days in which each patient took at least one rescue medication was significantly (p = 0.018) lesser in LG children than OG ones. In the global population, the cumulative days treated with rescue medication was significantly (p < 0.0001) higher in OG than in LG. There was no clinically relevant adverse event. The present study documented that prolonged Lertal® assumption was safe and able to significantly reduce, such as halving, the risk of AR exacerbation, their duration and the use of rescue medications, after the suspension of the one-month antihistamine treatment. Therefore, Lertal® could be envisaged as an effective preventive treatment in AR children able to guarantee long symptom-free time. Clinical trial registration: ClinicalTrials gov ID NCT03365648 . [\hyperlink{Xarelto}{PMID: 31319883}, Gianluigi Marseglia et al., 2019]

\hypertarget{pmid_28741653}{C}hloral hydrate is commonly used to sedate children for painless procedures. Children may recover more quickly after sedation with dexmedetomidine, which has a shorter half-life. We randomly allocated 196 children to chloral hydrate syrup 50 mg.kg [\hyperlink{Xarelto}{PMID: 28741653}, V M Yuen et al., 2017] Sedation is often required for children undergoing diagnostic procedures. Chloral hydrate has been one of the sedative drugs most used in children over the last 3 decades, with supporting evidence for its efficacy and safety. Recently, chloral hydrate was banned in Italy and France, in consideration of evidence of its carcinogenicity and genotoxicity. Dexmedetomidine is a sedative with unique properties that has been increasingly used for procedural sedation in children. Several studies demonstrated its efficacy and safety for sedation in non-painful diagnostic procedures. Dexmedetomidine's impact on respiratory drive and airway patency and tone is much less when compared to the majority of other sedative agents. Administration via the intranasal route allows satisfactory procedural success rates. Studies that specifically compared intranasal dexmedetomidine and chloral hydrate for children undergoing non-painful procedures showed that dexmedetomidine was as effective as and safer than chloral hydrate. For these reasons, we suggest that intranasal dexmedetomidine could be a suitable alternative to chloral hydrate. [\hyperlink{Xarelto}{PMID: 28741653}, Giorgio Cozzi et al., 2017]

\hypertarget{pmid_6937455}{H}aloperidol is safe and effective in children for relieving psychotic symptoms associated with childhood autism, schizophrenia and mental retardation. It is the drug of choice for Tourette's syndrome, and may be useful in nonpsychotic hyperactive or aggressive children to control acute episodes, or when the stimulants normally useful in hyperactive children are ineffective. Such children taking haloperidol not only become calmer, but are often better able to respond to other modalities of therapy and to school instruction. Dosage, initially low, is increased gradually to minimize drowsiness and extrapyramidal symptoms, the most common side effects. Haloperidol in children is usually well-tolerated. [\hyperlink{Xarelto}{PMID: 6937455}, A C Serrano et al., 1981]

\hypertarget{pmid_27560971}{O}bservational studies and anecdotal reports suggest that sertraline, a selective serotonin reuptake inhibitor, may improve language development in young children with fragile X syndrome (FXS). The authors evaluated the efficacy of 6 months of treatment with low-dose sertraline in a randomized, double-blind, placebo-controlled trial in 52 children with FXS aged 2 to 6 years. Eighty-one subjects were screened for eligibility, and 57 were randomized to sertraline (27) or placebo (30). Two subjects from the sertraline arm and 3 from the placebo arm discontinued. Intent-to-treat analysis showed no difference from placebo on the primary outcomes: the Mullen Scales of Early Learning (MSEL) expressive language (EL) age equivalent and Clinical Global Impression Scale-Improvement. However, analyses of secondary measures showed significant improvements, particularly in motor and visual perceptual abilities and social participation. Sertraline was well tolerated, with no difference in side effects between sertraline and placebo groups. No serious adverse events occurred. This randomized controlled trial of 6 months of sertraline treatment showed no primary benefit with respect to early EL development and global clinical improvement. However, in secondary exploratory analyses, there were significant improvements seen on motor and visual perceptual subtests, the cognitive T score sum on the MSEL, and on one measure of social participation on the Sensory Processing Measure-Preschool. Furthermore, post hoc analysis found significant improvement in early EL development as measured by the MSEL among children with autism spectrum disorder on sertraline. Treatment appears safe for this 6-month period in young children with FXS, but the authors do not know the long-term side effects of this treatment. These results warrant further studies of sertraline in young children with FXS using refined outcome measures as well as longer term follow-up studies to address long-term side effects of low-dose sertraline in early childhood. [\hyperlink{Xarelto}{PMID: 27560971}, Laura Greiss Hess et al., 2016]

\hypertarget{pmid_11552629}{T}he aim of the study was to research the efficiency of sertraline (zoloft) in depressions, anxious states and obsessive-compulsive disorders. Diagnosis of the mental disorders was carried out according to ICD-10. 72 children (59 boys, 13 girls) aged 6-18 years were treated. There were 32 inpatients and 40 outpatients. Therapy with sertraline was performed during 8 weeks with a gradual increase (titration) and individual selection of the dose from 12.5 to 100 mg/day. During the therapy clinical observation was combined with the patients' examination using Hamilton Depression Scale and Hamilton Anxiety Scale (HAM-D and HAM-A), and a Clinical Global Impression Scale (CGI). It was established that sertraline was very effective and safe drug in children (it has no influence on cognitive functions, has neither myorelaxing or sedative effects). Activity of this drug is characterized by quick manifestation of thymoanaleptic and anxiolytic effects. It mild depressive states 50 mg/day is a significant dose; in more severe depressions and obsessive-compulsive disorders the dose in juveniles was to 100 mg, the duration of the therapy was more than 2 months. [\hyperlink{Xarelto}{PMID: 11552629}, V M Voloshina et al., 2001]

\hypertarget{pmid_23257756}{E}fficacy and safety of sertraline (zoloft) have been assessed in 26 patients, aged from 7 to 15 years, with depressive states of different severity and psychopathological structure which are combined with obsessive-compulsive symptoms. Changes in patient's status are analyzed using psychometric scales during 6 weeks of treatment. It has been concluded that Zoloft is effective and safe drug for the treatment of mild to moderate depressive disorders concomitant to obsessive-compulsive disorders in children age. [\hyperlink{Xarelto}{PMID: 23257756}, A V Goriunov et al., 2012]

\hypertarget{pmid_9579287}{S}ertraline (Zoloft) is a selective serotonin reuptake inhibitor that is commonly used in adults in the treatment of mood and anxiety disorders. Whereas it also is used to treat these illnesses in children, it is not currently approved by the Food and Drug Administration for use in this population. Sertraline use has been increasing secondary to its efficacy and its more tolerable side effect profile than the tricyclic antidepressants. It is also much safer in overdose than the tricyclic antidepressants. Although there have been numerous reports of sertraline overdose in adults, reports in the pediatric population are much less common. We review the literature regarding sertraline overdose in children, describe a case of sertraline ingestion in a 22-month-old infant, and discuss the treatment of such an overdose. [\hyperlink{Xarelto}{PMID: 9579287}, G Catalano et al., ]

\hypertarget{pmid_1671951}{T}o examine whether antipyretic therapy in young children is associated with potential risks (interference with enhanced host defences at febrile temperatures) or benefits (improved comfort and behaviour), a randomised, double-blind, placebo-controlled trial of paracetamol was conducted among 225 children 6 months to 6 years of age who presented with acute (less than or equal to 4 days) fever (greater than or equal to 38 degrees C per rectum) without evident bacterial focus of infection. Parents were asked to give paracetamol liquid 10-15 mg/kg or placebo every 4 h as needed for fever and to avoid bathing, sponging, or other pharmacological agents. Parents kept temperature and symptom diaries and recorded changes in child comfort and behaviour according to a pretested, 5-category Likert-type questionnaire 1-2 h after every dose. There were no significant differences between treated and placebo groups in mean duration of subsequent fever (34.7 vs 36.1 h) or other symptoms (72.9 vs 71.7 h). Paracetamol-treated children were more likely to be rated by their parents as having at least a 1-category improvement in activity (38 vs 11\%; p = 0.005) and alertness (33 vs 12\%; p = 0.036) but no significant differences were noted in mood, comfort, appetite, or fluid intake. That overall improvement in behaviour and comfort with paracetamol was not impressive is underscored by the inaccuracy of parents' "guess" at the end of the trial as to which agent their child had received-45\% correct guesses for paracetamol and 52\% for placebo. The data suggest that the clinically relevant hazards and benefits of paracetamol antipyresis have been exaggerated. [\hyperlink{Xarelto}{PMID: 1671951}, M S Kramer et al., 1991]

\hypertarget{pmid_23131185}{T}he safety of a novel 0.5\% ivermectin lotion (IVL) and potential for ivermectin absorption after application was investigated in an open-label study in young children, and a human repeat insult patch test (HRIPT) and cumulative irritation test (CIT) assessed any potential for cumulative dermal irritation and contact sensitization. In the pharmacokinetic and safety study, 30 head louse-infested children ages 6 months to 3 years received a 10-minute application of IVL on day 1. Blood was collected before application; 0.5, 1, and 6 hours after rinsing; and on days 2 and 8. Samples from 20 subjects were assayed for ivermectin (test sensitivity 0.05 ng/mL). Liver panel and complete blood counts were completed for all subjects. For the HRIPT/CIT, occlusive patches containing IVL or vehicle control lotion (CL) were repeatedly applied to 220 healthy adult subjects to assess contact sensitization; for cumulative dermal irritation testing, additional patches with normal saline and sodium dodecyl sulfate (SDS) were applied to 36 subjects. In the open-label study, all detected ivermectin plasma concentrations were <1 ng/mL. No safety signals emerged, and treatment was well tolerated. In the HRIPT/CIT, IVL was significantly less irritating than normal saline and SDS, with no evidence of dermal irritation or sensitization in human skin. IVL was safe when applied topically, absorption was de minimus, there was no evidence of irritation or sensitization from repeated exposures, and results support the safety of topical IVL use in children as young as 6 months. [\hyperlink{Xarelto}{PMID: 23131185}, Lydie Hazan et al., ]

\hypertarget{pmid_2402648}{C}hloral hydrate has been used extensively to sedate children, but at Brooke Army Medical Center, other drug combinations were becoming increasingly popular due to a perception that chloral hydrate had a high rate of failure, especially with younger or neurologically impaired children. Therefore, 50 children were given the drug before a diagnostic study, and patient data and a sedation score were recorded on a worksheet. Of 50 children, 43 (86\%) were "successfully sedated" on the first attempt with no side effects. Children with neurologic disorders had a much greater (27\% vs 4\%) failure rate than "normal" children. The sedation rate did not significantly differ by age, sex, or initial drug dosage. The study suggest that chloral hydrate is a safe and effective oral sedative but that children with neurologic disorders may need alternative drugs for sedation. [\hyperlink{Xarelto}{PMID: 2402648}, P D Rumm et al., 1990]

\hypertarget{pmid_28827252}{C}eftriaxone is widely used in children in the treatment of sepsis. However, concerns have been raised about the safety of ceftriaxone, especially in young children. The aim of this review is to systematically evaluate the safety of ceftriaxone in children of all age groups. MEDLINE, PubMed, Cochrane Central Register of Controlled Trials, EMBASE, CINAHL, International Pharmaceutical Abstracts and adverse drug reaction (ADR) monitoring systems will be systematically searched for randomised controlled trials (RCTs), cohort studies, case-control studies, cross-sectional studies, case series and case reports evaluating the safety of ceftriaxone in children. The Cochrane risk of bias tool, Newcastle-Ottawa and quality assessment tools developed by the National Institutes of Health will be used for quality assessment. Meta-analysis of the incidence of ADRs from RCTs and prospective studies will be done. Subgroup analyses will be performed for age and dosage regimen. Formal ethical approval is not required as no primary data are collected. This systematic review will be disseminated through a peer-reviewed publication and at conference meetings. CRD42017055428. [\hyperlink{Xarelto}{PMID: 28827252}, Linan Zeng et al., 2017]

\hypertarget{pmid_2326439}{T}his paper reports on 350 pediatric patients who were studied over a 17-month period to determine the efficacy and safety of oral and intramuscular sedation techniques. The protocol using oral chloral hydrate, 50 mgm/kg, for infants under 1 year of age or intramuscular pentobarbital, 5 mgm/kg, for children over 1 year was found to be an effective, safe and fairly simple approach to pediatric sedation. Of the 350 sedated patients, 343 (98 percent) had satisfactory scans on the same day the examination was scheduled after a single dose or an initial dose and supplementary sedation. [\hyperlink{Xarelto}{PMID: 2326439}, J B Temme et al., ]

\hypertarget{pmid_2391756}{W}e administered norfloxacin (NFLX) to 16 children aged 3 to 14 year-old at the dose of 5.2 to 17.2 mg/kg/day. We evaluated the efficacy and safety of NFLX in 6 children with respiratory tract infections, 8 urinary tract infections, and 2 gastrointestinal tract infections. Efficacy rate of NFLX was 93.8\% and eradicated rate was 92.9\%. Any adverse effects were not observed. These results suggested that NFLX could be used safely to the children. [\hyperlink{Xarelto}{PMID: 2391756}, T Ihara et al., 1990]

\hypertarget{pmid_10709457}{I}n 1998, 77 cases of accidental ingestion of paracetamol paediatric syrup (Efferalgan) in children were notified to the Marseille Poison Centre. In a quarter of them, the alleged dose taken was greater than the toxic dose. Ingestion was mainly due to the child opening the bottle. The proximate marketing of a product with a child-proof top, which should allow the number of accidents to be reduced. Doctors and pharmacists should be informed rapidly, so that they can warn the families who still have the old type of bottle. [\hyperlink{Xarelto}{PMID: 10709457}, L de Haro et al., ]

\hypertarget{pmid_951517}{O}ral administration of solutions of pyrantel tartrate at 50, 75, 100 and 125 mg/kg body weight gave 43.9, 82.1, 92.8 and 99.1 per cent efficacy respecively. None of the chicks given 100 and 125 mg/kg body weight showed any sign of toxicity. [\hyperlink{Xarelto}{PMID: 951517}, E D Okon et al., 1976]

\hypertarget{pmid_2019938}{T}o test whether nebulized salbutamol (albuterol) is safe and efficacious for the treatment of young children with acute bronchiolitis, we enrolled 83 children (median age 6 months, range 1 to 21 months) in a randomized, double-blind clinical trial. Participants received two treatments at 30-minute intervals of either nebulized salbutamol (0.10 mg/kg in 2 ml 0.9\% saline solution) or a similar volume of 0.9\% saline solution placebo. Outcome measures were the respiratory rate, pulse oximetry, and a clinical score based on the degree of wheezing and retractions. Patients in the salbutamol arm had significantly greater improvement in clinical scores after the initial treatment (p = 0.04). There was no difference between the groups in oxygen saturation (p = 0.74); patients treated with salbutamol had a small increase in heart rate after two treatments (159 +/- 16 vs 151 +/- 16; p = 0.03). We conclude that salbutamol is safe and effective for the initial treatment of young children with acute bronchiolitis. [\hyperlink{Xarelto}{PMID: 2019938}, T P Klassen et al., 1991]

\hypertarget{pmid_21786126}{T}here is a paucity of evidence with regard to the safety of contrast medium administration at MRI in neonates and infants. To assess immediate adverse reactions in children younger than 18 months of age during routine clinical utilization of gadoteric acid (Gd-DOTA) in a cohort of patients with nonselected indications. One hundred and four neonates and infants were enrolled in a postmarketing survey with Gd-DOTA (Dotarem, Guerbet, Roissy, France) from a single pediatric hospital. A standardized questionnaire was used to collect the patient information. All included children, ages 3 days to 18 months, received one injection of Gd-DOTA (volume 0.6-4 ml). No immediate adverse event was reported. This postmarketing study involving neonates and infants suggests a favorable safety profile of Gd-DOTA in routine practice. [\hyperlink{Xarelto}{PMID: 21786126}, Sophie Emond et al., 2011] 37 febrile children aged between 3 months and 6 years were treated with paracetamol in a dose of 15--20 mg/kg by either oral elixir or rectal suppository. The rectal route was found to have an equal antipyretic effect and offers a practical alternative in those children for whom the oral route is not possible.20 [\hyperlink{Xarelto}{PMID: 21786126}, S Vernon et al., 1979]

\hypertarget{pmid_35206901}{I}ntranasal dexmedetomidine (IN DEX) is a relatively new sedative agent with supporting evidence on its efficacy and safety, which can be used for procedural sedation in children, and could have a major role in auditory brainstem response testing, especially in the case of non-cooperative children. The goal of this systematic review is to assess the role of IN DEX in ABR testing, evaluating the reported protocol, potential, and limits. We performed a comprehensive search strategy on PubMed, Scopus, and Google Scholar, including studies in English on the pediatric population, without time restrictions. Six articles, published between 2016 and 2021, were included in the systematic review. Sedation effectiveness was high across the studies, except for one study; 3 μg/kg was the dosing most often used. A comparison group was present in three studies, with oral chloral hydrate as the drug of comparison. Adverse effects were rarely reported. This systematic review showed how IN DEX can represent an adequate sedative for children undergoing ABR testing; larger and more rigorous trials are warranted in order to recommend its systematic utilization. [\hyperlink{Xarelto}{PMID: 35206901}, Pasquale Marra et al., 2022]

\hypertarget{pmid_22934167}{Y}oung children with fragile X syndrome (FXS) often experience anxiety, irritability, and hyperactivity related to sensory hyperarousal. However, there are no medication recommendations with documented efficacy for children under 5 years old of age with FXS. We examined data through a chart review for 45 children with FXS, 12-50 months old, using the Mullen Scales of Early Learning (MSEL) for baseline and longitudinal assessments. All children had clinical level of anxiety, language delays based on MSEL scores, and similar early learning composite (ELC) scores at their first visit to our clinic. Incidence of autism spectrum disorder (ASD) was similar in both groups. There were 11 children who were treated with sertraline, and these patients were retrospectively compared to 34 children who were not treated with sertraline by chart review. The baseline assessments were done at ages ranging from 18 to 44 months (mean 26.9, SD 7.99) and from 12 to 50 months (mean 29.94, SD 8.64) for treated and not treated groups, respectively. Mean rate of improvement in both expressive and receptive language development was significantly higher in the group who was treated with sertraline (P < 0.0001 and P = 0.0071, resp.). This data supports the need for a controlled trial of sertraline treatment in young children with FXS. [\hyperlink{Xarelto}{PMID: 22934167}, Tri Indah Winarni et al., 2012]

\hypertarget{pmid_16042782}{X}ylitol, a polyol sugar, has been shown to reduce dental caries when mixed with food or chewing gum. This study examines the taste acceptability of xylitol in milk as a first step toward measuring the effectiveness of xylitol in milk for the reduction of dental caries in a public health program. Three different types of milk (Ultra High Temperature (UHT), powder and evaporated) were tested for acceptability by 75 Peruvian children (25 per milk group, ages 4 to 7 years). Each group evaluated xylitol and sorbitol in one type of milk. In the first phase, each child was presented with a tray of four plastic cups containing 50 ml of milk with 0.021 g/ml xylitol, 0.042 g/ml xylitol, 0.042 g/ml sorbitol or no sugar. Each child was asked to taste the samples in a self-selected order. After tasting each sample, the child placed the milk cup in front of one of three cartoon faces (smile, frown or neutral) representing the child's response to the taste of each sample. In the second phase, the child was asked to rank order the milk samples within each category (smile, frown or neutral). Ranks within categories were then combined to obtain a rank ordering for all the test samples. The ranking from best to worst for the samples across categories (UHT, powder, evaporated) was xylitol (0.0.042 g/ml), sorbitol (0.042 g/ml), xylitol (0.021 g/ml) and milk alone (Friedman's ANOVA). Xylitol and sorbitol were preferred over milk alone, and xylitol (0.042 g/ml) was preferred to sorbitol (0.042 g/ml)(p < .05 sign test). Milk sweetened with xylitol is well accepted by Peruvian children ages 4-7 years. [\hyperlink{Xarelto}{PMID: 16042782}, Jorge L Castillo et al., 2005]

\hypertarget{pmid_8282390}{A} randomized double blind placebo controlled trial was carried out to study the effect of phenobarbitone (PB) in preventing recurrences of simple and atypical febrile convulsions among children in the age group 6 months to 6 years. Children with simple febrile convulsions were randomly allocated to receive either phenobarbitone or placebo. Children with atypical convulsions were treated with phenobarbitone, as a third group. Thirty children were admitted in each group. All the children were followed up for a period of twelve months. Recurrence of convulsions and side effects of PB were recorded. Recurrence occurred in only 7\% (95\% confidence interval: 1-22) of children on Phenobarbitone, suffering from either simple or atypical febrile convulsions, compared to 53\% (95\% confidence interval: 34-72) of children on placebo, suffering from simple febrile convulsions. With Phenobarbitone, 5\% of children had intolerable side effects. These results suggest that long term prophylaxis with phenobarbitone, even in simple febrile convulsions will be useful. [\hyperlink{Xarelto}{PMID: 8282390}, N Thilothammal et al., 1993]

\hypertarget{pmid_9402381}{T}he use of quinolones in children and accumulation of data on the pharmacodynamics of these drugs have been limited and delayed by concern regarding their chondrotoxicity. A comprehensive review of the findings in animals compared with the cumulative published findings in children and adolescents (>7,000 to date) allows the conclusion that such concern is not justified. Prospective controlled studies in children are justifiable in view of a continuing lack of correlation between findings in juvenile animals and those in children and because of the selected therapeutic advantages of the current and newer quinolones. [\hyperlink{Xarelto}{PMID: 9402381}, J E Burkhardt et al., 1997]

\hypertarget{pmid_29747541}{D}exrazoxane can prevent anthracycline-associated cardiotoxicity. However, in 2011, its use in children was contraindicated by the EMA over concerns of increased risk of infection, myelosuppression and second primary malignancies, and because its efficacy in children had not then been established. We review here the evidence published since 2011, which confirms that dexrazoxane is an effective cardioprotectant in children and adolescents, is not associated with an increased risk of second primary malignancies or excess early or late mortality and does not impair chemotherapy efficacy. Based on this evidence, the contraindication for children and adolescents requiring high doses of anthracyclines and at risk for cardiotoxicity was removed from the European labeling for dexrazoxane. [\hyperlink{Xarelto}{PMID: 29747541}, Peter Reichardt et al., 2018]

\section*{Xofluza}
\subsection*{Result}
\subsubsection*{Answer}

Unknown

\subsubsection*{{Explanation}}
\hypertarget{Xofluza}
After reviewing all available abstracts, there is no abstract that specifically addresses the safety of Xofluza (baloxavir marboxil) in children. None of the abstracts mention Xofluza or present targeted studies evaluating its safety in pediatric populations of any age range. Therefore, based on the abstracts provided, there is no evidence to affirm or refute the safety of Xofluza in children. The safety of Xofluza in children is therefore unknown according to the available data.

\subsection*{Abstracts}
\hypertarget{pmid_23257756}{E}fficacy and safety of sertraline (zoloft) have been assessed in 26 patients, aged from 7 to 15 years, with depressive states of different severity and psychopathological structure which are combined with obsessive-compulsive symptoms. Changes in patient's status are analyzed using psychometric scales during 6 weeks of treatment. It has been concluded that Zoloft is effective and safe drug for the treatment of mild to moderate depressive disorders concomitant to obsessive-compulsive disorders in children age. [\hyperlink{Xofluza}{PMID: 23257756}, A V Goriunov et al., 2012]

\hypertarget{pmid_37309365}{P}revious clinical trials established the efficacy and safety of sucrose-formulated recombinant factor (F) VIII (rFVIII-FS/Kogenate FS®/Helixate FS®) and octocog alfa (BAY 81-8973/Kovaltry®; LEOPOLD trials). To report the results of a post hoc subgroup analysis assessing efficacy and safety outcomes in patients with hemophilia A who were receiving rFVIII-FS prior to enrolling into the LEOPOLD I Part B and LEOPOLD Kids Part A clinical trials and switching to octocog alfa. LEOPOLD I Part B (NCT01029340) and LEOPOLD Kids Part A (NCT01311648) were octocog alfa Phase 3, multinational, open-label studies in patients with severe hemophilia A aged 12-65 years and ≤12 years, respectively. Annualized bleeding rate (ABR) was the efficacy endpoint for both studies. Safety endpoints included adverse events (AEs) and development of FVIII inhibitors. Of the 113 patients in both LEOPOLD trials, 40 (35.4\%) patients received rFVIII-FS prophylaxis pre-study and had data available for pre-study total ABR. In LEOPOLD I Part B (n = 22, 35.5\%), median (Q1; Q3) total ABR decreased from 2.5 (0.0; 9.0) pre-study to 1.0 (0.0; 6.8), and from 1.0 (0.0; 6.0) pre-study to 0.0 (0.0; 6.02) in LEOPOLD Kids Part A (n = 18, 35.3\%). Octocog alfa was well tolerated, and no patients had drug-related serious AEs or inhibitors. Treatment with octocog alfa prophylaxis appeared to have a favorable risk-benefit profile compared with rFVIII-FS and thus could be an effective and improved alternative strategy for individualized treatment for children, adolescent and adult patients with severe hemophilia A currently on rFVIII-FS treatment. [\hyperlink{Xofluza}{PMID: 37309365}, Gili Kenet et al., 2023]

\hypertarget{pmid_6214182}{T}he clinical efficacy and safety of the new oxacephalosporin moxalactam disodium were evaluated in 54 children with a variety of pediatric infections. Except for a terminally ill neutropenic leukemic patient with pneumonia and sepsis due to Pseudomonas aeruginosa who  died shortly after initiation of therapy, moxalactam treatment was effective in all patients. No recurrent infections were observed. The rate of clinical response to moxalactam appeared to be at least comparable to that of patients treated with traditional antibiotics. In vitro sensitivity testing demonstrated that all bacteria isolated except P aeruginosa were sensitive to moxalactam while Haemophilus influenzae was exquisitely sensitive. Side effects included thrombocytosis (five patients), transient SGPT elevations and eosinophilia (three each), fever with rash (one), and neutropenia (one). In one patient, superinfection with Streptococcus faecalis developed. We conclude that moxalactam may be a useful antibiotic in pediatrics, particularly for the treatment of infections due to H. influenzae and Enterobacteriaceae. Its role in infections caused by group B streptococcus and Pseudomonas awaits further studies. [\hyperlink{Xofluza}{PMID: 6214182}, R Yogev et al., 1982]

\hypertarget{pmid_23401309}{T}o report our experience on the safety and tolerability of moxifloxacin for treating children affected by pulmonary TB. Children receiving a moxifloxacin-containing anti-TB regimen were included in the study. Their medical records were revised at the end of follow-up. We describe nine children treated with moxifloxacin for pulmonary TB at Regina Margherita Children's Hospital (Turin, Italy) between 2007 and 2012. Moxifloxacin was administered orally at 10 mg/kg/day once daily (maximum dose = 400 mg/day) following World Health Organization indications. During treatment, patients were systematically assessed for the development of side effects. Eight children were considered cured at the end of treatment; one child was lost to follow-up after 3 months of treatment. Two children had side effects during treatment: one developed arthritis of the ankle; the other had liver toxicity, whose relationship with moxifloxacin could not be ruled out. We did not observe any case of QT prolongation, central nervous system disorders, growth defects or gastrointestinal disturbances. A moxifloxacin-containing regimen might be considered for the treatment of TB in children, especially for drug-resistant and extensive forms. However, vigilance for possible side effects is recommended, especially if other drugs are concomitantly used. Studies on wider populations are needed to better define the impact of long-term treatments with quinolones on children's growth and psychomotor development and to outline regulatory indications on moxifloxacin use in the pediatric setting. [\hyperlink{Xofluza}{PMID: 23401309}, Silvia Garazzino et al., 2014]

\hypertarget{pmid_16554175}{T}o evaluate the long-term efficacy, tolerability, and safety of oxcarbazepine (OXC) in children with epilepsy. We enrolled 36 patients (median age 7.75) with new diagnosis of partial epilepsy in an open prospective study. All type of epilepsy were included: 25 patients were affected by idiopathic epilepsy, eight by symptomatic epilepsy and three by cryptogenic epilepsy. Patients were then scheduled to come back for controls at 3 months (T1), 12 months (T2) and 24 months (T3) after the beginning of OXC-monotherapy (T0). At each control we evaluated patients through their seizure diary, a questionnaire on side effects, their level of 10-monohydroxy (MHD) metabolite and laboratory analysis. At T1, 21/36 patients (58.3\%) were seizure-free, 3/36 patients (8.3\%) showed an improvement higher than 50\%, 3/36 (8.3\%) lower than 50\%, while 2/36 worsened (5.6\%). In 7/36 (19.5\%) patients, no improvement was reported. At T2 13/18 patients (72.2\%) were seizure-free, 1/18 showed a response to therapy higher than 50\% while 2/18 worsened (11\%). In two patients no improvement was reported. A correspondence between MHD plasmatic levels and clinical response (r=0.49; p<0.05) was only registered at T1. An EEG normalization was observed in 25\% of cases. Side effects were reported in 25\% of cases, but symptoms progressively disappeared at follow-up. We can therefore conclude that OXC can be considered, for its efficacy and safety, as a first line drug in children with epilepsy. [\hyperlink{Xofluza}{PMID: 16554175}, E Franzoni et al., 2006]

\hypertarget{pmid_29498372}{S}cientific literature data on the experience of use of Proteflazid® (drops) and Immunoflazid® (syrup) for the treatment of viral diseases in children of the first six years of life are analysed in the article. A systematic review was conducted on the basis of postmarketing comparative clinical trials and long-term follow-up (during the period of 2002 to 2016) that involved about 1500 children (the intent-to-treat population comprised more than 800 of them). The safety and efficacy of the Proteflazid® (drops) and Immunoflazid® (syrup) usage in children for the treatment of viral infections have been proven. [\hyperlink{Xofluza}{PMID: 29498372}, Galina Beketova et al., 2018]

\hypertarget{pmid_10851644}{C}iprofloxacin clinical and bacteriological efficacies, as well as tolerability mainly with respect to chondrotoxicity were evaluated in the treatment of children with mucoviscidosis. The drug was shown to have high clinical and moderate bacteriological efficacies. As for its tolerability, adverse reactions chiefly associated with affection of the gastrointestinal tract, i.e. nausea, stomach pain, diarrhea, increased transaminase levels were recorded. The arthrotoxicity episode was single and transitory. The use of ciprofloxacin had no negative effect on the children growth. No chondrotoxic effect of ciprofloxacin in the treatment of children was observed which is explained in the paper. It is concluded that ciprofloxacin is in general an efficient and safe antibiotic useful for the treatment of children. [\hyperlink{Xofluza}{PMID: 10851644}, S S Postnikov et al., 2000]

\hypertarget{pmid_1494233}{C}efprozil (CFPZ, BMY-28100) was evaluated for its efficacy, safety and pharmacokinetics in children. CFPZ was effective against streptococcal pharyngitis, pneumococcal lower respiratory tract infections, staphylococcal skin infections and Escherichia coli urinary tract infections, but was less effective against lower respiratory tract infections and otitis media due to Haemophilus influenzae. No adverse reactions were encountered in 46 cases treated with CFPZ. With a premeal administration of 7.5 mg/kg, the Cmax was approximately 3.2 micrograms/ml and the T 1/2 beta was 1.4 hours. From the present study, CFPZ appears to be safe and effective against community-acquired childhood infections. [\hyperlink{Xofluza}{PMID: 1494233}, H Meguro et al., 1992]

\hypertarget{pmid_20415263}{T}he results of the multicentre clinical trials on cycloferon efficacy in children at the age from 4 to 16 years are presented. The prophylactic effect of the drug (2.9-7.2-fold decrease of the morbidity) with respect to the respiratory tract mono- and mixed infections was showen. The marked cytoprotective effect, evident from lower destruction of the epithelial cells and increased activity of the local nonspecific resistance factors (lysozyme, secretory immunoglobulin A) was observed. [\hyperlink{Xofluza}{PMID: 20415263}, M G Romantsov et al., 2009]

\hypertarget{pmid_3430725}{A} new oxacephem antibiotic, flomoxef sodium (FMOX, 6315-S), was studied for its clinical efficacy in the field of pediatrics. The treated patients were infants and children ranging from 6 months to 14 years old suffering from bacterial pneumonia in 3 cases, acute tonsillitis in 2 cases, acute enterocolitis in 2 cases, and cellulitis and urinary tract infection in 1 case each, a total of 9 cases. FMOX was administered at (levels of) 57-150 mg/kg in daily dose with durations of treatment ranging from 5 to 18 days. Clinical efficacies of good or excellent results were obtained in all cases (excellent in 4, good in 5). As an adverse reaction, eosinophilia was observed in 1 patient. This elevation is, however, normalized with the cessation of the treatment. [\hyperlink{Xofluza}{PMID: 3430725}, H Ogura et al., 1987]

\hypertarget{pmid_29356761}{T}his study was designed to evaluate primarily the safety and also the efficacy of moxifloxacin (MXF) in children with complicated intra-abdominal infections (cIAIs). In this multicenter, randomized, double-blind, controlled study, 451 pediatric patients aged 3 months to 17 years with cIAIs were treated with intravenous/oral MXF (N = 301) or comparator (COMP, intravenous ertapenem followed by oral amoxicillin/clavulanate; N = 150) for 5 to 14 days. Doses of MXF were selected based on the results of a Phase 1 study in pediatric patients (NCT01049022). The primary endpoint was safety, with particular focus on cardiac and musculoskeletal safety; clinical and bacteriologic efficacy at test of cure was also investigated. The proportion of patients with adverse events (AEs) was comparable between the 2 treatment arms (MXF: 58.1\% and COMP: 54.7\%). The incidence of drug-related AEs was higher in the MXF arm than in the COMP arm (14.3\% and 6.7\%, respectively). No cases of QTc interval prolongation-related morbidity or mortality were observed. The proportion of patients with musculoskeletal AEs was comparable between treatment arms; no drug-related events were reported. Clinical cure rates were 84.6\% and 95.5\% in the MXF and COMP arms, respectively, in patients with confirmed pathogen(s) at baseline. MXF treatment was well tolerated in children with cIAIs. However, a lower clinical cure rate was observed with MXF treatment compared with COMP. This study does not support a recommendation of MXF for children with cIAIs when alternative more efficacious antibiotics with better safety profile are available. [\hyperlink{Xofluza}{PMID: 29356761}, Stefan Wirth et al., 2018]

\hypertarget{pmid_7633153}{W}e evaluated the safety of ciprofloxacin administered in a dose of 15-25 mg/kg for 9-16 days, in a case series of 58 children who were between 8 months and 13 years of age. No arthropathy was observed during therapy and follow-up. Blinded evaluation of 22 pairs of nuclear magnetic resonance scans obtained before and between day 10 and 15 of therapy did not reveal any cartilage damage. After the first dose of ciprofloxacin (10 mg/kg), serum fluoride levels increased at 12 h in 15 of 19 (79\%) patients; 24-h urinary fluoride excretion was higher on day 7 compared with basal values in 16 of 18 (88.9\%) patients. Height z scores of 53 patients at a mean of 22.5 months of follow-up were not significantly different from basal scores (p = 0.12). In conclusion, ciprofloxacin may be recommended for use in children for short duration when effective alternative antibacterials are unavailable. However, there is a need for further studies to evaluate the tissue accumulation of fluoride and its potential to cause toxic effects. [\hyperlink{Xofluza}{PMID: 7633153}, K M Pradhan et al., 1995]

\hypertarget{pmid_2391756}{W}e administered norfloxacin (NFLX) to 16 children aged 3 to 14 year-old at the dose of 5.2 to 17.2 mg/kg/day. We evaluated the efficacy and safety of NFLX in 6 children with respiratory tract infections, 8 urinary tract infections, and 2 gastrointestinal tract infections. Efficacy rate of NFLX was 93.8\% and eradicated rate was 92.9\%. Any adverse effects were not observed. These results suggested that NFLX could be used safely to the children. [\hyperlink{Xofluza}{PMID: 2391756}, T Ihara et al., 1990]

\hypertarget{pmid_35200181}{T}he Food and Drug Administration has granted accelerated approval to vosoritide (Voxzogo) to treat children ages five years and older with achondroplasia who still have open epiphyses.Children prescribed vosoritide should have a meal and 240 to 300 mL of fluid in the hour prior to drug administration to prevent hypotensive episodes. [\hyperlink{Xofluza}{PMID: 35200181}, Diane S Aschenbrenner et al., 2022]

\hypertarget{pmid_22814964}{W}e report 6 pediatric cases of tuberculosis caused by Mycobacterium tuberculosis and treated them with levofloxacin or moxifloxacin in the mother-child unit of a university hospital in France between 2005 and 2011. We assess the clinical efficacy and safety of fluoroquinolones and the benefit-risk ratio for their use as second-line antituberculosis drugs in children and adolescents. [\hyperlink{Xofluza}{PMID: 22814964}, Jean-Vannak Chauny et al., 2012]

\hypertarget{pmid_25495591}{A} newly developed recombinant factor IX (BAX326(1) ) was investigated for prophylactic use in paediatric patients aged <12 years with severe (FIX level <1\%) or moderately severe (FIX level 1-2\%) haemophilia B. The aim of this prospective clinical trial was to assess the safety, haemostatic efficacy and pharmacokinetic profile of BAX326 in previously treated paediatric patients. BAX326 was administered as prophylaxis twice a week for a period of 6 months, and on demand for treatment of bleeds. Safety was assessed by the occurrence of related AEs, thrombotic events and immunologic assessments. Efficacy was evaluated by annualized bleeding rate (ABR), and by treatment response rating (excellent, good, fair, none). PK was assessed over 72 h. None of the 23 treated paediatric subjects had treatment-related SAEs or AEs. There were no thrombotic events, inhibitory or specific binding antibodies against FIX, rFurin or CHO protein. Twenty-six bleeds (19 non-joint vs. 7 joint bleeds) occurred (mean ABR 2.7 ± 3.14, median 2.0), of which 23 were injury-related. Twenty subjects (87\%) did not experience any bleeds of spontaneous aetiology. Haemostatic efficacy of BAX326 was excellent or good for >96\% of bleeds (100\% of minor, 88.9\% of moderate and 100\% of major bleeds); the majority (88.5\%) resolved after 1-2 infusions. Longer T1/2 and lower IR were observed in younger children (<6 years) compared to those aged 6 to 12 years. BAX326 administered as prophylactic treatment as well as for controlling bleeds is efficacious and safe in paediatric patients aged <12 years with haemophilia B. [\hyperlink{Xofluza}{PMID: 25495591}, T Urasinski et al., 2015]

\hypertarget{pmid_3430711}{F}lomoxef (FMOX, 6315-S), a new parenteral oxacephem antibiotic, was evaluated for its safety, efficacy and pharmacokinetics in children. Twenty-six patients with bacterial infections were treated with FMOX. Clinical efficacy rate was 92\% and bacteriological cure rate was 85\%. Three cases of infections due to methicillin-resistant Staphylococcus aureus were cured with FMOX therapy. No severe adverse reactions or abnormalities of laboratory test data were associated with FMOX therapy, although loose stools and diarrhea occurred frequently (23\%). Serum half-lives of FMOX after a single bolus injection of 9 infants and children were 0.77 +/- 0.31 hour and excretion into urine was rapid. From these experiences, FMOX appeared to be a safe and effective antibiotic when used in children with susceptible bacterial infections. [\hyperlink{Xofluza}{PMID: 3430711}, H Meguro et al., 1987]

\hypertarget{pmid_19597919}{F}ollowing a previous preliminary report on a group of children suffering from partial epilepsies, we present the final considerations on the same group in order to evaluate the long-term efficacy, tolerability and safety of oxcarbazepine (OXC). We enrolled 36 patients (mean age 8.5), between January 2003 and December 2004, with new diagnosis of partial epilepsy: 25 patients were affected by idiopathic partial epilepsy, eight by symptomatic epilepsy and three by cryptogenic epilepsy. Each patient was scheduled to attend the center four times after the initial examination: 3 months (T1), 12 months (T2), 24 (T3) months and 36 (T4) months after the beginning of OXC-monotherapy (T0). At the end of our study, 20 patients were seizure free (SF): nine stopped OXC because of SF for at least 2 years, 11 were still on therapy. One patient showed a reduction of seizure frequency >or=50\%, three were non responders (but still on therapy), nine stopped OXC due to a non-responder condition during follow-up before T4 and one because of adverse effects. At the end of the study no EEG focal abnormalities became generalized because of treatment. Normalization of EEG was observed in ten patients. Our preliminary findings have been confirmed. OXC can be considered an effective and well tolerated first line drug for long-term monotherapy in children with epilepsy, both for idiopathic and symptomatic/cryptogenic forms. [\hyperlink{Xofluza}{PMID: 19597919}, Emilio Franzoni et al., 2009]

\hypertarget{pmid_28236868}{S}evoflurane is often used in pediatric anesthesia and is associated with high incidence of psychomotor agitation. In such cases, dexmedetomidine (DEX) has been used, but its benefit and implications remain uncertain. We assessed the effects of DEX on agitation in children undergoing general anesthesia with sevoflurane. Meta-analysis of randomized clinical and double-blind studies, with children undergoing elective procedures under general anesthesia with sevoflurane, using DEX or placebo. We sought articles in English in PubMed database using the following terms: Dexmedetomidine, sevoflurane (Methyl Ethers/sevoflurante), and agitation (Psychomotor Agitation). Duplicate articles with children who received premedication and used active control were excluded. It was adopted random effects model with DerSimonian-Laird testing and odds ratio (OR) calculation for dichotomous variables, and standardized mean difference for continuous variables, with their respective 95\% confidence interval (CI). Of 146 studies identified, 10 were selected totaling 558 patients (282 in DEX group and 276 controls). The use of DEX was considered a protective factor for psychomotor agitation (OR=0.17; 95\% CI 0.13-0.23; p<0.0001) and nausea and vomiting in PACU (OR=0.49; 95\% CI 0.35-0.68; p<0.0001). Wake-up time and PACU discharge time were higher in the dexmedetomidine group. There was no difference between groups for extubation time and duration of anesthesia. Dexmedetomidine reduces psychomotor agitation during wake-up time of children undergoing general anesthesia with sevoflurane. [\hyperlink{Xofluza}{PMID: 28236868}, Marco Aurélio Soares Amorim et al., ]

\hypertarget{pmid_27157201}{S}evoflurane is often used in pediatric anesthesia and is associated with high incidence of psychomotor agitation. In such cases, dexmedetomidine (DEX) has been used, but its benefit and implications remain uncertain. We assessed the effects of DEX on agitation in children undergoing general anesthesia with sevoflurane. Meta-analysis of randomized clinical and double-blind studies, with children undergoing elective procedures under general anesthesia with sevoflurane, using DEX or placebo. We sought articles in English in PubMed database using the following terms: Dexmedetomidine, sevoflurane (Methyl Ethers/sevoflurante), and agitation (Psychomotor Agitation). Duplicate articles with children who received premedication and used active control were excluded. It was adopted random effects model with DerSimonian-Laird testing and odds ratio (OR) calculation for dichotomous variables, and standardized mean difference for continuous variables, with their respective 95\% confidence interval (CI). Of 146 studies identified, 10 were selected totaling 558 patients (282 in DEX group and 276 controls). The use of DEX was considered a protective factor for psychomotor agitation (OR=0.17; 95\% CI 0.13 to 0.23; p<0.0001) and nausea and vomiting in PACU (OR=0.49; 95\% CI 0.35 to 0.68; p<0.0001). Wake-up time and PACU discharge time were higher in the dexmedetomidine group. There was no difference between groups for extubation time and duration of anesthesia. Dexmedetomidine reduces psychomotor agitation during wake-up time of children undergoing general anesthesia with sevoflurane. [\hyperlink{Xofluza}{PMID: 27157201}, Marco Aurélio Soares Amorim et al., ]

\hypertarget{pmid_3866088}{A} clinical trial of ceftizoxime suppositories (CZX-S) was performed to evaluate the therapeutic effectiveness in children with bacterial infection. The subjects were 10 children comprising 4 with pneumonia, 3 with lacunar tonsillitis, 2 with pharyngitis, and 1 with UTI. They were given 1 suppository containing either 125 mg or 250 mg of CZX 2 to 4 times a day. The daily per kg body weight dose ranged from 17.1 to 60.0 mg. The result was "markedly effective" in 3, "effective" in 6, and "failure" was recorded in 1. Bacteriologically, successful eradication of causative organisms was confirmed in all the 4 children who underwent the test. No clinical side effects were observed. The only laboratory test abnormality recorded in a single patient was eosinophilia, which was not definitely ascribable to CZX-S. In conclusion, CZX-S have proved to be a clinically safe and effective antibiotic preparation in infantile infection, even in children whose treatment with conventional antibiotics is associated with difficulties. [\hyperlink{Xofluza}{PMID: 3866088}, T Hosoda et al., 1985]

\hypertarget{pmid_23700934}{T}wo hundred fifty patients, including 100 children with frequent and prolonged diseases at the age of 4 to 7 years, 76 children at the age of 7 to 18 years and 74 subjects at the age of 22 to 57 years were observed. The patients were treated with cycloferon in two courses with a 2-week interval according to the standard scheme. The tonsil surface microflora and its susceptibility to antibiotics were determined. Cycloferon lowered the Staphylococcus aureus titre and increased the culture susceptibility to benzylpenicillin, oxacillin, rifampicin, and erythromycin, reducing the variety of the fauces nonpathogenic microflora. The use of cycloferon induced no adverse (pathologic) reactions in 94.8\% of the cases. In 4.4\% of the children under school age the adverse reactions were transitory and did not require discontinuation of the drug use. Unforeseen reactions were recorded in 0.8\% of the children and the use of the drug in them was discontinued. The use of cycloferon in two courses with a 2-week interval according to the standard scheme is recommended for prophylaxis of acute respiratory diseases in the group of children with frequent and prolonged diseases during epidemiologically unfavourable periods and for complex therapy of rhinopharinx infections as an agent increasing efficacy of other antibacterials. [\hyperlink{Xofluza}{PMID: 23700934}, S A Lialikov et al., 2012]

\hypertarget{pmid_11847958}{I}nformation regarding the treatment of anthrax infection is scarce in adults and is even more limited in children. Children, however, may be at a greater risk for developing an infection and systemic disease if exposed to anthrax than adults. The Centers for Disease Control and Prevention (CDC) recommends the use of doxycycline or ciprofloxacin for prophylaxis and treatment in children. Doxycycline currently is not indicated for use in children < 8 years old, due to staining of teeth and inhibition of bone growth associated with tetracyclines. Doxycycline, however, may have less adverse effect on teeth than its precursors. Ciprofloxacin has a pediatric indication only when a child is potentially exposed to inhaled anthrax. Ciprofloxacin is contraindicated in pediatric patients because fluoroquinolones were shown to cause cartilage toxicity in immature animals. Although children of various ages have received ciprofloxacin, there are few reports of cartilage toxicity. Because anthrax is a potentially fatal infection, the benefits to using these antibiotics greatly outweigh the risks. Therefore, the use of these antibiotics in children can be recommended, despite the lack of adequate efficacy and safety studies in pediatric patients with anthrax. [\hyperlink{Xofluza}{PMID: 11847958}, Sandra Benavides et al., 2002]

\hypertarget{pmid_7289018}{C}efoxitin (CFX) was evaluated for its safety and efficacy in children. Fifteen patients were treated with 73-125 mg/kg per day of CFX by intravenous administrations. The diagnosis of the patients were acute pharyngitis (4), pneumonia (2), pertussis and pneumonia (1), urinary tract infection (3); and the remaining 5 patients were esteemed to have nonbacterial infections. All the 10 patients of bacterial infections were cured after the CFX therapy. The pathogens recovered were Streptococcus pyogenes (1), Streptococcus pneumoniae (3), Haemophilus influenzae (2), Escherichia coli (2), enteropathogenic Escherichia coli (1), and Klebsiella pneumoniae (1). All the strains isolated were susceptible to CFX, but the 2 isolates of Haemophilus influenzae had relatively high MIC values (12.5 mcg/ml). Diarrhea (3 cases) and transient neutropenia (1 case) were found to be associated with the CFX therapy. However, no severe adverse reactions were encountered. Half-life of the serum level was short (24.1 minutes) and excretion into the urine was rapid. CSF concentration obtained 30 minutes after an intravenous injection of 50 mg/kg of CFX in 1 case with inflamed meninges was considerably high (8.3 mcg/ml). CFX appears to be a safe and effective antibiotic when used in children with susceptible bacterial infections. [\hyperlink{Xofluza}{PMID: 7289018}, H Meguro et al., 1981]

\hypertarget{pmid_36174614}{S}urvivors of childhood cancer are at risk of anthracycline-induced cardiotoxicity, which might be prevented by dexrazoxane. However, concerns exist about the safety of dexrazoxane, and little guidance is available on its use in children. To facilitate global consensus, a working group within the International Late Effects of Childhood Cancer Guideline Harmonization Group reviewed the existing literature and used evidence-based methodology to develop a guideline for dexrazoxane administration in children with cancer who are expected to receive anthracyclines. Recommendations were made in consideration of evidence supporting the balance of potential benefits and harms, and clinical judgement by the expert panel. Given the dose-dependent risk of anthracycline-induced cardiotoxicity, we concluded that the benefits of dexrazoxane probably outweigh the risk of subsequent neoplasms when the cumulative doxorubicin or equivalent dose is at least 250 mg/m [\hyperlink{Xofluza}{PMID: 36174614}, Esmée C de Baat et al., 2022]

\section*{Cabtreo}
\subsection*{Result}
\subsubsection*{Answer}

Unknown

\subsubsection*{{Explanation}}
\hypertarget{Cabtreo}
After reviewing all available abstracts, there is no mention of Cabtreo in any of the abstracts provided. Therefore, there is no evidence from targeted studies regarding the safety or lack of safety of Cabtreo in children of any age range. As a result, the safety of Cabtreo for use in children is unknown based on the abstracts available.

\subsection*{Abstracts}
\hypertarget{pmid_12068839}{(}1) The choice of treatment for children with obsessive-compulsive disorder is difficult. Behaviour therapy and antidepressants have not been assessed adequately in this setting, and their efficacy seems limited. Clomipramine was the first antidepressant to show a degree of efficacy. (2) Sertraline is the first drug to be licensed in France for children aged from 6 to 17 years with obsessive-compulsive disorder. (3) According to our literature search, the evaluation file on sertraline in this indication mainly contains data from a double-blind placebo-controlled trial involving 187 children. After 3 months of treatment, sertraline was significantly more effective than placebo, although most children remained symptomatic. Direct comparison is lacking, but sertraline seems as effective as clomipramine. (4) However, 13\% of children receiving sertraline left this trial because of adverse events (3\% on placebo; p = 0.02). The short-term safety profile of sertraline in children is the same as in adults, i.e. mainly nausea, agitation, headache, insomnia and tremor. (5) We have no data on the effects of prolonged sertraline therapy in children, particularly on neuropsychological development. (6) The first-line treatment of obsessive-compulsive disorder is behaviour therapy. Sertraline, like clomipramine, is an option when behaviour therapy fails or is unfeasible. The choice between sertraline and clomipramine should be discussed case by case, according to their safety profiles; however, we have more experience with clomipramine, which should therefore be preferred over sertraline. [\hyperlink{Cabtreo}{PMID: 12068839}, Sertraline: new indication. May help children with obsessive-compulsive disorder., 2002]

\hypertarget{pmid_28827252}{C}eftriaxone is widely used in children in the treatment of sepsis. However, concerns have been raised about the safety of ceftriaxone, especially in young children. The aim of this review is to systematically evaluate the safety of ceftriaxone in children of all age groups. MEDLINE, PubMed, Cochrane Central Register of Controlled Trials, EMBASE, CINAHL, International Pharmaceutical Abstracts and adverse drug reaction (ADR) monitoring systems will be systematically searched for randomised controlled trials (RCTs), cohort studies, case-control studies, cross-sectional studies, case series and case reports evaluating the safety of ceftriaxone in children. The Cochrane risk of bias tool, Newcastle-Ottawa and quality assessment tools developed by the National Institutes of Health will be used for quality assessment. Meta-analysis of the incidence of ADRs from RCTs and prospective studies will be done. Subgroup analyses will be performed for age and dosage regimen. Formal ethical approval is not required as no primary data are collected. This systematic review will be disseminated through a peer-reviewed publication and at conference meetings. CRD42017055428. [\hyperlink{Cabtreo}{PMID: 28827252}, Linan Zeng et al., 2017]

\hypertarget{pmid_23515245}{T}o describe the rationale, design and first data from PATRO Children, a postmarketing surveillance of the long-term efficacy and safety of somatropin (Omnitrope(®)) for the treatment of children requiring growth hormone treatment. PATRO Children is a multicentre, open, longitudinal, noninterventional study being conducted in children's hospitals and specialised endocrinology clinics. The primary objective is to assess the long-term safety of Omnitrope(®) in routine clinical practice. Eligible patients are infants, children and adolescents (male or female) who are receiving treatment with Omnitrope(®) and who have provided informed consent. Patients who have been treated with another recombinant human growth hormone (rhGH) product before starting Omnitrope(®) are eligible for inclusion. All adverse events (AEs) are monitored and recorded, with particular emphasis on: long-term safety; the recording of malignancies; the occurrence and clinical impact of anti-hGH antibodies; the development of diabetes during Omnitrope(®) treatment in children short for gestational age (SGA); safety issues in patients with Prader-Willi syndrome (PWS). Efficacy assessments include auxological parameters, plus insulin-like growth factor-1 and insulin-like growth factor binding protein-3. As of September 2012, 1837 patients were enrolled in the study from 184 sites in 10 European countries. To date, efficacy data are reassuring and consistent with previous studies. In addition, there have been no confirmed cases of diabetes occurring under Omnitrope(®) treatment, no reports of malignancy and no safety issues in PWS patients. The efficacy and safety profile of Omnitrope(®) in the PATRO Children study so far are as expected. The ongoing study will extend the safety database for Omnitrope(®), and rhGH products more generally, in paediatric indications. Of particular interest, PATRO Children will add important information on the diabetogenic potential of rhGH in children born SGA, the risk of malignancies in children receiving rhGH, and AEs with a possible causal relationship to rhGH treatment in children with PWS. [\hyperlink{Cabtreo}{PMID: 23515245}, Roland Pfäffle et al., 2013]

\hypertarget{pmid_20819318}{A}llergic rhinitis (AR) and chronic idiopathic urticaria (CIU) are common causes of substantial illness and disability in preschool children. Antihistamines are commonly used to treat preschool children with these conditions, but their use is based mostly on extrapolated efficacy from adult populations; it is thus important to characterize the safety of antihistamines in the pediatric population. This study was designed to assess the safety of levocetirizine dihydrochloride oral liquid drops in infants and children with AR or CIU. Two multicenter, double-blind, randomized, parallel-group studies randomized infants aged 6-11 months (study 1, n = 69) and children aged 1-5 years (study 2, n = 173) to levocetirizine, 1.25 mg (q.d. or b.i.d., respectively), or placebo for 2 weeks, using a 2:1 ratio. Safety evaluations included treatment-emergent adverse events (TEAEs), vital signs, electrocardiographic (ECG) assessments, and laboratory tests. The overall incidence of TEAEs was similar between levocetirizine and placebo in both studies. Most TEAEs were mild or moderate in intensity. TEAEs prompted discontinuation of therapy in three patients receiving levocetirizine in study 1. No clinically relevant changes from baseline in vital signs or laboratory parameters were apparent in either study; changes from baseline in these evaluations were similar between groups. No significant changes were observed in ECG parameters, including corrected QT interval. Levocetirizine, 1.25 and 2.5 mg/day, was well tolerated in infants aged 6-11 months and in children aged 1-5 years, respectively, with AR or CIU. [\hyperlink{Cabtreo}{PMID: 20819318}, Frank Hampel et al., ]

\hypertarget{pmid_16028153}{B}ecause of concerns about arthrotoxicity, fluoroquinolones are restricted for use in children. This study describes the safety and efficacy of gatifloxacin when used for treatment of children with recurrent acute otitis media (ROM) or acute otitis media (AOM) treatment failure (AOMTF). We performed an analysis of 867 children included in 4 clinical trials who had ROM and/or AOMTF and were treated with gatifloxacin (10 mg/kg once daily for 10 days). Gatifloxacin had adverse event rates that were similar overall to those of a comparator antibiotic (amoxicillin-clavulanate), except for increased diarrhea in children <2 years old receiving amoxicillin-clavulanate. There was no evidence of arthrotoxicity, hepatotoxicity, alteration of glucose homeostasis, or central nervous system toxicity acutely or during 1 year follow-up in any child. Regarding efficacy, in 2 noncomparative trials, the gatifloxacin cure rate of AOM was 89\% (95\% confidence interval [CI], 83\%-95\%) at the test of cure (TOC) visit, 3-10 days after completion of therapy. In 2 comparative trials of gatifloxacin versus amoxicillin-clavulanate, the efficacy of gatifloxacin was 88\% (95\% CI, 82\%-94\%). Gatifloxacin led to better clinical outcomes than amoxicillin-clavulanate for AOMTF (91\% vs. 81\%; P=.029), for AOMTF and age <2 years old (89\% vs. 69\%; P=.009), and for severe AOM in children <2 years old (90\% vs. 75\%; P=.012). Among children with AOMTF previously treated with amoxicillin-clavulanate or ceftriaxone injections, gatifloxacin cure rates were high (88\% and 75\%, respectively). Gatifloxacin appears to be safe for children, with no evidence of producing arthrotoxicity in 867 children exposed to the antibiotic when used as treatment for ROM and AOMTF. [\hyperlink{Cabtreo}{PMID: 16028153}, Michael E Pichichero et al., 2005]

\hypertarget{pmid_17561929}{T}here are more than 40 H(1)-antihistamines available worldwide. Most of these medications have never been optimally studied in prospective, randomized, double-masked, placebo-controlled trials in children. The aim was to perform a long-term study of levocetirizine safety in young atopic children. In the randomized, double-masked Early Prevention of Asthma in Atopic Children Study, 510 atopic children who were age 12-24 months at entry received either levocetirizine 0.125 mg/kg or placebo twice daily for 18 months. Safety was assessed by: reporting of adverse events, numbers of children discontinuing the study because of adverse events, height and body mass measurements, assessment of developmental milestones, and hematology and biochemistry tests. The population evaluated for safety consisted of 255 children given levocetirizine and 255 children given placebo. The treatment groups were similar demographically, and with regard to number of children with: one or more adverse events (levocetirizine, 96.9\%; placebo, 95.7\%); serious adverse events (levocetirizine, 12.2\%; placebo, 14.5\%); medication-attributed adverse events (levocetirizine, 5.1\%; placebo, 6.3\%); and adverse events that led to permanent discontinuation of study medication (levocetirizine, 2.0\%; placebo, 1.2\%). The most frequent adverse events related to: upper respiratory tract infections, transient gastroenteritis symptoms, or exacerbations of allergic diseases. There were no significant differences between the treatment groups in height, mass, attainment of developmental milestones, and hematology and biochemistry tests. The long-term safety of levocetirizine has been confirmed in young atopic children. [\hyperlink{Cabtreo}{PMID: 17561929}, F Estelle R Simons et al., 2007]

\hypertarget{pmid_15073885}{T}his study determined the licensed and 'off label' (outside the terms of the licence) use of newly marketed medicines in children (2-11 years) and adolescents (12-17 years), by general practitioners in England. In addition, the incidence rates during the first month of therapy (ID(1)) for three adverse events, in these groups were compared with those of adults (> or =18 years). The use of these drugs was monitored in 63 individual prescription-event monitoring (PEM) studies, conducted to monitor the safety of these medicines. Patients and drug exposures were identified from dispensed prescriptions. Outcome data (events and demographic information) were obtained from questionnaires. Although only six of these 63 drugs were licensed for use in children, 44 of the 63 drugs were used to treat children. For the majority of the drugs there was no specific reference to adolescents in the data sheets therefore it has been assumed that the drugs were licensed for those aged > or =12 years unless specified otherwise; 55 have been taken as licensed for use in adolescents. Over 690,000 patients were included in the 63 PEM studies, 9081 (1.3\%) of these were children and 15,256 (2.2\%) were adolescents. 78\% of the 9081 children and 93\% of the 15,256 adolescents were treated with 'licensed' drugs. There was a significant difference in the incidence rate for rash and nausea/vomiting, two adverse events commonly reported during treatment with lamotrigine, between children and adolescents compared to adults. This survey has shown that although only a small proportion (10\%) of newly marketed drugs were licensed for use in children the majority of children (78\%) were treated with these licensed products but 22\% of children received drugs 'off label' during the first few years that the drug was marketed and a small number of children and adolescents were given drugs contraindicated in these age ranges. [\hyperlink{Cabtreo}{PMID: 15073885}, L V Wilton et al., 1999]

\hypertarget{pmid_23187987}{K}etamine is used intramuscularly or intravenously as a sedative when repairing the skin lacerations of children in many emergency departments (EDs). Nitrous oxide (N(2)O) has the advantages of being a sedative agent that does not require a painful injection and that offers shallower levels of sedation and a rapid recovery of mental state. We evaluated the clinical usefulness of N(2)O compared with intravenous ketamine when used for the repair of lacerations in children in the ED. From January to December 2009, we performed a prospective, randomized study at a single academic ED enrolling pediatric patients aged 3 to 10 years who needed primary repair of a laceration wound. The primary outcome was recovery time, which was defined as the time from completion of procedure to recovery of mental state. Other outcomes were sedation depth, pain scale, adverse effects, and satisfaction with sedation. There were 32 children who were randomly assigned. Recovery times were shorter in the N(2)O group compared with those in the ketamine group (median [interquartile range (IQR)], 0.0 minutes, [0.0-4.0 minutes] vs 21.5 minutes [12.5-37.5 minutes], P < 0.05). Sedation levels were deeper in the ketamine group than in the N2O group, but pain scales were comparable between groups. No difference was observed in the satisfaction scores by physicians, parents, or nurses. Nitrous oxide inhalation was preferable to injectable ketamine for pediatric patients because it is safe, allows for a faster recovery, maintains sufficient sedation time, and does not induce unnecessarily deep sedation. [\hyperlink{Cabtreo}{PMID: 23187987}, Jin Hee Lee et al., 2012]

\hypertarget{pmid_34350858}{P}ATRO Children is an international, observational, postmarketing surveillance study for a biosimilar recombinant human growth hormone (rhGH; somatropin, Omnitrope®; Sandoz), approved by the European Medicines Agency in 2006. We report safety and effectiveness data for patients with Turner syndrome (TS). The study population included infants, children, and adolescents with TS who received Omnitrope® treatment according to standard clinical practice. Adverse events (AEs) were monitored for safety evaluation, and height velocity (HV), height standard deviation score (HSDS), and HVSDS were calculated to evaluate treatment effectiveness. As of August 2019, 348 TS patients were enrolled from 130 centers. At baseline, 314 patients (90.2\%) were prepubertal and 284 patients (81.6\%) were rhGH treatment naïve. The mean (range) age at baseline was 9.0 (0.7-18.5) years, and mean (SD) treatment duration in the study was 38.5 (26.8) months. Overall, 170 patients (48.9\%) reported AEs, which were considered treatment related in 25 patients (7.2\%). One treatment-related serious AE was reported (intracranial hypertension). Mean ΔHSDS after 3 years of therapy was +1.17 in treatment-naïve prepubertal patients and +0.1 in pretreated prepubertal patients. In total, 51 patients (31.1\%) reached adult height (AH), 35 of whom were rhGH treatment naïve; in these patients, mean (SD) HSDS was -2.97 (1.03) at the start of Omnitrope® treatment, and they achieved a mean (SD) AHSDS of -2.02 (0.9). These data suggest that biosimilar rhGH is well tolerated and effective in TS patients managed in real-life clinical practice. Optimization of rhGH dose may contribute to a higher AH. [\hyperlink{Cabtreo}{PMID: 34350858}, Philippe Backeljauw et al., 2021]

\hypertarget{pmid_10563619}{T}o compare the safety and efficacy of add-on lamotrigine and placebo in the treatment of children and adolescents with partial seizures. Add-on and monotherapy lamotrigine is safe and effective in adults with partial seizures, and reports of preliminary uncontrolled trials suggest similar benefits in children. We studied 201 children with diagnoses of partial seizures of any subtype currently receiving stable conventional regimens of antiepileptic therapy at 40 study sites in the United States and France. After a baseline observation period (to confirm that more than four seizures occurred in each of two consecutive 4-week periods), patients were randomized to add-on lamotrigine or placebo therapy. A 6-week dose-escalation period was followed by a 12-week maintenance period. Compared with placebo, lamotrigine significantly reduced the frequency of all partial seizures and the frequency of secondarily generalized partial seizures in these treatment-resistant patients. The most commonly reported adverse events in the lamotrigine-treated patients were vomiting, somnolence, and infection; the frequency of these and other adverse events was similar to that in the placebo-treated group, with the exception of ataxia, dizziness, tremor, and nausea, which were more frequent in the lamotrigine-treated group. The frequency of withdrawals for adverse events was similar between groups. Two patients were hospitalized for skin rash, which resolved after discontinuation of lamotrigine therapy. Lamotrigine was effective for the adjunctive treatment of partial seizures in children and demonstrated an acceptable safety profile. [\hyperlink{Cabtreo}{PMID: 10563619}, M Duchowny et al., 1999]

\hypertarget{pmid_11673582}{I}nfantile spasms are a rare but devastating pediatric epilepsy that, outside the United States, is often treated with vigabatrin. The authors evaluated the efficacy and safety of vigabatrin in children with recent-onset infantile spasms. This 2-week, randomized, single-masked, multicenter study with a 3- year, open-label, dose-ranging follow-up study included patients who were younger than 2 years of age, had a diagnosed duration of infantile spasms of no more than 3 months, and had not previously been treated with adrenocorticotropic hormone, prednisone, or valproic acid. Patients were randomly assigned to receive low-dose (18-36 mg/kg/day) or high-dose (100-148 mg/kg/day) vigabatrin. Treatment responders were those who were free of infantile spasm for 7 consecutive days beginning within the first 14 days of vigabatrin therapy. Time to response to therapy was evaluated during the first 3 months, and safety was evaluated for the entire study period. Overall, 32 of 142 patients who were able to be evaluated for efficacy were treatment responders (8/75 receiving low-dose vigabatrin vs 24/67 receiving high doses, p < 0.001). Response increased dramatically after approximately 2 weeks of vigabatrin therapy and continued to increase over the 3-month follow-up period. Time to response was shorter in those receiving high-dose versus low-dose vigabatrin (p = 0.04) and in those with tuberous sclerosis versus other etiologies (p < 0.001). Vigabatrin was well tolerated and safe; only nine patients discontinued therapy because of adverse events. These results confirm previous reports of the efficacy and safety of vigabatrin in patients with infantile spasms, particularly among those with spasms secondary to tuberous sclerosis. [\hyperlink{Cabtreo}{PMID: 11673582}, R D Elterman et al., 2001]

\hypertarget{pmid_17611334}{T}o observe the effect of sevoflurane on the induction and maintenance of anaesthesia in children, and to evaluate its safety and effectiveness. Forty child patients who conformed to the selection standard were operated under anaesthesia with intubation.Without premedicant, all the patients inhaled 100\% oxygen(1L/min) and sevoflurane by mask, and escalated the concentration of sevoflurane (to the maximum concentration 7\%) until the lash reflex disappeared, and the maintenance concentration was controlled under 4\%. All the patients were intubated, together with vecuronium 0.1mg/kg. With little tract excretion, the achievement ratio of induction by sevoflurane was 100\%, and the children tolerated well. With stable hemodynajmics,1\% approximately 4.0\% maintenance concentration of sevoflurane during the operation showed effective anaesthesia, no decreased heart rate or blood pressure appeared, and all the patients' body temperature was normal. Sevoflurane for children induction can bring fewer stimuli in the respiratory tract,less cardiac vascular inhibition and palinesthesia time. Anaesthesia in children induced by sevoflurane is safe and effective. [\hyperlink{Cabtreo}{PMID: 17611334}, Xi-ying Zhang et al., 2007]

\hypertarget{pmid_23127263}{T}he correlation between lamotrigine serum concentration, efficacy, and toxicity in children is controversial. The database of the Clinical Pharmacology Laboratory at Assaf Harofeh Medical Center was retrospectively searched to identify lamotrigine serum concentrations in children aged 2-19 years with refractory epilepsy who received lamotrigine as monotherapy or polytherapy from 2007-2010. Data collected included age at epilepsy onset, additional antiepileptic drugs, lamotrigine dose, monthly seizure frequency before and after lamotrigine treatment, and side effects. Sixty blood samples were collected from 42 children aged 10.1 ± 4.9 years (range, 2-20 years). Seizure types included complex partial (n = 28), simple partial (n = 7), absence (n = 2), and generalized tonic-clonic (n = 23). Decreased seizure frequency was observed in 38 (63.3\%) patients. No correlation with lamotrigine serum concentration was evident, but seizure frequency was significantly influenced by age and lamotrigine dose. Side effects were reported in 21 (35\%) patients. Only diplopia was significantly correlated with lamotrigine serum concentration. Lamotrigine was more effective at lower doses and in older children. Lamotrigine serum concentration correlated significantly with diplopia, but not with other side effects or with clinical efficacy. Overall, lamotrigine is effective and safe in children with refractory epilepsy. [\hyperlink{Cabtreo}{PMID: 23127263}, Eli Heyman et al., 2012]

\hypertarget{pmid_6098721}{C}eftriaxone (Ro 13-9904, CTRX), developed by F. Hoffmann-La Roche Ltd. in Switzerland, was used for the pediatric infections and the following results were obtained. The mean blood level of CTRX in 2 children after a 60-minute intravenous drip infusion with 20 mg/kg was 58.6 micrograms/ml at 30 minutes, 75.0 micrograms/ml at 1 hour, 39.85 micrograms/ml at 2 hours, 27.74 micrograms/ml at 4 hours, 20.71 micrograms/ml at 6 hours, 11.72 micrograms/ml at 12 hours and 3.91 micrograms/ml at 24 hours while the half-life time was 5.9 hours in one child and 7.6 hours in the other. CTRX was used in 22 children with acute infections consisting of 3 with acute pharyngeal tonsillitis, 4 with acute bronchitis, 8 with bronchopneumonia, 6 with infections of skin soft tissue and 1 with salmonellosis. The results were excellent in 5 cases and good in 17, indicating an efficacy rate of 100\%. Out of 10 cases where the causative strains were detected, 4 cases were followed about the activities of the respective bacteria, i.e., H. influenzae, Streptococcus group A, S. aureus and Salmonella group B, all of which were eradicated after the end of administration. The daily dose of CTRX ranged from 30 to 50 mg/kg and generally a larger dose was used for serious infections. CTRX was administered twice daily in 20 out of 22 cases, by an intravenous injection in 4 and an intravenous drip infusion in 18, for 2 to 4 days in 16 and 5 to 8 1/2 days in 6. No clinical adverse reactions were observed while the laboratory test found a slight elevation of GOT in one and that of GOT and GPT in another. From the above results, CTRX was judged to be a highly useful drug for treatment of pediatric infections. [\hyperlink{Cabtreo}{PMID: 6098721}, M Minamitani et al., 1984]

\hypertarget{pmid_6098704}{F}undamental and clinical evaluation of ceftriaxone (CTRX) was performed in the pediatric field and the following results were obtained. The MIC of CTRX against E. coli isolated from urinary tract infections in children ranged from less than or equal to 0.024 to 0.39 mcg/ml except for 1 strain. CTRX was superior to other 3rd generation cephalosporins such as CPZ and LMOX, showing effectiveness also against ABPC-resistant bacteria. The clinical efficacy and bacteriological efficacy in 6 children consisting of 5 with respiratory tract infections and 1 with urinary tract infection were 83\% and 100\%, respectively. As to the adverse reaction, diarrhea was observed in 2 cases. The determination of PIVKA-II performed during the therapy with CTRX, which is observed when vitamin K is deficient, showed positiveness in 2 cases out of 6 cases including 1 which the clinical efficacy could not be evaluated. The test of platelet function in 3 cases found no inhibition of agglutination. Twice-daily administration with 20 mg/kg CTRX was considered to be a useful and safe method for treatment of bacterial infections in children, although attention should be taken not to cause vitamin K deficiency as in other 2nd and 3rd generation cephalosporins. [\hyperlink{Cabtreo}{PMID: 6098704}, K Sunakawa et al., 1984]

\hypertarget{pmid_10452767}{I}n very young children, H(1 )-receptor antagonists have not been adequately studied, although they are widely used and assumed to be safe. Our objective was to test the hypothesis that cetirizine would be as safe as placebo for long-term use in this population. In the prospective, double-blind, parallel-group, 18-month-long Early Treatment of the Atopic Child (ETAC) study, 817 children with atopic dermatitis who were 12 to 24 months old at study entry were randomized to receive either cetirizine 0.25 mg/kg or placebo twice daily. Safety was assessed by using the reports of all adverse events, diary cards, physical examinations, developmental assessments, electrocardiograms, blood hematology and chemistry tests, and urinalyses. The population evaluated for safety consisted of 399 children receiving cetirizine and 396 children receiving placebo. Drop-outs and serious events, including hospitalizations, occurred infrequently and were less common in the children receiving cetirizine than in those children receiving placebo, although the differences were not statistically significant. Most reported symptoms and events were mild and were attributed to intercurrent respiratory or gastrointestinal infections, exacerbations of allergic disorders, or age-related concerns rather than to medication-related adverse effects. There were no clinically relevant differences between the groups for neurologic or cardiovascular symptoms or events, growth, behavioral or developmental assessments, laboratory test results, or electrocardiograms, and no child receiving cetirizine therapy had prolongation of the QTc interval. The safety of cetirizine has been confirmed in this prospective study, the largest and longest randomized, double-blind, placebo-controlled safety investigation of any H(1 )-antagonist ever conducted in children and the longest prospective safety study of any H(1 )-antagonist ever conducted in any age group. [\hyperlink{Cabtreo}{PMID: 10452767}, F E Simons et al., 1999]

\hypertarget{pmid_3889818}{T}he safety and efficacy of captopril therapy in children with severe and refractory hypertension has been evaluated in a collaborative international study which enrolled a group of 73 patients, 15 years of age or younger. Most patients had hypertension associated with renal disease or vascular abnormalities. Captopril was administered for periods of less than 3 months to more than 1 year. A significant decrease in both systolic and diastolic blood pressures was produced by the administration of captopril, usually in conjunction with other antihypertensive agents (most commonly diuretics and/or beta-blockers). Systolic blood pressures were normalized in 62\% and 53\% and diastolic blood pressures in 56\% and 45\% of reported patients after the second and sixth months of captopril therapy, respectively. The response to captopril was sustained over a 12-month period. Adverse reactions were reported in 49\% of the 73 patients; 48\% of patients had experienced adverse reactions to other antihypertensive agents prior to entering the study. The reactions most frequently observed during captopril therapy were hypotension, vomiting, postural symptoms, anemia, rash, and anorexia. Leukopenia was reported in six patients, all of whom had renal impairment. Two of these patients had received concomitant therapy with immunosuppressants, and one had systemic lupus erythematosus. Captopril was discontinued in two of these six children. Statistically significant increases in mean serum urea nitrogen and potassium concentrations and decreases in mean serum CO2 levels were observed during the course of therapy. These effects could not be exclusively attributed to captopril administration as the study population received multidrug therapy and had significant intrinsic disease. Captopril was demonstrated to be an effective and safe drug for the treatment of children with severe hypertension. [\hyperlink{Cabtreo}{PMID: 3889818}, B L Mirkin et al., 1985]

\hypertarget{pmid_3897610}{I}n a total of 13 children with infections, ranging in age from 1 month to 6 years, cefminox (CMNX, MT-141), a new antibiotic of cephem group, was administered 14 times and its absorption, excretion, clinical results and safety were determined. Following intravenous drip infusion of CMNX, high blood level was achieved, with half-life of about 1 hour (0.77 to 1.13 hours). The urinary recovery rate was approximately 80\% within the first 6 hours after completion of administration. Clinical results were rated as effective in 8 out of 12 assessable cases (66.7\%). In any of the cases treated no side effects developed nor any abnormal changes in laboratory finding were observed. From these results, CMNX is considered to be a new antibiotic useful and safe for use in the field of pediatrics. [\hyperlink{Cabtreo}{PMID: 3897610}, K Tomimasu et al., 1985]

\hypertarget{pmid_16355266}{T}o review the literature on acute toxic exposure in children, excluding envenomations. MEDLINE review (emphasis on the past decade), including the American Academy of Clinical Toxicology and the European Association of Poison Centres and Clinical Toxicologists position statements and position papers (peer-reviewed information based on scientific evidence and broad consensus) on gastrointestinal decontamination, multiple-dose activated charcoal and urine alkalinization. Acute toxic exposure in children is a common event, mainly in children under six years of age. Death is rare. Although widely employed, there is no evidence that gastrointestinal decontamination and multiple-dose activated charcoal improve the outcome of poisoned patients. Very few efficient antidotes are used on a consistent basis, and some of them are very expensive and not available in Brazil. Ipecac syrup and cathartics should not be administered on a routine basis in acute toxic exposures in outpatient treatment. Excluding the contraindications, single-dose activated charcoal and gastric lavage may be considered within one hour of ingestion if a patient ingested a potentially toxic amount or a potentially lethal amount, respectively. Whole bowel irrigation, multiple-dose activated charcoal and urine alkalinization may be considered in a few situations. Fomepizole and octreotide are safe and efficient antidotes, which can be used in the treatment of alcohol (methanol and ethylene glycol) and sulfonylureas poisoning, respectively. [\hyperlink{Cabtreo}{PMID: 16355266}, Fábio Bucaretchi et al., 2005]

\hypertarget{pmid_24279906}{T}he prevalence of hypertension among children has been increasing. Community and Hospital Pharmacists are often challenged to provide an oral liquid extemporaneous formulation for pediatric patients, because there are no appropriate dosage drugs to the specific needs of the child. The objective of this study is to choose and develop suitable pediatric extemporaneous formulations for captopril and enalapril maleate and to determine their physicochemical stability. A survey was carried out to evaluate the extent of dispensation of these drugs in Hospitals in Spain. Stability studies of formulations have been studied according to ICH normative at 5, 25 and 40 °C. Three samples from each temperature were withdrawn and assessed for stability on days 0, 15, 30, 50 and 90 using a high-performance liquid chromatography (HPLC) mass spectrometer assay. Rheological studies were carried out to ensure the maintenance of the physical characteristics of these non-Newtonian fluids. Captopril and enalapril maleate formulations used the pure drug and were stable during 50 days at 5 °C. We have developed easy antihypertensive oral liquid extemporaneous formulations for pediatric patients with physical and chemical stability higher than those provided by the majority of Hospitals.  [\hyperlink{Cabtreo}{PMID: 24279906}, Marta Casas et al., 2015] An increase in the incidence of Salmonella typhi strains resistant to chloramphenicol, ampicillin and trimethoprim-sulfamethoxazole causing enteric fever in Egyptian children stimulated the evaluation of alternative drugs. Children with positive blood cultures were treated with cefixime, ceftriaxone or aztreonam, and the efficacy, safety and cost of these regimens were evaluated and compared. Cefixime (7.5 mg/kg) was given orally twice daily to 50 children for 14 days, ceftriaxone (50 to 70 mg/kg) was given im once daily for 5 days to 43 children and aztreonam (50 to 70 mg/kg) was given im every 8 hours for 7 days to 31 children. Children in the 3 groups were comparable with regard to age, sex, duration and severity of illness before admission. All children were cured. A significant difference (P < 0.05) in duration of treatment before becoming afebrile seemed to favor ceftriaxone (3.9 days) over aztreonam (5.5 days) and cefixime (5.3 days). During the 4-week follow-up period relapses occurred in 3 (6\%) children in the cefixime group, in 2 (5\%) in the ceftriaxone group and in 2 (6\%) in the aztreonam group. Safety and efficacy were comparable for all 3 drugs. Ceftriaxone was most cost-effective on an inpatient basis, because of a more rapid clinical cure, and cefixime was the most cost-effective on an outpatient basis, because of drug cost. [\hyperlink{Cabtreo}{PMID: 24279906}, N I Girgis et al., 1995]

\hypertarget{pmid_32986159}{T}iapride is commonly used in Europe for the treatment of tics. The aim of this study was to examine the relationship between dose and serum concentrations of tiapride and potential influential pharmacokinetic factors in children and adolescents. In addition, a preliminary therapeutic reference range for children and adolescents with tics treated with tiapride was calculated. Children and adolescents treated with tiapride at three university hospitals and two departments of child and adolescents psychiatry in Germany and Austria were included in the study. Patient characteristics, doses, serum concentrations, and therapeutic outcome were assessed during clinical routine care using standardised measures. In the 49 paediatric patients (83.7\% male, mean age = 12.5 years), a positive correlation was found between tiapride dose (median 6.9 mg/kg, range 0.97-19.35) and serum concentration with marked inter-individual variability. The variation in dose explained 57\% of the inter-patient variability in tiapride serum concentrations; age, gender, and concomitant medication did not contribute to the variability. The symptoms improved in 83.3\% of the patients. 27.1\% of the patients had mild or moderate ADRs. No patient suffered from severe ADRs. This study shows that tiapride treatment was effective and safe in most patients with tics. Compared with the therapeutic concentration range established for adults with Chorea Huntington, our data hinted at a lower lower limit (560 ng/ml) and similar upper limit (2000 ng/ml). [\hyperlink{Cabtreo}{PMID: 32986159}, Stefanie Fekete et al., 2021]

\hypertarget{pmid_22040192}{T}his study evaluated the potential benefits of a centrally acting selective serotonin reuptake inhibitor, sertraline, versus placebo for prevention of symptoms of posttraumatic stress disorder (PTSD) and depression in burned children. This is the first controlled investigation based on our review of the early use of a medication to prevent PTSD in children. Twenty-six children aged 6-20 were assessed in a 24-week double-blind placebo-controlled design. Each child received either flexibly dosed sertraline between 25-150 mg/day or placebo. At each reassessment, information was collected in compliance with the study medication, parental assessment of the child's symptomatology and functioning, and the child's self-report of symptomatology. The protocol was approved by the Human Studies Committees of Massachusetts General Hospital and Shriners Hospitals for Children. The final sample was 17 subjects who received sertraline versus 9 placebo control subjects matched for age, severity of injury, and type of hospitalization. There was no significant difference in change from baseline with child-reported symptoms; however, the sertraline group demonstrated a greater decrease in parent-reported symptoms over 8 weeks (-4.1 vs. -0.5, p=0.005), over 12 weeks (-4.4 vs. -1.2, p=.008), and over 24 weeks (-4.0 vs. -0.2, p=0.017). Sertraline was a safe drug, and it was somewhat more effective in preventing PTSD symptoms than placebo according to parent report but not child report. Based on this study, sertraline may prevent the emergence of PTSD symptoms in children. [\hyperlink{Cabtreo}{PMID: 22040192}, Frederick J Stoddard et al., 2011]

\hypertarget{pmid_27412538}{T}o investigate the efficacy and safety of lamotrigine monotherapy in children with epilepsy via a systematic review. PubMed, Cochrane, CNKI, VIP, CBM, Wanfang Data were searched for randomized controlled trials (RCTs) of lamotrigine monotherapy in children with epilepsy. Literature screening, data extraction, and quality assessment were performed according to the method recommended by Cochrane Collaboration. RevMan 5.2 software was used to conduct the Meta analysis. A total of 9 RCTs involving 1 016 participants were included. Lamotrigine yielded a significantly lower complete control rate of seizure than ethosuximide, but the complete control rate of seizure showed no significant differences between lamotrigine and carbamazepine/sodium valproate. Patients treated with lamotrigine had a significantly lower incidence rate of adverse events than those treated with carbamazepine, but the incidence rate of adverse events showed no significant differences between patients treated with lamotrigine and sodium valproate/carbamazepine. The drop-out rate showed no significant differences between the three treatment groups. Lamotrigine is an ideal alternative drug for children who do not respond to traditional antiepileptic medication or experience significant adverse reactions; however, more high-quality RCTs with a large sample size and a long follow-up time are needed to confirm these conclusions. [\hyperlink{Cabtreo}{PMID: 27412538}, Yan-Tao Liu et al., 2016]

\hypertarget{pmid_20401256}{A}cetaminophen has become the non-narcotic of choice for children because of concerns regarding the connection between acetylsalicylic acid exposure and Reye's syndrome. Ibuprofen, recently granted over-the-counter status for children over two years of age, offers another choice for treatment. The efficacy and safety of both drugs have been studied in numerous clinical trials. This paper reviews the published evidence about the efficacy and safety of acetaminophen and ibuprofen with regard to treating fever and mild to moderate pain in children. [\hyperlink{Cabtreo}{PMID: 20401256}, H N McCullough et al., 1998]

\section*{Ciclopirox}
\subsection*{Result}
\subsubsection*{Answer}

3 months to 9 years: Yes  
6 to 29 months: Yes  
Older than 9 years: Unknown  
Younger than 3 months: Unknown  

\subsubsection*{{Explanation}}
\hypertarget{Ciclopirox}
A review of the available abstracts reveals two targeted studies that directly address the safety of ciclopirox in children:

1. **Infants and Toddlers (6 to 29 months):**  
An open-label study evaluated the safety and efficacy of ciclopirox topical suspension 0.77\% in 44 children aged 6 to 29 months with diaper dermatitis due to Candida albicans. The medication was applied twice daily for one week. All adverse events were mild to moderate and considered not related to the study medication. The study concluded that ciclopirox was safe and effective in this age group [\hyperlink{pmid_15696982}{PMID: 15696982}, Elizabeth Gallup et al.].

2. **Infants to Young Children (3 months to 9 years):**  
A multicenter, open-label, phase III study assessed ciclopirox olamine cream 1\% in 21 children aged 3 months to 9 years with dermatomycosis. No adverse events were reported in 62\% of patients; among the mild and moderate adverse events, only one case of irritative dermatitis was possibly related to the treatment. Safety evaluation was rated as excellent in 95\% and good in 5\% of patients. The study concluded that ciclopirox olamine cream 1\% is a safe and feasible treatment for superficial cutaneous mycotic infections in children aged 3 months to 10 years [\hyperlink{pmid_25728876}{PMID: 25728876}, Elisabeth Gómez-Moyano et al.].

Other abstracts either do not specify pediatric populations or do not provide targeted safety data for children.

**Summary by Age Range:**
- **3 months to 9 years:** Evidence from a targeted study supports the safety of ciclopirox olamine cream 1\% for superficial fungal infections.
- **6 to 29 months:** Evidence from a targeted study supports the safety of ciclopirox topical suspension 0.77\% for diaper dermatitis due to Candida albicans.
- **Older children (9 years and above):** No targeted safety studies were identified in the abstracts for this age group.

\subsection*{Abstracts}
\hypertarget{pmid_15696982}{C}iclopirox is a broad-spectrum antifungal, antibacterial, and anti-inflammatory agent. This open-label study investigated the safety and efficacy of ciclopirox topical suspension 0.77\% in the treatment of diaper dermatitis due to Candida albicans (C. albicans). Forty-four male and female subjects aged 6 to 29 months were included in the study. Study medication was applied topically to the affected diaper area twice daily for 1 week. Subjects were clinically evaluated at baseline and days 3, 7, and 14 (7 days post-treatment). Safety and efficacy variables included adverse events, mycological culture studies, KOH tests, Severity Scores, and Global Evaluation of Clinical Response. All adverse events were mild to moderate and considered not related to the study medication. Treatment provided statistically significant improvement (P < .05) for both the rate of mycological cure and reduction of Severity Score at each time point compared with baseline. Ciclopirox was safe and effective in the treatment of diaper dermatitis due to C. albicans. [\hyperlink{Ciclopirox}{PMID: 15696982}, Elizabeth Gallup et al., ]

\hypertarget{pmid_18698269}{T}o evaluate the ototoxicity of ciclopirox-containing solution as an otologic preparation for the treatment of otomycosis. Ciclopirox is a synthetic antimycotic agent available in a variety of formulations to treat superficial fungal infections. Ciclopirox has demonstrated both fungicidal and fungistatic activity in vitro against a broad spectrum of pathogenic fungi. It also possesses a broad-spectrum antibacterial properties, anti-inflammatory, and antiedema effect. The ototoxic effect of ciclopirox-containing solutions has not been known, so the current study was designed to observe the ototoxic effect of this solution experimentally. Experiments were performed in 22 young male albino guinea pigs (weight, 450-550 g). The 10 animals in the experimental group received ciclopirox solution, and the control group was divided into two groups of six animals each. The first group received saline solution (negative control) and the second received gentamicin (40 mg/mL; ototoxic control). Under general anesthesia, pretreatment auditory brainstem responses (ABRs) from the right ears were obtained from the animals in all groups. The right tympanic membranes were totally perforated, and a small piece of Gelfoam was applied to the middle ear directly to the round window membrane. Ear solutions were applied through transcanal approach to the middle ear twice a day in 2 weeks. Twenty-two animals of perforated tympanic membrane were observed during a 2-week period. Posttreatment ABRs were obtained in all groups in a week after the last administration. Baseline ABR results were normal in right ears of all animals tested. Animals undergoing placement of Gelfoam with either ciclopirox solution or saline in the middle ear showed no changes in the ABR threshold. The gentamicin group showed a significant change in the ABR threshold. In the guinea pig, when applied topically to the middle ear, ciclopirox does not cause a reduction in the ABR threshold. Because its safety has not yet been confirmed in patients, caution should be observed when prescribing this agent. [\hyperlink{Ciclopirox}{PMID: 18698269}, Serdar Baylancicek et al., 2008]

\hypertarget{pmid_18305467}{W}e conducted a prospective, open-label multicenter trial to evaluate the efficacy and safety of treating children with frequently relapsing nephrotic syndrome with cyclosporine. Patients were randomly divided into two groups with both initially receiving cyclosporine for 6 months to maintain a whole-blood trough level between 80 and 100 ng/ml. Over the next 18 months, the dose was adjusted to maintain a slightly lower (60-80 ng/ml) trough level in Group A, while Group B received a fixed dose of 2.5 mg/kg/day. The primary end point was the rate of sustained remission with analysis based on the intention-to-treat principle. After 2 years, the rate of sustained remission was significantly higher while the hazard ratio for relapse was significantly lower in Group A as compared with Group B. Mild arteriolar hyalinosis of the kidney was more frequently seen in Group A than in Group B, but no patient was diagnosed with striped interstitial fibrosis or tubular atrophy. We conclude that cyclosporine given to maintain targeted trough levels is an effective and relatively safe treatment for children with frequently relapsing nephrotic syndrome. [\hyperlink{Ciclopirox}{PMID: 18305467}, K Ishikura et al., 2008]

\hypertarget{pmid_21144334}{C}yclosporine has been found to be effective and safe in many inflammatory skin disorders such as psoriasis and atopic dermatitis (AD), in adults and in children. Its use in paediatrics is still under scope. We present three patients who started cyclosporine but stopped due to complications. It is our aim to warn about potential side effects of cyclosporine and recommend cautious utilization. Two children, aged 4 and 13 years, with AD and one child, aged 2 years, with erythrodermic psoriasis, were treated with oral cyclosporine. developed secondary impetigo on the 6th day of treatment. Started topical corticosteroids and topical calcineurin inhibitors afterwards, with no relapses. developed herpetic infection, hepatic and renal impairment (eventual drug interaction) on the 4th day of treatment. THIRD CASE: Psoriasis and impetigo, treated with flucloxacillin, gentamicin. Generalized angioedema and urticariform lesions after 6 days of cyclosporine. Beta lactam hypersensitivity reaction under study. Eventual cyclosporine toxicity to consider. The data on cyclosporine use in children is still scarce. Use should be limited to cases with precise indication, after considering risks and benefits. [\hyperlink{Ciclopirox}{PMID: 21144334}, João Antunes et al., ]

\hypertarget{pmid_15985032}{C}iclopirox is an antifungal agent and is effective against both Gram-positive and Gram-negative bacteria. These properties may give ciclopirox an advantage over other antifungal agents in the treatment of interdigital tinea pedis with secondary bacterial infection (dermatophytosis complex). To evaluate the efficacy of ciclopirox 0.77\% gel in the treatment of tinea pedis interdigitalis with secondary bacterial infection in a prospective, randomized, double-blind, placebo-controlled clinical study. One hundred subjects were enrolled in this 8-week study (twice-daily ciclopirox, 40 subjects; once-daily ciclopirox, 40 subjects; twice-daily vehicle, 20 subjects). Mycologic sampling, bacterial swabs, and evaluations for symptoms and signs of tinea pedis were performed on a target webspace at baseline and at weeks 2, 4, and 8. Global evaluations were made by both investigator and subject at each visit. Ciclopirox gel applied once or twice daily significantly reduced the signs and symptoms at week 8, compared with vehicle (P<0.0036). The mycologic cure and complete cure rates were much higher for the ciclopirox regimens than for the vehicle regimen. Early reduction of bacterial counts was noted with the ciclopirox regimens. There was no significant difference in the adverse event rate between the ciclopirox groups and the placebo group. Ciclopirox 0.77\% gel, applied once or twice daily, is effective and safe in the treatment of tinea pedis interdigitalis with concomitant bacterial infection (dermatophytosis complex). [\hyperlink{Ciclopirox}{PMID: 15985032}, Aditya K Gupta et al., 2005]

\hypertarget{pmid_19617660}{C}yclosporine A is used in the treatment of idiopathic nephrotic syndrome. We conducted this study to evaluate the effect of cyclosporine and its combination with ketoconazole in Egyptian nephrotic children with steroid-resistant and steroid-dependant minimal change. Forty-eight children with minimal change lesions who received cyclosporine with or without ketoconazole were studied. Their mean age was 5.17 +/- 1.59 years, and they were 31 boys and 17 girls. The mean duration of the disease was 6.22 +/- 3.16 years. Thirty-one of the children were steroid dependent and 17 were steroid resistant. Cyclosporine treatment was commenced after remission was attained and adjusted to a target trough level of 100 ng/mL. The mean cyclosporine therapy at a dose of 2.07 +/- 0.91 mg/kg was administered for a mean of 25.75 +/- 1.95 months. Thirty-three patients received adjunctive ketoconazole therapy. Thirty-eight patients (79.2\%) responded well to cyclosporine. Steroid therapy could be discontinued in 43 patients (89.6\%), but 9 experienced relapse. Ten patients (20.8\%) were resistant to cyclosporine therapy. Fifteen patients received cyclosporine alone, while 33 received concomitant cyclosporine and ketoconazole. The response to cyclosporine was significantly better in those on ketoconazole. The economic effect of ketoconazole therapy was a reduction in the costs of cyclosporine treatment by 47.4\% at 1 year of treatment. Cyclosporine treatment in children with minimal change nephrotic syndrome is effective in preventing relapse and decreasing steroid toxicity. Its combination with low-dose ketoconazole is safe, reduces treatment costs, and improves the response to cyclosporine. [\hyperlink{Ciclopirox}{PMID: 19617660}, Alaa Sabry et al., 2009]

\hypertarget{pmid_25728876}{T}here is scarce information on the use of ciclopirox olamine in children. The aim of this study was to evaluate the efficacy and safety of ciclopirox olamine cream 1\% for the treatment of dermatomycosis in pediatric patients. A multicenter, non-randomized, open-label, phase iii study was conducted on patients aged 3 months to 9 years diagnosed with dermatomycosis confirmed by direct microscopy and culture, and treated with ciclopirox olamine cream 1\% for 28 days. Clinical and microbiological evaluations were performed before starting the treatment therapy, at 7, 14 and 28 days after starting the treatment, and 28 days after its completion. Twenty-one patients with a median age of 2.7 years (range 3 months-9 years) were included. The most frequent mycosis location was the inguinal region (72\%). The most frequently isolated etiological agent was Candida spp. (71\%). No adverse events were reported in 62\% of the patients. Among the mild and moderate reported adverse events, only one, irritative dermatitis, was considered as possibly related to the treatment. Safety evaluation was excellent in 95\% of the patients, and good in 5\%. After the first week of treatment, 12 patients out of 13 (92\%) showed a clinical improvement, and 5 out of 7 (71\%) had both clinical and mycological improvements. At the end of the treatment, clinical cure was observed in 7 out of 9 patients (78\%). No relapses occurred. Ciclopirox olamine cream 1\% is a safe and feasible treatment for superficial cutaneous mycotic infections, especially Candida spp. infection, in children aged between 3 months and 10 years. [\hyperlink{Ciclopirox}{PMID: 25728876}, Elisabeth Gómez-Moyano et al., ] the aim of this study was to report single centre experience with cyclosporine used in treatment of children with inflammatory bowel disease with regard to safety and efficacy. retrospective analysis included 23 patients, 21 with ulcerative colitis and 2 with Crohn's disease, aged 2.75 to 18.5 years. They were treated with cyclosporine during the last 5 years. Before cyclosporine therapy was started they received steroids and azathioprine. Cyclosporine treatment was given in severe steroid-resistant exacerbation of the disease (n = 10) or steroid-dependence (n = 13). Cyclosporine dose was set to obtain therapeutic levels (serum concentration > 100 ng/ml and < 200 ng/ml). Cyclosporine treatment was continued up to 2 months in 6 cases, 2 to 6 months in 8 patients and more than 6 months in 9 patients. Complications were reported in 2 patients: hirsutism and gingival hypertrophy. Cyclosporine treatment was stopped in the second case. None of the two patients with Crohn's disease improved during the treatment. Short-term improvement was observed in 11 patients with ulcerative colitis. Long-term recovery (> 6 months) was obtained in 6 cases. In 10 patients with severe exacerbation of ulcerative colitis colectomy was performed, in 4 of them elective surgery was performed when the clinical status improved. cyclosporine appears to be a safe and relatively effective treatment of ulcerative colitis in children. Cyclosporine is less effective in maintaining remission and it did not allow to avoid colectomy in severe exacerbation. Still case controlled studies are needed to show the efficacy of this treatment. [\hyperlink{Ciclopirox}{PMID: 25728876}, Piotr Socha et al., ]

\hypertarget{pmid_20530497}{W}e previously established a treatment protocol for conventional cyclosporine (Sandimmune, Novartis, Basel, Switzerland) in children with frequently relapsing nephrotic syndrome; ∼50\% of patients remained relapse free for 2 years, without serious adverse events. Recently, microemulsified cyclosporine (Neoral, Novartis), which has a more stable absorption profile than conventional cyclosporine, has been developed. We tested the hypothesis that microemulsified cyclosporine is at least as effective as conventional cyclosporine. To evaluate the safety and efficacy of microemulsified cyclosporine, a prospective, multicentre trial was conducted according to the previously established protocol, using microemulsified cyclosporine instead of conventional cyclosporine. The duration of treatment was 24 months. During the first 6 months, patients received microemulsified cyclosporine in a dose that maintained the trough level between 80 and 100 ng/mL of cyclosporine. For the next 18 months, the dose was adjusted to maintain a level between 60 and 80 ng/mL. A total of 62 patients (median age, 5.4 years; 48 males, 14 females) were studied. The frequency of relapse decreased from 4.6 ± 1.4 to 0.7 ± 1.5 times per year (P < 0.0001). The probability of relapse-free survival at Month 24 was 58.1\% (95\% confidence interval, 45.8-70.3\%). The probability of progression (to frequently relapsing nephrotic syndrome)-free survival at Month 24 was 88.5\% (95\% confidence interval, 80.4-96.5\%). Cyclosporine nephrotoxicity was detected in only 8.6\% of patients who underwent renal biopsy after 2 years of treatment. Antihypertensive agents were administered to 12.9\% of the patients to control hypertension without severe sequelae. Microemulsified cyclosporine administered according to our treatment protocol is safe and effective in children with frequently relapsing nephrotic syndrome. [\hyperlink{Ciclopirox}{PMID: 20530497}, Kenji Ishikura et al., 2010]

\hypertarget{pmid_2642107}{A}lthough cyclosporine has improved allograft survival in renal transplant patients, problems with drug toxicity remain, raising the question whether cyclosporine should be stopped at some point post-transplant. However, the relative safety of converting from cyclosporine to another immunosuppressive agent, or simply stopping cyclosporine remains an issue of debate and has not been evaluated in children. We have developed a protocol to convert children, who are 6 months post-transplant and have stable kidney function, from cyclosporine and prednisone to azathioprine and prednisone. Eleven children have undergone conversion because of suspected/potential nephrotoxicity or because of other difficulties with cyclosporine (expense, hirsutism). These children were compared with a control group of 12 children who met all criteria for conversion at 6 months but remained on cyclosporine. Allograft survival was similar in both groups but the children converted from cyclosporine experienced an improvement in renal function as measured by calculated creatinine clearance. There were no episodes of rejection for a period of 4 months post-conversion and all rejection episodes that developed subsequently occurred during or after the change from daily to alternate-day prednisone. We believe that conversion from cyclosporine to azathioprine can be accomplished safely in children with stable allograft function but long-term risks and benefits need further evaluation. [\hyperlink{Ciclopirox}{PMID: 2642107}, B A Kaiser et al., 1989]

\hypertarget{pmid_15334276}{C}iclopirox (Loprox) is a broad-spectrum antifungal medication that also has antibacterial and anti-inflammatory properties. Its main mode of action is thought to be its high affinity for trivalent cations, which inhibit essential co-factors in enzymes. Clinical trials have shown that ciclopirox gel is a successful treatment for seborrheic dermatitis of the scalp as well as for tinea pedis. Adverse effects are generally mild and include a skin-burning sensation, contact dermatitis, and pruritus. Ciclopirox is indicated in the US for the treatment of tinea pedis, tinea corporis, pityriasis versicolor, seborrheic dermatitis, and cutaneous candidiasis. [\hyperlink{Ciclopirox}{PMID: 15334276}, A K Gupta et al., ]

\hypertarget{pmid_8151150}{T}he efficacy and safety of aciclovir granules (containing 40\% w/w aciclovir) were evaluated in the treatment of chickenpox in otherwise healthy children. Patients presenting with chickenpox received aciclovir granules at a dose of 20 mg/kg four times daily for five to seven days. Overall 51 children received treatment with aciclovir. A further 53 patients receiving conventional symptomatic therapy acted as a control. In the aciclovir group the overall efficacy rate was 92.2\%. There were reductions in the numbers of lesions, fever, itching and the duration of symptoms. No adverse experiences were reported. Overall this formulation of aciclovir appears to be a safe and effective treatment for chickenpox in this patient population. However the need for anti-viral therapy in otherwise healthy children is still the subject of debate and it might be appropriate to identify sub-groups for whom such therapy is justified. [\hyperlink{Ciclopirox}{PMID: 8151150}, H Kamiya et al., 1994]

\hypertarget{pmid_15790671}{C}iclopirox is a topical antifungal agent of the hydroxypyridone class whose mode of action is poorly understood. In order to elucidate the mechanism of action of ciclopirox, we analysed the growth, cellular integrity, biochemical properties, viability and transcriptional profile of the polymorphic yeast Candida albicans following exposure to this antifungal agent. Multiple biochemical assays served to identify factors that were critical for antifungal activity and to identify proteins whose activities changed in drug-exposed cells. Genome-wide transcriptional profiling was used to identify genes that were up-regulated in response to the cellular effects of the drug. Ciclopirox inhibited growth of C. albicans yeast and hyphal cells in a dose-dependent manner. This effect was reduced (i) by the addition of iron ions or the metabolic inhibitor 2-deoxy-D-glucose to growth media, (ii) in media that lacked glucose, and (iii) for cells that were pre-incubated with hydrogen peroxide or menadione [which caused induction of proteins involved in detoxification of reactive oxygen species (ROS)]. In contrast, cells pre-cultured under poor oxygen conditions (which had decreased activity of proteins involved in ROS detoxification) were more susceptible to ciclopirox. Treatment with ciclopirox did not directly cause cell membrane damage and did not change intracellular levels of ATP. Finally, the transcriptional profiling pattern of drug-treated cells strongly resembled iron-limited conditions. These data indicate that metabolic activity, oxygen accessibility and iron levels are critical parameters in the mode of action of ciclopirox olamine. [\hyperlink{Ciclopirox}{PMID: 15790671}, Hans-Christian Sigle et al., 2005]

\hypertarget{pmid_15271196}{S}eborrheic dermatitis is a common inflammatory skin disorder. Yeasts of the genus Malassezia have been implicated in the etiology of seborrheic dermatitis, although this connection remains controversial. Ciclopirox is a synthetic, hydroxypyridone-derived, broad-spectrum antifungal agent with anti-inflammatory properties. A total of 499 US patients with seborrheic dermatitis of the scalp were randomized to apply either ciclopirox shampoo 1\% or vehicle twice weekly for 4 weeks. The main efficacy parameters were based on 6-point ordinal scales describing the disease's signs and symptoms (scaling, erythema and itching) and a 6-point scale providing a global evaluation of the status of seborrheic dermatitis. Ciclopirox was significantly better than vehicle in effectively treating seborrheic dermatitis. 'Effective treatment' (score of 0 or 1 for disease status, scaling and erythema) was achieved in 26.0\% of ciclopirox-treated patients compared with 12.9\% of vehicle-treated patients (P = 0.0001; OR: 2.383, 95\% CI: 1.494-3.799). The majority of subjects experienced adverse events that were mild in intensity, with skin and appendage reactions the most commonly reported, at similar frequency in both groups. Ciclopirox shampoo 1\% is effective and safe in the treatment of seborrheic dermatitis of the scalp. [\hyperlink{Ciclopirox}{PMID: 15271196}, Mark Lebwohl et al., 2004]

\hypertarget{pmid_14567368}{B}ACKGROUND Tinea pedis (athlete's foot) is the most common fungal infection in the general population. Ciclopirox, a broad-spectrum hydroxypyridone antifungal, has proven efficacy against the organisms commonly implicated in tinea pedis; Trichophyton rubrum, T.mentagrophytes and Epidermophyton floccosum. Two multicenter, double-blind, clinical studies compared the efficacy and safety of ciclopirox gel with that of its vehicle base in subjects with moderate interdigital tinea pedis with or without plantar involvement. Three hundred and seventy-four subjects were enrolled and randomized to one of two treatment groups: ciclopirox gel 0.77\%, or ciclopirox gel vehicle, applied twice daily for 28 days, with a final visit up to day 50. The primary efficacy variable was Treatment Success defined as combined mycological cure and clinical improvement >/= 75\%. Secondary measures of effectiveness were Global Clinical Response, Sign and Symptom Severity Scores, Mycological Evaluation (KOH examination and final culture result), Mycological Cure (negative KOH and negative final culture results) and Treatment Cure (combined clinical and mycological cure). At endpoint (final post-baseline visit), 60\% of the ciclopirox subjects achieved treatment success compared to 6\% of the vehicle subjects. At the same time point, 66\% of ciclopirox subjects compared with 19\% of vehicle subjects were either cleared or had excellent improvement. Pooled data showed that 85\% of ciclopirox subjects were mycologically cured, compared to only 16\% of vehicle subjects at day 43, 2 weeks post-treatment. Ciclopirox gel 0.77\% applied twice daily for 4 weeks is an effective treatment of moderate interdigital tinea pedis due to T. rubrum, T. mentagrophytes and E. floccosum and is associated with a low incidence of minor adverse effects. [\hyperlink{Ciclopirox}{PMID: 14567368}, Raza Aly et al., 2003]

\hypertarget{pmid_8518000}{F}ive children with multiple relapsing steroid-dependent nephrotic syndrome were treated with continuous cyclosporin for periods ranging from 18 to 48 months. Renal biopsy showed mild mesangial proliferation in three of the children and minimal change in two. All children previously had been treated with cyclophosphamide. Cyclosporin was started during remission at 5 mg/kg per day. If a relapse occurred the dose was increased until a trough blood level of 100-250 ng/mL (HPLC) was achieved. In the initial 12 months of treatment, the mean number of relapses decreased from 6.4 +/- 0.54 (s.d.) per annum to 1.6 +/- 1.3 per annum (P < 0.01). Cyclosporin was effective in maintaining long-term remission in four of the five patients. Side effects included hypertrichosis (5) and gum hyperplasia (1). The mean creatinine clearance decreased from 126 +/- 16 to 97 +/- 22 mL/min per 1.73 m2 (P = NS). A renal biopsy in all five patients after 12 months therapy showed no nephrotoxicity. A further biopsy in one patient after 4 years therapy showed interstitial fibrosis. Cyclosporin should be considered in children with steroid-dependent nephrotic syndrome who show signs of steroid toxicity and have only a short remission period after cyclophosphamide. Serial renal biopsies are recommended if prolonged therapy is used. [\hyperlink{Ciclopirox}{PMID: 8518000}, K L Webb et al., 1993]

\hypertarget{pmid_12836096}{I}n a systematic review and meta-analysis of randomized controlled trials (RCT), we aimed to evaluate the benefits and harms of all interventions for children with steroid-resistant nephrotic syndrome (SRNS). Nine RCTs involving 225 children were included. Cyclosporin when compared with placebo or no treatment significantly increased the number of children who achieved complete remission [3 trials, 49 children, relative risk (RR) for persistent nephrotic syndrome 0.64, 95\% confidence intervals (CI), 0.47-0.88]. There was no significant difference in the number of children who achieved complete remission between oral cyclophosphamide with prednisone and prednisone alone [2 trials, 91 children, RR 1.01, 95\% CI 0.74-1.36], between intravenous cyclophosphamide and oral cyclophosphamide [1 study, 11 children, RR 0.09, 95\% CI 0.01-1.39], and between azathioprine with prednisone and prednisone alone [1 trial, 31 children, RR 1.01, 95\% CI 0.77-1.32]. No RCTs were identified comparing combination regimens comprising high-dose steroids, alkylating agents or cyclosporin with single agents, placebo, or no treatment. Further adequately powered and well-designed RCTs are needed to confirm the efficacy of cyclosporin and to evaluate regimens of high-dose steroids with alkylating agents or cyclosporin for SRNS. [\hyperlink{Ciclopirox}{PMID: 12836096}, Doaa Habashy et al., 2003]

\hypertarget{pmid_8881900}{C}yclosporin has been shown to be effective in the treatment of adult atopic dermatitis, but there are no clinical trials evaluating its use in childhood. Atopic dermatitis is more common in children and the severe form can be associated with considerable morbidity. We report on 18 children with severe refractory atopic dermatitis who have been treated with cyclosporin on an open basis. The drug was given at an initial daily dose of 5 or 6 mg/kg and in some patients the dose was reduced according to response. Sixteen patients showed a good or excellent response to treatment, one a moderate response and one patient failed to improve. The treatment was well tolerated and there were no significant changes in serum creatinine or blood pressure. Long remission after withdrawal of treatment was seen in some patients, although most relapsed within a few weeks. We suggest that cyclosporin is an effective and safe short-term treatment for severe atopic dermatitis in childhood. [\hyperlink{Ciclopirox}{PMID: 8881900}, I Zaki et al., 1996]

\hypertarget{pmid_11107710}{C}iclopirox 8\% nail lacquer has recently become the first topical antifungal agent to be approved by the US Food and Drug Administration for the treatment of onychomycosis. This article reviews the results of the two pivotal clinical trials of this drug that have been performed in the United States as well as those that have been carried out in other countries. The two US studies were both double-blind, vehicle-controlled, parallel-group, multicenter studies designed to determine the efficacy and safety of ciclopirox nail lacquer in the treatment of mild-to-moderate onychomycosis of the toenails caused by dermatophytes. The combined results show a 34\% mycologic cure rate, as compared with 10\% for the placebo. Data from the ten studies conducted worldwide show a meta-analytic mean (+/- SE) mycologic cure rate of 52.6\% +/- 4.2\%. As expected for a topical agent, ciclopirox nail lacquer was found to be extremely safe, with mild, transient irritation at the site of application reported as the most common adverse event. Ciclopirox nail lacquer may also have potential for use in combination or adjunctive therapy. Further studies will help to better position this agent for the treatment of this widespread podiatric condition. [\hyperlink{Ciclopirox}{PMID: 11107710}, A K Gupta et al., ]

\hypertarget{pmid_11051135}{C}iclopirox is a synthetic hydroxypyridone antifungal agent. In contrast to the azoles, glucuronidation is the main metabolic pathway of ciclopirox; therefore interactions with drugs metabolized via the cytochrome P450 system are unlikely Ciclopirox is also distinct from the common systemic agents, which interfere with sterol biosynthesis. In fact, ciclopirox chelates trivalent cations (such as Fe3+), inhibits metal-dependent enzymes that are responsible for degradation of toxic metabolites in the fungal cells, and targets diverse metabolic (eg, respiratory) and energy producing processes in microbial cells. Ciclopirox is a broad spectrum antimicrobial with activity against all the usual dermatophytes as well as yeast and nondermatophyte molds. It has demonstrated activity against gram positive and negative bacteria, including resistant strains of Staphlococcus aureus. Ciclopirox exhibits fungal inhibitory activity (minimum inhibitory concentration < 4 microg/mL for dermatophytes) as well as fungicidal activity; to date resistance to the drug has not been identified. Ciclopirox has been formulated in a nail lacquer delivery system. After evaporation of volatile solvents in the lacquer, the concentration of ciclopirox in the remaining lacquer film reaches approximately 35\%, providing a high concentration gradient for penetration into the nail. Radiolabel data demonstrate penetration into infected nails after only 1 application of the lacquer. Ciclopirox nail lacquer is a topical product that provides an active fungicidal agent in a delivery system capable of promoting nail penetration. With repeated applications, the antifungal agent is homogeneously distributed through all layers of the toenail achieving concentrations of ciclopirox in excess of inhibitory and fungicidal concentrations for most pathogens. Although ciclopirox readily penetrates nails, very low levels of ciclopirox are recoverable systemically, even after chronic use. Ciclopirox nail lacquer 8\% is a topical product that provides an active fungicidal agent in a delivery system capable of penetrating nails. [\hyperlink{Ciclopirox}{PMID: 11051135}, M Bohn et al., 2000]

\hypertarget{pmid_1868743}{W}e report on our experience with acyclovir (Zovirax) capsules for the symptomatic treatment of chickenpox in two children and an adult. [\hyperlink{Ciclopirox}{PMID: 1868743}, C Marino et al., 1991]

\hypertarget{pmid_8647967}{S}evere atopic dermatitis (AD) remains difficult to treat. Cyclosporine is effective in adults but has not previously been investigated in children with AD. The aims were to investigate the efficacy, safety, and tolerability of cyclosporine in severe refractory childhood AD. Subjects 2 to 16 years of age were treated for 6 weeks with cyclosporine, 5 mg/kg per day, in an open study. Disease activity was monitored every 2 weeks by means of sign scores, visual analogue scales for symptoms, and quality-of-life questionnaires. Adverse events were monitored. Efficacy and tolerability were assessed with five-point scales. Twenty-seven children were treated. Significant improvements were seen in all measures of disease activity. Twenty-two showed marked improvement or total clearing. Quality of life improved for both the children and their families. Tolerability was considered good or very good in 25 subjects. Cyclosporine may offer an effective, safe, and well-tolerated short-term treatment option for children with severe AD. [\hyperlink{Ciclopirox}{PMID: 8647967}, J Berth-Jones et al., 1996]

\hypertarget{pmid_12895183}{S}eborrheic dermatitis is a common inflammatory skin disorder that usually occurs in patients with pre-existing seborrhea. The etiology of seborrheic dermatitis is uncertain. Typically, sites dense with sebaceous glands support growth of the lipophilic yeast Malassezia furfur. Ciclopirox (Loprox) gel is a hydroxypyridone, broad-spectrum antifungal agent proven effective against the yeast M. furfur. A multicenter, randomized, double-blind, vehicle controlled study of 178 subjects evaluated the efficacy of ciclopirox gel in treating seborrheic dermatitis of the scalp. One hundred and seventy-eight subjects were randomized to apply either ciclopirox gel 0.77\% twice daily, or vehicle twice daily for 28 days. Subjects' signs and symptoms of severity (erythema, scaling, pruritus and burning) were rated on a scale of 0-3 (none to severe); for inclusion, a minimum score of 4, for the sum of the individual ratings was required. Efficacy evaluations were performed at baseline, days 4, 8, 15, 22, 29, and at end-point (final visit, up to day 33). The primary efficacy variable was clinical response assessed by a global improvement, based on a scale of 0-5 (100\% clearance to flare of treatment area). Changes in signs/symptoms severity scores within the target lesion were also evaluated. Global evaluation scores demonstrated that significantly more ciclopirox-treated subjects achieved over 75\% improvement compared with vehicle at days 22, 29, and endpoint (P < 0.01). Change-from-baseline mean score for total signs and symptoms was significantly greater in ciclopirox subjects compared with vehicle subjects at the same time points as above (P < 0.001), as well as day 15 (P < 0.01). Twenty-nine percent of subjects rated ciclopirox as having excellent cosmetic acceptability. There were only mild adverse events, with the most common being burning sensation in 13\% of ciclopirox subjects and 9\% of vehicle subjects. Ciclopirox gel is effective and safe in the treatment of seborrheic dermatitis of the scalp. [\hyperlink{Ciclopirox}{PMID: 12895183}, Raza Aly et al., 2003]

\hypertarget{pmid_25059452}{C}yclosporine is a systemic therapy used for control of severe atopic dermatitis (AD) in children. Although traditionally recommended at a dose of 5 mg/kg/day for 6 months, a longer duration of treatment may be necessary to bring a child with active and severe disease into remission. There are few data on the short- and long-term effectiveness of longer courses of therapy. This was a retrospective chart review of children treated with cyclosporine at a Canadian hospital-affiliated clinic between 2000 and 2013. Fifteen patients with adequate follow-up were identified. Twelve (80\%) were male and the mean age at initiation of cyclosporine was 11.2 ± 3.4 years. The mean duration of cyclosporine therapy was 10.9 ± 2.7 months (range 7-15 months) at a starting dose of 2.8 ± 0.6 mg/kg/day. Of 12 patients (80\%) who responded to cyclosporine, 5 patients (42\%) had relapsed at a follow-up of 22.7 ± 15.0 months. The duration of therapy was longer in patients who did not relapse (17.7 ± 10.7 months) than in those who did (10.2 ± 2.7 months) (p = 0.06). Adverse events led to discontinuation in three patients (20\%) and included infection-related complications in two patients and reversible renal toxicity in one. These results suggest that a longer duration of low-dose cyclosporine may help decrease the risk of relapse in patients with severe AD who are resistant to topical therapies. [\hyperlink{Ciclopirox}{PMID: 25059452}, Cathryn Sibbald et al., ]

\hypertarget{pmid_23700934}{T}wo hundred fifty patients, including 100 children with frequent and prolonged diseases at the age of 4 to 7 years, 76 children at the age of 7 to 18 years and 74 subjects at the age of 22 to 57 years were observed. The patients were treated with cycloferon in two courses with a 2-week interval according to the standard scheme. The tonsil surface microflora and its susceptibility to antibiotics were determined. Cycloferon lowered the Staphylococcus aureus titre and increased the culture susceptibility to benzylpenicillin, oxacillin, rifampicin, and erythromycin, reducing the variety of the fauces nonpathogenic microflora. The use of cycloferon induced no adverse (pathologic) reactions in 94.8\% of the cases. In 4.4\% of the children under school age the adverse reactions were transitory and did not require discontinuation of the drug use. Unforeseen reactions were recorded in 0.8\% of the children and the use of the drug in them was discontinued. The use of cycloferon in two courses with a 2-week interval according to the standard scheme is recommended for prophylaxis of acute respiratory diseases in the group of children with frequent and prolonged diseases during epidemiologically unfavourable periods and for complex therapy of rhinopharinx infections as an agent increasing efficacy of other antibacterials. [\hyperlink{Ciclopirox}{PMID: 23700934}, S A Lialikov et al., 2012]

\section*{Panretin}
\subsection*{Result}
\subsubsection*{Answer}

Unknown

\subsubsection*{{Explanation}}
\hypertarget{Panretin}
After reviewing all available abstracts, there is no evidence from any abstract that specifically addresses the safety of Panretin (alitretinoin) in children. None of the abstracts provided are about Panretin or its use in pediatric populations. Therefore, based on the abstracts available, there is not enough data to determine whether Panretin is safe or unsafe for use in children of any age range. The safety of Panretin in children is unknown according to the evidence provided.

\subsection*{Abstracts}
\hypertarget{pmid_22040192}{T}his study evaluated the potential benefits of a centrally acting selective serotonin reuptake inhibitor, sertraline, versus placebo for prevention of symptoms of posttraumatic stress disorder (PTSD) and depression in burned children. This is the first controlled investigation based on our review of the early use of a medication to prevent PTSD in children. Twenty-six children aged 6-20 were assessed in a 24-week double-blind placebo-controlled design. Each child received either flexibly dosed sertraline between 25-150 mg/day or placebo. At each reassessment, information was collected in compliance with the study medication, parental assessment of the child's symptomatology and functioning, and the child's self-report of symptomatology. The protocol was approved by the Human Studies Committees of Massachusetts General Hospital and Shriners Hospitals for Children. The final sample was 17 subjects who received sertraline versus 9 placebo control subjects matched for age, severity of injury, and type of hospitalization. There was no significant difference in change from baseline with child-reported symptoms; however, the sertraline group demonstrated a greater decrease in parent-reported symptoms over 8 weeks (-4.1 vs. -0.5, p=0.005), over 12 weeks (-4.4 vs. -1.2, p=.008), and over 24 weeks (-4.0 vs. -0.2, p=0.017). Sertraline was a safe drug, and it was somewhat more effective in preventing PTSD symptoms than placebo according to parent report but not child report. Based on this study, sertraline may prevent the emergence of PTSD symptoms in children. [\hyperlink{Panretin}{PMID: 22040192}, Frederick J Stoddard et al., 2011]

\hypertarget{pmid_32928263}{T}his study aimed to explore the efficacy and safety of pantethine in children with pantothenate kinase-associated neurodegeneration (PKAN). A single-arm, open-label study was conducted. All subjects received pantethine during the 24-week period of treatment. The primary endpoints were change of the Unified Parkinson's Disease Rating Scale (UPDRS) I-III and Fahn-Marsden (FM) score from baseline to week 24 after treatment. Fifteen children with PKAN were enrolled, and all patients completed the study. After 24 weeks of treatment with pantethine at 60 mg/kg per day, there was no difference in either UPDRS I-III (t = 0.516, P = 0.614) or FM score (t = 0.353, P = 0.729) between the baseline and W24. Whereas the rates of increase in UPDRS I-III (Z = 2.614, p = 0.009) and FM scores (Z = 2.643, p = 0.008) were slowed. Four patients (26.7\%) were evaluated as "slightly improved" by doctors through blinded video assessment. Patients with lower baseline UPDRS I-III or FM scores were more likely to be improved. The quality of life of family members improved after pantethine treatment, evaluated by PedsQL TM 2.0 FIM scores, whereas the quality of life of the patients was unchanged at W24, evaluated by PedsQL TM 4.0 and PedsQL TM 3.0 NMM. Serum level of CoA was comparable between baseline and W24. There was no drug related adverse event during the study. Pantethine could not significantly improve motor function in children with PKAN after 24 weeks treatment, but it may delay the progression of motor dysfunction in our study. Pantethine was well-tolerated at 60 mg/kg per day. Clinical trial registration number at www.chictr.org.cn :ChiCTR1900021076, Registered 27 January2019, the first participant was enrolled 30 September 2018, and other 14 participants were enrolled after the trial was registered. [\hyperlink{Panretin}{PMID: 32928263}, Xuting Chang et al., 2020]

\hypertarget{pmid_29098909}{T}here is a few evidence-based information regarding the efficacy and safety of acitretin treatment in children with pustular psoriasis (PP). This study aimed to provide an additional evidence for this field. A retrospective study was undertaken for 15 children with PP who received acitretin in doses of 0.6-1.0 mg/kg/day for 4-6 weeks, the transition dose of 0.2-0.4 mg/kg/day for 4-6 weeks and maintenance dose of 0.2-0.3 mg/kg/day. Additionally, a literature review on this topic is conducted. Of 15 children with generalized PP (GPP, n = 10), palmoplantar psoriasis (PPP, n = 3), and acrodermatitis continua of Hallopeau (ACH, n = 2), 93.3\% (14/15) showed good response, only one case with ACH exhibited moderate response. During the 10-32 months of follow-up, acitrerin monotherapy for children cases with PP overall showed good efficacy and safety. In the literature review, a total of 107 childhood PP cases treated with acitretin in 21 studies were included in the analysis. The clinical effectiveness was obtained in 88.8\% (95/107) patients treated with acitretin as monotherapy or combination therapy, and most of cases (92.6\%, 100/107) treated by acitretin did not report side effects during the treatment and follow-up of acitretin. This study is just included a small sample sizes and no standardized studies were used in the literature. Acitretin therapy for children with PP (monotherapy or combination therapy), all showed a satisfactory therapeutic effect and safety, independent of the short or long-tern therapeutic procedures. [\hyperlink{Panretin}{PMID: 29098909}, Pingjiao Chen et al., 2018]

\hypertarget{pmid_9085807}{T}he results of this study show that postoperative patient-controlled pain therapy in children with piritramide is - in a similar way as with adults - a safe method involving a low incidence of side effects. A special pump parameter setting is required with larger bolus dose sizes and longer lockout intervals, not very different from the experience gained with adults, and which is based on other values than those recommended up to now with morphine for paediatric PCA. Side effects were rarely observed. The fear of respiratory depression constitutes no rational reason to deny the younger patients this form of analgesia provided that monitoring is guaranteed. [\hyperlink{Panretin}{PMID: 9085807}, G Petrat et al., 1997]

\hypertarget{pmid_22052632}{O}ctreotide is a synthetic somatostatin analogue which has been suggested for use in the management of acute pancreatitis, though its safety and effectiveness in the pediatric setting has not been extensively studied. we present a rare case of a 6.5-year-old female with acute lymphoblastic leukemia (ALL) and L-asparaginase (L-asp) induced pancreatitis, who developed epileptic seizures, possibly associated with octreotide administration. Her imaging and laboratory findings ruled out a leukemic involvement or infection of CNS. The EEG revealed repetitive sharp waves maximal on the frontal and temporal areas of the right hemisphere. The child was treated with diazepam and she continued with systemic anticonvulsant treatment with levetiracetam. After 2 weeks of conservative treatment, pancreatitis resolved and she continued her chemotherapy protocol. Levetiracetam treatment lasted 8 months. 7 months after the first episode, EEG was reported as normal, and the child completed the chemotherapy protocol without any further severe complications. Larger and well designed studies are needed to warrant the safety of octreotide in pediatric population. [\hyperlink{Panretin}{PMID: 22052632}, E Hatzipantelis et al., 2011]

\hypertarget{pmid_11673582}{I}nfantile spasms are a rare but devastating pediatric epilepsy that, outside the United States, is often treated with vigabatrin. The authors evaluated the efficacy and safety of vigabatrin in children with recent-onset infantile spasms. This 2-week, randomized, single-masked, multicenter study with a 3- year, open-label, dose-ranging follow-up study included patients who were younger than 2 years of age, had a diagnosed duration of infantile spasms of no more than 3 months, and had not previously been treated with adrenocorticotropic hormone, prednisone, or valproic acid. Patients were randomly assigned to receive low-dose (18-36 mg/kg/day) or high-dose (100-148 mg/kg/day) vigabatrin. Treatment responders were those who were free of infantile spasm for 7 consecutive days beginning within the first 14 days of vigabatrin therapy. Time to response to therapy was evaluated during the first 3 months, and safety was evaluated for the entire study period. Overall, 32 of 142 patients who were able to be evaluated for efficacy were treatment responders (8/75 receiving low-dose vigabatrin vs 24/67 receiving high doses, p < 0.001). Response increased dramatically after approximately 2 weeks of vigabatrin therapy and continued to increase over the 3-month follow-up period. Time to response was shorter in those receiving high-dose versus low-dose vigabatrin (p = 0.04) and in those with tuberous sclerosis versus other etiologies (p < 0.001). Vigabatrin was well tolerated and safe; only nine patients discontinued therapy because of adverse events. These results confirm previous reports of the efficacy and safety of vigabatrin in patients with infantile spasms, particularly among those with spasms secondary to tuberous sclerosis. [\hyperlink{Panretin}{PMID: 11673582}, R D Elterman et al., 2001]

\hypertarget{pmid_23515245}{T}o describe the rationale, design and first data from PATRO Children, a postmarketing surveillance of the long-term efficacy and safety of somatropin (Omnitrope(®)) for the treatment of children requiring growth hormone treatment. PATRO Children is a multicentre, open, longitudinal, noninterventional study being conducted in children's hospitals and specialised endocrinology clinics. The primary objective is to assess the long-term safety of Omnitrope(®) in routine clinical practice. Eligible patients are infants, children and adolescents (male or female) who are receiving treatment with Omnitrope(®) and who have provided informed consent. Patients who have been treated with another recombinant human growth hormone (rhGH) product before starting Omnitrope(®) are eligible for inclusion. All adverse events (AEs) are monitored and recorded, with particular emphasis on: long-term safety; the recording of malignancies; the occurrence and clinical impact of anti-hGH antibodies; the development of diabetes during Omnitrope(®) treatment in children short for gestational age (SGA); safety issues in patients with Prader-Willi syndrome (PWS). Efficacy assessments include auxological parameters, plus insulin-like growth factor-1 and insulin-like growth factor binding protein-3. As of September 2012, 1837 patients were enrolled in the study from 184 sites in 10 European countries. To date, efficacy data are reassuring and consistent with previous studies. In addition, there have been no confirmed cases of diabetes occurring under Omnitrope(®) treatment, no reports of malignancy and no safety issues in PWS patients. The efficacy and safety profile of Omnitrope(®) in the PATRO Children study so far are as expected. The ongoing study will extend the safety database for Omnitrope(®), and rhGH products more generally, in paediatric indications. Of particular interest, PATRO Children will add important information on the diabetogenic potential of rhGH in children born SGA, the risk of malignancies in children receiving rhGH, and AEs with a possible causal relationship to rhGH treatment in children with PWS. [\hyperlink{Panretin}{PMID: 23515245}, Roland Pfäffle et al., 2013]

\hypertarget{pmid_25023977}{I}n spite of the high occurrence of migraine headaches in school-age children, there are currently no approved and widely accepted pharmacologic agents for migraine prophylaxis in children. Our previous open-label study in children revealed the efficacy of cinnarizine, a calcium channel blocker, in migraine prophylaxis. A placebo-controlled trial was conducted to demonstrate the efficacy and safety of cinnarizine in the prophylaxis of migraine in children. A double-blind, placebo-controlled, parallel-group study conducted in a tertiary medical center in Tehran, Iran. Children (5-17 years) who experienced migraines with and without aura, as defined on the basis of 2004 International Headache Society criteria, were recruited into the study. Children were excluded if they had complicated migraine, epilepsy, or a history of use of migraine prophylactic agents. Each participant was randomly assigned to receive cinnarizine (a single 1.5 mg/kg/day dose in children weighing less than 30 kg and a single 50 mg dose in children weighing more than 30 kg, administered at bedtime) or placebo. The frequency, severity, and duration of headaches over the trial period were assessed and adverse effects were monitored. A total of 68 children (34 in each group) with migraine were enrolled and 62 participants completed the study. After 3 months of taking cinnarizine or placebo, children in both groups experienced significantly reduced frequency, severity, and duration of headaches compared with baseline measurements (P < 0.001). However, compared with 31.3\% of children in the placebo group, 60\% of children in the cinnarizine group reported more than 50\% reduction in monthly headache frequency (P = 0.023), suggesting that cinnarizine was significantly more effective than placebo in reducing the frequency of headaches. No serious adverse effects of the medications were observed in the treated children, including no abnormal weight gain or extrapyramidal signs. Our results indicate that the use of cinnarizine at doses administered in this study is effective and safe for prophylaxis of migraine headaches in children. [\hyperlink{Panretin}{PMID: 25023977}, Mahmoud Reza Ashrafi et al., 2014]

\hypertarget{pmid_1098353}{T}he effect of a combined pediatric preparation of brompheniramine maleate and phenylpropanolaminehydrochloride (Lunerin mite) was studied in 17 children (aged 4-14 years) with allergic rhinitis. A double-blind cross-over technique was used, and a score system was used for evaluation of symptoms and signs. The results show that the drug had a good effect with statistical differences on the 1 per cent level between the active preparation and the placebo. The effect was less pronounced in children aged more than 10 years, for whom a higher dose of the drug is recommended. The frequency of side-effects was negligible. [\hyperlink{Panretin}{PMID: 1098353}, M Kjellman et al., 1975]

\hypertarget{pmid_31982581}{A} recent 3-month double-blind, placebo-controlled study demonstrated efficacy and safety of pediatric prolonged-release melatonin (PedPRM) for insomnia in children with autism spectrum disorder. This study examined the long-term effects of PedPRM treatment on sleep, growth, body mass index, and pubertal development. Eighty children and adolescents (2-17.5 years of age; 96\% with autism spectrum disorder) who completed the double-blind, placebo-controlled trial were given 2 mg, 5 mg, or 10 mg PedPRM nightly up to 104 weeks, followed by a 2-week placebo period to assess withdrawal effects. Improvements in child sleep disturbance and caregiver satisfaction with child sleep patterns, quality of sleep, and quality of life were maintained throughout the 104-week treatment period (p < .001 versus baseline for all). During the 2-week withdrawal placebo period, measures declined compared with the treatment period but were still improved compared with baseline. PedPRM was generally safe; the most frequent treatment-related adverse events were fatigue (6.3\%), somnolence (6.3\%), and mood swings (4.2\%). Changes in mean weight, height, body mass index, and pubertal status (Tanner staging done by a physician) were within normal ranges for age with no evidence of delay in body mass index or pubertal development. Nightly PedPRM at optimal dose (2, 5, or 10 mg nightly) is safe and effective for long-term treatment in children and adolescents with autism spectrum disorder and insomnia. There were no observed detrimental effects on children's growth and pubertal development and no withdrawal or safety issues related to the use or discontinuation of the drug. Efficacy and Safety of Circadin in the Treatment of Sleep Disturbances in Children With Neurodevelopment Disabilities; https://clinicaltrials.gov/; NCT01906866. [\hyperlink{Panretin}{PMID: 31982581}, Beth A Malow et al., 2021]

\hypertarget{pmid_26719728}{S}ildenafil is a phosphodiesterase type-5 inhibitor approved for treatment of pulmonary arterial hypertension (PAH) in adults. Data from pediatric trials demonstrate a similar acute safety profile to the adult population but have raised concerns regarding the safety of long-term use in children. Interpretation of these trials remains controversial with major regulatory agencies differing in their recommendations - the US Food and Drug Administration recommends against the use of sildenafil for treatment of PAH in children, while the European Medicines Agency supports its use at "low doses". Here, we review the available pediatric data regarding dosing, acute, and long-term safety and efficacy of sildenafil for the treatment of PAH in children.  [\hyperlink{Panretin}{PMID: 26719728}, Andrew L Dodgen et al., 2015] This preliminary study examines the effectiveness and safety of selective serotonin reuptake inhibitors (SSRIs) for the treatment of panic disorder in children and adolescents. In a prospective open label study, 12 children and adolescents with panic disorder were treated with SSRIs, and if necessary, with benzodiazepines, for a period of 6-8 weeks and were followed for approximately 6 months. During the trial, clinician-based and self-report rating scales for anxiety and depression, functioning, and side effects, were administered. Using the Clinical Global Impression Scale (CGIS) 75\% of patients showed much to very much improvement with SSRIs without experiencing significant side effects. After controlling for changes in depressive symptoms, self-report and clinician-based anxiety scales also showed significant improvement. At the end of the trial, 67\% of patients no longer fulfilled criteria for panic disorder and 4 patients remained with significant residual symptoms. In conclusion, SSRIs appear to be a safe and promising for the treatment of children and adolescents with panic disorder, however, randomized controlled trials evaluating the effects of SSRIs and other interventions (e.g., cognitive therapy) for treating panic disorder in children and adolescents are warranted. It appears that until the SSRIs begin to exert their effects, a benzodiazepine adjunct treatment might be helpful for patients with severe panic disorder. [\hyperlink{Panretin}{PMID: 26719728}, J Renaud et al., 1999]

\hypertarget{pmid_10588965}{S}ecretin is a peptide hormone that stimulates pancreatic secretion. After recent publicity about a child with autism whose condition markedly improved after a single dose of secretin, thousands of children with autistic disorders may have received secretin injections. We conducted a double-blind, placebo-controlled trial of a single intravenous dose of synthetic human secretin in 60 children (age, 3 to 14 years) with autism or pervasive developmental disorder. The children were randomly assigned to treatment with an intravenous infusion of synthetic human secretin (0.4 microg per kilogram of body weight) or saline placebo. We used standardized behavioral measures of the primary and secondary features of autism, including the Autism Behavior Checklist, to assess the degree of impairment at base line and over the course of a four-week period after treatment. Of the 60 children, 4 could not be evaluated - 2 received secretin outside the study, and 2 did not return for follow-up. Thus, 56 children (28 in each group) completed the study. As compared with placebo, secretin treatment was not associated with significant improvements in any of the outcome measures. Among the children in the secretin group, the mean total score on the Autism Behavior Checklist at base line was 59.0 (range of possible values, 0 to 158, with a larger value corresponding to greater impairment), and among those in the placebo group it was 63.2. The mean decreases in scores over the four-week period were 8.9 in the secretin group and 17.8 in the placebo group (mean difference, -8.9; 95 percent confidence interval, -19.4 to 1.6; P=0.11). None of the children had treatment-limiting adverse effects. After they were told the results, 69 percent of the parents of the children in this study said they remained interested in secretin as a treatment for their children. A single dose of synthetic human secretin is not an effective treatment for autism or pervasive developmental disorder. [\hyperlink{Panretin}{PMID: 10588965}, A D Sandler et al., 1999]

\hypertarget{pmid_11436954}{P}aroxetine has repeatedly been shown to be effective in the treatment of panic disorder (PD) in adults, and, according to previous case observations, it may be useful in treating children and adolescents with PD as well. This preliminary naturalistic study examines effectiveness and safety of paroxetine in the treatment of children and adolescents with PD. A chart review was conducted on 18 patients with Diagnostic and Statistical Manual of Mental Disorders PD admitted to the Division of Child Neurology and Psychiatry and to the Department of Psychiatry at the University of Pisa. Paroxetine was given at an initial mean dosage of 8.9 +/- 2.1 mg/day and was gradually increased up to 40 mg/day, depending on clinical response and side effects. Clinical status was assessed with the Clinical Global Impression (CGI) and adverse effects were assessed retrospectively at each visit. Patients with final CGI-Improvement scores of 1 or 2 were considered responders. Mean paroxetine treatment duration was 11.7 +/- 8.3 months, with a mean final dosage of 23.9 +/- 9.8 mg/day (range, 10-40 mg/day). No patient had to interrupt the treatment because of side effects. Fifteen patients (83.3\%) were considered responders. The mean change on the CGI-Severity scale was statistically significant (p < 0.0001). Paroxetine was well tolerated and effective in the treatment of PD in these children and adolescents. [\hyperlink{Panretin}{PMID: 11436954}, G Masi et al., 2001]

\hypertarget{pmid_1737249}{I}njectable polytetrafluoroethylene (Polytef) has been used in many patients for vocal cord augmentation and for the management of urinary incontinence since the early 1960s and non-injectable forms have been used for sutures, hernia repair, replacement of the stapes, hip prostheses, cardiac valves and vascular grafts. Since 1984, many children have been treated with subureteric Polytef injection for the management of vesicoureteric reflex. Its use in young patients has heightened the concern about particle migration and carcinogenesis, particularly in view of the fact that the substance may be in the patient for decades. The available evidence does not confirm a significant carcinogenic effect in humans; rather it suggests that, if there is a risk, it is extremely low. However, human specimens, taken decades after the implantation of Polytef, and long-term, non-rodent animal experiments are needed to substantiate the probable safety of Polytef in children. [\hyperlink{Panretin}{PMID: 1737249}, P A Dewan et al., 1992]

\hypertarget{pmid_34021341}{P}alonosetron has demonstrated non-inferiority to ondansetron for prevention of chemotherapy-induced nausea and vomiting in pediatric patients in the United States and Europe. We conducted a single-arm registration study to evaluate the efficacy, safety and pharmacokinetics of palonosetron in pediatric patients in Japan. Key inclusion criteria were age of 28 days to 18 years and malignant disease for which initial highly emetogenic chemotherapy or moderately emetogenic chemotherapy was planned. Patients received palonosetron at 20 μg/kg over at least 30 s intravenously before the start of highly emetogenic chemotherapy or moderately emetogenic chemotherapy and received dexamethasone on Days 1-3. The primary endpoint was the proportion of patients achieving a complete response in the overall phase (0-120 h) in Course 1, and its threshold was set at 30\%. From December 2016 to June 2019, 60 patients were enrolled, and 58 received at least one dose of palonosetron. The proportion of patients achieving a complete response during the overall phase was 58.6\% (95\% confidence interval, 44.9\%-71.4\%), showing the primary endpoint was met (P < 0.0001). Treatment-related adverse events occurred in two patients (3.4\%). Regarding the pharmacokinetics of palonosetron, neither the plasma concentration immediately after administration nor the area under the plasma concentration-time curve from time 0 to infinity differed significantly among the age groups. We demonstrated the efficacy of palonosetron in pediatric patients receiving highly emetogenic chemotherapy or moderately emetogenic chemotherapy and confirmed the appropriateness of the 20 μg/kg dose, regardless of age, considering the safety and pharmacokinetic profiles. JapicCTI-163305, registered 6 June 2016. [\hyperlink{Panretin}{PMID: 34021341}, Junichi Hara et al., 2021]

\hypertarget{pmid_14566209}{T}his article provides an overview of the use of paroxetine in the treatment of mood and anxiety disorders in children and adolescents. Although not currently approved for use in patients younger than 18 years of age, the efficacy and safety of paroxetine have been studied in several pediatric mood and anxiety disorders. The epidemiology, diagnosis, and course of major depression, obsessive-compulsive disorder, social anxiety disorder, and panic disorder are discussed briefly. Current available data on the safety and efficacy of paroxetine based on double-blind, placebo-controlled trials and open-label studies for the treatment of mood and anxiety disorders in children and adolescents are reviewed. Clinical guidelines for the use of paroxetine in children and adolescents and recommendations regarding future directions of study are discussed. [\hyperlink{Panretin}{PMID: 14566209}, Karen Dineen Wagner et al., 2003]

\hypertarget{pmid_20944041}{I} frequently see children with scabies in my practice. A variety of medications are available to treat scabies. Permethrin is one of the most common medications used. Is permethrin a safe and effective option for children? Scabies is a common parasitic skin infection. It is highly prevalent in young children. Topical permethrin (5\% cream) is a safe and effective scabicide in children. It is recommended as a first-line therapy for patients older than 2 months of age. Because there are theoretical concerns regarding percutaneous absorption of permethrin in infants younger than 2 months of age, guidelines recommend 7\% sulfur preparation instead of permethrin. [\hyperlink{Panretin}{PMID: 20944041}, Lina Albakri et al., 2010]

\hypertarget{pmid_3002411}{T}he study was designed to determine whether the onset of action of pancuronium in infants and children could be accelerated by its administration in divided doses. Sixty paediatric patients (0-1 yr (n = 20); 1-3 yr (n = 20); 3-10 yr (n = 20)) were studied during nitrous oxide-oxygen-halothane anaesthesia using train-of-four stimulation, and the results were compared with data obtained previously in adults. The time to onset correlated with the patient's age, and an additional, small acceleration was produced following divided doses which did not alter the duration or pattern of neuromuscular blockade. It was concluded that, in children, divided doses of pancuronium are unlikely to offer important clinical advantages. [\hyperlink{Panretin}{PMID: 3002411}, J C Bevan et al., 1985]

\hypertarget{pmid_11729017}{T}he study compared the safety and efficacy of sertraline, a selective serotonin reuptake inhibitor, and placebo in the treatment of generalized anxiety disorder in children and adolescents. The study subjects were 22 children and adolescents age 5-17 years who met the DSM-IV criteria for generalized anxiety disorder according to the Anxiety Disorders Interview Schedule for Children-Revised and who had a Hamilton Anxiety Rating Scale score > or = 16. The patients underwent a 2-3-week prestudy evaluation period, followed by a 9-week double-blind treatment phase in which they were randomly assigned in blocks of four to receive either sertraline or pill placebo. The maximum dose of sertraline was 50 mg/day. Primary outcome measures were the Hamilton anxiety scale and the Clinical Global Impression scale. The Hamilton anxiety scale total score, psychic factor, and somatic factor and the Clinical Global Impression severity and improvement scales showed significant differences with treatment in favor of sertraline over placebo beginning at week 4. Self-report measures reflected these results at the end of treatment. The results of this double-blind, placebo-controlled trial suggest that sertraline at the daily dose of 50 mg is safe and efficacious for the treatment of generalized anxiety disorder in children and adolescents. [\hyperlink{Panretin}{PMID: 11729017}, M A Rynn et al., 2001]

\hypertarget{pmid_10452767}{I}n very young children, H(1 )-receptor antagonists have not been adequately studied, although they are widely used and assumed to be safe. Our objective was to test the hypothesis that cetirizine would be as safe as placebo for long-term use in this population. In the prospective, double-blind, parallel-group, 18-month-long Early Treatment of the Atopic Child (ETAC) study, 817 children with atopic dermatitis who were 12 to 24 months old at study entry were randomized to receive either cetirizine 0.25 mg/kg or placebo twice daily. Safety was assessed by using the reports of all adverse events, diary cards, physical examinations, developmental assessments, electrocardiograms, blood hematology and chemistry tests, and urinalyses. The population evaluated for safety consisted of 399 children receiving cetirizine and 396 children receiving placebo. Drop-outs and serious events, including hospitalizations, occurred infrequently and were less common in the children receiving cetirizine than in those children receiving placebo, although the differences were not statistically significant. Most reported symptoms and events were mild and were attributed to intercurrent respiratory or gastrointestinal infections, exacerbations of allergic disorders, or age-related concerns rather than to medication-related adverse effects. There were no clinically relevant differences between the groups for neurologic or cardiovascular symptoms or events, growth, behavioral or developmental assessments, laboratory test results, or electrocardiograms, and no child receiving cetirizine therapy had prolongation of the QTc interval. The safety of cetirizine has been confirmed in this prospective study, the largest and longest randomized, double-blind, placebo-controlled safety investigation of any H(1 )-antagonist ever conducted in children and the longest prospective safety study of any H(1 )-antagonist ever conducted in any age group. [\hyperlink{Panretin}{PMID: 10452767}, F E Simons et al., 1999]

\hypertarget{pmid_6114415}{T}he tubeless for diagnosing exocrine pancreatic insufficiency using Fluorescein-Dilaurate was first applied to children of three years and younger, and to children with malabsorption and/or maldigestion. Fluorescein-Dilaurate in this test is an indication for the excretion of the pancreas-specific arylesterases. The dosage and the amount of liquid to be drunk by adults were reduced. It was shown that - applied to children of more than three years- reliability and validity of the test were not perceptibly influenced by the practise modifications and by the existing malabsorption/maldigestion. The applicability of this method to children of three years and younger depends on the observance of certain rules. Thereby, the test becomes sufficiently practicable also in children of three years and younger. [\hyperlink{Panretin}{PMID: 6114415}, P Neis et al., 1981]

\hypertarget{pmid_17561929}{T}here are more than 40 H(1)-antihistamines available worldwide. Most of these medications have never been optimally studied in prospective, randomized, double-masked, placebo-controlled trials in children. The aim was to perform a long-term study of levocetirizine safety in young atopic children. In the randomized, double-masked Early Prevention of Asthma in Atopic Children Study, 510 atopic children who were age 12-24 months at entry received either levocetirizine 0.125 mg/kg or placebo twice daily for 18 months. Safety was assessed by: reporting of adverse events, numbers of children discontinuing the study because of adverse events, height and body mass measurements, assessment of developmental milestones, and hematology and biochemistry tests. The population evaluated for safety consisted of 255 children given levocetirizine and 255 children given placebo. The treatment groups were similar demographically, and with regard to number of children with: one or more adverse events (levocetirizine, 96.9\%; placebo, 95.7\%); serious adverse events (levocetirizine, 12.2\%; placebo, 14.5\%); medication-attributed adverse events (levocetirizine, 5.1\%; placebo, 6.3\%); and adverse events that led to permanent discontinuation of study medication (levocetirizine, 2.0\%; placebo, 1.2\%). The most frequent adverse events related to: upper respiratory tract infections, transient gastroenteritis symptoms, or exacerbations of allergic diseases. There were no significant differences between the treatment groups in height, mass, attainment of developmental milestones, and hematology and biochemistry tests. The long-term safety of levocetirizine has been confirmed in young atopic children. [\hyperlink{Panretin}{PMID: 17561929}, F Estelle R Simons et al., 2007]

\hypertarget{pmid_15039684}{W}e determine the efficacy of prophylactic phenytoin in preventing early posttraumatic seizures in children with moderate to severe blunt head injury. Children younger than 16 years and experiencing moderate to severe blunt head injury were randomized to receive phenytoin or placebo within 60 minutes of presentation at 3 pediatric trauma centers. The primary endpoint was posttraumatic seizures within 48 hours; secondary endpoints were survival and neurologic outcome 30 days after injury. A Bayesian decision-theoretic clinical trial design was used to determine the probability of remaining posttraumatic seizure free for each treatment group. One hundred two patients were enrolled, with a median age of 6.1 years. Sixty-eight percent were boys. The 2 treatment groups were well matched. During the 48-hour observation period, 3 (7\%) of 46 patients given phenytoin and 3 (5\%) of 56 patients given placebo experienced a posttraumatic seizure. There were no significant differences between the treatment groups in survival or neurologic outcome after 30 days. According to these results, the probability that phenytoin has the originally hypothesized effect of reducing the rate of early posttraumatic seizures by 12.5\% is 0.0053. The probability that phenytoin has any prophylactic efficacy is 0.383. The median effect size in this trial was -0.015 (seizure rate increased by 1.5\% in the phenytoin group), 95\% probability interval -0.127 to 0.091 (12.7\% higher rate of posttraumatic seizures to a 9.1\% lower rate of posttraumatic seizures with phenytoin). The rate of early posttraumatic seizures in children may be much lower than previously reported. Phenytoin did not substantially reduce that rate. [\hyperlink{Panretin}{PMID: 15039684}, Kelly D Young et al., 2004]

\hypertarget{pmid_34784012}{T}ransdermal fentanyl is not yet approved for pediatric and adolescent use in Japan. Serum fentanyl concentration and the safety and efficacy of once-a-day fentanyl citrate patch were investigated in pediatric and adolescent patients with cancer pain. In this open-label, uncontrolled study, cancer patients aged 2-19 years being treated with strong opioid analgesics were switched to fentanyl citrate patch for 2 weeks. Serum fentanyl concentration was measured at steady state, and severity of pain was evaluated. Eleven patients (four patients aged 2-5 years and seven patients aged 6-19 years) were enrolled. No patient received a dose exceeding 2 mg. Mean serum fentanyl concentrations after administration of 0.5 mg, 1 mg, and 2 mg were 144 pg/mL (n = 4), 277 pg/mL (n = 3), and 2070 pg/mL (n = 4), respectively. All patients were included in the efficacy and safety analysis, but one patient was excluded from the pharmacokinetic analysis because blood was sampled on the day after blood transfusion. A subgroup analysis showed that the mean serum fentanyl concentration tended to be higher in pre-school patients (aged 2-5 years) than in school-aged and adolescent patients (aged 6-19 years) and than in reports of adult patients (aged 20 years and above) who received the same dose. No respiratory adverse events were observed, and pain was well controlled. Fentanyl citrate patch tended to result in a higher serum fentanyl concentration in pre-school patients than in school-aged, adolescent, and adult patients who received the same dose. The patch provided adequate pain control, was well tolerated, and did not cause respiratory adverse events. JPRN-JapicCTI-183909. [\hyperlink{Panretin}{PMID: 34784012}, Eiso Hiyama et al., 2021]

\section*{iDose TR}
\subsection*{Result}
\subsubsection*{Answer}

Unknown

\subsubsection*{{Explanation}}
\hypertarget{iDose TR}
A review of the provided abstracts reveals that none of them are about iDose TR. Therefore, there is not enough data available in the abstracts to determine whether iDose TR is safe or unsafe for use in children of any age range. No targeted studies on the safety of iDose TR in children are present in the abstracts, and thus the safety profile for this medication in pediatric populations is unknown based on the available evidence.

\subsection*{Abstracts}
\hypertarget{pmid_17718820}{I}ntrathecal baclofen (ITB) therapy can be very effective in the treatment of intractable spasticity, but its effectiveness and safety have not yet been thoroughly studied in children with cerebral palsy (CP). The aims of this double-blind, randomized, placebo-controlled, dose-finding study were to select children eligible for continuous ITB infusion, to assess the effective ITB bolus dose, and to evaluate the effects, side effects, and complications. Outcome measures included the original Ashworth scale and the Visual Analogue Scale for individually formulated problems. We studied nine females and eight males, aged between 7 and 16 years (mean age 13y 2mo [SD 2y 9mo]). Twelve children had spastic CP and five had spastic-dyskinetic CP. One child was classified on the Gross Motor Function Classification System at Level III, two at Level I V, and 14 at Level V. The test treatment worked successfully for all 17 children with an effective ITB bolus dose of 12.5 microg in one, 20 microg in another, 25 microg in 10, and 50 microg in five children. ITB significantly reduced muscle tone, diminished pain, and facilitated ease of care. The placebo did not have these effects. Nine side effects of ITB were registered, including slight lethargy in seven children. Fourteen children had symptoms of lowered cerebrospinal fluid pressure. We conclude that ITB bolus administration is effective and safe for carefully selected children with intractable spastic CP. [\hyperlink{iDose TR}{PMID: 17718820}, Marjanke A Hoving et al., 2007]

\hypertarget{pmid_29046203}{T}o investigate the clinical effect and safety of anti-D immunoglobulin (anti-D) in the treatment of children with newly diagnosed acute immune thrombocytopenia (ITP) through a Meta analysis. PubMed, EMBASE, Cohrane Library, Ovid, CNKI, and Wanfang Data were searched for randomized controlled trials (RCTs) published up to April 2017. Review Manager 5.3 was used for the Meta analysis. Seven RCTs were included. The Meta analysis showed that after 72 hours and 7 days of treatment, the intravenous immunoglobulin (IVIG) group had a significantly higher percentage of children who achieved platelet count >20×10 Intravenous injection of anti-D may have a similar effect as IVIG in improving platelet count in children with acute ITP, but it may be slightly inferior to IVIG in the rate of platelet increase after treatment. The anti-D dose of 50 μg/kg may have a similar effect as 75 μg/kg. The recommended dose of anti-D for treatment of ITP is safe. [\hyperlink{iDose TR}{PMID: 29046203}, Wei Qin et al., 2017]

\hypertarget{pmid_35179098}{D}ata regarding immediate-release (IR)-tramadol exposures in children remain sparse. We aimed to investigate the incidence of IR-tramadol exposures in ≤6-year-old children, to describe the characteristics and resulting outcome of ingestions involving IR-tramadol alone, and to estimate a clinically relevant toxic dose in this population. Retrospective analysis of IR-tramadol exposures in ≤6-year-old children, collected by the French Poison Control Centers (PCCs) in 2003-2019. The incidence was estimated using IR-tramadol prescription data from the Health Improvement Network database (the French version of THIN). The Poison severity score (PSS) was used to grade severity. We found 1260 IR-tramadol exposures in ≤6-year-old children. The number of cases per 100,000 IR-tramadol-treated patients increased over time ( Despite increasing tramadol prescriptions in adults during the study period in France, oral exposure to IR-tramadol in ≤6-year-old children was rare but possibly responsible for severe toxicity. Children with no underlying disease and concomitant medication ingesting <7.4 mg/kg IR-tramadol alone could be observed at home. However, given the observed variability in the onset of seizures after tramadol ingestion, which can occur at ingested tramadol doses below 7.4 mg and even at therapeutic doses, parents or guardians should be specifically warned about the risk of seizures. [\hyperlink{iDose TR}{PMID: 35179098}, Weniko Caré et al., 2022]

\hypertarget{pmid_29601447}{I}traconazole is a broad-spectrum antifungal agent used for prophylaxis and treatment of fungal infections in immunocompromised children. Achieving the recommended target serum itraconazole trough concentration of ≥0.5 mg/L is challenging in children because of variation in itraconazole pharmacokinetics with age. We studied itraconazole use and treatment outcomes in a tertiary children's hospital. We did a 10-year retrospective review of medical records of children at the Royal Children's Hospital Melbourne who received oral itraconazole and had therapeutic drug monitoring (TDM). Overall, 81 children received 92 courses of oral itraconazole and had TDM. Of 222 TDM samples, 183 (82.4\%) were taken at the appropriate time (trough level at steady state). Patients ≤12 and >12 years of age required median doses of 6.2 and 3.9 mg/kg/d, respectively, to attain target trough levels (P < 0.001). Of children ≤12 years of age, 71.4\% required doses above the recommended dose of 5 mg/kg/d to achieve therapeutic levels, compared with 17.4\% of those >12 years of age. At least 1 subtherapeutic trough concentration was reported in 63 (76.8\%) courses; in only 18 (28.6\%) of these was the dose adjusted. Gastrointestinal symptoms [14/92 (15.2\%) courses] and hepatotoxicity [6/92 (6.5\%)] were the most frequent adverse events. Neither was associated with elevated trough levels. The poor attainment of target levels with current recommended dosing in children <12 years of age suggests that higher empiric doses are needed in this age group. The poor compliance with TDM guidelines highlights the need for better education about appropriate timing of sampling and dose adjustment. [\hyperlink{iDose TR}{PMID: 29601447}, Ying Hua Leong et al., 2019]

\hypertarget{pmid_21516022}{W}e conducted a study to evaluate the efficacy of intravenous (IV) anti-D against IV immunoglobulin (IVIG) in newly diagnosed immune thrombocytopenia (ITP) in children and to identify the clinical characteristics of the children most likely to benefit from one or the other treatment. Children (6 mo to 14 y) with newly diagnosed ITP and a platelet count <20,000/μL were treated either with a single bolus dose of 50 μg/kg IV anti-D or with 0.8 to 1 g/kg IVIG in a randomized manner. Twenty-five patients, mean age of 6.8 years, were treated either with IV anti-D (n=10) or with IVIG (n=15). Both drugs were equally efficient in raising the platelet count above 20,000/μL at 24 hours posttreatment. Children who presented with bleeding stage 1 or 2 (no mucosal bleeding) responded better to IVIG treatment, in terms of an increase in platelet count at 24 hours posttreatment (P=0.04). Hemoglobin drop was greater in the anti-D group (P=0.002). A single bolus dose of 50 μg/kg of IV anti-D is a safe and effective first-line treatment in newly diagnosed ITP in childhood and mucosal bleeding is a poor prognostic factor for treatment with IVIG. [\hyperlink{iDose TR}{PMID: 21516022}, Andromachi Papagianni et al., 2011]

\hypertarget{pmid_24565869}{U}se of inhaled tobramycin therapy for treatment of Pseudomonas aeruginosa infections in young children with cystic fibrosis (CF) is increasing. Safety data for pre-school children are sparse. The aim of this study was to assess the safety of tobramycin solution for inhalation (TOBI®-TSI) administered twice daily for 2 months/course concurrently to intravenous (IV) tobramycin during P. aeruginosa eradication therapy in children (0-5 years). Audiological assessment and estimation of glomerular filtration rate (GFR) was measured prior to any exposure and end of the study. Data were available from 142 patients who were either never exposed to aminoglycosides (n=39), exposed to IV aminoglycosides only (n=36) or exposed to IV+TSI (n=67). Median exposure to TSI was 113 days [59, 119]. Comparison of effects on audiometry results and GFR, showed no detectable difference between the groups. Use of TSI and IV tobramycin in pre-school children with CF was not associated with detectable renal toxicity or ototoxicity. [\hyperlink{iDose TR}{PMID: 24565869}, Stefanie Hennig et al., 2014]

\hypertarget{pmid_26903552}{C}oncerns have been raised of increased mortality risk in adulthood in certain patients who received growth hormone treatment during childhood. This study evaluated the safety of growth hormone treatment in childhood in everyday practice. NordiNet(®) International Outcome Study (IOS) is a noninterventional, observational study evaluating safety and effectiveness of Norditropin(®) (somatropin; Novo Nordisk A/S, Bagsvaerd, Denmark). Long-term safety data (1998-2013) were collected on 13 834 growth hormone treated pediatric patients with short stature. Incidence rates (IRs) of adverse events (AEs) defined as adverse drug reactions (ADRs), serious ADRs (SADRs), and serious AEs (SAEs) were calculated by mortality risk group (low/intermediate/high). The effect of growth hormone dose on IRs and the occurrence of cerebrovascular AEs were investigated by the risk group. We found that 61.0\% of patients were classified as low-risk, 33.9\% intermediate-risk, and 5.1\% high-risk. Three hundred and two AEs were reported in 261 (1.9\%) patients during a mean (s.d.) treatment duration of 3.9 (2.8) years. IRs were significantly higher in the high- vs the low-risk group (high risk vs low risk-ADR: 9.11 vs 3.14; SAE: 13.66 vs 1.85; SADR: 4.97 vs 0.73 events/1000 patient-years of exposure; P < 0.0001 for all). Except for SAEs in the intermediate-risk group (P = 0.0486) in which an inverse relationship was observed, no association between IRs and growth hormone dose was found. No cerebrovascular events were reported. We conclude that safety data from NordiNet(®) IOS do not reveal any new safety signals and confirm a favorable overall safety profile in accordance with other pediatric observational studies. No association between growth hormone dose and the incidence of AEs during growth hormone treatment in childhood was found. [\hyperlink{iDose TR}{PMID: 26903552}, Lars Sävendahl et al., 2016]

\hypertarget{pmid_21855592}{I}n order to compare the immunogenicity and safety of different doses of trivalent influenza vaccine (TIV) administered intradermallly (ID) with those evoked by a full dose of intramuscular (IM) virosomal-adjuvanted influenza vaccine (VA-TIV), 112 previously primed healthy children aged ≥ 3 years were randomised to receive 9 μg or 15 μg of each strain of ID-TIV, or a full IM dose (15 μg of each strain) of VA-TIV. The A/H1N1 and A/H3N2 seroconversion and seroprotection rates were ≥ 90\% and geometric mean titres (GMTs) increased 3.2-14.9 times without any statistically significant between-group differences; however, the seroconversion and seroprotection rates against the B strain were significantly higher in the children receiving either ID-TIV dose (p<0.05) without any differences between them. GMT against B virus was significantly higher in the children receiving the highest dose (p<0.05). Local reactions were significantly more common among the children receiving either ID-TIV dose (p<0.05), but systemic reactions were relatively uncommon in all three groups. Our findings suggest that ID-TIV with 15 μg of each viral antigen can confer a significant better protection against influenza than that obtained with the same dose of IM TIV in already primed children aged ≥ 3 years with an acceptable safety profile. The lower dose of ID-TIV needs further evaluation to analyze persistence of protection. [\hyperlink{iDose TR}{PMID: 21855592}, Susanna Esposito et al., 2011]

\hypertarget{pmid_33730099}{O}ral ivermectin is a safe broad spectrum anthelminthic used for treating several neglected tropical diseases (NTDs). Currently, ivermectin use is contraindicated in children weighing less than 15 kg, restricting access to this drug for the treatment of NTDs. Here we provide an updated systematic review of the literature and we conducted an individual-level patient data (IPD) meta-analysis describing the safety of ivermectin in children weighing less than 15 kg. A systematic review was conducted using the Preferred Reporting Items for Systematic Reviews and Meta-Analyses (PRISMA) for IPD guidelines by searching MEDLINE via PubMed, Web of Science, Ovid Embase, LILACS, Cochrane Database of Systematic Reviews, TOXLINE for all clinical trials, case series, case reports, and database entries for reports on the use of ivermectin in children weighing less than 15 kg that were published between 1 January 1980 to 25 October 2019. The protocol was registered in the International Prospective Register of Systematic Reviews (PROSPERO): CRD42017056515. A total of 3,730 publications were identified, 97 were selected for potential inclusion, but only 17 sources describing 15 studies met the minimum criteria which consisted of known weights of children less than 15 kg linked to possible adverse events, and provided comprehensive IPD. A total of 1,088 children weighing less than 15 kg were administered oral ivermectin for one of the following indications: scabies, mass drug administration for scabies control, crusted scabies, cutaneous larva migrans, myiasis, pthiriasis, strongyloidiasis, trichuriasis, and parasitic disease of unknown origin. Overall a total of 1.4\% (15/1,088) of children experienced 18 adverse events all of which were mild and self-limiting. No serious adverse events were reported. Existing limited data suggest that oral ivermectin in children weighing less than 15 kilograms is safe. Data from well-designed clinical trials are needed to provide further assurance. [\hyperlink{iDose TR}{PMID: 33730099}, Podjanee Jittamala et al., 2021]

\hypertarget{pmid_17195716}{W}e evaluated 19 children using 220-300 mg/m of indinavir (IDV) boosted with 100 mg ritonavir (RTV) (n = 12) or full-dose RTV (n = 7). Geometric mean (GM) (90\% confidence interval, CI) of IDV Ctrough in children who took IDV with 100 mg RTV (n = 12) was 0.17 (0.06-0.50) mg/L. For children who took IDV with full-dosage RTV, GM (90\% CI) was 0.40 (0.10-1.61) mg/L. C2hours were less than 10 mg/L in all subjects. Eighteen children had good virologic response. This report demonstrates that smaller IDV dosages given with RTV provide efficacious plasma concentrations and can be safely used. [\hyperlink{iDose TR}{PMID: 17195716}, Nottasorn Plipat et al., 2007]

\hypertarget{pmid_30046708}{T}he incidence of venous thromboembolism (VTE) in children has been increasing. Anticoagulants are the mainstay of treatment but are associated with bleeding events that may be life-threatening. Idarucizumab is a fragment antigen-binding (fab) that provides immediate, complete, and sustained reversal of dabigatran's anticoagulant effects in adults. This phase III, open-label, single-arm, multicenter, multinational trial will assess the safety of idarucizumab in children participating in two ongoing trials investigating dabigatran etexilate. Eligible patients will be children with VTE (aged 0-≤18 years; n = \textasciitilde{}5) with life-threatening or uncontrolled bleeding (group A), and children who require emergency surgery/urgent procedures for a condition other than bleeding (group B). Patients will receive idarucizumab up to 5 g as two consecutive intravenous infusions over 5-10 minutes each, as two 10-15-minute drips or as two bolus injections (15 minutes apart) and will be monitored for 30 days. The primary endpoint will be the safety of idarucizumab assessed by the occurrence of drug-related adverse events (including immune reactions) and all-cause mortality. Secondary endpoints will be the reversal of dabigatran anticoagulant effects assessed by changes in diluted thrombin time and ecarin clotting time, time to achieve complete reversal and the duration of the reversal and bleeding severity (group A). The formation of antidrug antibodies at 30 days post-dose and cessation of bleeding will also be assessed. This study will report the safety of idarucizumab in children with VTE who require rapid reversal of the anticoagulant effects of dabigatran. Clinical trial registration: NCT02815670. [\hyperlink{iDose TR}{PMID: 30046708}, Manuela Albisetti et al., 2018]

\hypertarget{pmid_23317951}{C}hildren's tuberculosis clinic, Houston, TX, United States. To determine the safety, adherence and efficacy of intermittent directly observed preventive therapy (DOPT). Retrospective cohort of children receiving intermittent DOPT for exposure to tuberculosis  (TB) or latent TB infection (LTBI) seen from 1989 to 2011 at one clinic. A total of 1383 children were treated for either TB exposure for 2-3 months (n = 935, 68\%) or LTBI for 9 months (n = 448, 32\%) with isoniazid 20-30 mg/kg/dose or rifampin 10-15  mg/kg/dose biweekly. All children with exposure and 411 (92\%) with LTBI were identified via contact investigations. Twelve (1.3\%) children with exposure experienced adverse effects (5 abdominal pain, 4 vomiting, 3 rash); 8 had transaminases evaluated and only 1 had elevated levels. Thirty  (6.7\%) children with LTBI experienced adverse effects (16 abdominal pain, 6 rash, 3 vomiting, 3 headache and 2 abdominal pain/vomiting); 19 had transaminases obtained and 2 had elevated transaminases. All transaminases normalized after the discontinuation of medication. Over 99\% of exposed  and 95.8\% of infected children completed treatment. One child, who had sickle cell anemia, was treated for LTBI and later developed TB disease. When compared to rates of disease progression by age, the efficacy of intermittent DOPT was 98\%. Intermittent DOPT in childhood TB  is safe, effective and offers high adherence rates. [\hyperlink{iDose TR}{PMID: 23317951}, A T Cruz et al., 2013]

\hypertarget{pmid_35106593}{T}he International Trachoma Initiative (ITI) provides azithromycin for mass drug administration (MDA) to eliminate trachoma as a public health problem. Azithromycin is given as tablets for adults and powder for oral suspension (POS) is recommended for children aged <7 y, children <120 cm in height (regardless of age) or anyone who reports difficulty in swallowing tablets. An observational assessment of MDA for trachoma was conducted to determine the frequency with which children aged 6 mo through 14 y received the recommended dose and form of azithromycin according to current dosing guidelines and to assess risk factors for choking and adverse swallowing events (ASEs). MDA was observed in three regions of Ethiopia and data were collected on azithromycin administration and ASEs. A total of 6477 azithromycin administrations were observed; 97.9\% of children received the exact recommended dose. Of children aged 6 mo to <7 y or <120 cm in height, 99.6\% received POS. One child experienced choking and 132 (2\%) experienced ≥1 ASEs. Factors significantly associated with ASEs included age 6-11 mo or 1-6 y, non-calm demeanor and requiring coaxing prior to drug administration. There is a high level of adherence to the revised azithromycin dosing guidelines and low incidence of choking and ASEs. [\hyperlink{iDose TR}{PMID: 35106593}, Allan M Ciciriello et al., 2022]

\hypertarget{pmid_32571538}{T}o evaluate the efficacy and safety of trigonal botulinum toxin A (BTX-A) injections for children with neurological detrusor overactivity (NDO) secondary to spinal cord injury (SCI). From February 2012 to December 2018, children with NDO secondary to SCI were enrolled. All patients received 200U BTX-A intradetrusor injections including the trigone. Videourodynamic study was performed at baseline and 12 weeks after injection. The primary outcome measures were the presence of vesicoureteral reflux and standardized urodynamic measures. Secondary outcomes included incontinence quality of life questionnaire (I-QoL), voiding volume, urinary incontinence episodes, and complete dryness. A total of 33 pediatric inpatients (28 male and 5 female) completed the study. No one developed VUR at week 0 or week 12. At the first instance of NDO, maximum detrusor pressure (PdetmaxFNDO) and NDO duration were reduced by 29.8\% and 31.8\%, respectively, whereas NDO volume (VFNDO) increased by 50.5\%,12 weeks after injection. Mean urinary incontinence episodes were reduced by 31.7\%, whereas voiding volume and I-QOL were increased by 52.9\% and 23.3\%. 3 patients reported mild transient hematuria during the first week after injection. Our results suggest that the use of bladder-trigone-including intradetrusor BTX-A injection does not induce VUR, and is safe and effective in children with NDO secondary to SCI. A prospective self-controlled trial LEVEL OF EVIDENCE: Level II. [\hyperlink{iDose TR}{PMID: 32571538}, Chen Hui et al., 2020]

\hypertarget{pmid_20096011}{T}his report documents our experience with intravenous immune globulin (IVIG) (1 g/kg, iv) and high-dose, anti-D immune globulin (anti-D) (75 microg/kg) as initial treatment for childhood immune thrombocytopenic purpura (ITP). The medical records of children diagnosed with ITP at a single institution between January 2003 and May 2008 were retrospectively reviewed. Participants received either IVIG or high-dose anti-D immune globulin as their initial treatment for ITP. For the 53 patients included for analysis, there was no statistical difference in efficacy between each group; however, patients who received anti-D experienced a higher rate of adverse drug reactions (ADRs), particularly chills and rigours, and 2 of 24 patients in the anti-D group developed severe anaemia requiring medical intervention. Patients who presented with mucosal bleeding had higher rates of treatment failure (32\%) compared to those who presented with dry purpura (6\%), regardless of treatment. Both IVIG and high-dose anti-D are effective first-line therapies for childhood ITP. However, we observed increased ADRs in the high-dose anti-D group in contrast to previously published reports. Further studies are needed to evaluate safety and premedications for high-dose anti-D and to determine the utility of using the presence of mucosal bleeding to predict treatment failure. [\hyperlink{iDose TR}{PMID: 20096011}, Ian Kane et al., 2010]

\hypertarget{pmid_17430480}{T}his single-centre, retrospective, observational pilot study was performed to evaluate the safety and efficacy of intravenous and oral itraconazole prophylaxis in paediatric haematopoietic stem cell transplantation (HCT). Study end-points were proven invasive fungal infection (IFI), survival, adverse reactions and graft-vs.-host disease (GVHD); 53 children and one young adult (median age 8.6 yr; range 0.4-18.3) transplanted between November 2001 and August 2004 were included in this study. Itraconazole was given intravenously from day +3 after HCT until oral medication became possible and continued until day +100 after HCT. Two proven new IFI in the itraconazole group (candidiasis, n = 1; aspergillosis, n = 1) were observed. After a median follow-up of 1.6 yr (0.3-6.1), six deaths (8\%) were seen; 24 patients (45\%) developed GVHD degree I-II, three children (6\%) had GVHD degree III-IV. In 11 of 53 patients (21\%), itraconazole prophylaxis was discontinued prematurely, mostly because of fever of unknown origin (n = 7). In total, 21 of 53 (40\%) of the children had abnormal results of laboratory investigations during the prophylaxis. The results of this pilot study indicate that itraconazole prophylaxis during HCT in children is feasible and safe, despite abnormal laboratory results. The efficacy in terms of prevention of IFI, however, has to be addressed in a prospective large-scale study. [\hyperlink{iDose TR}{PMID: 17430480}, L Grigull et al., 2007]

\hypertarget{pmid_34798685}{T}opical tacrolimus is used off-label in young children, but data are limited on its use in children under 2 years of age and for long-term treatment. To compare safety differences between topical tacrolimus (0.03\% and 0.1\% ointments) and topical corticosteroids (mild and moderate potency) in young children with atopic dermatitis (AD). We conducted a 36-month follow-up study with 152 young children aged 1-3 years with moderate to severe AD. The children were followed up prospectively, and data were collected on infections, disease severity, growth parameters, vaccination responses and other relevant laboratory tests were gathered. There were no significant differences between the treatment groups for skin-related infections (SRIs) (P = 0.20), non-SRIs (P = 0.20), growth parameters height (P = 0.60), body weight (P = 0.81), Eczema Area and Severity Index (EASI) (P = 0.19), vaccination responses (P = 0.62), serum cortisone levels (P = 0.23) or serum levels of interleukin (IL)-4, IL-10, IL-12, IL-31 and interferon-γ. EASI decreased significantly in both groups (P < 0.001). In the tacrolimus group, nine patients (11.68\%) had detectable tacrolimus blood concentrations at the 1-week visit. There were no malignancies or severe infections during the study, and blood eosinophil counts were similar in both groups. Topical tacrolimus (0.03\% and 0.1\%) and topical corticosteroids (mild and moderate potency) are safe to use in young children with moderate to severe AD, and have comparable efficacy and safety profiles. [\hyperlink{iDose TR}{PMID: 34798685}, A Salava et al., 2022]

\hypertarget{pmid_34350858}{P}ATRO Children is an international, observational, postmarketing surveillance study for a biosimilar recombinant human growth hormone (rhGH; somatropin, Omnitrope®; Sandoz), approved by the European Medicines Agency in 2006. We report safety and effectiveness data for patients with Turner syndrome (TS). The study population included infants, children, and adolescents with TS who received Omnitrope® treatment according to standard clinical practice. Adverse events (AEs) were monitored for safety evaluation, and height velocity (HV), height standard deviation score (HSDS), and HVSDS were calculated to evaluate treatment effectiveness. As of August 2019, 348 TS patients were enrolled from 130 centers. At baseline, 314 patients (90.2\%) were prepubertal and 284 patients (81.6\%) were rhGH treatment naïve. The mean (range) age at baseline was 9.0 (0.7-18.5) years, and mean (SD) treatment duration in the study was 38.5 (26.8) months. Overall, 170 patients (48.9\%) reported AEs, which were considered treatment related in 25 patients (7.2\%). One treatment-related serious AE was reported (intracranial hypertension). Mean ΔHSDS after 3 years of therapy was +1.17 in treatment-naïve prepubertal patients and +0.1 in pretreated prepubertal patients. In total, 51 patients (31.1\%) reached adult height (AH), 35 of whom were rhGH treatment naïve; in these patients, mean (SD) HSDS was -2.97 (1.03) at the start of Omnitrope® treatment, and they achieved a mean (SD) AHSDS of -2.02 (0.9). These data suggest that biosimilar rhGH is well tolerated and effective in TS patients managed in real-life clinical practice. Optimization of rhGH dose may contribute to a higher AH. [\hyperlink{iDose TR}{PMID: 34350858}, Philippe Backeljauw et al., 2021] mTOR inhibitors have activity in pediatric tumor models. A phase I trial of the mTOR inhibitor temsirolimus (TEM) with irinotecan (IRN) and temozolomide (TMZ) was conducted in children with recurrent/refractory solid tumors, including central nervous system (CNS) tumors. Escalating doses of intravenous (IV) TEM were administered on days 1 and 8 of 21-day cycles. IRN (50 mg/m(2)/dose escalated to a maximum of 90 mg/m(2)/dose) and TMZ (100 mg/m(2)/dose escalated to a maximum of 150 mg/m(2)/dose) were administered orally (PO) on days 1-5. When maximum tolerated doses (MTD) were identified, TEM frequency was increased to weekly. Seventy-one eligible pts (median age 10.9 years, range 1.0-21.5) with neuroblastoma (16), osteosarcoma (7), Ewing sarcoma (7), rhabdomyosarcoma (4), CNS (22) or other (15) tumors were enrolled. Dose-limiting hyperlipidemia occurred in two patients receiving oral corticosteroids. The protocol was subsequently amended to preclude chronic steroid use. The MTD was identified as TEM 35 mg/m(2) IV weekly, with IRN 90 mg/m(2) and TMZ 125 mg/m(2) PO on days 1-5. At higher dose levels, elevated serum alanine aminotransferase and triglycerides, anorexia, and thrombocytopenia were dose limiting. Additional ≥ grade 3 regimen-related toxicities included leukopenia, neutropenia, lymphopenia, anemia, and nausea/vomiting. Six patients had objective responses confirmed by central review; three of these had sustained responses through ≥ 14 cycles of therapy. The combination of TEM (35 mg/m(2)/dose IV weekly), IRN (90 mg/m(2)/dose days 1-5) and TMZ (125 mg/m(2)/dose days 1-5) administered PO every 21 days is well tolerated in children. Phase 2 trials of this combination are ongoing. [\hyperlink{iDose TR}{PMID: 34350858}, Rochelle Bagatell et al., 2014]

\hypertarget{pmid_8619355}{T}he purpose of this double-blind, randomized parallel trial was to evaluate and compare the clinical safety and the diagnostic efficacy of the new nonionic triiodinated contrast agent iobitridol (300 mg I/ml) with those of iohexol (300 mg I/ml) in CT examinations in children. Eighty children of either sex were included in the study. Of those, 40 patients received iobitridol, 40 iohexol. Mean volume injected i.v. was 1.8 ml/kg b.w. Adverse reactions occurring during 24 hours after the injection of the contrast agent were recorded. Image quality was judged good or excellent in all study examinations and all were diagnostically informative. There was no significant difference in clinical safety between the 2 groups and only adverse reactions of mild or moderate intensity were reported in both groups. Iobitridol appears to be a safe, well tolerated and effective contrast agent, when used for brain and body CT in children. [\hyperlink{iDose TR}{PMID: 8619355}, A Smets et al., 1996]

\hypertarget{pmid_29363561}{T}he traditional treatment of tuberculosis (TB) infection (9 months of daily isoniazid [9H]) is safe but completion rates of <50\% are reported. Shorter regimens (3 months of once-weekly isoniazid and rifapentine [3HP] or 4 months of daily rifampin [4R]) are associated with improved adherence in adults. This was a retrospective cohort study (2014-2017) of children (0-18 years old) seen at a children's TB clinic in a low-incidence nation. We compared the frequency of completion and adverse events (AEs) in children receiving 3HP, 4R, and 9H; the latter 2 regimens could be administered by families (termed self-administered therapy [SAT]) or as directly observed preventive therapy (DOPT); 3HP was always administered under DOPT. TB infection treatment was started in 667 children: 283 (42.4\%) 3HP, 252 (37.8\%) 9H, and 132 (19.8\%) 4R. Only 52\% of children receiving 9H via SAT completed therapy. Children receiving 3HP were more likely to complete therapy than the 9H (SAT) group (odds ratio [OR] 27.4, 95\% confidence interval [CI]: 11.8-63.7). Multivariate analyses found receipt of medication under DOPT (OR: 5.72, 95\% CI: 3.47-9.43), increasing age (OR: 1.09, 95\% CI: 1.02-1.17), and the absence of any AE (OR: 1.70, 95\% CI: 0.26-0.60) to be associated with completing therapy. AEs were more common in the 9H group (OR: 2.51, 95\% CI: 1.48-4.32). Two (0.9\%) children receiving 9H developed hepatotoxicity; no child receiving 3HP or 4R developed hepatotoxicity. Shorter regimens are associated with increased completion rates and fewer AEs than 9H. [\hyperlink{iDose TR}{PMID: 29363561}, Andrea T Cruz et al., 2018]

\hypertarget{pmid_32160320}{E}valuate technical success, tolerability, and safety of lidocaine iontophoresis and tympanostomy tube placement for children in an office setting. Prospective individual cohort study. This prospective multicenter study evaluated in-office tube placement in children ages 6 months through 12 years of age. Anesthesia was achieved via lidocaine/epinephrine iontophoresis. Tube placement was conducted using an integrated and automated myringotomy and tube delivery system. Anxiolytics, sedation, and papoose board were not used. Technical success and safety were evaluated. Patients 5 to 12 years old self-reported tube placement pain using the Faces Pain Scale-Revised (FPS-R) instrument, which ranges from 0 (no pain) to 10 (very much pain). Children were enrolled into three cohorts with 68, 47, and 222 children in the Operating Room (OR) Lead-In, Office Lead-In, and Pivotal cohorts, respectively. In the Pivotal cohort, there were 120 and 102 children in the <5 and 5- to 12-year-old age groups, respectively, with a mean age of 2.3 and 7.6 years, respectively. Bilateral tube placement was indicated for 94.2\% of children <5 and 88.2\% of children 5 to 12 years old. Tubes were successfully placed in all indicated ears in 85.8\% (103/120) of children <5 and 89.2\% (91/102) of children 5 to 12 years old. Mean FPS-R score was 3.30 (standard deviation [SD] = 3.39) for tube placement and 1.69 (SD = 2.43) at 5 minutes postprocedure. There were no serious adverse events. Nonserious adverse events occurred at rates similar to standard tympanostomy procedures. In-office tube placement in selected patients can be successfully achieved without requiring sedatives, anxiolytics, or papoose restraints via lidocaine iontophoresis local anesthesia and an automated myringotomy and tube delivery system. 2b Laryngoscope, 130:S1-S9, 2020. [\hyperlink{iDose TR}{PMID: 32160320}, Lawrence R Lustig et al., 2020]

\hypertarget{pmid_6540532}{P}reliminary data from a prospective randomized study of the use of a short course of adrenocorticosteroids in 73 children with ITP demonstrates a significant advantage of moderate dose (60 mg/m2/day p.o. X 21 days) prednisolone therapy in decreasing the duration of severe thrombocytopenia in most patients. The period of risk for serious bleeding, as reflected in the Rumpel-Leede test, was also significantly reduced. The number of children who developed chronic thrombocytopenia, although small in both groups, appeared to be uninfluenced by steroid therapy. No side effects or serious complications were noted in this trial. [\hyperlink{iDose TR}{PMID: 6540532}, J A Sartorius et al., 1984]

\hypertarget{pmid_30122515}{S}afety and efficacy of intravenous (IV) thrombolysis and endovascular therapy in children with acute ischemic stroke (AIS) are unknown to date. We aimed to review and synthesize currently available evidence on these acute recanalization therapies in pediatric stroke patients. We performed a systematic review and meta-analysis of all available data on safety and efficacy of acute treatment including thrombolysis and endovascular therapy in pediatric AIS patients aged <18 years. We searched the electronic databases Medline and Cochrane Library for eligible studies published from the earliest date available until August 31, 2016. Safety outcomes included intracerebral hemorrhage (ICH) post-treatment and in-hospital mortality. Efficacy outcomes included functional outcome 3-6 months after index stroke. We identified 222 records, of which 3 studies with a total of 16,987 pediatric stroke patients met our eligibility criteria of whom 181 received IV thrombolysis. No data exists from randomized trials and no data is available on endovascular thrombectomy. Risk of any ICH was increased in children receiving thrombolysis (risk ratio = 3.48, 95\%CI: 1.66-7.29; p = 0.001) compared with controls, with no evidence of heterogeneity (I Our analyses demonstrate a substantial lack of data on efficacy and safety of acute recanalization therapies in children with AIS. URL: http://www.crd.york.ac.uk/PROSPERO. Unique identifier: CRD42016047140. [\hyperlink{iDose TR}{PMID: 30122515}, Juliana T Pacheco et al., 2018]

\hypertarget{pmid_16316969}{I}ntravenous (IV) amiodarone has proven efficacy in adults. However, its use in children is based on limited retrospective data. A double-blind, randomized, multicenter, dose-response study of the safety and efficacy of IV amiodarone was conducted in 61 children (30 days to 14.9 years; median, 1.6 years). Children with incessant tachyarrhythmias (supraventricular arrhythmias [n=26], junctional ectopic tachycardia [JET, n=31], or ventricular arrhythmias [n=4]) were randomized to 1 of 3 dosing regimens (low, medium, or high: load plus 47-hour maintenance) with up to 5 open-label rescue doses. The primary efficacy end point was time to success. Of 229 patients screened, 61 were enrolled during 13 months by 27 of 48 centers in 7 countries. Median time to success was significantly related to dose (28.2, 2.6, and 2.1 hours for the low-, medium-, and high-dose groups, respectively; P=0.028). There was no significant association with dose for any arrhythmia subgroup, including JET, but the subgroups were too small for an accurate assessment. Adverse events (AEs) were common (87\%), leading to withdrawal of 10 patients. There were 5 deaths in the 30-day follow-up period (2 possibly related to the study drug). Dose-related AEs included hypotension (36\%), vomiting (20\%), bradycardia (20\%), atrioventricular block (15\%) and nausea (10\%). In children, the overall efficacy of IV amiodarone, as measured by time to success, was dose related but not significantly for any arrhythmia subgroup. AEs were common and appeared to be dose related. Although efficacious for critically ill patients, the dose-related risks of IV amiodarone should be taken into account when treating children with incessant arrhythmias. Prospective, placebo-controlled trials would be helpful in assessing antiarrhythmic drug efficacy in children, because their results may differ from retrospective series and adult studies. [\hyperlink{iDose TR}{PMID: 16316969}, J Philip Saul et al., 2005]

\section*{Azathioprine Sodium}
\subsection*{Result}
\subsubsection*{Answer}

Ages 2–18 years: Yes  
Ages <2 years: Unknown

\subsubsection*{{Explanation}}
\hypertarget{Azathioprine Sodium}
A review of the available abstracts reveals several targeted studies evaluating the safety of azathioprine (including azathioprine sodium) in children for various conditions, including inflammatory bowel disease (IBD), atopic dermatitis, and autoimmune hepatitis. The following summarizes the evidence by age range and indication:

1. **Children with Inflammatory Bowel Disease (IBD) (Ages: up to 18 years)**
   - Multiple studies specifically evaluated azathioprine in pediatric IBD populations. In a cohort of 107 children (age range not specified, but pediatric), azathioprine at 3 mg/kg was found to be safe and well-tolerated, with only 2 children discontinuing due to persistent adverse effects [\hyperlink{pmid_12656694}{PMID: 12656694}, D Fuentes et al., 2003]. Another retrospective study of 123 Italian children (mean age at start of therapy: 11.8 years) found that side effects occurred in 39\% but only 7\% experienced severe toxicity requiring discontinuation. The majority tolerated the drug, and it was considered efficacious in 70\% of cases [\hyperlink{pmid_12030954}{PMID: 12030954}, A Barabino et al., 2002]. A further study of 88 children (mean age 12 years) with Crohn’s disease found azathioprine was associated with prolonged remission and did not report significant safety concerns [\hyperlink{pmid_16954801}{PMID: 16954801}, Gerald J Jaspers et al., 2006].

2. **Children with Atopic Dermatitis (Ages: up to 14 years)**
   - Several studies focused on children with severe atopic dermatitis. In a prospective cohort of 82 children (mean age 8.3 years), adverse effects on blood indices occurred in 41\%, but only 2 children stopped therapy due to abnormal blood indices, and 2 due to clinical adverse effects. The authors concluded that oral azathioprine was associated with few pronounced adverse effects for the duration and dosage used [\hyperlink{pmid_25440430}{PMID: 25440430}, Nicholas R Fuggle et al., 2015]. Another retrospective study of 7 children (mean age 10 years) treated for up to 38 months found only mild, transient leukopenia in 2 patients, with no need to stop treatment [\hyperlink{pmid_20525484}{PMID: 20525484}, R M Martel et al., 2010]. A larger retrospective review of 48 children (age not specified, but pediatric) found no cases of neutropenia and considered the short-term adverse effect profile acceptable [\hyperlink{pmid_12174104}{PMID: 12174104}, L-A Murphy et al., 2002]. A prospective study of 12 children also found few adverse effects [\hyperlink{pmid_22892285}{PMID: 22892285}, Maura Caufield et al., 2013].

3. **Children with Autoimmune Hepatitis (Ages: up to 18 years)**
   - A retrospective analysis of 56 children (median age 11 years) with autoimmune hepatitis treated with azathioprine did not report significant safety concerns, and the focus was on optimizing therapy using metabolite monitoring [\hyperlink{pmid_30234756}{PMID: 30234756}, Rishi Bolia et al., 2018].

4. **Other Pediatric Indications**
   - A case report of a 13-year-old boy who ingested a large overdose of azathioprine reported only moderate, transient decreases in blood counts and no serious adverse events [\hyperlink{pmid_9629549}{PMID: 9629549}, C Krüger et al.].
   - A retrospective review of 95 children treated with azathioprine for dermatological conditions found that abnormal blood tests led to cessation in 3\% of cases, with no significant clinical side effects [\hyperlink{pmid_27435804}{PMID: 27435804}, Joy Yee et al., 2018].

5. **In Utero Exposure**
   - Studies of children exposed to azathioprine in utero (i.e., during pregnancy) are inconclusive regarding long-term safety, with some data suggesting possible developmental concerns but confounded by maternal illness and other factors [\hyperlink{pmid_32240856}{PMID: 32240856}, Cristina Belizna et al., 2020; \hyperlink{pmid_23139238}{PMID: 23139238}, Wendy Marder et al., 2013]. These do not address direct pediatric use.

**Summary by Age Range:**
- For children (up to 18 years), especially those with IBD, atopic dermatitis, or autoimmune hepatitis, multiple targeted studies affirm that azathioprine is generally safe and well-tolerated, with adverse effects being infrequent and rarely requiring discontinuation. Most studies included children from early childhood through adolescence (ages 2–18 years), with some specifying mean ages around 8–12 years.
- There is no evidence from these abstracts of targeted safety studies in infants (<2 years) for direct therapeutic use.
- In utero exposure studies do not provide sufficient evidence for or against safety in children treated directly with azathioprine.

**Conclusion:** Based on the available abstracts, azathioprine sodium is affirmed as safe for use in children (ages 2–18 years) for certain indications, with appropriate monitoring. Safety in infants (<2 years) is unknown based on these abstracts.

\subsection*{Abstracts}
\hypertarget{pmid_12656694}{A}zathioprine is widely used as maintenance therapy in children with moderate to severe inflammatory bowel disease (IBD). There is no data on safety at higher doses and its impact on growth and surgical morbidity in children. This retrospective cohort study included all children treated with azathioprine and diagnosed with IBD between 1996-2001. Outcome measures included indications for azathioprine use, adverse-effects and reasons for treatment discontinuation. Height and weight at diagnosis, treatment onset and current follow-up was recorded, and Z scores for height standardised for time. 107 children received azathioprine at 3 mg/kg. 61\% had Crohn's disease and 83\% started azathioprine within 2 years of diagnosis. Only 2/107 children had to stop azathioprine because of persistent adverse effects and 16/107 required surgery. There was a trend toward better growth in a group of children with Crohn's disease following treatment with high dose azathioprine therapy (P = 0.08). Azathioprine is a safe and well-tolerated maintenance therapy at 3 mg/kg for children with IBD. The prevalence of surgery and growth failure in a cohort of children with moderate to severe IBD appears less than previously reported. In children with Crohn's disease, growth velocity may be maximised by an emphasis on nutritional therapy and the use of high dose azathioprine. [\hyperlink{Azathioprine Sodium}{PMID: 12656694}, D Fuentes et al., 2003]

\hypertarget{pmid_25440430}{A}zathioprine is efficacious in the treatment of severe childhood atopic dermatitis; however, robust data on adverse effects in this population are lacking. We sought to assess adverse effects of azathioprine treatment in a pediatric atopic dermatitis cohort, and make recommendations for monitoring based on these data. Blood test results for all 82 children prescribed oral azathioprine for atopic dermatitis in our department between 2010 and 2012 were collated prospectively, and clinical notes were reviewed retrospectively. Mean age at commencing azathioprine was 8.3 years (SEM 0.4). Mean maximum doses were 2.4 mg/kg (SEM 0.1) and 1.5 mg/kg (SEM 0.1) for normal and reduced serum thiopurine-S-methyltransferase levels, respectively. Adverse effects on blood indices occurred in 34 of 82 patients (41\%), with pronounced effects in 18 of 82 (22\%) after a median time of 0.4 years. Two patients stopped therapy as a result of abnormal blood indices. Clinical adverse effects occurred in 16 of 82 (20\%), two resulting in cessation of therapy. Incidence of adverse effects was unaffected by age, sex, thiopurine-S-methyltransferase level, and drug dose on multivariate regression. Comparison with other studies is limited by varying definitions of adverse effects. Oral azathioprine was associated with few pronounced adverse effects for the duration of use and dosage in this cohort. Recommendations for monitoring are made. [\hyperlink{Azathioprine Sodium}{PMID: 25440430}, Nicholas R Fuggle et al., 2015]

\hypertarget{pmid_32240856}{A}zathioprine (AZA), an oral immunosuppressant, is safe during pregnancy. Some reports suggested different impairments in the offspring of mothers with autoimmune diseases (AI) exposed in utero to AZA. These observations are available from retrospective studies or case reports. However, data with respect to the long-term safety in the antenatally exposed child are still lacking. The aim of this study is to summarize the current knowledge in this field and to focus on the need for a prospective study on this population. We performed a PubMed search using several search terms. The actual data show that although the risk of congenital anomalies in offspring, as well as the infertility risk, are similar to those found in general population, there is a higher incidence of prematurity, of lower weight at birth and an intra-uterine delay of development. There is also an increased risk of materno- fetal infections, especially cytomegalovirus infection. Some authors raise the interrogations about neurocognitive impairment. Even though the adverse outcomes might well be a consequence of maternal illness and disease activity, interest has been raised about a contribution of this drug. However, the interferences between the external agent (in utero exposure to AZA), with the host (child genetic susceptibility, immune system anomalies, emotional status), environment (public health, social context, availability of health care), economic, social, and behavioral conditions, cultural patterns, are complex and represent confounding factors. In conclusion, it is necessary to perform studies on the medium and long-term outcome of children born by mothers with autoimmune diseases, treated with AZA, in order to show the safety of AZA exposure. Only large-scale population studies with long-term follow-up will allow to formally conclude in this field. TAKE HOME MESSAGES. [\hyperlink{Azathioprine Sodium}{PMID: 32240856}, Cristina Belizna et al., 2020]

\hypertarget{pmid_27435804}{S}ystemic oral immunomodulators azathioprine, methotrexate and cyclosporin are widely used in paediatric dermatology. Routine blood tests are performed to minimise drug-related adverse events. However, the frequency of monitoring tests may lead to significant fearful experiences for patients. We reviewed haematological abnormalities and clinical side-effects in a paediatric clinic population commencing immunomodulators for dermatological conditions, where haematological profiles are monitored less frequently than in current recommendations. A retrospective chart review of children started on azathioprine, methotrexate or cyclosporin for a dermatological condition between 2001-2015 from a primarily paediatric, private dermatology practice was performed. Blood tests were done at baseline, 1 month, 2 months and then 3-monthly for children on azathioprine. Children on methotrexate and cyclosporin had tests done at baseline, after 1 month and then 3-monthly. In total, 242 children were included in this study. Azathioprine, methotrexate and cyclosporin cohorts had 95, 97 and 50 patients treated for a mean duration of 656, 758 and 313 days, respectively. Isolated abnormal blood tests indicated the cessation of azathioprine in 3/95 (3\%), methotrexate in 5/97 (5\%) and cyclosporin in 2/50 (4\%) of patients. Abnormal blood test results were not associated with any reported clinical side-effects in the azathioprine (P = 0.726), methotrexate (P = 0.06) or cyclosporin groups (P = 0.250). In our experience, less frequent monitoring did not result in any significant adverse events over a 15-year period. We suggest that haematological monitoring during immunosuppressants use can be safely reduced from current recommendations. [\hyperlink{Azathioprine Sodium}{PMID: 27435804}, Joy Yee et al., 2018]

\hypertarget{pmid_16028153}{B}ecause of concerns about arthrotoxicity, fluoroquinolones are restricted for use in children. This study describes the safety and efficacy of gatifloxacin when used for treatment of children with recurrent acute otitis media (ROM) or acute otitis media (AOM) treatment failure (AOMTF). We performed an analysis of 867 children included in 4 clinical trials who had ROM and/or AOMTF and were treated with gatifloxacin (10 mg/kg once daily for 10 days). Gatifloxacin had adverse event rates that were similar overall to those of a comparator antibiotic (amoxicillin-clavulanate), except for increased diarrhea in children <2 years old receiving amoxicillin-clavulanate. There was no evidence of arthrotoxicity, hepatotoxicity, alteration of glucose homeostasis, or central nervous system toxicity acutely or during 1 year follow-up in any child. Regarding efficacy, in 2 noncomparative trials, the gatifloxacin cure rate of AOM was 89\% (95\% confidence interval [CI], 83\%-95\%) at the test of cure (TOC) visit, 3-10 days after completion of therapy. In 2 comparative trials of gatifloxacin versus amoxicillin-clavulanate, the efficacy of gatifloxacin was 88\% (95\% CI, 82\%-94\%). Gatifloxacin led to better clinical outcomes than amoxicillin-clavulanate for AOMTF (91\% vs. 81\%; P=.029), for AOMTF and age <2 years old (89\% vs. 69\%; P=.009), and for severe AOM in children <2 years old (90\% vs. 75\%; P=.012). Among children with AOMTF previously treated with amoxicillin-clavulanate or ceftriaxone injections, gatifloxacin cure rates were high (88\% and 75\%, respectively). Gatifloxacin appears to be safe for children, with no evidence of producing arthrotoxicity in 867 children exposed to the antibiotic when used as treatment for ROM and AOMTF. [\hyperlink{Azathioprine Sodium}{PMID: 16028153}, Michael E Pichichero et al., 2005]

\hypertarget{pmid_21555791}{A}n open-labelled, non-comparative study was conducted in 117 children aged 2-12 years to evaluate the efficacy and safety of azithromycin (20mg/ kg/day for 6 days) for the treatment of uncomplicated typhoid fever. Of the patients enrolled based on a clinical definition of typhoid fever, 109 (93.1\%) completed the study.Mean (SD) of duration of fever at presentation was 9.1(4.5) days. Clinical cure was seen in 102 (93.5\%) subjects, while 7 were withdrawn from the study because of clinical deterioration. Mean day of response was 3.45±1.97. BACTEC blood culture was positive for Salmonella typhi in 17/109 (15.5\%) and all achieved bacteriological cure. No serious adverse event was observed. Global well being assessed by the investigator and subjects was good in 95\% cases which was done at the end of the treatment. Azithromycin was found to be safe and efficacious for the management of uncomplicated typhoid fever. [\hyperlink{Azathioprine Sodium}{PMID: 21555791}, Anju Aggarwal et al., 2011]

\hypertarget{pmid_9629549}{W}e report on a 13 years 7 months old boy who ingested 650 mg azathioprine in a suicide attempt. His baseline medication was azathioprine and methotrexate for control of juvenile chronic polyarthritis. After the induction of vomiting by ipecacuanha sirup and administration of charcoal (1 g/kg), he was closely followed for haematological, hepatic, and renal side effects. During the following days, no serious adverse events were noted except that the thrombocyte (from 403,000 down to 199,000/microliter) and total leukocyte count decreased moderately (from 12,000 down to 7100/microliter). On the basis of this case report and the available literature, the potential acute toxicity of azathioprine and possible treatment modalities are discussed. [\hyperlink{Azathioprine Sodium}{PMID: 9629549}, C Krüger et al., ]

\hypertarget{pmid_31321320}{A}zithromycin is widely used in children not only in the treatment of individual children with infectious diseases, but also as mass drug administration (MDA) within a community to eradicate or control specific tropical diseases. MDA has also been reported to have a beneficial effect on child mortality and morbidity. However, concerns have been raised about the safety of azithromycin, especially in young children. The aim of this review is to systematically identify the safety of azithromycin in children of all ages. MEDLINE, PubMed, Cochrane Central Register of Controlled Trials, Embase, CINAHL, International Pharmaceutical Abstracts and adverse drug reaction (ADR) monitoring systems will be systematically searched for randomised controlled trials (RCTs), cohort studies, case-control studies, cross-sectional studies, case series and case reports evaluating the safety of azithromycin in children. The Cochrane risk of bias tool, Newcastle-Ottawa and quality assessment tools, and The Joanna Briggs Institute Critical Appraisal tools will be used for quality assessment. Meta-analyses will be conducted to the incidence of ADRs from RCTs if appropriate. Subgroup analyses will be performed in different age and azithromycin dosage groups. Formal ethical approval is not required as no primary data are collected. This systematic review will be disseminated through a peer-reviewed publication. CRD42018112629. [\hyperlink{Azathioprine Sodium}{PMID: 31321320}, Peipei Xu et al., 2019]

\hypertarget{pmid_14770073}{T}hree clinical trials have examined the efficacy and safety of single dose azithromycin (30 mg/kg) in children with uncomplicated acute otitis media (AOM). In the first trial, a small pilot study, the clinical and microbiologic efficacy of single dose azithromycin was comparable with that of 3-day azithromycin or single dose ceftriaxone. A second, non-comparative trial confirmed the clinical and microbiologic efficacy of the single dose regimen. The third study, a large double blind, double dummy trial, demonstrated comparable clinical success rates between single dose azithromycin and 10-day standard amoxicillin/clavulanate. The incidence of drug-related adverse events in patients treated with single dose azithromycin was low in all three trials and similar to rates that have been reported for other antimicrobial agents used for the treatment of patients with AOM. In the amoxicillin/clavulanate trial, compliance with single dose azithromycin was significantly better than with the amoxicillin/clavulanate regimen (P < 0.001). We conclude that a single dose of azithromycin (30 mg/kg) is safe and effective for the treatment of uncomplicated AOM in children. [\hyperlink{Azathioprine Sodium}{PMID: 14770073}, Adriano Arguedas et al., 2004]

\hypertarget{pmid_20525484}{I}n a small number of cases of childhood atopic dermatitis, topical therapy is ineffective, necessitating prolonged use of systemic immunosuppressants. Over the last few years, a better understanding of the metabolic pathways involved in azathioprine breakdown has enabled us to use this drug more safely. In this study, we evaluated the toxicity of azathioprine treatment adjusted to thiopurine methyltransferase activity in children with severe atopic dermatitis. We performed a retrospective study of the side effects of azathioprine therapy adjusted to thiopurine methyltransferase activity in children aged under 14 years with atopic dermatitis who were treated in the dermatology department of Hospital Universitario Insular de Gran Canaria in Gran Canaria, Spain. Side effects were evaluated by analysis of leukocyte count and transaminase levels at baseline, after 1 month of treatment, and every 3 months thereafter. During the last 4 years, 7 children (mean age, 10 years) with severe atopic dermatitis received azathioprine in our department. Mean duration of treatment was 12 months (range, 1 to 38 months). Only 2 patients presented mild transient leukopenia that did not require treatment to be suspended. Our experience shows that, when adjusted to thiopurine methyltransferase activity, azathioprine is a safe drug for the treatment of children with severe atopic dermatitis. However, clinical trials should be performed to compare the risk-benefit ratios of the different immunosuppressants used to treat these patients. [\hyperlink{Azathioprine Sodium}{PMID: 20525484}, R M Martel et al., 2010] 6-Mercaptopurine (6-MP) maintains remission in pediatric Crohn's disease (CD). Azathioprine, a prodrug of 6-MP, is used for maintenance of remission of CD in Europe. We evaluated to what extent azathioprine is used in newly diagnosed pediatric CD patients and whether maintenance of remission differed between patients using azathioprine or not. Charts of children (diagnosed 1998-2003, follow-up > or = 18 mo) were reviewed. Active disease was defined as Pediatric Crohn's Disease Activity Index (PCDAI) greater than 10 or systemic corticosteroid use. Remission was defined as PCDAI 10 or less without use of corticosteroids. Eighty-eight children (55M/33F, age 12 +/- 3 yr) were included. Seventy-two (82\%) patients received azathioprine during the follow-up period (38 +/- 17 mo). Patients diagnosed after 2000 received azathioprine significantly earlier during the course of disease compared with those diagnosed earlier (median, at 233 vs. 686 days; P < 0.05). At initial presentation, moderate-severe disease activity and prescription of corticosteroids were more prevalent in patients using azathioprine compared with nonazathioprine patients (75\% vs. 52\%; P < 0.05; and 89\% vs. 58\%; P < 0.005, respectively). Duration of corticosteroid use was longer in patients receiving azathioprine (232 vs. 168 days; P < 0.005). Median maintenance of first remission in patients who initially used corticosteroids, however, was longer in patients receiving azathioprine compared with nonazathioprine patients (PCDAI, 544 vs. 254 days, P = 0.08; corticosteroid free, 575 vs. 259 days, P < 0.05, respectively). We conclude that, since 2000, azathioprine is being introduced earlier in the treatment of newly diagnosed pediatric CD patients. The use of azathioprine is associated with prolonged maintenance of the first remission. [\hyperlink{Azathioprine Sodium}{PMID: 20525484}, Gerald J Jaspers et al., 2006]

\hypertarget{pmid_12174106}{T}here is a limited range of treatments for severe atopic dermatitis (AD). Azathioprine has often been used but there has been no randomized controlled trial of this drug to confirm its efficacy in AD. To establish or refute the efficacy of azathioprine in severe AD. To investigate the safety and tolerability of azathioprine in this patient population. We performed a double-blind, randomized, placebo-controlled, crossover trial of azathioprine in adult patients with severe AD. Each treatment period was of 3 months' duration. Treatments were azathioprine 2.5 mg kg(-1) day(-1) and matched placebo. Disease activity was monitored using the SASSAD sign score. In addition, severity of pruritus, sleep disturbance and disruption of work/daytime activity were monitored using visual analogue scales. Adverse events were recorded and haematological and biochemical monitoring was performed. Thirty-seven subjects were enrolled, mean age 38 years (range 17-73). Sixteen were withdrawn, 12 during azathioprine treatment and four during placebo treatment. The SASSAD score fell by 26\% during treatment with azathioprine vs. 3\% on placebo (P < 0.01). Pruritus, sleep disturbance and disruption of work/daytime activity all improved significantly on active treatment but not on placebo. The difference in mean improvement between azathioprine and placebo was significant for disruption of work/daytime activity (P < 0.02) but not for pruritus or sleep disturbance. Gastrointestinal disturbances were reported by 14 patients during azathioprine treatment and four were withdrawn as a result of severe nausea and vomiting. Leukopenia was observed in two patients and deranged liver enzymes in eight during treatment with azathioprine. Azathioprine is an effective and useful drug in severe AD although it is not always well-tolerated. Monitoring of the full blood count and liver enzymes is advisable during treatment. [\hyperlink{Azathioprine Sodium}{PMID: 12174106}, J Berth-Jones et al., 2002]

\hypertarget{pmid_8818855}{A}dverse events and laboratory abnormalities were monitored over 35 days after the commencement of treatment in 45 open studies conducted in Europe, South America, Africa and Asia. These studies were to assess the clinical efficacies of azithromycin and comparator antimicrobial agents in the treatment of paediatric acute bacterial infections. Children (6 months-16 years of age) had been treated with an oral suspension of azithromycin (10 mg/kg given once daily for 3 consecutive days) or with the approved oral dosing regimen of the comparator (amoxycillin, co-amoxiclav, cefixime, cefaclor, clarithromycin, erythromycin, penicillin V, cloxacillin, or roxithromycin). Adverse events were recorded in 232/2655 (8.7\%) children treated with azithromycin and in 180/1844 (9.8\%) who received comparator treatment. The majority of the treatment-related adverse events were classed as being of only mild or moderate severity and were gastrointestinal: 140 (5.3\%) in azithromycin- and 120 (6.5\%) in comparator-treated children. Co-amoxiclav was responsible for proportionately more of such events than any other agent. Treatment was discontinued prematurely due to an adverse event in 34 (1.3\%) azithromycin- and in 31 (1.7\%) comparator-treated children. Incidences of clinically significant laboratory abnormalities were low and occurred with comparable frequency in both treatment groups. The present analysis confirms that azithromycin can be safely used to treat bacterial infections in children of all ages. [\hyperlink{Azathioprine Sodium}{PMID: 8818855}, G Treadway et al., 1996]

\hypertarget{pmid_11136494}{T}he clinical effectiveness of amiodarone must be weighed against the likelihood of adverse effects. Adverse effects are less common in children than in adults, yet there have been no large studies assessing the efficacy and safety of amiodarone in the first 9 months of life. We sought to assess the safety and efficacy of amiodarone as primary therapy for supraventricular tachycardia in infancy. We evaluated the clinical course of 50 consecutive infants and neonates (1.0+/-1.5 months, 35 male) treated with amiodarone for supraventricular tachyarrhythmias between July 1994 and July 1999. At presentation, congenital heart disease, congestive heart failure, or ventricular dysfunction were present in 24\%, 36\%, and 44\% of the infants, respectively. Infants received a 7- to 10-day load of amiodarone at either 10 or 20 mg/kg/d. If this failed to control the arrhythmia, oral propranolol (2 mg/kg/d) was added. Patients were followed up for 16.0+/-13.0 months, and antiarrhythmic drugs were discontinued as tolerated. Rhythm control was achieved in all patients. Of the 34 patients who have reached 1 year of age, 23 (68\%) have remained free of arrhythmia, despite discontinuation of propranolol and amiodarone. Growth and development remained normal for age. Higher loading doses of amiodarone were associated with an increase in the corrected QT interval, but no proarrhythmia was seen. There were no side effects necessitating drug withdrawal. Amiodarone is an effective and safe therapy for tachycardia control in infancy. [\hyperlink{Azathioprine Sodium}{PMID: 11136494}, S P Etheridge et al., 2001]

\hypertarget{pmid_22892285}{A}zathioprine is prescribed as a corticosteroid-sparing agent for many inflammatory conditions, including refractory atopic dermatitis (AD). There are limited prospective data on its appropriate use and monitoring for children with AD. This study was designed to assess clinical response to azathioprine, determine the necessity for repeated measurement of thiopurine methyltransferase (TPMT) activity during treatment, and test the utility of measuring levels of the metabolites 6-thioguanine nucleotide and 6-methylmercaptopurine. Twelve children with severe, recalcitrant AD were treated with oral azathioprine and followed prospectively. Disease severity was determined by the SCORing AD index. Baseline TPMT activity was measured and this was repeated along with 6-thioguanine nucleotide and 6-methylmercaptopurine measurement at times of stable improvement, inadequate response, or change in response. Azathioprine therapy was associated with clinical improvement in all but 1 patient. There were few adverse effects. Three patients showed a significant change in TPMT activity during treatment: 2 had a mild decrease and 1 demonstrated enzyme inducibility with an increase from the intermediate to the normal activity range. These changes, but not 6-thioguanine nucleotide or 6-methylmercaptopurine levels, inversely correlated with the clinical response to therapy. Small sample size is a limitation. Azathioprine can be of benefit in the treatment of recalcitrant pediatric AD. Repeat assessment of TPMT activity may be helpful for evaluation of nonresponse or change in response and warrants further study. In contrast, measurement of thiopurine metabolites during treatment was not clinically useful. [\hyperlink{Azathioprine Sodium}{PMID: 22892285}, Maura Caufield et al., 2013]

\hypertarget{pmid_12174104}{A}topic eczema is a chronic inflammatory skin disease, which in most children can be adequately controlled using topical therapy. However, in a small number of children it is necessary to use systemic treatments to gain an acceptable level of disease control. To evaluate azathioprine as a treatment for severe atopic eczema in children, and the value of pretreatment thiopurine methyltransferase (TPMT) levels in the identification of patients at high risk of myelosuppression. Between January 1997 and May 2000, 91 children had erythrocyte TPMT assays with the intention of treating their atopic eczema with azathioprine. This study is based on retrospective examination of data taken from the hospital notes of these children, who had attended Great Ormond Street Hospital for Children and St John's Institute of Dermatology, London. The distribution of TPMT values corresponded closely to that previously described in adults. Forty-eight children were commenced on treatment with azathioprine. Twenty-eight had an excellent response to treatment, 13 had a good response and seven had a poor response. No patient developed neutropenia. Azathioprine may prove a very valuable treatment for severe atopic eczema in children. We consider its short-term adverse effect profile in children with normal TPMT activity to have been entirely acceptable with our treatment protocol. As result, we now feel confident to initiate therapy at dose levels of 2.5-3.5 mg kg(-1) in those with a normal TPMT level, and to reduce the frequency with which we undertake tests of bone marrow and liver function. [\hyperlink{Azathioprine Sodium}{PMID: 12174104}, L-A Murphy et al., 2002]

\hypertarget{pmid_8878240}{I}n this multicenter, open label trial the investigators evaluated the efficacy and safety of azithromycin suspension administered once daily for 5 days for the treatment of clinically and bacteriologically established acute otitis media. Two hundred eligible children with acute otitis media from 10 US centers were treated with 10 mg/kg of azithromycin oral suspension on Day 1, followed by 5 mg/kg once daily for the next 4 days. Tympanocentesis and subsequent culture of middle ear effusion were performed at baseline. Clinical efficacy was evaluated on Days 6, 11 and 30. Analysis of clinical efficacy in evaluable patients 11 days after the initiation of therapy showed that the rate of satisfactory responses (cured or improved) ranged from 79.6 to 82.4\% in patients infected with Streptococcus pneumoniae, Haemophilus influenzae, or Moraxella catarrhalis. Satisfactory clinical response at Day 30 was reported in 70\% of evaluable patients, and eradication of S. pneumoniae, H. influenzae and M. catarrhalis was presumed in 64 to 73\%. Relapses occurred in 14\% of the evaluable patients. Among the treated patients 8.5\% reported mild or moderate side effects. Azithromycin is an effective, safe and well-tolerated treatment for children with acute otitis media. [\hyperlink{Azathioprine Sodium}{PMID: 8878240}, J McCarty et al., 1996]

\hypertarget{pmid_33729325}{A}zathioprine is a common first-line therapy for neuromyelitis optica spectrum disorder (NMOSD). The aim of this study was to determine whether long-term treatment (>10 years) with azathioprine is safe in NMOSD. Methods: We conducted a retrospective medical record review of all patients at the School of Medicine of the University of São Paulo (São Paulo, Brazil) who fulfilled the 2015 international consensus diagnostic criteria for NMOSD and were treated with azathioprine for at least 10 years. Out of 375 patients assessed for eligibility, 19 were included in this analysis. These patients' median age was 44 years (range=28-61); they were mostly female (17/19) and AQP4-IgG seropositive (18/19). The median disease duration was 15 years (range=10-39) and most patients presented a relapsing clinical course (84.2\%). The median duration of treatment was 11.9 years (range=10.0-23.8). The median annualized relapse rates (ARR) pre- and post-treatment with azathioprine were 1 (range=0.1-2) and 0.1 (range=0-0.35); p=0.09. Three patients (15.7\%) had records of adverse events during the follow-up, which consisted of chronic B12 vitamin deficiency, pulmonary tuberculosis and breast cancer. Azathioprine may be considered a safe agent for long-term treatment (>10 years) of NMOSD, but continuous vigilance for infections and malignancies is required. [\hyperlink{Azathioprine Sodium}{PMID: 33729325}, Ana Beatriz Ayroza Galvão Ribeiro Gomes et al., 2021]

\hypertarget{pmid_8988412}{A}zithromycin (AZM), 10\% fine granules or 100 mg capsules, were given orally to 27 children with various pediatric infections. The results of the study are shown below. 1. Pharmacokinetic investigation. We studied plasma and urinary concentrations after 100 mg AZM capsules were given. One patient received 8.3 mg/kg of AZM once a day for 3 days, and AZM concentration in plasma was 0.033 microgram/ml 48 hours after the final dosing. Doses of 8.3 and 12.5 mg/kg body weight of AZM were respectively given to two patients once daily for 3 days. As a result, AZM concentrations in urine during a period between 96 and 120 hours post-dosing were 1.67 and 4.53 micrograms/ml, respectively, and urinary excretion rate in 120 hours after the first dosing was 10.54\% in the patient that was given 12.5 mg/kg. 2. Clinical investigation. Clinical efficacies were examined in 24 patients. Excellent results were obtained in 7 patients, good results in 14 patients, hence the clinical efficacy rate was 87.5\%. Bacteriologically, Haemophilus influenzae strains isolates from 2 patients were eradicated in 1 and decreased in the other. Safety was evaluated in 26 patients. An adverse reaction was observed in 1 patient (urticaria). Abnormal laboratory test results were observed in 2 patients, decreased WBC in 1 and elevation of eosinophils in the other. The above results suggest that AZM is a useful oral antibiotic for pediatric patients with infection with susceptible organisms. [\hyperlink{Azathioprine Sodium}{PMID: 8988412}, Y Kobayashi et al., 1996]

\hypertarget{pmid_8396100}{I}n this open study, a three-day regimen of azithromycin (single daily dose of 10 mg/kg) was compared with a ten-day regimen of amoxycillin paediatric suspension (30 mg/kg/day in three divided doses; children > 20 kg received 250 mg tid daily) in 154 children (aged 2-12 years) with a clinical diagnosis of acute otitis media (13 recurrent). Full clinical, bacteriological and laboratory safety assessments were performed during and after the study. Of the 77 azithromycin patients, 61 (79\%) were considered cured, 15 (19\%) improved and one (1\%) failed, compared with 45 (58\%) cured, 28 (36\%) improved and four (5\%) failed among the 77 amoxycillin patients. Excluding from analysis the 13 patients with recurrent otitis media, azithromycin was found to be significantly superior to amoxycillin (P = 0.003). The incidence of side-effects was low, with only two (3\%) and three (4\%) patients reporting adverse events with azithromycin and amoxycillin, respectively. These were gastrointestinal in nature and of mild or moderate severity, except for one case of severe diarrhoea in the amoxycillin group. No treatment-related abnormalities in the laboratory safety tests were observed, and no patients withdrew from therapy. A three-day regimen of azithromycin was therefore shown to be more effective than, and as well tolerated as, amoxycillin in the treatment of children with acute otitis media. [\hyperlink{Azathioprine Sodium}{PMID: 8396100}, E Mohs et al., 1993]

\hypertarget{pmid_24370666}{A}lthough paediatric patients frequently suffer from intoxications with atypical antipsychotics, the number of studies in young children, which have assessed the effects of acute exposure to this class of drugs, is very limited. The aim of this study was to achieve a better characterization of the acute toxicity profile in young children of the atypical antipsychotics clozapine, olanzapine, quetiapine, and risperidone. We performed a multicentre retrospective analysis of cases with atypical antipsychotics intoxication in children younger than 6 years, reported by physicians to German, Austrian, and Swiss Poisons Centres for the 9-year period between January 1, 2001 and December 31, 2009. One hundred and six cases (31 clozapine, 29 olanzapine, 12 quetiapine, and 34 risperidone) were available for analysis. Forty-seven of the children showed minor, 28 moderate, and 2 severe symptoms. Twenty-nine cases were asymptomatic. No fatalities were recorded. Symptoms predominantly involved the central nervous and cardiovascular systems. Minor reduction in vigilance (Glasgow Coma Scale score >9) (62 \%) was the most frequently reported symptom, followed by miosis (12 \%) and mild tachycardia (10 \%). Extrapyramidal motor symptoms were observed in one case (1 \%) after ingestion of risperidone. In most cases, surveillance and supportive care were sufficient to achieve a good outcome, and all children made full recovery. Paediatric antipsychotic exposure can result in significant poisoning; however, in most cases only minor or moderate symptoms occurred and were followed by complete recovery. Symptomatic patients should be monitored for central nervous system depression and an electrocardiogram should be obtained. [\hyperlink{Azathioprine Sodium}{PMID: 24370666}, Marianne Meli et al., 2014]

\hypertarget{pmid_34465327}{A}zithromycin has recently been shown to reduce all-cause childhood mortality in sub-Saharan Africa. One potential mechanism of this effect is via the anti-malarial effect of azithromycin, which may help treat or prevent malaria infection. This study evaluated short- and longer-term effects of azithromycin on malaria outcomes in children. Children aged 8 days to 59 months were randomized in a 1:1 fashion to a single oral dose of azithromycin (20 mg/kg) or matching placebo. Children were evaluated for malaria via thin and thick smear and rapid diagnostic test (for those with tympanic temperature ≥ 37.5 °C) at baseline and 14 days and 6 months after treatment. Malaria outcomes in children receiving azithromycin versus placebo were compared at each follow-up timepoint separately. Of 450 children enrolled, 230 were randomized to azithromycin and 220 to placebo. Children were a median of 26 months and 51\% were female, and 17\% were positive for malaria parasitaemia at baseline. There was no evidence of a difference in malaria parasitaemia at 14 days or 6 months after treatment. In the azithromycin arm, 20\% of children were positive for parasitaemia at 14 days compared to 17\% in the placebo arm (P = 0.43) and 7.6\% vs. 5.6\% in the azithromycin compared to placebo arms at 6 months (P = 0.47). Azithromycin did not affect malaria outcomes in this study, possibly due to the individually randomized nature of the trial. Trial registration This study is registered at clinicaltrials.gov (NCT03676751; registered 19 September 2018). [\hyperlink{Azathioprine Sodium}{PMID: 34465327}, Boubacar Coulibaly et al., 2021]

\hypertarget{pmid_23139238}{A}zathioprine (AZA) is recognized among immunosuppressive medications as relatively safe during pregnancy for women with systemic lupus erythematosus (SLE) requiring aggressive treatment. This pilot study aimed to determine whether SLE therapy during pregnancy was associated with developmental delays in offspring. This cohort study included SLE patients with at least one live birth postdiagnosis. Medical histories were obtained via interviews and chart review. Multiple logistic regression was used to examine associations between SLE therapy during pregnancy and maternal report of special educational (SE) requirements (as proxy for developmental delays) among offspring. Propensity scoring (incorporating corticosteroid use, lupus flare, and lupus nephritis) was used to account for disease severity. Of 60 eligible offspring from 38 mothers, 15 required SE services, the most common indication for which was speech delay. Seven (54\%) of the 13 children with in utero AZA exposure utilized SE services versus 8 (17\%) of 47 nonexposed children (P < 0.01). After adjustment for pregnancy duration, small for gestational age, propensity score, maternal education level, and antiphospholipid antibody syndrome, AZA was significantly associated with SE utilization occurring from age 2 years onward (odds ratio 6.6, 95\% confidence interval 1.0-43.3), and bordered on significance for utilization at any age or age <2 years. AZA exposure during SLE pregnancy was independently associated with increased SE utilization in offspring, after controlling for confounders. Further research is indicated to fully characterize developmental outcomes among offspring with in utero AZA exposure. Vigilance and early interventions for suspected developmental delays among exposed offspring may be warranted. [\hyperlink{Azathioprine Sodium}{PMID: 23139238}, Wendy Marder et al., 2013]

\hypertarget{pmid_30234756}{A}zathioprine (AZA) is the mainstay of maintenance therapy in pediatric autoimmune hepatitis (AIH). The use of thiopurines metabolites to individualize therapy and avoid toxicity has not, however, been clearly defined. Retrospective analysis of children ≤18 years diagnosed with AIH between January 2001 and 2016. Standard definitions were used for treatment response and disease flare. Thiopurine metabolite levels were correlated with the corresponding liver function test. A total of 56 children (32 girls) were diagnosed with AIH at a median age of 11 years (interquartile range [IQR] 9). No difference in 6-thioguanine-nucleotide (6-TG) levels (271[IQR 251] pmol/8 × 10 red blood cell vs 224 [IQR 147] pmol/8 × 10 red blood cell, P = 0.06) was observed in children in remission when compared with those who were not in remission. No correlation was observed between the 6-TG and alanine aminotransferase levels (r = -0.179, P = 0.109) or between 6-methyl-mercaptopurine (6-MMP) and alanine aminotransferase levels (r = 0.139, P = 0.213). The 6-MMP/6-TG ratio was significantly lower in patients who were in remission (2[7] vs 5 (10), P = 0.04). Using a quartile analysis, we found that having a ratio of <4 was significantly associated with being in remission with OR 2.50 (95\% confidence interval 1.02-6.10), P = 0.047. Use of allopurinol with low-dose AZA in 6 children with preferential 6-MMP production brought about remission in 5/6 (83.3\%). Thiopurine metabolite levels should be measured in patients with AIH who have experienced a loss of remission. A 6-MMP/6-TG ratio of <4 with the addition of allopurinol could be considered in these patients. [\hyperlink{Azathioprine Sodium}{PMID: 30234756}, Rishi Bolia et al., 2018]

\hypertarget{pmid_12030954}{T}o assess the efficacy and safety of azathioprine in a paediatric population with inflammatory bowel disease. One hundred and twenty-three Italian children treated with azathioprine were studied retrospectively. The treatment duration and causes of its discontinuation, side-effects and variation in corticosteroid dose were assessed. The mean age at inflammatory bowel disease diagnosis was 9.8 +/- 3.6 years, and at the start of azathioprine therapy 11.8 +/- 4.3 years. The mean duration of treatment was 19 +/- 16 months. Fifty patients (41\%) stopped treatment due to surgery (12\%), prolonged remission (11\%), non-response (7\%), severe side-effects (7\%) and poor compliance (3\%). Of the 73 patients (59\%) remaining on azathioprine, 11 had never been treated with corticosteroids, 27 were able to stop them and 35 were still on a very low daily dose (91\% < 0.3 mg/kg). The difference in the daily corticosteroid dose between the beginning of azathioprine treatment (1 +/- 0.6 mg/kg) and the conclusion of the study (0.18 +/- 0.16 mg/kg) was statistically significant. Side-effects were recorded in 48 of the 123 patients (39\%), but only eight required discontinuation of azathioprine. Azathioprine was efficacious in 70\% of patients, but ineffective in 20\% and induced severe toxicity in 7\%. Corticosteroids were stopped or markedly reduced in 62\% of patients, but they were never given in 9\%. [\hyperlink{Azathioprine Sodium}{PMID: 12030954}, A Barabino et al., 2002]

\section*{Bupivacaine Hydrochloride}
\subsection*{Result}
\subsubsection*{Answer}

Infants (<1 year): Yes (with caution, as some studies recommend further research and careful dosing)
Children (1–12 years): Yes
Adolescents (12–18 years): Yes

\subsubsection*{{Explanation}}
\hypertarget{Bupivacaine Hydrochloride}
A review of the available abstracts reveals multiple targeted studies evaluating the safety of Bupivacaine Hydrochloride in children across various age ranges and surgical contexts. Below is a summary by age group:

Infants (under 1 year, including neonates and infants up to 8 months):
- A study using hyperbaric bupivacaine for spinal anesthesia in children younger than 8 months documents its use but does not explicitly report on safety outcomes or adverse events, so safety cannot be definitively affirmed [\hyperlink{pmid_2265161}{PMID: 2265161}, S K Parkinson et al.].
- In infants weighing less than 12 kg, caudal anesthesia with bupivacaine (2.5 mg/kg) resulted in plasma concentrations below toxic values, and the technique was considered safe at prescribed doses, though caution was advised for small infants [\hyperlink{pmid_3729087}{PMID: 3729087}, J Camboulives et al., 1986].
- A retrospective cohort of 223 children (age not specified, but for congenital pyloric stenosis, typically infants) receiving caudal block with bupivacaine 0.25\% (4 mg/kg) reported a high success rate and a low incidence (1.3\%) of anesthetic complications, suggesting safety but recommending further prospective studies [\hyperlink{pmid_12060325}{PMID: 12060325}, Diana Moyao-García et al., 2002].

Children (1 year to 12 years):
- Multiple studies (including prospective, randomized, and pharmacokinetic studies) in children aged 1–12 years undergoing various surgeries (urologic, hernia, tonsillectomy, neurosurgery, and orthopedic procedures) consistently report that bupivacaine hydrochloride, when used at recommended doses, results in plasma concentrations below toxic thresholds and is not associated with significant adverse events or complications [\hyperlink{pmid_29520391}{PMID: 29520391}, Kyoung Lee et al., 2018; \hyperlink{pmid_6859502}{PMID: 6859502}, R L Eyres et al., 1983; \hyperlink{pmid_15200653}{PMID: 15200653}, Franco Puncuh et al., 2004; \hyperlink{pmid_28431423}{PMID: 28431423}, Santhanam Suresh et al., 2017; \hyperlink{pmid_24691852}{PMID: 24691852}, Mehmet Haksever et al., 2014; \hyperlink{pmid_1773485}{PMID: 1773485}, P St Louis et al., 1991; \hyperlink{pmid_3396545}{PMID: 3396545}, I Murat et al., 1988; \hyperlink{pmid_3706800}{PMID: 3706800}, P Rothstein et al., 1986; \hyperlink{pmid_2872745}{PMID: 2872745}, R Baghdassarian et al., 1986; \hyperlink{pmid_7102098}{PMID: 7102098}, U Hofmann et al., 1982].
- A large prospective study of 1132 children aged 6 months to 14 years found spinal anesthesia with hyperbaric bupivacaine to be effective and safe, with a low incidence of complications [\hyperlink{pmid_15200653}{PMID: 15200653}, Franco Puncuh et al., 2004].
- Studies specifically evaluating hemodynamic and cardiac effects in young children found no clinically significant changes or adverse effects [\hyperlink{pmid_29520391}{PMID: 29520391}, Kyoung Lee et al., 2018; \hyperlink{pmid_8880822}{PMID: 8880822}, M H Tsuji et al., 1996].

Adolescents (12–18 years):
- A phase 3, randomized trial in children 12 to <17 years (and 6 to <12 years) undergoing spine or cardiac surgery found that plasma bupivacaine levels after local infiltration remained below toxic thresholds, and adverse events were mild to moderate, supporting safety in this age group [\hyperlink{pmid_34534923}{PMID: 34534923}, Christopher F Tirotta et al., 2021].
- A randomized trial in children 6–18 years undergoing ambulatory urologic surgery found bupivacaine hydrochloride (alone or with liposomal bupivacaine) to be well-tolerated, with no significant safety concerns reported [\hyperlink{pmid_37871332}{PMID: 37871332}, Andrew T Gabrielson et al., 2024; \hyperlink{pmid_36997075}{PMID: 36997075}, Andrew T Gabrielson et al., 2023].

Summary:
Across the pediatric age spectrum (from infants to adolescents), multiple targeted studies affirm the safety of Bupivacaine Hydrochloride for various regional anesthesia techniques (caudal, epidural, spinal, nerve block, local infiltration) when used at recommended doses and with appropriate monitoring. Some studies in infants and neonates recommend caution and further research, but no study in the available abstracts reports a high incidence of serious adverse events or toxicity when used appropriately.

\subsection*{Abstracts}
\hypertarget{pmid_3706800}{A} pharmacokinetic evaluation of bupivacaine was carried out after intercostal nerve blocks performed on 28 occasions in 27 children varying in age from 3 months to 16 yr. Bupivacaine HCl, 0.5\%, with epinephrine 1:200,000 was employed. Doses of 2 mg/kg, 3 mg/kg, and 4 mg/kg resulted in peak whole blood arterial bupivacaine (base) concentrations (mean +/- SD) of 0.77 +/- 0.25 microgram/ml, 1.37 +/- 0.23 microgram/ml, and 1.87 +/- 0.53 microgram/ml, respectively. Calculated pharmacokinetic parameters (mean +/- SD) were the following: apparent volume of distribution (VD beta), 2.8 +/- 0.8 L/kg; steady-state volume of distribution (VDss), 2.7 +/- 0.7 L/kg; elimination half-life (t1/2 beta), 147 +/- 80 min; and total body clearance (Cl), 16.0 +/- 7.4 ml X min-1 X kg-1, or 382 +/- 201 ml X min-1 X m-2. Compared with data reported for adult patients, our data indicate that the volume of distribution is greater and clearance is more rapid in children than in adults. The absorption of local anesthetic from the intercostal space appears to be more rapid in children than adults. In an additional group of 11 children, the relationship of the bupivacaine blood:plasma concentration ratio (lambda) to hematocrit was investigated. Hematocrit in this group ranged from 30 to 59, and lambda varied from 0.47 to 0.82. There was a significant relationship between lambda and hematocrit defined by the equation lambda = -0.0079 Hct + 1.028 (r = 0.72, P less than 0.05). Reporting bupivacaine concentration in terms of plasma concentration may introduce an artifact that is dependent on the hematocrit, and we therefore suggest that whole blood concentration values be reported by investigators in the future. [\hyperlink{Bupivacaine Hydrochloride}{PMID: 3706800}, P Rothstein et al., 1986]

\hypertarget{pmid_6859502}{P}lasma bupivacaine concentrations were measured in 45 children, whose ages ranged from 4 months to 12 years, following administration of caudal epidural analgesia. Using 3 mg/kg of bupivacaine 0.25\%, mean blood levels of 1.2-1.4 microgram/ml were reached, which are well within the limits of projected toxic levels. Simultaneous arterial and venous sampling showed a small but significant difference between these two sampling sites fo the first fifteen minutes. [\hyperlink{Bupivacaine Hydrochloride}{PMID: 6859502}, R L Eyres et al., 1983]

\hypertarget{pmid_37871332}{W}e sought to determine if the addition of liposomal bupivacaine to bupivacaine hydrochloride improves opioid-free rate and postoperative pain scores among children undergoing ambulatory urologic surgery. A prospective, phase 3, single-blinded, single-center randomized trial with superiority design was conducted in children 6 to 18 years undergoing ambulatory urologic procedures between October 2021 and April 2023. Patients were randomized 1:1 to receive dorsal penile nerve block (penile procedures) or incisional infiltration with spermatic cord block (inguinal/scrotal procedures) with weight-based liposomal bupivacaine plus bupivacaine hydrochloride or bupivacaine hydrochloride alone. The primary outcome was opioid-free rate at 48 hours. Secondary outcomes included parents' postoperative pain measure scores, numerical pain scale scores, and weight-based opioid utilization at 48 hours and 10 to 14 days. We randomized 104 participants, with > 98\% (102/104) with complete follow-up data at 48 hours and 10 to 14 days. At interim analysis, there was no significant difference in opioid-free rate at 48 hours between arms (60\% in the intervention vs 62\% in the control group; estimated difference in proportion -1.9\% [95\% CI, -20\%-16\%];  The addition of liposomal bupivacaine to bupivacaine hydrochloride did not significantly improve opioid-sparing effect or postoperative pain compared with bupivacaine hydrochloride alone among children ≥ 6 years undergoing ambulatory urologic surgery. [\hyperlink{Bupivacaine Hydrochloride}{PMID: 37871332}, Andrew T Gabrielson et al., 2024]

\hypertarget{pmid_29520391}{L}ocal anesthetic agents such as bupivacaine and lidocaine are commonly used after surgery for pain control. The aim of this prospective study was to evaluate the safety of a mixture of bupivacaine and lidocaine in children who underwent urologic inguinal and scrotal surgery. Fifty-five patients who underwent pediatric urologic outpatient surgeries, were prospectively enrolled in this study. The patients were divided into three groups according to age (group I: under 2 years, group II: between 3-4 years, and group III: 5 years and above). Patients were further sub-divided into unilateral and bilateral groups. All patients were injected with a mixture of 0.5\% bupivacaine and 2\% lidocaine (2:1 volume ratio) at the surgical site, just before the surgery ended. Hemodynamic and electrocardiographic parameters were measured before local anesthesia, 30 minutes after administration of local anesthesia, and 60 minutes after administration. The patients' mean age was 40.5±39.9 months. All patients had normal hemodynamic and electrocardiographic parameters before local anesthesia, after 30 minutes, and after 60 minutes. Also, results of all intervals were within normal values, when analyzed by age and laterality. No mixture related adverse events (nausea, vomiting, pruritus, sedation, respiratory depression) or those related to electrocardiographic parameters (arrhythmias and asystole) were reported in any patients. A mixture of bupivacaine and lidocaine can be safely used in children undergoing urologic inguinal and scrotal surgery. An appropriate dose has no clinically significant hemodynamic or cardiac changes and adverse effects. [\hyperlink{Bupivacaine Hydrochloride}{PMID: 29520391}, Kyoung Lee et al., 2018]

\hypertarget{pmid_10551577}{B}upivacaine provides reliable, long-lasting anesthesia and analgesia when given via the caudal route. Ropivacaine is a newer, long-acting local anesthetic that (at a concentration providing similar pain relief) has less motor nerve blockade and may have less cardiotoxicity than bupivacaine. In a double-blind trial, 81 healthy children, undergoing ambulatory surgical procedures, were randomly allocated to receive caudal analgesia with either bupivacaine or ropivacaine, 0.25\%, 1 mVkg. All blocks were placed by an attending anesthesiologist or an anesthesia fellow after induction of general anesthesia. Data were available for 75 children. There were no significant differences between the two groups in baseline characteristics or in anesthesia, surgery, recovery room, or day surgery unit durations. The quality and duration of postoperative pain relief did not differ. Motor and sensory effects were similar. Time to first micturition did not differ. Ropivacaine (0.25\%, 1 ml/kg) provided adequate postoperative analgesia with no difference from bupivacaine (0.25\%, 1 ml/kg) in quality and duration of pain relief, motor and sensory effects, or time to first micturition in our study children. [\hyperlink{Bupivacaine Hydrochloride}{PMID: 10551577}, S Khalil et al., 1999]

\hypertarget{pmid_564643}{B}upivacaine (Marcaine) hydrochloride, a long-acting local anesthetic drug, was used in concentrations of 0.25, 0.5, or 0.75 percent with and without a vasoconstrictor, in amounts ranging from 25 to over 600 mg, for caudal, epidural (peridural), or peripheral nerve block for 11,080 surgical, obstetrical, diagnostic, or therapeutic procedures. Onset of anesthesia occurred in 4 to 10 minutes and maximum anesthesia in 15 to 35 minutes. Concentrations of 0.25, 0.5, and 0.75 percent consistently produced complete sensory anesthesia of the integumentary and musculoskeletal systems. With 0.25 and 0.5 percent, motor blockade ranged from minimal to complete. In intra-abdominal surgery, only 0.75 percent consistently produced profound muscle relaxation. Fifteen systemic toxic reactions occurred, but no untoward sequelae resulted from them. One inadvertent subarachnoid injection of 110 mg resulted in a total spinal block with an uneventful recovery. [\hyperlink{Bupivacaine Hydrochloride}{PMID: 564643}, D C Moore et al., ]

\hypertarget{pmid_9567153}{E}pidural anaesthesia is extremely useful in providing postoperative analgesia for children after surgery of the lower body. Although results on early pharmacokinetics in children have previously been reported, no data are available on the long-term effects of epidural anaesthesia. The aim of this investigation was the assessment of plasma bupivacaine levels in children with continuous epidural anaesthesia in the postoperative period. A catheter with an outer diameter of 0.63 mm was inserted through a 19G Tuohy cannula into the epidural space. A maximum dose of 0.4 mg/kg/h bupivacaine was administered for continuous epidural infusion. Careful monitoring was performed to detect early signs of local anaesthetic intoxication. Two milliliters of blood were obtained in each patient per day and nepholometric serum measurement were performed to determine alpha 1-acid glycoprotein and albumin levels. Bupivacaine plasma concentrations were assessed according to the method described by Sattler et al. [25]. Ten children were included in the investigation. The measured albumin and alpha 1-acid glycoprotein concentrations were within the range described by other investigators. At the onset of pain therapy maximum levels of 0.5 microgram/ml were recorded after a loading dose of bupivacaine and levels of up to 2.2 micrograms/ml were achieved following continuous infusion. There were no neurologic complications or signs of local anesthetic intoxication. In conclusion our results show that a dose of up to 0.4 mg/kg/h bupivacaine during continuous epidural infusion is not associated with toxic complications. Careful monitoring of the children by experienced staff is mandatory. [\hyperlink{Bupivacaine Hydrochloride}{PMID: 9567153}, A Scherhag et al., 1998]

\hypertarget{pmid_2872745}{B}upivacaine was utilized for postoperative analgesia in patients undergoing orchiopexy and hernia repair. In a study of 75 pediatric patients, ranging in ages from twelve months to twelve years, who had undergone orchiopexy and hernia repair during a three-year period, 42 received bupivacaine hydrochloride as a local infiltration block anesthesia to relieve postoperative pain; 33 patients did not receive bupivacaine. Patients receiving bupivacaine had less postoperative pain and were more comfortable when leaving the hospital within a few hours after surgery. [\hyperlink{Bupivacaine Hydrochloride}{PMID: 2872745}, R Baghdassarian et al., 1986]

\hypertarget{pmid_34534923}{T}o evaluate the pharmacokinetics and safety of liposomal bupivacaine in pediatric patients undergoing spine or cardiac surgery. Multicenter, open-label, phase 3, randomized trial (PLAY; NCT03682302). Operating room. Two separate age groups were evaluated (age group 1: patients 12 to <17 years undergoing spine surgery; age group 2: patients 6 to <12 years undergoing spine or cardiac surgery). Randomized allocation of liposomal bupivacaine 4 mg/kg or bupivacaine hydrochloride (HCl) 2 mg/kg via local infiltration at the end of spine surgery (age group 1); liposomal bupivacaine 4 mg/kg via local infiltration at the end of spine or cardiac surgery (age group 2). The primary and secondary objectives were to evaluate the pharmacokinetics (eg, maximum plasma bupivacaine concentrations [C Baseline characteristics were comparable across groups. Mean C Plasma bupivacaine levels following local infiltration with liposomal bupivacaine remained below the toxic threshold in adults (\textasciitilde{}2000-4000 ng/mL) across age groups and procedures. AEs were mild to moderate, supporting the safety of liposomal bupivacaine in pediatric patients undergoing spine or cardiac surgery. Clinical trial number and registry URL: ClinicalTrials.gov identifier: NCT03682302. [\hyperlink{Bupivacaine Hydrochloride}{PMID: 34534923}, Christopher F Tirotta et al., 2021]

\hypertarget{pmid_264254}{A} study was developed in an attempt to investigate the possible usefulness of the local anesthetic agent, bupivacaine hydrochloride, for oral surgery. The results show that bupivacaine hydrochloride is an effective local anesthetic agent. It has a rapid onset time, a high frequency of surgical anesthesia, a long duration, and a low incidence of side effects. In comparison to lidocaine, bupivacaine has a greater potency, a lower toxicity at equipotent doses a longer duration, a possible pain-free period after return of normal sensation, and it does not require a vasoconstrictor for consistent profoundness. [\hyperlink{Bupivacaine Hydrochloride}{PMID: 264254}, J L Laskin et al., 1977]

\hypertarget{pmid_36997075}{T}his study evaluates the tolerability and efficacy of preoperative dorsal penile nerve block with Exparel plus bupivacaine hydrochloride in children>6 years old undergoing ambulatory urologic surgery. We demonstrate that the drug combination is well-tolerated, with appropriate analgesic efficacy in the recovery room as well as at 48-hour and 10-14 day follow-up periods. These preliminary data justify the need to perform a prospective, randomized trial comparing Exparel plus bupivacaine hydrochloride to other common local anesthetic regimens used in pediatric urologic surgery. [\hyperlink{Bupivacaine Hydrochloride}{PMID: 36997075}, Andrew T Gabrielson et al., 2023]

\hypertarget{pmid_25185333}{B}upivacaine hydrochloride is frequently used in veterinary dental procedures to reduce the amount of general anesthesia needed and to reduce post-procedural pain. The aim of this study was to develop a novel method to test local anesthetic duration in mammals. Six infant pigs were placed under deep/surgical anesthesia with 3 \% isoflurane and oxygen while 0.5 ml of 0.5\% bupivacaine hydrochloride was injected to block the two greater palatine and the nasopalatine nerves. They were then maintained under light anesthesia with 0.5-1.0\% isoflurane. Beginning 15-minutes after the injection, 7 sites in the oral cavity were stimulated using a pointed dental waxing instrument, including 3 sites on the hard palate. The response, or lack of response, to the stimulus was recorded on video and in written record The bupivacaine hydrochloride injections lasted 1 to 3-hours before the animals responded to the sensory stimulation with a reflexive movement This study provides evidence that bupivacaine used to anesthetize the hard palate has a relatively short and variable duration of action far below what is expected based on its pharmacokinetic properties. [\hyperlink{Bupivacaine Hydrochloride}{PMID: 25185333}, Shaina Devi Holman et al., 2014]

\hypertarget{pmid_3729087}{S}erum concentrations of lidocaine and plasma concentrations of bupivacaine were measured so as to assess the risk of systemic toxicity following their administration by the caudal route in children, and study their pharmacokinetic profiles according to age. The serum concentrations of lidocaine were measured by immuno-enzymology in 37 children (23 +/- 13 kg) during the first hour after administration of 7 mg . kg-1. The plasma concentrations of bupivacaine were measured by high performance liquid chromatography in 40 children (18.03 +/- 8.90 kg) during the first hour after administration of 2.5 mg . kg-1. The greatest concentrations observed between 15 and 30 min after the injection were of 2.40 +/- 0.86 micrograms . ml for lidocaine and 0.93 +/- 0.44 microgram . ml-1 for bupivacaine. Higher values were observed in infants weighing less than 12 kg where they reached 2.89 +/- 0.72 and 1.52 +/- 0.68 micrograms . ml-1 respectively. These results showed that caudal anaesthesia with lidocaine (7 ml . kg-1) and bupivacaine (2.5 ml . kg-1) was a safe technique for children, giving average plasma concentrations inferior to toxic values. However, it seemed prudent not to give more than the prescribed doses in the small infant. [\hyperlink{Bupivacaine Hydrochloride}{PMID: 3729087}, J Camboulives et al., 1986]

\hypertarget{pmid_15200653}{S}pinal anaesthesia has been used in children for over 100 years and in the last two decades its popularity for newborns and infants has increased, but there are still unanswered questions with the technique. We evaluated the characteristics of spinal block including ease of performance, efficacy, adverse effects and complications in 1132 children, aged 6 months to 14 years, undergoing surgery in the lower part of the body. Local ethical committee approved the protocol of this prospective study, and parents gave written informed consent and older children their assent. All patients were sedated with midazolam, thiopental or propofol intravenously during spontaneous ventilation. No inhalation anaesthetics were used. Spinal block was performed with 0.5\% hyperbaric bupivacaine at a dose of 0.2 mg x kg(-1). Efficacy, safety and ease of performance of the spinal block were shown to be satisfactory in most children. Only 27 of the 1132 children needed any supplementation. The incidence and severity of complications was low. Only nine of 942 children, < 10 years of age and eight of 190 children, 10 years or older, developed hypotension. The incidences of postdural puncture headache, in five of the 1132 children, and backache, in nine of the 1132, were low. No other neurological complications were reported. Spinal anaesthesia with hyperbaric bupivacaine is a feasible anaesthetic method in children for surgery in the lower part of the body. [\hyperlink{Bupivacaine Hydrochloride}{PMID: 15200653}, Franco Puncuh et al., 2004]

\hypertarget{pmid_12060325}{S}ince 1970, bupivacaine 0.25\% in a dose of 4 mg x kg-1 (1.6 ml x kg-1) has been used at the Hospital Infantil de México for caudal block in children undergoing surgical correction of congenital pyloric stenosis (CPS). Although this dose is considered unsafe, in our experience, it has been associated with a high success rate and a low incidence of adverse events. This experience has not been previously documented. A retrospective cohort of patients undergoing surgical correction of CPS was studied. Nineteen patients received general anaesthesia while 223 received caudal block. The latter were then grouped according to the sedation technique. The rate of successful caudal blocks and complications were considered the major outcomes of the study, whereas the postsurgical fasting period and hospital stay were considered secondary outcomes. The rate of success of caudal block was 96\%. Anaesthetic complications related to bupivacaine were present in 1.3\%. Mortality occurred in the postoperatory period in one septic patient who also was suffering from gastroschisis that required general anaesthesia. Postoperatory fasting period and hospital stay tended to be higher with general anaesthesia than caudal block. However, of the 19 patients receiving general anaesthesia, five suffered serious comorbidity and nine were failed caudal blocks. Caudal block with bupivacaine 0.25\% (4 mg x kg-1) was associated with a low rate of anaesthetic complications. Further prospective studies to clarify the risks and benefits are required. [\hyperlink{Bupivacaine Hydrochloride}{PMID: 12060325}, Diana Moyao-García et al., 2002]

\hypertarget{pmid_10673893}{R}opivacaine is assumed to be less toxic than bupivacaine but there are no reports concerning its long-term use in paediatric anaesthesia. We report the use of ropivacaine for long-term epidural anaesthesia in a 21-month-old girl. In two consecutive periods of 3 days each, 0.5\% bupivacaine and 0.5\% or 0.75\% ropivacaine were administered to facilitate painful vaginal brachytherapy. The mean dose of bupivacaine increased from 1.05 to 1.32 mg kg-1 h-1 and that of ropivacaine increased from 1.40 to 3.86 mg kg-1 h-1. No toxic side effects were observed. We conclude that both epidural ropivacaine and bupivacaine were effective and safe during long-term epidural anaesthesia in this particular case. However, the doses were potentially toxic and should therefore be used with extreme caution. [\hyperlink{Bupivacaine Hydrochloride}{PMID: 10673893}, B Gustorff et al., 1999]

\hypertarget{pmid_8880822}{W}e studied the haemodynamic and cardiovascular effects of epidural anaesthesia with plain bupivacaine 0.75 ml.kg-1 in 13 unpremedicated ASA 1 children using measurements of heart rate, blood pressure and M-mode echocardiography. Under general anaesthesia, M-mode echocardiographic evaluation of left ventricular function in each patient was performed at four points (after general anaesthesia, point A; 5 min, 10 min and 25 min after epidural anaesthesia, point B; point C; and point D, respectively). Results were compared between point A and B, A and C, A and D, B and C, B and D, C and D. HR decreased significantly at 10 min (point C) and 25 min (point D) and MBP decreased at 5 min (point B) and 10 min (point D) compared to point A. No other M-mode cardiographic indices were changed at any point. Epidural anaesthesia using 0.25\% bupivacaine 0.75 ml.kg-1 did not affect LV function in young children. [\hyperlink{Bupivacaine Hydrochloride}{PMID: 8880822}, M H Tsuji et al., 1996]

\hypertarget{pmid_24691852}{T}he objective of this study is to compare the topical administration of bupivacaine hydrochloride, saline and bupivacaine hydrochloride infiltration on post-tonsillectomy pain in children. Sixty children undergoing tonsillectomy were enrolled in the study. Patients were randomized into three groups using sealed envelopes. Group 1 (n = 20) received topical 0.5 \% bupivacaine hydrochloride, group 2 (n = 20) received topical 0.9 \% NaCl (saline), and group 3 (n = 20) received 0.5 \% bupivacaine hydrochloride infiltrated around each tonsil. Pain was evaluated using McGrath's face scale. Pain scores in topical bupivacaine hydrochloride group was significantly lesser than the topical saline group at 5th, 13th, 17th and 21st hours, until the 6th day (p < 0.017). Moreover, pain scores of topical bupivacaine hydrochloride group was superior to bupivacaine hydrochloride infiltration group at 5th, 13th, 17th hours and 2nd, 3rd, 4th and 5th day (p < 0.017). There were significantly lesser morbidities in topical bupivacaine hydrochloride than saline group in 1st and 4th day (p < 0.017). Topical administration of bupivacaine hydrochloride proved to provide more efficient pain control than bupivacaine hydrochloride infiltration. [\hyperlink{Bupivacaine Hydrochloride}{PMID: 24691852}, Mehmet Haksever et al., 2014]

\hypertarget{pmid_28431423}{W}e evaluated blood bupivacaine concentrations in children having a single-shot sciatic and continuous femoral blocks after anterior cruciate ligament repair. Dried blood spot samples were analyzed for bupivacaine levels at 0, 5, 15, 30, 60, and 120 minutes and 4, 24, and 48 hours. The highest 99\% upper confidence interval limit was 135 ng/mL at the 4-hour evaluation point. The 99\% upper confidence interval was below potentially toxic levels (1500 ng/mL) across all sampling times. The risk of local anesthetic toxicity in pediatric patients receiving single-shot sciatic and continuous femoral nerve blocks is very low. [\hyperlink{Bupivacaine Hydrochloride}{PMID: 28431423}, Santhanam Suresh et al., 2017]

\hypertarget{pmid_3837258}{B}upivacaine induced caudal block as unique anaesthetic procedure in 80 children who underwent surgical correction of hypospadias, resulted of great efficacy and easy employment. Furthermore, the method permitted to avoid intubation and the use of general anaesthetic drugs with their relative complications and provided a great postoperative analgesia. [\hyperlink{Bupivacaine Hydrochloride}{PMID: 3837258}, M Heinen et al., ]

\hypertarget{pmid_3396545}{B}lood concentrations and pharmacokinetic parameters of bupivacaine were measured after epidural injection in children aged from 1 to 7 years. The children were allocated to two groups. In Group 1 (five children), pharmacokinetic parameters were measured after a single bolus injection of bupivacaine 0.25\% with adrenaline 1:200,000. In Group 2 (eight children), pharmacokinetic parameters were measured after the initial injection and the second injection. The same local anaesthetic was used. The volume of the second injection was half of the initial volume. In children of Group 1, maximum mean concentration (CPmax) was 0.64 +/- 0.05 microgram ml-1, time to maximum concentration (Tmax) 19.2 +/- 3.9 min, vascular absorption (T1/2 abs) 4.3 +/- 1.5 min, terminal half-life (T1/2 beta) 227 +/- 37.7 min, volume of distribution (Vd) 3.4 +/- 0.51 kg-1, and total body clearance (Clt) 11.0 +/- 2.0 ml min-1 kg-1. When compared to an adult's pharmacokinetic parameters, both Vd and Clt were increased, so that T1/2 beta remained essentially unchanged. In children of Group 2, the first repeat injection was performed at 110 +/- 6.9 min. Mean CPmax increased significantly (20\% after the second injection), whereas the values of the pharmacokinetic parameters measured did not differ significantly from those measured in children in Group 1. The results obtained in the present study demonstrate that the pharmacokinetic parameters of bupivacaine in children do not differ markedly from those reported in adults and that in the recommended dosage, the mean maximum concentrations, even after the second injection, are less than the presumed toxic levels. [\hyperlink{Bupivacaine Hydrochloride}{PMID: 3396545}, I Murat et al., 1988]

\hypertarget{pmid_7740909}{T}he authors discuss their experience with chloroprocaine for epidural anesthesia in five pediatric patients. While bupivacaine remains the most commonly used local anesthetic in children, recent reports of toxicity document the risks of this agent. The major advantage of chloroprocaine is its rapid metabolism, which thereby minimizes the risks of toxicity, especially in patients with preexisting problems such as young age or underlying hepatic dysfunction, which may limit the metabolism of local anesthetics of the amide class. In three cases, the epidural infusion was combined with the general anesthetic. The cases included hepatic resection, repair of bladder exstrophy, and correction of duodenal atresia. In two other cases, epidural anesthesia was used instead of general anesthesia in a former preterm infant who was undergoing inguinal herniorrhaphy and for lower extremity orthopedic procedures in a child with myotonic dystrophy. In all cases, chloroprocaine was chosen because of preexisting or associated conditions that might increase the risk of bupivacaine toxicity, such as hepatic resection, repeated dosing in a neonate, or the need for higher concentrations of local anesthetic to achieve adequate surgical conditions. Adequate intraoperative conditions were achieved in all five patients. No complications related to chloroprocaine epidural anesthesia were noted. This initial experience suggests that chloroprocaine offers an acceptable alternative to bupivacaine for epidural anesthesia in the pediatric population. [\hyperlink{Bupivacaine Hydrochloride}{PMID: 7740909}, J D Tobias et al., 1995]

\hypertarget{pmid_7102098}{T}he results of more than hundred caudal anaesthesias in surgical paediatric urology in children aged one to twelve years are reported. Modifying the method of Schulte-Steinberg, we used Bupivacain - CO2 0.5\%, and attained sufficient analgesia in all cases for whatever surgery applied. There were no complications except in one case where an overdose had been applied. We feel that the postoperative period is easier to bear because of the longer lasting local analgesia. Earlier onset of oral food intake is possible. [\hyperlink{Bupivacaine Hydrochloride}{PMID: 7102098}, U Hofmann et al., 1982]

\hypertarget{pmid_1773485}{T}he local anaesthetic bupivacaine could be very useful for analgesia in pediatric neurosurgery. Since systemic toxic reactions to bupivacaine are correlated with high plasma levels it was important, as an adjunct to clinical evaluation, to measure plasma bupivacaine. This report describes a high-performance liquid chromatography (HPLC) method for the quantitation of plasma bupivacaine. Sample preparation involves extraction into ether followed by back-extraction into HCl. After evaporation, the acid extract is redissolved and separated by reversed-phase chromatography. The assay is linear to 5 mg bupivacaine/L and shows excellent recovery and precision. With samples from children undergoing brain surgery following scalp infiltration with either 0.125\% or 0.25\% bupivacaine, plasma levels peak within 10 min, then fall rapidly to a plateau by 30 min. This plateau is maintained for at least 120 min. In no case did we find supposed toxic levels of bupivacaine. [\hyperlink{Bupivacaine Hydrochloride}{PMID: 1773485}, P St Louis et al., 1991]

\hypertarget{pmid_2265161}{P}revious studies have established the efficacy of spinal anesthesia in infants with hyperbaric lidocaine, hyperbaric tetracaine and isobaric and hyperbaric bupivacaine. Use of the commercially available hyperbaric spinal anesthesia solution of 0.75\% bupivacaine in 8.25\% dextrose has not heretofore been documented in this patient population. We present a series of 12 cases in which this hyperbaric bupivacaine solution with epinephrine was used for spinal anesthesia in children younger than 8 months of age. [\hyperlink{Bupivacaine Hydrochloride}{PMID: 2265161}, S K Parkinson et al., ]

\section*{Busulfan}
\subsection*{Result}
\subsubsection*{Answer}

Infants (<1 year): Yes  
Children (1-12 years): Yes  
Adolescents (12-18 years): Yes  
Neonates (<1 month): Unknown  

\subsubsection*{{Explanation}}
\hypertarget{Busulfan}
To determine if Busulfan is safe for use in children, I reviewed the available abstracts for targeted studies on safety in pediatric populations, focusing on specific age ranges and whether the studies affirm safety.

**Infants (<1 year) and Young Children (up to \textasciitilde{}2 years):**
- Several pharmacokinetic studies have been conducted in infants and very young children, showing that Busulfan disposition is age-dependent, with higher clearance and larger volume of distribution in this group [\hyperlink{pmid_8338934}{PMID: 8338934}, G Vassal et al., 1993; \hyperlink{pmid_24029650}{PMID: 24029650}, Radojka M Savic et al., 2013]. These studies emphasize the need for individualized dosing and therapeutic drug monitoring due to variability, but do not directly affirm safety.
- One study in children as young as 0.3 years (about 4 months) up to 17.2 years found that over 80\% of children achieved desired plasma exposure with a new dosing regimen, but the abstract does not explicitly state safety outcomes [\hyperlink{pmid_19049661}{PMID: 19049661}, L Nguyen et al., 2008].
- A study in 18 children aged 0.5 to 16 years using IV Busulfan with dose adjustment reported no new unexpected or unusual toxicity, and only one case of moderate veno-occlusive disease (VOD), concluding that IV Busulfan is "safe, convenient and feasible" in children [\hyperlink{pmid_17001185}{PMID: 17001185}, Juliette Zwaveling et al., 2006].
- A meta-analysis of 13 studies involving 548 pediatric patients (aged 0.3-18 years) found that Busulfan exposure above certain thresholds increased the risk of VOD, but did not report an unacceptably high rate of severe toxicity, suggesting that with proper monitoring, Busulfan can be used safely in children [\hyperlink{pmid_32312247}{PMID: 32312247}, Xinying Feng et al., 2020].

**Children (1-12 years):**
- Multiple studies specifically in children (mean ages ranging from 6 to 9 years) undergoing hematopoietic stem cell transplantation (HSCT) report that Busulfan is widely used, and with therapeutic drug monitoring and dose adjustment, toxicity is manageable and within acceptable limits [\hyperlink{pmid_32140913}{PMID: 32140913}, Abdullah Alsultan et al., 2020; \hyperlink{pmid_28801891}{PMID: 28801891}, Maura Faraci et al., 2018; \hyperlink{pmid_10642802}{PMID: 10642802}, M A Diaz et al., 1999].
- A large multicenter study of 954 pediatric patients (age not specified but described as "children") found a low incidence of seizures (1.3\%) with Busulfan-containing regimens, most of which were associated with pre-existing risk factors [\hyperlink{pmid_24201160}{PMID: 24201160}, Désirée Caselli et al., 2014].
- Another study in children aged 2 months to 3.6 years highlighted the need for pediatric-specific dosing and monitoring due to pharmacokinetic differences, but did not report unexpected toxicity [\hyperlink{pmid_2328321}{PMID: 2328321}, L B Grochow et al., 1990].

**Adolescents (12-18 years):**
- Studies often include adolescents up to 17 or 18 years, and the safety profile appears similar to that in younger children, with the main risks being VOD and seizures, both of which are manageable with monitoring and prophylaxis [\hyperlink{pmid_19049661}{PMID: 19049661}, L Nguyen et al., 2008; \hyperlink{pmid_32312247}{PMID: 32312247}, Xinying Feng et al., 2020].

**Summary:**
- There are multiple targeted studies in children, including infants, that address Busulfan use, dosing, and safety monitoring.
- The studies generally affirm that Busulfan can be used safely in children for conditioning prior to HSCT, provided that therapeutic drug monitoring and dose adjustments are performed to minimize toxicity.
- The main risks are VOD and seizures, but these are not unacceptably high and are considered manageable.
- There is no evidence from these abstracts of studies showing Busulfan is unsafe in children when used with appropriate monitoring.
- For neonates (<1 month), there is limited direct evidence, but infants as young as 0.3 months (about 9 days) have been included in some studies [\hyperlink{pmid_19049661}{PMID: 19049661}, L Nguyen et al., 2008; \hyperlink{pmid_32312247}{PMID: 32312247}, Xinying Feng et al., 2020].

**Animal Studies:**
- Some animal studies show toxicity in infant rats, but these do not directly translate to human pediatric safety and are not considered definitive for clinical safety in children [\hyperlink{pmid_31706849}{PMID: 31706849}, Henrique J C B Gouveia et al., 2020; \hyperlink{pmid_24791064}{PMID: 24791064}, Toko Ohira et al., 2014].

**Conclusion:**
- For infants (including <1 year), children (1-12 years), and adolescents (12-18 years), targeted studies affirm that Busulfan is safe for use when proper dosing and monitoring protocols are followed.
- For neonates (<1 month), while some infants in studies were as young as 0.3 months, explicit safety data for this age group is limited, so safety is less certain but not contradicted by available evidence.

\subsection*{Abstracts}
\hypertarget{pmid_32140913}{B}ackground Busulfan is an antineoplastic drug that is used widely as part of a conditioning regimen in pediatric patients undergoing hematopoietic stem cell transplantation. It has a narrow therapeutic index and highly variable pharmacokinetics; therefore therapeutic drug monitoring is recommended to optimize busulfan dosing. Objective To study the population pharmacokinetics of busulfan in Saudi pediatric patients to optimize its dosing. Settings King Abdullah Specialist Children's Hospital in Riyadh, Saudi Arabia. Methods This pharmacokinetic observational study was conducted between January 2016 and December 2018. All pediatric patients receiving IV busulfan and undergoing routine therapeutic drug monitoring were included. Population pharmacokinetics modeling was conducted using Monolix2019R1. Pharmacokinetic data of busulfan in children. Results The study included 59 patients and 513 samples. The mean ± SD age was 6.10 ± 3.17 years, and the dose administered was 0.994 ± 0.15 mg/kg. The mean ± SD Cmax and area under the curve (AUC) were 900.60 ± 402.8 ng/mL and 1031.14 ± 300.75 µM min, respectively. Based on our simulations, the European Medicines Agency recommended dose were adequate for most patient's groups to achieve the conventional target of an AUC [\hyperlink{Busulfan}{PMID: 32140913}, Abdullah Alsultan et al., 2020] Busulfan is widely used as a component of the myeloablative therapy in bone marrow transplantation. Recent studies have shown that the drug disposition is altered in children and is associated with less therapeutic effectiveness, lower toxicities, and higher rates of engraftment failure. We have evaluated the bioavailability of the drug in two groups of patients: eight children between 1.5 and 6 years of age and eight older children and adults between 13 and 60 years. Oral bioavailability showed a large interindividual variation. In children, the bioavailability ranged from 0.22 to 1.20, and for adults, it was within the range 0.47 to 1.03. The elimination half-life after intravenous administration in children (2.46 +/- 0.27 hours; mean +/- SD) did not differ from that obtained for adults (2.61 +/- 0.62 hours). However, busulfan clearance normalized to body weight was significantly higher in children (3.62 +/- 0.78 mL.min-1.kg-1) than that in adults (2.49 +/- 0.52 mL.min-1.kg-1). Also, the distribution volume normalized for body weight was significantly higher in children (0.74 +/- 0.10 L.kg-1) compared with 0.56 +/- 0.10 L. kg-1 in adults. The difference in clearance between children and adults was not statistically significant when normalized to body surface area, which most probably shows that busulfan dosage should be calculated on the basis of surface area rather than body weight. However, to avoid drug-related toxicities, drug monitoring and an individual dose adjustment should be considered because of the variability in busulfan bioavailability. [\hyperlink{Busulfan}{PMID: 32140913}, M Hassan et al., 1994]

\hypertarget{pmid_2328321}{C}hildren receive busulfan orally as part of myeloablative therapy before bone marrow transplantation for malignant and nonmalignant conditions. Children have been reported to have a low incidence of severe toxicity and significant rates of failure to achieve full engraftment. We evaluated the disposition of busulfan in children between 2 months and 3.6 years of age with lysosomal storage diseases, leukemia, and immunodeficiency disorders receiving oral doses of 1 or 2 mg/kg using a gas chromatographic assay. Peak concentrations were lower than those previously reported for adults, ranging from 1.4 to 5.2 mumol/L. The harmonic mean of the elimination half-life was 92 minutes, which is only slightly faster than that for adults (140 minutes). However, the area under the curve ranged from 400 to 1,000 (715 +/- 240) mumol.min/L, substantially lower than in adults receiving 1 mg/kg (range, 710 to 5,100 mumol.min/L; mean +/- SD, 2,180 +/- 1,200). The apparent volume of distribution (assuming complete bioavailability) ranged from 0.28 to 3.53 L/kg (1.42 +/- 0.86), which is more than twice that reported for adults (0.60 +/- 0.42). Busulfan clearance rate normalized to surface area is twice as high in children (200 +/- 100 mL/min/m2) as it is in adults (95 +/- 54 mL/min/m2). Alterations in bioavailability (absorption or first pass elimination) or in actual volume of distribution may account for these differences in drug disposition. The observed differences suggest the need for separate phase I dose escalation studies in children with accompanying pharmacokinetic assessment. [\hyperlink{Busulfan}{PMID: 2328321}, L B Grochow et al., 1990]

\hypertarget{pmid_28801891}{T}he aim of this report is to describe the experience in the management of busulphan-based conditioning regimen administered before hematopoietic stem cell transplantation (HSCT) in children. We report the values of the first dose AUC (area under the concentration-time curve, normal target between 3600 and 4800 ng·h/mL) in children treated with oral and intravenous busulphan, and we analyze the impact of some clinical variables in this cohort of patients. 82 children treated with busulphan before HSCT were eligible for the study: 57 received oral busulphan with a mean AUC of 3586 ng·h/mL, while 25 received intravenous busulphan with a mean AUC of 4158 ng·h/mL. Dose adjustment was based on first dose AUC. The dose was increased in 36 children (43.9\%) and decreased in 26 patients (31.7\%). Age at HSCT (P = 0.015), cumulative dose of busulphan as mg/m We concluded that older age at HSCT, intravenous administration of busulphan, cumulative, and prescribed dose of busulphan are associated with higher AUC levels. The absence of significant correlations between toxic events, graft failure, and AUC suggests the efficacy of busulphan concentrations monitoring in our patients. [\hyperlink{Busulfan}{PMID: 28801891}, Maura Faraci et al., 2018]

\hypertarget{pmid_19049661}{B}usulfan (Bu) is commonly used in preparative conditioning regimen prior to bone marrow transplantation in infants (< 1 year old), children and adolescents (up to 17 years old). The clinical development of an intravenous form of busulfan (Busilvex) was based on pharmacokinetic (PK) modeling and simulation techniques. A retrospective population PK analysis was initially performed from a first study in 24 pediatric patients (0.45-16.7 years old) and a log-linear relationship between body weight and Busilvex clearance was demonstrated with no age-dependency. For an optimal area under the curve (AUC) targeting, a new Bu dosing regimen [i.e. 5 dose levels (0.80 to 1.20 mg/kg) adjusted to 5 discrete weight categories] was developed and assessed through population PK-based simulations. The benefit from this new dosing strategy was validated in a second trial including 55 children (0.30-17.2 years old). This prospective trial confirmed the previous simulations: an efficient therapeutic targeting whatever the patient's age or body weight. Over 80\% of the children were within the desired plasma exposure window, and the initial PK model was validated on the confirmatory dataset. [\hyperlink{Busulfan}{PMID: 19049661}, L Nguyen et al., 2008]

\hypertarget{pmid_31706849}{B}usulfan is a bifunctional alkylating agent used for myeloablative conditioning and in the treatment of chronic myeloid leukemia due to its ability to cause DNA damage. However, in rodent experiments, busulfan presented a potential teratogenic and cytotoxic effect. Studies have evaluated the effects of busulfan on fetuses after administration in pregnancy or directly on pups during the lactation period. There are no studies on the effects of busulfan administration during pregnancy on offspring development after birth. We investigated the effects of busulfan on somatic and reflex development and encephalic morphology in young rats after exposure in pregnancy. The pregnant rats were exposed to busulfan (10 mg/kg, intraperitoneal) during the early developmental stage (days 12-14 of the gestational period). After birth, we evaluated the somatic growth, maturation of physical features and reflex-ontogeny during the lactation period. We also assessed the effects of busulfan on encephalic weight and cortical morphometry at 28 days of postnatal life. As a result, busulfan-induced pathological changes included: microcephaly, evaluated by the reduction of cranial axes, delay in reflex maturation and physical features, as well as a decrease in the morphometric parameters of somatosensory and motor cortex. Thus, these results suggest that the administration of a DNA alkylating agent, such as busulfan, during the gestational period can cause damage to the central nervous system in the pups throughout their postnatal development. [\hyperlink{Busulfan}{PMID: 31706849}, Henrique J C B Gouveia et al., 2020]

\hypertarget{pmid_10642802}{W}e conducted a prospective pilot study to assess the feasibility and safety of high-dose busulfan/melphalan as conditioning therapy prior to autologous PBPC transplantation in pediatric patients with high-risk solid tumors. From January 1995 to January 1999, 30 patients aged 2-21 years (median 8) were entered into the study. There were 14 females and 16 males. Diagnoses included neuroblastoma in 10 patients; Ewing's sarcoma and peripheral neuroectodermal tumor (PNET) in 15 patients and rhabdomyosarcoma in five patients. Treatment consisted of busulfan 16 mg/kg, orally over 4 days (from days -5 to -2) in 6 hourly divided doses, and melphalan at a dose of 140 mg/m2 given by intravenous infusion over 5 min on day -1. G-CSF mobilized PBPC were used as autologous stem-cell rescue. One patient developed a single generalized convulsion during busulfan therapy. The most relevant non-hematologic toxicity was gastrointestinal, manifesting as grade 2-3 mucositis and diarrhea in 12 patients. Two patients died of procedure-related complications, one from veno-occlusive disease of liver and multiorgan failure and the other from adult respiratory distress syndrome. Probability of treatment-related mortality was 6.6 +/- 4.5\%. With a median follow-up of 18 months (range, 1-48), 19 patients are alive and disease-free, the actuarial EFS at 4 years being 55 +/- 12\% for the whole group. We conclude that high-dose busulfan/melphalan for autologous transplantation in children with solid tumors is feasible even in small patients. It is well-tolerated, with an acceptable transplant-related mortality and has proven antitumor activity. [\hyperlink{Busulfan}{PMID: 10642802}, M A Diaz et al., 1999]

\hypertarget{pmid_22742881}{B}usulfan is widely used in a neuroblastoma setting, with several studies reporting marked inter-patient variability in busulfan pharmacokinetics and pharmacodynamics. The current study reports on the pharmacokinetics of oral versus intravenous (IV) busulfan in high-risk neuroblastoma patients treated on the European HR-NBL-1/SIOPEN study. Busulfan was administered four times daily for 4 days to children aged 0.7-13.1 years, either orally (1.45-1.55 mg/kg) or by the IV route (0.8-1.2mg/kg according to body weight strata). Blood samples were obtained prior to administration, 2, 4, and 6h after the start of administration on dose 1. Busulfan analysis was carried out by gas chromatography-mass spectrometry and data analysed using a NONMEM population pharmacokinetic approach. Busulfan plasma concentrations obtained from 38 patients receiving IV busulfan and 25 patients receiving oral busulfan, were fitted simultaneously using a one-compartment pharmacokinetic model. Lower variability in drug exposure was observed following IV administration, with a mean busulfan area under the plasma concentration versus time curve (AUC) of 1146 ± 187 μM.min (range 838-1622), as compared to 953 ± 290 μM.min (range 434-1427) following oral busulfan. A total of 87\% of children treated with IV busulfan achieved AUC values within the target of 900-1500 μM.min versus 56\% of patients following oral busulfan. Busulfan AUC values were significantly higher in HR-NBL-1/SIOPEN trial patients who experienced hepatic toxicity or veno-occlusive disease (VOD) (1177 ± 189 μM.min versus 913 ± 256 μM.min; p=0.0086). Further stratification based on route of administration suggested that the incidence of hepatic toxicity was related to both high busulfan AUC and oral drug administration. The reduced pharmacokinetic variability and improved control of busulfan AUC observed following IV administration support its utility within the ongoing HR-NBL-1/SIOPEN trial. [\hyperlink{Busulfan}{PMID: 22742881}, G J Veal et al., 2012]

\hypertarget{pmid_11966670}{I}ntravenous formulations of busulfan have recently become available. Although busulfan is used frequently in children as part of a myeloablative regimen prior to bone marrow transplantation, pharmacokinetic data on intravenous busulfan in children are scarce. The aim was to investigate intravenous busulfan pharmacokinetics in children and to suggest a limited sampling strategy in order to determine busulfan systemic exposure with the minimum of inconvenience and risk for the patient. Plasma pharmacokinetics after the first administration was investigated in six children using nonlinear mixed effect modelling. Pharmacokinetics showed little variability and were described adequately with a one-compartment model (population estimates CL,av=0.29 l h(-1) kg(-1); V,av=0.84 l kg(-1); t(1/2)=1.7-2.8 h). Combined with limited sampling and a Bayesian fitting procedure, the model can adequately estimate the systemic exposure to intravenous busulfan, which in children appears to be at the lower end of the adult range. Busulfan systemic exposure in children during intravenous administration can be estimated adequately with limited sampling and a Bayesian fitting procedure from a one-compartment model. Intravenous busulfan pharmacokinetics in children should be the subject of more research. [\hyperlink{Busulfan}{PMID: 11966670}, Serge Cremers et al., 2002]

\hypertarget{pmid_24791064}{B}usulfan is an antineoplastic bifunctional alkylating agent. We previously reported the busulfan-induced systemic histopathological changes in fetal rats and the sequence of brain lesions in fetal and infant rats. In the present study, in order to clarify the nature and sequence of busulfan-induced systemic histopathological changes in infant rats, 6-day-old male infant rats were subcutaneously administered 20 mg/kg of busulfan and histopathologically examined at 1, 2, 4, 7 and 14 days after treatment (DAT). As a result, histopathological changes characterized by pyknosis of component cells were observed in the heart, lungs, stomach, intestines, liver, kidneys, testes, epididymides, hematopoietic and lymphoid tissues, dorsal skin and femur as well as in the brain and eyes (data not shown in this paper). Such pyknosis transiently appeared until 7 DAT with prominence at 2 and/or 4 DAT in each tissue, except for the thymus, in which pyknosis peaked at 1 DAT. Most of the pyknotic nuclei were immunohistochemically positive for cleaved caspase-3, indicating that pyknotic cells were apoptotic. Different from the reports of fetal and adult rats, apoptosis was also found in cardiomyocytes and osteoblasts in infant rats.  [\hyperlink{Busulfan}{PMID: 24791064}, Toko Ohira et al., 2014] Busulphan levels in plasma were measured in 27 patients during conditioning therapy (1 mg/kg x 4 for 4 days) before bone marrow transplantation. The mean minimal concentration found in children aged less than 5 years (237 ng ml-1) was lower than that observed in adults or older children (607 and 573 ng ml-1, respectively). The AUC for the last dose was significantly lower in young children (2.315 h ng ml-1) than in adults or older children (6,134 and 5,937 h ng ml-1, respectively). The elimination half-life for the last dose in young children was shorter (2.05 h) than that in either adults (2.59 h) or older children (2.79 h). When the AUC was normalized for body surface area, the difference between young children and the other groups was smaller but remained statistically significant. The total body clearance was significantly higher in young children (7.3 ml min-1 kg-1) as compared with both older children and adults (3.02 and 2.7 ml min-1 kg-1, respectively). The plasma levels of busulphan showed circadian rhythmicity, especially in young children. The concentration measured during the night in some patients was up to 3-fold that observed during daytime. We conclude that the busulphan dosage for children must be reconsidered and that further studies are urgently needed to develop an optimal therapy. [\hyperlink{Busulfan}{PMID: 24791064}, M Hassan et al., 1991]

\hypertarget{pmid_7803883}{T}o review the current published studies evaluating the pharmacokinetics, clinical efficacy, safety, and toxicity of busulfan in pediatric and adult patients. English-language literature published between 1953 and 1993 was analyzed; pertinent literature was reviewed. Emphasis was placed on pharmacologic studies and clinical trials involving busulfan therapy both in myeloproliferative disorders and in conditioning regimens for autologous or allogeneic bone marrow transplantation. Data from both pediatric and adult studies were evaluated; emphasis was placed on the relationship between plasma concentrations of busulfan and its efficacy and toxicity. Busulfan has been used widely at conventional dosages (1-12 mg/d) for the treatment of patients with chronic myelogenous leukemia (CML). Busulfan at high doses (usually 16 mg/kg) given with other cytotoxic drugs (especially cyclophosphamide) is a common preparative regimen in patients undergoing allogeneic or autologous bone marrow transplantation (BMT) for acute or chronic leukemia and other nonmalignant disorders (e.g., hemoglobinopathies, inborn error of immune system, congenital metabolic disorders). Pharmacokinetics of high-dose busulfan are age-dependent. Busulfan systemic exposure and, thus, tissue and tumor exposure are lower in children than with adults. Relationships between toxicity (principally neutropenia, hepatic veno-occlusive disease, incidence of seizures) and drug exposure were found for busulfan. Busulfan is a useful, sufficiently safe drug in the treatment of patients with CML. At higher dosages, busulfan is a fundamental part of myeloablative therapies for patients undergoing BMT. As the pharmacokinetics and metabolism of busulfan is further understood, there is great potential for improving treatment outcome. An assessment of maximal tolerated exposure determined by therapeutic drug monitoring may decrease the incidence and lethality of regimen-related toxicities. [\hyperlink{Busulfan}{PMID: 7803883}, I Buggia et al., 1994]

\hypertarget{pmid_24201160}{B}usulphan (BU) is associated with neurotoxicity and risk of seizures. Hence, seizure prophylaxis is routinely utilized during BU administration for stem cell transplantation (SCT). We collected data on the incidence of seizures among children undergoing SCT in Italy. Fourteen pediatric transplantation centers agreed to report unselected data on children receiving BU as part of the conditioning regimen for SCT between 2005 and 2012. Data on 954 pediatric transplantation procedures were collected; of them, 66\% of the patients received BU orally, and the remaining 34\%, i.v. All the patients received prophylaxis of seizures, according to local protocols, consisting of different schedules and drugs. A total of 13 patients (1.3\%) developed seizures; of them, 3 had a history of epilepsy (or other seizure-related pre-existing condition); 3 had documented brain lesions potentially causing seizures per se; 1 had febrile seizures, 1 severe hypo-osmolality. In the remaining 5 patients, seizures were considered not explained and, thus, potentially related to BU administration. The incidence of seizures in children receiving BU-containing regimen was very low (1.3\%); furthermore, most of them had at least 1-either pre-existing or concurrent-associated risk factor for seizures.  [\hyperlink{Busulfan}{PMID: 24201160}, Désirée Caselli et al., 2014] Busulfan disposition is age-dependent with a higher clearance and a larger volume of distribution in children than in adults. The optimal dosage of busulfan needed to achieve bone marrow (BM) displacement in young children with malignant or nonmalignant disease remains to be defined. Using a gas chromatography-mass spectrometry assay, we evaluated plasma pharmacokinetics of busulfan in 33 children (median age, 9 months; range, 2 months to 2.75 years) with immune deficiencies, lysosomal storage diseases, acute leukemias, and malignant lymphohistiocytosis after an oral dose ranging from 0.9 to 2.6 mg/kg. The busulfan clearance (assuming a bioavailability of 1) ranged from 2.1 to 13.4 mL/min/kg with a mean of 6.8 mL/min/kg, which is higher than that reported in older children (4.5 mL/min/kg) and adults (2.9 mL/min/kg). Six children with lysosomal storage disease (5 with Hurler's disease, 1 with San Filippo's disease) had a prolonged elimination half-life (4.9 v 2.4 hours), a larger volume of distribution (3.4 v 1.2 L/kg) and a faster clearance (8.7 v 6.3 mL/min/kg) than the other 27 children. This suggests that a higher dose of busulfan will be required to achieve BM displacement in children with lysosomal storage disease. Over the dose range of 0.9 to 2.6 mg/kg, busulfan pharmacokinetics were linear. However, only 46\% of the interpatient variation in systemic exposure could be ascribed to the dose. Given the wide interpatient variability in busulfan disposition, dose adjustment and drug monitoring will be needed to achieve the optimal dosage of busulfan in young children. The plasma busulfan levels required to achieve BM displacement need to be defined, especially in lysosomal storage diseases. [\hyperlink{Busulfan}{PMID: 24201160}, G Vassal et al., 1993]

\hypertarget{pmid_17001185}{W}e studied the pharmacokinetics and clinical outcome of a new once-daily intravenous area under the curve-targeted dosing scheme for busulfan based on body surface area. Eighteen children undergoing busulfan-based conditioning for allogeneic stem cell transplantation were enrolled. The age of the children ranged from 0.5 to 16 years. For all children, the starting dose was 80 mg/m. Unlimited dose adjustment was allowed to reach the target area under the curve (3800 micromol/l . min). This target area under the curve was determined on the basis of a previous study in our hospital. Pharmacokinetic studies were performed after the first dose. The median area under the curve on day 1 was 2616 (range 1781-5040) micromol/l . min at a dose of 80 mg/m. This resulted in a median dose increment to 114 (range 62-168) mg/m to reach the target area under the curve. In only one patient, the dose was decreased. Donor engraftment was established in 14 out of 18 patients (78\%). Two of the four patients were successfully retransplanted. Relapse occurred in two patients (one died, one received additional treatment). Fourteen patients survived with a median follow-up of 1.6 years (1.0-2.2 years). The disease-free survival was 66\% (12 of 18 patients). Despite the high systemic peak levels, there was no new unexpected or unusual toxicity. Moderate veno-occlusive disease was seen in one patient only. We conclude that intravenous busulfan in children administered once daily is safe, convenient and feasible, and can be dosed surface-based, independent of age. There was very limited (liver) toxicity, but the rejection rate was relative high, which can be probably overcome by a higher exposure to busulfan. Future investigations should be aimed at further optimizing the target area under the curve of intravenous busulfan for specific patient groups. [\hyperlink{Busulfan}{PMID: 17001185}, Juliette Zwaveling et al., 2006]

\hypertarget{pmid_20677921}{H}igh-dose Busulfan in combination chemotherapy has been used commonly for hematopoietic stem cell transplantation. It crosses the blood-brain barrier and could cause seizure. Benzodiazepines have been used as anticonvulsant prophylaxis. This is a prospective study using oral lorazepam together with busulfan-based conditioning regimen in 30 children undergoing hematopoietic stem cell transplantation. The dose of lorazepam used ranged from 0.017 to 0.039 mg/kg (median = 0.026 mg/kg) per dose. None of the patients developed seizure while receiving oral lorazepam or within 72 hours of the last dose of Busulfan. Oral lorazepam was tolerated by the patients, but all patients needed dose reduction due to some adverse effects. In the authors' experience, oral lorazepam is a useful anticonvulsant prophylaxis for children receiving high-dose busulfan. [\hyperlink{Busulfan}{PMID: 20677921}, Amir Ali Hamidieh et al., 2010]

\hypertarget{pmid_24029650}{L}ittle information is currently available regarding the pharmacokinetics (PK) of busulfan in infants and small children to help guide decisions for safe and efficacious drug therapy. The objective of this study was to develop an algorithm for individualized dosing of i.v. busulfan in infants and children weighing ≤12 kg, that would achieve targeted exposure with the first dose of busulfan. Population PK modeling was conducted using intensive time-concentration data collected through the routine therapeutic drug monitoring of busulfan in 149 patients from 8 centers. Busulfan PK was well described by a 1-compartment base model with linear elimination. The important clinical covariates affecting busulfan PK were actual body weight and age. Based on our model, the predicted clearance of busulfan increases approximately 1.7-fold between 6 weeks to 2 years of life. For infants age <5 months, the model-predicted doses (mg/kg) required to achieve a therapeutic concentration at steady state of 600-900 ng/mL (area under the curve range, 900-1350 μM·min) were much lower compared with standard busulfan doses of 1.1 mg/kg. These results could help guide clinicians and inform better dosing decisions for busulfan in young infants and small children undergoing hematopoietic cell transplantation.  [\hyperlink{Busulfan}{PMID: 24029650}, Radojka M Savic et al., 2013] Busulphan (1, 4-bis [methanesulfonyl-y] butane) is a bi-functional alkylating agent that, in combination with cyclophosphamide, has been commonly used in conditioning regimens before hematological stem cell transplantation (HSCT) for nearly 20 years. Busulfan has a very narrow therapeutic index, and acute toxicity may be related to absorption and disposition of the drug and metabolites. Precise delivery of the oral formulation is compromised by erratic gastrointestinal absorption, particularly in infants and small children. An intravenous busulfan formula was approved nearly 40 years after the approval of the oral formulation. Busulfan levels expressed as the area under the concentration-time curve (AUC) higher than 1500 microM* minute were reported to increase the risk of developing veno-occlusive disease (VOD), while low levels may result in engraftment failure or disease relapse. VOD occurs in 11-40\% of patients undergoing HSCT and is associated with death in 3.3\% of patients. Measurement of individual plasma busulfan levels during oral or intravenous dosing to obtain an AUC is likely to provide the necessary elements to monitor the drug disposition, ensuring efficacy and preventing toxicity of patients undergoing HSCT. It is also important to consider the busulfan drug-drug interactions and adverse drug reactions that can develop during the therapeutic process. Busulfan therapeutic drug monitoring and dose-adjustment should be performed in specialized laboratories staffed by well-trained personnel. [\hyperlink{Busulfan}{PMID: 24029650}, Norberto Krivoy et al., 2008]

\hypertarget{pmid_32312247}{B}usulfan (Bu) is a key component of several conditioning regimens used before hematopoietic stem cell transplantation (HSCT). However, the optimum systemic exposure (expressed as the area under the concentration-time curve [AUC]) of Bu for clinical outcome in children is controversial. Research on pertinent literature was carried out at PubMed, EMBASE, Web of science, the Cochrane Library and ClinicalTrials.gov. Observational studies were included, which compared clinical outcomes above and below the area under the concentration-time curve (AUC) cut-off value, which we set as 800, 900, 1000, 1125, 1350, and 1500 μM × min. The primary efficacy outcome was notable in the rate of graft failure. In the safety outcomes, incidents of veno-occlusive disease (VOD) were recorded, as well as other adverse events. Thirteen studies involving 548 pediatric patients (aged 0.3-18 years) were included. Pooled results showed that, compared with the mean Bu AUC (i.e., the average value of AUC measured multiple times for each patient) of > 900 μM × min, the mean AUC value of < 900 μM × min significantly increased the incidence of graft failure (RR = 3.666, 95\% CI: 1.419, 9.467). The incidence of VOD was significantly decreased with the mean AUC < 1350 μM × min (RR = 0.370, 95\% CI: 0.205-0.666) and < 1500 μM × min (RR = 0.409, 95\% CI: 0182-0.920). In children, Bu mean AUC above the cut-off value of 900 μM × min (after every 6-h dosing) was associated with decreased rates of graft failure, while the cut-off value of 1350 μM × min were associated with increased risk of VOD, particularly for the patients without VOD prophylaxis therapy. Further well-designed prospective and multi centric randomized controlled trials with larger sample size are necessary before putting our result into clinical practices. [\hyperlink{Busulfan}{PMID: 32312247}, Xinying Feng et al., 2020]

\hypertarget{pmid_15181963}{S}ome studies have proved that intravenous busulfan with cyclophosphamide (used as a component of conditioning regimens for hematopoietic stem cell transplantation) is safer and has fewer complication than oral busulfan in adults, whereas the same proof in pediatric patients is only limited, with no reported data so far from Asian countries. In this study, we aimed to evaluate the efficacy and complications of IV busulfan in pediatric patients. Three pediatric patients with acute myeloid leukemia were treated by intravenous busulfan combined with cyclophosphamide to compare retrospectively with the treatment with oral busulfan plus intravenous cyclophosphamide in another three pediatric cases having transplantation in the same institute. The results showed that the intravenous busulfan-based regimen had better compliance and less adverse effects including mucositis, hepatic toxicity, transplant-related hepatic veno-occlusive disease, and acute graft-versus-host disease than oral busulfan-based treatment. The conditioning regimen of intravenous busulfan combined with cyclophosphamide is an acceptable alternative for pediatric patients with hematological malignancies in Taiwan. The long-term benefit and adverse effects of intravenous busulfan should be further explored. [\hyperlink{Busulfan}{PMID: 15181963}, Ming-Yang Lee et al., 2004]

\hypertarget{pmid_24174393}{B}usulfan (Bu) is a DNA-alkylating agent used for myeloablative conditioning in stem cell transplantation in children and adults. While the use of intravenous rather than oral administration of Bu has reduced inter-individual variability in plasma levels, toxicity still occurs frequently after hematopoietic stem cell transplantation (HSCT). Toxicity (especially hepatotoxic effects) of intravenous (IV) Bu may be related to both Bu and/or N,N-dimethylacetamide (DMA), the solvent of Bu. In this study, we assessed the relation between the exposure of Bu and DMA with regards to the clinical outcome in children from two cohorts. In a two-centre study Bu and DMA AUC (area under the curve) were correlated in pediatric stem cell recipients to the risk of developing SOS and to the clinical outcome. In patients receiving Bu four times per day Bu levels >1,500 µmol/L minute correlate to an increased risk of developing a SOS. In the collective cohort, summarizing data of all 53 patients of this study, neither high area under the curve (AUC) of Bu nor high AUC of DMA appears to be an independent risk factor for the development of SOS in children. In this study neither Bu nor DMA was observed as an independent risk factor for the development of SOS. To identify subgroups (e.g., infants), in which Bu or DMA might be risk factors for the induction of SOS, larger cohorts have to be evaluated. [\hyperlink{Busulfan}{PMID: 24174393}, Kornelius Kerl et al., 2014]

\hypertarget{pmid_9334898}{B}uspirone is a nonbenzodiazepine anxiolytic that has been effective in uncontrolled trials for treating childhood anxiety disorders. A 4-year-old boy with a history of laryngomalacia (congenital structural abnormality with airway collapse and obstruction on inhalation), pharyngeal dysphagia (difficulty in swallowing), poor weight gain, delayed self-feeding skills, and anxiety symptoms is described. An open trial of buspirone, increased gradually to 12.5 mg daily in divided doses over a period of 22 weeks, was associated with decreased anxiety, improved self-feeding skills, and weight gain. Based on parental reports, buspirone appeared to decrease separation and social anxiety, as well as anxiety associated with eating. Drug discontinuation was associated with symptom relapse, whereas drug readministration lead to the same clinical benefits that had been observed previously. The medication was well tolerated, and its benefits have persisted for over 1 year. No new recommendations can be made regarding the use of buspirone in preschool children or in the treatment of anxious behaviors adversely affecting medical conditions in children and adolescents. [\hyperlink{Busulfan}{PMID: 9334898}, G L Hanna et al., 1997]

\hypertarget{pmid_22455797}{T}he wide variability in pharmacokinetics of busulfan in children is one factor influencing outcomes such as toxicity and event-free survival. A meta-analysis was conducted to describe the pharmacokinetics of busulfan in patients from 0.1 to 26 years of age, elucidate patient characteristics that explain the variability in exposure between patients and optimize dosing accordingly. Data were collected from 245 consecutive patients (from 3 to 100 kg) who underwent haematopoietic stem cell transplantation (HSCT) in four participating centres. The inter-patient, inter-occasion and residual variability in the pharmacokinetics of busulfan were estimated with a population analysis using the nonlinear mixed-effects modelling software NONMEM VI. Covariates were selected on the basis of their known or theoretical relationships with busulfan pharmacokinetics and were plotted independently against the individual pharmacokinetic parameters and the weighted residuals of the model without covariates to visualize relations. Potential covariates were formally tested in the model. In a two-compartment model, body weight was the most predictive covariate for clearance, volume of distribution and inter-compartmental clearance and explained 65\%, 75\% and 40\% of the observed variability, respectively. The relationship between body weight and clearance was characterized best using an allometric equation with a scaling exponent that changed with body weight from 1.2 in neonates to 0.55 in young adults. This implies that an increase in body weight in neonates results in a larger increase in busulfan clearance than an increase in body weight in older children or adults. Clearance on the first day was 12\% higher than that of subsequent days (p < 0.001). Inter-occasion variability on clearance was 15\% between the 4 days. Based on the final pharmacokinetic-model, an individualized dosing nomogram was developed. The model-based individual dosing nomogram is expected to result in predictive busulfan exposures in patients ranging between 3 and 65 kg and thereby to a safer and more effective conditioning regimen for HSCT in children. [\hyperlink{Busulfan}{PMID: 22455797}, Imke H Bartelink et al., 2012]

\hypertarget{pmid_10570028}{T}he relationship between age and busulfan apparent oral clearance (Cl/F) expressed relative to adjusted ideal body weight and body surface area (bsa) was evaluated in 135 children aged 0 to 16 years undergoing hematopoietic stem cell transplantation for various disorders. Busulfan plasma levels were measured by gas chromatography-mass spectrometry after the first daily dose of the 4-day dosing regimen. Cl/F expressed relative to adjusted ideal body weight (ml/min/kg) and bsa (ml/min/m(2)) was lower in 9- to 16-year-old (y.o.) compared with 0- to 4-y.o. children (49 and 30\%; p<.001). We hypothesized that the greater busulfan Cl/F observed in young children was in part due to enhanced (first-pass intestinal) metabolism. Busulfan conjugation rate was compared in incubations with human small intestinal biopsy specimens from healthy young (1- to 3-y.o.) and older (9- to 17-y.o.) children. Villin content in biopsy specimens was determined by Western blot and busulfan conjugation rate was expressed relative to villin content to control for differences in epithelial cell content in pinch biopsies. Intestinal biopsy specimens from young children had a 77\% higher busulfan conjugation rate (p =.037) compared with older children. We have previously shown that glutathione-S-transferase (GST) A1-1 is the major isoform involved in busulfan conjugation, and that this enzyme is expressed uniformly along the length of adult small intestine. Thus, the greater busulfan conjugation activity in intestinal biopsies of the young children was most likely due to enhanced GSTA1-1 expression. We conclude that age dependence in busulfan Cl/F appears to result at least in part from enhanced intestinal GSTA1-1 expression in young children. [\hyperlink{Busulfan}{PMID: 10570028}, J P Gibbs et al., 1999]

\hypertarget{pmid_8932835}{B}usulphan pharmacokinetics were investigated in 20 children, who underwent bone marrow transplantation for either leukemia or inherited disorders. Busulphan (1.90-6.02 mg/kg/day) was administered orally as a single dose or twice daily. Busulphan kinetics were found to be linear within the studied range. Children with inherited disorders eliminated busulphan significantly faster after the first and the last dose with half-lives (t1/2) of 1.93 and 1.71 h, respectively compared to children with leukemia (3.16 and 2.70 h, respectively). The area under plasma concentration curves (AUCs, corrected for mg/kg) as an expression for the systemic exposure of busulphan were significantly higher in children with leukemia, 22.4 and 19.04 mumol/l.h (5527 and 4690 ng.h.ml-1) after the first and the last dose, respectively, compared to 11.2 and 8.2 mumol/l.h (2768 and 2029 ng.h.ml-1) found in children with inherited disorders. The present results confirm those reported by others, ie busulphan pharmacokinetics can be influenced by the underlying disease and its status. Our population pharmacokinetic analysis showed a negative correlation between the weight corrected clearance and the age in both groups of children. However, clearance was about 42\% higher in children with inherited disorders compared to those with leukemia. To estimate AUC for the first dose, we evaluated a limited sampling model based on three concentrations (1, 3 and 6 h). A high correlation (r = 0.998, P < 0.0001, n = 40) between the estimated and the determined AUC was found. The present model is reliable and adequate for studying more patients, with a long-term follow-up combined with drug monitoring in correlation with drug efficacy and toxicity to define the optimal busulphan dosage required. [\hyperlink{Busulfan}{PMID: 8932835}, M Hassan et al., 1996]

\section*{Butorphanol Tartrate}
\subsection*{Result}
\subsubsection*{Answer}

Ages 6 months to 17 years: Yes
Ages 1 to 4 years: Yes
Ages 1.5 to 13 years: Yes
Ages 6 weeks to 7 years: Yes
Adolescents: Yes

\subsubsection*{{Explanation}}
\hypertarget{Butorphanol Tartrate}
Based on the available abstracts, several targeted studies have evaluated the safety and efficacy of Butorphanol Tartrate in children across different age ranges:

1. **Ages 1 to 4 years**: A study of 103 children aged 1 to 4 years undergoing various surgeries assessed the use of butorphanol tartrate (moradol) as part of combined anesthesia. The study monitored hemodynamic parameters, stress hormones, and metabolic markers, and found stable parameters and no serious side effects. The authors concluded that butorphanol provided effective analgesia with minimal impact on adrenal and thyroid function, and no serious adverse effects were observed [\hyperlink{pmid_9045578}{PMID: 9045578}, T S Agzamkhodzhaev et al.].

2. **Ages 6 months and older**: A double-blind, placebo-controlled study in 60 children aged 6 months or older undergoing bilateral myringotomy and tube placement (BMT) found that transnasal butorphanol at 25 mcg/kg provided significant postoperative pain relief compared to placebo, with fewer children requiring rescue analgesia. No serious adverse effects were reported [\hyperlink{pmid_9710397}{PMID: 9710397}, R E Bennie et al., 1998].

3. **Ages 8 to 17 years**: A small case series of eight children aged 8 to 17 years receiving transnasal butorphanol for postoperative pain after orthopedic and plastic surgery reported adequate analgesia with only mild, transient side effects (nausea, dizziness, bitter taste, mild throat irritation), none of which precluded further use [\hyperlink{pmid_8521312}{PMID: 8521312}, J D Tobias et al., 1995].

4. **Ages 1.5 to 13 years**: A randomized, double-blind study compared butorphanol and morphine in 156 children aged 1.5 to 13 years after elective surgery. Both drugs provided similar analgesia, with butorphanol associated with less postoperative vomiting. No significant safety concerns were reported [\hyperlink{pmid_7628027}{PMID: 7628027}, W M Splinter et al., 1995].

5. **Ages 6 weeks to 7 years**: A prospective, randomized, double-blind study in 60 postoperative pediatric patients aged 6 weeks to 7 years compared epidural and intravenous butorphanol for prevention of morphine side effects. The overall incidence of side effects was similar across groups, with sedation more frequent in the butorphanol groups but not severe. No serious adverse events were reported [\hyperlink{pmid_7639376}{PMID: 7639376}, A G Bailey et al., 1994].

6. **Ages not specified, but pediatric**: A retrospective study compared epidural butorphanol/bupivacaine to fentanyl/bupivacaine in 191 children after urological procedures. Analgesia was similar, with fewer side effects (notably pruritus) in the butorphanol group. No cases of clinically significant respiratory depression were reported in the butorphanol group [\hyperlink{pmid_21269000}{PMID: 21269000}, Alexandra Szabova et al.].

7. **Adolescents**: A pilot study in 27 adolescents with postoperative orthopedic pain found that intramuscular butorphanol provided good or excellent pain relief in 89\% of patients, with only urinary retention as a notable adverse effect (possibly related to other medications or surgery). Tolerance and safety were rated as good or excellent in all patients [\hyperlink{pmid_3079008}{PMID: 3079008}, N L Steg et al., 1988].

In summary, multiple targeted studies in children from 6 months to 17 years have evaluated the safety of Butorphanol Tartrate for perioperative and postoperative pain management. These studies consistently report effective analgesia and a favorable safety profile, with no reports of serious or life-threatening adverse effects in the pediatric populations studied. Mild side effects such as sedation, nausea, and dizziness were observed but were not severe or limiting. Therefore, based on the evidence from these abstracts, Butorphanol Tartrate is affirmed as safe for use in children within the studied age ranges.

\subsection*{Abstracts}
\hypertarget{pmid_9045578}{T}he purpose of this study was to assess the efficacy of moradol (butorphanol tartrate) as an analgetic component of combined total anesthesia in children. The adequacy of anesthesia was assessed by echography, electrocardiography, and electroencephalography, measurements of stress hormones, electrolyte balance, and metabolic parameters at various stages of anesthesia and surgery. A total of 103 patients aged 1 to 4 years were examined, subjected to abdominal, thoracal, urological, orthopaedic, and ENT surgery. The stability of the basic hemodynamic parameters and some stress hormones in the blood during surgery indicated effective protection of the organism from surgical trauma, on the one hand, and evidenced the minimal effect of anesthesia on the adrenal and thyroid function, on the other. A single injection of moradol during induction narcosis was sufficient for 2.5 to 3 hours of surgery. No serious side effects of the drug were observed. Prolonged (10 hours on average) anesthesia persisted after the operation, this decreasing the postoperative use of analgesics. [\hyperlink{Butorphanol Tartrate}{PMID: 9045578}, T S Agzamkhodzhaev et al., ]

\hypertarget{pmid_8458045}{B}utorphanol tartrate, a synthetically derived opioid agonist-antagonist analgesic, was tested in a large group of postpartum women (N = 76) to assess the safety and analgesic efficacy of a recently approved transnasal preparation of this drug in the relief of postepisiotomy pain. The safety and efficacy of intravenous and intramuscular administration of butorphanol tartrate has been established over 14 years of clinical use. The new nasal spray dosage form offers a similar degree of efficacy with a rapid onset of action. Compared with the injectables and other drugs in this class, transnasal butorphanol has a longer duration of action (4 to 5 hours). In this double-blind, parallel-group, dose-response study, 76 female patients ages 17 to 37 years with moderate to severe postepisiotomy pain were randomly assigned to receive a single dose of transnasal butorphanol (0.25, 0.5, 1, or 2 mg) or placebo. The patients were evaluated for 6 hours. The results of the study indicate that the 1-mg and 2-mg doses were associated with greater efficacy compared with placebo using several markers for efficacy, including the pain relief score and time to remedication. The drug was well tolerated, dizziness and drowsiness being the most frequently reported adverse effects. Adverse effects appeared to be dose related. [\hyperlink{Butorphanol Tartrate}{PMID: 8458045}, T H Joyce et al., ]

\hypertarget{pmid_9710397}{M}ore than 70\% of children require analgesics after bilateral myringotomy and tube placement (BMT). Because anesthesia for BMT is generally provided by face mask without placement of an intravenous catheter, an alternative route for analgesia administration is needed. Transnasal butorphanol is effective in relieving postoperative pain in adults and children. The effectiveness of transnasal butorphanol for postoperative pain management in children undergoing BMT was studied. This double-blinded, placebo-controlled study compared the postoperative analgesic effects of transnasal butorphanol administered after the induction of anesthesia. Sixty children classified as American Society of Anesthesiologists physical status 1 or 2 who were aged 6 months or older and scheduled for elective BMT were randomized to receive transnasal placebo or 5, 15, or 25 microg/kg butorphanol. Postoperative pain was assessed using the Children's Hospital of Eastern Ontario Pain Scale (CHEOPS) on arrival in the postanesthesia care unit and at 5, 10, 15, 30, 45, and 60 min. The CHEOP scores were significantly less in the 25 microg/kg transnasal butorphanol group compared with controls. Significantly fewer children received rescue analgesia in the 25 microg/kg transnasal butorphanol group compared with controls (n = 1 and 8, respectively; P = 0.02). Transnasal butorphanol given in a dose of 25 microg/kg after induction of anesthesia provided adequate postoperative pain relief in children undergoing BMT. [\hyperlink{Butorphanol Tartrate}{PMID: 9710397}, R E Bennie et al., 1998]

\hypertarget{pmid_8521312}{T}he authors present their experience with transnasal butorphanol to provide analgesia following orthopaedic and plastic surgical procedures in children. Transnasal butorphanol was administered to eight patients ranging in age from eight to 17 years and in weight from 34 to 64 kg. Following the surgical procedure, the patient and a parent were instructed on how to use the medication. They were instructed to administer one spray into one nostril every three h as needed for pain. The quality of analgesia was assessed twice a day using a visual analogue score of 0 to 10 (0 = no pain, 10 = worst pain imaginable). Intranasal butorphanol provided adequate analgesia in all eight patients with visual analogue scores of zero to two. Adverse effects from the medication included one report of nausea, one complaint of transient dizziness, and two reports of a bitter taste and some mild throat irritation. None of these was severe enough to preclude its subsequent use. Our preliminary experience suggests that transnasal butorphanol may offer an alternative route of delivery when intravenous administration is not feasible. Future studies are needed to compare its efficacy to intravenous and non-parenteral routes of administration. It may prove to be useful in other situations when intravenous access is lacking such as prior to invasive procedures in the outpatient clinic or emergency room. [\hyperlink{Butorphanol Tartrate}{PMID: 8521312}, J D Tobias et al., 1995]

\hypertarget{pmid_21269000}{T}he aim of this retrospective study is to compare safety and efficacy of postoperative epidural butorphanol/bupivacaine with the gold-standard epidural analgesic infusion fentanyl/bupivacaine in children. With the Institutional Review Board's approval, the authors searched their Pain Management Database and divided children who received epidural analgesia into two groups. Each butorphanol group subject was matched with two fentanyl group subjects. Demographic data, pain scores, epidural interventions, epidural side effects, use of rescue opioid analgesia and adjuvant analgesics, causes of epidural failure, time of first oral intake and ambulation, and length of stay were statistically compared. A total of 191 patients were identified between 2000 and 2007; 58 in epidural butorphanol/bupivacaine and 133 in fentanyl/bupivacaine groups. Demographic data were comparable between the groups. The number of children with good pain control on postoperative days 1 and 2 in butorphanol (84 and 82 percent) and fentanyl (93 and 91 percent) groups were statistically similar (p = 0.06 and 0.13, respectively). Incidences of epidural side effects, especially pruritus, were significantly higher in the fentanyl group. Significantly more children in the butorphanol group required epidural rate changes when compared with those in the fentanyl group. Incidence of failed epidurals was significantly higher in the fentanyl group when compared with that in the butorphanol group. Clinically significant respiratory depression occurred in two children in the fentanyl group and in none of the children in the butorphanol group (p > 0.99). Epidural butorphanol provided similar analgesia to epidural fentanyl after urological procedures in children, but butorphanol caused less pruritus than fentanyl. Epidural analgesia with butorphanol/bupivacaine is effective in children undergoing urological procedures. When compared with epidural fentanyl, epidural butorphanol causes significantly less itching. [\hyperlink{Butorphanol Tartrate}{PMID: 21269000}, Alexandra Szabova et al., ]

\hypertarget{pmid_24370240}{T}o evaluate antinociceptive effects and pharmacokinetics of butorphanol tartrate after IM administration to American kestrels (Falco sparverius). Fifteen 2- to 3-year-old American kestrels (6 males and 9 females). Butorphanol (1, 3, and 6 mg/kg) and saline (0.9\% NaCl) solution were administered IM to birds in a crossover experimental design. Agitation-sedation scores and foot withdrawal response to a thermal stimulus were determined 30 to 60 minutes before (baseline) and 0.5, 1.5, 3, and 6 hours after treatment. For the pharmacokinetic analysis, butorphanol (6 mg/kg, IM) was administered in the pectoral muscles of each of 12 birds. In male kestrels, butorphanol did not significantly increase thermal thresholds for foot withdrawal, compared with results for saline solution administration. However, at 1.5 hours after administration of 6 mg of butorphanol/kg, the thermal threshold was significantly decreased, compared with the baseline value. Foot withdrawal threshold for female kestrels after butorphanol administration did not differ significantly from that after saline solution administration. However, compared with the baseline value, withdrawal threshold was significantly increased for 1 mg/kg at 0.5 and 6 hours, 3 mg/kg at 6 hours, and 6 mg/kg at 3 hours. There were no significant differences in mean sedation-agitation scores, except for males at 1.5 hours after administration of 6 mg/kg. Butorphanol did not cause thermal antinociception suggestive of analgesia in American kestrels. Sex-dependent responses were identified. Further studies are needed to evaluate the analgesic effects of butorphanol in raptors. [\hyperlink{Butorphanol Tartrate}{PMID: 24370240}, David Sanchez-Migallon Guzman et al., 2014]

\hypertarget{pmid_2782703}{T}he effects of butorphanol tartrate on arterial pressure, jejunal blood flow, vascular resistance, oxygen extraction, and oxygen uptake were determined in 10 anesthetized ponies ventilated with a mixture of halothane and 100\% oxygen, using isolated autoperfused jejunal segments. Physiologic saline solution or butorphanol tartrate (0.2 mg/kg of body weight) was administered as a single bolus into the left jugular vein. By 2 minutes, butorphanol decreased arterial blood pressure and intestinal blood flow, and increased intestinal oxygen extraction. However, intestinal vascular resistance and oxygen uptake were unaffected. Results of this study indicate that butorphanol tartrate induces a hypotension that secondarily decreases intestinal blood flow, but intestinal vascular resistance and metabolism are not adversely affected. We conclude that butorphanol tartrate does not compromise intestinal viability in halothane-anesthetized ponies and, therefore, may be a good analgesic choice for the equid destined for abdominal surgery. [\hyperlink{Butorphanol Tartrate}{PMID: 2782703}, J A Stick et al., 1989]

\hypertarget{pmid_359109}{B}utorphanol tartrate 1 mg and 2 mg were compared in 80 normal mothers at term in a double-blind study with meperidine hydrochloride 40 mg and 80 mg for the relief of pain in labour. Butorphanol was found to be as effective as meperidine in relieving pain in labour. The foetal condition, as measured by ECG monitoring, Apgar scores, time to sustained respiration, umbilical venous H+ (pH) and PCO2, and a general nursery survey were comparable for meperidine and butorphanol. No psychomimetic phenomena were seen. Assays indicated that both butorphanol and meperidine crossed the placenta. The mean concentration of butorphanol in neonatal serum was 0.84 times maternal serum at 1.5 to 3.5 hours after intramuscular administration of a single or two successive doses of butorphanol 1 mg or 2 mg to the mother. The mean concentrations for meperidine in neonatal serum was 0.89 times maternal serum at 0.85 to 3.6 hours after intramuscular administration of meperidine 40 mg or 80 mg to the mother. Neither analgesic caused severe depression of the infant except for one meperidine-treated case. [\hyperlink{Butorphanol Tartrate}{PMID: 359109}, A L Maduska et al., 1978]

\hypertarget{pmid_2830756}{B}utorphanol tartrate is a highly effective opioid agonist-antagonist analgesic with qualitative as well as quantitative differences from the pure agonists. These differences are thought to be due to interaction with a distinct subset of opioid receptors. Although it relieves severe pain, the drug does not usually elevate mood, and it may occasionally cause dysphoria. Counterbalancing its disadvantages is a wealth of clinical experience with the drug showing an impressive record of safety. Butorphanol produces limited respiratory depression and smooth muscle spasm, and both effects are reversible with naloxone. The most prominent side effect is sedation, a property that is generally quite useful in the perioperative period. Butorphanol is a weak morphine antagonist, so it may interact with agonists like morphine or fentanyl. The chief advantages of this agent are its low toxicity and very low potential for abuse. [\hyperlink{Butorphanol Tartrate}{PMID: 2830756}, C E Rosow et al., 1988]

\hypertarget{pmid_15613167}{B}utorphanol tartrate is a synthetic mixed agonist-antagonist opioid analgesic. Its transnasal dosage form, which may be self-administered when the use of an opioid analgesic is appropriate, was previously shown to provide rapid relief of migraine pain. In this double-blind, parallel-group, outpatient study, we compared butorphanol nasal spray 1 mg followed in 1 hour by an optional second 1-mg dose with the orally administered analgesic, Fiorinal with Codeine (one capsule containing butalbital 50 mg, caffeine 40 mg, aspirin 325 mg, and codeine phosphate 30 mg). Patients (N=321) were assigned by randomization to one of two treatment groups (butorphanol or Fiorinal with Codeine) and instructed to self-administer medication when migraine pain reached an intensity of moderate or severe and to record study-related events in a diary for 24 hours posttreatment. Efficacy analyses were performed on data from 275 patients who took study medication and returned a patient diary; 136 in the butorphanol group and 139 in the Fiorinal with Codeine group. During the first 2 hours after treatment, butorphanol was more effective than Fiorinal with Codeine in treating migraine pain as measured by pain intensity difference scores, percentage of responders (pain decreased to mild or none), percentage of pain-free patients, and degree of pain relief, with a more rapid time to onset of 15 minutes. A similar percentage of patients in the two groups used rescue medication during the first 4 hours, after which more butorphanol-treated than Fiorinal with Codeine-treated patients used rescue medication. Butorphanol patients had more side effects, less improvement in digestive symptoms, and less improvement in functional ability than Fiorinal with Codeine patients. [\hyperlink{Butorphanol Tartrate}{PMID: 15613167}, J Goldstein et al., ]

\hypertarget{pmid_8330463}{T}he safety, tolerance, and pharmacokinetics of transnasal butorphanol were evaluated in a double-blind, multiple-dose phase I study. A total of 18 subjects received either placebo (n = 6) or a single transnasal dose of 2 mg butorphanol tartrate on the first day and 1, 2, and 4 mg doses of butorphanol tartrate every 6 hours on days 2 through 6, 7 through 11, and 12 through 16, respectively. Safety assessment was performed on days 7, 12, and 17. Serial blood samples were collected on days 1, 6, 11, and 16, and the plasma was analyzed for unchanged butorphanol by a validated and specific radioimmunoassay. Butorphanol was rapidly absorbed and peak levels in plasma were generally attained within 1 hour after the nasal administration. The values of maximum concentration, minimum concentration, and area under the concentration versus time curve from time zero to the dosing interval [AUC(0-tau)] increased as the administered dose increased in a dose-proportional manner. The values of AUC from time zero to infinity after a single dose of 2 mg butorphanol tartrate, 10.9 ng.hr/ml, were identical to the values of AUC(0-tau) after a multiple administration of 2 mg dose, 10.4 ng.hr/ml. Mean elimination half-life value was 5.45 hours. Steady state was reached in fewer than eight doses when given every 6 hours. Transnasal butorphanol was well tolerated by all subjects. After repeated administration of transnasal butorphanol, no significant changes were observed in the nasal examination, which included evaluation of color, wetness, and thickness of nostril membrane, air flow, airway patency, and general nasal conditions.(ABSTRACT TRUNCATED AT 250 WORDS) [\hyperlink{Butorphanol Tartrate}{PMID: 8330463}, W C Shyu et al., 1993] A pilot study was conducted to evaluate the use of butorphanol, administered intramuscularly in 0.7-mg to 3-mg doses, in 27 adolescents with postoperative orthopedic pain. Butorphanol provided good or excellent pain relief in 24 (89\%) patients. The duration of the relief was about three to four hours. The only adverse effect experienced in more than one patient was urinary retention, possibly associated with the use of fentanyl, which was administered for balanced anesthesia, and/or with the surgical procedure (spinal fusion). Tolerance and safety were rated as good or excellent in 100\% of the patients. [\hyperlink{Butorphanol Tartrate}{PMID: 8330463}, N L Steg et al., 1988]

\hypertarget{pmid_7639376}{W}e performed a prospective, randomized, double-blinded study in 60 postoperative pediatric patients aged 6 wk to 7 yr to compare the efficacy of butorphanol given epidurally or intravenously in preventing the side effects of epidural morphine. Three groups of patients received 60 micrograms/kg epidural morphine; 20 patients also received epidural butorphanol 30 micrograms/kg, and 20 patients also received 30 micrograms/kg intravenous butorphanol. All patients were evaluated for analgesia, sedation, vomiting, urinary retention, pruritus, and respiratory depression for 24 h postoperatively. Although the overall incidence of side effects was not different in the three groups, the epidural butorphanol group had a significant decrease in severity of pruritus. Sedation was seen more frequently in the groups receiving butorphanol, but was most pronounced in the epidural butorphanol group. We conclude that butorphanol has little or no effect on the side effects of epidural morphine. [\hyperlink{Butorphanol Tartrate}{PMID: 7639376}, A G Bailey et al., 1994]

\hypertarget{pmid_3344597}{T}he analgesic efficacy and safety of butorphanol tartrate are discussed in 2 groups of patients who underwent urological procedures. The first group of 83 patients is presented as a retrospective review of the postoperative use of butorphanol. The second group of patients was involved in a double-blind, randomized comparative trial of butorphanol (2 or 4 mg) and meperidine (80 mg) for the relief of moderate to severe pain due to renal colic. Eighty-three patients with documented upper urinary tract calculi were evaluated for efficacy; 120 patients were evaluated for safety. Butorphanol 4 mg (i.m.) was more effective than butorphanol 2 mg (i.m.) and equivalent to meperidine 80 mg (i.m.). There were no statistically significant differences among the three treatment groups in regard to side effects. Overall, in the urology patients studied, butorphanol was found to be an effective and well tolerated agent that possesses important safety advantages when compared with the narcotic analgesics. [\hyperlink{Butorphanol Tartrate}{PMID: 3344597}, H H Henry et al., 1988]

\hypertarget{pmid_15334604}{B}utorphanol (17-cyclobutylmethyl-3,14-dihydroxymorphinan) tartrate (Stadol) is a mixed agonist-antagonist opioid analgesic agent that is about five to seven times as potent as morphine in analgesic effects. The chronic use of butorphanol produces physical dependence in humans and animals. Phosphorylation plays a very important role in developing butorphanol dependence; however, global phosphorylation events induced by chronic butorphanol administration have not been reported. The aim of this study is to determine the alteration of tyrosine phosphorylation of brain frontal cortical proteins in butorphanol-dependent rats using a proteomic approach. Dependence was produced by continuous intracerebroventricular (i.c.v.) infusion of butorphanol (26 nmol/microl/hr) for 72 hr via osmotic minipump in rats. Similar patterns of protein expression were detected by two-dimensional electrophoresis (2-DE) in brain frontal cortex of butorphanol-dependent and saline-treated control rats. All 65 phosphotyrosyl (p-Tyr) protein spots detected in pH 3-10 phosphotyrosine 2-DE of control rat brains were detected in butorphanol-dependent rat brains. The densities of most p-Tyr protein spots were increased in butorphanol-dependent rat brains compared to saline-treated control samples. Eighteen additional p-Tyr protein spots were detected in pH 3-10 2-DE images of butorphanol-dependent rat brains. Immobilized pH strips with three different narrow pH ranges were examined to improve the resolution of p-Tyr proteins in 2-DE gels. Fifty-three p-Tyr protein spots were identified as known proteins involved in cell cytoskeleton, cell metabolism, and cell signaling. This proteomic approach can provide useful information for understanding the complex mechanism of butorphanol dependence in vivo. [\hyperlink{Butorphanol Tartrate}{PMID: 15334604}, Seong-Youl Kim et al., 2004]

\hypertarget{pmid_10791933}{T}o evaluate effects of butorphanol tartrate and buprenorphine hydrochloride on withdrawal threshold to a noxious stimulus in conscious African grey parrots. 29 African grey parrots (Psittacus erithacus erithacus and Psittacus erithacus timneh). Birds were fitted with an electrode on the medial metatarsal region of the right leg, placed into a test box, and allowed to acclimate. An electrical stimulus (range, 0.0 to 1.46 mA) was delivered to each bird's foot through an aluminum perch. A withdrawal response was recorded when the bird lifted its foot from the perch or vigorously flinched its wings. Baseline threshold to a noxious electrical stimulus was determined. Birds then were randomly assigned to receive an i.m. injection of saline (0.9\% NaCl) solution, butorphanol (1.0 mg/kg of body weight), or buprenorphine (0.1 mg/kg), and threshold values were determined again. Butorphanol significantly increased threshold value, but saline solution or buprenorphine did not significantly change threshold values. Butorphanol had an analgesic effect, significantly increasing the threshold to electrical stimuli in African grey parrots. Buprenorphine at the dosage used did not change the threshold to electrical stimulus. Butorphanol provided an analgesic response in half of the birds tested. Butorphanol would be expected to provide analgesia to African grey parrots in a clinical setting. [\hyperlink{Butorphanol Tartrate}{PMID: 10791933}, J R Paul-Murphy et al., 1999]

\hypertarget{pmid_7886096}{I}n the present series of studies we examined the effect of butorphanol tartrate on food-reinforced operant responding in satiated rats. In the first experiment, 8.0 mg/kg butorphanol was administered subcutaneously, once per day for 4 days, to satiated rats responding under an fixed ratio 10 (FR 10) reinforcement schedule. In the second experiment, butorphanol (0, 0.3, 1.0, 3.0, 10.0 mg/kg) was administered to satiated rats responding under an FR 80 (first pellet) FR 3 (subsequent pellets) reinforcement schedule for 4 consecutive days. Repeated butorphanol administration increased total amount of food consumed over sessions in both experiments. Under the FR 80 schedule component, butorphanol initially increased latency to acquire the first pellet, an effect attenuated by repeated administration. Whereas vehicle administration was associated with consumption of relatively large quantities of food within the first 10 min of receiving the first pellet, butorphanol was associated with continued feeding as the session progressed. These data suggest that butorphanol-induced food intake is associated with maintenance rather than initiation of feeding. [\hyperlink{Butorphanol Tartrate}{PMID: 7886096}, J M Rudski et al., 1994]

\hypertarget{pmid_359382}{B}utorphanol tartrate, a synthetic morphinan which has analgesic agonist and antagonist properties, was compared with meperidine for balanced anaesthesia. The two agents were found to be comparable in efficacy, maintenance of cardiovascular stability, and incidence of side-effects. Butorphanol has the advantage of being non-narcotic and having a lower propensity for addiction. Because of its antagonist properties, there appears to be a limit to its depressant effects on respiration. [\hyperlink{Butorphanol Tartrate}{PMID: 359382}, L C Stehling et al., 1978]

\hypertarget{pmid_7628027}{T}he purpose of this study was to compare the side effects and efficacy of equianalgesic doses of morphine (M) and butorphanol (B) in children undergoing similar surgical procedures associated with moderate postoperative pain. We studied 156 healthy children aged 1.5-13 yr who underwent elective inguinal herniorrhaphy or orchidopexy. After induction of anaesthesia subjects were given 150 micrograms.kg-1 M or 30 micrograms.kg-1 B following a randomized, stratified, blocked and double-blind design. A standardized anaesthetic was administered, which included 1.5\% halothane, vecuronium, droperidol and mechanical ventilation. The postsurgical four-hour follow-up included assessment of pain, vomiting and respiratory depression. Pain was assessed with mCHEOPS and analgesics were administered when indicated in the recovery room. Each opioid was administered to a group of 78 patients. Within each group, 25 subjects had an iv induction, 21 children had an orchidopexy and 57 had inguinal hernia repairs. The groups were similar with respect to age, weight, and length of surgery. The choice of opioid did not affect recovery times from anaesthesia. Analgesic requirements were similar among the groups. Ten minutes after arrival in the recovery room the B-subjects had a lower pain score than the M-patients. Postoperative vomiting was less among the B-subjects: 14\% vs 28\%, P = 0.03. Two M-patients required an unscheduled admission to hospital because of vomiting. It is concluded that butorphanol has few advantages over morphine in the population studied. [\hyperlink{Butorphanol Tartrate}{PMID: 7628027}, W M Splinter et al., 1995]

\hypertarget{pmid_34853785}{A}sthma is the most common chronic disease in children, many of whom are managed solely with a short-acting β The aim of this study is to determine the efficacy and safety of as-needed budesonide-formoterol therapy compared with as-needed salbutamol in children aged 5 to 15 years with mild asthma, who only use a SABA. A 52-week, open-label, parallel group, phase III RCT will recruit 380 children aged 5 to 15 years with mild asthma. Participants will be randomised 1:1 to either budesonide-formoterol (Symbicort Rapihaler This is the first RCT to assess the safety and efficacy of as-needed budesonide-formoterol in children with mild asthma. The results will provide a much-needed evidence base for the treatment of mild asthma in children. [\hyperlink{Butorphanol Tartrate}{PMID: 34853785}, Lee Hatter et al., 2021]

\hypertarget{pmid_9153436}{B}utorphanol (Stadol, Bristol-Meyers Squibb, Princeton, NJ) is a synthetically derived opiate. As a nasal spray, it was approved for release in 1991 and was subsequently promoted as a safe treatment for migraine. Since then, there have been numerous reports of problems with butorphanol similar to those of any narcotic, especially dependence-addiction and major psychological disturbances. These problems have been documented by the Food and Drug Administration, but the information can be obtained only through the Freedom of Information Act. The experience with butorphanol indicates the need for physicians to have additional sources of information about drugs than are presently available. [\hyperlink{Butorphanol Tartrate}{PMID: 9153436}, M A Fisher et al., 1997]

\hypertarget{pmid_24592110}{P}arenteral opioids can be administered with ease at a very low cost with high efficacy as labour analgesia. However, there are insufficient data available to accept the benefits of parenteral opioids over other proven methods of labour analgesia. Butorphanol, a new synthetic opioid, has emerged as a promising agent in terms of efficacy and a better safety profile. This study investigates the effect of butorphanol as a labour analgesia to gather further evidence of its safety and efficacy to pave the way for its widespread use in low resource settings. One hundred low risk term consenting pregnant women were recruited to take part in a prospective cohort study. Intramuscular injections of butorphanol tartrate 1 mg (Butrum 1/2mg, Aristo, Mumbai, India) were given in the active phase of labour and repeated two hourly. Pain relief was noted on a 10-point visual pain analogue scale (VPAS). Obstetric and neonatal outcome measures were mode of delivery, duration of labour, Apgar scores at 1 and 5 minutes and Neonatal Intensive Care Unit admissions. Collected data were analysed for statistically significant pain relief between pre- and post-administration VPAS scores and also for the incidence of adverse outcomes. Pain started to decrease significantly within 15 minutes of administration and reached the nadir (3.08 SD0.51) at the end of two hours. The pain remained below four on the VPAS until the end of six hours and was still significantly low after eight hours. The incidence of adverse outcomes was low in the present study. Butorphanol is an effective parenteral opioid analgesic which can be administered with reasonable safety for the mother and the neonate. The study has the drawback of lack of control and small sample size. [\hyperlink{Butorphanol Tartrate}{PMID: 24592110}, Ajay Halder et al., 2013]

\hypertarget{pmid_23126609}{P}ain management is an important component of foal nursing care, and no objective data currently exist regarding the analgesic efficacy of opioids in foals. To evaluate the somatic antinociceptive effects of 2 commonly used doses of intravenous (i.v.) butorphanol in healthy foals. Our hypothesis was that thermal nociceptive threshold would increase following i.v. butorphanol in a dose-dependent manner in both neonatal and older pony foals. Seven healthy neonatal pony foals (age 1-2 weeks), and 11 healthy older pony foals (age 4-8 weeks). Five foals were used during both age periods. Treatments, which included saline (0.5 ml), butorphanol (0.05 mg/kg bwt) and butorphanol (0.1 mg/kg bwt), were administered i.v. in a randomised crossover design with at least 2 days between treatments. Response variables included thermal nociceptive threshold, skin temperature and behaviour score. Data within each age period were analysed using a 2-way repeated measures ANOVA, followed by a Holm-Sidak multiple comparison procedure if warranted. There was a significant (P<0.05) increase in thermal threshold, relative to Time 0, following butorphanol (0.1 mg/kg bwt) administration in both age groups. No significant time or treatment effects were apparent for skin temperature. Significant time, but not treatment, effects were evident for behaviour score in both age groups. Butorphanol (0.1 mg/kg bwt, but not 0.05 mg/kg bwt) significantly increased thermal nociceptive threshold in neonatal and older foals without apparent adverse behavioural effects. Butorphanol shows analgesic potential in foals for management of somatic painful conditions. [\hyperlink{Butorphanol Tartrate}{PMID: 23126609}, K T McGowan et al., 2013]

\hypertarget{pmid_9470973}{B}utorphanol is an opioid used as analgesic in humans and other species. In horses, it can cause locomotor stimulation at low doses. This drug is not well chromatographed by GC and so, it is necessary to transform it into a more suitable compound, which can be done by derivatization. The derivatization of a drug is used to impart volatility, masking polar groups to improve the results in gas chromatographic analysis. We have evaluated N,O-bis(trimethylsilyl)- trifluoracetamide (BSTFA)+ 1\% trimethylchlorsilane (TMCS) and N-methyl-N-trimethylsilil-trifluoroacetamide (MSTFA) as derivatizing reagents for butorphanol at 30, 60 and 80 degrees C during 15, 30 and 60 min. The effects of dilution of these reagents with toluene and the evaporation before the derivatization were tested. Both reagents can be used for butorphanol derivatization and analysis and the dilution and evaporation steps did not alter the final results. The best derivatization conditions were 15 min at 80 degrees C, although 60 degrees C, although 60 degrees C during 60 min were also suitable. [\hyperlink{Butorphanol Tartrate}{PMID: 9470973}, M H Andraus et al., ]

\hypertarget{pmid_22918011}{B}utorphanol tartrate, a mixed synthetic agonistantagonist opioid analgesic has been used for management of postoperative pain in minor and major surgical procedures.(14,20) Tramadol hydrochloride is a centrally acting opioid which is effectively used in postoperative pain in various minor and major surgeries. Twenty subjects selected randomly received butorphanol tartrate 1 mg intramuscular and 20 subjects received tramadol hydrochloride 50 mg intramuscular after the removal of mandibular third molars. Time of injection, amount of anesthetic injected, duration of surgery, adverse effects were recorded.(21) The mean amount of LA administered in butorphanol group was 2.6450 ml and in tramadol group was 2.640 ml respectively, the mean duration for surgery was 56.75 and 53.5 minutes for butorphanol and tramadol groups respectively which was statistically not significant. Pain assessment was done with VAS which showed mean of 19.2 and 15.5 mm (p = 0.001) which was significant for butorphanol and tramadol respectively after 12 hours. The mean time for rescue medication requirement was 5.9 hours (for tramadol) and 8.4 hours (for butorphanol). Effective analgesic activity was seen by butorphanol 1 mg intramuscular then tramadol 50 mg. Butorphanol 1 mg was more effective than tramadol 50 mg in respect to postoperative analgesia. [\hyperlink{Butorphanol Tartrate}{PMID: 22918011}, Syed Sirajul Hassan et al., 2012]

\section*{Cardiolite}
\subsection*{Result}
\subsubsection*{Answer}
Unknown
\subsubsection*{{Explanation}}
\hypertarget{Cardiolite}
A review of all available abstracts reveals that none of them are about Cardiolite (technetium-99m sestamibi), nor do any specifically address its safety in children. The abstracts provided discuss a variety of cardiac medications, imaging agents, and interventions in pediatric populations, but Cardiolite is not mentioned in any of them. Therefore, based on the abstracts available, there is not enough data to determine whether Cardiolite is safe or unsafe for use in children. The safety of Cardiolite in children is unknown according to the evidence provided.

\subsection*{Abstracts}
\hypertarget{pmid_11456146}{C}arvedilol has proven to be beneficial in a majority of adult patients with congestive heart failure. Although the experience from adult patients may be extrapolated to older children, symptomatic infants remain a subset for whom dosage, safety and efficacy need to be established. The purpose of this study was to assess whether treatment with carvedilol is efficacious and safe for infants with dilated cardiomyopathy who do not show satisfactory clinical improvement despite treatment with conventional medications. Eight infants with dilated cardiomyopathy (ejection fraction <30\%) who were symptomatic despite tailored treatment with decongestive medications, were enrolled in the study. Echocardiographic findings and heart failure symptom scores were analyzed before and after starting carvedilol. Patients were hospitalized and monitored for side-effects during up-titration of carvedilol. At a follow-up of 4.5+/-2.2 months, patients receiving carvedilol showed a significant improvement in the left ventricular ejection fraction (38.5+/-11\% v. 24.4+/-5\%), and heart failure symptom score (p<0.05). No adverse events related to carvedilol administration occurred. There were no deaths. Carvedilol is well tolerated in infants with dilated cardiomyopathy and there is significant improvement in their functional status. Optimal timing of starting therapy, dosage and long-term effects need to be investigated with multi-institutional trials. [\hyperlink{Cardiolite}{PMID: 11456146}, N Gachara et al., ]

\hypertarget{pmid_19378887}{C}arvedilol reduces mortality and hospitalization in adults with congestive heart failure. Limited information is available about its use in children. The objective of this study was to determine the dosing, efficacy and side effects of carvedilol for the management of dilated cardiomyopathy in children. Sixteen children with idiopathic dilated cardiomyopathy, aged 7 months to 138 months and with an ejection fraction less than 40\%, were treated with carvedilol. The average initial dose was 0.1 mg/kg/day and it was uptitrated to 0.4 mg/kg/day. After six months on carvedilol, there were improvements in clinical scoring system from an average of 2.94 to 2.50 (p<0.05), in mean fractional shortening from 17.2 +/- 6.1\% to 22.7 +/- 5.1\% (p<0.05), and in ejection fraction from 35.2 +/- 6.8\% to 43.1 +/- 11.2\% (p<0.05). No side effect was observed during the study period. Two patients died due to serious infection. Carvedilol in addition to standard therapy for dilated cardiomyopathy in children improves cardiac function and symptoms. It is well tolerated, with minimal adverse effects, but close monitoring is necessary. [\hyperlink{Cardiolite}{PMID: 19378887}, Hülya Askari et al., ]

\hypertarget{pmid_16582683}{T}his review addresses recent concerns about the cardiovascular safety of nonsteroidal anti-inflammatory drugs, the disease-modifying role of these drugs in ankylosing spondylitis, and their use in the understudied pediatric population. Several recent observational and controlled studies highlight the cardiovascular toxicity of rofecoxib, celecoxib, parecoxib, valdecoxib and naproxen. Concerns about cardiovascular safety raise questions about the chronic use of nonsteroidal anti-inflammatory drugs in patients with rheumatic diseases, including children. The risks of these drugs in the pediatric population are not well known and this review addresses the limited data available concerning nonsteroidal anti-inflammatory drug use in children. A recent trial in ankylosing spondylitis patients demonstrated continuous nonsteroidal anti-inflammatory drug use reduced the rate of syndesmophyte formation, suggesting that they may have a disease-modifying role in these patients. Nonsteroidal anti-inflammatory drugs have been in the spotlight this year. While preliminary evidence has supported novel roles for these drugs in ankylosing spondylitis and in cancer prevention, accumulating evidence shows that some cyclooxygenase-2 and perhaps all nonsteroidal anti-inflammatory drugs are associated with cardiovascular toxicity. Further research is needed to understand the magnitude and mechanism of this risk. Clinicians are compelled to weigh carefully the benefits and risks of therapy. Concerns about safety are balanced by optimism about their potential role in delaying the progression of ankylosing spondylitis. [\hyperlink{Cardiolite}{PMID: 16582683}, Stacy P Ardoin et al., 2006]

\hypertarget{pmid_37070139}{M}yocarditis in children is more common in clinical practice, which can cause different degrees of cardiac function damage. We investigated the effects of creatine phosphate in the treatment of myocarditis in children. Children in the control group were treated with sodium fructose diphosphate, and children in the observation group were treated with creatine phosphate on the basis of the control group. After treatment, the myocardial enzyme profile and cardiac function of children in the observation group were better than the control group. The total effective rate of treatment for children in the observation group was higher than that in the control group. In conclusion, creatine phosphate could significantly improve myocardial function, improve myocardial enzyme profile and reduce myocardial damage in children with pediatric myocarditis and had a high safety of use, which was worthy of clinical promotion. [\hyperlink{Cardiolite}{PMID: 37070139}, Shaoli Lin et al., 2023]

\hypertarget{pmid_2672245}{A}nimal experiments have shown that quinolones can cause irreversible damage to cartilage in strained joints of young animals. Since a similar effect in humans cannot be excluded, the new quinolones are contraindicated in children and pregnant women. Nalidixic acid was the first quinolone developed and was often used in pediatric patients for the treatment of urinary tract infections. This compound is also known to cause damage to juvenile cartilage in animal studies but is still licensed for pediatric use. In three different retrospective studies of children treated with nalidixic acid, no case of quinolone-associated arthropathy could be detected by careful X-ray examinations or comparisons with control patients. This disparity most probably can be explained by interspecies differences: human cartilage may be insensitive or only slightly sensitive to the arthropathic effects of quinolones. Careful prospective evaluations of the efficacy and safety of new quinolones in childhood infections are both medically indicated and ethically justified. [\hyperlink{Cardiolite}{PMID: 2672245}, D Adam et al., ]

\hypertarget{pmid_24441450}{K}etorolac has been used safely as an analgesic agent for children following cardiac surgery in selected populations. Controversy exists among institutions about the risks involved with this medication in this patient group. This article reviews the current literature regarding the safety of ketorolac for postoperative pain management in children after cardiac surgery. Specifically, concerns about renal dysfunction and increased bleeding risk are addressed. Additionally, the article details pharmacokinetics and potential benefits of ketorolac, such as its opioid-sparing effect. The literature reflects that the use of this medication is not well studied in certain pediatric cardiac patients such as neonates and those with single-ventricle physiology, and the safety of this medication in regards to these special populations is reviewed. In conclusion, ketorolac can be used in specific pediatric patients after cardiac surgery with minimal risk of bleeding or renal dysfunction with appropriate dosing and duration of use.  [\hyperlink{Cardiolite}{PMID: 24441450}, Meredith K Jalkut et al., ] Fluoroquinolone-induced joint/cartilage toxicity has been observed in juvenile animal studies and is species- and dose-specific with canines exhibiting the highest rate of arthralgias. These early observations led to the contraindication of fluoroquinolones in the pediatric population. Despite these recommendations fluoroquinolones continue to be prescribed for select children with difficult-to-treat infections for whom the benefit of quinolone therapy may outweigh the risk of cartilage toxicity. A review of retrospective and prospective safety data of ciprofloxacin-treated children showed that the rates of arthralgia and quinolone-induced cartilage toxicity were low. Episodes of arthralgia were mostly reversible based on published surveillance data in children. Recent data from Bayer's ciprofloxacin clinical trials database found that the incidence of arthralgia in children did not differ between the ciprofloxacin and nonquinolone antimicrobial control groups. The role of fluoroquinolones in the treatment of certain serious infections in children does not appear to be compromised by safety concerns when used appropriately. [\hyperlink{Cardiolite}{PMID: 24441450}, Richard Grady et al., 2003]

\hypertarget{pmid_27682546}{A} cardioprotective drug could help prevent long-term heart damage in children who receive chemotherapy, researchers say.  [\hyperlink{Cardiolite}{PMID: 27682546}, Drug limits heart damage in childhood chemotherapy., 2016] To determine whether oral midazolam is a safe and effective alternative to our current standard premedication for children with cyanotic congenital heart disease (CCHD), 30 children aged 1-6 yr, scheduled for elective cardiac surgery, were studied. The children were randomly assigned to one of two groups: Group I received oral midazolam 0.75 mg.kg-1 30 min before separation from their parents in the surgical waiting area, and Group II received oral or rectal pentobarbitone 2 mg.kg-1 at 90 min, and morphine 0.2 mg.kg-1 and atropine 0.02 mg.kg-1 im at 60 min before separation. Heart rate, haemoglobin oxygen saturation (SpO2) and anxiolysis and sedation scores were recorded at four times during the study: at baseline (immediately before premedication), immediately after administration of the premedication, at separation of children from parents in the waiting area and at the time of application of the face mask in the operating room. We found that in Group I, anxiolysis improved at separation from parents compared with baseline (P < 0.05) and sedation increased both at separation and on mask application (P < 0.05), whereas in Group II anxiolysis did not change at any time and sedation increased only at separation (P < 0.05). Intramuscular injection of morphine produced a transient decrease in mean SpO2 (from 84\% to 76\%) (P < 0.05) that did not occur after ingestion of oral midazolam. The results of this study indicate that oral midazolam is a safe and effective replacement for the standard premedication for children with CCHD undergoing cardiac surgery and avoids the decrease in SpO2 associated with im injections. [\hyperlink{Cardiolite}{PMID: 27682546}, M F Levine et al., 1993]

\hypertarget{pmid_9002122}{Q}uinolone-induced cartilage toxicity has been observed in experimental juvenile animal studies and is species- and dose-specific. Accordingly these findings have led to the contraindication of fluoroquinolones in children. Previous data in 634 adolescents and children treated with compassionate use ciprofloxacin demonstrated low rates of reversible arthralgia and a safety profile similar to that for adult patients. This report describes the safety findings in 1795 children who received 2030 treatment courses of intravenous or oral ciprofloxacin. The average doses of intravenous and oral ciprofloxacin in the study population were 8 and 25 mg/kg/day, respectively. Treatment-associated events were reported in 10.9\% of children receiving oral ciprofloxacin compared with 18.9\% among intravenous recipients. Overall arthralgia occurred during 31 ciprofloxacin treatment courses (1.5\%) and the majority of events were of mild to moderate severity and resolved without intervention. More than 60\% of arthralgia episodes were in children with cystic fibrosis. The adverse event pattern in children receiving ciprofloxacin in this analysis was similar to that observed in adults. Rates of reversible arthralgia were low and unchanged from previously published surveillance data in children. [\hyperlink{Cardiolite}{PMID: 9002122}, B Hampel et al., 1997]

\hypertarget{pmid_36174614}{S}urvivors of childhood cancer are at risk of anthracycline-induced cardiotoxicity, which might be prevented by dexrazoxane. However, concerns exist about the safety of dexrazoxane, and little guidance is available on its use in children. To facilitate global consensus, a working group within the International Late Effects of Childhood Cancer Guideline Harmonization Group reviewed the existing literature and used evidence-based methodology to develop a guideline for dexrazoxane administration in children with cancer who are expected to receive anthracyclines. Recommendations were made in consideration of evidence supporting the balance of potential benefits and harms, and clinical judgement by the expert panel. Given the dose-dependent risk of anthracycline-induced cardiotoxicity, we concluded that the benefits of dexrazoxane probably outweigh the risk of subsequent neoplasms when the cumulative doxorubicin or equivalent dose is at least 250 mg/m [\hyperlink{Cardiolite}{PMID: 36174614}, Esmée C de Baat et al., 2022] For many years quinolone-induced cartilage toxicity observed in experiments with some skeletally immature animals represented indisputable contraindication for the use of these promising antimicrobials in prepubertal patients. Our clinical, magnetic resonance imaging and histopathological monitoring of ciprofloxacin use, together with the published experiences of other groups, suggest that the quinolone antibiotics do not cause arthropathy in humans. Conditions that potentially qualify for quinolone use (especially ciprofloxacin) in children include oral antipseudomonal (or antistaphylococcal) therapy for pulmonary exacerbation in cystic fibrosis, and for complicated urinary tract, skeletal, aural and shunt infections. In addition to these rare indications, there is an urgent need in developing countries for availability of the new quinolones for treating children with endemic and epidemic shigellosis and invasive salmonellosis. At present these compounds are not approved for pediatric patients; therefore each such treatment must be part of a controlled study or respect the compassionate use regularities. [\hyperlink{Cardiolite}{PMID: 36174614}, U B Schaad et al., ]

\hypertarget{pmid_15261177}{C}arvedilol reduces mortality and hospitalization in adults with congestive heart failure. Limited information is available about its use in children. We reviewed the medical records of 24 children with dilated cardiomyopathy and left ventricular ejection fraction of <or=40\%, who were treated with carvedilol as adjunct therapy to angiotensin-converting enzyme inhibitors, digoxin and diuretics. Carvedilol was initiated 14.3 +/- 23.3 (mean +/- SD) months after the diagnosis of cardiomyopathy. Mean age at initiation of therapy was 7.2 +/- 6.4 years. The mean initial and maximum doses were 0.15 +/- 0.09 and 0.98 +/- 0.26 mg/kg/day. Adverse effects occurred in 5 patients (21\%). Two patients (8\%) required discontinuation of the drug within 5 weeks of the initial dose. The remaining 22 patients tolerated carvedilol for a mean follow-up period of 26.6 +/- 14.7 months. Among these 22 patients, mean left ventricular ejection fraction improved from 24.6 +/- 7.6\% to 42.2 +/- 14.2\% (p < 0.001), and mean sphericity index from 0.86 +/- 0.11 to 0.74 +/- 0.10 (p < 0.001). New York Heart Association functional class improved in 15 patients (68\%). One patient (4\%) died and 3 (14\%) were transplanted. Carvedilol, in addition to standard therapy for dilated cardiomyopathy in children improves cardiac function and symptoms; it is well tolerated, with minimal adverse effects, but close monitoring is necessary as it might worsen congestive heart failure and precipitate asthma. Control studies are necessary to assess the effect of carvedilol on mortality and hospitalization rates. [\hyperlink{Cardiolite}{PMID: 15261177}, Paolo Rusconi et al., 2004]

\hypertarget{pmid_16102652}{T}he use of fluoroquinolones in children is limited because of the potential of these agents to induce arthropathy in juvenile animals and to potentiate development of bacterial resistance. No quinolone-induced cartilage toxicity as described in animal experiments has been documented unequivocally in patients, but the risk fro rapid emergence of bacterial resistance associated with widespread, uncontrolled fluoroquinolones use in children is a realistic threat. Overall, the fluoroquinolones have been safe and effective in the treatment of selected bacterial infections in pediatric patients. There are clearly defined indications for these compounds in children who are ill. [\hyperlink{Cardiolite}{PMID: 16102652}, Urs B Schaad et al., 2005]

\hypertarget{pmid_16719889}{E}pidural analgesia in children is highly effective and safe; however, it has not enjoyed great popularity in surgery that requires cardiopulmonary bypass. A major concern is the possibility of damage to blood vessels with the epidural needle or catheter and epidural hematoma formation. There seems to be a low incidence of epidural hematoma if certain guidelines are followed, so that in children, epidural analgesia can be used in selected patients, with safety, when surgical repair requires cardiopulmonary bypass. Epidural morphine has been used for clinical pain relief in pediatric cardiac surgery. Improved pulmonary function, suppressed hormonal and metabolic stress responses, easy early tracheal extubation, and good analgesia and sedation that allows neurological examination to alert any possibles hidden complications, are the advantages. A dedicated medical team is essential in the perioperative management to achieve maximum benefit for these patients. [\hyperlink{Cardiolite}{PMID: 16719889}, Ramon Vilà et al., 2006]

\hypertarget{pmid_12904131}{I}n the last few years, there has been increasing pressure to use fluoroquinolones in paediatric patients, since these antibiotics offer the advantage of an oral treatment regimen on an out-patient basis. However, even although this class of antibiotics generally remains well-tolerated, the restriction of fluoroquinolone use in children on a compassionate basis, which derives from their potential to cause cartilage toxicity, limits the safety data in this population and suggests a cautious use. This review reports the data of the literature on the safety of fluoroquinolones in different districts, focusing on the side effects in children and drug interactions. Moreover, data available in the literature with regards to side effects in children are reported, with particular attention to their potential in arthropathy. [\hyperlink{Cardiolite}{PMID: 12904131}, Laura Cuzzolin et al., 2002]

\hypertarget{pmid_16028153}{B}ecause of concerns about arthrotoxicity, fluoroquinolones are restricted for use in children. This study describes the safety and efficacy of gatifloxacin when used for treatment of children with recurrent acute otitis media (ROM) or acute otitis media (AOM) treatment failure (AOMTF). We performed an analysis of 867 children included in 4 clinical trials who had ROM and/or AOMTF and were treated with gatifloxacin (10 mg/kg once daily for 10 days). Gatifloxacin had adverse event rates that were similar overall to those of a comparator antibiotic (amoxicillin-clavulanate), except for increased diarrhea in children <2 years old receiving amoxicillin-clavulanate. There was no evidence of arthrotoxicity, hepatotoxicity, alteration of glucose homeostasis, or central nervous system toxicity acutely or during 1 year follow-up in any child. Regarding efficacy, in 2 noncomparative trials, the gatifloxacin cure rate of AOM was 89\% (95\% confidence interval [CI], 83\%-95\%) at the test of cure (TOC) visit, 3-10 days after completion of therapy. In 2 comparative trials of gatifloxacin versus amoxicillin-clavulanate, the efficacy of gatifloxacin was 88\% (95\% CI, 82\%-94\%). Gatifloxacin led to better clinical outcomes than amoxicillin-clavulanate for AOMTF (91\% vs. 81\%; P=.029), for AOMTF and age <2 years old (89\% vs. 69\%; P=.009), and for severe AOM in children <2 years old (90\% vs. 75\%; P=.012). Among children with AOMTF previously treated with amoxicillin-clavulanate or ceftriaxone injections, gatifloxacin cure rates were high (88\% and 75\%, respectively). Gatifloxacin appears to be safe for children, with no evidence of producing arthrotoxicity in 867 children exposed to the antibiotic when used as treatment for ROM and AOMTF. [\hyperlink{Cardiolite}{PMID: 16028153}, Michael E Pichichero et al., 2005] 2-8\% of all children aged between 6 months and 5 years have febrile seizures. Often these seizures cease spontaneously, however depending on different national guidelines, 20-40\% of the patients would need therapeutic intervention. For seizures longer than 3-5 minutes application of rectal diazepam, buccal midazolam or sublingual lorazepam is recommended. Benzodiazepines may be ineffective in some patients or cause prolonged sedation and fatigue. Preclinical investigations in a rat model provided evidence that febrile seizures may be triggered by respiratory alkalosis, which was subsequently confirmed by a retrospective clinical observation. Further, individual therapeutic interventions demonstrated that a pCO2-elevation via re-breathing or inhalation of 5\% CO2 instantly stopped the febrile seizures. Here, we present the protocol for an interventional clinical trial to test the hypothesis that the application of 5\% CO2 is effective and safe to suppress febrile seizures in children. The CARDIF (CARbon DIoxide against Febrile seizures) trial is a monocentric, prospective, double-blind, placebo-controlled, randomized study. A total of 288 patients with a life history of at least one febrile seizure will be randomized to receive either carbogen (5\% CO2 plus 95\% O2) or placebo (100\% O2). As recurrences of febrile seizures mainly occur at home, the study medication will be administered by the parents through a low-pressure can fitted with a respiratory mask. The primary outcome measure is the efficacy of carbogen to interrupt febrile seizures. As secondary outcome parameters we assess safety, practicability to use the can, quality of life, contentedness, anxiousness and mobility of the parents. The CARDIF trial has the potential to develop a new therapy for the suppression of febrile seizures by redressing the normal physiological state. This would offer an alternative to the currently suggested treatment with benzodiazepines. This study is an example of academic translational research from the study of animal physiology to a new therapy. ClinicalTrials.gov identifier: NCT01370044. [\hyperlink{Cardiolite}{PMID: 16028153}, Stephanie Ohlraun et al., 2013]

\hypertarget{pmid_18571537}{T}o evaluate the efficiency and safety of clopidogrel treatment in children with heart disease. We conducted single center retrospective chart review of children with heart disease at the University Hospital, Leuven, Belgium, in whom clopidogrel was used. The indication, dosage, duration of therapy, and adverse events were examined. Clinical efficacy was defined by an absence of thrombotic events. 46 children were identified. The mean age of first clopidogrel dose was 4.9 +/- 4.1 years. The study dosage ranged from 0.1 to 0.7 mg/kg/day clopidogrel. Almost all patients received concomitant aspirin therapy. No thrombotic events developed. Skin bruising developed in almost every patient, suggesting that clopidogrel has an anti-platelet effect. 2 patients who were treated with concomitant warfarin had bleeding complications (severe epistaxis and gastrointestinal bleeding). Hematological abnormalities were documented in 1 patient who received clopidogrel for 1 year; they reversed with medication cessation. Clopidogrel therapy in a pediatric population appears to be relatively safe and effective; however, randomized, controlled prospective studies are needed to determine the true efficacy and safety of clopidogrel in children. [\hyperlink{Cardiolite}{PMID: 18571537}, Luc Mertens et al., 2008]

\hypertarget{pmid_11295713}{T}he objective was to determine the dosing, efficacy, and side effects of the nonselective beta-blocker carvedilol for the management of heart failure in children. Carvedilol use in addition to standard medical therapy for pediatric heart failure was reviewed at 6 centers. Children with dilated cardiomyopathy (80\%) and congenital heart disease (20\%), age 3 months to 19 years (n = 46), were treated with carvedilol. The average initial dose was 0.08 mg/kg, uptitrated over a mean of 11.3 weeks to an average maintenance dose of 0.46 mg/kg. After 3 months on carvedilol, there were improvements in modified New York Heart Association class in 67\% of patients (P =.0005, chi2 analysis) and improvement in mean shortening fraction from 16.2\% to 19.0\% (P =.005, paired t test). Side effects, mainly dizziness, hypotension, and headache, occurred in 54\% of patients but were well tolerated. Adverse outcomes (death, cardiac transplantation, and ventricular-assist device placement) occurred in 30\% of patients. Carvedilol as an adjunct to standard therapy for pediatric heart failure improves symptoms and left ventricular function. Side effects are common but well tolerated. Further prospective study is required to determine the effect of carvedilol on survival and to clearly define its role in pediatric heart failure therapy. [\hyperlink{Cardiolite}{PMID: 11295713}, L A Bruns et al., 2001]

\hypertarget{pmid_17182438}{P}rotecting the myocardium from the risk of acute ischemia during heart surgery is still an unsolved problem; the problem is even more open and more pressing in pediatric heart surgery. To meet this greater risk it is advisable to use a cardioplegic solution with a composition that is better suited to the particular morphofunctional conditions of the myocardium in the child, i.e., a solution offering greater protection. To this purpose the authors experimented with Celsior cardioplegic solution during heart surgery in children to evaluate the efficacy compared to the standard St. Thomas solution. In this comparative study 15 children were treated with Celsior cardioplegic solution and 15 others with St. Thomas cardioplegic solution. Each patient underwent 2 biopsies of the myocardium, the first before cardioplegic treatment and the second immediately after reperfusion. In both groups, focal lesions involving both the cardiomyocytes and the vascular-stromal structures were randomly found. The former had undergone a necrotic-regressive process with changes in the myofibrils and the mitochondria. The vascular-stromal structures showed changes in the permeability of the capillary endothelia, with interstitial edema. The results show the lesions to be similar in the 2 groups both on a quality and quantitative level. [\hyperlink{Cardiolite}{PMID: 17182438}, L Cuccurullo et al., ]

\hypertarget{pmid_9475697}{T}he calcium antagonist amlodipine may have the potential for expanded use in children owing to its physiochemistry and pharmacokinetic profile that facilitates once-daily dosing in a liquid formulation. Its safety and efficacy have not been previously evaluated in children. A retrospective analysis of 15 pediatric bone marrow transplant patients who had amlodipine incorporated into their antihypertensive drug regimen reveals significantly lower blood pressure as compared with baseline therapy (123.5+/-2.1 mmHg and 117.2+/-2.2 mmHg, systolic blood pressure before and during amlodipine, P<0.05; 81.5+/-1.8 mmHg and 75.5+/-2.6 mmHg, diastolic blood pressure before and during amlodipine, P<0.05). Amlodipine provided improved blood pressure control in this cohort and may provide a valuable pharmacologic alternative for treatment of pediatric hypertension. [\hyperlink{Cardiolite}{PMID: 9475697}, S Khattak et al., 1998]

\hypertarget{pmid_35979562}{T}he prospective Control of HEART rate in inFant and child tachyarrhythmia with reduced cardiac function Using Landiolol (HEARTFUL) study investigated the effectiveness and safety of landiolol, a short-acting β The HEARTFUL study has demonstrated the efficacy of landiolol, by reducing heart rate or terminating tachycardia, in pediatric patients with supraventricular tachyarrhythmias. Although serious ARs and concerns were not identified in this study, physicians should be always cautious of circulatory collapse due to hypotension. [\hyperlink{Cardiolite}{PMID: 35979562}, Koichi Sagawa et al., 2022]

\hypertarget{pmid_3190970}{C}hloral hydrate 25, 50 or 75 mg kg-1 or midazolam 0.4, 0.5 or 0.6 mg kg-1, all given by mouth in combination with atropine 0.03 mg kg-1, were compared as premedication in 248 children in a randomized, double-blind study. Chloral hydrate was significantly less palatable than midazolam. The anxiolytic effect of chloral hydrate 75 mg kg-1 was "good" in children younger than 5 yr, whereas the other doses of chloral hydrate, and all doses of midazolam, provided only "fair" anxiolysis in this age group. All doses of both premedicants provided good anxiolysis in the older children. A satisfactory antisialogogue effect was seen in 83-90\% of each group. About 20 min after extubation, restlessness was observed in 15-25\% of the younger children premedicated with chloral hydrate 25 mg kg-1 or with midazolam 0.4 or 0.6 mg kg-1. The mean total recovery score (0-10) based on activity, ventilation, heart rate, conscious level and colour ranged between 5.8 and 6.8 at 10 min and between 9 and 9.5 at 70 min after extubation in all groups. Midazolam 0.5 mg kg-1 is recommended for children less than 5 yr of age and midazolam 0.4-0.5 mg kg-1 for older ones. Chloral hydrate 75 mg kg-1 provided good anxiolysis in both age groups; however, it was less palatable than the midazolam. [\hyperlink{Cardiolite}{PMID: 3190970}, L Saarnivaara et al., 1988]

\hypertarget{pmid_33547021}{C}ardiac computed tomography (CT) is increasingly used in pediatric patients with congenital heart disease (CHD). Variability of practice and of comprehensive diagnostic risk across institutions is not known. Four centers prospectively enrolled consecutive pediatric CHD patients <18 years of age undergoing cardiac CT from January 6, 2017 to 1/30/2020. Patient characteristics, cardiac CT data and comprehensive diagnostic risk were compared by age and institutions. Risk categories included sedation and anesthesia use, vascular access, contrast exposure, cardiovascular medication, adverse events (AEs), and estimated radiation dose. Cardiac CT was performed in 1045 pediatric patients at a median (interquartile range, IQR) age of 1.7 years (0.3, 11.0). The most common indications were arterial abnormalities, suspected coronary artery anomalies, functionally single ventricle heart disease, and tetralogy of Fallot/pulmonary atresia. Sedation was used in 8\% and anesthesia in 11\% of patients. Peripheral vascular access was utilized for 93\%. Median contrast volume was 2 ​ml/kg. Beta blockers were administered in 11\% of cases and nitroglycerin in 2\% of cases. The median (IQR) total procedural dose length product (DLP) was 20 ​mGy∗cm (10, 50). Sedation, vascular access, contrast exposure, use of cardiovascular medications and radiation dose estimates varied significantly by institution and age (p ​< ​0.001). Seven minor adverse events (0.7\%) and no major adverse events were reported. Cardiac CT for CHD is safe in pediatric patients when appropriate CT technology and expertise are available. Scans can be acquired at relatively low radiation exposure with few minor adverse events. [\hyperlink{Cardiolite}{PMID: 33547021}, Mariana De Oliveira Nunes et al., ]

\section*{Ceftriaxone}
\subsection*{Result}
\subsubsection*{Answer}

Neonates (0–1 month): Unknown
Infants and Children (>1 month to 18 years): Yes

\subsubsection*{{Explanation}}
\hypertarget{Ceftriaxone}
Based on the abstracts provided, there is substantial evidence from targeted studies evaluating the safety of ceftriaxone in children across various age ranges. Below is a summary of the evidence by age group:

Neonates (0–1 month):
- One study specifically examined the pharmacokinetics and safety of ceftriaxone in 39 neonates treated for suspected sepsis. The study found no drug-associated changes in biochemical or hematological parameters and concluded that ceftriaxone is a safe and well-tolerated antibiotic for use in newborns [\hyperlink{pmid_4076254}{PMID: 4076254}, A Mulhall et al., 1985].
- However, another systematic review highlighted concerns about potential adverse effects in neonates, such as hyperbilirubinemia and cardiopulmonary events, especially when ceftriaxone is administered with calcium-containing solutions. The review concluded that the available literature had small sample sizes and poor external validity, making it impossible to estimate the magnitude of risk, and called for further controlled studies [\hyperlink{pmid_27718120}{PMID: 27718120}, Patrick C Donnelly et al., 2017].

Infants and Children (>1 month to 18 years):
- Multiple studies, including randomized and prospective trials, have evaluated the safety of ceftriaxone in infants and children for various infections (e.g., meningitis, sepsis, respiratory tract infections, typhoid fever, sickle cell disease, and others). These studies consistently report high efficacy and a low incidence of serious adverse effects. Most adverse events were mild and transient, such as diarrhea, rash, eosinophilia, thrombocytosis, and neutropenia, which resolved during or after therapy [\hyperlink{pmid_3912733}{PMID: 3912733}, F De Francesco et al.; \hyperlink{pmid_6314805}{PMID: 6314805}, R W Steele et al., 1983; \hyperlink{pmid_6330022}{PMID: 6330022}, T Chonmaitree et al., 1984; \hyperlink{pmid_6318653}{PMID: 6318653}, S C Aronoff et al., 1983; \hyperlink{pmid_6098721}{PMID: 6098721}, M Minamitani et al., 1984; \hyperlink{pmid_6098695}{PMID: 6098695}, I Nagamatsu et al., 1984; \hyperlink{pmid_3658622}{PMID: 3658622}, D Floret et al., 1987; \hyperlink{pmid_3985602}{PMID: 3985602}, B L Congeni et al., 1985; \hyperlink{pmid_3725639}{PMID: 3725639}, R Yogev et al.; \hyperlink{pmid_3969363}{PMID: 3969363}, S J Nelson et al.; \hyperlink{pmid_1810586}{PMID: 1810586}, S S Bakshi et al., 1991; \hyperlink{pmid_1823514}{PMID: 1823514}, S L Yen et al.].
- Several studies specifically mention that no serious side effects were observed, and ceftriaxone was considered safe and effective for use in children, including those with serious infections and in special populations such as children with leukemia or sickle cell disease [\hyperlink{pmid_3464362}{PMID: 3464362}, M R Rossi et al., 1986; \hyperlink{pmid_1810586}{PMID: 1810586}, S S Bakshi et al., 1991].
- Some studies report rare but potentially serious hypersensitivity reactions, including anaphylaxis and severe rash, indicating that while generally safe, clinicians should be vigilant for allergic reactions [\hyperlink{pmid_24592804}{PMID: 24592804}, D Shrestha et al., 2013; \hyperlink{pmid_24130394}{PMID: 24130394}, Russelian Arulraj et al.].
- Studies also report that ceftriaxone can cause transient, asymptomatic biliary sludge or nephrolithiasis, particularly with higher doses, longer duration, or in children over 12 months. These findings were not associated with clinical symptoms and did not require discontinuation of therapy in asymptomatic patients [\hyperlink{pmid_18246742}{PMID: 18246742}, Ahmet Soysal et al.; \hyperlink{pmid_24910741}{PMID: 24910741}, Azita Fesharakinia et al., 2013].

Summary:
- For neonates, the safety of ceftriaxone is uncertain due to conflicting evidence: one targeted study supports its safety, but a systematic review highlights unresolved safety concerns and insufficient data.
- For infants and children older than 1 month, multiple targeted studies affirm the safety of ceftriaxone, with rare serious adverse events and generally mild, self-limited side effects.

\subsection*{Abstracts}
\hypertarget{pmid_28827252}{C}eftriaxone is widely used in children in the treatment of sepsis. However, concerns have been raised about the safety of ceftriaxone, especially in young children. The aim of this review is to systematically evaluate the safety of ceftriaxone in children of all age groups. MEDLINE, PubMed, Cochrane Central Register of Controlled Trials, EMBASE, CINAHL, International Pharmaceutical Abstracts and adverse drug reaction (ADR) monitoring systems will be systematically searched for randomised controlled trials (RCTs), cohort studies, case-control studies, cross-sectional studies, case series and case reports evaluating the safety of ceftriaxone in children. The Cochrane risk of bias tool, Newcastle-Ottawa and quality assessment tools developed by the National Institutes of Health will be used for quality assessment. Meta-analysis of the incidence of ADRs from RCTs and prospective studies will be done. Subgroup analyses will be performed for age and dosage regimen. Formal ethical approval is not required as no primary data are collected. This systematic review will be disseminated through a peer-reviewed publication and at conference meetings. CRD42017055428. [\hyperlink{Ceftriaxone}{PMID: 28827252}, Linan Zeng et al., 2017]

\hypertarget{pmid_3912733}{A}A. have tested a new drug (Ceftriaxone) on 40 children affected by upper and lower respiratory tract infectious diseases. As shown by results, this new drug has been remarkably effective and easy to use since it may be administered once in a day; moreover, the tested drug has not caused any kind of tissue or parenchymal involvement. [\hyperlink{Ceftriaxone}{PMID: 3912733}, F De Francesco et al., ]

\hypertarget{pmid_6314805}{C}eftriaxone is an investigational cephalosporin with a half-life of five to eight hours. In an uncontrolled study, we evaluated its efficacy and safety in 30 pediatric and 12 young adult patients with serious bacterial infections. This agent was administered to children at a dosage of 50 to 75 mg/kg/day intravenously in two divided doses. Those with CNS infections received 100 mg/kg/day. In adults, the dosage was 1 g either once or twice daily. The diseases we treated included pneumonia (17), sepsis (eight), ventriculoperitoneal shunt infections (three), osteomyelitis (three), brain abscess (two), peritonitis (two), and miscellaneous (seven). Clinical cures were achieved in all cases, although one child with cystic fibrosis and Pseudomonas pneumonia had persistent colonization in his sputum. No serious side effects were observed. Although not the agent of choice for many of these pathogens, ceftriaxone appears to represent an important alternative to therapy. [\hyperlink{Ceftriaxone}{PMID: 6314805}, R W Steele et al., 1983]

\hypertarget{pmid_6330022}{T}he clinical efficacy and safety of ceftriaxone, a long half-life cephalosporin were evaluated in 48 children with a variety of serious bacterial infections. Clinical cure was achieved in 92\% (44 of 48) of patients. Peak serum bactericidal titres for Haemophilus influenzae type b, Streptococcus pneumoniae, Str. pyogenes and Escherichia coli were greater than or equal to 1:1024. Mean peak and trough ceftriaxone levels were 173 and 42 mg/l, respectively. Mild and transient diarrhoea was observed in 10\% of patients. Laboratory side effects encountered were eosinophilia, thrombocytosis and neutropenia in another 8\%. Ceftriaxone is a useful antibiotic for common childhood infections. Its prolonged half-life allows twice daily administration which reduces problems related to intravenous therapy as well as the cost and personnel time. [\hyperlink{Ceftriaxone}{PMID: 6330022}, T Chonmaitree et al., 1984]

\hypertarget{pmid_24592804}{C}eftriaxone is a widely used antibiotic in pediatric clinical practice. Usually ceftriaxone is well tolerated and serious adverse effect like anaphylaxis is rare. We report a near fatal anaphylaxis reaction in a child after the first dose of intravenous ceftriaxone who revived successfully. [\hyperlink{Ceftriaxone}{PMID: 24592804}, D Shrestha et al., 2013]

\hypertarget{pmid_3969363}{C}eftriaxone is a new parenteral cephalosporin with a prolonged half-life and an expanded Gram-negative spectrum. Before it can be used as a single agent for infections of unknown etiology, its efficacy in treating infections caused by Gram-positive organisms, particularly Staphylococcus aureus, must be proven. Ceftriaxone was administered to 12 children for treatment of infections due to S. aureus alone or in the presence of other organisms. Sites of infection included soft tissue, respiratory tract, bone and joint. Patients received ceftriaxone at 68 to 100 mg/kg/day in two doses for 3 to 20 days. Clinical and bacteriologic responses were satisfactory in all patients. One patient experienced abdominal pain during infusion and another developed a skin rash. Five patients had platelet counts of 500,000/mm3 or greater; four had an eosinophil count of 7\% or greater and one patient had transient neutropenia. These abnormalities resolved during or after therapy. Ceftriaxone was a safe and effective single antibiotic for the treatment of infections caused by S. aureus in children. [\hyperlink{Ceftriaxone}{PMID: 3969363}, S J Nelson et al., ]

\hypertarget{pmid_1918222}{C}eftriaxone is generally recognized as safe and effective when used as a single drug in the therapy of septicemia and other serious infections involving bacteremia in both adults and children. An advantage of ceftriaxone over other third-generation cephalosporins is its long serum half-life, which allows the drug to be given every 12 hours in children or less frequently in adults. [\hyperlink{Ceftriaxone}{PMID: 1918222}, M T Foster et al., 1991]

\hypertarget{pmid_6318653}{T}hirty-four patients aged 1 month to 19 years were treated with ceftriaxone for suspected bacterial infections. Bacterial pathogens were isolated from 25 children. The overall bacterial cure rate was 88\%, with an overall clinical response rate of 96\%. No side effects requiring cessation of therapy were observed. Ceftriaxone proved to be safe and effective in the treatment of serious infections in children. [\hyperlink{Ceftriaxone}{PMID: 6318653}, S C Aronoff et al., 1983]

\hypertarget{pmid_6098721}{C}eftriaxone (Ro 13-9904, CTRX), developed by F. Hoffmann-La Roche Ltd. in Switzerland, was used for the pediatric infections and the following results were obtained. The mean blood level of CTRX in 2 children after a 60-minute intravenous drip infusion with 20 mg/kg was 58.6 micrograms/ml at 30 minutes, 75.0 micrograms/ml at 1 hour, 39.85 micrograms/ml at 2 hours, 27.74 micrograms/ml at 4 hours, 20.71 micrograms/ml at 6 hours, 11.72 micrograms/ml at 12 hours and 3.91 micrograms/ml at 24 hours while the half-life time was 5.9 hours in one child and 7.6 hours in the other. CTRX was used in 22 children with acute infections consisting of 3 with acute pharyngeal tonsillitis, 4 with acute bronchitis, 8 with bronchopneumonia, 6 with infections of skin soft tissue and 1 with salmonellosis. The results were excellent in 5 cases and good in 17, indicating an efficacy rate of 100\%. Out of 10 cases where the causative strains were detected, 4 cases were followed about the activities of the respective bacteria, i.e., H. influenzae, Streptococcus group A, S. aureus and Salmonella group B, all of which were eradicated after the end of administration. The daily dose of CTRX ranged from 30 to 50 mg/kg and generally a larger dose was used for serious infections. CTRX was administered twice daily in 20 out of 22 cases, by an intravenous injection in 4 and an intravenous drip infusion in 18, for 2 to 4 days in 16 and 5 to 8 1/2 days in 6. No clinical adverse reactions were observed while the laboratory test found a slight elevation of GOT in one and that of GOT and GPT in another. From the above results, CTRX was judged to be a highly useful drug for treatment of pediatric infections. [\hyperlink{Ceftriaxone}{PMID: 6098721}, M Minamitani et al., 1984]

\hypertarget{pmid_3464362}{C}eftriaxone (CFX) is a new third-generation cephalosporin with interesting characteristics as regards both its antibacterial spectrum and kinetics which make it potentially useful in the empiric treatment of infections in neutropenic cancer patients. However, since its kinetic characteristics in children with leukemia are not known and its pharmacokinetics are reported to be altered in such patients, we studied ceftriaxone's activity in ten leukemic children with fever and neutropenia. Our findings seem to be confirm the potential efficacy of the drug also in this particular type of patient. [\hyperlink{Ceftriaxone}{PMID: 3464362}, M R Rossi et al., 1986]

\hypertarget{pmid_27718120}{C}eftriaxone is a third-generation cephalosporin with broad-spectrum activity against both Gram-positive and Gram-negative bacteria. Despite its effectiveness, its use for the treatment of infections in neonatal patients has been limited because of concern about its potential toxicity. Our aim was to review the literature for an association between ceftriaxone and cardiopulmonary events, hyperbilirubinemia, and pseudolithiasis among neonates. We searched PubMed and EMBASE and included studies that evaluated ceftriaxone safety in neonates. Study bias was evaluated in the following domains: exposure measurement, outcome measurement, attrition, generalizability, confounding, statistical analysis, and reporting. We included nine studies regarding ceftriaxone side effects, primarily spontaneous reports, published case reports, and small case series. Reports of bilirubin displacement attributed to ceftriaxone included increases in serum bilirubin necessitating antibiotic change in a subset of infants after administration of ceftriaxone. One study described self-resolving biliary sludge after ceftriaxone administration in six of 80 infants. Cardiopulmonary adverse events included a report of eight cardiopulmonary events related to concomitant ceftriaxone-calcium infusion, including seven infant deaths. Additional cardiopulmonary events reported included perinatal asphyxia, pulmonary hypertension, and thrombocytosis. However, the available literature had small sample sizes, poor external validity, and inconsistent outcome ascertainment methods, which made it impossible to estimate the magnitude of risk. Concomitant administration of intravenous ceftriaxone and calcium-containing solutions should be avoided in neonates. However, further controlled studies are needed to assess whether bilirubin displacement associated with the use of ceftriaxone is clinically relevant, particularly in healthy term and near-term neonates with mild hyperbilirubinemia. [\hyperlink{Ceftriaxone}{PMID: 27718120}, Patrick C Donnelly et al., 2017]

\hypertarget{pmid_3985602}{C}eftriaxone administered as a single daily dose of 50 mg/kg was evaluated in the treatment of 35 children with a variety of nonmeningitic bacterial infections. In two of the patients, the drug was discontinued before the response to the drug could be evaluated. All of the remaining patients had a satisfactory response. In 22 of the patients, plasma was available for the determination of ceftriaxone levels 1 h after a dose and immediately before the next dose. All but one of these patients had trough ceftriaxone levels which exceeded the MIC of the infecting organism, although marginally so for Staphylococcus aureus. Ceftriaxone appears to be safe and effective in the treatment of a variety of bacterial pathogens in children when administered at a single daily dose of 50 mg/kg. This drug may be especially useful in those patients in whom outpatient antibiotic therapy is contemplated or in whom maintenance of intravenous access is difficult. [\hyperlink{Ceftriaxone}{PMID: 3985602}, B L Congeni et al., 1985]

\hypertarget{pmid_18246742}{C}eftriaxone, a third-generation cephalosporin, is widely used for treating infection during childhood. It is mainly eliminated in the urine, but approximately 40\% of a given dose is unmetabolized and secreted into bile. The aim of this study was to investigate the frequency, clinical characteristics, and outcome of biliary sludge (BS) in addition to potential contributing risk factors in children who receive ceftriaxone. Biliary ultrasonography was performed at the time of randomization before ceftriaxone treatment was started, on the 5th and 10th days, and at the end of the treatment. If BS was detected, patients were followed-up weekly by sonographic examination until the BS or biliary lithiasis (BL) disappeared. A total of 114 children (56 girls, 58 boys; age range: 2-180 months, mean 47.5 +/- 46.3 mos) were enrolled in the study. Fourteen (12\%) subjects developed BS and 10 (9\%) developed BL on the 5th day of treatment. On the 10th day of treatment, 20 (18\%) subjects developed BS and 15 (13\%) developed BL. In total, 35 (31\%) of all subjects developed biliary precipitation (BP), of whom 20 (57\%) were diagnosed as BS and 15 (43\%) as BL. All subjects who developed BP were found to be asymptomatic during the course of therapy. Patient age over 12 months, daily total dose of ceftriaxone of more than 2 g, and duration of treatment longer than five days were found to be associated with BP. Ceftriaxone frequently causes transient BPs and its probability increases if the child is over 12 months of age, the dose is over 2 g/day, or the duration is over five days. Neither radiologic investigation nor the discontinuation of treatment with ceftriaxone is necessary as long as the patient is asymptomatic. [\hyperlink{Ceftriaxone}{PMID: 18246742}, Ahmet Soysal et al., ]

\hypertarget{pmid_3906584}{C}eftriaxone is an aminothiazolyl-oxyimino cephalosporin. It possesses the typical in vitro activity of a third-generation cephalosporin with excellent activity against many gram-negative aerobic bacilli: Escherichia coli; species of Proteus, Klebsiella, Morganella, Providencia and Citrobacter; and Enterobacter agglomerans. Ceftriaxone also has outstanding bactericidal action against pneumococci, group B streptococci, meningococci, gonococci and Hemophilus influenzae. In healthy volunteers, it has an exceptionally long serum half-life of 5.8-8.7 (mean 6.5) hours. It distributes well throughout all body spaces, including cerebrospinal fluid in the presence of inflammation. Dosage modification is necessary only when there is combined hepatic and renal dysfunction. Adverse reactions characteristic of cephalosporins have been observed with the administration of ceftriaxone. No unique toxicities have been identified, and hypoprothrombinemic bleeding is not part of the adverse reaction profile. Ceftriaxone has been used to treat serious bacterial infections in neonates, infants, children and adults. Bacteriologic and clinical success rates have consistently exceeded 90\%. The drug has also been used as single-dose chemoprophylaxis in coronary artery bypass, biliary tract, vaginal hysterectomy and prostatic surgery. Efficacy and safety were similar to multiple-dose cefazolin. Ceftriaxone warrants special consideration because its extended half-life allows for less frequent dosing than other antimicrobials. Significant cost savings can be realized with proper use of this antibiotic. [\hyperlink{Ceftriaxone}{PMID: 3906584}, T R Beam et al., ]

\hypertarget{pmid_1823514}{T}en children, diagnosed as having typhoid fever, were enrolled in this study between April and September, 1989. Ceftriaxone was administered intravenously, in two dosages adding to 50-100 mg/kg/day over as short a period as five days. The mean period of defervescence was 3.2 days. No adverse reactions to the drug occurred; all those fulfilling the prescribed course were cured. To date, no relapse has been reported nor has any patient become a chronic carrier. Shortterm use of Ceftriaxone had the advantages of rapid response, abscence of serious side effects, and low failure rate in treating children with typhoid fever. [\hyperlink{Ceftriaxone}{PMID: 1823514}, S L Yen et al., ]

\hypertarget{pmid_4076254}{T}he pharmacokinetics and safety of ceftriaxone were examined in 39 neonates who required antibiotics for clinically suspected sepsis. The drug was administered as a once daily dose of 50 mg/kg by the intravenous (IV) or intramuscular (IM) route. Ceftriaxone was assayed in 49 series of blood samples, 3 samples of cerebrospinal fluid (CSF) and 15 samples of urine by a microbiological technique. Blood was collected before, during and after treatment for biochemical analysis. Routine haematological investigations were also monitored. There was no significant difference between the maximum plasma concentrations following IV (153 +/- 39 mg/l) or IM (141 +/- 19 mg/l) administration (first dose). The mean elimination half-life, total body clearance, and volume of distribution following the first dose were 15.4 +/- 5.4 h, 0.28 +/- 0.12 ml/min per kg and 325 +/- 59 ml/kg respectively. Clearance increased with increasing postnatal age and body temperature (P less than 0.0002) and decreasing plasma creatinine concentration (P less than 0.005). Increasing plasma protein concentration was associated with a decrease in volume of distribution (P less than 0.001). There were no drug-associated changes in any of the biochemical or haematological parameters examined. Ceftriaxone is a safe and well tolerated antibiotic for use in the treatment of newborn sepsis and possibly meningitis. A once daily administration of 50 mg/kg by the IV and IM routes provides satisfactory plasma concentrations throughout the dosage interval whilst avoiding accumulation. [\hyperlink{Ceftriaxone}{PMID: 4076254}, A Mulhall et al., 1985]

\hypertarget{pmid_6098699}{C}eftriaxone CTRX was evaluated about its antibacterial activity against clinical isolates at our department and tried clinically in 10 children of 6 months to 10 years and 6 months of age. The antibacterial activity was equal to cefotaxime or higher while the clinical results were almost satisfactory. Three out of 4 strains were eradicated (75\%). As to the adverse reaction, eosinophilia was observed only in 1 case. [\hyperlink{Ceftriaxone}{PMID: 6098699}, H Hoshina et al., 1984]

\hypertarget{pmid_1810586}{C}hildren with sickle cell disease have a greatly increased potential for developing rapid and at times fatal sepsis from Streptococcus pneumoniae. Hospitalization and parenteral antibiotic treatment in all febrile children with sickle cell disease have thus become the standard of care at most sickle cell centers. As an alternative approach, we managed selected febrile children with sickle cell disease on an ambulatory basis with parenteral ceftriaxone to determine its safety and effectiveness in preventing sepsis and reducing the number of days of hospitalization. Twenty of 40 children who presented with significant fever met the study criteria and received ceftriaxone on an ambulatory basis. Three were subsequently hospitalized. Compared with a previous year, when all febrile children were admitted, ceftriaxone use reduced the days of hospitalization from 214 (6.3 +/- 1.6 days/patient) to 111 days (2.8 +/- 0.7 days/patient). The empiric use of ceftriaxone appears safe and effective, but it requires an expanded study over an extended period. [\hyperlink{Ceftriaxone}{PMID: 1810586}, S S Bakshi et al., 1991]

\hypertarget{pmid_3725639}{C}eftriaxone has a very long serum half-life and enhanced in vitro activity against common pediatric pathogens. Therefore we evaluated the efficacy and safety of once daily ceftriaxone therapy in 57 children with serious infections including: meningitis (26 patients); ventriculitis (3); pyelonephritis (7); osteomyelitis (6); abscess (4); septic arthritis (3); sepsis (2); and miscellaneous infections (6). The most common isolates were Haemophilus influenzae (23), Escherichia coli (9) and Staphylococcus aureus (8). Ceftriaxone was given intravenously or intramuscularly in a dose of 50 mg/kg for non-central nervous system (CNS) infections. Patients with CNS infections received an initial dose of 100 mg/kg followed by 80 mg/kg 12 hours later and once daily thereafter. In a limited number of patients no major differences in serum ceftriaxone concentrations were found after intravenous or intramuscular injection. Of 57 patients with pathogens isolated 55 were completely cured; in one patient with Klebsiella pneumoniae ventriculitis, intraventricular gentamicin was briefly added to the regimen. Another patient with an anaerobic liver abscess recovered after metronidazole was administered. In three patients a delayed response to ceftriaxone was noted. One patient with previous recurrent infections had a second episode of H. influenzae meningitis 22 days after cessation of therapy. Clinical side effects were noted in 10 of 71 patients (including 14 treated patients who had negative cultures). Seven patients had diarrhea, one each had fever or rash and one had fever, rash and arthralgia. Laboratory side effects in 16 of 71 patients included eosinophilia (7), thrombocytosis (7), elevated liver enzymes (4) and leukopenia and neutropenia (2).(ABSTRACT TRUNCATED AT 250 WORDS) [\hyperlink{Ceftriaxone}{PMID: 3725639}, R Yogev et al., ] The pharmacokinetics of ceftriaxone (Ro 13-9904, CTRX) was studied in 14 children receiving a dose of 10, 20 mg/kg or 1 g as a intravenous bolus. The mean half-lives of CTRX were 4.5, 6.3 +/- 0.5 and 5.2 +/- 0.7 hours, respectively, while the urinary recovery rates up to 12 hours were 51.7, 48.6 and 48.9\%. Forty-one patients, aged 2 months to 10 years, were treated with an intravenous dosage of 10 to 58 mg/kg CTRX every 12 hours for 2 to 29 days. The diseases consisted of upper respiratory tract infections (4), bronchitis (7), pneumonia (18), pyothorax (2), urinary tract infections (4), pertussis (4), meningitis (1) and endocarditis (1). Clinical cures were achieved in 38 cases, overall clinical response rate being 92.7\%. No serious side effects were observed, although mild diarrhea was seen in 2 cases. [\hyperlink{Ceftriaxone}{PMID: 3725639}, I Nagamatsu et al., 1984]

\hypertarget{pmid_24130394}{C}eftriaxone is a commonly used antibiotic in children for various infections like respiratory tract infection, urinary tract infection and enteric fever. Hypersensitive reactions following ceftriaxone therapy are uncommon but are potentially life-threatening. The rash can resemble viral exanthems and may lead to a delay in the recognition and prompt treatment. Here we report a 7-year-old boy who presented with fever and rash with emphasis on recognizing ceftriaxone hypersensitivity and its management.  [\hyperlink{Ceftriaxone}{PMID: 24130394}, Russelian Arulraj et al., ] Ceftriaxone is a third-generation cephalosporin used to treat infants with community-acquired pneumonia. Currently, there is a large variability in the amount of ceftriaxone used for this purpose in this particular age group, and an evidence-based optimal dose is still unavailable. Therefore, we investigated the population pharmacokinetics of ceftriaxone in infants and performed a developmental pharmacokinetic-pharmacodynamic analysis to determine the optimal dose of ceftriaxone for the treatment of infants with community-acquired pneumonia. A prospective, open-label pharmacokinetic study of ceftriaxone was conducted in infants (between 1 month and 2 years of age), adopting an opportunistic sampling strategy to collect blood samples and applying high-performance liquid chromatography to quantify ceftriaxone concentrations. Developmental population pharmacokinetic-pharmacodynamic analysis was conducted using nonlinear mixed effects modeling (NONMEM) software. Sixty-six infants were included, and 169 samples were available for pharmacokinetic analysis. A one-compartment model with first-order elimination matched the data best. Covariate analysis elucidated that age and weight significantly affected ceftriaxone pharmacokinetics. According to the results of a Monte Carlo simulation, with a pharmacokinetic-pharmacodynamic target of a free drug concentration above the MIC during 70\% of the dosing interval (70\%  [\hyperlink{Ceftriaxone}{PMID: 24130394}, Ya-Kun Wang et al., 2020] Ceftriaxone is a third-generation cephalosporin which is widely used for treatment of infection in children accompanied by complications like urinary tract lithiasis and gallbladder psudolithiasis or sludge. The aim of this study was to investigate the incidence and predisposing factors that contribute to these complications in children. This quasi-experimental and before- and after-study was conducted in 96 children who were hospitalized for treatment of different bacterial infections and received 50-100 mg/kg/day ceftriaxone divided into two equal doses intravenously under conditions of adequate hydration. Sonographic examinations of urinary tract and gallbladder were carried out before and after treatment for that purpose. Patients with positive sonographic findings after treatment were followed with serial sonographic examinations. Post-treatment sonography demonstrated nephrolithiasis in 6 (6.3\%) and gallbladder stone in one (1\%), all were asymptomatic. Comparison of the groups with and without nephrolithiasis demonstrated no significant differences with respect to age, body weight, diagnosis, season of hospitalization, dosage of drug and the duration of treatment. Nephrolithiasis had a significant relation with male gender (P=0.02). Our results showed that pediatric patients may develop small sized, asymptomatic renal stones during a 2-6 day course of normal or moderate dose of ceftriaxone therapy. Close monitoring of ceftriaxone treated patients especially on high dose long term therapy for nephrolithiasis and gallbladder psudolithiasis or sludge is recommended. [\hyperlink{Ceftriaxone}{PMID: 24130394}, Azita Fesharakinia et al., 2013]

\hypertarget{pmid_6323376}{C}eftriaxone is a new cephalosporin with a broad spectrum of antibacterial activity and unique serum and CSF pharmacokinetics. The drug was compared in a randomized fashion with ampicillin and chloramphenicol in the treatment of 19 children with Haemophilus influenzae type b meningitis. Ceftriaxone was also administered non-randomly to six other patients including three children with Gram-negative meningitis. Among the children with H. influenzae meningitis, no deaths were noted and the outcomes of the study and the control groups were similar. Ninety per cent of the isolates of H. influenzae were inhibited by 0.0625, 1 and 1 mg/l of ceftriaxone, ampicillin and chloramphenicol respectively. One child with pneumococcal meningitis and two children with meningococcal meningitis recovered rapidly and without incident during ceftriaxone therapy. Three children with Gram-negative meningitis caused by multiply-drug resistant organisms were bacteriologically cured within five days of the onset of therapy. Persistent pleocytosis and neurological disabilities were noted in two at the conclusion of therapy. Ceftriaxone, as a single agent, was comparable in efficacy with traditional antimicrobial therapy usually employed in childhood meningitis. [\hyperlink{Ceftriaxone}{PMID: 6323376}, S C Aronoff et al., 1984]

\hypertarget{pmid_3658622}{T}hirty-one infants and children aged 1 month to 15 years 3 months were treated with ceftriaxone once a day for the treatment of a meningitis related to Neisseria meningitidis (19 cases), haemophilus influenzae (7 cases), streptococcus pneumoniae (1 case), not identified bacteria (4 cases). All identified bacteria were sensitive to ceftriaxone. Twenty children were treated with 100 mg/kg/day, 11 with 50 mg/kg/day. CSF was sterile at the first control-generally performed 30 h after the onset of treatment-in all cases. Despite a great number of severe forms (fulminans purpura and septic shock; 11 cases; severe neurologic disturbances: 6 cases), all patients survived and recovered after a treatment of 9 to 22 days. Two infants exhibited neurologic sequelae: deafness, delayed development and hydrocephalus. Tolerance to ceftriaxone appeared to be good. With a 100 mg/kg/day dosage, mean CSF level at 6 h was 3.3 mg/l (0.8-7.7), on the first day of treatment. At the end of treatment, mean CSF level at 24h was 0.47 (0.15-2.5). With a 50 mg/kg/day dosage, mean CSF level at 6 h was 2,1 mg/l (1.1-3.9) in the first day of treatment. At the end of the treatment, mean CSF level at 24h was 0.22 mg/l (0.08-0.5). Once a day administration of ceftriaxone is adequate for the treatment of meningitis in infants and children. Though a 50 mg/kg/day dosage is probably sufficient in most cases, it seems to be more secure to use a 100 mg/kg/day dosage. [\hyperlink{Ceftriaxone}{PMID: 3658622}, D Floret et al., 1987]

\section*{Cefuroxime Sodium}
\subsection*{Result}
\subsubsection*{Answer}

Neonates (birth to 1 month): Yes
Infants and Children (1 month to 12 years): Yes
Adolescents (12–17 years): Yes

\subsubsection*{{Explanation}}
\hypertarget{Cefuroxime Sodium}
A review of the available abstracts reveals multiple targeted studies evaluating the safety of Cefuroxime Sodium (and its oral prodrug, cefuroxime axetil) in children across various age ranges. Below is a summary of the relevant evidence, organized by age group:

Neonates (birth to 1 month):
- A study of 28 neonates with suspected or proven infection (all with complications at birth or early neonatal life) received intramuscular or intravenous cefuroxime (50 mg/kg twice daily for 5 days). The study found significant clinical improvement in 27/28 neonates, no adverse clinical side effects, and no laboratory changes attributable to cefuroxime. The authors concluded that cefuroxime is a safe, well-tolerated, and rapidly absorbed drug for the treatment of neonates with suspected or proven infections [\hyperlink{pmid_7065695}{PMID: 7065695}, J de Louvois et al., 1982].

Infants and Children (1 month to 12 years):
- An open, multicenter trial involving 304 children aged 3 months to 12 years with acute upper respiratory infections and/or acute otitis media treated with cefuroxime axetil suspension reported a 93\% cure rate and only minor adverse reactions in 4.9\% of patients (most commonly vomiting). The study concluded that cefuroxime axetil suspension was safe and effective in this population [\hyperlink{pmid_8234058}{PMID: 8234058}, J Barliński et al.].
- A study of 84 children aged 3 months to 5 years with community-acquired pneumonia received intravenous cefuroxime followed by oral cefuroxime axetil suspension. Of 84 evaluable patients, 82 (97.6\%) were cured or improved, and no significant adverse effects were reported. The study confirmed the safety and efficacy of cefuroxime in children under 5 years [\hyperlink{pmid_8088980}{PMID: 8088980}, I Shalit et al., 1994].
- A randomized trial in 40 children aged 3 months to 5 years with parapneumonic pleural effusion or empyema compared cefuroxime to other antibiotics. Adverse effects attributed to cefuroxime were mild and infrequent, and the study concluded cefuroxime is effective and well-tolerated in this age group [\hyperlink{pmid_11969360}{PMID: 11969360}, G C Palacios et al., 2002].
- A randomized, unblinded study in 43 children aged 6 months to 12 years with early Lyme disease compared cefuroxime axetil to amoxicillin. All patients had a good outcome, with no long-term problems and only mild diarrhea in a small number of participants. No hypersensitivity reactions occurred. The study concluded that cefuroxime axetil seems to be safe and efficacious for children with early Lyme disease [\hyperlink{pmid_12042561}{PMID: 12042561}, Stephen C Eppes et al., 2002].
- A study of 36 children aged 3 months to 12 years receiving cefuroxime axetil suspension for respiratory or soft-tissue infections found that 3 of 35 were withdrawn due to adverse events (one drug-related hypersensitivity), but the majority tolerated the drug well, with only mild reactions in 9 children. The study supports favorable tolerability and safety [\hyperlink{pmid_1763541}{PMID: 1763541}, D A Powell et al., 1991].
- A retrospective analysis of 886 children aged 4 months to 17 years hospitalized for acute ENT diseases found cefuroxime therapy to be safe and effective, with a beneficial therapeutic effect in 98.9\% of patients [\hyperlink{pmid_15732828}{PMID: 15732828}, Magdalena Lapienis et al., 2004].
- Additional studies in children (various ages, including infants and up to 13.5 years) with respiratory, urinary, and soft-tissue infections consistently report good efficacy and no significant side effects [\hyperlink{pmid_434912}{PMID: 434912}, J A Kuzemko et al., 1979; \hyperlink{pmid_41955}{PMID: 41955}, K Ohnuma et al., 1979; \hyperlink{pmid_390177}{PMID: 390177}, Y Nakamura et al., 1979; \hyperlink{pmid_3494006}{PMID: 3494006}, J W Carson et al., 1987; \hyperlink{pmid_3877456}{PMID: 3877456}, W J Barson et al., 1985].

Summary:
- For neonates (birth to 1 month), targeted studies affirm the safety of cefuroxime sodium.
- For infants and children (1 month to 12 years), multiple targeted studies affirm the safety of cefuroxime sodium and cefuroxime axetil.
- For adolescents (12–17 years), one large retrospective study includes children up to 17 years and reports safety, but most detailed data are for younger children.

No studies were found that specifically address safety in children older than 12 years as a distinct group, but the available evidence up to 17 years (from the large retrospective study) supports safety.


\subsection*{Abstracts}
\hypertarget{pmid_513298}{H}aving resistance to beta-lactamase-producing strains and showing resistance to not only cephalosporin resistant strains of E. coli and Klebsiella but also to Citrobacter, Proteus and Enterobacter, Cefuroxime (CXM) was used in pediatric field for both fundamental and clinical studies. CXM was found to be a useful antibiotic in views of high clinical efficacy rate obtained and no side effect noted. As for the dose, the single dose of 25 mg/kg achieved sufficient blood levels. Also in view of good clinical effect, the dose of 25 mg/kg three or four times daily seems appropriate for treatment of children. [\hyperlink{Cefuroxime Sodium}{PMID: 513298}, M Hotta et al., 1979]

\hypertarget{pmid_3531565}{P}harmacokinetic and clinical studies of cefixime (CFIX) in children were done and the following results were obtained. Serum and urinary concentrations of CFIX were determined in 6 children aged 5 to 14 years given single doses of 1.5 or 6.0 mg/kg. Mean serum concentrations peaked at 4 hours after the administration of either 1.5 or 6.0 mg/kg, and respective peak values were 0.71 and 4.46 micrograms/ml. Biological half-lives for the low and the high doses were 5.28 and 4.45 hours, respectively. The 12-hours urinary recovery ranged from 7.0 to 13.8\% after administration of 1.5 mg/kg, and the 8-hours urinary recovery was 18.1\% after administration of 6.0 mg/kg. Therapeutic responses were recorded as excellent or good in 43 (97.7\%) of the children, comprising 13 with tonsillitis and 31 with scarlet fever. The microbiological effectiveness of CFIX on identified pathogens comprising 29 strains of S. pyogenes and 2 strains of S. aureus was satisfactory as evidence by a high eradication rate of 93.5\%. No clinical side effects were observed. Abnormal laboratory findings were elevation of GOT and/or GPT in 4 patients and eosinophilia in 1 patient. In conclusion, CFIX was found to be efficacious and safe for the treatment of bacterial infections in children. [\hyperlink{Cefuroxime Sodium}{PMID: 3531565}, T Nishimura et al., 1986]

\hypertarget{pmid_7933536}{W}e administrated cefodizime (40 mg/kg) to 13 patients with simple herniorrhaphy in the pediatric field and determined its concentrations in tissues and serums. The mean serum and tissue levels of cefodizime after administration were 43.1 +/- 13.3 micrograms/ml, and 23.1 +/- 6.4 micrograms/g, respectively, at 3 hours. Cefodizime concentrations of the tissue and serum were maintained at relatively high levels for many hours. The ratio of cefodizime concentrations in tissue to serum became high at 3 hours after administration, and this suggests that tissue concentrations decreased more slowly than serum levels, and cefodizime concentrations in tissue were maintained at fairly high levels over a long period. No side effects caused by cefodizime were observed. From pharmacokinetic and clinical observations, cefodizime appears to be a safe and effective injectable antibiotic for the treatment of infections in children. [\hyperlink{Cefuroxime Sodium}{PMID: 7933536}, K Matsuura et al., 1994]

\hypertarget{pmid_8234058}{T}he study aimed at assessing the clinical efficiency, safety, and tolerance of cefuroxime axetil suspension in the treatment of children with the acute upper respiratory infections and/or the acute otitis media. The trial was open, multicenter, involving 304 children aged between 3 months and 12 years. They were recruited from 18 general practice centers in Poland. Children were given cefuroxime axetil suspension in the dose of 10 mg/kg body weight (upper respiratory) or 15 mg/kg otitis media. max. 250 mg) bid. Children were examined prior to the treatment, 3-4 days following the start of therapy, 1-2 days after completion of the treatment, and followed-up for 14 days. Post-therapy examination has shown 93\% cure rate. During the follow-up period 0.77\% of patients relapsed. Only minor adverse reactions were reported by 4.9\% of patients. Most common complaint was vomiting. Cefuroxime axetil suspension was safe and effective therapy in the acute upper respiratory infections and the acute otitis media in childhood. [\hyperlink{Cefuroxime Sodium}{PMID: 8234058}, J Barliński et al., ]

\hypertarget{pmid_3866088}{A} clinical trial of ceftizoxime suppositories (CZX-S) was performed to evaluate the therapeutic effectiveness in children with bacterial infection. The subjects were 10 children comprising 4 with pneumonia, 3 with lacunar tonsillitis, 2 with pharyngitis, and 1 with UTI. They were given 1 suppository containing either 125 mg or 250 mg of CZX 2 to 4 times a day. The daily per kg body weight dose ranged from 17.1 to 60.0 mg. The result was "markedly effective" in 3, "effective" in 6, and "failure" was recorded in 1. Bacteriologically, successful eradication of causative organisms was confirmed in all the 4 children who underwent the test. No clinical side effects were observed. The only laboratory test abnormality recorded in a single patient was eosinophilia, which was not definitely ascribable to CZX-S. In conclusion, CZX-S have proved to be a clinically safe and effective antibiotic preparation in infantile infection, even in children whose treatment with conventional antibiotics is associated with difficulties. [\hyperlink{Cefuroxime Sodium}{PMID: 3866088}, T Hosoda et al., 1985]

\hypertarget{pmid_7065695}{T}he new broad spectrum cephalosporin, cefuroxime, was used to treat 28 neonates with suspected or proved infection. All of them had had complications at birth or in early neonatal life which were known to predispose to infection. The treatment regimen consisted of intramuscular or intravenous cefuroxime (50 mg/kg twice a day) for 5 days. Previously, such infants would have received gentamicin with penicillin or ampicillin. Pathogenic or potentially pathogenic bacteria were isolated from  7 (25\%) of them. All of these organisms were sensitive to cefuroxime. None of the babies had meningitis, but blood cultures from 2 gave positive results. There was significant clinical improvement in 27 of them after 5 days of treatment and each was well on discharge from hospital. Serum urea, total protein, albumin, and alanine transaminase levels were estimated before, during, and after cefuroxime treatment. There were no changes attributable to cefuroxime nor were any changes in haemoglobin, packed cell volume, or total differential white cell counts observed. There were no adverse clinical side effects. One hundred and ninety-four samples of serum were assayed for cefuroxime. The mean peak level after intramuscular injection (42.7 mg/l) was reached in 0.8 hours, and the mean trough level was 10.5 mg/l. The mean half-life of cefuroxime in infants aged less than 4 days was 5.8 hours. In 4 infants older than 8 days, it ranged from 1.6-3.8 hours. Half-life was not associated with birthweight. Cefuroxime is a safe, well-tolerated, and rapidly absorbed drug for the treatment of neonates with suspected or proved infections; it is a useful alternative to gentamicin, if the use of an aminoglycoside is not clearly indicated. [\hyperlink{Cefuroxime Sodium}{PMID: 7065695}, J de Louvois et al., 1982]

\hypertarget{pmid_8088980}{F}or children with acute respiratory infections in hospital, it is desirable to transfer from parenteral to oral therapy at the earliest opportunity. The introduction of a pediatric suspension of cefuroxime axetil provides a continuous course of one antibiotic with transition from injectable to oral therapy. This open study was designed to investigate the efficacy of cefuroxime in pediatric patients aged 3 months to 5 years with community-acquired pneumonia. Children had evidence of lobar pneumonia on chest X-ray, a white blood cell count of > 15,000/mm3 and a rectal temperature of > or = 38.5 degrees C on enrollment. Cefuroxime was given by i.v. injection at 75 mg/kg per day in three divided doses for 48-72 h followed by oral cefuroxime suspension at 30 mg/kg per day in two divided doses. Of 84 evaluable patients 82 (97.6\%) were cured or improved post-treatment, and of 74 evaluable children who returned for follow-up assessment 73 (98.6\%) remained well. Oral therapy with twice daily cefuroxime axetil suspension following 2-3 days of i.v. cefuroxime administration was confirmed as effective and safe treatment for lobar pneumonia in children under 5 years of age. [\hyperlink{Cefuroxime Sodium}{PMID: 8088980}, I Shalit et al., 1994]

\hypertarget{pmid_2693753}{C}linical usefulness of cefixime (CFIX), a new oral cephalosporin antibiotic, in pediatric field was investigated. The results obtained were summarized as follows. 1. The clinical efficacy of CFIX was investigated in a total of 138 children including 49 with upper respiratory tract infections (RTI), 22 with acute bronchitis, 18 with pneumonia, 19 with scarlet fever and 21 with urinary tract infections (UTI). 2. Clinical effectiveness was excellent in 58, good in 60, fair in 14 and poor in 3, with an overall efficacy rate of 87.4\%. The efficacy rate classified according to types of infection were 85.7\% in upper RTI, 89.5\% in acute bronchitis, 94.4\% in pneumonia, 78.9\% in scarlet fever, and 90.5\% in UTI. 3. Out of the suspected causative organisms, 43 strains of a total of 50 strains isolated were eradicated. The bacteriological eradication rate was 86.0\%. (Haemophilus influenzae 100\%, Haemophilus parainfluenzae 100\%, Streptococcus pyogenes 88.5\%, Escherichia coli 85.7\%). 4. One hundred forty four children were analyzed for side effect. Side effects were observed in 2 children (1.4\%) with diarrhea in 1 and anorexia in another. Abnormal laboratory test results were recorded in 4 children (3.3\%). The above results suggest that CFIX is a very useful new oral cephalosporin for the treatment of bacterial infections in children. [\hyperlink{Cefuroxime Sodium}{PMID: 2693753}, H Mikawa et al., 1989]

\hypertarget{pmid_434912}{C}efuroxime (25 mg/kg) given intravenously every four hours to 7 children with bacterial meningitis resulted in satisfactory therapeutic blood and CSF levels. All children made a full recovery and side effects were absent. [\hyperlink{Cefuroxime Sodium}{PMID: 434912}, J A Kuzemko et al., 1979]

\hypertarget{pmid_3761541}{W}e used cefixime (CFIX), a newly developed oral cephalosporin antibiotic, to treat 21 children with various infections. The results are summarized as follows. The serum half-lives of CFIX after an administration of 6 mg/kg to each of 2 children were 2.56 and 2.79 hours. The serum concentrations were high enough to ensure the therapeutic response. The clinical response was "excellent" in 16 children and "good" in 5, with a 100\% efficacy rate. No side effects were recorded. The only abnormal finding was slight eosinophilia in 1 child. [\hyperlink{Cefuroxime Sodium}{PMID: 3761541}, S Furukawa et al., 1986]

\hypertarget{pmid_3877456}{A}lthough it is used extensively in Europe, there is a limited amount of published data concerning pediatric clinical experience with cefuroxime in the United States. Thirty-six children, ranging from 3.5 to 57 months of age, received intravenous cefuroxime (75 mg/kg/day in three divided doses) for soft-tissue infections of the face or epiglottis. Infections treated included preseptal (19 patients) and buccal (13 patients) cellulitis and epiglottitis (four patients). Blood cultures were positive in 22 patients, yielding Haemophilus influenzae type b in 17 (four were beta-lactamase-positive), Streptococcus pneumoniae in four; and beta-lactamase-positive, nontypable H influenzae in one. An additional five patients with buccal cellulitis had negative blood cultures but H influenzae type b antigenuria. A satisfactory clinical response was noted in all patients, and repeated blood cultures performed in initially bacteremic patients were sterile. Cefuroxime therapy was well tolerated, and abnormal laboratory results were infrequent, except for absolute granulocytopenia (granulocytes, less than 1,500/cu mm), which occurred in six patients but could not be ascribed to a drug effect because of the uncontrolled design of our study. Treatment with cefuroxime appears to be a safe and effective therapy for pediatric soft-tissue infections due to H influenzae and S pneumoniae. [\hyperlink{Cefuroxime Sodium}{PMID: 3877456}, W J Barson et al., 1985]

\hypertarget{pmid_15732828}{T}he authors present results of retrospective clinical analysis of usefulness the cefuroxime therapy of acute ENT diseases in children. The study group consist of 886 patients, aged 4 m. to 17 year, hospitalized at the Department of Paediatric Otolaryngology between 1997-2002. The efficacy of therapy was estimated on the ground of 4 degree scale. Particular attention was paid on measuring an average time of intravenous and oral administration of drug and on side effects of treatment. The results of the study shown that cefuroxime therapy is safe and effective. Beneficial therapeutic effect was obtained in 98.9\% of patients. [\hyperlink{Cefuroxime Sodium}{PMID: 15732828}, Magdalena Lapienis et al., 2004]

\hypertarget{pmid_28827252}{C}eftriaxone is widely used in children in the treatment of sepsis. However, concerns have been raised about the safety of ceftriaxone, especially in young children. The aim of this review is to systematically evaluate the safety of ceftriaxone in children of all age groups. MEDLINE, PubMed, Cochrane Central Register of Controlled Trials, EMBASE, CINAHL, International Pharmaceutical Abstracts and adverse drug reaction (ADR) monitoring systems will be systematically searched for randomised controlled trials (RCTs), cohort studies, case-control studies, cross-sectional studies, case series and case reports evaluating the safety of ceftriaxone in children. The Cochrane risk of bias tool, Newcastle-Ottawa and quality assessment tools developed by the National Institutes of Health will be used for quality assessment. Meta-analysis of the incidence of ADRs from RCTs and prospective studies will be done. Subgroup analyses will be performed for age and dosage regimen. Formal ethical approval is not required as no primary data are collected. This systematic review will be disseminated through a peer-reviewed publication and at conference meetings. CRD42017055428. [\hyperlink{Cefuroxime Sodium}{PMID: 28827252}, Linan Zeng et al., 2017]

\hypertarget{pmid_3494006}{C}efuroxime axetil tablets were given to 12 children (aged 19 months to 13.5 years) for a total of 14 episodes of lower respiratory tract infection. Doses ranged from 15 to 32 mg/kg/day. Six infections were regarded as cured and seven improved. In four cases, Haemophilus influenzae was present at the end of treatment. Serum levels of cefuroxime showed great variability. Absorption and penetration of the drug into the lower respiratory mucosa may not be sufficient to kill organisms which are sensitive in vitro. Cefuroxime axetil tablets were acceptable to most children. [\hyperlink{Cefuroxime Sodium}{PMID: 3494006}, J W Carson et al., 1987]

\hypertarget{pmid_41955}{C}efuroxime, a new synthetic cephalosporin, was administered to 10 pediatric patients (6 with respiratory tract infection, 2 with urinary tract infection, 1 with sepsis of E. coli and 1 with enterocolitis). The clinical result was good and excellent in all the 10 cases. No side effect was observed in any of them. [\hyperlink{Cefuroxime Sodium}{PMID: 41955}, K Ohnuma et al., 1979]

\hypertarget{pmid_34834338}{C}efixime (CEF) is a cephalosporin included in the WHO Model List of Essential Medicines for Children. Liquid formulations are considered the best choice for pediatric use, due to their great ease of administration and dose-adaptability. Owing to its very low aqueous solubility and poor stability, CEF is only available as a powder for oral suspensions, which can lead to reduced compliance by children, due to its unpleasant texture and taste, and possible non-homogeneous dosage. The aim of this work was to develop an oral pediatric CEF solution endowed with good palatability, exploiting the solubilizing and taste-masking properties of cyclodextrins (CDs), joined to the use of amino acids as an auxiliary third component. Solubility studies indicated sulfobutylether-β-cyclodextrin (SBEβCD) and Histidine (His) as the most effective CD and amino acid, respectively, even though no synergistic effect on drug solubility improvement by their combined use was found. Molecular Dynamic and  [\hyperlink{Cefuroxime Sodium}{PMID: 34834338}, Marzia Cirri et al., 2021] Cefixime (CFIX), a new oral cephalosporin, was administered clinically at a daily dose of 3.4 mg/kg to 10.4 mg/kg to each of 12 children, aged from 2 months to 14 years old. An additional separate study was done to compare the serum and urinary levels of CFIX in 3 children when each was administered with 100 mg of the drug in capsule with the serum and urinary levels of the drug in the same children when each was given the same amount of drug in the form of 5\% granules. The results of these trials are summarized below. Peak serum levels of CFIX administered in capsules and 5\% granules averaged 1.4 micrograms/ml and 1.9 micrograms/ml, respectively. The half-life of the former was 5.13 hours, while that of the latter was 4.17 hours. The difference in the peak levels was statistically insignificant. The urinary excretion of CFIX in either form of the drug (capsules and granules) was about 14-18\% in 12 hours. In 9 cases of respiratory infections, therapeutic results were excellent in 3 cases, good in 6 cases, and the effective rate was 100\%. In 2 cases of urinary tract infection, results were excellent in 1 case and good in 1 case. The drug efficacy was poor in 1 case of purulent cervical lymphadenitis, probably caused by Staphylococcus aureus. No adverse reactions attributable to the drug were observed. CFIX may be expected to be a highly effective and safe agent in moderate respiratory and urinary tract infections of children. [\hyperlink{Cefuroxime Sodium}{PMID: 34834338}, N Nakayama et al., 1986]

\hypertarget{pmid_1763541}{T}he tolerability, safety, and efficacy of cefuroxime axetil suspension was studied in 36 children (aged 3 mo to 12 y) who had been hospitalized for respiratory tract or soft-tissue infections. After receiving parenteral antibiotics for a mean of 3.7 days, children were discharged home to receive cefuroxime axetil suspension at doses of 10, 15, or 20 mg/kg every 8 or 12 hours for a mean of 8.2 days. One child was lost to follow-up. Three of 35 evaluated patients were withdrawn from therapy because of adverse events, one of which was a drug-related hypersensitivity reaction. Of the 32 children who completed therapy, 9 developed mild reactions including oral thrush, diarrhea, or diaper dermatitis; none were withdrawn from therapy. Complete clinical cure occurred in 28 children (80 percent); 4 (11.4 percent) were clinically improved but still required an additional antibiotic within one week of completing therapy with cefuroxime axetil suspension. This favorable tolerability and safety of cefuroxime axetil suspension warrants further efficacy trials in pediatric patients. [\hyperlink{Cefuroxime Sodium}{PMID: 1763541}, D A Powell et al., 1991]

\hypertarget{pmid_390177}{C}efuroxime, a new cephalosporin C antibiotic, was administered to 15 children with respiratory tract infection, urinary tract infection, or subcutaneous tumour. The following results were obtained. 1) CXM 30 approximately 100 mg/kg/day were used in treatment of respiratory tract infection. Eight of the eleven patients treated responded to the therapy. 2) CXM 45 approximately 75 mg/kg/day were given to 3 patients with urinary tract infection. Excellent results were obtained in all these cases. 3) One patient with subcutaneous tumour responded to CXM therapy. 4) Clinical isolates from the foci involved, i.e., Staphylococcus aureus (4 strains), Group A Streptococcus hemolyticus (1 strain), Streptococcus pneumoniae (1 strain), Haemophilus influenzae (1 strain), and Escherichia coli (3 strains) were all eliminated by CXM therapy except 2 unassessable strains. 5) No noteworthy side effect was noted. [\hyperlink{Cefuroxime Sodium}{PMID: 390177}, Y Nakamura et al., 1979]

\hypertarget{pmid_37698043}{T}he World Health Organization recommends that infants be exclusively breastfed for the first 6 months. Antibiotics are among the most commonly prescribed drugs for pregnant and lactating women. The vast majority of drugs pass into breast milk, which may create a risk for the infant. In cases where drug exposure may pose a risk, breastfeeding should be discontinued. Therefore, the mother's drug use should be decided by considering the most accurate and recent data. Cefuroxime is a second-generation cephalosporin antibiotic with a broad spectrum of activity against Gram-negative and -positive microorganisms. In this study, we aimed to develop the LC-MS/MS method using salt-assisted liquid-liquid micro-extraction (SALLME) for the determination of cefuroxime in breast milk. The method was validated according to the European Medicines Agency (EMA) guidelines. Cefuroxime and the internal standard cefixime were extracted from plasma by a SALLME technique. The results obtained from the entire validation study are at an acceptable level according to the EMA criteria. The calibration curve of cefuroxime was between 25 and 1000 ng/ml, with correlation coefficients of >0.99. The lower limit of quantitation was 25 ng/ml for cefuroxime. Furthermore, the developed method was applied for the determination of cefuroxime in real patient breast milk. [\hyperlink{Cefuroxime Sodium}{PMID: 37698043}, Aykut Kul et al., 2023]

\hypertarget{pmid_1571464}{C}efotaxime has been used to treat serious bacterial infections in children since 1982. With the predominant use of cephalosporins in pediatrics, reports of adverse effects of certain compounds have increased. A retrospective review is presented of 2,243 cases of children receiving therapy with cefotaxime in order to evaluate the safety profile and efficacy of cefotaxime in the treatment of serious infections in hospitalized children. Overall, 57 (2.5\%) children experienced adverse reactions. These included local reactions in 6 (0.3\%), rash in 28 (1.2\%), diarrhea in 15 (0.97\%), vomiting in 10 (0.7\%), abdominal pain in 1 (0.1\%), headache in 3 (0.4\%), and drug fever in 1 (0.1\%). No cases of hemolytic anemia, bleeding, or hyperbilirubinemia were found. Efficacy of treatment for different disease categories ranged from 90.5\% to 100\%. The percentage of children in any treatment group with a particular laboratory abnormality following initiation of cefotaxime therapy ranged from 0\% to 2.6\%, and rates of superinfection with bacteria or Candida were 0.4\% to 1.7\%. Cefotaxime has the distinct advantage of high rates of efficacy and low rates of complications and superinfection among children hospitalized for serious infections. [\hyperlink{Cefuroxime Sodium}{PMID: 1571464}, R F Jacobs et al., 1992]

\hypertarget{pmid_10496153}{A}n open-labeled and randomized trial was conducted to compare the efficacy and safety of once daily cefpodoxime proxetil suspension (10mg/kg/day) and thrice daily cefaclor (45mg/kg/day) in the treatment of acute otitis media in children. A total of 57 children aged from 6 months to 9 years were enrolled; 23 were treated with cefpodoxime and 34 with cefaclor. Satisfactory clinical outcome, either cure or improvement, was achieved at the end of treatment in 90\% of patients in the cefaclor group and 95\% of patients in the cefpodoxime group (p > 0.05). Clinical recurrence was identified at the follow-up visits in one case of the cefaclor group (3\%), and none in the cefpodoxime group (p > 0.05). These drugs were well tolerated by 14/21 (67\%) in the cefpodoxime-treated group and 27/32 (84\%) in the cefaclor-treated group. The incidence of adverse events was slightly higher in the cefpodoxime group than in the cefaclor group, however the difference did not reach statistical significance (p > 0.05). The daily cost of once-daily cefpodoxime was lower than that of thrice-daily cefaclor. We conclude that cefpodoxime administered once daily is as effective and safe as cefaclor administered thrice daily in the treatment of acute otitis media in children. The less dosing frequency and lower daily price of cefpodoxime provide additional benefits. [\hyperlink{Cefuroxime Sodium}{PMID: 10496153}, H Y Tsai et al., 1998]

\hypertarget{pmid_3761545}{C}efixime (CFIX) was evaluated for pharmacokinetics, therapeutic effectiveness on infection, safety, and bacteriological effectiveness in pediatrics. The following is a summary of the results. Pharmacokinetics in 4 children, 2 each receiving a single dose of 1.5 mg or 6.0 mg per kg body weight, were examined. Peak serum CFIX concentrations after the dose of 1.5 mg/kg were 1.12 and 1.34 micrograms/ml, and the serum half-lives were 1.83 and 3.53 hours. For the children administered with 6.0 mg/kg of CFIX, the respective figures were 2.50 and 7.46 micrograms/ml, and 6.77 and 6.64 hours. The 12-hour urinary recoveries were 4.9 and 34.1\% and 9.4 and 25.4\% for the small and the large doses, respectively. Therapeutic effectiveness in 19 children with infections was "excellent" in 14 and "good" in 5, with an effectiveness rate of 100\%. Bacteriological effectiveness was evaluated in 10 children. Classified by causative organisms, 5 cases had H. influenzae, 2 each H. parainfluenzae and S. pyogenes, and 1 mixed infection by H. influenzae and S. pneumoniae. Only the H. influenzae in the child with mixed infection resisted the therapy, and all the other pathogens were successfully eradicated. No side effects were recorded. The only abnormal laboratory test finding attributed to CFIX was eosinophilia in 2 children. [\hyperlink{Cefuroxime Sodium}{PMID: 3761545}, M Miyazaki et al., 1986]

\hypertarget{pmid_12042561}{C}efuroxime axetil has been shown to have efficacy comparable to doxycycline in adults with early Lyme disease (LD). Because of toxicity, doxycycline is usually avoided in children. For children who are unable to tolerate amoxicillin, there is currently no proven alternative oral therapy for LD. This randomized, unblinded study compared 2 dosage regimens of cefuroxime axetil (20 mg/kg/d and 30 mg/kg/d) with amoxicillin (50 mg/kg/d), each given for 20 days. Children were enrolled if they were 6 months to 12 years of age, had erythema migrans, and met other eligibility requirements. Serologic testing occurred at entry and after 6 months. Follow-up evaluations for safety, tolerability, and efficacy occurred at 10 and 20 days, 6 months, and 1 year. Forty-three children were randomized (13 in the amoxicillin group, 15 in each cefuroxime axetil group); 39 completed 12 months of follow-up. At the completion of treatment, there was total resolution of erythema migrans in 67\% of the amoxicillin group, 92\% of the low-dose cefuroxime group, and 87\% of the high-dose cefuroxime group, and resolution of constitutional symptoms occurred in 100\%, 69\%, and 87\%, respectively. All patients had a good outcome, with no long-term problems associated with LD. One patient, who was well at the first 2 follow-up visits, was treated with doxycycline because of new constitutional symptoms. Mild diarrhea occurred in a small number of participants in each group (1 patient was diagnosed and treated for Clostridium difficile-associated diarrhea, which occurred after completing the full course of study medication). No hypersensitivity reactions occurred. The number of patients in this trial was not sufficient to demonstrate a statistically significant difference between the 3 groups; however, both amoxicillin and cefuroxime axetil seem to be safe, efficacious treatments for children with early LD. [\hyperlink{Cefuroxime Sodium}{PMID: 12042561}, Stephen C Eppes et al., 2002]

\hypertarget{pmid_11969360}{T}he aim of this study was to evaluate the efficacy of cefuroxime, compared with the combination of dicloxacillin/chloramphenicol, for the treatment of children with parapneumonic pleural effusion or empyema. Forty patients, aged 3 months to 5 years, with pleural effusion or empyema were randomized to receive cefuroxime (100 mg/kg/day) IV (n=20) or chloramphenicol (100 mg/kg/day) plus dicloxacillin (200 mg/kg/day) IV (n=20). Both groups were similar in age, days of illness, clinical and radiological findings, and etiology. Most patients (70\%) had an empyema at presentation. There was no difference in clinical outcomes, including days to defervescence, duration of respiratory distress, duration of chest tube drainage, and days to discharge from hospital. The complication rates were similar in both groups. Pleural thickening occurred in four patients, bronchopleural fistula in two, and loculated empyema in one patient of each treatment group. Adverse effects attributed to cefuroxime were mild and infrequent. These results suggest that cefuroxime is an effective and well-tolerated alternative for the treatment of children with pleural effusion and empyema. [\hyperlink{Cefuroxime Sodium}{PMID: 11969360}, G C Palacios et al., 2002]

\section*{Chloroprocaine Hydrochloride}
\subsection*{Result}
\subsubsection*{Answer}

Yes (infants up to 6 months, based on targeted clinical studies)
Unknown (children older than 6 months, as no targeted safety studies are identified in the abstracts)

\subsubsection*{{Explanation}}
\hypertarget{Chloroprocaine Hydrochloride}
A review of the available abstracts reveals several that specifically address the use of Chloroprocaine Hydrochloride (2-chloroprocaine) in children, particularly infants. Here is a summary of the relevant evidence:

1. **Infants (up to 6 months):**
   - A retrospective cohort comparison evaluated continuous epidural infusions of 1.5\% 2-chloroprocaine versus 0.1\% ropivacaine in full-term infants 6 months or younger following thoracotomy. The study included 26 infants in the chloroprocaine group. The primary outcome was morphine administration postoperatively. The results showed that 2-chloroprocaine was not inferior to ropivacaine, and no specific safety concerns or complications related to chloroprocaine were reported in this age group [\hyperlink{pmid_26306545}{PMID: 26306545}, Wallis T Muhly et al., 2015].

2. **Infants and Children (general pediatric population):**
   - A review article discusses the use of 2-chloroprocaine for regional anesthesia in infants and children, highlighting its rapid metabolism and potential advantages in neonates and infants. The review presents dosing regimens and discusses applications, but does not report new safety data or targeted prospective studies. It supports the use of 2-chloroprocaine as an attractive alternative, especially in the presence of liver impairment or when higher infusion rates are needed [\hyperlink{pmid_28321983}{PMID: 28321983}, Giorgio Veneziano et al., 2017].
   - A case report describes a brief episode of local anesthetic systemic toxicity following 3\% 2-chloroprocaine in an infant, with complete recovery and no long-term effects. The authors note that such events are infrequent and support continued use in infants, with the caveat that volume should be minimized [\hyperlink{pmid_27089835}{PMID: 27089835}, Maria A Hernandez et al., 2016].
   - An earlier case series describes the use of chloroprocaine for epidural anesthesia in five pediatric patients (including neonates and infants), with no complications reported. The authors suggest chloroprocaine is an acceptable alternative to bupivacaine in the pediatric population [\hyperlink{pmid_7740909}{PMID: 7740909}, J D Tobias et al., 1995].

3. **Preclinical (animal) data:**
   - A study in juvenile rats evaluated the safety of single maximum tolerated doses of intrathecal 2-chloroprocaine at different postnatal ages (P7, P14, P21). No evidence of developmental neurotoxicity was found, but the authors caution that results cannot be extrapolated to repeated dosing or prolonged infusion [\hyperlink{pmid_34801587}{PMID: 34801587}, Suellen M Walker et al., 2022].

**Summary by Age Range:**
- For infants up to 6 months, there is targeted clinical data (retrospective cohort and case series) supporting the safety of chloroprocaine hydrochloride for epidural anesthesia, with no significant complications reported.
- For older children, the evidence is less direct but includes case series and review articles suggesting safety, with rare and transient adverse events.
- There are no abstracts reporting targeted studies showing chloroprocaine hydrochloride is unsafe in children.
- There is no evidence from the abstracts for safety or unsafety in repeated or prolonged use beyond the studied settings.


\subsection*{Abstracts}
\hypertarget{pmid_7740909}{T}he authors discuss their experience with chloroprocaine for epidural anesthesia in five pediatric patients. While bupivacaine remains the most commonly used local anesthetic in children, recent reports of toxicity document the risks of this agent. The major advantage of chloroprocaine is its rapid metabolism, which thereby minimizes the risks of toxicity, especially in patients with preexisting problems such as young age or underlying hepatic dysfunction, which may limit the metabolism of local anesthetics of the amide class. In three cases, the epidural infusion was combined with the general anesthetic. The cases included hepatic resection, repair of bladder exstrophy, and correction of duodenal atresia. In two other cases, epidural anesthesia was used instead of general anesthesia in a former preterm infant who was undergoing inguinal herniorrhaphy and for lower extremity orthopedic procedures in a child with myotonic dystrophy. In all cases, chloroprocaine was chosen because of preexisting or associated conditions that might increase the risk of bupivacaine toxicity, such as hepatic resection, repeated dosing in a neonate, or the need for higher concentrations of local anesthetic to achieve adequate surgical conditions. Adequate intraoperative conditions were achieved in all five patients. No complications related to chloroprocaine epidural anesthesia were noted. This initial experience suggests that chloroprocaine offers an acceptable alternative to bupivacaine for epidural anesthesia in the pediatric population. [\hyperlink{Chloroprocaine Hydrochloride}{PMID: 7740909}, J D Tobias et al., 1995]

\hypertarget{pmid_10851644}{C}iprofloxacin clinical and bacteriological efficacies, as well as tolerability mainly with respect to chondrotoxicity were evaluated in the treatment of children with mucoviscidosis. The drug was shown to have high clinical and moderate bacteriological efficacies. As for its tolerability, adverse reactions chiefly associated with affection of the gastrointestinal tract, i.e. nausea, stomach pain, diarrhea, increased transaminase levels were recorded. The arthrotoxicity episode was single and transitory. The use of ciprofloxacin had no negative effect on the children growth. No chondrotoxic effect of ciprofloxacin in the treatment of children was observed which is explained in the paper. It is concluded that ciprofloxacin is in general an efficient and safe antibiotic useful for the treatment of children. [\hyperlink{Chloroprocaine Hydrochloride}{PMID: 10851644}, S S Postnikov et al., 2000]

\hypertarget{pmid_2402648}{C}hloral hydrate has been used extensively to sedate children, but at Brooke Army Medical Center, other drug combinations were becoming increasingly popular due to a perception that chloral hydrate had a high rate of failure, especially with younger or neurologically impaired children. Therefore, 50 children were given the drug before a diagnostic study, and patient data and a sedation score were recorded on a worksheet. Of 50 children, 43 (86\%) were "successfully sedated" on the first attempt with no side effects. Children with neurologic disorders had a much greater (27\% vs 4\%) failure rate than "normal" children. The sedation rate did not significantly differ by age, sex, or initial drug dosage. The study suggest that chloral hydrate is a safe and effective oral sedative but that children with neurologic disorders may need alternative drugs for sedation. [\hyperlink{Chloroprocaine Hydrochloride}{PMID: 2402648}, P D Rumm et al., 1990]

\hypertarget{pmid_21531030}{C}hloral hydrate (CH) is an oral sedative widely used to sedate infants and young children during auditory brainstem response (ABR) testing. The aim of this study was to record effectiveness, complications and safety of CH as a sedative for ABR. From January of 2003 until December of 2007, 1903 children were tested for ABR, 568 of them being under the age of 6 months. CH (8\%) was used for sedation at a dose of 40 mg/kg with a repeat dose, if necessary, for an adequate sedation, in 20-30 min. We recorded the effectiveness of CH as a sedative for ABR examination, as well as all complications related to the use of CH such as vomiting, rash, hyperactivity, respiratory distress and apnea. The statistical method used was the absolute and percentage frequency distribution of the occurrences. Sedation with CH was necessary to perform testing in 1591 (83.6\%) of the examined children. However, in the population of the examined infants, only 341 (60\%) were sedated with CH, because the remaining 227 (40\%) fell asleep by themselves. Complications included hyperactivity in 152 children (8\%), minor respiratory distress in 10 children (0.4\%), vomiting in 217 children (11.4\%), apnea in 4 children (0.2\%) and rash in 10 children (0.4\%). The complications of hyperactivity, vomiting and rash resolved without any medical treatment. The apnea cases were managed effectively by supplying ventilation to the children via a mask in the presence of an anesthesiologist. The use of CH at a dose of 40 mg/kg up to 80 mg/kg is safe and effective when administered in a setting with adequate equipment and the presence of well trained personnel. [\hyperlink{Chloroprocaine Hydrochloride}{PMID: 21531030}, Eirini Avlonitou et al., 2011]

\hypertarget{pmid_2026812}{C}hloral hydrate is commonly used to sedate children before CT. However, no prospective study has been published of the safety and efficacy of chloral hydrate at high dose levels for children undergoing CT. We define high dose levels of oral chloral hydrate to be 80-100 mg/kg, with a maximum total dose of 2 g. High dose chloral hydrate sedation was administered orally to 295 children for 326 CT examinations. Adverse reactions occurred in 7\% of the children, with vomiting being the most common (4.3\% of children). Hyperactivity and respiratory symptoms each occurred in less than 2\% of children. Prolonged sedation ( greater than 2 h) was not encountered in our series. Sedation was successful in producing motion free CT examinations, so that in 303 (93\%) of the cases, no repeat CT scans were needed. We conclude that high dose oral chloral hydrate provides safe and effective sedation for children undergoing CT. [\hyperlink{Chloroprocaine Hydrochloride}{PMID: 2026812}, S B Greenberg et al., ]

\hypertarget{pmid_28827252}{C}eftriaxone is widely used in children in the treatment of sepsis. However, concerns have been raised about the safety of ceftriaxone, especially in young children. The aim of this review is to systematically evaluate the safety of ceftriaxone in children of all age groups. MEDLINE, PubMed, Cochrane Central Register of Controlled Trials, EMBASE, CINAHL, International Pharmaceutical Abstracts and adverse drug reaction (ADR) monitoring systems will be systematically searched for randomised controlled trials (RCTs), cohort studies, case-control studies, cross-sectional studies, case series and case reports evaluating the safety of ceftriaxone in children. The Cochrane risk of bias tool, Newcastle-Ottawa and quality assessment tools developed by the National Institutes of Health will be used for quality assessment. Meta-analysis of the incidence of ADRs from RCTs and prospective studies will be done. Subgroup analyses will be performed for age and dosage regimen. Formal ethical approval is not required as no primary data are collected. This systematic review will be disseminated through a peer-reviewed publication and at conference meetings. CRD42017055428. [\hyperlink{Chloroprocaine Hydrochloride}{PMID: 28827252}, Linan Zeng et al., 2017]

\hypertarget{pmid_28741653}{C}hloral hydrate is commonly used to sedate children for painless procedures. Children may recover more quickly after sedation with dexmedetomidine, which has a shorter half-life. We randomly allocated 196 children to chloral hydrate syrup 50 mg.kg [\hyperlink{Chloroprocaine Hydrochloride}{PMID: 28741653}, V M Yuen et al., 2017] Chloral hydrate (CH), as a sedation agent, is widely used in children for diagnostic or therapeutic procedures. However, it has not come into the market and is currently only used as hospital preparation in China. This review aims to systematically evaluate the efficacy of CH in children of all age groups for sedation before medical procedures. Seven electronic databases and three clinical trial registry platforms were searched and the deadline was September 2018. Randomized controlled trials (RCTs) evaluating the efficacy of CH for sedation in children were included by two reviewers. The extracted information included success rate of sedation, sedation latency and sedation duration. The Cochrane risk of bias tool was applied to assess the risk of bias. The outcomes were analyzed by Review Manager 5.3 software and expressed as relative risks (RR) or Mean Difference (MD) with 95\% confidence interval (CI). Heterogeneity was assessed with I-squared (I A total of 24 RCTs involving 3564 children of CH for sedation were included in the meta-analysis. Compared to placebo group, CH group had a significant increase in success rate of sedation when used for painless and painful procedure (RR=4.15, 95\% CI [1.21, 14.24], P=0.02; RR=1.28, 95\% CI [1.17, 1.40], P<0.01), which included 22 and 455 children for this analysis, respectively. Compared to midazolam group, CH group had a significant increase in success rate of sedation (RR=1.63, 95\% CI [1.48, 1.79], I From the extrapolation of the existing literature, CH oral solution is an appropriate effective alternative for sedation in pediatrics. [\hyperlink{Chloroprocaine Hydrochloride}{PMID: 28741653}, Zhe Chen et al., 2019]

\hypertarget{pmid_24445981}{T}o compare efficacy and safety of chloral hydrate (CH), chloral hydrate and promethazine (CH + P) and chloral hydrate and hydroxyzine (CH + H) in electroencephalography (EEG) sedation. In a parallel single-blinded randomized clinical trial, ninety 1-7 y-old uncooperative kids who were referred to Pediatric Neurology Clinic of Shahid Sadoughi University, Yazd, Iran from April through August 2012, were randomly assigned to receive 40 mg/kg of chloral hydrate or 40 mg/kg of chloral hydrate and 1 mg/kg of promethazine or 40 mg/kg of chloral hydrate and 2 mg/kg of hydroxyzine. The primary endpoint was efficacy in sufficient sedation (obtaining four Ramsay sedation score) and successful completion of EEG. Secondary endpoint was clinical adverse events. Thirty nine girls (43.3 \%) and 51 boys (56.7 \%) with mean age of 3.34 ± 1.47 y were assessed. Sufficient sedation and completion of EEG were achieved in 70 \% (N = 21) of chloral hydrate group, in 83.3 \% (N = 25) of CH + H group and in 96.7 \% (N = 29) of CH + P group (p = 0.02). Mild clinical adverse events including vomiting [16.7 \% (N = 5) in CH, 6.7 \% (N = 2) in CH + P, 6.7 \% (N = 2) in CH + H], agitation in 3.3 \% of CH + P (N = 1) group and mild transient hypotension in 3.3 \% of CH + H (N = 1) group occurred. Safety of these three sedation regimens was not statistically significant different (p = 0.14). Combination of chloral hydrate-antihistamines can be used as the most effective and safe sedation regimen in drug induced sleep electroencephalography of kids. [\hyperlink{Chloroprocaine Hydrochloride}{PMID: 24445981}, Razieh Fallah et al., 2014]

\hypertarget{pmid_22246409}{C}hloral hydrate (CH) is safe and effective for sedation of suitable children. The purpose of this study was to assess whether adequate sedation is achieved with reduced CH doses. We retrospectively recorded outpatient CH sedations over 1 year. We defined standard doses of CH as 50 mg/kg (infants) and 75 mg/kg (children >1 year). A reduced dose was defined as at least 20\% lower than the standard dose. In total, 653 children received CH sedation (age, 1 month-3 years 10 months), 42\% were given a reduced initial dose. Augmentation dose was required in 10.9\% of all children, and in a higher proportion of children >1 year (15.7\%) compared to infants (5.7\%; P < 0.001). Sedation was successful in 96.7\%, and more frequently successful in infants (98.3\%) than children >1 year (95.3\%; P = 0.03). A reduced initial dose had no negative effect on outcome (P = 0.19) or time to sedation. No significant complications were seen. We advocate sedation with reduced CH doses (40 mg/kg for infants; 60 mg/kg for children >1 year of age) for outpatient imaging procedures when the child is judged to be quiet or sleepy on arrival. [\hyperlink{Chloroprocaine Hydrochloride}{PMID: 22246409}, Jennifer Bracken et al., 2012]

\hypertarget{pmid_16520840}{C}hloral hydrate is generally considered to be a safe hypnotic drug, and is commonly used for short-term sedation before diagnostic procedures. Its irritant actions to the mucous membranes are usually limited. We report a rare complication of chloral hydrate overdose in an infant. An 8-month-old male infant became unconscious and required ventilation support after an overdose of chloral hydrate was administered to provide sedation for an ophthalmologic examination. White plaques and sloughing of the oropharyngeal mucosa were observed on the next day. Esophagogastroscopy revealed severe corrosive lesions on the whole esophagus. The child recovered after supportive treatment and his oral intake remained well without dysphagia after 1 year. This report illustrates the potential corrosive effect of chloral hydrate. Strict attention should be paid to the dosing and administration protocol of chloral hydrate in infants. The condition of the oropharyngeal mucosa should be carefully monitored after chloral hydrate administration. [\hyperlink{Chloroprocaine Hydrochloride}{PMID: 16520840}, Yu-Cheng Lin et al., 2006]

\hypertarget{pmid_26306545}{C}ontinuous thoracic epidural analgesia is useful in the management of infants following thoracotomy. Concerns about drug accumulation and toxicity limit the amount of amide local anesthetics that can be delivered. Continuous epidural infusions of the ester local anesthetic chloroprocaine result in little drug accumulation allowing for higher infusion rates. We retrospectively compared patients managed with 1.5\% 2- chloroprocaine or 0.1\% ropivacaine epidural infusions to determine if the increased infusion rate resulted in similar or improved analgesia. This retrospective cohort comparison consisted of full term infants 6 months or younger who underwent thoracotomy for congenital lung lesion resection. Patients were included if they were managed with either a 1.5\% 2-chloroprocaine (Group C) (n = 26) or 0.1\% ropivacaine (Group R) (n = 28) infusion administered through a caudally placed thoracic epidural catheter. The primary outcome was morphine administration at 0-24 h. Patients were similar in age, weight, length of stay, epidural location and duration. There was weak evidence for a difference in morphine use in the first 24 h in Group C compared to Group R (P = 0.08) but no difference 24-48 h. Group C was more commonly managed with ketorolac at 0-24 h (P = 0.03) and 24-48 h (P =< 0.01). The use of 2-chloroprocaine for continuous epidural infusion in infants following thoracotomy was not inferior to ropivacaine and there was weak evidence for a reduction in opioid consumption in the first 24 h postoperatively. However, the 2-chloroprocaine group was more likely to receive ketorolac. [\hyperlink{Chloroprocaine Hydrochloride}{PMID: 26306545}, Wallis T Muhly et al., 2015]

\hypertarget{pmid_24447296}{C}hloral hydrate is the most commonly used sedative for paediatric diagnostic procedures in China with a success rate of around 80\%. Intranasal dexmedetomidine is used for rescue sedation in our centre. This prospective investigation evaluated 213 children aged one month to 10 years who were not adequately sedated following administration of chloral hydrate. Children were randomly assigned to receive rescue intranasal dexmedetomidine at 1 μg.kg(-1) (group 1), 1.5 μg.kg(-1) (group 2) or 2 μg.kg(-1) (group 3). The sedation level was assessed every 10 min using a modified observer's assessment of alertness/sedation scale. Successful rescue sedation in groups 1, 2 and 3 were 56 (83.6\%), 66 (89.2\%) and 51 (96.2\%), respectively. Increasing the rescue dose was associated with an increased success rate with an odds ratio of 4.12 (95\% CI 1.13-14.98), p = 0.032. We conclude that intranasal dexmedetomidine is effective for sedation in children who do not respond to chloral hydrate.  [\hyperlink{Chloroprocaine Hydrochloride}{PMID: 24447296}, B L Li et al., 2014] Continuous epidural infusions are an effective and safe method of providing anesthesia and postoperative analgesia in infants and children with multiple advantages over systemic medications, including earlier tracheal extubation, decreased perioperative stress response, earlier return of bowel function, and decreased exposure to volatile anesthetic agents with uncertain long-term neurocognitive effects. Despite these benefits, local anesthetic toxicity remains a concern in neonates and infants because of their decreased metabolic capacity for amide local anesthetics. Chloroprocaine, an ester local anesthetic agent, which is rapidly metabolized in plasma at all ages, is an attractive alternative for this special population, particularly in the presence of superimposed liver impairment or when higher infusion rates are needed for surgical incisions stretching many dermatomes. The current manuscript reviews the literature pertaining to the use of 2-chloroprocaine for regional anesthesia in infants and children. Dosing regimens are presented and the applications of 2-chloroprocaine in this population are discussed. [\hyperlink{Chloroprocaine Hydrochloride}{PMID: 24447296}, Giorgio Veneziano et al., 2017]

\hypertarget{pmid_28242616}{A}lthough chloral hydrate (CH) has been used as a sedative for decades, it is not widely accepted as a valid choice for ophthalmic examinations in uncooperative children. This study aimed to systematically review the literature on the drug's safety and efficacy. We searched PubMed, EMBASE, ISI Web of Science, Scopus, CENTRAL, Google Scholar and Trip database to 1 October 2015, using the keywords 'chloral hydrate', 'paediatric' and 'procedural sedation OR diagnostic sedation'. A meta-analysis of randomised controlled trials (RCTs) was performed. A total of 6961 articles were screened and 104 were included in the review. Thirteen of these concerned paediatric ophthalmic examination, while 13 others were RCTs and were meta-analysed. CH was reported to have been administered in a total of 24 265 sedation episodes in children aged from <1 month to 18 years. The meta-analysis showed CH had a higher OR (2.95, 95\% CI 1.09 to 7.99) for successful sedation compared to other sedatives, but significant limitations apply. The commonest reported adverse events (AE) were not serious (eg, paradoxical reaction or transient vomiting) and required no intervention. Severe AE, including two deaths, were related to comorbidity, overdose or aspiration. Despite the paucity of high quality evidence, the existing literature suggests that the use of CH for procedural sedation in children appears to be an effective alternative to general anaesthesia, and it can be safe when administered in the hospital setting with appropriate monitoring and vigilance for intervention. [\hyperlink{Chloroprocaine Hydrochloride}{PMID: 28242616}, Asimina Mataftsi et al., 2017]

\hypertarget{pmid_20112608}{C}hloral hydrate is generally considered a safe sedative-hypnotic drug, and is commonly used for sedation of infants and young children before diagnostic procedures. Even chloral hydrate administered within the recommended maximal dose limits can cause serious morbidity and mortality. Here the authors describe a four-month-old girl with a life-threatening central nervous system and respiratory depression after administration of a therapeutic dose of chloral hydrate. The patient gradually recovered with supportive treatment including ventilation therapy. [\hyperlink{Chloroprocaine Hydrochloride}{PMID: 20112608}, Emre Ceçen et al., ]

\hypertarget{pmid_31264154}{C}hlordecone was used intensively as an insecticide in the French West Indies. Because of its high persistence, the resulting contamination of food and water has led to chronic exposure of the general population as evidenced by its presence in the blood of people of Guadeloupe, in particular in pregnant women and newborns, and in maternal breast milk. Chlordecone is recognized as a reproductive and developmental toxicant, is neurotoxic and carcinogenic in rodents, and is considered as an endocrine-disrupting compound with well-established estrogenic and progestogenic properties both in vitro and in vivo. The question arises of its potential consequences on child neurodevelopment following prenatal and childhood exposure, in particular on behavioral sexual dimorphism in childhood. We followed 116 children from the TIMOUN mother-child cohort study in Guadeloupe, who were examined at age 7. These children were invited to participate in a 7-min structured play session in which they could choose between different toys considered as feminine, masculine, or neutral. The play session was video recorded, and the percentage of the time spent playing with feminine or masculine toys was calculated. We estimated associations between playtime and prenatal exposure to chlordecone (assessed by concentration in cord blood) or childhood exposure (determined from concentrations in child blood obtained at the 7-year follow-up), taking into account confounders and co-exposures to other environmental chemicals. We used a two-group regression model to take into account sex differences in play behavior. Our results do not indicate any modification in sex-typed toy preference among 7-year-old children in relation with either prenatal or childhood exposure to chlordecone. [\hyperlink{Chloroprocaine Hydrochloride}{PMID: 31264154}, Sylvaine Cordier et al., 2020]

\hypertarget{pmid_11847958}{I}nformation regarding the treatment of anthrax infection is scarce in adults and is even more limited in children. Children, however, may be at a greater risk for developing an infection and systemic disease if exposed to anthrax than adults. The Centers for Disease Control and Prevention (CDC) recommends the use of doxycycline or ciprofloxacin for prophylaxis and treatment in children. Doxycycline currently is not indicated for use in children < 8 years old, due to staining of teeth and inhibition of bone growth associated with tetracyclines. Doxycycline, however, may have less adverse effect on teeth than its precursors. Ciprofloxacin has a pediatric indication only when a child is potentially exposed to inhaled anthrax. Ciprofloxacin is contraindicated in pediatric patients because fluoroquinolones were shown to cause cartilage toxicity in immature animals. Although children of various ages have received ciprofloxacin, there are few reports of cartilage toxicity. Because anthrax is a potentially fatal infection, the benefits to using these antibiotics greatly outweigh the risks. Therefore, the use of these antibiotics in children can be recommended, despite the lack of adequate efficacy and safety studies in pediatric patients with anthrax. [\hyperlink{Chloroprocaine Hydrochloride}{PMID: 11847958}, Sandra Benavides et al., 2002]

\hypertarget{pmid_20527137}{O}nly a few corticosteroids for topical use have proven safe and effective in pediatric populations down to 3 months of age. The authors report the results of a study designed to assess the efficacy and safety of hydrocortisone butyrate (HCB) 0.1\% in lipocream (LCr) vehicle in infants and children. A total of 264 boys and girls 3 months to less than 18 years old, with stable, mild to moderate atopic dermatitis affecting at least 10\% body surface area applied HCB 0.1\% in LCr or LCr alone twice daily for up to 1 month without occlusion. Primary end-points included: percent of patients who achieved treatment success based on physician global assessments. Secondary endpoint included: difference in pruritus and Eczema Area and Severity Index (EASI) at day 29. Treatment was significant (P < 0.001) for HCB 0.1\% LCr over vehicle. No serious nor significant adverse events were reported. Results are representative of a short duration treatment for a chronic disease. HCB 0.1\% in LCr is more effective than its vehicle in pediatric populations down to 3 months of age without significant adverse events when used twice a day for up to 1 month. [\hyperlink{Chloroprocaine Hydrochloride}{PMID: 20527137}, William Abramovits et al., ]

\hypertarget{pmid_27089835}{R}egional anesthesia use in pediatric patients has a good safety profile. 2-Chloroprocaine is used frequently in infants due to rapid onset, lack of accumulation, and rapid plasma degradation. We present a case of local anesthetic systemic toxicity following the administration of 3\% 2-chloroprocaine through a paravertebral catheter in an infant. The episode lasted 40 s followed by complete recovery. The infrequent reporting of local anesthetic systemic toxicity and limited duration of symptoms supports the continued use of 2-chloroprocaine in infants. Volume should be restricted to the smallest amount providing analgesia. [\hyperlink{Chloroprocaine Hydrochloride}{PMID: 27089835}, Maria A Hernandez et al., 2016]

\hypertarget{pmid_33655976}{C}hildren evaluated in the emergency department for head trauma often undergo computed tomography (CT), with some uncooperative children requiring pharmacological sedation. Chloral hydrate (CH) is a sedative that has been widely used, but its rectal use for child sedation after head trauma has rarely been studied. The objective of this study was to document the safety and efficacy of rectal CH sedation for cranial CT in young children.We retrospectively studied all the children with head trauma who received rectal CH sedation for CT in the emergency department from 2016 to 2019. CH was administered rectally at a dose of 50 mg/kg body weight. When sedation was achieved, CT scanning was performed, and the children were monitored until recovery. The sedative safety and efficacy were analyzed.A total of 135 children were enrolled in the study group, and the mean age was 16.05 months. The mean onset time was 16.41 minutes. Successful sedation occurred in 97.0\% of children. The mean recovery time was 71.59 minutes. All of the vital signs were within normal limits after sedation, except 1 (0.7\%) with transient hypoxia. There was no drug-related vomiting reaction in the study group. Adverse effects occurred in 11 patients (8.1\%), but all recovered completely. Compared with oral CH sedation, rectal CH sedation was associated with quicker onset (P < .01), higher success rate (P < .01), and lower adverse event rate (P < .01).Rectal CH sedation can be a safe and effective method for CT imaging of young children with head trauma in the emergency department. [\hyperlink{Chloroprocaine Hydrochloride}{PMID: 33655976}, Quanmin Nie et al., 2021]

\hypertarget{pmid_15951862}{D}iagnostic and therapeutic procedures in children are made easier using sedation. However, there is no consensus about which drug should be used to achieve this. Furthermore, none of the drugs used for sedation are risk free. The aim of this work is to study sedation indications, effectiveness, and safety at our center. A prospective observational study conducted at the Pediatric Day Care Unit, King Fahad National Guard Hospital, Riyadh, Saudi Arabia. The study covered 17.5 weeks in 2 periods: May 9th 1999 to June 13th 1999 and October 31st 2001 to February 11th 2002. Children <12 years were included. Collected data included demographics, indication, drug dosing and outcome. Data were reported as mean +/- SD. We included 148 patients, age 38 +/- 30 months. Adequate sedation was achieved in 79\% after initial chloral hydrate (CH) dose of 56.9 +/- 9.3 mg/kg, in 95\% after adding 18.5 +/- 6.4 mg/kg CH and in 96\% after adding second drug. Compared to nonrespondents, first CH dose respondents were younger and lower in weight. The CH side effects were few and mild. Chloral hydrate is a safe and effective agent for sedation in children with an age and weight dependent response. [\hyperlink{Chloroprocaine Hydrochloride}{PMID: 15951862}, Omar M Hijazi et al., 2005]

\hypertarget{pmid_34801587}{S}pinally-administered local anesthetics provide effective perioperative anesthesia and/or analgesia for children of all ages. New preparations and drugs require preclinical safety testing in developmental models. We evaluated age-dependent efficacy and safety following 1 \% preservative-free 2-chloroprocaine (2-CP) in juvenile Sprague-Dawley rats. Percutaneous lumbar intrathecal 2-CP was administered at postnatal day (P)7, 14 or 21. Mechanical withdrawal threshold pre- and post-injection evaluated the degree and duration of sensory block, compared to intrathecal saline and naive controls. Tissue analyses one- or seven-days following injection included histopathology of spinal cord, cauda equina and brain sections, and quantification of neuronal apoptosis and glial reactivity in lumbar spinal cord. Following intrathecal 2-CP or saline at P7, outcomes assessed between P30 and P72 included: spinal reflex sensitivity (hindlimb thermal latency, mechanical threshold); social approach (novel rat versus object); locomotor activity and anxiety (open field with brightly-lit center); exploratory behavior (rearings, holepoking); sensorimotor gating (acoustic startle, prepulse inhibition); and learning (Morris Water Maze). Maximum tolerated doses of intrathecal 2-CP varied with age (1.0 μL/g at P7, 0.75 μL/g at P14, 0.5 μL/g at P21) and produced motor and sensory block for 10-15 min. Tissue analyses found no significant differences across intrathecal 2-CP, saline or naïve groups. Adult behavioral measures showed expected sex-dependent differences, that did not differ between 2-CP and saline groups. Single maximum tolerated in vivo doses of intrathecal 2-CP produced reversible spinal anesthesia in juvenile rodents without detectable evidence of developmental neurotoxicity. Current results cannot be extrapolated to repeated dosing or prolonged infusion. [\hyperlink{Chloroprocaine Hydrochloride}{PMID: 34801587}, Suellen M Walker et al., 2022]

\hypertarget{pmid_7633153}{W}e evaluated the safety of ciprofloxacin administered in a dose of 15-25 mg/kg for 9-16 days, in a case series of 58 children who were between 8 months and 13 years of age. No arthropathy was observed during therapy and follow-up. Blinded evaluation of 22 pairs of nuclear magnetic resonance scans obtained before and between day 10 and 15 of therapy did not reveal any cartilage damage. After the first dose of ciprofloxacin (10 mg/kg), serum fluoride levels increased at 12 h in 15 of 19 (79\%) patients; 24-h urinary fluoride excretion was higher on day 7 compared with basal values in 16 of 18 (88.9\%) patients. Height z scores of 53 patients at a mean of 22.5 months of follow-up were not significantly different from basal scores (p = 0.12). In conclusion, ciprofloxacin may be recommended for use in children for short duration when effective alternative antibacterials are unavailable. However, there is a need for further studies to evaluate the tissue accumulation of fluoride and its potential to cause toxic effects. [\hyperlink{Chloroprocaine Hydrochloride}{PMID: 7633153}, K M Pradhan et al., 1995]

\hypertarget{pmid_18278305}{C}hloral hydrate and hydroxyzine are a drug combination frequently used by practitioners to sedate pediatric dental patients, but their effectiveness has not been compared to a negative control group in humans. The aim of this crossover, double-blinded study was to evaluate the effect of these drugs compared to a placebo, administered to young children for dental treatment. Thirty-five dental sedation sessions were carried out on 12 uncooperative ASA I children aged less than 5 years old. In each session patients were randomly assigned to groups P (placebo), CH (chloral hydrate 75 mg/kg) and CHH (chloral hydrate 50 mg/kg plus hydroxyzine 2.0 mg/kg). Vital signs and behavioral variables were evaluated every 15 min. Comparisons were statistically analyzed using Friedman and Wilcoxon tests. P, CH and CHH had no differences concerning vital signs, except for breathing rate. All vital signs were in the normal range. CH and CHH promoted more sleep in the first 30 min of treatment. Overall behavior was better in CH and CHH than in P. CH, CHH and P were effective in 62.5\%, 61.5\% and 11.1\% of the cases, respectively. Chloral hydrate was safe and relatively effective, causing more satisfactory behavioral and physiological outcomes than a placebo. [\hyperlink{Chloroprocaine Hydrochloride}{PMID: 18278305}, Luciane Ribeiro de Rezende Sucasas da Costa et al., 2007]

\section*{Cisatracurium Besylate}
\subsection*{Result}
\subsubsection*{Answer}

Yes (Infants ≥1 month to 1 year)
Yes (Children >1 year to 16 years)

\subsubsection*{{Explanation}}
\hypertarget{Cisatracurium Besylate}
A review of the available abstracts reveals multiple targeted studies evaluating the safety and efficacy of Cisatracurium Besylate in children across various age ranges. Below is a summary of the relevant evidence, organized by age group:

Infants (1 month to 1 year):
- Several studies specifically included infants. For example, one study evaluated 27 infants (aged 1–23 months) and 24 children (aged 2–12.5 years) who received 0.15 mg/kg cisatracurium. Complete neuromuscular blockade was achieved in all but one infant, and the drug was described as effective and well tolerated, with negligible changes in blood pressure and heart rate [\hyperlink{pmid_11069329}{PMID: 11069329}, T Taivainen et al., 2000].
- Another study compared infants (mean age 0.7 years, range 0.3–1.0 years) and children (mean age 4.9 years, range 3.1–9.6 years) and found similar potency and infusion requirements, with no mention of adverse safety signals [\hyperlink{pmid_11388529}{PMID: 11388529}, J de Ruiter et al., 2001].
- A further study compared onset, duration, and recovery in 15 infants and 15 children, noting that infants are more sensitive to cisatracurium, but no safety concerns were raised; neuromuscular monitoring was recommended due to interindividual differences [\hyperlink{pmid_12125308}{PMID: 12125308}, S Soltész et al., 2002].

Children (1 year to 12 years):
- Multiple studies included children in this age range. For example, a study of 21 children aged 2–12 years found that cisatracurium provided good or excellent intubation conditions, sufficient relaxation, and no appreciable hemodynamic fluctuations or side effects [\hyperlink{pmid_10584360}{PMID: 10584360}, E G Agavelian et al.].
- Another study of 181 infants and children aged 1 month to 12 years found that cisatracurium 0.15 mg/kg produced acceptable intubating conditions in the great majority of patients, with negligible changes in heart rate and arterial pressure [\hyperlink{pmid_17238881}{PMID: 17238881}, George H Meakin et al., 2007].
- A pharmacokinetic and pharmacodynamic study in 9 children aged 1–6 years found results comparable to other pediatric studies, with no safety concerns noted [\hyperlink{pmid_16492821}{PMID: 16492821}, Karynn Imbeault et al., 2006].
- A randomized study in 32 children aged 2–12 years found negligible changes in blood pressure or heart rate and described cisatracurium as safe and promising [\hyperlink{pmid_8880817}{PMID: 8880817}, O A Meretoja et al., 1996].
- A cost comparison study in 200 children aged 2–10 years undergoing dental surgery used cisatracurium and other agents, with no safety issues reported [\hyperlink{pmid_11473856}{PMID: 11473856}, W M Splinter et al., 2001].
- A study in critically ill children aged 3 months to 16 years (median 4.1 years) compared cisatracurium and vecuronium infusions, finding faster recovery with cisatracurium and no mention of safety concerns [\hyperlink{pmid_15815895}{PMID: 15815895}, Margarita Burmester et al., 2005].

Summary:
Across these studies, cisatracurium besylate was administered to infants (as young as 1 month) and children (up to 16 years) in both elective and critical care settings. The studies consistently report effective neuromuscular blockade, good intubating conditions, and negligible hemodynamic or other adverse effects. No study reported significant safety concerns or adverse events attributable to cisatracurium in these pediatric populations. Several studies explicitly describe the drug as "safe" or "well tolerated" in children.

Therefore, based on the abstracts reviewed, there is targeted evidence affirming the safety of cisatracurium besylate for use in infants (≥1 month) and children up to 16 years of age.

\subsection*{Abstracts}
\hypertarget{pmid_10584360}{C}isatracurium besilate was used in 21 children aged 2-12 years (ASA classes I-II) for intravenous anesthesia anesthetic + nitrous oxide + oxygen (group 1, 11 pts) and halothane + nitrous oxide + oxygen (group 2, 11 pts). In group 1 the initial nimbex dose was 0.15 mg/kg, in group 2 0.12 mg/kg, which created good or excellent conditions for intubation within 2 min and induced a neuromuscular blocking (NMB) for 44.2 +/- 4.7 and 37.6 +/- 5.9 min, respectively. NMB was maintained by bolus injections of nimbex in doses of 0.03 mg/kg (group 1) and 0.02 mg/kg (group 2). The duration of myoplegia was 18.6 +/- 3.6 and 13.2 +/- 2.3 min, respectively. Clinically the relaxation was sufficient throughout the operation. Neuromuscular conduction recovered spontaneously in all cases, extubation was carried out 42.2 +/- 3.8 min after the last bolus injection of nimbex in group 1 and after 35.4 +/- 7.4 min in group 2. No appreciable fluctuations of hemodynamic parameters or other side effects (skin hyperemia or bronchial spasm) were observed during anesthesia. Studies of the benzyl isoquinoline myorelaxant cisatracurium besilate demonstrated that this effective and safe drug with an average duration of effect can be used in pediatric anesthesiology. [\hyperlink{Cisatracurium Besylate}{PMID: 10584360}, E G Agavelian et al., ]

\hypertarget{pmid_9129870}{C}isatracurium besilate (besylate) is a nondepolarising neuromuscular blocking agent with an intermediate duration of action. It is the R-cis, R'-cis isomer of atracurium besilate and is approximately 3-fold more potent than the mixture of isomers that constitute the parent drug. The ED95 for cisatracurium besilate (dose required to produce 95\% suppression of twitch response to nerve stimulation) in adults is 0.05 mg/kg during N2O/O2 opioid anaesthesia. As for atracurium besilate, the primary route of elimination of cisatracurium besilate is by spontaneous degradation. Cisatracurium besilate is not associated with dose-related histamine release (at bolus doses of < or = 8 x ED95) and, consistent with this, has demonstrated cardiovascular stability in both healthy patients (< or = 8 x ED95) and those with coronary artery disease (< or = 6 x ED95). In clinical trials, cisatracurium besilate has been used successfully to facilitate intubation (at 2 to 4 x ED95) and as a muscle relaxant during surgery and in intensive care. Compared with vecuronium, cisatracurium besilate was associated with a significantly faster recovery after continuous infusion in patients in intensive care. Relative to atracurium besilate, cisatracurium besilate has a lower propensity to cause histamine release is more potent but has a slightly longer onset time at equipotent doses. It also offers a more predictable recovery profile than vecuronium after prolonged use in patients in intensive care. Thus, comparative data provide some indication of the potential of cisatracurium besilate as an intermediate-duration neuromuscular blocking agent but further comparisons with other like agents are required to define precisely its relative merits. [\hyperlink{Cisatracurium Besylate}{PMID: 9129870}, H M Bryson et al., 1997]

\hypertarget{pmid_11388529}{T}o determine the effect of age on the dose-response relation and infusion requirement of cisatracurium besylate in pediatric patients, 32 infants (mean age, 0.7 yr; range, 0.3-1.0 yr) and 32 children (mean age, 4.9 yr; range, 3.1-9.6 yr) were studied during thiopentone-nitrous oxideoxygen-narcotic anesthesia. Potency was determined using a single-dose (20, 26, 33, or 40 microg/kg) technique. Neuromuscular block was assessed by monitoring the electromyographic response of the adductor pollicis to supramaximal train-of-four stimulation of the ulnar nerve at 2 Hz. Least-squares linear regression analysis of the log-probit transformation of dose and maximal response yielded median effective dose (ED50) and 95\% effective dose (ED95) values for infants (29+/-3 microg/kg and 43+/-9 microg/kg, respectively) that were similar to those for children (29+/-2 microg/kg and 47+/-7 microg/kg, respectively). The mean infusion rate necessary to maintain 90-99\% neuromuscular block during the first hour in infants (1.9+/-0.4 microg x kg(-1) x min(-1); range: 1.3-2.5 microg x kg(-1) x min(-1)) was similar to that in children (2.0+/-0.5 microg x kg(-1) x min(-1); range: 1.3-2.9 microg x kg(-1) x min(-1)). The authors conclude that cisatracurium is equipotent in infants and children when dose is referenced to body weight during balanced anesthesia. [\hyperlink{Cisatracurium Besylate}{PMID: 11388529}, J de Ruiter et al., 2001]

\hypertarget{pmid_9606456}{T}he stability of cisatracurium besylate was studied. Cisatracurium (as besylate) 2 mg/mL in 5- and 10-mL unopened vials and 10 mg/mL in 20-mL unopened vials, as well as 3 mL of solution from additional 2-mg/mL vials, repackaged in 3-mL sealed plastic syringes, was stored at 4 and 23 degrees C in the dark and in normal fluorescent room light. Admixtures of cisatracurium (as besylate) 0.1, 2, or 5 mg/mL in polyvinyl chloride (PVC) minibags of 5\% dextrose injection or 0.9\% sodium chloride injection were stored at 4 and 23 degrees C in normal fluorescent room light. Triplicate samples for each storage condition were taken initially and at 1, 3, 5, 7, 14, 21, and 30 days; samples from vials were also removed at 45 and 90 days. Solutions were stored in sterile vials at -70 degrees C and then thawed at room temperature before analysis of chemical stability by high-performance liquid chromatography. Physical stability was assessed as well. Cisatracurium besylate was physically stable in all samples throughout the study. Cisatracurium (as besylate) 2 mg/mL exhibited drug losses at 23 degrees C in vials at 45 days and in syringes at 30 days. Cisatracurium (as besylate) 0.1, 2, and 5 mg/mL in 5\% dextrose injection and in 0.9\% sodium chloride injection was stable for at least 30 days at 4 degrees C, but substantial drug losses occurred at 23 degrees C. Admixtures prepared with cisatracurium (as besylate) 0.1 mg/mL and with 5\% dextrose injection exhibited the greatest losses. Cisatracurium besylate was stable in most samples for at least 30 days at 4 and 23 degrees C; admixtures containing cisatracurium (as besylate) 0.1 or 2 mg/mL exhibited substantial drug loss at 23 degrees C. [\hyperlink{Cisatracurium Besylate}{PMID: 9606456}, Q A Xu et al., 1998]

\hypertarget{pmid_16492821}{W}e studied the pharmacokinetics and pharmacodynamics of cisatracurium in 9 children (mean weight, 17.1 kg) aged 1-6 yr (mean, 3.75 yr) during propofol-nitrous oxide anesthesia. Neuromuscular monitoring was performed. Venous samples were taken before injection of a 0.1 mg/kg dose of cisatracurium and then at 2, 5, 10, 30, 60, 90, and 120 min. Cisatracurium plasma concentrations were determined by high performance liquid chromatography. Onset time was 2.5 +/- 0.8 min, recovery to 25\% of baseline twitch height was 37.6 +/- 10.2 min, and the 25\%-75\% recovery index was 10.9 +/- 3.7 min. Distribution and elimination half-lives were 3.5 +/- 0.9 min and 22.9 +/- 4.5 min, respectively. Steady-state volume of distribution (0.207 +/- 0.031 L/kg) and total body clearance (6.8 +/- 0.7 mL/min/kg) were significantly larger than those published for adults. Pharmacodynamic results were comparable to those obtained in pediatric studies during halothane or opioid anesthesia with the exception of a longer recovery to 25\% baseline. Although the plasma-effect compartment equilibration rate constant was twofold faster (0.115 +/- 0.025 min(-1)) than that published for cisatracurium in adults, the effect compartment concentration corresponding to 50\% block was similar (129 +/- 27 ng/mL). [\hyperlink{Cisatracurium Besylate}{PMID: 16492821}, Karynn Imbeault et al., 2006]

\hypertarget{pmid_9262747}{T}he compatibility of cisatracurium besylate with 91 other drugs during simulated Y-site injection was studied. Five milliliters of cisatracurium 0.1, 2, and 5 mg/mL (as besylate) in 5\% dextrose injection was combined with 5 mL of each of 91 drugs in 5\% dextrose injection or 0.9\% sodium chloride injection. All combinations were prepared in duplicate and stored at approximately 23 degrees C. Samples were visually examined under normal laboratory fluorescent light and, if there was no obvious visual incompatibility, under high-intensity monodirectional light. Turbidity was measured as well. Particle sizing and counting was performed for selected combinations. All evaluations were performed at intervals up to four hours. Cisatracurium besylate at all three concentrations was compatible with most of the drugs tested. However, one drug (cefoperazone) was incompatible with cisatracurium besylate at all three concentrations, 14 (including many cephalosporins) were incompatible with cisatracurium besylate 2 and 5 mg/ mL, and 12 were incompatible with cisatracurium 5 mg/ mL. During simulated Y-site administration, cisatracurium 0.1, 2, and 5 mg/mL (as besylate) in 5\% dextrose injection was compatible with 64 of 91 drugs for four hours at approximately 23 degrees C. Twenty-seven drugs were incompatible with cisatracurium besylate at one or more concentrations. [\hyperlink{Cisatracurium Besylate}{PMID: 9262747}, L A Trissel et al., 1997]

\hypertarget{pmid_12125308}{T}o compare the onset, duration and maximum effect of 0.1 mg/kg cisatracurium during balanced anesthesia with sevoflurane and remifentanil between infants and children. We measured the time course of the neuromuscular blockade in 15 infants and 15 children by electromyography. Anesthesia was induced with propofol/remifentanil and maintained with sevoflurane (constant 2\% endtidal) and remifentanil according to the patients individual requirements. After injection of 0.1 mg/kg cisatracurium we measured the following parameters: onset time: time between the beginning of injection of cisatracurium and maximum T1 depression, clinical duration: time between injection of the drug and recovery of T1 to 25\%, recovery index: time between recovery of T1 from 25\% to 75\%. TOFR 0.9: time between injection of cisatracurium and recovery of the train-of-four ratio to 90\%. In addition, we determined the maximum neuromuscular blockade Tmax after 0.1 mg/kg cistracurium. Both groups differed significantly with regard to onset time and clinical duration. In the infants, the onset time was shorter (74 s vs. 198 s) and the clinical duration longer (55 min vs. 41 min) compared to the older children. The TOFR 0.9 was 73 min (range 56-86 min) in the group of the infants and 59 min (range 43-72 min) in the group of the older children (p < 0.001). Tmax was 100\% (range 97-100\%) in the infants and 98\% (range 92-100\%) in the children (p < 0.01). However, the recovery index was comparable in both groups (21 vs. 16 min). Infants are substantially more sensitive to cisatracurium than children, which can be demonstrated in a significantly shorter onset time, a prolonged clinical duration and a delayed neuromuscular recovery. As there exist large interindividual differences, we recommend the use of neuromuscular monitoring in the routine practice of pediatric anesthesia. [\hyperlink{Cisatracurium Besylate}{PMID: 12125308}, S Soltész et al., 2002]

\hypertarget{pmid_18929279}{T}o describe, in pediatric patients, the effects of three doses of cisatracurium during nitrous oxide-propofol anesthesia and to determine if larger doses result in faster onset time. College hospital. 75 ASA physical status I and II children, aged 15 to 60 months, undergoing surgery for hypospadias or undescendent testis. Patients were randomly assigned to one of three groups according to the dose of cisatracurium: Group A = 0.1 mg/kg (two x effective dose), Group B = 0.15 mg/kg (three x effective dose), and Group C = 0.2 mg/kg (4 x effective dose). Neuromuscular block was assessed with TOF-Guard (Biometer International, Odense, Denmark) accelerometry. Onset time (from cisatracurium injection to maximal depression of time to first twitch), duration of peak effect (time from cisatracurium injection to 5\% recovery of time to first twitch), duration of clinical action (time from cisatracurium injection to 25\% recovery of time to first twitch), and recovery index (recovery of time to first twitch from 25\% to 75\%) were recorded. Cisatracurium had no effect on heart rate or blood pressure at any dose. Compared with Group A, onset times in Groups B and C were shorter; and durations of peak effect and clinical action in Groups B and C were longer (P < 0.01) than those in Group A. There was no difference in recovery index among the three groups. There was no difference in onset times between Groups B and C. Compared with Group B, durations of peak effect and clinical action in Group C were longer (P < 0.05 or P < 0.01). Four times the effective dose of cisatracurium did not significantly shorten onset time beyond that produced with three times the effective dose in young children. [\hyperlink{Cisatracurium Besylate}{PMID: 18929279}, WangNing ShangGuan et al., 2008]

\hypertarget{pmid_17238881}{T}he aims of the present study were to determine the tracheal intubating conditions, onset time, duration of action, and hemodynamic responses following the administration of cisatracurium 0.15 mg x kg(-1) to infants and children. One hundred and eighty-one infants and children aged 1 month to 12 years were randomized to two groups to receive anesthesia with nitrous oxide-oxygen-halothane (group H) or nitrous oxide-oxygen-thiopental-fentanyl (group TF). Intubation conditions were assessed 120 s after cisatracurium administration using a 4-part scale. Neuromuscular transmission was monitored by recording the evoked compound electromyogram of the adductor pollicis. The proportion of patients with excellent or good intubating conditions was similar in both groups (88 of 90, 98\% in group H; 85 of 90, 94\% in group TF). However, there was a significantly greater proportion of excellent intubating conditions in group H (79 of 90, 88\%) compared with group TF (65 of 90, 72\%) (P = 0.01) and recovery time was significantly longer in group H compared with group TF (P < 0.001). There was also a higher proportion of excellent intubating conditions in infants compared with older subjects (P = 0.02) and a shorter onset time (P < 0.001) and longer recovery time (P < 0.001) in younger compared with older patients. Changes in heart rate and arterial pressure were negligible 1 min following the cisatracurium administration. Cisatracurium 0.15 mg x kg(-1) produces acceptable intubating conditions at 120 s in the great majority of infants and children. Anesthesia background and age have significant effects on intubating conditions and duration of action of cisatracurium. [\hyperlink{Cisatracurium Besylate}{PMID: 17238881}, George H Meakin et al., 2007]

\hypertarget{pmid_8880817}{C}isatracurium, 51W89, is one of the ten stereoisomers of Tracrium which, unlike atracurium, has been reported to have a lack of histamine mediated cardiovascular effects at doses as high as 8 x ED95 in adults. We compared the time-course of neuromuscular effects of 80 micrograms.kg-1 or 100 micrograms.kg-1 cisatracurium during N2O-O2-halothane or N2O-O2-opioid anaesthesia, respectively, in 32 children 2-12 years old. Neuromuscular function was monitored by evoked adductor pollicis EMG. Even-numbered patients (n = 16) were allowed to obtain full spontaneous recovery of neuromuscular function and odd-numbered patients (n = 16) received neostigmine 45 micrograms.kg-1 together with glycopyrrolate at the time of 25\% EMG recovery. Data are expressed as median with 10th to 90th percentile range. Cisatracurium had an onset time (time from administration to maximal effect) of 2.2 (1.7-3.8) or 2.3 (1.8-4.9) min, a clinical duration (time to 25\% EMG recovery) of 34 (22-40) or 27 (24-33) min, and a spontaneous 25-75\% recovery time (time from 25 to 75\% EMG recovery) of 11 (9-13) or 11 (7-12) min during halothane or balanced anaesthesia, respectively (NS). Train-of-four ratio recovered to 0.70 in 2.5 (1.8-3.0) or 3.2 (2.1-4.3) min following neostigmine during halothane or balanced anaesthesia, respectively (NS). Changes in blood pressure or heart rate following cisatracurium were negligible. We regard cisatracurium as a safe and promising intermediate duration muscle relaxant the effects of which can easily be reversed with neostigmine. [\hyperlink{Cisatracurium Besylate}{PMID: 8880817}, O A Meretoja et al., 1996]

\hypertarget{pmid_9989341}{C}isatracurium besilate, one of the 10 stereoisomers that comprise atracurium besilate, is a nondepolarising neuromuscular blocking agent with an intermediate duration of action. Following a 5- to 10-sec intravenous bolus dose of cisatracurium besilate to healthy young adult surgical patients, elderly patients and patients with renal or hepatic failure, the concentration versus time profile of cisatracurium besilate is best characterised by a 2-compartment model. The volume of distribution (Vd) of cisatracurium besilate is small because of its relatively large molecular weight and high polarity. Cisatracurium besilate undergoes Hofmann elimination, a process dependent on pH and temperature. Unlike atracurium besilate, cisatracurium besilate does not appear to be degraded directly by ester hydrolysis. Hofmann elimination, an organ independent elimination pathway, occurs in plasma and tissue, and is responsible for approximately 77\% of the overall elimination of cisatracurium besilate. The total body clearance (CL), steady-state Vd and elimination half-life of cisatracurium besilate in patients with normal organ function are approximately 0.28 L/h/kg (4.7 ml/min/kg), 0.145 L/kg and 25 minutes, respectively. The magnitude of interpatient variability in the CL of cisatracurium besilate is low (16\%), a finding consistent with the strict physiological control of the factors that effect the Hofmann elimination of cisatracurium besilate (i.e. temperature and pH). There is a unique relationship between plasma clearance and Vd because the primary elimination pathway for cisatracurium besilate is not dependent on organ function. There are minor differences in the pharmacokinetics of cisatracurium besilate in various patient populations. These differences are not associated with clinically significant differences in the recovery profile of cisatracurium besilate, but may be associated with differences in the time to onset of neuromuscular block. [\hyperlink{Cisatracurium Besylate}{PMID: 9989341}, D F Kisor et al., 1999]

\hypertarget{pmid_15815895}{T}o evaluate and compare the efficacy, infusion rate and recovery profile of vecuronium and cisatracurium continuous infusion in critically ill children requiring mechanical ventilation. Prospective, randomised, double-blind, single-centre study in critically ill children in a paediatric intensive care unit in a tertiary children's hospital. Thirty-seven children from 3 months to 16 years old (median 4.1 year) were randomised to receive either drug; those already receiving more than 6 h of neuromuscular blocking drugs were excluded. The Train-of-Four (TOF) Watch maintained neuromuscular blockade to at least one twitch in the TOF response. Recovery time was measured from cessation of infusion until spontaneous TOF ratio recovery of 70\%. The cisatracurium infusion rate in nineteen children averaged 3.9+/-1.3 microg kg(-1) min(-1) with a median duration of 63 h (IQR 23-88). The vecuronium infusion rate in 18 children averaged mean 2.6+/-1.3 microg kg(-1) min(-1) with a median duration of 40 h (IQR 27-72). Median time to recovery was significantly shorter with cisatracurium (52 min, 35-73) than with vecuronium (123 min, 80-480). Prolonged recovery of neuromuscular function (>24 h) occurred in one child (6\%) on vecuronium. Recovery of neuromuscular function after discontinuation of neuromuscular blocking drug infusion in children is significantly faster with cisatracurium than vecuronium. Neuromuscular monitoring was not sufficient to eliminate prolonged recovery in children on vecuronium infusions. [\hyperlink{Cisatracurium Besylate}{PMID: 15815895}, Margarita Burmester et al., 2005]

\hypertarget{pmid_11069329}{W}e studied the neuromuscular and cardiovascular effects of a single, rapidly administered intravenous dose of cisatracurium 0.15 mg.kg(-1) in 27 infants (aged 1-23 months) and 24 children (aged 2-12.5 years). After midazolam premedication, anaesthesia was induced and maintained with thiopental and alfentanil in addition to nitrous oxide in oxygen. Neuromuscular function was monitored by evoked adductor pollicis electromyography. At least 15 min after intubation, each patient received cisatracurium 0.15 mg.kg(-1) over 5 s. Complete neuromuscular blockade was produced by this dose in all but one infant. The mean (SD) onset time of maximal blockade was more rapid in infants [2.0 (0.8) min] than in children [3.0 (1.2) min], p = 0. 0011. The clinical duration of action of cisatracurium (recovery of evoked response to 25\% of control) was significantly longer in infants [43.3 (6.2) min] than in children [36.0 (5.4) min], p < 0.0001. Once neuromuscular function started to recover, the rate of recovery was similar in both age groups. Changes in blood pressure and heart rate after the administration of cisatracurium were negligible in both age groups. Cisatracurium, at a dose of 0.15 mg. kg(-1), was effective and well tolerated in infants and children. [\hyperlink{Cisatracurium Besylate}{PMID: 11069329}, T Taivainen et al., 2000]

\hypertarget{pmid_15839217}{T}he authors have studied the effects of the muscular relaxants rocuronium bromide and cysatracurium besylate on neuromuscular conduction, respiration mechanics, and hemodynamics in 120 children during endosurgical operations. The paper comparatively assesses the muscular relaxants and the procedures of their use, which made it possible to ensure controlled and deep relaxation. [\hyperlink{Cisatracurium Besylate}{PMID: 15839217}, V V Makushkin et al., ]

\hypertarget{pmid_28827252}{C}eftriaxone is widely used in children in the treatment of sepsis. However, concerns have been raised about the safety of ceftriaxone, especially in young children. The aim of this review is to systematically evaluate the safety of ceftriaxone in children of all age groups. MEDLINE, PubMed, Cochrane Central Register of Controlled Trials, EMBASE, CINAHL, International Pharmaceutical Abstracts and adverse drug reaction (ADR) monitoring systems will be systematically searched for randomised controlled trials (RCTs), cohort studies, case-control studies, cross-sectional studies, case series and case reports evaluating the safety of ceftriaxone in children. The Cochrane risk of bias tool, Newcastle-Ottawa and quality assessment tools developed by the National Institutes of Health will be used for quality assessment. Meta-analysis of the incidence of ADRs from RCTs and prospective studies will be done. Subgroup analyses will be performed for age and dosage regimen. Formal ethical approval is not required as no primary data are collected. This systematic review will be disseminated through a peer-reviewed publication and at conference meetings. CRD42017055428. [\hyperlink{Cisatracurium Besylate}{PMID: 28827252}, Linan Zeng et al., 2017]

\hypertarget{pmid_31784219}{C}isatracurium besylate has been determined by fast and highly sensitive spectrofluorimetric method based on measuring the fluorescence intensity of its methanolic solution at 312 nm after excitation at 230 nm (Method I). The linearity occurred over the concentration range of 10.0-130.0 ng/mL with detection limit of 1.07 ng/mL. The method was further extended for the determination of the studied drug in spiked human plasma with good percentage recoveries (97.43-103.50\%). Cisatracurium is co-administered with nalbuphine during surgery. The simultaneous determination of both drugs was based on synchronous spectrofluorimetric technique. First derivative synchronous spectrofluorimetric amplitude was measured in methanol at Δ λ = 60 nm and each drug could be estimated at the zero crossing point of the other. Hence, cisatracurium could be measured at 284.6 nm while nalbuphine at 276.3 nm (Method II). The method was linear over the ranges of 50.0-750.0 ng/mL and 0.5-7.0 μg/mL with the detection limits of 2.16 ng/mL and 0.04 μg/mL for cisatracurium and nalbuphine, respectively. The method was further extended for the simultaneous determination of both drugs in spiked human urine with mean percentage recoveries of 99.99 ± 2.06 and 99.53 ± 6.17 for cisatracurium and nalbuphine, respectively. Both methods were validated in agreement with Guidelines adopted by International Council of Harmonization (ICH). [\hyperlink{Cisatracurium Besylate}{PMID: 31784219}, Mona E El Sharkasy et al., 2020]

\hypertarget{pmid_12542606}{G}astroesophageal reflux is a common problem in infancy. Cisapride is a commonly used therapy for gastroesophageal reflux in children. In view of recent concern regarding adverse effects this study aims to evaluate the benefits and risks of cisapride for the treatment of gastroesophageal reflux in children. A meta-analysis of randomized controlled trials of cisapride using a random-effects model. Ten trials involving 415 children were identified. There was no evidence of a significant reduction in vomiting severity with cisapride as measured by a clinical score (five trials, standardized weighted mean difference -0.18; 95\% confidence interval (CI) -0.51 to 0.15). Twenty-four-hour esophageal pH monitoring data showed the mean reflux index was significantly lower in the children treated with cisapride compared with controls (five trials, weighted mean difference -6.24; 95\% CI -8.81 to -3.67). With cisapride treatment, there was no reduction in the mean number of reflux episodes lasting greater than 5 min (three trials, weighted mean difference -0.72; 95\% CI -1.92 to 0.47) or in the number of children with esophagitis at final follow up compared with baseline (two trials, relative risk 0.80; 95\% CI 0.40 to 1.61). There was no significant difference in reported side-effects or adverse events (six trials, relative risk 1.16; 95\% CI 0.95 to 1.41). No clinically important benefits of cisapride in children with gastroesophageal reflux have been demonstrated. Nor was there any evidence of adverse or harmful events. [\hyperlink{Cisatracurium Besylate}{PMID: 12542606}, Jacqueline R Dalby-Payne et al., 2003]

\hypertarget{pmid_22378696}{A}nti-seizure prophylaxis is routinely utilized during busulfan administration for HSCT. We evaluated the feasibility and efficacy of levetiracetam in children undergoing HSCT. A total of 28 children and young adults received levetiracetam during HSCT and the outcomes and costs were compared to a historical, but similar cohort of 25 patients who had received fosphenytoin. Levetiracetam was well tolerated and was efficacious in preventing seizures. Cost of drug, administration, and monitoring were also similar among the two groups. Due to non-induction of the hepatic cytochrome P450 enzymes, levetiracetam may lead to better dose assurance of busulfan in targeted dose regimens for HSCT. [\hyperlink{Cisatracurium Besylate}{PMID: 22378696}, Sandeep Soni et al., 2012]

\hypertarget{pmid_9831411}{T}his paper provides a comprehensive review of the current knowledge on cisapride in different clinical conditions in children: different manifestations of gastro-oesophageal reflux, such as (excessive) regurgitation, oesophagitis, chronic respiratory disease or uncontrolled asthma, cystic fibrosis, chronic dyspepsia, constipation and pseudo-obstruction, and as an aid to small bowel capsule-biopsy. It discusses, in depth, the safety profile of cisapride in paediatric patients. [\hyperlink{Cisatracurium Besylate}{PMID: 9831411}, Y Vandenplas et al., 1998]

\hypertarget{pmid_9297378}{C}isatracurium (51W89) is one of the ten stereoisomers of atracurium, accounting for about 15\% of the racemate. The ED95 of cisatracurium was determined to be about 50 micrograms/kg (cation, molecular weight 929), while the ED95 of atracurium (besylate salt, molecular weight 1245) was 250 micrograms/kg. Thus, on a molar basis in adult patients, cisatracurium is about 3.5 times as potent as the racemic atracurium mixture. We compared atracurium with cisatracurium in healthy adult patients and found an almost identical pharmacodynamic profile. In children, an ED95 of about 40 micrograms/kg was determined, while a 1-min-longer onset of cisatracurium was found in geriatric than in young adult patients. The presence of chronic renal failure did not prolong the duration of action of cisatracurium. The recovery of neuromuscular transmission from a cisatracurium infusion of up to 145 h was investigated in intensive care unit patients. Their time from the end of infusion to a train-of-four ratio > 0.7 (68 +/- 18 min) was on average only some 70\% longer than after an infusion of cisatracurium for 2 h in normal surgical patients. In another study, no signs of histamine release nor any clinically relevant cardiovascular effects of cisatracurium were found in doses up to eight times ED95. [\hyperlink{Cisatracurium Besylate}{PMID: 9297378}, H Mellinghoff et al., 1997]

\hypertarget{pmid_34462863}{C}ommunity-acquired pneumonia (CAP)/community-acquired bacterial pneumonia (CABP) and complicated skin and soft tissue infection (cSSTI)/acute bacterial skin and skin structure infection (ABSSSI) represent major causes of morbidity and mortality in children. β-Lactams are the cornerstone of antibiotic treatment for many serious bacterial infections in children; however, most of these agents have no activity against methicillin-resistant Staphylococcus aureus (MRSA). Ceftaroline fosamil, a β-lactam with broad-spectrum in vitro activity against Gram-positive pathogens (including MRSA and multidrug-resistant Streptococcus pneumoniae) and common Gram-negative organisms, is approved in the European Union and the United States for children with CAP/CABP or cSSTI/ABSSSI. Ceftaroline fosamil has completed a pediatric investigation plan including safety, efficacy, and pharmacokinetic evaluations in patients with ages ranging from birth to 17 years. It has demonstrated similar clinical and microbiological efficacy to best available existing treatments in phase III-IV trials in patients aged ≥ 2 months to < 18 years with CABP or ABSSSI, with a safety profile consistent with the cephalosporin class. It is also approved in the European Union for neonates with CAP or cSSTI, and in the US for neonates with ABSSSI. Ceftaroline fosamil dosing for children (including renal function adjustments) is supported by pharmacokinetic/pharmacodynamic modeling and simulations in appropriate age groups, and includes the option of 5- to 60-min intravenous infusions for standard doses, and a high dose for cSSTI patients with MRSA isolates, with a ceftaroline minimum inhibitory concentration of 2-4 mg/L. Considered together, these data suggest ceftaroline fosamil may be beneficial in the management of CAP/CABP and cSSTI/ABSSSI in children. [\hyperlink{Cisatracurium Besylate}{PMID: 34462863}, Susanna Esposito et al., 2021]

\hypertarget{pmid_11473856}{T}he purpose of this investigation was to compare the costs of intermediate-acting neuromuscular blocking drugs in children during routine ambulatory surgery. We studied 200 healthy, 2-10-yr-old children undergoing elective dental restorative surgery. During Part 1 of the study, children received an inhaled anesthetic with halothane and nitrous oxide, whereas in Part 2, the anesthetic was IV propofol with nitrous oxide. The study drugs were atracurium, cisatracurium, mivacurium, rocuronium, and vecuronium. Patients were initially administered 2x the effective dose for 95\% of the study drug. After recovery to 10\% of baseline neuromuscular function, the neuromuscular blockade was rigidly maintained with an infusion of the study drug at about 10\% of baseline function. Neuromuscular drug costs were approximated as drug usage x cost/unit. The initial drug costs were not substantially different for both Parts 1 and 2, but over time, mivacurium became the most expensive drug and cisatracurium the least expensive. In conclusion, based on current costs, cisatracurium is the least expensive intermediate-acting neuromuscular drug. For children undergoing minor ambulatory procedures of 1-2 h, and continuous intraoperative neuromuscular blockade is indicated, cisatracurium currently is the least expensive drug. [\hyperlink{Cisatracurium Besylate}{PMID: 11473856}, W M Splinter et al., 2001]

\hypertarget{pmid_3866088}{A} clinical trial of ceftizoxime suppositories (CZX-S) was performed to evaluate the therapeutic effectiveness in children with bacterial infection. The subjects were 10 children comprising 4 with pneumonia, 3 with lacunar tonsillitis, 2 with pharyngitis, and 1 with UTI. They were given 1 suppository containing either 125 mg or 250 mg of CZX 2 to 4 times a day. The daily per kg body weight dose ranged from 17.1 to 60.0 mg. The result was "markedly effective" in 3, "effective" in 6, and "failure" was recorded in 1. Bacteriologically, successful eradication of causative organisms was confirmed in all the 4 children who underwent the test. No clinical side effects were observed. The only laboratory test abnormality recorded in a single patient was eosinophilia, which was not definitely ascribable to CZX-S. In conclusion, CZX-S have proved to be a clinically safe and effective antibiotic preparation in infantile infection, even in children whose treatment with conventional antibiotics is associated with difficulties. [\hyperlink{Cisatracurium Besylate}{PMID: 3866088}, T Hosoda et al., 1985]

\hypertarget{pmid_9719722}{C}isatracurium is one of ten isomers that form the racemic mix of atracurium (51W89 or 1 R-cis, 1'R-cis atracurium). It is three times more potent than atracurium itself and hemodynamically stable thanks to its scarce release of histamine. Cisatracurium is hydrolyzed mainly by the pathway of Hofmann (77\%) and to a lesser degree it is metabolized by organ-dependent modes (mainly by the kidney (16\%)). Dose therefore hardly needs to be changed for elderly patients or those with liver, kidney or cardiovascular disease. The calculated ED95 is 0.05 mg.kg-1 (0.04 mg.kg-1 in children), although a dose two to four times greater is used under clinical conditions to shorten tracheal intubation time because of low onset of blockade, particularly in comparison with rocuronium. The period of deep blockade (lack of response to neurostimulation) is prolonged by the higher dose, but recovery is dose-independent and recovery indices are similar. Cisatracurium has proven useful in intensive care because of its hemodynamic stability, which is comparable to that of steroid derivatives but with faster recovery from blockade once administration is discontinued. Its metabolism predominantly through Hofmann's pathway, with less laudanosine formation than is produced by atracurium, is also appreciated. Cisatracurium is described as the nondepolarizing muscle relaxant of choice for medium-to-long-term surgery on hemodynamically unstable patients or those with kidney or liver disease, and for neuromuscular blockade in intensive care. [\hyperlink{Cisatracurium Besylate}{PMID: 9719722}, J R Ortiz et al., ]

\hypertarget{pmid_8788285}{T}o establish whether cisapride is beneficial in children with intractable constipation, an open trial was performed. Chronically constipated children who had failed at least 12 weeks of medical therapy received cisapride at a dose of 0.2 mg/kg/dose TID for 12 weeks. Children with pelvic floor dyssynergia were excluded. Patients were followed prospectively for at least 12 months. Thirty children were initially enrolled, and 27 (14 boys, 13 girls) completed the study. At the end of 12 weeks of cisapride treatment, there was a significant increase in the number of bowel movements per week (1.43 +/- 0.52 to 6.48 +/- 4.16; p < 0.05) and significant decreases in the number of accidents per day (2.86 +/- 2.71 to 0.52 +/- 1.23; p < 0.05) and doses of laxatives used per week (14.33 +/- 5.84 to 3.37 +/- 7.10; p < 0.05). Encopresis disappeared in 65.2\% of cases (p < 0.0001) and improved in 26\%. Sixty-nine percent of the patients stopped using laxatives (p < 0.001). After 12 weeks 18 patients (66.6\%) were asymptomatic, seven (25.9\%) showed some improvement in bowel movement frequency and number of accidents, and two (7.4\%) showed no improvement. The cisapride was well tolerated. After long-term follow-up (20 +/- 9.8 months), 37\% of patients had recovered (asymptomatic and off laxatives and cisapride) and 29.6\% were still asymptomatic but were using laxatives or cisapride. There were no differences in baseline characteristics between recovered and nonrecovered patients. We conclude that cisapride is effective in the treatment of some children with intractable constipation without pelvic floor dyssynergia. [\hyperlink{Cisatracurium Besylate}{PMID: 8788285}, S Nurko et al., 1996]

\section*{Clonidine Hydrochloride}
\subsection*{Result}
\subsubsection*{Answer}

Neonates/Infants (<1 year): Unknown  
Children (1–12 years): Yes  
Adolescents (13–18 years): Yes  

\subsubsection*{{Explanation}}
\hypertarget{Clonidine Hydrochloride}
A review of the available abstracts reveals multiple targeted studies evaluating the safety of clonidine hydrochloride in children across various age ranges and indications. Below is a summary of the evidence, organized by age group and indication:

Neonates and Infants:
- One pilot study investigated clonidine for neonatal narcotic abstinence syndrome (NNAS) in infants. Six of seven infants responded to clonidine (3-4 mcg/kg/day) with no toxic side effects observed at the dosage used. The authors concluded clonidine "may be a safe therapeutic agent" for NNAS, but emphasized that its use remains investigational and that further studies are needed to determine relative safety and efficacy compared to other regimens [\hyperlink{pmid_6597462}{PMID: 6597462}, E L Hoder et al., 1984].
- A large retrospective study of IV clonidine infusion for analgosedation in critically ill children (age 0-18 years, median age 12.9 months) found that while bradycardia and hypotension were common, these were not clinically significant and did not require intervention. Younger age was a risk factor for bradycardia, but overall, clonidine was considered hemodynamically well tolerated in this high-risk population [\hyperlink{pmid_29912068}{PMID: 29912068}, Niina Kleiber et al., 2018].
- A review article notes that while clonidine appears safe in pediatric anesthesia, there is an urgent need for high-quality trials in children younger than 12 months and those with hemodynamic instability [\hyperlink{pmid_31045639}{PMID: 31045639}, Arash Afshari et al., 2019].

Children (Ages 1–12 years):
- Multiple randomized controlled trials and observational studies have evaluated oral and IV clonidine for premedication, sedation, ADHD, tic disorders, and pain management in children aged 1–12 years. These studies consistently report that clonidine is generally well tolerated, with the most common side effects being mild-to-moderate somnolence and occasional bradycardia, which rarely required intervention [\hyperlink{pmid_21241954}{PMID: 21241954}, Rakesh Jain et al., 2011; \hyperlink{pmid_18182964}{PMID: 18182964}, W Burleson Daviss et al., 2008; \hyperlink{pmid_26083572}{PMID: 26083572}, Peter G Larsson et al., 2015; \hyperlink{pmid_37203849}{PMID: 37203849}, K-D Zeng et al., 2023; \hyperlink{pmid_11448249}{PMID: 11448249}, V Agarwal et al., 2001; \hyperlink{pmid_8836273}{PMID: 8836273}, K Nishina et al., 1996; \hyperlink{pmid_8561317}{PMID: 8561317}, K Mikawa et al., 1996; \hyperlink{pmid_12673018}{PMID: 12673018}, Kenji Sumiya et al., 2003].
- A large study of 1,507 children receiving oral or IV clonidine as premedication found a low incidence of clinically significant bradycardia, suggesting that fear of bradycardia should not preclude its use [\hyperlink{pmid_26083572}{PMID: 26083572}, Peter G Larsson et al., 2015].
- In ADHD and tic disorder studies, clonidine was found to have a favorable safety profile compared to other medications, with adverse events being mostly mild and transient [\hyperlink{pmid_21241954}{PMID: 21241954}, Rakesh Jain et al., 2011; \hyperlink{pmid_37203849}{PMID: 37203849}, K-D Zeng et al., 2023].
- Several studies on compounded and extemporaneous pediatric formulations confirm the chemical and microbiological stability of clonidine preparations, using excipients considered safe for all pediatric age groups [\hyperlink{pmid_30124097}{PMID: 30124097}, V Merino-Bohórquez et al., 2019; \hyperlink{pmid_22580108}{PMID: 22580108}, A L de Goede et al., 2012; \hyperlink{pmid_34465373}{PMID: 34465373}, Jumpei Saito et al., 2021; \hyperlink{pmid_36186250}{PMID: 36186250}, Maya Wasilewski et al., 2022].

Adolescents (Ages 13–18 years):
- Studies including adolescents (up to 17 years) for ADHD and other indications report similar safety findings as in younger children, with mild-to-moderate adverse events and no significant cardiovascular concerns [\hyperlink{pmid_21241954}{PMID: 21241954}, Rakesh Jain et al., 2011; \hyperlink{pmid_21555501}{PMID: 21555501}, Scott H Kollins et al., 2011].

Poisoning and Overdose:
- Multiple case reports and retrospective studies document that accidental or intentional clonidine overdose in children can cause significant toxicity (lethargy, bradycardia, hypotension, respiratory depression), and rare fatalities have been reported [\hyperlink{pmid_6523727}{PMID: 6523727}, J T Algren et al., 1984; \hyperlink{pmid_11929375}{PMID: 11929375}, Wendy Klein-Schwartz et al., 2002; \hyperlink{pmid_6849304}{PMID: 6849304}, M Artman et al., 1983; \hyperlink{pmid_33036823}{PMID: 33036823}, Michael S Toce et al., 2021; \hyperlink{pmid_19655285}{PMID: 19655285}, Mariya Farooqi et al., 2009; \hyperlink{pmid_11483818}{PMID: 11483818}, M J Romano et al., 2001]. However, these events are related to overdose or medication errors, not to therapeutic use in controlled studies.

Summary by Age Range:
- Neonates/Infants (<1 year): Limited targeted studies suggest possible safety in specific settings (e.g., NNAS, PICU sedation), but more high-quality trials are needed, especially in those <12 months.
- Children (1–12 years): Multiple targeted studies affirm safety for various indications (ADHD, premedication, sedation, tic disorder, pain), with monitoring for bradycardia and somnolence recommended.
- Adolescents (13–18 years): Targeted studies affirm safety for ADHD and other indications, with similar side effect profiles as in younger children.

In conclusion, based on the abstracts, clonidine hydrochloride is affirmed as safe for use in children (1–18 years) for several indications when used at therapeutic doses and with appropriate monitoring. Safety in neonates and infants (<1 year) is less well established and remains an area for further research.

\subsection*{Abstracts}
\hypertarget{pmid_30124097}{P}ediatric patients present changing physiological features. Because of the lack of land suitable for commercial management, pediatric specialties very often need to prepare extemporaneous formulations to improve the dosage and administration of drugs for children. Oral liquid formulations are the most suitable for pediatric patients. Clonidine is widely used in the pediatric population for opioid withdrawal, hypertensive crisis, attention deficit disorders and hyperactivity syndrome, and as an analgesic in neuropathic cancer pain. The objective was to study the physicochemical and microbiological stability and determine the shelf life of an oral solution containing 20 µg/mL clonidine hydrochloride in different storage conditions (5 ± 3 °C, 25 ± 3 °C, and 40 ± 2 °C). Using raw material with excipients safe for all pediatric age groups, two oral liquid formulations of clonidine hydrochloride were designed (with and without preservatives). Solutions stored at 5 ± 3 °C (with and without preservatives) were physically and microbiologically stable for at least 90 days in closed containers and for 42 days after opening. Two oral solutions of clonidine hydrochloride 20 µg/mL were developed for pediatric use from raw materials that are readily available and easy to process, containing safe excipients that are stable over a long period of time. [\hyperlink{Clonidine Hydrochloride}{PMID: 30124097}, V Merino-Bohórquez et al., 2019]

\hypertarget{pmid_37203849}{T}he current research was designed to assess the efficacy of clonidine in the treatment of children with tic disorder co-morbid with attention deficit hyperactivity disorder. A total of 154 children with tic disorder co-morbid with attention deficit hyperactivity disorder admitted to our hospital from July 2019 to July 2022 were recruited and assigned to receive either methylphenidate hydrochloride plus haloperidol (observation group) or clonidine (experimental group), with 77 cases in each group. Outcome measures included clinical efficacy, Yale Global Tic Severity Scale (YGTSS) scores, Conners Parent Symptom Questionnaire (PSQ) scores, and adverse events. Clonidine was associated with markedly higher clinical efficacy vs. methylphenidate hydrochloride plus haloperidol (p<0.05). Clonidine offered more significant mitigation of the tic disorder vs. methylphenidate hydrochloride plus haloperidol, as evinced by the lower kinetic tic scores, vocal tic scores, and total scores (p<0.05). Children exhibited markedly milder tic symptoms after clonidine monotherapy vs. those with dual therapy of methylphenidate hydrochloride and haloperidol, suggested by the lower scores of character problems, learning problems, psychosomatic disorders, hyperactivity/impulsivity, anxiety index, and hyperactivity index (p<0.05). Clonidine features a higher safety profile than methylphenidate hydrochloride plus haloperidol by reducing the incidence of adverse events (p<0.05). Clonidine effectively alleviates tic symptoms, reduces attention deficit and hyperactivity/impulsivity in children with tic disorder co-morbid attention deficit hyperactivity disorder, and features a high safety profile. [\hyperlink{Clonidine Hydrochloride}{PMID: 37203849}, K-D Zeng et al., 2023]

\hypertarget{pmid_6523727}{C}lonidine hydrochloride (CH) is an antihypertensive drug with complex pharmacologic activity including central and peripheral alpha-adrenergic stimulation and CNS depression. We reviewed the records of 5 children admitted to our Pediatric Intensive Care Unit following accidental ingestion of CH. All patients presented with lethargy or stupor, beginning 20-60 minutes after ingestion. Respiratory depression or apnea occurred in 4, requiring endotracheal intubation in 2 and mechanical ventilation in 1. All 5 developed mild to moderate hypertension, and 3 developed asymptomatic bradycardia. The dose of CH ingested was estimated to be 0.2-0.4 mg in 4 out of 5 patients. Treatment consisted of efforts to prevent absorption of CH from the GI tract and supportive care. All signs of CH toxicity resolved within 6-14 hours. Four patients were transferred from ICU within 24 hours and discharged home the following day. One patient developed post-extubation stridor and atelectasis. Significant toxicity occurred even though the amount of CH ingested was relatively small in at least 4 or 5 patients. Transient hypertension occurred early in the hospital course of all patients and resolved without treatment. Hypotension and symptomatic bradycardia were not observed. Apnea was the most serious abnormality observed. All patients recovered without significant morbidity. [\hyperlink{Clonidine Hydrochloride}{PMID: 6523727}, J T Algren et al., 1984]

\hypertarget{pmid_21241954}{T}his study examined the efficacy and safety of clonidine hydrochloride extended-release tablets (CLON-XR) in children and adolescents with attention-deficit/hyperactivity disorder (ADHD). This 8-week, placebo-controlled, fixed-dose trial, including 3 weeks of dose escalation, of patients 6 to 17 years old with ADHD evaluated the efficacy and safety of CLON-XR 0.2 mg/day or CLON-XR 0.4 mg/day versus placebo in three separate treatment arms. Primary endpoint was mean change in ADHD Rating Scale-IV (ADHD-RS-IV) total score from baseline to week 5 versus placebo using a last observation carried forward method. Secondary endpoints were improvement in ADHD-RS-IV inattention and hyperactivity/impulsivity subscales, Conners Parent Rating Scale-Revised: Long Form, Clinical Global Impression of Severity, Clinical Global Impression of Improvement, and Parent Global Assessment from baseline to week 5. Patients (N = 236) were randomized to receive placebo (n = 78), CLON-XR 0.2 mg/day (n = 78), or CLON-XR 0.4 mg/day (n = 80). Improvement from baseline in ADHD-RS-IV total score was significantly greater in both CLON-XR groups versus placebo at week 5. A significant improvement in ADHD-RS-IV total score occurred between groups as soon as week 2 and was maintained throughout the treatment period. In addition, improvement in ADHD-RS-IV inattention and hyperactivity/impulsivity subscales, Conners Parent Rating Scale-Revised: Long Form, Clinical Global Impression of Improvement, Clinical Global Impression of Severity, and Parent Global Assessment, occurred in both treatment groups versus placebo. The most common treatment-emergent adverse event was mild-to-moderate somnolence. Changes on electrocardiogram were minor and reflected the known pharmacology of clonidine. Clonidine hydrochloride extended-release tablets were generally well tolerated by patients in the study and significantly improved ADHD symptoms in this pediatric population. [\hyperlink{Clonidine Hydrochloride}{PMID: 21241954}, Rakesh Jain et al., 2011]

\hypertarget{pmid_12166288}{W}e report a case of effective treatment with clonidine ointment for herpetic neuralgia in a child. Clonidine hydrochloride is an alpha 2-agonist. It is generally administered intravenously, intramuscularly, intrathecally and orally. However, there have been only a few reports on transdermal usage. In our department, we have investigated the analgesic effect of topical application of clonidine in adults, and we have obtained sufficient evidence on the effects of clonidine. Therefore, we decided to use clonidine to a child. A 9-year-old child who had undergone BMT and developed herpes zoster was experiencing severe pain, itch, and insomnia. Many drugs were ineffective in relieving the pain, itch, and insomnia. To remove the symptoms, we tried clonidine ointment. Immediate improvement was observed in all the symptoms. Therefore, clonidine ointment was thought to be effective and we decided to prescribe clonidine ointment and amitriptyline hydrochloride. The application of clonidine showed no side effects such as bradycardia and low blood pressure. We conclude that clonidine can be administered to children without causing side effects. [\hyperlink{Clonidine Hydrochloride}{PMID: 12166288}, Reiko Hagihara et al., 2002]

\hypertarget{pmid_11929375}{T}o analyze the trends, demographics, and toxic effects associated with pediatric clonidine hydrochloride exposures reported to poison centers. Retrospective. Clonidine-only exposures followed up to known outcome in children younger than 19 years reported to the American Association of Poison Control Center's database from January 1, 1993, through December 31, 1999. Frequency of exposures over time, acuity, reason, symptoms, management site, treatment, and outcome. There were 10 060 reported exposures with 57\% reported for children younger than 6 years, 34\% for children between 6 and 12 years old, and 9\% for adolescents between 13 and 18 years old. In 1999 there were 2.5 times as many exposures as in 1993. In 6- through 12-year-olds, clonidine was the child's medication in 35\% of the exposures, compared with 10\% in children younger than 6 years and 26\% in adolescents. The proportion of cases involving the child's medication increased over 7 years. While unintentional overdose was most common in children younger than 6 years, therapeutic errors and suicide attempts predominated in 6- through 12-year-olds and adolescents, respectively. In 6042 symptomatic children (60\%), the most common symptoms were lethargy (80\%), bradycardia (17\%), hypotension (15\%), and respiratory depression (5\%). Most exposures resulted in no effect (40\%) or minor effects (39\%). Moderate effects occurred in 1907 children (19\%), major effects in 230 children (2\%); there was 1 fatality in a 23-month-old. While most of the clonidine exposures resulted in minimal toxic effects, serious toxic effects and death can occur. The trend toward increasing the number of exposures in children, especially with evidence of toxic effects in children receiving clonidine therapeutically, is cause for concern. [\hyperlink{Clonidine Hydrochloride}{PMID: 11929375}, Wendy Klein-Schwartz et al., 2002]

\hypertarget{pmid_6597462}{C}lonidine hydrochloride, an alpha 2-adrenergic agonists, was used to treat seven infants who were passively addicted to narcotics because of maternal methadone maintenance. In six of seven infants, the major symptoms of narcotic withdrawal were ameliorated after a total daily oral dose of 3-4 micrograms/kg/day was achieved. One infant failed to respond. No toxic side effects of clonidine were observed at the dosage level used. The results of this pilot study suggest that clonidine may be a safe therapeutic agent for the treatment of neonatal narcotic abstinence syndrome (NNAS). Clonidine treatment of NNAS remains strictly investigational at this time. The relative efficacy and safety of clonidine versus other currently used drug regimens for NNAS also remains to be determined. [\hyperlink{Clonidine Hydrochloride}{PMID: 6597462}, E L Hoder et al., 1984]

\hypertarget{pmid_36186250}{C}lonidine hydrochloride is an antihypertensive, centrally acting α2 adrenergic agonist with various pediatric indications. For pediatric patients, 20-mcg clonidine hydrochloride capsules can be compounded from commercial tablets or from a pre-compounded titrated powder. These methods should be compared to ensure the best quality for the high-risk patients, and a beyond-use date should be established. Eight experimental batches were made from commercial tablets and 8 were made from microcrystalline cellulose (MCC)-based titrated powders. Quality controls were performed to determine the best compounding protocol. Stability study was conducted on capsules compounded with the best method. Of 8 batches manufactured from commercial tablets, 7 were compliant for both clonidine mean content and content uniformity, whereas 7 of 8 batches manufactured from titrated powders were not. A clonidine loss during compounding was evidenced by surface sampling analyses. Clonidine hydrochloride 20-mcg capsules' mean content remained higher than 90\% of initial content for 1 year when stored at 25°C with 60\% relative humidity and protected from light. Commercial tablets should be preferred to 1\% clonidine hydrochloride and MCC titrated powder made from the active pharmaceutical ingredient. Twenty-microgram clonidine hydrochloride capsules made from commercial tablets are stable for 1 year when stored under managed ambient storage condition. [\hyperlink{Clonidine Hydrochloride}{PMID: 36186250}, Maya Wasilewski et al., 2022]

\hypertarget{pmid_26083572}{C}lonidine has been advocated as a valid alternative for premedication in children but one of the few limitations is its association with reduced heart rate (HR), which thus raises the question of the safety of clonidine as premedication in children. The aim of this study was to investigate the incidence of bradycardia in children premedicated with oral or intravenous clonidine as compared to children not receiving pharmacologic premedication. An open, nonrandomized, observational study design was used. During the preoperative assessment visit the children were prescribed no premedication, intravenous or oral clonidine. On arrival to the operating room (OR) HR was recorded by connecting the patient to standard monitoring with pulseoximetry and/or Electrocardiogram. The primary outcome measure was the number of patients with a HR below 85\% of the lower limit of the normal range (1st centile), which was defined as bradycardia that might need clinical intervention. One thousand five hundred and seven patients were included in the analysis. 600 and 85 patients did not receive any premedication (Group 0), 305 patients received iv Clonidine (Group CIV), and 517 patients were given oral Clonidine (Group CPO). One patient in Group 0 (0.15\%; 95\% CI: 0-0.81\%), none in Group CIV (0\%; 95\% CI: 0.00-0.98\%), and 5 patients in Group CPO (0.97\%; 95\% CI: 0.31-2.24\%) were observed to have a HR of <85\% of the 1st centile. The incidence of bradycardia following oral or intravenous premedication with clonidine in a pediatric population scheduled for anesthesia is low. Thus, it does not appear rational to refrain from using clonidine as premedication in children only due to fear of bradycardia. [\hyperlink{Clonidine Hydrochloride}{PMID: 26083572}, Peter G Larsson et al., 2015]

\hypertarget{pmid_29912068}{C}lonidine is an antihypertensive drug used for analgosedation in the PICU. Lack of reliable data on its hemodynamic tolerance limits its use. This study explores the hemodynamic tolerance of IV clonidine infusion in a broad population of children with high severity of disease. Retrospective analysis of prospectively collected data. A tertiary and quaternary referral PICU. Critically ill children age 0-18 years old who received an IV clonidine infusion for analgosedation of at least 1 hour. None. The primary endpoints were the prevalences of bradycardia and hypotension. Secondary endpoints were changes in heart rate, blood pressure, Vasoactive-Inotropic Score, COMFORT Behavior score (a sedation scoring scale), and body temperature during the infusion. The association of bradycardia with other hemodynamic variables was explored, as well as potential risk factors for severe bradycardia. One-hundred eighty-six children (median age, 12.9 mo [interquartile range, 3.5-60.6 mo]) receiving a maximum median clonidine infusion of 0.7 µg/kg/hr (interquartile range, 0.3-1.5) were included. Severe bradycardia and systolic hypotension occurred in 72 patients (40.2\%) and 105 patients (58\%), respectively. Clonidine-associated bradycardia was hemodynamically well tolerated, as it was not related with hypotension and the need for vasoactive drugs decreased in parallel with a sedation score guided clonidine infusion rate increase. Younger age was the only identified risk factor for clonidine-associated bradycardia. Although administration of clonidine is often associated with bradycardia and hypotension, these complications do not seem clinically significant in a mixed PICU population with a high degree of disease severity. Clonidine may have a vasoactive-inotropic sparing effect. [\hyperlink{Clonidine Hydrochloride}{PMID: 29912068}, Niina Kleiber et al., 2018]

\hypertarget{pmid_8836273}{O}ral clonidine given as a premedicant in adults has been shown to reduce intraoperative inhalation anaesthetic requirements and provide perioperative haemodynamic stability. We conducted the current study to ascertain whether or not these beneficial effects of clonidine can be reproduced in children. In a prospective, randomized, double-blind, controlled clinical trial, 60 children (ASA I) aged 5-11 yr, received placebo (control), 2 micrograms kg-1 clonidine, or 4 micrograms kg-1 clonidine orally 105 min before induction of anaesthesia. Anaesthesia was induced with halothane, nitrous oxide in oxygen via mask and maintained with halothane and 60\% nitrous oxide in oxygen. The halothane concentration was titrated to the concentration required to maintain haemodynamic stability (defined as 20\% of blood pressure (BP) and heart rate (HR)) for maintenance of anaesthesia. The end-tidal concentration of halothane was monitored throughout anaesthesia. On completion of surgery, nitrous oxide and halothane were discontinued. Following confirmation of recovery from anaesthesia and muscle relaxation, the endotracheal tube was removed. Higher inspired concentrations of halothane (\%) were required in the control and 2 micrograms kg-1 clonidine-treated groups (mean SD: 1.1 +/- 0.2 and 1.0 +/- 0.2, respectively) than in the 4 micrograms kg-1 clonidine-treated group (0.6 +/- 0.1) for haemodynamic stability (P < 0.05). Clonidine, 4 micrograms kg-1, significantly reduced the intraoperative lability (coefficient of variation) of systolic and diastolic BP and HR compared with the other two regimens. Oral clonidine premedication at a dose of 4 micrograms kg-1 provided intraoperative haemodynamic stability and reduced anaesthetic requirements in children. However, we are unable to extrapolate these observations to younger children and infants. [\hyperlink{Clonidine Hydrochloride}{PMID: 8836273}, K Nishina et al., 1996]

\hypertarget{pmid_11448249}{A} 12-week, double-blind, randomized, placebo-controlled trial of oral clonidine in three fixed doses (4, 6, and 8 mcg/kg/day) using a crossover design was conducted with 10 children who had hyperkinetic disorder (mean age 7.6 years +/-.54). All had comorbid mental retardation. Both parents' ratings on the Parent Symptom Questionnaire and clinicians' ratings on the Hillside Behaviour Rating Scale showed a marked dose-related response to clonidine in hyperactivity, impulsivity, and inattention. Drowsiness was a common side effect of clonidine. It wore off by the 2nd to 4th week in most cases. Thus, clonidine is a safe and effective medication in young hyperkinetic children with comorbid mental retardation. [\hyperlink{Clonidine Hydrochloride}{PMID: 11448249}, V Agarwal et al., 2001]

\hypertarget{pmid_34465373}{C}lonidine hydrochloride is used to treat sedative agent withdrawals, malignant hypertension, and anesthesia complications. Clonidine is also prescribed off-label to pediatric patients at a dose of 1 μg/kg. The commercially available enteral form of clonidine, Catapres® tablets, is often compounded into a powder form by pharmacists to achieve dosage adjustments for administration to pediatric patients. However, the stability and quality of compounded clonidine powder have not been verified. The objectives of this study were to formulate a 0.2 mg/g oral clonidine hydrochloride powder and assess the stability and physical properties of this compounded product in storage. A 0.2 mg/g clonidine powder was prepared by adding lactose monohydrate to crushed and filtrated clonidine tablets. The powder was stored in polycarbonate amber bottles or coated paper packages laminated with cellophane and polyethylene. The stability of clonidine at 25 °C ± 2 °C and 60\% ± 5\% relative humidity was examined over a 120-d period in "bottle (closed)," "bottle (in use)," and "laminated paper" storage conditions. Drug dissolution and powder X-ray diffraction analysis were conducted to assess physicochemical stabilities. Validated liquid chromatography-diode array detection was used to detect and quantify clonidine and its degradation product, 2,6-dichloroaniline (2,6-DCA). Clonidine content was maintained between 90.0 and 110.0\% of the initial contents in all packaging and storage conditions. After 120 d of storage, 2,6-DCA was not detected, and no crystallographic and dissolution changes were observed. Compounded clonidine powder stability was maintained for 120 d at 25 °C ± 2 °C and 60\% ± 5\% relative humidity. This information may contribute to the management of clonidine compounded powder in community and hospital pharmacies in Japan. [\hyperlink{Clonidine Hydrochloride}{PMID: 34465373}, Jumpei Saito et al., 2021]

\hypertarget{pmid_22580108}{M}any drugs are unavailable in suitable paediatric dosage forms. We describe the development and validation of a stable paediatric oral formulation of clonidine hydrochloride 50 μg/ml, allowing individualised paediatric dosing and easy administration. Stability of the extemporaneously compounded formulation of clonidine hydrochloride was assessed using a validated HPLC method. Clonidine hydrochloride was stable in the buffered aqueous solution at room temperature for up to 9 months. The described formulation is chemically stable for at least 9 months when stored in brown 100 ml PET bottles at room temperature, enabling adequate oral treatment in paediatric patients. [\hyperlink{Clonidine Hydrochloride}{PMID: 22580108}, A L de Goede et al., 2012]

\hypertarget{pmid_12673018}{C}lonidine hydrochloride has been used for pre-anesthetic medication to provide a pre-operative sedation in pediatric surgery. The purpose of this study is to determine the plasma clonidine concentration, which gives satisfactory sedation in pediatric surgery. Sixteen pediatric patients (age: 1-11 years, weight: 9-33 kg) received either 2 or 4 microg/kg of clonidine lollipop before entering the operating room. Plasma clonidine concentrations were determined 120 min after administration of clonidine lollipop. Pre-operative sedation was evaluated by 5-point scoring systems at entering the operating room. The changes in systolic blood pressure (SBP), diastolic blood pressure (DBP) and heart rate (HR) were also assessed before and after administration of clonidine lollipop. The patients with satisfactory sedation had higher plasma clonidine concentration than that of the patients with unsatisfactory sedation (0.45+/-0.16 ng/ml vs. 0.26+/-0.16 ng/ml, p<0.05). The clonidine concentrations in the satisfactory group ranged from 0.28 to 0.81 ng/ml. There was no significant difference in hemodynamic parameters (SBP, DBP and HR) before and after administration of clonidine lollipop in both satisfactory and unsatisfactory sedation groups. Plasma clonidine concentration of 0.3-0.8 ng/ml would be sufficient to produce satisfactory sedation without changes in hemodynamic parameters in pediatric surgery. [\hyperlink{Clonidine Hydrochloride}{PMID: 12673018}, Kenji Sumiya et al., 2003]

\hypertarget{pmid_18182964}{T}o examine the safety and tolerability of clonidine used alone or with methylphenidate in children with attention-deficit/hyperactivity disorder (ADHD). In a 16-week multicenter, double-blind trial, 122 children with ADHD were randomly assigned to clonidine (n = 31), methylphenidate (n = 29), clonidine and methylphenidate (n = 32), or placebo (n = 30). Doses were flexibly titrated up to 0.6 mg/day for clonidine and 60 mg/day for methylphenidate (both with divided dosing). Groups were compared regarding adverse events and changes from baseline to week 16 in electrocardiograms and vital signs. There were more incidents of bradycardia in subjects treated with clonidine compared with those not treated with clonidine (17.5\% versus 3.4\%; p =.02), but no other significant group differences regarding electrocardiogram and other cardiovascular outcomes. There were no suggestions of interactions between clonidine and methylphenidate regarding cardiovascular outcomes. Moderate or severe adverse events were more common in subjects on clonidine (79.4\% versus 49.2\%; p =.0006) but not associated with higher rates of early study withdrawal. Drowsiness was common on clonidine, but generally resolved by 6 to 8 weeks. Clonidine, used alone or with methylphenidate, appears safe and well tolerated in childhood ADHD. Physicians prescribing clonidine should monitor for bradycardia and advise patients about the high likelihood of initial drowsiness. [\hyperlink{Clonidine Hydrochloride}{PMID: 18182964}, W Burleson Daviss et al., 2008]

\hypertarget{pmid_6849304}{C}lonidine hydrochloride poisoning in children has become more frequent with increasing availability of this drug. We report four cases of accidental clonidine poisoning that demonstrate the various signs and symptoms of clonidine poisoning. The most frequent and significant toxic effects are depression of consciousness, bradycardia, hypotension, and respiratory depression. Ventilatory support must be available if apnea occurs. Bradycardia can be treated with atropine sulfate, epinephrine chloride, dopamine hydrochloride, or tolazoline hydrochloride. Hypotension is treated with intravenous fluids and dopamine, reserving tolazoline for refractory cases. Hypothermia is common but is of minor clinical significance. Paradoxical hypertension should be treated with tolazoline. Clonidine may not be detected in body fluids by routine toxicology-screening procedures, so poisoning should be suspected on clinical grounds. [\hyperlink{Clonidine Hydrochloride}{PMID: 6849304}, M Artman et al., 1983]

\hypertarget{pmid_16301230}{C}lonidine is effective in treating sevoflurane-induced postanesthesia agitation in children. We conducted a study on 169 children to quantify the risk reduction of clonidine agitation in patients admitted to our day-surgery pediatric clinic. Children were randomly allocated to receive clonidine 2 mug/kg or placebo before general anesthesia with sevoflurane that was also supplemented with a regional or central block. An observer blinded to the anesthetic technique assessed recovery variables and the presence of agitation. Pain and discomfort scores were significantly decreased in the clonidine group; the incidence of agitation was reduced by 57\% (P = 0.029) and the incidence of severe agitation by 67\% (P = 0.064). Relative risks for developing agitation and severe agitation were 0.43 (95\% confidence interval, 0.24-0.78) and 0.32 (0.09-1.17), respectively. Clonidine produces a substantial reduction in the risk of postsevoflurane agitation in children. [\hyperlink{Clonidine Hydrochloride}{PMID: 16301230}, Simonetta Tesoro et al., 2005]

\hypertarget{pmid_19655285}{C}lonidine is frequently prescribed to children. Clonidine overdose in children has resulted in major clinical effects and deaths. A 3.5-year-old male with a history of a seizure disorder and night terrors presented following difficulty walking, excessive sleeping, agitation when awake, and possible seizure activity. Chronic medications were valproic acid (VPA) and clonidine. On presentation, he alternated between poor responsiveness and agitation, with initial vitals: blood pressure, BP 144/76 mmHg; heart rate, 65 bpm; respiratory rate, 18 bpm; temperature 99.5 degrees F; and pulse oximetry 96\% on room air. VPA level was 35 microg/mL. A toxicology consult the next day noted a dry mouth, 2-mm pupils, intermittent gasping, and central nervous system (CNS) depression, with a diagnostic impression of clonidine overdose. The caregiver had been giving 1 mL (0.1 mg) qd of a pharmacy-compounded clonidine suspension by a provided syringe. The pharmacy procedure record agreed with the physicians order. The amount dispensed was a 30-day supply but the bottle was empty on day 19, leading us to suspect a possible accelerated dosing error. The concentration in the bottle thus could not be confirmed. The child slowly returned to his baseline state over 48 hours. A serum clonidine level drawn approximately 18 hours after his last dose later returned at 300 ng/mL (reference range = 0.5-4.5 ng/mL). Compounding and liquid dosing errors are common in children and may result in massive overdoses. There was an accelerated dosing error, but whether a compounding or suspension error or even an acute overdose occurred as well is unknown. Particular care should be taken with medications that have low therapeutic indices, that are extemporaneously compounded, and are prepared as liquids, where medication errors are more likely. [\hyperlink{Clonidine Hydrochloride}{PMID: 19655285}, Mariya Farooqi et al., 2009]

\hypertarget{pmid_3053867}{W}e evaluated the tolerance and effectiveness of the oral clonidine test for GH in 75 children, 84\% with hyposomia and 16\% with other diseases. The test was well tolerated, since 97\% of the examined children had no side effects with the exception of occasional drowsiness, pallor and myosis of short duration. Two of the children at the end of the test, had more severe symptoms 30 min after (deep asthenia, pallor and a further small blood pressure drop) which however, resolved after 4-6 h. No correlation was observed between the clinical picture and the drops in blood pressure and/or plasma cortisol in the children examined. We confirm the effectiveness of the clonidine test in the release of GH since in our study we observed no negative false subnormal responses. [\hyperlink{Clonidine Hydrochloride}{PMID: 3053867}, R De Angelis et al., 1988]

\hypertarget{pmid_33036823}{P}ediatric clonidine ingestions frequently result in emergency department visits and admission for cardiac monitoring. Detailed information on the clinical course and specifically time of vital sign abnormalities of these patients is lacking. The objective of this study was to provide descriptive analysis of the rates and times to vital sign abnormalities, treatment, disposition, and outcomes in a single-center cohort of pediatric patients with report of clonidine poisoning. We performed a retrospective cohort study of patients younger than 21 years who presented to a large, urban, tertiary care center with a report of single substance clonidine exposure between January 2004 and November 2017. Patients were dichotomized into younger (≤9 years or younger) and older (10-21 years) groups based on the expected physiologic and psychologic differences between older and younger children. Eighty-eight patients met our inclusion criteria. Younger patients (≤9 years or younger; n = 47) were more likely to be exposed to someone else's medication (53\%) and older patients (10-21 years; n = 41) overwhelmingly (85\%) were exposed to their own medication. Thirty-nine (45\%) became bradycardic, 27 (32\%) became bradypneic, and 38 (44\%) became hypotensive. Eighty percent of patients had depressed mental status. Thirty-three (38\%) patients received at least one dose of naloxone (median 0.07 mg/kg; interquartile range 0.03-0.11 mg/kg). Of those who received naloxone, 50\% had a documented clinical response. In this study of patients at a pediatric tertiary referral center, pediatric patients with report of clonidine exposures were likely to exhibit altered mental status and frequently develop vital sign abnormalities. Naloxone exhibited some effectiveness; given its wide safety margin, high-dose naloxone should be used in critically poisoned non-opioid-dependent patients. Because adolescents are much more likely to ingest their own clonidine medication, counseling with parents and other caregivers regarding safe medication storage is paramount. [\hyperlink{Clonidine Hydrochloride}{PMID: 33036823}, Michael S Toce et al., 2021]

\hypertarget{pmid_31045639}{C}lonidine, an α2-receptor agonist is a widely used drug in pediatrics with a large scope of indications ranging from prevention of postoperative emergence agitation, analgesia, anxiolysis, sedation, weaning to shivering. In the era of 'opioid-free' medicine with much attention be directed toward increasing problems with opioid use, clonidine due to its global availability, low cost and safety profile has become an even more interesting option. Increasing evidence from randomised clinical trials support the use of clonidine in healthy children in the perioperative setting. Clonidine appears to significantly reduce postoperative emergence agitation, opioid consumption, shivering, nausea and vomiting. In addition, emerging evidence support the use of clonidine for sedation of critically ill children in ICUs. In this review, the current evidence for clonidine in pediatrics is described and analyzed including a meta-analysis for prevention of emergence agitation. Clonidine appears a safe and beneficial drug with moderate to high-quality evidence supporting its use in pediatric anesthesia. However, for some indications and populations such as children younger than 12 months old and those with hemodynamic instability, there is an urgent need for high-quality trials. [\hyperlink{Clonidine Hydrochloride}{PMID: 31045639}, Arash Afshari et al., 2019]

\hypertarget{pmid_21555501}{T}o assess the efficacy and safety of clonidine hydrochloride extended-release tablets (CLON-XR) combined with stimulants (ie, methylphenidate or amphetamine) for attention-deficit/hyperactivity disorder (ADHD). In this phase 3, double-blind, placebo-controlled trial, children and adolescents with hyperactive- or combined-subtype ADHD who had an inadequate response to their stable stimulant regimen were randomized to receive CLON-XR or placebo in combination with their baseline stimulant medication. Predefined efficacy measures evaluated change from baseline to week 5. Safety was assessed by spontaneously reported adverse events, vital signs, electrocardiogram recordings, and clinical laboratory values. Improvement from baseline for all efficacy measures was evaluated using analysis of covariance. Of 198 patients randomized, 102 received CLON-XR plus stimulant and 96 received placebo plus stimulant. At week 5, greater improvement from baseline in ADHD Rating Scale IV (ADHD-RS-IV) total score (95\% confidence interval: -7.83 to -1.13; P = .009), ADHD-RS-IV hyperactivity and inattention subscale scores (P = .014 and P = .017, respectively), Conners' Parent Rating Scale scores (P < .062), Clinical Global Impression of Severity (P = .021), Clinical Global Impression of Improvement (P = .006), and Parent Global Assessment (P = .001) was observed in the CLON-XR plus stimulant group versus the placebo plus stimulant group. Adverse events and changes in vital signs in the CLON-XR group were generally mild. The results of this study suggest that CLON-XR in combination with stimulants is useful in reducing ADHD in children and adolescents with partial response to stimulants. [\hyperlink{Clonidine Hydrochloride}{PMID: 21555501}, Scott H Kollins et al., 2011]

\hypertarget{pmid_11483818}{A} 5-year-old child who weighed 17.5 kg received 50 mg of clonidine. The amount ingested was confirmed by analysis of the suspension administered (clonidine HCl 9.78 mg/mL). To our knowledge, this represents the largest ingestion in a child and the largest ingestion on a milligram per kilogram basis in the medical literature. The child's initial presentation included hyperventilation, an unusual feature of clonidine toxicity. The child was discharged without sequela 42 hours after admission. A serum concentration of clonidine 17 hours postingestion was 64 ng/mL, the highest reported to date in a pediatric patient. The intoxication was traced to a pharmacy compounding error in which milligrams were substituted for micrograms. Increased prescribing of clonidine in young children coupled with the requirement to compound clonidine in a suspension and the narrow therapeutic index suggests that the frequency of severe ingestions in children will increase in the future. [\hyperlink{Clonidine Hydrochloride}{PMID: 11483818}, M J Romano et al., 2001]

\hypertarget{pmid_8561317}{C}lonidine is an effective preanesthetic medication in children, providing a preoperative sedative effect. The analgesic properties of the drug have been well documented in adults. The current study was designed to investigate the effect of oral clonidine given preoperatively on postoperative pain in children undergoing minor surgery. In a prospective, randomized, controlled clinical trial, 90 children aged 5-12 yr undergoing elective ophthalmic, urologic, and otologic surgery received placebo (control), clonidine 2 micrograms/kg, or clonidine 4 micrograms/kg. These drugs were administered 105 min before the estimated time of induction of anesthesia and followed by treatment with oral atropine 0.03 mg/kg 60 min before anesthesia. Anesthesia was induced and maintained with halothane and nitrous oxide in oxygen. Postoperative pain was assessed by a blinded observer using an objective pain scale (OPS). Clonidine 4 micrograms/kg provided lower OPS (highest) scores during 12 h after surgery and reduced requirement for postoperative supplementary analgesic (diclofenac suppository) compared with the other two regimens. These data suggest that oral clonidine premedication (4 micrograms/kg) is a possible approach to facilitating postoperative analgesia in children undergoing minor surgery. [\hyperlink{Clonidine Hydrochloride}{PMID: 8561317}, K Mikawa et al., 1996]

\section*{Colistimethate Sodium}
\subsection*{Result}
\subsubsection*{Answer}

Yes (Neonates, including preterm and extremely low birth weight)
Yes (Infants and children, 1 month to 18 years)
Yes (Children with cystic fibrosis aged ≥6 years, for inhaled use)

\subsubsection*{{Explanation}}
\hypertarget{Colistimethate Sodium}
A review of the available abstracts reveals multiple targeted studies evaluating the safety of Colistimethate Sodium (colistin) in children, including neonates, infants, and older pediatric patients. Below is a summary by age range:

Neonates (including preterm and extremely low birth weight):
- Several studies specifically evaluated intravenous Colistimethate Sodium in neonates with multidrug-resistant Gram-negative infections. These studies report favorable clinical outcomes, low rates of nephrotoxicity, and no neurotoxicity, concluding that Colistimethate Sodium appears safe and effective in this population [\hyperlink{pmid_21245777}{PMID: 21245777}, Mamta Jajoo et al., 2011; \hyperlink{pmid_26868136}{PMID: 26868136}, Manar Al-Lawama et al., 2016].
- Another study in critically ill neonates found that while the standard dose may be suboptimal for achieving therapeutic plasma concentrations, no significant safety concerns were reported [\hyperlink{pmid_27276179}{PMID: 27276179}, Narongsak Nakwan et al., 2016].

Infants and Children (from 1 month to 18 years):
- Multiple prospective and retrospective studies, as well as case series, have evaluated the safety of intravenous Colistimethate Sodium in children with severe infections due to multidrug-resistant Gram-negative bacteria. These studies consistently report that Colistimethate Sodium is generally well tolerated, with low rates of nephrotoxicity and no significant neurotoxicity [\hyperlink{pmid_25691180}{PMID: 25691180}, Poddutoor Preetham Kumar et al., 2015; \hyperlink{pmid_20003141}{PMID: 20003141}, Solmaz Celebi et al., 2010; \hyperlink{pmid_20119725}{PMID: 20119725}, Elias Iosifidis et al., 2010; \hyperlink{pmid_27994915}{PMID: 27994915}, Ayşe Karaaslan et al., 2016; \hyperlink{pmid_19116601}{PMID: 19116601}, Matthew E Falagas et al., 2009].
- A randomized trial in children with ventilator-associated pneumonia or central line-associated bloodstream infection found no significant increase in nephrotoxicity with a loading dose regimen [\hyperlink{pmid_31956632}{PMID: 31956632}, Shiva Fatehi et al.].
- Inhaled Colistimethate Sodium was also reported as safe in a small case series of critically ill children [\hyperlink{pmid_20658485}{PMID: 20658485}, Matthew E Falagas et al., 2010].
- In children with cystic fibrosis aged ≥6 years, a phase III randomized trial of inhaled Colistimethate Sodium found it to be well tolerated, with adverse event rates similar to comparator therapy [\hyperlink{pmid_23135343}{PMID: 23135343}, Antje Schuster et al., 2013].

Pharmacokinetic studies:
- Several pharmacokinetic studies in children and neonates highlight variability in drug exposure and suggest that standard dosing may be suboptimal for achieving therapeutic concentrations, but these studies do not report significant safety concerns [\hyperlink{pmid_33782000}{PMID: 33782000}, Charalampos Antachopoulos et al., 2021; \hyperlink{pmid_30722017}{PMID: 30722017}, Mong How Ooi et al., 2019; \hyperlink{pmid_25874300}{PMID: 25874300}, V I Zakharevich et al., 2015; \hyperlink{pmid_32179149}{PMID: 32179149}, Noppadol Wacharachaisurapol et al., 2020].

Summary:
Based on the abstracts, targeted studies in neonates (including preterm and extremely low birth weight), infants, and children up to 18 years of age affirm that Colistimethate Sodium is generally safe for use in these populations when used for severe infections due to multidrug-resistant Gram-negative bacteria. The most commonly reported adverse effect is nephrotoxicity, which appears to be infrequent and often associated with other risk factors or concomitant nephrotoxic drugs. No studies in the abstracts report that Colistimethate Sodium is unsafe in children. However, optimal dosing remains an area of ongoing research.

\subsection*{Abstracts}
\hypertarget{pmid_25691180}{T}o observe the safety and efficacy of Colistimethate sodium in children infected with gram-negative bacteria, susceptible only to colistimethate sodium. This prospective observational study done over 2 years observed children who received colistin for >48 h, for renal failure as defined by p-RIFLE criteria. Out of 68 children, 52 (76.5\%) survived. There were three children with evidence of acute kidney injury and none had neurotoxicity. Serum creatinine significantly decreased at 48 h and at end of treatment, from that at beginning of therapy (P=0.007). Colistimethate sodium is effective against carbapenem-resistant Gram-negative bacteria, and is safe in children. [\hyperlink{Colistimethate Sodium}{PMID: 25691180}, Poddutoor Preetham Kumar et al., 2015]

\hypertarget{pmid_20003141}{T}he aim of the present study was to assess the efficacy and safety of colistimethate sodium therapy in multidrug-resistant nosocomial infections caused by Pseudomonas aeruginosa or Acinetobacter baumannii in neonates and children. Pediatric patients hospitalized at the Uludag University Hospital who had nosocomial infections caused by multidrug-resistant P. aeruginosa or A. baumannii, were enrolled in the study. Colistimethate sodium at a dosage of 50-75 x 10(3) U/kg per day was given i.v. divided into three doses. Fifteen patients received 17 courses of colistimethate sodium for the following infections: ventilator-associated pneumonia (n= 14), catheter-related sepsis (n= 1) and skin and soft-tissue infection (n= 2). The mean age of patients was 53.2 + 74.7 months (range, 8 days-15 years) and 60\% were male. Mortality was 26.6\%. Colistimethate sodium appears to be safe and effective for the treatment of severe infections caused by multidrug-resistant P. aeruginosa or A. baumannii in pediatric patients. [\hyperlink{Colistimethate Sodium}{PMID: 20003141}, Solmaz Celebi et al., 2010]

\hypertarget{pmid_20585114}{U}sing a liquid chromatography-tandem mass spectrometry method, the serum and cerebrospinal fluid (CSF) concentrations of colistin were determined in patients aged 1 months to 14 years receiving intravenous colistimethate sodium (60,000 to 225,000 IU/kg of body weight/day). Only in one of five courses studied (a 14-year-old receiving 225,000 IU/kg/day) did serum concentrations exceed the 2 microg/ml CLSI/EUCAST breakpoint defining susceptibility to colistin for Pseudomonas and Acinetobacter. CSF colistin concentrations were <0.2 microg/ml but increased in the presence of meningitis (approximately 0.5 microg/ml or 34 to 67\% of serum levels). [\hyperlink{Colistimethate Sodium}{PMID: 20585114}, Charalampos Antachopoulos et al., 2010]

\hypertarget{pmid_23135343}{T}o assess efficacy and safety of a new dry powder formulation of inhaled colistimethate sodium in patients with cystic fibrosis (CF) aged ≥6 years with chronic Pseudomonas aeruginosa lung infection. A prospective, centrally randomised, phase III, open-label study in patients with stable CF aged ≥6 years with chronic P aeruginosa lung infection. Patients were randomised to Colobreathe dry powder for inhalation (CDPI, one capsule containing colistimethate sodium 1 662 500 IU, twice daily) or three 28-day cycles with twice-daily 300 mg/5 ml tobramycin inhaler solution (TIS). Study duration was 24 weeks. 380 patients were randomised. After logarithmic transformation of data due to a non-normal distribution, adjusted mean difference between treatment groups (CDPI vs TIS) in change in forced expiratory volume in 1 s (FEV1\% predicted) at week 24 was -0.98\% (95\% CI -2.74\% to 0.86\%) in the intention-to-treat population (n=373) and -0.56\% (95\% CI -2.71\% to 1.70\%) in the per protocol population (n=261). The proportion of colistin-resistant isolates in both groups was ≤1.1\%. The number of adverse events was similar in both groups. Significantly more patients receiving CDPI rated their device as 'very easy or easy to use' (90.7\% vs 53.9\% respectively; p<0.001). CDPI demonstrated efficacy by virtue of non-inferiority to TIS in lung function after 24 weeks of treatment. There was no emergence of resistance of P aeruginosa to colistin. Overall, CDPI was well tolerated. TRIAL REG NO: EudraCT 2004-003675-36. [\hyperlink{Colistimethate Sodium}{PMID: 23135343}, Antje Schuster et al., 2013]

\hypertarget{pmid_21245777}{N}osocomial infection due to multidrug-resistant Gram-negative pathogens in intensive care units is a challenge for clinicians and microbiologists, and has led to resurgence of parenteral colistin use in the last decade. Safety and efficacy data regarding intravenous colistin (colistimethate) use in neonates is sparse. We present our experience of efficacy and safety of colistimethate in the treatment of sepsis in critically sick term and preterm neonates. The records of the neonates who received colistimethate in a neonatal intensive care unit of a tertiary care center from January 2009 to December 2009 were reviewed. Eighteen critically sick neonates (10 term and 8 preterm) received 21 courses of colistimethate (dose ranging from 50,000 to 75,000 IU/kg/d) for treatment of pneumonia, blood stream infections, meningitis, and empyema thoracis. The isolated pathogens in decreasing order of frequency were Acinetobacter baumannii, Klebsiella pneumoniae, Pseudomonos aeruginosa, and Enterobacter. Mean duration of colistimethate was 13.1 days/course (range: 5-21 days). At least one other antibiotic was coadministered in all courses. A favorable clinical outcome occurred in 16 of 21 (76\%) courses, 5 patients died due to severe sepsis with multiple organ dysfunction. Microbiologic clearance was documented in 17 courses. Increase in serum creatinine by > 0.5 mg/dL above baseline in 2 babies was associated with the presence of multiple organ dysfunction syndrome in both and coadministration of netilmicin in one. Colistimethate intravenous administration appears to be safe and efficacious for multidrug-resistant Gram-negative infections in neonates, including preterm and extremely low birth weight neonates. [\hyperlink{Colistimethate Sodium}{PMID: 21245777}, Mamta Jajoo et al., 2011]

\hypertarget{pmid_28741653}{C}hloral hydrate is commonly used to sedate children for painless procedures. Children may recover more quickly after sedation with dexmedetomidine, which has a shorter half-life. We randomly allocated 196 children to chloral hydrate syrup 50 mg.kg [\hyperlink{Colistimethate Sodium}{PMID: 28741653}, V M Yuen et al., 2017] Limited pharmacokinetic (PK) data suggest that currently recommended pediatric dosages of colistimethate sodium (CMS) by the Food and Drug Administration and European Medicines Agency may lead to suboptimal exposure, resulting in plasma colistin concentrations that are frequently <2 mg/liter. We conducted a population PK study in 17 critically ill patients 3 months to 13.75 years (median, 3.3 years) old who received CMS for infections caused by carbapenem-resistant Gram-negative bacteria. CMS was dosed at 200,000 IU/kg/day (6.6 mg colistin base activity [CBA]/kg/day; 6 patients), 300,000 IU/kg/day (9.9 mg CBA/kg/day; 10 patients), and 350,000 IU/kg/day (11.6 mg CBA/kg/day; 1 patient). Plasma colistin concentrations were determined using ultraperformance liquid chromatography combined with electrospray ionization-tandem mass spectrometry. Colistin PK was described by a one-compartment disposition model, including creatinine clearance, body weight, and the presence or absence of systemic inflammatory response syndrome (SIRS) as covariates ( [\hyperlink{Colistimethate Sodium}{PMID: 28741653}, Charalampos Antachopoulos et al., 2021] The increasing frequency of infections caused by multidrug-resistant (MDR) Gram-negative bacteria has led to the reappraisal of colistimethate use. We present a case series of critically ill pediatric patients without cystic fibrosis who received intravenous colistimethate treatment. All available relevant medical records were reviewed. Seven children without cystic fibrosis (mean age 7.7 years; 2 female), admitted to the intensive care unit of a tertiary-care pediatric hospital in Athens, Greece, were identified to have received intravenous colistimethate during October 2004 to May 2008. MDR Acinetobacter baumannii, Pseudomonas aeruginosa, and/or Klebsiella pneumoniae were isolated from blood and/or bronchial secretions specimens in 6 of 7 reported patients. All isolates were susceptible to colistin. All 7 patients received intravenous colistimethate in a dosage of 5 mg/kg daily (divided in 3 equal doses, administered every 8 hours). Five children received colistimethate for 10 days and the remaining 2 for 2 and 23 days, respectively. The infections caused by MDR Gram-negative bacteria were improved in 6 children with microbiologically documented infections. Five of the 7 children were discharged from the ICU. The remaining 2 children died (1 of them had received colistimethate for 2 days); their death was not attributed to MDR Gram-negative infection. No nephrotoxicity or other type of toxicity of colistimethate was noted in this case-series. Although the small number of included cases precludes any firm conclusions, our study suggests that colistimethate may have a role for the treatment of infections caused by MDR Gram-negative bacteria in critically ill pediatric patients. [\hyperlink{Colistimethate Sodium}{PMID: 28741653}, Matthew E Falagas et al., 2009]

\hypertarget{pmid_31956632}{P}harmacokinetic and clinical studies recommend applying loading dose of colistin for the treatment of severe infections in the critically ill adults. Pharmacokinetic studies of colistin in children also highlight the need for a loading dose. However, there are no clinical studies evaluating the effectiveness of colistin loading dose in children. In a randomized trial, children with ventilator-associated pneumonia or central line-associated bloodstream infection (CLABSI) for whom colistin was initiated, were enrolled. Patients were randomized into two groups; loading dose and conventional dose treatment arms. In the conventional treatment arm, colistimethate sodium was initiated with maintenance dose. In the loading dose group, colistimethate sodium was commenced with a loading dose of 150,000 international unit/kg, then on the maintenance dose. Both treatment arms also received meropenem as combination therapy. Primary outcomes were overall efficacy, clinical improvement and microbiological cure. Secondary outcomes were colistin-induced nephrotoxicity and development of resistance. Thirty children completed this study. There was a significantly higher overall efficacy in the group received loading dose (42.9 vs. 6.3\%,  This preliminary study suggests that colistin loading dose might have some benefits in critically ill children, specifically in children with CLABSI. Further trials are required to elucidate colistin best dosing strategy in critically ill children with severe infections. [\hyperlink{Colistimethate Sodium}{PMID: 31956632}, Shiva Fatehi et al., ]

\hypertarget{pmid_27994915}{T}he emergence of infections due to multidrug-resistant  In this study, we aimed to evaluate the clinical efficacy and safety of colistin use in critically ill pediatric patients. This study has a retrospective study design. Sixty-one critically ill children were identified through the department's patient files archive during the period from January 2011 to November 2014. Twenty-nine females and thirty-two males with a mean±standard deviation (SD) age of 61±9 months (range 0-216, median 12 months) received IV colistin due to MDR-GNB infections. Bacteremia (n=23, 37.7\%) was the leading diagnosis, followed by pneumonia (n=19, 31\%), clinical sepsis (n=7, 11.4\%), wound infection (n=6, 9.8\%), urinary tract infection (n=5, 8.1\%) and meningitis (n=1, 1.6\%). All of the isolates were resistant to carbapenems; however, all were susceptible to colistin. The isolated microorganisms in decreasing order of frequency were:  Colistin appears to be a safe and efficacious drug for treating MDR-GNB infections in children. [\hyperlink{Colistimethate Sodium}{PMID: 27994915}, Ayşe Karaaslan et al., 2016]

\hypertarget{pmid_30722017}{I}ntravenous colistin is widely used to treat infections in pediatric patients. Unfortunately, there is a paucity of pharmacological information to guide the selection of dosage regimens. The daily dose recommended by the US Food and Drug Administration (FDA) and European Medicines Agency (EMA) is the same body weight-based dose traditionally used in adults. The aim was to increase our understanding of the patient factors that influence the plasma concentration of colistin, and assess the likely appropriateness of the FDA and EMA dosage recommendations. There were 5 patients, with a median age of 1.75 (range 0.1-6.25) years, a median weight of 10.7 (2.9-21.5) kg, and a median creatinine clearance of 179 (44-384) mL/min/1.73m2, who received intravenous infusions of colistimethate each 8 hours. The median daily dose was 0.21 (0.20-0.21) million international units/kg, equivalent to 6.8 (6.5-6.9) mg of colistin base activity per kg/day. Plasma concentrations of colistimethate and formed colistin were subjected to population pharmacokinetic modeling to explore the patient factors influencing the concentration of colistin. The median, average, steady-state plasma concentration of colistin (Css,avg) was 0.88 mg/L; individual values ranged widely (0.41-3.50 mg/L), even though all patients received the same body weight-based daily dose. Although the daily doses were \textasciitilde{}33\% above the upper limit of the FDA- and EMA-recommended dose range, only 2 patients achieved Css,avg ≥2mg/L; the remaining 3 patients had Css,avg <1mg/L. The pharmacokinetic covariate analysis revealed that clearances of colistimethate and colistin were related to creatinine clearance. The FDA and EMA dosage recommendations may be suboptimal for many pediatric patients. Renal functioning is an important determinant of dosing in these patients. [\hyperlink{Colistimethate Sodium}{PMID: 30722017}, Mong How Ooi et al., 2019]

\hypertarget{pmid_3168097}{G}olytely solution is now commonly used in preoperative bowel preparation or in colonoscopy and barium enema in adults. Studies have demonstrated its effectiveness and good acceptance in regards to clinical as well as biological point of view. In children, it has been used more recently, but since 1984 several teams agree to find the method excellent. Our study aimed to confirm there was no electrolytic movement caused by golytely and that using it without reserve in children was possible, even in the very young ones. Children are generally very sensible to those movements and mainly as they have a general anaesthesia in the hours following the golytely administration for investigation or surgery. Up to now, 54 children from 4 months to 18 years aged have been studied. Besides the good quality of the preparation noted by the operator and the good clinical tolerance, no significant change of the sodium, potassium, chloride, creatinine and proteins has been noticed. Only urea has decreased very lightly but not out of norms. These results confirm that golytely is safe and effective in preparing the bowel in children. [\hyperlink{Colistimethate Sodium}{PMID: 3168097}, A Bichet-Sicard et al., 1988]

\hypertarget{pmid_37283451}{D}ata regarding the treatment of childhood granulomatous periorificial dermatitis (CGPD) using oral therapies are limited. This study included 31 Chinese children with CGPD treated with oral roxithromycin. After 12 weeks of treatment, 90.3\% of the patients recovered, and there were no severe adverse effects. Our results suggest that oral roxithromycin is an effective and safe treatment for CGPD. [\hyperlink{Colistimethate Sodium}{PMID: 37283451}, Senfen Wang et al., ]

\hypertarget{pmid_27276179}{I}n this study, we sought to evaluate the pharmacokinetics of colistin after intravenous administration of colistimethate sodium (CMS) in the critically ill neonates with Gram-negative bacterial infections. A single intravenous dose of CMS [approximately 150,000 IU/kg, equivalent to 5 mg/kg colistin base activity (CBA)] was administered to 7 critically ill neonates. Mean (±SD) maximum plasma colistin concentration and area under the time-concentration curve from 0 to infinity were 3.0 ± 0.7 µg/mL and 25.3 ± 10.4 µg·h/mL, respectively. Time to maximum concentration, half-life, apparent volume of distribution and clearance were 1.3 ± 0.9 hours, 9.0 ± 6.5 hours, 7.7 ± 9.3 L/kg and 0.6 ± 0.3 L/h/kg, respectively. After a dose regimen of 5 mg/kg CBA every 24 hours, the average concentration expected at steady state is 1.1 ± 0.4 µg/mL. In critically ill neonates, a single intravenous dose of 5 mg CBA/kg (approximately 150,000 IU/kg of CMS) resulted in suboptimal plasma concentrations of colistin. According to our pharmacokinetics data, the dosage of CMS currently used in critically ill neonates is insufficient. [\hyperlink{Colistimethate Sodium}{PMID: 27276179}, Narongsak Nakwan et al., 2016]

\hypertarget{pmid_26868136}{N}eonatal sepsis caused by multidrug-resistant gram-negative bacteria has been reported in different parts of the world. It is a major threat to neonatal care, carrying a high rate of morbidity and mortality. While Colistin is the treatment of choice, few studies have reported its use in neonatal patients. A retrospective descriptive study of all neonatal patients who had multidrug-resistant Acinetobacter sepsis and were treated with Colistin over a 2-year period. Patients' charts and hospital laboratory data were reviewed. During the study period, 21 newborns were treated with Colistin. All had sepsis evident by positive blood culture and clinical signs of sepsis. The median gestational age and birth weight were 33 weeks (26-39) and 1700 g (700-3600), respectively. Nine (43 \%) were very low birth weight infants. Eighteen (86 \%) were preterm infants. Nineteen (91 \%) newborns survived. No renal impairment is documented in any of our patients. Fourteen (67 \%) of our patients had elevated eosinophil counts following Colistin treatment, for those patients, the average eosinophilic counts ± standard deviation before and after Colistin therapy were 149.08 ± 190.38 to 1193 ± 523.29, respectively, with a p value of less than 0.0001. Our study showed that Colistin was both effective and safe for treating multidrug-resistant Acinetobacter neonatal sepsis. This is a retrospective study. No universal protocol was used for the patients. The factors that might affect the response or cause side effects are difficult to evaluate. [\hyperlink{Colistimethate Sodium}{PMID: 26868136}, Manar Al-Lawama et al., 2016]

\hypertarget{pmid_15951862}{D}iagnostic and therapeutic procedures in children are made easier using sedation. However, there is no consensus about which drug should be used to achieve this. Furthermore, none of the drugs used for sedation are risk free. The aim of this work is to study sedation indications, effectiveness, and safety at our center. A prospective observational study conducted at the Pediatric Day Care Unit, King Fahad National Guard Hospital, Riyadh, Saudi Arabia. The study covered 17.5 weeks in 2 periods: May 9th 1999 to June 13th 1999 and October 31st 2001 to February 11th 2002. Children <12 years were included. Collected data included demographics, indication, drug dosing and outcome. Data were reported as mean +/- SD. We included 148 patients, age 38 +/- 30 months. Adequate sedation was achieved in 79\% after initial chloral hydrate (CH) dose of 56.9 +/- 9.3 mg/kg, in 95\% after adding 18.5 +/- 6.4 mg/kg CH and in 96\% after adding second drug. Compared to nonrespondents, first CH dose respondents were younger and lower in weight. The CH side effects were few and mild. Chloral hydrate is a safe and effective agent for sedation in children with an age and weight dependent response. [\hyperlink{Colistimethate Sodium}{PMID: 15951862}, Omar M Hijazi et al., 2005]

\hypertarget{pmid_27083755}{T}o review the evidence for the efficacy and safety of colchicine in children with pericarditis. Systematic review. The following databases were searched for studies about colchicine in children with pericarditis (June 2015): Cochrane Central, Medline, EMBASE and LILACS. All observational and experimental studies on humans with any length of follow-up and no limitations on language or publication status were included. The outcomes studied were recurrences of pericarditis and adverse events. Two authors extracted data and assessed quality of included studies using the Cochrane risk of bias tool for non-randomised trials. Two case series and nine case reports reported the use of colchicine in a total of 86 children with pericarditis. Five articles including 74 paediatric patients were in favour of colchicine in preventing further pericarditis recurrences. Six studies including 12 patients showed that colchicine did not prevent recurrences of pericarditis. No randomised controlled trials (RCTs) were found. Although colchicine is an established treatment for pericarditis in adults, it is not routinely used in children. There is not enough evidence to support or discourage the use of colchicine in children with pericarditis. Further research in the form of large double-blind RCTs is needed to establish the efficacy of colchicine in children with pericarditis. [\hyperlink{Colistimethate Sodium}{PMID: 27083755}, Samer Alabed et al., 2016]

\hypertarget{pmid_20645317}{P}ediatric ulcerative colitis (UC) has a more severe phenotype, reflected by more extensive disease and a higher rate of acute severe exacerbations. The pooled steroid-failure rate among 291 children from five studies is 34\% (95\% confidence interval [CI]: 27\%-41\%). It is suggested that corticosteroids should be dosed between 1-1.5 mg/kg up to 40-60 mg daily. Food restriction has a limited role in severe UC and should be generally discouraged in children who do not have a surgical abdomen. Appraisal of radiologic findings in children must recognize the variation in colonic width with age and size. Data suggest that the Pediatric UC Activity Index (PUCAI), determined at day 3, should be used to screen for patients likely to fail corticosteroids (>45 points), and at day 5 to dictate the introduction of second-line therapy (>65-70 points). Cyclosporine is successful in children with severe colitis but its use should be restricted to 3-4 months while bridging to thiopurine treatment (pooled short-term success rate 81\% [95\% CI: 76\%-86\%]; n = 94 from eight studies). Infliximab may be as effective as cyclosporine (75\% pooled short-term response (95\% CI: 67\%-83\%); n = 126, six studies) with a pooled 1-year response of 64\% (95\% CI: 56\%-72\%). In toxic megacolon, in patients refractory to one salvage medical therapy, and in chronic severe disease, colectomy may be preferred. Decision-making regarding colectomy in children must consider the toxicity of medication consumed over many future years, the quality of life and self-image associated with either choice, as well as both functional outcomes and, in females, fertility following pouch procedures. [\hyperlink{Colistimethate Sodium}{PMID: 20645317}, Dan Turner et al., 2011]

\hypertarget{pmid_32179149}{U}se of colistin in children is rising in line with the increase of multidrug-resistant Gram-negative bacteria (MDR-GNB). In adults, a colistin loading dose is recommended to achieve therapeutic concentrations within 12-24 h. Here we aimed to describe the pharmacokinetic (PK) parameters of a loading dose versus a recommended initial dose of intravenous colistimethate sodium (CMS) in paediatric patients. A prospective, open-label, PK study was conducted in paediatric patients (age 2-18 years) with normal renal function. Patients (n = 20) were randomly assigned to receive either a CMS loading dose (LD group) of 4 mg of colistin base activity (CBA)/kg/dose or a standard initial dose (NLD group) of 2.5 mg (12-h interval) or 1.7 mg (8-h interval) of CBA/kg/dose. Serial blood samples were collected. Plasma concentrations of formed colistin were measured by LC-MS/MS. PK parameters were reported. Acute kidney injury (AKI) was monitored by serum creatinine and urine NGAL. The median (interquartile range) age and body weight were 8.5 (3.5-11.3) years and 21.5 (13.5-20.0) kg. The mean (standard deviation) of first-dose PK parameters of the LD group versus the NLD group were: C [\hyperlink{Colistimethate Sodium}{PMID: 32179149}, Noppadol Wacharachaisurapol et al., 2020] Emergence of multidrug-resistant Gram-negative nosocomial pathogens has led to resurgence of colistin use. Safety and efficacy data regarding colistin use in pediatric patients are sparse, while optimal dosage has not been defined. We present a case series of neonates and children without cystic fibrosis treated with various doses of colistin intravenously. The records of patients who received colistin in a tertiary-care hospital from January 2007 to March 2009 were reviewed. Thirteen patients (median age 5 years, range 22 days to 14 years) received 19 courses of colistin as treatment of pneumonia, central nervous system infection, bacteremia, or complicated soft tissue infection. The isolated pathogens were Acinetobacter baumannii, Enterobacter cloacae, Klebsiella pneumoniae, Pseudomonas aeruginosa, and Stenotrophomonas maltophilia. Daily dose of colistin (colistimethate) ranged between 40,000 and 225,000 IU/kg. Duration of administration ranged from 1 to 133 days. Other antimicrobials were co-administered in 18/19 courses. Increase of serum creatinine in one patient was associated with co-administration of colistin and gentamicin. Sixteen of 19 courses had a favorable outcome, and only two of the three deaths were infection-related. Colistin intravenous administration appears well tolerated even at higher than previously recommended doses and of prolonged duration. [\hyperlink{Colistimethate Sodium}{PMID: 32179149}, Elias Iosifidis et al., 2010]

\hypertarget{pmid_28275979}{S}edation is often required for children undergoing diagnostic procedures. Chloral hydrate has been one of the sedative drugs most used in children over the last 3 decades, with supporting evidence for its efficacy and safety. Recently, chloral hydrate was banned in Italy and France, in consideration of evidence of its carcinogenicity and genotoxicity. Dexmedetomidine is a sedative with unique properties that has been increasingly used for procedural sedation in children. Several studies demonstrated its efficacy and safety for sedation in non-painful diagnostic procedures. Dexmedetomidine's impact on respiratory drive and airway patency and tone is much less when compared to the majority of other sedative agents. Administration via the intranasal route allows satisfactory procedural success rates. Studies that specifically compared intranasal dexmedetomidine and chloral hydrate for children undergoing non-painful procedures showed that dexmedetomidine was as effective as and safer than chloral hydrate. For these reasons, we suggest that intranasal dexmedetomidine could be a suitable alternative to chloral hydrate. [\hyperlink{Colistimethate Sodium}{PMID: 28275979}, Giorgio Cozzi et al., 2017]

\hypertarget{pmid_21531030}{C}hloral hydrate (CH) is an oral sedative widely used to sedate infants and young children during auditory brainstem response (ABR) testing. The aim of this study was to record effectiveness, complications and safety of CH as a sedative for ABR. From January of 2003 until December of 2007, 1903 children were tested for ABR, 568 of them being under the age of 6 months. CH (8\%) was used for sedation at a dose of 40 mg/kg with a repeat dose, if necessary, for an adequate sedation, in 20-30 min. We recorded the effectiveness of CH as a sedative for ABR examination, as well as all complications related to the use of CH such as vomiting, rash, hyperactivity, respiratory distress and apnea. The statistical method used was the absolute and percentage frequency distribution of the occurrences. Sedation with CH was necessary to perform testing in 1591 (83.6\%) of the examined children. However, in the population of the examined infants, only 341 (60\%) were sedated with CH, because the remaining 227 (40\%) fell asleep by themselves. Complications included hyperactivity in 152 children (8\%), minor respiratory distress in 10 children (0.4\%), vomiting in 217 children (11.4\%), apnea in 4 children (0.2\%) and rash in 10 children (0.4\%). The complications of hyperactivity, vomiting and rash resolved without any medical treatment. The apnea cases were managed effectively by supplying ventilation to the children via a mask in the presence of an anesthesiologist. The use of CH at a dose of 40 mg/kg up to 80 mg/kg is safe and effective when administered in a setting with adequate equipment and the presence of well trained personnel. [\hyperlink{Colistimethate Sodium}{PMID: 21531030}, Eirini Avlonitou et al., 2011]

\hypertarget{pmid_25874300}{T}he prevalence of hospital strains of P. aeruginosa, A. baumannii and K. pneumoniae characterizing by multiple drug resistance to overwhelming majority of antibiotics anew evoked an increased interest to colistin. However, until now there is not enough information concerning pharmacokinetics of colistin to optimize dosage of this pharmaceutical. The study was carried out to analyze pharmacokinetics of both colistin and sodium colistimitate in children with chemically induced neutropenia. To quantitatively detect colistin in blood serum the technique of highly effective fluid chromatography with mass spectrometry was applied. The concentration of colistin was detected in 21 children with chemically induced neutropenia (13 patients with septicemia, 8 children of control group) after intravenous injection of sodium colistimitate. The significant variability of pharmacokinetic parameters of colistin was established both in patients with septicemia and in control group. The technique of highly effective fluid chromatography with mass spectrometry can be applied for therapeutic medicinal monitoring and optimization regimen of dosage. [\hyperlink{Colistimethate Sodium}{PMID: 25874300}, V I Zakharevich et al., 2015]

\hypertarget{pmid_20658485}{D}ata regarding the role of inhaled colistin in critically ill pediatric patients without cystic fibrosis are scarce. Three children (one female), admitted to the intensive care unit (ICU) of a tertiary-care pediatric hospital in Athens, Greece, during 2004-2009 received inhaled colistin as monotherapy for tracheobronchitis (two children), and as adjunctive therapy for necrotizing pneumonia (one child). Colistin susceptible Acinetobacter baumannii and Pseudomonas aeruginosa were isolated from the cases' bronchial secretions specimens. All three children received inhaled colistin at a dosage of 75 mg diluted in 3 ml of normal saline twice daily (1,875,000 IU of colistin daily), for a duration of 25, 32, and 15 days, respectively. All three children recovered from the infections. Also, a gradual reduction, and finally total elimination of the microbial load in bronchial secretions was observed during inhaled colistin treatment in the reported cases. All three cases were discharged from the ICU. No bronchoconstriction or any other type of toxicity of colistin was observed. In conclusion, inhaled colistin was effective and safe for the treatment of two children with tracheobronchitis, and one child with necrotizing pneumonia. Further studies are needed to clarify further the role of inhaled colistin in pediatric critically ill patients without cystic fibrosis. [\hyperlink{Colistimethate Sodium}{PMID: 20658485}, Matthew E Falagas et al., 2010]

\hypertarget{pmid_33655976}{C}hildren evaluated in the emergency department for head trauma often undergo computed tomography (CT), with some uncooperative children requiring pharmacological sedation. Chloral hydrate (CH) is a sedative that has been widely used, but its rectal use for child sedation after head trauma has rarely been studied. The objective of this study was to document the safety and efficacy of rectal CH sedation for cranial CT in young children.We retrospectively studied all the children with head trauma who received rectal CH sedation for CT in the emergency department from 2016 to 2019. CH was administered rectally at a dose of 50 mg/kg body weight. When sedation was achieved, CT scanning was performed, and the children were monitored until recovery. The sedative safety and efficacy were analyzed.A total of 135 children were enrolled in the study group, and the mean age was 16.05 months. The mean onset time was 16.41 minutes. Successful sedation occurred in 97.0\% of children. The mean recovery time was 71.59 minutes. All of the vital signs were within normal limits after sedation, except 1 (0.7\%) with transient hypoxia. There was no drug-related vomiting reaction in the study group. Adverse effects occurred in 11 patients (8.1\%), but all recovered completely. Compared with oral CH sedation, rectal CH sedation was associated with quicker onset (P < .01), higher success rate (P < .01), and lower adverse event rate (P < .01).Rectal CH sedation can be a safe and effective method for CT imaging of young children with head trauma in the emergency department. [\hyperlink{Colistimethate Sodium}{PMID: 33655976}, Quanmin Nie et al., 2021]

\section*{Cortrosyn}
\subsection*{Result}
\subsubsection*{Answer}

Unknown

\subsubsection*{{Explanation}}
\hypertarget{Cortrosyn}
A review of the available abstracts reveals the following evidence regarding the safety of Cortrosyn (cosyntropin) in children:

1. One targeted study evaluated the use of Cortrosyn depot in 14 children with severe and frequent asthmatic attacks, aged (age range not specified, but all were children). The study monitored plasma cortisol responses and clinical improvement over a 3-month period with intermittent intramuscular injections. The abstract reports clinical improvement and changes in cortisol levels but does not mention any significant adverse events or safety concerns. However, the sample size is small, and the abstract does not explicitly state that a comprehensive safety assessment was performed, nor does it affirm safety in a broad pediatric population or across specific age ranges [\hyperlink{pmid_6291354}{PMID: 6291354}, J O Bolodeoku et al., 1982].

2. Another study investigated the use of cosyntropin in extremely low birth weight infants (500-999 g, tested between 18 and 28 days of birth) to assess adrenal function. The study compared two doses of cosyntropin and evaluated cortisol responses. The abstract focuses on the diagnostic utility of different doses and does not report any adverse events or safety concerns related to the administration of cosyntropin. However, it does not explicitly affirm safety, nor does it provide a comprehensive safety assessment for this age group [\hyperlink{pmid_16159938}{PMID: 16159938}, Kristi L Watterberg et al., 2005].

No other abstracts specifically address the safety of Cortrosyn (cosyntropin) in children, nor do they provide targeted, affirmative evidence of safety or unsafety in any pediatric age group.

In summary, while there are studies in children and infants involving Cortrosyn (cosyntropin), none of the abstracts provide definitive, targeted evidence affirming its safety or unsafety in children according to the strict criteria defined (i.e., a targeted safety study affirming safety or unsafety). Therefore, the safety of Cortrosyn in children remains unknown based on the available abstracts.

\subsection*{Abstracts}
\hypertarget{pmid_29053119}{T}o assess the efficacy and safety of сortexin in the treatment of children with cerebral palsy (CP) combined with epilepsy. Eighty-four patients (55 boys and 29 girls), aged from 1 to 11 years, with CP combined with epilepsy received cortexin together with antiepileptic drugs (AEDs). Cortexin was administered in doses of 5-10 mg depending on the patient's age and body weight intramuscularly during hospitalization. Cortexin as add-on to AEDs reduced for more than two times the number of seizures, along with improvement of motor function, in 31 (36.9\%) patients. The improvement of motor function, but without a significant decrease in epileptic seizures, was achieved in 15 (17.8\%) of the patients. Reduction of epileptic seizures frequency (>2 times), but without a significant effect on motor function, was observed in 14 cases (16.7\%). Twenty-three patients (27.4\%) did not respond the therapy. The aggravation of epileptic seizures during cortexin therapy was observed in only 1 girl with West syndrome (1.2\%), and this was significantly lower than the probability of seizures aggravation on AED. Polypeptide nootropic medication cortexin demonstrated efficacy and safety as adjunctive therapy in children with CP combined with epilepsy. [\hyperlink{Cortrosyn}{PMID: 29053119}, A A Kholin et al., ]

\hypertarget{pmid_28827252}{C}eftriaxone is widely used in children in the treatment of sepsis. However, concerns have been raised about the safety of ceftriaxone, especially in young children. The aim of this review is to systematically evaluate the safety of ceftriaxone in children of all age groups. MEDLINE, PubMed, Cochrane Central Register of Controlled Trials, EMBASE, CINAHL, International Pharmaceutical Abstracts and adverse drug reaction (ADR) monitoring systems will be systematically searched for randomised controlled trials (RCTs), cohort studies, case-control studies, cross-sectional studies, case series and case reports evaluating the safety of ceftriaxone in children. The Cochrane risk of bias tool, Newcastle-Ottawa and quality assessment tools developed by the National Institutes of Health will be used for quality assessment. Meta-analysis of the incidence of ADRs from RCTs and prospective studies will be done. Subgroup analyses will be performed for age and dosage regimen. Formal ethical approval is not required as no primary data are collected. This systematic review will be disseminated through a peer-reviewed publication and at conference meetings. CRD42017055428. [\hyperlink{Cortrosyn}{PMID: 28827252}, Linan Zeng et al., 2017]

\hypertarget{pmid_6291354}{T}reatment trial over a period of 3 months was conducted with intermittent intramuscular injections of Cortrosyn depot in fourteen children with severe and frequent asthmatic attacks. The basal plasma cortisol levels were generally high, but higher than normal in four (29\%) of the patients. At a period of 24 h after the initial Cortrosyn injection was administered, plasma cortisol increases ranging between 4-52 micrograms/100 ml above the basal levels were recorded. At a period of 1 week after the end of daily injections for 1 week, increases of plasma cortisol ranging between 3-71 micrograms/100 ml above the basal levels were observed and presumed to be a reflection of an associated adrenal hypertrophy resulting from repetitive daily Cortrosyn injections. The highest increases at this stage were observed in the youngest patients with the severest asthmatic attacks, but not in their older counterparts. At the end of the trial treatment, clinical improvement was associated with lowered plasma cortisol levels compared with the elevated basal values. [\hyperlink{Cortrosyn}{PMID: 6291354}, J O Bolodeoku et al., 1982]

\hypertarget{pmid_20527137}{O}nly a few corticosteroids for topical use have proven safe and effective in pediatric populations down to 3 months of age. The authors report the results of a study designed to assess the efficacy and safety of hydrocortisone butyrate (HCB) 0.1\% in lipocream (LCr) vehicle in infants and children. A total of 264 boys and girls 3 months to less than 18 years old, with stable, mild to moderate atopic dermatitis affecting at least 10\% body surface area applied HCB 0.1\% in LCr or LCr alone twice daily for up to 1 month without occlusion. Primary end-points included: percent of patients who achieved treatment success based on physician global assessments. Secondary endpoint included: difference in pruritus and Eczema Area and Severity Index (EASI) at day 29. Treatment was significant (P < 0.001) for HCB 0.1\% LCr over vehicle. No serious nor significant adverse events were reported. Results are representative of a short duration treatment for a chronic disease. HCB 0.1\% in LCr is more effective than its vehicle in pediatric populations down to 3 months of age without significant adverse events when used twice a day for up to 1 month. [\hyperlink{Cortrosyn}{PMID: 20527137}, William Abramovits et al., ]

\hypertarget{pmid_11862174}{T}opical corticosteroids are useful for the treatment of pediatric dermatoses. However, concerns regarding possible systemic and topical toxicities have limited the use of moderate-potency corticosteroids in children. Our purpose was to characterize the safety of fluticasone propionate cream in children. Children between 3 months and 5 years 11 months (n = 32) and 3 up to 6 years of age (n = 19) with moderate to severe atopic dermatitis (> or =35\% body surface area; mean body surface area treated, 64\%) were treated with fluticasone propionate cream, 0.05\% twice daily for 3 to 4 weeks. Serum cortisol response, fluticasone levels, skin changes, and adverse events were analyzed. Mean cortisol levels were similar at baseline (13.76 +/- 6.94 microg/dL prestimulation and 30.53 +/- 7.23 microg/dL poststimulation) and at end of treatment (12.32 +/- 6.92 microg/dL prestimulation and 28.84 +/- 7.16 microg/dL poststimulation). Only 2 of 43 children had end-treatment poststimulation values less than 18.0 microg/dL. No significant adverse cutaneous effects were noted. Fluticasone propionate cream 0.05\% appears to be safe for the treatment of severe eczema for up to 4 weeks in children 3 months of age and older. [\hyperlink{Cortrosyn}{PMID: 11862174}, S F Friedlander et al., 2002]

\hypertarget{pmid_31321320}{A}zithromycin is widely used in children not only in the treatment of individual children with infectious diseases, but also as mass drug administration (MDA) within a community to eradicate or control specific tropical diseases. MDA has also been reported to have a beneficial effect on child mortality and morbidity. However, concerns have been raised about the safety of azithromycin, especially in young children. The aim of this review is to systematically identify the safety of azithromycin in children of all ages. MEDLINE, PubMed, Cochrane Central Register of Controlled Trials, Embase, CINAHL, International Pharmaceutical Abstracts and adverse drug reaction (ADR) monitoring systems will be systematically searched for randomised controlled trials (RCTs), cohort studies, case-control studies, cross-sectional studies, case series and case reports evaluating the safety of azithromycin in children. The Cochrane risk of bias tool, Newcastle-Ottawa and quality assessment tools, and The Joanna Briggs Institute Critical Appraisal tools will be used for quality assessment. Meta-analyses will be conducted to the incidence of ADRs from RCTs if appropriate. Subgroup analyses will be performed in different age and azithromycin dosage groups. Formal ethical approval is not required as no primary data are collected. This systematic review will be disseminated through a peer-reviewed publication. CRD42018112629. [\hyperlink{Cortrosyn}{PMID: 31321320}, Peipei Xu et al., 2019]

\hypertarget{pmid_10851644}{C}iprofloxacin clinical and bacteriological efficacies, as well as tolerability mainly with respect to chondrotoxicity were evaluated in the treatment of children with mucoviscidosis. The drug was shown to have high clinical and moderate bacteriological efficacies. As for its tolerability, adverse reactions chiefly associated with affection of the gastrointestinal tract, i.e. nausea, stomach pain, diarrhea, increased transaminase levels were recorded. The arthrotoxicity episode was single and transitory. The use of ciprofloxacin had no negative effect on the children growth. No chondrotoxic effect of ciprofloxacin in the treatment of children was observed which is explained in the paper. It is concluded that ciprofloxacin is in general an efficient and safe antibiotic useful for the treatment of children. [\hyperlink{Cortrosyn}{PMID: 10851644}, S S Postnikov et al., 2000]

\hypertarget{pmid_8132376}{L}ike all fluoroquinolones, ciprofloxacin causes articular damage in juvenile animals. Consequently, this drug was not recommended for children or pregnant women. However, due to its antibacterial effectiveness and convenience of oral administration, ciprofloxacin is now increasingly used for the treatment of certain infectious conditions in children and adolescents aged less than 18 years. In this paper the published literature on this subject is reviewed. Up to now, data are available on more than 1,500 paediatric patients who were given ciprofloxacin, two-thirds of whom were suffering from acute infectious bronchopulmonary exacerbations of cystic fibrosis, mainly due to Pseudomonas aeruginosa. The effectiveness of oral ciprofloxacin for this indication compared well to that of standard intravenous combination regimens. The majority of the remaining published trials was conducted in children with multiresistant typhoid fever; the administration of ciprofloxacin was successful in up to 100\% of the cases. The safety profile of ciprofloxacin in children and adolescents was very similar to that observed in adult patients. Adverse events were noted in 5-15\%, with gastrointestinal, skin and central nervous system reactions being the most common. Reversible arthralgia occurred in 36 out of 1,113 patients with cystic fibrosis, and in no case could cartilage damage be demonstrated by radiographic procedures. Thus, publication data clearly suggest that the administration of ciprofloxacin to children is effective and safe, but there is a need for further prospective, well-controlled clinical trials. [\hyperlink{Cortrosyn}{PMID: 8132376}, R Kubin et al., ]

\hypertarget{pmid_17300659}{C}orticosteroids are currently the first line of treatment for patients with atopic dermatitis. In the pediatric population however, the potential impact of adrenal suppression is always an important safety concern. Twenty boys and girls, 5-12 years of age, with normal adrenal function and a history of atopic dermatitis were maximally treated three times daily with a lipid-rich, moisturizing formulation of hydrocortisone butyrate 0.1\% for up to 4 weeks. At the conclusion of the 4-week treatment period, cosyntropin injection stimulation testing showed no evidence of adrenal suppression. In addition, the therapy was noted to be highly efficacious, with a clinical success rate of 80\% (Physician Global Score of (0) clear or (1) almost clear). No local side effects associated with prolonged use of topical corticosteroids were reported. In summary, this study supports the contention that this lipid-rich, moisturizing formulation of hydrocortisone butyrate 0.1\% was a well-tolerated and beneficial treatment for atopic dermatitis, demonstrating no adrenal suppression in the pediatric population aged 5-12 years. The relevance of these findings for children below 5 years of age, because of difference in body mass/surface area ratios, remains to be determined. [\hyperlink{Cortrosyn}{PMID: 17300659}, Lawrence Eichenfield et al., ]

\hypertarget{pmid_18977585}{T}o prospectively study the efficacy and safety of intraparotid gland injection of Botulinum neurotoxin serotype A (Dysport) for the treatment of sialorrhea (drooling) in children with cerebral palsy (CP). Twenty-four children, ages 21 months to 7 years, were recruited and randomized to receive either treatment with 100U Botulinum toxin or placebo. Rating scales for the frequency and severity of drooling were performed at the time of injection, at 1 month, and at baseline prior to the second injection. A second set of injections of either 140U of drug or placebo was given 4 months later, and the same rating scales were used. Eight patients declined the second injection. Due to high dropouts in the placebo group in second set of injections, statistical analysis was performed for the results of the initial injection only. Scores of the median frequency (p=0.034) and severity (p=0.026) of drooling were reduced in the treatment group. Median total score also declined in the treatment group (p=0.027). After the second injection, five out of nine patients injected with the drug showed a decline in the total score; including three patients who did not respond to the first injection. Only two patients experienced transient increase in drooling after the treatment with the drug. Botulinum toxin is an effective and safe treatment option for drooling in children with CP. [\hyperlink{Cortrosyn}{PMID: 18977585}, Ali H Alrefai et al., 2009]

\hypertarget{pmid_1917049}{S}ix hundred and thirty four adolescents and children aged three days to 17 years treated with ciprofloxacin on a compassionate basis were analysed for drug safety. 62\% of the ciprofloxacin courses were given to patients with respiratory tract infection, primarily those with acute pulmonary exacerbation of cystic fibrosis. The mean daily oral dose was 25.2 mg/kg body weight. The duration of treatment ranged from one to 880 days (mean 22.8 days). Because of the arthropathogenic potential of quinolones in juvenile animals special emphasis was placed on the evaluation of musculoskeletal adverse events. Arthralgia considered by the treating physicians to be related to ciprofloxacin was reported in eight children, all of whom were females. Arthralgia resolved in all children. Some of these children were given subsequent courses of ciprofloxacin with no complaints of arthralgia. Overall, the safety profile of ciprofloxacin in children is not substantially different from that of adults. [\hyperlink{Cortrosyn}{PMID: 1917049}, V Chyský et al., ]

\hypertarget{pmid_21159531}{C}urrently there is much controversy whether to treat idiopathic facial palsy with corticosteroids with sparse data on the natural course of the disease in children. We performed a prospective study on all children <15 years of age who were admitted to our unit for facial palsy between 1st July 1998 to 30th June 2008. All patients received a standardized work-up and follow-up. Therapy consisted of symptomatic treatment either with (in case of neuroborreliosis) or without a 14 day course of intravenous antibiotics. 106 patients were included in our study. The calculated incidence for facial palsy was 21.1/100000/year for children <15 years. The incidence for neuroborreliosis (NB) in this age group was calculated to be 4.9/100000/year. The overall rate of complete recovery was 97.6\% with significantly faster recovery in younger children and in patients with NB as compared to idiopathic facial palsy. Both patients with incomplete recovery were at least 14 years old and presented late in the course of the disease. Based on the rate of 97.6\% spontaneous complete recovery we believe that the routine use of corticosteroids in children with facial palsy is not justified, unless there is new data from controlled trials in children. [\hyperlink{Cortrosyn}{PMID: 21159531}, Andreas Christoph Jenke et al., 2011]

\hypertarget{pmid_7633153}{W}e evaluated the safety of ciprofloxacin administered in a dose of 15-25 mg/kg for 9-16 days, in a case series of 58 children who were between 8 months and 13 years of age. No arthropathy was observed during therapy and follow-up. Blinded evaluation of 22 pairs of nuclear magnetic resonance scans obtained before and between day 10 and 15 of therapy did not reveal any cartilage damage. After the first dose of ciprofloxacin (10 mg/kg), serum fluoride levels increased at 12 h in 15 of 19 (79\%) patients; 24-h urinary fluoride excretion was higher on day 7 compared with basal values in 16 of 18 (88.9\%) patients. Height z scores of 53 patients at a mean of 22.5 months of follow-up were not significantly different from basal scores (p = 0.12). In conclusion, ciprofloxacin may be recommended for use in children for short duration when effective alternative antibacterials are unavailable. However, there is a need for further studies to evaluate the tissue accumulation of fluoride and its potential to cause toxic effects. [\hyperlink{Cortrosyn}{PMID: 7633153}, K M Pradhan et al., 1995]

\hypertarget{pmid_25692259}{P}ediatric shock is associated with significant morbidity and limited evidence suggests treatment with corticosteroids. The objective of this study was to describe practice patterns and outcomes associated with corticosteroid use in children with shock. We conducted a retrospective, cohort study in four pediatric intensive care units (PICU) in Canada. Patients aged newborn to 17 years admitted to PICU with shock between January 2010 and June 2011 were eligible. 364 patients were included. The frequency of hydrocortisone administration was 22.3\% overall (95\% CI: 18.0, 26.5) and 59.4\% in patients who received at least 60 cc/kg of fluid and were on two or more vasoactive agents. Patients administered hydrocortisone had higher PRISM scores (19, IQR 11-24 versus 9, IQR 5-16; P < 0.0001), higher inotrope scores (15, IQR 10-25 versus 7.5, IQR 3.3-10.6, P < 0.0001) and were more likely to have received 60 cc/kg of fluid resuscitation (59.3\% versus 33.6\%, OR 2.88, 95\% CI: 2.09, 3.96). In an adjusted analysis, patients who received hydrocortisone spent more time on vasoactive infusions (64 versus 34  hours, hazard ratio 0.72, 95\% CI: 0.62, 0.84) and had a higher incidence of positive cultures between day 4 and day 28 post admission (24.7\% versus 14.5\%, OR 1.79, 95\% CI: 1.58, 2.04). Hydrocortisone administration was associated with longer time on vasopressors and increased incidence of positive cultures even after correcting for illness severity. Caution should be exercised in administering hydrocortisone for shock until there is clear evidence for benefit in this patient population. [\hyperlink{Cortrosyn}{PMID: 25692259}, Kusum Menon et al., 2015]

\hypertarget{pmid_9002122}{Q}uinolone-induced cartilage toxicity has been observed in experimental juvenile animal studies and is species- and dose-specific. Accordingly these findings have led to the contraindication of fluoroquinolones in children. Previous data in 634 adolescents and children treated with compassionate use ciprofloxacin demonstrated low rates of reversible arthralgia and a safety profile similar to that for adult patients. This report describes the safety findings in 1795 children who received 2030 treatment courses of intravenous or oral ciprofloxacin. The average doses of intravenous and oral ciprofloxacin in the study population were 8 and 25 mg/kg/day, respectively. Treatment-associated events were reported in 10.9\% of children receiving oral ciprofloxacin compared with 18.9\% among intravenous recipients. Overall arthralgia occurred during 31 ciprofloxacin treatment courses (1.5\%) and the majority of events were of mild to moderate severity and resolved without intervention. More than 60\% of arthralgia episodes were in children with cystic fibrosis. The adverse event pattern in children receiving ciprofloxacin in this analysis was similar to that observed in adults. Rates of reversible arthralgia were low and unchanged from previously published surveillance data in children. [\hyperlink{Cortrosyn}{PMID: 9002122}, B Hampel et al., 1997]

\hypertarget{pmid_11761817}{D}ietary elimination is a treatment of first choice in food hypersensitivity. Such therapy is not always enough to stop the disease and introduction of pharmacological treatment is necessary. In prevention and long term treatment antiallergic drugs are recommended. The aim of the study was to assess efficacy and safety of oral sodium cromoglycate in treatment of food hypersensitivity in the youngest children. In our study we examined: the group of 25 children aged 6 months-3 years treated with oral cromolyn sodium during the period 4-20 weeks and 29 children aged 6 months-3 years treated with ketotifen. Symptoms from skin, digestive and respiratory tract, behaviour status were evaluated for drugs efficacy. Cromolyn and ketotifen effected a significant decrease in total symptoms score. The treatment was well tolerated. No serious side effects were noted. The incidents of skin rash, disquiet during the night, diarrhoea and urticaria were only 8 percent. Sodium cromoglycate is safe and effective drug in treatment of food allergy in children; specially in symptoms from gastrointestinal tract and multi-organs allergy. [\hyperlink{Cortrosyn}{PMID: 11761817}, E Zur et al., 2001]

\hypertarget{pmid_19057445}{T}o study the efficacy of low-dose intravenous hydrocortisone therapy in the management of pediatric septic shock with respect to the time taken for shock reversal and requirement of inotropes. Open label randomized pilot study. Pediatric intensive care unit of a tertiary care pediatric center in a third world country. Thirty-eight children, 2 months-12 yrs of age, with septic shock unresponsive to fluid therapy alone. Intravenous hydrocortisone 5 mg/kg/day in four divided doses followed by half the dose for a total duration of 7 days or normal saline (similar amount in a similar manner) for the same duration. There was a trend toward earlier reversal of shock (median 49.5 vs. 70 hrs, p = 0.65, Mann-Whitney U test) and lower inotropes requirement (median \{lsqb;10th-90th centile\{rsqb; inotropes score: 20 \{lsqb;15-60\{rsqb; vs. 50 \{lsqb;20-80\{rsqb;, p = 0.15) in the hydrocortisone-treated patients as compared with controls, although the difference was not statistically significant. Mortality rate was similar in both groups. Our data, although, inconclusive favor the need for a study with a larger sample size to clearly define role of low-dose hydrocortisone in pediatric septic shock in developing countries, while taking in consideration effect of malnutrition, delayed presentations, and their interactions with the hypothalamic-pituitary-adrenocortical axis. [\hyperlink{Cortrosyn}{PMID: 19057445}, Harsha T Valoor et al., 2009]

\hypertarget{pmid_8295811}{A} prospective, randomized, single (investigator) blind multicenter study was performed to compare the safety and efficacy of clarithromycin and cefadroxil oral suspensions in the treatment of mild to moderate skin and skin structure infections in children. Male and female patients ages 6 months to 12 years were enrolled at 24 study centers in the United States. Patients had signs and symptoms consistent with mild to moderate skin or skin structure infections judged suitable for oral antimicrobial therapy. Clarithromycin oral suspension was given to 118 children in a dose of 7.5 mg/kg (maximum of 500 mg) twice daily; cefadroxil oral suspension was given to 113 children in a dose of 15 mg/kg (maximum of 1000 mg) twice daily. Among clinically evaluable patients clinical success rates (cure plus improvement) were 96\% (71 of 74) for clarithromycin and 98\% (83 of 85) for cefadroxil (P = 0.664). Bacteriologic cure rates in evaluable clarithromycin and cefadroxil patients were 96\% (72 of 75) and 99\% (89 of 90), respectively (P = 0.331). Pathogen eradication rates based on 204 evaluable pathogens were 97\% in the clarithromycin group and 99\% in the cefadroxil group (P = 0.326). Adverse events were mild or moderate and were reported in 25\% of clarithromycin and 35\% of cefadroxil patients (P = 0.085). In both groups adverse events involved primarily the digestive tract. No significant laboratory changes were noted. Clarithromycin oral suspension appears to be a safe and effective alternative to cefadroxil for the treatment of pediatric skin and skin structure infections. [\hyperlink{Cortrosyn}{PMID: 8295811}, A A Hebert et al., 1993]

\hypertarget{pmid_2292542}{T}he Cystic Fibrosis Clinic at the Royal Belfast Hospital for Sick Children has treated 31 children with ciprofloxacin, for serious pseudomonas infection in cystic fibrosis, and carefully monitored the safety and acceptability of the drug. Initially, eight very ill children were treated on a named-patient basis, with an encouraging clinical response and few adverse effects. Children aged 10-18 years were included in a study of four consecutive exacerbations of respiratory disease, comparing (i) oral ciprofloxacin in each episode with (ii) ciprofloxacin alternating with intravenous azlocillin and tobramycin. Other children with cystic fibrosis were subsequently treated with ciprofloxacin, as the need arose. In all the groups very few adverse reactions were found; in particular only one child developed arthralgia. A total of 202 children in the UK have been treated with ciprofloxacin on a named-patient basis, and their clinicians have reported 46 adverse events that may have been drug-related. Overall ciprofloxacin appears to be safe and effective in children but concern about the possible occurrence of arthropathy remains and long term follow-up of these children may be necessary. [\hyperlink{Cortrosyn}{PMID: 2292542}, A Black et al., 1990]

\hypertarget{pmid_17273119}{T}o review the findings and discuss the implications of studies on the use of low-dose corticosteroids in septic shock. A critical appraisal of "Low-dose hydrocortisone improves shock reversal and reduces cytokine levels in early hyperdynamic septic shock" by Oppert et al. (Crit Care Med 2005; 33:2457-2464) with literature review. Previous studies have shown that low-dose corticosteroids shorten duration of shock in adults with sepsis, which is confirmed by the results of Oppert et al. The benefit on mortality is much less clear. Review of the literature casts doubt on whether these data can be extrapolated to children. There is some, albeit limited, evidence for the benefit of low-dose steroids in adults with sepsis. No supporting data are available for the pediatric population; therefore, a randomized controlled trial in septic children is needed. [\hyperlink{Cortrosyn}{PMID: 17273119}, Sandrijn M van Schaik et al., 2007]

\hypertarget{pmid_16159938}{V}arious cosyntropin doses are used to test adrenal function in premature infants, without consensus on appropriate dose or adequate response. The objective of this study was to test the cortisol response of extremely low birth weight infants to different cosyntropin doses and evaluate whether these doses differentiate between groups of infants with clinical conditions previously associated with differential response to cosyntropin. The design was a prospective, nested study conducted within a randomized clinical trial of low-dose hydrocortisone from November 1, 2001, to April 30, 2003. The setting was nine newborn intensive care units. The patients included infants with 500-999 g birth weight. The drug used was cosyntropin, at 1.0 or 0.1 microg/kg, given between 18 and 28 d of birth. We measured the cortisol response to cosyntropin. Two hundred seventy-six infants were tested. Previous hydrocortisone treatment did not suppress basal or stimulated cortisol values. Cosyntropin, at 1.0 vs. 0.1 microg/kg, yielded higher cortisol values (P < 0.001) and fewer negative responses (2 vs. 21\%). The higher dose, but not the lower dose, showed different responses for girls vs. boys (P = 0.02), infants receiving enteral nutrition vs. not (P < 0.001), infants exposed to chorioamnionitis vs. not (P = 0.04), and those receiving mechanical ventilation vs. not (P = 0.02), as well as a positive correlation with fetal growth (P = 0.03). A response curve for the 1.0-microg/kg dose for infants receiving enteral nutrition (proxy for clinically well infants) showed a 10th percentile of 16.96 microg/dl. Infants with responses less than the 10th percentile had more bronchopulmonary dysplasia and longer length of stay. A cosyntropin dose of 0.1 microg/kg did not differentiate between groups of infants with clinical conditions that affect response. We recommend 1.0 microg/kg cosyntropin to test adrenal function in these infants. [\hyperlink{Cortrosyn}{PMID: 16159938}, Kristi L Watterberg et al., 2005]

\hypertarget{pmid_8336748}{I}n order to study the adverse reaction of a new, inhaled steroid (Fluticasone) on the pituitary-adrenocortical axis in asthmatic children, we investigated 7 children (aged 7 to 15 years) before and during treatment with Fluticason (100-200 micrograms/day). For the dosage tested, we found no depression of adrenal function, neither in circadian cortisol secretion nor in hCRH-stimulation-test. Another 7 asthmatic children under treatment with Budesonide (800 micrograms/day) were examined by the same tests. They equally did not show an adrenocortical suppression. However, in 4 other children under therapy with oral prednisone (2.5 to 7.5 mg/day), there was a marked suppression on adrenocortical function, even with low doses of the steroid. We conclude that Fluticasone (as well as Budesonide) in the above dosages represent a safe therapy for bronchial asthma in children. [\hyperlink{Cortrosyn}{PMID: 8336748}, A Hoffmann-Streb et al., 1993]

\hypertarget{pmid_33611671}{T}he use of corticosteroids in the treatment of steroid-sensitive nephrotic (SSNS) syndrome in children has evolved surprisingly slowly since the ISKDC consensus over 50 years ago. From a move towards longer courses of corticosteroid to treat the first episode in the 1990s and 2000s, more recent large, well-designed randomized controlled trials (RCTs) have unequivocally shown no benefit from an extended course, although doubt remains whether this applies across all age groups. With regard to prevention of relapses, daily ultra-low-dose prednisolone has recently been shown to be more effective than low-dose alternate-day prednisolone. Daily low-dose prednisolone for a week at the time of acute viral infection seems to be effective in the prevention of relapses but the results of a larger RCT are awaited. Recently, corticosteroid dosing to treat relapses has been questioned, with data suggesting lower doses may be as effective. The need for large RCTs to address the question of whether corticosteroid doses can be reduced was the conclusion of the authors of the recent corticosteroid therapy for nephrotic syndrome in children Cochrane update. This review summarizes development in thinking on corticosteroid use in SSNS and makes suggestions for areas that merit further scrutiny. [\hyperlink{Cortrosyn}{PMID: 33611671}, Martin T Christian et al., 2022]

\hypertarget{pmid_9366699}{T}o compare the safety and efficacy of ofloxacin otic solution with those of Cortisporin otic solutions (neomycin sulfate, polymyxin B sulfate, and hydrocortisone) in otitis externa in adults and children. Two randomized, evaluator-blind, multicenter trials, 1 each in children and adults. Twenty-three primary care and referral ambulatory care sites per trial. A total of 314 adults (12 years and older) and 287 children (younger than 12 years). Of the total, data for 247 adults and 227 children were considered clinically evaluable (CE), and those for 98 children and 98 adults were microbiologically evaluable (ME). Ofloxacin (adults, 0.5 mL; children, 0.25 mL) twice daily or Cortisporin (adults, 0.2 mL; children, 0.15 mL) 4 times daily for 10 days. The CE subjects were cured if all signs and symptoms resolved at posttherapy (days 11-13) and test-of-cure (days 17-20) visits. The ME subjects had microbiological and clinical successes if they were cured and had microbiological eradication or presumed eradication. Cure was observed in 82\% and 97\% of CE adults and children treated with ofloxacin and 84\% and 95\% of CE adults and children treated with Cortisporin, respectively. The most common pathogens at the pretherapy visit were Pseudomonas aeruginosa, Staphylococcus aureus, and enteric bacilli. There were no statistically significant differences in clinical or microbiological and clinical cure or in the rates of adverse events between treatment groups. Ofloxacin given twice daily is as safe and effective as Cortisporin given 4 times daily for otitis externa. The bacteriological findings and treatment responses do not differ between adults and children. [\hyperlink{Cortrosyn}{PMID: 9366699}, R N Jones et al., 1997]

\hypertarget{pmid_378798}{D}uring 1978 I carried out a clinical trial in my pediatric practice with the new corticoid dermaticum Alfason, which contains the active substance hydrocortisone-17 abutyrate--an ester of hydrocortisone not previously used therapeutically. To begin with Alfason was given in an open trial to 100 children and later, because of the favourable results obtained, it was employed in another group of 100 children in a double-blind trial against a fluorinated corticoid, a substance which up to then I had most often prescribed. The results showed a good to very good effect in over 90\% of the cases treated with either drug, there was no substance related statistical difference. According to the literature the ratio of success regarding side effects is particularly favourable for Alfason so that its use in the treatment of corticoid-requiring dermatoses in childhood can be recommended. [\hyperlink{Cortrosyn}{PMID: 378798}, R T Rieder et al., 1979]

\section*{Cyklokapron}
\subsection*{Result}
\subsubsection*{Answer}

Unknown

\subsubsection*{{Explanation}}
\hypertarget{Cyklokapron}
Based on the abstracts available, there is no evidence specifically about Cyklokapron (tranexamic acid) in children. All the abstracts provided are about other medications, such as cyclosporine, ciprofloxacin, ceftriaxone, cyclopentolate, cycloferon, ciclopirox, and acyclovir, but none mention Cyklokapron or tranexamic acid. Therefore, there is not enough data available in the abstracts to determine whether Cyklokapron is safe or unsafe for use in children in any age range.

\subsection*{Abstracts}
\hypertarget{pmid_2642107}{A}lthough cyclosporine has improved allograft survival in renal transplant patients, problems with drug toxicity remain, raising the question whether cyclosporine should be stopped at some point post-transplant. However, the relative safety of converting from cyclosporine to another immunosuppressive agent, or simply stopping cyclosporine remains an issue of debate and has not been evaluated in children. We have developed a protocol to convert children, who are 6 months post-transplant and have stable kidney function, from cyclosporine and prednisone to azathioprine and prednisone. Eleven children have undergone conversion because of suspected/potential nephrotoxicity or because of other difficulties with cyclosporine (expense, hirsutism). These children were compared with a control group of 12 children who met all criteria for conversion at 6 months but remained on cyclosporine. Allograft survival was similar in both groups but the children converted from cyclosporine experienced an improvement in renal function as measured by calculated creatinine clearance. There were no episodes of rejection for a period of 4 months post-conversion and all rejection episodes that developed subsequently occurred during or after the change from daily to alternate-day prednisone. We believe that conversion from cyclosporine to azathioprine can be accomplished safely in children with stable allograft function but long-term risks and benefits need further evaluation. [\hyperlink{Cyklokapron}{PMID: 2642107}, B A Kaiser et al., 1989]

\hypertarget{pmid_19617660}{C}yclosporine A is used in the treatment of idiopathic nephrotic syndrome. We conducted this study to evaluate the effect of cyclosporine and its combination with ketoconazole in Egyptian nephrotic children with steroid-resistant and steroid-dependant minimal change. Forty-eight children with minimal change lesions who received cyclosporine with or without ketoconazole were studied. Their mean age was 5.17 +/- 1.59 years, and they were 31 boys and 17 girls. The mean duration of the disease was 6.22 +/- 3.16 years. Thirty-one of the children were steroid dependent and 17 were steroid resistant. Cyclosporine treatment was commenced after remission was attained and adjusted to a target trough level of 100 ng/mL. The mean cyclosporine therapy at a dose of 2.07 +/- 0.91 mg/kg was administered for a mean of 25.75 +/- 1.95 months. Thirty-three patients received adjunctive ketoconazole therapy. Thirty-eight patients (79.2\%) responded well to cyclosporine. Steroid therapy could be discontinued in 43 patients (89.6\%), but 9 experienced relapse. Ten patients (20.8\%) were resistant to cyclosporine therapy. Fifteen patients received cyclosporine alone, while 33 received concomitant cyclosporine and ketoconazole. The response to cyclosporine was significantly better in those on ketoconazole. The economic effect of ketoconazole therapy was a reduction in the costs of cyclosporine treatment by 47.4\% at 1 year of treatment. Cyclosporine treatment in children with minimal change nephrotic syndrome is effective in preventing relapse and decreasing steroid toxicity. Its combination with low-dose ketoconazole is safe, reduces treatment costs, and improves the response to cyclosporine. [\hyperlink{Cyklokapron}{PMID: 19617660}, Alaa Sabry et al., 2009]

\hypertarget{pmid_21144334}{C}yclosporine has been found to be effective and safe in many inflammatory skin disorders such as psoriasis and atopic dermatitis (AD), in adults and in children. Its use in paediatrics is still under scope. We present three patients who started cyclosporine but stopped due to complications. It is our aim to warn about potential side effects of cyclosporine and recommend cautious utilization. Two children, aged 4 and 13 years, with AD and one child, aged 2 years, with erythrodermic psoriasis, were treated with oral cyclosporine. developed secondary impetigo on the 6th day of treatment. Started topical corticosteroids and topical calcineurin inhibitors afterwards, with no relapses. developed herpetic infection, hepatic and renal impairment (eventual drug interaction) on the 4th day of treatment. THIRD CASE: Psoriasis and impetigo, treated with flucloxacillin, gentamicin. Generalized angioedema and urticariform lesions after 6 days of cyclosporine. Beta lactam hypersensitivity reaction under study. Eventual cyclosporine toxicity to consider. The data on cyclosporine use in children is still scarce. Use should be limited to cases with precise indication, after considering risks and benefits. [\hyperlink{Cyklokapron}{PMID: 21144334}, João Antunes et al., ]

\hypertarget{pmid_18305467}{W}e conducted a prospective, open-label multicenter trial to evaluate the efficacy and safety of treating children with frequently relapsing nephrotic syndrome with cyclosporine. Patients were randomly divided into two groups with both initially receiving cyclosporine for 6 months to maintain a whole-blood trough level between 80 and 100 ng/ml. Over the next 18 months, the dose was adjusted to maintain a slightly lower (60-80 ng/ml) trough level in Group A, while Group B received a fixed dose of 2.5 mg/kg/day. The primary end point was the rate of sustained remission with analysis based on the intention-to-treat principle. After 2 years, the rate of sustained remission was significantly higher while the hazard ratio for relapse was significantly lower in Group A as compared with Group B. Mild arteriolar hyalinosis of the kidney was more frequently seen in Group A than in Group B, but no patient was diagnosed with striped interstitial fibrosis or tubular atrophy. We conclude that cyclosporine given to maintain targeted trough levels is an effective and relatively safe treatment for children with frequently relapsing nephrotic syndrome. [\hyperlink{Cyklokapron}{PMID: 18305467}, K Ishikura et al., 2008]

\hypertarget{pmid_28827252}{C}eftriaxone is widely used in children in the treatment of sepsis. However, concerns have been raised about the safety of ceftriaxone, especially in young children. The aim of this review is to systematically evaluate the safety of ceftriaxone in children of all age groups. MEDLINE, PubMed, Cochrane Central Register of Controlled Trials, EMBASE, CINAHL, International Pharmaceutical Abstracts and adverse drug reaction (ADR) monitoring systems will be systematically searched for randomised controlled trials (RCTs), cohort studies, case-control studies, cross-sectional studies, case series and case reports evaluating the safety of ceftriaxone in children. The Cochrane risk of bias tool, Newcastle-Ottawa and quality assessment tools developed by the National Institutes of Health will be used for quality assessment. Meta-analysis of the incidence of ADRs from RCTs and prospective studies will be done. Subgroup analyses will be performed for age and dosage regimen. Formal ethical approval is not required as no primary data are collected. This systematic review will be disseminated through a peer-reviewed publication and at conference meetings. CRD42017055428. [\hyperlink{Cyklokapron}{PMID: 28827252}, Linan Zeng et al., 2017] the aim of this study was to report single centre experience with cyclosporine used in treatment of children with inflammatory bowel disease with regard to safety and efficacy. retrospective analysis included 23 patients, 21 with ulcerative colitis and 2 with Crohn's disease, aged 2.75 to 18.5 years. They were treated with cyclosporine during the last 5 years. Before cyclosporine therapy was started they received steroids and azathioprine. Cyclosporine treatment was given in severe steroid-resistant exacerbation of the disease (n = 10) or steroid-dependence (n = 13). Cyclosporine dose was set to obtain therapeutic levels (serum concentration > 100 ng/ml and < 200 ng/ml). Cyclosporine treatment was continued up to 2 months in 6 cases, 2 to 6 months in 8 patients and more than 6 months in 9 patients. Complications were reported in 2 patients: hirsutism and gingival hypertrophy. Cyclosporine treatment was stopped in the second case. None of the two patients with Crohn's disease improved during the treatment. Short-term improvement was observed in 11 patients with ulcerative colitis. Long-term recovery (> 6 months) was obtained in 6 cases. In 10 patients with severe exacerbation of ulcerative colitis colectomy was performed, in 4 of them elective surgery was performed when the clinical status improved. cyclosporine appears to be a safe and relatively effective treatment of ulcerative colitis in children. Cyclosporine is less effective in maintaining remission and it did not allow to avoid colectomy in severe exacerbation. Still case controlled studies are needed to show the efficacy of this treatment. [\hyperlink{Cyklokapron}{PMID: 28827252}, Piotr Socha et al., ]

\hypertarget{pmid_8881900}{C}yclosporin has been shown to be effective in the treatment of adult atopic dermatitis, but there are no clinical trials evaluating its use in childhood. Atopic dermatitis is more common in children and the severe form can be associated with considerable morbidity. We report on 18 children with severe refractory atopic dermatitis who have been treated with cyclosporin on an open basis. The drug was given at an initial daily dose of 5 or 6 mg/kg and in some patients the dose was reduced according to response. Sixteen patients showed a good or excellent response to treatment, one a moderate response and one patient failed to improve. The treatment was well tolerated and there were no significant changes in serum creatinine or blood pressure. Long remission after withdrawal of treatment was seen in some patients, although most relapsed within a few weeks. We suggest that cyclosporin is an effective and safe short-term treatment for severe atopic dermatitis in childhood. [\hyperlink{Cyklokapron}{PMID: 8881900}, I Zaki et al., 1996]

\hypertarget{pmid_17133159}{Q}uinolone-induced arthropathic toxicity in weight-bearing joints observed in juvenile animals during preclinical testing has largely restricted the routine use of ciprofloxacin in the pediatric age group. As histopathologic, radiologic and magnetic resonance imaging monitoring evidence has gathered supporting the safety of fluoroquinolones in children, many pediatricians have started to prescribe quinolones to some patients on a compassionate basis. The objective of this study was to ascertain the safety of ciprofloxacin in preterm neonates <33 weeks gestational age treated at Dhaka Shishu (Children) Hospital in Bangladesh. Long-term follow up was done to monitor the growth and development of preterm infants who were administered intravenous ciprofloxacin in the neonatal period. Ciprofloxacin was used only as a life-saving therapy in cases of sepsis produced by bacterial agents resistant to other antibiotics. Another group of preterm neonates with septicemia who were not exposed to ciprofloxacin, but effectively treated with other antibiotics and followed up, were matched with cases for gender, gestational age and birth weight and included as a comparison group. Forty-eight patients in the ciprofloxacin group and 66 patients in the comparison group were followed up for a mean of 24.7 +/- 18.5 months and 21.6 +/- 18.8 months, respectively. No osteoarticular problems or joint deformities were observed in the ciprofloxacin group during treatment or follow up. No differences in growth and development between the groups were found. Ciprofloxacin is a safe therapeutic option for newborns with sepsis produced by multiply resistant organisms. [\hyperlink{Cyklokapron}{PMID: 17133159}, A S M Nawshad Uddin Ahmed et al., 2006] to evaluate the efficiency of the cycloferon (in tablets) in treatment of frequently ill children (FIC) during seasonal acute respiratory infections and estimate its safety for children and adults. Research had open character. Under supervision there were 411 children of different age groups and 74 adults. 250 persons (100 frequently ill children from 4 to 7 years old , 76 - from 7 to 18 and 74 adults) were treated with cycloferon under the standard regimen. Control group included 235 FIC. It was found that the preventive courses of cycloferon administered during seasonal acute respiratory infections significantly reduced number of day offs taken by parents for sick 5 year old and younger FIC. The cycloferon administration in 94,8 \% of cases was not accompanied by pathological symptoms. [\hyperlink{Cyklokapron}{PMID: 17133159}, S Lialikau et al., 2013]

\hypertarget{pmid_8151150}{T}he efficacy and safety of aciclovir granules (containing 40\% w/w aciclovir) were evaluated in the treatment of chickenpox in otherwise healthy children. Patients presenting with chickenpox received aciclovir granules at a dose of 20 mg/kg four times daily for five to seven days. Overall 51 children received treatment with aciclovir. A further 53 patients receiving conventional symptomatic therapy acted as a control. In the aciclovir group the overall efficacy rate was 92.2\%. There were reductions in the numbers of lesions, fever, itching and the duration of symptoms. No adverse experiences were reported. Overall this formulation of aciclovir appears to be a safe and effective treatment for chickenpox in this patient population. However the need for anti-viral therapy in otherwise healthy children is still the subject of debate and it might be appropriate to identify sub-groups for whom such therapy is justified. [\hyperlink{Cyklokapron}{PMID: 8151150}, H Kamiya et al., 1994]

\hypertarget{pmid_15696982}{C}iclopirox is a broad-spectrum antifungal, antibacterial, and anti-inflammatory agent. This open-label study investigated the safety and efficacy of ciclopirox topical suspension 0.77\% in the treatment of diaper dermatitis due to Candida albicans (C. albicans). Forty-four male and female subjects aged 6 to 29 months were included in the study. Study medication was applied topically to the affected diaper area twice daily for 1 week. Subjects were clinically evaluated at baseline and days 3, 7, and 14 (7 days post-treatment). Safety and efficacy variables included adverse events, mycological culture studies, KOH tests, Severity Scores, and Global Evaluation of Clinical Response. All adverse events were mild to moderate and considered not related to the study medication. Treatment provided statistically significant improvement (P < .05) for both the rate of mycological cure and reduction of Severity Score at each time point compared with baseline. Ciclopirox was safe and effective in the treatment of diaper dermatitis due to C. albicans. [\hyperlink{Cyklokapron}{PMID: 15696982}, Elizabeth Gallup et al., ]

\hypertarget{pmid_8132376}{L}ike all fluoroquinolones, ciprofloxacin causes articular damage in juvenile animals. Consequently, this drug was not recommended for children or pregnant women. However, due to its antibacterial effectiveness and convenience of oral administration, ciprofloxacin is now increasingly used for the treatment of certain infectious conditions in children and adolescents aged less than 18 years. In this paper the published literature on this subject is reviewed. Up to now, data are available on more than 1,500 paediatric patients who were given ciprofloxacin, two-thirds of whom were suffering from acute infectious bronchopulmonary exacerbations of cystic fibrosis, mainly due to Pseudomonas aeruginosa. The effectiveness of oral ciprofloxacin for this indication compared well to that of standard intravenous combination regimens. The majority of the remaining published trials was conducted in children with multiresistant typhoid fever; the administration of ciprofloxacin was successful in up to 100\% of the cases. The safety profile of ciprofloxacin in children and adolescents was very similar to that observed in adult patients. Adverse events were noted in 5-15\%, with gastrointestinal, skin and central nervous system reactions being the most common. Reversible arthralgia occurred in 36 out of 1,113 patients with cystic fibrosis, and in no case could cartilage damage be demonstrated by radiographic procedures. Thus, publication data clearly suggest that the administration of ciprofloxacin to children is effective and safe, but there is a need for further prospective, well-controlled clinical trials. [\hyperlink{Cyklokapron}{PMID: 8132376}, R Kubin et al., ]

\hypertarget{pmid_8647967}{S}evere atopic dermatitis (AD) remains difficult to treat. Cyclosporine is effective in adults but has not previously been investigated in children with AD. The aims were to investigate the efficacy, safety, and tolerability of cyclosporine in severe refractory childhood AD. Subjects 2 to 16 years of age were treated for 6 weeks with cyclosporine, 5 mg/kg per day, in an open study. Disease activity was monitored every 2 weeks by means of sign scores, visual analogue scales for symptoms, and quality-of-life questionnaires. Adverse events were monitored. Efficacy and tolerability were assessed with five-point scales. Twenty-seven children were treated. Significant improvements were seen in all measures of disease activity. Twenty-two showed marked improvement or total clearing. Quality of life improved for both the children and their families. Tolerability was considered good or very good in 25 subjects. Cyclosporine may offer an effective, safe, and well-tolerated short-term treatment option for children with severe AD. [\hyperlink{Cyklokapron}{PMID: 8647967}, J Berth-Jones et al., 1996]

\hypertarget{pmid_25059452}{C}yclosporine is a systemic therapy used for control of severe atopic dermatitis (AD) in children. Although traditionally recommended at a dose of 5 mg/kg/day for 6 months, a longer duration of treatment may be necessary to bring a child with active and severe disease into remission. There are few data on the short- and long-term effectiveness of longer courses of therapy. This was a retrospective chart review of children treated with cyclosporine at a Canadian hospital-affiliated clinic between 2000 and 2013. Fifteen patients with adequate follow-up were identified. Twelve (80\%) were male and the mean age at initiation of cyclosporine was 11.2 ± 3.4 years. The mean duration of cyclosporine therapy was 10.9 ± 2.7 months (range 7-15 months) at a starting dose of 2.8 ± 0.6 mg/kg/day. Of 12 patients (80\%) who responded to cyclosporine, 5 patients (42\%) had relapsed at a follow-up of 22.7 ± 15.0 months. The duration of therapy was longer in patients who did not relapse (17.7 ± 10.7 months) than in those who did (10.2 ± 2.7 months) (p = 0.06). Adverse events led to discontinuation in three patients (20\%) and included infection-related complications in two patients and reversible renal toxicity in one. These results suggest that a longer duration of low-dose cyclosporine may help decrease the risk of relapse in patients with severe AD who are resistant to topical therapies. [\hyperlink{Cyklokapron}{PMID: 25059452}, Cathryn Sibbald et al., ]

\hypertarget{pmid_12836096}{I}n a systematic review and meta-analysis of randomized controlled trials (RCT), we aimed to evaluate the benefits and harms of all interventions for children with steroid-resistant nephrotic syndrome (SRNS). Nine RCTs involving 225 children were included. Cyclosporin when compared with placebo or no treatment significantly increased the number of children who achieved complete remission [3 trials, 49 children, relative risk (RR) for persistent nephrotic syndrome 0.64, 95\% confidence intervals (CI), 0.47-0.88]. There was no significant difference in the number of children who achieved complete remission between oral cyclophosphamide with prednisone and prednisone alone [2 trials, 91 children, RR 1.01, 95\% CI 0.74-1.36], between intravenous cyclophosphamide and oral cyclophosphamide [1 study, 11 children, RR 0.09, 95\% CI 0.01-1.39], and between azathioprine with prednisone and prednisone alone [1 trial, 31 children, RR 1.01, 95\% CI 0.77-1.32]. No RCTs were identified comparing combination regimens comprising high-dose steroids, alkylating agents or cyclosporin with single agents, placebo, or no treatment. Further adequately powered and well-designed RCTs are needed to confirm the efficacy of cyclosporin and to evaluate regimens of high-dose steroids with alkylating agents or cyclosporin for SRNS. [\hyperlink{Cyklokapron}{PMID: 12836096}, Doaa Habashy et al., 2003]

\hypertarget{pmid_10917383}{T}o review the pharmacokinetics, efficacy, and safety of fluoroquinolones in children. A MEDLINE search (January 1966-March 1998) was conducted for relevant literature. Data from compassionate use and published studies were reviewed for the assessment of pharmacokinetics, efficacy, and safety of fluoroquinolones in children. Fluoroquinolones have a broad spectrum coverage of gram-positive and gram-negative bacteria, including Pseudomonas aeruginosa and intracellular organisms. Fluoroquinolones are well absorbed from the gastrointestinal tract, have excellent tissue penetration, low protein binding, and long elimination half-lives. These antibiotics are effective in treating various infections and are well tolerated in adults. However, the use of fluoroquinolones in children has been restricted due to potential cartilage damage that occurred in research with immature animals. Fluoroquinolones have been used in children on a compassionate basis. Ciprofloxacin is the most frequently used fluoroquinolone in children, most often in the treatment of pulmonary infection in cystic fibrosis as well as salmonellosis and shigellosis. Other uses include chronic suppurative otitis media, meningitis, septicemia, and urinary tract infection. Safety data of fluoroquinolones in children appear to be similar to those in adults. Fluoroquinolones are associated with tendinitis and reversible arthralgia in adults and children. However, direct association between fluoroquinolones and arthropathy remains uncertain. Fluoroquinolones have been found to be effective in treating certain infections in children. Additional research is needed to define the optimal dosage regimens in pediatric patients. Although fluoroquinolones appear to be well tolerated, further investigations are needed to determine the risk of arthropathy in children. However, their use in children should not be withheld when the benefits outweigh the risks. [\hyperlink{Cyklokapron}{PMID: 10917383}, A A Alghasham et al., 2000]

\hypertarget{pmid_2113819}{A} 4.5 year old boy with cerebral palsy presented with seizures associated with facial flushing and tachycardia following the instillation of 1\% cyclopentolate, a commonly used mydriatic in paediatric practice. He had no prior history of convulsions. This case demonstrates the uncommon, though serious, atropine-like side effect of cyclopentolate eyedrops (Cyclogyl, Alcon) in usual dosage in a brain damaged child without an epileptic focus. [\hyperlink{Cyklokapron}{PMID: 2113819}, D A Fitzgerald et al., 1990]

\hypertarget{pmid_19740527}{E}noxaparin, a low molecular weight heparin (LMWH), is frequently used for the prevention and treatment of thromboembolic complications in infants and children (Sutor et al., 2004 [1]). Injection pain and the fear and anxiety associated with needle phobia in the pediatric population are well documented. Best practice pediatric pain management standards of care recommend mitigating the child's pain experience whenever possible. The use of topical anesthetics such as liposomal-lidocaine 4\% results in a rapid onset of anesthesia, minimal blanching, without vasoconstriction (Koh et al., 2004 [2]) or risk of methemoglobinemia. Topical lidocaine has been used to reduce the injection pain of enoxaparin, but there is no data available examining whether it will interfere with the absorption of LMWH. To determine if the topical lidocaine, Maxilene, interferes with enoxaparin absorption as measured by peak anti-Xa levels. Infants and children clinically prescribed enoxaparin were eligible for this study. Children in group 1 were pre-treated with Maxilene prior to enoxaparin injection on day 1 with no Maxilene pre-treatment on day 2. For group 2, the order was reversed. Peak anti-Xa levels were measured following each enoxaparin dose and were compared between the groups. 26 children of ages 14d-16 y (median 6.7 months) were enrolled. Anti-Xa levels following topical lidocaine administration were 0.070 U/mL (95\%CI 0.025; 0.114) lower than without prior topical lidocaine administration. Anti-Xa levels on the second day were on average 0.013 U/mL (95\%CI -0.066; 0.040) higher compared to day one regardless of the order of topical lidocaine administration. There were no reported incidences of local reactions such as redness, hives or blanching. Topical lidocaine (Maxilene) administration before enoxaparin injection results in a small, clinically non-significant, reduction in anti-Xa levels. [\hyperlink{Cyklokapron}{PMID: 19740527}, S M Duncan et al., 2010]

\hypertarget{pmid_16243226}{T}his study demonstrates the efficacy of cyclosporine included in a regimen for the treatment of steroid-resistant chronic inflammatory demyelinating polyradiculoneuropathy in two children. Clinical response was characterized by either decreased frequency of recurrent weakness or normalized motor function. Nerve conduction studies and monitoring of cyclosporine levels were included in the serial follow-up evaluations, and their results were used in formulating a treatment plan. One of the two children, who had been monitored for 56 months since the onset of the disease, was able to maintain normal muscle strength without recurrent weakness for 39 months, with 5 mg/kg daily of cyclosporine. The other child, who had been taking prednisolone 0.3 mg/kg daily and cyclosporine 5 mg/kg daily, regained ambulation without support while demonstrating a reduction of recurrent weakness. None had adverse effects caused by cyclosporine therapy. We conclude that cyclosporine is an effective drug in the treatment of children with steroid-resistant chronic inflammatory demyelinating polyradiculoneuropathy. [\hyperlink{Cyklokapron}{PMID: 16243226}, Anannit Visudtibhan et al., 2005]

\hypertarget{pmid_8545564}{W}e evaluated safety and tolerance of acyclovir ACV per os in immunocompetent children affected by chicken-pox admitted to our department from January 1993 to December 1994. 183 subjects (102 males and 81 females) aged between 0 and 14 years were treated by ACV (80 mg/kg/daily in 4 divided doses): 88 children were treated within 24 hours and 95 subjects within 48 hours from the onset of symptoms. The control group consisted of 83 children (52 males and 31 females) aged between 0 to 14 years. In all patients routine blood-test were performed and in those with respiratory illness Chest-Rx was also done. We evaluated clinical course, degree of eruption, the appearance and kind of complications, duration of hospitalization, the compliance and the potential consequences on specific antibody response. Our results show a faster improvement of clinical symptoms in treated patients with respect to the control group with shortening of the period of the fever, itch and appearance of new vescicles. The percentage of complications was lower in treated than in untreated patients. 16 cases tested for specific antibody response showed protective titers six months after treatment. In conclusion, ACV administered per os within 48 hours from onset of exanthema causes reduction of the period and the degree of general symptoms and exanthema, a lower incidence of complications even if non statistically significant. The drug is safe and well-tolerated. [\hyperlink{Cyklokapron}{PMID: 8545564}, S Catania et al., ]

\hypertarget{pmid_12775319}{C}yclosporin is known to be highly effective in the treatment of psoriasis in adults. It has also proved effective and well tolerated in the treatment of severe childhood atopic dermatitis. Psoriasis in childhood is relatively unusual but by no means rare and on occasions the disease can be very difficult to control in this age group. The use of cyclosporin for psoriasis in childhood has received scarcely any attention and in the few cases that have been reported the results have been inconsistent. Three children aged from 7 to 11 years with severe psoriasis resistant to topical agents were treated with cyclosporin. The highest dose required was 3.5 mg/kg per day. The duration of treatment ranged from 6 weeks to 4 months. Cyclosporin was effective and generally well tolerated. Treatment was interrupted in one case due to nausea and diarrhoea. None of the patients developed hypertension or renal impairment. The potential role of cyclosporin in severe childhood psoriasis is discussed. [\hyperlink{Cyklokapron}{PMID: 12775319}, C M Perrett et al., 2003]

\hypertarget{pmid_3880820}{C}yclosporine and prednisone were used in combination to produce immunosuppression in 18 pediatric recipients of renal allografts. Ten children received cadaveric kidneys and eight received kidneys from living related donors. With a mean follow-up of 16.5 months (range 7 to 33 months), the patient survival rate is 100\% (18 of 18) and the graft survival rate is 83\% (15 of 18). Two grafts were lost for nonimmunologic reasons. Currently the group mean (+/- SE) serum creatinine concentration is 1.22 +/- 0.11 mg/dl and creatinine clearance is 69.3 +/- 4.79 ml/min/1.73 m2. Cyclosporine nephrotoxicity has not caused irreversible allograft injury nor led to graft loss in this population. The incidence of treated rejection episodes has been 39\% (seven of 18). Only 39\% (seven of 18) of children have required hospital readmissions since the initial transplant discharge. There have been no opportunistic infections. In the 15 children with functioning grafts, some linear growth has occurred in 10 of 11 prepubertal and two of four postpubertal patients. Cyclosporine and prednisone have constituted a safe, efficacious immunosuppressive regimen for pediatric renal allograft recipients. Longer follow-up will be necessary to confirm whether these advantages persist beyond 2 years. [\hyperlink{Cyklokapron}{PMID: 3880820}, S B Conley et al., 1985]

\hypertarget{pmid_20415263}{T}he results of the multicentre clinical trials on cycloferon efficacy in children at the age from 4 to 16 years are presented. The prophylactic effect of the drug (2.9-7.2-fold decrease of the morbidity) with respect to the respiratory tract mono- and mixed infections was showen. The marked cytoprotective effect, evident from lower destruction of the epithelial cells and increased activity of the local nonspecific resistance factors (lysozyme, secretory immunoglobulin A) was observed. [\hyperlink{Cyklokapron}{PMID: 20415263}, M G Romantsov et al., 2009]

\hypertarget{pmid_32165600}{T}here are some randomized trials which have already evaluated different calcineurin inhibitors (CNIs), especially comparing Tacrolimus and Cyclosporine, as immunosuppressant agents in children. However, their findings have been occasionally conflicting and thus debatable. Therefore, the evidence on safety and efficacy of immunosuppressive therapy after kidney transplantation in children has been inconclusive and argued to date. This study was aimed to compare the benefits and disadvantages of tacrolimus versus cyclosporine as the primary immunosuppression after renal transplantation in children. A systematic review and meta-analysis was done. An electronic literature review was conducted to identify appropriate studies. The outcomes were presented as relative risk, with 95\% confidence intervals. Five qualified randomized controlled trials were included in this systematic review. Tacrolimus was insignificantly superior to cyclosporine considering the total effect size of graft loss (RR = 0.67, 95\% CI: 0.40 - 1.11; P > .05) and acute rejection (RR = 0.79, 95\% CI: 0.59 - 1.05; P > .05). On the contrary, cyclosporine seemed to be insignificantly superior to tacrolimus regarding mortality rate (RR = 1.06, 95\% CI: 0.59 - 1.90; P > .05). Admitting the study limitations mainly because of the nature and case study size of the included trials, it can be concluded from our systematic review results that Tacrolimus seems insignificantly superior to Cyclosporine respecting graft loss and acute rejection. However, Cyclosporine was shown to be insignificantly superior regarding mortality rate. However additional studies with a larger sample size are highly recommended. [\hyperlink{Cyklokapron}{PMID: 32165600}, Yalda Ravanshad et al., 2020]

\hypertarget{pmid_2292542}{T}he Cystic Fibrosis Clinic at the Royal Belfast Hospital for Sick Children has treated 31 children with ciprofloxacin, for serious pseudomonas infection in cystic fibrosis, and carefully monitored the safety and acceptability of the drug. Initially, eight very ill children were treated on a named-patient basis, with an encouraging clinical response and few adverse effects. Children aged 10-18 years were included in a study of four consecutive exacerbations of respiratory disease, comparing (i) oral ciprofloxacin in each episode with (ii) ciprofloxacin alternating with intravenous azlocillin and tobramycin. Other children with cystic fibrosis were subsequently treated with ciprofloxacin, as the need arose. In all the groups very few adverse reactions were found; in particular only one child developed arthralgia. A total of 202 children in the UK have been treated with ciprofloxacin on a named-patient basis, and their clinicians have reported 46 adverse events that may have been drug-related. Overall ciprofloxacin appears to be safe and effective in children but concern about the possible occurrence of arthropathy remains and long term follow-up of these children may be necessary. [\hyperlink{Cyklokapron}{PMID: 2292542}, A Black et al., 1990]

\section*{Daunorubicin Hydrochloride}
\subsection*{Result}
\subsubsection*{Answer}

Infants (<1 year): Yes  
Children (1-16 years): Yes  

\subsubsection*{{Explanation}}
\hypertarget{Daunorubicin Hydrochloride}
A review of the available abstracts reveals multiple targeted studies evaluating the safety of Daunorubicin Hydrochloride (including liposomal and standard formulations) in children with cancer, particularly acute lymphoblastic leukemia (ALL) and brain tumors. The evidence is summarized below by age range:

Infants (<1 year):
- One pharmacokinetic study included infants as young as 0.05 years (about 18 days) up to 1.88 years, with dose adjustments for those <6 months and 6-12 months. The study found no age-dependency in pharmacokinetics, and while infants <6 months experienced more infections during induction, other toxicities were similar to older children. No specific safety concerns unique to infants were reported, but the sample size was small and the focus was on pharmacokinetics and general toxicity, not comprehensive safety outcomes [\hyperlink{pmid_19731319}{PMID: 19731319}, Georg Hempel et al., 2010].

Children (1-16 years):
- Multiple studies specifically evaluated the safety and toxicity of Daunorubicin in children with ALL and other malignancies:
    - A randomized study in children with ALL (age not specified, but pediatric) compared 1-hour vs. 24-hour infusions and found no difference in efficacy or major safety signals, suggesting both regimens are tolerable [\hyperlink{pmid_17525906}{PMID: 17525906}, G Escherich et al.].
    - A prospective cohort of 110 children (1 month-16 years) found a high incidence of anthracycline-induced cardiotoxicity (14\% acute, 25\% within a year), with some deaths due to severe cardiac dysfunction. Cardiotoxicity was dose-related and higher with combination therapy (doxorubicin + daunorubicin) [\hyperlink{pmid_24259388}{PMID: 24259388}, Abdul Sattar Shaikh et al., 2013].
    - Another study in 14 children (median age 5, range 3-12) with brain tumors treated with liposomal daunorubicin found mild bone marrow depression and some cardiotoxicity (13.8\% decrease in left ventricular function for the group; one child stopped due to significant cardiac dysfunction). The drug was described as "well-tolerated and effective" at the studied dose [\hyperlink{pmid_10100273}{PMID: 10100273}, R J Lippens et al.].
    - A long-term follow-up (median 22 years) of ALL survivors treated with low-dose daunorubicin (4 x 25 mg/m\textasciicircum{}2) found no clinical cardiac abnormalities after more than 20 years [\hyperlink{pmid_18648960}{PMID: 18648960}, C A J Brouwer et al., 2007].
    - Several studies confirm that cardiotoxicity is a known, dose-dependent risk in children, and that monitoring is essential. Some studies suggest daunorubicin may be less cardiotoxic than doxorubicin [\hyperlink{pmid_26304888}{PMID: 26304888}, Elizabeth A M Feijen et al., 2015; \hyperlink{pmid_27858183}{PMID: 27858183}, Hiroki Hori et al., 2017].

Summary:
- Daunorubicin Hydrochloride has been studied in children from infancy through adolescence, primarily in the context of cancer treatment.
- The studies affirm that it is used in children and that its safety profile is well-characterized: the main risk is dose-dependent cardiotoxicity, which can be severe and even fatal in some cases, but is generally manageable with appropriate monitoring and dose limitations.
- There is no evidence from these abstracts that Daunorubicin is categorically unsafe in children; rather, it is considered safe for use with careful monitoring and dose management, as is standard in pediatric oncology.
- For infants (<1 year), data are more limited, but no unique safety signals were identified in the available pharmacokinetic/toxicity study.
- For children (1-16 years), multiple studies affirm its use and safety profile, with the caveat of known cardiotoxicity risk.

Therefore, based on the abstracts, Daunorubicin Hydrochloride is considered safe for use in children (with the standard oncologic precautions), but with a clear risk of cardiotoxicity that requires monitoring.

\subsection*{Abstracts}
\hypertarget{pmid_10100273}{L}iposomal daunorubicin (DaunoXome = DNX) has been used in 14 children with recurrent or progressive growing brain tumor. DNX was given as a 1-h intravenous infusion with a dose of 60 mg/m2, once every 4 weeks, up to a cumulative dose of 600 mg/m2. At 3-month intervals the tumor process was evaluated on MRI or CT scan. Tumor response and toxicity of DNX were recorded according to the WHO guidelines. In 6 of the children a response has been established: 2 had complete responses, of which one relapsed again after 3 months; in 3 children a partial response was found. Two children showed stable disease. In 6 children the tumors grew progressively. In all responding children a remarkable subjective response was found. The toxicity of DNX at this dose was mild with a mild bone marrow depression and a slight but certain cardiotoxicity in 3 children. For the whole group the left ventricular function decreased with 13.8\%. In 1 child the DNX treatment was stopped because of a decrease of the shortening fraction to 20\%. In 4 children some hair loss was observed at the end of the treatment. In 3 children mental depression occurred that was associated with the administration of DNX. DNX is a well-tolerated and effective drug in the treatment of slowly progressive or recurrent brain tumors in children. [\hyperlink{Daunorubicin Hydrochloride}{PMID: 10100273}, R J Lippens et al., ]

\hypertarget{pmid_23146307}{D}aunorubicin is a chemotherapeutic antibiotic of the anthracycline family used for the treatment of many type of cancers when doxorubicin or other less effective drugs cannot be used. The aim of the present study was labeling of Daunorubicin with (99m)Tc, quality control, characterization, and biodistribution of radiolabeled Daunorubicin. Labeling efficiency was determined by ascending paper chromatography. All the experiments were performed at room temperature (25°C±2°C). More than 96\% labeling efficiency with (99m)Tc was achieved at pH 5-6, 2-4 μg stannous chloride and 300 μg of ligand in few minutes. The characterization of the compound was performed by using HPLC, electrophoresis and shake flask assay. Electrophoresis indicates that Tc-99m-Daunorubicin is neutral, HPLC confirms the single specie of the labeled compound, while shake flask assay confirms high lipophilicity. The biodistribution studies of (99m)Tc-Daunorubicin were performed in rats. Significantly higher accumulation of (99m)Tc-Daunorubicin was seen in brain of normal rats. Scintigraphy was also indicating higher accumulation of (99m)Tc-Daunorubicin in brain of normal rabbits. [\hyperlink{Daunorubicin Hydrochloride}{PMID: 23146307}, A R Faheem et al., 2013]

\hypertarget{pmid_7758517}{D}oxorubicin and daunorubicin are effective anticancer agents in children, however, their therapeutic value is limited by myocardial cardiotoxicity. In 14 children (median age 5.0 years, range 3-12) prospective studies were performed using pulsed Doppler echocardiography to assess the changes in left ventricular systolic and diastolic filling dynamics. None of these children developed cardiomyopathy. M-mode echocardiographic systolic parameters and Doppler transmitral flow velocities were analysed at baseline, after a cumulative anthracycline dose of 138 +/- 26 mg/m2 (second examination) and after 240 +/- 15 mg/m2 (third examination). At the second examination the acceleration time/ejection time ratio was significantly reduced (P < 0.01), but this was no longer evident at the third examination. There was no significant change of peak velocity over aortic valve, pre-ejection period and change of velocity over time. In contrast, three diastolic parameters changed significantly; the late over early inflow velocity (P < 0.05), mitral valve late time velocity integral (P < 0.01 at the second and P < 0.05 at the third examination) and the ratio A-TVI/TVI (P < 0.025 and P < 0.01). At the third examination the velocity of the A wave was also significantly increased. CONCLUSION In anthracycline treated children left ventricular diastolic function deteriorates before systolic function. Diastolic function parameters should be used rather than systolic parameters to monitor these patients. [\hyperlink{Daunorubicin Hydrochloride}{PMID: 7758517}, K Schmitt et al., 1995]

\hypertarget{pmid_17525906}{D}aunorubicin (DNR) is one of the most important drugs in treatment of acute lymphoblastic leukemia (ALL). Prolonged infusions of anthracyclines are less cardiotoxic but it has not been investigated whether the in vivo leukemic cell kill is equivalent to short-term infusions. In the cooperative treatment study COALL-92 for childhood ALL 178 patients were randomized to receive in a therapeutic window a single dose of 36 mg/m (2) DNR either as a 1-h (85 patients) or 24-h infusion (93 patients). Daily measurements of white blood cell count (WBC) and peripheral blood smears for seven days could be evaluated centrally in 101 patients (1-h: 43 patients, 24-h: 58 patients). The proportional decline of blasts at day 7 after DNR infusion showed no statistically significant difference between the two treatment arms. At day 3 the median percentage of blasts was less than 10\%, at day 7 less than 2\% for either the 1-h or 24-h infusion. Twelve patients (1-h: 5 patients, 24-h: 7 patients) had an absolute number of more than 1000 blasts per mul peripheral blood (PB) at day 7 after DNR infusion (DNR poor responders). Kaplan-Meier analysis showed an equal probability of EFS for the short- and long-term infusion group (24-h: 83\%+/-5; 1-h: 81+/-6) after a median observation time of 12.3 years. We conclude that in children with ALL a 24-h infusion of DNR has the same in vivo cytotoxicity for leukemic cells as a 1-h infusion. This offers the possibility to use prolonged infusions with hopefully less cardiotoxicity without loss of efficacy. [\hyperlink{Daunorubicin Hydrochloride}{PMID: 17525906}, G Escherich et al., ]

\hypertarget{pmid_24259388}{T}o identify anthracycline-induced acute (within 1 month) and early-onset chronic progressive (within 1 year) cardiotoxicity in children younger than 16 years of age with childhood malignancies at a tertiary care centre of Pakistan. Prospective cohort study. Aga Khan University, Karachi, Pakistan. 110 children (aged 1 month-16 years). Anthracycline (doxorubicin and/or daunorubicin). All children who received anthracycline as chemotherapy and three echocardiographic evaluations (baseline, 1 month and 1 year) between July 2010 and June 2012 were prospectively analysed for cardiac dysfunction. Statistical analysis including systolic and diastolic functions at baseline, 1 month and 1 year was carried out by repeated measures analysis of variance. Mean age was 74±44 months and 75 (68.2\%) were males. Acute lymphoblastic leukaemia was seen in 70 (64\%) patients. Doxorubicin alone was used in 59 (54\%) and combination therapy was used in 35 (32\%). A cumulative dose of anthracycline <300 mg/m(2) was used in 95 (86\%). Fifteen (14\%) children developed cardiac dysfunction within a month and 28 (25\%) children within a year. Of these 10/15 (66.6\%) and 12/28 (43\%) had isolated diastolic dysfunction, respectively, while 5/15 (33.3\%) and 16/28 (57\%) had combined systolic and diastolic dysfunction. Seven (6.4\%) patients expired due to severe cardiac dysfunction. Eight of 59 (13.5\%) children showed dose-related cardiotoxicity (p=<0.001). Cardiotoxicity was also high when the combination of doxorubicin and daunorubicin was used (p=0.004). Incidence of anthracycline-induced cardiotoxicity is high. Long-term follow-up is essential to diagnose its late manifestations. [\hyperlink{Daunorubicin Hydrochloride}{PMID: 24259388}, Abdul Sattar Shaikh et al., 2013]

\hypertarget{pmid_26304888}{C}umulative anthracycline dose is one of the strongest predictors of heart failure (HF) after cancer treatment. However, the differential risk for cardiotoxicity between daunorubicin and doxorubicin has not been rigorously evaluated among survivors of childhood cancer. These risks, which are based on hematologic toxicity, are currently assumed to be approximately equivalent. Data from 15,815 survivors of childhood cancer who survived at least 5 years were used. Survivors were from the Emma Children's Hospital/Academic Medical Center (n = 1,349), the National Wilms Tumor Study (n = 364), the St Jude Lifetime Cohort Study (n = 1,695), and the Childhood Cancer Survivor Study (n = 12,407). The hazard ratio (HR) for clinical HF through age 40 years for doses of daunorubicin and doxorubicin (per 100-mg/m(2) increments) was estimated by using Cox regression adjusted for sex, age at diagnosis, treatment with other anthracycline agents and chest radiation, and cohort membership. In total, 5,144 (32.5\%) patients received doxorubicin as part of their cancer treatment, whereas 2,243 (14.7\%) received daunorubicin. On the basis of 271 occurrences of HF during a median follow-up time after cohort entry of 17.3 years (range, 0.0 to 35.0 years), the cumulative incidence of HF at age 40 years was 3.2\% (95\% CI, 2.8\% to 3.7\%). The average ratio of HRs for daunorubicin to doxorubicin was 0.45 (95\% CI, 0.23 to 0.73). A similar ratio was obtained by using a linear dose-response model, which yielded an HR of 0.49 (95\% CI, 0.28 to 0.70). Compared with doxorubicin, daunorubicin was less cardiotoxic among survivors of childhood cancer than most current guidelines suggest. This may have implications for follow-up guidelines. The feasibility of substitution of doxorubicin with daunorubicin in childhood cancer treatment protocols to reduce cardiotoxicity should be additionally investigated. [\hyperlink{Daunorubicin Hydrochloride}{PMID: 26304888}, Elizabeth A M Feijen et al., 2015]

\hypertarget{pmid_19731319}{T}here is an extreme paucity of pharmacokinetic data for anticancer agents in infants. Therefore, we aimed at characterizing the pharmacokinetics for daunorubicin in infants and examined their relationship to age, body weight, and body surface area. Leukemia patients treated according to the Interfant 99 protocol received 30 mg/m(2) daunorubicin, with dose reduction to 3/4 for patients 6-12 months old and 2/3 for patients <6 months, respectively. Plasma samples from 21 patients (aged 0.05-1.88 years) were collected and analyzed for daunorubicin and daunorubicinol. Samples from 12 children (age 1.6-18.8 years), who received daunorubicin in an earlier investigation, were used for pharmacokinetic model building using the software NONMEM. Plasma concentration time profiles could be described using a two compartment model. Daunorubicin clearance was 43.9 L hr(-1) m(-2) +/- 65\% and central volume of distribution 16.4 L m(-2) +/- 46\%, whereas apparent clearance of daunorubicinol was 19.1 L hr(-1) m(-2) +/- 32\% and apparent volume of distribution 228 L m(-2) +/- 80\% (mean +/- interindividual variability). No age-dependency in any of the BSA-normalized pharmacokinetic parameters was observed. Consequently, due to the empirical dose reduction in infants the overall exposure to daunorubicinol in infants was smaller than would be expected from older children. Patients aged <6 months experienced more infections in the induction phase than the group aged 6-12 months at diagnosis. Other toxicities were similar in both groups. We observed no indication of an age-dependency in the pharmacokinetics of daunorubicin. Pediatr Blood Cancer 2010;54:355-360. [\hyperlink{Daunorubicin Hydrochloride}{PMID: 19731319}, Georg Hempel et al., 2010]

\hypertarget{pmid_19555570}{T}o study relationship between daunorubicin (DNR) pharmacokinetics and efficacy and toxicity in children with acute leukemia. (1) The concentration of DNR in plasma of children with acute leukemia was determined by high performance liquid chromatography (HPLC)-fluorescence detection method. Plasma was sampled frequently from the start of the infusion till the end of 24 h. DNR pharmacokinetics was studied by determination of the concentrations. (2) Efficacy and toxicity were monitored in each period after chemotherapy. Laboratory studies included examination of bone marrow, white blood cell count, electrocardiogram, echocardiogram, myocardial enzymogram, the liver and kidney function. (1) DNR was eliminated from plasma in a two-compartment manner. The maximum concentration was seen 1 - 3 h after infusion. Cmax was 63.50 microg/L. Tmax was 1.45 h. The concentration decreased quickly to a low level of about 11.52 microg/L from the end of 2 hours infusion. There was a large inter-individual difference in pharmacokinetic parameters of DNR. The difference of CL was 9-fold, AUC was 8-fold, Cmax was 5-fold. (2) CL of male patients [57.99 L/(h.m(2))] was significantly lower than that of female patients [93.71 L/(h.m(2))] (P < 0.05). Tmax of children older than 6 years was 1.1 h, and that of children younger than 6 years was 1.8 h (P < 0.05); Cmax of children older than 6 years was 90.77 microg/L, younger than 6 years was 57.44 microg/L (P < 0.05). (1) There is a large inter-individual difference in pharmacokinetic parameters of DNR in children. It may predict individual variety of efficacy and toxicity. Therapeutic drug monitoring is important. (2) Male patients and children older than 6 years had a higher bioavailability and lower metabolism, toxicity may easily occur in such children, therefore they may need lower dose. [\hyperlink{Daunorubicin Hydrochloride}{PMID: 19555570}, Yi-Na Sun et al., 2009]

\hypertarget{pmid_7779709}{D}aunorubicin (DNR) is a major front-line drug in the treatment of childhood acute lymphoblastic leukaemia (ALL). Previously, we showed that in vitro resistance to DNR at diagnosis is related to a poor long-term clinical outcome in childhood ALL and that relapsed ALL samples are more resistant to DNR than untreated ALL samples. In cell line studies, idarubicin (IDR), aclarubicin (ACR) and mitoxantrone (MIT) showed a (partial) lack of cross-resistance to the conventional anthracyclines DNR and doxorubicin (DOX), but clinical studies in childhood ALL have been inconclusive about the suggested lack of cross-resistance. In the present study we determined the in vitro cross-resistance pattern between DNR, DOX, IDR, ACR and MIT in 48 untreated and 39 relapsed samples from children with ALL using the MTT assay. The relapsed ALL group was about twice as resistant to DNR, DOX, IDR, ACR and MTT as the untreated ALL group. Thus, resistance developed to all five drugs. We found a significant cross-resistance between DNR, DOX, IDR, ACR and MIT, although in some individual cases in vitro anthracycline cross-resistance was less pronounced. We conclude that IDR, ACR and MIT cannot circumvent in vitro resistance to DNR in childhood ALL. Clinical studies may still prove whether IDR, ACR or MIT has a more favourable toxicity profile than DNR. [\hyperlink{Daunorubicin Hydrochloride}{PMID: 7779709}, E Klumper et al., 1995]

\hypertarget{pmid_29794839}{E}vidence supports a significant reduction in the incidence of intraventricular hemorrhage (IVH) in preterm infants receiving delayed umbilical cord clamping (DCC). This study evaluated clinical feasibility, efficacy, and safety outcomes in preterm infants (<36 weeks' gestational age) who received DCC following a practice change implementation intended to reduce the incidence of IVH. Infants receiving DCC (45-60 seconds) were compared with a sample of infants receiving immediate umbilical cord clamping (<15 seconds) in a retrospective chart review (N = 354). The primary outcome measure was the prevalence of IVH. Secondary safety outcome measures of 1- and 5-minute Apgar scores, axillary temperature on neonatal intensive care unit admission, and initial 24-hour bilirubin level were also evaluated. Gestational age was examined for its effect on outcomes. Although the small number of infants with IVH precluded the ability to detect statistical significance, our raw data suggest DCC is efficacious in reducing the risk for IVH. For infants 29 or less weeks' gestational age, admission axillary temperature was significantly higher in those who received DCC. No differences were found in 1- and 5-minute Apgar scores, 24-hour bilirubin level, or hematocrit level between the two groups. Infants more than 29 weeks' gestational age who received DCC had significantly higher 1-minute Apgar scores, temperature, and 24-hour bilirubin level. Clinicians should advocate for the implementation of DCC as part of the resuscitative process for preterm neonates. Future studies are needed to evaluate the effect of DCC on other clinical outcomes and to investigate umbilical cord milking as an alternative approach to DCC. [\hyperlink{Daunorubicin Hydrochloride}{PMID: 29794839}, Christen Fenton et al., 2018]

\hypertarget{pmid_2522789}{T}he neuromuscular and cardiovascular effects of doxacurium chloride (BW A938U) were evaluated in 27 children (2-12 yr) anaesthetized with 1\% halothane and nitrous oxide in oxygen. In nine children the incremental technique was used to establish a cumulative dose-response curve by train-of-four stimulation. The remaining children received either 30 or 50 micrograms kg-1 of the drug as a single bolus. The median ED50 and ED95 of doxacurium in children were 19 and 32 micrograms kg-1, respectively. No clinically significant change in heart rate or arterial pressure occurred. Following doxacurium 30 micrograms kg-1 and 50 micrograms kg-1, recovery to 25\% of control occurred in 25 (SEM 6) and 44 (3) min, respectively. The recovery index (25-75\% of control) was 27 (2) min. The duration of action of doxacurium is similar to that of tubocurarine and dimethyl-tubocurarine in children. Compared with adults, children seem to require more doxacurium (microgram kg-1) to achieve a comparable degree of neuromuscular depression, and they recover more rapidly. [\hyperlink{Daunorubicin Hydrochloride}{PMID: 2522789}, N G Goudsouzian et al., 1989]

\hypertarget{pmid_24211979}{D}oxorubicin hydrochloride is widely used to treat various types of cancers. Its therapeutic and side effects are well documented. However, the developmental toxicity of doxorubicin has not been previously described. Lethal and sublethal effects on embryo-larval stages of zebrafish in a study of the developmental toxicity of doxorubicin were observed. Zebrafish embryos were exposed to different concentrations (0-100 mg/L) of doxorubicin between 4 and 120 h post fertilization, and zebrafish larvae were exposed to different concentrations (0-200 mg/L) of doxorubicin for 96 h. The markers about the development toxicity of doxorubicin in zebrafish were observed under a stereomicroscope. Higher doxorubicin concentrations mainly caused acute lethal effects, and lower doxorubicin concentrations mainly caused sublethal effects, such as multiple malformations in embryos and larvae. Moreover, with the increase of doxorubicin concentration, the malformation rate increased. The heart rate of embryos was accelerated at lower concentrations of doxorubicin (≤ 10 mg/L) and decelerated at higher concentrations (≥ 25 mg/L). The hatching rate and body length were inhibited at higher concentrations of doxorubicin (≥ 25 mg/L).In conclusion, doxorubicin has serious developmental toxicity and this raises a concern for developmental effects of doxorubicin in clinical practice.  [\hyperlink{Daunorubicin Hydrochloride}{PMID: 24211979}, Can Chang et al., 2014] The anthracyclines daunorubicin (DNR) and doxorubicin (DOX) are among the most important drugs in the treatment of childhood acute lymphoblastic leukemia, however there are conflicting in vitro data about the comparative efficacy and equivalent doses of both anthracyclines. To address the question of in vivo efficacy of both anthracyclines, patients enrolled in the CoALL 07-03 trial were randomized to receive one single dose of either doxorubicin 30 mg/m(2) , daunorubicin 30 mg/m(2) , or daunorubicin 40 mg/m(2) upfront induction therapy. Children with newly diagnosed B-Precursor ALL or T-ALL were eligible for the randomized comparison. From the percentage of blasts and the white blood cell count (WBC) the absolute number of leukemic cells per µl peripheral blood (PB) was calculated and the initial value before DOX/DNR infusion equated as 100\%. Main target criterion of this study was the leukemic cell decrease from Day 0 to Day 7. Seven hundred forty three patients were randomized: 247 to the DOX; 252 to the DNR 30 mg/m(2) ; and DNR to the 40 mg/m(2) arm. The in vivo response was similar in all three treatment arms with a comparable blast decline in the peripheral blood. The percentages of patients with a clear non-response (M3 marrow) and moreover, the level of minimal residual disease (MRD) on Day 15 or at the end of induction were similar. In vivo efficacy of a single dose daunorubicin 30 or 40 mg/m(2) is similar to that of doxorubicin given in a dose of 30 mg/m(2) . [\hyperlink{Daunorubicin Hydrochloride}{PMID: 24211979}, Gabriele Escherich et al., 2013]

\hypertarget{pmid_34048168}{D}aunorubicine, a type of anthracycline, is a drug commonly used in cancer chemotherapy that increases survival rate but consequently compromises with cardiovascular outcomes in some patients. Thus, preventing the early progression of cardiotoxicity is important to improve the treatment outcome in childhood acute lymhoblastic leukemia (ALL). The present study aimed to identify the risk factors in anthracycline-induced early cardiotoxicity in childhood ALL. This retrospective study was conducted by observing ALL-diagnosed children from 2014 to 2019 in Dr. Soetomo General Hospital. There were 49 patients who met the inclusion criteria and were treated with chemotherapy using Indonesian Childhood ALL Protocol 2013. Echocardiography was performed by pediatric cardiologists to compare before and at any given time after anthracycline therapy. Early cardiotoxicity was defined as a decline of left ventricle ejection fraction (LVEF) greater than 10\% with a final LVEF < 53\% during the first year of anthracycline administration.  Risk factors such as sex, age, risk stratification group, and cumulative dose were identified by using multiple logistic regression. Diagnostic performance of cumulative anthracycline dose was evaluated by receiver operating characteristic (ROC) curve. Early anthracycline-induced cardiotoxicity was observed in 5 out of 49 patients. The median cumulative dose of anthracycline was 143.69±72.68 mg/m2. Thirty-three patients experienced a decreasing LVEF. The factors associated with early cardiomyopathy were age of ≥ 4 years (PR= 1.128; 95\% CI: 1.015-1.254; p= 0.001), high risk group (PR= 1.135; 95\% CI: 1.016-1.269; p= 0.001), and cumulative dose of ≥120 mg / m2 (CI= 1.161; 95\% CI:1.019-1.332). Age of ≥ 4 years, risk group, and cumulative dose of ≥120 mg/m2 are significant risk factors for early cardiomyopathy in childhood ALL. [\hyperlink{Daunorubicin Hydrochloride}{PMID: 34048168}, Sunny Mariana Samosir et al., 2021]

\hypertarget{pmid_30231396}{A}nthracyclines (doxorubicin, daunorubicin, epirubicin, and idarubicin) are among the most potent chemotherapeutic agents and have truly revolutionized the management of childhood cancer. They form the backbone of chemotherapy regimens used to treat childhood acute lymphoblastic leukemia, acute myeloid leukemia, Hodgkin lymphoma, Ewing sarcoma, osteosarcoma, and neuroblastoma. More than 50\% of children with cancer are treated with anthracyclines. The clinical utility of anthracyclines is compromised by dose-dependent cardiotoxicity, manifesting initially as asymptomatic cardiac dysfunction and evolving irreversibly to congestive heart failure. Childhood cancer survivors are at a five- to 15-fold increased risk for congestive heart failure compared with the general population. Once diagnosed with congestive heart failure, the 5-year survival rate is less than 50\%. Prediction models have been developed for childhood cancer survivors (i.e., after exposure to anthracyclines) to identify those at increased risk for cardiotoxicity. Studies are currently under way to test risk-reducing strategies. There remains a critical need to identify patients with childhood cancer at diagnosis (i.e., prior to anthracycline exposure) such that noncardiotoxic therapies can be contemplated. [\hyperlink{Daunorubicin Hydrochloride}{PMID: 30231396}, Saro Armenian et al., 2018]

\hypertarget{pmid_17242696}{T}his review systematically assessed the evidence on the clinical and cost-effectiveness of cardioprotection against the toxic effects of anthracyclines given to children with cancer. We searched eight electronic databases, including Medline and the Cochrane Library, from inception to January 2006 for systematic reviews and randomised controlled trials that reported death, heart failure, arrhythmias or measures of cardiac performance associated with cardioprotective technologies compared with standard treatment in children treated for cancer with anthracyclines. Economic evaluations were also sought. Inclusion criteria, data extraction and quality assessment were undertaken by standard methodology. Four randomised controlled trials met the inclusion criteria of the review; each had methodological limitations. No economic evaluations were identified. Studies were combined through narrative synthesis. One trial found that continuous infusion of doxorubicin did not offer any cardioprotection over rapid infusion. One suggested that continuous infusion of daunorubicin provoked less cardiotoxicity than rapid infusion. One concluded that dexrazoxane reduces cardiac injury during doxorubicin therapy and one reported a protective effect of coenzyme Q(10) on cardiac function during anthracycline therapy. The evidence on the effectiveness of cardioprotective technologies in children is limited in quality and quantity thus making conclusions difficult. This is surprising given the importance of anthracycline use in children with cancer. Further long-term research, which includes relevant outcome measures, is needed to determine whether technologies influence the development of cardiac damage without limiting the antitumour efficacy of anthracyclines. [\hyperlink{Daunorubicin Hydrochloride}{PMID: 17242696}, J Bryant et al., 2007]

\hypertarget{pmid_36174614}{S}urvivors of childhood cancer are at risk of anthracycline-induced cardiotoxicity, which might be prevented by dexrazoxane. However, concerns exist about the safety of dexrazoxane, and little guidance is available on its use in children. To facilitate global consensus, a working group within the International Late Effects of Childhood Cancer Guideline Harmonization Group reviewed the existing literature and used evidence-based methodology to develop a guideline for dexrazoxane administration in children with cancer who are expected to receive anthracyclines. Recommendations were made in consideration of evidence supporting the balance of potential benefits and harms, and clinical judgement by the expert panel. Given the dose-dependent risk of anthracycline-induced cardiotoxicity, we concluded that the benefits of dexrazoxane probably outweigh the risk of subsequent neoplasms when the cumulative doxorubicin or equivalent dose is at least 250 mg/m [\hyperlink{Daunorubicin Hydrochloride}{PMID: 36174614}, Esmée C de Baat et al., 2022] The killed oral cholera vaccine Dukoral is recommended for adults and only children over 2 years of age, although cholera is seen frequently in younger children and there is an urgent need for a vaccine for them. Since decreased immunogenicity of oral vaccines in children in developing countries is a critical problem, we tested interventions to enhance responses to Dukoral. We evaluated the effect on the immune responses by temporarily withholding breast-feeding or by giving zinc supplementation. Two doses of Dukoral consisting of killed cholera vibrios and cholera B subunit were given to 6-18 months old Bangladeshi children (n=340) and safety and immunogenicity studied. Our results showed that two doses of the vaccine were safe and induced antibacterial (vibriocidal) antibody responses in 57\% and antitoxin responses in 85\% of the children. Immune responses were comparable after intake of one and two doses. Temporary withholding breast-feeding for 3 h before immunization or supplementation with 20 mg of zinc per day for 42 days resulted in increased magnitude of vibriocidal antibodies (77\% and 79\% responders, respectively). Administration of vaccines without buffer or in water did not result in reduction of vibriocidal responses. This study demonstrates that the vaccine is safe and immunogenic in children under 2 years of age and that simple interventions can enhance immune responses in young children. [\hyperlink{Daunorubicin Hydrochloride}{PMID: 36174614}, Tanvir Ahmed et al., 2009]

\hypertarget{pmid_23166343}{D}oxorubicin, effective against many malignancies, is limited by cardiotoxicity. Continuous-infusion doxorubicin, compared with bolus-infusion, reduces early cardiotoxicity in adults. Its effectiveness in reducing late cardiotoxicity in children remains uncertain. We determined continuous-infusion doxorubicin cardioprotective efficacy in long-term survivors of childhood acute lymphoblastic leukemia (ALL). The Dana-Farber Cancer Institute ALL Consortium Protocol 91-01 enrolled pediatric patients between 1991 and 1995. Newly diagnosed high-risk patients were randomly assigned to receive a total of 360 mg/m(2) of doxorubicin in 30 mg/m(2) doses every 3 weeks, by either continuous (over 48 hours) or bolus-infusion (within 15 minutes). Echocardiograms at baseline, during, and after doxorubicin therapy were blindly remeasured centrally. Primary outcomes were late left ventricular (LV) structure and function. A total of 102 children were randomized to each treatment group. We analyzed 484 serial echocardiograms from 92 patients (n = 49 continuous; n = 43 bolus) with ≥1 echocardiogram ≥3 years after assignment. Both groups had similar demographics and normal baseline LV characteristics. Cardiac follow-up after randomization (median, 8 years) showed changes from baseline within the randomized groups (depressed systolic function, systolic dilation, reduced wall thickness, and reduced mass) at 3, 6, and 8 years; there were no statistically significant differences between randomized groups. Ten-year ALL event-free survival rates did not differ between the 2 groups (continuous-infusion, 83\% versus bolus-infusion, 78\%; P = .24). In survivors of childhood high-risk ALL, continuous-infusion doxorubicin, compared with bolus-infusion, provided no long-term cardioprotection or improvement in ALL event-free survival, hence provided no benefit over bolus-infusion. [\hyperlink{Daunorubicin Hydrochloride}{PMID: 23166343}, Steven E Lipshultz et al., 2012]

\hypertarget{pmid_27858183}{A}nthracyclines are used to treat childhood acute lymphoblastic leukemia (ALL). Even when administered at low doses, these agents are reported to cause progressive cardiac dysfunction. We conducted a clinical trial comparing the toxicities of two anthracyclines, pirarubicin (THP) and daunorubicin (DNR), in the treatment of childhood ALL. The results from our study that relate to acute and late toxicities are reported here. 276 children with B-ALL were enrolled in the trial from April 1997 to March 2002 and were randomly assigned to receive a regimen including either THP (25 mg/m Acute hematological toxicity in the early phase was more significant in the THP arm. Based on ultrasound cardiography, cardiac function was impaired in both groups during the follow-up period, but there was no significant difference between the groups except for a greater decline in fractional shortening on ultrasound cardiography in the DNR arm. While acute hematological toxicity was more significant in the THP arm, THP also appeared to be less cardiotoxic. However, the evaluation of late cardiotoxicity was limited because only a few subjects were followed beyond 10 years after ALL onset. Considering that the THP regimen produced an EFS rate comparable with that of the DNR regimen, the efficacy and toxicity of THP at reduced doses should be studied in order to identify potentially safer regimens. [\hyperlink{Daunorubicin Hydrochloride}{PMID: 27858183}, Hiroki Hori et al., 2017]

\hypertarget{pmid_20819318}{A}llergic rhinitis (AR) and chronic idiopathic urticaria (CIU) are common causes of substantial illness and disability in preschool children. Antihistamines are commonly used to treat preschool children with these conditions, but their use is based mostly on extrapolated efficacy from adult populations; it is thus important to characterize the safety of antihistamines in the pediatric population. This study was designed to assess the safety of levocetirizine dihydrochloride oral liquid drops in infants and children with AR or CIU. Two multicenter, double-blind, randomized, parallel-group studies randomized infants aged 6-11 months (study 1, n = 69) and children aged 1-5 years (study 2, n = 173) to levocetirizine, 1.25 mg (q.d. or b.i.d., respectively), or placebo for 2 weeks, using a 2:1 ratio. Safety evaluations included treatment-emergent adverse events (TEAEs), vital signs, electrocardiographic (ECG) assessments, and laboratory tests. The overall incidence of TEAEs was similar between levocetirizine and placebo in both studies. Most TEAEs were mild or moderate in intensity. TEAEs prompted discontinuation of therapy in three patients receiving levocetirizine in study 1. No clinically relevant changes from baseline in vital signs or laboratory parameters were apparent in either study; changes from baseline in these evaluations were similar between groups. No significant changes were observed in ECG parameters, including corrected QT interval. Levocetirizine, 1.25 and 2.5 mg/day, was well tolerated in infants aged 6-11 months and in children aged 1-5 years, respectively, with AR or CIU. [\hyperlink{Daunorubicin Hydrochloride}{PMID: 20819318}, Frank Hampel et al., ]

\hypertarget{pmid_36961611}{C}ardiotoxicity is a major concern following doxorubicin (DOX) use in the treatment of malignancies. We aimed to investigate whether deferoxamine (DFO) can prevent acute cardiotoxicity in children with cancer who were treated with DOX as part of their chemotherapy. Sixty-two newly-diagnosed pediatric cancer patients aged 2-18 years with DOX as part of their treatment regimens were assigned to three groups: group 1 (no intervention, n = 21), group II (Deferoxamine (DFO) 10 times DOX dose, n = 20), and group III (DFO 50 mg/kg, n = 21). Patients in the intervention groups were pretreated with DFO 8-h intravenous infusion in each chemotherapy course during and after completion of DOX infusion. Conventional and tissue Doppler echocardiography, serum concentrations of human brain natriuretic peptide (BNP), and cardiac troponin I (cTnI) were checked after the last course of chemotherapy. Sixty patients were analyzed. The level of cTnI was < 0.01 in all patients. Serum BNP was significantly lower in group 3 compared to control subjects (P = 0.036). No significant differences were observed in the parameters of Doppler echocardiography. Significant lower values of tissue Doppler late diastolic velocity at the lateral annulus of the tricuspid valve were noticed in group 3 in comparison with controls. By using Pearson analysis, tissue Doppler systolic velocity of the septum showed a marginally significant negative correlation with DOX dose (P = 0.05, r = - 0.308). No adverse effect was reported in the intervention groups. High-dose DFO (50 mg/kg) may serve as a promising cardioprotective agent at least at the molecular level in cancer patients treated with DOX. Further multicenter trials with longer follow-ups are needed to investigate its protective role in delayed DOX-induced cardiac damage. Trial registration IRCT, IRCT2016080615666N5. Registered 6 September 2016, http://www.irct.ir/IRCT2016080615666N5 . [\hyperlink{Daunorubicin Hydrochloride}{PMID: 36961611}, Kosar Rahimi et al., 2023]

\hypertarget{pmid_18648960}{I}n children with cancer a well-known risk factor for cardiotoxicity is a high cumulative dose of anthracyclines, but little is known about cardiac function in low-dose anthracycline-treated survivors. Also, it is unclear if a safe anthracycline-dose exists at all. Cardiac function was assessed in 23 long-term ALL-survivors with a median follow-up of 22 years (range 19.5-24.5) post-treatment. Age at diagnosis and current age were 5.0 (2.0-14.0) and 29.0 (24.0-39.0) years. All 23 survivors were treated according to DCLSG protocol ALL-5, including 18-25 Gy cranial irradiation. Thirteen of them received 4 x 25 mg/m(2) daunorubicin by randomization. Cardiac evaluation included blood pressure measurement, echocardiography, and (24 h-) electrocardiogram. Results were compared with an earlier assessment at median 12 years post-treatment. None of the survivors had cardiac abnormalities. Cardiac status of daunorubicin-treated survivors showed no deterioration compared with the previous assessment in 1995. CONCLUSION AND IMPLICATION FOR CANCER SURVIVORS: After prolonged follow-up (more than 20 years post-treatment), ALL-survivors treated with low dose daunorubicin had no clinical relevant deterioration of cardiac function. [\hyperlink{Daunorubicin Hydrochloride}{PMID: 18648960}, C A J Brouwer et al., 2007]

\hypertarget{pmid_34431211}{A}BVD (doxorubicin, bleomycin,vinblastine, and dacarbazine) is not a standard regimen in children due to concerns regarding late effects. However, no studies have evaluated long-term toxicities of ABVD in children. Total 154 pediatric Hodgkin lymphoma (HL) survivors uniformly treated with ABVD were clinically followed up as per institutional protocol. All participants were evaluated for cardiac, pulmonary, and thyroid function abnormalities by multigated acquisition scan (MUGA) scan, spirometry with diffusion capacity of lung for the uptake of carbon monoxide (DLCO), and thyroid profile test, respectively, at a single time point. Predictors of toxicity were also analyzed. The median duration of follow-up of the cohort was 10.3 years (6.04-16.8). No secondary malignant neoplasm (SMN) or symptomatic cardiac/pulmonary toxicities were detected. Nine patients (5.9\%) had left ventricular ejection fraction (LVEF) <55\%. Subclinical and overt hypothyroidism were observed in 78 (50.6\%) and 16 (10.4\%) survivors, respectively. Abnormal spirometry and reduced DLCO was observed in 43.2\% and 42.0\% survivors, respectively. Receiving neck radiation was significantly associated with thyroid dysfunction (odds ratio [OR] 16.04, p < .001); age ≥10 years predicted reduced DLCO (OR 4.12, p = .001). Sixty-three and 33 patients had one and two late adverse effects, respectively; receiving neck radiation predicted development of multiple late effects (proportional OR 4.72, p < 0.001). Cumulative dose of chemotherapy did not predict toxicity. Overall, ABVD appears safe in children at a relatively short follow-up. Long-term safety data are required before it can be adopted for treating pediatric HL patients. Children receiving neck radiation require close follow-up. [\hyperlink{Daunorubicin Hydrochloride}{PMID: 34431211}, Abhenil Mittal et al., 2021]

\hypertarget{pmid_11493818}{C}ontrolled intubation in the pediatric emergency department (ED) requires a paralytic agent that is safe, efficacious, and of rapid onset. The safety of succinylcholine has been challenged, leading some clinicians to use vecuronium as an alternative. Rocuronium's onset is similar to that of succinylcholine. To evaluate the safety and efficacy of rocuronium for controlled intubation with paralysis (CIP) in the pediatric ED. A retrospective, observational study reviewed the records of patients less than 15 years of age, who received controlled intubation with paralytics at two Dallas EDs. The patients received either vecuronium or rocuronium. The study included 84 patients (vecuronium 19, rocuronium 65). Complications were similar between the two groups. Rocuronium had a shorter time from administration to intubation when compared to vecuronium (P < 0.05). Rocuronium is as safe and efficacious as vecuronium for CIP in the pediatric ED. [\hyperlink{Daunorubicin Hydrochloride}{PMID: 11493818}, D R Mendez et al., 2001]

\section*{Depo-Medrol}
\subsection*{Result}
\subsubsection*{Answer}

Unknown

\subsubsection*{{Explanation}}
\hypertarget{Depo-Medrol}
To determine if Depo-Medrol (methylprednisolone acetate) is safe for use in children, I reviewed all available abstracts for targeted pediatric safety studies.

1. [\hyperlink{pmid_7243223}{PMID: 7243223}, K M Zinn et al., 1981] – This is a case report of a 4-year-old girl who suffered severe ocular complications (retinal tear, detachment, cataract) after accidental intraocular injection of Depo-Medrol. This is not a targeted safety study, but it does highlight potential for severe toxicity with intraocular use in children.

2. [\hyperlink{pmid_7065483}{PMID: 7065483}, A A Ashoor et al., 1982] – This study evaluated a single intramuscular injection of Depo-Medrol for allergic rhinitis. The abstract does not specify the age range of the patients, only that they were "Saudi personnel." There is no explicit mention that children were included, so it cannot be considered a targeted pediatric safety study.

3. [\hyperlink{pmid_1092294}{PMID: 1092294}, H A Anderson et al., 1975] – This double-blind study of methylprednisolone acetate (Depo-Medrol) for post-tonsillectomy pain included 165 patients. The abstract does not specify the age range, but tonsillectomy is commonly performed in children. However, without explicit mention of age or a focus on pediatric safety, this cannot be considered a definitive pediatric safety study.

4. [\hyperlink{pmid_25389841}{PMID: 25389841}, Yoshihiro Yonekawa et al., 2009] – This is a case report of severe retinal toxicity after inadvertent intravitreal injection of Depo-Medrol during retinal detachment repair. The age of the patient is not specified, and this is not a targeted safety study.

5. [\hyperlink{pmid_1938281}{PMID: 1938281}, A Loewenstein et al., 1991] – This animal study in rabbits examined the effects of Depo-Medrol and its vehicle on the visual system. It is not a pediatric human safety study.

Summary:
- There are no abstracts presenting a targeted, controlled safety study of Depo-Medrol in children for any indication.
- The available case reports suggest potential for severe toxicity with intraocular use, but these are not systematic safety studies.
- Some studies mention use in populations that may include children, but do not specify age ranges or focus on pediatric safety.

Therefore, based on the abstracts available, the safety of Depo-Medrol in children is unknown, as no targeted pediatric safety studies are present.

\subsection*{Abstracts}
\hypertarget{pmid_7243223}{A} four-year-old girl, while undergoing a repeat excisional biopsy for a progressive orbital hemangioma, sustained an accidental intraocular injection of methylprednisolone acetate (Depo-Medrol) resulting in a mass of intravitreal and subretinal depot material as well as a retinal tear, retinal detachment, and rapid cataract formation. Pars plana lensectomy, vitrectomy, xenon arc endophotocoagulation and air fluid-gas exchange techniques were used to restore the ocular tissues for visual functions in the affected eye and to save the globe from the devastating toxic effects of the vehicles in the depot corticosteroid preparation. [\hyperlink{Depo-Medrol}{PMID: 7243223}, K M Zinn et al., 1981]

\hypertarget{pmid_37409838}{C}hildren usually need sedation or even anaesthesia for magnetic resonance imaging (MRI) studies. As there is no universally accepted method for this purpose we undertook a prospective, randomised comparison of propofol and dexmedetomidine in children aged 1 to 10 years. After Institutional Board approval and parents' informed consent 64 ASA status I or II children scheduled for MRI scan were enrolled. Patients were premedicated with intravenous (IV) midazolam (0.1 mg kg -1 ) and ketamine (1 mg kg -1 ) and randomised to propofol (P) or dexmedetomidine (D) group. A propofol bolus of 1 mg kg -1 followed by infusion of 4 mg kg -1 h -1 , or dexmedetomidine 1 µg kg -1 followed by 2 µg kg -1 h-1 infusion were used. Heart rate, SpO 2 and non-invasive blood pressure were monitored and recorded at 5 min intervals. Results were compared by means of standard statistical methods. Both dexmedetomidine and propofol after premedication with ketamine and midazolam are suitable for MRI sedation, although propofol use results in shorter recovery time. Less interventions are needed when dexmedetomidine is used. [\hyperlink{Depo-Medrol}{PMID: 37409838}, Viktor Mark Brzózka et al., 2023]

\hypertarget{pmid_36418946}{T}o evaluate the safety and effectiveness of different dosages of intranasal Dexmedetomidine (DEX) in combination with oral midazolam for sedation of young children during brain MRI examination. Included in this prospective single-blind randomized controlled trial were 156 children aged from 3 months to 6 years and weighing from 4 to 20 Kg with ASA I-II who underwent brain MRI examination between March 2021 and February 2022. Using the random number table method, they were divided into group A (using 3 ug/kg intranasal DEX plus 0.2 mg/Kg oral midazolam) and group B (using 2 ug/kg intranasal DEX plus 0.2 mg/Kg oral Midazolam). The one-time success rate of sedation, sedation onset time, recovery time, overall sedation time, and occurrence of adverse reactions during MRI examination were compared between the two groups. The heart rate (HR), mean arterial pressure (MAP), and percutaneous SpO The one-time success rate of sedation in group A and B was 88.31\% and 79.75\% respectively, showing no significant difference between the two groups (P>0.05). The sedation onset time in group A was 24.97±16.94 min versus 27.92±15.83 min in group B, and the recovery time was 61.88±22.18 min versus 61.16±28.16 min, both showing no significance difference between the two groups (P>0.05). Children in both groups exhibited good drug tolerance without presenting nausea and vomiting, hypoxia, or bradycardia and hypotension that needed clinical interventions. There was no significant difference in the occurrence of abnormal HR, MAP or other adverse reactions between the two groups (P>0.05). 3 ug/kg or 2 ug/kg intranasal DEX in combination with 0.2 mg/kg oral Midazolam both are safe and effective for sedation of children undergoing MRI examination with the advantages of fast-acting and easy application. It was registered at the Chinese Clinical Trial Registry ( ChiCTR1800015038 ) on 02/03/2018. [\hyperlink{Depo-Medrol}{PMID: 36418946}, Hongbin Gu et al., 2022]

\hypertarget{pmid_12213857}{D}epot GnRH agonists are widely used for the treatment of precocious puberty. Leuprorelin 3-month depot is currently used in adults but has not been evaluated in children. We evaluated the efficacy of this new formulation (11.25 mg every 3 months), for the suppression of gonadotropic activation and pubertal signs in children with central precocious puberty. We included 44 children (40 girls) with early-onset pubertal development in a 6-month open trial. The inclusion criteria were clinical pubertal development before the age of 8 (girls) or 10 (boys), advanced bone age, enlarged uterus (>36 mm), testosterone more than 1.7 nmol/liter (boys), and pubertal response of LH to GnRH (peak >5 IU/liter). The principal criterion for efficacy assessment, GnRH-stimulated LH peak less than 3 IU/liter, was met in 81 of 85 (95\%) of the tests performed at months 3 and 6. The remaining four values were slightly above the threshold. The levels of sex steroids were also significantly reduced and clinical pubertal development was arrested. Plasma leuprorelin levels, measured every 30 d, were essentially stable after d 60. Local intolerance was noted after 10 of 86 injections (12\%), and was mild in four cases, moderate in five cases, and severe in one. Among these 10 events, 4 consisted in local pain at injection's site. In conclusion, leuprorelin 3-month depot efficiently inhibits the gonadotropic axis in 95\% of children with central precocious puberty studied for a 6-month period. This regimen allows the reduction of the number of yearly injections from 12 to 4. [\hyperlink{Depo-Medrol}{PMID: 12213857}, Jean-Claude Carel et al., 2002]

\hypertarget{pmid_26858095}{S}edation is increasingly used to facilitate procedures on children in emergency departments (EDs). This overview of systematic reviews (SRs) examines the safety and efficacy of sedative agents commonly used for procedural sedation in children in the ED or similar settings. We followed standard SR methods: comprehensive search; dual study selection, quality assessment, data extraction. We included SRs of children (1 month to 18 years) where the indication for sedation was procedure-related and performed in the ED. Fourteen SRs were included (210 primary studies). The most data were available for propofol (six reviews/50,472 sedations) followed by ketamine (7/8,238), nitrous oxide (5/8,220), and midazolam (4/4,978). Inconsistent conclusions for propofol were reported across six reviews. Half concluded that propofol was sufficiently safe; three reviews noted a higher occurrence of adverse events, particularly respiratory depression (upper estimate 1.1\%; 5.4\% for hypotension requiring intervention). Efficacy of propofol was considered in four reviews and found adequate in three. Five reviews found ketamine to be efficacious and seven reviews showed it to be safe. All five reviews of nitrous oxide concluded it is safe (0.1\% incidence of respiratory events); most found it effective in cooperative children. Four reviews of midazolam made varying recommendations. To be effective, midazolam should be combined with another agent that increases the risk of adverse events (upper estimate 9.1\% for desaturation, 0.1\% for hypotension requiring intervention). This comprehensive examination of an extensive body of literature shows consistent safety and efficacy for nitrous oxide and ketamine, with very rare significant adverse events for propofol. There was considerable heterogeneity in outcomes and reporting across studies and previous reviews. Standardized outcome sets and reporting should be encouraged to facilitate evidence-based recommendations for care. [\hyperlink{Depo-Medrol}{PMID: 26858095}, Lisa Hartling et al., 2016]

\hypertarget{pmid_28476033}{S}everal studies have reported the use of dexmedetomidine (DEX) plus opioids for flexible bronchoscopy in both adults and children. To determine whether DEX plus sufentanil (SF) is safe for children, 142 children undergoing flexible bronchoscopy were assigned to one of three groups, each of which received the same SF loading dose and similar DEX and SF maintenance doses, but different loading doses of DEX: DS1 (DEX 0.5 μg·kg-1), DS2 (DEX 1.0 μg·kg-1), and DS3 (DEX 1.5 μg·kg-1). The Ramsay sedation scale was maintained at 3 in all groups. Results showed that anesthesia onset time was shorter, and the perioperative hemodynamic profile was more stable, in the DS3 group. The number of intraoperative movements was also lowest in the DS3 group. The time to first dose of rescue midazolam and lidocaine was significantly longer, but the total corresponding accumulated doses were lower in the DS3 group. Although the time to recovery prior to discharge from the post anesthesia care unit was longer, the overall incidence of tachycardia was lower in the DS3 group, and it received the highest bronchoscopist satisfaction score among the three groups. We therefore conclude that high-dose DEX plus SF can be safely and efficaciously used in children undergoing flexible bronchoscopy. [\hyperlink{Depo-Medrol}{PMID: 28476033}, Xiujing Dang et al., 2017]

\hypertarget{pmid_31050796}{F}requently, infants and children require sedation to facilitate noninvasive procedures and imaging studies. Propofol and dexmedetomidine are used to achieve deep procedural sedation in children. The objective of this study was to compare the clinical safety and efficacy of propofol versus dexmedetomidine in pediatric patients undergoing sedation in a pediatric sedation unit. A retrospective analysis of patients sedated with either propofol or dexmedetomidine in a pediatric sedation unit by pediatric emergency physicians was performed. Both medications were dosed per protocol with propofol 2 mg/kg induction and 150 μg · kg A total of 2432 children were included- 1503 who received propofol and 929 who received dexmedetomidine. Propofol and dexmedetomidine resulted in successful completion of the study in 98.8\% and 99.7\%, respectively ( Propofol use led to significantly shorter recovery times, with an increased need for airway management, but rates of bag-mask ventilation (2.3\%), airway obstruction (1.1\%), and desaturation (1.6\%) were low. No patients required intubation. Propofol is a reasonable alternative to dexmedetomidine, with a clinically acceptable safety profile. [\hyperlink{Depo-Medrol}{PMID: 31050796}, Nicole M Schacherer et al., 2019]

\hypertarget{pmid_33531939}{P}ediatric anxiety and restlessness may create issues and difficulties in performing accurate diagnostic studies even noninvasive ones, such as radiological imaging. There are some agents that will help to get this goal. This study aimed to compare the intranasal effect of dexmedetomidine (DEX) and midazolam (MID) for sedation parameters of children undergoing computerized tomography (CT) imaging. A double-blind clinical trial was conducted on 162 eligible children who underwent CT imaging. These patients were divided into two groups including MID ( Decreasing in MBP and HR was higher in DEX group than MID group ( Using 3 μg/kg intranasal DEX for sedation of 1-6-year-old children was a suitable method to undergo noninvasive studies such as CT imaging. Intranasal DEX is superior to MID due to higher sedation satisfactory, faster starting effect of sedation, and lower side effects and complications. Nevertheless, in children with hemodynamic instability DEX is not an appropriate choice. [\hyperlink{Depo-Medrol}{PMID: 33531939}, Reza Azizkhani et al., ]

\hypertarget{pmid_7065483}{E}ffects of a single intramuscular depot-corticosteroid in allergic rhinitis were studied at clinical and light microscopic levels in Saudi personnel with allergic rhinitis. Clinical studies and biopsies were performed before and after treatment. Following a single intramuscular injection of Depo-Medrol all patients experienced relief and remained symptom-free thereafter for several months. The roentgenographic appearance of the sinuses, eosinophilia of blood and nasal secretion, skin reactivity and the microscopic features exhibited improvement. These findings suggest that a single intramuscular injection of Depo-Medrol is a satisfactory agent for the treatment of allergic rhinitis. [\hyperlink{Depo-Medrol}{PMID: 7065483}, A A Ashoor et al., 1982]

\hypertarget{pmid_1938281}{P}eriocular injections of corticosteroids play an important role in the management of various ophthalmologic diseases. The Depo-Medrol vehicle, injected into the vitreous, was shown to be toxic to the lens and to the retina when applied at double strength. The authors examined the effects of Depo-Medrol and one of the components of its vehicle, myristyl-gamma-picolinium chloride (MGP), on the functional integrity of the rabbit visual system. Visual function was assessed objectively from the electroretinogram (ERG) and the visual evoked potential (VEP). The experimental drugs were injected into the vitreous humor of one eye while saline was injected into the fellow eye for control. Depo-Medrol did not produce any measurable effects on the ERG or the VEP. When MGP solutions were injected in concentrations at least twice as large as that in the Depo-Medrol, significant reductions in the light- and dark-adapted ERG responses were seen. The effects of the drug on the ERG responses was seen as early as 3 days postinjection and developed to its maximal level within 1-2 weeks. No ERG recovery was seen over a period of more than 2 months. The VEP, elicited by applying light stimuli to the experimental eye, was characterized by low amplitude and delayed implicit time compared with the response obtained from the control eye. [\hyperlink{Depo-Medrol}{PMID: 1938281}, A Loewenstein et al., 1991]

\hypertarget{pmid_25661272}{W}e hypothesized that postoperative sedation with dexmedetomidine/fentanyl would be effective in infants and neonates with congenital heart disease and pulmonary arterial hypertension (PAH). Children who were <36 months of age, had congenital heart disease with PAH, and had been treated at our hospital between October 2011 and April 2013 (n = 187) were included in this retrospective study. Either dexmedetomidine/fentanyl (Group Dex) or midazolam/fentanyl (Group Mid) was used for postoperative sedation. The main outcome variables included delirium scores, supplemental sedative/analgesic drugs, ventilator use, and sedation time. Baseline demographics and clinical characteristics were similar between the two groups. The Pediatric Anesthesia Emergence Delirium scale (5.2 ± 5.3 vs. 7.1 ± 5.2 in the Dex and Mid groups, respectively; P = 0.016) and the incidence of delirium (18.2 vs. 32.0 \% in the Dex and Mid groups, respectively; P = 0.039) were significantly lower in the Dex group than in the Mid group. Total sufentanil, midazolam, and propofol doses given during the operation did not differ between the two groups. Group Dex patients required significantly lower doses of adjunctive sedative/analgesic drugs than group Mid patients in the cardiac intensive care unit (CICU; midazolam, P = 0.007; morphine, P < 0.001). In conclusion, we found no differences between dexmedetomidine/fentanyl and midazolam/fentanyl in terms of the duration of sedation, mechanical ventilator use, and CICU stay in children with PAH. However, patients in the Dex group required a lower additional sedative/analgesic drugs and had a lower incidence of delirium than patients in the Mid group. [\hyperlink{Depo-Medrol}{PMID: 25661272}, Li Jiang et al., 2015]

\hypertarget{pmid_17474950}{T}he current study evaluates the hemodynamic and respiratory effects of dexmedetomidine (DEX) when administered to children anesthetized with sevoflurane (SEVO) or desflurane (DES). After tracheal intubation and spontaneous ventilation, DEX (0.5 microg x kg(-1)) was administered over 5 min. Heart rate (HR), systolic blood pressure (sBP), diastolic blood pressure (dBP), and endtidal carbon dioxide (P(E)CO(2)) were monitored and recorded prior to DEX (time 0) and again at 5, 10, and 15 min after DEX. The cohort included 80 children (1-12 years of age) anesthetized with SEVO (n = 40) or DES (n = 40). The lowest HR from time 0 to time 15 was less in patients anesthetized with SEVO compared with DES (104 +/- 16 b x min(-1) in the SEVO/DEX group vs 120 +/- 17 b x min(-1) in the DES/DEX group, < 0.01). Although both sBP and dBP decreased following the administration of DEX to patients anesthetized with either SEVO or DES, there was no difference in sBP or dBP between the two groups. Likewise, no evidence was found for changes in the P(E)CO(2) during the study period. The administration of DEX (0.5 microg x kg(-1)) results in a lower HR in patients anesthetized with SEVO compared with DES. No evidence was found for differences in sBP, dBP, or P(E)CO(2) during spontaneous ventilation with 1 MAC of SEVO vs DES. [\hyperlink{Depo-Medrol}{PMID: 17474950}, Eric Deutsch et al., 2007]

\hypertarget{pmid_1092294}{T}here are numerous studies using various agents to reduce posttonsillectomy morbidity. Due to lack of conclusive results a double-blind study using methylprednisolone acetate (Depo Medrol) was conducted. A total of 165 patients were randomly divided into two groups. Following tonsillectomy 2.5 ml of solution was injected into the base of each tonsillar fossa. The study group received 20 mg of deposteroid to each fossa while the control group got normal saline. The deposteroid reduced postoperative pain but did not significantly alter other factors contributing to morbidity such as difficulty in swallowing or resumption of a normal diet. The deposteroid appeared to have no effect on the rate of healing of the tonsillar fossa. [\hyperlink{Depo-Medrol}{PMID: 1092294}, H A Anderson et al., 1975]

\hypertarget{pmid_22165748}{B}ACKROUND/AIM: Sedation is necessary in children undergoing magnetic resonance imaging (MRI) to ensure motionless. The success of sedation is typically measured by two factors: safety (lack of adverse events) and effectiveness of the procedure (successful completion of the diagnostic examination). Propofol is frequently used to induce deep sedation in children. However, increased doses of propofol may lead to oversedation and respiratory depression. The aim of the study was to investigate sedation in children using propofol with midazolam in regard to efficacy, adverse events and time to return to presedation functional status. We investigated 24 children prospectively. Sedation was introduced with a single bolus of intravenous (iv) midazolam 0.1 mg/kg followed by repeated small iv boluses of propofol until sufficient depth of sedation was obtained. The outcome of sedation was measured by the induction time, sedation time, need for additional sedation, respiratory events, cardiovascular events and sedation failure. Median age of children was 4.72 +/- 3.06 (1.1-12.3) years and their body weight was 21.3 +/- 11.9 (11-60) kg. Average propofol bolus dose for induction was 1.76 +/- 0.9 (0.5-4) mg/kg. The induction time was 8.88 +/- 2.92 (5-15) min, and sedation time 28.39 +/- 8.42 (20-50) min. Additional sedation was necessary in 3 (12.5\%) patients. Unsucesfull sedation or significant adverse events were not observed. The presented sedation technique for children undergoing ambulatory MRI of the brain is safe and adequate. This sedation regiment provides short induction time, fast recovery, stable cardiorespiratory conditions and rarely demans additional sedation. [\hyperlink{Depo-Medrol}{PMID: 22165748}, Jasna Jevdjić et al., 2011]

\hypertarget{pmid_32145737}{I}ntranasal dexmedetomidine (DEX), as a novel sedation method, has been used in many clinical examinations of infants and children. However, the safety and efficacy of this method for electroencephalography (EEG) in children is limited. In this study, we performed a large-scale clinical case analysis of patients who received this sedation method. The purpose of this study was to evaluate the safety and efficacy of intranasal DEX for sedation in children during EEG. This was a retrospective study. The inclusion criteria were children who underwent EEG from October 2016 to October 2018 at the Children's Hospital affiliated with Chongqing Medical University. All the children received 2.5 μg·kg A total of 3475 cases were collected and analysed in this study. The success rate of the initial dose was 87.0\% (3024/3475 cases), and the success rate of intranasal sedation rescue was 60.8\% (274/451 cases). The median sedation onset time was 19 mins (IQR: 17-22 min), and the sedation recovery time was 41 mins (IQR: 36-47 min). The total incidence of adverse events was 0.95\% (33/3475 cases), and no serious adverse events occurred. Intranasal DEX (2.5 μg·kg [\hyperlink{Depo-Medrol}{PMID: 32145737}, Hang Chen et al., 2020] Methylprednisolone acetate (Depo-Medrol, Pfizer, New York) is a depot corticosteroid that is commonly injected periorbitally to treat various ophthalmologic conditions. Accidental intravitreal injections secondary to globe perforations have resulted in rapid retinal toxicity. To their knowledge, the authors report the first case of inadvertent intravitreal methylprednisolone acetate injection during pars plana vitrectomy. Report of a case of inadvertent intravitreal injection of methylprednisolone acetate, mistaken as triamcinolone acetonide, during repeated retinal detachment repair. The affected eye had loss of vision, afferent pupillary defect, optic nerve atrophy, retinal necrosis, retinal vascular damage, and recurrent retinal detachment. Methylprednisolone acetate administered during vitrectomy causes severe retinal toxicity and complicates retinal detachment repair. It is important to use measures to avoid erroneous intravitreal injections during vitrectomy. [\hyperlink{Depo-Medrol}{PMID: 32145737}, Yoshihiro Yonekawa et al., 2009]

\hypertarget{pmid_28442954}{A} deep level of sedation is required for magnetic resonance imaging (MRI) in children to ensure optimum image quality. The present study was conducted to evaluate the efficacy and safety of dexmedetomidine versus propofol for sedation in children undergoing MRI. A total of sixty children aged 2-10 years, having physical status 1 or 2 according to the American Society of Anesthesiologists, undergoing MRI were included in the study. Group D: ( The mean time for onset of sedation in Group D was much longer than in Group  Propofol had an advantage of providing rapid onset of sedation and quicker recovery time. Dexmedetomidine resulted in a better preservation of respiratory rate and oxygen saturation, so it may be more suitable in children who are prone to respiratory depression. Hence, both the drugs could achieve required sedation in children posted for MRI. [\hyperlink{Depo-Medrol}{PMID: 28442954}, Kirti Kamal et al., ]

\hypertarget{pmid_7788009}{W}e evaluated the pituitary and gonadal suppression in 40 girls and nine boys treated with depot leuprorelin (3.75 mg sc if body weight > or = 20 kg, 1.87 mg if body weight < 20 kg) every 28 days for central precocious puberty. Gonadal suppression was obtained in most of the children with this dose: 3 months after initiation of the treatment, 85\% of children had a peak plasma luteinizing hormone response to gonadotropin-releasing hormone < 3 IU/l and the gonadal axis remained suppressed throughout the duration of the study (up to 24 months). Four patients required higher doses of leuprorelin to achieve suppression. In two girls, a cutaneous reaction to the drug was associated with incomplete suppression and the treatment had to be interrupted. Plasma leuprorelin levels tended to increase from day 3 to day 28 after injection. Residual leuprorelin levels measured 28 days after injection were stable during the first year of the study. We conclude that an initial dose of depot leuprorelin of 3.75 mg sc every 28 days is efficient in most children with central precocious puberty. [\hyperlink{Depo-Medrol}{PMID: 7788009}, J C Carel et al., 1995]

\hypertarget{pmid_31543589}{W}e studied the efficacy and safety of different total intravenous anesthesia used for pediatric magnetic resonance imaging (MRI). Children of 1-7 years age ( Initiation of scan was 100\% successful with median induction time of 10 min. Maintenance of sedation was successful in 100\% with dexmedetomidine and 97.4\% with propofol infusion. Recovery time (25 min v/s 30 min), discharge time (35 min v/s 60 min), and total care duration (80 min v/s 105 min) were significantly less with propofol as compared to dexmedetomidine ( Dexmedetomidine 1μg/kg, ketamine 1 mg/kg, and propofol 1 mg/kg provide good conditions for initiation of MRI. Although dexmedetomidine at 0.7μg/kg/h and propofol at 3 mg/kg/h are safe and effective for maintenance, propofol provides faster recovery. [\hyperlink{Depo-Medrol}{PMID: 31543589}, Bhuvaneswari Balasubramanian et al., ]

\hypertarget{pmid_36620110}{M}agnetic resonance imaging (MRI) under sedation requires faster recovery for early discharge and feeding resumption in children with neuropsychiatric disorders. The use of dexmedetomidine alone results in delayed recovery. Propofol, when used alone, can cause hypotension and respiratory depression. A new regimen for sedation was evaluated by exploiting the properties of these drugs, to allow faster recovery with minimal adverse events. One hundred and fifty children aged 2-12 years requiring MRI were randomly allocated to these three groups. Group P ( Recovery following sedation in Group PD (15 ± 7.0 min) and Group P (17.35 ± 7.4 min) were comparable and significantly ( The regimen with propofol bolus and dexmedetomidine infusion provided adequate sedation and better recovery characteristics in children aged 2-12 years without systemic complications, as compared to the use of either agent alone. [\hyperlink{Depo-Medrol}{PMID: 36620110}, Shwethashri Kondavagilu Ramaprasannakumar et al., ]

\hypertarget{pmid_10648310}{A}n IM combination of meperidine, promethazine, and chlorpromazine (DPT) has been given as sedation for pediatric procedures for more than 40 years. We compared this IM combination to oral (PO) ketamine/midazolam in children having cardiac catheterization. A total of 51 children, ages 9 mo to 10 yr, were enrolled and randomized in this double-blinded study. All children received an IM injection at time zero and PO fluid 15 minutes later. We observed acceptance of medication, onset of sedation and sleep, and sedative efficacy. The cardiorespiratory changes were evaluated. Sedation was supplemented with IV propofol as required. Recovery time, parental satisfaction, and patient amnesia were assessed. Ketamine/midazolam given PO was better tolerated (P < 0.0005), had more rapid onset (P < 0.001), and provided superior sedation (P < 0.005). Respiratory rate decreased after IM DPT only. Heart rate and shortening fraction were stable. Oxygen saturation and mean blood pressure decreased minimally in both groups. Supplemental propofol was more frequently required (P < or = 0.02) and in larger doses (P < 0.05) after IM DPT. Parental satisfaction ratings were higher (P < 0.005) and amnesia was more reliably obtained (P = 0.007) with PO ketamine/midazolam. Two patients needed airway support after the PO medication, as did two other patients when PO ketamine/midazolam was supplemented with IV propofol. Although PO ketamine/midazolam provided superior sedation and amnesia compared to IM DPT, this regimen may require the supervision of an anesthesiologist for safe use. Oral medication can be superior to IM injections for sedating children with congenital heart disease; however, the safety of all medications remains an issue. [\hyperlink{Depo-Medrol}{PMID: 10648310}, S M Auden et al., 2000]

\hypertarget{pmid_30971402}{I}ntranasal, intramuscular, and intravenous (IV) dexmedetomidine routes have been used successfully for pediatric MRI studies. We designed this retrospective study to determine efficacy and safety of buccal dexmedetomidine for pediatric MRI sedation. Medical records were reviewed of outpatient children ages 5 to 18 years who received buccal dexmedetomidine with or without oral midazolam for MRI sedation at a freestanding children's hospital sedation program in 2015 and 2016. A total of 220 outpatient encounters received buccal dexmedetomidine for MRI. Mean age of the cohort was 10.1 ± 2.6 years (range: 5-18.7). Buccal dexmedetomidine dose administered was a mean of 2.20 ± 0.38 μg/kg (range: 0.88-3.19). Of the 220 sedation encounters, 179 (81.4\%) patients had satisfactory sedation with buccal dexmedetomidine with or without oral midazolam: 84 had buccal dexmedetomidine as the sole sedative, 95 had satisfactory sedation when buccal dexmedetomidine and oral midazolam (mean: 0.33 ± 0.07 mg/kg; range: 0.21-0.53) were given together, 1 (0.4\%) had satisfactory sedation when intranasal fentanyl and midazolam were administered in addition to buccal dexmedetomidine, and 35 (15.9\%) required IV sedatives to achieve satisfactory sedation. All patients completed their MRI successfully except 5 (2.2\%): 2 encounters were sedation failures, 2 IV sedations developed severe upper airway obstruction, and 1 IV sedation experienced MRI contrast anaphylaxis. In a selected population of pediatric patients, buccal dexmedetomidine with or without midazolam provides adequate sedation for most MRI studies with few adverse effects, but given a failure rate of almost 20\%, modifications to buccal dexmedetomidine dosing should be investigated. [\hyperlink{Depo-Medrol}{PMID: 30971402}, Juan P Boriosi et al., 2019]

\hypertarget{pmid_35989987}{A}lthough numerous intravenous sedative regimens have been documented, the ideal non-parenteral sedation regimen for magnetic resonance imaging (MRI) has not been determined. This prospective, interventional study aimed to investigate the efficacy and safety of buccal midazolam in combination with intranasal dexmedetomidine in children undergoing MRI. Children between 1 month and 10 years old requiring sedation for MRI examination were recruited to receive buccal midazolam 0.2 mg⋅kg Sedation with dexmedetomidine-midazolam was administered to 530 children. The successful sedation rate was 95.3\% (95\% confidence interval: 93.5-97.1\%) with the initial sedation regimens and 97.7\% (95\% confidence interval: 96.5-99\%) with a rescue dose of 2 μg⋅kg In MRI examinations, the addition of buccal midazolam to intranasal dexmedetomidine is associated with a high success rate and a good safety profile. This non-parenteral sedation regimen can be a feasible and convenient option for short-duration MRI in children between 1 month and 10 years. [\hyperlink{Depo-Medrol}{PMID: 35989987}, Bi Lian Li et al., 2022]

\hypertarget{pmid_29445339}{D}esmopressin is used to treat primary nocturnal enuresis in children. Over the years, various formulations of desmopressin were commercialized of which the sublingual melt tablet is preferred in the pediatric population, despite the lack of full PK studies in this population. A full PK study was performed in growing conventional piglets to evaluate if this juvenile animal model can provide supplementary information to complement the information gap in the pediatric population. A desmopressin sublingual melt tablet (120 μg) was administered to 32 male piglets aged 8 days, 4 weeks, 7 weeks, and 6 months (each group  [\hyperlink{Depo-Medrol}{PMID: 29445339}, Elke Gasthuys et al., 2018] The GnRH analog leuprolide acetate has been shown to be effective in the treatment of precocious puberty when given as a daily sc injection. We studied the effectiveness of a single im dose of a new depot form of leuprolide in suppressing estradiol and gonadotropin secretion in children with precocity. Five girls with previously untreated precocity showed significant decreases in basal serum estradiol and FSH levels and in peak LH levels (after GnRH testing) 30 days after a single im dose of leuprolide acetate for depot suspension. Mean peak FSH levels also fell greatly, but the difference was not significant. No adverse effects were noted during the first 4-6 months of monthly im injections. Depot im leuprolide appears to be effective in suppressing estradiol and gonadotropin secretion, and may be a useful method of treating children with central precocious puberty. [\hyperlink{Depo-Medrol}{PMID: 29445339}, M Kappy et al., 1989]

\section*{Dopamine Hydrochloride}
\subsection*{Result}
\subsubsection*{Answer}

Yes (neonates, infants, children)
Unknown (adolescents)

\subsubsection*{{Explanation}}
\hypertarget{Dopamine Hydrochloride}
A review of the available abstracts reveals several that specifically address the use of dopamine hydrochloride in children, including neonates and infants.

1. One abstract [\hyperlink{pmid_2682552}{PMID: 2682552}, V Bhatt-Mehta et al., 1989] provides a detailed review of dopamine hydrochloride's pharmacologic effects, pharmacokinetics, and adverse effects in neonates, infants, and older children. It notes that dopamine is widely used in these populations for shock and cardiac failure, with dose-dependent effects. The abstract mentions that adverse effects such as tachycardia and cardiac arrhythmias can occur, especially at high doses in neonates, but does not report any unique or severe safety concerns that would preclude its use. The discussion is evidence-based and specific to pediatric populations, including neonates and preterm infants.

2. Another abstract [\hyperlink{pmid_875819}{PMID: 875819}, J S Wright et al., 1977] describes the use of dopamine hydrochloride in postoperative management of open-heart surgical patients, both children and adults. It reports a very low incidence of undesirable responses and a high frequency of circulatory improvement, with careful control of infusion rate and monitoring. The abstract affirms that, in the dosages employed, dopamine was free of most undesirable side effects seen with other catecholamines.

3. A third abstract [\hyperlink{pmid_2358894}{PMID: 2358894}, V Bhatt-Mehta et al., 1990] focuses on the physical compatibility and chemical stability of dopamine when co-administered with other drugs in newborn infants. While the primary focus is on drug compatibility, the context affirms that dopamine is used in newborn infants in clinical practice.

4. An additional abstract [\hyperlink{pmid_10870004}{PMID: 10870004}, S Watarida et al., 2000] evaluates the safety of docarpamine, a dopamine prodrug, in children undergoing open heart surgery. While this is not dopamine hydrochloride itself, the study design includes children stabilized on intravenous dopamine before transitioning to oral docarpamine, and no adverse cardiac events were reported during dopamine use.

None of the abstracts report a targeted pediatric study that demonstrates dopamine hydrochloride is unsafe in children. The available evidence from these abstracts supports that dopamine hydrochloride has been studied in neonates, infants, and older children, and is considered safe for use in these populations when administered with appropriate monitoring and dosing.

No abstracts were found that specifically address safety in adolescents as a distinct group, but the studies include "older children," which may overlap with early adolescence.

In summary, based on the abstracts reviewed, dopamine hydrochloride is considered safe for use in neonates, infants, and children, with safety affirmed by targeted studies in these age groups.

\subsection*{Abstracts}
\hypertarget{pmid_2682552}{D}opamine hydrochloride is widely used to increase blood pressure, cardiac output, urine output, and peripheral perfusion in neonates, infants, and older children with shock and cardiac failure. Its pharmacologic effects are dose dependent, and at low, intermediate, and high dosages include dilation of renal, mesenteric, and cerebral vasculature; inotropic response in the myocardium; and increases in peripheral and renal vascular resistance, respectively. The inotropic response is diminished in neonates compared with older children and adults due to maturational differences in norepinephrine stores. The clearance of dopamine varies widely in the pediatric population, depending on age. Its elimination half-life is approximately 2 minutes in full-term neonates and older children, and may be as long as 4-5 minutes in preterm infants. Due to immaturity of the autonomic nervous system, the drug may produce some adverse respiratory responses at high dose in neonates, the most common being tachycardia and cardiac arrhythmias. Dobutamine resembles dopamine chemically and is an analog of isoproterenol. It is relatively cardioselective at dosages used in clinical practice, with its main action being on beta 1-adrenergic receptors. Unlike dopamine, it does not have any effect on specific dopaminergic receptors. Dobutamine is used to increase cardiac output in infants and children with circulatory failure. Its elimination half-life is about 2 minutes in adults and older children. No information is available about its pharmacokinetics in neonates and infants. Adverse effects such as an increase in heart rate usually occur at high dosages. [\hyperlink{Dopamine Hydrochloride}{PMID: 2682552}, V Bhatt-Mehta et al., 1989]

\hypertarget{pmid_17614751}{D}oxapram hydrochloride, a respiratory stimulant, has several undesirable side effects during high-dose administration, including second-degree atrioventricular (AV) block and QT prolongation. In Japan, this drug is contraindicated for newborn infants. Recent studies, however, have demonstrated the efficacy and safety of doxapram therapy for apnea of prematurity (AOP) using lower doses than those previously tested. As a result, approximately 60\% of Japanese neonatologists continue to use this drug. This study used surface ECG recordings to assess the cardiac safety of low-dose doxapram hydrochloride (0.2 mg/kg/h) in fifteen premature very-low-birth-weight infants with idiopathic AOP. Cardiac intervals and number of apnea episodes were compared before and after drug administration. Low-dose doxapram hydrochloride resulted in approximately 90\% reduction in the frequency of apnea without side effects. None of the infants developed QT or PR prolongation, arrhythmia, or other conduction disorders. In addition, there was no change in the slope of QT/RR before versus after administration of doxapram hydrochloride. We conclude that low-dose administration of doxapram hydrochloride did not have any undesirable effects on myocardial depolarization and repolarization. [\hyperlink{Dopamine Hydrochloride}{PMID: 17614751}, Masafumi Miyata et al., 2007]

\hypertarget{pmid_875819}{D}opamine hydrochloride (Intropin) has been used in the postoperative management of open-heart surgical patients, both children and adults. It has been associated with a very low incidence of undesirable responses and a high frequency of circulatory improvement associated with diuresis. Its administration requires careful control of infusion rate and measurement of response. In the dosages employed, it has been free of most of the undesirable side effects of other commonly used catecholamines. [\hyperlink{Dopamine Hydrochloride}{PMID: 875819}, J S Wright et al., 1977]

\hypertarget{pmid_2358894}{D}opamine hydrochloride is widely used to increase blood pressure, cardiac output, and peripheral perfusion in critically ill newborn infants and children with shock and congestive heart failure. These patients often require numerous other intravenous drugs such as dobutamine, tolazoline, and theophylline concurrently, but have limited venous access. As a result, two or more of these drugs may be administered through the same intravenous site. The objective of our study was to determine the physical compatibility and chemical stability of dopamine with dobutamine, tolazoline, and theophylline using simulated conditions encountered in the neonatal intensive care unit. Dopamine, dobutamine, tolazoline, and theophylline were studied at concentrations of 120 mg/100 mL, 120 mg/100 mL, 400 mg/100 mL, and 400 mg/500 mL, respectively, in 5\% dextrose injection. The flow rate of dopamine was 0.3 mL/h while all combination drugs were run at 1 mL/h. Aliquots were collected at hourly intervals for 5 hours. A simultaneous static experiment was also performed by mixing dopamine with each combination drug in a ratio of 1:3 and allowing these to stand at room temperature. Samples were obtained at 0.5-hour intervals for 3 hours. Each aliquot was inspected visually for any change in color and clarity and analyzed in triplicate for dopamine content using high performance liquid chromatographic technique with electrochemical detection. Linear regression analysis was performed on the mean values of dopamine concentrations to assess its degradation. Dopamine was found to be physically and chemically stable with dobutamine, tolazoline, and theophylline. Thus, dopamine can be infused concurrently with any of these drugs in 5\% dextrose injected at frequently used concentrations in newborn infants. [\hyperlink{Dopamine Hydrochloride}{PMID: 2358894}, V Bhatt-Mehta et al., 1990]

\hypertarget{pmid_28142338}{N}euroleptic drug molindone hydrochloride is a dopamine D2/D5 receptor antagonist and it is in late stage development for the treatment of impulsive aggression in children and adolescents who have attention deficient/hyperactivity disorder (ADHD). This new indication for this drug would expand the target population to include younger patients, and therefore, toxicity assessments in juvenile animals were undertaken in order to determine susceptibility differences, if any, between this age group and the adult rats. Adult rats were administered molindone by oral gavage for 13 weeks at dose levels of 0, 5, 20, or 60 mg/kg-bw/day. Juvenile rats were dosed for 8 weeks by oral gavage at dose levels of 0, 5, 25, 50, or 75 mg/kg-bw/day. Standard toxicological assessments were made using relevant study designs in consultation with FDA. Treatment-related elevation in serum cholesterol and triglycerides and decreases in glucose levels were observed in both the age groups. Organ weight changes included increases in liver, adrenal gland and seminal vesicles/prostate weights. Decreases in uterine weights were also observed in adult females exposed to the top two doses with sufficient exposure. In juveniles, sexual maturity parameters secondary to decreased body weights were observed, but, were reversed. In conclusion, the adverse effects noted in reproductive tissues of adults were attributed to hyperprolactinemia and these changes were not considered to be relevant to humans due to species differences in hormonal regulation of reproduction. On the whole, there were no remarkable differences in the toxicity profile of the drug between the two age groups. [\hyperlink{Dopamine Hydrochloride}{PMID: 28142338}, Gopala Krishna et al., 2017]

\hypertarget{pmid_942230}{K}etamine hydrochloride 2 mg/kg, together with atropine 0.2 mg, has been given intravenously on 100 occasions on a general paediatric ward. No serious side effects occurred. Dreams followed in 4 children but did not reduce acceptability of the drug. In our hands it has greatly reduced the pain and distress of children undergoing many routine medical procedures, particularly the dread which builds up when these have to be repeated in the same child. It has also produced close to ideal conditions for the operator, and probably increased his efficiency by reducing the emotional strain which occurs when doing painful things to a frightened patient. [\hyperlink{Dopamine Hydrochloride}{PMID: 942230}, E Elliott et al., 1976]

\hypertarget{pmid_26499007}{T}o evaluate the efficacy and safety of Drotaverine hydrochroride in children with recurrent abdominal pain. Double blind, randomized placebo-controlled trial. Pediatric Gastroenterology clinic of a teaching hospital. 132 children (age 4-12 y) with recurrent abdominal pain (Apley Criteria) randomized to receivedrotaverine (n=66) or placebo (n=66) orally. Children between 4-6 years of age received 10 mL syrup orally (20 mg drotaverine hydrochloride or placebo) thrice daily for 4 weeks while children >6 years of age received one tablet orally (40 mg drotaverine hydrochloride or placebo) thrice daily for 4 weeks. Primary: Number of episodes of pain during 4 weeks of use of drug/placebo and number of pain-free days. Secondary: Number of school days missed during the study period, parental satisfaction (on a Likert scale), and occurrence of solicited adverse effects. Reduction in number of episodes of abdominal pain [mean (SD) number of episodes 10.3 (14) vs 21.6 (32.4); P=0.01] and lesser school absence [mean (SD) number of school days missed 0.25 (0.85) vs 0.71 (1.59); P=0.05] was noticed in children receiving drotaverine in comparison to those who received placebo. The number of pain-free days, were comparable in two groups [17.4 (8.2) vs 15.6 (8.7); P=0.23]. Significant improvement in parental satisfaction score was noticed on Likert scale by estimation of mood, activity, alertness, comfort and fluid intake. Frequency of adverse events during follow-up period was comparable between children receiving drotaverine or placebo (46.9\% vs 46.7\%; P=0.98). Drotaverine hydrochloride is an effective and safe pharmaceutical agent in the management of recurrent abdominal pain in children. [\hyperlink{Dopamine Hydrochloride}{PMID: 26499007}, Manish Narang et al., 2015]

\hypertarget{pmid_17542008}{T}here is growing evidence to support the use of early central cholinergic enhancement to improve cognitive functioning in individuals with Down syndrome (DS). This report summarizes preliminary safety and cognitive efficacy data for seven children (8-13 years) with DS who participated in a 22-week, open-label trial of donepezil hydrochloride. Donepezil was dosed once daily at 2.5 mg and, based on tolerability, increased to 5 mg/day. Safety assessments were conducted at Week 1 (baseline), Week 8 (2.5 mg donepezil), Week 16 (5 mg) and Week 22 (after the donepezil had been discontinued). Measures of cognitive function were administered at each visit, encompassing the following domains: memory; attention; mood; and adaptive functioning. Donepezil was well tolerated at the 2.5 and 5 mg doses. The side effects were mild, transient, and consistent with the adverse events noted with cholinesterase inhibitors. Some children showed improvement on measures of memory (NEPSY Memory for Names and Narrative Memory) and sustained attention to tasks (Conners' Parent Rating Scales), although increased irritability and/or assertiveness were noted in some patients. Overall, this clinical report series adds to our initial findings of language gains in children with DS treated with donepezil. It also supports the need for larger, double-blind studies of the safety and efficacy of donepezil and other cholinesterase inhibitors for children with DS. [\hyperlink{Dopamine Hydrochloride}{PMID: 17542008}, Gail A Spiridigliozzi et al., 2007]

\hypertarget{pmid_10453184}{T}he purpose of this study was to evaluate whether or not dopamine (DA) can penetrate to the central nervous system (CNS) from the blood in the infantile period in rats. In a preliminary experiment, we administered a 50 mg/kg dose of DA hydrochloride, intraperitoneally, to 7-day-old rats (DA 50 mg/kg group), obtaining cerebrospinal fluid (CSF) both before and at 5, 10, 20, 30, 60 and 120 min after administration. The CSF levels of DA and its main metabolites, 3,4-dihydroxyphenylacetic acid (DOPAC) and homovanillic acid (HVA), were then measured. Next, we investigated the DA transfer from blood to the CNS by administering doses of 1, 5, 10 and 30 mg/kg DA hydrochloride (DA 1, 5, 10 and 30 mg/kg groups). In these groups, CSF samples were obtained only at 10 and/or 60 min after DA administration, based on the results of the DA 50 mg/kg group. The DA concentrations in CSF significantly increased compared with values before DA administration in the DA 50 mg/kg group. The DA concentrations in the DA 30 mg/kg group, DOPAC concentrations in the DA 5, 10 and 30 mg/kg groups, and HVA concentrations in all groups were significantly higher than in the control (saline injection) group. These findings suggest easy DA transfer from blood to the CNS and immaturity of the blood-brain barrier for DA in the infantile period in rats. [\hyperlink{Dopamine Hydrochloride}{PMID: 10453184}, H Miyaguchi et al., 1999]

\hypertarget{pmid_2522789}{T}he neuromuscular and cardiovascular effects of doxacurium chloride (BW A938U) were evaluated in 27 children (2-12 yr) anaesthetized with 1\% halothane and nitrous oxide in oxygen. In nine children the incremental technique was used to establish a cumulative dose-response curve by train-of-four stimulation. The remaining children received either 30 or 50 micrograms kg-1 of the drug as a single bolus. The median ED50 and ED95 of doxacurium in children were 19 and 32 micrograms kg-1, respectively. No clinically significant change in heart rate or arterial pressure occurred. Following doxacurium 30 micrograms kg-1 and 50 micrograms kg-1, recovery to 25\% of control occurred in 25 (SEM 6) and 44 (3) min, respectively. The recovery index (25-75\% of control) was 27 (2) min. The duration of action of doxacurium is similar to that of tubocurarine and dimethyl-tubocurarine in children. Compared with adults, children seem to require more doxacurium (microgram kg-1) to achieve a comparable degree of neuromuscular depression, and they recover more rapidly. [\hyperlink{Dopamine Hydrochloride}{PMID: 2522789}, N G Goudsouzian et al., 1989]

\hypertarget{pmid_16224736}{T}o conduct an updated review of the mechanisms of action, pharmacokinetics, clinical effectiveness and safety of atomoxetine in the treatment of the symptoms of ADHD. Atomoxetine is the first of the group of non-stimulant drugs to be approved by the US Food and Drug Administration to treat this disorder in children, adolescents and adults. Atomoxetine has a direct effect on noradrenalin and dopamine concentrations by exerting a strong and highly selective inhibiting action on the pre-synaptic noradrenalin transporter, with a minimum affinity for other transporters and receptors. After adjustment of the dosage for body weight, the pharmacokinetic parameters are similar across all age and gender groups. Maximal plasma concentration is reached one to two hours after oral administration. Data concerning the effectiveness and safety from the clinical trials and studies reported in the literature are discussed. Atomoxetine is an effective and well-tolerated drug when used for the pharmacological treatment of ADHD symptoms. Despite being a drug that has only recently been developed, evidence from the large number of comparative studies that have been carried out endorse its widespread use in the treatment of this syndrome. [\hyperlink{Dopamine Hydrochloride}{PMID: 16224736}, J D Velásquez-Tirado et al., ]

\hypertarget{pmid_11392342}{A} growing body of literature implicates interactions between glutamatergic and neostriatal dopaminergic neurotransmitter systems in the development and expression of impulsivity, hyperactivity, and stereotypy. Amantadine hydrochloride, a drug used in young children for influenza prophylaxis, acts both as an indirect dopamine agonist as well as an N-methyl-D-aspartate (NMDA) receptor antagonist. Thus an open clinical trial of this medication for the treatment of symptoms of impulse control disorders in children was performed. A total of eight children (seven with neurodevelopmental disorders and all inpatients) with target behaviors refractory to other treatments were selected after parental informed consent. All patients were male and ranged in age from 4 to 12 years. Outcome was based on subjective consensus clinical ratings by the multidisciplinary treatment team. For four of the children, amantadine was associated with marked clinical improvement. In the remainder, improvement was also observed. Amantadine was well tolerated. On the basis of this experience, it appears that amantadine hydrochloride or related NMDA antagonists may warrant additional study in this and related populations. [\hyperlink{Dopamine Hydrochloride}{PMID: 11392342}, B H King et al., 2001]

\hypertarget{pmid_14749146}{A}ttention-deficit/hyperactivity disorder (ADHD) occurs in approximately 3\% to 10\% of the pediatric population. Most of the drugs typically used to treat ADHD are stimulants, which, because of their addictive properties and potential for abuse, are controlled substances. Although these drugs are the mainstay of treatment for ADHD, nearly one third of patients may not respond to or be able to tolerate them. Atomoxetine hydrochloride, a nonstimulant approved by the US Food and Drug Administration for the treatment of ADHD, may provide an alternative to the use of stimulants. The goal of this review was to describe the chemistry, mechanism of action, pharmacokinetics, drug interactions, and efficacy and safety profiles of atomoxetine in pediatric and adult patients with ADHD, as well as relevant pharmacoeconomic considerations. Relevant publications were identified through a search of the English-language literature indexed on PreMEDLINE and MEDLINE (1966-May 2003) using the search terms atomoxetine, tomoxetine, and LY139603. These terms were also applied to the Google search engine. All articles were reviewed for suitability for inclusion. The manufacturer of atomoxetine provided both published and unpublished data. In the data reviewed, atomoxetine was more efficacious than placebo in patients with ADHD (P<0.05 to P<0.01). Therapeutic doses ranged from 45 mg or placebo (P<0.05). These results add support to the hypothesis that atomoxetine may not cause the increase in dopamine concentrations in the nucleus accumbens that is associated with pleasurable effects and abuse potential. [\hyperlink{Dopamine Hydrochloride}{PMID: 14749146}, Joshua Caballero et al., 2003]

\hypertarget{pmid_15951862}{D}iagnostic and therapeutic procedures in children are made easier using sedation. However, there is no consensus about which drug should be used to achieve this. Furthermore, none of the drugs used for sedation are risk free. The aim of this work is to study sedation indications, effectiveness, and safety at our center. A prospective observational study conducted at the Pediatric Day Care Unit, King Fahad National Guard Hospital, Riyadh, Saudi Arabia. The study covered 17.5 weeks in 2 periods: May 9th 1999 to June 13th 1999 and October 31st 2001 to February 11th 2002. Children <12 years were included. Collected data included demographics, indication, drug dosing and outcome. Data were reported as mean +/- SD. We included 148 patients, age 38 +/- 30 months. Adequate sedation was achieved in 79\% after initial chloral hydrate (CH) dose of 56.9 +/- 9.3 mg/kg, in 95\% after adding 18.5 +/- 6.4 mg/kg CH and in 96\% after adding second drug. Compared to nonrespondents, first CH dose respondents were younger and lower in weight. The CH side effects were few and mild. Chloral hydrate is a safe and effective agent for sedation in children with an age and weight dependent response. [\hyperlink{Dopamine Hydrochloride}{PMID: 15951862}, Omar M Hijazi et al., 2005]

\hypertarget{pmid_10870004}{D}ocarpamine is a dopamine prodrug which has been selected from a large number of dopamine derivatives in order to develop an orally effective dopamine. The pharmacokinetics after oral administration of docarpamine have not yet been studied in children undergoing open heart surgery. This study examined the effects of docarpamine on hemodynamics and evaluated its safety in 11 children undergoing open heart surgery for congenital heart disease. This study began when the patientOs postoperative condition was stabilized by continuous dopamine infusion into the vein at a rate of 5 micro g/kg/min. The patients were administered 40 mg/kg of docarpamine every 8 hours, and hemodynamics were measured every 4 hours for 16 hours after the initial docarpamine administration. Immediately after the initial docarpamine administration, the dose of dopamine was reduced to 3 micro g/kg/min. Infusion of dopamine was stopped 8 hours after the initial docarpamine administration. Systemic systolic and diastolic blood pressure and heart rate showed no significant changes. Mean right atrial pressure decreased 4 hours after docarpamine administration. Mixed venous oxygen saturation and mean velocity of circumferential fiber shortening increased significantly after docarpamine administration. No significant changes were observed in urine volume. All patients could be weaned from dopamine within 8 hours. No changes were observed in ECG, and no arrhythmia-inducing action was noted. Our study indicates that 40 mg/kg oral doses of docarpamine produce plasma dopamine concentration equivalent to those of a 3 to 5 micro g/kg/min dopamine infusion. Our data suggest that docarpamine is a safe and effective drug for children who have undergone open heart surgery. [\hyperlink{Dopamine Hydrochloride}{PMID: 10870004}, S Watarida et al., 2000]

\hypertarget{pmid_2402648}{C}hloral hydrate has been used extensively to sedate children, but at Brooke Army Medical Center, other drug combinations were becoming increasingly popular due to a perception that chloral hydrate had a high rate of failure, especially with younger or neurologically impaired children. Therefore, 50 children were given the drug before a diagnostic study, and patient data and a sedation score were recorded on a worksheet. Of 50 children, 43 (86\%) were "successfully sedated" on the first attempt with no side effects. Children with neurologic disorders had a much greater (27\% vs 4\%) failure rate than "normal" children. The sedation rate did not significantly differ by age, sex, or initial drug dosage. The study suggest that chloral hydrate is a safe and effective oral sedative but that children with neurologic disorders may need alternative drugs for sedation. [\hyperlink{Dopamine Hydrochloride}{PMID: 2402648}, P D Rumm et al., 1990]

\hypertarget{pmid_8703459}{T}o evaluate neuromuscular potency of doxacurium during balanced anesthesia in pediatric patients. Prospective, consecutive sample trial. Operating room at a university hospital. 15 infants (1 to 11 months), 20 children (3 to 10 years), and 20 adolescents (13 to 19 years). Anesthesia was induced and maintained with thiopental, alfentanil, and nitrous oxide in oxygen. No volatile drugs were used at any time during the study. The neuromuscular function was recorded as adductor pollicis electromyography evoked by a train-of-four stimulation at 20-second intervals. A cumulative log-dose probit-response curve of doxacurium was established for every patient. ED50 and ED95 doses of doxacurium (14 micrograms/kg and 25 micrograms/kg in infants, 26 micrograms/kg and 53 micrograms/kg in children, and 20 micrograms/kg and 41 micrograms/kg in adolescents, respectively) were smallest in infants and greatest in children (p < 0.05 between each pair of groups by analysis of variance and Scheffe's F-test). Potency of doxacurium was greatest in infants and least in children. We suggest that doxacurium can be administered safely in infants, and with dosages close to those reported in adults. Children's dose requirement was almost 50\% greater than that of infants. [\hyperlink{Dopamine Hydrochloride}{PMID: 8703459}, T R Taivainen et al., 1996]

\hypertarget{pmid_20040824}{T}o assess the long-term safety and tolerability of atomoxetine hydrochloride in children and adolescents with attention-deficit/hyperactivity disorder treated for > or = 3 years. Data from 13 double-blind, placebo-controlled trials and 3 open-label extension studies were pooled. Outcome measures were patient-reported treatment-emergent adverse events (AEs); discontinuations due to AEs, serious AEs, and changes in body weight, height, vital signs, electrocardiogram, and hepatic function tests. In total, 714 patients were treated with atomoxetine for > or = 3 years (mean follow-up 4.8 years [SD 1.1 years]), including a subset of 508 treated for > or = 4 years (mean follow-up 5.3 years [SD 0.8 years]). Most subjects were younger than 12 years at entry (73.8\%), male (78.4\%), and white (88.9\%). The mean final daily dose of atomoxetine was 1.35 mg/kg (SD 0.37 mg/kg). No new or unexpected AEs were observed compared with acute-phase treatment. Less than 6\% of patients exhibited aggressive/hostile behaviors, and less than 1.6\% reported suicidal ideation/behavior. No clinically significant effects were seen on growth rate, vital signs, or electrocardiographic parameters, and < or = 2\% of patients showed potentially clinically significant hepatic changes. Atomoxetine was safe and well tolerated for children and adolescents with > or = 3 and/or > or = 4 years of treatment. [\hyperlink{Dopamine Hydrochloride}{PMID: 20040824}, Craig Donnelly et al., 2009]

\hypertarget{pmid_8708264}{P}emoline, a dopamine agonist, is effective in children with attention deficit hyperactivity disorder (ADHD), but its efficacy in adults is unknown. The authors studied the efficacy and safety of pemoline, using retrospective chart review of treated students with ADHD over a 2-year period. Forty students met diagnostic and treatment criteria; pemoline was associated with much improved or very much improved Clinical Global Impression symptoms scores in 70\% of the students during a treatment period of 14 or more days. Severity of illness scores dropped from 4.11 to 3.01 between baseline and subsequent evaluation. Nine evaluable patients had adverse events, most commonly headaches, insomnia, and decreased appetite. Five additional students, who failed to meet the treatment-duration criterion, terminated because of severe initial insomnia. The authors concluded that pemoline is effective and safe in students with ADHD and has a lower abuse potential than methylphenidate and dextroamphetamine, the other two widely used, structurally dissimilar compounds, but controlled studies may be necessary before any final conclusions are reached. [\hyperlink{Dopamine Hydrochloride}{PMID: 8708264}, E Heiligenstein et al., 1996]

\hypertarget{pmid_22246409}{C}hloral hydrate (CH) is safe and effective for sedation of suitable children. The purpose of this study was to assess whether adequate sedation is achieved with reduced CH doses. We retrospectively recorded outpatient CH sedations over 1 year. We defined standard doses of CH as 50 mg/kg (infants) and 75 mg/kg (children >1 year). A reduced dose was defined as at least 20\% lower than the standard dose. In total, 653 children received CH sedation (age, 1 month-3 years 10 months), 42\% were given a reduced initial dose. Augmentation dose was required in 10.9\% of all children, and in a higher proportion of children >1 year (15.7\%) compared to infants (5.7\%; P < 0.001). Sedation was successful in 96.7\%, and more frequently successful in infants (98.3\%) than children >1 year (95.3\%; P = 0.03). A reduced initial dose had no negative effect on outcome (P = 0.19) or time to sedation. No significant complications were seen. We advocate sedation with reduced CH doses (40 mg/kg for infants; 60 mg/kg for children >1 year of age) for outpatient imaging procedures when the child is judged to be quiet or sleepy on arrival. [\hyperlink{Dopamine Hydrochloride}{PMID: 22246409}, Jennifer Bracken et al., 2012]

\hypertarget{pmid_20644039}{P}ropranolol hydrochloride has been prescribed for decades in the pediatric population for a variety of disorders, but its effectiveness in the treatment of infantile hemangiomas (IHs) was only recently discovered. Since then, the use of propranolol for IHs has exploded because it is viewed as a safer alternative to traditional therapy. We report the cases of 3 patients who developed symptomatic hypoglycemia during treatment with propranolol for their IHs and review the literature to identify other reports of propranolol-associated hypoglycemia in children to highlight this rare adverse effect. Although propranolol has a long history of safe and effective use in infants and children, understanding and recognition of deleterious adverse effects is critical for physicians and caregivers. This is especially important when new medical indications evolve as physicians who may not be as familiar with propranolol and its adverse effects begin to recommend it as therapy. [\hyperlink{Dopamine Hydrochloride}{PMID: 20644039}, Kristen E Holland et al., 2010]

\hypertarget{pmid_31230222}{T}he safety and efficacy of a novel combination treatment of AChE inhibitors and choline supplement was initiated and evaluated in children and adolescents with autism spectrum disorder (ASD). Safety and efficacy were evaluated on 60 children and adolescents with ASD during a 9-month randomized, double-blind, placebo-controlled trial comprising 12 weeks of treatment preceded by baseline evaluation, and followed by 6 months of washout, with subsequent follow-up evaluations. The primary exploratory measure was language, and secondary measures included core autism symptoms, sleep and behavior. Significant improvement was found in receptive language skills 6 months after the end of treatment as compared to placebo. The percentage of gastrointestinal disturbance reported as a side effect during treatment was higher in the treatment group as compared to placebo. The treatment effect was enhanced in the younger subgroup (younger than 10 years), occurred already at the end of the treatment phase, and was sustained at 6 months post treatment. No significant side effects were found in the younger subgroup. In the adolescent subgroup, no significant improvement was found, and irritability was reported statistically more often in the adolescent subgroup as compared to placebo. Combined treatment of donepezil hydrochloride with choline supplement demonstrates a sustainable effect on receptive language skills in children with ASD for 6 months after treatment, with a more significant effect in those under the age of 10 years. [\hyperlink{Dopamine Hydrochloride}{PMID: 31230222}, Lidia V Gabis et al., 2019]

\hypertarget{pmid_17803435}{T}he aim of this study was to evaluate the safety of olopatadine hydrochloride ophthalmic solution 0.2\% in children and adolescents 3-17 years of age. In this 6-week, randomized, double-masked safety evaluation, eligible subjects with asymptomatic eyes underwent in-office visits at weeks 1, 3, and 6 and were contacted by telephone at weeks 2, 4, and 5. Qualified subjects were assigned randomly in a 2:1 ratio of olopatadine 0.2\% to vehicle (identical formation without the active ingredient) for dosing on a once-daily schedule. Safety parameters assessed included adverse events, visual acuity, ocular signs (slit-lamp assessments), dilated fundus examinations, intraocular pressure (IOP), pulse, and blood pressure. An evaluation of 126 subjects (age range, 3-17) revealed no clinically relevant treatment-related changes in visual acuity, IOP, slit-lamp assessments, fundus examinations, or cardiovascular parameters. All adverse events reported were mild or moderate. Olopatadine 0.2\% administered once-daily for 6 weeks is safe and well tolerated in children and adolescent patients. [\hyperlink{Dopamine Hydrochloride}{PMID: 17803435}, Steven J Lichtenstein et al., 2007]

\hypertarget{pmid_404123}{A}lthough the FDA recommends imipramine hydrochloride (IMI) only for temporary relief of symptoms of enuresis nocturna (EN), the drug has been applied to a number of other pediatric situations, including the Hyperkinetic Syndrome (HS), childhood depression, somnambulism and pavor nocturnus, school phobia, petit mal epilepsy, allergies, autism, encorpresis and head-banging. We have reviewed the literature, with particular attention to the pharmacokinetics of IMI in children, and its putative mechanisms of action. The drug probably works through a number of different actions, and the futher delineation of these will be of considerable heuristic value. We review the toxic effects of IMI treatment and IMI poisoning in children, and the pediatric literature concerning other antidepressant drugs and lithium carbonate (Li). [\hyperlink{Dopamine Hydrochloride}{PMID: 404123}, C T Gualtieri et al., 1977]

\hypertarget{pmid_2295577}{F}luoxetine hydrochloride is the first selective serotonin uptake inhibitor introduced commercially in the United States. This report describes preliminary clinical experience with fluoxetine in 10 children and adolescents, aged 8 to 15 years, with primary obsessive compulsive disorder (OCD) or Tourette's syndrome (TS) plus OCD. In general, fluoxetine, which was administered from 4 to 20 weeks at a dosage of 10 or 40 mg per day, was well tolerated. Adverse effects included behavioral agitation/activation in four patients and mild gastrointestinal symptoms in two patients. No abnormalities were noted in the seven children who had follow-up EKGs. Five of the 10 patients (50\%) were considered responders; their obsessive-compulsive symptoms decreased substantially during treatment with fluoxetine. Responder rates were similar in the primary OCD (two of four, 50\%) and TS + OCD (three of six, 50\%) groups. In conclusion, short-term fluoxetine administration appears to be safe in children and adolescents. Placebo-controlled trials are needed to further assess the efficacy of fluoxetine. [\hyperlink{Dopamine Hydrochloride}{PMID: 2295577}, M A Riddle et al., 1990]

\section*{Erythromycin Lactobionate}
\subsection*{Result}
\subsubsection*{Answer}

Yes (preterm neonates ≤15 days old, ≤1500g)
Unknown (other pediatric age groups)

\subsubsection*{{Explanation}}
\hypertarget{Erythromycin Lactobionate}
To determine if Erythromycin Lactobionate is safe for use in children, I reviewed the available abstracts for targeted studies on its safety in pediatric populations. The following summarizes the relevant evidence by age group:

1. **Preterm Neonates (≤15 days old, birth weight ≤1500g):**
   - A study specifically investigated erythromycin lactobionate in 14 preterm neonates (≤15 days old, ≤1500g) for treatment of Ureaplasma urealyticum. Doses of 25 or 40 mg/kg/day were administered intravenously for 10 days. The study reports: "No adverse effects thought to be related to administration of erythromycin were observed. These preliminary findings showed that erythromycin is well-tolerated, has favorable pharmacokinetic activity in the preterm neonate and should be further investigated for treatment of ureaplasmal infections" [\hyperlink{pmid_8036045}{PMID: 8036045}, K B Waites et al., 1994]. This provides direct evidence of safety in this specific neonatal population, though the sample size is small and the authors recommend further investigation.

2. **Critically Ill Children (age not specified, but not neonates):**
   - A randomized controlled trial evaluated erythromycin lactobionate (10 mg/kg IV) in 74 critically ill children to facilitate feeding tube placement. The study reports no mention of adverse effects or safety concerns, but the primary outcome was efficacy, not safety. The abstract does not explicitly affirm safety, nor does it report adverse events [\hyperlink{pmid_11334066}{PMID: 11334066}, V Gharpure et al.]. Therefore, this study does not provide definitive evidence for or against safety in this group.

3. **Other Pediatric Populations:**
   - Several studies examine erythromycin (various formulations, including ethylsuccinate and estolate) in infants, children, and adolescents for different indications. These studies generally report good tolerability and no serious adverse effects, but do not specifically address erythromycin lactobionate or are not targeted safety studies for this formulation.

4. **General Note:**
   - One review notes that "no serious adverse effects have been reported in studies in which erythromycin has been used for its prokinetic effects, although fatal reactions have followed the intravenous administration of erythromycin to neonates in antibiotic doses" [\hyperlink{pmid_11328252}{PMID: 11328252}, J I Curry et al., 2001]. However, this is a general statement and not a targeted safety study.

**Summary:**  
- For preterm neonates (≤15 days old, ≤1500g), there is a small targeted study supporting the safety of erythromycin lactobionate for short-term use.
- For other pediatric age groups, there is insufficient targeted safety data for erythromycin lactobionate specifically; thus, safety is unknown.

\subsection*{Abstracts}
\hypertarget{pmid_11328252}{E}rythromycin has been used as an antibiotic for more than four decades, but only in the last 10 years have other therapeutic benefits of this agent been exploited. Animal and human studies have demonstrated a prokinetic effect on the gastrointestinal tract at sub-antimicrobial doses (typically a quarter or less of the antibiotic dose). A limited number of studies have been performed in children to investigate this action. A review of this literature is particularly pertinent given the frequency of clinical problems related to gastrointestinal dysmotility in children and the limited availability of prokinetic agents in paediatric practice, compounded by the recent withdrawal of cisapride. The prokinetic effects of erythromycin have been investigated in infants with dysmotility associated with prematurity, in low birth-weight infants recovering from abdominal surgery, and in older children with a variety of other gastrointestinal disorders. Only one randomized placebo-controlled trial has been conducted. All except one of these studies have shown a beneficial effect of erythromycin in either promoting tolerance of enteral feeds or enhancing a measured index of gastrointestinal motility. Erythromycin appears to be equally effective when given orally (as ethylsuccinate or estolate) or intravenously (as lactobionate). Significantly, no serious adverse effects have been reported in studies in which erythromycin has been used for its prokinetic effects, although fatal reactions have followed the intravenous administration of erythromycin to neonates in antibiotic doses. [\hyperlink{Erythromycin Lactobionate}{PMID: 11328252}, J I Curry et al., 2001]

\hypertarget{pmid_6349401}{A} double-blind placebo-controlled trial of erythromycin ethylsuccinate was conducted in 65 infants and young children hospitalized with acute nonspecific gastroenteritis. Etiologic agents included rotaviruses (29\%), Campylobacter jejuni (17\%), "classical" enteropathogenic Escherichia coli (12\%), enterotoxigenic E. coli (11\%), Salmonella (9\%), Shigella (2\%), and Giardia lamblia (2\%). No pathogens were obtained from 25 (38\%) children. Treatment with erythromycin had no effect on the course of the illness in terms of the time required for hydration, stool frequency and temperature to return to normal, or for vomiting to be abolished. Children treated with erythromycin, however, experienced a marginally, but significantly (P less than 0.05), shorter period of abnormal stool consistency compared with control subjects. This effect was most pronounced in children from whom no enteropathogens were isolated. [\hyperlink{Erythromycin Lactobionate}{PMID: 6349401}, R M Robins-Browne et al., 1983]

\hypertarget{pmid_14502372}{T}he efficacy of erythromycin was assessed in the treatment of 14 children aged 4 to 13 years with refractory chronic constipation, and presenting megarectum and fecal impaction. A double-blind, placebo- controlled, crossover study was conducted at the Pediatric Gastroenterology Outpatient Clinic of the University Hospital. The patients were randomized to receive placebo for 4 weeks followed by erythromycin estolate, 20 mg kg-1 day-1, divided into four oral doses for another 4 weeks, or vice versa. Patient outcome was assessed according to a clinical score from 12 (most severe clinical condition) to 0 (complete recovery). At enrollment in the study and on the occasion of follow-up medical visits at two-week intervals, patient score and laxative requirements were recorded. During the first 30 days, the mean SD clinical score for the erythromycin group (N = 6) decreased from 8.2+/-2.3 to 2.2+/-1.0 while the score for the placebo group (N = 8) decreased from 7.8+/-2.1 to 2.9+/-2.8. During the second crossover phase, the score for patients on erythromycin ranged from 2.9+/-2.8 to 2.4+/-2.1 and the score for the patients on placebo worsened from 2.2+/-1.0 to 4.3+/-2.3. There was a significant improvement in score when patients were on erythromycin (P < 0.01). Mean laxative requirement was lower when patients ingested erythromycin (P < 0.05). No erythromycin-related side effects occurred. Erythromycin was useful in this group of severely constipated children. A larger trial is needed to fully ascertain the prokinetic efficacy of this drug as an adjunct in the treatment of severe constipation in children. [\hyperlink{Erythromycin Lactobionate}{PMID: 14502372}, M A Bellomo-Brandão et al., 2003]

\hypertarget{pmid_8036045}{E}rythromycin is receiving renewed attention as an alternative for treatment of neonatal infections caused by Ureaplasma urealyticum because of recently proved abilities of this organism to produce systemic disease in this population. Although erythromycin has been used clinically for almost 40 years, very little is known about its activity in the preterm neonate. Fourteen neonates, birth weights < or = 1500 g and < or = 15 days of age, from whom U. urealyticum was isolated from the lower respiratory tract were randomized to receive erythromycin lactobionate either 25 or 40 mg/kg/day in four divided doses at 6-hour intervals scheduled for a total of 10 days. Blood samples collected at multiple time points after initial and steady state doses were assayed for erythromycin by liquid chromatography. Minimal inhibitory concentrations (MICs) of erythromycin for the U. urealyticum isolates were determined. MICs ranged from 0.031 to 2 micrograms/ml; MIC90 = 2 micrograms/ml. Serum erythromycin concentrations met or exceeded most MICs, with peak values of 3.05 to 3.69 and 1.92 to 2.9 micrograms/ml for the 40- and 25-mg/kg/day dosage groups, respectively. Pharmacokinetic parameters were calculated after the initial dose and at steady state for both dosage groups and compared. No adverse effects thought to be related to administration of erythromycin were observed. These preliminary findings showed that erythromycin is well-tolerated, has favorable pharmacokinetic activity in the preterm neonate and should be further investigated for treatment of ureaplasmal infections. [\hyperlink{Erythromycin Lactobionate}{PMID: 8036045}, K B Waites et al., 1994]

\hypertarget{pmid_8504017}{R}esults of studies conducted to characterise local, systemic, reproductive, and mutagenic effects indicate that the new macrolide antimicrobial clarithromycin is well tolerated within reasonable multiples of the intended clinical dose. No adverse effects of clarithyromycin on male or female fertility, perinatal, or postnatal reproduction were indicated by data from rabbits, mice, rats and macaques. No evidence of mutagenic potential was revealed from various in vitro and in vivo study methodologies. Evidence of low potential for ototoxicity, oculotoxicity, hepatotoxicity and nephrotoxicity was provided in studies involving rats, dogs and primates. In agreement with studies with other macrolides, venous irritation potential for the intravenous lactobionate salt formulation was substantial in rabbit studies. In addition, the safety profile of this agent has been evaluated on the basis of adverse reactions and abnormal laboratory values seen in phase I, II and III international clinical trials conducted in adults. The most frequently reported adverse reactions occurring in 3768 patients receiving clarithromycin in phase II and III trials were nausea (3.8\%), diarrhoea (3.0\%), abdominal pain (1.9\%) and headache (1.7\%). Adverse reactions were not serious and were usually rapidly reversible. The incidence of adverse reactions did not vary with gender, race or age. Adverse reaction rates were comparable to or less than those of comparator beta-lactams and macrolides. Overall, clarithromycin appears to be a safe and well-tolerated macrolide antimicrobial agent. [\hyperlink{Erythromycin Lactobionate}{PMID: 8504017}, D R Guay et al., 1993]

\hypertarget{pmid_18789096}{C}hronic bullous disease of childhood is the commonest acquired blistering disorder of children. Erythromycin has been reported to be beneficial for this condition. A three question survey was e-mailed to all members of the British Society for Paediatric Dermatology to assess the incidence, preferred treatments and experience of oral erythromycin in treating chronic bullous disease of childhood. A second, more detailed questionnaire was sent to members who had used erythromycin. Forty patients were reported to have been treated over the previous 2 years. The preferred treatment was dapsone. Erythromycin alone had been used in five children as first-line oral treatment. In three of these patients the initial improvement was graded as either "good" or "complete resolution." This benefit was only sustained in one child, with the other two relapsing between 4 and 12 weeks. In a further eight children, erythromycin had been used with other oral agents. In five of these children, erythromycin was associated with long-term benefit. These results suggest that erythromycin is unlikely to produce sustained improvement in chronic bullous disease of childhood when used as a sole first-line agent. However, erythromycin can cause an initial improvement, which may be useful whilst awaiting results of diagnostic tests and may confer benefit when used with other systemic treatments. [\hyperlink{Erythromycin Lactobionate}{PMID: 18789096}, Paul Farrant et al., ]

\hypertarget{pmid_20687891}{T}he number of children suffering from atopic eczema has increased over the past 30 years especially in children between the ages of 2 and 5 years. These is a significant group of eczematous children that are resistant to standard therapy. Babies and children with eczema suffer pain, irritation and disfigurement from the dermatitis. In this study, we have followed 14 cases of pediatric patients (ages of 8 months to 64 months) with a history of resistant eczema for a period of at least six months. All of these children received 300 mg to 500 mg standardized Lactobacillus rhamnosus cell lysate daily as an immunobiotic supplement. The results of this open label non-randomized clinical observation showed a substantial improvement in quality of life, skin symptoms and day- and nighttime irritation scores in children with the supplementation of Lactobacillus rhamnosus lysate. There were no intolerance or adverse reactions observed in these children. Lactobacillus rhamnosus cell lysate may thus be used as a safe and effective immunobiotic for the treatment and prevention of childhood eczema and possible other types of atopy (allergic diseases). [\hyperlink{Erythromycin Lactobionate}{PMID: 20687891}, Ba X Hoang et al., 2010]

\hypertarget{pmid_36827282}{E}rythromycin is a macrolide antibiotic that is also prescribed off-label in premature neonates as a prokinetic agent. There is no oral formulation with dosage and/or excipients adapted for these high-risk patients. Clinical studies of erythromycin as a prokinetic agent were reviewed. Capsules of 20 milligrams of erythromycin were compounded with microcrystalline cellulose. Erythromycin capsules were analyzed using the chromatographic method described in the United States Pharmacopoeia which was found to be stability-indicating. The stability of 20 mg erythromycin capsules stored protected from light at room temperature was studied for one year. 20 mg erythromycin capsules have a beyond use date not lower than one year. 20 milligrams erythromycin capsules can be compounded in batches of 300 unities in hospital pharmacy with a beyond-use-date of one year at ambient temperature protected from light. [\hyperlink{Erythromycin Lactobionate}{PMID: 36827282}, Patrick Thevin et al., 2023]

\hypertarget{pmid_792407}{E}rythromycin continues to be a valuable and useful antimicrobial agent in children. Its low index of toxicity, freedom from sensitization, and reliable absorption and when administered orally contribute to make it an attractive agent in the treatment of a variety of minor respiratory and skin infections, especially in those situations where real or potential allergy to penicillin exists. Additional major uses are in the eradication of the carrier state in whooping cough and in diphtheria, especially in those instances when oral therapy can be tolerated. Dispite use over more than two decades, resistance developing in formerly susceptible organisms has not been a problem and thus seems unlikely to become so in the future. [\hyperlink{Erythromycin Lactobionate}{PMID: 792407}, C M Ginsburg et al., 1976]

\hypertarget{pmid_31321320}{A}zithromycin is widely used in children not only in the treatment of individual children with infectious diseases, but also as mass drug administration (MDA) within a community to eradicate or control specific tropical diseases. MDA has also been reported to have a beneficial effect on child mortality and morbidity. However, concerns have been raised about the safety of azithromycin, especially in young children. The aim of this review is to systematically identify the safety of azithromycin in children of all ages. MEDLINE, PubMed, Cochrane Central Register of Controlled Trials, Embase, CINAHL, International Pharmaceutical Abstracts and adverse drug reaction (ADR) monitoring systems will be systematically searched for randomised controlled trials (RCTs), cohort studies, case-control studies, cross-sectional studies, case series and case reports evaluating the safety of azithromycin in children. The Cochrane risk of bias tool, Newcastle-Ottawa and quality assessment tools, and The Joanna Briggs Institute Critical Appraisal tools will be used for quality assessment. Meta-analyses will be conducted to the incidence of ADRs from RCTs if appropriate. Subgroup analyses will be performed in different age and azithromycin dosage groups. Formal ethical approval is not required as no primary data are collected. This systematic review will be disseminated through a peer-reviewed publication. CRD42018112629. [\hyperlink{Erythromycin Lactobionate}{PMID: 31321320}, Peipei Xu et al., 2019]

\hypertarget{pmid_22348492}{E}rythromycin is generally used as a prokinetic agent for the treatment of feeding intolerance in preterm infants; however, results from previous studies significantly vary due to different medication dosages, routes of administration, and therapy durations. The effectiveness and safety of intermediate-dose oral erythromycin in very low birth weight (VLBW) infants with feeding intolerance was examined in this study. Between November 2007 and August 2009, 45 VLBW infants with feeding intolerance, who were all at least 14 days old, were randomly allocated to a treatment group and administered 5mg/kg oral erythromycin every 6hours for 14 days (n=19). Another set of randomly selected infants was allocated to the control group, which was not administered erythromycin (n=26). The number of days required to achieve full enteral feeding (36.5±7.4 vs. 54.7±23.3 days, respectively; p=0.01), the duration of parenteral nutrition (p<0.05), and the time required to achieve a body weight ≥2500g (p<0.05) were significantly shorter in the erythromycin group compared with the control group. The incidence of parenteral nutrition-associated cholestasis (PNAC) and necrotizing enterocolitis (NEC) ≥ stage II after 14 days of treatment were significantly lower (p<0.05) in the erythromycin group. No significant differences were observed in terms of the incidences of sepsis, bronchopulmonary dysplasia, or retinopathy of prematurity. No adverse effects were associated with erythromycin treatment. Intermediate-dose oral erythromycin is effective and safe for the treatment of feeding intolerance in VLBW infants. The incidences of PNAC and ≥ stage II NEC were significant lower in the erythromycin group. [\hyperlink{Erythromycin Lactobionate}{PMID: 22348492}, Yan-Yan Ng et al., 2012]

\hypertarget{pmid_8295812}{C}larithromycin is a new macrolide antibiotic that is active in vitro against a variety of organisms that are responsible for acute otitis media in children. The parent compound is metabolized to microbiologically active 14-hydroxy clarithromycin, which is especially active against Haemophilus influenzae. The safety and efficacy of clarithromycin and amoxicillin suspensions were compared in the treatment of acute otitis media in children 1 to 12 years of age inclusive. This was a Phase III, single blind (investigator-blind), randomized, multicenter clinical trial. Clarithromycin oral suspension was given in a dose of 7.5 mg/kg (maximum, 500 mg) twice daily, and amoxicillin suspension in a dose of 20 mg/kg (maximum, 750 mg) was given twice daily for 7 to 10 days in a 1:1 ratio. Clinical evaluations were performed pretreatment, within 48 hours posttreatment and 10 to 14 days posttreatment. Myringotomy was performed in every child to obtain a microbiologic sample pretreatment and at subsequent visits as clinically indicated. A total of 79 children were enrolled, 39 in the clarithromycin and 40 in the amoxicillin treatment group. Thirty-two children were excluded from the efficacy analysis for various reasons. Clinical success (cure and improvement) rates at 0 to 4 days posttreatment were 93\% for clarithromycin and 90\% for amoxicillin (P > 0.999). Altogether 17 children (10 receiving clarithromycin, 7 receiving amoxicillin) experienced some adverse event, with gastrointestinal disorders being the most common complaint. No clinically significant differences in laboratory tests were found between the groups.(ABSTRACT TRUNCATED AT 250 WORDS) [\hyperlink{Erythromycin Lactobionate}{PMID: 8295812}, J S Pukander et al., 1993] We evaluated 260 previously healthy children ages 3 through 12 years who had clinical signs and symptoms of pneumonia, radiographically confirmed. Patients were randomized 1:1 to a 10-day course of either clarithromycin suspension 15 mg/kg/day divided twice a day or erythromycin suspension 40 mg/kg/day divided twice a day or three times a day. Evidence of infection with Chlamydia pneumoniae was detected in 28\% (74) of patients: 13\% (34) by nasopharyngeal culture and 18\% (48) by serology with the microimmunofluorescence assay. Evidence of infection with Mycoplasma pneumoniae was detected in 27\% (69) of patients: 20\% (53) by nasopharyngeal culture or polymerase chain reaction and 17\% (44) by serology with the use of enzyme-linked immunosorbent assay. Serologic confirmation of infection was observed in 23\% (8) and 53\% (28) of patients with bacteriologically detected C. pneumoniae and M. pneumoniae, respectively. Treatment with clarithromycin vs. erythromycin, respectively, yielded the following outcomes: clinical success 98\% (121 of 124) vs. 95\% (105 of 110); radiologic success 98\% (109 of 111) vs. 94\% (92 of 110); and eradication by pathogen, C. pneumoniae 79\% (15 of 19) vs. 86\% (12 of 14) and M. pneumoniae 100\% (9 of 9) vs. 100\% (4 of 4). Adverse events were primarily gastrointestinal occurring in almost one-fourth of patients in both groups, and were mild to moderate in severity. Clarithromycin and erythromycin were similarly effective and safe for the treatment of radiographically proved, community-acquired pneumonia in children older than 2 years old.(ABSTRACT TRUNCATED AT 250 WORDS) [\hyperlink{Erythromycin Lactobionate}{PMID: 8295812}, S Block et al., 1995] Studies were conducted on in vivo pharmacokinetics of clarithromycin (TE-031, A-56268) in children and also on the efficacy and the safety of this macrolide antibiotic in the treatment of bacterial infections in children. The results obtained are summarized as follows: 1. TE-031 granules were orally administered to 5 children in a dosage of 5 mg/kg before meal. Maximum drug concentrations (range: 0.29-2.0 micrograms/ml) in the serum occurred during a period from 30 minutes to 1 hour after administration, but there were clear differences in blood concentrations among the individuals. 2. TE-031 granules were orally administered in a average dosage of 20 mg/kg/day to a total of 17 patients, consisting of 14 children with respiratory tract infections and 3 children with intestinal infections. The clinical efficacy evaluation resulted in 10 excellent cases, 6 good cases and 1 fair case, for an efficacy rate of 94.1\%. 3. Studies on the bacterial efficacy were carried out for 10 cases. The TE-031 bacteriological efficacy evaluation showed elimination in 7 cases, a decreased bacterial count in 2 cases, and no change in 1 case. The elimination rate was, thus, 70.0\%. Elimination rates according to different species of bacteria were 66.7\% (2 of 3 strains) for Staphylococcus aureus, 100\% for both Streptococcus pneumoniae (3 of 3) and Streptococcus pyogenes (1 of 1), and 42.9\% (3 of 7) for Haemophilus influenzae. 4. There were no symptoms which were attributable to side effects of the TE-031 therapy. The only laboratory test abnormality detected was eosinophilia in 1 patient. [\hyperlink{Erythromycin Lactobionate}{PMID: 8295812}, M Hayashi et al., 1989]

\hypertarget{pmid_6885176}{A} double-blind study was designed to test the hypothesis that local side-effects during i. v. administration of erythromycin lactobionate depend on the drug concentration and that they can therefore be minimized by dissolving erythromycin in a larger infusion volume. Forty healthy students were assigned in a randomized sequence to four 30 min infusions: 120 and 250 ml of erythromycin lactobionate (1 g in 0.9\% NaCl) and 120 and 250 ml of placebo (0.9\% NaCl). An unexpectedly high incidence (95\% and 80\% for the infusion volumes of 120 and 250 ml, respectively) of severe systemic side-effects was observed during the first 79 infusions. Because all of these systemic side-effects were associated with the infusion of erythromycin, the study was terminated at this point. Side-effects included abdominal cramps, nausea, vomiting, dizziness and profuse sweating. The postulated positive effect of lower erythromycin concentrations in the infusion on local side-effects (pain at the infusion site, erythema) was marginal (63\% vs. 45\%). Compared to the systemic side-effects, the problem of local tolerance is less important. In young adults, 30 min infusions of 1 g erythromycin lactobionate are associated with a high incidence of systemic side-effects which may be due to an age-dependent effect of the drug on smooth muscle. [\hyperlink{Erythromycin Lactobionate}{PMID: 6885176}, R Putzi et al., ]

\hypertarget{pmid_10724028}{C}hildren infected with Chlamydia pneumoniae sometimes experience lower respiratory tract infections such as pneumonia and bronchitis. Although numerous anti-microbial compounds have been reported to be active against the organism, most of them have not been in a clinical trial in infants and children with C. pneumoniae infection. Clarithromycin has been shown to express anti-chlamydial effects in vitro. In this study, we evaluated the clinical anti-C. pneumoniae properties of clarithromycin in children with mainly lower respiratory tract infection. We administered clarithromycin orally to 21 infants and children at a dose of 10-15 mg/kg/day divided into two or three doses for 4-21 days. Clinical symptoms, roentgenographic and laboratory abnormal findings improved. The overall clinical efficacy rate was 85.7\% (18 of 21 cases). Administration of clarithromycin was considered to be a suitable treatment for improving lower respiratory infections in infants and children caused by C. pneumoniae. [\hyperlink{Erythromycin Lactobionate}{PMID: 10724028}, K Numazaki et al., 2000]

\hypertarget{pmid_3534749}{E}rythromycin ethyl succinate is an antibiotic frequently administered in pediatrics. According to some authors, this drug sharply decreases the fecal count of enterobacteria. The fecal flora of 12 infants less than one year old, treated by erythromycin ethyl succinate for 7 to 10 days was studied by differential count. A variable effect was observed on enterobacteria: a 10(3) to 10(5) fold reduction in 9 cases with a final count superior or equal to 10(4) per gram of feces, with or without coming back to the initial count; in 3 cases no modification. MIC of enterobacteria and concentrations of erythromycin in feces were not predictives of flora variation. Anaerobic flora was weakly modified. No implantation of potentially-pathogenic bacteria or multi-resistant or highly erythromycin resistant enterobacteria occurred. Thus, erythromycin ethyl succinate is valuable in pediatrics as it does not disturb barrier effects. But its use for selective decontamination of gut must be discussed depending on pharmacologic form and posology administered. [\hyperlink{Erythromycin Lactobionate}{PMID: 3534749}, M J Butel et al., 1986]

\hypertarget{pmid_7049959}{F}ollowing a study in which the etiology of nearly 70\% of 142 cases of pneumonia in children could be determined using a combination of bacteriological and serological methods, the effect of erythromycin ethylsuccinate was compared with that of amoxicillin in a randomized study on 120 cases of pneumonia. We first examined the tracheal secretion microbiologically and determined other serological parameters and clinical data. The tracheal secretion was sterile in only 19\% of the cases. We were able to identify the etiology in 64\% of the cases using a combination of microbiological and serological methods. A discontinuation of therapy and acceptable side-effects were considerably more frequent with amoxicillin than with erythromycin ethylsuccinate (75 mg/kg body weight). The advantages of erythromycin, especially for the initial therapy of pneumonia, and the improvements in diagnosis resulting from the examination of the tracheal secretion will be discussed. [\hyperlink{Erythromycin Lactobionate}{PMID: 7049959}, H Ruhrmann et al., 1982]

\hypertarget{pmid_21269858}{A} randomized, double-blind, double-dummy, multicenter international study was conducted to assess the clinical and bacteriologic response, safety, and compliance of a single 60-mg/kg dose of azithromycin extended-release (ER) versus a 10-day regimen of amoxicillin/clavulanate 90/6.4 mg/kg per day in children with acute otitis media at high risk of persistent or recurrent middle ear infection. Children aged 3 to 48 months were enrolled and stratified into two age groups (≤ 24 months and >24 months). Pretreatment tympanocentesis was performed at all sites and was repeated during treatment at selected sites. The primary endpoint, clinical response at the test-of-cure visit in the bacteriologic eligible population, was achieved in 80.5\% of children in the azithromycin ER group and 84.5\% of children in the amoxicillin/clavulanate group (difference-3.9\%; 95\% confidence interval-10.4, 2.6). Bacteriologic eradication was 82.6\% in the azithromycin ER group and 92\% in the amoxicillin/clavulanate group (p=0.050). Children who received amoxicillin/clavulanate had significantly higher rates of dermatitis and diarrhea, a greater burden of adverse events, and a lower rate of compliance to study drug compared to those who received azithromycin ER. A single 60-mg/kg dose of azithromycin ER provides near equivalent effectiveness to a 10-day regimen of amoxicillin/clavulanate 90/6.4 mg/kg per day in the treatment of children with acute otitis media. [\hyperlink{Erythromycin Lactobionate}{PMID: 21269858}, Adriano Arguedas et al., 2011]

\hypertarget{pmid_6975059}{W}e studied the pharmacokinetics of erythromycin estolate and ethylsuccinate suspensions in infants under 4 months of age who were being treated for chlamydial infections or pertussis. We conducted our studies after the initial dose of 10 mg/kg and subsequently during steady-state treatment. The estolate preparation resulted in higher peak concentrations in sera, and its absorption and elimination half-lives were longer. Peak concentrations occurred 3 h after a dose with the estolate preparation and 1 h after a dose with the ethylsuccinate preparation. The area under the curve for the estolate preparation was about three times greater than that for the ethylsuccinate preparation. Based on these findings, we recommend that erythromycin estolate suspensions be given to young infants at 8- or 12-h intervals (30 mg/kg per day in three divided doses or 20 mg/kg per day in two divided doses) and that erythromycin ethylsuccinate is best given at 6-h intervals (40 mg/kg per day in four divided doses). [\hyperlink{Erythromycin Lactobionate}{PMID: 6975059}, P Patamasucon et al., 1981]

\hypertarget{pmid_26569091}{H}ematopoietic cell transplantation (HCT) has become a standard treatment for many adult and pediatric conditions. Emerging evidence suggests that perturbations in the microbiota diversity increase recipients' susceptibilities to gut-mediated conditions such as diarrhea, infection and acute GvHD. Probiotics preserve the microbiota and may minimize the risk of developing a gut-mediated condition; however, their safety has not been evaluated in the setting of HCT. We evaluated the safety and feasibility of the probiotic, Lactobacillus plantarum (LBP), in children and adolescents undergoing allogeneic HCT. Participants received once-daily supplementation with LBP beginning on day -8 or -7 and continued until day +14. Outcomes were compliance with daily administration and incidence of LBP bacteremia. Administration of LBP was feasible with 97\% (30/31, 95\% confidence interval (CI) 83-100\%) of children receiving at least 50\% of the probiotic dose (median 97\%; range 50-100\%). We did not observe any case of LBP bacteremia (0\% (0/30) with 95\% CI 0-12\%). There were not any unexpected adverse events related to LBP. Our study provides preliminary evidence that administration of LBP is safe and feasible in children and adolescents undergoing HCT. Future steps include the conduct of an approved randomized, controlled trial through Children's Oncology Group. [\hyperlink{Erythromycin Lactobionate}{PMID: 26569091}, E J Ladas et al., 2016]

\hypertarget{pmid_1778860}{A} randomized single-blind study of the effects of erythromycin and roxithromycin on chlamydial conjunctivitis was performed on a group of patients, comprising 28 newborns and 27 adults. Treatment used was either 200 mg of erythromycin ethylsuccinate or 50 mg of roxithromycin daily, divided into two doses for the neonatal group or for the adult group, 1000 mg of erythromycin stearate or 300 mg of roxithromycin daily divided into two doses. All patients were treated for ten days. Clinically nine of the neonates and 13 of the adults had unilateral conjunctivitis, whilst the remaining cases were bilateral. At follow-up one month after commencing therapy, all but one (erythromycin-treated) of the 28 neonates and three (two of whom were erythromycin-treated) of the 27 adults were cured. However, 16 (nine neonates and seven adults) were culture-positive for Chlamydia trachomatis in samples from eye and/or nasopharynx. The culture-positive group comprised ten cases (four neonates and six adults) who had been treated with erythromycin and six (five neonates and one adult) with roxithromycin. No major side effects of the therapy were seen. The study indicates that there was no difference in the clinical cure rate for the two drugs either in neonates or in adults. However, the isolation rate of chlamydiae in the adult group differed, with 12 (92\%) of the 13 roxithromycin-treated cases becoming culture-negative, whilst this was true for only eight (57\%) of the 14 erythromycin-treated cases (P less than 0.007). [\hyperlink{Erythromycin Lactobionate}{PMID: 1778860}, K Stenberg et al., 1991]

\hypertarget{pmid_11334066}{E}rythromycin enhances gastric emptying and has been suggested to facilitate nasoenteric feeding tube placement in adults. Our primary objective was to evaluate the effect of erythromycin on the transpyloric passage of feeding tubes in critically ill children, and second, to evaluate the effect of erythromycin on the distal migration of duodenal feeding tubes. Seventy-four children were randomly assigned to receive erythromycin lactobionate (10 mg/kg) IV or equal volume of saline placebo 60 minutes before passage of a flexible weighted tip feeding tube. Abdominal radiographs were obtained 4 hours later to assess tube placement. If the tube was proximal to the third part of the duodenum, two additional doses of erythromycin/placebo were administered 6 hours apart. Those receiving additional doses had repeat radiographs 14 to 18 hours after tube placement. The number of postpyloric feeding tubes was similar in the erythromycin and placebo treated groups 4 hours after tube insertion (23/37 vs 27/37, p = .5). Of those with prepyloric tubes at 4 hours, none in the erythromycin group and 3 in the placebo group had the tube migrate to the postpyloric position by 14 to 18 hours (p < .05). Of those with postpyloric tubes proximal to the third part of the duodenum at 4 hours, additional doses of erythromycin did not cause more tubes to advance further into the intestine than did placebo (p = .6). Erythromycin does not facilitate transpyloric passage of feeding tubes in critically ill children. The distal migration of duodenal tubes further into the small bowel is also not enhanced by erythromycin. [\hyperlink{Erythromycin Lactobionate}{PMID: 11334066}, V Gharpure et al., ]

\hypertarget{pmid_2526241}{A} new oral macrolide, clarithromycin (TE-031, A-56268), was evaluated for its safety, efficacy and pharmacokinetics in 33 children. TE-031 was effective in all cases of mycoplasmal pneumonia, pneumococcal pneumonia, streptococcal pharyngitis, pertussis and Campylobacter gastroenteritis. The pharmacokinetic availability of TE-031 granule and tablets was much better than the older macrolides; serum half-lives of TE-031 averaged 3.2 +/- 0.25 hours (for the granule preparation). No clinical adverse reaction was encountered, but cases of transient mild elevation of the serum GPT (2 cases) and eosinophilia (2 cases) were encountered. From these preliminary data, TE-031 seems to have a place in the treatment of pediatric infectious diseases. [\hyperlink{Erythromycin Lactobionate}{PMID: 2526241}, H Meguro et al., 1989]

\hypertarget{pmid_7359612}{E}rythromycin is considered one of the safest antibiotics in common use today. In its otolaryngologic use, the authors have found it effective in treating acute suppurative sinusitis and occasionally otitis media, when combined with sulfonamides. There are few complications of erythromycin administration. Probably the least generally acknowledged of these is ototoxicity. There have been three reports of six cases with ototoxic complications from erythromycin, primarily from administration of its intravenous form. The authors present a case study of an 18 year old girl in severe renal failure, who suffered a reversible sensorineural hearing loss from high doses of an oral erythromycin preparation. The clinical manifestations of this case are compared to those previously reported. [\hyperlink{Erythromycin Lactobionate}{PMID: 7359612}, P Thompson et al., 1980]

\section*{Fomepizole}
\subsection*{Result}
\subsubsection*{Answer}

Unknown

\subsubsection*{{Explanation}}
\hypertarget{Fomepizole}
To determine if Fomepizole is safe for use in children, I reviewed all available abstracts for evidence of targeted pediatric safety studies. According to your criteria, a drug is considered "safe for use in children" only if a targeted study has been done about safety in children and that study affirms its safety.

Here is a summary of the relevant abstracts, organized by age range:

Infants (0–1 year):
- One case report describes a 5-month-old boy with severe ethylene glycol poisoning successfully treated with fomepizole (seven doses, no hemodialysis). The infant made a complete recovery with no change in renal function. The authors state, "fomepizole seemed safe and effective in a case of severe ethylene glycol poisoning, without the need for hemodialysis" but also note it is "not yet approved for this indication in the child" [\hyperlink{pmid_15329167}{PMID: 15329167}, Thierry Detaille et al., 2004]. This is a single case report, not a targeted safety study.

- Another case report describes a 7-month-old female with acetaminophen toxicity treated with fomepizole and NAC, who fully recovered. The authors state, "Although randomized trials are lacking, this case suggests that fomepizole may safely provide additional benefit in pediatric patients at risk for severe acetaminophen toxicity" [\hyperlink{pmid_37681263}{PMID: 37681263}, Lesley Pepin et al., 2023]. Again, this is a single case report, not a targeted safety study.

Toddlers and Young Children (1–6 years):
- A case report describes a 2-year-old child with ethylene glycol poisoning managed with fomepizole, who fully recovered [\hyperlink{pmid_22605809}{PMID: 22605809}, Gayle Hann et al., 2012]. This is a single case report, not a targeted safety study.

- Another case report describes a 3-year-old boy with methanol poisoning treated with fomepizole, with an uneventful course and no side effects. The authors provide an overview of all cases of pediatric poisoning in which fomepizole was used and state, "Fomepizole seems to be a safe and valid alternative to ethanol in cases of paediatric methanol poisoning" [\hyperlink{pmid_15578220}{PMID: 15578220}, Nathalie De Brabander et al., 2005]. This is a case report with a literature overview, not a targeted safety study.

- A case report of a 6-year-old girl with ethylene glycol poisoning treated with fomepizole describes transient nystagmus as a side effect, but the child recovered uneventfully. The authors state, "Fomepizole has been incompletely evaluated in the pediatric population, and the nature and occurrence of adverse events have not been described adequately. The use of fomepizole appeared safe in this patient although she developed transient nystagmus" [\hyperlink{pmid_11192468}{PMID: 11192468}, J G Benitez et al., 2000].

Children and Adolescents (up to 18 years):
- Several review articles and case series mention the use of fomepizole in children, but all note the lack of targeted pediatric safety studies. For example, one review states, "Fomepizole is both efficacious and safe in the pediatric population, but is not recommended during pregnancy" [\hyperlink{pmid_27147840}{PMID: 27147840}, Bruno Mégarbane et al., 2010], but does not cite a targeted pediatric safety study.

- Another review states, "Data on comparative efficacy of fomepizole versus ethanol and data on administration of fomepizole in children are limited" [\hyperlink{pmid_11899949}{PMID: 11899949}, Deon P Druteika et al., 2002].

- A case report and literature review from 1999 notes, "the first article reporting the use of fomepizole in a pediatric ethylene glycol exposure was published" and that "the therapy of ethylene glycol poisoning in children is likely to change" [\hyperlink{pmid_10349109}{PMID: 10349109}, J F Wiley et al., 1999], but again, this is not a targeted safety study.

Summary:
Across all age ranges, the available evidence consists of case reports and small case series, not targeted safety studies. While these reports generally describe favorable outcomes and minimal adverse effects, they do not constitute definitive evidence of safety as defined by your criteria. No abstract describes a prospective or retrospective study specifically designed to assess the safety of fomepizole in children.

Therefore, based on the abstracts available, the safety of fomepizole in children is unknown.

\subsection*{Abstracts}
\hypertarget{pmid_11134452}{F}omepizole (4-methylpyrazole, Antizol) is being increasingly used in the treatment of ethylene glycol toxicity in adults. Little experience exists with this drug, however, in the pediatric population. We present a case of ethylene glycol poisoning in a child where use of fomepizole averted intravenous ethanol infusion and hemodialysis, limited the duration of intensive care monitoring, and decreased the overall cost of treatment. [\hyperlink{Fomepizole}{PMID: 11134452}, E W Boyer et al., 2001]

\hypertarget{pmid_11581485}{F}omepizole (4-methylpyrazole; Antizol) is used increasingly in the treatment of methanol toxicity in adults. Little experience exists with this drug in the pediatric population, however. We present a case of methanol poisoning in a child in whom the use of fomepizole averted intravenous ethanol infusion and the attendant side effects of this therapy. [\hyperlink{Fomepizole}{PMID: 11581485}, M J Brown et al., 2001]

\hypertarget{pmid_11192468}{F}omepizole is an alcohol dehydrogenase inhibitor used to treat ethylene glycol poisoning in adults, with only one report describing the use of fomepizole in the pediatric population. We report a case of nystagmus associated with fomepizole treatment of a 6-year-old female who ingested ethylene glycol 15 hours prior to admission. A previously healthy 6-year-old presented to the emergency department mottled, comatose, and with Kussmaul respirations. Initial arterial blood gases: pH 7.11, PO2 200, HCO3 2, base excess -29, and within 20 minutes her pH dropped to 7.03. The patient was responsive to pain only. Initially, crystalluria without fluorescence was observed in the emergency department; 2 hours after admission, the urine fluoresced under Wood's light. Laboratory data were significant for increased anion and osmolar gaps. She was fluid-resuscitated, NaHCO3, thiamine, and pyridoxine were administered, and she was admitted to the pediatric intensive care unit. Within 4 hours of admission, a loading dose of fomepizole (15 mg/kg) was infused due to the severity of the patient's clinical status. Hemodialysis was initiated but discontinued temporarily due to catheter thrombus formation. The initial (3-hour postadmission) ethylene glycol concentration was 13 mg/dL. She developed coarse vertical nystagmus within 2 hours of fomepizole infusion. The ethylene glycol concentration was 5 mg/dL 3 hours after hemodialysis which then was discontinued. No further fomepizole was administered and the child recovered uneventfully. There was no evidence of the more frequently cited adverse events, such as headache, nausea, and dizziness. Fomepizole has been incompletely evaluated in the pediatric population, and the nature and occurrence of adverse events have not been described adequately. The use of fomepizole appeared safe in this patient although she developed transient nystagmus. [\hyperlink{Fomepizole}{PMID: 11192468}, J G Benitez et al., 2000]

\hypertarget{pmid_22891985}{F}omepizole has been utilized with remarkable success for ethylene glycol and methanol poisonings in children and adults. However, very little information is available regarding the safe and effective use of fomepizole in pregnancy. The goal of this research was to utilize an animal model to investigate the kinetics of fomepizole in pregnancy. Male and pregnant Sprague-Dawley rats, which were obtained at 19 days gestation, were administered fomepizole 15 mg/kg intraperitoneally. The animals were anesthetized and blood, liver, kidney, and fetus samples were collected at 1-24 hours post administration. Tissue samples were homogenized, deproteinized and prepared by solid phase extraction. Fomepizole concentrations from tissue and serum samples were analyzed using high pressure liquid chromatography. Between males and pregnant females, tissue and serum fomepizole levels were similar. Fomepizole concentrations in whole fetal tissue were similar to those in the maternal liver and kidney tissue. Fetal fomepizole concentrations were fivefold higher than maternal serum concentrations. The zero order elimination rate of fomepizole from maternal serum was 7.6 mol/L/h, which was slightly slower than the elimination rate in male rats (12.9 mol/L/h). Elimination of fomepizole from the fetus followed a similar time course to that in the maternal tissues. Elevated concentrations of fomepizole were detected in the fetus following maternal administration. Although the levels of fomepizole in the fetal tissue would imply significant protection against fetal formation of toxic alcohol metabolites, further research is needed to explore the long-term effects of fomepizole on the fetus. [\hyperlink{Fomepizole}{PMID: 22891985}, Rebeca Gracia et al., 2012]

\hypertarget{pmid_15329167}{T}o report a case of a massive ingestion of ethylene glycol in an infant successfully treated by fomepizole without hemodialysis. Descriptive case report. Pediatric intensive care unit. A 5-mo-old boy who ingested 200 mL of an antifreeze solution. Antidotal therapy with a total of seven doses of fomepizole administered intravenously with an interval of 12 hrs (15 mg/kg as loading dose, then 10 mg/kg). Hemodialysis was not performed. Iterative determination of ethylene glycol concentration was obtained in blood and urine. Kinetics were calculated for ethylene glycol and fomepizole elimination. The infant made a complete recovery with no change in renal function. Although not yet approved for this indication in the child, fomepizole seemed safe and effective in a case of severe ethylene glycol poisoning, without the need for hemodialysis. [\hyperlink{Fomepizole}{PMID: 15329167}, Thierry Detaille et al., 2004]

\hypertarget{pmid_34697779}{F}omepizole is an anti-metabolite therapy that is used to diminish the toxicity from methanol or ethylene glycol. Although its elimination kinetics have been well described in healthy human subjects, the elimination in poisoned patients have only been described in a few isolated cases. This study was designed to relate the elimination of fomepizole in a series of poisoned patients to that in healthy humans. Plasma samples from 26 patients in the clinical trials of the use of fomepizole for methanol and ethylene glycol poisoning were analyzed for fomepizole concentrations. The elimination of fomepizole was assessed after individual doses, both during and without intermittent hemodialysis. In methanol- and ethylene glycol-poisoned patients, fomepizole had a volume of distribution of 0.66-0.68 L/kg. After repeated doses of fomepizole, the minimum trough concentration averaged 86-109 µmol/L, which is 10 times higher than the minimum therapeutic concentration. In healthy human subjects, fomepizole elimination follows Michaelis-Menten kinetics and has been calculated as zero-order elimination rates. Zero-order elimination rates averaged 13 and 17 μmol/L/h in methanol and ethylene glycol patients, respectively, compared to 6-19 μmol/L/h in healthy subjects. Elimination during intermittent hemodialysis followed first-order kinetics, with a half-life of 3 h. Plasma concentrations during the repeated dosing confirmed that the recommended dosing schedule, with and without intermittent hemodialysis, maintained therapeutic concentrations throughout the treatments. Fomepizole elimination in poisoned patients at therapeutic plasma concentrations appears be similar to that reported previously in healthy human subjects. [\hyperlink{Fomepizole}{PMID: 34697779}, Kenneth McMartin et al., 2022]

\hypertarget{pmid_10349109}{E}thylene glycol is a serious toxin that children frequently ingest. Diagnosis and treatment of this poisoning are challenging and frequently involve the use of novel therapies. In the past year, fomepizole (4-methylpyrazole) has been approved for use as an antidote in the treatment of ethylene glycol poisoning in adults, and the first article reporting the use of fomepizole in a pediatric ethylene glycol exposure was published. As a result, the therapy of ethylene glycol poisoning in children is likely to change from the traditional approach of ethanol administration coupled with hemodialysis to the administration of fomepizole with or without hemodialysis. [\hyperlink{Fomepizole}{PMID: 10349109}, J F Wiley et al., 1999]

\hypertarget{pmid_27147840}{E}thylene glycol (EG) and methanol are responsible for life-threatening poisonings. Fomepizole, a potent alcohol dehydrogenase (ADH) inhibitor, is an efficient and safe antidote that prevents or reduces toxic EG and methanol metabolism. Although no study has compared its efficacy with ethanol, fomepizole is recommended as a first-line antidote. Treatment should be started as soon as possible, based on history and initial findings including anion gap metabolic acidosis, while awaiting measurement of alcohol concentration. Administration is easy (15 mg/kg-loading dose, either intravenously or orally, independent of alcohol concentration, followed by intermittent 10 mg/kg-doses every 12 hours until alcohol concentrations are <30 mg/dL). There is no need to monitor fomepizole concentrations. Administered early, fomepizole prevents EG-related renal failure and methanol-related visual and neurological injuries. When administered prior to the onset of significant acidosis or organ injury, fomepizole may obviate the need for hemodialysis. When dialysis is indicated, 1 mg/kg/h-continuous infusion should be provided to compensate for its elimination. Side-effects are rarely serious and with a lower occurrence than ethanol. Fomepizole is contraindicated in case of allergy to pyrazoles. It is both efficacious and safe in the pediatric population, but is not recommended during pregnancy. In conclusion, fomepizole is an effective and safe first-line antidote for EG and methanol intoxications.  [\hyperlink{Fomepizole}{PMID: 27147840}, Bruno Mégarbane et al., 2010] Orphan Medical has developed fomepizole as a potential treatment for both ethylene glycol and methanol poisoning. The drug was launched as Antizol in January 1998 for the treatment of ethylene glycol poisoning [273949] after US marketing approval was grantedin December 1997 [271563]. It has also received US approval for methanol poisoning [393217] and UK approval for ethylene glycol poisoning [329495]. In 1999, Orphan Medical's partner, Cambridge Laboratories, intended to pursue European approval under the mutual recognition procedure [329495]. However, by September 2000, Cambridge Laboratories had discontinued their involvement with fomepizole and IDIS World Medicines had licensed the rights to distribute the drug in the UK [412142]. In February 2000, the Canadian Therapeutic Products Programme (TPP) granted fomepizole Priority Review, provided that an NDA was submitted by March 14, 2000 [354665]. In August 2000, the TPP accepted this NDA and set a target date for approval in the fourth quarter of 2000 [379474]. The TPP granted fomepizole a Notice of Compliance permitting the sale of fomepizole in Canada in December 2000. The company's marketing partner in Canada, Paladin Labs had launched fomepizole by January 2001 [396953]. In June 2000, Tucker Anthony Cleary Gull stated that the Orphan Drug status which Orphan Medical had obtained for fomepizole would provide marketing exclusivity through December 2004. The analysts also stated that fomepizole had accounted for 40\% of Orphan Medical's revenue in financial year 1999, although +/- 30\% of sales were estimated to be due to stockpiling [409606]. [\hyperlink{Fomepizole}{PMID: 27147840}, P Hantson et al., 2001]

\hypertarget{pmid_18344099}{F}omepizole is available intravenously (i.v.) for the treatment of methanol and ethylene glycol poisoning. Few studies demonstrate that fomepizole achieves effective serum concentrations after i.v. or oral (p.o.) use. The objective was to describe the comparative pharmacokinetics of fomepizole after a single p.o. and i.v. dose. This was a prospective, randomized, crossover trial in 10 healthy volunteers. Each received 15 mg/kg fomepizole, p.o. and by 30 minute i.v. infusion. Serum was collected at 0, 0.25, 0.5, 1, 2, 4, 7, 12, 24, 36, and 48 hours (h) and stored at -70 degrees C. Candidate models were fit to the i.v. and p.o. data, simultaneously, using iterative 2-stage analysis weighted by the estimated inverse observation variance. Time above the MEC (T>MEC) was determined by numeric integration of the fitted functions using 10 micromoles/L as the minimum effective concentration (MEC). Seven females and 3 males were enrolled. Sole complaints included headache and dizziness in 3 subjects and 10/10 reported an unpleasant taste. The final PK model was 2-compartment with 0-order i.v. and 1(st)-order p.o. input (following a fitted TLag) and Michaelis-Menten elimination. p.o. fomepizole was rapidly absorbed with a bioavailability of approximately 100\%. The Km was 0.935+/-0.98 micromoles/L and the Vmax was 18.57+/-9.58 micromoles/L/h. T>MEC was 32 h with agreement between p.o. and i.v. dosing. This is the first study that effectively determines a human Vmax and Km for p.o. and i.v. fomepizole. p.o. and i.v. administration of fomepizole result in similar pharmacokinetic parameters. [\hyperlink{Fomepizole}{PMID: 18344099}, Jeanna Marraffa et al., 2008]

\hypertarget{pmid_15578220}{M}ethanol poisoning is not frequently observed in children; however, without treatment, serious intoxication can be complicated by visual impairment, coma, metabolic acidosis, respiratory and circulatory insufficiency and death. Treatment in a paediatric intensive care is therefore compulsory. Methanol is metabolised in the liver by alcohol dehydrogenase to the toxic metabolites formaldehyde and formic acid. Classically, ethanol is given as a competitive inhibitor in order to avoid the formation of these compounds. We report on the use of fomepizole (4-methylpyrazole),a new and potent inhibitor of alcohol dehydrogenase, in a 3-year-old boy after the intake of a toxic amount of methanol. The course was uneventful and the use of fomepizole was not accompanied by any side-effects. An overview is given of all cases of paediatric poisoning in which fomepizole was used. Fomepizole seems to be a safe and valid alternative to ethanol in cases of paediatric methanol poisoning. [\hyperlink{Fomepizole}{PMID: 15578220}, Nathalie De Brabander et al., 2005]

\hypertarget{pmid_1494233}{C}efprozil (CFPZ, BMY-28100) was evaluated for its efficacy, safety and pharmacokinetics in children. CFPZ was effective against streptococcal pharyngitis, pneumococcal lower respiratory tract infections, staphylococcal skin infections and Escherichia coli urinary tract infections, but was less effective against lower respiratory tract infections and otitis media due to Haemophilus influenzae. No adverse reactions were encountered in 46 cases treated with CFPZ. With a premeal administration of 7.5 mg/kg, the Cmax was approximately 3.2 micrograms/ml and the T 1/2 beta was 1.4 hours. From the present study, CFPZ appears to be safe and effective against community-acquired childhood infections. [\hyperlink{Fomepizole}{PMID: 1494233}, H Meguro et al., 1992]

\hypertarget{pmid_10485727}{F}omepizole is an effective alternative to ethanol in the treatment of ethylene glycol poisoning. In a series of 38 acute poisonings without renal failure, fomepizole obviated the need for haemodialysis. [\hyperlink{Fomepizole}{PMID: 10485727}, S W Borron et al., 1999]

\hypertarget{pmid_25251104}{T}o evaluate the efficacy and safety of piperacillin/tazobactam (PIPC/TAZ) or cefepime (CFPM) monotherapy for febrile neutropenia (FN) in children, a total of 53 patients with 213 febrile episodes were randomly treated with either PIPC/TAZ 337.5 mg/kg/day, or CFPM 100 mg/kg/day. No significant differences were observed in the success rates of the PIPC/TAZ and CFPM treatments (62.1\% vs. 59.1\%, P = 0.650). Furthermore, no differences were noted in the rates of new infection and mortality, and no serious adverse effects occurred in either of groups. Both PIPC/TAZ and CFPM were effective and safe as an empirical therapy for FN in children. Pediatr Blood Cancer 2015;62:356-358. © 2014 Wiley Periodicals, Inc. [\hyperlink{Fomepizole}{PMID: 25251104}, Hirozumi Sano et al., 2015]

\hypertarget{pmid_37681263}{A}cetaminophen overdose is common in the pediatric population. N-acetylcysteine (NAC) is effective at preventing liver injury in most patients when started shortly after the overdose. Delays to therapy increase risk of hepatotoxicity and liver failure that may necessitate organ transplant. Animal studies have demonstrated fomepizole may provide added benefit in acetaminophen overdose because of its ability to block the metabolic pathway that produces the toxic acetaminophen metabolite and downstream inhibition of oxidative stress pathways that lead to cell death. Several adult case reports describe use of fomepizole in patients at higher risk for poor outcomes despite NAC. We describe a case of a 7-month-old female who presented in acute liver failure with persistently elevated acetaminophen concentration secondary to repeated supratherapeutic doses of acetaminophen to manage fever. Fomepizole and NAC antidotes were used in the management of the patient. She fully recovered despite demonstrating multiple markers of poor outcome on initial presentation. Although randomized trials are lacking, this case suggests that fomepizole may safely provide additional benefit in pediatric patients at risk for severe acetaminophen toxicity. [\hyperlink{Fomepizole}{PMID: 37681263}, Lesley Pepin et al., 2023]

\hypertarget{pmid_18818954}{A} randomized, open, coordinated multi-center trial compared the bacteriological and clinical efficacy and safety of orally administered ceftibuten and trimethoprim-sulfamethoxazole (TMP-SMX) in children with febrile urinary tract infection (UTI). Children aged 1 month to 12 years presenting with presumptive first-time febrile UTI were eligible for enrollment. A 2:1 assignment to treatment with ceftibuten 9 mg/kg once daily (n = 368) or TMP-SMX (3 mg + 15 mg)/kg twice daily (n = 179) for 10 days was performed. Escherichia coli was recovered in 96\% of the cases. Among the E. coli isolates, 14\% were resistant to TMP-SMX but none to ceftibuten. In the modified intention-to-treat population, the bacteriological elimination rates at follow-up did not differ significantly between patients treated with ceftibuten and those treated with TMP-SMX [91 vs. 95\%, with a 95\% confidence interval (CI) for difference of -9.7 to 1.0]. However, the clinical cure rate was significantly higher among those treated with ceftibuten (93 vs. 83\%, with a 95\% CI for difference of 2.4 to 17.0). Adverse events were similar for both regimens and consisted mainly of gastrointestinal disturbances. In conclusion, ceftibuten is a safe and effective drug for the empirical treatment of febrile UTI in young children. [\hyperlink{Fomepizole}{PMID: 18818954}, Staffan Mårild et al., 2009]

\hypertarget{pmid_25482738}{D}uring an outbreak of mass methanol poisonings in the Czech Republic in 2012-2013, fomepizole was applied as an alternative antidote to ethanol. We present the laboratory data, clinical features, adverse reactions, and treatment outcomes in all patients treated with fomepizole. Combined retrospective and prospective case series study in 25 patients, median age 50 (16-73) years, 18 males and 7 females. There were 24\% fatalities, 36\% survivors without health impairment, and 40\% survivors with sequelae. All the patients who died were comatose on admission; the mortality was 50\% among patients in a coma. The median intensive care unit length of stay was six (2-22) days. The median total dose of fomepizole was 2 (1-9) g. Complications were observed in 7/25 cases: aspiration pneumonia (4), sepsis (2), bleeding (2), malignant arrhythmia (1), delirium tremens (1), and rebound of acidosis (1). The patients who survived without impairment were less acidotic than those who died or survived with sequelae (P<0.01). No difference in serum methanol and formate was found between the three groups. There is no evidence whether fomepizole is a more efficient antidote than ethanol with regards to the hospital mortality. The possibility of delirium tremens in the patients with a history of chronic alcohol abuse has to be taken in consideration. The benefits of fomepizole were indirect: no need to monitor serum ethanol's level during the hemodialysis in severely poisoned patients and less working overload on ICU doctors treating several poisoned patients simultaneously. [\hyperlink{Fomepizole}{PMID: 25482738}, Sergey Zakharov et al., 2014]

\hypertarget{pmid_10497633}{F}omepizole (4-methylpyrazole, 4-MP, Antizol) is a potent inhibitor of alcohol dehydrogenase that was approved recently by the US Food and Drug Administration (FDA) for the treatment of ethylene glycol poisoning. Although ethanol is the traditional antidote for ethylene glycol poisoning, it has not been studied prospectively. Furthermore, the FDA has not approved the use of ethanol for this purpose. Case reports and a prospective case series indicate that the intravenous (i.v.) administration of fomepizole every 12 hours prevents renal damage and metabolic abnormalities associated with the conversion of ethylene glycol to toxic metabolites. Currently, there are insufficient data to define the relative role of fomepizole and ethanol in the treatment of ethylene glycol poisoning. Fomepizole has clear advantages over ethanol in terms of validated efficacy, predictable pharmacokinetics, ease of administration, and lack of adverse effects, whereas ethanol has clear advantages over fomepizole in terms of long-term clinical experience and acquisition cost. The overall comparative cost of medical treatment using each antidote requires further study. [\hyperlink{Fomepizole}{PMID: 10497633}, D G Barceloux et al., 1999]

\hypertarget{pmid_30360666}{C}hronic idiopathic nausea (CIN) and functional dyspepsia (FD) cause considerable strain on many children's lives and their families. Areas covered: This study aims to systematically assess the evidence on efficacy and safety of pharmacological treatments for CIN or FD in children. CENTRAL, EMBASE, and Medline were searched for Randomized Controlled Trials (RCTs) investigating pharmacological treatments of CIN and FD in children (4-18 years). Cochrane risk of bias tool was used to assess methodological quality of the included articles. Expert commentary: Three RCTs (256 children with FD, 2-16 years) were included. No studies were found for CIN. All studies showed considerable risk of bias, therefore results should be interpreted with caution. Compared with baseline, successful relief of dyspeptic symptoms was found for omeprazole (53.8\%), famotidine (44.4\%), ranitidine (43.2\%) and cimetidine (21.6\%) (p = 0.024). Compared with placebo, famotidine showed benefit in global symptom improvement (OR 11.0; 95\% CI 1.6-75.5; p = 0.02). Compared with baseline, mosapride versus pantoprazole reduced global symptoms (p = 0.011; p = 0.009). One study reported no occurrence of adverse events. This systematic review found no evidence to support the use of pharmacological drugs to treat CIN or FD in children. More high-quality clinical trials are needed. AP-FGID: Abdominal Pain Related Functional Gastrointestinal Disorders; BART: Biofeedback-Assisted Relaxation Training; CIN: Chronic Idiopathic Nausea; COS: Core Outcomes Sets; EPS: Epigastric Pain Syndrome; ESPGHAN: European Society for Pediatric Gastroenterology Hepatology and Nutrition; FAP: Functional Abdominal Pain; FD: Functional Dyspepsia; GERD: Gastroesophageal Reflux Disease; GES: Gastric Electrical Stimulation; H [\hyperlink{Fomepizole}{PMID: 30360666}, Pamela D Browne et al., 2018] To assess the efficacy and safety of fomepizole, a competitive alcohol dehydrogenase inhibitor, in methanol poisoning and to test the hypothesis that fomepizole obviates the need for hemodialysis in selected patients. Retrospective clinical study in three intensive care units in university-affiliated teaching hospitals. All methanol-poisoned patients admitted to these ICUs and treated with fomepizole from 1987-1999 (n=14). The median plasma methanol concentration was 50 mg/dl (range 4-146), anion gap 22.1 mmol/l (11.8-42.2), arterial pH 7.34 (7.11-7.51), and bicarbonate 17.5 mmol/l (3.0-25.0). Patients received oral or intravenous fomepizole until blood methanol was undetectable. The median cumulative dose was 1250 mg (500-6000); the median number of twice daily doses was 2 (1-16). Four patients underwent hemodialysis for visual impairment present on admission. Four patients with plasma methanol concentrations of 50 mg/dl or higher and treated without hemodialysis recovered fully. Patients without pretreatment visual disturbances recovered, with no sequelae in any case. There were no deaths. Fomepizole was safe and well tolerated, even in the case of prolonged treatment. Analysis of methanol toxicokinetics in five patients demonstrated that fomepizole was effective in blocking methanol's toxic metabolism. Fomepizole appears safe and effective in the treatment of methanol-poisoned patients. If our results are confirmed in prospective analyses, hemodialysis may prove unnecessary in patients presenting without visual impairment or severe acidosis. [\hyperlink{Fomepizole}{PMID: 30360666}, B Mégarbane et al., 2001]

\hypertarget{pmid_22605809}{T}his case report describes the presentation and management of a 2-year-old child who ingested a potentially fatal amount of ethylene glycol (EG). There are few published cases worldwide of EG poisoning in children managed with fomepizole. All cases described in the literature were managed in a paediatric intensive care unit. In this case, the child presented irritable, pale and confused with high anion gap metabolic acidosis. As there were no paediatric intensive care beds available in the region, the child was successfully managed in a high dependency area in our district general hospital. The child fully recovered and was discharged home in 7 days. The authors believe that multi-disciplinary team management and the use of fomepizole contributed to the positive outcome and this case raised many useful learning points. [\hyperlink{Fomepizole}{PMID: 22605809}, Gayle Hann et al., 2012]

\hypertarget{pmid_27057183}{F}ebrile seizure is the most common neurologic problem in children between 3 months to 5 years old. Two to five percent of children aged less than five yr old will experience it at least one time. This type of seizure is age dependent and its recurrence rate is about 33\% overalls and 50\% in children less than one yr old. The prophylactic treatment is still controversial, so we conducted a randomized controlled clinical trial to find out the effectiveness of continuous phenobarbital versus intermittent diazepam for febrile seizure. This clinical trial was conducted in the Department of Pediatric Neurology, Babol University of Medical Sciences, Babol, Iran between March 2008 and October 2010. All children from 6 month to 5 yr old referred to Amirkola Children's Hospital, Babol, Iran were enrolled in the study. Children with febrile seizure that had indication for prophylaxis but did not receive any prophylaxis previously were enrolled in the study. For prophylactic anti convulsion therapy, patients were divided randomly in two groups. One group received continuous phenobarbital and another treated with intermittent diazepam whenever the children experienced an episode of febrile illness for up to one year after their last convulsion. Of all 145 studied cases, the recurrent rate in children under prophylaxis with diazepam was 11/71 and in phenobarbital group was 17/74. There was no significant difference in the recurrence rate in both groups. There was no significant difference in the effectiveness of phenobarbital and diazepam in prevention of recurrent in febrile seizure and we think that in respect of lower complication rate in diazepam administration, it cloud be better choice than phenobarbital. [\hyperlink{Fomepizole}{PMID: 27057183}, Mohammadreza Salehiomran et al., 2016]

\hypertarget{pmid_34585641}{F}omepizole is the preferred antidote for treatment of methanol and ethylene glycol poisoning, acting by inhibiting the formation of the toxic metabolites. Although very effective, the price is high and the availability is limited. Its availability is further challenged in situations with mass poisonings. Therefore, a 50\% reduced maintenance dose for fomepizole during continuous renal replacement therapy (CRRT) was suggested in 2016, based on pharmacokinetic data only. Our aim was to study whether this new dosing for fomepizole during CRRT gave plasma concentrations above the required 10 µmol/L. Secondly, we wanted to study the elimination kinetics of fomepizole during CRRT, which has never been studied before. Prospective observational study of adult patients treated with fomepizole and CRRT. We collected samples from arterial line (pre-filter) = plasma concentration, post-filter and dialysate for fomepizole measurements. Fomepizole was measured using high-pressure liquid chromatography with a reverse phase column. Ten patients were included in the study. Seven were treated with continuous veno-venous hemodialysis (CVVHD) and three with continuous veno-venous hemodiafiltration (CVVHDF). Ninety-eight percent of the plasma samples were above the minimum plasma concentration of 10 µmol/L. Fomepizole was removed during CRRT with a median saturation/sieving coefficient of 0.85 and dialysis clearance of 28 mL/min. Fomepizole was eliminated during CCRT. The new dosing recommendations for fomepizole and CRRT appeared safe, by maintaining the plasma concentration above the minimum value of 10 µmol/L. Based on these data, the fomepizole maintenance dose during CRRT could be reduced to half as compared to intermittent hemodialysis. [\hyperlink{Fomepizole}{PMID: 34585641}, Yvonne E Lao et al., 2022]

\hypertarget{pmid_25449223}{T}o systematically review literature assessing efficacy and safety of pharmacologic treatments in children with abdominal pain-related functional gastrointestinal disorders (AP-FGIDs). MEDLINE and Cochrane Database were searched for systematic reviews and randomized controlled trials investigating efficacy and safety of pharmacologic agents in children aged 4-18 years with AP-FGIDs. Quality of evidence was assessed using Grades of Recommendation, Assessment, Development and Evaluation approach. We included 6 studies with 275 children (aged 4.5-18 years) evaluating antispasmodic, antidepressant, antireflux, antihistaminic, and laxative agents. Overall quality of evidence was very low. Compared with placebo, some evidence was found for peppermint oil in improving symptoms (OR 3.3 (95\% CI 0.9-12.0) and for cyproheptadine in reducing pain frequency (relative risk [RR] 2.43, 95\% CI 1.17-5.04) and pain intensity (RR 3.03, 95\% CI 1.29-7.11). Compared with placebo, amitriptyline showed 15\% improvement in overall quality of life score (P = .007) and famotidine only provides benefit in global symptom improvement (OR 11.0; 95\% CI 1.6-75.5; P = .02). Polyethylene glycol with tegaserod significantly decreased pain intensity compared with polyethylene glycol only (RR 3.60, 95\% CI 1.54-8.40). No serious adverse effects were reported. No studies were found concerning antidiarrheal agents, antibiotics, pain medication, anti-emetics, or antimigraine agents. Because of the lack of high-quality, placebo-controlled trials of pharmacologic treatment for pediatric AP-FGIDs, there is no evidence to support routine use of any pharmacologic therapy. Peppermint oil, cyproheptadine, and famotidine might be potential interventions, but well-designed randomized controlled trials are needed. [\hyperlink{Fomepizole}{PMID: 25449223}, Judith J Korterink et al., 2015]

\hypertarget{pmid_11899949}{T}o systematically review English-language articles on fomepizole administration in patients with ethylene glycol poisoning. MEDLINE, EMBASE, Current Contents, and PubMed. Search terms were fomepizole, 4-methylpyrazole, and ethylene glycol. The search was supplemented with a bibliographic review of all relevant articles. All published reports of fomepizole administration in patients with ethylene glycol poisoning were reviewed, irrespective of study design. We identified one clinical trial and subsequent pharmacokinetic study, one case series, and 13 case reports. Fomepizole has been investigated in 70 patients in open, unblinded studies. Most patients received an intravenous loading dose, with subsequent variable maintenance doses every 12 hours until plasma ethylene glycol levels became undetectable. Additional hemodialysis treatment generally was administered when patients had renal insufficiency or ethylene glycol levels above 50 mg/dl. Many patients had detectable ethanol levels either because of coadministration or as a result of adjunctive treatment at a referring center. Poorer patient outcomes, such as death and renal insufficiency, were associated with later clinical presentation time after ingestion. At therapeutic fomepizole levels (> 8.6 mg/ml), the half-life of ethylene glycol was prolonged to over 19 hours. Fomepizole appeared to be well tolerated by most patients. Fomepizole is an effective alcohol dehydrogenase inhibitor that decreases production of ethylene glycol metabolites. Reduced mortality and morbidity are undetermined because of the small number of patients evaluated to date. Data on comparative efficacy of fomepizole versus ethanol and data on administration of fomepizole in children are limited. [\hyperlink{Fomepizole}{PMID: 11899949}, Deon P Druteika et al., 2002]

\section*{Gadoteridol}
\subsection*{Result}
\subsubsection*{Answer}

10–18 years: Yes
Children (age not specified): Yes
Infants and children under 10 years: Unknown

\subsubsection*{{Explanation}}
\hypertarget{Gadoteridol}
Based on the available abstracts, several targeted studies have evaluated the safety of gadoteridol in children:

1. A phase III open-label clinical trial included 103 children with suspected neurologic disease (age range not specified, but "children" is used) and found only minor, self-resolving adverse events in two children. The study concluded that gadoteridol demonstrated excellent safety and efficacy for use in children with suspected intracranial or spinal disease [\hyperlink{pmid_8430186}{PMID: 8430186}, W S Ball et al., 1993].

2. Another study specifically enrolled 13 children aged 10 to 18 years who underwent MRI with gadoteridol. No minor or major reactions were observed, and the study concluded that gadoteridol is a safe and excellent contrast agent for use in MRI in this age group [\hyperlink{pmid_8496990}{PMID: 8496990}, S E Byrd et al., 1993].

3. A phase IIIB open-label multicenter clinical trial included 22 pediatric patients (age not specified, but all described as "children") with CNS neoplasms. No clinically relevant changes in vital signs or laboratory values were attributed to gadoteridol, and no systemic complaints were reported. The study concluded that intravenous gadoteridol was safe in children [\hyperlink{pmid_1501959}{PMID: 1501959}, J F Debatin et al., 1992].

4. Another multicenter clinical trial involved 101 pediatric patients (age not specified, but all described as "children") and found gadoteridol suitable for enhanced MRI detection, localization, and characterization of CNS pathology in children. No specific safety concerns were reported [\hyperlink{pmid_1506153}{PMID: 1506153}, W S Ball et al., 1992].

5. A study on high-dose gadoteridol (up to 0.3 mmol/kg) included 67 patients (age not specified, but the context suggests inclusion of children) and found only minor, self-resolving adverse effects in two patients. The study concluded that gadoteridol can be safely administered up to a cumulative dose of 0.3 mmol/kg [\hyperlink{pmid_8073972}{PMID: 8073972}, W T Yuh et al., 1994].

6. A general safety study of gadoteridol included animal and in vitro studies, as well as reproduction studies in rats and rabbits, but did not provide targeted pediatric human data [\hyperlink{pmid_1506157}{PMID: 1506157}, R A Soltys et al., 1992].

None of the abstracts specifically address the safety of gadoteridol in infants or children under 10 years of age. The studies that do specify age ranges focus on children aged 10–18 years, while others refer to "children" without further age breakdown. Therefore, based on the abstracts, gadoteridol is affirmed as safe for use in children aged 10–18 years, and in general pediatric populations where age is not further specified. However, the safety in infants and children under 10 years old is not definitively established in the abstracts provided.


\subsection*{Abstracts}
\hypertarget{pmid_8430186}{A} phase III open-label clinical trial was conducted at 11 institutions to determine the safety and efficacy of gadoteridol in children suspected of having neurologic disease. One hundred three children were included in the safety analysis; 92 were evaluated for efficacy (76 intracranial and 16 spinal examinations). Three adverse events were reported in two children. All adverse events were considered minor and resolved spontaneously without treatment or sequelae. In a comparison of enhanced T1-weighted magnetic resonance images with unenhanced T1- and T2-weighted images, enhancement of disease was noted in 70\% of the intracranial and 38\% of the spinal examinations. Additional diagnostic information was reported in 82\% of the postcontrast intracranial examinations and 62\% of the spinal examinations. Use of this additional information contributed to a potential modification of patient diagnosis in 48\% of intracranial and 20\% of spinal cases with additional information. These results indicate excellent safety and efficacy for use of gadoteridol in children with suspected intracranial or spinal disease. [\hyperlink{Gadoteridol}{PMID: 8430186}, W S Ball et al., 1993]

\hypertarget{pmid_26045036}{G}adoteric acid is a paramagnetic gadolinium macrocyclic contrast agent approved for use in MRI of cerebral and spinal lesions and for body imaging. To investigate the safety and efficacy of gadoteric acid in children by extensively reviewing clinical and post-marketing observational studies. Data were collected from 3,810 children (ages 3 days to 17 years) investigated in seven clinical trials of central nervous system (CNS) imaging (n = 141) and six post-marketing observational studies of CNS, musculoskeletal and whole-body MR imaging (n = 3,669). Of these, 3,569 children were 2-17 years of age and 241 were younger than 2 years. Gadoteric acid was generally administered at a dose of 0.1 mmol/kg. We evaluated image quality, lesion detection and border delineation, and the safety of gadoteric acid. We also reviewed post-marketing pharmacovigilance experience. Consistent with findings in adults, gadoteric acid was effective in children for improving image quality compared with T1-W unenhanced sequences, providing diagnostic improvement, and often influencing the therapeutic approach, resulting in treatment modifications. In studies assessing neurological tumors, gadoteric acid improved border delineation, internal morphology and contrast enhancement compared to unenhanced MR imaging. Gadoteric acid has a well-established safety profile. Among all studies, a total of 10 children experienced 20 adverse events, 7 of which were thought to be related to gadoteric acid. No serious adverse events were reported in any study. Post-marketing pharmacovigilance experience did not find any specific safety concern. Gadoteric acid was associated with improved lesion detection and delineation and is an effective and well-tolerated contrast agent for use in children. [\hyperlink{Gadoteridol}{PMID: 26045036}, Csilla Balassy et al., 2015]

\hypertarget{pmid_21786126}{T}here is a paucity of evidence with regard to the safety of contrast medium administration at MRI in neonates and infants. To assess immediate adverse reactions in children younger than 18 months of age during routine clinical utilization of gadoteric acid (Gd-DOTA) in a cohort of patients with nonselected indications. One hundred and four neonates and infants were enrolled in a postmarketing survey with Gd-DOTA (Dotarem, Guerbet, Roissy, France) from a single pediatric hospital. A standardized questionnaire was used to collect the patient information. All included children, ages 3 days to 18 months, received one injection of Gd-DOTA (volume 0.6-4 ml). No immediate adverse event was reported. This postmarketing study involving neonates and infants suggests a favorable safety profile of Gd-DOTA in routine practice. [\hyperlink{Gadoteridol}{PMID: 21786126}, Sophie Emond et al., 2011]

\hypertarget{pmid_1506157}{T}o support clinical use of gadoteridol (0.5 M) injection, a battery of in vitro and in vivo safety studies was conducted. In mice, the acute intravenous LD50 for gadoteridol (0.5 M) injection was 11 to 14 mmol/kg, and the intravenous minimal lethal dose in rats was greater than 10 mmol/kg. In 2-week studies with gadoteridol, no serious effects were observed in mice given 3 mmol/kg or dogs given 1.5 mmol/kg daily. In a series of reproduction studies, no treatment-related adverse effects on fertility, reproductive performance, or postnatal development were seen in rats at doses of 1.5 mmol/kg or less, and no teratogenic effects were observed at doses as high as 6 mmol/kg in rabbits and 10 mmol/kg in rats. In an in vitro test, gadoteridol did not demonstrate any potential to hemolyze human erythrocytes when incubated in high concentrations with whole blood, suggesting there is little probability gadoteridol will cause hemolysis in vivo. A substantial margin of safety exists for the clinical use of gadoteridol in magnetic resonance imaging procedures. [\hyperlink{Gadoteridol}{PMID: 1506157}, R A Soltys et al., 1992]

\hypertarget{pmid_25114540}{G}adobutrol is a 1-molar gadolinium-based contrast agent with well-characterized safety and efficacy for magnetic resonance imaging (MRI) in adults and children ≥ 2 years old. This observational study assessed gadobutrol-enhanced MRI in children < 2 years of age. Sixty infants (mean age 11.1 months) underwent MRI using gadobutrol at standard dose of 0.1 mL/kg (0.1 mmol/kg) body weight. MRI examinations included brain, spine, and neck (n = 24), subcutaneous soft tissues (n = 14), chest, abdomen, and pelvis (n = 12), musculoskeletal system (n = 7) and vascular system (n = 3). No patients experienced adverse events related to gadobutrol injection. In 57 patients with confirmed diagnoses, gadobutrol-enhanced MRI provided findings consistent with confirmed pathologies. This study indicates that gadobutrol at a standard dose for MRI is safe in patients aged < 2 years and provides diagnostic information for multiple pathologies.  [\hyperlink{Gadoteridol}{PMID: 25114540}, Ravi Bhargava et al., 2013] Gadobutrol is a gadolinium-based contrast agent, uniquely formulated at 1.0 mmol/ml. Although there is extensive safety evidence on the use of gadobutrol in adults, few studies have addressed the safety and tolerability of gadobutrol in pediatric patients. This subanalysis of data from the GARDIAN study evaluated the safety and use of gadobutrol in pediatric patients (age <18 years). The GARDIAN study was a large phase IV non-interventional prospective multicenter post-authorization safety study performed in Europe, Asia, North America and Africa. A total of 23,708 patients were included who were scheduled to undergo cranial or spinal MRI, liver or kidney MRI, or MR angiography with gadobutrol enhancement. The primary study endpoint was the overall incidence of adverse drug reactions (ADRs) and serious adverse events (SAEs) following gadobutrol administration. The GARDIAN study included 1,142 children (age <18 years) who received gadobutrol at a mean dose of 0.13 (range 0.04-0.50) mmol/kg body weight. Gadobutrol was well tolerated in these children, with low rates of ADRs (0.5\%) and no SAEs, consistent with results in adults enrolled in the GARDIAN study. Rates of adverse events and ADRs were unrelated to pediatric age or gadobutrol weight-adjusted dose. There were no symptoms suggestive of nephrogenic systemic fibrosis. Investigators rated the contrast quality of gadobutrol-enhanced images as good or excellent in 97.8\% of pediatric patients, similar to the main study population. Gadobutrol is very well tolerated and provides excellent contrast quality at the recommended weight-adjusted dose in children (age <18 years), similar to the profile in adults. [\hyperlink{Gadoteridol}{PMID: 25114540}, Katja Glutig et al., 2016]

\hypertarget{pmid_8496990}{T}his article reports the results of clinical testing in pediatric patients of a new contrast agent, gadoteridol injection (ProHance), developed by Squibb Diagnostic as a nonionic gadolinium agent for use in magnetic resonance imaging (MRI). Thirteen children (four girls and nine boys) ranging in age from 10 to 18 years were enrolled in the study. The children had MR studies of the brain and/or spine with T1-weighted, T2-weighted, and postgadoteridol injection T1-weighted sequences. Five children had primary brain or spine neoplasms, three children had metastatic disease to the central nervous system, one child had a recurrent brain neoplasm and spinal canal metastasis, one child had an arteriovenous malformation, and two children were normal on the MRI studies. No minor or major reactions to gadoteridol injection developed in the 13 patients. Gadoteridol injection provided excellent delineation and enhancement of the arteriovenous malformation and all of the primary and secondary neoplasms of the central nervous system except for one case of a grade 1 glioma of the midbrain. Gadoteridol injection is a safe and excellent contrast agent for use in MRI. [\hyperlink{Gadoteridol}{PMID: 8496990}, S E Byrd et al., 1993]

\hypertarget{pmid_1501959}{T}wenty-two pediatric patients with known CNS neoplasms underwent magnetic resonance (MR) imaging before and after intravenous injection of 0.1 mmol/kg gadoteridol injection as part of a Phase IIIB open label multicenter clinical trial. Intravenous administration of this neutral, nonionic contrast agent was found to be safe in children. No clinically relevant changes in vital signs or laboratory values (including complete blood count, blood chemistry, serum electrolytes, thyroid and metabolic panel and clotting function) were attributed to the administration of gadoteridol injection. There were no systemic complaints. The imaging characteristics of gadoteridol in pediatric CNS disease appeared similar to those of gadopentetate dimeglumine. Contrast enhancement was present in 17 of 22 patients (77\%). The administration of gadoteridol injection provided additional clinically relevant information including improved visualization and delineation of the primary lesion, detection of additional lesions, determination of tumor recurrence and narrowing the list of differential considerations in all 17 enhancing studies as well as in 2 of 5 studies without signal intensity enhancement. The very low toxicity, inherent to this nonionic low osmolal paramagnetic contrast formulation may allow administration of increased doses at increased infusion rates for an increased number of indications with improved sensitivity. [\hyperlink{Gadoteridol}{PMID: 1501959}, J F Debatin et al., 1992]

\hypertarget{pmid_6937455}{H}aloperidol is safe and effective in children for relieving psychotic symptoms associated with childhood autism, schizophrenia and mental retardation. It is the drug of choice for Tourette's syndrome, and may be useful in nonpsychotic hyperactive or aggressive children to control acute episodes, or when the stimulants normally useful in hyperactive children are ineffective. Such children taking haloperidol not only become calmer, but are often better able to respond to other modalities of therapy and to school instruction. Dosage, initially low, is increased gradually to minimize drowsiness and extrapyramidal symptoms, the most common side effects. Haloperidol in children is usually well-tolerated. [\hyperlink{Gadoteridol}{PMID: 6937455}, A C Serrano et al., 1981]

\hypertarget{pmid_8929382}{T}he safety and efficacy of intravenous gadodiamide injection, 0.1 mmol/kg body weight, have been evaluated in an open label, non-comparative as to drug, phase III clinical trial in 50 children from 6 months to 13 years of age, referred for MRI requiring the injection of a contrast medium. The central nervous system and other body areas were examined with T1 sequences before and after intravenous injection of the contrast medium. Overall safety was very good and no clinically relevant changes were evident as regards heart rate and venous blood oxygen saturation after injection. No adverse event or discomfort was experienced by conscious patients that could with certainty be related to the contrast medium, but slight movements were observed in two sedated patients that could be related to the injection. Comparing pre- and post-injection images, additional diagnostic information could be obtained from the latter in 41 patients (82\%). In these images, the number of lesions detected increased and they were generally better delineated and their size more easily estimated. The results of this trial indicate that gadodiamide injection is safe and effective for MRI examinations in children. [\hyperlink{Gadoteridol}{PMID: 8929382}, S Hanquinet et al., 1996]

\hypertarget{pmid_27315460}{I}n the mouse, when a tympanic perforation is present, gadoteridol does not seem to cause ototoxicity. Gadodiamide may cause mild ototoxicity other than toxicity to the outer hair cells of the cochlea. Endolymphatic hydrops have been visualized through intra-tympanic injection of gadolinium-based contrast agents (GBCAs) and three-dimensional fluid-attenuated inversion recovery (3-D FLAIR) magnetic resonance imaging. However, reports on the safety of GBCAs are limited. This study aimed to assess ototoxicity of gadoteridol and gadodiamide. In a prospective, randomized, controlled trial, myringotomies in the left ear were performed in 20 male C57 BL/6 mice. After testing the baseline auditory brainstem response (ABR) (range = 8-32 kHz), the test solution (gadoteridol, gadodiamide, saline, or cisplatin) was injected into the left ear. ABR testing was repeated 14 days after test solution application. In morphological experiments, images of post-mortem surface preparations were assessed for cochlear hair cell status. At 14 days following gadoteridol application, there was no significant change in ABR thresholds at 8, 16, or 32 kHz. Gadodiamide application caused a significant change in the ABR threshold at 8 kHz. Apparent cochlear hair cell loss was not observed in the surface preparation after gadoteridol or gadodiamide application. [\hyperlink{Gadoteridol}{PMID: 27315460}, Hiroshi Nonoyama et al., 2016]

\hypertarget{pmid_1506153}{T}his study assesses the efficacy of gadoteridol for contrast-enhanced magnetic resonance imaging (MRI) in children. Patients were examined by MRI before and after receiving 0.10 mmol/kg gadoteridol. Blinded and unblinded readers analyzed brain and spine MRI studies from a multicenter clinical trial involving 101 patients at 11 sites. Ninety-two cases (76 brain, 16 spine) were evaluated by unblinded investigators, and 91 cases (76 brain, 15 spine) were evaluated by three neuroradiologists unaffiliated with any investigational site and blinded to clinical information. Unblinded readers noted enhancement of brain pathology in 70\% of cases versus 50\% to 67\% among blinded readers. Unblinded readers determined that additional diagnostic information was available after contrast in 82\% of brain studies (average, 64\% for blinded readers) and would have changed patient diagnoses in 48\% of these studies (average, 46\% for blinded readers). In spine cases, enhancement of pathology was noted in 38\% (unblinded) and 33\% to 40\% (blinded). Additional diagnostic information was available after contrast in 63\% of spine studies (unblinded), or an average of 58\% (blinded), and patient diagnoses would have changed in 20\% (unblinded), or an average of 59\% (blinded). Gadoteridol is suitable for enhanced MRI detection, localization, and characterization of central nervous system pathology in children. [\hyperlink{Gadoteridol}{PMID: 1506153}, W S Ball et al., 1992]

\hypertarget{pmid_1506151}{T}he use of paramagnetic contrast agents has improved the diagnostic sensitivity and specificity of magnetic resonance imaging (MRI) for evaluating diseases of the central nervous system. To assess the safety and imaging properties of the nonionic, gadolinium-based MRI contrast agent gadoteridol, 151 patients and controls were evaluated for safety, and 118 patients with cerebral or spinal pathology were evaluated for imaging efficacy. Precontrast T1- and T2-weighted spin-echo images and postcontrast (0.10 mmol/kg) T1-weighted spin-echo images were read by unblinded investigators at each site. The rate of adverse events possibly or probably related to gadoteridol was 4.0\% (vasodilation [facial flushing], 1 patient; nausea, 3 patients; urticaria, 2 patients). Laboratory changes were reported in 6.0\%. None of these events or changes was considered to be clinically significant. Contrast enhancement was noted in 75\% of cases with brain pathology and 64\% of cases involving spine lesions. Gadoteridol is safe in routine clinical use at a dose of 0.10 mmol/kg and provides improved lesion detection compared to plain MRI. [\hyperlink{Gadoteridol}{PMID: 1506151}, M Seiderer et al., 1992]

\hypertarget{pmid_28932122}{G}adobutrol is a gadolinium (Gd)-based contrast agent for magnetic resonance imaging (MRI). In India, gadobutrol is approved for MRI of the central nervous system (CNS), liver, kidneys, breast and for MR angiography for patients 2 years and older. The standard dose for all age groups is 0.1 mmol/kg body weight. The safety profile has been demonstrated in 42 clinical phase 2 to 4 studies (>6800 patients), 7 observational studies, and by assessing pharmacovigilance data of 29 million applications. Furthermore, studies in children, adults, and elderly and in patients with impaired liver or kidney function did not show any increased adverse event rate. Diagnostic efficacy was demonstrated in numerous studies and various indications, such as diseases of the CNS, peripheral and supra-aortic vessels, kidneys, liver, and breast. [\hyperlink{Gadoteridol}{PMID: 28932122}, Jan Endrikat et al., 2017]

\hypertarget{pmid_16028153}{B}ecause of concerns about arthrotoxicity, fluoroquinolones are restricted for use in children. This study describes the safety and efficacy of gatifloxacin when used for treatment of children with recurrent acute otitis media (ROM) or acute otitis media (AOM) treatment failure (AOMTF). We performed an analysis of 867 children included in 4 clinical trials who had ROM and/or AOMTF and were treated with gatifloxacin (10 mg/kg once daily for 10 days). Gatifloxacin had adverse event rates that were similar overall to those of a comparator antibiotic (amoxicillin-clavulanate), except for increased diarrhea in children <2 years old receiving amoxicillin-clavulanate. There was no evidence of arthrotoxicity, hepatotoxicity, alteration of glucose homeostasis, or central nervous system toxicity acutely or during 1 year follow-up in any child. Regarding efficacy, in 2 noncomparative trials, the gatifloxacin cure rate of AOM was 89\% (95\% confidence interval [CI], 83\%-95\%) at the test of cure (TOC) visit, 3-10 days after completion of therapy. In 2 comparative trials of gatifloxacin versus amoxicillin-clavulanate, the efficacy of gatifloxacin was 88\% (95\% CI, 82\%-94\%). Gatifloxacin led to better clinical outcomes than amoxicillin-clavulanate for AOMTF (91\% vs. 81\%; P=.029), for AOMTF and age <2 years old (89\% vs. 69\%; P=.009), and for severe AOM in children <2 years old (90\% vs. 75\%; P=.012). Among children with AOMTF previously treated with amoxicillin-clavulanate or ceftriaxone injections, gatifloxacin cure rates were high (88\% and 75\%, respectively). Gatifloxacin appears to be safe for children, with no evidence of producing arthrotoxicity in 867 children exposed to the antibiotic when used as treatment for ROM and AOMTF. [\hyperlink{Gadoteridol}{PMID: 16028153}, Michael E Pichichero et al., 2005]

\hypertarget{pmid_30450703}{S}edation is often required for young children during transthoracic echocardiography. Dexmedetomidine and ketamine are two sedatives that are commonly used in children for procedural sedation, but they have some disadvantages when they are used alone. The aim of this retrospective study was to analyze the effects and safety of intranasal sedation with a combination of dexmedetomidine and ketamine during transthoracic echocardiography in young children and to analyze risk factors for sedation failure. After IRB approval, we retrospectively evaluated data on patients who underwent echocardiography between May 2016 and August 2017 utilizing a combination of dexmedetomidine 2 μg/kg and ketamine 1 mg/kg. We collected information including heart rate, pulse oxygen saturation, sedation onset time, exam time, recovery time, and adverse reactions. Stepwise logistic regression analyses were performed to analyze the risk factors for sedation failure. Sedation was successful in 2212 patients (96\%) and took effect in 15.7 (IQR: 10-23) min, while sedation failed in 92 patients. Cyanotic heart disease, history of sedation failure, history of congenital heart disease surgery, and fever were independent risk factors for sedation failure. Most of the patients in this study had an American Society of Anesthesiologists (ASA) grade of II to III, but no severe adverse reactions were observed. Intranasal sedation with a combination of dexmedetomidine and ketamine is effective and appears to have an acceptable safety profile for young children during transthoracic echocardiography. [\hyperlink{Gadoteridol}{PMID: 30450703}, Jianxia Liu et al., 2019]

\hypertarget{pmid_26858095}{S}edation is increasingly used to facilitate procedures on children in emergency departments (EDs). This overview of systematic reviews (SRs) examines the safety and efficacy of sedative agents commonly used for procedural sedation in children in the ED or similar settings. We followed standard SR methods: comprehensive search; dual study selection, quality assessment, data extraction. We included SRs of children (1 month to 18 years) where the indication for sedation was procedure-related and performed in the ED. Fourteen SRs were included (210 primary studies). The most data were available for propofol (six reviews/50,472 sedations) followed by ketamine (7/8,238), nitrous oxide (5/8,220), and midazolam (4/4,978). Inconsistent conclusions for propofol were reported across six reviews. Half concluded that propofol was sufficiently safe; three reviews noted a higher occurrence of adverse events, particularly respiratory depression (upper estimate 1.1\%; 5.4\% for hypotension requiring intervention). Efficacy of propofol was considered in four reviews and found adequate in three. Five reviews found ketamine to be efficacious and seven reviews showed it to be safe. All five reviews of nitrous oxide concluded it is safe (0.1\% incidence of respiratory events); most found it effective in cooperative children. Four reviews of midazolam made varying recommendations. To be effective, midazolam should be combined with another agent that increases the risk of adverse events (upper estimate 9.1\% for desaturation, 0.1\% for hypotension requiring intervention). This comprehensive examination of an extensive body of literature shows consistent safety and efficacy for nitrous oxide and ketamine, with very rare significant adverse events for propofol. There was considerable heterogeneity in outcomes and reporting across studies and previous reviews. Standardized outcome sets and reporting should be encouraged to facilitate evidence-based recommendations for care. [\hyperlink{Gadoteridol}{PMID: 26858095}, Lisa Hartling et al., 2016]

\hypertarget{pmid_26197466}{G}onadotropin-releasing hormone analogues are generally regarded as safe drugs. Gonadorelin acetate has been widely used for the diagnosis of central precocious puberty, and life-threatening reactions to gonadorelin acetate are extremely rare. Herein, we described - to the best of our knowledge - the first pediatric case in which severe anaphylaxis was encountered after intravenous gonadorelin acetate administration. An 8-year-old girl who was diagnosed with central precocious puberty was receiving triptorelin acetate treatment uneventfully for 6 months. In order to evaluate the efficacy of the treatment, an LH-RH stimulation test with gonadorelin acetate was planned. Within 3 min after intravenous administration of gonadorelin acetate, she lost consciousness and tonic seizures began in her hands and feet. She was immediately treated with epinephrine, diphenhydramine, and fluids. Her vital signs recovered within 30 min. Based on the results, anaphylaxis should be anticipated and the administration of these drugs should be performed in a setting that is equipped to deal with systemic reactions.  [\hyperlink{Gadoteridol}{PMID: 26197466}, Onur Akın et al., 2015] The European Medicine Agency recommendations limiting codeine use in children have created a void in managing moderate pain. We review the evidence on the pharmacokinetic, pharmacodynamic and safety profile of tramadol, a possible substitute for codeine. Tramadol appears to be safe in both paediatric inpatients and outpatients. It may be appropriate to limit the current use of tramadol to monitored settings in children with risk factors for respiratory depression, subject to further safety evidence. [\hyperlink{Gadoteridol}{PMID: 26197466}, Pierluigi Marzuillo et al., 2014]

\hypertarget{pmid_8073972}{T}o assess the efficacy and safety profile of high-dose (0.3 mmol/kg cumulative dose) gadoteridol in patients with suspected central nervous system metastatic disease. We studied 67 patients using an incremental-dose technique. Patient monitoring included a medical history, physical examination, vital signs, and extensive laboratory tests within 24 hours before and after the MR examination. Precontrast T1- and T2-weighted spin-echo studies were performed, followed by intravenous injection of 0.1 mmol/kg of gadoteridol. T1-weighted images were acquired immediately after and at 10 and 20 minutes after injection. At 30 minutes an additional 0.2 mmol/kg of gadoteridol was administered (0.3-mmol/kg cumulative dose), and T1-weighted images were acquired. Cases demonstrating abnormal MR findings were assessed for efficacy by unblinded and blinded reviewers and were analyzed quantitatively. Three adverse effects in two patients were considered to be related to gadoteridol administration. No adverse effects were serious; all self-resolved. Forty-nine cases showed abnormal MR findings and were included in the efficacy analysis. A significantly greater number of lesions was seen on the high-dose as opposed to the standard-dose images. Blinded and unblinded readers identified 5 and 8 patients, respectively, with solitary lesions on standard-dose examination and multiple lesions on high-dose examination. Two patients who had normal standard-dose findings had lesions identified on high-dose studies. Quantitative analysis of 133 lesions in 45 patients demonstrated significant increases in lesion signal intensity on high-dose studies when compared with standard-dose studies. Gadoteridol can be safely administered up to a cumulative dose of 0.3 mmol/kg. High-dose contrast studies provide improved lesion detectability and additional diagnostic information over studies performed in the same patients with a 0.1-mmol/kg dose and aid in patient diagnosis and treatment. High-dose gadoteridol study may facilitate the care of patients with suspected central nervous system metastasis. [\hyperlink{Gadoteridol}{PMID: 8073972}, W T Yuh et al., 1994]

\hypertarget{pmid_34698441}{T}here is a paucity of data regarding the safety of the practice of sedation for oro-dental trauma in paediatric emergency departments (ED). A previous study reported the safety of intramuscular ketamine administered as a single agent. In the paediatric ED of a tertiary trauma centre in Israel, one of two ketamine-based regimens is used for sedating children with intraoral injuries according to the physician's discretion: a single dose of intramuscular ketamine or a combination of ketamine and propofol (KP) intravenously. The aim of this study was to assess the safety of KP sedation in children undergoing emergency treatment of oro-dental injuries in this paediatric ED. The primary outcome was sedation adverse events that required intervention (SAERI): prolonged oxygen desaturation and apnoea, laryngospasm, hypotension, bradycardia, partial or complete airway obstruction, and pulmonary aspiration. During the 2 years study period, 17 children were sedated with KP, 20 with intramuscular ketamine and 29 with nitrous oxide. Patients who were treated with ketamine-based sedation or with nitrous oxide sedation had a median (interquartile range, IQR) age of 3 (2-4) years and 7 (5-9) years, respectively. No SAERI occurred in patients who were sedated with intramuscular ketamine. One (3.4\%) SAERI was reported in a patient who was sedated with N [\hyperlink{Gadoteridol}{PMID: 34698441}, Leon Bilder et al., 2022] To demonstrate that gadodiamide injection is a safe and efficient contrast agent for MRI in infants younger than 6 months of age. The authors designed a phase III multicenter nonrandomized study using a control group. Gadodiamide injection at a dosage of 0.1 mmol/kg body weight was administered to 39 children; 20 received no contrast. The mean age was 10.6 weeks in the contrast group and 9.3 weeks in the control group. MR examinations, blood (serum creatinine, S-ASAT, S-ALAT, S-bilirubin, alkaline phosphatase) and urine (proteins, blood, others) sampling before sedation and after examination, heart rate (electrocardiography) and oxygen saturation (pulse oximetry) during examination, adverse events, and efficacy parameters were analyzed. In the contrast group, 18 (51.4\%) children had 31 abnormal changes in one or more of the safety parameters and vital signs. In the control group there were 16 (80.0\%) children with 19 abnormal changes. Gadodiamide injection had no negative influence on the safety parameters. No serious adverse events occurred, and only three clinically relevant adverse events (elevation of S-ALAT and S-ASAT, elevation of bilirubin) in two patients in the contrast group and one event (vomiting) in one patient in the control group were documented. The benefit of the contrast medium was clearly shown for all evaluated parameters. Gadodiamide injection is safe, well tolerated, and effective in infants younger than 6 months of age. [\hyperlink{Gadoteridol}{PMID: 34698441}, L Martí-Bonmatí et al., 2000]

\hypertarget{pmid_25746065}{S}ildenafil (Revatio®) and tadalafil (Adcirca®) are specific inhibitors of the phosphodiesterase-5 enzyme and produce pulmonary vasodilation by inhibiting the breakdown of cyclic guanosine monophosphate (cGMP) in the walls of pulmonary arterioles. We focus on the efficacy and safety of sildenafil and tadalafil in the treatment of pulmonary hypertension (PH) in children through a PubMed literature search. Although used since 1999 in the treatment of PH in children, it is only in the past few years that robust evidence for the use of sildenafil has emerged principally in the pivotal STARTS-1 study. The open-label extension of this study, STARTS-2, has revealed safety concerns substantiated by FDA post marketing surveillance leading to recommendations to use lower doses. More recently, tadalafil has been introduced allowing once daily dosing with apparently similar efficacy to sildenafil in children. Recently there have been suggestions that sildenafil and tadalafil may have a place in treating muscular dystrophy. [\hyperlink{Gadoteridol}{PMID: 25746065}, Alan G Magee et al., 2015]

\hypertarget{pmid_7772422}{I}n a prospective, randomized, blind study, we assessed the effectiveness of droperidol 20 micrograms kg-1 i.v., given at induction of anaesthesia, in preventing postoperative vomiting in paediatric day-case patients. We studied 270 children, aged 1-15 yr, undergoing body surface surgery. There was a significant reduction in the incidence of vomiting in the recovery room (1.4\% vs 9.2\%, P < 0.005) and in the day ward (9.4\% vs 18.3\%, P < 0.05) in patients receiving droperidol. There was no significant difference on the journey home (9.5\% vs 17.83\%, ns) or at home (16.7\% vs 10.3\%, ns). There was also a reduction in the severity of vomiting in the droperidol group. There were no adverse side effects. [\hyperlink{Gadoteridol}{PMID: 7772422}, D V Lunn et al., 1995]

\hypertarget{pmid_37812485}{T}his review describes the pharmacokinetics, efficacy, and safety of gadopiclenol, a new macrocyclic gadolinium-based contrast agent (GBCA) recently approved by the Food and Drug Administration at the dose of 0.05 mmol/kg. Gadopiclenol is a high relaxivity contrast agent that shares similar pharmacokinetic characteristics with other macrocyclic GBCAs, including a predominant renal excretion. In pediatric patients aged 2-17 years, the pharmacokinetic parameters (assessed through a population pharmacokinetics model) were comparable to those observed in adults, indicating no need for age-based dose adjustment. For contrast-enhanced magnetic resonance imaging (MRI) of the central nervous system (CNS) and body indications, gadopiclenol at 0.05 mmol/kg was shown to be noninferior to gadobutrol at 0.1 mmol/kg in terms of 3 lesion visualization parameters (ie, lesion border delineation, internal morphology, and contrast enhancement). Moreover, for contrast-enhanced MRI of the CNS, compared with gadobenate dimeglumine at 0.1 mmol/kg, gadopiclenol exhibited superior contrast-to-noise ratio at 0.1 mmol/kg and comparable contrast-to-noise ratio at 0.05 mmol/kg. A pooled safety analysis of 1047 participants showed a favorable safety profile for gadopiclenol. Comparative studies showed that the incidence and nature of adverse drug reactions with gadopiclenol were comparable to those observed with other GBCAs. Importantly, no significant safety concerns were identified in pediatric and elderly patients, as well as in patients with renal impairment. Overall, these findings support the clinical utility and safety of gadopiclenol for MRI in adult and pediatric patients aged 2 years and older in CNS and body indications. [\hyperlink{Gadoteridol}{PMID: 37812485}, Jing Hao et al., 2023]

\section*{Glycopyrrolate}
\subsection*{Result}
\subsubsection*{Answer}

Ages <1 year: Unknown  
Ages 1–3 years: Unknown  
Ages 3–18 years: Yes  

\subsubsection*{{Explanation}}
\hypertarget{Glycopyrrolate}
Based on the abstracts reviewed, there are several targeted studies evaluating the safety of glycopyrrolate in children for various indications, primarily for sialorrhea (drooling) associated with neurological conditions, as well as for hyperhidrosis and perioperative use. The evidence is summarized below by age range and indication:

Ages 3–16 years (Sialorrhea/Drooling):
- Multiple studies, including randomized controlled trials and open-label studies, specifically evaluated glycopyrrolate in children aged 3–16 years with neurological conditions causing drooling. These studies consistently found glycopyrrolate to be effective and generally well tolerated, though anticholinergic side effects (dry mouth, constipation, etc.) were common and led to discontinuation in a minority of patients (about 20–28\%) [\hyperlink{pmid_22003294}{PMID: 22003294}, Marian L Evatt et al., 2011; \hyperlink{pmid_22298950}{PMID: 22298950}, Robert S Zeller et al., 2012; \hyperlink{pmid_22646067}{PMID: 22646067}, Karly P Garnock-Jones et al., 2012; \hyperlink{pmid_8790123}{PMID: 8790123}, P A Blasco et al., 1996; \hyperlink{pmid_11115305}{PMID: 11115305}, R J Mier et al., 2000; \hyperlink{pmid_9069045}{PMID: 9069045}, L M Stern et al., 1997; \hyperlink{pmid_9729704}{PMID: 9729704}, S J Bachrach et al., 1998]. The US FDA has approved glycopyrrolate for this age group for this indication, and the studies affirm its safety profile in this population.

Ages 3–18 years (Sialorrhea/Drooling):
- An additional noncomparative phase III study included children up to 18 years and found similar efficacy and tolerability [\hyperlink{pmid_22646067}{PMID: 22646067}, Karly P Garnock-Jones et al., 2012].

Ages 1 month–18 years (Perioperative Use):
- Randomized controlled trials and pharmacokinetic studies have evaluated glycopyrrolate for perioperative indications (e.g., as an anticholinergic premedicant, to prevent bradycardia, or to reduce secretions) in children as young as 1 month. These studies found glycopyrrolate to be effective and did not report significant safety concerns, though the primary focus was not long-term use [\hyperlink{pmid_17242078}{PMID: 17242078}, Alan R Tait et al., 2007; \hyperlink{pmid_8060629}{PMID: 8060629}, P Rautakorpi et al., 1994; \hyperlink{pmid_7137551}{PMID: 7137551}, R K Mirakhur et al., 1982; \hyperlink{pmid_7126399}{PMID: 7126399}, R K Mirakhur et al., 1982; \hyperlink{pmid_7446931}{PMID: 7446931}, G W Black et al., 1980; \hyperlink{pmid_6871778}{PMID: 6871778}, J Lerman et al., 1983].

Ages <1 year:
- There is limited data for infants under 1 year. One pharmacokinetic study included children under 1 year and found no significant safety concerns with a single intravenous dose, but this was not a long-term safety study [\hyperlink{pmid_8060629}{PMID: 8060629}, P Rautakorpi et al., 1994]. Another small case series reported use in infants as young as 5 months for breath-holding spells, with apparent benefit and no major safety issues, but the sample size was very small (n=4) and not sufficient to definitively establish safety [\hyperlink{pmid_25869368}{PMID: 25869368}, Jason Williams et al., 2015].

Ages 9–18 years (Hyperhidrosis):
- Retrospective and institutional reviews of oral glycopyrrolate for hyperhidrosis in children and adolescents (mean age of initiation \textasciitilde{}14.8 years, range 9–18) found it to be effective and generally well tolerated, with side effects (mainly dry mouth and eyes) in a minority of patients [\hyperlink{pmid_22405644}{PMID: 22405644}, Amy S Paller et al., 2012; \hyperlink{pmid_24266878}{PMID: 24266878}, Monique G Kumar et al.]. A topical formulation (glycopyrronium tosylate) was also studied in children as young as 9 years, with a low risk of anticholinergic adverse events when used as directed [\hyperlink{pmid_33433785}{PMID: 33433785}, David M Pariser et al., 2021].

Summary:
- For children aged 3–18 years, especially those with neurological conditions and sialorrhea, there is strong evidence from targeted studies affirming the safety (with expected anticholinergic side effects) of glycopyrrolate.
- For children under 3 years, especially infants, the safety profile is less well established, with only limited pharmacokinetic and case series data.
- For perioperative and hyperhidrosis indications, studies in children (including some as young as 1 month for perioperative use and 9 years for hyperhidrosis) suggest glycopyrrolate is generally safe, but the evidence is less robust than for sialorrhea in the 3–18 year age group.

\subsection*{Abstracts}
\hypertarget{pmid_22003294}{E}xcessive drooling may complicate the care of children with chronic neurological conditions by socially isolating both patients and families and by causing secondary dermatitis and infection. Normal control of saliva requires normal integrity of oral structures, normal oropharyngeal sensation, and motor functioning, as well as normal cognitive awareness and rate of salivary production. Glycopyrrolate is an anticholinergic medication with a quaternary structure that recently received Food and Drug Administration approval to treat sialorrhea due to neurological problems in children ages 3-16 years. This review summarizes the few published studies of safety and efficacy of glycopyrrolate for drooling in children with chronic neurological conditions. [\hyperlink{Glycopyrrolate}{PMID: 22003294}, Marian L Evatt et al., 2011]

\hypertarget{pmid_11115305}{T}o determine the safety and efficacy of glycopyrrolate in the treatment of developmentally disabled children with sialorrhea. Placebo-controlled, double-blind, crossover dose-ranging study. Outpatient facilities in 2 pediatric hospitals. Thirty-nine children with both developmental disabilities and excessive and bothersome sialorrhea. Parent and investigator evaluation of change in sialorrhea and adverse effects. Glycopyrrolate in doses of 0.10 mg/kg per dose is effective at controlling sialorrhea. Even at low doses, 20\% of children may exhibit adverse effects severe enough to require discontinuation. Glycopyrrolate is effective in the control of excessive sialorrhea in children with developmental disabilities. Approximately 20\% of children given glycopyrrolate may experience substantial adverse effects, enough to require discontinuation of medication. Arch Pediatr Adolesc Med. 2000;154:1214-1218. [\hyperlink{Glycopyrrolate}{PMID: 11115305}, R J Mier et al., 2000]

\hypertarget{pmid_9069045}{A} study was undertaken to assess the efficacy of an oral anticholinergic drug, glycopyrrolate, in the management of drooling in children and young adults with disabilities. Glycopyrrolate was used by 24 children and young adults for up to 28 months. Parents/carers were asked to complete a questionnaire on the effects of the drug on severity and frequency of drooling and to report any side-effects. Twenty-two questionnaires were returned. There was a statistically significant decrease in both severity and frequency of drooling with minimal side-effects reported. In this preliminary study, glycopyrrolate was found to be an effective and well-tolerated addition to the management of drooling in children with disabilities. [\hyperlink{Glycopyrrolate}{PMID: 9069045}, L M Stern et al., 1997]

\hypertarget{pmid_24266878}{P}rimary hyperhidrosis is a common disorder affecting children and adolescents, and it can have a significant negative psychosocial effect. Treatment for pediatric hyperhidrosis tends to be limited by low efficacy, low adherence, and poor tolerance. Oral glycopyrrolate is emerging as a potential second-line treatment option, but experience with safety, efficacy, and dosing is especially limited in children. We present an institutional review of 12 children with severe, refractory hyperhidrosis treated with oral glycopyrrolate; 11 (92\%) noted improvement and 9 (75\%) would recommend oral glycopyrrolate to their friends. No significant side effects were noted. Our retrospective analysis suggests that oral glycopyrrolate is safe and effective in children with hyperhidrosis.  [\hyperlink{Glycopyrrolate}{PMID: 24266878}, Monique G Kumar et al., ] Chronic drooling (sialorrhea) is a common dysfunction in children with neurologic disorders such as cerebral palsy. Glycopyrrolate oral solution, an anticholinergic agent, is the first drug treatment approved in the US for drooling in children with neurologic conditions. This article reviews the clinical efficacy and tolerability of glycopyrrolate oral solution in pediatric patients with neurologic conditions and provides an overview of the pharmacological properties of the drug. In a phase III, randomized, double-blind, multicenter trial, children (aged 3-16 years; n = 36) with problem drooling associated with neurologic conditions and receiving glycopyrrolate oral solution had a significantly (p < 0.01) greater modified Teacher's Drooling Scale (mTDS) response rate at 8 weeks (primary endpoint) than those receiving placebo (73.7\% vs 17.6\%). At 24 weeks in an additional, noncomparative, phase III study, 52.3\% of glycopyrrolate oral solution recipients (aged 3-18 years; n = 137) had an mTDS response (primary endpoint); the response rate was consistently above 50\% at all 4-weekly timepoints, aside from the first assessment at week 4 (40.3\%). In general, glycopyrrolate oral solution was well tolerated in clinical trials. The majority of adverse events were within expectations as characteristic anticholinergic outcomes. [\hyperlink{Glycopyrrolate}{PMID: 24266878}, Karly P Garnock-Jones et al., 2012]

\hypertarget{pmid_8060629}{T}o investigate the pharmacokinetics of glycopyrrolate in children. Open study with three parallel groups. Pediatric surgery department at a university hospital. 26 healthy ASA physical status I children undergoing minor surgery. Patients were assigned to 1 of 3 groups: under 1 year of age (Group 1, n = 8), between 1 and 3 years of age (Group 2, n = 7), and over 3 years of age (Group 3, n = 11). Glycopyrrolate 5 micrograms/kg was given as a single intravenous (i.v.) injection before induction of general anesthesia. Blood samples (for determination of drug concentrations in plasma) were collected via venous cannula inserted into the contralateral antecubital vein. ECG was observed continuously, blood pressure was measured with an automatic noninvasive device, and blood samples were taken just before and at 2, 4, 6, 10, 15, 30, 60, 120, 180, 240, 360, and 480 minutes after injection of glycopyrrolate. Glycopyrrolate concentrations in plasma were determined with a radioreceptor assay. The only significant difference in the pharmacokinetic parameters was the shortened elimination half-life in patients between 1 and 3 years of age. Glycopyrrolate 5 micrograms/kg i.v. did not cause any significant alterations in heart rate. There were no significant changes in the distribution volume or clearance of glycopyrrolate in children of different ages. The shortened elimination half-life in children between 1 and 3 years of age is of minor clinical importance. [\hyperlink{Glycopyrrolate}{PMID: 8060629}, P Rautakorpi et al., ]

\hypertarget{pmid_17242078}{T}wo recent studies have identified copious secretions as an independent risk factor for perioperative adverse events in children who present for elective surgery in the presence of an upper respiratory tract infection (URI). We designed this study, therefore, to determine whether the administration of the anticholinergic drug, glycopyrrolate, to children with URIs would reduce the incidence of adverse perioperative respiratory events. One hundred thirty children (1 mo to 18 yr of age) who presented for elective surgery with a URI were randomized to receive either 0.01 mg/kg glycopyrrolate or placebo and were followed for the appearance and severity of any perioperative respiratory adverse events. The two groups were similar with respect to demographics, presenting URI symptoms, anesthetic management, and surgical procedure. In the intention-to-treat analysis, there were no statistical differences in the incidence or severity of perioperative respiratory adverse events between the glycopyrrolate and placebo groups (45.2\% vs 37.5\% respectively, P = NS). Furthermore, there were no differences in outcome between the two groups when children with congestion and secretions were analyzed separately (45.0\% vs 37.0\%, respectively). However, compared with the placebo group, children in the glycopyrrolate group had significantly shorter discharge times (83.9 min vs 111.4 min, P = 0.024), and significantly less postoperative nausea and vomiting (10.7\% vs 33.3\%, P = 0.005). These results suggest that glycopyrrolate, administered after induction of anesthesia to children with URIs, does not reduce the incidence of perioperative respiratory adverse events, and thus may not be clinically indicated for routine use in this population. [\hyperlink{Glycopyrrolate}{PMID: 17242078}, Alan R Tait et al., 2007]

\hypertarget{pmid_6859501}{G}lycopyrrolate is a quaternary ammonium compound with indications for use similar to those for atropine. Because of the quaternary nature, it is poorly absorbed when taken orally and penetrates neither placental nor blood-brain barriers. When given by the parenteral route, the cardio-vagal blocking action of glycopyrrolate is twice that of atropine while inhibition of salivation is 5-6 times greater. The use of glycopyrrolate for premedication provides a therapeutic margin 2-3 times wider than that of atropine. Glycopyrrolate administered with neostigmine to antagonise the residual neuromuscular blockade of non-depolarising relaxants has advantages over atropine because the pharmacodynamic profile is more suited to that of neostigmine. The abrupt changes in cardiac rate, therefore, become minimal. If glycopyrrolate, 5 micrograms/kg-1, is injected intravenously just before the induction of anaesthesia, severe bradycardia is inhibited when repeated doses of succinylcholine are used. Although the alkalinising effect on gastric secretions has not been substantially verified, glycopyrrolate does provide long lasting bronchodilatation from its blocking action on smooth muscle. Only a few studies with glycopyrrolate in children have yet been published. However, it appears that this drug provides no real advantages over atropine when used in paediatric anaesthesia. [\hyperlink{Glycopyrrolate}{PMID: 6859501}, D A Cozanitis et al., 1983]

\hypertarget{pmid_9729704}{F}ifty-four parents/caretakers of children with cerebral palsy were surveyed regarding their use of antisialorrheic medication for excessive drooling. Glycopyrrolate was used by 37 of 41 respondents, with significant improvement in drooling noted in the vast majority (95\%) of cases as indicated by a five-point rating scale. Side effects (dry mouth, thick secretions, urinary retention, or flushing) surfaced in almost half (44\%) of the patients but necessitated discontinuation of pharmacologic treatment in less than a third. While larger clinical studies are needed, our preliminary data indicate a trial of glycopyrrolate should be considered in children with cerebral palsy where drooling is a significant problem. [\hyperlink{Glycopyrrolate}{PMID: 9729704}, S J Bachrach et al., 1998]

\hypertarget{pmid_9783332}{B}ased on plasma levels determined with a radioreceptor assay and following a single oral (50 micrograms/kg) and intravenous (5 micrograms/kg) administration of glycopyrrolate in six healthy children operated twice during a several weeks period, a negligible and variable oral bioavailability was found (3.3; 1.3-13.3\%) (median;range). No significant changes in heart rate after oral or intravenous administration of the drug could be seen. Oral glycopyrrolate appears to have no place in paediatric premedication. [\hyperlink{Glycopyrrolate}{PMID: 9783332}, P Rautakorpi et al., 1998]

\hypertarget{pmid_8481249}{G}lycopyrrolate is an anticholinergic agent used to dry oral secretions and has been advocated for routine use with transesophageal echocardiography (TEE). To evaluate the safety and efficacy of glycopyrrolate for this unique application, a prospective double-blind placebo-controlled study of glycopyrrolate was performed in 61 patients who were awake while undergoing TEE. Thirty patients were randomized to the standard dose of glycopyrrolate (0.2 mg intravenously), and 31 patients received 1 ml of saline solution as placebo. Intravenous midazolam was used for sedation in all but one patient. Heart rate, electrocardiogram, blood pressure, and oxygen saturation were continuously monitored before, during, and after TEE. The patients scored their comfort immediately after TEE and were interviewed at 24 hours for side effects. The operator scored the ease of performing the TEE. No complications occurred in either group. Changes in vital signs and oxygen saturation were similar in both groups. The operator ease and patient comfort was similar in both groups. A significantly higher incidence of the following side effects was observed at 24 hours in patients who received glycopyrrolate versus those who received placebo: sore throat, 63\% versus 19\%; dry mouth, 43\% versus 6\%; and urinary retention, 16\% versus 0\% (p < 0.05 for all). No benefit from glycopyrrolate was noted in operator ease or patient comfort. In conclusion, glycopyrrolate is not recommended for routine use when performing TEE on patients who are awake. [\hyperlink{Glycopyrrolate}{PMID: 8481249}, J Gorcsan et al., ]

\hypertarget{pmid_22298950}{T}o evaluate the efficacy of glycopyrrolate oral solution (1 mg/5 mL) in managing problem drooling associated with cerebral palsy and other neurologic conditions. Thirty-eight patients aged 3-23 years weighing at least 27 lb (12.2 kg) with severe drooling (clothing damp 5-7 days/week) were randomized to glycopyrrolate (n = 20), 0.02-0.1 mg/kg three times a day, or matching placebo (n = 18). Primary efficacy endpoint was responder rate, defined as percentage showing ≥3-point change on the modified Teacher's Drooling Scale (mTDS). Responder rate was significantly higher for the glycopyrrolate (14/19; 73.7\%) than for the placebo (3/17; 17.6\%) group (P = 0.0011), with improvements starting 2 weeks after treatment initiation. Mean improvements in mTDS at week 8 were significantly greater in the glycopyrrolate than in the placebo group (3.94 ± 1.95 vs 0.71 ± 2.14 points; P < 0.0001). In addition, 84\% of physicians and 100\% of parents/caregivers regarded glycopyrrolate as worthwhile compared with 41\% and 56\%, respectively, for placebo (P ≤ 0.014). Most frequently reported treatment-emergent adverse events (glycopyrrolate vs placebo) were dry mouth, constipation, and vomiting. Children aged 3-16 years with problem drooling due to neurologic conditions showed a significantly better response, as assessed by mTDS, to glycopyrrolate than to placebo.  CLINICALTRIALS.GOV IDENTIFIER: NCT00425087. [\hyperlink{Glycopyrrolate}{PMID: 22298950}, Robert S Zeller et al., 2012]

\hypertarget{pmid_7446931}{A}tropine and glycopyrrolate were compared when given in a mixture with neostigmine for the reversal of non-depolarising neuromuscular block in children. Glycopyrrolate was an effective antimuscarinic agent and could be safely used as an alternative to atropine, although the advantages in this age group were not as marked as have been observed in adults. [\hyperlink{Glycopyrrolate}{PMID: 7446931}, G W Black et al., 1980]

\hypertarget{pmid_33433785}{G}lycopyrronium tosylate (GT; Qbrexza The objective of this study was to compare the pharmacokinetics and safety of GT to oral glycopyrrolate (phase I study) and assess the relationship between glycopyrronium pharmacokinetics and anticholinergic-related adverse events or efficacy with population pharmacokinetics using data from two phase II studies. In the phase I study, study staff applied GT to axillae of patients with primary axillary hyperhidrosis (aged 9-65 years) once daily (5 days); oral glycopyrrolate was administered to healthy adults (aged 18-65 years) every 8 hours (15 days). In the phase II studies (NCT02016885 [20 December, 2013], NCT02129660 [2 May, 2014]), adults with primary axillary hyperhidrosis applied topical glycopyrronium (0.8-3.2\%) or vehicle to axillae once daily (4 weeks). Pharmacokinetic and adverse event data were collected in all studies. Glycopyrronium pharmacokinetic parameters were similar between adult and pediatric patients treated with GT; there was no evidence of accumulation. Systemic absorption of glycopyrronium was lower with GT vs oral glycopyrrolate. No anticholinergic-related adverse events occurred with GT in the phase I study, while dry mouth and nasal dryness occurred with oral glycopyrrolate; anticholinergic adverse events occurred in the phase II studies. In the population pharmacokinetic analysis, frequency/severity of anticholinergic-related adverse events increased with higher glycopyrronium concentration; no relationship was observed between efficacy and pharmacokinetic measures. These studies indicate limited absorption of GT compared to oral glycopyrrolate and a low risk of anticholinergic adverse events with proper GT administration when following instructions for use (wipe each underarm once with same cloth, wash hands, avoid ocular contact). [\hyperlink{Glycopyrrolate}{PMID: 33433785}, David M Pariser et al., 2021]

\hypertarget{pmid_8790123}{T}o describe the use of glycopyrrolate in the control of drooling in children and young adults with cerebral palsy and related neurodevelopmental disabilities. Prospective, open-label study of drug dosage parameters, response to therapy, and side effects. Follow-up ranged from 8 months to 4 years. Outpatient clinic of a rehabilitation hospital that is a regional referral center for children with disabilities. Forty children and young adults with motor and/or cognitive disabilities who were experiencing drooling to a severe degree. Treatment with oral glycopyrrolate. Change in the quantity of drooling and side effects associated with treatment. Thirty-six patients (90\%) had reduced drooling in response to medication; 2 (5\%) could not be assessed and 2 (5\%) received no benefit. Side effects resulted in discontinuation of treatment in 11 (28\%). Overall, 26 (65\%) continued to receive drug therapy because of the perceived benefit. The final effective dose ranged widely from 0.01 to 0.82 mg/kg per day. Glycopyrrolate therapy safely and effectively decreased but rarely abolished drooling in patients with cerebral palsy and related neurodevelopmental disabilities. The dose range was surprisingly broad. Side effects, although generally minor and predictable, often led to discontinuation of drug therapy. [\hyperlink{Glycopyrrolate}{PMID: 8790123}, P A Blasco et al., 1996]

\hypertarget{pmid_8720721}{G}lycopyrrolate, an anticholinergic agent that does not cross the blood-brain barrier, has several indications, but its mydriatic effect has never been tested. This study was carried out in order to compare the mydriatic effect of glycopyrrolate 0.5\% to that of atropine sulfate 1\%. Glycopyrrolate 0.5\% and atropine 1.0\% were instilled separately in the eyes of albino rabbits. Pupil diameter and intra-ocular pressure were monitored. Mydriasis was noted within 5 min of glycopyrrolate instillation, reached near-maximal level at 15 min and persisted for 1 week. Glycopyrrolate 0.5\% showed a faster, stronger and more persistent mydriatic effect than atropine 1.0\%. Administration of glycopyrrolate 0.5\% solution b.i.d. for 1 week did not affect intra-ocular pressure or produce any adverse reaction. Glycopyrrolate solution has the potential to deliver an ocular anticholinergic effect without causing associated central anticholinergic hazards. [\hyperlink{Glycopyrrolate}{PMID: 8720721}, D Varssano et al., 1996]

\hypertarget{pmid_10365009}{W}e have tested the hypotheses that glycopyrrolate, administered immediately before induction of subarachnoid anaesthesia for elective Caesarean section, reduces the incidence and severity of nausea, with no adverse effects on neonatal Apgar scores, in a double-blind, randomized, controlled study. Fifty women received either glycopyrrolate 200 micrograms or saline (placebo) i.v. during fluid preload, before induction of spinal anaesthesia with 2.5 ml of 0.5\% isobaric bupivacaine. Patients were questioned directly regarding nausea at 3-min intervals throughout operation and asked to report symptoms as they arose. The severity of nausea was assessed using a verbal scoring system and was treated with increments of i.v. ephedrine and fluids. Patients in the group pretreated with glycopyrrolate reported a reduction in the frequency (P = 0.02) and severity (P = 0.03) of nausea. Glycopyrrolate also reduced the severity of hypotension, as evidenced by reduced ephedrine requirements (P = 0.02). There were no differences in neonatal Apgar scores between groups. [\hyperlink{Glycopyrrolate}{PMID: 10365009}, D Ure et al., 1999]

\hypertarget{pmid_22405644}{P}rimary focal hyperhidrosis not uncommonly begins during the first two decades of life, and can have a profound effect on quality of life. Few treatment options have been studied in children. We sought to evaluate the response to oral glycopyrrolate in pediatric patients. Records of pediatric patients with hyperhidrosis seen at a pediatric hospital in a 10-year period were reviewed retrospectively and, if possible, parents and patients were also interviewed. The efficacy and adverse effects of oral glycopyrrolate were assessed. In all, 31 children took at least one dose of oral glycopyrrolate. All had daily hyperhidrosis that affected their quality of life and were resistant or intolerant of aluminum salts. The mean age of hyperhidrosis onset was 10.3 years, and mean age of initiation of glycopyrrolate was 14.8 years. At a mean dosage of 2 mg daily, 90\% of patients experienced improvement, which was major in 71\% of responders. Improvement occurred within hours of administration and disappeared within a day of discontinuation. Duration of treatment averaged 2.1 years (range to 10 years). Side effects were noted by 29\% of children, most commonly dry mouth (26\%) and eyes (10\%), and were dose-related. One patient developed blurred vision, which resolved with dosing below 5 mg/d; one patient experienced palpitations and discontinued the medication. This was a retrospective analysis of a limited number of pediatric patients. Oral glycopyrrolate is a cost-effective, painless second-line therapy for children and adolescents with primary focal hyperhidrosis that impacts their quality of life. [\hyperlink{Glycopyrrolate}{PMID: 22405644}, Amy S Paller et al., 2012]

\hypertarget{pmid_27354782}{T}he purpose of this study was to confirm the efficacy and safety of twice-daily glycopyrrolate 15.6 µg, a long-acting muscarinic antagonist, in patients with stable, symptomatic, chronic obstructive pulmonary disease (COPD) with moderate-to-severe airflow limitation. The GEM1 study was a 12-week, multicenter, double-blind, parallel-group, placebo-controlled study that randomized patients with stable, symptomatic COPD with moderate-to-severe airflow limitation to twice-daily glycopyrrolate 15.6 µg or placebo (1:1) via the Neohaler(®) device. The primary objective was to demonstrate superiority of glycopyrrolate versus placebo in terms of forced expiratory volume in 1 second area under the curve between 0 and 12 hours post morning dose at week 12. Other outcomes included additional spirometric end points, transition dyspnea index, St George's Respiratory Questionnaire, COPD Assessment Test, rescue medication use, and symptoms reported by patients via electronic diary. Safety was also assessed during the study. Of the 441 patients randomized (glycopyrrolate, n=222; placebo, n=219), 96\% of patients completed the planned treatment phase. Glycopyrrolate demonstrated statistically significant (P<0.001) improvements in lung function versus placebo. Glycopyrrolate showed statistically significant improvement in the transition dyspnea index focal score, St George's Respiratory Questionnaire total score, COPD Assessment Test score, rescue medication use, and daily total symptom score versus placebo at week 12. Safety was comparable between the treatment groups. Significant improvement in lung function, dyspnea, COPD symptoms, health status, and rescue medication use suggests that glycopyrrolate is a safe and effective treatment option as maintenance bronchodilator in patients with stable, symptomatic COPD with moderate-to-severe airflow limitation. [\hyperlink{Glycopyrrolate}{PMID: 27354782}, Craig LaForce et al., 2016]

\hypertarget{pmid_7137551}{A}tropine 15 micrograms/kg and glycopyrrolate 5 or 10 microgram/kg were studied as anticholinergic premedicants in groups of 20 children each. A control group of 20 children did not receive anticholinergic premedication. Both atropine and the higher dose of glycopyrrolate produced significant increases in heart rate prior to induction of anaesthesia. The subsequent increase during the process of induction was less than in those who had not received an anticholinergic drug or glycopyrrolate 5 micrograms/kg. Dysrhythmias during induction of anaesthesia occurred slightly less frequently in the patients given atropine or the higher dose of glycopyrrolate. Although the incidence was similar in these two groups, ventricular ectopic beats occurred less frequently following the use of glycopyrrolate. The control of secretions was also superior with this anticholinergic premedicant. [\hyperlink{Glycopyrrolate}{PMID: 7137551}, R K Mirakhur et al., 1982]

\hypertarget{pmid_6871778}{T}o determine whether intravenous atropine and glycopyrrolate are equally effective in preventing succinylcholine-induced heart rate changes, we studied the heart rate during the first 78 seconds of anaesthesia in 40 children anaesthetized with either thiopentone, atropine (0.02 mg X kg-1) and succinylcholine (2 mg X kg-1), or thiopentone, glycopyrrolate (0.01 mg X kg-1) and succinylcholine (2 mg X kg-1). Each treatment group was divided into four subgroups which differed only in the interval (6, 10, 15, 20 seconds) between injection of atropine or glycopyrrolate and succinylcholine. During the 54 seconds after succinylcholine, the mean heart rate of each subgroup decreased transiently and then returned to the pre-induction heart rate or higher. There was no difference in either the magnitude or the duration of the decrease in heart rate or the subsequent increase in heart rate between respective subgroups. Bradycardia occurred in only two patients, both of whom received glycopyrrolate. We conclude that atropine (0.02 mg X kg-1) and glycopyrrolate (0.01 mg X kg-1) are equally effective in attenuating succinylcholine-induced changes in heart rate in children. [\hyperlink{Glycopyrrolate}{PMID: 6871778}, J Lerman et al., 1983]

\hypertarget{pmid_7126399}{T}he frequency and nature of the oculocardiac reflex and its prevention by atropine or glycopyrrolate (i.m. and i.v.) has been studied in 160 children undergoing surgery for the correction of squint. Ninety per cent of those given no anticholinergic premedication exhibited the reflex. This was decreased to about 50\% in those receiving the drugs i.m. Glycopyrrolate 7.5 micrograms kg-1 and atropine 15 micrograms kg-1 i.v. were effective in most instances, the latter being slightly better. However, glycopyrrolate was associated with tachycardia of smaller magnitude. The reflex was observed more often following traction on the medial rectus muscle. [\hyperlink{Glycopyrrolate}{PMID: 7126399}, R K Mirakhur et al., 1982]

\hypertarget{pmid_20385892}{S}ialorrhea affects approximately 75\% of patients with Parkinson disease (PD). Sialorrhea is often treated with anticholinergics, but central side effects limit their usefulness. Glycopyrrolate (glycopyrronium bromide) is an anticholinergic drug with a quaternary ammonium structure not able to cross the blood-brain barrier in considerable amounts. Therefore, glycopyrrolate exhibits minimal central side effects, which may be an advantage in patients with PD, of whom a significant portion already experience cognitive deficits. To determine the efficacy and safety of glycopyrrolate in the treatment of sialorrhea in patients with PD. We conducted a 4-week, randomized, double-blind, placebo-controlled, crossover trial with oral glycopyrrolate 1 mg 3 times daily in 23 patients with PD. The severity of the sialorrhea was scored on a daily basis by the patients or a caregiver with a sialorrhea scoring scale ranging from 1 (no sialorrhea) to 9 (profuse sialorrhea). The mean (SD) sialorrhea score improved from 4.6 (1.7) with placebo to 3.8 (1.6) with glycopyrrolate (p = 0.011). Nine patients (39.1\%) with glycopyrrolate had a clinically relevant improvement of at least 30\% vs 1 patient (4.3\%) with placebo (p = 0.021). There were no significant differences in adverse events between glycopyrrolate and placebo treatment. Oral glycopyrrolate 1 mg 3 times daily is an effective and safe therapy for sialorrhea in Parkinson disease. This study provides Class I evidence that glycopyrrolate 1 mg 3 times daily is more effective than placebo in reducing sialorrhea in patients with Parkinson disease during a 4-week study. [\hyperlink{Glycopyrrolate}{PMID: 20385892}, M E L Arbouw et al., 2010]

\hypertarget{pmid_25869368}{B}reath-holding spells are a common childhood disorder that typically present before 12 months of age. Whereas most cases are benign, some patients have very severe cases associated with bradycardia that can progress from asystole to syncope and seizures. Treatment studies have implicated the use of several therapies, such as oral iron, fluoxetine, and pacemaker implantation. This is a retrospective study of patients treated with glycopyrrolate for pallid breath-holding spells. Clinical data from 4 patients referred to pediatric cardiology who saw therapeutic benefit from treatment using glycopyrrolate were reviewed to evaluate for clinical response to the drug. Two twin patients, whose symptoms began at 5 months of age, experienced a decrease in breath-holding frequency after 1 month. A patient diagnosed at 7 months of age experienced a decrease in frequency of spells. A patient diagnosed at 10 months of age reported cessation of syncope shortly after initiation of glycopyrrolate and complete resolution of breath-holding spells during prolonged treatment. This case study of 4 patients with pallid breath-holding offers evidence that glycopyrrolate may be beneficial in treating breath-holding spells and has a safer side-effect profile than pacemaker implantation.  [\hyperlink{Glycopyrrolate}{PMID: 25869368}, Jason Williams et al., 2015] Remifentanil is recommended for use in procedures with painful intraoperative stimuli but minimal postoperative pain. However, bradycardia and hypotension are known side-effects. We evaluated haemodynamic effects of i.v. glycopyrrolate during remifentanil-sevoflurane anaesthesia for cardiac catheterization of children with congenital heart disease. Forty-five children undergoing general anaesthesia with remifentanil and sevoflurane were randomly allocated to receive either saline, glycopyrrolate 6 microg kg(-1) or glycopyrrolate 12 microg kg(-1). After induction of anaesthesia with sevoflurane, i.v. placebo or glycopyrrolate was administered. An infusion of remifentanil at the rate of 0.15 microg kg(-1) min(-1) was started, sevoflurane continued at 0.6 MAC and cisatracurium 0.2 mg kg(-1) was given. Heart rate (HR) and non-invasive arterial pressures were monitored and noted every minute for the first 10 min and then every 2.5 min for subsequent maximum of 45 min. Baseline HR [mean (SD)] of 117 (20) beats min(-1) decreased significantly from 12.5 min onwards after starting the remifentanil infusion in the control group [106 (18) at 12.5 min and 99 (16) beats min(-1) at 45 min]. In the groups receiving glycopyrrolate, no significant decrease in HR was noticed. Glycopyrrolate at 12 microg kg(-1) induced tachycardia between 5 and 9 min after administration. Systolic and diastolic arterial pressures decreased gradually, but there were no significant differences in the pressures between groups. I.V. glycopyrrolate 6 microg kg(-1) prevents bradycardia during general anaesthesia with remifentanil and sevoflurane for cardiac catheterization in children with congenital heart disease. Administering 12 microg kg(-1) of glycopyrrolate temporarily induces tachycardia and offers no additional advantage. [\hyperlink{Glycopyrrolate}{PMID: 25869368}, K Reyntjens et al., 2005]

\section*{Granisetron Hydrochloride}
\subsection*{Result}
\subsubsection*{Answer}

Ages <1 year: Unknown
Ages 1–18 years: Yes

\subsubsection*{{Explanation}}
\hypertarget{Granisetron Hydrochloride}
A review of the available abstracts reveals multiple targeted studies evaluating the safety of Granisetron Hydrochloride in children across various age ranges and clinical scenarios. Below is a summary of the evidence by age group:

Ages 1–16 years:
- A multicenter, double-blind, dose-ranging study in 294 children aged 1–16 years receiving chemotherapy found both 20 and 40 microg/kg oral granisetron to be effective and safe, with no significant safety concerns reported [\hyperlink{pmid_10779814}{PMID: 10779814}, M Mabro et al., 2000].
- Another study in children aged 1–23 years (median 7.7 years) receiving chemotherapy found both 10 and 40 microg/kg IV granisetron to be well tolerated, with no significant adverse events [\hyperlink{pmid_17372773}{PMID: 17372773}, Su G Berrak et al., 2007].

Ages 1–18 years:
- A study of 30 children aged 3–18 years receiving chemotherapy reported no serious adverse events with IV granisetron (20 microg/kg/dose) [\hyperlink{pmid_8037341}{PMID: 8037341}, S J Jacobson et al., 1994].
- A study in 88 children aged 2–17 years receiving ifosfamide therapy found granisetron (20 microg/kg IV) to be superior to chlorpromazine-dexamethasone, with fewer adverse effects and no extrapyramidal reactions [\hyperlink{pmid_7844684}{PMID: 7844684}, K Hählen et al., 1995].

Ages 2–5 years:
- In a randomized trial of 80 children aged 2–5 years undergoing surgery, IV granisetron (10 microg/kg) was effective in preventing shivering, with no significant adverse events [\hyperlink{pmid_22313076}{PMID: 22313076}, Ahmed A Eldaba et al., 2012].

Ages 4–10 years:
- Multiple randomized, double-blind, placebo-controlled studies in children aged 4–10 years undergoing surgery (including tonsillectomy, strabismus repair, and hernia repair) found oral and IV granisetron at doses of 20–80 microg/kg to be effective and safe, with no clinically important or serious adverse events reported [\hyperlink{pmid_9861127}{PMID: 9861127}, Y Fujii et al., 1998; \hyperlink{pmid_10084103}{PMID: 10084103}, Y Fujii et al., 1999; \hyperlink{pmid_10434165}{PMID: 10434165}, Y Fujii et al., 1999; \hyperlink{pmid_8807169}{PMID: 8807169}, Y Fujii et al., 1996; \hyperlink{pmid_11903942}{PMID: 11903942}, Yoshitaka Fujii et al., 2002; \hyperlink{pmid_11210871}{PMID: 11210871}, Y Fujii et al., 2001].
- A study in 54 children (age not specified, but described as pediatric tonsillectomy patients) found IV granisetron (10 microg/kg) reduced vomiting with no safety concerns [\hyperlink{pmid_9233110}{PMID: 9233110}, D Carnahan et al., 1997].

Ages 1–18 years (cancer patients):
- Several studies in children with cancer (ages 1–18) receiving chemotherapy found granisetron (10–40 microg/kg, oral or IV) to be effective and well tolerated, with no serious adverse events [\hyperlink{pmid_7752999}{PMID: 7752999}, A W Craft et al., 1995; \hyperlink{pmid_10478183}{PMID: 10478183}, T Sawada et al., 1999; \hyperlink{pmid_10087313}{PMID: 10087313}, Y Tsuchida et al., 1999; \hyperlink{pmid_10533454}{PMID: 10533454}, Y Komada et al., 1999; \hyperlink{pmid_8037342}{PMID: 8037342}, Y Miyajima et al., 1994].

Ages 1 year (case report):
- A case report of a 1-year-old girl with neuroblastoma who tolerated granisetron after ondansetron anaphylaxis supports its safety in this individual case [\hyperlink{pmid_20921907}{PMID: 20921907}, Hacı Ahmet Demir et al., 2010].

Other findings:
- One study in 22 children found that IV granisetron (40 microg/kg) caused transient, asymptomatic changes in ECG (QT and QTc dispersion), but no clinical adverse events [\hyperlink{pmid_15803017}{PMID: 15803017}, Mustafa Buyukavci et al., 2005].
- No studies reported serious or life-threatening adverse events attributable to granisetron in children.

Summary:
There is robust evidence from multiple targeted studies affirming the safety of Granisetron Hydrochloride in children aged 1–18 years, including those as young as 1 year, for both chemotherapy-induced and postoperative nausea and vomiting. The studies consistently report no serious adverse events, and granisetron is generally well tolerated. For children under 1 year, no targeted safety data are available in the abstracts reviewed.

\subsection*{Abstracts}
\hypertarget{pmid_10779814}{T}his multicentric double-blind, dose-ranging study was to compare efficacy and safety of two oral doses of granisetron solution in the prevention of chemotherapy-induced emesis in children with malignant diseases : 294 children, aged 1 to 16, treated with a moderately or highly emetogenic chemotherapy were randomly assigned to receive oral granisetron either 20 microg/kg (n = 143) or 40 microg/kg (n = 151) before and 6 to 12 hours after the start of chemotherapy. Fifty-one percent of patients treated with 20 microg/kg bd of oral granisetron solution achieved a complete response (no vomiting, no worse than mild nausea, no rescue therapy and no withdrawal during the specified period) and 59\% achieved a major response (no more than one episode of vomiting, no worse than mild nausea, no rescue therapy and no withdrawal during the specified period). There was no difference between the two oral doses of granisetron. Treatment was rated as good or very good by investigators in 70\% of cases. In conclusion, oral granisetron suspension either at 20 microg/kg bd or at 40 microg/kg bd showed good efficacy and safety in the prevention of chemotherapy-induced emesis in children with malignant diseases. Oral granisetron solution can be used as prophylaxis of emesis in children receiving moderately or highly emetogenic chemotherapy. [\hyperlink{Granisetron Hydrochloride}{PMID: 10779814}, M Mabro et al., 2000]

\hypertarget{pmid_7752999}{T}he safety and efficacy of the new 5HT-3 antagonist granisetron as an antiemetic in children with cancer was evaluated in 40 children at a single dose of 40 micrograms/kg. No adverse affects attributable to the granisetron were noted. The overall major and complete response rate was 82.5\% and this was highest in the younger children. Only 2 patients showed no response. Pharmacokinetic studies showed associations between some pharmacokinetic parameters and age which were no longer apparent after normalisation for body weight. Granisetron is an effective and very well-tolerated antiemetic and appears to be an important addition to the supportive care available for children with cancer. [\hyperlink{Granisetron Hydrochloride}{PMID: 7752999}, A W Craft et al., 1995]

\hypertarget{pmid_8037341}{T}his study was undertaken to evaluate the safety and efficacy of granisetron (a 5-hydroxytryptamine. antagonist) in children with malignant disease who had previously experienced unacceptable nausea and vomiting and/or adverse effects associated with standard antiemetic therapy. Thirty children 3-18 years of age who were receiving anticancer chemotherapy were enrolled in the study. Patients received a prophylactic dose of granisetron before chemotherapy and two subsequent doses as needed. If further antiemetics were required, standard therapy was given and those patients were classified as treatment failures. Patients received granisetron during one to three cycles of chemotherapy; a total of 66 courses were given. Eighty-seven percent of patients had good control of nausea and vomiting with granisetron alone; 90\% of patients elected to receive granisetron with subsequent chemotherapy. No loss of efficacy was noted with repeated cycles in 21 patients. No serious adverse events occurred. Intravenous granisetron (20 micrograms/kg/dose) appears to be a safe and effective drug for pediatric patients receiving emetogenic chemotherapy. [\hyperlink{Granisetron Hydrochloride}{PMID: 8037341}, S J Jacobson et al., 1994]

\hypertarget{pmid_20921907}{A} 1-year-old girl with stage-IV neuroblastoma developed ondansetron hydrochloride anaphylaxis. Safe use of granisetron as an antiemetic after skin prick, oral, and intravenous challenge tests is presented. We present this case to emphasize that ondansetron hydrochloride may cause a serious anaphylactic reaction. In such a case, granisetron may be given to patients as an antiemetic after some provocative tests performed. [\hyperlink{Granisetron Hydrochloride}{PMID: 20921907}, Hacı Ahmet Demir et al., 2010]

\hypertarget{pmid_10087313}{T}he efficacy of granisetron hydrochloride 20 microg/kg and 40 microg/kg were compared using a cross-over method to determine the optimal dose in children with solid tumors receiving high-dose chemotherapy. Granisetron controlled the onset of vomiting in 17 of 23 patients (73.9\%) who were given 40 microg/kg of granisetron, while 8 of 21 patients (38.1\%) were free of vomiting in the 20 microg/kg group. The average frequency of vomiting was 7.22 times in the 20 microg/kg dose versus 4.44 times in the 40 microg/kg dose. No safety problems were associated with either dose. The 40 microg/kg dose of granisetron appears to be more optimal. [\hyperlink{Granisetron Hydrochloride}{PMID: 10087313}, Y Tsuchida et al., 1999]

\hypertarget{pmid_22313076}{T}his study evaluates the effect of prophylactic granisetron on the incidence of postoperative shivering after spinal anaesthesia in children. Eighty children, American Society of Anesthesiologists physical status I to II and aged two to five years were scheduled for surgery of the lower limb under spinal anaesthesia. The children were randomised to receive 10 µg/kg granisetron diluted in 10 ml saline 0.9\% intravenously (group 1, n=40) or placebo (10 ml 0.9\% saline, group 2, n=40) to be given over five minutes just before spinal puncture. Shivering, core temperature and the levels of motor and sensory block were assessed. No patients shivered in group 1. However, six patients shivered in Group 2 (P=0.025). There were no significant differences in the other measured variables between the groups. Granisetron is an effective agent to prevent shivering after spinal anaesthesia in children from two to five years of age. [\hyperlink{Granisetron Hydrochloride}{PMID: 22313076}, Ahmed A Eldaba et al., 2012]

\hypertarget{pmid_9861127}{W}e have studied the efficacy of granisetron, a selective 5-hydroxytryptamine type 3 receptor antagonist, administered orally for the prevention of postoperative vomiting after tonsillectomy in children. In a randomized, double-blind, placebo-controlled study, 160 paediatric patients, ASA 1, aged 4-10 yr, received placebo or granisetron (20, 40 or 80 micrograms kg-1) (n = 40 each) orally, 1 h before surgery. A standard general anaesthetic technique was used throughout. A complete response, defined as no emesis and no need for another rescue antiemetic during the first 24 h after anaesthesia, occurred in 40\%, 48\%, 85\% and 90\% of patients who had received placebo, or granisetron 20, 40 or 80 micrograms kg-1, respectively (P < 0.05; overall Fisher's exact probability test). There were no clinically important adverse events. We conclude that preoperative oral granisetron, in doses more than 40 micrograms kg-1, was effective for the prevention of postoperative vomiting in children. [\hyperlink{Granisetron Hydrochloride}{PMID: 9861127}, Y Fujii et al., 1998]

\hypertarget{pmid_10478183}{G}ranisetron has been used widely for the prevention and treatment of nausea and vomiting associated with anticancer drugs in adult patients with cancer. This multi-center open study was conducted to study the efficacy, safety and usefulness of granisetron in children with cancer. Among 166 evaluable patients, the efficacy rate (percentage of "remarkably effective" or "effective") was 84.9\% and the usefulness rate (percentage of "extremely useful" or "useful") was 87.3\%. No serious adverse effects were observed. As granisetron 40 micrograms/kg had an excellent antiemetic effect and a high degree of safety against nausea and vomiting associated with anticancer drugs, it was shown to be useful for children with cancer. [\hyperlink{Granisetron Hydrochloride}{PMID: 10478183}, T Sawada et al., 1999]

\hypertarget{pmid_17372773}{G}ranisetron is a safe and effective prophylaxis for nausea and vomiting associated with moderate to highly emetogenic chemotherapy. Few trials have been conducted to determine the optimal effective dose of granisetron in children with cancer. The objective of this report was to compare two doses of granisetron in patients with optic pathway tumors receiving moderately emetogenic doses of carboplatin. In this double-blind, crossover, randomized study, antiemetic efficacy and tolerability of two dose levels (10 and 40 microg/kg) of granisetron in the prevention of acute and delayed nausea/emesis were compared in children and young adults. A total of 18 patients (13 boys) aged 1-23 years (median 7.7 years) treated with a moderately emetogenic dose of carboplatin were randomly assigned to receive either 10 or 40 microg/kg of slow granisetron intravenous (i.v.) infusions at alternating cycles of chemotherapy in a blinded fashion until the end of the study period or until their chemotherapy regimen ended. In this way, the patients acted as their own controls. Patients in the granisetron 10 and 40 microg/kg groups received 104 and 121 cycles of chemotherapy, respectively. There was no significant difference in antiemetic efficacy in terms of nausea and emesis between the dose groups in the first 5 days of chemotherapy. The treatment was well tolerated. We conclude that granisetron 10 and 40 microg/kg have comparable efficacy in controlling carboplatin-induced acute and delayed nausea/emesis and is well tolerated in children and young adults. [\hyperlink{Granisetron Hydrochloride}{PMID: 17372773}, Su G Berrak et al., 2007]

\hypertarget{pmid_11210871}{G}ranisetron, a selective 5-hydroxytryptamine type 3 receptor antagonist, is effective for the prevention of vomiting after tonsillectomy in children. Ramosetron (Nasea; Yamanouchi; Tokyo, Japan), another new antagonist of 5-hydroxytryptamione type 3 receptor, has more potent and longer-acting properties than granisetron (Kytril; Smith Kline Beecham, London, UK) against cisplatin-induced emesis. This study was undertaken to compare the efficacy and safety of granisetron and ramosetron for the prevention of vomiting after pediatric tonsillectomy. Prospective, randomized, double-blinded study. Ninety pediatric patients, aged 4 to 10 years, received intravenously granisetron 40 microg/kg or ramosetron 6 microg/kg (n = 45 each) at the end of surgery. The same standard general anesthetic technique and postoperative analgesia were used throughout. Emetic episodes and safety assessment were performed during the first 24-hour period and the next 24-hour period after anesthesia. The rates of patients being emesis-free during the period from 0 to 24 hours after anesthesia were 89\% with granisetron and 93\% with ramosetron, respectively (P = .357); the corresponding rates during the period from 24 to 48 hours after anesthesia were 71\% and 93\%, respectively (P = .006). No clinically serious adverse events attributable to the study drugs were observed in any of the groups. Ramosetron is a better antiemetic than granisetron for the long-term prevention of postoperative vomiting in children undergoing general anesthesia for tonsillectomy. [\hyperlink{Granisetron Hydrochloride}{PMID: 11210871}, Y Fujii et al., 2001]

\hypertarget{pmid_8037342}{I}n a prospective crossover study, we evaluated the safety and antiemetic activity of granisetron, a 5-hydroxytryptamine3 (5-HT3) receptor antagonist, compared with conventional antiemetics regimen, including metoclopramide, in pediatric cancer patients. Twenty-two children with malignant diseases were enrolled. The chemotherapy included cytarabine 3 g/m2 (regimen A), cisplatin 90 mg/m2 (regimen B), and actinomycin D 900 micrograms/m2 plus ifosfamide 3 g/m2 (regimen C). Granisetron 40 micrograms/kg was infused over 30 min just before each chemotherapy treatment. A complete response was obtained more often with granisetron than with conventional antiemetics (59.1\% vs. 0\%, p < 0.001). In terms of efficacy by chemotherapy type, complete response with granisetron was obtained in eight of 10 patients with regimen A, three of eight with regimen B, and two of four with regimen C. Major efficacy (vomiting fewer than two times) was also obtained more with granisetron than with conventional antiemetics (81.8\% vs. 4.6\%, p < 0.001). The number of vomiting episodes in the first 24 h was less with granisetron than with conventional antiemetics (1.1 +/- 1.46 vs. 9.0 +/- 4.97, p < 0.001). Normal appetite and activity were retained in more patients with granisetron than with conventional antiemetics. Extrapyramidal reactions, akathisia, and sedation were not seen in any case with granisetron. Granisetron 40 micrograms/kg is well tolerated and more effective than are conventional antiemetic regimens containing metoclopramide for children receiving cancer chemotherapy. [\hyperlink{Granisetron Hydrochloride}{PMID: 8037342}, Y Miyajima et al., 1994]

\hypertarget{pmid_10925688}{W}e investigated the antiemetic effect, safety and usefulness of granisetron hydrochloride tablets on nausea and vomiting induced by oral anticancer drugs used in chemotherapy for gastric cancer and colorectal cancer. In the present trial, oral administration of granisetron hydrochloride was performed during 5 days after nausea or vomiting. 1) Clinically, the effective rate of granisetron hydrochloride (the percentage of cases in which the drug was assessed as "Remarkably effective" or "Effective") was more than 75\% on each day of administration. There were no adverse events or abnormal laboratory tests. 2) In terms of usefulness, granisetron hydrochloride was rated "Extremely useful" or "Useful" in 17 out of 23 cases (78.2\%). The above results have shown that granisetron hydrochloride tablets, administrated orally once daily at a dose of 2 mg, have an excellent antiemetic effect, and that this is a safe and useful drug. [\hyperlink{Granisetron Hydrochloride}{PMID: 10925688}, S Togo et al., 2000] 5-HT3 receptor antagonists, including granisetron and ondansetron, are widely used in the prophylactic treatment of chemotherapy-induced nausea and vomiting. Although the cardiac safety of granisetron and ondansetron has been investigated in several adult studies, there is no report investigating the effects of those agents on electrocardiography (ECG) in children. The effects of intravenously infused (over 30 seconds) 0.1 mg/kg ondansetron and 40 microg/kg granisetron on ECG were assessed in 22 children receiving high-dose methotrexate therapy for acute lymphoblastic leukemia. The ECG recording was obtained at before and just after the infusion, and repeated at 1, 3, 6, and 24 hours of treatment. Granisetron administration resulted in a statistically significant decrease of mean heart rate at 1 and 3 hours, and significant prolongation of mean QT and QTc dispersions at 1 hour of infusion. In patients treated with ondansetron, no meaningful change was observed. In conclusion, intravenous granisetron but not ondansetron causes clinically asymptomatic and transient changes on ECG measurements in children receiving high-dose methotrexate therapy. [\hyperlink{Granisetron Hydrochloride}{PMID: 10925688}, Mustafa Buyukavci et al., 2005]

\hypertarget{pmid_8546476}{W}e, in the Department of Obstetrics and Gynecology, Kansai Medical College, conducted an evaluation of the usefulness and safety of granisetron hydrochloride used for nausea and vomiting due to chemotherapy in patients with gynecological malignant tumors, with an additional study of the efficacy of different regimens. The subjects were 9 patients in whom 16 courses of CAP therapy were given (group A) and 13 patients in whom 24 courses of CAP therapy were given (group B). Granisetron hydrochloride 3 mg/body was administered by intravenous drip in the two groups before chemotherapy. Clinical symptoms of nausea, vomiting, and anorexia were observed for 2 days after anticancer drugs were administered in order to evaluate its efficacy. The percentage of patients who responded as "effective" or better was 90.0\%. In different regimens, the efficacy was 93.8\% in group A and 87.5\% in group B. These results indicated clinically high usefulness in both groups. No side effects related to granisetron hydrochloride were found in this study. [\hyperlink{Granisetron Hydrochloride}{PMID: 8546476}, M Kitada et al., 1996]

\hypertarget{pmid_10084103}{T}his study was undertaken to determine the minimum effective dose of granisetron, 5-hydroxytryptamine type 3 receptor antagonist, for the prevention of post-operative vomiting in children undergoing general inhalational anaesthesia for surgery (inguinal hernia and phimosis). In a randomized, double-blind manner, 120 children, ASA physical status I, aged 4-10 years, were assigned to receive placebo (saline) or granisetron at three different doses (20 micrograms kg-1, 40 micrograms kg-1, 100 micrograms kg-1) intravenously immediately after inhalation induction of anaesthesia (n = 30 of each). A complete response, defined as no emesis and no need for another rescue antiemetic during the first 24 h after anaesthesia, occurred in 57\% with placebo, 67\% with granisetron 20 micrograms kg-1, 90\% with granisetron 40 micrograms kg-1 and 90\% with granisetron 100 micrograms kg-1 respectively (P < 0.05; overall Fisher's exact probability test). No clinically important adverse events were observed in any of the groups. Our results suggest that granisetron 40 micrograms kg-1 is the minimum effective dose for the prevention of emesis after paediatric surgery, and that increasing its dose to 100 micrograms kg-1 provides no demonstrable benefit. [\hyperlink{Granisetron Hydrochloride}{PMID: 10084103}, Y Fujii et al., 1999]

\hypertarget{pmid_7844684}{T}o compare the efficacy and safety of intravenously administered granisetron with those of chlorpromazine plus dexamethosone in the prevention of ifosmamide-induced emesis in children with malignant disease. Eighty-eight children, aged 2 to 17 years, were scheduled for ifosfamide therapy (> or = 3 gm/m2) for 2 or 3 consecutive days. On each day, children received granisetron, 20 microgram/kg intravenously, before ifosfamide therapy, plus up to two more doses within 24 hours if required, or chlorpromazine, 0.3 to 0.5 mg/kg intravenously, every 4 to 6 hours, plus dexamethasone, 2 mg/m2 intravenously every 8 hours. During the initial 24 hours, significantly fewer episodes of vomiting were recorded after granisetron administration (median number, 1.5 vs 7.0; p = 0.001), and the percentages of children having no more than one vomiting episode (51\% granisetron vs 21\% chlorpromazine-dexamethasone) and no worse than mild nausea (67\% granisetron vs 38\% chlorpromazine-dexamethasone) were lower after granisetron therapy (p < 0.01). Fewer children had sedation with granisetron (2 vs 19; p < 0.001); there were no extrapyramidal reactions during granisetron therapy compared with two during control therapy. Granisetron was superior to chlorpromazine-dexamethasone antiemetic therapy for children receiving ifosfamide therapy and deserves further study during other chemotherapy regimens. [\hyperlink{Granisetron Hydrochloride}{PMID: 7844684}, K Hählen et al., 1995]

\hypertarget{pmid_10434165}{T}his study was undertaken to compare the efficacy and safety of granisetron, a 5-hydroxytryptamine type 3 receptor antagonist, and dexamethasone and each drug alone for the prevention of post-operative vomiting by children, with no history of motion sickness and/or previous post-operative vomiting, undergoing general inhalational anaesthesia for surgery (inguinal hernia and phimosis). In a randomized, double-blind manner, 150 children, ASA physical status 1, aged 4-10 years, were assigned to receive granisetron 40 mg kg-1, dexamethasone 150 mg kg-1, or granisetron 40 mg kg-1 plus dexamethasone 150 mg kg-1 intravenously immediately after inhalation induction of anaesthesia (n = 50 of each). A complete response, defined as no emesis and no need for another rescue anti-emetic during the first 24 h after anaesthesia, was 86\% with granisetron, 68\% with dexamethasone and 98\% with granisetron plus dexamethasone, respectively (P < 0.05; overall Fisher's exact probability test). No clinically serious adverse events were observed in any of the groups. In conclusion, prophylactic therapy with combined granisetron and dexamethasone was more effective than was each anti-emetic alone for the prevention of vomiting after paediatric surgery. [\hyperlink{Granisetron Hydrochloride}{PMID: 10434165}, Y Fujii et al., 1999]

\hypertarget{pmid_9512674}{T}he safety and efficacy of granisetron (10 micrograms/kg and 40 micrograms/kg) were evaluated during a second (n = 393) and third (n = 200) cycle of chemotherapy in this multicenter, double-blind, randomized, parallel-group study. Granisetron was administered as a single intravenous dose before the start of cisplatin chemotherapy (> or = 60 mg/m2). Total control (no vomiting, no retching, no nausea, and no use of antiemetic rescue medication) after the first 24 hr following chemotherapy was achieved in 40\% and 49\% of patients in Cycles 2 and 3, respectively, for the 10 micrograms/kg group, and in 42\% and 38\% of patients in Cycles 2 and 3, respectively, for the 40 micrograms/kg group. Both dose levels of granisetron were well tolerated. The results demonstrate comparable efficacy between the 10 micrograms/kg and 40 micrograms/kg doses of granisetron in preventing nausea and vomiting during repeat cycles of high-dose cisplatin-based chemotherapy. The results of this study show that granisetron 10 micrograms/kg is safe and well tolerated, and remains effective with repeat cycle use. [\hyperlink{Granisetron Hydrochloride}{PMID: 9512674}, H L Ritter et al., 1998]

\hypertarget{pmid_25813292}{T}he objective of this study was to elucidate the possible toxic effects on the fetal tissues after exposure to two clinically relevant concentrations of granisetron. Primary cells were isolated from human fetal organs of 16-19 weeks gestational age and treated with 3 ng/mL or 30 ng/mL of granisetron. Cell cycle progression was evaluated by flow cytometry. ELISA was used to detect alterations in major apoptotic proteins. Up to 10\% apoptosis in cardiac tissue was observed following treatment with 30 ng/mL granisetron. Neither concentration of granisetron caused alteration in cell cycle progression or alterations in apoptotic proteins in any of the other tissues. At 30 ng/mL granisetron concentration had the potential to induce up to 10\% apoptosis in cardiac tissue; clinical significance needs further evaluation. At granisetron 3 ng/mL there was no detectable toxicity or on any fetal tissue in this study. Further research is needed to confirm these preliminary findings and determine if clinically significant. [\hyperlink{Granisetron Hydrochloride}{PMID: 25813292}, Judith A Smith et al., 2015]

\hypertarget{pmid_10533454}{T}his randomised study was performed to assess the anti-emetic efficacy and tolerability of two-dose regimens of granisetron in children with leukaemia. 49 children with leukaemia were treated with three consecutive courses of high-dose methotrexate or cytarabine regimen. During the first course, patients were evaluated regarding the emetogenicity of each regimen. They were randomised in a crossover manner to receive 20 or 40 micrograms/kg of granisetron before the second and third course of chemotherapy. Neither emesis nor severe appetite loss were observed in over 80\% of patients within the first 24 h in all treatment groups. There was no significant difference in the anti-emetic efficacy between the two-dose regimens of granisetron. However, complete protection was achieved less frequently on days 2 and 3. Older children and girls appeared to be less well protected. No adverse events attributable to granisetron were observed. Granisetron dose regimens of 20 and 40 micrograms/kg are, comparably, well tolerated and effective in controlling chemotherapy-induced emesis in the first 24 h, though this protection fails thereafter, particularly in older patients and girls. [\hyperlink{Granisetron Hydrochloride}{PMID: 10533454}, Y Komada et al., 1999]

\hypertarget{pmid_7552896}{T}he stability and sterility of granisetron hydrochloride in 5\% dextrose injection or 0.9\% sodium chloride injection when stored in a disposable elastomeric infusion device were studied. Granisetron was diluted to 0.02 mg/mL (as the hydrochloride salt) in 5\% dextrose chloride injection. The solution was placed in the drug reservoir of a disposable elastomeric infusion device and refrigerated at 4 degrees C for 14 days. A total of eight pumps were prepared, four containing granisetron 0.02 mg/mL in 5\% dextrose injection and four containing granisetron 0.02 mg/mL in 0.9\% sodium chloride injection. The solutions were assayed for granisetron concentration by stability-indicating high-performance liquid chromatography at 0 hours, 24 hours, 48 hours, 7 days, and 14 days. Each solution was inspected for clarity, color, and precipitation, and sterility testing was performed. Throughout the study, the mean concentration of granisetron remaining was more than 92\% of the initial concentration both in 5\% dextrose injection and in 0.9\% sodium chloride injection. Individual solutions in 0.9\% sodium chloride injection consistently maintained more than 90\% of the initial drug concentration for only seven days. No microbial growth was detected. No precipitation, color change, or haziness was seen. Granisetron 0.02 mg/mL (as the hydrochloride salt) was stable and free of microbial growth in 0.9\% sodium chloride injection for up to 7 days and stable and free of microbial growth in 5\% dextrose injection for up to 14 days when stored at 4 degrees C in a disposable elastomeric infusion device. [\hyperlink{Granisetron Hydrochloride}{PMID: 7552896}, K C Chung et al., 1995]

\hypertarget{pmid_9117791}{T}he compatibility of granisetron hydrochloride with selected other drugs during simulated Y-site administration was studied. Five milliliters of granisetron 50 micrograms/mL (as the hydrochloride) in 5\% dextrose injection was combined with 5 mL of each of 91 secondary additives, including antineoplastics, anti-infectives, and supportive care drugs, in 5\% dextrose injection or (if necessary to avoid incompatibility with the diluent) 0.9\% sodium chloride injection. Visual examinations were performed with the unaided eye in fluorescent light and in high-intensity monodirectional light to enhance visualization of small particles and low-level turbidity. The turbidity of each solution was measured as well. Particle sizing and counting were performed for selected solutions. Evaluations were performed initially and at one and four hours. Nearly all the test drugs were compatible with granisetron during the four-hour observation period. The granisetron-amphotericin B combination had an unacceptable increase in turbidity upon being mixed. During simulated Y-site administration, granisetron 50 micrograms/mL (as the hydrochloride) in 5\% dextrose injection was compatible with 90 of 91 drugs and combination drugs for four hours at room temperature; the exception was amphotericin B. [\hyperlink{Granisetron Hydrochloride}{PMID: 9117791}, L A Trissel et al., 1997]

\hypertarget{pmid_8807169}{T}his study was to identify the minimum effective dose of granisetron, a selective 5-hydroxytryptamine type 3 receptor antagonist, to prevent postoperative vomiting in children who have undergone strabismus repair, tonsillectomy or tonsillectomy with adenoidectomy. In a randomized, double-blind fashion, 80 healthy children aged 4-10 yr were assigned to receive either placebo (saline) or granisetron in a dose of 20, 40 or 80 micrograms.kg-1 iv immediately following the induction of anaesthesia. All subjects received a standardized anaesthetic, which consisted of sevoflurane in nitrous oxide and oxygen. Rescue antiemetics were administered if two or more episodes of vomiting occurred. Postoperative pain was treated with acetaminophene pr or pentazocine iv. During the first 24 hr after anaesthesia, the frequencies of retching and vomiting were recorded in a standardized fashion by nursing staff while subjects were in a hospital. There were no differences among four treatment groups with regard to subject characteristics, surgical procedures, anaesthetic and postoperative management or adverse effects. The frequencies of these symptoms were as follows: 65\%, 60\%, 20\% and 15\% after administration of placebo, granisetron 20, 40 or 80 micrograms.kg-1. Three children who had received either placebo or granisetron 20 micrograms.kg-1 required another rescue antiemetic drug, whereas none who had received granisetron 40 or 80 micrograms.kg-1 needed rescue drugs. Granisetron 40 micrograms.kg-1 is an effective antiemetic for preventing retching and vomiting following strabismus repair and tonsillectomy in children. Increasing the dose to 80 micrograms.kg-1 provided no demonstrable benefit in reducing postoperative emesis. [\hyperlink{Granisetron Hydrochloride}{PMID: 8807169}, Y Fujii et al., 1996]

\hypertarget{pmid_11903942}{W}e evaluated the efficacy of granisetron, 5-hydroxytryptamine type 3 receptor antagonist, given orally, preoperatively, for the prevention of postoperative vomiting in children undergoing general anaesthesia for surgery (inguinal hernia, phimosis-circumcision). In a randomized, double-blinded manner, 100 children, ASA physical status I, aged 4-11 years, received orally placebo or granisetron at three different doses (20 microg x kg(-1), 40 microg x kg(-1), 80 microg x kg(-1)) 60 min before surgery (n=25 of each). The same standard general anaesthetic technique was used. The percentage of patients being emesis-free during 0-6 h after anaesthesia was 56\% with placebo, 64\% with graniseron 20 microg x kg(-1) (P=0.773), 88\% with granisetron 40 microg x kg(-1) (P=0.027) and 92\% with granisetron 80 microg x kg(-1) (P=0.01); the corresponding rate during 6-24 h after anaesthesia was 60\%, 68\% (P=0.768), 92\% (P=0.02) and 92\% (P=0.02) (P-values versus placebo). No clinically serious adverse events were observed in any of the groups. In summary, preoperative oral granisetron 40 microg x kg(-1) is effective for the prevention of vomiting following paediatric surgery (inguinal hernia, phimosis-circumcision). Increasing the doses to 80 microg x kg(-1) provides no demonstrable additional benefit. [\hyperlink{Granisetron Hydrochloride}{PMID: 11903942}, Yoshitaka Fujii et al., 2002]

\hypertarget{pmid_9233110}{A} high incidence of vomiting after tonsillectomy is consistently reported in the literature. Multiple trials with different pharmacological agents and alternative anesthetic techniques have been used in an attempt to decrease this, but results have been inconclusive and adverse effects occur frequently. The purpose of this study was to examine the safety and efficacy of granisetron (Kytril) in the prevention of postoperative vomiting in children undergoing tonsillectomy. Fifty-four patients were included in this study. The study population was divided into an experimental (n = 28) and control group (n = 26) using a randomized double-blinded technique. The experimental group was treated with granisetron 10 micrograms/kg intravenously (IV) and the control group received saline solution IV. Episodes of vomiting were recorded throughout the hospital stay and for the 24 hours after surgery. A reduction in the incidence and severity of vomiting was shown to be significant in the experimental group, both in the hospital and at 24 hours postoperatively. High patient satisfaction was also reported in this group. Clinically, the prophylactic administration of granisetron intraoperatively in pediatric tonsillectomy patients should be considered to reduce the incidence and severity of postoperative vomiting. [\hyperlink{Granisetron Hydrochloride}{PMID: 9233110}, D Carnahan et al., 1997]

\section*{Ifex}
\subsection*{Result}
\subsubsection*{Answer}

Unknown

\subsubsection*{{Explanation}}
\hypertarget{Ifex}
A review of the provided abstracts reveals that none of them are about Ifex (ifosfamide). The abstracts discuss the safety and efficacy of other medications in children, such as norfloxacin, flomoxef, cefixime, dexmedetomidine, infliximab, and ibuprofen, but there is no mention of Ifex or ifosfamide in any of the abstracts. Therefore, based on the available abstracts, there is not enough data to determine whether Ifex is safe or unsafe for use in children.

\subsection*{Abstracts}
\hypertarget{pmid_2391756}{W}e administered norfloxacin (NFLX) to 16 children aged 3 to 14 year-old at the dose of 5.2 to 17.2 mg/kg/day. We evaluated the efficacy and safety of NFLX in 6 children with respiratory tract infections, 8 urinary tract infections, and 2 gastrointestinal tract infections. Efficacy rate of NFLX was 93.8\% and eradicated rate was 92.9\%. Any adverse effects were not observed. These results suggested that NFLX could be used safely to the children. [\hyperlink{Ifex}{PMID: 2391756}, T Ihara et al., 1990]

\hypertarget{pmid_3430711}{F}lomoxef (FMOX, 6315-S), a new parenteral oxacephem antibiotic, was evaluated for its safety, efficacy and pharmacokinetics in children. Twenty-six patients with bacterial infections were treated with FMOX. Clinical efficacy rate was 92\% and bacteriological cure rate was 85\%. Three cases of infections due to methicillin-resistant Staphylococcus aureus were cured with FMOX therapy. No severe adverse reactions or abnormalities of laboratory test data were associated with FMOX therapy, although loose stools and diarrhea occurred frequently (23\%). Serum half-lives of FMOX after a single bolus injection of 9 infants and children were 0.77 +/- 0.31 hour and excretion into urine was rapid. From these experiences, FMOX appeared to be a safe and effective antibiotic when used in children with susceptible bacterial infections. [\hyperlink{Ifex}{PMID: 3430711}, H Meguro et al., 1987]

\hypertarget{pmid_32145737}{I}ntranasal dexmedetomidine (DEX), as a novel sedation method, has been used in many clinical examinations of infants and children. However, the safety and efficacy of this method for electroencephalography (EEG) in children is limited. In this study, we performed a large-scale clinical case analysis of patients who received this sedation method. The purpose of this study was to evaluate the safety and efficacy of intranasal DEX for sedation in children during EEG. This was a retrospective study. The inclusion criteria were children who underwent EEG from October 2016 to October 2018 at the Children's Hospital affiliated with Chongqing Medical University. All the children received 2.5 μg·kg A total of 3475 cases were collected and analysed in this study. The success rate of the initial dose was 87.0\% (3024/3475 cases), and the success rate of intranasal sedation rescue was 60.8\% (274/451 cases). The median sedation onset time was 19 mins (IQR: 17-22 min), and the sedation recovery time was 41 mins (IQR: 36-47 min). The total incidence of adverse events was 0.95\% (33/3475 cases), and no serious adverse events occurred. Intranasal DEX (2.5 μg·kg [\hyperlink{Ifex}{PMID: 32145737}, Hang Chen et al., 2020] We have evaluated the effectiveness and safety of norfloxacin (NFLX) in 18 children with infectious diseases. Doses ranging from 5.6 to 18.8 mg/kg/day for t.i.d. or q.i.d. were used. The causative bacteria were Campylobacter jejuni in 3 cases, Salmonella typhimurium in 1, Staphylococcus aureus in 2, Haemophilus influenzae in 1 and unknown in 11 cases. Except 1 strain of S. aureus, all the bacteria mentioned above were eradicated. Clinical effects were excellent in 8, good in 8, poor in 2, and the total efficacy rate was 88.9\%. No side effects nor abnormal laboratory test results were observed. These results have shown that the NFLX is a usefull drug for infectious diseases in the pediatric field. [\hyperlink{Ifex}{PMID: 32145737}, T Okada et al., 1990]

\hypertarget{pmid_33505889}{T}he long-term efficacy and safety of infliximab (IFX) in children with ulcerative colitis (UC) have not been well-evaluated. Here, we reviewed the long-term durability and safety of IFX in our single center pediatric cohort with UC. This retrospective study included 20 children with UC who were administered IFX. For induction, 5 mg/kg IFX was administered at weeks 0, 2, and 6, followed by every 8 weeks for maintenance. The dose and interval of IFX were adjusted depending on clinical decisions. Corticosteroid (CS)-free remission without dose escalation (DE) occurred in 30\% and 25\% of patients at weeks 30 and 54, respectively. Patients who achieved CS-free remission without DE at week 30 sustained long-term IFX treatment without colectomy. However, one-third of the patients discontinued IFX treatment because of a primary nonresponse, and one-third experienced secondary loss of response (sLOR). IFX durability was higher in patients administered IFX plus azathioprine for >6 months. Four of five patients with very early onset UC had a primary nonresponse. Infusion reactions (IRs) occurred in 10 patients, resulting in discontinuation of IFX in four of these patients. No severe opportunistic infections occurred, except in one patient who developed acute focal bacterial nephritis. Three patients developed psoriasis-like lesions. IFX is relatively safe and effective for children with UC. Clinical remission at week 30 was associated with long-term durability of colectomy-free IFX treatment. However, approximately two-thirds of the patients were unable to continue IFX therapy because of primary nonresponse, sLOR, IRs, and other side effects. [\hyperlink{Ifex}{PMID: 33505889}, Hirotaka Shimizu et al., 2021]

\hypertarget{pmid_3430716}{T}he new antibiotic flomoxef (FMOX, 6315-S) was administered to 38 children. The results obtained are summarized as follows. 1. In 3 cases of children administered with FMOX (20 mg/kg) by intravenous drip infusion for 30 minutes, the mean T1/2 (beta) was 0.96 hour and the mean 6-hour urinary excretion was 95.5\%. 2. The antibiotic was administered to a total of 38 patients with bronchopneumonia, lacunar tonsillitis, upper respiratory tract infection complicated with brain tumor, otitis media, urinary tract infection, purulent meningitis, subcutaneous and hyponychial abscess, cervical lymphadenitis, or bacterial enteritis. The treatment was markedly effective in 24 cases, effective in 13, fair in 1, and ineffective in none. The efficacy rate was 97.4\%. From our results, this drug appears to be particularly effective to bronchopneumonia, upper respiratory tract infection and urinary tract infection. 3. None of the children showed clinical symptoms indicating side effects of the drug. These results showed that FMOX is a drug that can be safely used in the pediatric field as well as for adults. [\hyperlink{Ifex}{PMID: 3430716}, M Ito et al., 1987]

\hypertarget{pmid_3761534}{A} new beta-lactamase-stable oral cephem antibiotic, cefixime (CFIX), was evaluated for safety, efficacy and pharmacokinetics in children. CFIX was effective in 19 of 20 cases (95\%) with bacterial infections. The drug was especially effective against the cases of pneumonia due to beta-lactamase-producing H. influenzae or B. catarrhalis. Pharmacokinetic parameters of CFIX (3 mg/kg) with premeal administration were as follows: Kel 0.328 +/- 0.066 hr-1, T 1/2 2.14 +/- 0.36 hrs, AUC 10.9 +/- 8.7 micrograms X hr/ml, and Vd/F 1.64 +/- 1.42 L/kg. In most of the cases tested, the urinary excretion rate in 12 hours was 5 to 17\%. A dose of 3 mg/kg twice daily seems to be adequate for a regular treatment. [\hyperlink{Ifex}{PMID: 3761534}, H Meguro et al., 1986]

\hypertarget{pmid_2391760}{W}e have evaluated norfloxacin (NFLX) tablets for therapeutic effectiveness and safety in children. The results are summarized as follows. 1. A clinical study was performed on 14 children with infections, including 12 with urinary tract infections and 2 with acute bronchitis. Doses ranging from 1.7 to 5.4 mg/kg body weight were given b.i.d. or t.i.d.. Lengths of treatment ranged from 3 to 15 days. The therapeutic responses were considered "excellent" in 8 and "good" in 5, with an efficacy rate of 93\%. 2. Side effects were observed in 2 cases, one with light-headed feeling and one with vomiting. In clinical laboratory tests, eosinophilia was found in 2 cases and GOT was slightly elevated in 1 case. It has been concluded that NFLX is a usable drug for the treatment of bacterial infections in children. [\hyperlink{Ifex}{PMID: 2391760}, H Morita et al., 1990]

\hypertarget{pmid_3430725}{A} new oxacephem antibiotic, flomoxef sodium (FMOX, 6315-S), was studied for its clinical efficacy in the field of pediatrics. The treated patients were infants and children ranging from 6 months to 14 years old suffering from bacterial pneumonia in 3 cases, acute tonsillitis in 2 cases, acute enterocolitis in 2 cases, and cellulitis and urinary tract infection in 1 case each, a total of 9 cases. FMOX was administered at (levels of) 57-150 mg/kg in daily dose with durations of treatment ranging from 5 to 18 days. Clinical efficacies of good or excellent results were obtained in all cases (excellent in 4, good in 5). As an adverse reaction, eosinophilia was observed in 1 patient. This elevation is, however, normalized with the cessation of the treatment. [\hyperlink{Ifex}{PMID: 3430725}, H Ogura et al., 1987]

\hypertarget{pmid_3430714}{P}harmacokinetical, bacteriological and clinical studies of flomoxef (FMOX, 6315-S), a new cephem antibiotic, were conducted in the pediatric field. 1. Mean drug concentrations in the blood after intravenous one shot injection of FMOX 20 mg/kg to 5 children (aged 5-8 years) were 39.7 micrograms/ml at 1/4 hour, 24.1 micrograms/ml at 1/2 hour, 12.2 micrograms/ml at 1 hour, 4.7 micrograms/ml at 2 hours, 1.1 microgram/ml at 4 hours, and 0.3 microgram/ml at 6 hours. The mean half-life in blood was 0.65 hour. Mean concentrations in urine was 3,558 micrograms/ml during 0-2 hours after intravenous injection, 568 micrograms/ml during 2-4 hours, and 117 micrograms/ml during 4-6 hours. The mean 6-hour urinary recovery rate was 72.8\%. 2. Clinically, FMOX was administered to 32 children (5 months to 9 years) with infections, i.e. 29 with pneumonia, 1 each with acute purulent tonsillitis, acute purulent lymphadenitis and cellulitis. The treatment was excellent in 24 cases and good in 8. Thus, the efficacy rate was 100\%. 3. The bacteriological effect of FMOX was investigated using 11 clinical isolates which were considered to be causative organisms, i.e. 1 strain of Staphylococcus aureus, 1 strain of Streptococcus pneumoniae, 8 strains of Haemophilus influenzae and 1 strain of Haemophilus parainfluenzae. Except one strain (H. influenzae) which was just decreased, all the bacteria were eliminated. 4. No side effect was found at all. As abnormal laboratory findings, elevation of GOT was found in 1 case, thrombocytosis in 2, and eosinophilia in 1 but all the changes were slight and were normalized by the time of re-examination. The above results suggest that FMOX is a useful and safe drug for the pediatric practice just as in the adult field. [\hyperlink{Ifex}{PMID: 3430714}, N Iwai et al., 1987]

\hypertarget{pmid_29140940}{I}nfliximab (IFX) infusion may lead to development of anti-IFX antibodies, and subsequent infusion reactions (IRs). The safety of rapid IFX infusion administered over 60 minutes has been under-investigated in children with inflammatory bowel disease. In a multicenter study, the frequency and nature of rapid infusion-associated IRs were examined. The medical records of all consecutive children with inflammatory bowel disease receiving rapid IFX infusions between January 2014 and December 2016 were reviewed. Poisson regression analysis was used to identify possible associated factors with IRs. A total of 4120 rapid infusions for 453 children (median age 16 yrs [interquartile range 13.8-17.8], 289 males, 374 with Crohn's disease) were included. One hundred thirty-five participants (29.8\%) received rapid IFX infusion for induction and maintenance while the rest received rapid IFX infusion after a median of 5 (interquartile range 4-9) standard infusions. The median dose of IFX using rapid protocol was 8 mg/kg/infusion (interquartile range 6-10). Two hundred sixty-seven (59\%) patients received 1 or more premedications and 161 (35.5\%) participants received concomitant immunosuppression. Twenty-one participants (4.6\%) had IRs with rapid infusions and 2 participants discontinued IFX because of IRs (0.4\%). Antihistamine premedications were associated with less frequent IR (adjusted relative risk = 0.30; 95\% confidence interval, 0.14-0.64; P = 0.002). In children with inflammatory bowel disease, rapid IFX infusion administered over 60 minutes is safe and well-tolerated. Antihistamine premedications may reduce frequency of IRs (see Video Abstract, Supplemental Digital Content 1, http://links.lww.com/IBD/B632). [\hyperlink{Ifex}{PMID: 29140940}, Wael El-Matary et al., 2017]

\hypertarget{pmid_3430724}{F}lomoxef (FMOX, 6315-S) was evaluated pharmacologically and clinically in its application to bacterial infections in children. 1. Pharmacokinetics: A bullous intravenous injection of FMOX at 20 mg/kg body weight gave a peak serum concentration of 114.6 +/- 34.4 micrograms/ml in 1 minute after the injection and T1/2 of beta-phase was 0.86 +/- 0.15 hours. 2. Bacteriological effectiveness: MIC of FMOX against Staphylococcus aureus except resistant strains (10(6) cells/ml) was below 0.39 micrograms/ml and against Haemophilus influenzae, Escherichia coli and Klebsiella pneumoniae (10(6) cells/ml) were below 0.78 microgram/ml. 3. Clinical effectiveness: Clinical effectiveness of FMOX was excellent in 14 cases, good in 2 cases and fairly good in 1 case among 17 cases of bacterial infections examined. An increase in eosinophilic leukocytes was observed in 1 case but no other clinical adverse effects were detected. These findings indicate that FMOX is a useful and safe antibiotic as a first choice against bacterial infections in children. [\hyperlink{Ifex}{PMID: 3430724}, T Morimoto et al., 1987]

\hypertarget{pmid_28476033}{S}everal studies have reported the use of dexmedetomidine (DEX) plus opioids for flexible bronchoscopy in both adults and children. To determine whether DEX plus sufentanil (SF) is safe for children, 142 children undergoing flexible bronchoscopy were assigned to one of three groups, each of which received the same SF loading dose and similar DEX and SF maintenance doses, but different loading doses of DEX: DS1 (DEX 0.5 μg·kg-1), DS2 (DEX 1.0 μg·kg-1), and DS3 (DEX 1.5 μg·kg-1). The Ramsay sedation scale was maintained at 3 in all groups. Results showed that anesthesia onset time was shorter, and the perioperative hemodynamic profile was more stable, in the DS3 group. The number of intraoperative movements was also lowest in the DS3 group. The time to first dose of rescue midazolam and lidocaine was significantly longer, but the total corresponding accumulated doses were lower in the DS3 group. Although the time to recovery prior to discharge from the post anesthesia care unit was longer, the overall incidence of tachycardia was lower in the DS3 group, and it received the highest bronchoscopist satisfaction score among the three groups. We therefore conclude that high-dose DEX plus SF can be safely and efficaciously used in children undergoing flexible bronchoscopy. [\hyperlink{Ifex}{PMID: 28476033}, Xiujing Dang et al., 2017]

\hypertarget{pmid_2693753}{C}linical usefulness of cefixime (CFIX), a new oral cephalosporin antibiotic, in pediatric field was investigated. The results obtained were summarized as follows. 1. The clinical efficacy of CFIX was investigated in a total of 138 children including 49 with upper respiratory tract infections (RTI), 22 with acute bronchitis, 18 with pneumonia, 19 with scarlet fever and 21 with urinary tract infections (UTI). 2. Clinical effectiveness was excellent in 58, good in 60, fair in 14 and poor in 3, with an overall efficacy rate of 87.4\%. The efficacy rate classified according to types of infection were 85.7\% in upper RTI, 89.5\% in acute bronchitis, 94.4\% in pneumonia, 78.9\% in scarlet fever, and 90.5\% in UTI. 3. Out of the suspected causative organisms, 43 strains of a total of 50 strains isolated were eradicated. The bacteriological eradication rate was 86.0\%. (Haemophilus influenzae 100\%, Haemophilus parainfluenzae 100\%, Streptococcus pyogenes 88.5\%, Escherichia coli 85.7\%). 4. One hundred forty four children were analyzed for side effect. Side effects were observed in 2 children (1.4\%) with diarrhea in 1 and anorexia in another. Abnormal laboratory test results were recorded in 4 children (3.3\%). The above results suggest that CFIX is a very useful new oral cephalosporin for the treatment of bacterial infections in children. [\hyperlink{Ifex}{PMID: 2693753}, H Mikawa et al., 1989]

\hypertarget{pmid_1784081}{B}asic and clinical studies of flomoxef (6315-S, FMOX) were performed in the pediatric surgical field. The results obtained are summarized as follows: 1. FMOX was administered to 7 pediatric patients with biliary atresia (FMOX 20 mg/kg, i.v.d.). Peak biliary levels of FMOX were obtained at 1 hour after finishing administration by drip infusion, and were higher than those in blood 1 hours after finishing administration by drip infusion. 2. Urinary excretion was excellent, and urinary recovery rates were 57.8-97.8\%. 3. FMOX was administered to 5 patients in the pediatric surgical field. One case was phlegmon, and other 4 cases were premature babies for postoperative prophylactic use. Clinical results were excellent in 1 case, good in 4 cases, with an overall efficacy rate of 100\%. No clinical and laboratory side effects due to the administration FMOX were observed. It was concluded that FMOX was a safe and effective antibiotic in the pediatric surgical field. [\hyperlink{Ifex}{PMID: 1784081}, J Yura et al., 1991]

\hypertarget{pmid_2391750}{O}ral new quinolone, norfloxacin (NFLX, AM-715), was evaluated for its safety, efficacy and pharmacokinetics in children. 1. NFLX was effective in 88.0\% of 25 cases infected with Haemophilus influenzae, Pseudomonas aeruginosa, Escherichia coli, Campylobacter jejuni, Staphylococcus aureus including methicillin-resistant strains, and other bacteria. 2. After single oral administration of 50 mg and 100 mg NFLX tablet at fasting, mean peak values of serum concentration were 0.35, 0.48 microgram/ml and T1/2 values were 2.2, 2.7 hours, respectively. 3. No adverse reactions suggestive for arthropathy were encountered with NFLX therapy with daily doses of 4.6-35.7 mg/kg (maximum 600 mg per day) and duration of 3 to 19 days. From these preliminary data, NFLX seems to have a place in the treatment of pediatric infectious diseases. [\hyperlink{Ifex}{PMID: 2391750}, H Meguro et al., 1990]

\hypertarget{pmid_24749085}{I}nfliximab (IFX) is considered safe and effective for the treatment of ulcerative colitis (UC) in both adults and children. The aim of this study was to evaluate the short- and long-term clinical course of IFX in Korean children with UC. Pediatric patients with UC who had received IFX infusions between November 2007 and May 2013 at Samsung Medical Center were retrospectively investigated. The clinical efficacy of IFX treatment was evaluated at 8 weeks (short term) and 54 weeks (long term) after the initiation of IFX treatment using the Pediatric Ulcerative Colitis Activity Index (PUCAI). The degree of response to IFX treatment was defined as complete response (PUCAI score=0), partial response (decrement of PUCAI score≥20 points), and non-response (decrement of PUCAI score <20 points). Adverse events associated with IFX treatment were also investigated. Eleven pediatric patients with moderate to severe UC had received IFX. The remission rate after IFX treatment was 46\% (5/11) and 82\% (9/11) at 8 weeks and 54 weeks after IFX treatment, respectively. All patients who were steroid-dependent before treatment with IFX achieved remission at 54 weeks and were able to stop treatment with corticosteroids, while all steroid-refractory patients failed to achieve remission at 54 weeks after treatment with IFX. Response to IFX treatment after 8 weeks may predict a favorable long-term response to IFX treatment in Korean pediatric UC patients. [\hyperlink{Ifex}{PMID: 24749085}, Jong Min Kim et al., 2014]

\hypertarget{pmid_3531565}{P}harmacokinetic and clinical studies of cefixime (CFIX) in children were done and the following results were obtained. Serum and urinary concentrations of CFIX were determined in 6 children aged 5 to 14 years given single doses of 1.5 or 6.0 mg/kg. Mean serum concentrations peaked at 4 hours after the administration of either 1.5 or 6.0 mg/kg, and respective peak values were 0.71 and 4.46 micrograms/ml. Biological half-lives for the low and the high doses were 5.28 and 4.45 hours, respectively. The 12-hours urinary recovery ranged from 7.0 to 13.8\% after administration of 1.5 mg/kg, and the 8-hours urinary recovery was 18.1\% after administration of 6.0 mg/kg. Therapeutic responses were recorded as excellent or good in 43 (97.7\%) of the children, comprising 13 with tonsillitis and 31 with scarlet fever. The microbiological effectiveness of CFIX on identified pathogens comprising 29 strains of S. pyogenes and 2 strains of S. aureus was satisfactory as evidence by a high eradication rate of 93.5\%. No clinical side effects were observed. Abnormal laboratory findings were elevation of GOT and/or GPT in 4 patients and eosinophilia in 1 patient. In conclusion, CFIX was found to be efficacious and safe for the treatment of bacterial infections in children. [\hyperlink{Ifex}{PMID: 3531565}, T Nishimura et al., 1986]

\hypertarget{pmid_19023899}{I}nfliximab (IFX) is efficacious in inducing remission in severe forms of pediatric Crohn's disease (CD). Adult studies indicate that IFX is also safe and well tolerated as maintenance therapy. The present study aimed to evaluate in a prospective manner the efficacy and safety of IFX as maintenance therapy of severe pediatric CD comparing scheduled and "on demand" treatment strategies. Forty children with CD (nonpenetrating, nonstricturing as well as penetrating forms, mean age: 13.9 +/- 2.2 years) with a severe flare-up (Harvey-Bradshaw Index [HBI] > or =5, erythrocyte sedimentation rate [ESR] >20 mm/h) despite well-conducted immunomodulator therapy (n = 36 azathioprine, n = 1 mercaptopurine, n = 3 methotrexate) combined with steroids were included in this randomized, multicenter, open-label study. Three IFX infusions (5 mg/kg) were administered at week (W)0/W2/W6. At W10, clinical remission (HBI <5) and steroid withdrawal were analyzed and IFX responders were randomized to maintenance therapy over 1 year: group A, scheduled every 2 months; group B, "on demand" on relapse. In all, 34/40 children came into remission during IFX induction therapy (HBI: 6.7 +/- 2.5 (WO) vs. 1.1 +/- 1.5 (W10); P < 0.001). At the end of phase 2, 15/18 (83\%) patients were in remission in group A compared to 8/13 (61\%) children in group B (P < 0.01), with a mean HBI of 0.5 versus 3.2 points (group A versus B, P = 0.011). In group A, 3/13 (23.1\%) children experienced a relapse compared to 11/12 (92\%) children in group B. No severe adverse event occurred during this trial. IFX is well tolerated and safe as maintenance therapy for pediatric CD, with a clear advantage when used on a scheduled 2-month basis compared to an "on demand" basis. [\hyperlink{Ifex}{PMID: 19023899}, Frank M Ruemmele et al., 2009]

\hypertarget{pmid_33514404}{D}espite its recognized efficacy and tolerability profile, during the last decade a rise of adverse events following ibuprofen administration in children has been reported, including a possible role in worsening the clinical course of infections. Our aim was to critically evaluate the safety of ibuprofen during the course of pediatric infectious disease in order to promote its appropriate use in children. Ibuprofen is associated with severe necrotizing soft tissue infections (NSTI) during chickenpox course. Pre-hospital use of ibuprofen seems to increase the risk of complicated pneumonia in children. Conflicting data have been published in septic children, while ibuprofen in the setting of Cystic Fibrosis (CF) exacerbations is safe and efficacious. No data is yet available for ibuprofen use during COVID-19 course. Ibuprofen should not be recommended for chickenpox management. Due to possible higher risks of complicated pneumonia, we suggest caution on its use in children with respiratory symptoms. While it remains unclear whether ibuprofen may have harmful effects during systemic bacterial infection, its administration is recommended in CF course. Despite the lack of data, it is seems cautious to prefer the use of paracetamol during COVID-19 acute respiratory distress syndrome in children. [\hyperlink{Ifex}{PMID: 33514404}, Lucia Quaglietta et al., 2021]

\hypertarget{pmid_3430721}{P}harmacokinetic and clinical studies were performed on flomoxef (FMOX, 6315-S), a new oxacephem antibiotic, as follows. 1. Pharmacokinetics Serum concentrations of FMOX were measured in 2 cases given 20 mg/kg bolus injection. In the 2 cases, peak concentrations of the drug were 44.3 and 197 micrograms/ml at 15 minutes, T1/2 (beta) were 0.76 and 0.47 hour and AUC were 44.8 and 169.5 micrograms.hr/ml, respectively. Urinary recovery rates for these cases during 6 hours were 83.1 and 54.9\%, respectively. The extremely high peak serum concentration in one case may be attributed to dehydration. 2. Clinical efficacy FMOX was administrated intravenously to 12 patients, 6 with pneumonia, 2 with cellulitis, 1 each with bronchitis, tonsillitis, purulent lymphadenitis and subcutaneous abscess, in doses of 55.0-120.0 mg/kg (average 82.2 mg/kg) t.i.d. for 4-13 days (average 6.2 days). The overall efficacy rate was 100\%, with excellent responses in 10 and good in 2. Bacteriological efficacy was excellent; 4 of 5 strains were eradicated and 1 strain was decreased. No clinical side effect was observed. Laboratory abnormality was observed in 1 case with transient eosinophilia. The above results suggested that FMOX would be an useful antibiotic for treating pediatric bacterial infections. [\hyperlink{Ifex}{PMID: 3430721}, T Hosoda et al., 1987]

\hypertarget{pmid_28089968}{I}nfliximab (IFX) is a monoclonal tumor necrosis factor-α-inhibiting antibody used in children with refractory arthritis and uveitis. Immunogenicity is associated with a lack of clinical response and infusion reactions in adults; data on immunogenicity in children treated with IFX for rheumatic diseases are scarce. We aimed to describe the prevalence of anti-IFX antibodies and determine co-factors associated with anti-IFX antibodies in children with inflammatory rheumatic and ocular diseases. Consecutive children treated between August 2009 and August 2012 with IFX at our department were included. Blood samples were collected every 6 months before IFX infusion and tested for anti-IFX antibodies by radioimmunoassay. Patients' charts were retrospectively reviewed for clinical features and analyzed for associations with anti-IFX antibodies. Anti-IFX antibodies occurred in 14/62 children (23\%) and 32/253 blood samples (12.6\%) after a mean treatment time of 1084 days (range 73-3498). Infusion reactions occurred in 10/62 (16\%) children during the treatment period. With continuation of IFX, anti-IFX antibodies disappeared in 7/14 children. In the bivariate analysis, the occurrence of anti-IFX antibodies was associated with younger age at IFX treatment start (mean age 7.01 vs 9.88 yrs, p = 0.003) and infusion reactions (OR 15.0), while uveitis as treatment indication was protective against development of anti-IFX antibodies (OR 0.17), likely because of higher IFX doses. In the multivariate logistic regression, all 3 covariates remained highly significant. Anti-IFX antibodies occurred commonly at any time during IFX treatment. Anti-IFX antibodies were associated with younger age at IFX start, infusion reactions, and arthritis as treatment indication. [\hyperlink{Ifex}{PMID: 28089968}, Florence A Aeschlimann et al., 2017]

\hypertarget{pmid_1784075}{S}tudies on pharmacokinetics and clinical effects of flomoxef (FMOX), a parenteral oxacephem antibiotic, were carried out in neonates. The results obtained are summarized as follows. 1. Mean peak serum concentrations of FMOX upon single administrations at doses of 20 mg/kg and 40 mg/kg were 33.3 +/- 7.33 micrograms/ml and 68.9 micrograms/ml, respectively. 2. Mean urinary recovery rates of FMOX in the first 6 hours after administration of the above doses were 35.2\% and 48.3\%, respectively. 3. FMOX was administered to 4 cases including 1 prophylactic case, 1 case each with aspiration pneumonia and sepsis, hypodermic abscess of the head, and itrauterine infection, at a dose of 20-30 mg/kg 2 or 3 times a day. Clinically, excellent results were obtained in 3 cases including an methicillin-resistant Staphylococcus aureus case. 4. No side effects nor abnormal laboratory test results were observed. [\hyperlink{Ifex}{PMID: 1784075}, H Sato et al., 1991]

\hypertarget{pmid_32443029}{I}nfliximab (IFX), a monoclonal antibody directed against tumor necrosis factor alpha is a potent treatment option for inflammatory bowel disease (IBD). Dosing regimens in children are extrapolated from adult data using a fixed, weight-based dose, which is often not adequate. While clinical trials have focused on safety and efficacy, there is limited data on pharmacokinetic characteristics and immunogenicity of IFX in children. The objective was to provide a systematic overview of current literature on pharmacokinetic and immunogenicity of IFX in children with IBD, to assess the validity of current adult to pediatric dosing extrapolation. A literature search identified publications up to October 2018. Eligibility criteria were study population consisting of children and/or adolescents with IBD, report of IFX trough levels and/or antibodies-to IFX, full text article or abstract, article in English, and original data. Initial electronic search yielded 2360 potentially relevant articles, with 1831 remaining after removal of duplicates. An additional search yielded another 202 potentially relevant articles. Of the 2033 retrieved articles, 2000 articles were excluded based on title, abstract, or eligibility criteria. Clearance of IFX was increased in young children and children with extensive disease, leading to lower trough levels after extrapolated dosing of 5 mg/kg, antibodies-to IFX emergence, and subsequent reduced efficacy. Adult to pediatric weight-based dosing extrapolation is often inadequate. We provide several considerations for optimal dosing of IFX in children and adolescents with IBD. [\hyperlink{Ifex}{PMID: 32443029}, Dwight A Winter et al., 2020]

\hypertarget{pmid_2391761}{N}orfloxacin (NFLX, AM-715), a new synthetic antibacterial agent, was administered to 18 child patients with infectious diseases. The patients included 5 boys and 13 girls from 3 to 14 years of ages. They were given orally dosage ranging 5.2-17.9 mg/kg/day for 4 to 14 days. Clinical efficacies were excellent in 1 case, good in 16 cases, unknown in 1 case, hence the total efficacy rate was determined to be 100\%. There were no cases which showed side effects of the drug and no abnormal laboratory test values were observed during the treatment. [\hyperlink{Ifex}{PMID: 2391761}, K Okada et al., 1990]

\section*{Isoproterenol Hydrochloride}
\subsection*{Result}
\subsubsection*{Answer}

Yes (Infants and children under 6 years, based on inhalation/IV use for asthma, with monitoring)  
Yes (Children 6 years and older, based on inhalation/IV use for asthma, with monitoring)  
Yes (Children 2 months to 10 years, based on use as a diagnostic agent under anesthesia, with monitoring)  

\subsubsection*{{Explanation}}
\hypertarget{Isoproterenol Hydrochloride}
A review of the available abstracts reveals several studies specifically addressing the safety of Isoproterenol Hydrochloride in children, particularly in the context of asthma and as a diagnostic agent during anesthesia. Below is a summary of the relevant evidence, organized by age range where possible:

1. **Children with Severe Asthma Attacks (General Pediatric Population, including under 6 years):**
   - A study of 22 children (age not specified, but described as "children") with 32 episodes of severe asthma attacks treated with continuous isoproterenol inhalation found the therapy to be effective and safe, especially in the early stage of severe attacks. Heart rates decreased as symptoms improved, and no significant adverse effects were detected [\hyperlink{pmid_1444819}{PMID: 1444819}, Y Adachi et al., 1992].
   - Another study compared efficacy and safety in 31 children divided into two groups: under 6 years (n=20) and over 7 years (n=22). Continuous isoproterenol inhalation was effective in both groups, with no mention of significant adverse effects. The therapy was slightly less effective in younger children, but still considered safe [\hyperlink{pmid_1492792}{PMID: 1492792}, Y Adachi et al., 1992].
   - A retrospective review of 11 children (8 months to 15 years) with respiratory failure due to status asthmaticus treated with intravenous isoproterenol reported that all cases were successfully treated without complications, and the therapy was described as safe and effective [\hyperlink{pmid_2013558}{PMID: 2013558}, M S Victoria et al., 1991].
   - In contrast, a prospective study of 20 admissions for severe childhood asthma found evidence of cardiotoxicity (elevated CPK-MB and ECG changes) in all six admissions where isoproterenol infusion was used. The authors recommend serial cardiac monitoring for all children receiving isoproterenol for severe asthma, indicating a risk of cardiotoxicity [\hyperlink{pmid_1956735}{PMID: 1956735}, J F Maguire et al., 1991].
   - A study of 23 asthmatic children (age not specified) compared three methods of isoproterenol administration and found all methods similarly effective in reversing bronchospasm, with no mention of significant adverse effects [\hyperlink{pmid_328233}{PMID: 328233}, M Loren et al., 1977].
   - A small study of 11 asthmatic children (9 to 16 years) found that inhaled isoproterenol produced bronchodilation but was less effective than albuterol, with no overt cardiovascular effects reported [\hyperlink{pmid_6846920}{PMID: 6846920}, M R Littner et al., 1983].

2. **Children Undergoing Anesthesia (Ages 2 months to 10 years, and infants):**
   - Several studies evaluated intravenous isoproterenol as a marker for epidural test dosing in children under anesthesia. In a study of 72 children (mean age 2.8 ± 1.7 years), isoproterenol produced a reliable heart rate response with minimal adverse effects (one transient dysrhythmia) [\hyperlink{pmid_8712442}{PMID: 8712442}, S Kozek-Langenecker et al., 1996].
   - Another study of 44 children (2 months to 10 years) found that isoproterenol produced a graded increase in heart rate with no episodes of hypotension or arrhythmia. The authors note that the application appears promising but recommend further evaluation of neurotoxicity and reliability [\hyperlink{pmid_8418721}{PMID: 8418721}, M Perillo et al., 1993].
   - A dose-response study in 36 children (0.5–8 years) found that isoproterenol was effective as a marker for epidural test dosing, with no mention of significant adverse effects [\hyperlink{pmid_9728825}{PMID: 9728825}, S A Kozek-Langenecker et al., 1998].

3. **Children with Cardiac Preexcitation (Ages not specified, but pediatric):**
   - A retrospective review of 151 pediatric patients undergoing electrophysiologic study under anesthesia found that low-dose isoproterenol significantly affected accessory pathway conduction. The study did not report on safety or adverse effects, focusing instead on electrophysiologic outcomes [\hyperlink{pmid_21156771}{PMID: 21156771}, Jeremy P Moore et al., 2011].

**Summary by Age Range:**
- **Infants and Young Children (under 6 years):** Multiple studies (including those with children as young as 8 months) report isoproterenol to be effective and generally safe for acute asthma management, with rare or no significant adverse effects in most reports. However, one study found evidence of cardiotoxicity in children receiving isoproterenol infusions for severe asthma, recommending cardiac monitoring.
- **Children (6 years and older):** Similar findings as above, with isoproterenol being effective and generally safe for asthma and as a diagnostic agent under anesthesia, with rare adverse effects.
- **General Pediatric Population (2 months to 10 years, and up to 15 years):** Studies in anesthetized children and those with severe asthma generally report isoproterenol as safe, with rare adverse effects, but with a caution for potential cardiotoxicity in severe cases.

**Conclusion:** There are targeted studies evaluating the safety of Isoproterenol Hydrochloride in children, particularly for acute asthma and as a diagnostic agent during anesthesia. Most studies affirm its safety when used appropriately, but at least one study highlights a risk of cardiotoxicity in severe asthma cases, warranting cardiac monitoring. Therefore, while isoproterenol is generally considered safe for use in children in these contexts, caution and monitoring are advised, especially in severe cases or with intravenous infusion.

\subsection*{Abstracts}
\hypertarget{pmid_1444819}{T}he aim of this study was to evaluate the efficacy and safety of continuous isoproterenol inhalation therapy for asthma attacks in children. We used l-body isoproterenol (Proternol L) in 22 children with 32 episodes of severe attacks. One of them did not respond to this therapy, and two had complications (atelectasis and pneumothorax). Twenty-nine cases were divided into three subgroups according to their clinical scores; A) scores less than or equal to 4, which meant that they were in the early stage of severe attack (n = 9), B) scores 5-6, which meant impending respiratory failure (n = 17), C) scores greater than or equal to 7, which meant respiratory failure (n = 3). The values of SpO2 at the start of this therapy were 94.8, 91.5, 82.0\%, respectively. The more severe their attacks were, the lower their SpO2 levels were. The periods until their scores became zero were 0.78, 6.3, 17.2 hours, respectively. There were significant differences between each period respectively (p less than 0.001, p less than 0.01). Heart rates decreased when their symptoms improved, and other adverse effects were not detected. These results suggest that this therapy is effective and safe for children with severe asthma attacks, especially in the early stage. [\hyperlink{Isoproterenol Hydrochloride}{PMID: 1444819}, Y Adachi et al., 1992]

\hypertarget{pmid_8712442}{A}n epidural test dose containing epinephrine does not reliably produce hemodynamic responses in children under halothane anesthesia. The purpose of this study was to determine hemodynamic responses to intravenous isoproterenol in both awake and halothane-anesthetized children. After obtaining institutional review board approval and parental informed consent, 72 ASA physical status 1 or 2 children (2.8 +/- 1.7 yr) undergoing elective minor surgery were studied before and during anesthesia with 1.2 minimum alveolar concentration halothane. A bolus containing 0.25 mg/ kg bupivacaine and 0.05 microgram/kg, 0.075 microgram/kg, or 0.1 microgram/kg isoproterenol, or bupivacaine and saline was injected via a peripheral arm vein to simulate intravascular injection of an epidural test dose. Before induction of anesthesia, all patients showed a positive test response after isoproterenol injection (heart rate increase > or = 20 beats/min). During anesthesia, 79\% of patients receiving 0.05 microgram/kg, 89\% of patients receiving 0.075 microgram/kg, and 100\% of patients receiving 0.1 microgram/kg met the criterion of a positive test response. Among each treatment group, all infants showed a positive test response. Blood pressure did not differ among the groups at any time. Transient benign dysrhythmias occurred in only one patient under halothane anesthesia receiving 0.075 microgram/kg isoproterenol. Isoproterenol at a dose of 0.1 microgram/kg is a sensitive indicator for intravascular injection of a test dose in children anesthetized with halothane and nitrous oxide. Isoproterenol at a dose of 0.05 microgram/kg approximates a minimal effective dose in awake children and in infants. After detailed studies on neural toxicity, isoproterenol could be of value as an epidural test agent in children. [\hyperlink{Isoproterenol Hydrochloride}{PMID: 8712442}, S Kozek-Langenecker et al., 1996]

\hypertarget{pmid_9728825}{I}soproterenol has been suggested as an alternative marker for epidural test dosing in children receiving halothane anesthesia. The purpose of this prospective, randomized, double-blind study was to determine the chronotropic response to IV isoproterenol in sevoflurane-anesthetized children. Thirty-six ASA physical status I children (0.5-8 yr) were anesthetized with either halothane or sevoflurane at 1 minimum alveolar anesthetic concentration adjusted for age in 70\% nitrous oxide. Patients received incremental IV injections of isoproterenol until their heart rate increased > or = 20 bpm above baseline. The minimal effective dose of isoproterenol required to produce an increase of > or = 20 bpm was 55 ng/kg (42-72 ng/kg; 95\% confidence interval) in sevoflurane-anesthetized children and 32 ng/kg (26-38 ng/kg; 95\% confidence interval) in halothane-anesthetized children (P < 0.05). This dose-response study suggests that sevoflurane antagonizes beta-adrenergic-mediated chronotropic responses to isoproterenol more than halothane. These observations also suggest that larger doses of isoproterenol will be necessary for epidural test dosing in children receiving sevoflurane rather than halothane anesthesia. Isoproterenol has been suggested as an alternative marker for epidural test dosing in children receiving halothane anesthesia. This isoproterenol dose-response study indicates that larger doses of isoproterenol will be necessary for epidural test dosing in children undergoing sevoflurane rather than halothane anesthesia. [\hyperlink{Isoproterenol Hydrochloride}{PMID: 9728825}, S A Kozek-Langenecker et al., 1998]

\hypertarget{pmid_8418721}{T}he purpose of this study was to determine if isoproterenol would be an effective marker of intravascular injection in anesthetized children. Forty-four ASA 1 children, aged 2 mo to 10 yr, were randomly assigned to two groups. Children in group 1 (n = 21) received 0.05 microgram/kg isoproterenol, and children in group 2 (n = 23) received 0.075 microgram/kg isoproterenol. A blinded observer continuously recorded heart rate and arterial blood pressure. Measurements were recorded before the surgical incision at steady-state halothane concentration of 1.2 minimum alveolar concentration adjusted for age. Isoproterenol produced a graded increase in heart rate with mean maximum increases of 16.5 +/- 8.7 beats/min in group 1 and 21.5 +/- 9.2 beats/min in group 2. No episodes of hypotension and arrhythmia were noted. Isoproterenol, 0.075 microgram/kg, is more sensitive but still is an imperfect marker of an intravascular injection. It produces a heart rate increase in 96\% of children anesthetized with halothane and nitrous oxide in 50\% oxygen. The application of isoproterenol as an epidural test dose appears promising, but cannot be recommended until its full reliability and neurotoxicity are evaluated. [\hyperlink{Isoproterenol Hydrochloride}{PMID: 8418721}, M Perillo et al., 1993]

\hypertarget{pmid_619057}{T}o determine the effectiveness of oral propranolol in children, we administered 0.5 to 4.0 mg/kg/day of the drug to 64 children (age one day to 20 years); 41 with cardiac dysrhythmias, six with isiopathic hypertrophic subaortic stenosis, and 17 with paroxysmal hypoxemic spells associated with right ventricular infundibular obstruction. A new liquid form of propranolol (10 mg/ml) was administered to 37 of the younger patients, and tablets were given to the other 27. Propranolol improved the dysrhythmia in 31 of 41 patients, being notably effective in supraventricular tachycardia and ventricular tachycardia associated with a prolonged QT interval. The drug also eliminated symptoms attributed to IHSS in six of six patients and abolished hypoxemic spells in 12 of 17. The liquid and tablets were equally effective; and the liquid had the advantage of allowing for accurate dose changes in younger children. We conclude that oral propranolol is an excellent drug for use in pediatric patients with certain types of cardiac disease. [\hyperlink{Isoproterenol Hydrochloride}{PMID: 619057}, P Gillette et al., 1978]

\hypertarget{pmid_21788220}{P}ropranolol hydrochloride is a safe and effective medication for treating infantile hemangiomas (IHs), with decreases in IH volume, color, and elevation. Forty children between the ages of 9 weeks and 5 years with facial IHs or IHs in sites with the potential for disfigurement were randomly assigned to receive propranolol or placebo oral solution 2 mg/kg per day divided 3 times daily for 6 months. Baseline electrocardiogram, echocardiogram, and laboratory evaluations were performed. Monitoring of heart rate, blood pressure, and blood glucose was performed at each visit. Children younger than 6 months were admitted to the hospital for monitoring after their first dose at weeks 1 and 2. Efficacy was assessed by performing blinded volume measurements at weeks 0, 4, 8, 12, 16, 20, and 24 and blinded investigator scoring of photographs at weeks 0, 12, and 24. IH growth stopped by week 4 in the propranolol group. Significant differences in the percent change in volume were seen between groups, with the largest difference at week 12. Significant decrease in IH redness and elevation occurred in the propranolol group at weeks 12 and 24 (P = .01 and .001, respectively). No significant hypoglycemia, hypotension, or bradycardia occurred. One child discontinued the study because of an upper respiratory tract infection. Other adverse events included bronchiolitis, gastroenteritis, streptococcal infection, cool extremities, dental caries, and sleep disturbance. Propranolol hydrochloride administered orally at 2 mg/kg per day reduced the volume, color, and elevation of focal and segmental IH in infants younger than 6 months and children up to 5 years of age. [\hyperlink{Isoproterenol Hydrochloride}{PMID: 21788220}, Marcia Hogeling et al., 2011]

\hypertarget{pmid_20644039}{P}ropranolol hydrochloride has been prescribed for decades in the pediatric population for a variety of disorders, but its effectiveness in the treatment of infantile hemangiomas (IHs) was only recently discovered. Since then, the use of propranolol for IHs has exploded because it is viewed as a safer alternative to traditional therapy. We report the cases of 3 patients who developed symptomatic hypoglycemia during treatment with propranolol for their IHs and review the literature to identify other reports of propranolol-associated hypoglycemia in children to highlight this rare adverse effect. Although propranolol has a long history of safe and effective use in infants and children, understanding and recognition of deleterious adverse effects is critical for physicians and caregivers. This is especially important when new medical indications evolve as physicians who may not be as familiar with propranolol and its adverse effects begin to recommend it as therapy. [\hyperlink{Isoproterenol Hydrochloride}{PMID: 20644039}, Kristen E Holland et al., 2010]

\hypertarget{pmid_2013558}{B}etween January 1985 and December 1988, a total of 701 children were admitted to The Methodist Hospital, Brooklyn, NY for treatment of acute asthma. Eleven of these patients (age range between 8 months and 15 years) went into respiratory failure. All cases of respiratory failure were successfully treated with intravenous isoproterenol. Only one patient needed mechanical ventilation. Treatment with isoproterenol was safe and effective without any complications. We present our experience in the use of isoproterenol in treating children with respiratory failure secondary to status asthmaticus. A brief review of the literature is included. [\hyperlink{Isoproterenol Hydrochloride}{PMID: 2013558}, M S Victoria et al., 1991]

\hypertarget{pmid_1492792}{T}he aim of this study was to evaluate the efficacy of continuous isoproterenol inhalation therapy for severe asthma attacks in younger children, compared with its efficacy in older children. We used l-body isoproterenol (Proternol L) in 31 children with 42 episodes of severe attacks. They were divided into two group according to age: 20 cases under 6 years old (Group A), and 22 cases over 7 years old (Group B). All of the patients except for one in Group B, eventually improved with this therapy. Wood's clinical scores for Group A were significantly higher than those for group B (p < 0.01). In 22 cases whose scores were 5-6, their SpO2 values at the onset of this therapy were 90.8 +/- 3.17 in group A and 92.4 +/- 3.82\% in group B. The improvement time of group A (13.6 +/- 16.2 hours) was significantly longer than that of group B (2.5 +/- 5.66, p < 0.01). The nebulized isoproternol doses for group A were 0.47 +/- 0.168 and for group B 0.26 +/- 0.096 mg/kg/saline 500 ml. The dose for group A was significantly higher than that for group B (p < 0.01). We concluded that continuous isoproterenol inhalation therapy was effective even in younger children. But the degree of efficacy was slightly lower in younger children, although they inhaled higher doses of isoproterenal than older children. [\hyperlink{Isoproterenol Hydrochloride}{PMID: 1492792}, Y Adachi et al., 1992]

\hypertarget{pmid_21156771}{R}apid anterograde conduction in the setting of ventricular preexcitation is associated with an increased risk of sudden cardiac death. The effect of isoproterenol in this setting is unclear, particularly in younger anesthetized patients. The aim of this study was to determine the effect of isoproterenol on accessory-pathway conduction in children undergoing general anesthesia and its role in the risk-stratification process. The records of 151 pediatric patients with preexcitation undergoing electrophysiologic study under propofol anesthesia during a 5-year period were reviewed. Data included accessory-pathway effective refractory period, minimum 1:1 accessory pathway conduction with atrial pacing, and shortest preexcited R-R interval in atrial fibrillation. Measurements were repeated after administration of low-dose isoproterenol (mean, 0.013 μg/kg per min; range, 0.003 to 0.027). All accessory-pathway characteristics were significantly shortened with isoproterenol (P<0.001). Accessory-pathway effective refractory period increased modestly with age, both in the baseline state (r=0.172, P=0.04) and with isoproterenol (r=0.267, P<0.01) as did minimum 1:1 accessory pathway conduction with atrial pacing (r=0.178, P=0.034, and r=0.175, P<0.01, respectively). Accessory-pathway effective refractory period ≤250 ms was observed in only 5\% of patients at baseline vs 25\% after isoproterenol, and Shortest preexcited R-R interval in atrial fibrillation ≤250 ms was noted in 16\% vs 41\%. Tachycardia was induced in 48 of 151 patients before and in 102 of 151 after isoproterenol. In anesthetized children with ventricular preexcitation, accessory pathways display shorter conduction properties at younger ages and important adrenergic sensitivity at all ages. Use of low-dose isoproterenol resulted in a substantial increase in the number of patients who would otherwise meet typical criteria for ablation. [\hyperlink{Isoproterenol Hydrochloride}{PMID: 21156771}, Jeremy P Moore et al., 2011]

\hypertarget{pmid_30659785}{P}ropranolol is an effective method of treatment for infantile hemangiomas (IH). A recent concern is a shift of the therapy into outpatient settings. The aim of the study was to evaluate the safety of initiating and maintaining propranolol therapy for IH. The study involved 55 consecutive children with IH being treated with propranolol. The patients were assessed in the hospital at the initiation of the therapy and later in outpatient settings during and after the therapy. Each time, the following monitoring methods were used: physical examination, cardiac ultrasound (ECHO), electrocardiography (ECG), blood pressure (BP), heart rate (HR), and biochemical parameters: blood count, blood glucose, aspartate transaminase (AST), alanine transaminase (ALT), and ionogram. The therapeutic dose of propranolol was 2.0 mg/kg/day divided into 2 doses. Four children were excluded during the qualification or the initiation of propranolol; a total of 51 patients were subject to the final analysis. All the children presented clinical improvement. There was a significant reduction in the mean HR values only at the initiation of propranolol. There were no changes in HR during the course of the therapy. Blood pressure values were within normal limits. Both systolic and diastolic values decreased in the first 3 months. Bradycardia and hypotension were observed sporadically, and they were asymptomatic. Electrocardiography did not show significant deviations. The pathological findings of the ECHO scans were not a contraindication to continuing the therapy. There were no changes in biochemical parameters. Apart from 1 symptomatic case of hypoglycemia, other low glucose episodes were asymptomatic and sporadic. The observed adverse effects were mild and the propranolol dose had to be adjusted in only 6 cases. Propranolol is effective, safe and well-tolerated by children with IH. The positive results of the safety assessment support the strategy of initiating propranolol in outpatient settings. Future studies are needed to assess the benefits of the therapy in ambulatory conditions. [\hyperlink{Isoproterenol Hydrochloride}{PMID: 30659785}, Lidia Babiak-Choroszczak et al., 2019]

\hypertarget{pmid_328233}{T}weinty-three asthmatic children had severe sudden bronchospasm due to numerous factors. Baseline values for the peak expiratory flow rate were less than 25 percent of predicted. Utilizing an analysis of variance, three methods of administering isoproterenol hydrochloride (hand-held Freon-propelled nebulization, continuous nebulization, and intermittent positive-pressure breathing [IPPB]) were compared and found to be similar in reversing the bronchospasm (F = 1.56; degrees of freedom, 2/44; P = 0.22). There were no patients whose condition consistently improved with IPPB over the methods of therapy using simple nebulization. Therapy with IPPB did not offer any advantage over simple nebulization in patients with severe, reversible airway obstruction. [\hyperlink{Isoproterenol Hydrochloride}{PMID: 328233}, M Loren et al., 1977]

\hypertarget{pmid_25009634}{T}he aim of the present study was to investigate the effects of propranolol and isoproterenol on the growth curve of infantile hemangioma endothelial cells (IHECs)  [\hyperlink{Isoproterenol Hydrochloride}{PMID: 25009634}, Yalin Zhu et al., 2014] Propranolol hydrochloride is a nonselective β-blocker that is used for the treatment of hypertension, arrhythmia, and angina pectoris. In Japan, it was recently approved for the treatment of childhood arrhythmia. It has been observed to produce drastic involution of infantile hemangiomas. The aim of this prospective study was to examine propranolol's superiority to classical therapy with pulsed dye laser and/or cryosurgery in treating proliferating infantile hemangiomas. Fifteen patients between the ages of 1 and 4 months with proliferating infantile hemangiomas received grinded propranolol tablets 2 mg/kg per day divided in three doses. Twelve patients with proliferating infantile hemangiomas receiving pulsed dye laser and/or cryosurgery were enrolled as controls. Baseline electrocardiogram, echocardiogram, and chest x-ray were performed. Monitoring of heart rate, blood pressure, and blood glucose was performed every 2 weeks. Efficacy was assessed by performing blinded volume measurements and taking photographs at every visit. Propranolol induced significantly earlier involution and redness reduction of infantile hemangiomas, compared to pulsed dye laser and cryosurgery. Adverse effects such as hypoglycemia, hypotension, or bradycardia did not occur. The dramatic response of infantile hemangiomas to propranolol and few side effects suggest that early treatment of infantile hemangiomas could result in decreased disfigurement. Propranolol should be considered as a first-line treatment of infantile hemangiomas. [\hyperlink{Isoproterenol Hydrochloride}{PMID: 25009634}, Shinji Kagami et al., 2013]

\hypertarget{pmid_1956735}{W}e prospectively evaluated 20 patient admissions for severe exacerbation of childhood asthma at The Children's Hospital, Boston, to detect evidence of cardiotoxicity. Evidence of cardiotoxicity was found in all six patient admissions for which isoproterenol infusion was utilized. This included marked elevation of serum creatine phosphokinase isoenzyme (CPK-MB) levels and electrocardiogram abnormalities consistent with transient myocardial ischemia. Peak serum CPK-MB levels were significantly lower and electrocardiogram abnormalities were significantly less frequent during 14 patient admissions for which isoproterenol infusion was not utilized. Risk factors associated with cardiotoxicity included tachycardia, hypercapnia, acidosis, and intravenous isoproterenol therapy. We conclude that cardiotoxicity is not infrequent during therapy for severe exacerbations of childhood asthma. Electrocardiograms and measurement of serum CPK-MB levels are sensitive, useful, and readily obtained indicators of cardiotoxicity. Abnormalities of these studies may detect cardiotoxicity prior to the occurrence of more blatant or catastrophic manifestations of cardiotoxicity. We therefore recommend serial monitoring of serum CPK-MB levels and electrocardiograms for all children requiring an admission to the intensive care unit for management of severe asthmatic exacerbation. [\hyperlink{Isoproterenol Hydrochloride}{PMID: 1956735}, J F Maguire et al., 1991]

\hypertarget{pmid_31266444}{P}ropranolol hydrochloride is the first-line agent recommended for the treatment of infantile hemangiomas (IH). Serious adverse effects of propranolol therapy for hemangiomas are infrequent. We report a case presented in deep hypoglycemic coma during his treatment with propranolol for IH. Through our case report and the review of the literature, we aimed to underline the importance of recognizing adverse effects during propranolol therapy. Although propranolol has a long history of safe and effective use in infants and children, pediatricians should be aware that life-threatening adverse effects can happen during propranolol therapy for IH. Early identification of these adverse effects can be of great importance for patient management and prognosis. It must certainly be noted that not just early identification among doctors, but education for parents is crucial. [\hyperlink{Isoproterenol Hydrochloride}{PMID: 31266444}, Ilirjana Bakalli et al., 2019]

\hypertarget{pmid_3712718}{A} previously well 2-year-old child presented with seizures and ventricular tachycardia shortly after playing with an aerosol can of a well-known proprietary deodorant. She required intensive care and survived without sequelae. The propellants used in this product were isobutane, n-butane, and propane. The propellants have been thought to be safer than the previously used Freons, which were known to be cardiotoxic and neurotoxic. Significant exposure was confirmed by the detection of n-butane and isobutane in the patient's serum. We conclude that unintentional exposure to non-Freon aerosol propellants in a nonconfined space can be hazardous to children. Aerosol cans should be considered to represent toxic hazards and should be kept out of reach of children. [\hyperlink{Isoproterenol Hydrochloride}{PMID: 3712718}, S Wason et al., 1986]

\hypertarget{pmid_19706583}{I}nfantile hemangiomas (IHs) are the most-common soft-tissue tumors of infancy. We report the use of propranolol to control the growth phase of IHs. Propranolol was given to 32 children (21 girls; mean age at onset of treatment: 4.2 months) after clinical and ultrasound evaluations. After electrocardiographic and echocardiographic evaluations, propranolol was administered with a starting dose of 2 to 3 mg/kg per day, given in 2 or 3 divided doses. Blood pressure and heart rate were monitored during the first 6 hours of treatment. In the absence of side effects, treatment was continued at home and the child was reevaluated after 10 days of treatment and then every month. Ultrasound measurements were performed after 60 days of treatment. Immediate effects on color and growth were noted in all cases and were especially dramatic in cases of dyspnea, hemodynamic compromise, or palpebral occlusion. In ulcerated IHs, complete healing occurred in <2 months. Objective clinical and ultrasound evidence of longer-term regression was seen in 2 months. Systemic corticosteroid treatment could be stopped within a few weeks. Treatment was administered for a mean total duration of 6.1 months. Relapses were mild and responded to retreatment. Side effects were limited and mild. One patient discontinued treatment because of wheezing. Propranolol administered orally at 2 to 3 mg/kg per day has a consistent, rapid, therapeutic effect, leading to considerable shortening of the natural course of IHs, with good clinical tolerance. [\hyperlink{Isoproterenol Hydrochloride}{PMID: 19706583}, Véronique Sans et al., 2009]

\hypertarget{pmid_23271298}{T}he clinical efficacy and safety of topical propranolol hydrochloride gel in the treatment of superficial infantile hemangiomas (IHs) were assessed. Fifty-one cases of IHs from Oct. 2010 to Sept. 2011 were subjected to the topical propranolol hydrochloride gel intervention in Fuzhou General Hospital of Nanjing Military Commands, China. Changes in size, texture, color, peak systolic velocity of the hemangiomas, resistance index and adverse effects were observed. The results were evaluated by using Achauer system, and responses of IHs to pranpronolol were considered scale II (poor) in 4 patients (17.24\%), scale II (moderate) in 18 patients (24.14\%), scale III (good) in 22 patients (44.83\%) and scale IV (excellent) in 7 patients (13.79\%). The response of superficial hemangiomas was significantly better than other hemangiomas (P<0.05), and no differences in response were found among different primary sites (P>0.05). Our study indicates that topical application of 3\% propranolol hydrochloride gel is effective and safe in treating IHs. [\hyperlink{Isoproterenol Hydrochloride}{PMID: 23271298}, Lie Wang et al., 2012]

\hypertarget{pmid_23578266}{T}he aim of this study is to formulate an extemporaneous pediatric oral solution of propranolol hydrochloride (PRO) 2 mg/ml for the therapy of infantile haemangioma or hypertension in a target age group of 1 month to school children and to evaluate its stability. A citric acid solution and/or a citrate-phosphate buffer solution, respectively, were used as the vehicles to achieve pH value of about 3, optimal for the stability of PRO. In order to mask the bitter taste of PRO, simple syrup was used as the sweetener. All solutions were stored in tightly closed brown glass bottles at 5 ± 3 °C and/or 25 ± 3 °C, respectively. The validated HPLC method was used to evaluate the concentration of PRO and a preservative, sodium benzoate, at time intervals of 0-180 days. All preparations were stable at both storage temperatures with pH values in the range of 2.8-3.2. According to pharmacopoeial requirements, the efficacy of sodium benzoate 0.05 \% w/v was proved (Ph.Eur., 5.1.3). The preparation formulated with the citrate-phosphate buffer, in our experience, had better palatability than that formulated with the citric acid solution. propranolol hydrochloride pediatric preparation extemporaneous preparation solution stability testing HPLC. [\hyperlink{Isoproterenol Hydrochloride}{PMID: 23578266}, Sylva Klovrzová et al., 2013]

\hypertarget{pmid_32532590}{I}nfantile hepatic hemangioendothelioma (IHHE) is a benign liver tumor, associated with hypothyroidism and vascular malformations along the skin, brain, digestive tract and other organs. Here, we determined a single-center patient cohort by evaluating the effectiveness and safety of propranolol and sirolimus for the treatment of IHHE. We performed a monocentric and observational study, based on clinical data obtained from 20 cases of IHHE treated with oral propranolol and sirolimus at the Shanghai Children's Medical Center (SCMC), between December 2017 and April 2019. All cases were confirmed by abdominal enhanced CT examination (18/20, 90\%) and sustained decrease of alpha fetoprotein (AFP) (2/20, 10\%). Propranolol treatment was standardized as once a day at 1.0mg/kg for patients younger than 2 months, and twice a day at 1.0mg/kg (per dose) for patients older than 2 months. Sirolimus was used to treat refractory IHHE patients after 6 months of propranolol treatment, and initial dosing was at 0.8mg/m The effective rate of propranolol for the treatment of children with IHHE was 85\% (17/20). In most cases, the AFP levels gradually decreased into the normal range. A complete response (CR) was achieved in 3 cases, partial response (PR) for 14 cases, progressive disease (PD) for 2 cases and stable disease (SD) was only detected once. Lesions decreased in two PD patients after administration of oral sirolimus. No serious adverse reactions were observed. This study indicates that both propranolol and sirolimus were effective drugs for the treatment of children with IHHE at SCMC. [\hyperlink{Isoproterenol Hydrochloride}{PMID: 32532590}, Ruicheng Tian et al., ]

\hypertarget{pmid_21601311}{I}nfantile hemangioma (IH) is a frequently encountered tumor with a potentially complicated course. Recently, propranolol was discovered to be an effective treatment option. To describe the effects and side effects of propranolol treatment in 28 children with (complicated) IH. A protocol for treatment of IH with propranolol was designed and implemented. Propranolol was administered to 28 children (21 girls and 7 boys, mean age at onset of treatment: 8.8 months). All 28 patients had a good response. In two patients, systemic corticosteroid therapy was tapered successfully after propranolol was initiated. Propranolol was also an effective treatment for hemangiomas in 4 patients older than 1 year of age. Side effects that needed intervention and/or close monitoring were not dose dependent and included symptomatic hypoglycemia (n = 2; 1 patient also taking prednisone), hypotension (n = 16, of which 1 is symptomatic), and bronchial hyperreactivity (n = 3). Restless sleep (n = 8), constipation (n = 3) and cold extremities (n = 3) were observed. Clinical studies are necessary to evaluate the incidence of side effects of propranolol treatment of IH. Propranolol appears to be an effective treatment option for IH even in the nonproliferative phase and after the first year of life. Potentially harmful adverse effects include hypoglycemia, bronchospasm, and hypotension. [\hyperlink{Isoproterenol Hydrochloride}{PMID: 21601311}, Marlies de Graaf et al., 2011]

\hypertarget{pmid_16542772}{A}nticholinergic treatment combined with intermittent catheterisation is the cornerstone of the conservative treatment strategy in children with neurogenic detrusor overactivity, which in most cases is due to congenital causes. Efficacy, tolerability and safety of propiverine hydrochloride were evaluated retrospectively in these children. At four specialized outpatient clinics, all children's records were scrutinized for first-line propiverine hydrochloride treatment, or second- or third-line treatment after failure of a non-selective alpha-blocker (phenoxybenzamine) and/or other anticholinergics (oxybutynin, trospium chloride). The primary efficacy outcomes were urodynamic parameters, with clinical symptoms as secondary outcomes. Statistical analysis was performed by paired t-tests (significance level p < 0.05). Altogether 74 children and adolescents (40 boys, 34 girls; age range 11 months-19 years) were treated with propiverine hydrochloride (average duration 2 years and approximately 4 months; individual dose range 5-75 mg). The primary efficacy outcome parameters improved significantly: maximum cystometric capacity 161.2 [standard deviation (SD) 97.3] to 252.2 ml (SD 117.2), p < 0.001; maximum detrusor pressure 43.8 (SD 39.2) to 27.1 cm H(2)O (SD 26.4), p = 0.002; bladder compliance 7.6 (SD 6.4) to 17.0 ml/cm H(2)O (SD 16.2), p < 0.001. Phasic detrusor overactivity was abolished by 63\%; incontinence resolved by 54\%. One patient spontaneously reported a typical anticholinergic adverse event, which resolved after dose reduction. No safety concerns were documented. Propiverine hydrochloride is effective in neurogenic detrusor overactivity in children and adolescents, even in some of those cases unresponsive to other anticholinergics. The low incidence rate (<1.5\%) of adverse events evidences a favourable risk-benefit profile of propiverine hydrochloride, considering in particular the total documented treatment and surveillance period of 171 patient years and nine months. [\hyperlink{Isoproterenol Hydrochloride}{PMID: 16542772}, Ute Grigoleit et al., 2006]

\hypertarget{pmid_6846920}{A}lthough albuterol aerosol is an effective bronchodilator in adult asthmatics its efficacy and safety are not fully established in asthmatic children. Albuterol (100 and 200 micrograms), isoproterenol (70 and 140 micrograms) and placebo were administered from metered dose aerosol cannisters to 11 asthmatic children (9 to 16 years of age). Our findings indicate that in asthmatic children albuterol aerosol causes bronchodilation for up to six hours and greater bronchodilation than comparable doses of inhaled isoproterenol and does not produce overt effects on the cardiovascular system, significant changes in arterial partial pressure of oxygen or excessive tremor. [\hyperlink{Isoproterenol Hydrochloride}{PMID: 6846920}, M R Littner et al., 1983]

\hypertarget{pmid_29149854}{P}ropranolol has become the first-line treatment for complicated Infantile Hemangioma (IH), showing so far a good risk-benefit profile. We report the case of a toddler, on propranolol, who suffered cardiac arrest during an acute viral infection. She had a neurally-mediated syncope that progressed to asystole, probably because of concurrent factors as dehydration, beta-blocking and probably individual susceptibility to vaso-vagal phenomena. In fact a significant history of breath-holding spells was consistent with vagal hyperactivity. The number of patients treated with propranolol for IHs will increase and sharing experience will help to better define the safety profile of this drug. [\hyperlink{Isoproterenol Hydrochloride}{PMID: 29149854}, Alvise Tosoni et al., 2017]

\section*{Levophed}
\subsection*{Result}
\subsubsection*{Answer}

Unknown

\subsubsection*{{Explanation}}
\hypertarget{Levophed}
After reviewing all available abstracts, there are no studies specifically about Levophed (norepinephrine) in children. The abstracts provided cover a range of medications, including leuprolide, levetiracetam, lidocaine, levocetirizine, and others, but none address Levophed or its safety profile in pediatric populations. Therefore, based on the abstracts available, there is not enough data to determine whether Levophed is safe or unsafe for use in children of any age range.

\subsection*{Abstracts}
\hypertarget{pmid_24926950}{W}e have recently demonstrated short-term (6-month) efficacy and safety of leuprolide acetate 3-month depot 11.25 and 30 mg in children with central precocious puberty (CPP). To assess long-term (36-month) hypothalamic-pituitary-gonadal axis suppression and safety of leuprolide acetate 3-month depot 11.25 and 30 mg in children with CPP. Open-label, 36-month extension. Twenty pediatric endocrine centers. Seventy-two children (mean age, 8.5 ± 1.6 y; 65 females) with CPP completed and showed maintenance of LH suppression after a 6-month lead-in study. Leuprolide acetate depot (11.25 or 30 mg) administered im every 3 months. Peak-stimulated LH, estradiol, T, growth rate, pubertal progression, and adverse events (AEs). Twenty-nine of 34 subjects in the 11.25-mg group and 36 of 38 subjects in the 30-mg group had LH values < 4 mIU/mL after day 1 at all time points. All seven subjects who escaped LH suppression at any time still maintained sex steroid concentrations at prepubertal levels and showed no signs of pubertal progression. AEs were comparable between groups, with injection site pain being the most common (26.4\% overall). No AE led to discontinuation of study drug. The safety profile over 36 months was comparable to that observed during the 6-month pivotal study. The two doses of leuprolide acetate 3-month depot were associated with an acceptable safety profile and provided maintenance of LH suppression in the majority of children with CPP during the 36 months of the study or until readiness for puberty. [\hyperlink{Levophed}{PMID: 24926950}, Peter A Lee et al., 2014]

\hypertarget{pmid_17258474}{L}evetiracetam (LEV) is a novel antiepileptic drug (AED) that has recently obtained marketing authorisation for use in children. The purpose of this study was to assess the efficacy, tolerability and retention rate of LEV in children with refractory epilepsies. It is a retrospective multicentre observational study reporting the use of LEV in 200 children, aged 0.3-19 years (median 9-years-old) over a 4-year period. All of the patients included in the study had refractory epilepsy with a median age of onset of epilepsy of 3 years (range 0-13 years). The 38\% had failed and withdrawn 3 or more AEDs previously and 24\% were taking at least 2 other AEDs in addition to LEV. The 47\% had focal, and 58\% had symptomatic epilepsies. The LEV dose ranged from 8 to 100 mg/kg/day (mean 39 mg/kg). The study comprised 215 person years of LEV exposure. LEV was well tolerated with a retention rate of 49\% at 1 year. No serious adverse events were reported with possibly related adverse events reported in only 24\% of patients (mainly emotional or behavioural changes). At more than 2, 6 and 12 months, worthwhile improvement (>50\% seizure reduction) was noted in 60, 40 and 32\%, including seizure freedom in 14, 14 and 5\%, respectively. Our results confirm the efficacy and tolerability of LEV in children with refractory epilepsies and demonstrate good response and retention rates at 12 months. It represents the largest cohort of paediatric patients published so far on LEV with a 1-year follow-up. [\hyperlink{Levophed}{PMID: 17258474}, D Peake et al., 2007]

\hypertarget{pmid_17006857}{L}evetiracetam (LEV) is the latest drug approved in the European Union for use in polytherapy in children over 4 years of age with partial epileptic seizures that are resistant to other antiepileptic drugs. AIM. To report our experience of associating LEV in children with medication resistant epileptic seizures. We conducted an open, observational, respective study involving 133 children with refractory epilepsies: 106 with focal seizures and 27 with other types of seizures. LEV was associated over a period of more than 6 months and we evaluated its repercussion on the frequency of the seizures and the side effects related to the drug. With average doses of LEV of 1,192 +/- 749 mg/day the frequency of the seizures was reduced by over 50\% in 58.6\% of cases and seizures were quelled in 15.8\% of patients. Side effects were produced in 27.8\% of cases, and were usually transient or tolerable; these effects led to withdrawal of LEV in only eight cases (6.02\%). In 37 children (27.8\%), their relatives noted an improvement in their social behaviour and cognitive abilities. a) LEV is an effective drug that is well tolerated in children with refractory epilepsy; b) Its effectiveness in different types of seizures indicates a broad therapeutic spectrum; and c) LEV can even condition favourable secondary effects, a circumstance that has been reported only exceptionally in the case of other antiepileptic drugs. [\hyperlink{Levophed}{PMID: 17006857}, J L Herranz et al., ]

\hypertarget{pmid_14669138}{L}evetiracetam (LEV) is the latest antiepileptic drug (AED) to be marketed, and is indicated for use in association in adults with focal seizures. The purpose of this study is to report on our experience of administering LEV to children and adolescents with pharmacoresistant epilepsies. Retrospective open trial involving the observation of 43 children and adolescents with refractory epilepsies, using associated LEV for more than 6 months on an individual basis, the aim of which was to evaluate the repercussions on the frequency of the seizures, together with the adverse and beneficial side effects of LEV administration. With mean doses of LEV of 45.01 +/- 33.02 mg/kg/day the frequency of seizures was reduced by >50\% in 65\% of patients, while seizures were completely eradicated in 14\% of patients; adverse side effects were reported in 28\% of patients, although these were usually transient or tolerable, as LEV administration only had to be stopped for this reason in two cases (4.65\%). Relatives noted an improvement in social behaviour and in cognitive skills in the case of 15 children (34.9\%). 1. LEV is an effective drug that is well tolerated in children and adolescents with refractory epilepsies; 2. Its effectiveness in different types of seizures suggests a broad therapeutic spectrum; 3. LEV is a well tolerated drug with favourable side effects, a fact that is rarely reported with regard to other AED. [\hyperlink{Levophed}{PMID: 14669138}, J L Herranz et al., ]

\hypertarget{pmid_25972500}{T}his systematic review aimed to assess the safety and efficacy of antiretroviral options for postexposure prophylaxis (PEP). Recognizing the limited data on the safety and efficacy of antiretroviral drugs for PEP in children, this review was extended to include consideration of data on the use of antiretroviral drugs for treatment of infants and children living with human immunodeficiency virus. The PEP literature was assessed to identify studies reporting safety and completion rates for children given PEP, and this information was complemented by safety and efficacy data for drugs used in antiretroviral therapy. The proportion of patients experiencing each outcome was calculated and data were pooled using random-effects meta-analysis. Three prospective cohort studies reported outcomes of children given zidovudine (ZDV) plus lamivudine (3TC) as a 2-drug PEP regimen. The proportion of children completing the full 28-day course of PEP was 64.0\% (95\% confidence interval [CI], 41.2\%-86.8\%), whereas the proportion discontinuing due to adverse events was 4.5\% (95\% CI, .4\%-8.6\%). One randomized trial compared abacavir (ABC) plus lamivudine (3TC) and ZDV+3TC as part of a dual or triple first-line antiretroviral therapy regimen; this study showed better efficacy in the ABC-containing combinations and no difference in the time to first serious adverse event. Three randomized trials compared lopinavir/ritonavir (LPV/r) to nevirapine (NVP) for antiretroviral therapy and showed a lower risk of treatment discontinuations associated with LPV/r vs NVP (hazard ratio, 0.56 [95\% CI, .41-.75]) but no difference in drug-related adverse events. The overall quality of the evidence was rated as very low. This review supports ZDV+3TC+LPV/r as the preferred 3-drug regimen for PEP in children. [\hyperlink{Levophed}{PMID: 25972500}, Martina Penazzato et al., 2015]

\hypertarget{pmid_16193496}{T}he purpose of this study was to evaluate the safety and efficacy of single-isomer (R)-albuterol (levalbuterol, LEV) in children aged 2-5 years. Children aged 2-5 years (n = 211) participated in this multicenter, randomized, double-blind study of 21 days of t.i.d. LEV (0.31 mg or 0.63 mg without regard to weight), racemic albuterol (RAC, 1.25 mg for children <33 pounds (lb); 2.5 mg for children >/=33 lb), or placebo (PBO). Endpoints included adverse-event (AE) reporting, safety parameters, peak expiratory flow (PEF), the Pediatric Asthma Questionnaire(c) (PAQ), and the Pediatric Asthma Caregiver's Quality of Life Questionnaire (PACQLQ). Baseline disease severity was generally mild in all groups, as defined by PAQ scores that ranged from 6.3-7.3 on a scale of 0-27 and 1.5 days/week of uncontrolled asthma. After treatment, the PAQ decreased in all groups (P = NS). In the subset of subjects able to perform PEF (51.7\%), all active treatments improved in-clinic PEF after the first dose (mean +/- SD: PBO, 1.4 +/- 20.8; LEV 0.31 mg, 12.4 +/- 12; LEV 0.63 mg, 16.7 +/- 15.4; RAC, 18.0 +/- 16.5 l/min; P < 0.01). PACQLQ measurements improved more than the minimally important difference only in the LEV-treated groups, and were significant in children <33 lb (P < 0.05). Asthma exacerbations occurred primarily in children >/=33 lb, and one serious asthma exacerbation occurred in the 2.5-mg RAC group. RAC and LEV 0.63 mg, but not LEV 0.31 mg or placebo, led to significant increases in ventricular heart rate. In this study of levalbuterol in children aged 2-5 years with asthma, LEV was generally well-tolerated, and in children able to perform PEF, led to significant bronchodilation compared with placebo. [\hyperlink{Levophed}{PMID: 16193496}, David P Skoner et al., 2005]

\hypertarget{pmid_21094062}{N}eonatal seizures are common, especially in prematurity. Phenobarbital (PB) currently represents the antiepileptic drug (AED) of choice, despite being related to increased neuronal apoptosis in animal models and cognitive impairment in human subjects. Levetiracetam (LEV) may have a more favorable profile since it does not cause neuronal apoptosis in infant rodents. In a prospective feasibility study, LEV was applied as first-line treatment in 38 newborns with EEG-confirmed seizures, after ruling out hypoglycemia, hypocalcaemia, hypomagnesaemia and pyridoxin dependency. Initial intravenous doses of 10 mg/kg LEV were gradually increased to 30 mg/kg over 3 days with a further titration to 45-60 mg/kg at the end of the week. Acute intervention with up to 2 intravenous doses of PB 20 mg/kg was tolerated during LEV titration. LEV was switched to oral as soon as the infants' condition allowed. Based on clinical observation, EEG tracings (aEEG/routine EEGs), and lab data, drug safety and anticonvulsant efficacy were assessed over 12 months. In 19 newborns a single PB dose of 20 mg/kg was administered, while 3 newborns received 2 PB doses. 30 infants were seizure free under LEV at the end of the first week and 27 remained seizure free at four weeks, while EEGs markedly improved in 24 patients at 4 weeks. In 19 cases, LEV was discontinued after 2-4 weeks, while 7 infants received LEV up to 3 months. No severe adverse effects were observed. These results illustrate the safety of LEV treatment in neonatal seizures, including prematurity and suggest LEV anticonvulsant efficacy. Additional PB treatment admittedly constitutes a methodological shortcoming due to the prolonged anticonvulsive efficacy of PB. Double blind prospective controlled studies and long-term evaluation of cognitive outcome are called for. [\hyperlink{Levophed}{PMID: 21094062}, Georgia Ramantani et al., 2011]

\hypertarget{pmid_27011634}{T}o report the effectiveness and safety of intravenous (IV) levetiracetam (LEV) in the treatment of critically ill children with acute repetitive seizures and status epilepticus (SE) in a children's hospital. We retrospectively analyzed data from children treated with IV LEV. The mean age of the 108 children was 69.39 ± 46.14 months (1-192 months). There were 58 (53.1\%) males and 50 (46.8\%) females. LEV load dose was 28.33 ± 4.60 mg/kg/dose (10-40 mg/kg). Out of these 108 patients, LEV terminated seizures in 79 (73.1\%). No serious adverse effects were observed but agitation and aggression were developed in two patients, and mild erythematous rash and urticaria developed in one patient. Antiepileptic treatment of critically ill children with IV LEV seems to be effective and safe. Further study is needed to elucidate the role of IV LEV in critically ill children. [\hyperlink{Levophed}{PMID: 27011634}, Faruk Incecik et al., ]

\hypertarget{pmid_17204435}{T}o assess the efficacy, tolerability and safety of Levetiracetam (LEV) therapy, we identified 21 (15 male; 6 female) patients with a history of benign epilepsy with centrotemporal spikes (BECTS), with and without secondarily generalization in children and adolescents aged between 5.0 and 12.1 years. LEV was administered as a first drug (number of patients=9) or converted after previous treatment with other AEDs (number of patients=12). The patients were subdivided into two groups: "newly diagnosed" patients and "converted" patients. Patients were followed up for 12 months and all patients were able to continue on LEV treatment. At the end of follow-up (12 months), all patients were seizure free or showed a reduction of seizures >50\%. LEV dosage ranged from 1000 to 2500mg/daily. Overall, 100\% of patients completed the 12 months study, without any important side effect. Somnolence and irritability occurred in two (9.5\%) patients. Our results support findings that LEV monotherapy is effective and well tolerated in children with BECTS. Prospective, large, long-term double-blind studies are needed to confirm these findings. [\hyperlink{Levophed}{PMID: 17204435}, A Verrotti et al., 2007]

\hypertarget{pmid_11879369}{T}he pharmacokinetics of the novel antiepileptic drug (AED) levetiracetam and its major metabolite, ucb L057, were studied in children with partial seizures in a multicenter, open-label, single-dose study. Twenty-four children (15 boys, nine girls), 6 to 12 years old, received a single dose of levetiracetam (20 mg/kg) as an adjunct to their stable regimen of a single concomitant AED, followed by a 24-h pharmacokinetic evaluation. In children, the half-lives of levetiracetam and its metabolite ucb L057 were 6.0 +/- 1.1 and 8.1 +/-2.7 hours, respectively. The Cmax and area under the curve (AUC) of levetiracetam equated for a 1-mg/kg dose were lower in children (Cmax, norm=1.33 plus minus 0.35 microg/ml; AUCnorm=12.4 +/- 3.5 microg/h/ml) than in adults (Cmax, norm=1.38 +/- 0.05 microg/ml; AUCnorm=11.48 +/- 0.63 microg/h/ml), whereas the renal clearance was higher. The apparent body clearance (1.43 +/- 0.36 ml/min/kg) was approximately 30-40\% higher in children than in adults. Levetiracetam was generally well tolerated. On the basis of these data, a daily maintenance dose equivalent to 130-140\% of the usual daily adult maintenance dosage (1,000-3,000 mg/day) in two divided doses, on a weight-normalized level (mg/kg/day) is initially recommended. Clinical efficacy trials in children are ongoing with dosages of 20 to 60 mg/kg/day. [\hyperlink{Levophed}{PMID: 11879369}, J M Pellock et al., 2001]

\hypertarget{pmid_30679288}{T}herapeutic schedules for treating neonatal seizures remain elusive. First-line treatment with phenobarbital is widely supported but without strong scientific evidence. Levetiracetam (LEV) is an emerging and promising antiepileptic drug (AED). The aim of this phase II trial is to determine the benefits of LEV by applying a strict methodology and to estimate the optimal dose of LEV as a first-line AED to treat seizures in newborns suffering from hypoxic-ischaemic encephalopathy. LEVNEONAT-1 is an open and sequential LEV dose-finding study. The optimal dose is that which is estimated to be associated with a toxicity not exceeding 10\% and an efficacy higher than 60\%. Efficacy is defined by a seizure burden reduction of 80\% after the loading dose. Four increasing dose regimens will be assessed including one loading dose of 30, 40, 50 or 60 mg/kg followed by eight maintenance doses (ie, a quarter of the loading dose) injected every 8 hours. A two-patient cohort will be necessary at each dose level to consider an upper dose level assignment. The maximal sample size expected is 50 participants with a minimum of 24 patients or fewer in the case of a high rate of toxicity. Patients will be recruited in five neonatal intensive care units beginning in October 2017 and continuing for 2 years. In parallel, the LEV pharmacokinetics will be measured five times (ie, 30 min; 4 and 7 hours after the loading dose; 1-3 hours and 12-18 hours after the last maintenance dose). Ethics approval has been obtained from the regional ethical committee (2016-R25) and the French Drug Safety Agency (160652A-31). The results will be published in a peer-reviewed journal. The results will also be presented at medical meetings. NCT02229123; Pre-results. [\hyperlink{Levophed}{PMID: 30679288}, Geraldine Favrais et al., 2019]

\hypertarget{pmid_11916787}{I}n this randomized, double-blinded, placebo-controlled study, we evaluated the safety and efficacy of lidocaine iontophoresis for the prevention of pain associated with venipuncture in 59 children aged 6-17 yr. Children received either lidocaine HCl 2\% with epinephrine 1:100,000 (Active) or the same formulation without lidocaine (Placebo) via a 20 mA/min iontophoretic treatment. Pain during venipuncture was assessed by the subject, parent, and nurse using a 100-mm visual analog scale. Median (interquartile range) visual analog scale scores were significantly lower in the Active versus Placebo groups: subject, 7.0 (2.0-20.8) versus 31.0 (12.0-48.0), P < 0.001; nurse, 5.0 (2.2-10.8) versus 24.0 (9.0-47.0), P < 0.001; and parent, 3.0 (0.8-7.2) versus 20.0 (4.5-43.0), P < 0.002, respectively. Similarly, higher median satisfaction scores were given to the Active versus Placebo group by the three evaluators. Of the 59 subjects completing the study, 10 subjects experienced a total of 12 adverse events that were all graded as mild. In conclusion, lidocaine iontophoresis is safe in children, reduces discomfort associated with venipuncture, and increases satisfaction when compared with the placebo. In this randomized, double-blinded, placebo-controlled study, we found that dermal anesthesia with lidocaine HCl 2\% combined with epinephrine 1:100,000 administered via iontophoresis in children is achieved in 8.8 +/- 2.1 min, reduces discomfort associated with venipuncture, is safe, and increases satisfaction when compared with the placebo. [\hyperlink{Levophed}{PMID: 11916787}, John B Rose et al., 2002]

\hypertarget{pmid_17561929}{T}here are more than 40 H(1)-antihistamines available worldwide. Most of these medications have never been optimally studied in prospective, randomized, double-masked, placebo-controlled trials in children. The aim was to perform a long-term study of levocetirizine safety in young atopic children. In the randomized, double-masked Early Prevention of Asthma in Atopic Children Study, 510 atopic children who were age 12-24 months at entry received either levocetirizine 0.125 mg/kg or placebo twice daily for 18 months. Safety was assessed by: reporting of adverse events, numbers of children discontinuing the study because of adverse events, height and body mass measurements, assessment of developmental milestones, and hematology and biochemistry tests. The population evaluated for safety consisted of 255 children given levocetirizine and 255 children given placebo. The treatment groups were similar demographically, and with regard to number of children with: one or more adverse events (levocetirizine, 96.9\%; placebo, 95.7\%); serious adverse events (levocetirizine, 12.2\%; placebo, 14.5\%); medication-attributed adverse events (levocetirizine, 5.1\%; placebo, 6.3\%); and adverse events that led to permanent discontinuation of study medication (levocetirizine, 2.0\%; placebo, 1.2\%). The most frequent adverse events related to: upper respiratory tract infections, transient gastroenteritis symptoms, or exacerbations of allergic diseases. There were no significant differences between the treatment groups in height, mass, attainment of developmental milestones, and hematology and biochemistry tests. The long-term safety of levocetirizine has been confirmed in young atopic children. [\hyperlink{Levophed}{PMID: 17561929}, F Estelle R Simons et al., 2007]

\hypertarget{pmid_24756362}{A} pharmacokinetic substudy was conducted within a phase 3 clinical trial that evaluated the efficacy and safety of two leuprolide acetate 3-month depot formulations in children with central precocious puberty (CPP), where the pharmacokinetics of leuprolide and the exposure-response relationship between leuprolide concentration and the probability of luteinizing hormone (LH) suppression were assessed. Children diagnosed with CPP (N = 42 in each dosing cohort), who were treatment naïve or previously treated, received a total of two intramuscular injections of either leuprolide acetate depot 11.25 or 30 mg formulations administered 3 months apart. Serial blood samples were collected for leuprolide concentration determination in a subset of subjects (N = 24 in each cohort). One-way analysis of covariance was used to assess dose proportionality. The probability of LH suppression (peak-stimulated LH concentrations <4 mIU/mL) exposure-response relationship was modelled using repeated measures logistic regression. The predicted probability of LH suppression and the corresponding 95 \% confidence interval at the mean leuprolide concentration of each dose group and at each time of measurement were computed. Mean leuprolide concentrations between weeks 4 and 12 for 11.25 and 30 mg doses were relatively constant and dose proportional, with no accumulation of leuprolide upon repeated administration. Body weight and age were not found to be significant covariates on leuprolide pharmacokinetics. Higher leuprolide concentrations were associated with higher probability of LH suppression and both doses provided LH suppression levels <4 mIU/mL. Leuprolide pharmacokinetics were characterized for 11.25 and 30 mg 3-month depot injections. An exposure-response model was developed to link leuprolide concentrations and probability of peak-stimulated LH suppression. [\hyperlink{Levophed}{PMID: 24756362}, Nael M Mostafa et al., 2014]

\hypertarget{pmid_9165509}{T}he role of lamotrigine (LTG) in childhood epilepsy is emerging. We evaluated the efficacy and adverse effects of LTG in an open, prospective study of 56 children with generalized epilepsies. Six (11\%) children became seizure-free, and 24 (43\%) had greater than 50\% reduction in seizure frequency. LTG was effective against a broad range of generalized seizure types. Three of 15 patients with Lennox-Gastaut syndrome achieved complete seizure control and eight demonstrated 50 to 99\% improvement in seizure control. Increase in seizures (7) and rash (5) were the most common side effects. After valproate was discontinued, LTG therapy was resumed, with no recurrence of rash in any patient. This study suggests that LTG may be a useful drug in the treatment of generalized epilepsies in children. [\hyperlink{Levophed}{PMID: 9165509}, K Farrell et al., 1997]

\hypertarget{pmid_19740527}{E}noxaparin, a low molecular weight heparin (LMWH), is frequently used for the prevention and treatment of thromboembolic complications in infants and children (Sutor et al., 2004 [1]). Injection pain and the fear and anxiety associated with needle phobia in the pediatric population are well documented. Best practice pediatric pain management standards of care recommend mitigating the child's pain experience whenever possible. The use of topical anesthetics such as liposomal-lidocaine 4\% results in a rapid onset of anesthesia, minimal blanching, without vasoconstriction (Koh et al., 2004 [2]) or risk of methemoglobinemia. Topical lidocaine has been used to reduce the injection pain of enoxaparin, but there is no data available examining whether it will interfere with the absorption of LMWH. To determine if the topical lidocaine, Maxilene, interferes with enoxaparin absorption as measured by peak anti-Xa levels. Infants and children clinically prescribed enoxaparin were eligible for this study. Children in group 1 were pre-treated with Maxilene prior to enoxaparin injection on day 1 with no Maxilene pre-treatment on day 2. For group 2, the order was reversed. Peak anti-Xa levels were measured following each enoxaparin dose and were compared between the groups. 26 children of ages 14d-16 y (median 6.7 months) were enrolled. Anti-Xa levels following topical lidocaine administration were 0.070 U/mL (95\%CI 0.025; 0.114) lower than without prior topical lidocaine administration. Anti-Xa levels on the second day were on average 0.013 U/mL (95\%CI -0.066; 0.040) higher compared to day one regardless of the order of topical lidocaine administration. There were no reported incidences of local reactions such as redness, hives or blanching. Topical lidocaine (Maxilene) administration before enoxaparin injection results in a small, clinically non-significant, reduction in anti-Xa levels. [\hyperlink{Levophed}{PMID: 19740527}, S M Duncan et al., 2010]

\hypertarget{pmid_37287398}{S}eizures are common in critically ill children and neonates, and these patients would benefit from intravenous (IV) antiseizure medications with few adverse effects. We aimed to assess the safety profile of IV lacosamide (LCM) among children and neonates. This retrospective multicenter cohort study examined the safety of IV LCM use in 686 children and 28 neonates who received care between January 2009 and February 2020. Adverse events (AEs) were attributed to LCM in only 1.5\% (10 of 686) of children, including rash (n = 3, .4\%), somnolence (n = 2, .3\%), and bradycardia, prolonged QT interval, pancreatitis, vomiting, and nystagmus (n = 1, .1\% each). There were no AEs attributed to LCM in the neonates. Across all 714 pediatric patients, treatment-emergent AEs occurring in >1\% of patients included rash, bradycardia, somnolence, tachycardia, vomiting, feeling agitated, cardiac arrest, tachyarrhythmia, low blood pressure, hypertension, decreased appetite, diarrhea, delirium, and gait disturbance. There were no reports of PR interval prolongation or severe cutaneous adverse reactions. When comparing children who received a recommended versus a higher than recommended initial dose of IV LCM, there was a twofold increase in the risk of rash in the higher dose cohort (adjusted incidence rate ratio = 2.11, 95\% confidence interval = 1.02-4.38). This large observational study provides novel evidence demonstrating the tolerability of IV LCM in children and neonates. [\hyperlink{Levophed}{PMID: 37287398}, Susan L Fong et al., 2023]

\hypertarget{pmid_29179233}{S}eizures are the most common neurological complication in neonatal intensive care units. Phenobarbital (PB) remains the first-line antiepileptic drug (AED) for neonatal seizures despite known neurotoxicity. Levetiracetam (LEV) is a newer AED not approved for neonates. Retrospective and pilot studies have investigated the use of LEV in neonatal seizures. Our objective was to compare the efficacy of LEV to PB in neonatal seizures based upon published data. We searched PubMed to perform a systematic review. We found no studies of LEV with comparison or control groups; therefore, we utilized data from two randomized controlled trials of PB as our comparison group. Five studies of LEV met all inclusion/exclusion criteria. The pooled sample size for LEV was 102 (48 received primary LEV, 54 received secondary LEV). The pooled sample size for primary PB was 52. Complete or near-complete seizure cessation was achieved as follows: primary LEV 37/48 (77\%), secondary LEV 34/54 (63\%), and primary PB 24/52 (46\%). Our findings suggest that LEV may be at least as or more effective for neonatal seizures as PB. Our review, though limited, is the first to examine LEV efficacy compared with PB in neonates. [\hyperlink{Levophed}{PMID: 29179233}, Daryl C McHugh et al., 2018]

\hypertarget{pmid_26013703}{L}evetiracetam, a second-generation anti-epileptic drug (AED) with a good efficacy and safety profile, is licensed as monotherapy for adults and children older than 16 years with focal seizures with or without secondary generalization. However, it is increasingly being used off-label in younger children. We critically reviewed the available evidence and discuss the present status of levetiracetam monotherapy in children 0-16 years old. We systematically searched the literature using PubMed, Web of Science and Embase up to August 2014 for articles on levetiracetam monotherapy in children. Keywords were levetiracetam, monotherapy and child*. The titles and abstracts of 532 articles were evaluated by AW, of which 480 were excluded. The full texts of the other 52 articles were assessed for relevance. We covered one review, one opinion statement and 32 studies in this review, including four randomized controlled trials, ten open-label prospective studies, eight retrospective studies, and ten case reports. The formal evidence for levetiracetam monotherapy in children is minimal: it is potentially efficacious or effective as initial monotherapy in children with benign epilepsy with centrotemporal spikes. In all of the published studies, however, efficacy and tolerability of levetiracetam seemed to be good and comparable to other AEDs. The data of 32 studies on levetiracetam monotherapy in children were insufficient to confirm that levetiracetam is effective as initial monotherapy for different types of seizures and/or epilepsy syndromes. There is still an urgent need for well designed trials to justify the widespread use of levetiracetam monotherapy in children of all ages. [\hyperlink{Levophed}{PMID: 26013703}, Amerins Weijenberg et al., 2015]

\hypertarget{pmid_26876768}{T}here are limited data on the use of the antiepileptic drug (AED) levetiracetam for the treatment of infants. To prospectively evaluate the safety of levetiracetam oral solution and its impact on epilepsy severity in infants with different seizure types. This noninterventional post-authorization safety study included patients 1-11 months of age. Patients' treatment - levetiracetam dose, and addition, withdrawal or changes in the doses of concomitant medications and AEDs - was at the discretion of the physician. The primary variable was treatment-emergent adverse events (TEAEs). Of 101 infants, 75 completed and 26 discontinued the study. Mean age was 6.0 months, 50 were male, most (80\%) took 1 ≥ concomitant AED and had cryptogenic or symptomatic epilepsy that was focal (38.6\%) or generalized (20.8\%), particularly frontal lobe epilepsy (20.0\%) or West syndrome/infantile spasms (20.0\%). Among known aetiologies, congenital factors (22.8\%) such as dysplastic lesions or perinatal events (17.8\%) were predominant. Overall, 54.5\% of patients had ≥ 1 TEAE. Five patients experienced drug-related TEAEs - convulsion, irritability, somnolence and hypotonia, all listed in the product label, with the exception of hypotonia, which was reported for one patient and resolved without any change in study medication. Seven patients discontinued due to TEAEs, mainly due to infantile spasms and respiratory disorders. At study end, 71.8\% of patients showed improvement in epilepsy severity, 18.8\% remained stable and 9.4\% showed worsening. Levetiracetam did not appear to have a negative effect on growth parameters. In this prospective study, which included the largest number of patients in this age range so far, levetiracetam was found to be well tolerated and efficacious for the treatment of infants with epilepsy. [\hyperlink{Levophed}{PMID: 26876768}, Alexis Arzimanoglou et al., 2016]

\hypertarget{pmid_37023853}{L}evetiracetam (Lev) is an antiepileptic drug that has been increasingly used in the epilepsy pediatric population in recent years, but its pharmacokinetic behavior in pediatric population needs to be characterized clearly. Clinical trials for the pediatric drug remain difficult to conduct due to ethical and practical factors. The purpose of this study was to use the physiologically based pharmacokinetic (PBPK) model to predict changes in plasma exposure of Lev in pediatric patients and to provide recommendations for dose adjustment. A PBPK model of Lev in adults was developed using PK-Sim® software and extrapolated to the entire age range of the pediatric population. The model was evaluated using clinical pharmacokinetic data. The results showed the good fit between predictions and observations of the adult and pediatric models. The recommended doses for neonates, infants and children are 0.78, 1.67 and 1.22 times that of adults, respectively. Moreover, at the same dose, plasma exposure in adolescents was similar to that of adults. The PBPK models of Lev for adults and pediatrics were successfully developed and validated to provide a reference for the rational administration of drugs in the pediatric population. [\hyperlink{Levophed}{PMID: 37023853}, Wenxin Shao et al., 2023]

\hypertarget{pmid_24904873}{W}e evaluated the efficacy, safety and psychological aspect of monthly administrations of the gonadotropin-releasing hormone agonists (GnRHa), leuprolide acetate depot (Luphere depot 3.75 mg), in patients with precocious puberty. A total of 54 girls with central precocious puberty were administered with leuprolide acetate (Luphere depot 3.75 mg) every four weeks over 24 weeks. We evaluated the percentage of children exhibiting a suppressed luteinizing hormone (LH) response to GnRH (LH peak≤3 IU/L), peak LH/follicle stimulating hormone (FSH) ratio of GnRH stimulation test less than 1, change in bone age/chronologic age ratio, change in the Tanner stage and change in eating habit and psychological aspect. (1) The percentage of children exhibiting a suppressed LH response to GnRH, defined as an LH peak≤3 IU/L at 24 weeks was 96.3 \% (52/54). (2) The percentage of children exhibiting peak LH/FSH ratio<1 at 24 weeks of the study was 94.4 \% (51/54). (3) The ratio of bone age and chronological age significantly declined from 1.27±0.07 to 1.24±0.01 after the 6 months of the study. (4) The mean Tanner stage manifested a significant change 2.3±0.48 at baseline, down to 1.70±0.61 at 24 weeks. (5) Based on the questionnaires, the score for eating habits showed a significant change from the baseline 34.0±6.8 to 31.3±6.8. (6) The psychological assessment did not exhibit a significant difference except with scores for sociability, problem behavior total score and other problems. The leuprolide 3.75 mg (Luphere depot) is useful and safety for treating children with central precocious puberty. [\hyperlink{Levophed}{PMID: 24904873}, You Jin Kim et al., 2013]

\hypertarget{pmid_37008014}{I}nfants exposed to opioids  [\hyperlink{Levophed}{PMID: 37008014}, Sarah A Beyeler et al., 2023] Studies on the efficacy and tolerability of rufinamide in infants and young children are scarce. Here we report on an open, retrospective, and pragmatic study about safety and efficacy of rufinamide in children aged less than four years, in terms of seizures types and epilepsy syndromes. Forty children (mean age 39.5 months; range 22-48) were enrolled in the study. The mean follow-up period was 12.2 months (range 5-21). Rufinamide was initiated at a mean age of 26.7 months (range 12-42). Final rufinamide mean dosage was 31.5 mg/kg/day if associated with valproic acid and 44.2 mg/kg/day if not. The highest seizure reduction rate was observed in the epileptic spasms (46\%) and drop attacks (42\%) groups. Seizure reduction was also observed in tonic seizures (35\%) and in the focal seizure (30\%) groups. In terms of epilepsy syndrome, rufinamide was effective in Lennox-Gastaut syndrome. Results were very poor for those affected by Dravet's syndrome. Globally, responder rate was 27.5\%, including two (5\%) patients seizure-free. Adverse reactions occurred in 37.5\% of children and were mainly represented by vomiting, drowsiness, irritability, and anorexia. Discontinuation rate due to treatment-emergent adverse events was 15\%. The present study concludes that rufinamide may be a safe and effective drug for a broad range of seizures and epilepsy syndromes in infants and young children and represents a valid therapeutic option in this population. [\hyperlink{Levophed}{PMID: 37008014}, Salvatore Grosso et al., 2014]

\hypertarget{pmid_19325512}{I}ntravenous (IV) levetiracetam (LEV) is approved for use in patients older than 16 years and may be useful in critically ill children, although there is little data available regarding pharmacokinetics. We aim to investigate the safety, an appropriate dosing, and efficacy of IV LEV in critically ill children. We describe a cohort of critically ill children who received IV LEV for status epilepticus, including refractory or nonconvulsive status, or acute repetitive seizures. There were no acute adverse effects noted. Children had temporary cessation of ongoing refractory status epilepticus, termination of ongoing nonconvulsive status epilepticus, cessation of acute repetitive seizures, or reduction in epileptiform discharges with clinical correlate. IV LEV was effective in terminating status epilepticus or acute repetitive seizures and well tolerated in critically ill children. Further study is needed to elucidate the role of IV LEV in critically ill children. [\hyperlink{Levophed}{PMID: 19325512}, Nicholas S Abend et al., 2009]

\section*{Methylergonovine Maleate}
\subsection*{Result}
\subsubsection*{Answer}

Unknown

\subsubsection*{{Explanation}}
\hypertarget{Methylergonovine Maleate}
A review of the available abstracts reveals the following evidence regarding the safety of Methylergonovine Maleate in children:

1. One abstract describes a prospective, controlled observational study of mothers treated with methylergonovine while breastfeeding, comparing outcomes in their infants (aged 1-3 years at follow-up) to a control group. The study found no effect of methylergonovine on lactation, no difference in neonatal complications, and no increase in adverse long-term outcomes in infants exposed to methylergonovine through breastfeeding. The authors conclude that their data support the continuation of breastfeeding in women requiring methylergonovine treatment [\hyperlink{pmid_27846760}{PMID: 27846760}, Oded Gilad et al., 2017]. However, this study only addresses indirect exposure (via breast milk), not direct administration to children.

2. Another abstract reports a case of accidental direct administration of methylergonovine to a term neonate (0-28 days old). The infant developed oxygen desaturation requiring supplemental oxygen for several hours but was discharged in good condition after 72 hours without further complications. The authors note that ergot alkaloids in neonates can cause serious adverse events, including respiratory depression, seizures, and death, and emphasize the need for vigilance to prevent accidental administration [\hyperlink{pmid_27765939}{PMID: 27765939}, Bryan M Corbett et al., 2016]. This is a single case report and does not constitute a targeted safety study.

3. No other abstracts provide targeted studies evaluating the safety of methylergonovine maleate in children (by direct administration) for any age group.

Summary by age range:
- Neonates (0-28 days): No targeted safety studies; a case report describes adverse effects after accidental administration.
- Infants and children (1 month to 3 years): One observational study of indirect exposure via breast milk found no adverse effects, but this does not establish safety for direct use.
- Older children: No data available.

Based on the abstracts, there are no targeted studies affirming the safety or unsafety of methylergonovine maleate for direct use in children of any age group. Therefore, the safety of methylergonovine maleate in children is unknown.

\subsection*{Abstracts}
\hypertarget{pmid_27846760}{T}o evaluate maternal and breastfed infant's outcome following post-partum maternal use of methylergonovine. A prospective, controlled observational study design was used. Mothers who contacted Beilinson Teratology Information Service (BELTIS) were followed by phone interview. Data on lactation, neonatal symptoms and outcomes at the age of 1-3 years were obtained. Mothers' breastfeeding while treated with methylergonovine and their infants were compared to a matched control group of breastfeeding mothers using a drug known to be safe during lactation (amoxicillin). Follow-up was obtained for 38 of 42 women (90.5\%). Of whom, six stopped breastfeeding because of concerns regarding drug treatment and three refused to participate. The remaining 29 women and infant pairs were compared to a control group of 58 women and their infants. Comparison showed no effect of methylergonovine on lactation and similarly showed no difference in rate of neonatal complications (p = 1). At time of follow-up there were no differences in growth or in adverse neurodevelopment outcomes (p = 0.26). No increase in adverse long-term outcomes was found in infants exposed to methylergonovine through breastfeeding. Our data in conjunction with previous estimates of very low drug exposure support continuation of breastfeeding in women requiring treatment with methylergonovine. [\hyperlink{Methylergonovine Maleate}{PMID: 27846760}, Oded Gilad et al., 2017]

\hypertarget{pmid_27765939}{B}ACKGROUND Methylergonovine is an ergot alkaloid used to treat post-partum hemorrhage secondary to uterine atony. Mistaking methylergonovine for vitamin K with accidental administration to the neonate is a rare iatrogenic illness occurring almost exclusively in the delivery room setting. Complications of ergot alkaloids in neonates include respiratory depression, seizures, and death. CASE REPORT A term infant was inadvertently given 0.1 mg of methylergonovine intramuscularly in the right thigh. The error was only noted when the vial of medication was scanned, after administration, identifying it as methylergonovine rather than vitamin K. The local poison center was notified, and the infant was transferred to the neonatal intensive care unit for observation. Two hours after transfer, the infant was noted to have oxygen desaturations and required oxygen via nasal cannula. Supplemental oxygen was continued for 4 hours until the neonate was able to maintain normal oxygen saturations in room air. Feeding was started by 10 hours of life, and the infant was discharged home in good condition after a 72-hour stay without further complications. CONCLUSIONS Because of the potential for serious adverse events, vigilance is required to prevent accidental administration of methylergonovine to the neonate as a result of possible confusion with vitamin K in the early post-partum period. [\hyperlink{Methylergonovine Maleate}{PMID: 27765939}, Bryan M Corbett et al., 2016]

\hypertarget{pmid_26861518}{I}nfantile hemangioma is the most common benign vascular tumor of childhood that has a tendency for spontaneous involution. The aim of this study was to evaluate the efficacy of topical timolol maleate in the treatment of superficial infantile hemangioma and associated side effects during the course of treatment. Four boys and five girls with a median age of 5 months were reviewed at 2-week intervals for a period of 16 weeks. A decrease in size, color, and consistency were noted. Adverse effects caused by timolol maleate were noted and managed. Of nine cases, two patients showed excellent response, five showed good response, one showed partial response, and one had poor response. Topical timolol maleate is safe and effective in the treatment of infantile hemangioma.  [\hyperlink{Methylergonovine Maleate}{PMID: 26861518}, Abhijeet Kumar Jha et al., ] The antimetabolite methotrexate has been shown in placebo-controlled trials to be effective in adults with rheumatoid arthritis. Methotrexate may also be effective in children with resistant juvenile rheumatoid arthritis, but the supporting data are from uncontrolled trials. Centers in the United States and the Soviet Union participated in this randomized, controlled, double-blind trial designed to evaluate the effectiveness and safety of orally administered methotrexate. Patients received one of the following treatments each week for six months: 10 mg of methotrexate per square meter of body-surface area (low dose), 5 mg of methotrexate per square meter (very low dose), or placebo. The use of prednisone (less than or equal to 10 mg per day) and two nonsteroidal antiinflammatory drugs was also allowed. The 127 children (mean age, 10.1 years) had a mean duration of disease of 5.1 years; 114 qualified for the analysis of efficacy. According to a composite index of several response variables, 63 percent of the children who received low-dose methotrexate improved, as compared with 32 percent of those in the very-low-dose group and 36 percent of those in the placebo group (P = 0.013). As compared with the placebo group, the low-dose group also had significantly larger mean reductions from base line in the number of joints with pain on motion (-11.0 vs. -7.1), the pain-severity score (-19 vs. -11.5), the number of joints with limited motion (-5.4 vs. -0.7), and the erythrocyte sedimentation rate (-19.0 vs. -6 mm per hour). In the methotrexate groups only three children had the drug discontinued because of mild-to-moderate side effects; none had severe toxicity. Methotrexate given weekly in low doses is an effective treatment for children with resistant juvenile rheumatoid arthritis, and at least in the short term this regimen is safe. [\hyperlink{Methylergonovine Maleate}{PMID: 26861518}, E H Giannini et al., 1992]

\hypertarget{pmid_27588127}{T}he aim of the present study was to assess the efficacy and safety of topical timolol maleate combined with oral propranolol for parotid infantile hemangiomas. Between October 2012 and April 2014, propranolol was administered orally at a dose of 1.0-1.5 mg/kg/day to 22 infants with proliferating hemangiomas in the Department of Oral and Maxillofacial Surgery (Hospital of Stomatology, China Medical University, Shenyang, Liaoning, China). A small amount of 0.5\% timolol maleate eye drop solution was topically applied with medical cotton swabs to the area of the lesion twice a day, every 12 h. The study group consisted of 9 males and 13 females, aged 2-9 months, with a median age of 4.7 months. The lesions were all located in the parotid region, and measured between 3.5×4×0.5 and 7×8×3 cm in volume. The planned duration of therapy was 6-8 months, or the two drugs were stopped when complete regression of the lesions was obtained. The therapeutic outcomes and safety were assessed by the change in the size and color of the tumor, and the presence of adverse effects throughout the course of treatment. The mean duration of therapy was 21.1 weeks and ranged from 3 to 8 months. Of the 22 patients, 16 demonstrated an excellent response, 6 showed a good response and 2 displayed a moderate response. No major collateral effects were observed. Overall, oral propranolol combined with topical timolol maleate may be used as the first-line therapeutic choice in the treatment of infantile parotid mixed hemangioma. [\hyperlink{Methylergonovine Maleate}{PMID: 27588127}, Shuang Tong et al., 2016]

\hypertarget{pmid_30465439}{T}o investigate by meta-analysis the efficacy of gelatin tannate (GT), a mucosal barrier protector, in children with acute gastroenteritis. A comprehensive literature search was conducted. Studies were selected according to PICO: Participants: children aged 0-12 years with acute diarrhea; Intervention: GT; Comparison: oral rehydration solution and/or placebo; Outcomes: diarrhea-related outcomes. Three published randomized controlled trials were identified of pediatric diarrhea treated with GT (n = 203) or control (n = 204). GT significantly (p < 0.01) reduced stool frequency at 12 h in two randomized controlled trials. A significant treatment effect (risk ratio = 0.74; p < 0.01) in favor of GT was found for the exploratory composite outcome of 'diarrhea or liquid stools at 24 h' in three studies. Risk ratios in a single study which reported the percentage of patients with liquid stools at 12, 24 and 48 h favored GT at all time points. No significant differences were found between GT and control for patients with diarrhea at 12 or 24 h or for duration of diarrhea. GT improved stool frequency and stool consistency in children with acute diarrhea, although further well-controlled studies would be useful to confirm a beneficial treatment effect. [\hyperlink{Methylergonovine Maleate}{PMID: 30465439}, Marina Aloi et al., 2019]

\hypertarget{pmid_22238470}{C}hildren have a lower response rate to antimonial drugs and higher elimination rate of antimony (Sb) than adults. Oral miltefosine has not been evaluated for pediatric cutaneous leishmaniasis. A randomized, noninferiority clinical trial with masked evaluation was conducted at 3 locations in Colombia where Leishmania panamensis and Leishmania guyanensis predominated. One hundred sixteen children aged 2-12 years with parasitologically confirmed cutaneous leishmaniasis were randomized to directly observed treatment with meglumine antimoniate (20 mg Sb/kg/d for 20 days; intramuscular) (n = 58) or miltefosine (1.8-2.5 mg/kg/d for 28 days; by mouth) (n = 58). Primary outcome was treatment failure at or before week 26 after initiation of treatment. Miltefosine was noninferior if the proportion of treatment failures was ≤15\% higher than achieved with meglumine antimoniate (1-sided test, α = .05). Ninety-five percent of children (111/116) completed follow-up evaluation. By intention-to-treat analysis, failure rate was 17.2\% (98\% confidence interval [CI], 5.7\%-28.7\%) for miltefosine and 31\% (98\% CI, 16.9\%-45.2\%) for meglumine antimoniate. The difference between treatment groups was 13.8\%, (98\% CI, -4.5\% to 32\%) (P = .04). Adverse events were mild for both treatments. Miltefosine is noninferior to meglumine antimoniate for treatment of pediatric cutaneous leishmaniasis caused by Leishmania (Viannia) species. Advantages of oral administration and low toxicity favor use of miltefosine in children. NCT00487253. [\hyperlink{Methylergonovine Maleate}{PMID: 22238470}, Luisa Consuelo Rubiano et al., 2012]

\hypertarget{pmid_23850529}{T}he purpose of this study was to examine the risks of acute coronary syndrome (ACS) and acute myocardial infarction (AMI) that are associated with methylergonovine maleate (Methergine; Novartis Pharmaceuticals Corporation, Plantation, FL) use in a large database of inpatient delivery admissions in the United States. We conducted a retrospective cohort study using data from the Premier Perspective Database and identified 2,233,630 women who were hospitalized for delivery between 2007 and 2011 (approximately one-seventh of all US deliveries during this period). Exposure was defined by a charge code for methylergonovine during the delivery hospitalization. Study outcomes included ACS and AMI. Propensity score matching was used to address potential confounding. Methylergonovine was administered to 139,617 patients (6.3\%). Overall, 6 patients (0.004\%) who were exposed to methylergonovine and 52 patients (0.002\%) who were not exposed to methylergonovine had an ACS. Four patients (0.003\%) who were exposed to methylergonovine and 44 patients (0.002\%) in the not-exposed group had an AMI. After propensity score matching, the relative risk for ACS that was associated with methylergonovine exposure was 1.67 (95\% confidence interval [CI], 0.40-6.97), and the risk difference was 1.44 per 100,000 patients (95\% CI, -2.56 to 5.45); the relative risk for AMI that was associated with methylergonovine exposure was 1.00 (95\% CI, 0.20-4.95), and the risk difference was 0.00 per 100,000 patients (95\% CI, -3.47 to 3.47). Despite studying a very large proportion of US deliveries, we did not find a significant increase in the risk of ACS or AMI in women who received methylergonovine compared with those who did not; estimates were increased only modestly or not at all. The upper limit of the 95\% CI of our analysis suggests that treatment with methylergonovine would result in no more than 5 additional cases of ACS and 3 additional cases of AMI per 100,000 exposed patients. [\hyperlink{Methylergonovine Maleate}{PMID: 23850529}, Brian T Bateman et al., 2013]

\hypertarget{pmid_22187405}{W}ithin the last two decades low molecular weight heparins (LMWH) have gained increasing widespread use as anticoagulants in children. The use of LMWH has been implemented into clinical care even though there is a lack of firm evidence on the efficacy and safety of LMWH in this population due to the absence of sufficiently powered randomized controlled trials. In the absence of clinical trials, we performed a meta-analysis of available single-arm studies using LMWH in children. A systematic search of electronic databases (Medline, EMBASE, OVID, Web of Science, The Cochrane Library) for studies published from 1980 to 2010 was conducted using keywords in combination both as MeSH terms and text words. Two authors independently screened citations and those meeting a priori defined inclusion criteria were retained. Data on year of publication, study design, country of origin, number of patients, ethnicity, venous thromboembolic events type, and frequency of recurrence and major bleedings were abstracted. Pooled incidence rates (IR) including 95\% confidence intervals (95\% CIs) on efficacy and safety data of LMWH administration on primary prophylaxis, as well as on secondary prophylaxis in children following symptomatic thromboembolism (TE) were shown. We included 2251 pediatric patients derived from 35 single-arm studies from 12 study countries who were eligible for analysis in the present systematic review. Pooled incidence rates (95\% CI) to develop first TE on primary prophylaxis, further TE event on LMWH secondary prophylaxis, or a major bleeding event on LMWH were 0.047 (0.023 to 0.091), 0.052 (0.037 to 0.073) for efficacy, and 0.054 (0.039 to 0.074) for safety (treatment data only), respectively. Efficacy and safety data are comparable with adult data. The present systematic review suggests that use of LMWH in children as primary prophylaxis and in treatment of symptomatic thrombosis is effective and safe. However, properly designed randomized controlled trials are needed. [\hyperlink{Methylergonovine Maleate}{PMID: 22187405}, Christoph Bidlingmaier et al., 2011]

\hypertarget{pmid_31573668}{P}rilocaine/lidocaine is widely used as local anesthetic in children for cannulation and minor surgical procedures. Usually it is unproblematic but it is important to adhere to recommended dose to avoid serious complications. Excessive amount of prilocaine/lidocaine, large application area, prolonged application time or repeated application can, especially in infants, cause methemoglobinemia with clinical symptoms. In severe cases intensive care and antidote treatment with Methylene blue may be required. We report three infants who were overdosed with prilocaine/lidocaine, two of them due to incorrect use after circumcision and one premature baby where prilocaine/lidocaine was not removed in time. Two of the babies had MetHb levels > 33\% and were seriously affected with hypoxia, tachycardia and fatigue. After Methylene blue was given the infants recovered within 15 minutes and MetHb levels returned to normal. [\hyperlink{Methylergonovine Maleate}{PMID: 31573668}, Cornelia Kjellgard et al., 2019]

\hypertarget{pmid_32453920}{S}hock refractory to fluid and catecholamine therapy has significant morbidity and mortality in children. The use of methylene blue to treat refractory shock in children is not well described. We aim to collect and summarize the literature and define physicians' practice patterns regarding the use of methylene blue to treat shock in children. We conducted a systematic search of MEDLINE, Embase, PubMed, Web of Science, Cochrane for studies involving the use of methylene blue for catecholamine-refractory shock from database inception to 2019. Collected studies were analyzed qualitatively. To describe practice patterns of methylene blue use, we electronically distributed a survey to U.S.-based pediatric critical care physicians. We assessed physician knowledge and experience with methylene blue. Survey responses were quantitatively and qualitatively evaluated. Pediatric critical and cardiac care units. Patients less than or equal to 25 years old with refractory shock treated with methylene blue. None. One-thousand two-hundred ninety-three abstracts met search criteria, 139 articles underwent full-text review, and 24 studies were included. Studies investigated refractory shock induced by a variety of etiologies and found that methylene blue was generally safe and increased mean arterial blood pressure. There is overall lack of studies, low number of study patients, and low quality of studies identified. Our survey had a 22.5\% response rate, representing 125 institutions. Similar proportions of physicians reported using (40\%) or never even considering (43\%) methylene blue for shock. The most common reasons for not using methylene blue were unfamiliarity with this drug, its proper dosing, and lack of evidentiary support. Methylene blue appears safe and may benefit children with refractory shock. There is a stark divide in familiarity and practice patterns regarding its use among physicians. Studies to formally assess safety and efficacy of methylene blue in treating pediatric shock are warranted. [\hyperlink{Methylergonovine Maleate}{PMID: 32453920}, Andrea V Otero Luna et al., 2020]

\hypertarget{pmid_35176737}{T}he scientific evidence of methotrexate (MTX) in children with severe plaque psoriasis is scarce. To retrospectively evaluate the efficacy and safety of oral MTX in children with severe plaque psoriasis in a single center in China. We enrolled 42 children with severe plaque psoriasis who were administrated MTX. Efficacy was evaluated by the psoriasis area and severity index (PASI) score, physician global assessment (PGA) score, and body surface area (BSA) score. The Children's Dermatology Life Quality Index (CDLQI) score and safety data were recorded. Among 42 children (22 males, 20 females), the mean age was 11.2 years old. The initial weight-based dosage of oral MTX ranged from 0.1 to 0.3 mg/kg weekly. Overall, 80.6 and 47.2\% of patients achieved PASI75 (at least 75\% improvement from baseline in PASI score) and PASI90 (at least 90\% improvement from baseline in PASI score) at week 12, respectively. 72.2\% of patients achieved PGA 0/1 at week 12. BSA and PGA scores significantly decreased from baseline from week 4, accompanied by CDLQI score improvement from week 8. The steady effect of MTX could be reached at week 16. Elevated liver enzymes (28.6\%) and infections (28.6\%) were the most common side effects. Relapse was recorded in 9 (30.0\%) of 30 patients, and the mean posttherapy disease-free interval was 7.2 months. MTX is an effective and safe option for children with severe plaque psoriasis with adequate monitoring. [\hyperlink{Methylergonovine Maleate}{PMID: 35176737}, Zhaoyang Wang et al., 2022]

\hypertarget{pmid_32292451}{T}he role of methyl prednisolone in longitudinal extensive transverse myelitis in children is not completely discovered in developing country like Pakistan. So this is the first study which aimed to evaluate the efficacy of methyl prednisolone in longitudinal extensive transverse myelitis in children. This is quasi experimental hospital based descriptive prospective study. The data was collected from 34 children admitted in Paediatric Neurology department through Outpatient/emergency department in Children's Hospital and the Institute of Child Health, Lahore for period of one year from January 2018 to December 2018. The children full filling the inclusion criteria were observed before and after giving injection methyl prednisolone 30mg/kg/dose (maximum dose one Gram irrespective of the body weight) once daily for five days in the form of intravenous infusion. Complete recovery was seen in 41.2\% while 58.8\% showed partial recovery. The correlation of response to treatment (recovery) with gender, area of spinal cord involvement, muscle power and autonomic dysfunction is found at significance level of five percent according to Chi square test. Early consideration and administration of methyl prednisolone in longitudinally extensive transverse myelitis in children can be beneficial and can help to reduce the morbidity. [\hyperlink{Methylergonovine Maleate}{PMID: 32292451}, Muhammad Azeem Ashfaq et al., ]

\hypertarget{pmid_16284689}{M}ethotrexate (MTX) remains a mainstay in the treatment of children with hematological malignancies. The availability of an antidote/rescue agent, leucovorin (LV) has allowed escalation of MTX doses to achieve enormous plasma concentrations, compared with plasma folate. However, a recent review of more than 40 trials for children with ALL concluded that the addition of high dose MTX (HDMTX) in many different doses and schedules did not improve CNS therapy and made only minor improvements in systemic therapy for children with ALL [11]. Some assessment suggested that by HDMTX benefits only limited amount of children with ALL. Recent treatment schedules vary markedly in terms of timing, dosing and scheduling of MTX and/or leukovorin, which may leave us uncertain with ideas such as "how should we best use HDMTX and LV?" or "why are we still using such by industry recommended doses of MTX?" The answer of how best to incorporate HDMTX and/or LV into ALL treatment plans is still not known and further clinical and pharmacological studies dealing with still controversial systemic MTX issue are actual even now, after more than 5 decades of clinical experiences with the MTX in pediatric oncology. [\hyperlink{Methylergonovine Maleate}{PMID: 16284689}, J Sterba et al., 2005]

\hypertarget{pmid_21372841}{M}ercaptopurine has been used in continuing treatment of childhood acute lymphoblastic leukaemia since the mid 1950s. Recent advances in the understanding of thiopurine pharmacology indicated that thioguanine (TG) might be more effective than mercaptopurine (MP). The US and UK cooperative groups began randomised thiopurine trials and agreed prospectively to a meta-analysis. All randomised trials of TG versus MP were sought, and data on individual patients were analysed by standard methods. Combining three trials (from US, UK and Germany), the overall event-free survival (EFS) was not significantly improved with TG (odds ratio (OR)=0.89; 95\% confidence interval 0.78-1.03). Apparent differences in results between trials may be partly explained by the different types of patients studied. The larger treatment effect reported in males in the US trial was confirmed in the other trials. There was heterogeneity between sex/age subgroups (P=0.001), with significant EFS benefit of TG only observed for males aged <10 years old (OR=0.70; 0.58-0.84), although this did not result in a significant difference in overall survival (OR=0.83; 0.62-1.10). Additional toxicity occurs with TG. Mercaptopurine remains the standard thiopurine of choice, but further study of TG may be warranted to determine whether it could benefit particular subgroups. [\hyperlink{Methylergonovine Maleate}{PMID: 21372841}, G Escherich et al., 2011]

\hypertarget{pmid_34397332}{T}he present review was carried out to describe publications on the use of methylene blue (MB) in pediatrics and neonatology, discussing dose, infusion rate, action characteristics, and possible benefits for a pediatric patient group. The research was performed on the data sources PubMed, BioMed Central, and Embase (updated on Aug 31, 2020) by two independent investigators. The selected articles included human studies that evaluated MB use in pediatric or neonatal patients with vasoplegia due to any cause, regardless of the applied methodology. The MB use and 0 to 18-years-old patients with vasodilatory shock were the adopted criteria. Exclusion criteria were the use of MB in patients without vasoplegia and patients ≥ 18-years-old. The primary endpoint was the increase in mean arterial pressure (MAP). Side effects and dose were also evaluated. Eleven studies were found, of which 10 were case reports, and 1 was a randomized clinical study. Only two of these studies were with neonatal patients (less than 28 days-old), reporting a small number of cases (1 and 6). All studies described the positive action of MB on MAP, allowing the decrease of vasoactive amines in several of them. No severe side effects or death related to the use of the medication were reported. The maximum dose used was 2 mg/kg, but there was no consensus on the infusion rate and drug administration timing. Finally, no theoretical or experimental basis sustains the decision to avoid MB in children claiming it can cause pulmonary hypertension. The same goes for the concern of a possible deleterious effect on inflammatory distress syndrome. [\hyperlink{Methylergonovine Maleate}{PMID: 34397332}, Walusa A Gonçalves-Ferri et al., 2022]

\hypertarget{pmid_11224849}{A} child with malaria from a chloroquine-resistant area received an accidental overdose of chloroquine administered by a parent. Application of pharmacokinetics permitted definitive treatment with mefloquine in a safe and effective manner. [\hyperlink{Methylergonovine Maleate}{PMID: 11224849}, J A Lowry et al., 2001]

\hypertarget{pmid_23136875}{M}ethylphenidate is a centrally acting sympathomimetic used for the treatment of attention deficit/hyperactivity disorder in children and adolescents and for narcolepsy in adults. Despite the growing use among adult women, no reliable data on the prevalence of use during pregnancy have been published, and safety during pregnancy has not been established. We systematically reviewed available data on birth outcome after human in utero exposure to methylphenidate. Systematic searches in PubMed/Embase were performed from origin to August 2012, and data from Michigan Medicaid recipients, The Collaborative Perinatal Project and the Swedish Birth Registry were evaluated. Excluding three case reports, a total of 180 children exposed to methylphenidate in utero during first trimester were identified, among whom, four children with major malformations were observed. Methylphenidate exposure during pregnancy does not appear to be associated with a substantially (i.e. more than twofold) increased risk of congenital malformations. [\hyperlink{Methylergonovine Maleate}{PMID: 23136875}, Dorthe Dideriksen et al., 2013]

\hypertarget{pmid_26582874}{C}hildren account for 7\%-20\% of cutaneous leishmaniasis cases in Iran, but there are few safety data to guide pediatric antiparasitic therapy. We evaluated the clinical and laboratory tolerance of the systemic pentavalent antimonial compound meglumine antimoniate, in 70 Iranian children with cutaneous leishmaniasis. Adverse effects were similar to those seen in adults.  [\hyperlink{Methylergonovine Maleate}{PMID: 26582874}, Pouran Layegh et al., 2015] The use of antihistamine therapy in children for the management of upper respiratory tract infections remains a topic of debate. In this study, we focused on evaluating the effectiveness of promethazine (Phenergan), a first-generation H1 receptor antagonist and sedative, in addressing preoperative and intra-operative sequelae in cleft surgeries. A single-centered, parallel, randomized, double-blinded controlled clinical trial was conducted on 128 children aged 2 to 4 years undergoing cleft palate surgery under general anesthesia. The case group received Phenergan syrup orally twice a day for three days, while the control group received a placebo. Primary outcomes measured preoperative anxiety levels using a children's fear scale, while secondary outcomes assessed preoperative sleep quality and cough rate through objective scales. Intraoperative heart rate was monitored using an ECG connected to a monitor. The results demonstrated that the administration of promethazine resulted in a 34\% reduction in anxiety levels, a 46\% reduction in cold and cough, a 38\% improvement in sleep score, and stable heart rates throughout the surgery compared to the control group. Based on these findings, promethazine is considered a safe premedication option for children undergoing cleft palate surgeries; given its benefits outweigh its adverse effects. [\hyperlink{Methylergonovine Maleate}{PMID: 26582874}, Vedha Vivigdha A et al., 2023]

\hypertarget{pmid_17136402}{W}e performed a pharmacokinetic evaluation of methotrexate (MTX) in infants with acute lymphoblastic leukemia enrolled on the Pediatric Oncology Group (POG) 9407 Infant Leukemia Study to evaluate the effects of age on MTX pharmacokinetics and pharmacodynamics. A pharmacokinetic database of 61 patients was developed by combining MTX data obtained from 16 patients in a pharmacokinetic sub-study with data obtained for clinical care in other patients enrolled on the POG 9407 protocol. The data were analyzed for the first dose of MTX given to patients in induction/intensification therapy. Patients received MTX (4 g/m2) over 24 h at week 4 of therapy. Toxicity data were also reviewed to evaluate the incidence of common MTX toxicities during the first 6 weeks of therapy (the induction/intensification phase). Steady-state clearance (mean+/-standard deviation) for infants aged 0-6 months was 89+/-32 ml/min/m2 compared to 111+/-40 for infants aged 7-12 months (P=0.030). In the subgroup of infants aged 0-3 months the mean steady-state clearance was 84+/-30 ml/min/m2 (P=0.026 vs. the 7-12-month group). The incidence of renal toxicity (all grades) during induction/intensification therapy was 23\% in the 0-3 months age group compared to 0\% (for n=27) in the group 7-12 months of age (P=0.029). There were no significant differences in hepatoxicity or mucous membrane toxicity between age groups. A modest difference in steady-state MTX clearance is observed between younger infants (0-6 months) and older infants (7-12 months). Very young infants (0-3 months) also experienced a slightly higher incidence of renal toxicity during induction/intensification therapy. Steady-state clearance for the older infants is similar to values reported for children in other studies. [\hyperlink{Methylergonovine Maleate}{PMID: 17136402}, Patrick A Thompson et al., 2007]

\hypertarget{pmid_7792222}{O}ver the past four years terbinafine has become established as an effective systemic antimycotic agent with an excellent safety profile. However, experience with its use in children is very limited. We report the effective treatment of five children with oral terbinafine. [\hyperlink{Methylergonovine Maleate}{PMID: 7792222}, V Goulden et al., 1995]

\hypertarget{pmid_30414269}{I}nfantile haemangiomas (IH) are soft swellings of the skin that occur in 3-10\% of infants. When haemangiomas occur in high-risk areas or when complications develop, active intervention is necessary. To update a Cochrane Review assessing the interventions for the management of IH in children. We searched for randomized controlled trials in CENTRAL, MEDLINE, Embase, LILACS, AMED, PsycINFO, CINAHL and six trials registers up to February 2017. We included 28 trials (1728 participants) assessing 12 interventions. We downgraded evidence from high to moderate/low for issues related to risk of bias and imprecision. Oral propranolol (3 mg kg Our key results indicate that oral propranolol and topical timolol maleate are more beneficial than placebo in terms of clearance or other measures of resolution, or both, without an increase in harm. [\hyperlink{Methylergonovine Maleate}{PMID: 30414269}, M Novoa et al., 2019]

\hypertarget{pmid_24139067}{T}he purpose of this study was to assess the safety and efficacy of mitoxantrone (MX) in pediatric patients with aggressive multiple sclerosis (MS). A retrospective analysis on pediatric MS patients treated with MX was performed with regards to demographic/clinical parameters and magnetic resonance imaging (MRI) findings. 19 definite pediatric MS cases with mean ± SD age of 15.4 ± 2.8 years underwent 20 mg MX for control of their severe/frequent relapses, high EDSS score or new and active brain MRI lesions. After a median [IQR] follow-up period of 30[12-60] months, 14 cases (73\%) were relapse free; the EDSS score decreased by at least 0.5 in 16 cases (84.2\%); and gadolinium-enhancing lesion volume fell by 84.2\% in 16 cases. Adverse events included nausea and vomiting, fatigue, alopecia, palpitation, cardiomyopathy and mild leukopenia. All adverse events were mild and transient. Our results suggest MX is a good candidate for treatment of children with worsening RRMS and SPMS. Recommendations regarding patient selection, treatment administration, and close follow-up should be considered. Continuing research is needed to establish its efficacy and safety profile in a multinational collaboration with careful follow-up of adverse events. [\hyperlink{Methylergonovine Maleate}{PMID: 24139067}, Masoud Etemadifar et al., 2014]

\hypertarget{pmid_32229154}{T}he present study evaluated the exposure of children aged from one to 36 months to seven groups of mycotoxins, in the context of the infant French Total Diet Study (iTDS). Exposure was then compared to the health-based guidance values (HBGVs) for each mycotoxin. The value of the 90th percentile of exposure to nivalenol, patulin, fumonisins and zearalenone was less than 40\% of the HBGV considered relevant for children. On the other hand, a risk could not be excluded for ochratoxin A and aflatoxins as exposure was close to the HBGV for ochratoxin A and the margin of exposure was much lower than the critical margin of 10,000 for aflatoxins. The HBGVs for toxins T2 and HT2, and for deoxynivalenol (DON) and its acetylated compounds were exceeded. Five percent to 10\% of the children aged 5-12 months exceeded the HBGV considering the lower bound hypothesis for toxins T2 and HT2 and 7.5\%-27\% of the children aged 5 months and above exceeded the HBGV for DON. Consequently, the exposure of young children raises safety concerns for T2/HT2 and DON. Efforts should therefore be pursued to decrease their exposure to these molecules. [\hyperlink{Methylergonovine Maleate}{PMID: 32229154}, Karine Vin et al., 2020]

\section*{Octreoscan}
\subsection*{Result}
\subsubsection*{Answer}

Unknown

\subsubsection*{{Explanation}}
\hypertarget{Octreoscan}
A review of the available abstracts reveals only one that directly addresses the use of Octreoscan (111In-pentreotide) in a pediatric patient. In this case report, an 11-year-old boy with a bladder pheochromocytoma underwent Octreoscan imaging. The abstract states: "OctreoScan data correlated well with other radiologic studies, operative findings and with the final diagnosis, validating its use on pediatric patients" [\hyperlink{pmid_18450243}{PMID: 18450243}, José R Fournier et al., 2008]. However, this is a single case report and does not constitute a targeted safety study in a pediatric population or any specific age range.

No other abstracts provide targeted safety studies of Octreoscan in children, nor do they report on adverse events or safety outcomes in a pediatric cohort. Therefore, based on the abstracts available, there is no definitive evidence from targeted safety studies affirming that Octreoscan is safe or unsafe for use in children of any age range. The safety of Octreoscan in children remains unknown according to the evidence provided in the abstracts.

\subsection*{Abstracts}
\hypertarget{pmid_18450243}{A}n eleven year old boy presented with headaches and dizziness associated to micturition. On radiologic imaging, he was found with a bladder mass. The biochemical work up was suggestive of pheochromocytoma. An OctreoScan (111In-pentreotide) was used to rule out metastatic extension or other extra-adrenal locations of the pheochromocytoma. OctreoScan data correlated well with other radiologic studies, operative findings and with the final diagnosis, validating its use on pediatric patients. [\hyperlink{Octreoscan}{PMID: 18450243}, José R Fournier et al., 2008]

\hypertarget{pmid_10326811}{T}o determine the safety and efficacy of ofloxacin otic solution in the treatment of acute otorrhea in children with tympanostomy tubes. Multicenter study with an open-label, prospective ofloxacin arm and retrospective historical and current practice arms. Ear, nose, and throat pediatric and general practice clinics and office-based practices. Children younger than 12 years with acute purulent otorrhea of presumed bacterial origin and tympanostomy tubes. Instillation of 0.3\% ofloxacin, 0.25 mL, twice daily for 10 days in the prospective arm; review of medical records in the retrospective arms. The primary index of clinical efficacy was absence (cure) or presence (failure) of otorrhea at 10 to 14 days after therapy. The primary index of microbiologic efficacy (in the ofloxacin arm only) was eradication of pathogens isolated at baseline. Safety was evaluated in the ofloxacin arm only. Significantly more clinically evaluable ofloxacin-treated subjects were cured (84.4\%; 119/141) than were historical practice subjects (64.2\%; 140/218) (P< or =.001) or current practice subjects (70\%; 33/47) (P< or =.03). All baseline pathogens were eradicated in 103 (96.3\%) of 107 microbiologically evaluable ofloxacin subjects. Adverse events considered "possibly" or "probably" treatment related occurred in 29 (12.8\%) of 226 ofloxacin-treated subjects. Ofloxacin is safe and significantly more effective than treatments used in historical or current practice for acute purulent otorrhea in children with tympanostomy tubes. [\hyperlink{Octreoscan}{PMID: 10326811}, J E Dohar et al., 1999]

\hypertarget{pmid_9204352}{A} double-blind, parallel-group trial was performed in 96 children in order to evaluate and compare the safety and efficacy of iopentol (Imagopaque, Nycomed imaging AS, Oslo, Norway) with those of routinely used contrast media. Ten children below 1 year of age received iopentol and eight received iohexol (Omnipaque, Nycomed Imaging AS, Oslo, Norway) (both 300 mg I/ml) in a random manner. Seventy-eight children (39 in each group), between one and 10 years of age, received iopentol 300 or diatrizoate (Urografin, Schering AG, Berlin, Germany) 292 mg I/ml for urography. No adverse events were observed among the children below 1 year of age. In the group between one and 10 years there was a statistically and clinically significant difference (p = 0.007) in the incidence of adverse events between the iopentol (2.6\%) and diatrizoate (25.6\%) groups. All adverse events were of mild or moderate intensity. No serious reactions were encountered. The overall quality of visualization was judged to be diagnostic for all patients. This study supports the use of non-ionic contrast media in pediatric patients and confirms that iopentol is a safe and effective contrast medium well suited for children from 0-10 years of age. [\hyperlink{Octreoscan}{PMID: 9204352}, P Lanning et al., 1997]

\hypertarget{pmid_9848042}{T}he ideal oral wound care analgesic for children should be palatable, provide potent analgesia of rapid onset and short duration, and require minimal, yet appropriate, monitoring. With use of a double-blinded crossover design, we compared the efficacy and safety of oral transmucosal fentanyl citrate (OTFC) (approximately 10 micrograms/kg) with the efficacy and safety of oral hydromorphone (60 micrograms/kg) in 14 pediatric inpatients (ages 4 to 17 years) undergoing daily burn wound care in a ward setting. Pulse oximetry, vital signs, side effects, patient pain scores, and observer scores for cooperation, anxiety, and sedation were recorded. Pulse oximetry, vital signs, cooperation, sedation, incidence of nausea or vomiting, and the amount of time it took to resume normal activities were similar in both treatment groups. OTFC resulted in improved pain scores before wound care and improved anxiolysis during wound care, but at other points it was similar in effect to hydromorphone. We conclude that OTFC is a safe and effective analgesic, that it may provide minor improvements in analgesia and anxiolysis compared with hydromorphone, and that it offers a palatable alternative route of opioid administration without intravenous access for wound care procedures in children. [\hyperlink{Octreoscan}{PMID: 9848042}, S R Sharar et al., ]

\hypertarget{pmid_17803435}{T}he aim of this study was to evaluate the safety of olopatadine hydrochloride ophthalmic solution 0.2\% in children and adolescents 3-17 years of age. In this 6-week, randomized, double-masked safety evaluation, eligible subjects with asymptomatic eyes underwent in-office visits at weeks 1, 3, and 6 and were contacted by telephone at weeks 2, 4, and 5. Qualified subjects were assigned randomly in a 2:1 ratio of olopatadine 0.2\% to vehicle (identical formation without the active ingredient) for dosing on a once-daily schedule. Safety parameters assessed included adverse events, visual acuity, ocular signs (slit-lamp assessments), dilated fundus examinations, intraocular pressure (IOP), pulse, and blood pressure. An evaluation of 126 subjects (age range, 3-17) revealed no clinically relevant treatment-related changes in visual acuity, IOP, slit-lamp assessments, fundus examinations, or cardiovascular parameters. All adverse events reported were mild or moderate. Olopatadine 0.2\% administered once-daily for 6 weeks is safe and well tolerated in children and adolescent patients. [\hyperlink{Octreoscan}{PMID: 17803435}, Steven J Lichtenstein et al., 2007]

\hypertarget{pmid_15763288}{O}tomicroscopic examination with suctioning of ears or other procedures is frequently uncomfortable especially for children. Anxiety and pain with lack of cooperation may result in trauma to the ear, incompletion of the examination, delayed diagnosis and treatment and need for completion of the examination under general anesthesia. The purpose of this study was to evaluate the efficacy and safety of utilizing nitrous oxide-oxygen inhalation for sedation and analgesia in otologic examination and minor surgical procedures performed on the uncooperative child at the outpatient clinic. In a prospective pilot case series study conducted at the Pediatric Otolaryngology outpatient clinic of a tertiary medical center, nitrous oxide-oxygen inhalation was administered by the examining otolaryngologist and the assisting nurse. The study group included children over 2 years old, for which an accurate diagnosis of ear pathology could not be made or a minor surgical procedure could not be tolerated because of anxiety and lack of cooperation. Completion of the indicated procedure was successful in 21 of 24 patients (88\%). Full cooperation, where no restraint was necessary was achieved in 20 of 24 patients (83\%). The mean rank pain scores, evaluated separately by the patient, parent and staff, were in the mild pain range using a 0-10 coding for Faces Pain Rating Scale. The mean procedure time was 8.9 min. An adverse reaction, vomiting, occurred in one patient. Twenty-one of 24 parents stated that they would repeat the procedure if necessary. This pilot study shows the potential usefulness of nitrous oxide-oxygen inhalation administered by an otolaryngologist in the outpatient clinic. Alleviation of pain and anxiety and avoiding the need for physical restraint is an important goal that can be achieved with this form of sedation. [\hyperlink{Octreoscan}{PMID: 15763288}, Gadi Fishman et al., 2005]

\hypertarget{pmid_34050849}{T}his study investigated the safety and efficacy of orthokeratology in myopic children in Japan. Prospective clinical trial. This study enrolled myopic children aged 6-16 years with a spherical equivalent of -1.00 D to -4.00 D and astigmatism of -1.5 D or lower, whose parents could manage contact lens use and could provide written informed consent. The children were treated with orthokeratology lenses (BREATH-O CORRECT R, Universal View Co., Ltd.) for 3 years. Slit-lamp findings, visual acuity, intraocular pressure, subjective refraction, corneal topography, corneal endothelial cell density, corneal thickness, and axial length were regularly assessed. This study included 96 eyes of 48 patients (average age, 10.7 ± 2.08 years). The average baseline spherical equivalent was -2.46 ± 0.97 D. The average baseline uncorrected visual acuity was 0.74 ± 0.32 logMAR, with significant improvement to -0.08 ± 0.18 logMAR at 4 weeks and 0.02 ± 0.21 logMAR at 3 years (both p < 0.001, Dunnett's test). The average baseline corneal endothelial cell density was 3053 ± 181 cells/mm Orthokeratology lens (BREATH-O CORRECT R) use in children demonstrated good efficacy and safety during 3 years of follow-up. [\hyperlink{Octreoscan}{PMID: 34050849}, Tomoko Goto et al., 2021]

\hypertarget{pmid_30707286}{O}esophageal atresia and tracheo-oesophageal atresia require surgical repair in early infancy. These children have significant disease-related morbidity requiring frequent radiological examinations resulting in an increased malignancy risk. A single-centre, retrospective review was performed of radiation exposure in children with OA/TOF born 2011-2015. Medical records were reviewed to determine the number and type of imaging studies involving ionising radiation exposure enabling the calculation of the estimated effective dose per child over the first year of life. Forty-nine children were included. Each child underwent a median of 19 (IQR 11.5-35) imaging studies, which were primarily plain radiography (median = 14, IQR 7-26.5). The overall median estimated effective dose per patient was 4.7 (IQR 3.0-9.4) mSv, with the majority of radiation exposure resulting from fluoroscopic imaging (median 3.3 mSv, IQR 2.2-6.0). 'Routine' postoperative oesophagrams showed no leak in 35/36 (97\%) with the remaining study showing an insignificant leak that did not alter management. Careful consideration should be given to the use of imaging in OA/TOF to minimise morbidity in these vulnerable infants. Oesophagrams in children without the symptoms of anastomotic leak or stricture should be discontinued. Standardisation of monitoring protocols with regard to radiation exposure should be considered. [\hyperlink{Octreoscan}{PMID: 30707286}, Kiera Roberts et al., 2019]

\hypertarget{pmid_22052632}{O}ctreotide is a synthetic somatostatin analogue which has been suggested for use in the management of acute pancreatitis, though its safety and effectiveness in the pediatric setting has not been extensively studied. we present a rare case of a 6.5-year-old female with acute lymphoblastic leukemia (ALL) and L-asparaginase (L-asp) induced pancreatitis, who developed epileptic seizures, possibly associated with octreotide administration. Her imaging and laboratory findings ruled out a leukemic involvement or infection of CNS. The EEG revealed repetitive sharp waves maximal on the frontal and temporal areas of the right hemisphere. The child was treated with diazepam and she continued with systemic anticonvulsant treatment with levetiracetam. After 2 weeks of conservative treatment, pancreatitis resolved and she continued her chemotherapy protocol. Levetiracetam treatment lasted 8 months. 7 months after the first episode, EEG was reported as normal, and the child completed the chemotherapy protocol without any further severe complications. Larger and well designed studies are needed to warrant the safety of octreotide in pediatric population. [\hyperlink{Octreoscan}{PMID: 22052632}, E Hatzipantelis et al., 2011]

\hypertarget{pmid_11176590}{O}torrhea occurs in 21 to 50\% of all children with tympanostomy tubes in the United States. More than 1 million children annually undergo tubomyringotomy, constituting placement of more than 2 million tympanostomy tubes each year. The organisms typically responsible for otorrhea are the same as those that cause otitis media in very young children, including Streptococcus pneumonia, Haemophilus influenzae and Moraxella catarrhalis. Drainage from tympanostomy tubes in older children involves organisms that colonize the external auditory canal, the most common being Pseudomonas aeruginosa and Staphylococcus aureus. Ofloxacin (Floxin otic), a newer fluoroquinalone antibiotic, has several advantages over other agents available for the treatment of otorrhea caused by acute otitis media in patients with tympanostomy tubes. The twice daily dosing regimen encourages better patient adherence to therapy, which is likely to improve treatment efficacy. Ofloxacin has not been associated with ototoxicity in animal models or in children participating in the clinical trials. It provides coverages for a wide range of pathogens, including Pseudomonas sp., and is indicated for use in children > or =1 year old and currently approved for patients > or =12 years with chronic suppurative otitis media. Ofloxacin applied topically in children with tympanostomy tubes in place and purulent otorrhea is as efficacious as oral amoxicillin/clavulanate (Augmentin) therapy. Other currently available therapeutic options are discussed. [\hyperlink{Octreoscan}{PMID: 11176590}, E L Goldblatt et al., 2001]

\hypertarget{pmid_8985547}{T}o evaluate two photoscreeners in a childhood population. Double-masked study. One hundred and thirteen children aged between 11 and 44 months with either normal vision or known visual disorders were photoscreened without cycloplegia by the Otago and Dortmans (prototype) photoscreeners. Each child had a full ophthalmological examination either on the day of screening or in the proceeding six months. Photoscreen images were reviewed by an independent observer for indicators of amblyopiogenic risk factors, and compared to the full ophthalmological examination to determine sensitivity and specificity for each instrument. The Otago photoscreener returned a sensitivity of 70\% and specificity of 82\% for the detection of amblyopiogenic risk factors. The Dortmans photoscreener returned a sensitivity of 70\% and specificity of 90\%. Both photoscreeners were portable and easily operated. Children can be screened successfully for amblyopiogenic risk factors with these photoscreening systems. Further evaluation is required to determine specificity in a normal population. This would also provide information on the potential usefulness of photoscreeners in a cost effective childhood vision screening program. [\hyperlink{Octreoscan}{PMID: 8985547}, C D Cooper et al., 1996]

\hypertarget{pmid_27188702}{T}o investigate the efficacy, safety, and microbiology of a thermosensitive otic suspension of ciprofloxacin (OTO-201) in children with bilateral middle ear effusion undergoing tympanostomy tube placement. Two randomized, double-blind, sham-controlled phase 3 trials. Patients were randomized to intratympanic OTO-201 or sham. Children with bilateral middle ear effusion undergoing tympanostomy tube placement. Studies evaluated 532 patients (6 months to 17 years old) in a combined analysis of efficacy (treatment failure: presence of otorrhea, otic or systemic antibiotic use, lost to follow-up, missed visits), safety (audiometry, otoscopy, tympanometry), and microbiology. There was a lower cumulative proportion of treatment failures in patients receiving OTO-201 vs tympanostomy tubes alone (1) on days 4, 8, 15, and 29; (2) on day 15, primary end point (23.0\% vs 45.1\%; age-adjusted odds ratio, 0.341; P < .001; reduction in relative risk, 49\%); and (3) on day 15, blinded-assessor otorrhea treatment failure (7.0\% vs 19.4\%; age-adjusted odds ratio, 0.303; P < .001; reduction in relative risk, 64\%). Per-protocol and subgroup analyses (baseline demographics, pathogen type, culture status, effusion type, microbiologic response) supported these findings. There were no drug-related serious adverse events; the most frequent treatment-emergent adverse events in both groups were pyrexia, postoperative pain, nasopharyngitis, cough, and upper respiratory tract infection. OTO-201 administration had no evidence of increased tube occlusion and no negative effect on audiometry, tympanometry, or otoscopy. Combined analysis of 2 phase 3 trials demonstrated a lower cumulative proportion of treatment failures through day 15 compared with TT alone when OTO-201 was administered intratympanically for otitis media with bilateral middle ear effusion at time of tympanostomy tube placement. [\hyperlink{Octreoscan}{PMID: 27188702}, Albert H Park et al., 2016]

\hypertarget{pmid_25512818}{A}nticholinergics are a key element in treating neurogenic detrusor overactivity, but only limited data are available in the pediatric population, thus limiting the application to children even for oxybutynin chloride (OC), a prototype drug. This retrospective study was designed to provide data regarding the efficacy, tolerability, and safety of OC in the pediatric population (0-15 years old) with spinal dysraphism (SD). Records relevant to OC use for neurogenic bladder were gathered and scrutinized from four specialized clinics for pediatric urology. The primary efficacy outcomes were maximal cystometric capacity (MCC) and end filling pressure (EFP). Data on tolerability, compliance, and adverse events (AEs) were also analyzed. Of the 121 patient records analyzed, 41 patients (34\%) received OC at less than 5 years of age. The range of prescribed doses varied from 3 to 24 mg/d. The median treatment duration was 19 months (range, 0.3-111 months). Significant improvement of both primary efficacy outcomes was noted following OC treatment. MCC increased about 8\% even after adjustment for age-related increases in MCC. Likewise, mean EFP was reduced from 33 to 21 cm H2O. More than 80\% of patients showed compliance above 70\%, and approximately 50\% of patients used OC for more than 1 year. No serious AEs were reported; constipation and facial flushing consisted of the major AEs. OC is safe and efficacious in treating pediatric neurogenic bladder associated with SD. The drug is also tolerable and the safety profile suggests that adjustment of dosage for age may not be strictly observed. [\hyperlink{Octreoscan}{PMID: 25512818}, Jung Hoon Lee et al., 2014]

\hypertarget{pmid_31958794}{O}zenoxacin is a topical antibiotic approved in Europe to treat non-bullous impetigo in adults and children aged ≥6 months. This analysis evaluated the efficacy and safety of ozenoxacin in paediatric patients by age group. Pooled data for patients aged 6 months to <18 years who had participated in a phase I or in two phase III clinical trials of ozenoxacin 1\% cream were analysed by age group: 0.5-<2, 2-<6, 6-<12, and 12-<18 years. The combined population comprised 529 patients with non-bullous impetigo treated with ozenoxacin (n = 239), vehicle (n = 201), or retapamulin as internal validation control (n = 89). Studies were well matched for extent and severity of impetigo and therapeutic schedule (twice daily application for 5 days). The clinical success rate after 5 days' treatment (day 6-7, end of therapy), and microbiological success rates after 3-4 days' treatment and at the end of therapy, were significantly higher with ozenoxacin than vehicle (p < 0.0001 for all comparisons). Clinical and bacterial eradication rates were higher with ozenoxacin than vehicle in each age group. No safety concerns were identified with ozenoxacin. One (0.3\%) of 327 plasma samples exceeded the lower limit of quantification for ozenoxacin, but the low concentration indicated negligible systemic absorption. This combined analysis supports the efficacy and safety of ozenoxacin administered twice daily for 5 days. Ozenoxacin 1\% cream is a new option to consider for treatment of non-bullous impetigo in children aged 6 months to <18 years. [\hyperlink{Octreoscan}{PMID: 31958794}, Antonio Torrelo et al., 2020]

\hypertarget{pmid_16028153}{B}ecause of concerns about arthrotoxicity, fluoroquinolones are restricted for use in children. This study describes the safety and efficacy of gatifloxacin when used for treatment of children with recurrent acute otitis media (ROM) or acute otitis media (AOM) treatment failure (AOMTF). We performed an analysis of 867 children included in 4 clinical trials who had ROM and/or AOMTF and were treated with gatifloxacin (10 mg/kg once daily for 10 days). Gatifloxacin had adverse event rates that were similar overall to those of a comparator antibiotic (amoxicillin-clavulanate), except for increased diarrhea in children <2 years old receiving amoxicillin-clavulanate. There was no evidence of arthrotoxicity, hepatotoxicity, alteration of glucose homeostasis, or central nervous system toxicity acutely or during 1 year follow-up in any child. Regarding efficacy, in 2 noncomparative trials, the gatifloxacin cure rate of AOM was 89\% (95\% confidence interval [CI], 83\%-95\%) at the test of cure (TOC) visit, 3-10 days after completion of therapy. In 2 comparative trials of gatifloxacin versus amoxicillin-clavulanate, the efficacy of gatifloxacin was 88\% (95\% CI, 82\%-94\%). Gatifloxacin led to better clinical outcomes than amoxicillin-clavulanate for AOMTF (91\% vs. 81\%; P=.029), for AOMTF and age <2 years old (89\% vs. 69\%; P=.009), and for severe AOM in children <2 years old (90\% vs. 75\%; P=.012). Among children with AOMTF previously treated with amoxicillin-clavulanate or ceftriaxone injections, gatifloxacin cure rates were high (88\% and 75\%, respectively). Gatifloxacin appears to be safe for children, with no evidence of producing arthrotoxicity in 867 children exposed to the antibiotic when used as treatment for ROM and AOMTF. [\hyperlink{Octreoscan}{PMID: 16028153}, Michael E Pichichero et al., 2005]

\hypertarget{pmid_16532329}{C}hylothorax is a rare but life-threatening condition in children. To date, there is no commonly accepted treatment protocol. Somatostatin and octreotide have recently been used for treating chylothorax in children. We set out to summarise the evidence on the efficacy and safety of somatostatin and octreotide in treating young children with chylothorax. Systematic review: literature search (Cochrane Library, EMBASE and PubMed databases) and literature hand search of peer reviewed articles on the use of somatostatin and octreotide in childhood chylothorax. Thirty-five children treated for primary or secondary chylothorax (10/somatostatin, 25/octreotide) were found. Ten of the 35 children had been given somatostatin, as i.v. infusion at a median dose of 204 microg/kg/day, for a median duration of 9.5 days. The remaining 25 children had received octreotide, either as an i.v. infusion at a median dose of 68 microg/kg/day over a median 7 days, or s.c. at a median dose of 40 microg/kg/day and a median duration of 17 days. Side effects such as cutaneous flush, nausea, loose stools, transient hypothyroidism, elevated liver function tests and strangulation-ileus (in a child with asplenia syndrome) were reported for somatostatin; transient abdominal distension, temporary hyperglycaemia and necrotising enterocolitis (in a child with aortic coarctation) for octreotide. A positive treatment effect was evident for both somatostatin and octreotide in the majority of reports. Minor side effects have been reported, however caution should be exercised in patients with an increased risk of vascular compromise as to avoid serious side effects. Systematic clinical research is needed to establish treatment efficacy and to develop a safe treatment protocol. [\hyperlink{Octreoscan}{PMID: 16532329}, Charles C Roehr et al., 2006]

\hypertarget{pmid_27910218}{O}ctreotide is a synthetic peptide analog of naturally occurring somatostatin. Octreotide is used off-label in children <6 years of age for hyperinsulinism, chylothorax, and gastrointestinal bleeding. There is a lack of controlled data on efficacy or potential adverse events from this off-label use. Three pediatric hospitals participated in this study. Patients were hospitalized January 2007-December 2010 and administered octreotide for congenital hyperinsulinism (CHI) at least 1 day. Variables assessed included octreotide dosage, patient demographics, medical interventions, concomitant medicines, serious adverse events (SAEs) including necrotizing enterocolitis (NEC), and mortality. The 103 patient sample had a median gestational age of 38 weeks. During the study period, two patients died: one from NEC and the other from cardiomyopathy/sepsis. There were 11 other SAEs in the 101 surviving patients. This study highlights potential risks in administering octreotide off-label. This study, like several other published studies, has highlighted NEC in a full-term infant treated with octreotide. It is important to study the efficacy and the safety of octreotide for hyperinsulinism. In the interim, it might be prudent to prescribe octreotide in CHI neonates only in the absence of other risk factors for NEC. Copyright © 2016 John Wiley \& Sons, Ltd. [\hyperlink{Octreoscan}{PMID: 27910218}, Ann W McMahon et al., 2017]

\hypertarget{pmid_32258344}{O}zenoxacin is a topical antibiotic approved in the United States for treatment of impetigo in adults and children age ≥2 months. This analysis evaluated the efficacy and safety of ozenoxacin in specific pediatric age groups. Data for children aged 2 months to <18 years recruited from eight countries who had participated in phase 1 and 3 trials of ozenoxacin were extracted and analyzed by age range. Across studies, 644 pediatric patients with impetigo received ozenoxacin 1\% cream (n = 287) or vehicle (n = 247). One study included retapamulin 1\% ointment as the internal validity control (n = 110). The clinical success rate at the end of treatment and bacterial eradication rates after 3 to 4 days of treatment and at the end of treatment were significantly higher with ozenoxacin than vehicle (all  The results of this analysis suggest that ozenoxacin 1\% cream is an effective and safe treatment for impetigo in pediatric patients aged 2 months to <18 years. [\hyperlink{Octreoscan}{PMID: 32258344}, Adelaide A Hebert et al., 2020]

\hypertarget{pmid_7862469}{T}o investigate the efficacy and safety of oral transmucosal fentanyl (OTFC) in providing analgesia and sedation for painful diagnostic procedures in children. Randomized, placebo-controlled clinical trial. Forty-eight children referred to the University Connecticut Division of Pediatric Hematology/Oncology for bone marrow aspiration or lumbar puncture were randomized to receive either OTFC (15 to 20 micrograms/kg) or a placebo lollipop. Thirty minutes after administration, the procedure was begun. An anesthesiologist monitored the child's heart rate, blood pressure, and oxygen saturation every 10 minutes. At the conclusion of the procedure, the nurse, the child's parent, and all children over 8 years of age were asked to rate the pain associated with the procedure using a 1 to 10 visual analogue scale. Young children (less than 8) used a modified scale, the Oucher, yielding a 0 to 5 score. Significant differences in pain ratings between the OTFC and placebo groups were noted on the pain scores of the parents (P = .005), nurses (P = .001), younger children (P = .006), and older children (P = .013), and median pain scores in the OTFC group were reduced to tolerable levels. Vomiting (P = .003) and itching (P = .001) were more common in the OTFC group, but no clinically significant vital sign deviations occurred. OTFC is safe and effective for use in relieving the pain of pediatric procedures, but frequency of vomiting may restrict its clinical usefulness. [\hyperlink{Octreoscan}{PMID: 7862469}, N L Schechter et al., 1995]

\hypertarget{pmid_26296929}{T}his exploratory clinical trial evaluated the safety and clinical activity of a novel, sustained-exposure formulation of ciprofloxacin microparticulates in poloxamer (OTO-201) administered during tympanostomy tube placement in children. Double-blind, randomized, prospective, placebo- and sham-controlled, multicenter Phase 1b trial in children (6 months to 12 years) with bilateral middle ear effusion requiring tympanostomy tube placement. Patients were randomized to intraoperative OTO-201 (4 mg or 12 mg), placebo, or sham (2:1:1 ratio). Eighty-three patients (52 male/31 female; mean age, 2.80 years) were followed for safety (otoscopic exams, cultures, audiometry, and tympanometry) and clinical activity, defined as treatment failure (physician-documented otorrhea and/or otic or systemic antibiotic use ≥3 days post surgery). At baseline, 14.3\% to 36.8\% of children showed positive cultures of middle ear effusion samples in at least 1 ear. Through day 15, treatment failures accounted for 14.3\%, 15.8\%, 45.5\%, and 42.9\% of patients (OTO-201 4 mg, OTO-201 12 mg, placebo, and sham, respectively); treatment failure reductions for OTO-201 doses were significant compared to pooled control (P values = .023 and .043, respectively). Observed OTO-201 safety profile was indistinguishable from placebo or sham. Results of this first clinical trial suggest that OTO-201 was well tolerated and shows preliminary clinical activity in treating tympanostomy tube otorrhea. [\hyperlink{Octreoscan}{PMID: 26296929}, Eric A Mair et al., 2016]

\hypertarget{pmid_9742537}{T}hirty min prior to anaesthetic induction for surgery, children aged 4-12 years old were given a 10 micrograms.kg-1 oral transmucosal fentanyl citrate (OTFC) and were instructed to suck the OTFC until pruritus appeared (Group 2) or until the entire dose was consumed (Group 1). Sedation, apprehension and cooperation scores were rated, and vital signs including oxygen saturation were monitored until anaesthetic induction. The results showed that pruritus was present in 76\% of children; however; in all but one child, it occurred after the OTFC had been completely consumed. There were no significant changes in oxygen saturation, but respiratory rate decreased from 19.6 +/- 1.7 to 18.4 +/- 1.3. Activity decreased significantly; however, cooperation and apprehension did not change. The conclusion was that pruritus cannot be used as an endpoint for OTFC effectiveness; however, OTFC dosed at 10 micrograms.kg-1 is effective in providing sedation without causing clinically significant changes in vital signs or oxygen saturation. [\hyperlink{Octreoscan}{PMID: 9742537}, B Ginsberg et al., 1998]

\hypertarget{pmid_16554175}{T}o evaluate the long-term efficacy, tolerability, and safety of oxcarbazepine (OXC) in children with epilepsy. We enrolled 36 patients (median age 7.75) with new diagnosis of partial epilepsy in an open prospective study. All type of epilepsy were included: 25 patients were affected by idiopathic epilepsy, eight by symptomatic epilepsy and three by cryptogenic epilepsy. Patients were then scheduled to come back for controls at 3 months (T1), 12 months (T2) and 24 months (T3) after the beginning of OXC-monotherapy (T0). At each control we evaluated patients through their seizure diary, a questionnaire on side effects, their level of 10-monohydroxy (MHD) metabolite and laboratory analysis. At T1, 21/36 patients (58.3\%) were seizure-free, 3/36 patients (8.3\%) showed an improvement higher than 50\%, 3/36 (8.3\%) lower than 50\%, while 2/36 worsened (5.6\%). In 7/36 (19.5\%) patients, no improvement was reported. At T2 13/18 patients (72.2\%) were seizure-free, 1/18 showed a response to therapy higher than 50\% while 2/18 worsened (11\%). In two patients no improvement was reported. A correspondence between MHD plasmatic levels and clinical response (r=0.49; p<0.05) was only registered at T1. An EEG normalization was observed in 25\% of cases. Side effects were reported in 25\% of cases, but symptoms progressively disappeared at follow-up. We can therefore conclude that OXC can be considered, for its efficacy and safety, as a first line drug in children with epilepsy. [\hyperlink{Octreoscan}{PMID: 16554175}, E Franzoni et al., 2006]

\hypertarget{pmid_10435125}{A} total of 305 children, five to 16 months of age, were treated from 1983-1984 with ventilation tubes-Shah vent Teflon tube-inserted under local anaesthesia for recurrent acute otitis media (RAOM) or otitis media with effusion (OME). The final study group comprised 281 children (92.1 per cent) monitored prospectively for five years, 185 in the OME-group and 96 in the RAOM-group. For the first insertion of tubes the average ventilation period was 15.4 months. Re-tympanostomy, with adenoidectomy simultaneously at the first time was performed in 99 ears (35.2 per cent); once in 27.0 per cent, twice in five per cent, and three times in 3.2 per cent. Mastoidectomy due to otorrhoea was performed in three ears (1.1 per cent). The children in the OME-group were at higher risk of repeated post-tympanostomy otorrhoea episodes than children in the RAOM-group. These episodes of otorrhoea during the first insertion of ventilation tubes significantly increased both the tube extrusion rate and the need for subsequent re-tympanostomies. No major complications were caused by the tympanostomy procedure as such. It is concluded that early tympanostomy is a safe procedure in young children with RAOM or OME. However, parents should be carefully informed of risks of post-tympanostomy otorrhoea and recurrent disease after insertion of ventilation tubes necessitating subsequent tube insertion, especially in children with OME. [\hyperlink{Octreoscan}{PMID: 10435125}, H Valtonen et al., 1999]

\hypertarget{pmid_29464579}{T}his study aimed to assess efficacy and safety of oxcarbazepine (OXC) oral suspension in pediatric patients aged 2-5 years with partial seizures (PS) and/or generalized tonic-clonic seizures (GTCS) in real-world clinical practice in China. This 26-week, prospective, single-arm, multicenter, observational study recruited pediatric patients aged 2-5 years with PS or GTCS suitable for OXC oral suspension treatment based on physicians' judgments from 11 medical centers in China. Enrolled subjects started OXC oral suspension treatment as monotherapy or in combination with other antiepileptic drugs. Primary efficacy outcome was the percentage of pediatric subjects achieving ≥ 50\% seizure frequency reduction at the end of the 26-week treatment. Secondary efficacy-related parameters and safety parameters such as adverse events (AEs) and serious AEs (SAEs) were also monitored during the 26-week treatment period. Six hundred and six pediatric patients were enrolled and 531 (87.6\%) completed the study. After 26 weeks of treatment, 93.3\% subjects achieved ≥ 50\% seizure frequency reduction, and 81.8\% achieved 100\% seizure frequency reduction compared to baseline. Among different seizure types, OXC was effective in all subjects with simple PS and in > 90\% of subject with other type of seizure present in the study. AEs were observed in 49 (8.1\%) subjects. Only three subjects experienced SAE. Rash (n = 18, 2.97\%) was the most common AE. Only 17 subjects discontinued due to AEs. This study, reporting the real-world data, further confirms the efficacy and good safety profile of OXC oral suspension in Chinese pediatric patients aged 2-5 years with PS and/or GTCS. [\hyperlink{Octreoscan}{PMID: 29464579}, Jiong Qin et al., 2018]

\hypertarget{pmid_11685525}{T}his study evaluates the benefits of and indications for the orbito-cranial approach (OCA) in pediatric patients. The authors report their recent experience of using the OCA in 9 pediatric patients, 6 boys and 3 girls. The patients' ages ranged from 3 to 17 years (mean 9.6+/-5.16 years). Follow-up periods varied between 6 and 21 months (mean 12.6+/-5.9 months). Five patients were operated on for craniopharyngiomas, 2 for chiasmatic-hypothalamic astrocytomas, 1 for a recurrent hypothalamic gangliocytoma, and 1 for a hypothalamic hamartoma. In 7 cases a neuronavigation system (BrainLab) was utilized. The lesions were removed totally in 5 patients, near-totally in 1, subtotally in 2, and partially in 1 patient. An average increase of 30\% in the area of vertical exposure significantly decreased the need for brain retraction. There was no mortality in this series. The only complications connected with the surgical approach were transient subgaleal cerebro-spinal fluid collections in 7 of 9 children and a subgaleal-peritoneal shunt placement in another patient. Our experience with this series of patients suggests that the OCA is as safe and beneficial in pediatric patients as it is in adults. It facilitates tumor removal by providing shorter access to and better exposure of the suprasellar area, thereby minimizing brain retraction. [\hyperlink{Octreoscan}{PMID: 11685525}, V Siomin et al., 2001]

\section*{Pyridoxine Hydrochloride}
\subsection*{Result}
\subsubsection*{Answer}

Infants and children (0-12 years): Yes
Adolescents (13-18 years): Unknown

\subsubsection*{{Explanation}}
\hypertarget{Pyridoxine Hydrochloride}
To determine if Pyridoxine Hydrochloride is safe for use in children, I reviewed all available abstracts for targeted studies on safety in pediatric populations. The following relevant studies were identified:

1. A randomized, double-blind, placebo-controlled trial in 48 children with moderate or severe atopic dermatitis administered pyridoxine hydrochloride 50 mg once daily for 4 weeks. No statistically significant difference in clinical benefit was found, but no adverse effects or safety concerns were reported in the children who received pyridoxine hydrochloride [\hyperlink{pmid_8555030}{PMID: 8555030}, D C Mabin et al., 1995].

2. A randomized trial in 107 children hospitalized after febrile convulsions, with 55 children receiving pyridoxine hydrochloride (20 mg twice daily) and 52 receiving placebo for up to one year. The study did not show efficacy in preventing recurrent convulsions, but no safety concerns or adverse effects were reported in the pyridoxine group [\hyperlink{pmid_7009019}{PMID: 7009019}, J McKiernan et al., 1981].

3. A study of 90 infants and children with recurrent convulsions (primarily due to acute infectious diseases) compared high-dose pyridoxine (30 or 50 mg/kg/day IV) to controls. No adverse effects of pyridoxine were apparent during the observation period, and the authors concluded that pyridoxine is effective, safe, and well-tolerated as an adjunct to antiepileptic drugs in children [\hyperlink{pmid_9308977}{PMID: 9308977}, F Y Jiao et al., 1997].

4. A pilot, randomized, open-label trial in children aged 3 months to 3 years with infantile spasms compared high-dose pyridoxine (30 mg/kg/day) plus prednisolone to prednisolone alone. The adverse effects were comparable in both groups, and the study states that high-dose pyridoxine may be safe in children with infantile spasms [\hyperlink{pmid_29547159}{PMID: 29547159}, Vedavathi Kunnanayaka et al.].

5. A double-blind study with 76 asthmatic children followed for five months indicated significant improvement in asthma following pyridoxine therapy (200 mg daily). The abstract notes a lack of toxicity of this vitamin at relatively high doses (1 or 1.5 g), suggesting safety, though the primary focus is on efficacy and pharmacological activity [\hyperlink{pmid_94242}{PMID: 94242}, M Garcia et al.].

6. High-dose vitamin B6 (pyridoxine-HCl, 300 mg/kg/day orally) was used as initial treatment in 17 children with infantile spasms. No serious adverse reactions were noted; side effects were mainly gastrointestinal and reversible after dose reduction [\hyperlink{pmid_8330589}{PMID: 8330589}, J Pietz et al.].

7. A case report describes the effective use of IV pyridoxine in a 32-month-old child with acute poisoning and uncontrollable seizures, with no mention of adverse effects [\hyperlink{pmid_6866642}{PMID: 6866642}, E Hartemann et al.].

8. A survey and commentary on the availability of parenteral pyridoxine in pediatric emergency departments notes its use in children for acute seizures secondary to pyridoxine dependency or responsiveness, but does not provide new safety data [\hyperlink{pmid_16160662}{PMID: 16160662}, Sidney M Gospe et al., 2005].

Summary by age range:
- Infants and children (up to 12 years): Multiple studies, including randomized controlled trials and case reports, have administered pyridoxine hydrochloride in various doses (from 20 mg/day to 300 mg/kg/day) for conditions such as atopic dermatitis, febrile convulsions, recurrent seizures, and infantile spasms. Across these studies, no serious adverse effects or safety concerns were reported, and side effects (when present) were mild and reversible. Therefore, based on the available abstracts, pyridoxine hydrochloride appears to be safe for use in children, including infants as young as 3 months, when used in the studied contexts and dosages.

- Adolescents: No targeted safety studies in adolescents (13-18 years) were identified in the abstracts.

Conclusion: There is evidence from targeted studies affirming the safety of pyridoxine hydrochloride in infants and children up to 12 years of age. Safety in adolescents is unknown based on the available abstracts.

\subsection*{Abstracts}
\hypertarget{pmid_8555030}{A} previous study has reported benefit when pyridoxine hydrochloride was given to patients with atopic dermatitis. To investigate this in children, we performed a randomized, double-blind, parallel-group, placebo-controlled trial. Forty-eight children with moderate or severe atopic dermatitis were recruited and, of those who completed the study, 19 received pyridoxine hydrochloride 50 mg once daily for 4 weeks and 22 received placebo. Disease activity was monitored by clinical severity scores measuring the extent and degrees of erythema recorded by the investigator and symptom scores (daytime itch and nocturnal sleep disturbance) recorded by parents. There was no statistically significant difference between the two groups at the end of treatment. We have been unable to demonstrate clinical benefit from pyridoxine supplementation in children with atopic dermatitis. [\hyperlink{Pyridoxine Hydrochloride}{PMID: 8555030}, D C Mabin et al., 1995]

\hypertarget{pmid_7009019}{A} total of 107 children who had been hospitalized following a febrile convulsion were enrolled into the trial. By random allocation, 55 children were treated with pyridoxine hydrochloride (20 mgs twice daily) and the remaining 52 children were treated with a placebo until there had been either a further convulsion or a year had passed without recurrence. Eighty children were adequately followed up and of these, 17 had a recurrent febrile convulsion while receiving medication. Recurrences occurrences occurred in 7 of the 38 children receiving pyridoxine and in 10 of the 42 children receiving placebo (X2 = .346, p greater than 0.5). Initial tryptophan load tests had been abnormal in 34 children, and of these, recurrences occurred in 3 of the 17 who received pyridoxine and in 3 of the 17 who received placebo. It has yet to be shown that pyridoxine supplementation protects children from recurrent febrile convulsions. [\hyperlink{Pyridoxine Hydrochloride}{PMID: 7009019}, J McKiernan et al., 1981]

\hypertarget{pmid_8010205}{T}he purpose of this prospective study was to evaluate the safety and efficacy of thioridazine as an adjunct to chloral hydrate sedation when children undergoing MR imaging are difficult to sedate. All 87 children in the study either could not be sedated with chloral hydrate alone or were mentally retarded. Thioridazine (2-4 mg/kg) was administered orally 2 hr before and chloral hydrate (50-100 mg/kg) was administered orally 30 min before the 104 MR examinations. All children were monitored by continuous pulse oximetry. All images were individually evaluated by pediatric radiologists and were graded acceptable if they contained only minimal motion artifact or no motion artifact. Studies were considered successful only when 95\% or more of the images were acceptable. MR imaging was successful in 93 (89\%) of 104 examinations. The success rate for children entered into the study because of prior failure of chloral hydrate sedation was not significantly different from the success rate for children with mental retardation. A tendency for increasing failure rate with age was not significant. No serious complications occurred during the study. The most common adverse reaction, transient reduced oxygen saturation, was seen in five children. Other adverse effects encountered were vomiting in four children, hyperactivity in two children, transient tachycardia in one child, and prolonged sedation in one child. No child required hospitalization because of an adverse reaction to sedation. The study indicates that thioridazine is a safe and effective adjunct to chloral hydrate when a child undergoing MR imaging is difficult to sedate. [\hyperlink{Pyridoxine Hydrochloride}{PMID: 8010205}, S B Greenberg et al., 1994]

\hypertarget{pmid_9308977}{T}o determine the efficacy of pyridoxine in treating seizures, 90 infants and children with recurrent convulsions primarily due to acute infectious diseases were enrolled in the present study. Forty patients were treated with high-dose pyridoxine (30 or 50 mg/kg/day) by intravenous infusion, and 50 subjects served as controls. Antiepileptic drugs and other therapies were similar in the two groups except for pyridoxine. Clinical efficacy criteria were based on the frequency of convulsions per day and on the duration of individual seizures after therapy was initiated. The results indicated that total response rates in the pyridoxine group and control group were 92.5\% and 64\%, respectively (chi-square = 14.68, P < .001). After initiation of therapy, seizures resolved after 2.4 +/- 1.4 days in the pyridoxine group and after 3.7 +/- 2.0 days in the control group (t = 3.67, P < .001). No adverse effects of pyridoxine were apparent during the observation period. We conclude that pyridoxine is an effective, safe, well-tolerated, and relatively inexpensive adjunct to routine antiepileptic drugs for treatment of recurrent seizures in children. [\hyperlink{Pyridoxine Hydrochloride}{PMID: 9308977}, F Y Jiao et al., 1997]

\hypertarget{pmid_10323625}{Y}oung children often appear bothered by ear pain during ascent and descent while traveling on commercial airplanes. While pseudoephedrine hydrochloride is effective in decreasing the risk for earache in adults with recurrent air travel-associated ear pain, such use in children has not been studied. To assess the efficacy and side effects of prophylactic pseudoephedrine in children traveling by air. A placebo-controlled, double-blind clinical trial. Children aged 6 months to 6 years were included in this study. Pseudoephedrine hydrochloride (1 mg/kg body weight) or placebo was administered 30 to 60 minutes prior to departure on commercial air flights. Caregivers noted historical details and the degree of apparent ear pain, drowsiness, and excitability with ascent and descent. Ninety-one flights involving 50 children were studied, with ear pain being reported in 13 (14\%) of flights. Ear pain was not associated with a history of air travel-associated ear pain, recent ear infection, or recent upper airway symptoms. Pseudoephedrine use was not associated with a decrease in ear pain during either ascent (4\% with pseudoephedrine vs 5\% with placebo; P approximately 1.00) or descent (12\% with pseudoephedrine vs. 13\% with placebo; P approximately 1.00). Pseudoephedrine use was, however, linked to drowsiness at takeoff (60\% with pseudoephedrine vs. 27\% with placebo; P = .003) but not at landing (P = .39). Treatment was not associated with excitability at takeoff (P = .09) or landing (P approximately 1.00). Ear pain is not uncommon in children traveling by commercial aircraft. The predeparture use of pseudoephedrine does not decrease the risk for in-flight ear pain in children but is associated with drowsiness. [\hyperlink{Pyridoxine Hydrochloride}{PMID: 10323625}, B J Buchanan et al., 1999]

\hypertarget{pmid_16160662}{P}yridoxine is a recommended antidote that should be available in emergency departments (EDs). A pediatric use of this preparation is the treatment of acute seizures secondary to pyridoxine dependency or responsiveness. Two cases of children with pyridoxine-dependent and pyridoxine-responsive seizures whose treatment was affected by the unavailability of pyridoxine in local EDs are presented. These cases prompted the development of a survey to ascertain the availability of parenteral pyridoxine in the pharmacies and EDs of both children's and general hospitals in the United States. A survey of 203 pharmacy directors in 100 pediatric hospitals (42 self-governing and 58 within a hospital) and 103 general hospitals was conducted. The questionnaire asked for the number of licensed beds and whether injectable pyridoxine was on the formulary and stocked by the ED. The overall response rate was 73\% (83\% pediatric and 64\% general hospitals). Injectable pyridoxine was on the formulary of 99\% of pediatric hospitals and 91\% of general hospitals (P = 0.044). Of those hospitals that had pyridoxine on the formulary, the availability of injectable pyridoxine in EDs was low in both pediatric (20.7\%) and general hospitals (16.7\%). Given the number of possible uses of parenteral pyridoxine in the ED, it is suggested that there is a case for all pediatric and general hospital pharmacies to have it on the formulary and further for all EDs in these hospitals to have injectable pyridoxine available for immediate use. [\hyperlink{Pyridoxine Hydrochloride}{PMID: 16160662}, Sidney M Gospe et al., 2005]

\hypertarget{pmid_7857353}{T}he efficacy and safety of pidotimod ((R)-3-[(S)-(5-oxo-2-pyrrolidinyl)carbonyl]-thiazolidine-4-carboxylic acid, PGT/1A, CAS 121808-62-6) were rated in a child population with a remote history of recurrent respiratory infections (RRI). This randomized double-blind multicenter clinical trial versus placebo, stratified by age groups, involved 748 children recruited in 69 Medical Centres. The trial consisted of a 60-day treatment period and a 90-day follow-up. At the end of the treatment period the pidotimod group showed a significant decrease in the number of RRI episodes and associated symptoms vs control group. As a consequence, there was a significant decrease in the number of days of absence from kindergarten or school and in the consumption of antibiotics and symptomatic drugs. Safety was good. The effect of the drug persisted after its withdrawal throughout the whole 90-day follow-up period. During this period there was a significantly lower RRI incidence rate in the pidotimod group than in the placebo group (p < 0.01). Because of its efficacy and safety, pidotimod may be rated as an excellent drug in the RRI management in children. [\hyperlink{Pyridoxine Hydrochloride}{PMID: 7857353}, P Careddu et al., 1994]

\hypertarget{pmid_16176855}{S}tudies of children indicate that exposure of the general population to low levels of polychlorinated dibenzo-p-dioxins and dibenzofurans (PCDD/Fs) does not result in any clinical evidence of disease, although accidental exposure to high levels either before or after birth have led to a number of developmental deficits. Breast-fed infants have higher exposures than formula-fed infants, but studies consistently find that breast-fed infants perform better on developmental neurologic tests than their formula-fed counterparts, supporting the well-recognized benefits of breast feeding. Children receive higher exposures to PCDD/Fs from food than adults on a body-weight basis but those exposures are below the World Health Organization's tolerable daily intake. Laboratory rodents appear to be at least an order of magnitude more sensitive than humans to the aryl hydrocarbon receptor-mediated effects of these substances, which makes them poor surrogates for predicting quantitative risks but makes them good models for establishing safe levels of human exposure by organizations mandated to protect public health. Any exposure limit for PCDD/Fs based on developmental toxicity in sensitive laboratory animals can be expected to be especially protective of human health, including the health of infants and children. Because body burdens and environmental levels continue to decline, it is unlikely that children alive today in the USA will experience exposures to PCDD/Fs that are injurious to their health. [\hyperlink{Pyridoxine Hydrochloride}{PMID: 16176855}, Gail Charnley et al., 2006]

\hypertarget{pmid_29547159}{W}est syndrome is a catastrophic epilepsy syndrome characterized by infantile spasms, hypsarrhythmia, and developmental arrest or regression. The aim of this study was to explore the role of pyridoxine in the management of infantile spasms. This was a pilot, randomized, open-label trial conducted at a tertiary level hospital from November 2012 to March 2014. Children aged 3 months to 3 years presenting with infantile spasms in clusters (at least 1 cluster/day) with hypsarrhythmia or its variants on electroencephalogram (EEG) were enrolled. The study participants were randomized to receive either oral prednisolone (4 mg/kg/day) alone or 30 mg/kg/day of pyridoxine with oral prednisolone. The primary outcome measure was the proportion of children who achieved spasm freedom for 48 h on day-14 after treatment initiation, as per parental reports, in both the groups. The adverse effects were also monitored. The study was registered with clinicaltrials.gov (ClinicalTrials.gov Identifier: NCT01828437). Sixty-two children were randomized into the two groups with comparable baseline characteristics. The proportion of children with spasm cessation on day-14 was similar in the two groups (39 vs. 37\%, P = 0.98). The adverse effects were comparable in both the groups. The combination of pyridoxine with oral prednisolone was not found to be a beneficial therapy as compared to prednisolone alone in the treatment of infantile spasms in this pilot study. However, high dose pyridoxine may be safe in children with infantile spasms. [\hyperlink{Pyridoxine Hydrochloride}{PMID: 29547159}, Vedavathi Kunnanayaka et al., ]

\hypertarget{pmid_7857354}{T}he efficacy and safety of a new synthetic immunostimulant pidotimod ((R)-3-[(S)-(5-oxo-2-pyrrolidinyl) carbonyl]-thiazolidine-4-carboxylic acid, PGT/1A, CAS 121808-62-6) in recurrent infections of the primary airways were assessed in a group of 416 children with a history of recurrent respiratory infections (RRI). This was a double-blind randomized trial of pidotimod vs. placebo, consisting of a treatment period of 60 days and a follow-up period of 3 months. A reduction in the duration and frequency of infectious episodes in the group of children treated with pidotimod (one 400 mg oral bottle daily) was observed which was statistically different from the placebo group. The protective effect produced by pidotimod was also confirmed by a series of recordings made over the five-month observation period, which showed a significant reduction in the number of days of fever, the severity of the signs and symptoms of acute episodes, use of antibiotics and antipyretic drugs and absence from school or nursery school. Safety was excellent. [\hyperlink{Pyridoxine Hydrochloride}{PMID: 7857354}, D Passali et al., 1994]

\hypertarget{pmid_28590988}{V}ilazodone hydrochloride is the first member in a new class of antidepressants called indolealkylamines and was approved for use in the United States in 2011 for major depressive disorder. It has a combined mechanism of action of a selective serotonin reuptake inhibitor and a partial agonist of serotonin 5-HT1A receptors. It has not been approved for use in the pediatric population, and toxicity from exploratory vilazodone ingestion has been rarely described to date. We describe 2 children with laboratory-confirmed vilazodone ingestions that led to significant toxicity including refractory status epilepticus in 1 patient and likely transient seizure activity in the other. Both patients required multiple doses of benzodiazepines; in the more severe case, barbiturates were added to control seizure activity. These children returned to baseline and had no prolonged neurologic complications. Pediatric experience with vilazodone is limited; however, the literature demonstrates 3 additional case reports of children experiencing seizure after vilazodone ingestion. With the 2 new cases presented here, it seems prudent to educate prescribers and families of the potential dangers of ingestion of vilazodone tablets by young children. [\hyperlink{Pyridoxine Hydrochloride}{PMID: 28590988}, Jeannine Del Pizzo et al., 2018]

\hypertarget{pmid_3686545}{P}yridoxine hydrochloride was gavaged to 2 groups of pregnant Wistar rats from day 0 to 13 and day 6 to 15 of gestation at doses of 100, 200, 400 and 800 mg/kg. A higher number of implantations, live pups and corpora lutea were observed in the treated rats, but a significant reduction in the body weights of the pups was noticed in the groups treated with 400 and 800 mg/kg. No other adverse effects on implantation and pregnancy were noticed. No evidence of dismorphogenic effects was seen. [\hyperlink{Pyridoxine Hydrochloride}{PMID: 3686545}, M R Marathe et al., 1987]

\hypertarget{pmid_94242}{P}yridoxine, one of the B vitamins, has been shown to be useful in the treatment of childhood bronchial asthma by Collip et al. (1975). A double-blind study with 76 asthmatic children followed for five months indicated significant improvement in asthma following pyridoxine therapy (200 mg daily) and a reduction in dosage of bronchodilators and cortisone. Other reports have shown that nicotinamide, another B vitamin shows inhibitory activity in rat mast cell degranulation and histamine release (Bekier et al. 1974, Wiczolkowska and Maslinski, 1975, 1976). These results induced us to investigate if pyridoxine, like nicotinamide or disodium cromoglycate, exhibits pharmacological inhibitory activity in rat mast cell degranulation and histamine release induced by antigen or other non-immunological stimulants. We found that pyridoxine at concentrations of 10 (-3) M, or greater significantly inhibited rat mast cell degranulation and histamine release induced by phospholipase A, compound 48/80, antigen (egg albumin) or a mixture of dextran and phosphatidyl serine, respectively. In these experimental models, pyridoxine shows a pharmacological profile similar to nicotinamide and disodium cromoglycate, although weaker than the latter. In spite of this, the lack of toxicity of this vitamin at relatively high doses (1 or 1.5 g), the possibility that other mechanisms of action may be involved, such as the improvement in tryptophan metabolism reported by Collip following pyridoxine therapy, suggest that this vitamine merits additional research. [\hyperlink{Pyridoxine Hydrochloride}{PMID: 94242}, M Garcia et al., ]

\hypertarget{pmid_21516020}{H}ydroxyurea is a safe and efficacious medication for children with sickle cell disease (SCD). Our objective was to compare health-related quality of life (HRQL) between children taking hydroxyurea and those not taking hydroxyurea. We conducted a retrospective cohort study of children with SCD who had completed the PedsQL 4.0 at Duke University Medical Center or the Midwest Sickle Cell Center. Our primary outcome was HRQL in children receiving hydroxyurea therapy compared with those not receiving hydroxyurea. One hundred and ninety-one children with SCD were included in the study. Children in the hydroxyurea group had higher self-reported Total PedsQL median scores than children in the no hydroxyurea group (P=0.04). Child self-reported physical functioning scores were significantly higher for children in the hydroxyurea group (P=0.01). In conclusion, children with SCD who received hydroxyurea therapy reported better overall HRQL and better physical HRQL than children who did not receive this therapy despite disease severity. Further research assessing the impact of hydroxyurea therapy on HRQL, such as prospective assessment over time, would aid in our understanding of the effectiveness of hydroxyurea for individual children. Ultimately, this may aid in decreasing the barriers to the use of hydroxyurea. [\hyperlink{Pyridoxine Hydrochloride}{PMID: 21516020}, Courtney D Thornburg et al., 2011]

\hypertarget{pmid_17941284}{T}he safety of fexofenadine has been examined extensively in adults and school-age children. However, the safety of fexofenadine in children younger than 6 years has not been reported to date. To compare the safety and tolerability of twice-daily fexofenadine hydrochloride, 30 mg, and placebo in preschool children aged 2 to 5 years with allergic rhinitis. This was a multicenter, double-blind, randomized, placebo-controlled, parallel-group study, conducted between February 29, 2000, and June 14, 2001. Participants were randomized to either fexofenadine hydrochloride, 30 mg, or placebo twice daily for a 2-week period. To facilitate dosing, capsule content was mixed with applesauce (approximately 10 mL). Safety assessments depended on date of entry into the study because of an amendment to the protocol. Before the amendment, assessments included physical examination, vital signs reporting (oral temperature, heart rate, and respiratory rate), and adverse event (AE) reporting. After the amendment, safety assessments included laboratory testing (blood chemistry and hematology profiles), physical examination, 12-lead electrocardiography, and vital signs (oral temperature, blood pressure, heart rate, and respiratory rate) and AE reporting. Treatment-emergent AEs were observed in 116 of 231 participants receiving placebo and 111 of 222 receiving fexofenadine. These AEs were possibly related to study medication in 19 (8.2\%) and 21 (9.5\%) of the participants receiving placebo and fexofenadine, respectively, and most frequently involved the digestive system. No clinically relevant differences in laboratory measures, vital signs, and physical examinations were observed. The findings show that fexofenadine hydrochloride, 30 mg, is well tolerated and has a good safety profile in children aged 2 to 5 years with allergic rhinitis. [\hyperlink{Pyridoxine Hydrochloride}{PMID: 17941284}, Henry Milgrom et al., 2007]

\hypertarget{pmid_11423814}{P}yridoxine hydrochloride, the antidote for isonicotinic acid hydrazide (INH)--induced seizures, is available in solution at a concentration of 100 mg/mL at a pH of less than 3. Pyridoxine is often infused rapidly in large doses for INH-induced seizures. Effects of pyridoxine infusion on base deficit in amounts given for INH poisoning have not been studied in human subjects. We hypothesized that this infusion would result in transient worsening of acidosis. We conducted a randomized, controlled crossover trial in human volunteers. Five healthy volunteers (mean age, 35 years; range, 29 to 43 years) were randomized to receive intravenous placebo (50 mL of normal saline solution) or 5 g of pyridoxine (50 mL) over 5 minutes. A peripheral intravenous catheter was established in each arm, and a heparinized venous blood sample was obtained for base deficit at baseline and 3, 6, 10, 20, and 30 minutes after infusion. After at least a 1-week washout period, the volunteers were assigned to the alternate arms of the experiments, thus acting as their own control subjects. Data were analyzed by using the 2-tailed paired t test, controlling for multiple comparisons. No difference was noted between groups at baseline. A statistically significant increased base deficit was noted after the pyridoxine infusion versus control at 3 to 20 minutes but not at 30 minutes (P =.1). Maximal mean increase in base deficit (2.74 mEq/L) was noted at 3 minutes. A transient increase in base deficit occurs after the infusion of 5 g of pyridoxine in normal volunteers. [\hyperlink{Pyridoxine Hydrochloride}{PMID: 11423814}, F Lovecchio et al., 2001]

\hypertarget{pmid_18365604}{P}seudoephedrine hydrochloride (PEH) is a sympathomimetic agent that is widely used in common cold disease in children. Though side effects of PEH are well known, it is preferred by many pediatricians in order to benefit from its symptomatic relief in common cold disease. A case of acute urinary retention due to PEH in a three-year-old boy is reported. The aim of this case report is to emphasize the clinical importance and differential diagnosis of PEH overdose in children and to discuss the appropriate treatment approach to PEH overdose in the emergency department. [\hyperlink{Pyridoxine Hydrochloride}{PMID: 18365604}, Tutku Soyer et al., ]

\hypertarget{pmid_2295577}{F}luoxetine hydrochloride is the first selective serotonin uptake inhibitor introduced commercially in the United States. This report describes preliminary clinical experience with fluoxetine in 10 children and adolescents, aged 8 to 15 years, with primary obsessive compulsive disorder (OCD) or Tourette's syndrome (TS) plus OCD. In general, fluoxetine, which was administered from 4 to 20 weeks at a dosage of 10 or 40 mg per day, was well tolerated. Adverse effects included behavioral agitation/activation in four patients and mild gastrointestinal symptoms in two patients. No abnormalities were noted in the seven children who had follow-up EKGs. Five of the 10 patients (50\%) were considered responders; their obsessive-compulsive symptoms decreased substantially during treatment with fluoxetine. Responder rates were similar in the primary OCD (two of four, 50\%) and TS + OCD (three of six, 50\%) groups. In conclusion, short-term fluoxetine administration appears to be safe in children and adolescents. Placebo-controlled trials are needed to further assess the efficacy of fluoxetine. [\hyperlink{Pyridoxine Hydrochloride}{PMID: 2295577}, M A Riddle et al., 1990]

\hypertarget{pmid_32925756}{B}romhexine hydrochloride tablets may be effective in the treatment of Coronavirus disease 2019 (COVID-19) in children. This study will further evaluate the efficacy and safety of bromhexine hydrochloride tablets in the treatment of COVID-19 in children. The following electronic databases will be searched, with all relevant randomized controlled trials (RCTs) up to August 2020 to be included: PubMed, Embase, Web of Science, the Cochrane Library, China National Knowledge Infrastructure (CNKI), the Chongqing VIP China Science and Technology Database (VIP), Wanfang, the Technology Periodical Database, and the Chinese Biomedical Literature Database (CBM). As well as the above, Baidu, the International Clinical Trials Registry Platform (ICTRP), Google Scholar, and the Chinese Clinical Trial Registry (ChiCTR) will also be searched to obtain more comprehensive data. Besides, the references of the included literature will also be traced to supplement our search results and to obtain all relevant literature. This systematic review will evaluate the current status of bromhexine hydrochloride in the treatment of COVID-19 in children, to evaluate its efficacy and safety. This study will provide the latest evidence for evaluating the efficacy and safety of bromhexine hydrochloride in the treatment of COVID-19 in children. CRD42020199805. The private information of individuals will not be published. This systematic review will also not involve endangering participant rights. Ethical approval is not available. The results may be published in peer-reviewed journals or disseminated at relevant conferences. [\hyperlink{Pyridoxine Hydrochloride}{PMID: 32925756}, Yuying Wang et al., 2020]

\hypertarget{pmid_17063023}{E}vidence on the caries-preventive effect of chlorhexidine (CHX) among high-risk children is inconclusive, possibly because obscured by fluoride exposure. We investigated the effect of CHX among initially 3-year-old subjects whose baseline d(3)ft was = 0 and whose only regular fluoride exposure came from toothpaste. The subjects were assigned to three groups: high-risk test (HRT, n = 70), high-risk control (HRC, n = 71), and low-risk control (LRC, n = 70). Risk classification was based on salivary mutans streptococcal levels (MS, </>or=1.0 x 10(5) cfu/ml). Basic measures (oral hygiene, dietary counselling every 4 months) were given to all groups. HRT also underwent CHX gel applications for 3 consecutive days at 3-month intervals for 15 months. Eighteen months after baseline d(3)ft increments and proportions of children with d(3)ft increment >or=1 (\%d(3)ft increment >or=1) among all groups were assessed. Anti-MS effect on high-risk children and caries-preventive effect on all children were statistically analysed by residual change analysis (MS), non-parametric tests and logistic regression analysis (caries). No differences were found between the groups in basic programme compliance. CHX significantly reduced MS levels. \%d(3)ft increment >or=1 and mean d(3)ft increments were 34.3\%, 0.56 (HRT), 32.4\%, 0.54 (HRC) and 11.4\%, 0.11 (LRC), with HRT/HRC values statistically significantly higher than LRC values and no significant difference between HRT and HRC. HRT children were not less likely to show new lesions than HRC children (OR = 1.09; 95\% confidence interval 0.54-2.19), while high-risk children were 4 times more likely to show new lesions than low-risk children (OR = 3.71; 95\% confidence interval 1.53-9.03). CHX gel applications showed moderate anti-MS effect but negligible caries-preventive effect. [\hyperlink{Pyridoxine Hydrochloride}{PMID: 17063023}, S Petti et al., 2006]

\hypertarget{pmid_10851644}{C}iprofloxacin clinical and bacteriological efficacies, as well as tolerability mainly with respect to chondrotoxicity were evaluated in the treatment of children with mucoviscidosis. The drug was shown to have high clinical and moderate bacteriological efficacies. As for its tolerability, adverse reactions chiefly associated with affection of the gastrointestinal tract, i.e. nausea, stomach pain, diarrhea, increased transaminase levels were recorded. The arthrotoxicity episode was single and transitory. The use of ciprofloxacin had no negative effect on the children growth. No chondrotoxic effect of ciprofloxacin in the treatment of children was observed which is explained in the paper. It is concluded that ciprofloxacin is in general an efficient and safe antibiotic useful for the treatment of children. [\hyperlink{Pyridoxine Hydrochloride}{PMID: 10851644}, S S Postnikov et al., 2000]

\hypertarget{pmid_8330589}{H}igh-dose vitamin B6 (pyridoxine-HCl, 300 mg/kg/day orally) was introduced as the initial treatment of recently manifested infantile spasms in 17 children (13 symptomatic cases with identified brain lesion and 4 cryptogenic cases). 5 of 17 children (2 cryptogenic, 2 with severe pre/perinatal brain damage and one with Sturge-Weber syndrome) were classified as responders to high-dose vitamin B6. In all 5 cases the response to vitamin B6 occurred within the first 2 weeks of treatment and within 4 weeks all patients were free of seizures. Two patients developed other seizures (partial seizures, etiologically unclear blinking attacks), but no relapse of infantile spasms was observed among the five responders to vitamin B6. No serious adverse reactions were noted. Side effects were mainly gastrointestinal symptoms, which were reversible after reduction of the dosage. Considering the life-threatening side effects of treatment with ACTH/corticosteroids or valproate, a controlled clinical trial with high-dose vitamin B6 would appear justified to either prove or disprove efficacy. [\hyperlink{Pyridoxine Hydrochloride}{PMID: 8330589}, J Pietz et al., ]

\hypertarget{pmid_34506721}{T}he contents of perchlorate and chlorate were determined in a total of 278 samples of infant formulas marketed in China. The associated health risk via infant and young child formulas consumption for 0-36 month old children in China was also assessed. The contents of perchlorate and chlorate were measured by a validated method with LC-MS and the limit of detection (LOD) was 1.5 μg kg [\hyperlink{Pyridoxine Hydrochloride}{PMID: 34506721}, Qing Liu et al., 2021] A case of acute poisoning in a 32 months old child, with generalised and uncontrollable seizures is reported. Pyridoxine IV is efficacely used in such poisoning. [\hyperlink{Pyridoxine Hydrochloride}{PMID: 34506721}, E Hartemann et al., ]

\hypertarget{pmid_28741653}{C}hloral hydrate is commonly used to sedate children for painless procedures. Children may recover more quickly after sedation with dexmedetomidine, which has a shorter half-life. We randomly allocated 196 children to chloral hydrate syrup 50 mg.kg [\hyperlink{Pyridoxine Hydrochloride}{PMID: 28741653}, V M Yuen et al., 2017]

\section*{Terbutaline Sulfate}
\subsection*{Result}
\subsubsection*{Answer}

Infants (<1 year): Unknown
Children (1–16 years): Yes

\subsubsection*{{Explanation}}
\hypertarget{Terbutaline Sulfate}
Based on the abstracts available, there are multiple targeted studies evaluating the safety of Terbutaline Sulfate in children with asthma, across various age ranges. Here is a summary by age group:

Infants (<1 year):
- One case report describes a five-month-old girl who developed focal seizures after taking terbutaline sulfate syrup, with no other cause found. The seizures stopped after discontinuing the drug, suggesting a possible adverse effect [\hyperlink{pmid_37719479}{PMID: 37719479}, Falah Nadeem et al., 2023]. However, this is a single case report and not a controlled safety study, so definitive safety or lack thereof cannot be established for this age group.

Toddlers and Young Children (1–5 years):
- A randomized, controlled trial in children aged 2–5 years with acute asthma found that continuous nebulization of terbutaline sulfate was effective and showed evidence of absorption, but did not report significant adverse effects [\hyperlink{pmid_9677633}{PMID: 9677633}, J P Lotufo et al., 1998].
- Another study included children as young as 1 year and 7 months (up to 10 years) and found nebulized terbutaline (5 mg) to be safe and effective for acute asthma, with no significant changes in pulse rate or blood pressure [\hyperlink{pmid_2284942}{PMID: 2284942}, B F Lee et al.].
- A study in children aged 3–6 years compared terbutaline delivery methods and found no significant safety concerns [\hyperlink{pmid_2677624}{PMID: 2677624}, J Pendergast et al., 1989].
- A pharmacokinetic study in children (age not specified, but context suggests young children) found that a dose of 0.25 mg (max 5 mg) was safe and non-toxic [\hyperlink{pmid_6838255}{PMID: 6838255}, R Dinwiddie et al., 1983].

Children (6–12 years):
- Multiple studies in children aged 4–13 years, 3–10 years, and 6–14 years found inhaled, oral, and intravenous terbutaline to be safe and effective for asthma, with only mild or no significant side effects reported [\hyperlink{pmid_2751767}{PMID: 2751767}, P Phanichyakarn et al., 1989; \hyperlink{pmid_8875584}{PMID: 8875584}, E Ståhl et al., 1996; \hyperlink{pmid_14689023}{PMID: 14689023}, D Solé et al.; \hyperlink{pmid_3128951}{PMID: 3128951}, J Portnoy et al., 1988; \hyperlink{pmid_7308338}{PMID: 7308338}, M I Blackhall et al., 1981; \hyperlink{pmid_16203672}{PMID: 16203672}, Mamatha Kambalapalli et al., 2005; \hyperlink{pmid_7985255}{PMID: 7985255}, B Hertz et al., 1994; \hyperlink{pmid_7211775}{PMID: 7211775}, F E Simons et al., 1981; \hyperlink{pmid_15572812}{PMID: 15572812}, Prakash Chandra et al., 2004; \hyperlink{pmid_6586481}{PMID: 6586481}, G Lönnerholm et al., 1984; \hyperlink{pmid_3789327}{PMID: 3789327}, G Wettrell et al., 1986].
- One study in children aged 1–12 years with severe asthma found intravenous terbutaline to be safe, though some patients required potassium supplementation for hypokalemia [\hyperlink{pmid_16203672}{PMID: 16203672}, Mamatha Kambalapalli et al., 2005].
- Another study in children aged 1–12 years compared terbutaline to other treatments for severe asthma and found it less safe than magnesium sulfate (due to hypokalemia in two patients), but still effective [\hyperlink{pmid_25164315}{PMID: 25164315}, Sunit Singhi et al., 2014].

Adolescents (13–16 years):
- Some studies included children up to 14 or 16 years old and found terbutaline to be safe and effective for asthma, with only mild side effects [\hyperlink{pmid_334490}{PMID: 334490}, W J Davis et al., 1977; \hyperlink{pmid_7308338}{PMID: 7308338}, M I Blackhall et al., 1981].

Summary:
- For children aged 1–16 years, there are multiple controlled studies affirming the safety of terbutaline sulfate for short-term use in acute asthma, with only mild and manageable side effects (e.g., tremor, mild tachycardia, occasional hypokalemia).
- For infants (<1 year), there is insufficient controlled data; a single case report suggests possible risk of seizures, but this is not definitive.
- For children under 1 year, safety is unknown. For children 1–16 years, evidence supports safety for use in asthma, with monitoring for known side effects.

\subsection*{Abstracts}
\hypertarget{pmid_9677633}{T}his study investigated the efficacy of a system for continuous nebulization of terbutaline sulphate in the treatment of acute asthmatic crises in children. The equipment consisted of a condensation nebulizer attached to a 40 liter acrylic tent placed around the patient's head. A prospective, randomized and open clinical trial was conducted. Twenty eight children, 2 to 5 year-old, in acute asthmatic crises were selected. Fourteen were nebulized with terbutaline sulphate while in the control group the aerosolization was proceeded only with half diluted physiologic serum. All patients were administered aminophyline intravenously. The parameter used to evaluate the efficacy of the terbutaline sulphate nebulizing system was clinical improvement measured by the Wood-Downes Score. Two additional parameters indicating terbutaline sulphate absorption were used: reduction of potassium seric levels and positive chronotropic effect. The group treated with terbutaline sulphate showed greater clinical improvement than control group at the 12 hour protocol evaluation as well as lower seric potassium level. A positive chronotropic effect was also observed at the final protocol evaluation. The data showed, preliminarily, that (a) the system for continuous nebulization of terbutaline sulphate was effective in treatment of children's acute asthmatic crises, and (b) there was evidence attesting to the absorption of terbutaline sulphate by the children treatment with it. [\hyperlink{Terbutaline Sulfate}{PMID: 9677633}, J P Lotufo et al., 1998]

\hypertarget{pmid_37719479}{T}erbutaline sulfate is a beta-adrenergic receptor agonist. More specific for B2 receptors, it is used as a bronchodilator in asthma. Its known side effects can include dizziness, tremors, and tachycardia. However, seizures are not among the commonly reported side effects. This is the case of a five-month-old girl who presented with focal seizures after the intake of terbutaline sulfate syrup. Other causes of the seizures were excluded through history and investigations, including an EEG and electrolyte panel. The seizures stopped on cessation of the terbutaline sulfate with no recurrence, leading us to believe that the focal seizures were an adverse effect of the terbutaline sulfate. A high index of suspicion for drug-related adverse effects should therefore be kept for a child with new onset focal seizures. [\hyperlink{Terbutaline Sulfate}{PMID: 37719479}, Falah Nadeem et al., 2023]

\hypertarget{pmid_2751767}{T}hirty asthmatic children, aged 4 to 13 years, 22 boys and 8 girls, were studied during acute asthmatic attacks. Each group of 15 children received either a 0.01 mg/kg subcutaneous injection of terbutaline or 2 puffs from terbutaline pressurized aerosol (0.25 mg/puff) inhaler through a 750-ml volumetric spacer. A slightly greater increase in PEFR following injection compared with inhalation throughout the 6 hours study period was observed. Significant increases in systolic blood pressure and pulse rate were observed only after injection. Therefore, it was concluded that inhaled terbutaline is safe and effective for treating children over 4 years of age with acute bronchospasm and has less cardiovascular side effects than injected terbutaline. [\hyperlink{Terbutaline Sulfate}{PMID: 2751767}, P Phanichyakarn et al., 1989]

\hypertarget{pmid_8875584}{T}he purpose of this study was to investigate the relative effectiveness of 0.25 mg, 0.5 mg, and 1.0 mg of terbutaline, administered via Turbuhaler, in children with mild to moderate asthma, and to register peak inspiratory flow rates through Turbuhaler (PIFTBH). Thirty-seven children in Portugal (one center) and 45 children in Sweden (one center) aged 3-10 years participated in two separate, double-blind, placebo-controlled, crossover, and randomized studies of the same design. Because of differences in other therapies for asthma and climate, combination of the two studies into one metanalysis did not appear appropriate. The children inhaled 0.25 mg, 0.5 mg, and 1.0 mg terbutaline sulfate and placebo t.i.d. for consecutive 2-week periods without washout periods. Peak expiratory flow rates (PEF) were measured at home before and 15 minutes after each inhalation in the morning, afternoon, and evening. PIFTBH was measured twice at each of four clinic visits. At the Portuguese center the increases in mean morning PEF from before to after inhalation were 32 L/min after 0.25 mg, 35 L/min after 0.5 mg, and 40 L/min after 1.0 mg. The corresponding figures in Sweden were 26 L/min, 31 L/min, and 29 L/min after 0.25 mg, 0.5 mg, and 1.0 mg, respectively. For children 3-6 years, mean values for PIFTBH were 60 L/min in Portugal (n = 15), and 58 L/min in Sweden (n = 23). In the 7-10 year group the mean PIFTBH was 72 L/min (n = 22) in Portugal, and 68 L/min (n = 22) in Sweden. We conclude that inhalation of terbutaline sulfate via Turbuhaler at a small dose of 0.25 mg resulted in good bronchodilation and was comparable to inhalations of 0.5 mg and 1.0 mg in children aged 3-10 years with mild to moderate asthma. PIFTBH were comparable to values previously recorded in healthy 6-year-old and older children and in adult asthmatic patients. [\hyperlink{Terbutaline Sulfate}{PMID: 8875584}, E Ståhl et al., 1996]

\hypertarget{pmid_334490}{F}orty-eight children with known asthma (ranging in age from 2 to 16 years) were studied during an acute attack. Each received either terbutaline or epinephrine subcutaneously in a random double-blind fashion. Measurement of heart rate, respiratory rate, and systemic arterial systolic and diastolic blood pressures and careful clinical assessment of obstruction of the airway were made before and at 15, 30, and 60 minutes after the administration of the drugs. Appreciable and significant clinical improvement was noted in 19 of the 24 patients in both groups and was of comparable magnitude. A small, but significant, increase in heart rate was noted in those patients requiring only one injection of terbutaline, suggesting that the drug's selectivity for the lung is relative not absolute. The present study demonstrates that terbutaline is an effective bronchodilator drug in acute childhood asthma. [\hyperlink{Terbutaline Sulfate}{PMID: 334490}, W J Davis et al., 1977]

\hypertarget{pmid_7041287}{T}erbutaline sulphate (Bricanyl; Keatings) aerosol or placebo aerosol was administered in a randomized fashion to 26 asthmatic children with proven exercise-induced asthma. The children were then subjected to the modified standard exercise challenge test involving running on the level for 6 minutes. Terbutaline sulphate aerosol had a marked protective effect against exercise-induced asthma in these children. Compared with placebo, a significant reduction in exercise-induced bronchospasm was achieved. The improved design of the mouthpiece, incorporating a newly introduced 'misting tube' enabled the children to handle the apparatus easily. The need to synchronize the activation of the aerosol with inhalation was eliminated. Terbutaline aerosol can be recommended to protect children affected by exercise-induced asthma. The preparation can be given prior to the exercise challenge and will offer prolonged and adequate protection against exercise asthma. [\hyperlink{Terbutaline Sulfate}{PMID: 7041287}, E G Weinberg et al., 1982]

\hypertarget{pmid_7308338}{T}en asthmatic children received regular daily therapy with terbutaline aerosol for 50 weeks. No evidence was found for adverse effects of this drug on growth, bone marrow, liver function or the cardiovascular system. All children had improved respiratory function throughout the year. In acute experiments carried out in 12 symptom-free asthmatic children with 0.5 mg terbutaline, it was demonstrated that the improvement in respiratory function, i.e. increase in FEV1, MMEF25-75\%, FVC and PEF, was quick in onset, was maintained for at least 60 min and was not accompanied by effects on pulse rate. Thus, the bronchodilator aerosol, terbutaline, can be safely used as a regular daily therapy in asthmatic children aged 7 to 14 years. [\hyperlink{Terbutaline Sulfate}{PMID: 7308338}, M I Blackhall et al., 1981]

\hypertarget{pmid_18818954}{A} randomized, open, coordinated multi-center trial compared the bacteriological and clinical efficacy and safety of orally administered ceftibuten and trimethoprim-sulfamethoxazole (TMP-SMX) in children with febrile urinary tract infection (UTI). Children aged 1 month to 12 years presenting with presumptive first-time febrile UTI were eligible for enrollment. A 2:1 assignment to treatment with ceftibuten 9 mg/kg once daily (n = 368) or TMP-SMX (3 mg + 15 mg)/kg twice daily (n = 179) for 10 days was performed. Escherichia coli was recovered in 96\% of the cases. Among the E. coli isolates, 14\% were resistant to TMP-SMX but none to ceftibuten. In the modified intention-to-treat population, the bacteriological elimination rates at follow-up did not differ significantly between patients treated with ceftibuten and those treated with TMP-SMX [91 vs. 95\%, with a 95\% confidence interval (CI) for difference of -9.7 to 1.0]. However, the clinical cure rate was significantly higher among those treated with ceftibuten (93 vs. 83\%, with a 95\% CI for difference of 2.4 to 17.0). Adverse events were similar for both regimens and consisted mainly of gastrointestinal disturbances. In conclusion, ceftibuten is a safe and effective drug for the empirical treatment of febrile UTI in young children. [\hyperlink{Terbutaline Sulfate}{PMID: 18818954}, Staffan Mårild et al., 2009]

\hypertarget{pmid_3128951}{T}welve children with severe asthma were treated in an intensive care unit with continuously nebulized terbutaline at doses between 1.0 and 12.0 mg/hour. All patients showed improvement in blood gases, pulse, and respiratory rates. None experience significant side effects. The duration of therapy ranged from 1 to 24 hours (mean = 8.3 hours), and all were able to leave the intensive care unit within one day. The use of continuously nebulized terbutaline appears to be safe and effective for the treatment of severe asthma in children in this limited experience. [\hyperlink{Terbutaline Sulfate}{PMID: 3128951}, J Portnoy et al., 1988]

\hypertarget{pmid_14689023}{F}orty seven children (6-14 years), with an acute mild or moderate attack of asthma (clinical score 3 or FEV1 > 50\% of the predicted), were treated with terbutaline sulphate, by inhalation route with a dry powder inhaler (Turbuhaler - 0,5 mg - group T; N=27, or by a nebulizer 1\% solution-in saline-compressed air (6 l/min.) group S; N=20. The children were evaluated at 5, 15, 25 and 30 minutes after the initial treatment. In both groups a significant fall of the clinical score (starting at 15 minutes) (p < 0.05) and a significant improvement of the FEV(1), VC and FEF25-75\% (starting at 5 minutes), were observed (p < 0.05). There were no significant changes in heart rates, respiratory rates and blood pressure (p > 0.05). At the end of the first treatment, the number of patients with a FEV(1) < 80\% was similar in both groups (T = 13/27 and S = 10/20). The same treatment was repeated, and all the children showed a marked improvement, except for one boy of the group T was hospitalized. In conclusion, children with mild or moderate acute attacks of asthma can be treated up to a week with an inhalation of dry powder, resulting in adequate bronchodilatation without important side effects. [\hyperlink{Terbutaline Sulfate}{PMID: 14689023}, D Solé et al., ]

\hypertarget{pmid_6838255}{P}lasma terbutaline levels and peak expiratory flow rates were measured in 5 asthmatic children using doses of 0.25 and 0.075 mg/kg. The higher dose resulted in safe, non-toxic plasma levels and returned the peak expiratory flow rate to normal. This dose (maximum 5 mg) is safe in children. [\hyperlink{Terbutaline Sulfate}{PMID: 6838255}, R Dinwiddie et al., 1983] (1) To determine the effect of intravenous terbutaline in children with acute severe asthma on parameters like heart rate, blood pressure, electrocardiogram and serum electrolytes; (2) to assess the safety profile and to evaluate the outcome of children treated with intravenous terbutaline for acute severe asthma. Retrospective study of admission records of children admitted with acute severe asthma who needed intravenous terbutaline. Children's Hospital at the Leicester Royal Infirmary, UK. 77 children with acute severe asthma admitted between April 1999 and October 2002. There was a significant increase in heart rate and significant fall in diastolic blood pressure in this cohort. Four patients required inotropic support. None of the patients had cardiac arrhythmias. Potassium supplements were required in 10 patients due to hypokalaemia. All patients improved and none required initiation of ventilation after commencing terbutaline. There was no mortality in this cohort. Terbutaline was found to be safe for use in this patient group in doses ranging between 1 and 5 microg/kg/min. Intravenous terbutaline was found to be a useful adjunct in those who failed to respond to standard initial therapy. [\hyperlink{Terbutaline Sulfate}{PMID: 6838255}, Mamatha Kambalapalli et al., 2005]

\hypertarget{pmid_6374804}{A} bronchodilator aerosol, terbutaline sulphate, was administered to 18 asthmatic children, mean age 8.0 years (range 4.9-13.7 years) in a cross-over trial using either a common actuator or a 750-ml collapsible spacer. With the spacer, peak expiratory flow rate was significantly higher (p less than 0.05) at 5, 20, and 60 min after administration of 0.25 mg terbutaline sulphate. The mean maximum value during this period was 92\% of the predicted normal value with the spacer compared to 86\% with the common actuator. The difference was significant (p less than 0.01). This study confirms previous findings that a 750-ml spacer is beneficial in bronchodilator aerosol therapy in children with asthma. [\hyperlink{Terbutaline Sulfate}{PMID: 6374804}, K G Hidinger et al., 1984]

\hypertarget{pmid_6586480}{S}even asthmatic children (8-12 years) were given terbutaline sulphate intravenously (5.5 micrograms/kg) and orally (50 micrograms/kg) one week apart. Unchanged terbutaline was measured in plasma and urine. In urine, conjugates were also assayed. The intravenous plasma concentration-time curve declined in a multiexponential manner. The terminal half-life ranged from 8.8 to 15.8 (mean 12.1) h. Body clearance (mean +/- SD) was 3.76 +/- 0.86, renal clearance 2.42 +/- 0.49 mL/min/kg. The volume of distribution at steady state was 1.57 +/- 0.19 L/kg. The extrapolated recovery of intact terbutaline in urine was 65.4 +/- 6.5\% of the dose and the total recovery 80.8 +/- 8.4\%. After oral administration, the recovery of intact terbutaline in urine was 6.2 +/- 1.1\%. Absorption was on average 33\%, but because of a mean first-pass elimination of 70\%, bioavailability was 9.5 +/- 2.4\%. It seems that children as a group have shorter terminal half-lives than adults and slightly higher weight-corrected clearances. [\hyperlink{Terbutaline Sulfate}{PMID: 6586480}, C Hultquist et al., 1984]

\hypertarget{pmid_2284942}{T}o evaluate the therapeutic effect of nebulized terbutaline in children with acute asthma, 21 children, aged 1 year and 7 months to 10 years, with acute asthma, were enrolled into this study, during the period from July to December 1989. Each patient received nebulized terbutaline (Bricanyl) 5 mg/dose over 10 minutes. The respiratory rate, pulse rate, blood pressure, peak expiratory flow rate and clinical severity score were recorded before, and at 10 minutes after treatment. Comparing with the data before treatment, respiratory rate, peak expiratory flow rate and clinical severity score at 10 minutes after treatment showed significant improvement (p value less than 0.05; less than 0.0005; less than 0.0001, respectively), but pulse rate and blood pressure did not differ significantly. It was concluded that administration of nebulized terbutaline, at a dose of 5 mg, was both safe and effective in treating acute asthma, and may be used as the first line measure in treating acute asthma in children. [\hyperlink{Terbutaline Sulfate}{PMID: 2284942}, B F Lee et al., ]

\hypertarget{pmid_7985255}{W}e wanted to assess the protective effects on exercise-induced asthma as well as the clinical efficacy and safety of increasing doses of a new sustained-release formulation of terbutaline sulphate in 17 asthmatic children aged 6-12 years (mean 9 years). Placebo, 2, 4, and 6 mg terbutaline were given b.i.d. for 14 days in a randomized, double-blind, cross-over design. At the end of each two week period, an exercise test was performed and plasma terbutaline was measured. Compared with placebo, no significant effect was seen on asthma symptoms monitored at home, or on exercise-induced asthma. The percentage falls in FEV1 after the exercise test were 36, 35, 27 and 28\%, after placebo, 4, 8 and 12 mg terbutaline/day, respectively. A small but statistically significant dose-related increase was seen in morning and evening peak expiratory flow (PEF) recordings. It is concluded that continuous treatment, even with high doses or oral terbutaline, does not offer clinically useful protection against exercise-induced asthma. [\hyperlink{Terbutaline Sulfate}{PMID: 7985255}, B Hertz et al., 1994]

\hypertarget{pmid_7211775}{I}n a double-blind dose response study in 26 children, 3, 6, or 12 microgram/kg of terbutaline sulfate was compared with 10 microgram/kg of epinephrine administered subcutaneously. In the first hour after injection, all doses of terbutaline and epinephrine resulted in improvement in mean clinical score, mean forced vital capacity, mean forced expiratory volume in the first second, and mean forced expiratory flow from 25\% to 75\% of vital capacity. Terbutaline epinephrine. However, while adverse effects following terbutaline were clinically imperceptible, epinephrine produced unpleasant headache and excitement in a few patients. Terbutaline did not change mean PaO2 or PaCO2 significantly in a subgroup of patients. The 12 microgram/kg dose of terbutaline was superior to 3 or 6 microgram/kg in relieving obstruction to airflow measured at the midportion of the vital capacity. This dose caused tremor in some children, but the tremor was not apparent to patients or their parents. [\hyperlink{Terbutaline Sulfate}{PMID: 7211775}, F E Simons et al., 1981]

\hypertarget{pmid_15572812}{T}o compare the clinical efficacy and side effects of terbutaline and salbutamol administered by metered dose inhaler and holding chamber in the mild to moderate acute exacerbations of asthma in children. The study subjects were children in the age group of 5- 15 years who presented with a mild or moderate acute exacerbation of asthma. Baseline assessment included clinical parameters and spirometry. The children were then randomized to receive salbutamol or terbutaline. Three puffs each of either 100 mcg salbutamol or 250 mcg of terbutaline were administered using 750 ml holding chamber with valve. Thirty minutes after drug administration, the children were reevaluated for clinical parameters and spirometry. Of the total 60 subjects studied, 31 were administered terbutaline and 29 salbutamol. The baseline spirometric parameters were comparable. After drug administration, all the studied variables showed significant improvement within each group. However, there were no statistically significant differences when the two groups were compared with each other. There was no significant difference in the side effects between two groups. Terbutaline and salbutamol, when administered by MDI with holding chamber, are equally efficacious in children with mild or moderate acute exacerbation of asthma. [\hyperlink{Terbutaline Sulfate}{PMID: 15572812}, Prakash Chandra et al., 2004]

\hypertarget{pmid_6586481}{N}ine children, 10-15 years old with chronic asthma, were treated for weekly periods with gradually increasing doses of oral terbutaline sulphate: 45, 79, 118 and 166 micrograms/kg (mean values) 3 times daily. There was a linear relationship between dose and steady-state plasma concentration of terbutaline within patients, but the plasma levels varied 3-fold between patients taking similar doses. Symptom score, peak expiratory flow rate (PEFR) and volume of air expelled in the first second of forced expiration (FEV1) improved with increasing doses, and the need for inhalation therapy decreased. An increase in pulse rate and tremor was measurable at all dose levels, but reported side-effects were few and mild. Linear regression analysis showed a statistically significant relationship between the plasma concentration of terbutaline and the effect on FEV1 (p less than 0.01) and PEFR (p less than 0.05) within patients. [\hyperlink{Terbutaline Sulfate}{PMID: 6586481}, G Lönnerholm et al., 1984]

\hypertarget{pmid_3789327}{T}he effect of terbutaline sulphate in slow-release (SR) tablets (Bricanyl Depot), 5 mg twice daily, was compared with that of terbutaline sulphate in ordinary tablets (Bricanyl), 2.5 mg three times daily, in a double-blind, randomized, cross-over study during 2 consecutive weeks in 10 asthmatic children. Plasma concentrations and urinary excretion of terbutaline were measured at various times during both treatment periods. The SR tablets produced a higher mean plasma concentration in the morning and a smaller peak-trough variation over the day than the ordinary ones. No differences between the two treatments were observed concerning FEV1 (forced expiratory volume in 1 s). Tremor, measured with an opto-electronic tremorgraph, was about the same for two treatments and not significantly different from tremor seen in healthy children. The reported side effects were less frequent in the SR tablet period. [\hyperlink{Terbutaline Sulfate}{PMID: 3789327}, G Wettrell et al., 1986]

\hypertarget{pmid_2019938}{T}o test whether nebulized salbutamol (albuterol) is safe and efficacious for the treatment of young children with acute bronchiolitis, we enrolled 83 children (median age 6 months, range 1 to 21 months) in a randomized, double-blind clinical trial. Participants received two treatments at 30-minute intervals of either nebulized salbutamol (0.10 mg/kg in 2 ml 0.9\% saline solution) or a similar volume of 0.9\% saline solution placebo. Outcome measures were the respiratory rate, pulse oximetry, and a clinical score based on the degree of wheezing and retractions. Patients in the salbutamol arm had significantly greater improvement in clinical scores after the initial treatment (p = 0.04). There was no difference between the groups in oxygen saturation (p = 0.74); patients treated with salbutamol had a small increase in heart rate after two treatments (159 +/- 16 vs 151 +/- 16; p = 0.03). We conclude that salbutamol is safe and effective for the initial treatment of young children with acute bronchiolitis. [\hyperlink{Terbutaline Sulfate}{PMID: 2019938}, T P Klassen et al., 1991]

\hypertarget{pmid_15247700}{M}any children with urological disease require long-term treatment with antibiotics. In many cases the choice of medical instead of surgical management hinges on the implied safety of certain drugs. Recently some groups have advocated subureteral injection procedures to avoid long-term antibiotics for low grade reflux. We present a concise and relevant review on the use and adverse reactions of nitrofurantoin, trimethoprim and sulfamethoxazole in children. We reviewed the literature regarding the safety and toxicity of these drugs. Information regarding absorption, excretion and dosing was also gathered to explain better the mechanisms of toxicity. Adverse reactions in children reported in the literature related to nitrofurantoin are gastrointestinal disturbance (4.4/100 person-years at risk), cutaneous reactions (2\% to 3\%), pulmonary toxicity (9 patients), hepatoxicity (12 patients and 3 deaths), hematological toxicity (12 patients), neurotoxicity and an increased rate of sister chromatid exchanges. Adverse reactions in children related to trimethoprim/sulfamethoxazole are almost exclusively due to the sulfamethoxazole component, including cutaneous reactions (1.4 to 7.4 events per 100 person-years at risk), hematological toxicity (0\% to 72\% of patients) and hepatotoxicity (5 patients). The majority of adverse reactions were found in children on full dose therapy and not prophylaxis. The use of nitrofurantoin, trimethoprim and sulfamethoxazole is safe in children for long-term prophylactic therapy. The antibiotic safety issue should not be misconstrued as an argument for surgical therapy, whether minimally invasive or not. Adverse reactions exist to these medicines but they are less common than seen in adults, presumably because of the lower dose used for therapy, and the lack of significant comorbidities and drug interactions in children. Serious side effects are extremely rare and most are reversible by discontinuing therapy. The extremely low potential for significant adverse reactions should be discussed with parents. [\hyperlink{Terbutaline Sulfate}{PMID: 15247700}, Edward Karpman et al., 2004]

\hypertarget{pmid_2677624}{W}e compared the use of terbutaline sulphate that was delivered by a nebulizer with its delivery by a Nebuhaler at two dose levels in 27 children (nine children per group) of between three and six years of age with acute asthma. No significant difference was found in the mean baseline clinical score among the three groups, and a significant decline occurred in the mean clinical scores in all groups by 15 minutes which was maintained to 60 minutes after the dose was administered. The decline that was achieved with delivery of the drug by way of a Nebuhaler (at either dose level) was not significantly different from that with a nebulizer, although cooperation with Nebuhaler usage was not universal in the age-group. [\hyperlink{Terbutaline Sulfate}{PMID: 2677624}, J Pendergast et al., 1989]

\hypertarget{pmid_25164315}{T}his study compared the efficacy of intravenous magnesium sulphate, terbutaline and aminophylline for children with acute, severe asthma poorly responsive to standard initial treatment. We enrolled 100 children, aged one to 12 years, who had failed to respond to initial standard treatment for acute, severe asthma, in this randomised controlled trial. They received either intravenous magnesium sulphate, terbutaline or aminophylline. Responses were monitored using a modified Clinical Asthma Severity (CAS) score. The primary outcome was treatment success, defined as a reduction in the CAS of four points or more 1 h after starting the intervention. The magnesium sulphate group had higher treatment success (33/34, 97\%) than the terbutaline and aminophylline groups (both 23/33, 70\%) (p = 0.006) and faster resolution of retractions, wheeze and dyspnoea (p < 0.001). No adverse events occurred among patients receiving magnesium sulphate, but two patients receiving terbutaline had hypokalemia and nine patients receiving aminophylline had nausea and, or, vomiting. Adding a single dose of Intravenous magnesium sulphate to inhaled beta2-agonists and corticosteroids was more effective, and safer, than using terbutaline or aminophylline when treating a child with acute severe asthma poorly responsive to initial treatment. [\hyperlink{Terbutaline Sulfate}{PMID: 25164315}, Sunit Singhi et al., 2014]

\hypertarget{pmid_18611612}{T}he safety and efficacy of cefetamet pivoxil, an oral cephalosporin of the third generation, have been studied in open, prospective, randomized comparative, clinical trials including 301 toddlers (children aged 1 to 2 years) with upper and lower respiratory tract infections, and urinary tract infections. Cefetamet pivoxil (CAT) syrup formulation was given to 177 toddlers either in the standard dose of 10 mg/kg b.i.d. [n = 116] or 20 mg/kg b.i.d. [n = 61] and 124 toddlers have been treated with comparator drugs [cefaclor, n = 98; phenoxymethylpenicillin, n = 18; amoxicillin plus clavulanic acid; n = 8]. The treatment period was 7 days mainly, except for pharyngotonsillitis for which the treatment duration was 7 or 10 days. The assessment of treatment was based on clinical signs and symptoms primarily in infections of lower respiratory tract and acute otitis media, whereas in patients with pharyngotonsillitis and acute urinary tract infections the bacteriological findings were the main evaluation criteria. The overall therapeutic outcome was successful in 148 (95.4\%) of the 155 toddlers to whom CAT was administered and in 87 (85.3\%) out of 102 toddlers receiving standard drugs. Adverse events of mild to moderate severity, mainly of gastro-intestinal type (vomiting or diarrhoea) occurred in 14.7\% in the patient group receiving CAT, 11.2\% in the toddlers receiving the standard dose of CAT, and in 12.9\% with the comparator drugs. From the data presented it is concluded that cefetamet pivoxil is efficient and safe in toddlers presenting with community-acquired respiratory and urinary infections mainly caused by S. pneumoniae, H. influenzae, Group A beta-haemolytic streptococci, M. catarrhalis, E. coli, Proteus spp. and K. pneumoniae. [\hyperlink{Terbutaline Sulfate}{PMID: 18611612}, A Chibante et al., 1994]

\section*{Zinc Chloride}
\subsection*{Result}
\subsubsection*{Answer}

Unknown

\subsubsection*{{Explanation}}
\hypertarget{Zinc Chloride}
Based on the abstracts available, there is very limited data specifically addressing the safety of zinc chloride in children. Most studies on zinc supplementation in children use other zinc salts, such as zinc sulfate, zinc acetate, or zinc gluconate, and these have been shown to be safe and effective in various pediatric populations for conditions like diarrhea, pneumonia, and Wilson's disease. However, these results cannot be extrapolated to zinc chloride.

There are two abstracts that specifically mention zinc chloride in children:

1. One case report describes a child who ingested a zinc chloride solution used in lead soldering. The child experienced bleeding from esophageal erosions and hematemesis, but no adverse systemic effects were observed. The elevated serum zinc was normalized with chelation therapy [\hyperlink{pmid_6784612}{PMID: 6784612}, J L Potter et al., 1981]. This report describes an accidental poisoning, not therapeutic use, and highlights the corrosive and toxic potential of zinc chloride when ingested.

2. Another case report details a 10-year-old girl who accidentally ingested an acid soldering flux solution containing 30\% to <60\% zinc chloride. She suffered severe gastric corrosion, developed an antral stricture, and required surgical intervention. No significant systemic effects were noted, but the local caustic injury was severe, and long-term follow-up was recommended due to the risk of malignancy in the damaged stomach [\hyperlink{pmid_9574776}{PMID: 9574776}, A Yamataka et al., 1998].

No abstracts were found that studied the safety of zinc chloride as a supplement or therapeutic agent in children. The only data available are from accidental ingestions, which resulted in significant local tissue injury and required medical intervention.

Therefore, based on the abstracts reviewed, there is no evidence from targeted studies affirming the safety of zinc chloride for use in children in any age group. The available evidence suggests that zinc chloride is a corrosive agent and can cause significant harm if ingested.

For all pediatric age ranges, the safety of zinc chloride as a supplement or therapeutic agent is unknown, and the available case reports suggest potential for harm rather than safety.

\subsection*{Abstracts}
\hypertarget{pmid_11241029}{T}he objectives were to evaluate appropriate doses of zinc acetate and its efficacy for the maintenance management of Wilson's disease in pediatric cases. Pediatric patients of 1 to 5 years of age were given 25 mg of zinc twice daily; patients of 6 to 15 years of age, if under 125 pounds body weight, were given 25 mg of zinc three times daily; and patients 16 years of age or older were given 50 mg of zinc three times daily. Patients were followed for efficacy (or over-treatment) until their 19th birthday by measuring levels of urine and plasma copper, urine and plasma zinc and through liver function tests and quantitative speech and neurologic scores. Patients were followed for toxicity by measures of blood counts, blood biochemistries, urinalysis, and clinical follow-up. Thirty-four patients, ranging in ages from 3.2 to 17.7 years of age, were included in the study. All doses met efficacy objectives of copper control, zinc levels, neurologic improvement, and maintenance of liver function except for episodes of poor compliance. No instance of over-treatment was encountered. Four patients exhibited mild and transient gastric disturbance from the zinc. Zinc therapy in pediatric patients appears to have a mildly adverse effect on the high-density lipoprotein/total cholesterol ratio, contrary to results of an earlier large study of primarily adults. In conclusion, zinc is effective and safe for the maintenance management of pediatric cases of Wilson's disease. Our data are strongest in children above 10 years of age. More work needs to be done in very young children, and the cholesterol observations need to be studied further. [\hyperlink{Zinc Chloride}{PMID: 11241029}, G J Brewer et al., 2001]

\hypertarget{pmid_18326612}{M}ultiple studies have shown the benefits of zinc supplementation among young children in high-risk populations. However, the optimal dose and safe upper level of zinc have not been determined. The objectives of this study were to measure the effects of different doses of supplemental zinc on the plasma zinc concentration, morbidity, and growth of young children; to detect any adverse effects of 10 mg supplemental Zn on markers of copper or iron status; and to determine whether any adverse effects are alleviated by providing copper with zinc. This randomized, double-masked, community-based intervention trial was conducted in 631 Ecuadorian children who were 12-30 mo old at baseline and who had initial length-for-age z scores <-1.3. Children received 1 of 5 daily supplements for 6 mo: 3, 7, or 10 mg Zn as zinc sulfate, 10 mg Zn + 0.5 mg Cu as copper sulfate, or placebo. The change in plasma zinc concentration from baseline was positively related to the zinc dose (P < 0.001). Zinc supplementation, including doses as low as 3 mg/d, reduced the incidence of diarrhea by 21-42\% (P < 0.01). There were no other significant group-wise differences. Zinc supplementation with a dose as low as 3 mg/d increased plasma zinc concentrations and reduced diarrhea incidence in the study population. There were no observed adverse effects of 10 mg Zn/d on indicators of copper or iron status. The current tolerable upper level of zinc recommended by the Institute of Medicine should be reassessed for young children. [\hyperlink{Zinc Chloride}{PMID: 18326612}, Sara E Wuehler et al., 2008]

\hypertarget{pmid_20335624}{Z}inc supplementation has proven beneficial in the treatment of acute child diarrhea and appears to enhance linear growth. There is a theoretical risk of anemia in zinc-supplemented children due to inhibited iron transport via decreased copper absorption. Although many zinc supplementation trials have included hematological measures, the potential effect of zinc on these outcomes has not been quantitatively evaluated in a comprehensive review. We performed a systematic review of randomized trials that examined the effect of zinc supplementation on hemoglobin concentrations in apparently healthy children ages 0-15 y and conducted a random effects meta-analysis of weighted mean differences (WMD) of change in hemoglobin concentrations before and after supplementation. Twenty-one randomized, controlled trials representing 3869 participants were included in the meta-analysis. The duration of treatment ranged from 4 to 15 mo; doses were typically 10-20 mg/d. Zinc supplementation did not affect changes in hemoglobin concentrations (pooled WMD: 0.8 g/L; 95\% CI: -0.6, 2.2; P = 0.27). There was no evidence for effect modification by age, zinc dosage, duration of treatment, type of control, baseline hemoglobin status, geographical or healthcare setting, or quality of the studies. These results suggest that zinc supplementation at doses typically used in randomized trials is a safe intervention with regards to hemoglobin concentrations. Some benefits might exist among children with severe anemia or zinc deficiency, which warrant further evaluation. [\hyperlink{Zinc Chloride}{PMID: 20335624}, Louise H Dekker et al., 2010]

\hypertarget{pmid_6784612}{T}his brief report describes the clinical course and management of a child who ingested a zinc chloride solution used in a lead soldering process. Injury was limited to bleeding from esophageal erosions and hematemesis. No adverse systemic effects were observed, although serum zinc levels were markedly elevated. A single small dosage of calcium disodium edetate (150 mg dissolved in 75 ml 1:5 normal saline) was effective in normalizing the serum zinc level. [\hyperlink{Zinc Chloride}{PMID: 6784612}, J L Potter et al., 1981]

\hypertarget{pmid_675401}{S}uccessful therapy with zinc sulphate is reported in 3 children suffering from acroedematitis enteropathica. [\hyperlink{Zinc Chloride}{PMID: 675401}, I L Rubin et al., 1978]

\hypertarget{pmid_19472600}{Z}inc supplementation trials carried out among children have produced variable results, depending on the specific outcomes considered and the initial characteristics of the children who were enrolled. We completed a series of meta-analyses to examine the impact of preventive zinc supplementation on morbidity; mortality; physical growth; biochemical indicators of zinc, iron, and copper status; and indicators of behavioral development, along with possible modifying effects of the intervention results. Zinc supplementation reduced the incidence of diarrhea by approximately 20\%, but the impact was limited to studies that enrolled children with a mean initial age greater than 12 months. Among the subset of studies that enrolled children with mean initial age greater than 12 months, the relative risk of diarrhea was reduced by 27\%. Zinc supplementation reduced the incidence of acute lower respiratory tract infections by approximately 15\%. Zinc supplementation yielded inconsistent impacts on malaria incidence, and too few trials are currently available to allow definitive conclusions to be drawn. Zinc supplementation had a marginal 6\% impact on overall child mortality, but there was an 18\% reduction in deaths among zinc-supplemented children older than 12 months of age. Zinc supplementation increased linear growth and weight gain by a small, but highly significant, amount. The interventions yielded a consistent, moderately large increase in mean serum zinc concentrations, and they had no significant adverse effects on indicators of iron and copper status. There were no significant effects on children's behavioral development, although the number of available studies is relatively small. The available evidence supports the need for intervention programs to enhance zinc status to reduce child morbidity and mortality and to enhance child growth. Possible strategies for delivering preventive zinc supplements are discussed. [\hyperlink{Zinc Chloride}{PMID: 19472600}, Kenneth H Brown et al., 2009]

\hypertarget{pmid_9574776}{Z}inc chloride is a powerful corrosive agent. Reports of zinc chloride ingestion are uncommon, and there is little information about its toxicity and management. The authors report the clinical course of a 10-year-old girl who accidentally ingested an acid soldering flux solution (pH, 3.0; zinc chloride, 30\% to < 60\%). Systemic effects after the ingestion were unremarkable except for lethargy. Thus, chelation therapy was not considered. Severe gastric corrosion was caused by local caustic action. An antral stricture of the stomach approximately 3 weeks after the ingestion developed, and she underwent a modified Heineke-Mikulicz antropyloroplasty. Postoperatively, she made an uneventful recovery. On follow-up, although she was tolerating a normal diet, results of a barium meal showed her stomach to be totally aperistaltic. Results of a nuclear medicine study showed moderately delayed gastric emptying. Careful long-term follow-up is necessary, because there is potential risk for malignancy in the damaged stomach. [\hyperlink{Zinc Chloride}{PMID: 9574776}, A Yamataka et al., 1998]

\hypertarget{pmid_24475083}{T}here is no official consensus regarding zinc therapy in pre-symptomatic children with Wilson Disease (WD); more data is needed. To investigate the safety and efficacy of zinc gluconate therapy for Chinese children with pre-symptomatic WD. We retrospectively analyzed pre-symptomatic children receiving zinc gluconate in a single Chinese center specialized in pediatric hepatology. Short-term follow-up data on safety and efficacy were presented, and effects of different zinc dosages were compared. 30 children (21 males) aged 2.7 to 16.8 years were followed for up to 4.4 years; 26 (87\%) children had abnormal ALT at baseline. Most patients (73\%) received higher than the currently recommended dose of elemental zinc. Zinc gluconate significantly reduced mean ALT (p<0.0001), AST (p<0.0001), GGT (p<0.0001) levels after 1 month, and urinary copper excretion after 6 months (p<0.0054). Mean direct bilirubin levels dropped significantly at 1 month (p = 0.0175), 3 months (p = 0.0010), and 6 months (p = 0.0036). Serum zinc levels gradually increased and reached a significantly higher level after 6 months (p<0.0026), reflecting good compliance with the therapy. Complete blood count parameters did not change throughout the analysis period. 8 children experienced mild and transient gastrointestinal side effects. The higher zinc dose did not affect treatment response and was not associated with different or increased side effects when compared to conventional zinc dose. In our cohort, zinc gluconate therapy for Chinese children with pre-symptomatic WD was effective, and higher initial dose of elemental zinc had the same level of efficacy as the conventional dose. [\hyperlink{Zinc Chloride}{PMID: 24475083}, Kuerbanjiang Abuduxikuer et al., 2014]

\hypertarget{pmid_36515300}{T}he effects of locally applied zinc chloride (ZnCl [\hyperlink{Zinc Chloride}{PMID: 36515300}, Kevin Innella et al., 2023] Zinc phosphide is a chemical compound that is frequently used as a rodenticide; it is a highly toxic product that is widely used, among other spaces, at home. Given that it is a highly commercialized pesticide and that there is no antidote, it is mandatory to establish favorably the clinical manifestations of the intoxication. The aim was to describe the epidemiological and clinical profile of children intoxicated with zinc phosphide attended in a toxicological center of a tertiary referral hospital. Cross-sectional, retrospective and observational study based on the medical records of 36 pediatric patients attended from 2005 to 2015 at the Centro de Información y Atención Toxicológica from Hospital General "Dr. Gaudencio González Garza", which belongs to the Instituto Mexicano del Seguro Social. The study didn't show a prevalence of gender; 66\% of patients were children between ages 1 and 2. 96\% of patients were healthy and three adolescents used the product with suicidal purposes. Zinc phosphide exposure occurred at home. Toxicity was characterized by hypotension, hypoglycemia, metabolic acidosis, abdominal pain, nausea, and vomiting; none of the patients died. In addition, neither required mechanical ventilation nor hemodialysis. The lack of knowledge of the potential toxicity of zinc phosphide and the fact that is easily reached at home allow the exposure to this product; it is an absolutely preventable risk. [\hyperlink{Zinc Chloride}{PMID: 36515300}, María Carmen Socorro Sánchez-Villegas et al., 2017]

\hypertarget{pmid_27872827}{T}o evaluate the role of zinc as add on treatment to the "recommended treatment" of nephrotic syndrome (NS) in children. All the published literature through the major databases including Medline/Pubmed, Embase, and Google Scholar were searched till 31 Of 54 citations retrieved, a total of 6 RCTs were included. Zinc was used at a dose of 10-20 mg/d, for the duration that varied from 6-12 mo. Compared to placebo, zinc reduced the frequency of relapses, induced sustained remission/no relapse, reduced the proportion of infection episodes associated with relapse with a mild adverse event in the form of metallic taste. The GRADE evidence generated was of "very low-quality". Zinc may be a useful additive in the treatment of childhood NS. The evidence generated mostly was of "very low-quality". We need more good quality RCTs in different country setting as well different subgroups of children before any firm recommendation can be made. [\hyperlink{Zinc Chloride}{PMID: 27872827}, Girish Chandra Bhatt et al., 2016]

\hypertarget{pmid_22392179}{D}iarrhea and pneumonia are the leading causes of illness and death in children <5 years of age. Zinc supplementation is effective for treatment of acute diarrhea and can prevent pneumonia. In this trial, we measured the efficacy of zinc when given to children hospitalized and treated with antibiotics for severe pneumonia. We enrolled 610 children aged 2 to 35 months who presented with severe pneumonia defined by the World Health Organization as cough and/or difficult breathing combined with lower chest indrawing. All children received standard antibiotic treatment and were randomized to receive zinc (10 mg in 2- to 11-month-olds and 20 mg in older children) or placebo daily for up to 14 days. The primary outcome was time to cessation of severe pneumonia. Zinc recipients recovered marginally faster, but this difference was not statistically significant (hazard ratio = 1.10, 95\% CI 0.94-1.30). Similarly, the risk of treatment failure was slightly but not significantly lower in those who received zinc (risk ratio = 0.88 95\% CI 0.71-1.10). Adjunct treatment with zinc reduced the time to cessation of severe pneumonia and the risk of treatment failure only marginally, if at all, in hospitalized children. [\hyperlink{Zinc Chloride}{PMID: 22392179}, Sudha Basnet et al., 2012]

\hypertarget{pmid_25825293}{Z}inc deficiency has been estimated to result in more than 450,000 child deaths annually by increasing the risk of diarrhea and pneumonia mortality. Trials of daily supplemental zinc have shown preventive benefits in childhood diarrhea with a 20\% reduction in incidence. Use of zinc in treatment of diarrhea has also been successful in shortening the duration of the episode by 10\% and reducing the number of prolonged episodes. The World Health Organization recommends that zinc supplements be used for 10-14 days for every episode of childhood diarrhea along with oral hydration and feeding. Large-scale effectiveness trials of these recommendations in Bangladesh and India have found a reduction in hospitalizations due to diarrhea and pneumonia and in child mortality. Trials have also demonstrated a reduction in the incidence childhood pneumonia with zinc supplements and some, but not all, studies have found a therapeutic benefit of zinc as adjunctive treatment along with antibiotics as well. Preventive zinc also improves the growth of children in developing countries and reduces total deaths in 1-to 4-year-old children by 18\%. Zinc supplementation is an intervention with proven effectiveness and broad application to address pneumonia and diarrhea, the two most important childhood infectious diseases globally.  [\hyperlink{Zinc Chloride}{PMID: 25825293}, Robert E Black et al., 2012] To determine whether continuing with zinc supplementation after zinc treatment (ZT) of an acute diarrhoea episode will result in additional clinical benefits beyond ZT alone. Children 6-23 months of age, living in an urban slum in Dhaka, Bangladesh with acute childhood diarrhoea (ACD), were enrolled in a randomized, double-blind field trial. All children received 10 days of ZT (20 mg/day) and were then randomized to zinc (10 mg/day) or placebo supplementation for 3 months. Weekly follow-up of all children occurred over a period of 9 months. A total of 353 subjects were enrolled, with 93\% of the zinc supplemented and 96\% of the placebo children followed for 9 months. The incidence density of ACD among those receiving zinc supplementation compared to those receiving placebo was reduced by 28\% (2.64 vs.3.66 episodes/p-y follow-up) over the 3 months while on supplementation and by 21\% (2.05 vs.2.59 episodes/p-y follow-up) over the 9 months of follow-up. There was no observed effect on the incidence of acute respiratory infections (ARIs) or on growth. Zinc supplementation after treatment provides additional preventive ACD benefits to children in early childhood. Larger, effectiveness trials of this strategy are warranted. [\hyperlink{Zinc Chloride}{PMID: 25825293}, Charles P Larson et al., 2010]

\hypertarget{pmid_15951862}{D}iagnostic and therapeutic procedures in children are made easier using sedation. However, there is no consensus about which drug should be used to achieve this. Furthermore, none of the drugs used for sedation are risk free. The aim of this work is to study sedation indications, effectiveness, and safety at our center. A prospective observational study conducted at the Pediatric Day Care Unit, King Fahad National Guard Hospital, Riyadh, Saudi Arabia. The study covered 17.5 weeks in 2 periods: May 9th 1999 to June 13th 1999 and October 31st 2001 to February 11th 2002. Children <12 years were included. Collected data included demographics, indication, drug dosing and outcome. Data were reported as mean +/- SD. We included 148 patients, age 38 +/- 30 months. Adequate sedation was achieved in 79\% after initial chloral hydrate (CH) dose of 56.9 +/- 9.3 mg/kg, in 95\% after adding 18.5 +/- 6.4 mg/kg CH and in 96\% after adding second drug. Compared to nonrespondents, first CH dose respondents were younger and lower in weight. The CH side effects were few and mild. Chloral hydrate is a safe and effective agent for sedation in children with an age and weight dependent response. [\hyperlink{Zinc Chloride}{PMID: 15951862}, Omar M Hijazi et al., 2005]

\hypertarget{pmid_24906347}{Z}inc is an essential micronutrient important for growth and for normal function of the immune system. Many children in developing countries have inadequate zinc nutrition. Routine zinc supplementation reduces the risk of respiratory infections and diarrhea, the two leading causes of morbidity and mortality in young children worldwide. In childhood diarrhea oral zinc also reduces illness duration and risk of persistent episodes. Oral zinc is therefore recommended for the treatment of acute diarrhea in young children. The results from the studies that have measured the therapeutic effect of zinc on acute respiratory infections, however, are conflicting. Moreover, the results of therapeutic zinc for childhood malaria also are so far not promising.This paper gives a brief outline of the current evidence from clinical trials on therapeutic effect of oral zinc on childhood respiratory infections, pneumonia and malaria and also of new evidence of the effect on serious bacterial illness in young infants.  [\hyperlink{Zinc Chloride}{PMID: 24906347}, Sudha Basnet et al., 2015] In a previous study, children aged 2-5 years old in Bangladesh were supplemented orally with a single dose of Vitamin A (200,000 IU) and a placebo for zinc (zinc equivalent to 20 mg of elemental zinc) everyday for 42 days (group A), zinc and a placebo for Vitamin A (group Z), zinc and Vitamin A (group AZ) or both placebos (group P). All children were orally immunised with two doses of the killed cholera vaccine containing whole cells and a recombinant B subunit of cholera toxin (CT). The number of children who responded with > or = 4-fold vibriocidal antibody (a proxy indicator of protection against cholera) was significantly greater among the zinc-supplemented groups than among the non-zinc-supplemented groups, while Vitamin A supplementation did not appear to have any effect. The sera from these children were assayed for antibody to CT. Antibody to CT is known to exert a synergistic protective effect against cholera in animal studies, and offer significantly higher short-term protection against cholera and significant short-term protection against enterotoxigenic Escherichia coli diarrhoea in humans on oral immunisation with the cholera vaccine. Children who received zinc had significantly reduced levels of serum antibodies to CT than children who received placebos only. Factorial analysis showed a trend for zinc showing a reduction in the number of children responding with CT-antibody, while Vitamin A did not appear to have any effect. Thus, zinc enhanced vibriocidal antibody response, but suppressed CT-antibody response, suggesting that zinc supplementation has different modulating effects on vibriocidal antibody response and CT-antibody response. [\hyperlink{Zinc Chloride}{PMID: 24906347}, Firdausi Qadri et al., 2004]

\hypertarget{pmid_12324294}{Z}inc supplementation in young children has been associated with reductions in the incidence and severity of diarrheal diseases, acute respiratory infections, and malaria. The objective was to evaluate the potential role of zinc as an adjunct in the treatment of acute, uncomplicated falciparum malaria; a multicenter, double-blind, randomized placebo-controlled clinical trial was undertaken. Children (n = 1087) aged 6 mo to 5 y were enrolled at sites in Ecuador, Ghana, Tanzania, Uganda, and Zambia. Children with fever and >or=2000 asexual forms of Plasmodium falciparum/ micro L in a thick blood smear received chloroquine and were randomly assigned to receive zinc (20 mg/d for infants, 40 mg/d for older children) or placebo for 4 d. There was no effect of zinc on the median time to reduction of fever (zinc group: 24.2 h; placebo group: 24.0 h; P = 0.37), a >or=75\% reduction in parasitemia from baseline in the first 72 h in 73.4\% of the zinc group and in 77.6\% of the placebo group (P = 0.11), and no significant change in hemoglobin concentration during the 3-d period of hospitalization and the 4 wk of follow-up. Mean plasma zinc concentrations were low in all children at baseline (zinc group: 8.54 +/- 3.93 micro mol/L; placebo group: 8.34 +/- 3.25 micro mol/L), but children who received zinc supplementation had higher plasma zinc concentrations at 72 h than did those who received placebo (10.95 +/- 3.63 compared with 10.16 +/- 3.25 micro mol/L, P < 0.001). Zinc does not appear to provide a beneficial effect in the treatment of acute, uncomplicated falciparum malaria in preschool children. [\hyperlink{Zinc Chloride}{PMID: 12324294},  et al., 2002]

\hypertarget{pmid_1792743}{Z}inc sulfate-enriched lactic acid lactobacterin was used in the combined treatment of 23 children with celiac disease, aged from 1 to 10 years. A group of 23 children with celiac disease who received lactic acid lactobacterin without zinc were used as control. The patients treated with lactobacterin containing zinc showed a higher increase in body mass, total protein and zinc levels in the blood serum and elevated activity of metalloenzymes-ceruloplasmin and cytochrome oxidase. [\hyperlink{Zinc Chloride}{PMID: 1792743}, I D Uspenskaia et al., ]

\hypertarget{pmid_12052800}{T}o evaluate the effect of daily zinc supplementation in children on the incidence of acute lower respiratory tract infections and pneumonia. Double masked, randomised placebo controlled trial. A slum community in New Delhi, India. 2482 children aged 6 to 30 months. Daily elemental zinc, 10 mg to infants and 20 mg to older children or placebo for four months. Both groups received single massive dose of vitamin A (100 000 IU for infants and 200 000 IU for older children) at enrollment. All households were visited weekly. Any children with cough and lower chest indrawing or respiratory rate 5 breaths per minute less than the World Health Organization criteria for fast breathing were brought to study physicians. At four months the mean plasma zinc concentration was higher in the zinc group (19.8 (SD 10.1) v 9.3 (2.1) micromol/l, P<0.001). The proportion of children who had acute lower respiratory tract infection during follow up was no different in the two groups (absolute risk reduction -0.2\%, 95\% confidence interval -3.9\% to 3.6\%). Zinc supplementation resulted in a lower incidence of pneumonia than placebo (absolute risk reduction 2.5\%, 95\% confidence interval 0.4\% to 4.6\%). After correction for multiple episodes in the same child by generalised estimating equations analysis the odds ratio was 0.74, 95\% confidence interval 0.56 to 0.99. Zinc supplementation substantially reduced the incidence of pneumonia in children who had received vitamin A. [\hyperlink{Zinc Chloride}{PMID: 12052800}, Nita Bhandari et al., 2002]

\hypertarget{pmid_24967861}{Z}inc deficiency is common in children among populations in developing areas. Zinc deficiency alters the immune system and the resistance to infections. To evaluate the effect of two zinc compounds in the prevention of acute respiratory infection and acute diarrhea. Randomized triple-blind community trial with 301 children between 2-5 years of age from six child daycare centers in Medellin, Colombia. Children were distributed in three groups receiving zinc amino acid chelate, zinc sulfate and placebo five days a week for 16 weeks. Daily symptoms of respiratory infection, acute diarrhea and side effects were evaluated. The incidence of respiratory infection was lower with zinc amino acid chelate (1.42 per 1,000 child-days) compared with placebo (3.3 per 1,000 child-days) (RR=0.43, 95\% CI: 0.196 to 0.950, p=0.049) and with zinc sulfate (1.57 per 1,000 child-days) (RR=0.90, 95\% CI 0.382 to 2.153, p=0.999). The incidence of acute diarrhea with zinc amino acid chelate (0.15 per 1,000 child-days) was lower than with placebo (0.49 per 1,000 child-days) (RR=0.32, 95\% CI 0.006 to 3.990, p=0.346) and with zinc sulfate (0.78 per 1,000 child-days) (RR=0.20, 95\% CI: 0.0043 to 1.662, p=0.361). Zinc amino acid chelate had a better effect in reducing the incidence of acute respiratory infection and acute diarrhea in preschool children when compared with the other groups. [\hyperlink{Zinc Chloride}{PMID: 24967861}, Juliana Sánchez et al., ]

\hypertarget{pmid_21713083}{Z}inc supplementation is a critical new intervention for treating diarrheal episodes in children. Recent studies suggest that administration of zinc along with new low osmolarity oral rehydration solutions / salts (ORS), can reduce the duration and severity of diarrheal episodes for up to three months. The World Health Organization (WHO) and UNICEF recommend daily 20 mg zinc supplements for 10 - 14 days for children with acute diarrhea, and 10 mg per day for infants under six months old, to curtail the severity of the episode and prevent further occurrences in the ensuing -two to three months, thereby decreasing the morbidity considerably. This article reviews the available evidence on the efficacy and safety of zinc supplementation in pediatric diarrhea and convincingly concludes that zinc supplementation has a beneficial impact on the disease outcome. [\hyperlink{Zinc Chloride}{PMID: 21713083}, Chaitali Bajait et al., 2011]

\hypertarget{pmid_19888335}{Z}inc treatment of childhood diarrhea has the potential to save 400,000 under-five lives per year in lesser developed countries. In 2004 the World Health Organization (WHO)/UNICEF revised their clinical management of childhood diarrhea guidelines to include zinc. The aim of this study was to monitor the impact of the first national campaign to scale up zinc treatment of childhood diarrhea in Bangladesh. Between September 2006 to October 2008 seven repeated ecologic surveys were carried out in four representative population strata: mega-city urban slum and urban nonslum, municipal, and rural. Households of approximately 3,200 children with an active or recent case of diarrhea were enrolled in each survey round. Caretaker awareness of zinc as a treatment for childhood diarrhea by 10 mo following the mass media launch was attained in 90\%, 74\%, 66\%, and 50\% of urban nonslum, municipal, urban slum, and rural populations, respectively. By 23 mo into the campaign, approximately 25\% of urban nonslum, 20\% of municipal and urban slum, and 10\% of rural under-five children were receiving zinc for the treatment of diarrhea. The scale-up campaign had no adverse effect on the use of oral rehydration salt (ORS). Long-term monitoring of scale-up programs identifies important gaps in coverage and provides the information necessary to document that intended outcomes are being attained and unintended consequences avoided. The scale-up of zinc treatment of childhood diarrhea rapidly attained widespread awareness, but actual use has lagged behind. Disparities in zinc coverage favoring higher income, urban households were identified, but these were gradually diminished over the two years of follow-up monitoring. The scale up campaign has not had any adverse effect on the use of ORS. Please see later in the article for the Editors' Summary. [\hyperlink{Zinc Chloride}{PMID: 19888335}, Charles P Larson et al., 2009]

\hypertarget{pmid_9401251}{I}n a zinc supplementation trial (with a significant impact on diarrheal morbidity), to evaluate effect of zinc supplementation on cellular immune status before and after 120 days of supplementation. A double blind, randomized controlled trial with immune assessment at baseline and after 120 days on supplement. Community based study in an urban slum population. Randomly selected children (zinc 38, control 48), had a Multitest CMI skin test at both times. In 66 children (zinc 22, control 34), proportions of CD3, CD4, CD8, CD16, CD20 cells and the CD/CD8 ratio were also estimated using a whole blood lysis method and flowcytometry. Zinc gluconate to provide elemental zinc 10 mg daily and 20 mg during diarrhea. Regarding CMI, the percentage of anergic or hypoergic children (using induration score) decreased from 67\% to 47\% in the zinc group, while in the control group it remained unchanged (73\% vs 71\%) (p = 0.05). The percentage of children deteriorating between first and second tests was significantly lower in the zinc group (13\% vs 33\%, p = 0.03). Regarding lymphocyte subsets, the zinc group had a significantly higher rise in the geometric means of CD3 (25\%, p = 0.02), CD4 (64\% p = 0.001), and CD4/CD8 ratio (73\% p = 0.004) with no difference in CD8 and CD20. The rise in CD4 was significantly higher in the zinc as compared to the control group; the ratio of geometric means was 1.45 (95\% CI, 1.03-2.01). Zinc supplementation improves cellular immune status, which may have been one of the mechanisms for observed impact of zinc supplementation on diarrheal morbidity. [\hyperlink{Zinc Chloride}{PMID: 9401251}, S Sazawal et al., 1997]

\hypertarget{pmid_16155274}{R}ecent studies reported that zinc significantly reduced the duration and volume of acute watery diarrhea in children aged > or = 4 mo, but there were no data specifically on infants aged < 6 mo. This study investigated the effect of zinc on the duration of illness and the stool quantity in acute watery diarrhea of infants aged 1-6 mo by comparing a 20 mg Zn/d dose with a 5 mg Zn/d dose. Infants hospitalized with at least some dehydration (by World Health Organization classification) were enrolled in a double-blind, randomized, placebo-controlled trial. Infants were randomly assigned to receive 20 mg Zn (acetate)/d, 5 mg Zn/d, or placebo for the duration of illness. Two hundred seventy-five infants were enrolled between 20 September 1998 and 18 December 2000. Neither diarrhea duration nor mean stool volume differed between groups. There were no significant differences in fluid intake, the need for unscheduled intravenous fluid, weight gain, or vomiting rates between the groups. Zinc supplementation did not affect diarrhea duration or stool volume in young infants. Young infants tolerated both zinc doses. A beneficial effect on subsequent illness cannot be ruled out. [\hyperlink{Zinc Chloride}{PMID: 16155274}, W Abdullah Brooks et al., 2005]

\section*{Arranon}
\subsection*{Result}
\subsubsection*{Answer}
Unknown
\subsubsection*{{Explanation}}
\hypertarget{Arranon}
After reviewing all available abstracts, there is no evidence of a targeted safety study of Arranon (also known as nelarabine) in children. None of the abstracts provided discuss Arranon or nelarabine, nor do they present data on its safety or efficacy in pediatric populations or any specific age range of children. Therefore, based solely on the abstracts available, the safety of Arranon for use in children is unknown.

\subsection*{Abstracts}
\hypertarget{pmid_24300768}{T}he objective of the present investigation was to estimate the effectiveness and safety of rinorin used for the prevention and treatment of infectious complications of allergic rhinitis in the children. A total of 70 children varying in the age from 5 to 15 years and presenting with the mild and moderate-to-severe form of seasonal allergic rhinitis were included in the study; they were divided into two groups. The study failed to reveal statistically significant difference between the groups in terms of adverse reactions to the treatment with rinorin. During the 21 day study period, infectious rhinitis was diagnosed in 2 (5.7\%) children given rinorin compared with 5 (14.3\%) ones in the control group (p>0.05). However, the patients of the latter group suffered a more severe form of this infectious complication of allergic rhinitis. It is concluded that rinorin is characterized by a high level of safety even though the antimicrobial potency of benzalkonium chloride, its active ingredient, needs further studies and evaluation.  [\hyperlink{Arranon}{PMID: 24300768}, E P Karpova et al., 2013] Prophylactic efficiency and safety of anaferon (pediatric formulation) in children aging 1?month to 4 years, including sickly children, was proven. The use of the preparation in children reduced the incidence of acute respiratory infections, alleviated the course of the disease, and decreased the incidence of detection of viral antigens in nasal meatuses. [\hyperlink{Arranon}{PMID: 24300768}, E S Erman et al., 2009]

\hypertarget{pmid_19441606}{I}ntranasal corticosteroids (INSs) are the most effective treatment for allergic rhinitis (AR). However, available INS safety and efficacy data in children younger than 6 years are limited. To report the first well-controlled study assessing the safety and efficacy of an INS in children aged 2 to 5 years with perennial AR. In a 4-week, multicenter, double-blind, parallel-group study, patients were randomized to receive triamcinolone acetonide aqueous nasal spray (TAA AQ), 110 microg once daily, or placebo. A subset of children continued into a 6-month, open-label phase. Efficacy end points included total nasal symptom scores. Safety measures included reports of adverse events, morning serum cortisol levels before and after cosyntropin infusion, and growth as measured using office stadiometry. A total of 474 patients were randomized to receive TAA AQ (n = 236) or placebo (n = 238); 436 entered the open-label extension phase. Adjusted mean (SE) changes from baseline during the double-blind period in instantaneous and reflective total nasal symptom scores were -2.28 (0.16) and -2.31 (0.15), respectively, in the TAA AQ group (P = .09) vs -1.92 (0.16) and -1.87 (0.15) in the placebo group (P = .03). Adverse event rates were comparable between treatment groups. There was no significant change from baseline in serum cortisol levels after cosyntropin infusion at study end. The distribution of children by stature-for-age percentile remained stable during the study. Use of TAA AQ, 110 microg once daily, for up to 6 months offers a favorable efficacy to safety ratio in children aged 2 to 5 years with perennial AR. [\hyperlink{Arranon}{PMID: 19441606}, Steven Weinstein et al., 2009]

\hypertarget{pmid_18939734}{G}uidelines recommend treatment with intranasal corticosteroids for patients with allergic rhinitis (AR), but concerns remain about possible adverse effects. To present the 1- and 2-year growth results for children with AR treated with triamcinolone acetonide aqueous nasal spray. Thirty-nine children (aged 6.1-14.3 years at study entry) were treated with triamcinolone acetonide aqueous for 1 year, and a subset of 30 children completed a second year of treatment. The dose was physician titered to achieve control over AR symptoms. For each child, statural heights at baseline and at the 1- and 2-year (where available) visits, together with growth rates, were measured and were compared with predicted values. There were no significant differences between measured and predicted heights at the 1- and 2-year visits. The mean (SD) measured--predicted difference was 0.3 (2.2) cm (95\% confidence interval, -0.4 to 1.0 cm) at the 1-year visit and 0.5 (3.0) cm (95\% confidence interval, -0.6 to 1.6 cm) at the 2-year visit. Mean differences in measured and predicted growth rates were nonsignificant at the 1- and 2-year visits. Triamcinolone acetonide aqueous titered to control AR symptoms and given for 1 or 2 years had no significant effect on statural growth in children with AR. [\hyperlink{Arranon}{PMID: 18939734}, David P Skoner et al., 2008]

\hypertarget{pmid_7315036}{T}he results of using the Soviet drug aminalon in combined therapy of 143 children aged 7 months to 16 years suffering from generalized meningococcal infection are presented. According to the data of catamnestic observation the use of the drug in the early convalescence period contributed to a reduction of the incidence of residual manifestations in the patients, and prevented development of the cerebrospinal syndrome. In the late restorative period, the repeated courses of aminalon in combination with vitamins and other biostimulants are indicated in children with asthenohypodynamic states or having difficulties of learning at school. [\hyperlink{Arranon}{PMID: 7315036}, R I Ivanova et al., 1981]

\hypertarget{pmid_28497918}{T}he nature of allergic rhinitis (AR) in preschool aged children remains incompletely characterized. This study aimed to investigate the prevalence of AR and its associated risk factors in preschool-aged children and to assess the clinical utility of fractional exhaled nitric oxide (FeNO). This general population-based, cross-sectional survey included 933 preschool-aged (3- to 7-year-old) children from Korea. Current AR was defined as having nasal symptoms within the last 12 months and physician-diagnosed AR. The prevalence of current AR in preschool children was 17.0\% (156/919). Mold exposure (adjusted odds ratio [aOR], 1.67; 95\% confidence interval [CI], 1.15-2.43) and the use of antibiotics (aOR, 1.97; 95\% CI, 1.33-2.90) during infancy were associated with an increased risk of current AR, whereas having an older sibling (aOR, 0.52; 95\% CI, 0.35-0.75) reduced the risk. Children with current atopic AR had significantly higher geometric mean levels of FeNO compared to those with non-atopic rhinitis (12.43; range of 1standard deviation [SD], 7.31-21.14 vs 8.25; range of 1SD, 5.62-12.10, P=0.001) or non-atopic healthy children (8.58; range of 1SD, 5.51-13.38, P<0.001). The FeNO levels were higher in children with current atopic AR compared with atopic healthy children (9.78; range of 1SD, 5.97-16.02, P=0.083). Mold exposure and use of antibiotics during infancy increases the risk of current AR, whereas having an older sibling reduces it. Children with current atopic AR exhibit higher levels of FeNO compared with non-atopic rhinitis cases, suggesting that FeNO levels may be a useful discriminatory marker for subtypes of AR in preschool children. [\hyperlink{Arranon}{PMID: 28497918}, Jisun Yoon et al., 2017]

\hypertarget{pmid_18702885}{A}llergic rhinitis (AR) is a common chronic condition in children and may impact a child's quality of life. Increasing treatment compliance may improve quality of life. An oral suspension of fexofenadine hydrochloride (HCl) has been developed to ease administration to children and may, therefore, improve treatment compliance. The purpose of this study was to assess the pharmacokinetic behavior, safety, and tolerability of a single dose of fexofenadine HCl oral suspension administered to children aged 2-5 years with allergic rhinitis. Children (aged 2-5 years) with AR were recruited in a multicenter, open-label, single-dose study. Fexofenadine HCl (30 mg) was administered as a 6-mg/mL suspension (5 mL). Plasma samples were collected up to 24 hours postdose. Adverse events (AEs); electrocardiograms (ECGs); vital signs; and clinical laboratory tests for hematology, blood chemistry, and urinalysis were analyzed to evaluate safety and tolerability. Fifty subjects completed the study. Mean maximum plasma concentration of fexofenadine was 224 ng/mL, and mean area under the plasma concentration curve was 898 ng . hour/mL. Treatment-emergent AEs were mild in intensity and reported in a total of seven subjects. No trends or clinically meaningful changes in mean ECG, vital sign, or clinical laboratory test data occurred during the study. In children aged 2-5 years, the exposure after a 30-mg dose of fexofenadine HCl suspension was similar to the exposures previously seen after a 30- and 60-mg dose of fexofenadine HCl in children aged 6-11 years and in adults, respectively. The suspension was also well tolerated. [\hyperlink{Arranon}{PMID: 18702885}, Nathan Segall et al., ]

\hypertarget{pmid_28827252}{C}eftriaxone is widely used in children in the treatment of sepsis. However, concerns have been raised about the safety of ceftriaxone, especially in young children. The aim of this review is to systematically evaluate the safety of ceftriaxone in children of all age groups. MEDLINE, PubMed, Cochrane Central Register of Controlled Trials, EMBASE, CINAHL, International Pharmaceutical Abstracts and adverse drug reaction (ADR) monitoring systems will be systematically searched for randomised controlled trials (RCTs), cohort studies, case-control studies, cross-sectional studies, case series and case reports evaluating the safety of ceftriaxone in children. The Cochrane risk of bias tool, Newcastle-Ottawa and quality assessment tools developed by the National Institutes of Health will be used for quality assessment. Meta-analysis of the incidence of ADRs from RCTs and prospective studies will be done. Subgroup analyses will be performed for age and dosage regimen. Formal ethical approval is not required as no primary data are collected. This systematic review will be disseminated through a peer-reviewed publication and at conference meetings. CRD42017055428. [\hyperlink{Arranon}{PMID: 28827252}, Linan Zeng et al., 2017]

\hypertarget{pmid_21106710}{I}nformation on the use of oral bisphosphonate agents to treat pediatric osteogenesis imperfecta (OI) is limited. The objective of the investigation was to study the efficacy and safety of daily oral alendronate (ALN) in children with OI. We conducted a multicenter, double-blind, randomized, placebo-controlled study. One hundred thirty-nine children (aged 4-19 yr) with type I, III, or IV OI were randomized to either placebo (n = 30) or ALN (n = 109) for 2 yr. ALN doses were 5 mg/d in children less than 40 kg and 10 mg/d for those 40 kg and greater. Spine areal bone mineral density (BMD) z-score, urinary N-telopeptide of collagen type I, extremity fracture incidence, vertebral area, iliac cortical width, bone pain, physical activity, and safety parameters were measured. ALN increased spine areal BMD by 51\% vs. a 12\% increase with placebo (P < 0.001); the mean spine areal BMD z-score increased significantly from -4.6 to -3.3 (P < 0.001) with ALN, whereas the change in the placebo group (from -4.6 to -4.5) was insignificant. Urinary N-telopeptide of collagen type I decreased by 62\% in the ALN-treated group, compared with 32\% with placebo (P < 0.001). Long-bone fracture incidence, average midline vertebral height, iliac cortical width, bone pain, and physical activity were similar between groups. The incidences of clinical and laboratory adverse experiences were also similar between the treatment and placebo groups. Oral ALN for 2 yr in pediatric patients with OI significantly decreased bone turnover and increased spine areal BMD but was not associated with improved fracture outcomes. [\hyperlink{Arranon}{PMID: 21106710}, L M Ward et al., 2011]

\hypertarget{pmid_16968954}{A}llergic rhinitis (AR) is currently the most common chronic disease in children in the United States, affecting as many as 40\% of the children in the population. AR can substantially decrease a child's quality of life and can contribute to school absenteeism. Intranasal corticosteroids (INS) are thought to be the most effective treatment choice for controlling the symptoms of AR. However, parent and physician concerns over INS safety in children have often led to INS being used as a second-line treatment option. More recently, safety studies have shown that the newer INS agents have improved safety profiles compared with older INS agents. The newer INS drugs have been found to have minimal adverse effects on growth and hypothalamic-pituitary-adrenal-axis function in children, which is a concern when prescribing corticosteroids. This review will discuss the pathophysiology, diagnosis, and classification of AR. The mechanism of action, efficacy, and safety of INS will also be discussed by focusing on clinical evidence. Furthermore, other considerations, such as parent or caregiver education and patient compliance, will be reviewed. [\hyperlink{Arranon}{PMID: 16968954}, Jacqueline Kaari et al., 2006]

\hypertarget{pmid_15951902}{T}o present the evidence regarding the safety of nebulization with 3-5 ml of adrenaline (1:1000) for the treatment of children with acute inflammatory airway obstruction. An electronic search was undertaken, using mainly Medline databases (January of 1949-July of 2004). The study inclusion criteria for this review were: 1) randomized clinical trial; 2) Patients (up to 12 yrs) with diagnosis of bronchiolitis or laryngotracheobronchitis; 3) use of adrenaline (1:1000) by nebulization. The principal data extracted from the trials included adrenaline dosages and their effects on heart rate and blood pressure and any other side-effects. Seven clinical trials with a total of 238 patients were included for this review. Two of the five trials in which larger dosages (> or = 3 ml) of adrenaline were used demonstrated a significant increase in heart rate. The mean increase in heart rate varied from seven to 21 beats per minute, up to 60 minutes after treatment. The highest incidence of pallor was observed in one trial with 21 children treated by nebulization with 3 ml of adrenaline (47.6\% in the adrenaline group vs. 14.3\% in the salbutamol group, 30 minutes after treatment). Two clinical trials failed to observe a significant effect on blood pressure from nebulization with adrenaline (4 and 5 ml). Evidence shows that nebulization with 3 to 5 ml of adrenaline (1:1000) is a safe therapy, with minor side-effects, for children with acute inflammatory airway obstruction. [\hyperlink{Arranon}{PMID: 15951902}, Linjie Zhang et al., ]

\hypertarget{pmid_2183804}{A} 6-month double-blind, parallel, randomized, placebo-controlled multicenter trial of auranofin (0.15-0.20 mg/kg/day) was conducted in 231 children with juvenile rheumatoid arthritis (JRA) in the United States and in the Union of Soviet Socialist Republics. Approximately 80\% of the children had polyarticular disease. The auranofin-treated patients showed greater mean decreases from baseline in 11 of the 12 articular disease indices measured than did the placebo-treated subjects after 3 months of therapy, and in 9 of the 12 indices after 6 months. However, the actual intergroup mean differences were relatively small and were not statistically significant. According to the physician's global assessment, 69\% of the auranofin-treated patients and 61\% of the placebo-treated patients demonstrated clinically significant improvement from baseline after 6 months (P = 0.24). Children whose disease onset occurred less than 2 years prior to entry improved more than did those who had arthritis for a longer period. In addition, those with polyarticular involvement at baseline improved more than did patients with mild disease. However, these relationships were observed in both the auranofin- and placebo-treated groups, and again, there were no significant intergroup differences. Diarrhea was the most common adverse effect of auranofin. We conclude that the clinical efficacy of auranofin is modestly higher than that of placebo in the treatment of JRA, as evidenced by the consistent trends observed in the data. However, the magnitude of the individual intergroup differences is not statistically significant. Auranofin appears to be very safe in children with JRA. [\hyperlink{Arranon}{PMID: 2183804}, E H Giannini et al., 1990]

\hypertarget{pmid_34607935}{T}he over-the-counter nasal decongestant oxymetazoline (eg, Afrin) is used in the pediatric population for a variety of conditions in the operating room setting. Given its vasoconstrictive properties, it can have cardiovascular adverse effects when systemically absorbed. There have been several reports of cardiac and respiratory complications related to use of oxymetazoline in the pediatric population. Current US Food and Drug Administration approval for oxymetazoline is for patients ≥6 years of age, but medical professionals may elect to use it short-term and off label for younger children in particular clinical scenarios in which the potential benefit may outweigh risks (eg, active bleeding, acute respiratory distress from nasal obstruction, acute complicated sinusitis, improved surgical visualization, nasal decongestion for scope examination, other conditions, etc). To date, there have not been adequate pediatric pharmacokinetic studies of oxymetazoline, so caution should be exercised with both the quantity of dosing and the technique of administration. In the urgent care setting, emergency department, or inpatient setting, to avoid excessive administration of the medication, medical professionals should use the spray bottle in an upright position with the child upright. In addition, in the operating room setting, both monitoring the quantity used and effective communication between the surgeon and anesthesia team are important. Further studies are needed to understand the systemic absorption and effects in children in both nonsurgical and surgical nasal use of oxymetazoline. [\hyperlink{Arranon}{PMID: 34607935}, Richard Cartabuke et al., 2021]

\hypertarget{pmid_23715494}{T}his study was designed to analyse the effectiveness of combined treatment of chronic adenoiditis in the children with the use of rinorin (Orion, Finland) in comparison with the traditional methods for the management of this condition either combined with irrigation therapy or without it. The results of the study indicate that the application of rinorin enhance the effectiveness of the treatment due to the substantial reduction of the manifestation of clinical symptoms and the frequency of relapses. The patients describe rinorin as a modern convenient-to-use preparation superior to the traditional medicines for the treatment of adenoiditis which improved medication compliance. [\hyperlink{Arranon}{PMID: 23715494}, Iu L Soldatskiĭ et al., 2013]

\hypertarget{pmid_20027347}{P}rophylactic treatment with anaferon (pediatric formulation) in children groups reduced total morbidity and incidence of acute respiratory viral infections and shortened the duration of fever, intoxication, and catarrhal syndromes. No allergic and other reactions caused by administration of the preparation were noted. [\hyperlink{Arranon}{PMID: 20027347}, M V Kudin et al., 2009]

\hypertarget{pmid_29490769}{T}he safety of a novel intranasal formulation of azelastine hydrochloride (AZE) and fluticasone propionate (FP) has been established in adults and adolescents with allergic rhinitis but not in children <12 years old. To evaluate the safety and tolerability of an intranasal formulation of AZE and FP in children ages 4-11 years with allergic rhinitis. The study was a randomized, 3-month, parallel-group, open-label design. Qualified patients were randomized in a 3:1 ratio to AZE/FP (n = 304) or fluticasone propionate (FP) (n = 101), one spray per nostril twice daily, and to one of three age groups: ≥4 to <6 years, ≥6 to <9 years, and ≥9 to <12 years. Safety was assessed by child- or caregiver-reported adverse events, nasal examinations, vital signs, and laboratory assessments. The incidence of treatment-related adverse events (TRAEs) was low in both the AZE/FP (16\%) and FP-only (12\%) groups after 90 days' continuous use. Epistaxis was the most frequently reported TRAE in both groups (AZE/FP, 9\%; FP, 9\%), followed by headache (AZE/FP, 3\%; FP, 1\%). All other TRAEs in the AZE/FP group were reported by ≤1\% of the children. The majority of TRAEs were of mild intensity and resolved spontaneously. Results of nasal examinations showed an improvement over time in both groups, with no cases of mucosal ulceration or nasal septal perforation. There were no unusual or unexpected changes in laboratory parameters or vital signs. The intranasal formulation of AZE and FP was safe and well tolerated after 3 months' continuous use in children with allergic rhinitis.The study was registered on <ext-link xmlns:xlink="http://www.w3.org/1999/xlink" ext-link-type="uri" xlink:href="http://ClinicalTrials.gov">ClinicalTrials.gov</ext-link> (NCT01794741). [\hyperlink{Arranon}{PMID: 29490769}, William Berger et al., 2018]

\hypertarget{pmid_16911649}{T}o evaluate the efficacy of topical racemic adrenaline (RA) (Micronefrin; Bird Products, Palm Springs, CA, USA) in the control of intraoperative bleeding and the prevention of postoperative bleeding, laryngeal spasm and postoperative pain in adenoidectomy among children <6 years of age. Prospective, randomised, blinded and placebo-controlled trial. Kanta-Hame Central Hospital, a district referral center in Finland. A consecutive sample of 93 children undergoing outpatient adenoidectomy. Patients were randomised to receive topical gauze sponges soaked in either 1:500 RA or 0.9\% sodium chloride (physiological saline) for 3 min after adenoidectomy. Amount of intraoperative bleeding (surgeons' subjective estimate), need for additional packings, need for electrocautery, laryngeal spasm, postoperative bleeding and pain, duration of procedure and duration of patients' stay in the operation room (OR). Adrenaline significantly decreased surgeons' subjective estimate of the amount of intraoperative bleeding (proportion of patients with significant decrease 67 versus 21\%, P < 0.001), reduced the mean number of packings needed (0.6 versus 1.2, P < 0.001) and use of electrocautery (22 versus 45\%, P = 0.015), and shortened the mean duration of the procedure (13 versus 18 min, P = 0.043) and the mean stay in the OR (31 versus 35 min, P = 0.058). The impact of adrenaline was even more pronounced among patients with extensive adenoids and/or profuse intraoperative bleeding. A slight elevation of heart rate was observed more often in the adrenaline group (P = 0.043). Use of topical adrenaline can be recommended in adenoidectomy among children. It helps control the intraoperative bleeding, reduces the use of electrocautery and shortens the durations of procedure and stay in the OR. [\hyperlink{Arranon}{PMID: 16911649}, H Teppo et al., 2006]

\hypertarget{pmid_6432407}{A}n open label, non-controlled trial of six-month duration was designed to determine the safety and efficacy of auranofin in the treatment of 13 children with polyarticular JRA. Adverse reactions were observed in 5 of the 13 patients (38\%) but only in one was it serious enough to discontinue treatment. None of the patients developed diarrhea or hematologic abnormalities. Therapeutic response was evaluated in the 11 patients who completed the six-month treatment. According to the final overall assessment 9 of the 11 children had improved, one remained unchanged and one worsened. After four months of treatment serum gold levels in 11 patients ranged between 28 and 59 micrograms/dl, with a mean value of 34 micrograms/dl. There was no correlation between serum gold levels and the frequency and severity of side effects. [\hyperlink{Arranon}{PMID: 6432407}, O Garcia-Morteo et al., 1984]

\hypertarget{pmid_32974732}{W}e investigated the efficacy and safety of apararenone (MT-3995), a non-steroidal compound with mineralocorticoid receptor agonist activity, in patients with stage 2 diabetic nephropathy (DN). The study had two parts: a dose-response, parallel-group, randomized, double-blind, placebo-controlled, multicenter, phase 2, 24-week study and an open-label, uncontrolled, 28-week extension study. Primary and secondary endpoints were the 24-week percent change from baseline in urine albumin to creatine ratio (UACR) and 24- and 52-week UACR remission rates. Safety parameters were changes from baseline in estimated glomerular filtration rate (eGFR) and serum potassium at 24 and 52 weeks, and incidences of adverse events (AEs) and adverse drug reactions (ADRs). In the dose-response period, 73 patients received placebo and 73, 74, and 73 received apararenone 2.5 mg, 5 mg, and 10 mg, respectively. As a percentage of baseline, mean UACR decreased to 62.9\%, 50.8\%, and 46.5\% in the 2.5 mg, 5 mg, and 10 mg apararenone groups, respectively, at week 24 (placebo: 113.7\% at week 24; all P < 0.001 vs placebo). UACR remission rates at week 24 were 0.0\%, 7.8\%, 29.0\%, and 28.1\% in the placebo and apararenone 2.5 mg, 5 mg, and 10 mg groups, respectively. eGFR tended to decrease and serum potassium tended to increase, but these events were not clinically significant. AE incidence increased with dose while ADR incidence did not. The UACR-lowering effect of apararenone administered once daily for 24 weeks in patients with stage 2 DN was confirmed, and the 52-week administration was safe and tolerable. NCT02517320 (dose-response study) and NCT02676401 (extension study). [\hyperlink{Arranon}{PMID: 32974732}, Takashi Wada et al., 2021]

\hypertarget{pmid_21612039}{T}he post-marketing surveillance of meropenem for children was conducted between May 2004 and September 2006. The safety and the efficacy were analyzed in 1210 cases and 1004 cases, respectively. The results of this surveillance were as follows: The incidence of adverse drug reactions (ADRs) associated with use of meropenem (including abnormal laboratory findings) was 14.3\% (173 cases), and the main ADRs were hepatic function abnormal, alanine aminotransferase increased, and aspartate aminotransferase increased, which were similar to these observed in the clinical study. And the efficacy was 88.6\% (890 cases). [\hyperlink{Arranon}{PMID: 21612039}, Koji Wakisaka et al., 2011]

\hypertarget{pmid_11136494}{T}he clinical effectiveness of amiodarone must be weighed against the likelihood of adverse effects. Adverse effects are less common in children than in adults, yet there have been no large studies assessing the efficacy and safety of amiodarone in the first 9 months of life. We sought to assess the safety and efficacy of amiodarone as primary therapy for supraventricular tachycardia in infancy. We evaluated the clinical course of 50 consecutive infants and neonates (1.0+/-1.5 months, 35 male) treated with amiodarone for supraventricular tachyarrhythmias between July 1994 and July 1999. At presentation, congenital heart disease, congestive heart failure, or ventricular dysfunction were present in 24\%, 36\%, and 44\% of the infants, respectively. Infants received a 7- to 10-day load of amiodarone at either 10 or 20 mg/kg/d. If this failed to control the arrhythmia, oral propranolol (2 mg/kg/d) was added. Patients were followed up for 16.0+/-13.0 months, and antiarrhythmic drugs were discontinued as tolerated. Rhythm control was achieved in all patients. Of the 34 patients who have reached 1 year of age, 23 (68\%) have remained free of arrhythmia, despite discontinuation of propranolol and amiodarone. Growth and development remained normal for age. Higher loading doses of amiodarone were associated with an increase in the corrected QT interval, but no proarrhythmia was seen. There were no side effects necessitating drug withdrawal. Amiodarone is an effective and safe therapy for tachycardia control in infancy. [\hyperlink{Arranon}{PMID: 11136494}, S P Etheridge et al., 2001]

\hypertarget{pmid_3044374}{T}he safety and efficacy of auranofin in the long-term treatment of children with juvenile rheumatoid arthritis was investigated in an open study of 14 patients. Twelve patients completed at least 12 months of treatment, and 7 patients completed 36 months of treatment. Classic parameters of disease activity showed improvement over baseline values after 6 months of treatment, and laboratory indices remained stable or improved throughout the study. Auranofin was well tolerated; the frequency of adverse effects was lower in these patients than has been previously reported in either adults or children whose arthritis has been treated with injectable gold. [\hyperlink{Arranon}{PMID: 3044374}, R Marcolongo et al., 1988]

\hypertarget{pmid_8862938}{T}he efficacy and safety of the nasally administered drug Allergodil in the treatment of allergic rhinitis were evaluated in a prospective drug monitoring programme conducted in Germany. Data from 489 children under the age 13 were included. The study was designed to gain knowledge about Allergodil in a normal clinical setting. Dosing was at the judgement of the investigator bearing in mind data sheet recommendations, i.e. one spray-puff (0.14 mg) per nostril twice daily. Patients were treated for four weeks. The occurrence of ten nasal, eye and throat symptoms was rated (0 = never, 1 = sometimes, 2 = often). All symptoms showed a statistically significant improvement at the final visit, as did the overall sums of the scores. These changes were clinically significant. Overall assessment of efficacy by the physicians and the patients was very good and good in more than 85\% of patients. 70\% of patients required no concomitant medication. 13.5\% of patients experienced adverse events, mostly mild or moderate in severity. Safety and tolerance were assessed as very good and good in more than 97\% of cases. No sedation was seen. With respect to both efficacy and safety, there were no differences between patients younger than 6 years and those aged 6-12 years. In conclusion, these results suggest that Allergodil is an effective treatment of the symptoms of allergic rhinitis in children. The subgroup of 48 young patients studied shows that Allergodil was safe and well tolerated in patients aged 2-6 years. [\hyperlink{Arranon}{PMID: 8862938}, W Lassig et al., 1996]

\hypertarget{pmid_26978049}{T}o assess the efficacy and safety of children tenoten in the treatment of children and adolescents with anxiety disorders. It was conducted a multicenter, double-blind, placebo-controlled trial of the drug tenoten children at a dose of 1 tablet 3 times a day for 12 weeks. The study included 98 patients (boys and girls from 5 to 15 years with a confirmed diagnosis of anxiety disorder), randomized into two groups: the first included 48 patients treated tenotenom children, in the second - 50 patients receiving placebo. Tenoten children has a strong anti-anxiety effect both on the results of self-assessment of patients, and on the reports of parents. This anxiolytic activity of the drug manifested most significantly in children aged 5 to 7 years. In addition, in patients 8-15 years of treatment spent tenotenom children to regress the symptoms of anxiety disorders by anxiety subscales SCAS «Separation anxiety», «panic attacks and agoraphobia» and «social phobia». Throughout the course of treatment tenoten children have been no adverse events. [\hyperlink{Arranon}{PMID: 26978049}, N N Zavadenko et al., 2015]

\hypertarget{pmid_19639035}{I}nhaled short-acting bronchodilators are recommended for the quick relief of bronchospasm symptoms in children including those less than five years of age. However, limited safety data is available in this young population. Safety data were analyzed from a randomized, double-blind, parallel group, placebo-controlled multicenter, study evaluating albuterol HFA 90microg or 180microg versus placebo three times a day for 4 weeks using a valved holding chamber, Aerochamber Plus and facemask in children birth </=24 months old with a history of bronchospasm. THE OVERALL INCIDENCE OF ADVERSE EVENTS (AE) DURING TREATMENT WAS: albuterol 90microg (59\%), albuterol 180microg (76\%) and placebo (71\%). The most frequently reported AEs were pyrexia in 7 (24\%), 2 (7\%), and 3 (11\%) subjects in the albuterol 180microg, albuterol 90microg, and placebo groups, respectively. Upper respiratory tract infection (URTI) occurred in 5 (17\%) and 3 (11\%) subjects in the albuterol 180microg and placebo groups, respectively. Sinus tachycardia occurred in 5 (17\%), 2 (7\%) and 2 (7\%) subjects receiving albuterol 180microg, albuterol 90microg and placebo, respectively. One subject in each of the albuterol treatment groups experienced drug related agitation and/or restlessness or mild sinus arrhythmia. No drug-related QT prolongation or abnormal serum potassium and glucose levels were reported in the albuterol treatment groups. This study provides additional albuterol HFA safety information for the treatment of children aged birth </=24 months with a history of bronchospasm. [\hyperlink{Arranon}{PMID: 19639035}, James A Hedrick et al., 2009]

\section*{Reclast}
\subsection*{Result}
\subsubsection*{Answer}

Unknown

\subsubsection*{{Explanation}}
\hypertarget{Reclast}
A review of the provided abstracts reveals that none of them are about Reclast (zoledronic acid) or its use in children. The abstracts discuss the safety and efficacy of various medications in pediatric populations, including montelukast, statins, tacrolimus, rituximab, rilonacept, and others, but there is no mention of Reclast or any study specifically evaluating its safety in children. Therefore, based on the abstracts available, there is not enough data to determine whether Reclast is safe or unsafe for use in children in any age range.

\subsection*{Abstracts}
\hypertarget{pmid_11759189}{T}his 6-month, open-label extension study of a previously described base study compared oral montelukast with inhaled beclomethasone in terms of safety, forced expiratory volume in one second (FEV1) measurements, parent and patient satisfaction with treatment, asthma-related medical resource utilization, school absenteeism, and parental work loss in children with asthma. A total of 124 of 266 asthmatic children, 6 to 11 years of age, who enrolled in the base study entered a 6-month open-label extension study (74 boys, 50 girls) and were re-randomized (2:1 ratio) to receive once-daily oral montelukast (n = 83) or inhaled beclomethasone 100 mcg three times daily (n = 41). Children were evaluated in the clinic prior to re-randomization (Month 0) and at regular visits at 1, 3, and 6 months. Children and their parents showed a significantly higher overall satisfaction for montelukast at 6 months than for inhaled beclomethasone (p = 0.001 and p < 0.05, respectively). According to parents, montelukast was more convenient (p < 0.001), less difficult to use (p = 0.005), and was used as instructed more of the time (p = 0.006) compared with beclomethasone. Oral corticosteroid use was similar in the montelukast (13\% of patients) and beclomethasone (17\%) treatment groups. The montelukast treatment group was more adherent with their regimen than the inhaled beclomethasone treatment group; almost twice as many children on montelukast compared with inhaled beclomethasone were highly compliant (82\% versus 45\%). The two study groups were similar with respect to overall safety, change in FEV1, asthma-related medical resource utilization, school absenteeism, and parental work loss. Montelukast represents a safe and effective asthma treatment regimen to which children with asthma are more likely to adhere. [\hyperlink{Reclast}{PMID: 11759189}, J F Maspero et al., 2001]

\hypertarget{pmid_19449366}{M}ontelukast is a potent leukotriene-receptor antagonist administered once daily that provides clinical benefit in the treatment of asthma and allergic rhinitis in children and adults. Because of its wide use as a pediatric controller, there is a need for a further review of the safety and tolerability of montelukast in children. To summarize safety and tolerability data for montelukast from previously reported as well as from unpublished placebo-controlled, double-blind, pediatric studies and their active-controlled open-label extension/extended studies. These studies evaluated 2,751 pediatric patients 6 months to 14 years of age with persistent asthma, intermittent asthma associated with upper respiratory infection, or allergic rhinitis. These patients were enrolled in seven randomized, placebo-controlled, double-blind registration and post-registration studies and three active-controlled open-label extension/extended studies conducted by Merck Research Laboratories between 1995 and 2004. Montelukast was well tolerated in all studies. Clinical and laboratory adverse experiences for patients treated with montelukast were generally mild and transient. The most frequent clinical adverse events for all treatments (placebo, montelukast, active control/usual care) in virtually all studies were upper respiratory infection, worsening asthma, pharyngitis, and fever. The clinical and laboratory safety profile for montelukast was similar to that observed for placebo or active control/usual care therapies. The safety profile of montelukast did not change with long-term use. [\hyperlink{Reclast}{PMID: 19449366}, Hans Bisgaard et al., 2009]

\hypertarget{pmid_23631461}{C}urrent American Academy of Pediatrics Guidelines recommended that statins should be considered as a first-line agent in children as early as 8 years of age. The aim of our work is to assess the safety of 3-hydroxy-3-methylglutaryl coenzyme A reductase inhibitors in children with hypercholesterolaemia. Controlled studies in children show that statin monotherapy is efficacious, well tolerated and safe in the short-time. Unfortunately, these studies have relatively short-term follow-up periods, and therefore, long-term safety remains unclear. [\hyperlink{Reclast}{PMID: 23631461}, Norman Lamaida et al., 2013]

\hypertarget{pmid_10922144}{T}o date, only one study of chronic use of a leukotriene receptor antagonist in children has been published. The efficacy and safety of montelukast in children 6-14 years of age with asthma (n = 336) was studied during an 8-week, double-blind, placebo-controlled trial. There was significantly greater improvement in forced expired volume in 1 sec (FEV(1)) from baseline for the montelukast group (8. 23\%) compared to the placebo group (3.58\%). There was a significant decrease in use of beta agonists for symptom relief and a significant decrease in the percentage of days and percentage of patients with asthma exacerbations. An asthma-specific quality of life questionnaire revealed significant overall improvement in quality of life and significant improvement in the quality of life domains for symptoms, activity, and emotions. Adverse effects were not significantly different for montelukast than for placebo, with the exception of allergic rhinitis which was more prevalent in the placebo group. A 6-month open follow-up of patients from the above study was undertaken. Effects of montelukast on FEV(1) were consistent over the 6 months, with the increase in FEV(1) not significantly different from a small control group treated with beclomethasone. Quality of life remained significantly improved throughout the open treatment period. In conclusion, leukotriene receptor antagonists are of value for the treatment of children with asthma. [\hyperlink{Reclast}{PMID: 10922144}, A Becker et al., 2000]

\hypertarget{pmid_10741880}{T}he aim of this article is to review data on the efficacy and safety of montelukast in the treatment of children with asthma. Available published literature, including published abstracts, is reviewed. In patients aged 6 to 14 years with asthma (n = 27), montelukast 5mg demonstrated a significant decrease in exercise-induced bronchoconstriction 20 to 24 hours postdose after 2 days of treatment. For children with chronic asthma, only one study of the regular use of a leukotriene receptor antagonist has been published. The efficacy and safety of montelukast in children aged 6 to 14 years with asthma (n = 336) were studied during an 8-week, double-blind, placebocontrolled trial. There was a significantly greater improvement in forced expiratory volume in 1 second (FEV1) from baseline for the montelukast group (8.23\%) compared with the placebo group (3.58\%). There was a significant decrease in the use of a 3-agonist for symptom relief, as well as in the percentage of days and percentage of patients with asthma exacerbations. An asthma specific quality-of-life (QOL) questionnaire revealed a significant overall improvement in QOL and a significant improvement in the QOL domains for symptoms, activity and emotions in montelukast recipients. There was no significant difference between montelukast and placebo recipients in the frequency of adverse events, with the exception of allergic rhinitis, which was more prevalent in the placebo group. An open label follow-up of patients from the above study was undertaken. The effect of montelukast on FEV1 was consistent for up to 1.4 years, with the increase in FEV1 being not significantly different from that in a small control group treated with inhaled beclomethasone dipropionate. QOL remained significantly improved during the open treatment period. Montelukast appears effective and safe for the treatment of children with asthma. [\hyperlink{Reclast}{PMID: 10741880}, A Becker et al., 2000]

\hypertarget{pmid_34798685}{T}opical tacrolimus is used off-label in young children, but data are limited on its use in children under 2 years of age and for long-term treatment. To compare safety differences between topical tacrolimus (0.03\% and 0.1\% ointments) and topical corticosteroids (mild and moderate potency) in young children with atopic dermatitis (AD). We conducted a 36-month follow-up study with 152 young children aged 1-3 years with moderate to severe AD. The children were followed up prospectively, and data were collected on infections, disease severity, growth parameters, vaccination responses and other relevant laboratory tests were gathered. There were no significant differences between the treatment groups for skin-related infections (SRIs) (P = 0.20), non-SRIs (P = 0.20), growth parameters height (P = 0.60), body weight (P = 0.81), Eczema Area and Severity Index (EASI) (P = 0.19), vaccination responses (P = 0.62), serum cortisone levels (P = 0.23) or serum levels of interleukin (IL)-4, IL-10, IL-12, IL-31 and interferon-γ. EASI decreased significantly in both groups (P < 0.001). In the tacrolimus group, nine patients (11.68\%) had detectable tacrolimus blood concentrations at the 1-week visit. There were no malignancies or severe infections during the study, and blood eosinophil counts were similar in both groups. Topical tacrolimus (0.03\% and 0.1\%) and topical corticosteroids (mild and moderate potency) are safe to use in young children with moderate to severe AD, and have comparable efficacy and safety profiles. [\hyperlink{Reclast}{PMID: 34798685}, A Salava et al., 2022]

\hypertarget{pmid_12107585}{T}o describe a new test for vesicoureteric reflux in children and assess patient preference compared to indirect radionuclide cystography. One hundred and three toilet-trained children aged between 2.1 and 15.6 years underwent percutaneous injection of 10-20 MBq of 99m-technetium-labelled mercapto-acetyl-triglycine (MAG3) into the full bladder after the application of anaesthetic cream. Gamma camera images of the bladder and renal areas were recorded during a 5-min resting period and during micturition. All procedures were successful, 97 with a single stab. Fifty-four of the 66 children who expressed a preference preferred percutaneous suprapubic injection to intravenous injection. Images were easy to interpret and there were no indeterminate results. Of 200 renal units, 33 refluxed during the resting phase and 31 during micturition. In 24 renal units, reflux was only demonstrated during the resting phase. Reflux was significantly associated with abnormalities on dimercaptosuccinic acid (DMSA) scans ( P<0.001). The new technique of direct percutaneous radionuclide cystography is described. It was well tolerated by patients. It detects reflux during the resting phase that would be missed on the indirect study and avoids doubt as to whether activity in the renal areas is due to reflux or excretion. Free pertechnetate could be used and would be much cheaper. [\hyperlink{Reclast}{PMID: 12107585}, A Graham Wilkinson et al., 2002]

\hypertarget{pmid_34742600}{L}idocaine and prilocaine are local anesthetics, a class of medications which are frequently used in clinical medicine to minimize pain in a variety of procedures. They are commonly found in over-the-counter products such as topical anesthetic creams advertised to relieve localized muscle and joint pain. While safe and well-tolerated when used appropriately, an overdose of these anesthetics increases the risk for local anesthetic systemic toxicity (LAST), which in severe cases can present with seizures, cardiac dysrhythmias, and ultimately cardiovascular collapse. The reduced muscle mass of pediatric patients puts them at an increased risk of LAST due to the depot effect of the systemically absorbed anesthetic. Methemoglobinemia may also be associated with local anesthetic toxicity. Our case involves a previously healthy 15-month-old female who presented to one of our networks' emergency departments in status epilepticus following an accidental ingestion of a tube of 2.5\% lidocaine/2.5\% prilocaine cream. Her seizure activity was initially resistant to intraosseous benzodiazepine administration, but ultimately resolved following administration of lipid emulsion and sodium bicarbonate. Additionally, the patient had refractory hypoxia on the monitor which resolved shortly after administration of methylene blue. After stabilization, the patient was transferred to the Pediatric ICU and ultimately made a complete recovery. LAST is a life-threatening presentation which requires early recognition by clinicians, as well as an understanding of the appropriate treatment modalities. We review the assessment and management of LAST, with special focus on the pediatric patient. [\hyperlink{Reclast}{PMID: 34742600}, Kathleen McMahon et al., 2022]

\hypertarget{pmid_24997231}{L}ocust bean gum (LBG) is a galactomannan polysaccharide used as thickener in infant formulas with the therapeutic aim to treat uncomplicated gastroesophageal reflux (GER). Since its use in young infants below 12weeks of age is not explicitly covered by the current scientific concept of the derivation of health based guidance values, the present integrated safety review aimed to compile all the relevant preclinical toxicological studies and to combine them with substantial evidence gathered from the clinical paediatric use as part of the weight of evidence supporting the safety in young infants below 12weeks of age. LBG was demonstrated to have very low toxicity in preclinical studies mainly resulting from its indigestible nature leading to negligible systemic bioavailability and only possibly influencing tolerance. A standard therapeutic level of 0.5g/100mL in thickened infant formula is shown to confer a sufficiently protective Margin of Safety. LBG was not associated with any adverse toxic or nutritional effects in healthy term infants, while there are limited case-reports of possible adverse effects in preterms receiving the thickener inappropriately. Altogether, it can be concluded that LBG is safe for its intended therapeutic use in term-born infants to treat uncomplicated regurgitation from birth onwards.  [\hyperlink{Reclast}{PMID: 24997231}, Leo Meunier et al., 2014] Pruritus is a severe symptom accompanying chronic cholestasis. It can be debilitating and difficult to control. In children, first-line treatments are ursodeoxycholic acid and rifampicin. Refractory pruritus may require invasive therapies including liver transplantation. Clinical trials based on small samples of adult patients suggest that serotonin reuptake inhibitors can improve pruritus in cholestatic or uremic disease. We performed a prospective, multicenter study to assess efficiency and safety of the serotonin reuptake inhibitor sertraline in treating children with refractory cholestatic pruritus. Twenty children experiencing refractory cholestatic pruritus related to Alagille syndrome or progressive familial intrahepatic cholestasis were included from 4 centers between 2007 and 2014, and treated with sertraline at a starting dose of 1 mg · kg · day and thereafter individually adapted up to 4 mg · kg · day. Before and after 3 months with therapy, pruritus was assessed using a visual itching scale graded on 10 points, a skin scratch marks score and a sleeping impairment score. Sertraline was prescribed at a median daily dose of 2.2 mg · kg · day. After 3 months, pruritus improved in 14 out of 20 treated patients, and the median itching score decreased significantly from 8/10 (5-10) to 5/10 (2-10). Likewise, skin scratch marks and sleep quality improved in 9 of these 14 patients. Nonsevere adverse events were reported in 6 children, leading to treatment discontinuation in 3. Our data suggest that sertraline may constitute a useful drug in the management of refractory cholestatic pruritus in children. [\hyperlink{Reclast}{PMID: 24997231}, Alice Thébaut et al., 2017]

\hypertarget{pmid_11074186}{T}his is a multicenter, open-label, add-on trial, investigating the safety and efficacy of ganaxolone (GNX) in a population of children with refractory infantile spasms, or with continuing seizures after a prior history of infantile spasms. A total of 20 children aged 7 months to 7 years were enrolled in this dose-escalation study, after baseline seizure frequencies were established. Concomitant antiepilepsy drugs were maintained throughout the study period. The dose of GNX was progressively increased to 36 mg/kg/d (or to the maximally tolerated dose) over a period of 4 weeks, then maintained for 8 weeks before tapering and discontinuation. Seizure diaries were maintained by the families, and spasm frequency was compared with the baseline period. The occurrence of adverse events was clinically monitored, and global evaluations of seizure severity and response to treatment were obtained. A total of 16 of the 20 subjects completed the study, 15 of whom had refractory infantile spasms at the time of study enrollment. Spasm frequency was reduced by at least 50\% in 33\% of these subjects, with an additional 33\% experiencing some improvement (25-50\% reduction in spasm frequency). Ganaxolone was well tolerated, and adverse events attributed to GNX were generally mild. Ganaxolone was safe and effective in treating this group of refractory infantile spasms patients in an open-label, add-on trial. Further investigation with randomized, controlled study design is warranted. [\hyperlink{Reclast}{PMID: 11074186}, J F Kerrigan et al., 2000]

\hypertarget{pmid_17937851}{T}o determine the therapeutic effectiveness and safety of polyethylene glycol 4000 (forlax) in the treatment of constipation in children over 8 years old. This study was designed as a randomized, positive medicine (lactulose) controlled multicenter trial. A total of 216 children with constipation from 8-18 years old from 7 hospitals across China who were matched with a uniform entry criteria were enrolled in this study. The 216 patients were randomized to receive either oral forlax (20 g/d, n=105) or lactulose (15 mL/d, n=111) for 2 weeks. The therapeutic effects, including bowel movement frequency, stool consistency, clinical complete remission rate of constipation and abdominal symptoms, and the safety of forlax and lactulose were evaluated at 1 and 2 weeks of treatment. The median weekly frequency of bowel movement in the forlax group increased by 4 and 5 times respectively after 1 and 2 weeks of treatment, and increased by 3 and 4 times in the lactulose group (P < 0.05). The stool consistency of the two groups was both improved significantly after treatment. The Bristol score of stool consistency of the forlax and lactulose groups were 3.41+/-1.11 and 3.64+/-1.33 respectively (P < 0.05) after 1 week of treatment, and were 4.26+/-0.89 and 3.63+/-1.33 respectively (P < 0.05) after 2 weeks of treatment. The clinical complete remission rate of constipation in the forlax and lactulose groups was 70\% and 40\% respectively (P < 0.05) by week 1 of treatment, and that was 72\% and 41\% respectively (P < 0.05) by week 2 of treatment. Abdominal pain disappeared in 75\% of patients in the forlax group but in only 57\% in the lactulose group by week 2 of treatment (P < 0.05). No serious adverse events happened and no abnormalities were found in laboratory tests and physical examinations in the two groups after medication. Forlax is safe and effective in the treatment of constipation in children over 8 years old. [\hyperlink{Reclast}{PMID: 17937851}, Bao-Xi Wang et al., 2007]

\hypertarget{pmid_22370535}{P}ranlukast (PLK) is a leukotriene receptor antagonist (LTRA) that has been approved for treatment of asthma in patients of all ages and allergic rhinitis (AR) in adults but not for AR in children in Japan. This randomized, double-blind, placebo-controlled, crossover study used an artificial  exposure chamber (OHIO Chamber) to investigate the efficacy and safety of PLK in children from 10 to 15 years old with seasonal AR (SAR) due to Japanese cedar (JC) pollen. Eighty-four subjects were enrolled and randomized to the treatment arm and 74 were included in the per protocol set. Subjects  received either PLK dry syrup (DS) or placebo for 1 week. They were challenged with JC pollen in the OHIO Chamber for 3 hours. Total nasal symptom scores (TNSSs) were recorded every 30 minutes during the exposure. PLK DS treatment suppressed the TNSS changes from baseline significantly when  compared with placebo. The difference in the least square means in TNSS between the PLK DS-treated group and placebo group was -0.37 (95\% CI, -0.54, -0.20) with a value of p < 0.0001, showing that PLK DS significantly suppressed the nasal symptoms. Regarding specific  nasal symptoms, PLK DS significantly suppressed sneezing, nasal discharge, and nasal obstruction. The effect of PLK DS on nasal obstruction was most prominent, with significant improvement relative to placebo beginning 60 minutes after the start of exposure. No serious adverse events were  reported during the study. In this study, PLK DS is effective and safe for treatment in children with SAR. [\hyperlink{Reclast}{PMID: 22370535}, Ken-ichiro Wakabayashi et al., ]

\hypertarget{pmid_33070945}{I}diopathic nephrotic syndrome is the most common glomerular disease in children, but there are still some difficulties in treating childhood steroid-dependent or steroid-resistant nephrotic syndrome (SDNS/SRNS). Rituximab (RTX) might be an effective and safe choice. Studies were searched from PubMed, Web of Science, Cochrane library and some Chinese databases up to April 2020. Only randomized controlled trials (RCT) were included. Of 1383 screened articles, 6 RCTs with 334 participants were included. RTX was better than the control group at improving relapse-free rate in the short term [RR (risk ratio) (95\% CI (confidence interval)), 1.84(1.41, 2.39)]. As for long-term, RTX did not show significant improvement [RR (95\% CI), 4.43(.57, 34.67)]; but in subgroup analysis, RTX was still better than conventional drugs and tacrolimus [RR (95\% CI), 9.91(1.95, 50.52) and 1.42(1.15, 1.75), respectively]. And there was a difference between the two groups of prednisolone dose after treatment [MD (mean difference) (95\% CI), -.22(-.36, -.09) mg/kg/d)]. However, RTX did not significantly improve serum albumin and creatinine [MD (95\% CI), 3.46(-1.40, 8.32)g/L and -3.66(-11.79, 4.48)μmol/L, respectively]. No significant differences between the RTX and the control group were found in total adverse events (AEs) or serious AEs. Childhood SDNS/SRNS patients appear to benefit from RTX in relapse-free rate and dose of prednisolone use. Also, RTX did not significantly increase the incidence of AEs. But RTX did not show improvements in biological indicators, more studies are required to explain the effect of RTX. [\hyperlink{Reclast}{PMID: 33070945}, Dan Chang et al., 2021]

\hypertarget{pmid_34156376}{R}ilonacept (Araclyst) has been approved to treat recurrent pericarditis and to reduce the risk of recurrence in adults and children 12 years of age and older. The drug is given subcutaneously. [\hyperlink{Reclast}{PMID: 34156376}, Diane S Aschenbrenner et al., 2021]

\hypertarget{pmid_22969980}{T}he aim of this study was to evaluate the efficacy and safety of polyethylene glycol 4000 (PEG 4000) for the treatment of constipation in children over 8 years of age. A total of 216 children from 7 hospitals were enrolled. A total of 105 patients received oral PEG 4000 (20 g/day) and 111 patients received oral lactulose (15 ml/day) for 2 weeks. The stool frequency, stool consistency and abdominal pain of the patients were monitored. In the PEG group, following one week and two weeks of treatment, the median weekly stool frequency improved from 2 times prior to treatment to 6 and 7 times, respectively, following treatment. The clinical remission rates of the PEG and lactulose groups following one week of treatment were 70.48 and 39.64\%, respectively, and following two weeks of treatment were 72.38 and 41.44\%, respectively. Abdominal pain disappeared in 74.6\% of patients following two weeks of PEG 4000 treatment. No significant clinical adverse effects or abnormalities in the laboratory tests were observed in the two treatment groups. In conclusion, PEG 4000 is a safe and more effective drug compared to lactulose for the treatment of constipation in children. [\hyperlink{Reclast}{PMID: 22969980}, Yishi Wang et al., 2012]

\hypertarget{pmid_26364765}{A}lthough second-generation antihistamines, such as bepotastine besilate, are recommended as a first-line treatment option for adult perennial allergic rhinitis (PAR), few non-sedating second-generation antihistamines are safe for children. A double-blind, placebo-controlled, comparative study of 473 pediatric PAR patients (7 - 15 years old) to determine the superiority and safety of bepotastine besilate (10 mg twice daily) relative to placebo for improved total and individual nasal symptom scores compared with baseline. Subjects were randomized to placebo (n = 233) or bepotastine besilate (n = 240, 10 mg orally twice daily for 2 weeks). Interference of daily life by PAR was assessed by measuring change in individual nasal symptom scores from baseline. Bepotastine besilate was superior to placebo in terms of total nasal symptom scores, with improved overall nasal symptoms of PAR compared with baseline values. Subgroup analyses demonstrated bepotastine besilate was effective irrespective of age, sex or body weight. No clinically significant adverse drug reactions often observed with first-generation antihistamines were reported and no difference in adverse events between groups was observed. Bepotastine besilate is effective and safe for pediatric PAR patients aged 7 - 15 years, and has a significant clinical impact on PAR. ClinicalTrials.gov identifier: NCT01861522 ( https://clinicaltrials.gov/ct2/show/NCT01861522 ). [\hyperlink{Reclast}{PMID: 26364765}, Kimihiro Okubo et al., 2015]

\hypertarget{pmid_24210192}{T}his study analyzes the efficacy and safety of a retrievable, fully covered self-expanding metal stent (cSEMS) in the treatment of refractory benign esophageal restenosis in children. This is a retrospective analysis of the application of a newly designed cSEMS in treating refractory benign postoperative restenosis in five children with ages ranging from 16 months to 8 years. Efficacy and safety were evaluated during the follow-up period. cSEMS with or without an antireflux valve at the distal end were successfully placed and removed in five children. These five patients were followed up for 4-12 months after stent removal. Among the five patients, ulcerative stricture was observed in two patients because of reflux esophagitis, while three patients showed no signs of stricture recurrence. Stent migration was observed in three patients, two of which required the stent to be reset. The narrow esophagus was successfully expended to a diameter of 12-13 mm. Besides the observation of mild granulation tissue growth in one case, no severe complications were observed during surgery and after stent placement. Our study suggests that a retrievable, fully covered SEMS is safe and partially effective for treating refractory benign postoperative esophageal restenosis in children during short-term observation. [\hyperlink{Reclast}{PMID: 24210192}, Jie Zhang et al., 2013]

\hypertarget{pmid_17401268}{T}he present study aimed at verifying the safety and efficacy of rifampicin in ameliorating pruritus in cholestatic children. Twenty-three Egyptian children (14 boys and 9 girls), suffering from intractable pruritus of cholestasis, were included. Rifampicin was started at a dose of 10 mg/Kg/day in two divided doses and increased gradually to a maximum of 20 mg/Kg/day if there was no response. Liver function tests were followed up weekly. Seventeen patients (74\%) showed improvement of pruritus with rifampicin. None of the patients showed any deterioration in liver functions. Rifampicin in a dose of 10-20 mg/Kg/day is safe and effective in ameliorating uncontrollable pruritus in children with persistent cholestasis. [\hyperlink{Reclast}{PMID: 17401268}, Hanaa El-Karaksy et al., 2007]

\hypertarget{pmid_22218843}{I}nfections with viruses causing upper respiratory tract infection (URI) are associated with increased leukotriene levels in the upper airways. Montelukast, a selective leukotriene-receptor antagonist, is an effective treatment of asthma and allergic rhinitis. To determine whether prophylactic treatment with montelukast reduces the incidence and severity of URI in children. A randomized, double-blind, placebo-controlled study was performed in 3 primary care pediatric ambulatory clinics in Israel. Healthy children aged 1 to 5 years were randomly assigned in a 1:1 ratio to receive 12-week treatment with 4 mg oral montelukast or look-alike placebo. Patients were excluded if they had a previous history of reactive airway disease. A study coordinator contacted the parents by phone once a week to obtain information regarding the occurrence of acute respiratory episodes. The parents received a diary card to record any acute symptoms of URI. The primary outcome measure was the number of URI episodes. Three hundred children were recruited and randomly assigned into montelukast (n = 153) or placebo (n = 147) groups. One hundred thirty-one (85.6\%) of the children treated with montelukast and 129 (87.7\%) of the children treated with placebo completed 12 weeks of treatment. The number of weeks in which URI was reported was 30.4\% in children treated with montelukast and 30.7\% in children treated with placebo. There was no significant difference in any of the secondary variables between the groups. In preschool-aged children, 12-week treatment with montelukast, compared with placebo, did not reduce the incidence of URI. [\hyperlink{Reclast}{PMID: 22218843}, Eran Kozer et al., 2012]

\hypertarget{pmid_28827252}{C}eftriaxone is widely used in children in the treatment of sepsis. However, concerns have been raised about the safety of ceftriaxone, especially in young children. The aim of this review is to systematically evaluate the safety of ceftriaxone in children of all age groups. MEDLINE, PubMed, Cochrane Central Register of Controlled Trials, EMBASE, CINAHL, International Pharmaceutical Abstracts and adverse drug reaction (ADR) monitoring systems will be systematically searched for randomised controlled trials (RCTs), cohort studies, case-control studies, cross-sectional studies, case series and case reports evaluating the safety of ceftriaxone in children. The Cochrane risk of bias tool, Newcastle-Ottawa and quality assessment tools developed by the National Institutes of Health will be used for quality assessment. Meta-analysis of the incidence of ADRs from RCTs and prospective studies will be done. Subgroup analyses will be performed for age and dosage regimen. Formal ethical approval is not required as no primary data are collected. This systematic review will be disseminated through a peer-reviewed publication and at conference meetings. CRD42017055428. [\hyperlink{Reclast}{PMID: 28827252}, Linan Zeng et al., 2017]

\hypertarget{pmid_26451086}{M}ontelukast (MT) is a leukotriene D4 antagonist. It is an effective and safe medicine for the prophylaxis and treatment of chronic asthma. It is also used to prevent acute exercise-induced bronchoconstriction and as a symptomatic relief of seasonal allergic rhinitis and perennial allergic rhinitis. The aim of this study was to evaluate the bioequivalence (BE) of two drug products: generic MT 5 mg chewable tablets versus the branded drug Singulair(®) pediatric 5 mg chewable tablets among Mediterranean volunteers. An open-label, randomized two-period crossover BE design was conducted in 32 healthy male volunteers with a 9-day washout period between doses and under fasting conditions. The drug concentrations in plasma were quantified by using a newly developed and fully validated liquid chromatography tandem mass spectrometry method, and the pharmacokinetic parameters were calculated using a non-compartmental model. The ratio for generic/branded tablets using geometric least squares means was calculated for both the MT products. The relationship between concentration and peak area ratio was found to be linear within the range 6.098-365.855 ng/mL. The correlation coefficient (R (2)) was always greater than 0.99 during the course of the validation. Statistical comparison of the main pharmacokinetic parameters showed no significant difference between the generic and branded products. The point estimates (ratios of geometric means) were 101.2\%, 101.6\%, and 98.11\% for area under the curve (AUC)0→last, AUC0→inf, and C max, respectively. The 90\% confidence intervals were within the predefined limits of 80.00\%-125.00\% as specified by the US Food and Drug Administration and European Medicines Agency for BE studies. Broncast(®) pediatric chewable tablets (5 mg/tablet) are bioequivalent to Singulair(®) pediatric chewable tablets (5 mg/tablet), with a similar safety profile. This suggests that these two formulations can be considered interchangeable in clinical practice. [\hyperlink{Reclast}{PMID: 26451086}, Abdel Naser Zaid et al., 2015]

\hypertarget{pmid_19194156}{S}ugammadex reverses neuromuscular blockade by chemical encapsulation of rocuronium. This phase IIIA study explored efficacy and safety of sugammadex in infants (28 days to 23 months), children (2-11 yr), adolescents (12-17 yr), and adults (18-65 yr). Anesthetized patients (American Society of Anesthesiologists class 1-2) received 0.6 mg/kg rocuronium and were randomized to receive sugammadex (0.5, 1.0, 2.0, or 4.0 mg/kg) or placebo at reappearance of T2. Neuromuscular monitoring was performed using acceleromyography. Primary endpoint was time from sugammadex/placebo administration to recovery of the train-of-four ratio to 0.9. Adverse events and electrocardiograms were recorded, and blood samples were collected for safety and determination of sugammadex and rocuronium plasma concentrations. A dose-response relation was demonstrated in children (n = 22), adolescents (n = 28), and adults (n = 26), but not infants because of the small sample size (n = 8). After placebo, median recovery time of train-of-four to 0.9 was 21.0, 19.0, 23.4, and 28.5 min in infants, children, adolescents, and adults, respectively. After 2.0 mg/kg sugammadex train-of-four 0.9 was attained in 0.6, 1.2, 1.1, and 1.2 min, respectively. The sugammadex plasma concentrations were similar for the children, adolescent, and adult age groups across the dose range. Sugammadex was well tolerated: No reoccurrence of blockade, inadequate reversal, significant QT prolongation, or other abnormalities were observed. Sugammadex is a new reversal agent that rapidly, effectively, safely, and with similar recovery times reverses rocuronium-induced neuromuscular blockade in children, adolescents, adults, and the small number of infants studied. [\hyperlink{Reclast}{PMID: 19194156}, Benoît Plaud et al., 2009] (1) Respiratory distress and seizures developed in an 18-month-old boy following brief exposure to low-strength (17.6\%) N,N-diethyl-m-toluamide (DEET). A review of the literature revealed 17 reports of DEET-induced encephalopathy in children. The objective of this study was to test the hypothesis that the potential toxicity of DEET is high and that available repellents containing DEET, irrespective of their strength, are not safe when applied to children's skin. (2) Although this is a case report, we used the features of published reports of DEET-induced encephalopathy in children to support the diagnosis, since the evidence that the child's illness was caused by DEET was circumstantial. In the following case analysis, clinical reports of children < 16 years old have been reviewed and analyzed in an effort to relate direct DEET toxicity to various clinical, demographic, and toxic compound exposure factors (Fisher's exacttest and logistic regression analysis). (3) DEET-induced encephalopathy in children (56\% girls) followed not only ingestion or repeated and extensive application of repellents, but also a brief exposure to DEET (45\%). Of those who reported a dermal exposure, 33\% reported an exposure to a product containing DEET < 20\%. Seizures, the most prominent symptom (72\%), were significantly more frequent when DEET solutions were applied to the skin (P<0.01). Mortality (16.6\%) did not correlate significantly with the concentration of the DEET liquid used, duration of skin exposure, pattern of use, age, or sex. (4) Data of this case analysis suggest that repellents containing DEET are not safe when applied to children's skin and should be avoided in children. Additionally, since the potential toxicity of DEET is high, less toxic preparations should be probably substituted for DEET-containing repellents, whenever possible. [\hyperlink{Reclast}{PMID: 19194156}, G Briassoulis et al., 2001]

\hypertarget{pmid_37382130}{T}o study the efficacy and safety of repeated application of rituximab (RTX) at a low dose (200 mg/m A randomized controlled trial was conducted for 29 children with FRNS/SDNS who received systemic treatment in the Department of Nephrology, Anhui Provincial Children's Hospital, from September 2020 to December 2021. These children were divided into a recommended dose group ( After RTX treatment, both the low dose group and the recommended dose group achieved B-lymphocyte depletion and had significant reductions in the number of relapses and glucocorticoid dose ( Repeated RTX treatment at a low dose has comparable clinical efficacy and safety to that at the recommended dose and can significantly reduce the number of FRNS/SDNS relapses and the amount of glucocorticoids used, with little adverse effect throughout the treatment cycle. Therefore, it holds promise for clinical application. [\hyperlink{Reclast}{PMID: 37382130}, Ying Zhu et al., 2023]

\end{document}
