\documentclass{article}
\usepackage[letterpaper, portrait, margin=1in]{geometry}
\usepackage{hyperref}\usepackage[utf8]{inputenc}
\hypersetup{colorlinks=true}
\title{Drug Safety Output}
\begin{document}
\maketitle
\section*{Clindamycin Phosphate}
\subsection*{Result}
\subsubsection*{Answer}

Neonates and infants (<1 year): Yes  
Children (1–14 years): Yes  
Children <3 years: Yes  
Adolescents (12+ years, topical use): Yes  

\subsubsection*{{Explanation}}
\hypertarget{Clindamycin Phosphate}
Based on the available abstracts, several targeted studies have evaluated the safety of clindamycin phosphate in children across different age ranges and indications:

Neonates and Infants (<1 year, including premature and term infants):
- Multiple pharmacokinetic and safety studies have been conducted in neonates and infants. One study of 62 infants with postnatal ages <121 days (median gestational age 28 weeks) found no adverse events related to clindamycin use, and dosing regimens were proposed based on pharmacokinetic modeling [\hyperlink{pmid_26926644}{PMID: 26926644}, Daniel Gonzalez et al., 2016]. Another large retrospective cohort study of 4089 infants receiving clindamycin before postnatal day 121 found that higher clindamycin exposure was not associated with increased odds of death or non-laboratory adverse events, though there was a marginal increase in necrotizing enterocolitis within 7 days in those with the highest exposure [\hyperlink{pmid_31725114}{PMID: 31725114}, Rachel G Greenberg et al., 2020]. Additional studies recommend dose adjustments for neonates due to slower clearance and longer half-life, but do not report significant safety concerns [\hyperlink{pmid_6470871}{PMID: 6470871}, M J Bell et al., 1984; \hyperlink{pmid_3737273}{PMID: 3737273}, G Koren et al., 1986].

Children (1–14 years):
- Several studies have evaluated clindamycin phosphate (alone or in combination) for various infections in children. In a study of 29 children with osteomyelitis (age not specified but described as children), clindamycin phosphate was used for up to nine weeks with no diarrhea or enterocolitis reported [\hyperlink{pmid_910760}{PMID: 910760}, W Rodriguez et al., 1977].
- For malaria, studies in children aged 1–14 years using clindamycin in combination with fosmidomycin reported that the combination was generally well tolerated, though there were some concerns about neutropenia and hemoglobin drops in a subset [\hyperlink{pmid_16870763}{PMID: 16870763}, Steffen Borrmann et al., 2006]. In children aged 7–14 years, the combination was well tolerated with no serious adverse events [\hyperlink{pmid_15478056}{PMID: 15478056}, Steffen Borrmann et al., 2004; \hyperlink{pmid_14976608}{PMID: 14976608}, Steffen Borrmann et al., 2004].
- For topical use in acne, large randomized controlled trials in adolescents (12 years and older) and pooled analyses (including 2915 adolescents) found clindamycin phosphate (alone or in combination) to have a favorable safety and tolerability profile, with adverse event rates similar to vehicle or comparators [\hyperlink{pmid_24704684}{PMID: 24704684}, Brigitte Dréno et al.; \hyperlink{pmid_26345297}{PMID: 26345297}, Guy Webster et al., 2015; \hyperlink{pmid_18805603}{PMID: 18805603}, Diane Thiboutot et al., 2008; \hyperlink{pmid_15696985}{PMID: 15696985}, Alan R Shalita et al.].
- In a study of 103 children with streptococcal pharyngitis, clindamycin palmitate was as effective as penicillin, with some rash observed but no serious adverse events [\hyperlink{pmid_4208902}{PMID: 4208902}, M Stillerman et al., 1973].

Children <3 years:
- A study specifically in children <3 years with malaria found the combination of fosmidomycin and clindamycin to be well tolerated, with most adverse events related to malaria recurrence rather than the drug, but efficacy was poor [\hyperlink{pmid_22430959}{PMID: 22430959}, Miguel Lanaspa et al., 2012].

Summary:
- For neonates and infants (including premature), targeted studies support the safety of clindamycin phosphate when dosed appropriately, with no major safety signals.
- For children aged 1–14 years, both systemic and topical clindamycin phosphate have been studied, with good safety profiles reported in targeted pediatric studies.
- For children <3 years, safety appears acceptable in the available studies, though efficacy for malaria was suboptimal in this age group.
- For adolescents (12+), topical clindamycin phosphate is well studied and safe.

No studies were found that specifically demonstrated clindamycin phosphate to be unsafe in children.


\subsection*{Abstracts}
\hypertarget{pmid_910760}{C}lindamycin phosphate was used in the treatment of 29 children with osteomyelitis of whom 25 had an acute and four a chronic type of infection. The usual dose was 50 mg/kg/day intravenously for approximately three weeks followed by oral clindamycin palmitate at home in a dose of 30 mg/kg/day for an additional six weeks. Staphylococcus aureus was isolated in 22 of 29 cases: 96\% of strains were penicillin resistant. The clinical and bacteriologic results in the present series were good to excellent. There was prompt clinical and bacteriologic response shortly after initiation of clindamycin therapy. Good bone penetration of the drug was observed. Long-term evaluation revealed satisfactory clinical and roentgenographic progress in all patients. No diarrhea or manifestations of enterocolitis appeared in any patient in spite of high doses of the drug for intervals up to nine weeks. [\hyperlink{Clindamycin Phosphate}{PMID: 910760}, W Rodriguez et al., 1977]

\hypertarget{pmid_26926644}{C}lindamycin may be active against methicillin-resistant Staphylococcus aureus, a common pathogen causing sepsis in infants, but optimal dosing in this population is unknown. We performed a multicenter, prospective pharmacokinetic (PK) and safety study of clindamycin in infants. We analyzed the data using a population PK analysis approach and included samples from two additional pediatric trials. Intravenous data were collected from 62 infants (135 plasma PK samples) with postnatal ages of <121 days (median [range] gestational age of 28 weeks [23 to 42] and postnatal age of 17 days [1 to 115]). In addition to body weight, postmenstrual age (PMA) and plasma protein concentrations (albumin and alpha-1 acid glycoprotein) were found to be significantly associated with clearance and volume of distribution, respectively. Clearance reached 50\% of the adult value at PMA of 39.5 weeks. Simulated PMA-based intravenous dosing regimens administered every 8 h (≤32 weeks PMA, 5 mg/kg; 32 to 40 weeks PMA, 7 mg/kg; >40 to 60 weeks PMA, 9 mg/kg) resulted in an unbound, steady-state concentration at half the dosing interval greater than a MIC for S. aureus of 0.12 μg/ml in >90\% of infants. There were no adverse events related to clindamycin use. (This study has been registered at ClinicalTrials.gov under registration no. NCT01728363.). [\hyperlink{Clindamycin Phosphate}{PMID: 26926644}, Daniel Gonzalez et al., 2016]

\hypertarget{pmid_6470871}{T}he pharmacokinetics of intravenously administered clindamycin phosphate was studied in 40 children less than 1 year of age. Mean peak serum concentrations were 10.92 micrograms/ml in premature infants less than 4 weeks of age, 10.45 micrograms/ml in term infants greater than 4 weeks, and 12.69 micrograms/ml in term infants less than 4 weeks of age. Mean trough concentrations were 5.52, 2.8, and 3.03 micrograms/ml, respectively, in the same groups. Serum half-life was significantly longer (8.68 vs 3.60 hours) in premature compared with term infants less than 4 weeks of age. Both premature and term infants less than 4 weeks had significantly decreased clearance when compared with infants greater than 4 weeks (0.294 and 0.678, respectively, vs 1.58 L/hr). Clearance was significantly greater (1.919 vs 0.310 L/hr) and serum half-life less (1.75 vs 7.57 hours) in infants with body weight greater than 3.5 kg. On the basis of these data it is recommended that in infants greater than 4 weeks or greater than 3.5 kg, intravenous clindamycin dosage be 20 mg/kg/day in four divided doses. In premature neonates less than 4 weeks, the dose should be reduced to 15 mg/kg/day in three divided doses. Term infants greater than 1 week of age may also receive 20 mg/kg/day in four doses. [\hyperlink{Clindamycin Phosphate}{PMID: 6470871}, M J Bell et al., 1984]

\hypertarget{pmid_16870763}{F}osmidomycin plus clindamycin was shown to be efficacious in the treatment of uncomplicated Plasmodium falciparum malaria in a small cohort of pediatric patients aged 7 to 14 years, but more data, including data on younger children with less antiparasitic immunity, are needed to determine the potential value of this new antimalarial combination. We conducted a single-arm study to improve the precision of efficacy estimates for an oral 3-day fixed-ratio combination of fosmidomycin and clindamycin at 30 and 10 mg/kg of body weight, respectively, every 12 hours for the treatment of uncomplicated P. falciparum malaria in 51 pediatric outpatients aged 1 to 14 years. Fosmidomycin plus clindamycin was generally well tolerated, but relatively high rates of treatment-associated neutropenia (8/51 [16\%]) and falls of hemoglobin concentrations of > or =2 g/dl (7/51 [14\%]) are of concern. Asexual parasites and fever were cleared within median periods of 42 h and 38 h, respectively. All patients who could be evaluated were parasitologically and clinically cured by day 14 (49/49; 95\% confidence interval [CI], 93 to 100\%). The per-protocol, PCR-adjusted day 28 cure rate was 89\% (42/47; 95\% CI, 77 to 96\%). Efficacy appeared to be significantly reduced in children aged 1 to 2 years, with a day 28 cure rate of only 62\% for this small subgroup (5/8). The inadequate efficacy in children of <3 years highlights the need for continued systematic studies of the current dosing regimen, which should include randomized trial designs. [\hyperlink{Clindamycin Phosphate}{PMID: 16870763}, Steffen Borrmann et al., 2006]

\hypertarget{pmid_4208902}{C}lindamycin palmitate and potassium phenoxymethyl penicillin were evaluated in 103 children with upper respiratory illnesses and pharyngeal group A streptococci, from November 1970 to July 1971. The children were assigned randomly by weight to one of the antibiotic regimens given orally for 10 days. Clindamycin palmitate and potassium phenoxymethyl penicillin dosages were 75 and 125 mg, respectively, in 5 ml tid for children weighing less than 25 kg, and 150 and 250 mg, respectively, in 10 ml bid for children weighing 25 kg or more. Recurrences of the original streptococcal group A, M, and T types within 3 weeks after the end of treatment were classified as failures. The failure rates were: clindamycin palmitate, 10\% (5 of 52), and potassium phenoxymethyl penicillin, 18\% (9 of 51). Possible drug-related rashes were observed in 8 of 52 clindamycin palmitate-treated patients. The geometric mean minimal inhibitory concentrations of clindamycin and penicillin against 103 isolates of group A streptococci were 0.033 and 0.007 mug/ml, respectively. The serum concentrations about 70 min after ingesting 150 mg of clindamycin palmitate averaged 3.8 mug/ml and after 250 mg of potassium phenoxymethyl penicillin averaged 0.9 mug/ml. Clindamycin palmitate was as effective as potassium phenoxymethyl penicillin in eradicating group A streptococci from the pharynx in tid and bid regimens. Nevertheless, because of its rash-producing tendency in some patients and higher cost, clindamycin palmitate should not be preferred to penicillin for treatment of streptococcal sore throat in the non-penicillin-allergic patient. [\hyperlink{Clindamycin Phosphate}{PMID: 4208902}, M Stillerman et al., 1973]

\hypertarget{pmid_28339377}{C}lindamycin is an effective antibiotic in the treatment of infections caused by certain gram-positive and gram-negative anaerobic microorganisms. While manufactured forms of the drug for pediatric use are available, there are instances when a compounded liquid dosage form is essential to meet unique patient needs. The purpose of this study was to determine the chemical stability of clindamycin hydrochloride in the PCCA base SuspendIt, a sugar-free, paraben- free, dye-free, and gluten-free thixotropic vehicle containing a natural sweetener obtained from the monk fruit. It thickens upon standing to minimize settling of any insoluble drug particles and becomes fluid upon shaking to allow convenient pouring during administration to the patient. A robust stability-indicating high-performance liquid chromatographic assay for the determination of clindamycin hydrochloride in SuspendIt was developed and validated. This assay was used to determine the chemical stability of the drug in SuspendIt. Samples were prepared and stored under three different temperature conditions (5°C, 25°C, and 40°C), and assayed using the high-performance liquid chromatographic assay at pre-determined intervals over an extended period of time as follows: 7, 14, 30, 45, 60, 91, 120, and 182 days at each designated temperature. Physical data such as pH, viscosity, and appearance were also monitored. The study showed that drug concentration did not go below 90\% of the label claim (initial drug concentration) at all three temperatures studied, barring isolated experimental errors. Viscosity and pH values also did not change significantly. Some variations in viscosity were attributed to the thixotropic nature of the vehicle. This study demonstrates that clindamycin hydrochloride is physically and chemically stable in SuspendIt for 182 days in the refrigerator and at room temperature, thus providing a viable, compounded alternative for clindamycin hydrochloride in a liquid dosage form, with an extended beyond-use date to meet patient needs. [\hyperlink{Clindamycin Phosphate}{PMID: 28339377}, Yashoda V Pramar et al., ]

\hypertarget{pmid_143716}{C}lindamycin phosphate was administered intravenously to 41 patients with different types of infections including osteomyelitis, septicaemia and soft tissue infections. All bacterial strains tested showed low MIC values for clindamycin. Maximum serum concentrations after 600 mg intravenously were 6.0--29.0 microgram/ml, after 300 mg intravenously 2.6--26.0 microgram/ml. The therapeutic effect of the drug was considered good in 26 of 31 patients with proven or probable bacterial aetiology. Side effects were noted in 16 of the 41 patients. However, in only 5 of these the treatment had to be terminated, all due to pruritic rashes. In the 7 cases with diarrhoea as side effect, the symptoms were mild and of short duration. [\hyperlink{Clindamycin Phosphate}{PMID: 143716}, H Hugo et al., 1977]

\hypertarget{pmid_31725114}{D}espite the absence of adequate safety or efficacy data, clindamycin is widely prescribed in the neonatal intensive care unit. We evaluated the association between clindamycin exposure and adverse events, as well as antibiotic effectiveness in infants. This was a retrospective cohort study of infants receiving clindamycin before postnatal day 121 who were discharged from a Pediatrix Medical Group neonatal intensive care unit (1997-2015). Using a previously developed pharmacokinetic model, we performed simulations to predict clindamycin exposure based on available dosing data. We used multivariable logistic regression to evaluate the association between clindamycin exposure and safety outcomes during and after clindamycin therapy. We reported the proportion of infants with methicillin-resistant Staphylococcus aureus (MRSA) bacteremia and clearance of MRSA bacteremia. A total of 4089 infants received clindamycin at a median (25th-75th percentile) dose of 15 mg/kg/d (12-16). Clearance increased with older gestational age. Infants with the highest total clindamycin exposure had marginally increased odds of necrotizing enterocolitis within 7 days (adjusted odds ratio = 1.95 [1.04-3.63]), but exposure was not associated with death, sepsis, seizures, intestinal perforation or intestinal strictures. Of 25 infants who had MRSA bacteremia, 19 (76\%) cleared the infection by the end of the clindamycin course. Higher clindamycin exposure was not associated with increased odds of death or nonlaboratory adverse events. The use of pharmacokinetic models combined with available electronic health record data offers a valuable, cost-effective approach to analyzing the safety and effectiveness of drugs in infants when large-scale trials are not feasible. [\hyperlink{Clindamycin Phosphate}{PMID: 31725114}, Rachel G Greenberg et al., 2020]

\hypertarget{pmid_2292542}{T}he Cystic Fibrosis Clinic at the Royal Belfast Hospital for Sick Children has treated 31 children with ciprofloxacin, for serious pseudomonas infection in cystic fibrosis, and carefully monitored the safety and acceptability of the drug. Initially, eight very ill children were treated on a named-patient basis, with an encouraging clinical response and few adverse effects. Children aged 10-18 years were included in a study of four consecutive exacerbations of respiratory disease, comparing (i) oral ciprofloxacin in each episode with (ii) ciprofloxacin alternating with intravenous azlocillin and tobramycin. Other children with cystic fibrosis were subsequently treated with ciprofloxacin, as the need arose. In all the groups very few adverse reactions were found; in particular only one child developed arthralgia. A total of 202 children in the UK have been treated with ciprofloxacin on a named-patient basis, and their clinicians have reported 46 adverse events that may have been drug-related. Overall ciprofloxacin appears to be safe and effective in children but concern about the possible occurrence of arthropathy remains and long term follow-up of these children may be necessary. [\hyperlink{Clindamycin Phosphate}{PMID: 2292542}, A Black et al., 1990]

\hypertarget{pmid_24949994}{C}lindamycin is commonly prescribed to treat children with skin and skin-structure infections (including those caused by community-acquired methicillin-resistant Staphylococcus aureus (CA-MRSA)), yet little is known about its pharmacokinetics (PK) across pediatric age groups. A population PK analysis was performed in NONMEM using samples collected in an opportunistic study from children receiving i.v. clindamycin per standard of care. The final model was used to optimize pediatric dosing to match adult exposure proven effective against CA-MRSA. A total of 194 plasma PK samples collected from 125 children were included in the analysis. A one-compartment model described the data well. The final model included body weight and a sigmoidal maturation relationship between postmenstrual age (PMA) and clearance (CL): CL (l/h) = 13.7 × (weight/70)(0.75) × (PMA(3.1)/(43.6(3.1) + PMA(3.1))); V (l) = 61.8 × (weight/70). Maturation reached 50\% of adult CL values at \textasciitilde{}44 weeks PMA. Our findings support age-based dosing.  [\hyperlink{Clindamycin Phosphate}{PMID: 24949994}, D Gonzalez et al., 2014] The combination of fosmidomycin and clindamycin (F/C) is effective in adults and older children for the treatment of malaria and could be an important alternative to existing artemisinin-based combinations (ACTs) if proven to work in younger children. We conducted an open-label clinical trial to assess the efficacy, safety, and tolerability of F/C for the treatment of uncomplicated P. falciparum malaria in Mozambican children <3 years of age. Aqueous solutions of the drugs were given for 3 days, and the children were followed up for 28 days. The primary outcome was the PCR-corrected adequate clinical and parasitological response at day 28. Secondary outcomes included day 7 and 28 uncorrected cure rates and fever (FCT) and parasite (PCT) clearance times. Fifty-two children were recruited, but only 37 patients were evaluable for the primary outcome. Day 7 cure rates were high (94.6\%; 35/37), but the day 28 PCR-corrected cure rate was 45.9\% (17/37). The FCT was short (median, 12 h), but the PCT was longer (median, 72 h) than in previous studies. Tolerability was good, and most common adverse events were related to the recurrence of malaria. The poor efficacy observed for the F/C combination may be a consequence of the new formulations used, differential bioavailability in younger children, naturally occurring variations in parasite sensitivity to the drugs, or an insufficient enhancement of their effects by naturally acquired immunity in young children. Additional studies should be conducted to respond to the many uncertainties arising from this trial, which should not discourage further evaluation of this promising combination. [\hyperlink{Clindamycin Phosphate}{PMID: 24949994}, Miguel Lanaspa et al., 2012]

\hypertarget{pmid_9194107}{M}ore data on the efficacy and safety of ciprofloxacin in pediatric cystic fibrosis patients are needed. One hundred eight pediatric cystic fibrosis patients (ages 5 to 17 years) with acute bronchopulmonary exacerbations entered a randomized multicenter trial designed to compare the safety and efficacy of antipseudomonas therapy with oral ciprofloxacin (15 mg/kg twice daily; maximum dosage 750 mg twice daily) or intravenous ceftazidime plus tobramycin (CAZ/TM) for 14 days. Clinical improvement was observed in 93\% of patients treated with oral ciprofloxacin and in 96\% of those receiving parenteral therapy. Transient suppression of Pseudomonas aeruginosa was achieved in 63\% of patients at the end of the course of iv CAZ/TM therapy and in 24\% receiving ciprofloxacin. Ultrasound examination and nuclear magnetic resonance imaging scans showed no evidence of cartilage toxicity in any of the ciprofloxacin-treated patients. Musculoskeletal adverse events were reported with similar frequency in the two groups of patients (7\% in the group receiving ciprofloxacin therapy and 11\% in the IV CAZ/TM group). The only sustained musculoskeletal symptom was a case of synovitis in a patient receiving parenteral CAZ/TM. Ciprofloxacin thus appears to be safe and effective for use in young patients with bronchopulmonary exacerbation of cystic fibrosis. [\hyperlink{Clindamycin Phosphate}{PMID: 9194107}, D A Richard et al., 1997]

\hypertarget{pmid_634877}{C}lindamycin phosphate is an antibiotic which is effective against both Staphylococcus aureus and the anaerobic organisms. In thirteen patients, its concentration following joint replacement was measured by the agar diffusion method. In bone, the concentration was (mean +/- s.e. mean) 5.01 microgram/ml +/- 1.16, N=10; in capsule, 3.29 microgram/ml +/- 0.71, N=12; measured between 1.75 and 3.75 hr after intramuscular and intravenous injections, and in drainage fluid it amounted to 4.61 microgram/ml +/- 0.38, N=11 in 48 hr. Two patients developed diarrhoea which settled within a short period. [\hyperlink{Clindamycin Phosphate}{PMID: 634877}, P Baird et al., 1978]

\hypertarget{pmid_3737273}{W}e studied 12 newborn infants (gestational ages 26-39 wk [mean +/- SD, 30.6 +/- 4.7]; birth weight 640-2700 g, [mean, 1,322 +/- 688]; postnatal age 1-24 days [mean, 9.6 +/- 8.5]) who received clindamycin phosphate for suspected or proven necrotizing enterocolitis (ten patients) or suspected anaerobic septicemia (two patients) in doses of 3.2-11 mg/kg every six hours. Range of mean serum concentration of clindamycin at steady state was between 12.7 and 40 micrograms/ml (therapeutic range = 2-10 micrograms/ml). High concentrations could be attributed to elimination T1/2 (6.3 +/- 2.1 hr) 100\% longer than in older children or adults. Clindamycin clearance (61.6 +/- 31.6 hr ml/kg/hr) was lower than in older children or adults. Because of the observed prolongation in T1/2 and correspondingly lower clearance, the IV dose of clindamycin in newborn infants should be reduced to 15-20 mg/kg/day given in four daily doses. [\hyperlink{Clindamycin Phosphate}{PMID: 3737273}, G Koren et al., 1986]

\hypertarget{pmid_15478056}{I}t has been demonstrated that fosmidomycin has good tolerability and rapid onset of action, but late recrudescences preclude its use alone; in vitro, clindamycin has been shown to act synergistically with fosmidomycin against Plasmodium falciparum. We conducted a study in pediatric outpatients with P. falciparum malaria in Gabon to evaluate the efficacy and safety of an oral combination of fosmidomycin-clindamycin of 30 mg/kg and 10 mg/kg of body weight, respectively, every 12 h. Patients 7-14 years old were recruited in cohorts of 10. The first 10 patients were treated for 5 days. The duration of treatment was then incrementally shortened in intervals of 1 day if >85\% of the patients in a cohort were cured by day 14. All dosing regimens were well tolerated, and no serious adverse events occurred. Asexual parasites and fever rapidly cleared in all patients. Cure ratios of 100\% on day 14 were achieved with treatment durations of 5 (10/10 patients), 4 (10/10 patients), 3 (10/10 patients), and 2 days (10/10 patients); 1 day of treatment led to a cure ratio of 50\% (5/10 patients). Fosmidomycin-clindamycin is safe and well tolerated, and short-course regimens achieved high efficacy in children with P. falciparum malaria. Fosmidomycin-clindamycin is a promising novel treatment option for malaria. [\hyperlink{Clindamycin Phosphate}{PMID: 15478056}, Steffen Borrmann et al., 2004]

\hypertarget{pmid_26345297}{T}o investigate the cutaneous safety and tolerability of clindamycin phosphate 1.2\%/benzoyl peroxide 3.75\% gel in moderate-to-severe acne patients. A safety assessment of 498 patients with moderate-to-severe acne receiving clindamycin phosphate 1.2\%/benzoyl peroxide 3.75\% gel or vehicle for 12 weeks. The vast majority (80-95\%) of patients reported no cutaneous safety or tolerability problems throughout the study. Mean scores for both active and vehicle were all <1 (where l=mild) and reduced over the duration of the study. When scaling, erythema, itching, burning, or stinging was reported it was generally mild. Moderate or severe reactions to clindamycin phosphate 1.2\%/benzoyl peroxide 3.75\% gel were rare and generally seen early in treatment. There were eight reports (3.3\%) of moderate erythema, four reports (1.7\%) of moderate scaling, three reports (1.2\%) of moderate itching, and one report of moderate burning (0.4\%) at Week 4. There was one report (0.4\%) of severe erythema and one report (0.4\%) of severe burning (both at Week 4), with one report (0.4\%) of severe stinging at Week 12. There were no substantive differences seen in cutaneous tolerability among treatment groups and younger patients tended to have milder reactions. It is not possible to determine the contributions of the individual active ingredients. Clindamycin phosphate 1.2\%/benzoyl peroxide 3.75\% gel has a favorable safety and tolerability profile with very low incidence of moderate or severe reactions. [\hyperlink{Clindamycin Phosphate}{PMID: 26345297}, Guy Webster et al., 2015]

\hypertarget{pmid_1191404}{C}lindamycin phosphate, a new semisynthetic antibiotic that is effective in the treatment of toxoplasmosis and of infections caused by Gram-positive bacteria, was found to be highly concentrated in the choroid, iris, and retina of the pigmented rabbit eye after a single intramuscular injection of 75 mg/kg. Drug levels considered adequate for the control of most ocular infections were detectable in the iris, choroid, and retina 24 hours after injection, at which time serum levels were negligible. Subconjunctival injection of clindamycin phosphate also produced sustained high levels of drug in the choroid, iris, and retina; but when 150 mg was injected in a volume of 1 ml, corneal edema and severe inflammation of the conjunctiva resulted. Lesser amounts (15 to 35 mg) injected subconjunctivally produced adequate ocular tissue levels without damage to the conjunctiva or cornea. [\hyperlink{Clindamycin Phosphate}{PMID: 1191404}, K F Tabbara et al., 1975]

\hypertarget{pmid_14976608}{F}osmidomycin is a new antimalarial drug with a novel mechanism of action. Studies in Africa that have evaluated fosmidomycin as monotherapeutic agent demonstrated its excellent tolerance, but 3-times-daily treatment regimens of >or=4 days were required to achieve radical cure, prompting further research to identify and validate a suitable combination partner to enhance its efficacy. We conducted a randomized, controlled, open-label study to evaluate the efficacy and safety of fosmidomycin combined with clindamycin (n=12; 30 and 5 mg/kg body weight every 12 h for 5 days, respectively), compared with fosmidomycin alone (n=12; 30 mg/kg body weight every 12 h for 5 days) and clindamycin alone (n=12; 5 mg/kg body weight every 12 h for 5 days) for the clearance of asymptomatic Plasmodium falciparum infections in schoolchildren in Gabon aged 7-14 years. Asexual parasites were rapidly cleared in children treated with fosmidomycin-clindamycin (median time, 18 h) and fosmidomycin alone (25 h) but slowly in children treated with clindamycin alone (71 h; P=.004). However, only treatment with fosmidomycin-clindamycin or clindamycin alone led to the radical elimination of asexual parasites as measured by day 14 and 28 cure rates of 100\%. Asexual parasites reappeared by day 28 in 7 children who received fosmidomycin (day 14 cure rate, 92\% [11/12; day 28 cure rate, 42\% [5/12]). All regimens were well tolerated, and no serious adverse events occurred. The combination of fosmidomycin and clindamycin is well tolerated and superior to either agent on its own with respect to the rapid and radical clearance of P. falciparum infections in African children. [\hyperlink{Clindamycin Phosphate}{PMID: 14976608}, Steffen Borrmann et al., 2004]

\hypertarget{pmid_24704684}{T}he efficacy and safety of clindamycin phosphate 1.2\%/tretinoin 0.025\% (Clin-RA) were evaluated in three 12-week randomised studies. To perform a pooled analysis of data from these studies to evaluate Clin-RA's efficacy and safety in a larger overall population, in subgroups of adolescents and according to acne severity. 4550 patients were randomised to Clin-RA, clindamycin, tretinoin and vehicle. Evaluations included percentage change in lesions, treatment success rate, proportions of patients with ≥50\% or ≥80\% continuous reduction in lesions, adverse events and cutaneous tolerability. In the overall population, the percentage reduction in inflammatory, non-inflammatory and total lesions and the treatment success rate were significantly greater with Clin-RA compared with clindamycin, tretinoin and vehicle alone (all p<0.01). The percentage reduction in all types of lesions was also significantly greater with Clin-RA in the adolescent subgroup (2915 patients, p<0.002) and in patients with mild/moderate acne (3662 patients, p<0.02) versus comparators. In patients with severe acne (n = 880), the percentage reduction in all lesion types was significantly greater with Clin-RA versus vehicle (p<0.0001). A greater proportion of Clin-RA treated patients had a ≥50\% or ≥80\% continuous reduction in all types of lesions at week 12 compared with clindamycin, tretinoin and vehicle. Adverse event frequencies in the active and vehicle groups were similar. Baseline-adjusted mean tolerability scores over time were <1 (mild) and similar in all groups. Clin-RA is safe, has superior efficacy to its component monotherapies and should be considered as one of the first-line therapies for mild-to-moderate facial acne. [\hyperlink{Clindamycin Phosphate}{PMID: 24704684}, Brigitte Dréno et al., ]

\hypertarget{pmid_557118}{T}wenty eight patients were treated with parenteral clindamycin-2-phosphate in the field of surgery, and good response was obtained in a series of superficial soft tissue infection, especially caused by staphylococci, with a daily dose of 300 mg. Serum level and urinary excretion were also investigated in four healthy male volunteers. [\hyperlink{Clindamycin Phosphate}{PMID: 557118}, Y Shiraha et al., 1977]

\hypertarget{pmid_18805603}{W}e sought to evaluate efficacy, safety, and tolerability of a combination of clindamycin phosphate 1.2\% and benzoyl peroxide 2.5\% (clindamycin-BPO 2.5\%) aqueous gel in moderate to severe acne vulgaris. A total of 2813 patients, aged 12 years or older, were randomized to receive clindamycin-BPO 2.5\%, individual active ingredients, or vehicle in two identical, double-blind, controlled 12-week, 4-arm studies evaluating safety and efficacy (inflammatory and noninflammatory lesion counts) using Evaluator Global Severity Score and subject self-assessment. Clindamycin-BPO 2.5\% demonstrated statistical superiority to individual active ingredients and vehicle in reducing both inflammatory and noninflammatory lesions and acne severity. Visibly greater improvement was observed by patients with clindamycin-BPO 2.5\% as early as week 2. No substantive differences were seen in cutaneous tolerability among treatment groups and less than 1\% of patients discontinued treatment because of adverse events. Data from controlled studies may differ from clinical practice. Clindamycin-BPO 2.5\% provides statistically significant greater efficacy than individual active ingredients and vehicle with a highly favorable safety and tolerability profile. [\hyperlink{Clindamycin Phosphate}{PMID: 18805603}, Diane Thiboutot et al., 2008]

\hypertarget{pmid_1494233}{C}efprozil (CFPZ, BMY-28100) was evaluated for its efficacy, safety and pharmacokinetics in children. CFPZ was effective against streptococcal pharyngitis, pneumococcal lower respiratory tract infections, staphylococcal skin infections and Escherichia coli urinary tract infections, but was less effective against lower respiratory tract infections and otitis media due to Haemophilus influenzae. No adverse reactions were encountered in 46 cases treated with CFPZ. With a premeal administration of 7.5 mg/kg, the Cmax was approximately 3.2 micrograms/ml and the T 1/2 beta was 1.4 hours. From the present study, CFPZ appears to be safe and effective against community-acquired childhood infections. [\hyperlink{Clindamycin Phosphate}{PMID: 1494233}, H Meguro et al., 1992]

\hypertarget{pmid_9797245}{C}lindamycin, which is usually used in combination with pyrimethamine, has been proven effective in the treatment of cerebral toxoplasmosis in human immunodeficiency virus-infected patients. However, it is not known if clindamycin achieves inhibitory concentrations at the site of infection. Also, it has been hypothesized that the activity of clindamycin against Toxoplasma gondii may be due, at least in part, to a metabolite. We evaluated the penetration of clindamycin and its major metabolite, N-demethylclindamycin (NDC), into cerebrospinal fluid (CSF) of AIDS patients undergoing lumbar puncture for diagnostic purposes. A single, 1,200-mg dose of clindamycin was administered as a 45-min intravenous infusion beginning at 1.5 or 2.5 h before CSF sampling. The concentrations of clindamycin in CSF ranged from 0.091 to 0.429 mg/liter at 1.5 h and from 0.120 to 0.283 mg/liter at 2.5 h following the beginning of the infusion. The concentrations of clindamycin in CSF were well above the 50\% inhibitory concentration of 0.001 mg/liter and the parasiticidal concentration of 0.006 mg/liter. NDC was undetectable both in plasma and in CSF. Our study provides a pharmacokinetic rationale for the clinical efficacy of clindamycin in the treatment of cerebral toxoplasmosis. [\hyperlink{Clindamycin Phosphate}{PMID: 9797245}, G Gatti et al., 1998]

\hypertarget{pmid_26322295}{T}he aim of the present study was to evaluate a non-destructive fabrication method in for the development of sustained-release poly (L, D-lactic acid)-based biodegradable clindamycin phosphate implants for the treatment of ocular toxoplasmosis. The rod-shaped intravitreal implants with an average length of 5 mm and a diameter of 0.4 mm were evaluated for their physicochemical parameters. Scanning electron microscopy (SEM), differential scanning calorimetry (DSC), Fourier-transform infrared (FTIR), and nuclear magnetic resonance (1H NMR) studies were employed in order to study the characteristics of these formulations. Drug content uniformity test confirmed the uniformity in different implant batches. Furthermore, the DSC, FTIR, and 1H NMR studies proved that the fabrication process did not have any destructive effects either on the drug or on the polymer structures. These studies showed that the developed sustained-release implants could be of interest for long-term sustained intraocular delivery of clindamycin, which can provide better patient compliance and also have good potential in terms of industrial feasibility. [\hyperlink{Clindamycin Phosphate}{PMID: 26322295}, Lana Tamaddon et al., 2015]

\hypertarget{pmid_15696985}{C}lindamycin phosphate is the most widely used topical antibacterial agent for acne treatment. Treatment of patients with mild to moderate acne vulgaris with a new foam formulation (clindamycin foam, CF) for 12 weeks was at least as effective as clindamycin gel (CG) based on the Investigator's Static Global Assessment (ISGA) score. CF was superior to CG based on the reduction from baseline in total (P = .0014), inflammatory (P = .0478), and noninflammatory (P = .0037) acne lesion counts. Additionally, CF achieved efficacy that was superior to that of vehicle foam based on ISGA score (P = .0025) and all 3 lesion counts (all P < .05). Adverse experiences in the active treatment groups were mild or moderate and transient in nature. Thus the foam formulation of clindamycin is a safe and effective acne treatment; the unique foam delivery vehicle may offer cosmetic benefits to the patient and thus increase compliance. [\hyperlink{Clindamycin Phosphate}{PMID: 15696985}, Alan R Shalita et al., ]

\section*{Tazarotene}
\subsection*{Result}
\subsubsection*{Answer}

Ages <9: Unknown  
Ages 9–12: Yes  
Ages 13–17: Yes  

\subsubsection*{{Explanation}}
\hypertarget{Tazarotene}
To determine if tazarotene is safe for use in children, I reviewed the provided abstracts for evidence of targeted safety studies in pediatric populations. The following abstracts are relevant:

1. **[\hyperlink{pmid_34076400}{PMID: 34076400}, Lawrence J Green et al., 2021]**  
   This abstract describes two phase 3 trials and pooled post hoc analyses of tazarotene 0.045\% lotion for moderate-to-severe acne. Patients aged ≥9 years were included, with specific age groups analyzed (13–19, 20–29, ≥30). The results state: "No notable age-related patterns were found for safety outcomes, though tazarotene-treated males of all age groups reported fewer adverse events than females. Tazarotene 0.045\% lotion is efficacious and well tolerated in female and male adolescents and adults with moderate-to-severe acne." This provides direct evidence of a targeted safety study in children aged 9–19 years, affirming safety in this age group.

2. **[\hyperlink{pmid_34459694}{PMID: 34459694}, Leon H Kircik et al., 2022]**  
   This abstract summarizes phase-1, -2, and -3 studies of tazarotene 0.045\% lotion, including participants aged ≥9 years (≥12 years in phase-2). It states: "The results from these five studies show that the tolerability, safety, and patient satisfaction of topical tazarotene 0.045\% lotion, combined with its efficacy, make it an important option for the treatment of acne." This supports safety in children aged 9 years and older.

3. **[\hyperlink{pmid_23652892}{PMID: 23652892}, Steven R Feldman et al., 2013]**  
   This abstract reports on two randomized, double-blind, vehicle-controlled studies of tazarotene foam, 0.1\% in adults and adolescents with acne vulgaris. Participants were aged 12 to 45 years. The study found tazarotene foam to have a "safe and acceptable tolerability profile." This provides evidence for safety in adolescents aged 12–17 years.

4. **[\hyperlink{pmid_21473199}{PMID: 21473199}, A S M Zakaria et al., 2010]**  
   This study enrolled 67 patients with facial acne in the age range of 13–30 years, treated with topical tazarotene 0.1\% cream. The abstract concludes: "Topical tazarotene cream (0.1\%) is a effective and safe treatment option for acne vulgaris affecting face." This supports safety in adolescents aged 13–17 years.

5. **[\hyperlink{pmid_10388959}{PMID: 10388959}, A R Shalita et al., 1999]**  
   This study included 375 patients aged 14 to 44 years with facial acne vulgaris. It concludes: "Tazarotene gel was an effective, safe, and generally well-tolerated therapy for the treatment of acne vulgaris." This supports safety in adolescents aged 14–17 years.

6. **[\hyperlink{pmid_23545935}{PMID: 23545935}, James Q Del Rosso et al., 2013]**  
   This review mentions "multiple studies are available supporting the effective and safe use of topical tazarotene for each of its indications," but does not specify pediatric age groups in the abstract.

7. **[\hyperlink{pmid_17642573}{PMID: 17642573}, D G Saple et al.]**  
   This study evaluated tazarotene gel (0.1\%) in Indian patients with acne vulgaris, but the abstract does not specify the age range of participants.

**Summary by Age Range:**
- **Ages 9–12:** [\hyperlink{pmid_34076400}{PMID: 34076400}, 34459694] provide evidence of safety for tazarotene 0.045\% lotion in children as young as 9 years old.
- **Ages 13–17:** Multiple studies ([\hyperlink{pmid_34076400}{PMID: 34076400}, 34459694, 23652892, 21473199, 10388959]) affirm safety for tazarotene (various formulations) in this age group.
- **Ages <9:** No abstracts provide targeted safety data for children under 9 years old.

**Other Formulations and Indications:**  
Most pediatric data are for acne, not psoriasis or other indications. The evidence is strongest for tazarotene 0.045\% lotion, with some support for foam and cream formulations.

**Conclusion:**  
Tazarotene is affirmed as safe for use in children aged 9 years and older for acne, based on targeted safety studies. Safety is unknown for children under 9 years old, and for indications other than acne.

\subsection*{Abstracts}
\hypertarget{pmid_9270553}{O}ral retinoids are effective in the treatment of psoriasis, but their use is limited by concerns for teratogenic potential and systemic side effects. Tazarotene is a novel acetylenic retinoid undergoing clinical trials for the topical treatment of mild-to-moderate plaque psoriasis. The safety and tolerability of tazarotene 0.1\% and 0.05\% gels were examined in a series of preclinical and clinical trials. In preclinical studies topically applied tazarotene gel was nonmutagenic, noncarcinogenic, and nonteratogenic. Tazarotene gel was not sensitizing, phototoxic, or photosensitizing in a series of studies in human volunteers. Treatment-related systemic adverse effects were not observed in clinical trials involving approximately 2000 patients treated with tazarotene 0.1\% or 0.05\% gel for periods of up to 1 year. Adverse effects appear limited to manageable, mainly mild-to-moderate local skin irritation. [\hyperlink{Tazarotene}{PMID: 9270553}, R Marks et al., 1997]

\hypertarget{pmid_21473199}{T}azarotene is a new 3rd generation topical acetylenic retinoid. It normalizes keratinocyte differentiation, reduces keratinocyte proliferation and decreases expression of inflammatory markers. Tazarotene was approved by U.S.F.D.A. in 1997 for acne vulgaris. To evaluate the efficacy and safety of topical tazarotene 0.1\% cream in the treatment of facial acne. 67 patients with facial acne in the age range of 13-30 years were enrolled in the study. Purposive sampling was done. Patients were treated with topical Tazarotene cream (0.1\%) and were instructed to apply the medication as a thin film over the affected area in the evening once daily for 12 weeks. Follow-up was done at 2nd, 4th week, 8th week and at 12th week. Of the 67 patients, 53\% got remission, 9\% had good response, 34\% had poor response and there was no response in 4\% of the patients by 12 weeks of treatment. Among the patients, 9 (13.43\%) developed mild side effects. Topical tazarotene cream (0.1\%) is a effective and safe treatment option for acne vulgaris affecting face. It is mostly effective in grade-1 and grade-2 acne. [\hyperlink{Tazarotene}{PMID: 21473199}, A S M Zakaria et al., 2010]

\hypertarget{pmid_9585866}{T}azarotene is a topical retinoid that appears to exert its effects via retinoic acid receptors. It normalises differentiation and proliferation of keratinocytes and has an anti-inflammatory effect. Topical tazarotene 0.05\% or 0.1\% gel was effective in the treatment of plaque psoriasis in clinical trials and its therapeutic effect was maintained for at least 12 weeks after treatment discontinuation in some patients. In one study in patients with psoriasis, tazarotene had similar efficacy to fluocinonide in reducing plaque elevation, but not erythema. In another study, tazarotene was reported to be less effective than fluocinonide. Combination treatment with tazarotene plus a mid- or high-potency corticosteroid was more effective in the treatment of psoriasis than tazarotene alone. Topical tazarotene 0.1\% gel significantly reduced lesion counts in patients with mild to moderate facial acne vulgaris. Skin irritation is a common adverse event with topical tazarotene, but it is mainly of mild to moderate severity. Tazarotene is not recommended for use in women who are, or may become, pregnant. [\hyperlink{Tazarotene}{PMID: 9585866}, R H Foster et al., 1998]

\hypertarget{pmid_23545935}{T}azarotene is a synthetic retinoid that, depending on the concentration and vehicle, is approved by the US Food and Drug Administration for the topical treatment of acne vulgaris (AV) and plaque psoriasis. Tazarotene is also used as adjunctive treatment for specified clinical manifestations of chronically photodamaged skin (facial fine wrinkling, mottled facial hypopigmentation and hyperpigmentation, and benign facial lentigines), along with comprehensive skin care and photoprotection from sunlight. The gel formulation was released in the United States in 1997, with the cream formulation made available in 2000. Multiple studies are available supporting the effective and safe use of topical tazarotene for each of its indications. This article provides an overview of the pharmacology of topically applied tazarotene, discussing in particular up-to-date information on the efficacy, tolerability, and safety of topical tazarotene for AV, including monotherapy and combination therapy studies. Topical tazarotene 0.1\% in both formulations is highly effective in reducing both inflammatory and noninflammatory acne lesions, and can be used in combination with other topical agents, including formulations containing benzoyl peroxide or dapsone 5\% gel. Although many patients tolerate the use of topical tazarotene without significant issues or concerns, some patients experience application-site tolerability reactions, which can usually be managed with proper skin care and are less frequent with the cream formulation. [\hyperlink{Tazarotene}{PMID: 23545935}, James Q Del Rosso et al., 2013]

\hypertarget{pmid_34076400}{T}wo identical phase 3 trials (NCT03168321 and NCT03168334) and pooled post hoc analyses have established efficacy and safety of a polymeric tazarotene 0.045\% lotion formulation in patients with moderate-to-severe acne. Presented here are post hoc analyses that further examine efficacy and safety of tazarotene 0.045\% lotion by age and sex. Patients aged ≥ 9 years with moderate-to-severe acne (score 3 or 4 on the Evaluator's Global Severity Score [EGSS]) were equally randomized to once-daily tazarotene 0.045\% lotion or vehicle lotion for 12 weeks. Efficacy outcomes included inflammatory/noninflammatory lesion counts and treatment success (proportion of participants achieving ≥ 2-grade reduction from baseline in EGSS and score of 0 [clear] or 1 [almost clear]). Adolescent and adult females (n=1,013) and males (n=529) were subdivided into 3 age groups: 13–19, 20–29, and ≥30 years. At week 12, large least-squares mean percent reductions in inflammatory and noninflammatory lesions were observed across all 3 tazarotene-treated age groups in males and females (range, -50.2\% to -64.8\%). Treatment success rates ranged from 23.6\% to 38.4\%. Across all efficacy assessments, significant differences between tazarotene and vehicle (P<0.05) were generally observed in the younger male and female participants (13–19 and 20–29). No notable age-related patterns were found for safety outcomes, though tazarotene-treated males of all age groups reported fewer adverse events than females. Tazarotene 0.045\% lotion is efficacious and well tolerated in female and male adolescents and adults with moderate-to-severe acne. J Drugs Dermatol. 2021;20(6):608-615. doi:10.36849/JDD.6070. [\hyperlink{Tazarotene}{PMID: 34076400}, Lawrence J Green et al., 2021]

\hypertarget{pmid_12942109}{T}azarotene is an acetylenic retinoid which is metabolised to tazarotenic acid and which binds selectively to the retinoid receptors RARbeta and RARgamma. The safety, toxicity and pharmacokinetics of oral tazarotene were determined over 12 weeks of treatment in 34 patients with advanced cancer. Commonly seen toxicities were mucocutaneous symptoms, musculoskeletal pain and headache. Dose-limiting toxicities were hypercalcaemia, hypertriglyceridaemia and musculoskeletal pain. The maximum tolerated dose of tazarotene in this schedule is 25.2 mg day(-1). Plasma concentrations of tazarotenic acid were found to peak rapidly within 1-3 h of dosing and thereafter declined quickly. The C(max) and AUC values on day 0, and weeks 2 and 4 were similar indicating no drug accumulation. The dose-normalised C(max) and AUC values at different dose levels and different study days appeared to be similar indicating linear pharmacokinetics. No objective responses were seen, although stable disease was seen in six out of eight evaluable patients receiving the three highest dose levels of tazarotene (16.8, 25.2 or 33.4 mg day(-1)). We conclude that oral tazarotene is well tolerated when administered daily for 12 weeks, has a favourable toxicity profile compared with other retinoids and merits further investigation as an anticancer therapy. [\hyperlink{Tazarotene}{PMID: 12942109}, P H Jones et al., 2003]

\hypertarget{pmid_23652892}{T}azarotene 0.1\% gel and cream are effective topical treatments for acne. Tazarotene foam, 0.1\% was developed to provide an alternative, safe, and effective formulation. To evaluate efficacy and tolerability of tazarotene foam, 0.1\% in adults and adolescents with acne vulgaris. Two randomized, double-blind, vehicle-controlled, parallel-group studies were conducted at 39 centers in the United States and Canada. The first study involved 744 participants and the second 742, aged 12 to 45 years, who were randomized to receive treatment with either tazarotene foam, 0.1\% or vehicle foam once daily for 12 weeks. Lesion counts, Investigator's Static Global Assessments (ISGA), and Subject's Global Assessments (SGA) were evaluated at baseline and weeks 2, 4, 8, and 12. Tolerability was monitored throughout the study. At week 12 in both studies, treatment with tazarotene foam led to greater decreases from baseline in mean absolute and percentage change in lesion counts (noninflammatory, inflammatory, and total), greater proportion of participants with ≥2-grade improvement in ISGA score, and greater proportion of participants with ISGA score of 0 or 1 than vehicle treatment (P<.001 for all). Only application-site skin irritation and dryness were reported by >5\% of participants in active treatment groups in both studies. The efficacy and tolerability of tazarotene foam were not compared directly with those of other formulations. Tazarotene foam, 0.1\% significantly reduced the number and severity of acne lesions after 12 weeks and had a safe and acceptable tolerability profile. [\hyperlink{Tazarotene}{PMID: 23652892}, Steven R Feldman et al., 2013]

\hypertarget{pmid_10827404}{T}azarotene is the first receptor-selective retinoid indicated for the topical treatment of plaque psoriasis. It is being used clinically in combination with other topical antipsoriatic treatments, although its stability in the presence of these products has not been examined extensively. This study examines the compatibility of tazarotene 0.05\% gel with 17 other topical products used in the treatment of psoriasis, assessed over a 2-week period. Tazarotene showed minimal degradation (<10\%) at 0, 8, 24, and 48 hours after compounding with each of the 17 products. In addition, after 1 and 2 weeks, degradation of tazarotene remained less than 10\% for 15 of the 17 products tested. Tazarotene appeared to have minimal impact on the stability of the other products. These results suggest that tazarotene gel can be successfully coprescribed with a range of commonly used topical psoriasis treatments without adversely affecting the chemical stability of either agent. [\hyperlink{Tazarotene}{PMID: 10827404}, D Hecker et al., 2000]

\hypertarget{pmid_9270554}{T}azarotene is the first of a new generation of acetylenic retinoids developed for the topical treatment of mild-to-moderate plaque psoriasis. Controlled clinical trials have demonstrated that once-daily tazarotene 0.05\% and 0.1\% gels are effective in improving and reducing clinical signs and symptoms of psoriasis on trunk and limb lesions and difficult-to-treat elbow and knee plaques. Tazarotene has a rapid onset of action indicated by significant improvements as early as the first week of treatment. Sustained beneficial effects have been observed in some patients for up to 12 weeks after the cessation of therapy. Compared with twice-daily fluocinonide 0.05\% cream, once-daily tazarotene 0.05\%, and 0.1\% gels were similarly effective in reducing plaque elevation. Once-daily tazarotene 0.05\% and 0.1\% gels demonstrated a more prolonged therapeutic effect after discontinuation than twice-daily fluocinonide cream. Tazarotene is generally well tolerated, with adverse events limited to local irritation. Tazarotene appears to be an effective addition to the currently available treatments for plaque psoriasis. [\hyperlink{Tazarotene}{PMID: 9270554}, G D Weinstein et al., 1997]

\hypertarget{pmid_9787989}{T}azarotene is the first topical retinoid demonstrated to be both effective and tolerable in the treatment of psoriasis. The clinical efficacy and safety of topical tazarotene have been investigated in vehicle-controlled and active-controlled studies. Ongoing studies are evaluating its use in combination with other antipsoriasis medications. Tazarotene has been demonstrated to be significantly more effective than vehicle, and comparable in efficacy to fluocinonide 0.05\%, but with a more sustained therapeutic effect after treatment is stopped. Its adverse effects consist primarily of mild to moderate local irritation with no reports of treatment-related systemic adverse effects. Combining tazarotene with a mid- to high-potency corticosteroid gives greater efficacy with fewer adverse effects than either agent used alone. The use of tazarotene in combination with phototherapy or calcipotriene is currently being investigated. Overall, topical tazarotene is a highly useful addition to the array of options available for the topical treatment of mild to moderate plaque psoriasis. [\hyperlink{Tazarotene}{PMID: 9787989}, M Lebwohl et al., 1998]

\hypertarget{pmid_31037297}{T}azarotene is internationally accepted common name for ethyl 6-[(4,4-dimethylthiochroman-6-yl)ethynyl]nicotinate. It is a synthetic retinoid used for the topical treatment of mild to moderate plaque psoriasis, acne vulgaris and photo aging. To ensure the quality of drug product and drug substance, a LC-MS compatible UHPLC method was developed for quantification of drug and its related substances. Stationary phase with fused core particle technology is used for the separation of impurities. Limit of quantification and limit of detection of the method are 0.1 and 0.03\%, respectively. Precision of the method for Tazarotene and all its related substances is less than 2.2\% RSD. The correlation coefficient is >0.999. Accuracy of method is ranged from 95.3\% to 107.0\%. Application of this method in stability analysis has been demonstrated by analyzing stressed samples. Experimental design is used for the verification of robustness of the method. To ensure the safety, an in silico toxicity of the drug and its related substances were determined using TOPKAT and DEREK toxicity predictions Both UHPLC and in silico methods were validated as per the ICH Q2 and ICH M7 guidelines, which will enable a rapid product development of Tazarotene topical formulations while ensuring the safety and quality of product. [\hyperlink{Tazarotene}{PMID: 31037297}, Nvvss Narayana Murty Nagulakonda et al., 2019]

\hypertarget{pmid_17642573}{T}azarotene is a new third generation topical acetylenic retinoid. The present study was conducted to evaluate the efficacy and safety of tazarotene gel (0.1\%) in Indian patients of acne vulgaris. The present study was a prospective, open, multicentric, phase III trial. The duration of study was 14 weeks, including a 12-week active treatment period, preceded by a 2-week washout phase. Patients applied 0.1\% tazarotene gel as a thin film over the affected area once daily in the evening. The efficacy was evaluated by analyzing changes in the number of facial acne lesions and patient's and physicians' global assessment. The efficacy parameters were assessed at baseline, visits 2, 4, 8, and 12 weeks. Tolerability and safety was assessed by physical examination, laboratory parameters and evaluation of adverse events. A total of 126 patients in 6 centers completed the study. At the end of the 8th and 12th weeks, the mean number of inflammatory lesions reduced by 70.6\% and 86.1\%, non-inflammatory lesions by 81.5\% and 92\%, and total lesion count 75.6\% and 88.8\% respectively from baseline. Also, 90.7\% and 93.6\% of total study cases showed complete to moderate clearance of acne lesions according to physicians at the end of the 8th and 12th weeks. This study confirms the efficacy and safety of tazarotene gel (0.1\%) in Indian patients of acne vulgaris. [\hyperlink{Tazarotene}{PMID: 17642573}, D G Saple et al., ]

\hypertarget{pmid_9777791}{T}he safety profile of tazarotene is superior to that of orally administered retinoids. The limited percutaneous penetration of tazarotene limits its systemic absorption and this, combined with its rapid metabolism in the skin and the plasma to the more water-soluble active metabolite, tazarotenic acid, helps prevent accumulation of the drug in fat containing tissues. Urinary and fecal elimination are virtually complete within 2 to 3 days and 7 days after dosing, respectively. Tazarotene also exhibits no indication of mutagenicity, carcinogenicity, phototoxic potential, photoallergic potential, or contact sensitization. [\hyperlink{Tazarotene}{PMID: 9777791}, R Marks et al., 1998]

\hypertarget{pmid_9449910}{T}o determine the safety and efficacy of topically applied tazarotene gel in the treatment of mild to moderate psoriatic plaques. Two multicenter, double-blind, randomized studies of 6- and 8-week duration, with an 8-week follow-up in the second study. Medical center outpatient dermatology services. One hundred fifty-three adults with 2 bilateral target plaques on the trunk, legs, or arms. Vehicle gel or 0.01\% and 0.05\% tazarotene gel administered twice daily to 45 patients (study A), or 0.05\% and 0.1\% tazarotene gel administered either once or twice daily to 108 patients (study B). Treatment success and plaque elevation, scaling, and erythema vs time. The 0.01\% tazarotene gel showed minimal efficacy. Applications of 0.05\% and 0.1\% tazarotene gels administered once or twice daily, resulted in significant improvements in plaque elevation, scaling, erythema, and overall clinical severity as early as 1 week. Treatment success rates (defined as > 75\% improvement from baseline) were 45\% with 0.05\% tazarotene gel vs 13\% with vehicle gel after 6 weeks of treatment (P < .05; study A) and ranged from 48\% to 63\% with the various tazarotene treatment regimens after 8 weeks of treatment (study B). These improvements were evident at the 8-week follow-up. Treatment-related adverse effects were generally limited to mild or moderate local irritation and were less frequent with the treatment regimen administered once daily. The 0.05\% and 0.1\% tazarotene gels demonstrated significant efficacy in the treatment of mild to moderate psoriatic plaques that persisted after cessation of treatment. [\hyperlink{Tazarotene}{PMID: 9449910}, G G Krueger et al., 1998]

\hypertarget{pmid_10388959}{R}etinoids reverse the abnormal pattern of keratinization seen in acne vulgaris. Tazarotene is the first of a novel family of topical receptor-selective acetylenic retinoids. This study evaluates the safety and efficacy of topical tazarotene 0.1\% and 0.05\% gels, in comparison to vehicle gel, applied once daily for 12 weeks, in the treatment of mild-to-moderate facial acne vulgaris. A total of 446 patients with facial acne vulgaris were enrolled, and 375 patients, ranging in age from 14 to 44 years, were evaluable in this multicenter, double-blind, randomized study. In comparison to vehicle gel, treatment with tazarotene 0.1\% gel resulted in significantly greater reductions in noninflammatory and total lesion counts at all follow-up visits, and inflammatory lesion counts at Week 12. Tazarotene 0.05\% gel resulted in significantly greater reductions in noninflammatory and total lesion counts than vehicle gel at Weeks 8 and 12. At Week 12, treatment success rates were 68\% and 51\% for tazarotene 0.1\% and 0.05\%, respectively (40\% for vehicle gel). Tazarotene gel was an effective, safe, and generally well-tolerated therapy for the treatment of acne vulgaris. [\hyperlink{Tazarotene}{PMID: 10388959}, A R Shalita et al., 1999]

\hypertarget{pmid_9787988}{T}he majority of patients with psoriasis exhibit a mild to moderate form of the disease. While topically applied agents represent the ideal therapeutic approach for these patients, our treatment options have been hampered by their inconsistent or limited efficacy, expense, and cosmetic properties. Furthermore, some of the topically applied preparations have the potential for systemic absorption and associated adverse events. Tazarotene is the first synthetically developed retinoid indicated for the topical treatment of patients with psoriasis. Pharmacologically, tazarotene affects the three primary abnormalities associated with psoriasis: it normalizes epidermal differentiation, it exhibits a potent antiproliferative effect, and it decreases epidermal inflammation. Following topical application, tazarotene is rapidly metabolized in the skin to tazarotenic acid, its primary and active metabolite. While tazarotenic acid achieves measurable levels in the systemic circulation, no drug-related hematologic, ophthalmologic, or metabolic adverse events have been observed. Only a minimal amount of tazarotene is absorbed into the circulation. While animal studies have found tazarotene to be nonmutagenic and nonteratogenic, women of child-bearing potential should be counseled regarding the potential risks of retinoid use during pregnancy. In summary, animal and clinical patient trials have found tazarotene to be effective, safe, and well-tolerated. The availability of such an agent in a topical formulation is a welcome treatment alternative for the management of patients with psoriasis. [\hyperlink{Tazarotene}{PMID: 9787988}, M Duvic et al., 1998]

\hypertarget{pmid_33133344}{A} concern with the increasing use of prescription drugs during pregnancy is teratogenic risk. This risk is undetermined for most drugs approved in the United States (US) from 2000 to 2010. Acne and psoriasis are chronic diseases that typically occur during the child-bearing years, and as topical retinoids are recommended for both acne and psoriasis treatment, is it possible for women to be exposed to a topical retinoid during pregnancy. Pharmacokinetic studies show relatively low systemic exposure from topical retinoids, but the exposure levels that could lead to teratogenicity in humans are unknown. Tazarotene, a topical retinoid, was US Food and Drug Administration (FDA) approved for both acne and psoriasis using pharmacokinetic data from psoriasis studies, which estimated the data based on use of tazarotene on up to 20\% body surface area. As such, under both the previous and current FDA pregnancy labeling, tazarotene is not recommended for use during pregnancy. The goal of this literature review was to provide historical context for the pregnancy labeling rule for tazarotene compared with other approved retinoids and gather available data on tazarotene- and retinoid-related pregnancy outcomes. While there are case reports of topical tretinoin and adapalene exposure  [\hyperlink{Tazarotene}{PMID: 33133344}, George Han et al., 2020] Oral tazarotene, an acetylenic retinoid, is in clinical development for the treatment of psoriasis. The disposition and biotransformation of tazarotene were investigated in six healthy male volunteers, following a single oral administration of a 6 mg (100 microCi) dose of [14C]tazarotene, in a gelatin capsule. Blood levels of radioactivity peaked 2 h postdose and then rapidly declined. Total recovery of radioactivity was 89.2+/-8.0\% of the administered dose, with 26.1+/-4.2\% in urine and 63.0+/-7.0\% in feces, within 7 days of dosing. Only tazarotenic acid, the principle active metabolite formed via esterase hydrolysis of tazarotene, was detected in blood. One major urinary oxidative metabolite, tazarotenic acid sulfoxide, accounted for 19.2+/-3.0\% of the dose. The majority of radioactivity recovered in the feces was attributed to tazarotenic acid representing 46.9+/-9.9\% of the dose and only 5.82+/-3.84\% of dose was excreted as unchanged tazarotene. Thus following oral administration, tazarotene was rapidly absorbed and underwent extensive hydrolysis to tazarotenic acid, the major circulating species in the blood that was then excreted unchanged in feces. A smaller fraction of tazarotenic acid was further metabolized to an inactive sulfoxide that was excreted in the urine. [\hyperlink{Tazarotene}{PMID: 33133344}, Mayssa Attar et al., 2005]

\hypertarget{pmid_9591815}{A} new class of topical receptor-selective acetylenic retinoids, the first of which is tazarotene, has been developed. Our purpose was to compare the safety, efficacy, and duration of therapeutic effect of 12 weeks of once-daily tazarotene 0.1\% and 0.05\% gel with that of twice-daily fluocinonide 0.05\% cream in the treatment of patients with plaque psoriasis. Three hundred forty-eight patients with plaque psoriasis were enrolled and 275 patients completed a multicenter, investigator-masked, randomized, parallel-group clinical trial. Both tazarotene gels were as effective as fluocinonide in reducing plaque elevation after 1 week of treatment, and tazarotene 0.1\% gel was similar to fluocinonide in reducing scaling of trunk/limb lesions at all study weeks except week 4. Tazarotene 0. 1\% gel was similar to fluocinonide in reducing scaling of knee/elbow lesions at weeks 8 and 12. Fluocinonide had a significantly greater effect on erythema than tazarotene at weeks 2 through 8. However, treatments were not significantly different at week 12, and tazarotene demonstrated significantly better maintenance of therapeutic effect after cessation of therapy. Tazarotene 0.1\% and 0.05\% gels were safe and effective in the treatment of mild-to-moderate plaque psoriasis. [\hyperlink{Tazarotene}{PMID: 9591815}, M Lebwohl et al., 1998]

\hypertarget{pmid_34459694}{T}opical retinoids are recommended for acne treatment, but their use may be limited by irritation or dermatitis. Herein is an overview of the dermal sensitization, safety, tolerability, and participant satisfaction data from phase-1, -2, and -3 studies of lower-dose tazarotene 0.045\% polymeric emulsion lotion. Two phase-1, single-blind, vehicle-controlled dermal safety studies were conducted in healthy participants aged ≥18 years. One phase-2 (NCT02938494) and two phase-3 studies (NCT03168334; NCT03168321) were double-blind, randomized, and vehicle-controlled over 12 weeks in participants aged ≥9 years (≥12 years, phase-2) with moderate-to-severe acne. A total of 2029 participants (tazarotene 0.045\% lotion or vehicle) were included across the 5 studies (safety populations:  The results from these five studies show that the tolerability, safety, and patient satisfaction of topical tazarotene 0.045\% lotion, combined with its efficacy, make it an important option for the treatment of acne. [\hyperlink{Tazarotene}{PMID: 34459694}, Leon H Kircik et al., 2022]

\hypertarget{pmid_11004622}{T}azarotene, a potent acetylenic retinoid for topical use, might be expected to benefit photodamaged skin, including improving the classical signs of fine wrinkles, mottled hyperpigmentation, and roughness. Our purpose was to determine the efficacy and safety of tazarotene 0.1\% gel in the treatment of photodamaged dorsal forearm skin. Ten healthy female volunteers, aged 45 to 65 years, with moderately photodamaged forearm skin applied tazarotene 0.1\% gel to one arm and vehicle gel to the other once daily for 12 weeks. The study was a double-blind, randomized, paired-comparison evaluation conducted at a single site. Tazarotene showed beneficial effects for several efficacy variables. It was more efficacious than vehicle in reducing skin roughness and fine wrinkling based on objective measurements. Tazarotene also corrected epidermal atrophy and atypia and improved skin hydration properties. In this 12-week pilot study tazarotene redressed abnormalities associated with photo-damaged skin. [\hyperlink{Tazarotene}{PMID: 11004622}, J Sefton et al., 2000]

\hypertarget{pmid_23456673}{T}azarotene, a retinoid pro-drug, is available in gel, cream and foam for the topical treatment of acne vulgaris. This single-centre, randomized, open-label study assessed relative bioavailability of its active metabolite tazarotenic acid after dosing of tazarotene foam or gel. Subjects with moderate-to-severe acne received a mean, once-daily dose of 3.7 g tazarotene foam or gel applied to face, chest, upper back and shoulders. Blood samples were collected pre-dose on multiple days and multiple time points over a 72-h period to measure plasma tazarotenic acid and tazarotene. Mean tazarotenic acid area under the plasma concentration-time curve (AUC) and maximum measured plasma concentration (Cmax) values were significantly higher for gel versus foam. Cmax occurred within 5-6 h after dosing, with an apparent terminal elimination half-life (t½) of 18-22 h. Accumulation was observed upon repeated dosing with steady-state conditions achieved at day 20. Mean tazarotene concentrations were also higher following gel application versus foam. Both foam and gel demonstrated an acceptable safety profile. Tazarotene foam, 0.1 \% is an alternative to gel with less systemic exposure. [\hyperlink{Tazarotene}{PMID: 23456673}, Michael Jarratt et al., 2013]

\hypertarget{pmid_30124724}{T}opical corticosteroids (TCS) are the mainstay of psoriasis treatment. Safety concerns may limit use. Combination with tazarotene may optimize efficacy and minimize safety and tolerability concerns. Investigate safety and efficacy of halobetasol propionate 0.01\%/tazarotene 0.045\% (HP/TAZ) lotion in moderate-to-severe plaque psoriasis. Two multicenter, randomized, double-blind, vehicle-controlled phase 3 studies (N=418). Subjects randomized (2:1) to HP/TAZ lotion or vehicle once-daily for 8 weeks, 4-week follow-up. Primary efficacy assessment: treatment success (at least a 2-grade improvement from baseline in IGA score and 'clear' or 'almost clear'). Safety and treatment emergent AEs evaluated throughout. HP/TAZ lotion demonstrated statistically significant superiority over vehicle as early as week 2 (P equals 0.002). By week 8, 40.6\% of subjects were treatment successes compared with 9.9\% on vehicle (P less than 0.001). A third of subjects remained treatment successes post-treatment. HP/TAZ lotion was also superior in reducing psoriasis signs and symptoms, and Body Surface Area (BSA) involvement. Most frequently reported treatment related AEs were contact dermatitis (6.3\%), application site pain (2.6\%), and pruritus (2.2\%). No data were collected beyond the 4-week follow-up. HP/TAZ lotion provides synergistic efficacy that is both rapid and sustained, with good tolerability and safety over 8 weeks use. J Drugs Dermatol. 2018;17(8):855-861. [\hyperlink{Tazarotene}{PMID: 30124724}, Jeffrey L Sugarman et al., 2018]

\hypertarget{pmid_23696711}{T}azarotene foam is the first topical retinoid foam approved for the treatment of acne vulgaris. To review the safety and efficacy studies of tazarotene foam in the treatment of moderate to severe acne. Five Phase I safety studies in normal controls are reviewed and two Phase III safety and efficacy studies in patients with moderate to severe acne are reviewed. Tazarotene foam, 0.1\% was effective in the treatment of moderate to severe acne vulgaris with an acceptable tolerability profile. This treatment provides a cosmetically favorable alternative for topical retinoid therapy. [\hyperlink{Tazarotene}{PMID: 23696711}, Erica L Epstein et al., 2013]

\hypertarget{pmid_12734506}{T}azarotene in a gel formulation is widely used in the treatment of psoriasis. To determine the efficacy and safety of tazarotene 0.1\% and 0.05\% creams in the treatment of psoriasis. A total of 1303 patients participated in 2 clinical trials. Patients applied tazarotene creams 0.1\% and 0.05\% or vehicle once daily to all psoriatic lesions for 12 weeks followed by a 12-week posttreatment period. Both creams were significantly more effective than vehicle on the basis of an overall assessment of psoriasis, a global response to treatment, and reduction in plaque elevation and scaling. Therapeutic effect was maintained during the posttreatment period. Common adverse events included signs and symptoms of skin irritation. Tazarotene creams were associated with significant reductions in the severity of the clinical signs of psoriasis and were found to be safe with acceptable tolerability. Tazarotene cream 0.1\% was generally more effective, although slightly less well tolerated, than the 0.05\% cream. [\hyperlink{Tazarotene}{PMID: 12734506}, Gerald D Weinstein et al., 2003]

\section*{Halobetasol Propionate}
\subsection*{Result}
\subsubsection*{Answer}

Unknown

\subsubsection*{{Explanation}}
\hypertarget{Halobetasol Propionate}
After reviewing all available abstracts, there are several studies evaluating the safety and efficacy of halobetasol propionate (HP) in the treatment of psoriasis and other dermatoses. However, none of these studies specifically mention the inclusion of children or provide age ranges that would indicate pediatric subjects were studied. The abstracts consistently refer to "subjects" or "patients" without specifying pediatric populations, and there is no mention of targeted safety studies in children or infants.

For example, the phase 3 studies of halobetasol propionate 0.01\% lotion in moderate-to-severe plaque psoriasis included 430 subjects, but the age range is not specified, and there is no indication that children were included or analyzed separately [\hyperlink{pmid_30365586}{PMID: 30365586}, Lawrence J Green et al., 2018; \hyperlink{pmid_30893392}{PMID: 30893392}, Jeffrey L Sugarman et al., 2019; \hyperlink{pmid_33683083}{PMID: 33683083}, Seemal R Desai et al., 2021; \hyperlink{pmid_32845589}{PMID: 32845589}, Fran E Cook-Bolden et al., 2020]. Similarly, studies of halobetasol propionate 0.05\% cream or ointment do not specify pediatric populations [\hyperlink{pmid_1757613}{PMID: 1757613}, H I Katz et al., 1991; \hyperlink{pmid_1757614}{PMID: 1757614}, C A Guzzo et al., 1991].

One review of a combination product (halobetasol propionate and tazarotene) notes that it is approved for adults and does not discuss pediatric use [\hyperlink{pmid_32606876}{PMID: 32606876}, Vidhatha Reddy et al., 2020]. Another study on a novel halobetasol propionate nanocarrier formulation reports in vitro and in vivo safety, but does not specify the age of subjects or mention children [\hyperlink{pmid_31170512}{PMID: 31170512}, Paulina Carvajal-Vidal et al., 2019].

In summary, based on the abstracts available, there are no targeted studies evaluating the safety of halobetasol propionate in children. Therefore, the safety of halobetasol propionate in children is unknown.

\subsection*{Abstracts}
\hypertarget{pmid_30365586}{T}opical corticosteroids (TCS) are the mainstay of psoriasis treatment; long-term safety concerns limiting consecutive use of potent TCS to 2-4 weeks. Investigate safety and efficacy of halobetasol propionate 0.01\% lotion in moderate-to-severe plaque psoriasis. Two multicenter, randomized, double-blind, vehicle-controlled phase 3 studies (N=430). Subjects randomized (2:1) to halobetasol propionate 0.01\% lotion or vehicle once-daily for 8 weeks, 4-week posttreatment follow-up. Primary efficacy assessment: treatment success (at least a 2-grade improvement from baseline in Investigator Global Assessment [IGA] score and 'clear' or 'almost clear') at week 8. Safety and treatment emergent adverse events (AEs) evaluated throughout. Halobetasol propionate 0.01\% lotion demonstrated statistically significant superiority over vehicle as early as week 2. By week 8, 36.5\% (Study 1) and 38.4\% (Study 2) of subjects were treatment successes compared with 8.1\% and 12.0\% on vehicle (P less than 0.001). Halobetasol propionate 0.01\% lotion was also superior in reducing psoriasis signs and symptoms, body surface area (BSA), and improving quality of life. Halobetasol propionate 0.01\% lotion was well-tolerated with no treatment-related AEs greater than 1\%. Study did not include subjects with BSA greater than 12. Halobetasol propionate 0.01\% lotion was associated with significant reductions in the severity of the clinical signs of psoriasis, without the safety concerns of a longer treatment course. J Drugs Dermatol. 2018;17(10):1062-1069. [\hyperlink{Halobetasol Propionate}{PMID: 30365586}, Lawrence J Green et al., 2018]

\hypertarget{pmid_6937455}{H}aloperidol is safe and effective in children for relieving psychotic symptoms associated with childhood autism, schizophrenia and mental retardation. It is the drug of choice for Tourette's syndrome, and may be useful in nonpsychotic hyperactive or aggressive children to control acute episodes, or when the stimulants normally useful in hyperactive children are ineffective. Such children taking haloperidol not only become calmer, but are often better able to respond to other modalities of therapy and to school instruction. Dosage, initially low, is increased gradually to minimize drowsiness and extrapyramidal symptoms, the most common side effects. Haloperidol in children is usually well-tolerated. [\hyperlink{Halobetasol Propionate}{PMID: 6937455}, A C Serrano et al., 1981]

\hypertarget{pmid_30893392}{P}otent topical corticosteroids (TCSs) are the mainstay of psoriasis treatment. Safety concerns have limited use to 2 to 4 weeks. The objective of our study was to investigate the safety and efficacy of once-daily halobetasol propionate (HP) lotion 0.01\% in moderate to severe plaque psoriasis through 2 multicenter, randomized, double-blind, vehicle-controlled phase 3 studies (N=430). Participants were randomized (2:1) to HP lotion 0.01\% or vehicle once daily for 8 weeks, followed by 4 weeks of follow-up. The primary efficacy assessment was treatment success (at least a 2-grade improvement in baseline investigator global assessment [IGA] score and a score of 0 [clear] or 1 [almost clear]). Additional assessments included improvement in psoriasis signs and symptoms, body surface area (BSA), and a composite score of IGA×BSA. Safety and treatment-emergent adverse events (AEs) were evaluated throughout. We found that HP lotion 0.01\% demonstrated statistically significant superiority over vehicle as early as week 2 and also was superior in reducing psoriasis signs and symptoms and BSA involvement. [\hyperlink{Halobetasol Propionate}{PMID: 30893392}, Jeffrey L Sugarman et al., 2019]

\hypertarget{pmid_33683083}{P}soriasis is a chronic, inflammatory disease that may differ in prevalence and clinical presentation among patients from various racial and ethnic groups. Two phase 3 studies demonstrated efficacy and safety of halobetasol propionate (HP) 0.01\% lotion in the treatment of moderate-to-severe plaque psoriasis (NCT02514577, NCT02515097). These post hoc analyses evaluated HP 0.01\% lotion in Hispanic participants. Participants were randomized (2:1) to receive once-daily HP or vehicle lotion for 8 weeks, with a 4-week posttreatment follow-up. Post hoc efficacy assessments in Hispanic participants (HP, n=76; vehicle, n=43) included treatment success (\&ge;2‑grade improvement in Investigator\&rsquo;s Global Assessment and score of \&lsquo;clear\&rsquo; or \&lsquo;almost clear\&rsquo;), psoriasis signs, and affected body surface area (BSA). Treatment-emergent adverse events (TEAEs) were evaluated. At week 8, 38.8\% of participants achieved treatment success with HP versus 10.3\% on vehicle (P=0.001). HP‑treated participants achieved greater improvements in psoriasis signs, compared with vehicle (P\&lt;0.01 all). HP group had a greater reduction in affected BSA versus vehicle (P=0.001). Treatment-related TEAEs with HP were application site infection and dermatitis (n=1 each). Once-daily HP 0.01\% lotion was associated with significant reductions in disease severity in Hispanic participants with moderate-to-severe psoriasis, with good tolerability and safety over 8 weeks. J Drugs Dermatol. 2021;20(3):252-258. doi:10.36849/JDD.5698. [\hyperlink{Halobetasol Propionate}{PMID: 33683083}, Seemal R Desai et al., 2021]

\hypertarget{pmid_30659785}{P}ropranolol is an effective method of treatment for infantile hemangiomas (IH). A recent concern is a shift of the therapy into outpatient settings. The aim of the study was to evaluate the safety of initiating and maintaining propranolol therapy for IH. The study involved 55 consecutive children with IH being treated with propranolol. The patients were assessed in the hospital at the initiation of the therapy and later in outpatient settings during and after the therapy. Each time, the following monitoring methods were used: physical examination, cardiac ultrasound (ECHO), electrocardiography (ECG), blood pressure (BP), heart rate (HR), and biochemical parameters: blood count, blood glucose, aspartate transaminase (AST), alanine transaminase (ALT), and ionogram. The therapeutic dose of propranolol was 2.0 mg/kg/day divided into 2 doses. Four children were excluded during the qualification or the initiation of propranolol; a total of 51 patients were subject to the final analysis. All the children presented clinical improvement. There was a significant reduction in the mean HR values only at the initiation of propranolol. There were no changes in HR during the course of the therapy. Blood pressure values were within normal limits. Both systolic and diastolic values decreased in the first 3 months. Bradycardia and hypotension were observed sporadically, and they were asymptomatic. Electrocardiography did not show significant deviations. The pathological findings of the ECHO scans were not a contraindication to continuing the therapy. There were no changes in biochemical parameters. Apart from 1 symptomatic case of hypoglycemia, other low glucose episodes were asymptomatic and sporadic. The observed adverse effects were mild and the propranolol dose had to be adjusted in only 6 cases. Propranolol is effective, safe and well-tolerated by children with IH. The positive results of the safety assessment support the strategy of initiating propranolol in outpatient settings. Future studies are needed to assess the benefits of the therapy in ambulatory conditions. [\hyperlink{Halobetasol Propionate}{PMID: 30659785}, Lidia Babiak-Choroszczak et al., 2019]

\hypertarget{pmid_1757613}{T}he efficacy and safety of halobetasol propionate 0.05\% cream, an ultra high-potency corticosteroid preparation, was evaluated in a double-blind, vehicle-controlled, paired comparison study. Patients' psoriatic lesions were evaluated before treatment and after 1 and 2 weeks of twice-daily treatment with halobetasol propionate and vehicle. Response measures (plaque elevation, erythema, scaling, and pruritus) were evaluated with a 4-point severity scale whereby the sum provided a total score. Patient self-assessment measures were obtained at the 2-week visit by categorizing his or her global responses to queries about each treatment's "effectiveness" and "overall rating." All efficacy parameters, as judged by the physician, showed statistically significant (p = 0.0001) treatment differences favoring halobetasol propionate at both week 1 and week 2 evaluations. Patient global responses for "effectiveness" and "overall rating" favored halobetasol propionate 0.05\% cream over vehicle after 2 weeks of use. No systemic adverse drug effects were reported during the study. No patient was discontinued from the study because of an adverse event, and there was no evidence of skin atrophy after 2 weeks of treatment with either agent. Patient reports of "stings" or "burns" were equally distributed between the active and vehicle treatment groups. This trial demonstrates that halobetasol propionate 0.05\% cream is clinically beneficial and without evidence of significant risk in the treatment of plaque psoriasis. [\hyperlink{Halobetasol Propionate}{PMID: 1757613}, H I Katz et al., 1991]

\hypertarget{pmid_34918994}{T}o evaluate the safety of initiating and maintaining propranolol therapy for infantile hemangioma (IH) and the safety of different doses. The retrospective analysis included 336 consecutive cases of infants with IH treated between January 2016 and October 2017. The patients were assessed in the hospital at the initiation of the therapy and later in outpatient settings during the therapy. The monitoring included blood pressure (BP), heart rate (HR), blood glucose, hepatic and renal function, myocardial enzymes and serum lipids. Cardiac examinations in the outpatient follow-up included electrocardiography, ultrasound echocardiography, height, weight and head circumference. Propranolol decreased BP and HR at the initiation of treatment. The incidences of sinus bradycardia and hypoglycemia increased with the time of administration. Mean height, weight and head circumference were not affected during the treatment. The incidence of PR prolongation was 0\%-5.7\%. The effect of propranolol on the cardiovascular system, metabolism and physical development was not affected by its dose. Oral propranolol is a safe treatment for IH. Serious side effects were not observed. Attention should be paid to the side effects during clinical treatment. [\hyperlink{Halobetasol Propionate}{PMID: 34918994}, Lu Yu et al., 2022]

\hypertarget{pmid_32845589}{I}ntroduction: Psoriasis is a chronic, immune-mediated skin disease that is associated with sex-related differences. Two double-blind, vehicle-controlled, phase 3 studies evaluated halobetasol propionate (HP) 0.01\% lotion for the treatment of moderate-to-severe localized plaque psoriasis; pooled post hoc analyses investigated efficacy and safety in male and female subgroups. Methods: Participants were randomized (2:1) to once-daily HP or vehicle lotion for 8-weeks of double-blind treatment, with a 4-week posttreatment follow-up. Post hoc efficacy assessments in male (n=253) and female (n=177) subgroups included treatment success (≥2‑grade improvement in Investigator's Global Assessment [IGA] score and score of 'clear' or 'almost clear'), treatment success in psoriasis signs (erythema, plaque elevation, and scaling) at the target lesion, and change in affected body surface area (BSA). Treatment-emergent adverse events (TEAEs) were evaluated. Results: At week 8, rates of IGA-rated treatment success were significantly greater for HP versus vehicle in males (34.0\% vs 6.4\%) and females (42.7\% vs 14.6\%; P<0.001 both). Treatment success in each psoriasis sign approached or exceeded 50\% for HP-treated males and females, with all differences versus vehicle statistically significant (P<0.001). Percent reduction in affected BSA was significantly greater for HP versus vehicle in males (34.9\% vs 6.7\%) and females (35.6\% vs 4.6\%; P<0.001 both). Five HP treatment-related TEAEs (all application site-related) were reported through week 8. Conclusions: HP lotion was associated with significant reductions in disease severity in male and female participants with moderate-to-severe psoriasis, with good tolerability and safety over 8 weeks of once-daily use. In the overall pooled population, results were similar. J Drugs Dermatol. 2020;19(8): doi:10.36849/JDD.2020.5250. [\hyperlink{Halobetasol Propionate}{PMID: 32845589}, Fran E Cook-Bolden et al., 2020]

\hypertarget{pmid_1757614}{T}he efficacy and safety of 0.05\% halobetasol propionate ointment were evaluated in patients with chronic atopic or other eczematous dermatoses in two vehicle-controlled, double-blind studies: a paired-comparison study in 124 patients (study A) and a parallel-group study in 100 patients (study B). In study A, patients applied both treatments twice daily for 2 weeks and were evaluated by investigators on days 0, 7, and 14 with 0 to 3 severity scales and by self-assessment with two 5-step end-of-treatment rating scales. In study B, patients applied treatments twice daily for 2 weeks, and investigators made evaluations on days 0, 3, 7, and 14 with 0 to 6 scales and also made a 5-step end-of-treatment physician's global assessment. In study A, both severity scores and patient ratings favored halobetasol propionate significantly on days 7 (p less than or equal to 0.0013) and 14 (p less than 0.0001); in study B, severity scores on days 3 (p less than or equal to 0.045, pruritus, erythema, and overall lesion severity), 7, and 14 (p less than 0.001, all comparisons) also favored halobetasol propionate significantly, and global assessments showed complete resolution or marked improvement for 83\% of patients using halobetasol propionate versus 28\% of those using vehicle (p less than 0.0001). No instances of systemic effects or skin atrophy were reported in either study. We conclude that 0.05\% halobetasol propionate ointment is highly effective and well tolerated in the treatment of the conditions studied, with the rapid action and high degree of clearing associated with superpotent corticosteroid formulations. [\hyperlink{Halobetasol Propionate}{PMID: 1757614}, C A Guzzo et al., 1991]

\hypertarget{pmid_27688361}{G}iven the widespread use of propranolol in infantile hemangioma (IH) it was considered essential to perform a systematic review of its safety. The objectives of this review were to evaluate the safety profile of oral propranolol in the treatment of IH. We searched Embase and Medline databases (2007-July 2014) and unpublished data from the manufacturer of Hemangiol/Hemangeol (marketed pediatric formulation of oral propranolol; Pierre Fabre Dermatologie, Lavaur, France). Selected studies included ≥10 patients treated with oral propranolol for IH and that either reported ≥1 adverse event or effect (AE) or planned to capture AEs. Data capture was standardized and extracted study design, demographic characteristics, IH characteristics, intervention, and safety outcomes. AEs were assigned a system organ class and preferred term. A total of 83 of 398 identified literature records met the inclusion criteria, covering 3766 propranolol-treated patients. The manufacturer's data for 3 pooled clinical trials (435 propranolol-treated patients) and 1 Compassionate Use Program (1661 patients) were included. AE data were reported for 1945 of 5862 propranolol-treated patients. The most frequently reported AEs included a range of sleep disturbances, peripheral coldness, and agitation. The most serious AEs (atrioventricular block, bradycardia, hypotension, bronchospasm/bronchial hyperreactivity, and hypoglycemia-related seizures) were managed by decreasing doses or temporary/permanent discontinuation of propranolol. Limitations included the variety of included study designs; monitoring, collection, and reporting of AE data; small sample sizes for some articles; and the wide scope of review. Oral propranolol is well tolerated if appropriate pretreatment assessments and within-treatment monitoring are performed to exclude patients with contraindications and to minimize serious side effects during treatment. [\hyperlink{Halobetasol Propionate}{PMID: 27688361}, Christine Léaute-Labrèze et al., 2016]

\hypertarget{pmid_31688260}{B}eta-blocker (Propanolol or Timolol maleate) treatment of infantile hemangiomas (IH) is a safe and effective treatment in the outpatient setting. The authors report a single surgeon's initial experience with setting up an outpatient service of beta-blocker treatment for head and neck IH at a tertiary children's hospital. A prospective study of children with head and neck IHs commenced in January 2015 with the end point being December 2018. Each child started either oral propranolol (2 mg/kg/day) or topical Timolol 0.5\%. Thirty-eight patients commenced a beta-blocker during the study duration. The mean age at time of starting therapy was 9 months (range 3 weeks to 116 months). Four patients were older than 12 months at commencement. The mean duration of treatment was 9 months. The response to treatment was excellent or complete in 29\% (n = 11), good in 50\% (n = 18) and mild in 10\% (n = 4). The non response rate was 10\% (n = 4). No major adverse effects occurred but 29\% (n = 11) experienced minor side effects. Low dose propranolol and topical Timolol is been safe and easy to use for surgeons who may not be regular prescribers or unfamiliar with treating children with IHs with beta-blocker therapy. In patient monitoring is unnecessary and parents can be taught easily to recognise side effects. Treating children from the start builds a trusting relationship with the family before the child requesting cosmetic revision of the fibro-fatty remnant. [\hyperlink{Halobetasol Propionate}{PMID: 31688260}, Shiba Sinha et al., ]

\hypertarget{pmid_28043186}{O}ral propranolol has been recently approved for infantile hemangiomas (IHs), but potential side effects stay a challenge. We sought to make an additional assessment on oral propranolol safety for this indication. Prospective study included 108 infants consecutively treated for IHs at the University Children's Hospital Tirsova, Belgrade from January 2010 to December 2013. Propranolol was administered orally at a daily dose of 0.5 mg/kg and doubled every 48 hours in the absence of side effects until reaching the maximum dose of 2 mg/kg daily. Systolic and diastolic blood pressure and heart rate were measured every 48 hours with clinical observation. Heart rate was monitored by standard electrocardiogram (ECG) and 48-hour Holter ECG. Statistically significant, but asymptomatic decreases in systolic blood pressure and heart rate recorded by Holter ECG were observed during the first doubling of dose and then remained stable. Arrhythmias were not detected. Despite mild sleep disturbance observed in 31\% of infants in the hospital milieu, Holter monitoring indicated circadian rhythm maintenance. Oral propranolol for IHs does not remarkably affect heart rhythm including circadian variations throughout hospital initiation. Therefore, there is no necessity for Holter monitoring in additional safety assessment. [\hyperlink{Halobetasol Propionate}{PMID: 28043186}, Jelena Petrovic et al., 2017]

\hypertarget{pmid_29149854}{P}ropranolol has become the first-line treatment for complicated Infantile Hemangioma (IH), showing so far a good risk-benefit profile. We report the case of a toddler, on propranolol, who suffered cardiac arrest during an acute viral infection. She had a neurally-mediated syncope that progressed to asystole, probably because of concurrent factors as dehydration, beta-blocking and probably individual susceptibility to vaso-vagal phenomena. In fact a significant history of breath-holding spells was consistent with vagal hyperactivity. The number of patients treated with propranolol for IHs will increase and sharing experience will help to better define the safety profile of this drug. [\hyperlink{Halobetasol Propionate}{PMID: 29149854}, Alvise Tosoni et al., 2017]

\hypertarget{pmid_23082876}{P}ediatric patients undergoing hematopoietic stem cell transplantation (HSCT) are at high risk of acquiring fungal infections. Antifungal prophylaxis shortly after transplantation is therefore indicated, but data for pediatric patients under 12 years of age are scarce. To address this issue, we retrospectively assessed the safety, feasibility, and initial efficacy of prophylactic posaconazole in children. 60 consecutive pediatric patients with a median age of 6.0 years who underwent allogeneic HSCT between August 2007 and July 2010 received antifungal prophylaxis with posaconazole in the outpatient setting. 28 pediatric patients received an oral suspension at 5 mg/kg body weight b.i.d., and 32 pediatric patients received the suspension at 4 mg/kg body weight t.i.d. The observation period lasted from start of treatment with posaconazole until its termination (maximum of 200 days post-transplant). Pediatric patients who received posaconazole at 4 mg/kg body weight t.i.d. had a median trough level of 383 μg/L. Patients who received posaconazole at 5 mg/kg body weight b.i.d. had a median trough level of 134 μg/L. Both regimens were well tolerated without severe side effects. In addition, no proven or probable invasive mycosis was observed. Posaconazole was a well-tolerated, safe, and effective oral antifungal prophylaxis in pediatric patients who underwent high-dose chemotherapy and HSCT. Posaconazole at a dosage of 12 mg/kg body weight divided in three doses produced consistently higher morning trough levels than in patients who received posaconazole 5 mg/kg body weight b.i.d. Larger prospective trials are needed to obtain reliable guidelines for antifungal prophylaxis in children after HSCT. [\hyperlink{Halobetasol Propionate}{PMID: 23082876}, Michaela Döring et al., 2012]

\hypertarget{pmid_8712442}{A}n epidural test dose containing epinephrine does not reliably produce hemodynamic responses in children under halothane anesthesia. The purpose of this study was to determine hemodynamic responses to intravenous isoproterenol in both awake and halothane-anesthetized children. After obtaining institutional review board approval and parental informed consent, 72 ASA physical status 1 or 2 children (2.8 +/- 1.7 yr) undergoing elective minor surgery were studied before and during anesthesia with 1.2 minimum alveolar concentration halothane. A bolus containing 0.25 mg/ kg bupivacaine and 0.05 microgram/kg, 0.075 microgram/kg, or 0.1 microgram/kg isoproterenol, or bupivacaine and saline was injected via a peripheral arm vein to simulate intravascular injection of an epidural test dose. Before induction of anesthesia, all patients showed a positive test response after isoproterenol injection (heart rate increase > or = 20 beats/min). During anesthesia, 79\% of patients receiving 0.05 microgram/kg, 89\% of patients receiving 0.075 microgram/kg, and 100\% of patients receiving 0.1 microgram/kg met the criterion of a positive test response. Among each treatment group, all infants showed a positive test response. Blood pressure did not differ among the groups at any time. Transient benign dysrhythmias occurred in only one patient under halothane anesthesia receiving 0.075 microgram/kg isoproterenol. Isoproterenol at a dose of 0.1 microgram/kg is a sensitive indicator for intravascular injection of a test dose in children anesthetized with halothane and nitrous oxide. Isoproterenol at a dose of 0.05 microgram/kg approximates a minimal effective dose in awake children and in infants. After detailed studies on neural toxicity, isoproterenol could be of value as an epidural test agent in children. [\hyperlink{Halobetasol Propionate}{PMID: 8712442}, S Kozek-Langenecker et al., 1996]

\hypertarget{pmid_27043724}{O}ral propranolol is now established as the first-line treatment for infantile haemangiomas, and used in up to 20 \% of all cases. Propranolol use in infants is most commonly instigated in a controlled environment to monitor for potential serious adverse events such as hypoglycaemia and hypotension. Two test doses are recommended, the first one of 300 μg/kg followed by 2-hourly monitoring. On the subsequent day, a further dose of 650 μg/kg is administered with the same monitoring. A dose of 2 mg/kg divided into three is started from the next day. Parents/carers need to be warned of common adverse effects, of which disturbed sleep is the commonest. Treatment is recommended for up to a year to avoid rebound growth and the need to restart the treatment.  [\hyperlink{Halobetasol Propionate}{PMID: 27043724}, Robert H Taylor et al., 2016] Halobetasol propionate and tazarotene lotion 0.01\%/0.045\% (HP/TAZ) is a topical medication approved for the treatment of plaque psoriasis in adults. As a treatment modality, HP/TAZ has a combinatory therapeutic effect because it contains both a corticosteroid (HP) and a retinoid (TAZ) component. Here, we review the important clinical efficacy and safety data derived from pivotal clinical trials for HP/TAZ in the treatment of plaque psoriasis. We also discuss the mechanism of action, dosage guidelines, pharmacokinetics/pharmacodynamics, and clinical considerations for HP/TAZ, including why HP/TAZ should be avoided in pregnant patients. [\hyperlink{Halobetasol Propionate}{PMID: 27043724}, Vidhatha Reddy et al., 2020]

\hypertarget{pmid_31170512}{H}alobetasol propionate (HB) is considered a super potent drug in the group of topical corticosteroids. HB has anti-inflammatory activity, vasoconstriction properties, and due to its high skin penetration, it can cause systemic side effects. To improve its characteristics, enhance topical effectiveness and reduce penetration to systemic circulation, a study to optimize and characterize a HB-loaded lipid nanocarrier (HB-NLC) has been made by high-pressure homogenization method. The formulation is composed by HB, surfactant, glyceryl distearate and capric glycerides. The optimized HB-NLC containing 0.01\% of HB and 3\% of total lipid shows an average size below 200 nm with a polydispersity index ≪0.2 and an encapsulation efficiency ≫90\%. The in vitro and in vivo tests indicate that the HB-NLC is not toxic, is well tolerated and has an anti-inflammatory effect because they decrease the production of Interleukins in keratinocytes and monocytes. HB-NLC is considered an alternative treatment for skin inflammatory disorders. [\hyperlink{Halobetasol Propionate}{PMID: 31170512}, Paulina Carvajal-Vidal et al., 2019]

\hypertarget{pmid_25753275}{P}ropranolol has been recently approved by health authorities to treat infantile haemangiomas (IH). Propranolol is indicated in infants less than 5months of age with an IH requiring systemic therapy: IH at life-threatening and/or functional risk, painful ulcerated IH and IH that may cause permanent disfigurement. Propranolol should be initiated by physicians who have expertise in the diagnosis, treatment and management of IH. In addition, the first intake and every escalation should be administrated in a controlled clinical setting where adequate facilities for handling of adverse reactions, including those requiring urgent measures, are available. Then a monthly monitoring with dose adjustment weight is mandatory by the family doctor. Parents should be informed of the risk of hypoglycaemia and bronchoconstriction, especially during respiratory infectious outbreaks. The recommended duration of treatment is 6months without tapering. Relapses are possible necessitating a second course of 3 to 6months of treatment.  [\hyperlink{Halobetasol Propionate}{PMID: 25753275}, C Léauté-Labrèze et al., 2015] Infantile haemangiomas (IH) are the most common benign tumours in children. They are characterised by rapid growth during the first year of life followed by spontaneous regression during childhood. Indications for treatment are functional impairment, bleeding/ulceration, rapid growth and severe aesthetic risk. Recently, systemic treatment with propranolol has become the first-line therapy. The objective of this study was to assess the efficacy of propranolol in the treatment of IH and to investigate whether treatment with a low dose of 1 mg/kg/day was sufficient. This study was retrospective and based on a review of children treated for IH with propranolol from the 2010-2012 period at Rigshospitalet. Overall, propranolol was effective in all but one child (97\%). The majority of the children (84\%) were treated with an initial dose of 1 mg/kg/day, which was considered sufficient in most cases (71\%). Children who started treatment before five months of age had a significantly better response than children who started treatment at a later age. No relation was found between location of IH and the effect of treatment. There were only few and mild side effects. Propranolol is effective in the treatment of IH and it has only few and mild side effects. In most cases, a low dose of 1 mg/kg/day was sufficient. Early initiation of treatment is recommended as the response to treatment was better in younger children and because early initiation helps prevent large residual changes. not relevant. not relevant. [\hyperlink{Halobetasol Propionate}{PMID: 25753275}, Ida Gillberg Andersen et al., 2014]

\hypertarget{pmid_24849505}{T}o evaluate the safety and efficacy of our institutional beta-blocker protocol for treatment of complicated infantile hemangiomas (IH). A retrospective descriptive study of 76 infants/children with IH treated with oral propranolol at the Children's Hospital of Philadelphia between June 2008 and August 2010 was performed, assessing both the safety and efficacy of propranolol. Based on preliminary data showing hemangioma recrudescence off-treatment, we reviewed 9 additional patients with recrudescence between August 2010 and December 2011. Mild adverse events included asymptomatic bradycardia, gastrointestinal symptoms, asymptomatic hypotension, cool hands/feet, asymptomatic hypoglycemia, and sleep disturbance. Sixteen patients had recrudescence of IH off-treatment, with propranolol discontinued at a median age of 14 months (interquartile range 10-15 months). Propranolol appears to be associated with minor, not severe symptomatic adverse events. Propranolol appears to be effective in treating complicated IH. Recrudescence can occur off-treatment, even with discontinuing propranolol as late as 15 months of age. [\hyperlink{Halobetasol Propionate}{PMID: 24849505}, Derek H Chu et al., 2014]

\hypertarget{pmid_23680605}{T}here has been widespread interest surrounding the use of beta-blockers (i.e. propranolol, timolol, nadolol, acebutolol) in the treatment of infantile hemangiomas (IH). To review literature evaluating treatment of IH with propranolol. We conducted a literature search on PubMed and investigated for case reports, case series, and controlled trials by using search terms including hemangioma and propranolol. Data suggest that beta-blockers are efficacious in cutaneous, orbital, subglottic, and hepatic hemangiomas and assist in the resolution of ulcerated hemangiomas. Improvement has also been documented in children with PHACE syndrome. Propranolol produces favorable results in children who do not respond to steroids and with no long-term adverse effects. Propranolol should be administered with caution due to rare but serious side effects including hypoglycemia, wheezing, hypotension, and bradycardia. Additionally, recurrence of lesions following the cessation of treatment has been documented. Although large-scale randomized controlled trials must be conducted in order to further evaluate the safety and the possible role of propranolol in the treatment of IH, the reviewed literature suggests that propranolol carries promise as a potential replacement for corticosteroids as first-line therapy or as a part of a multimodal approach. [\hyperlink{Halobetasol Propionate}{PMID: 23680605}, Nivedita Gunturi et al., 2013]

\hypertarget{pmid_22129321}{D}ata regarding the use of propranolol in pediatrics are limited despite its widespread use in adults. Since 1984, Propranolol has been used for the prevention of portal hypertensive hemorrhage in pediatric patients. Recently it has been also used for the management of hemangiomas in addition to other indications. The purpose of this review is to evaluate safety and efficacy of propranolol use in the pediatric population, highlighting the most important reported side effects, warnings and precautions. [\hyperlink{Halobetasol Propionate}{PMID: 22129321}, Mortada El-Shabrawi et al., 2011]

\hypertarget{pmid_25385271}{I}nfantile haemangiomas (IH) are neoplastic proliferations of endothelial cells which occur with an incidence of 10-12\%. IH rapidly growing and found in cosmetically sensitive areas or complicated with ulcerations are of special concern of parents. A review of medical charts was performed for newborns treated with propranolol because of IH between 2012 and 2013. There were two boys and two girls, referred to our department at the age of 2-3 weeks. Children were commenced on propranolol 0.5 mg/kg daily and closely monitored. The dosage was increased up to a maximum of 2 mg/kg/d and was maintained until the lesion had involuted or showed good result. The minimal dosage required to achieve involution was 1.5-2.0 mg/kg/d. No rebound growth or complications were observed. Three patients showed excellent response with resolution of the lesion. Fourth patient showed good result with >50\% reduction of IH. Propranolol at 1.5-2.0 mg/kg/d is effective and safe for treating IH in our series of newborn patients. Treatment should be maintained until the lesion is involuted or shows good cosmetic result. Still there is need for larger scale studies confirming the safety and efficacy of propranolol in treatment of haemangiomas in newborns. [\hyperlink{Halobetasol Propionate}{PMID: 25385271}, Marzanna Oksiuta et al., 2016]

\hypertarget{pmid_23340697}{W}e aimed to assess the efficacy and safety of low-dose propranolol for treatment of infantile hemangiomas (IHs) in China. Our prospective study included data from 89 patients with IH, aged 1-12 months. Plasma renin activity, angiotensin II, and aldosterone were measured before initiation of propranolol therapy. Patients were administered propranolol (0.75-1 mg/kg/day) under close observation. The volume, texture, and color of lesions were used to evaluate efficacy. Safety endpoints included heart rate, systolic and diastolic blood pressures, alanine transaminase, aspartate transaminase, thyroid function tests, and fasting blood glucose. Adverse effects were recorded. Mean plasma angiotensin II concentration in patients with IH was higher than that in age-matched healthy children, whereas mean plasma renin activity was lower. Mean aldosterone level was higher at 1-3 months but lower at 4-12 months, than values reported previously. After propranolol therapy for 6 months, IH regression was classed as grade IV in 44 patients (49.4 \%), grade III in 21 patients (23.6 \%), and grade II in 24 patients (27.0 \%); none were grade I. Mild adverse effects, including diarrhea, restless sleep, nausea, cold extremities, and hypoglycemia, occurred in 12 patients (13.5 \%). Slight decreases in heart rate and blood pressure occurred in all patients (p < 0.05). The IHs of four patients (4.5 \%) relapsed after treatment cessation at 4-5 months. Low-dose propranolol is effective and safe for Chinese children with IH, and larger-scale studies are merited. Mechanisms underlying IH pathogenesis, and possible involvement of the renin-angiotensin-aldosterone system, deserve study. [\hyperlink{Halobetasol Propionate}{PMID: 23340697}, Xiaorong Ma et al., 2013]

\section*{Fluticasone Propionate And Salmeterol Xinafoate}
\subsection*{Result}
\subsubsection*{Answer}

Yes (6 months to <5 years)
Yes (4–11 years)
Yes (5–16 years)

\subsubsection*{{Explanation}}
\hypertarget{Fluticasone Propionate And Salmeterol Xinafoate}
A review of the available abstracts reveals multiple targeted studies evaluating the safety of Fluticasone Propionate and Salmeterol Xinafoate (FP/SAL) in children with asthma. The evidence is summarized by age range as follows:

Infants and Preschool Children (6 months to <5 years):
- A randomized, double-blind, multicenter trial in children aged 8 months to 4 years (n=300) compared FP/SAL to FP alone for 8 weeks, followed by 16 weeks of open-label FP/SAL. No new safety signals were seen, and no clinically significant differences in safety were noted between groups [\hyperlink{pmid_30556939}{PMID: 30556939}, Shigemi Yoshihara et al., 2019].
- An open-label study in 35 children aged 6 months to 5 years with mild-to-moderate asthma found that FP/SAL improved symptoms with no safety concerns; adverse events were not regarded as treatment-related [\hyperlink{pmid_27273710}{PMID: 27273710}, S Yoshihara et al., 2016].
- A retrospective study in 50 children under 5 years old found significant efficacy and only a 3.4\% reduction in height percentile (not statistically significant), concluding the combination was highly efficacious and safe, though it called for further prospective studies [\hyperlink{pmid_15360067}{PMID: 15360067}, Sudhir Sekhsaria et al., 2004].

Children Aged 4–11 Years:
- A randomized, multicenter, double-blind, active-controlled study in 203 children aged 4–11 years with persistent asthma compared FP/SAL (100/50 mcg) to FP alone for 12 weeks. The safety profile of FP/SAL was similar to FP alone, with both treatments well tolerated and no significant differences in adverse events or laboratory findings [\hyperlink{pmid_16095144}{PMID: 16095144}, Randolph Malone et al., 2005].
- A randomized, double-blind, double-dummy, parallel-group study in 257 children (age not specified, but described as "children") found both combination and concurrent therapy with FP/SAL to be well-tolerated with comparable adverse event profiles [\hyperlink{pmid_10922131}{PMID: 10922131}, N J Van den Berg et al., 2000].
- A meta-analysis of 12 RCTs (n=9,859) in children (age not specified, but described as "children") found no significant differences in drug-related adverse events between FP/SAL and FP alone [\hyperlink{pmid_37740997}{PMID: 37740997}, Hua Li et al., 2023].
- Additional studies in children with persistent asthma or cough variant asthma (age ranges 4–17 and unspecified) found FP/SAL to be well tolerated with no new safety concerns [\hyperlink{pmid_19382218}{PMID: 19382218}, David Pearlman et al., 2009; \hyperlink{pmid_37587710}{PMID: 37587710}, Junting Liu et al., 2023].

Children Aged 5–16 Years:
- A prospective observational study in 84 children aged 5–16 years with moderate persistent asthma found that FP/SAL was associated with improved asthma control and no safety concerns were reported [\hyperlink{pmid_29274305}{PMID: 29274305}, Nulma S Jentzsch et al.].

Summary:
There are multiple targeted studies, including randomized controlled trials, open-label studies, and meta-analyses, specifically evaluating the safety of Fluticasone Propionate and Salmeterol Xinafoate in children as young as 6 months up to 16 years. Across these studies, FP/SAL was consistently found to be well tolerated, with a safety profile similar to FP alone and no new or significant safety concerns identified. Some studies in the youngest age group (under 5 years) note the need for further prospective research, but the available data affirm safety in the studied populations.

\subsection*{Abstracts}
\hypertarget{pmid_30556939}{F}luticasone propionate 50 μg/salmeterol xinafoate 25 μg (FP/SAL) is widely used in adults and children with asthma, but there is sparse information on its use in very young children. This was a randomized, double-blind, multicentre, controlled trial conducted in children aged 8 months to 4 years. During a 2-week run-in period, they all received FP twice daily. At randomization, they commenced FP/SAL or FP twice daily for 8 weeks. All were then given FP/SAL only, in a 16-week open-label study continuation. Medications were inhaled through an AeroChamber Plus with attached face mask. The primary end-point was mean change in total asthma symptom scores from baseline to the last 7 days of the double-blind period. Analyses were undertaken in all children randomized to treatment and who received at least one dose of study medication. Three hundred children were randomized 1:1 to receive FP/SAL or FP. Mean change from baseline in total asthma symptom scores was -3.97 for FP/SAL and -3.01 with FP. The between-group difference was not statistically significant (P = 0.21; 95\% confidence interval: -2.47, 0.54). No new safety signals were seen with FP/SAL. This is the first randomized, double-blind study of this size to evaluate FP/SAL in very young children with asthma. FP/SAL did not show superior efficacy to FP; no clear add-on effect of SAL was demonstrated. No clinically significant differences in safety were noted with FP/SAL usage. [\hyperlink{Fluticasone Propionate And Salmeterol Xinafoate}{PMID: 30556939}, Shigemi Yoshihara et al., 2019]

\hypertarget{pmid_27273710}{C}linical evidences of inhaled salmeterol/fluticasone propionate combination (SFC) therapy are insufficient in early childhood asthma. To examine the effects of SFC50, a combination product of salmeterol xinafoate (50 μg/day) and fluticasone propionate (100 μg/day), in infants and preschool children with asthma. The study was conducted at 31 sites in Japan. 35 patients (6 months to 5 years old) with asthma insufficiently controlled by inhaled corticosteroids (100 μg/day) were initiated to treat with SFC50 twice a day for 12 weeks with pressurized metered dose inhalers. The efficacy of SFC50 was assessed using nighttime sleep disorder score as the primary endpoint and the other efficacy measurements. The safety measurement included the incidences of adverse event (AE). Mean patient age was 3.1 years, and 94.2\% had mild-to-moderate persistent asthma (atopic type: 65.7\%). Nighttime sleep disorder scores, assessed by a nighttime sleep diary, significantly decreased after treatment with SFC50 throughout the study period (p<0.01). SFC50 also significantly improved other efficacy outcomes including asthma symptom score, frequency of short-acting beta-agonist treatment, frequency of unscheduled visits to clinic, frequency of exacerbation due to virus infection, asthma control score and patient QOL score (p<0.01). AEs of cold, upper respiratory inflammation and asthmatic attack occurred in each of the 3 patients (8.6\%); however, these were not regarded as treatment-related AEs. SFC50 improved nighttime sleep disorder score and other efficacy outcome measures with no safety concerns. The results suggest that SFC50 treatment is useful to control the mild-to-moderate asthma in infant and preschool-aged children. [\hyperlink{Fluticasone Propionate And Salmeterol Xinafoate}{PMID: 27273710}, S Yoshihara et al., 2016]

\hypertarget{pmid_37620110}{F}luticasone propionate/salmeterol xinafoate (FP/SAL) is an inhaled corticosteroid (ICS) and long-acting β Compared with the adult population, fewer clinical studies have investigated the efficacy of FP/SAL in paediatric patients with moderate and moderate-to-severe asthma. In this review, we synthesise the available evidence for the efficacy and safety of FP/SAL in the paediatric population, compared with other available therapies indicated for asthma in children. A literature review identified randomised controlled trials and observational studies of FP/SAL in the paediatric population with moderate-to-severe asthma. The Medline database was searched using PubMed (https://pubmed.ncbi.nlm.nih.gov/), with no publication date restrictions. Search strategies were developed and refined by authors. Selected articles were screened for clinical outcome data (exacerbation reduction, nocturnal awakenings, lung function, symptom control, rescue medication use and safety) and a table of key parameters developed. Improvements in asthma outcomes with FP/SAL include reduced risk of asthma-related emergency department visits and hospitalisations, protection against exercise-induced asthma and improvements in measures of lung function. Compared with FP monotherapy, greater improvements in measures of lung function and asthma control are reported. In addition, reduced incidence of exacerbations, hospitalisations and rescue medication use is observed with FP/SAL compared with ICS and leukotriene receptor antagonist therapy. Furthermore, FP/SAL therapy can reduce exposure to both inhaled and oral corticosteroids. FP/SAL is a reliable treatment option in patients not achieving control with ICS monotherapy or a different ICS/LABA combination. Evidence shows that FP/SAL is well tolerated and has a similar safety profile to FP monotherapy. Thus, FP/SAL provides an effective option for the management of moderate-to-severe asthma in the paediatric population. [\hyperlink{Fluticasone Propionate And Salmeterol Xinafoate}{PMID: 37620110}, Paulo Marcio Pitrez et al., 2023]

\hypertarget{pmid_10922131}{T}he aim of this study was to compare the efficacy and safety in children of salmeterol (50 microg twice daily) plus fluticasone propionate (100 microg twice daily) when delivered together via a single Diskus inhaler (Seretide; combination therapy) or concurrently using two separate Diskus inhalers (concurrent therapy). In a multicenter, randomized, double-blind, double-dummy, parallel-group study, 257 children with reversible airways obstruction who remained symptomatic on inhaled corticosteroids (200-500 microg daily) alone were randomized to combination or concurrent therapy for 12 weeks. Efficacy was assessed by measuring daily peak expiratory flow (PEF), symptom scores, and rescue salbutamol use. In addition, lung function tests were performed at each clinic visit. Safety assessments included monitoring of adverse events and morning serum cortisol concentrations. The primary efficacy parameter (mean morning PEF) increased during treatment in both groups; adjusted mean changes were 33 and 28 L/min for the combination and concurrent therapies, respectively. The 90\% confidence interval for the difference in mean morning PEF between treatment groups was within the +15 L/min criterion for clinical equivalence. Similarly, there were improvements in pulmonary function, symptom score, and rescue salbutamol use during treatment in both groups, with no significant differences between the combination and concurrent therapy groups for any of these secondary efficacy parameters. Both treatment regimens were well-tolerated and had comparable adverse event profiles. Mean morning serum cortisol levels increased similarly in both groups during the study. In conclusion, salmeterol and fluticasone propionate therapy given as a new combination product is as safe and effective in children with asthma as the same drugs given concurrently via separate inhalers. [\hyperlink{Fluticasone Propionate And Salmeterol Xinafoate}{PMID: 10922131}, N J Van den Berg et al., 2000]

\hypertarget{pmid_29274305}{T}here is a scarcity of studies that assessed the association between adherence to combination therapy and asthma control in pediatric patients. The authors investigated the association between adherence to fluticasone propionate/salmeterol xinafoate combination-metered aerosol and the level of asthma control in children. This was a prospective observational study of 84 patients aged 5-16 years with moderate persistent asthma, who remained uncontrolled despite the use of 1000μg/day of inhaled nonextrafine-hydrofluoric alkane-beclomethasone dipropionate in the three months prior to study enrollment. Participants were prescribed two daily doses of FP (125μg)/salmeterol xinafoate (25μg) combination by metered aerosol/spacer for six months. Adherence rates were assessed using the device's dose counter after the 2nd, 4th, and 6th months of follow up. Asthma control was assessed using a simplified Global Initiative for Asthma 2014 Report classification. Mean adherence rates after the second, fourth, and sixth months were 87.8\%, 74.9\%, and 62.1\% respectively, for controlled asthma, and 71.7\%, 56.0\%, and 47.6\% respectively, for uncontrolled asthma (all p-values≤0.03). The proportion of children achieving asthma control increased to 42.9\%, 67.9\% and 89.3\% after the 2nd, 4th and 6th months of follow-up, respectively (p≤0.001). Adherence rates between 87.8\% in the 2nd month and 62.1\% in the 6th month were strong determinants of asthma control. [\hyperlink{Fluticasone Propionate And Salmeterol Xinafoate}{PMID: 29274305}, Nulma S Jentzsch et al., ]

\hypertarget{pmid_9257086}{S}almeterol xinafoate is a selective beta 2-adrenoceptor agonist indicated for the maintenance treatment of adults and children with asthma. When administered as a dry powder or aerosol, salmeterol produces bronchodilation for at least 12 hours and protects against methacholine and exercise-induced bronchoconstriction. Salmeterol is not recommended for the treatment of acute exacerbations of asthma. Recent clinical studies have demonstrated the efficacy and tolerability of inhaled salmeterol in the management of asthma in children. Salmeterol improved symptom control and lung function more effectively than placebo or regularly administered salbutamol. In children who were symptomatic despite regular inhaled corticosteroid therapy, the addition of salmeterol to treatment produced a significant improvement in morning and evening peak expiratory flow and forced expiratory volume in 1 second, and a significant reduction in the incidence of asthma exacerbations compared with placebo. Notably, the long duration of action of salmeterol makes it particularly suitable for the prevention of nocturnal asthma symptoms and exercise-induced asthma (EIA) in children. Current data suggest that salmeterol should not be used as a substitute for corticosteroid therapy in children, but rather as an adjunct to therapy. Thus, salmeterol may be a suitable adjunct to therapy in children with asthma receiving inhaled corticosteroids. In addition, salmeterol also has a potentially important role in the prevention of EIA and nocturnal asthma symptoms. [\hyperlink{Fluticasone Propionate And Salmeterol Xinafoate}{PMID: 9257086}, J C Adkins et al., 1997]

\hypertarget{pmid_37740997}{A}sthma is a chronic respiratory disease that affects millions of children worldwide and can impair their quality of life and development. Inhaled glucocorticoids are the mainstay of asthma treatment, but some children require step-up therapy with additional drugs to achieve symptom control. Fluticasone propionate and salmeterol (FSC) has been shown to reduce asthma exacerbations and improve lung function in adults. However, the evidence for its efficacy and safety in children is limited. This study aims to provide a comprehensive basis for treatment selection by summarizing existing clinical randomized controlled trials (RCTs) on the efficacy of FSC compared to fluticasone propionate (FP) monotherapy in children with asthma who require step-up treatment. Five online databases and three clinical trial registration platforms were systematically searched. The effect size and corresponding 95\% confidence interval (CI) were calculated based on the heterogeneity among the included studies. Twelve RCTs were identified and a total of 9, 859 patients were involved. The results of the meta-analysis revealed that the use of FSC was associated with a greater reduction in the incidence of asthma exacerbations than FP alone when the dose of FP was the same or when the duration of treatment exceeded 12 weeks. In addition, FSC resulted in a greater proportion of time with asthma-free and without the use of albuterol compared to FP alone when the duration of treatment exceeded 12 weeks. No significant differences were observed between FSC and FP alone in the incidence of drug-related adverse events and other adverse events. Both FSC and FP alone are viable options for the initial selection of step-up treatment in asthmatic children. While, FSC treatment demonstrates a greater likelihood of reducing asthma exacerbations which is particularly important for reducing the personnel, social and economic burden in children requiring step-up asthma treatment. [\hyperlink{Fluticasone Propionate And Salmeterol Xinafoate}{PMID: 37740997}, Hua Li et al., 2023]

\hypertarget{pmid_15360067}{T}he incidence of asthma in children under age 5 is higher than in any other segment of the population. Current NAEPP guidelines recommend treatment of some asthmatics in this age group with the combination of an inhaled corticosteroid and a long-acting beta2-agonist even though this practice has never been studied with children younger than 4. This retrospective study analyzes the efficacy and safety of a combination of fluticasone propionate (FP) and salmeterol (SA) in children under 5. Fifty patients who started using FP/SA before the age of 60 months were included in the analysis. To determine efficacy, we tracked the change in emergency room visits, hospitalizations, and the frequency of wheezing as a result of treatment. Emergency room visits were reduced from 78 to 5 (p<0.001), hospitalizations were reduced from 43 to 2 (p<0.001) and frequency of wheezing, daily, weekly, or monthly, was also reduced significantly (p<0.003). In terms of safety, there was only a 3.4\% reduction in height percentile (p=0.37). Combination therapy is highly efficacious and safe for asthmatics under the age of 5. A well-designed prospective study is necessary to further evaluate the benefits and risks of this treatment method. [\hyperlink{Fluticasone Propionate And Salmeterol Xinafoate}{PMID: 15360067}, Sudhir Sekhsaria et al., 2004]

\hypertarget{pmid_10867258}{T}he physico-chemical properties of two anti-asthmatic drugs, salmeterol xinafoate and fluticasone propionate, have been studied in both aqueous and non-aqueous solvent environments. Ultraviolet-visible (UV-Vis) spectroscopy, fluorescence spectroscopy and electrospray ionisation mass spectrometry (ESI-MS) have been used to characterise the interaction of the drugs in 70:30 (v/v) methanol/water solutions. First derivative UV-Vis spectra measurements indicate that an interaction takes place between the two drugs in a binary solvent system. Fluorescence studies indicate that an increase in the concentration of fluticasone propionate results in a decrease in the fluorescence signal of the salmeterol for mixed solutions of the drugs. Analysis of a mixture of the two drug solutions using mass spectrometry also shows evidence of salmeterol-fluticasone propionate interaction and dimer formation with respect to both the salmeterol and the fluticasone propionate. Model metered dose inhalers (MDI) of both individual samples and mixtures of the drugs were formulated as suspensions in solvent CFC-113. The extent of deposition onto different inhaler components, such as the aluminium alloy canister, Teflon coated canister and the metering valve was evaluated by high-performance liquid chromatography (HPLC) of the methanol/water washings of the deposited drug(s). Changing the nature of the surface properties of the container resulted in a significant difference in the extent of deposition. The deposition of the individual drugs was found to increase as the dispersion concentration of the drug increases. However, the formulation based on a combination of the two drugs was found to show different deposition behaviour compared to the individual drug formulations. The deposition of the drugs, onto the aluminium alloy canister and the metering valve, decreases as the combined dispersion concentration of the two drug increases. [\hyperlink{Fluticasone Propionate And Salmeterol Xinafoate}{PMID: 10867258}, Y Michael et al., 2000]

\hypertarget{pmid_14730655}{O}ur objective was to evaluate the efficacy and safety of two doses of fluticasone propionate (FP) in young children with recurrent wheezing and risk factors for asthma. Our study design was a randomized, double-blind, placebo-controlled comparison of inhaled FP 50 mcg twice daily (FP 100) and 125 mcg twice daily (FP 250), for 6 months. Outcome measures included number of wheezing episodes, days on albuterol, height standard deviation score (height SDS), osteocalcin (OC), bone alkaline phosphatase fraction (AKP), insulin-like growth factor-binding protein 3 (IGFBP-3), and serum levels of cortisol (SC). Our subjects were 30 patients, aged 7-24 months. Mean wheezing episodes were 6.0 +/- 1.9, 1.9 +/- 1.9, and 2.8 +/- 1.2; mean days of albuterol use were 24.3 +/- 1.3, 6.5 +/- 0.8, and 9.1 +/- 0.8, per patient for placebo, FP100, and FP250 groups, respectively. There was a significant reduction in clinical outcome in the two FP groups compared to placebo (P < 0.01). No significant correlations were found between FP dosage and height SDS, OC, AKP, IGFBP-3, and SC. In conclusion, in young children with asthmatic symptoms, FP at 50 and 125 mcg b.i.d. for 6 months significantly improved respiratory symptoms without causing significant side effects on growth and bone metabolism. [\hyperlink{Fluticasone Propionate And Salmeterol Xinafoate}{PMID: 14730655}, Alejandro M Teper et al., 2004]

\hypertarget{pmid_27622699}{T}o evaluate the dose-response, efficacy, and safety of fluticasone furoate (FF; 25 µg, 50 µg, and 100 µg), administered once daily in the evening during a 12-week treatment period to children with inadequately controlled asthma. This was a Phase IIb, multicenter, stratified, randomized, double-blind, double-dummy, parallel-group, placebo- and active-controlled study in children aged 5-11 years with inadequately controlled asthma. The study comprised a 4-week run-in period, 12-week treatment period, and 1-week follow-up period. Children were randomized to receive either placebo once daily, fluticasone propionate (FP) 100 µg twice daily, FF 25 µg, FF 50 µg, or FF 100 µg each once daily in the evening. Primary endpoint was the mean change from baseline in daily morning peak expiratory flow (PEF) averaged over weeks 1-12. Adverse events (AEs) also were investigated. In total, 593 children were included in the intent-to-treat population. The difference vs placebo in change from baseline daily morning PEF averaged over weeks 1-12 was statistically significant for the FF 25, FF 50, FF 100, and FP 100 groups (18.6 L/min, 19.5 L/min, 12.5 L/min, and 14.0 L/min, respectively; P < .001 for all). The incidence of AEs was greater in the FF groups (32\%-36\%) than in the placebo group (29\%); the most frequent AE was cough. FF and FP resulted in significant improvements in morning PEF compared with placebo, suggesting that they are effective treatments for children with inadequately controlled asthma. All treatments were well tolerated; no new safety concerns were identified. ClinicalTrials.gov:NCT01563029. [\hyperlink{Fluticasone Propionate And Salmeterol Xinafoate}{PMID: 27622699}, Amanda J Oliver et al., 2016]

\hypertarget{pmid_19725893}{L}ong acting beta(2)-agonists (LABA) are widely used in children with asthma. Data from adults suggest that there is tachyphylaxis particularly to the bronchoprotective effects of LABA. There are no data in children. To determine whether LABA are subject to tachyphylaxis in school-aged children. Children were eligible for participation if they remained symptomatic on 400 microg of beclometasone dipropionate equivalent/day. Participants undertook a 4-wk run in period with open-label fluticasone 100 microg BD via Diskus. Children were then randomized to receive fluticasone 100 microg BD or salmeterol/fluticasone 50/100 microg BD via Diskus in a double-blind manner. Children underwent spirometry, cold air challenge and salbutamol reversibility testing at baseline, 4 and 8 wk. 37/42 children completed the study. There were significant improvements in basal FEV1 (\% predicted) in the salmeterol/fluticasone group (n = 21) (+6.4\% (95\% CI: 2.4-10.5) p = 0.0033) but not in the fluticasone group (n = 16) [+1.2 (95\% CI: -3.4 to 5.8) p = 0.5900]. There was a non-significant reduction in fall in FEV1 provoked by cold air in both groups. There was a significant lessening in the acute salbutamol response after 8 wk in the salmeterol/fluticasone group [-11.4\% (95\% CI: -17.6 to -5.2) p = 0.0010] but not in the fluticasone group [-1.6\% (95\% CI: -9.8 to 6.6) p = 0.6827]. Salmeterol/fluticasone therapy significantly improves basal FEV(1) in asthmatic children however, there is negligible additional bronchoprotection by week 4 of treatment and there is significant attenuation of salbutamol responsiveness when compared with fluticasone alone. Some of this reduction in salbutamol response may relate to the concurrent improvements in baseline lung function. [\hyperlink{Fluticasone Propionate And Salmeterol Xinafoate}{PMID: 19725893}, W D Carroll et al., 2010]

\hypertarget{pmid_27128459}{T}his multi-center, randomized, double-blind, placebo-controlled, two-way crossover study characterized the safety, tolerability, pharmacokinetics, and pharmacodynamics of fluticasone furoate (FF) in children (5-11 years) with persistent asthma. Twenty-seven children received inhaled FF 100 µg or placebo via the ELLIPTA™ dry powder inhaler once daily for 14 days, with a ≥7 day washout period. Adverse events (AEs) were reported by eight (31\%) and four (16\%) subjects during FF 100 µg and placebo treatment, respectively. Headache was reported by three subjects during FF 100 µg treatment and by no subjects during placebo treatment, all other AEs were reported by only one subject on either treatment; there were no serious AEs. Following repeat dosing, the arithmetic mean (SD) FF Cmax was 26.71 pg/mL (9.16) at 31 minutes post-dose. Arithmetic mean (SD) FF AUC(0-t) was 121.44 pg h/mL (83.04). Arithmetic mean values for weighted mean (SD) serum cortisol (0-12 hours) on day 14 were 56.49 (16.51) and 67.57 (20.66) ng/mL for FF 100 µg and placebo, respectively. No clinically significant effect of FF on serum cortisol levels was observed. FF was well tolerated. Pharmacokinetic profiles were well defined and did not differ between age groups in the study population, and no clinically significant suppression of serum cortisol was observed.  [\hyperlink{Fluticasone Propionate And Salmeterol Xinafoate}{PMID: 27128459}, Amanda Oliver et al., 2014] For children older than 5 years with asthma who remain symptomatic despite inhaled corticosteroid (ICS) therapy, the preferred treatment is to add an inhaled long-acting beta2-agonist vs increasing the ICS dose. To compare the safety of twice-daily treatment with inhaled fluticasone propionate plus the inhaled long-acting beta2-agonist salmeterol with that of fluticasone propionate used alone in children aged 4 to 11 years with persistent asthma. A randomized, multicenter, double-blind, active-controlled, parallel-group study in 203 children with persistent asthma who were symptomatic during ICS therapy. Patients received fluticasone propionate-salmeterol (100/50 microg) or fluticasone propionate (100 microg) alone twice daily for 12 weeks. The safety profile of fluticasone propionate-salmeterol was similar to that of fluticasone propionate alone. The overall incidence of adverse events was 59\% for fluticasone propionate-salmeterol and 57\% for fluticasone propionate. Both treatments were well tolerated. Two patients receiving fluticasone propionate-salmeterol and 5 receiving fluticasone propionate withdrew from the study because of worsening asthma. Changes in heart rate, blood pressure, and laboratory variables were infrequent and were similar between treatments. No patients had clinically significant abnormal electrocardiographic findings during treatment. Geometric mean 24-hour urinary cortisol excretion at baseline and after 12 weeks of treatment was comparable within and between groups; no patient in either group had abnormally low 24-hour urinary cortisol excretion after 12 weeks of treatment. The incidence of withdrawals due to asthma exacerbations was 2\% in the fluticasone propionate-salmeterol group and 5\% in the fluticasone propionate group. In pediatric patients with persistent asthma, fluticasone propionate-salmeterol twice daily was well tolerated, with a safety profile similar to that of fluticasone propionate used alone. [\hyperlink{Fluticasone Propionate And Salmeterol Xinafoate}{PMID: 27128459}, Randolph Malone et al., 2005]

\hypertarget{pmid_19382218}{E}xercise is a common trigger in children with persistent asthma and inhaled corticosteroids have been shown to effectively treat clinical manifestations of persistent asthma, including protection from decrements in lung function caused by exercise. The goal of this study was to evaluate the effectiveness of fluticasone propionate/salmeterol 100/50 mcg compared with fluticasone propionate 100 mcg for the prevention of airflow limitation triggered by standardized exercise challenge in pediatric and adolescent patients with persistent asthma. Multicenter, randomized, double-blind, parallel group trial of 248 subjects with persistent asthma (age 4-17 years) randomized to receive fluticasone propionate/salmeterol (100/50 mcg twice daily) or fluticasone propionate alone (100 mcg twice daily) via Diskus for 4 weeks. Exercise challenge tests were performed during screening and approximately 8 hr after administration of the blinded study medication on Treatment Day 28. After 4 weeks of therapy both treatments provided protection following exercise challenge. The protection estimated by the maximal fall in FEV(1) was significantly better for fluticasone propionate/salmeterol (9.5 +/- 0.8\% [mean +/- SE]) compared with fluticasone propionate alone (12.7 +/- 1.1\%, P = 0.021). Statistically significant differences were not observed for asthma rescue-free days and asthma symptom-free days. Chronic dosing with fluticasone propionate/salmeterol in a single device provides superior protection compared with an inhaled corticosteroid alone in protecting against exercise-induced asthma in children with persistent asthma. [\hyperlink{Fluticasone Propionate And Salmeterol Xinafoate}{PMID: 19382218}, David Pearlman et al., 2009]

\hypertarget{pmid_29490769}{T}he safety of a novel intranasal formulation of azelastine hydrochloride (AZE) and fluticasone propionate (FP) has been established in adults and adolescents with allergic rhinitis but not in children <12 years old. To evaluate the safety and tolerability of an intranasal formulation of AZE and FP in children ages 4-11 years with allergic rhinitis. The study was a randomized, 3-month, parallel-group, open-label design. Qualified patients were randomized in a 3:1 ratio to AZE/FP (n = 304) or fluticasone propionate (FP) (n = 101), one spray per nostril twice daily, and to one of three age groups: ≥4 to <6 years, ≥6 to <9 years, and ≥9 to <12 years. Safety was assessed by child- or caregiver-reported adverse events, nasal examinations, vital signs, and laboratory assessments. The incidence of treatment-related adverse events (TRAEs) was low in both the AZE/FP (16\%) and FP-only (12\%) groups after 90 days' continuous use. Epistaxis was the most frequently reported TRAE in both groups (AZE/FP, 9\%; FP, 9\%), followed by headache (AZE/FP, 3\%; FP, 1\%). All other TRAEs in the AZE/FP group were reported by ≤1\% of the children. The majority of TRAEs were of mild intensity and resolved spontaneously. Results of nasal examinations showed an improvement over time in both groups, with no cases of mucosal ulceration or nasal septal perforation. There were no unusual or unexpected changes in laboratory parameters or vital signs. The intranasal formulation of AZE and FP was safe and well tolerated after 3 months' continuous use in children with allergic rhinitis.The study was registered on <ext-link xmlns:xlink="http://www.w3.org/1999/xlink" ext-link-type="uri" xlink:href="http://ClinicalTrials.gov">ClinicalTrials.gov</ext-link> (NCT01794741). [\hyperlink{Fluticasone Propionate And Salmeterol Xinafoate}{PMID: 29490769}, William Berger et al., 2018]

\hypertarget{pmid_10428919}{T}he safety profile of fluconazole was assessed for 562 children (ages, 0 to 17 years) comprising 323 males and 239 females. The data are derived from 12 clinical studies of fluconazole as prophylaxis or treatment for a variety of fungal infections in predominantly immunocompromised patients. Most children received multiple doses of fluconazole in the range of 1 to 12 mg/kg of body weight; a few received single doses. Administration was mainly by oral suspension or intravenous injection. Overall, 58 (10.3\%) children reported 80 treatment-related side effects. The most common side effects were associated with the gastrointestinal tract (7.7\%) or skin (1.2\%). Self-limiting, treatment-related side effects affecting the liver and biliary system were reported in three patients (0.5\%). Overall, 18 patients (3.2\%) discontinued treatment due to side effects, mainly gastrointestinal symptoms. Dose and age did not appear to influence the incidence and pattern of side effects. Treatment-related laboratory abnormalities were uncommon, the most frequent being transient elevated alanine aminotransferase (4.9\%), aspartate aminotransferase (2.7\%), and alkaline phosphatase (2.3\%) levels. Although 98.6\% of patients were taking concomitant medications, no clinical or laboratory interactions were observed. The safety profile of fluconazole was compared with those of other antifungal agents, mostly oral polyenes, by using a subset of data from five controlled studies. Side effects were reported by more patients treated with fluconazole (45 of 382; 11.8\%) than by those patients treated with comparable agents (25 of 381; 6.6\%); vomiting and diarrhea were the most common events in both groups. The incidence and type of treatment-related laboratory abnormalities were similar for the two groups. In conclusion, fluconazole was well tolerated by the pediatric population, many of whom were suffering from severe underlying disease and were taking a variety of concurrent medications. The safety profile of fluconazole in children mirrors the excellent safety profile seen in adults. [\hyperlink{Fluticasone Propionate And Salmeterol Xinafoate}{PMID: 10428919}, V Novelli et al., 1999]

\hypertarget{pmid_37587710}{T}his study was to evaluate the clinical efficacy and safety of fluticasone/ salmeterol inhalation powder plus Huaiqihuang Granules for children with cough variant asthma (CVA). From June 2019 to May 2021, 60 children with CVA were hospitalized to the Pediatrics Department of Cangzhou Central Hospital and randomized to the observation (fluticasone/salmeterol inhalation powder plus huaiqihuang granules) and control group (fluticasone/salmeterol inhalation powder) using the random number table method. The outcome measures include clinical efficacy, forced vital capacity (FVC), forced expiratory volume per second (FEV1), peak expiratory flow (PEF), FeNO, high-sensitivity C-reactive protein (hs-CRP), interleukin-17 (IL-17) and IL-23, airway anatomical indicators and T lymphocyte subsets levels. Both groups exhibited remarkable improvements in FVC, FEV1, PEF and FeNO and hs-CRP, IL-17 and IL-23, with higher FVC, FEV1 and PEF and lower FeNO, hs-CRP, IL-17 and IL-23 in the observation group (all P<0.05). Significantly higher levels of CD4+ and CD4+/CD8+ were observed in the observation group versus control group, but lower airway wall thickness, basement membrane thickness, total airway wall area and CD8+ in the observation group (all P<0.05). Fluticasone/salmeterol inhalation powder plus Huaiqihuang Granules improves lung function, FeNO and airway inflammation in children with CVA and boosts cellular and humoral immune function. [\hyperlink{Fluticasone Propionate And Salmeterol Xinafoate}{PMID: 37587710}, Junting Liu et al., 2023]

\hypertarget{pmid_9704834}{F}luticasone propionate is a novel and potent corticosteroid. It seems to have an improved therapeutic index on the basis of studies on skin thinning and suppression of hypothalamic-pituitary-adrenal axis. We assessed the efficacy and safety of fluticasone propionate (FP) 0.05\% cream once daily as compared with clobetasone butyrate (CB) 0.05\% cream twice daily in children with atopic dermatitis (AD). Twenty-two children (3 to 8 years old) with moderately active AD received either FP once daily or CB twice daily. Severity of AD was scored weekly by means of the modified Scoring of Atopic Dermatitis system (SCORAD) and treatment was either stopped when skin lesions were almost cleared (SCORAD < 9) or after 4 weeks. Cortisol excretion was determined by means of 24-hour urine before and after treatment. Twenty-one children completed the study. After 1 week of treatment, mean SCORAD significantly decreased in both treatment groups. After 2, 3, and 4 weeks cumulatively, 8, 12, and 16 children, respectively, were clinically healed (SCORAD < 9). No significant differences in efficacy were observed between the two treatments. Urinary cortisol excretion was not altered by either of the treatments. Two weeks after discontinuation of active treatment, mean SCORAD had increased to 22, but still was significantly lower than that at the beginning of the study. Once-daily treatment with FP is as safe and effective as twice-daily treatment with CB in children with AD. All children experienced an exacerbation of AD within 2 weeks after treatment was withdrawn, indicating the need for long-term "intermittent" treatment. [\hyperlink{Fluticasone Propionate And Salmeterol Xinafoate}{PMID: 9704834}, A Wolkerstorfer et al., 1998]

\hypertarget{pmid_29857783}{T}he efficacy and safety of fluticasone propionate/formoterol fumarate pressurized metered-dose inhaler (pMDI) (fluticasone/formoterol; Flutiform A double-blind, double-dummy, parallel group, multicenter study. Patients, aged 5-<12 years with persistent asthma ⩾ 6 months and forced expiratory volume in 1 s (FEV A total of 512 patients were randomized: fluticasone/formoterol, 169; fluticasone, 173; fluticasone/salmeterol, 170. Fluticasone/formoterol was superior to fluticasone for the primary endpoint: change from predose FEV This study supports the efficacy and safety of fluticasone/formoterol in a pediatric asthma population and its superiority to fluticasone. [\hyperlink{Fluticasone Propionate And Salmeterol Xinafoate}{PMID: 29857783}, Anna Płoszczuk et al., ]

\hypertarget{pmid_9877002}{S}almeterol xinafoate is a highly selective beta2-adrenoceptor for the maintenance treatment of asthma in adults and children. To review the pharmacokinetics, clinical pharmacology, and therapeutic properties of a recently introduced, long acting antiasthmatic drug. Recent English-language publications were selected using Medline as database. Salmeterol's pharmacokinetics, clinical pharmacology, and therapeutic properties are reviewed and aspects related to salmeterol's unusual duration of action, its high potency, beta2-selectivity, possible antiinflammatory actions, its interaction with other drugs, low systemic adverse effects, dosage, and administration are also discussed. Salmeterol is a safe long-acting beta2-agonist very useful for maintenance treatment of asthma. [\hyperlink{Fluticasone Propionate And Salmeterol Xinafoate}{PMID: 9877002}, A Buchwald et al., 1998]

\hypertarget{pmid_17523720}{T}here is comparatively little information on asthma management in China. A multicentre, randomised, open-label, parallel-group, 6-week treatment study was conducted to evaluate the efficacy and safety of salmeterol/fluticasone propionate combination treatment in Chinese adult asthmatic patients. 398 patients with a documented history of moderate-to-severe asthma were randomised to treatment. Salmeterol 50mug/fluticasone propionate 100mug twice daily for 6 weeks via Accuhaler((R)) (Diskustrade mark) inhaler and budesonide 400mug twice daily for 6 weeks via Turbuhaler((R)) inhaler. Morning peak expiratory flow (PEF) was investigated as the primary efficacy endpoint; evening PEF, use of salbutamol (albuterol) as rescue medication, and day- and night-time asthma symptom scores were secondary efficacy endpoints. Safety was assessed according to adverse events recorded. Over the 6-week treatment period, salmeterol/fluticasone propionate led to a significantly greater increase in morning (p < 0.0001) and evening (p = 0.0066) PEF compared with budesonide. Moreover, the significant benefit of salmeterol/fluticasone propionate was evident from the first week. Similarly, salmeterol/fluticasone propionate led to significantly greater improvements in the use of rescue medication and day- and night-time asthma symptom scores, compared with budesonide. Both treatments were well tolerated, with a similar incidence (23\%) of adverse events in both treatment groups and no serious adverse events. Salmeterol/fluticasone propionate 50mug /100mug twice daily was significantly more effective than budesonide 400mug twice daily in improving lung function and reducing symptoms and use of rescue medication in Chinese asthmatic patients who were poorly controlled on low-dose inhaled corticosteroids. This confirms the findings of superior efficacy of this combination product over budesonide in other populations. [\hyperlink{Fluticasone Propionate And Salmeterol Xinafoate}{PMID: 17523720}, Nan Shan Zhong et al., 2004]

\hypertarget{pmid_9647268}{S}almeterol xinafoate is a long-acting, highly selective, beta2-adrenergic agonist that produces bronchodilation and clinically significant improvement in pulmonary function for up to 12 hours in patients with asthma. To evaluate the impact on asthma-specific quality of life, efficacy, and safety of salmeterol versus albuterol in adult patients with mild-to-moderate persistent asthma. A randomized, double-blind, double-dummy, parallel-group, multicenter study was conducted in 539 adult asthma patients over 12 weeks. Patients were randomized to receive either salmeterol 42 microg via metered-dose inhaler twice daily or albuterol 180 microg four times daily. Upon entry into the study, 46\% of patients were being treated with an inhaled corticosteroid and were allowed to continue treatment throughout the study. Pulmonary function and asthma symptoms were monitored daily, and patients completed the Asthma Quality of Life Questionnaire (AQLQ) at baseline and after 4, 8, and 12 weeks of treatment. Treatment with salmeterol twice daily produced significantly greater improvements from baseline in all quality of life domain ("Activity Limitation," "Asthma Symptoms," "Emotional Function," "Environmental Exposure") scores and in the global AQLQ score at 12 weeks (P < or = .038) compared with albuterol treatment four times daily. Pulmonary function and asthma symptoms were also significantly improved with salmeterol compared with albuterol. Salmeterol 42 microg administered twice daily is significantly more effective than albuterol 180 microg four times daily for improving asthma-specific quality of life, controlling asthma symptoms, and improving pulmonary function in patients with mild-to-moderate persistent asthma. Furthermore, those improvements were maintained over a 12-week period. [\hyperlink{Fluticasone Propionate And Salmeterol Xinafoate}{PMID: 9647268}, S E Wenzel et al., 1998]

\hypertarget{pmid_26581314}{T}he present paper reviews pharmacokinetics and systemic activity of a new patent of fluticasone fumarate/vilanterol trifenatate and summarises the efficacy data in children, adolescents and adults with asthma. Bioavailability of oral deposition of fluticasone furoate is approximately 1\%, of oral and pulmonary deposition 15\%. Fluticasone furoate 400, 600 and 800 µg have been associated with reductions in 24h urine cortisol excretion in adults, whereas several studies on fluticasone furoate/ vilanterol trifenatate 100/25 µg and 200/25 µg once daily found no suppressive effects. Bronchodilation was detected in adults with asthma from 5 minutes after vilanterol trifenatate was inhaled and up to 24 hours after. Five large clinical trials which were sponsored by the manufacturer GlaxoSmithKline provided evidence that dry powder fluticasone furoate/vilanterol trifenatate 100/25 µg and 200/25 µg once daily are efficacious in asthma in patients ≥ 12 years of age. It remains to be proven, however, that once daily dosing may improve asthma control as compared to twice daily dosing. Efficacy and the systemic activity potential for hypothalamic-pituitary-adrenal and growth suppression of fluticasone furoate have not been established in children. The potential for systemic activity of fluticasone furoate in children may be assessed by knemometry. [\hyperlink{Fluticasone Propionate And Salmeterol Xinafoate}{PMID: 26581314}, Ole D Wolthers et al., 2015]

\hypertarget{pmid_8640153}{T}he investigations were carried out on 13 children with seasonal rhinitis, and in 10 children with perennial rhinitis. The fluticasone propionate (Flixonase-Glaxo) was introduced in acute state of disease: one dose to each nostril (100 micrograms) for the patient 7-11 years old, and in the patients under 11 years old--two doses to each nostril (200 micrograms) in the morning, during 3 weeks. The clinical effects were established using score-system in 0-3 points scale which included essential symptoms like: itching, sneezing, nasal blockade, rhinorrhoea, mucosal oedema and eyes irritation. Very good and good effects of the treatment in patients with seasonal rhinitis in 93\% of children was observed. In the group of patients with perennial rhinitis, very good and good results, was observed in 80\% of children. Easy dosage, high efficacy, very nice smell and no side-effects make this medicine very usefull in the treatment of allergic rhinitis in children. [\hyperlink{Fluticasone Propionate And Salmeterol Xinafoate}{PMID: 8640153}, J Hofman et al., 1995]

\section*{Mometasone Furoate}
\subsection*{Result}
\subsubsection*{Answer}

Ages 2–4: Yes (intranasal/nasal spray, topical)  
Ages 5–11: Yes (intranasal/nasal spray, inhaled, topical)  
Ages 12–17: Yes (intranasal/nasal spray, topical)  

\subsubsection*{{Explanation}}
\hypertarget{Mometasone Furoate}
A review of the available abstracts reveals multiple targeted studies evaluating the safety of mometasone furoate in children across various age ranges and indications (nasal, inhaled, and topical use). Below is a summary by age range and formulation:

Ages 2–11 years (Intranasal/Nasal Spray):
- Several studies specifically evaluated intranasal mometasone furoate in children as young as 2 years old for allergic rhinitis, adenoidal hypertrophy, and sleep-disordered breathing. These studies consistently report that mometasone furoate is well tolerated, with adverse events similar to placebo and no significant systemic effects (e.g., no suppression of cortisol or increase in intraocular pressure) [\hyperlink{pmid_11449207}{PMID: 11449207}, J Dibildox et al., 2001; \hyperlink{pmid_12943488}{PMID: 12943488}, Eric Schenkel et al., 2003; \hyperlink{pmid_18681494}{PMID: 18681494}, Claudine M Baldwin et al., 2008; \hyperlink{pmid_25404102}{PMID: 25404102}, R Bhargava et al., 2014; \hyperlink{pmid_23010795}{PMID: 23010795}, Mohamed A Bitar et al., 2013; \hyperlink{pmid_21461060}{PMID: 21461060}, Yong Gi Jung et al., 2011]. No significant side effects were observed in these pediatric populations.

Ages 5–11 years (Inhaled for Asthma):
- Randomized controlled trials and meta-analyses have evaluated inhaled mometasone furoate (alone or in combination with formoterol) in children aged 5–11 years with asthma. These studies found the drug to be well tolerated, with adverse event rates similar to or lower than comparator groups, and no significant safety concerns [\hyperlink{pmid_32022483}{PMID: 32022483}, Cindy L J Weinstein et al., 2020; \hyperlink{pmid_34849118}{PMID: 34849118}, Yanling Song et al., 2021; \hyperlink{pmid_27740721}{PMID: 27740721}, Niran J Amar et al., 2017].

Ages 12–17 years (Intranasal/Nasal Spray):
- A post hoc analysis of a randomized controlled trial in adolescents (12–17 years) with seasonal allergic rhinitis found that mometasone furoate nasal spray was well tolerated, with adverse event rates similar to placebo [\hyperlink{pmid_19467175}{PMID: 19467175}, Robert Anolik et al.].

Ages 2–12 years (Topical for Skin Conditions):
- A randomized controlled trial in children with atopic dermatitis (ages not specified but described as "children") found that topical mometasone furoate cream was well tolerated, with no significant suppression of plasma cortisol [\hyperlink{pmid_2033138}{PMID: 2033138}, H J Vernon et al., 1991]. Another large study in boys with phimosis (ages not specified but described as "children") found topical mometasone furoate to be effective and safe [\hyperlink{pmid_36389352}{PMID: 36389352}, Guanglun Zhou et al., 2022].

General Pediatric Population:
- Reviews and meta-analyses confirm that mometasone furoate, in various formulations, has been extensively studied in children and is associated with a favorable safety profile, minimal systemic absorption, and no significant effect on growth or hypothalamic-pituitary-adrenal axis function at recommended doses [\hyperlink{pmid_17408308}{PMID: 17408308}, Myron Zitt et al., 2007; \hyperlink{pmid_29411351}{PMID: 29411351}, Fabrizio Spada et al., 2018].

Summary:
There is robust evidence from targeted pediatric studies affirming the safety of mometasone furoate in children as young as 2 years for intranasal use, 5 years for inhaled use, and for topical use in children (with some studies specifying ages 2–12 years). No studies in the abstracts reviewed reported significant safety concerns or adverse systemic effects in these age groups.

\subsection*{Abstracts}
\hypertarget{pmid_9806113}{M}ometasone furoate is a synthetic corticosteroid which has been evaluated for intranasal use in the treatment of adults and children with allergic rhinitis. In several large, well-controlled clinical trials, mometasone furoate 200 micrograms administered once daily as an aqueous intranasal spray was significantly more effective than placebo in controlling the symptoms associated with moderate to severe seasonal or perennial allergic rhinitis. Mometasone furoate was as effective as twice-daily beclomethasone dipropionate or once-daily fluticasone propionate in the treatment of perennial allergic rhinitis, and was as effective as twice-daily beclomethasone dipropionate and slightly more effective than once-daily oral loratadine in the treatment of seasonal allergic rhinitis. Mometasone furoate was also as effective as twice-daily beclomethasone dipropionate or once-daily budesonide, and significantly more effective than placebo in the prophylaxis of seasonal allergic rhinitis. The onset of action of mometasone furoate was approximately 7 hours in patients with seasonal allergic rhinitis. Mometasone furoate was as well tolerated as beclomethasone dipropionate, fluticasone propionate and budesonide in clinical trials, with an overall incidence of adverse events similar to placebo. Adverse events were generally mild to moderate and of limited duration. The most common adverse events associated with mometasone furoate therapy were nasal irritation and/or burning, headache, epistaxis and pharyngitis. Intranasal or oral mometasone furoate had no detectable effect on hypothalamic-pituitary-adrenal axis function in studies of < or = 1 year in duration. Mometasone furoate is a well tolerated intranasal corticosteroid with minimal systemic activity and an onset of action of < or = 7 hours. It is effective in the prophylaxis and treatment of seasonal allergic rhinitis and the treatment of perennial allergic rhinitis in patients with moderate to severe symptoms. [\hyperlink{Mometasone Furoate}{PMID: 9806113}, S V Onrust et al., 1998]

\hypertarget{pmid_18681494}{M}ometasone furoate (Nasonex) is a high-potency intranasal corticosteroid available for the treatment and/or prophylaxis of the nasal symptoms of seasonal allergic rhinitis (SAR) and perennial allergic rhinitis (PAR). In the EU, it is approved for use in patients aged > or =6 years and, in the US, it is approved as a treatment in patients aged > or =2 years and as prophylaxis in those > or =12 years of age.Extensive experience in both clinical trials and the clinical practice setting has firmly established the efficacy and good tolerability profile of intranasal mometasone furoate in children and adults with PAR or SAR. Thus, intranasal mometasone furoate is a useful first-line option for the treatment and prophylactic management of these conditions, including in children as young as 2 years of age in some countries and 6 years of age in others. [\hyperlink{Mometasone Furoate}{PMID: 18681494}, Claudine M Baldwin et al., 2008]

\hypertarget{pmid_32022483}{A}sthma affects over 6 million children in the United States alone. This study investigated the efficacy and long-term safety of mometasone furoate-formoterol (MF/F) and MF monotherapy in children with asthma. This phase 3, multicenter, randomized controlled trial evaluated metered-dose inhaler twice daily (BID) dosing with MF/F 100/10 µg or MF 100 µg in children, aged 5 to 11 years, with a history of asthma for greater than or equal to 6 months and confirmed bronchodilator reversibility, who were adequately controlled on inhaled corticosteroid/long-acting beta-agonist combination therapy for greater than or equal to 4 weeks. After a 2-week run-in on MF 100 µg BID, eligible patients received 24 weeks of double-blind treatment and were followed for safety up to 26 weeks. The primary efficacy endpoint was the change from baseline in AM postdose 60-minute AUC \%predicted FEV1\% across 12 weeks of treatment. A total of 181 participants received at least one dose of MF/F (n = 91) or MF (n = 90). MF/F was superior to MF across the 12-week evaluation period, with a treatment advantage of 5.21 percentage points (P < .001). Superior onset of action with MF/F over MF was achieved as early as 5 minutes postdose on day 1. Overall, approximately 50\% of participants experienced one or more treatment-emergent adverse events, with fewer occurring in the MF/F group. In children 5 to 11 years of age with persistent asthma, the addition of F to MF was well tolerated and provided significant, rapid, and sustained improvement in lung function compared with MF alone. [\hyperlink{Mometasone Furoate}{PMID: 32022483}, Cindy L J Weinstein et al., 2020]

\hypertarget{pmid_24871808}{M}ometasone furoate as a nasal spray is an effective treatment for seasonal allergic rhinitis (SAR). An aqueous mometasone nasal spray containing the same active substance and excipients as the originator product (reference mometasone) has been developed. This study was designed to establish therapeutic equivalence of test mometasone to reference mometasone and superiority over placebo for the treatment of SAR in adults. In this multicenter, randomized, double-blind, placebo- and active-controlled, fixed-dose study, patients aged ≥18 years with SAR were randomized 2:2:1 to reference mometasone, test mometasone, or placebo for 28 days. Patients recorded nasal and ocular symptoms daily. The primary end point was change from baseline in the pooled 24-hour reflective total nasal symptom score (rTNSS). Safety and tolerability included evaluation by adverse events (AEs), physical (including nasal) examinations, vital signs assessments, laboratory evaluations, and change in concomitant medications. Four hundred two patients received reference mometasone (n = 156), test mometasone (n = 163), or placebo (n = 83). The intent-to-treat population (ITT) comprised 399 patients, and the per-protocol (PP) population comprised 327 patients. The 95\% confidence intervals for the treatment difference (reference minus test mometasone) in change from baseline in pooled 24-hour rTNSS were within prespecified equivalence limits for the PP and ITT populations. Both active treatments showed superiority over placebo (p = 0.0019-0.0087). No significant difference was seen between test mometasone and reference mometasone for any secondary efficacy variables. Treatment-emergent AE incidence was low. No deaths or serious AEs were reported. The test mometasone is efficacious in the treatment of SAR in adults and shows a favorable safety profile. The results indicate that the test mometasone is therapeutically equivalent to the reference mometasone. [\hyperlink{Mometasone Furoate}{PMID: 24871808}, Piotr Kuna et al., ]

\hypertarget{pmid_12943488}{M}ometasone furoate aqueous nasal spray (NS; Nasonex, Schering Corporation), is a synthetic corticosteroid approved for the prophylaxis and treatment of seasonal allergic rhinitis (SAR) and the treatment of perennial allergic rhinitis (PAR) in patients >or= 12 years of age, and for the treatment of SAR and PAR in children as young as 2 years of age. Studies demonstrate that mometasone furoate NS is a potent, clinically effective and well-tolerated intranasal corticosteroid with negligible systemic activity and which offers the convenience of once-daily dosing. [\hyperlink{Mometasone Furoate}{PMID: 12943488}, Eric Schenkel et al., 2003]

\hypertarget{pmid_34849118}{T}he influence of mometasone furoate for paediatric asthma remains controversial. We conducted a systematic review and meta-analysis to explore the efficacy and safety of mometasone furoate for paediatric asthma. We have searched PubMed, Embase, Web of science, EBSCO, and Cochrane library databases through October 2019 for randomized controlled trials assessing the effect of mometasone furoate versus placebo for paediatric asthma. This meta-analysis was performed using the random-effects model. Four RCTs were included in the meta-analysis. Overall, as compared to placebo for paediatric asthma, mometasone furoate is associated with substantially increased predicted forced expiratory volume in 1 s (FEV Mometasone furoate may be effective and safe for paediatric asthma. [\hyperlink{Mometasone Furoate}{PMID: 34849118}, Yanling Song et al., 2021]

\hypertarget{pmid_11449207}{I}ntranasal mometasone furoate (MF) has been extensively studied in adults and has been found to be safe and effective therapy for the treatment of allergic rhinitis. Several studies have now been conducted on pediatric patients. In all, 990 pediatric patients given mometasone furoate nasal spray (MFNS) have been studied in phase I, II, and III clinical trials. In a dose-ranging study, 5 doses of nasal spray (25, 100, and 200 microg MFNS daily and 168 microg beclomethasone dipropionate daily) were compared with placebo. The 100- and 200-microg daily doses of MFNS were found to be more effective than 168 microg beclomethasone dipropionate or 25 microg MFNS given daily. MFNS (100 microg once daily) was chosen as the appropriate dose. In clinical efficacy and safety trials, MFNS was given to 381 patients 3 to 11 years of age for 4 weeks (357 patients received 100 microg MFNS daily for 6 months) and was found to decrease symptom scores from baseline significantly better than placebo. The long-term safety of MFNS was also studied in 166 patients treated for one year; no significant changes in intraocular pressure were detected. Cosyntropin stimulation showed no decreases in cortisol. In adults, nasal mucosa showed improvement in appearance of epithelium and reduction of inflammatory infiltrates, and there were no signs of nasal atrophy. [\hyperlink{Mometasone Furoate}{PMID: 11449207}, J Dibildox et al., 2001]

\hypertarget{pmid_36678822}{M}ometasone furoate (MF) is a medium-potency synthetic glucocorticosteroid with anti-inflammatory, antipruritic, and vasoconstrictive properties. However, its role in the treatment of ocular inflammation has not yet been explored. This work investigated the anti-inflammatory activity of MF in ocular tissues. First, the in vivo safety of the intravitreal (IVT) injection of MF (80, 160, and 240 µg) was evaluated via clinical examination (including the assessment of intraocular pressure), electroretinography (ERG), and histopathology. Second, MF was tested in an experimental model of bacillus Calmette-Guérin (BCG)-induced uveitis in Wistar rats. Intraocular inflammation was then evaluated via a slit-lamp and fundus examination, ERG, histopathology, and the quantification of pro-inflammatory markers. Intravitreal MF showed no toxicity in all the investigated doses, with 160 µg leading to attenuated disease progression and improvement in clinical, morphological, and functional parameters. There was a significant reduction in the levels of inflammatory markers (myeloperoxidase, interleukins 6 and 1β, CXCL-1, and tumor necrosis factor-alpha) when compared to the levels in untreated animals. Therefore, MF should be further investigated as a promising drug for the treatment of ocular inflammation. [\hyperlink{Mometasone Furoate}{PMID: 36678822}, Nayara Almeida Lage et al., 2023]

\hypertarget{pmid_17408308}{T}he development of corticosteroids that are delivered directly to the nasal mucosa has alleviated much of the concern about the systemic adverse effects associated with oral corticosteroid therapy. However, given the high potency of these drugs and their widespread use in the treatment of allergic rhinitis, it is important to ensure that intranasal corticosteroids have a favourable benefit-risk ratio. One agent that typifies the systemic safety found in the majority of intranasal corticosteroids is mometasone furoate nasal spray, a potent and effective treatment for seasonal and perennial allergic rhinitis and nasal polyposis. Mometasone furoate does not reach high systemic concentrations or cause clinically significant adverse effects. Results from pharmacokinetic studies in adults and children suggest that systemic exposure to mometasone furoate after intranasal administration is negligible. This is probably because of the inherently low aqueous solubility of mometasone furoate, which allows only a small fraction of the drug to cross the nasal mucosa and enter the bloodstream, and because a large amount of the administered drug is swallowed and undergoes extensive first-pass metabolism. There is no clinical evidence that mometasone furoate nasal spray suppresses the function of the hypothalamus-pituitary-adrenal axis when the drug is administered at clinically relevant doses (100-200 microg/day); consequently, mometasone furoate nasal spray has not been associated with growth inhibition in children. The safety and tolerability of mometasone furoate nasal spray have been rigorously assessed in clinical trials involving approximately 4,500 patients, with epistaxis, headache and pharyngitis being the most common adverse effects associated with treatment in adolescents and adults. The clinical effectiveness of mometasone furoate nasal spray, coupled with its agreeable safety and tolerability profile, confirms its favourable benefit-risk ratio. [\hyperlink{Mometasone Furoate}{PMID: 17408308}, Myron Zitt et al., 2007]

\hypertarget{pmid_25404102}{T}o study the role of mometasone furoate aqueous nasal spray for the management of adenoidal hypertrophy in children with more than 50 per cent obstruction, and to assess its impact on change in quality of life. A prospective, randomised, double-blind, interventional placebo-controlled study was conducted. A total of 100 children aged 2-12 years completed treatment and follow up. The symptoms and degree of obstruction were evaluated by nasopharyngoscopy conducted pre-treatment and 24 weeks post-treatment. Subjects received mometasone furoate nasal spray at a daily dose of 200 µg for 8 weeks, followed by a dose of 200 µg on alternate days for 16 weeks. RESULTS were compared with those of a matched control group who were given saline nasal spray. With mometasone treatment, there was an 89.8 per cent reduction in clinical symptom score, and the degree of obstruction dropped from 87 to 72 per cent (p < 0.0001). A statistically significant change in quality of life scores was seen in patients treated with the mometasone nasal spray (score change of 37.47) as compared with those given saline nasal spray (score change of 11.25) (p = 0.0001). Mometasone nasal spray appears to be effective in treating children with obstructive adenoids. [\hyperlink{Mometasone Furoate}{PMID: 25404102}, R Bhargava et al., 2014]

\hypertarget{pmid_27925610}{M}ometasone furoate (MMF) is a modern glucocorticoid of the 4th generation, which has been proven not only for inhalation but also for cutaneous treatment. Due to its lipophilic character, it is mainly used in ointments and creams with an outer lipophilic phase (W/O type). However, this study investigated the cutaneous cytotoxicology of MMF and tried to characterize its pharmacokinetic effects on the skin using an O/W preparation. An HPLC method has been developed and validated for the detection of MMF in cutaneous tissue, and concentration-time curves of MMF were created after cutaneous application on unaffected as well as lesional skin. Cytotoxicological characterization was carried out using scratch assays on keratinocytes and cutaneous fibroblasts. Results showed that the condition of the skin had no significant impact on the cutaneous bioavailability of MMF, but the intrinsic effect of the O/W vehicle could be utilized in periods of acute inflammation. Cytotoxicological data gave no new indications regarding the safety of MMF. [\hyperlink{Mometasone Furoate}{PMID: 27925610}, Johannes Wohlrab et al., 2016]

\hypertarget{pmid_3058398}{M}ometasone furoate (Elocon) is a newly formulated and unique medium-potency synthetic 17-heterocyclic corticosteroid. The efficacy and safety of the ointment and cream formulations (0.1 percent) of the corticosteroid, administered once daily, were compared with those of the ointment and cream formulations of fluocinolone acetonide 0.025 percent administered three times daily and triamcinolone acetonide 0.1 percent administered twice daily in four multicenter clinical studies. They were conducted involving psoriasis patients with chronic and moderate to severe disease. Evaluation of change in disease sign scores indicated that mometasone ointment, applied once daily, was significantly more effective (P less than 0.01) than fluocinolone ointment, applied three times daily, and triamcinolone ointment, applied twice daily. The cream formulation of mometasone was significantly more effective (p less than 0.001) than fluocinolone cream, applied three times daily, and equivalent to triamcinolone cream, applied twice daily. The incidence of local adverse experiences following treatment with the ointment or cream formulations of mometasone was minimal. Mometasone ointment and cream provide a highly effective once-a-day treatment for moderate to severe psoriasis with minimal risk of side effects. [\hyperlink{Mometasone Furoate}{PMID: 3058398}, R S Medansky et al., 1988]

\hypertarget{pmid_21461060}{T}o evaluate efficacy of short term intranasal corticosteroid (mometasone furoate) treatment in pediatric sleep-disordered breathing (SDB) patients. A prospective, observational study was done. A total of 41 children (2-11 years old) were enrolled into this study. All patients received 4-weeks course of mometasone furoate 100 µg/day treatment. They were evaluated at pretreatment and immediately after treatment with obstructive sleep apnea (OSA)-18 quality of life survey and lateral neck X-ray. Also, the assessment of each patients included history, skin prick test or CAP test, and sinus radiography. We compared the OSA-18 survey score and adenoidal-nasopharyngeal (AN) ratio between before and after treatment. Total OSA-18 score and AN ratio decreased significantly after treatment regardless of allergy, sinusitis, and obesity (P=0.003, P=0.006). There was no complication after treatment of mometasone furoate. Pediatric SDB patients with adenoid hypertrophy could be effectively treated with 4-weeks course of mometasone furoate. Allergy, obesity, and sinusitis did not affect on the result of treatment. [\hyperlink{Mometasone Furoate}{PMID: 21461060}, Yong Gi Jung et al., 2011]

\hypertarget{pmid_19618993}{M}ometasone furoate has been available for clinical use, starting with a dermatologic preparation, for nearly 20 years. An inhaled format of the drug for management of asthma had been in development during the last decade and has been available for clinical use for 6 years as a dry powder inhaler delivering either 100 mcg or 200 mcg per dose. It has a long half-life and is suitable for daily dosing. The drug is approved for use in the USA for the treatment of asthma in patients aged 4 years or over. Mometasone furoate is a topically potent glucocorticoid with a favorable risk-benefit profile. A wide variety of randomized clinical trials have shown the drug to have a clinically beneficial effect on asthma comparable to fluticasone propionate, and to permit the reduction or withdrawal of oral glucocorticoid therapy in patients with asthma. Mometasone furoate has approximately 1\% oral bioavailability but does produce systemic glucocorticoid effects from the drug released from the lung and its metabolites. These effects are minimal when mometasone is used appropriately at low or moderate doses. [\hyperlink{Mometasone Furoate}{PMID: 19618993}, Robert L Cowie et al., 2009]

\hypertarget{pmid_9860036}{M}ometasone furoate is a potent glucocorticoid that can markedly inhibit proinflammatory Th2 cytokines in vitro. An aqueous nasal spray formulation has been shown to be clinically active in reducing the symptoms of perennial and seasonal allergic rhinitis. To determine whether pretreatment with mometasone furoate 200 microg once daily decreases specific indices of early and late phase nasal inflammation compared with placebo. A randomized, double-blind, placebo-controlled crossover study was conducted using nasal provocation with ragweed antigen in 21 patients with ragweed-induced allergic rhinitis out of the ragweed season; the treatment period was 2 weeks. Symptom scores, rhinoprobe cytology, and nasal lavage fluid were collected during early and late phase periods for nasal cytokines (interleukin, 1, 4, 5, 6, and 8) and leukotriene B4 determinations using ELISA and RIA. Mean nasal symptom scores and sneezing frequency were consistently lower with mometasone furoate compared with placebo. Treatment was associated with a statistically significant early phase (30-minute time point) reduction in nasal lavage histamine levels compared with placebo (14.3 versus 20.2 ng/mL, P = .02). Within-treatment comparisons suggested that mometasone furoate reduced the antigen-induced late-phase response for IL-6, IL-8, and eosinophils compared with pretreatment. There were similar, but smaller, changes seen in the placebo group for these measurements. There were no statistically significant changes following antigen challenge in IL-1, IL-4, IL-5, LTB4, or in other nasal cytology parameters. These results suggest that the clinical activity of mometasone furoate nasal spray in seasonal allergic rhinitis is likely due, in part, to a reduction in the levels of histamine in nasal secretions related to the early phase response, and reductions in IL-6, IL-8, and eosinophils during the late phase response. [\hyperlink{Mometasone Furoate}{PMID: 9860036}, M Frieri et al., 1998]

\hypertarget{pmid_10825787}{T}he objective of the study was to valuation if the use of topical Mometasone Furoate for the treatment of rhinitis provokes an elevation of the intraocular pressure. It was comparative, double blind, experimental, prospective and longitudinal. To measure the intraocular pressure of the patients at the third week, sixth week, three month, six month and one year. There were some modifications in the intraocular pressure, but without exceeding the normal parameters. Mometasone Furoate is safe and does not cause an increment in the intraocular pressure. [\hyperlink{Mometasone Furoate}{PMID: 10825787}, D Bross Soriano et al., ]

\hypertarget{pmid_19874229}{M}ometasone furoate (MF) is a topical glucocorticoid used for atopic dermatitis, allergic rhinitis, and bronchial asthma. To elucidate the usefulness of MF, the dissociation between local anti-inflammatory effects and systemic effects of MF was compared with that of beclomethasone 17,21-dipropionate (BDP). MF was more potent than BDP in croton oil-induced ear edema tests in mice. Oral systemic effects of MF were inversely lower than that of BDP on thymolysis, plasma corticosterone lowering, and suppression of body weight gain in mice. These results indicate that MF has a higher therapeutic index and superior clinical usefulness as a topical glucocorticoid compared to BDP. [\hyperlink{Mometasone Furoate}{PMID: 19874229}, Masami Ogawa et al., 2009]

\hypertarget{pmid_19467175}{S}easonal allergic rhinitis (SAR) is common in adolescents. However, few studies have investigated the effectiveness of intranasal corticosteroids (INSs) for nasal and ocular symptoms of SAR solely in adolescents. The purpose of this study was to determine the safety and efficacy of the INS mometasone furoate nasal spray (MFNS) in adolescents; a post hoc analysis was conducted of adolescents who had participated in a study with adults. Data were analyzed retrospectively for subjects aged 12-17 years with moderate or severe SAR randomized to mometasone furoate, 200 mcg once daily (n = 86), or placebo (n = 82) for 15 days in a multicenter, double-blind, placebo-controlled study. Symptom scores (0 = none to 3 = severe) were recorded in diaries twice daily. End points included changes from baseline in total nasal symptom score (TNSS), individual nasal symptom score (rhinorrhea, congestion, itching, and sneezing), and total ocular symptom score (TOSS). Over 15 days, a significantly greater decrease from baseline in mean TNSS was observed in subjects receiving mometasone furoate (-2.47; -28.8\%) compared with those receiving placebo (-0.9; -9.6\%; p < 0.001). Significant improvement versus placebo was seen for each full day of treatment. Mometasone furoate significantly improved individual nasal symptoms (p < or = 0.03) and TOSS (p = 0.011) versus placebo. The incidence of adverse events was similar for both treatment groups. MFNS, 200 mcg once daily, is an effective and well-tolerated treatment for symptoms of SAR in adolescents. [\hyperlink{Mometasone Furoate}{PMID: 19467175}, Robert Anolik et al., ]

\hypertarget{pmid_23010795}{T}his study aimed at observing the efficacy of mometasone fuorate monohydrate nasal spray on obstructive adenoids in children and identifying the characteristics of responders using a pilot study including children aged 2-11 years, with evidence of more than 50 \% obstruction. Allergic rhinitis and nasal obstruction were evaluated on baseline (V0), 6- (V1), and 12-week (V2) visits. Degree of obstruction was evaluated by nasopharyngoscopy at V0 and V2. Subjects received 100 μg mometasone fuorate daily. Results were compared with those of a matching control group. Nineteen children (8 females, 11 males; 2.25-8.50 years old, mean 4.24 years, median 4.00 years) completed treatment and follow-up adequately. There was 58 \% reduction in a clinical score assessing the severity of adenoidal obstruction (P < 0.05), 56 \% reduction in severity of obstructive symptom (P < 0.05), and 75 \% reduction in allergic rhinitis score (P < 0.05) between V0 and V1. No further significant improvement was noticed between V1 and V2. The degree of obstruction dropped from 85 to 61 \% as noted on endoscopy (P < 0.05). None in the control group showed spontaneous decrease or resolution of the symptoms. Age of patients, allergic rhinitis score, and severity of the clinical score had no impact on the response parameters. No side effects were observed. Mometasone furoate monohydrate nasal spray appears to be effective in treating children with obstructive adenoids. The effect seems to be independent of the presence of mild intermittent allergic rhinitis, the age of patient, or the severity of symptoms. [\hyperlink{Mometasone Furoate}{PMID: 23010795}, Mohamed A Bitar et al., 2013]

\hypertarget{pmid_27740721}{M}ometasone furoate (MF), delivered via dry-powder inhaler (DPI) QD in the evening (PM), is a treatment option for pediatric patients with asthma. We evaluated MF delivered via a metered-dose inhaler (MDI), in children ages 5-11 years with persistent asthma. This was a 12-week double-blind, double-dummy, placebo-controlled trial. Pateints were randomized to the following treatments: MF-MDI 50 mcg BID, MF-MDI 100 mcg BID, MF-MDI 200 mcg BID, MF-DPI 100 mcg QD PM, and placebo. The primary analysis assessed MF-MDI doses versus placebo, on the change in \%-predicted forced expiratory volume in one second (FEV For change from baseline in \%-predicted FEV In children ages 5-11 years with persistent asthma, all three doses of MF-MDI (50, 100, and 200 mcg BID) demonstrated significant improvement in FEV [\hyperlink{Mometasone Furoate}{PMID: 27740721}, Niran J Amar et al., 2017] Derivatives of hydrocortisone, such as mometasone furoate, a (2') furoate-17 ester with chlorine substitutions at positions 9 and 21, have been designed to improve efficacy and reduce the incidence of adverse effects. An extensive literature search of MEDLINE, Embase and other databases was conducted to review the safety and efficacy of various formulations of topical mometasone furoate. Mometasone furoate exhibits high potency with greater anti-inflammatory activity and a longer duration of action than betamethasone. In clinical trials, mometasone furoate shows comparable or significantly better efficacy, depending on the comparator, in all indications studied in both adults and children. It is well tolerated with only transient, mild to moderate local adverse effects. It is characterised by low systemic availability due to its high lipophilicity, low percutaneous absorption and rapid hepatic biotransformation, and consequently has no significant effect on the hypothalamic-pituitary-adrenal axis. The molecular biotransformation of mometasone furoate in the skin results in a lower affinity with dermal cells than epidermal cells, which contributes to its low atrophogenicity. Sensitisation to mometasone furoate is low. Overall, mometasone furoate is a highly efficacious potent corticosteroid with a low risk of both local and systemic adverse effects. [\hyperlink{Mometasone Furoate}{PMID: 27740721}, Fabrizio Spada et al., 2018]

\hypertarget{pmid_2033138}{W}e conducted a 6-week randomized, blinded study that compared mometasone furoate 0.1\% cream, applied once daily, and hydrocortisone 1.0\% cream, applied twice daily, in 48 children with moderate to severe atopic dermatitis. Mometasone furoate, a moderate-potency steroid, produced significantly greater improvement than the low-potency hydrocortisone used twice daily. The difference in therapeutic response was particularly evident in patients with involvement of more than 25\% of their body surface area. Morning plasma cortisol levels were assayed before treatment, after 1 week of therapy, and at the end of the clinical trial. Plasma cortisol levels were transiently suppressed in one child who was treated with hydrocortisone and in none of the children treated with mometasone. [\hyperlink{Mometasone Furoate}{PMID: 2033138}, H J Vernon et al., 1991]

\hypertarget{pmid_9357385}{M}ometasone furoate (Nasonex), in a new once-daily aqueous nasal spray formulation, has been shown to be as effective and well-tolerated as twice-daily beclomethasone dipropionate aqueous nasal spray in treating symptoms of seasonal allergic rhinitis and perennial rhinitis. To compare the effectiveness and tolerability of mometasone furoate to placebo and of fluticasone propionate aqueous nasal spray, all treatments administered once-daily, in patients with perennial rhinitis. This was a 3-month, randomized, double-blind, double dummy, parallel group study in 550 patients, aged 12 to 77 years, at 25 centers in Canada, Latin America, and Europe. Patients allergic to at least one perennial allergen, with confirmed allergy history, skin test positivity, and moderate to severe symptomatology, were eligible to receive one of the following treatments, once daily in the morning: mometasone furoate 200 micrograms, fluticasone propionate 200 micrograms, or placebo. The primary efficacy variable was the change from baseline in total AM plus PM diary nasal symptom score over the first 15 days of treatment. Four hundred fifty-nine patients were valid for efficacy. For the primary efficacy variable, mometasone furoate was significantly (P < .01) more effective than placebo and was not statistically different from fluticasone propionate (percent reductions from baseline were 37, 39, and 22 for mometasone furoate, fluticasone propionate, and placebo, respectively). Generally, similar trends were seen for physician-evaluated total nasal symptoms, and patient-rated and physician-rated overall condition and response to therapy. Overall, mometasone furoate was at least as effective as fluticasone propionate at equivalent doses. There was no evidence of tachyphylaxis. All treatments were well tolerated. Mometasone furoate and fluticasone propionate adequately controlled symptoms of perennial rhinitis and were well tolerated. [\hyperlink{Mometasone Furoate}{PMID: 9357385}, M Mandl et al., 1997]

\hypertarget{pmid_36389352}{T}wice daily 0.1\% mometasone furoate is an effective treatment for phimosis in children. However, mometasone furoate has an important therapeutic advantage because it is effective in once-daily applications. This study was to compare the efficacy of two different topical 0.1\% mometasone furoate regimens for the treatment of symptomatic severe phimosis in pediatric patients. A total of 1,689 patients with symptomatic severe phimosis classified by the Kikiros system were prospectively enrolled in the study from March 2018 to February 2021. A total of 855 patients received 0.1\% mometasone furoate twice-daily (BID group) and 834 patients received 0.1\% mometasone furoate once-daily (QD group) for 4 weeks. A total of 1,595 boys completed the treatment (798 and 797 in the BID and QD groups, respectively). The success rate of the BID group was higher than that of the QD group at the end of week 2 (44.8\% vs. 33.3\%,  Topical application of 0.1\% mometasone furoate once-daily or twice-daily for 4 weeks had comparable efficacy in children with symptomatic severe phimosis. A once a day regimen may be more suitable for children. Topical steroid application is more effective in children with low-grade phimosis than those with high-grade phimosis. [\hyperlink{Mometasone Furoate}{PMID: 36389352}, Guanglun Zhou et al., 2022]

\hypertarget{pmid_9305231}{M}ometasone furoate (Nasonex), in a new once-daily aqueous nasal spray formulation, has been shown to be as effective and well-tolerated as twice-daily beclomethasone dipropionate aqueous nasal spray in treating symptoms of seasonal allergic rhinitis and perennial rhinitis. To compare the effectiveness and tolerability of mometasone furoate to placebo and to fluticasone propionate aqueous nasal spray, all treatments administered once-daily, in patients with perennial rhinitis. This was a 3-month, randomized, double-blind, double dummy, parallel group study in 550 patients, aged 12 to 77 years, at 25 centers in Canada, Latin America, and Europe. Patients allergic to at least one perennial allergen, with confirmed allergy history, skin test positivity, and moderate to severe symptomatology, were eligible to receive one of the following treatments, once daily in the morning: mometasone furoate 200 micrograms, fluticasone propionate 200 micrograms, or placebo. The primary efficacy variable was the change from baseline in total AM plus PM diary nasal symptom score over the first 15 days of treatment. Four hundred fifty-nine patients were valid for efficacy. For the primary efficacy variable, mometasone furoate was significantly (P < .01) more effective than placebo and was not statistically different from fluticasone propionate (percent reductions from baseline were 37, 39, and 22 for mometasone furoate, fluticasone propionate, and placebo, respectively). Generally, similar trends were seen for physician-evaluated total nasal symptoms, and patient-rated and physician-rated overall condition and response to therapy. Overall, mometasone furoate was at least as effective as fluticasone propionate at equivalent doses. There was no evidence of tachyphylaxis. All treatments were well tolerated. Mometasone furoate and fluticasone propionate adequately controlled symptoms of perennial rhinitis and were well tolerated. [\hyperlink{Mometasone Furoate}{PMID: 9305231}, M Mandl et al., 1997]

\section*{Hydrocortisone Acetate And Pramoxine Hydrochloride}
\subsection*{Result}
\subsubsection*{Answer}

Unknown

\subsubsection*{{Explanation}}
\hypertarget{Hydrocortisone Acetate And Pramoxine Hydrochloride}
A review of the available abstracts reveals the following:

- There are several studies on the use of hydrocortisone (including hydrocortisone acetate) in children, covering various indications such as congenital adrenal hyperplasia, atopic dermatitis, and otitis externa. These studies generally affirm the safety of hydrocortisone in pediatric populations, sometimes down to infancy, when used as monotherapy or in combination with other agents (e.g., antibiotics in ear drops) [\hyperlink{pmid_32888021}{PMID: 32888021}, Uta Neumann et al., 2021; \hyperlink{pmid_7601950}{PMID: 7601950}, J Jorizzo et al., 1995; \hyperlink{pmid_9366699}{PMID: 9366699}, R N Jones et al., 1997].

- There are studies on pramoxine hydrochloride, but the only relevant abstract is a double-blind randomized controlled trial comparing pramoxine hydrochloride 1\% and hydrocortisone acetate 1\% in a mucoadhesive foam base for episiotomy discomfort. This study was conducted in postpartum women, not children [\hyperlink{pmid_6397856}{PMID: 6397856}, I A Greer et al., 1984].

- One study evaluated a topical combination of hydrocortisone acetate 2.5\% and pramoxine hydrochloride 1\% for pruritus, but this was conducted only in adults (subjects age 18 and older) [\hyperlink{pmid_21386957}{PMID: 21386957}, Leon H Kircik et al., 2011].

- No abstract describes a targeted safety study of the combination of hydrocortisone acetate and pramoxine hydrochloride in children of any age group.

Therefore, based strictly on the abstracts available, there is no evidence from targeted pediatric safety studies for the combination of hydrocortisone acetate and pramoxine hydrochloride. The safety of this combination in children is unknown.

\subsection*{Abstracts}
\hypertarget{pmid_36417367}{T}o assess the 5-year neurocognitive outcomes of children born extremely preterm exposed to prophylactic hydrocortisone to improve survival without bronchopulmonary dysplasia. This was a prespecified secondary analysis of the PREMILOC clinical trial (trial registration: EudraCT no. 2007-002041-20, NCT00623740). The primary outcome was full-scale IQ based on the Wechsler Preschool and Primary Scale of Intelligence. Among 109 surviving children recruited at the Robert Debré Children's Hospital, Paris, outcome data were available for 42 out of 56 infants (75\%) in the group treated with hydrocortisone and 41 out of 53 (77\%) in the placebo group. Mean scores were not significantly different between the two groups on full-scale IQ (hydrocortisone: 91.9 [SD = 13.9], placebo: 86.3 [SD = 15.4]; mean difference = 5.7, 95\% confidence interval [CI] = -1.0 to 12.3, p = 0.10); however, working memory and retention ability were significantly better in the group treated with hydrocortisone. In a multivariate logistic regression including potential confounding variables, hydrocortisone treatment was significantly associated with a greater chance to survive at 5 years of age with a full-scale IQ equal to or greater than 90 compared to placebo (adjusted odds ratio = 4.26, 95\% CI = 1.47-12.36, p = 0.008). This exploratory analysis provides reassuring data regarding the long-term neurodevelopmental safety of prophylactic hydrocortisone in infants born extremely preterm. [\hyperlink{Hydrocortisone Acetate And Pramoxine Hydrochloride}{PMID: 36417367}, Clémence Trousson et al., 2023]

\hypertarget{pmid_35153845}{W}e identified the first-generation antihistamine hydroxyzine as the earliest and most frequently prescribed drug affecting the central nervous system in children under the age of 5 years in the province of British Columbia, Canada (1. 1\% prevalence). Whereas, the antagonism of H1-receptors exerts anti-pruritic effects in atopic dermatitis and diaper rash, animal studies suggest an adverse association between reduced neurotransmission of histamine and psychomotor behavior. In order to investigate hydroxyzine safety, we characterized the longitudinal patterns of hydroxyzine use in children under the age of 5 years and determined mental- and psychomotor disorders up to the age of 10 years. We found significantly higher rates of ICD-9 and ICD-10 codes for disorders such as tics (307), anxiety (300) and disturbance of conduct (312) in frequent users of hydroxyzine. Specifically, repeat prescriptions of hydroxyzine compared to a single prescription show an increase in tic disorder, anxiety and disturbance of conduct by odds ratios of: 1.55 (95\%CI: 1.23-1.96); 1.34 (95\%CI: 1.05-1.70); and 1.34 (95\%CI: 1.08-1.66) respectively in children up to the age of 10 years. Furthermore, a non-significant increased trend was found for ADHD (314) and disturbance of emotions (313). This is the first study reporting an association between long-term neurodevelopmental adverse effects and early use of hydroxyzine. Controlled studies are required in order to prove a causal relationship and to confirm the safety of hydroxyzine in the pediatric population. For the time being, we suggest the shortest possible duration for hydroxyzine use in preschool-age children. [\hyperlink{Hydrocortisone Acetate And Pramoxine Hydrochloride}{PMID: 35153845}, Hans J Gober et al., 2021]

\hypertarget{pmid_6397856}{A} double-blind randomised controlled trial, comparing pramoxine hydrochloride 1 per cent and hydrocortisone acetate 1 per cent in a mucoadhesive foam base, with simple aqueous foam (B.P.), in relieving episiotomy discomfort and episiotomy healing in 40 patients was carried out. Simple aqueous foam was more effective with regard to wound healing and episiotomy discomfort as measured by analgesic consumption. Pramoxine and hydrocortisone foam offers no advantage over simple aqueous foam in the treatment of post partum episiotomy discomfort. [\hyperlink{Hydrocortisone Acetate And Pramoxine Hydrochloride}{PMID: 6397856}, I A Greer et al., 1984]

\hypertarget{pmid_25692259}{P}ediatric shock is associated with significant morbidity and limited evidence suggests treatment with corticosteroids. The objective of this study was to describe practice patterns and outcomes associated with corticosteroid use in children with shock. We conducted a retrospective, cohort study in four pediatric intensive care units (PICU) in Canada. Patients aged newborn to 17 years admitted to PICU with shock between January 2010 and June 2011 were eligible. 364 patients were included. The frequency of hydrocortisone administration was 22.3\% overall (95\% CI: 18.0, 26.5) and 59.4\% in patients who received at least 60 cc/kg of fluid and were on two or more vasoactive agents. Patients administered hydrocortisone had higher PRISM scores (19, IQR 11-24 versus 9, IQR 5-16; P < 0.0001), higher inotrope scores (15, IQR 10-25 versus 7.5, IQR 3.3-10.6, P < 0.0001) and were more likely to have received 60 cc/kg of fluid resuscitation (59.3\% versus 33.6\%, OR 2.88, 95\% CI: 2.09, 3.96). In an adjusted analysis, patients who received hydrocortisone spent more time on vasoactive infusions (64 versus 34  hours, hazard ratio 0.72, 95\% CI: 0.62, 0.84) and had a higher incidence of positive cultures between day 4 and day 28 post admission (24.7\% versus 14.5\%, OR 1.79, 95\% CI: 1.58, 2.04). Hydrocortisone administration was associated with longer time on vasopressors and increased incidence of positive cultures even after correcting for illness severity. Caution should be exercised in administering hydrocortisone for shock until there is clear evidence for benefit in this patient population. [\hyperlink{Hydrocortisone Acetate And Pramoxine Hydrochloride}{PMID: 25692259}, Kusum Menon et al., 2015]

\hypertarget{pmid_20040824}{T}o assess the long-term safety and tolerability of atomoxetine hydrochloride in children and adolescents with attention-deficit/hyperactivity disorder treated for > or = 3 years. Data from 13 double-blind, placebo-controlled trials and 3 open-label extension studies were pooled. Outcome measures were patient-reported treatment-emergent adverse events (AEs); discontinuations due to AEs, serious AEs, and changes in body weight, height, vital signs, electrocardiogram, and hepatic function tests. In total, 714 patients were treated with atomoxetine for > or = 3 years (mean follow-up 4.8 years [SD 1.1 years]), including a subset of 508 treated for > or = 4 years (mean follow-up 5.3 years [SD 0.8 years]). Most subjects were younger than 12 years at entry (73.8\%), male (78.4\%), and white (88.9\%). The mean final daily dose of atomoxetine was 1.35 mg/kg (SD 0.37 mg/kg). No new or unexpected AEs were observed compared with acute-phase treatment. Less than 6\% of patients exhibited aggressive/hostile behaviors, and less than 1.6\% reported suicidal ideation/behavior. No clinically significant effects were seen on growth rate, vital signs, or electrocardiographic parameters, and < or = 2\% of patients showed potentially clinically significant hepatic changes. Atomoxetine was safe and well tolerated for children and adolescents with > or = 3 and/or > or = 4 years of treatment. [\hyperlink{Hydrocortisone Acetate And Pramoxine Hydrochloride}{PMID: 20040824}, Craig Donnelly et al., 2009]

\hypertarget{pmid_21386957}{I}tch is the most common symptom among patients presenting to the dermatology clinic. Scratching can cause mechanical trauma to the skin, further damaging the epidermal barrier and its function. This damage can facilitate the introduction of microbes that complicate the presenting disease and its management. Pruritus has a negative influence on quality of life. Initiation of treatment that can safely and effectively manage pruritus may provide immediate benefits to the patient. A novel topical formulation of hydrocortisone acetate 2.5\% and pramoxine hydrochloride 1\% in a hydrophilic lotion base is indicated for the management of pruritus. However, the rate of onset of antipruritic effects has not been well studied. This single-center, open-label, pilot study involved 11 subjects age 18 and older. All subjects applied hydrocortisone acetate 2.5\% and pramoxine hydrochloride 1\% lotion four times daily for one day. Severity of itch as measured by the visual analog scale decreased significantly following one day of medication use. The change in mean visual analog scale from baseline was -2.16±2.78 (P=0.0275), representing a mean percentage reduction of 31.74±42.11 (P=0.0315). Topical application of hydrocortisone acetate 2.5\% and pramoxine hydrochloride 1\% lotion provides a significant reduction in pruritus as rated by patients using the visual analog scale with a single day of use. Early onset of action to decrease itch is expected to improve the patient's treatment experience and increase the level of long-term adherence. [\hyperlink{Hydrocortisone Acetate And Pramoxine Hydrochloride}{PMID: 21386957}, Leon H Kircik et al., 2011]

\hypertarget{pmid_20527137}{O}nly a few corticosteroids for topical use have proven safe and effective in pediatric populations down to 3 months of age. The authors report the results of a study designed to assess the efficacy and safety of hydrocortisone butyrate (HCB) 0.1\% in lipocream (LCr) vehicle in infants and children. A total of 264 boys and girls 3 months to less than 18 years old, with stable, mild to moderate atopic dermatitis affecting at least 10\% body surface area applied HCB 0.1\% in LCr or LCr alone twice daily for up to 1 month without occlusion. Primary end-points included: percent of patients who achieved treatment success based on physician global assessments. Secondary endpoint included: difference in pruritus and Eczema Area and Severity Index (EASI) at day 29. Treatment was significant (P < 0.001) for HCB 0.1\% LCr over vehicle. No serious nor significant adverse events were reported. Results are representative of a short duration treatment for a chronic disease. HCB 0.1\% in LCr is more effective than its vehicle in pediatric populations down to 3 months of age without significant adverse events when used twice a day for up to 1 month. [\hyperlink{Hydrocortisone Acetate And Pramoxine Hydrochloride}{PMID: 20527137}, William Abramovits et al., ]

\hypertarget{pmid_19101215}{D}espite modern perinatal intensive care techniques, chronic lung disease remains a problem in preterm-born infants. The most commonly and almost exclusively prescribed drug to treat this disorder is dexamethasone. Corticosteroids improve short-term respiratory function; however, many side-effects have been reported and the adverse long-term effects of dexamethasone on neurodevelopment are particularly alarming. Hydrocortisone could be a suitable alternative for dexamethasone, if equally effective with fewer side-effects. This review evaluates the current literature on neonatal hydrocortisone treatment for chronic lung disease with regards to long-term neurodevelopmental outcome and cardiovascular effects. The neurodevelopmental studies do not show any adverse effects of hydrocortisone on neurocognitive and motor outcome, nor on incidence of brain abnormalities on magnetic resonance imaging or on long-lasting programming effects on the hypothalamus-pituitary-adrenal axis. At school age, cardiovascular stress response was the same in hydrocortisone-treated children compared with a reference group. Hydrocortisone seems a safe alternative to dexamethasone, but more double-blind randomised studies are needed. [\hyperlink{Hydrocortisone Acetate And Pramoxine Hydrochloride}{PMID: 19101215}, Karin J Rademaker et al., 2009]

\hypertarget{pmid_8329789}{T}o report the first five cases of amphotericin B overdose with secondary cardiac complications in a pediatric population. Treatment is also presented. Hospital. Two infants and three children inpatients receiving amphotericin B. Cardiac complications were observed in five pediatric patients who received between 4.6 and 40.8 mg/kg/d of amphotericin B. Cardiac arrest occurred in all patients, and four patients died. A detailed description of the cardiac event is provided for one patient who was on a cardiac monitor during the adverse reaction. Hydrocortisone prophylaxis and verapamil therapy were the primary therapies used in patient 1 (the only survivor). Evaluation of the literature provides substantial evidence for the use of hydrocortisone in prevention of cardiac arrhythmias. Amphotericin B overdose can be fatal in children and infants. The presentation in humans appears similar to that in dogs where cardiac arrhythmias occurred at doses of 5-15 mg/kg. Hydrocortisone may decrease the incidence of mortality associated with cardiac arrhythmias in children receiving amphotericin B overdoses. Animal studies are necessary to evaluate this observation and potential disadvantages of hydrocortisone usage. [\hyperlink{Hydrocortisone Acetate And Pramoxine Hydrochloride}{PMID: 8329789}, J D Cleary et al., 1993]

\hypertarget{pmid_26775450}{E}xtemporaneously prepared liquid dosage forms are needed to administer required medications in infants and young children. The goal of this study was to evaluate the stability of levothyroxine, doxycycline, hydrocortisone, and pravastatin in extemporaneously prepared suspensions stored in plastic prescription bottles under refrigeration and room temperature. Levothyroxine (25 mcg/mL), doxycycline (5 mg/mL), hydrocortisone (2 mg/mL), and pravastatin (10 mg/mL) were each prepared in two groups of suspensions. All of these suspensions were stored in plastic prescription bottles under refrigeration and at room temperature. Levothyroxine was stable for two weeks at 4°C but only one week at 25°C in both suspensions. Doxycycline was stable for two weeks in both suspensions at 4°C and 25°C. Hydrocortisone was stable for the entire two-week study period in both suspensions at both 4°C and 25°C. Pravastatin was stable for the one-week study period in both suspensions at both 4°C and 25°C. These results can be used to offer age-appropriate extemporaneously prepared medications to infants and young children when no suitable commercially available liquid formulations are available. [\hyperlink{Hydrocortisone Acetate And Pramoxine Hydrochloride}{PMID: 26775450}, Milap C Nahata et al., ]

\hypertarget{pmid_18345402}{H}ydrocortisone acetate is usually employed in the treatment of classic congenital adrenal hyperplasia (CAH) due to 21-hydroxylase deficiency. In Brazil, however, oral hydrocortisone acetate is only available from manipulation pharmacies. Prednisolone has stable oral pharmaceutical formulations commercially available, with the advantage of a single daily dose. The aim of this study was to compare the efficacy of oral prednisolone and oral hydrocortisone in the treatment of CAH due to 21-hydroxylase deficiency. Fifteen patients with mean (SD) chronological age of 7.2 (3.6) years, were evaluated in two consecutive 1-year periods. In the first year, hydrocortisone (17.5 mg/m2/day, divided in three doses) was used in the treatment, followed by the use of prednisolone (3 mg/m2/day, once in the morning) in the second year. The comparison between the two treatments was assessed after a one-year treatment period by: variation of height standard deviation score (SDS) (delta Height SDS), variation of height SDS according to bone age (delta BA SDS), variation of body mass SDS (delta BMI SDS) and serum levels of androstenedione. No significant difference was observed in relation to the delta Height SDS, delta BA SDS and delta BMI SDS. No significant difference was observed in the serum levels of androstenedione. We conclude that the efficacy of prednisolone administered once a day orally is comparable to the oral use of hydrocortisone three times a day. Oral prednisolone may be an option for patients with CAH due to 21-hydroxylase deficiency. [\hyperlink{Hydrocortisone Acetate And Pramoxine Hydrochloride}{PMID: 18345402}, Flavia M Leite et al., 2008]

\hypertarget{pmid_29490769}{T}he safety of a novel intranasal formulation of azelastine hydrochloride (AZE) and fluticasone propionate (FP) has been established in adults and adolescents with allergic rhinitis but not in children <12 years old. To evaluate the safety and tolerability of an intranasal formulation of AZE and FP in children ages 4-11 years with allergic rhinitis. The study was a randomized, 3-month, parallel-group, open-label design. Qualified patients were randomized in a 3:1 ratio to AZE/FP (n = 304) or fluticasone propionate (FP) (n = 101), one spray per nostril twice daily, and to one of three age groups: ≥4 to <6 years, ≥6 to <9 years, and ≥9 to <12 years. Safety was assessed by child- or caregiver-reported adverse events, nasal examinations, vital signs, and laboratory assessments. The incidence of treatment-related adverse events (TRAEs) was low in both the AZE/FP (16\%) and FP-only (12\%) groups after 90 days' continuous use. Epistaxis was the most frequently reported TRAE in both groups (AZE/FP, 9\%; FP, 9\%), followed by headache (AZE/FP, 3\%; FP, 1\%). All other TRAEs in the AZE/FP group were reported by ≤1\% of the children. The majority of TRAEs were of mild intensity and resolved spontaneously. Results of nasal examinations showed an improvement over time in both groups, with no cases of mucosal ulceration or nasal septal perforation. There were no unusual or unexpected changes in laboratory parameters or vital signs. The intranasal formulation of AZE and FP was safe and well tolerated after 3 months' continuous use in children with allergic rhinitis.The study was registered on <ext-link xmlns:xlink="http://www.w3.org/1999/xlink" ext-link-type="uri" xlink:href="http://ClinicalTrials.gov">ClinicalTrials.gov</ext-link> (NCT01794741). [\hyperlink{Hydrocortisone Acetate And Pramoxine Hydrochloride}{PMID: 29490769}, William Berger et al., 2018]

\hypertarget{pmid_32888021}{C}hildren with congenital adrenal hyperplasia (CAH) and adrenal insufficiency (AI) require daily hydrocortisone replacement with accurate dosing. Prospective study of efficacy and safety of hydrocortisone granules in children with AI and CAH monitored by 17-OHP (17-hydroxyprogesterone) saliva profiles. Seventeen children with CAH (9 male) and 1 with hypopituitarism (male), aged from birth to 6 years, had their hydrocortisone medication changed from pharmacy compounded capsules to hydrocortisone granules. Patients were followed prospectively for 2 years. In children with CAH, the therapy was adjusted by 17-OHP salivary profiles every 3 months. The following parameters were recorded: hydrocortisone dose, height, weight, pubertal status, adverse events, and incidence of adrenal crisis. The study medication was given thrice daily, and the median duration of treatment (range) was 795 (1-872) days, with 150 follow-up visits. Hydrocortisone doses were changed on 40/150 visits, with 32 based on salivary measurements and 8 on serum 17-OHP levels. The median daily mg/m2 hydrocortisone dose (range) at study entry for the different age groups 2-8 years, 1 month to 2 years, <28 days was 11.9 (7.2-15.5), 9.9 (8.6-12.2), and 12.0 (11.1-29.5), respectively, and at end of the study was 10.2 (7.0-14.4), 9.8 (8.9-13.1), and 8.6 (8.2-13.7), respectively. There were no trends for accelerated or reduced growth. No adrenal crises were observed despite 193 treatment-emergent adverse events, which were mainly common childhood illnesses. This first prospective study of glucocorticoid treatment in children with AI and CAH demonstrates that accurate dosing and monitoring from birth results in hydrocortisone doses at the lower end of the recommended dose range and normal growth, without occurrence of adrenal crises. [\hyperlink{Hydrocortisone Acetate And Pramoxine Hydrochloride}{PMID: 32888021}, Uta Neumann et al., 2021]

\hypertarget{pmid_7601950}{D}esonide, a class 6 nonfluorinated topical corticosteroid, has been available for more than two decades. Hydrocortisone is widely used in the treatment of dermatoses in children. Our purpose was to compare the safety and efficacy of desonide ointment and 1.0\% hydrocortisone ointment in children with atopic dermatitis. One hundred thirteen children (mean age, 4.8 years) with mild to moderate atopic dermatitis were enrolled in a multicenter, randomized, investigator-masked, parallel-group study. Treatments were applied twice daily for 5 weeks and extended to 6 months in 36 of the patients. Signs of atrophy were evaluated. Efficacy was determined by measuring global improvement, erythema, lichenification, excoriations, oozing or crusting, pruritus, and induration. No differences in safety were observed between hydrocortisone and desonide. The investigator's global assessment of improvement significantly favored desonide over hydrocortisone during 3 months of treatment (p < 0.05). Desonide ointment showed greater efficacy, produced more rapid improvement, and demonstrated an equivalent cutaneous safety profile when compared with 1\% hydrocortisone ointment for up to 6 months. [\hyperlink{Hydrocortisone Acetate And Pramoxine Hydrochloride}{PMID: 7601950}, J Jorizzo et al., 1995]

\hypertarget{pmid_18598837}{T}o demonstrate clinical equivalence (statistical noninferiority) of topical ciprofloxacin and hydrocortisone (CHC, Cipro HC) and topical neomycin/polymyxin b/hydrocortisone (NPH, Cortisporin) with systemic amoxicillin (AMX, Amoxil), for treatment of acute otitis externa (AOE). Randomized, active-control, observer-blind, multicenter trial. Altogether, 206 patients were enrolled (CHC, 106; NPH + AMX, 100). Patients were > or =1 year of age, had AOE >2 days with at least mild symptoms, and gave informed consent. All were evaluable for safety, and 151 were evaluable for efficacy. Ciprofloxacin and hydrocortisone 3 drops twice daily for 7 days (adults and children) or NPH 4 drops (adults) or 2 drops (children) with AMX 250 mg (adults and children) 3 times daily for 10 days, as directed in approved product labeling. The primary efficacy variable was response to therapy 7 days after treatment ended (test of cure). Secondary variables included time to end of pain, symptom scores (otalgia and tenderness) and microbiological eradication. Noninferiority was declared if the lower confidence limit around the measurement difference was above -10 (nearer zero). Response to therapy was higher for CHC (95.71\% vs 89.83\%) but was statistically noninferior (lower confidence limit, -4.98) to NPH + AMX. Median time to end of pain was 6 days for both groups. Noninferiority was declared for symptom scores at all measurement periods and for microbiological eradication. No serious adverse events related to treatment were reported. Ciprofloxacin and hydrocortisone is clinically equivalent to NPH + AMX for the treatment of AOE in adults and children. However, low systemic exposure, absence of ototoxicity, and less frequent dosing clearly favor Cipro HC. [\hyperlink{Hydrocortisone Acetate And Pramoxine Hydrochloride}{PMID: 18598837}, Peter S Roland et al., ]

\hypertarget{pmid_31903560}{T}o compare the efficacy and safety of prednisolone/prednisone and adrenocorticotropic hormone (ACTH) in the treatment of infantile spasms using a meta-analysis of randomized controlled trials (RCTs). In a systematic literature search of electronic databases (MEDLINE, Embase, the Cochrane Library), we identified RCTs that assessed prednisolone/prednisone compared with ACTH/tetracosactide in patients with infantile spasms. The electroclinical response and adverse events were evaluated. Six RCTs (616 participants) were included in the meta-analysis. Compared with prednisolone/prednisone, ACTH/tetracosactide was not superior in terms of cessation of spasms at day 14 (relative risk 1.19, 95\% confidence interval [CI] 0.74-1.92), day 42 (relative risk 1.02, 95\% CI 0.63-1.65), and resolution of hypsarrhythmia on electroencephalogram (relative risk 1.14, 95\% CI 0.71-1.81); the incidences of common adverse reactions caused by ACTH/tetracosactide were not lower than that of prednisolone/prednisone for irritability (relative risk 0.79, 95\% CI 0.57-1.10), increased appetite (relative risk 0.78, 95\% CI 0.57-1.08), weight gain (relative risk 0.86, 95\% CI 0.56-1.32), and gastrointestinal upset (relative risk 0.60, 95\% CI 0.35-1.02), though it seemed less frequent. Prednisolone/prednisone elicits a similar electroclinical response as ACTH for infantile spasms, which indicates that it can be an alternative to ACTH for treating infantile spasms. What this paper adds Prednisolone/prednisone is as effective as adrenocorticotropic hormone (ACTH) in electroclinical response of infantile spasms. Prednisolone/prednisone and ACTH cause similar and tolerable adverse effects, whose incidences are comparable. High-dose prednisone/prednisolone might be preferable to low dose for achieving freedom from spasms. [\hyperlink{Hydrocortisone Acetate And Pramoxine Hydrochloride}{PMID: 31903560}, Shaojun Li et al., 2020]

\hypertarget{pmid_8336748}{I}n order to study the adverse reaction of a new, inhaled steroid (Fluticasone) on the pituitary-adrenocortical axis in asthmatic children, we investigated 7 children (aged 7 to 15 years) before and during treatment with Fluticason (100-200 micrograms/day). For the dosage tested, we found no depression of adrenal function, neither in circadian cortisol secretion nor in hCRH-stimulation-test. Another 7 asthmatic children under treatment with Budesonide (800 micrograms/day) were examined by the same tests. They equally did not show an adrenocortical suppression. However, in 4 other children under therapy with oral prednisone (2.5 to 7.5 mg/day), there was a marked suppression on adrenocortical function, even with low doses of the steroid. We conclude that Fluticasone (as well as Budesonide) in the above dosages represent a safe therapy for bronchial asthma in children. [\hyperlink{Hydrocortisone Acetate And Pramoxine Hydrochloride}{PMID: 8336748}, A Hoffmann-Streb et al., 1993]

\hypertarget{pmid_2615293}{A} course of 4-5 intra-articular injections was given to 25 children aged 4-15 years with juvenile rheumatoid arthritis: 20\% dimexide++ solution in combination with hydrocortisone (2 ml) was administered into the right knee joint and hydrocortisone (12.5 mg) into the left knee joint once a week. Dimexide++ solution combination with hydrocortisone proved to be most effective: all signs of inflammation subsided, the joint function was restored and there were no untoward reactions. [\hyperlink{Hydrocortisone Acetate And Pramoxine Hydrochloride}{PMID: 2615293}, N I Melikhova et al., 1989]

\hypertarget{pmid_33991205}{C}orticosteroids and hyaluronidase are trialed for treating phimosis in children. We carried out the present network meta-analysis to compare the therapeutic effect of these drugs. Electronic databases were searched for appropriate randomized clinical trials. Odds ratio (OR) with 95\% confidence intervals (95\% CI) was used as the effect estimate. A random-effects model was used for generating the pooled estimates. Rankogram plot was used for ranking the drugs. Proportions of patients with remission (partial/complete) and with complete remission. Mometasone (OR 6.53, 95\% CI 2.85, 14.96), betamethasone/hyaluronidase (OR 12.1, 95\% CI 4.27, 34.49), triamcinolone (OR 19.15, 95\% CI 4.47, 81.96), dexamethasone (OR 21.38, 95\% CI 5.71, 79.98), betamethasone (OR 23.02, 95\% CI 6.92, 79.54), hydrocortisone (OR 23.2, 95\% CI 5.91, 91.02) and methylprednisolone (OR 50.47, 95\% CI 4.45, 572.72) were observed with significantly higher proportions of patients with remission (partial/complete) compared to placebo. Dexamethasone, triamcinolone, betamethasone, betamethasone/hyaluronidase, clobetasol, mometasone, and hydrocortisone were observed with significantly higher proportions of patients with complete remission compared to placebo. Beclomethasone was not observed to be superior to either placebo or other drugs. Rankogram plot revealed methylprednisolone followed by hydrocortisone had the maximum statistical probability of being 'the best' in the pool for remission and betamethasone followed by hydrocortisone for complete remission. Topical methylprednisolone, hydrocortisone, and betamethasone were observed with better clinical resolution of phimosis compared to other corticosteroids. Very high potent corticosteroids like beclomethasone and clobetasol were not observed with superior benefits compared to other corticosteroids. Considering low-potency, hydrocortisone shall be preferred until further evidence emerges. [\hyperlink{Hydrocortisone Acetate And Pramoxine Hydrochloride}{PMID: 33991205}, Kannan Sridharan et al., 2021]

\hypertarget{pmid_9366699}{T}o compare the safety and efficacy of ofloxacin otic solution with those of Cortisporin otic solutions (neomycin sulfate, polymyxin B sulfate, and hydrocortisone) in otitis externa in adults and children. Two randomized, evaluator-blind, multicenter trials, 1 each in children and adults. Twenty-three primary care and referral ambulatory care sites per trial. A total of 314 adults (12 years and older) and 287 children (younger than 12 years). Of the total, data for 247 adults and 227 children were considered clinically evaluable (CE), and those for 98 children and 98 adults were microbiologically evaluable (ME). Ofloxacin (adults, 0.5 mL; children, 0.25 mL) twice daily or Cortisporin (adults, 0.2 mL; children, 0.15 mL) 4 times daily for 10 days. The CE subjects were cured if all signs and symptoms resolved at posttherapy (days 11-13) and test-of-cure (days 17-20) visits. The ME subjects had microbiological and clinical successes if they were cured and had microbiological eradication or presumed eradication. Cure was observed in 82\% and 97\% of CE adults and children treated with ofloxacin and 84\% and 95\% of CE adults and children treated with Cortisporin, respectively. The most common pathogens at the pretherapy visit were Pseudomonas aeruginosa, Staphylococcus aureus, and enteric bacilli. There were no statistically significant differences in clinical or microbiological and clinical cure or in the rates of adverse events between treatment groups. Ofloxacin given twice daily is as safe and effective as Cortisporin given 4 times daily for otitis externa. The bacteriological findings and treatment responses do not differ between adults and children. [\hyperlink{Hydrocortisone Acetate And Pramoxine Hydrochloride}{PMID: 9366699}, R N Jones et al., 1997]

\hypertarget{pmid_2490059}{A} retrospective survey reports the use of articaine hydrochloride as an anesthetic in children under 4 years of age. Data was collected by a record audit in two pediatric dentistry offices. Articaine anesthetic was administered to 211 patients, 29 having additional administrations of the agent. In some instances, the dosages exceeded the recommended concentrations for older children. No adverse systemic adverse reactions were noted on the charts or known to the clinicians. The present report provides initial evidence for the use of articaine in children under 4 years of age. [\hyperlink{Hydrocortisone Acetate And Pramoxine Hydrochloride}{PMID: 2490059}, G Z Wright et al., ]

\hypertarget{pmid_18977976}{T}he goal was to investigate cardiovascular responses to a psychosocial stressor in school-aged, formerly premature boys and girls who had been treated neonatally with dexamethasone or hydrocortisone because of chronic lung disease. We compared corticosteroid-treated, formerly preterm infants with formerly preterm infants who had not been treated neonatally with corticosteroids (reference group). Children performed the Trier Social Stress Test for Children, which includes a public speaking task and a mental arithmetic task. Blood pressure was recorded continuously before, during, and after the stress test. Plasma norepinephrine levels were determined before the test, directly after the stress task, and after recovery. Overall, in response to stress, girls had significantly larger changes in systolic blood pressure and mean arterial pressure and in stroke volume and cardiac output, compared with boys. Boys exhibited larger total peripheral resistance responses, compared with girls. The hydrocortisone group did not differ significantly from the reference group in any of the outcome measures. However, dexamethasone-treated children had smaller stress-induced increases in systolic and mean arterial blood pressure than did hydrocortisone-treated children. In addition, the dexamethasone group showed smaller increases in stroke volume and blunted norepinephrine responses to stress, compared with children in the reference group. Correction for gender did not affect these results. The differences in cardiovascular stress responses between girls and boys are consistent with known gender differences in adult cardiovascular stress responses. Our data demonstrate that neonatal treatment with dexamethasone has long-term consequences for the cardiovascular and noradrenergic stress responses; at school age, the cardiovascular stress response was blunted in dexamethasone-treated children. Hydrocortisone-treated children did not differ from the reference group, which suggests that hydrocortisone might be a safe alternative to dexamethasone for treating chronic lung disease of prematurity. [\hyperlink{Hydrocortisone Acetate And Pramoxine Hydrochloride}{PMID: 18977976}, Rosa Karemaker et al., 2008]

\hypertarget{pmid_17941284}{T}he safety of fexofenadine has been examined extensively in adults and school-age children. However, the safety of fexofenadine in children younger than 6 years has not been reported to date. To compare the safety and tolerability of twice-daily fexofenadine hydrochloride, 30 mg, and placebo in preschool children aged 2 to 5 years with allergic rhinitis. This was a multicenter, double-blind, randomized, placebo-controlled, parallel-group study, conducted between February 29, 2000, and June 14, 2001. Participants were randomized to either fexofenadine hydrochloride, 30 mg, or placebo twice daily for a 2-week period. To facilitate dosing, capsule content was mixed with applesauce (approximately 10 mL). Safety assessments depended on date of entry into the study because of an amendment to the protocol. Before the amendment, assessments included physical examination, vital signs reporting (oral temperature, heart rate, and respiratory rate), and adverse event (AE) reporting. After the amendment, safety assessments included laboratory testing (blood chemistry and hematology profiles), physical examination, 12-lead electrocardiography, and vital signs (oral temperature, blood pressure, heart rate, and respiratory rate) and AE reporting. Treatment-emergent AEs were observed in 116 of 231 participants receiving placebo and 111 of 222 receiving fexofenadine. These AEs were possibly related to study medication in 19 (8.2\%) and 21 (9.5\%) of the participants receiving placebo and fexofenadine, respectively, and most frequently involved the digestive system. No clinically relevant differences in laboratory measures, vital signs, and physical examinations were observed. The findings show that fexofenadine hydrochloride, 30 mg, is well tolerated and has a good safety profile in children aged 2 to 5 years with allergic rhinitis. [\hyperlink{Hydrocortisone Acetate And Pramoxine Hydrochloride}{PMID: 17941284}, Henry Milgrom et al., 2007]

\hypertarget{pmid_6244198}{T}his study compares the effects on the hypothalamo-pituitary adrenal (HPA) axis of two dosage schedules of hydrocortisone 17-butyrate and hydrocortisone ointments in 20 children suffering from eczema. Children with moderately extensive eczema received either 30 g of 0.1\% hydrocortisone 17-butyrate or 30 g of 1\% hydrocortisone ointment weekly for 4 weeks without occlusion. Children with extensive eczema received either 60 g of hydrocortisone 17-butyrate or 60 g of hydrocortisone weekly for 4 weeks. All four groups showed some clinical improvement. Although many of the children appeared to have some impairment of adrenal function prior to entering the trial, no further significant depression of the HPA axis as reflected by the plasma ACTH levels and a 30-min Synacthen test was observed during the trial. On the basis of 4 weeks' treatment, hydrocortisone 17-butyrate did not have a significantly different effect on the HPA axis in children from that of hydrocortisone. [\hyperlink{Hydrocortisone Acetate And Pramoxine Hydrochloride}{PMID: 6244198}, R H Marten et al., 1980]

\hypertarget{pmid_20871859}{C}lassic congenital adrenal hyperplasia affects approximately 1 in 15,000 children. Current treatment strategies using multiple daily doses of hydrocortisone lead to suboptimal outcomes. We tested the hypothesis that nocturnal administration of dexamethasone will suppress the hypothalamic-pituitary-adrenal axis more effectively than standard hydrocortisone treatment by blocking the inherent diurnal secretion of ACTH. We performed a pilot study of five prepubertal patients comparing CAH control during two 24-hour hospitalizations, one on hydrocortisone and the other on dexamethasone. The patterns of adrenal suppression differed markedly between hydrocortisone and nocturnal dexamethasone, with significant suppression of the morning rise in ACTH, 17-hydroxyprogesterone, and androstenedione while on dexamethasone. On hydrocortisone therapy, there is a marked variation in ACTH and adrenal hormones depending on time of day and timing of hydrocortisone administration. Longer-term studies are needed to investigate the lowest effective dose and potential toxicity of nocturnal dexamethasone to determine its utility as a therapy for CAH. [\hyperlink{Hydrocortisone Acetate And Pramoxine Hydrochloride}{PMID: 20871859}, Andrew Dauber et al., 2010]

\section*{Erythromycin}
\subsection*{Result}
\subsubsection*{Answer}

Neonates and infants <43 days: No  
Infants 43–90 days: Yes  
Children >3 months: Yes  

\subsubsection*{{Explanation}}
\hypertarget{Erythromycin}
A review of the available abstracts reveals multiple targeted studies evaluating the safety of erythromycin in children across various age groups. Below is a summary by age range:

Neonates and Infants (0–3 months):
- Several studies specifically address the use of erythromycin in neonates and young infants. A large retrospective cohort study found that exposure to oral erythromycin in the first 14 days of life is associated with a significantly increased risk of infantile hypertrophic pyloric stenosis (IHPS), with an adjusted odds ratio (aOR) of 13.3 (95\% CI, 6.80-25.9). The risk remains elevated, though lower, for exposures between 15 and 42 days of life (aOR 4.10, 95\% CI, 1.69-9.91). No increased risk was found for exposures after 43 days of life [\hyperlink{pmid_25687145}{PMID: 25687145}, Matthew D Eberly et al., 2015]. This association is confirmed in other reviews, which note that while erythromycin may be effective as a prokinetic agent in preterm infants, the risk of IHPS is a significant concern, and long-term safety outcomes are not fully evaluated [\hyperlink{pmid_19218823}{PMID: 19218823}, Pak C Ng et al., 2009]. One pharmacokinetic study in preterm neonates found erythromycin to be well-tolerated in the short term, with no adverse effects observed in a small sample, but this does not address the risk of IHPS [\hyperlink{pmid_8036045}{PMID: 8036045}, K B Waites et al., 1994]. Therefore, targeted studies show erythromycin is not safe for use in neonates and young infants due to the risk of IHPS.

Infants and Young Children (>3 months to 1 year):
- The risk of IHPS appears to decrease after 43 days of life, with no association found in the 43–90 day age group [\hyperlink{pmid_25687145}{PMID: 25687145}, Matthew D Eberly et al., 2015]. Other studies in infants (e.g., those with gastroenteritis or being treated for infections) report no serious adverse effects, and side effects are generally mild and gastrointestinal in nature [\hyperlink{pmid_6349401}{PMID: 6349401}, R M Robins-Browne et al., 1983; \hyperlink{pmid_3149884}{PMID: 3149884}, J Goldfarb et al., 1988]. However, the sample sizes are small, and the studies are not powered to detect rare adverse events like IHPS. Thus, while no major safety concerns are reported after 43 days of life, the evidence is less robust, and long-term safety is not fully established.

Children (>1 year to adolescence):
- Multiple randomized controlled trials and comparative studies in children older than 1 year (including ages up to 15 years) consistently report that erythromycin is generally well-tolerated, with adverse events primarily being mild gastrointestinal symptoms (nausea, vomiting, diarrhea). No serious or life-threatening adverse effects are reported in these studies [\hyperlink{pmid_9124837}{PMID: 9124837}, J J Roord et al., 1996; \hyperlink{pmid_7667050}{PMID: 7667050}, S Block et al., 1995; \hyperlink{pmid_1337553}{PMID: 1337553}, R Manfredi et al., 1992; \hyperlink{pmid_8195854}{PMID: 8195854}, R Salzberg et al., 1993; \hyperlink{pmid_3149884}{PMID: 3149884}, J Goldfarb et al., 1988; \hyperlink{pmid_14502372}{PMID: 14502372}, M A Bellomo-Brandão et al., 2003]. One study in children aged 4–13 years found no erythromycin-related side effects in a double-blind, placebo-controlled trial [\hyperlink{pmid_14502372}{PMID: 14502372}, M A Bellomo-Brandão et al., 2003]. Another study in children aged 5 months to 13 years reported only mild diarrhea in a few cases [\hyperlink{pmid_3149884}{PMID: 3149884}, J Goldfarb et al., 1988]. These findings support the safety of erythromycin in children older than 1 month (and especially older than 3 months), with the caveat that gastrointestinal side effects are common.

Summary:
- Neonates and infants <43 days: Erythromycin is not safe due to a significantly increased risk of IHPS.
- Infants 43–90 days: No increased risk of IHPS found, but evidence is less robust; safety appears acceptable based on available data.
- Children >3 months: Multiple studies affirm safety, with mild gastrointestinal side effects being the most common adverse events.

\subsection*{Abstracts}
\hypertarget{pmid_792407}{E}rythromycin continues to be a valuable and useful antimicrobial agent in children. Its low index of toxicity, freedom from sensitization, and reliable absorption and when administered orally contribute to make it an attractive agent in the treatment of a variety of minor respiratory and skin infections, especially in those situations where real or potential allergy to penicillin exists. Additional major uses are in the eradication of the carrier state in whooping cough and in diphtheria, especially in those instances when oral therapy can be tolerated. Dispite use over more than two decades, resistance developing in formerly susceptible organisms has not been a problem and thus seems unlikely to become so in the future. [\hyperlink{Erythromycin}{PMID: 792407}, C M Ginsburg et al., 1976]

\hypertarget{pmid_31321320}{A}zithromycin is widely used in children not only in the treatment of individual children with infectious diseases, but also as mass drug administration (MDA) within a community to eradicate or control specific tropical diseases. MDA has also been reported to have a beneficial effect on child mortality and morbidity. However, concerns have been raised about the safety of azithromycin, especially in young children. The aim of this review is to systematically identify the safety of azithromycin in children of all ages. MEDLINE, PubMed, Cochrane Central Register of Controlled Trials, Embase, CINAHL, International Pharmaceutical Abstracts and adverse drug reaction (ADR) monitoring systems will be systematically searched for randomised controlled trials (RCTs), cohort studies, case-control studies, cross-sectional studies, case series and case reports evaluating the safety of azithromycin in children. The Cochrane risk of bias tool, Newcastle-Ottawa and quality assessment tools, and The Joanna Briggs Institute Critical Appraisal tools will be used for quality assessment. Meta-analyses will be conducted to the incidence of ADRs from RCTs if appropriate. Subgroup analyses will be performed in different age and azithromycin dosage groups. Formal ethical approval is not required as no primary data are collected. This systematic review will be disseminated through a peer-reviewed publication. CRD42018112629. [\hyperlink{Erythromycin}{PMID: 31321320}, Peipei Xu et al., 2019]

\hypertarget{pmid_11328252}{E}rythromycin has been used as an antibiotic for more than four decades, but only in the last 10 years have other therapeutic benefits of this agent been exploited. Animal and human studies have demonstrated a prokinetic effect on the gastrointestinal tract at sub-antimicrobial doses (typically a quarter or less of the antibiotic dose). A limited number of studies have been performed in children to investigate this action. A review of this literature is particularly pertinent given the frequency of clinical problems related to gastrointestinal dysmotility in children and the limited availability of prokinetic agents in paediatric practice, compounded by the recent withdrawal of cisapride. The prokinetic effects of erythromycin have been investigated in infants with dysmotility associated with prematurity, in low birth-weight infants recovering from abdominal surgery, and in older children with a variety of other gastrointestinal disorders. Only one randomized placebo-controlled trial has been conducted. All except one of these studies have shown a beneficial effect of erythromycin in either promoting tolerance of enteral feeds or enhancing a measured index of gastrointestinal motility. Erythromycin appears to be equally effective when given orally (as ethylsuccinate or estolate) or intravenously (as lactobionate). Significantly, no serious adverse effects have been reported in studies in which erythromycin has been used for its prokinetic effects, although fatal reactions have followed the intravenous administration of erythromycin to neonates in antibiotic doses. [\hyperlink{Erythromycin}{PMID: 11328252}, J I Curry et al., 2001]

\hypertarget{pmid_18789096}{C}hronic bullous disease of childhood is the commonest acquired blistering disorder of children. Erythromycin has been reported to be beneficial for this condition. A three question survey was e-mailed to all members of the British Society for Paediatric Dermatology to assess the incidence, preferred treatments and experience of oral erythromycin in treating chronic bullous disease of childhood. A second, more detailed questionnaire was sent to members who had used erythromycin. Forty patients were reported to have been treated over the previous 2 years. The preferred treatment was dapsone. Erythromycin alone had been used in five children as first-line oral treatment. In three of these patients the initial improvement was graded as either "good" or "complete resolution." This benefit was only sustained in one child, with the other two relapsing between 4 and 12 weeks. In a further eight children, erythromycin had been used with other oral agents. In five of these children, erythromycin was associated with long-term benefit. These results suggest that erythromycin is unlikely to produce sustained improvement in chronic bullous disease of childhood when used as a sole first-line agent. However, erythromycin can cause an initial improvement, which may be useful whilst awaiting results of diagnostic tests and may confer benefit when used with other systemic treatments. [\hyperlink{Erythromycin}{PMID: 18789096}, Paul Farrant et al., ]

\hypertarget{pmid_7359612}{E}rythromycin is considered one of the safest antibiotics in common use today. In its otolaryngologic use, the authors have found it effective in treating acute suppurative sinusitis and occasionally otitis media, when combined with sulfonamides. There are few complications of erythromycin administration. Probably the least generally acknowledged of these is ototoxicity. There have been three reports of six cases with ototoxic complications from erythromycin, primarily from administration of its intravenous form. The authors present a case study of an 18 year old girl in severe renal failure, who suffered a reversible sensorineural hearing loss from high doses of an oral erythromycin preparation. The clinical manifestations of this case are compared to those previously reported. [\hyperlink{Erythromycin}{PMID: 7359612}, P Thompson et al., 1980]

\hypertarget{pmid_34447818}{R}espiratory infections in children are common pediatric diseases caused by pathogens that invade the respiratory system. Children are considerably susceptible to  To analyze the clinical efficacy of different antibiotics in treating pediatric respiratory mycoplasma infections. We included 106 children with a confirmed diagnosis of respiratory mycoplasma infection who were admitted to our hospital from April 2017 to July 2019 and grouped them using a random number table. Among them, 53 children each received clarithromycin or erythromycin. The clinical efficacy of both drugs was evaluated and compared. We performed the multiplex polymerase chain reaction (MP-PCR) test and determined the MP-PCR negative rate in children after the end of the treatment course. We compared the incidence of toxic and side effects, including nausea, diarrhea, and abdominal pain; further, we recorded the length of hospitalization, antipyretic time, and drug costs. Additionally, we evaluated and compared the compliance of the children during treatment. The erythromycin group showed a significantly higher total effective rate of clinical treatment than the clarithromycin group. MP-PCR test results showed that the clarithromycin group had a significantly higher MP-PCR negative rate than the erythromycin group. Moreover, children in the clarithromycin group had shorter fever time, shorter hospital stays, and lower drug costs than those in the erythromycin group. The clarithromycin group had a significantly higher overall drug adherence rate than the erythromycin group. The incidence of toxic and side effects was significantly lower in the clarithromycin group than in the erythromycin group ( Our findings indicate that clarithromycin has various advantages over erythromycin, including higher application safety, stronger mycoplasma clearance, and higher medication compliance in children; therefore, it can be actively promoted. [\hyperlink{Erythromycin}{PMID: 34447818}, Mei-Ying Zhang et al., 2021]

\hypertarget{pmid_36827282}{E}rythromycin is a macrolide antibiotic that is also prescribed off-label in premature neonates as a prokinetic agent. There is no oral formulation with dosage and/or excipients adapted for these high-risk patients. Clinical studies of erythromycin as a prokinetic agent were reviewed. Capsules of 20 milligrams of erythromycin were compounded with microcrystalline cellulose. Erythromycin capsules were analyzed using the chromatographic method described in the United States Pharmacopoeia which was found to be stability-indicating. The stability of 20 mg erythromycin capsules stored protected from light at room temperature was studied for one year. 20 mg erythromycin capsules have a beyond use date not lower than one year. 20 milligrams erythromycin capsules can be compounded in batches of 300 unities in hospital pharmacy with a beyond-use-date of one year at ambient temperature protected from light. [\hyperlink{Erythromycin}{PMID: 36827282}, Patrick Thevin et al., 2023]

\hypertarget{pmid_3534749}{E}rythromycin ethyl succinate is an antibiotic frequently administered in pediatrics. According to some authors, this drug sharply decreases the fecal count of enterobacteria. The fecal flora of 12 infants less than one year old, treated by erythromycin ethyl succinate for 7 to 10 days was studied by differential count. A variable effect was observed on enterobacteria: a 10(3) to 10(5) fold reduction in 9 cases with a final count superior or equal to 10(4) per gram of feces, with or without coming back to the initial count; in 3 cases no modification. MIC of enterobacteria and concentrations of erythromycin in feces were not predictives of flora variation. Anaerobic flora was weakly modified. No implantation of potentially-pathogenic bacteria or multi-resistant or highly erythromycin resistant enterobacteria occurred. Thus, erythromycin ethyl succinate is valuable in pediatrics as it does not disturb barrier effects. But its use for selective decontamination of gut must be discussed depending on pharmacologic form and posology administered. [\hyperlink{Erythromycin}{PMID: 3534749}, M J Butel et al., 1986]

\hypertarget{pmid_14502372}{T}he efficacy of erythromycin was assessed in the treatment of 14 children aged 4 to 13 years with refractory chronic constipation, and presenting megarectum and fecal impaction. A double-blind, placebo- controlled, crossover study was conducted at the Pediatric Gastroenterology Outpatient Clinic of the University Hospital. The patients were randomized to receive placebo for 4 weeks followed by erythromycin estolate, 20 mg kg-1 day-1, divided into four oral doses for another 4 weeks, or vice versa. Patient outcome was assessed according to a clinical score from 12 (most severe clinical condition) to 0 (complete recovery). At enrollment in the study and on the occasion of follow-up medical visits at two-week intervals, patient score and laxative requirements were recorded. During the first 30 days, the mean SD clinical score for the erythromycin group (N = 6) decreased from 8.2+/-2.3 to 2.2+/-1.0 while the score for the placebo group (N = 8) decreased from 7.8+/-2.1 to 2.9+/-2.8. During the second crossover phase, the score for patients on erythromycin ranged from 2.9+/-2.8 to 2.4+/-2.1 and the score for the patients on placebo worsened from 2.2+/-1.0 to 4.3+/-2.3. There was a significant improvement in score when patients were on erythromycin (P < 0.01). Mean laxative requirement was lower when patients ingested erythromycin (P < 0.05). No erythromycin-related side effects occurred. Erythromycin was useful in this group of severely constipated children. A larger trial is needed to fully ascertain the prokinetic efficacy of this drug as an adjunct in the treatment of severe constipation in children. [\hyperlink{Erythromycin}{PMID: 14502372}, M A Bellomo-Brandão et al., 2003]

\hypertarget{pmid_9124837}{T}he efficacies and safeties of a 3-day, 3-dose course of azithromycin (10 mg/kg of body weight per day) and a 10-day, 30-dose course of erythromycin (40 mg/kg/day) for the treatment of acute lower respiratory tract infections in children were compared in an open randomized multicenter study. Sixty-eight of 85 evaluable patients (80\%) had radiologically proven pneumonia, and 20\% had bronchitis. Treatment success defined as cure or major improvement was achieved in 42 of 45 (93\%) azithromycin recipients versus 36 of 40 (90\%) erythromycin recipients. Adverse events were reported in 12 of 45 and 6 of 40 of the patients treated with azithromycin and erythromycin, respectively, a difference which was not statistically significant. In conclusion, a 3-day course of azithromycin is as effective as a 10-day course of erythromycin in the treatment of community-acquired lower respiratory tract infections in children, with comparable safety and acceptability profiles. This shorter treatment course might have a beneficial effect on compliance, especially in the pediatric age group. [\hyperlink{Erythromycin}{PMID: 9124837}, J J Roord et al., 1996]

\hypertarget{pmid_6349401}{A} double-blind placebo-controlled trial of erythromycin ethylsuccinate was conducted in 65 infants and young children hospitalized with acute nonspecific gastroenteritis. Etiologic agents included rotaviruses (29\%), Campylobacter jejuni (17\%), "classical" enteropathogenic Escherichia coli (12\%), enterotoxigenic E. coli (11\%), Salmonella (9\%), Shigella (2\%), and Giardia lamblia (2\%). No pathogens were obtained from 25 (38\%) children. Treatment with erythromycin had no effect on the course of the illness in terms of the time required for hydration, stool frequency and temperature to return to normal, or for vomiting to be abolished. Children treated with erythromycin, however, experienced a marginally, but significantly (P less than 0.05), shorter period of abnormal stool consistency compared with control subjects. This effect was most pronounced in children from whom no enteropathogens were isolated. [\hyperlink{Erythromycin}{PMID: 6349401}, R M Robins-Browne et al., 1983]

\hypertarget{pmid_7049959}{F}ollowing a study in which the etiology of nearly 70\% of 142 cases of pneumonia in children could be determined using a combination of bacteriological and serological methods, the effect of erythromycin ethylsuccinate was compared with that of amoxicillin in a randomized study on 120 cases of pneumonia. We first examined the tracheal secretion microbiologically and determined other serological parameters and clinical data. The tracheal secretion was sterile in only 19\% of the cases. We were able to identify the etiology in 64\% of the cases using a combination of microbiological and serological methods. A discontinuation of therapy and acceptable side-effects were considerably more frequent with amoxicillin than with erythromycin ethylsuccinate (75 mg/kg body weight). The advantages of erythromycin, especially for the initial therapy of pneumonia, and the improvements in diagnosis resulting from the examination of the tracheal secretion will be discussed. [\hyperlink{Erythromycin}{PMID: 7049959}, H Ruhrmann et al., 1982]

\hypertarget{pmid_1820902}{E}rythromycin pharmacokinetics was studied in neonates (less than 1 month), infants (1-12 months) and other children (1-12 years) after the drug rectal and intravenous administration. The areas under the erythromycin serum concentration-time curves (AUC) were practically independent on children's age following the intravenous drug administration, but not its rectal administration. There was a distinct age dependency of the AUC parameter in the latter case. The increase of children's age was resulted in enhancement of the erythromycin total clearance, reduction of the steady-state volume of distribution and of the mean residence time. The extent of absolute bioavailability of rectally administered erythromycin was increased from 28 per cent in neonates to 36 per cent in infants and to 54 per cent in children greater than 1 year. Alteration of the mean absorption time parameter was reflected the delayed absorption of erythromycin in neonates. [\hyperlink{Erythromycin}{PMID: 1820902}, L S Stratchunsky et al., 1991]

\hypertarget{pmid_3149884}{T}he safety and efficacy of a new topical antiinfective agent, mupirocin, was compared with that of oral erythromycin ethylsuccinate in the treatment of impetigo in children. Sixty-two children aged 5 months to 13 years with impetigo were assigned to be treated with either mupirocin in three daily applications or erythromycin ethylsuccinate (40 mg/kg of body weight per day divided into four doses) according to a randomized treatment schedule. On the initial visit, exudate or cleansed infected sites or both were cultured and therapy was begun. All patients were treated for 8 days. Patients were seen again on days 4 to 5 of therapy, at the end of therapy, and 7 days after the end of therapy. Sites of infection were comparable between the groups, as were bacteriologic responses. At the first visit, 24 of 30 children in the mupirocin group and 14 of 32 children in the erythromycin group were cured or had at least a 75\% reduction in size of the lesions. At the end of the study, all 29 of the children in the mupirocin group who came to follow-up, compared with 27 of 29 in the erythromycin group, were cured. Side effects were few. Five children in the erythromycin group developed mild diarrhea. Thus, mupirocin appears to be safe and effective in treating impetigo in children. Our data show a trend toward more rapid clinical response with mupirocin than with erythromycin. [\hyperlink{Erythromycin}{PMID: 3149884}, J Goldfarb et al., 1988]

\hypertarget{pmid_8036045}{E}rythromycin is receiving renewed attention as an alternative for treatment of neonatal infections caused by Ureaplasma urealyticum because of recently proved abilities of this organism to produce systemic disease in this population. Although erythromycin has been used clinically for almost 40 years, very little is known about its activity in the preterm neonate. Fourteen neonates, birth weights < or = 1500 g and < or = 15 days of age, from whom U. urealyticum was isolated from the lower respiratory tract were randomized to receive erythromycin lactobionate either 25 or 40 mg/kg/day in four divided doses at 6-hour intervals scheduled for a total of 10 days. Blood samples collected at multiple time points after initial and steady state doses were assayed for erythromycin by liquid chromatography. Minimal inhibitory concentrations (MICs) of erythromycin for the U. urealyticum isolates were determined. MICs ranged from 0.031 to 2 micrograms/ml; MIC90 = 2 micrograms/ml. Serum erythromycin concentrations met or exceeded most MICs, with peak values of 3.05 to 3.69 and 1.92 to 2.9 micrograms/ml for the 40- and 25-mg/kg/day dosage groups, respectively. Pharmacokinetic parameters were calculated after the initial dose and at steady state for both dosage groups and compared. No adverse effects thought to be related to administration of erythromycin were observed. These preliminary findings showed that erythromycin is well-tolerated, has favorable pharmacokinetic activity in the preterm neonate and should be further investigated for treatment of ureaplasmal infections. [\hyperlink{Erythromycin}{PMID: 8036045}, K B Waites et al., 1994]

\hypertarget{pmid_1337553}{T}he efficacy and tolerability of azithromycin and erythromycin in the treatment of acute respiratory tract infections in children were compared in an open, multicenter, randomized trial. A total of 151 children, aged from 2 months to 14 years, suffering from upper airways infections (60), or lower respiratory tract infections (91), were randomized to be treated either with azithromycin, 10 mg/Kg/day per os once daily for 3 or 10 mg/Kg/day 1 and 5 mg/Kg/days 2-5 (77 patients) or with erythromycin, 50 mg/Kg/day thrice daily for at least 7 days (74 patients). The two treatment groups did not significantly differ as to sex, age, weight, type and severity of infection, and infecting pathogens. Clinical evaluation was performed prior to therapy, on treatment days 1, 3, 5 and 7, and on day 10. Microbiological and laboratory assessment were carried out at baseline and after the end of therapeutic course. Chest X-ray and serologic assays for Mycoplasma pneumoniae infection were obtained in patients suspected to have lower respiratory tract infections. At the end of therapy, clinical cure was achieved in 73 out of 77 patients (94.8\%) in the azithromycin group, and in 60/72 evaluable subjects (83.3\%) in the erythromycin group. A significantly more rapid remission of several illness-related signs and symptoms was observed in patients treated with azithromycin. A total of 75 bacterial pathogens were isolated at baseline microbiological examination; at the end of the therapeutic course bacteriological eradication was obtained in 34/34 cases (100\%) treated with azithromycin, and in 40/41 children (97.5\%) treated with erythromycin.(ABSTRACT TRUNCATED AT 250 WORDS) [\hyperlink{Erythromycin}{PMID: 1337553}, R Manfredi et al., 1992] In spite of vaccination programmes, whooping cough epidemics continue to occur. The disease affects all age groups, although its severity is greatest in the young, with infants being particularly vulnerable. Erythromycin is generally accepted as the drug of choice both for treatment and for prophylaxis during epidemics. Roxithromycin is a macrolide with pharmacokinetic advantages over erythromycin; it is well absorbed, produces high serum concentrations, has a long half-life and penetrates respiratory secretions well. There are no accepted standards for testing the sensitivity of Bordetella pertussis to antibiotics, and reports of the activity of roxithromycin and erythromycin are variable. Using Isosensitest agar supplemented with 5\% horse blood and an inoculum of 10(4) cfu, 88 strains of B. pertussis were tested for their sensitivity to roxithromycin, erythromycin, rifampicin and trimethoprim/sulphamethoxazole. The range of MICs was 0.12-0.5 mg/L for both roxithromycin and erythromycin. Roxithromycin was bactericidal, with an MBC of 1 mg/L (as compared with 0.5 mg/L for erythromycin). Since roxithromycin is well tolerated by children when used for respiratory tract infections, the good in-vitro activity against B. pertussis, combined with its favourable pharmacokinetics, suggest it may be a good candidate for use in the treatment and prophylaxis of whooping cough. [\hyperlink{Erythromycin}{PMID: 1337553}, M Brett et al., 1998]

\hypertarget{pmid_3429384}{R}oxithromycin sachets of 50 mg were given to 304 infants and children, aged 2 months to 14 years, suffering from respiratory and skin infections treated in 25 hospitals in France and one in Greece. The dosage range was from 2.5 to 5.0 mg/kg/12 h and the mean duration of therapy was 8.9 days. The cure rate was 89\% of the 266 children evaluable for clinical efficacy and 90.3\% of the 50 bacteriologically identified cases. The overall bacteriological efficacy was 82\%. The antibiotic was well accepted by the 90\% of the 304 children, while in 6.9\% an adverse effect was reported, mainly vomiting. There were no toxic effects. Roxithromycin should be considered as an effective and safe oral antibiotic to treat children with upper and lower respiratory tract and skin infections due to common pathogens. [\hyperlink{Erythromycin}{PMID: 3429384}, D A Kafetzis et al., 1987]

\hypertarget{pmid_25687145}{U}se of oral erythromycin in infants is associated with infantile hypertrophic pyloric stenosis (IHPS). The risk with azithromycin remains unknown. We evaluated the association between exposure to oral azithromycin and erythromycin and subsequent development of IHPS. A retrospective cohort study of children born between 2001 and 2012 was performed utilizing the military health system database. Infants prescribed either oral erythromycin or azithromycin as outpatients in the first 90 days of life were evaluated for development of IHPS. Specific diagnostic and procedural codes were used to identify cases of IHPS. A total of 2466 of 1 074 236 children in the study period developed IHPS. Azithromycin exposure in the first 14 days of life demonstrated an increased risk of IHPS (adjusted odds ratio [aOR], 8.26; 95\% confidence interval [CI], 2.62-26.0); exposure between 15 and 42 days had an aOR of 2.98 (95\% CI, 1.24-7.20). An association between erythromycin and IHPS was also confirmed. Exposure to erythromycin in the first 14 days of life had an aOR of 13.3 (95\% CI, 6.80-25.9), and 15 to 42 days of life, aOR 4.10 (95\% CI, 1.69-9.91). There was no association with either macrolide between 43 and 90 days of life. Ingestion of oral azithromycin and erythromycin places young infants at increased risk of developing IHPS. This association is strongest if the exposure occurred in the first 2 weeks of life, but persists although to a lesser degree in children between 2 and 6 weeks of age. [\hyperlink{Erythromycin}{PMID: 25687145}, Matthew D Eberly et al., 2015]

\hypertarget{pmid_18327427}{T}his study aimed to evaluate the efficacy and safety of clarithromycin and erythromycin in the treatment of community-acquired pneumonia in children. Children with community-acquired pneumonia were randomly assigned to receive 10-day regimens of either clarithromycin 15 mg/kg/day, twice a day, or erythromycin 30-50 mg/kg/day, four times daily. A total of 97 children entered this study, including 26 with Mycoplasma pneumoniae infection, 15 with Chlamydia pneumoniae infection, and 6 with mixed mycoplasma and chlamydia infections. Fifty and 47 children received clarithromycin and erythromycin treatment, respectively. Three children withdrew from the study because the identified pathogens were resistant to the study drugs. All 47 children with mycoplasma or chlamydia infection were cured clinically. Delayed defervescence, defined as a fever lasting for more than 72 h after treatment, was observed in 4 of 22 clarithromycin-treated children (18\%) and in 3 of 15 erythromycin-treated children (20\%) [p>0.05]. Gastrointestinal side effects, including vomiting, abdominal pain and diarrhea, were observed in 3 of 50 children (6\%) receiving clarithromycin and in 11 of 49 children (22\%) receiving erythromycin (p=0.039). Excluding children with abnormal pretreatment liver function, abnormal liver function after treatment was observed in only one child, treated with erythromycin. Post-treatment eosinophil and platelet counts were significantly elevated after treatment in both groups. Clarithromycin showed efficacy equivalent to erythromycin for the treatment of mycoplasma or chlamydia pneumonia in children. However, the tolerability of clarithromycin was superior to that of erythromycin. [\hyperlink{Erythromycin}{PMID: 18327427}, Ping-Ing Lee et al., 2008]

\hypertarget{pmid_19218823}{M}ilk intolerance due to functional gastrointestinal (GI) dysmotility is a common problem in preterm infants. In the past decade, erythromycin has been used for its motilinomimetric effect to facilitate enteral feeding in preterm infants. Although earlier studies suggested that erythromycin is an effective prokinetic agent, recent randomized control trials (RCTs) reveal conflicting findings. This review assesses the evidence from all RCTs performed to date on erythromycin for preterm infants. The results suggest that oral erythromycin administered in intermediate or high doses as a rescue treatment is associated with a shorter time to attain full enteral feeding and decrease in the duration of requirement for parenteral nutrition. More importantly, the outcome study further indicates that oral erythromycin can reduce the incidence of parenteral nutrition-associated cholestasis by almost 50\% and decreases the incidence of recurrent septicemia. None of the RCTs reported any sinister adverse effects, in particular, hypertrophic infantile pyloric stenosis or fatal cardiac arrhythmia. Nonetheless, as long-term outcomes have not been fully evaluated, neonatologists should use this treatment cautiously and selectively in preterm infants with moderately severe GI dysmotility. [\hyperlink{Erythromycin}{PMID: 19218823}, Pak C Ng et al., 2009]

\hypertarget{pmid_7667050}{W}e evaluated 260 previously healthy children ages 3 through 12 years who had clinical signs and symptoms of pneumonia, radiographically confirmed. Patients were randomized 1:1 to a 10-day course of either clarithromycin suspension 15 mg/kg/day divided twice a day or erythromycin suspension 40 mg/kg/day divided twice a day or three times a day. Evidence of infection with Chlamydia pneumoniae was detected in 28\% (74) of patients: 13\% (34) by nasopharyngeal culture and 18\% (48) by serology with the microimmunofluorescence assay. Evidence of infection with Mycoplasma pneumoniae was detected in 27\% (69) of patients: 20\% (53) by nasopharyngeal culture or polymerase chain reaction and 17\% (44) by serology with the use of enzyme-linked immunosorbent assay. Serologic confirmation of infection was observed in 23\% (8) and 53\% (28) of patients with bacteriologically detected C. pneumoniae and M. pneumoniae, respectively. Treatment with clarithromycin vs. erythromycin, respectively, yielded the following outcomes: clinical success 98\% (121 of 124) vs. 95\% (105 of 110); radiologic success 98\% (109 of 111) vs. 94\% (92 of 110); and eradication by pathogen, C. pneumoniae 79\% (15 of 19) vs. 86\% (12 of 14) and M. pneumoniae 100\% (9 of 9) vs. 100\% (4 of 4). Adverse events were primarily gastrointestinal occurring in almost one-fourth of patients in both groups, and were mild to moderate in severity. Clarithromycin and erythromycin were similarly effective and safe for the treatment of radiographically proved, community-acquired pneumonia in children older than 2 years old.(ABSTRACT TRUNCATED AT 250 WORDS) [\hyperlink{Erythromycin}{PMID: 7667050}, S Block et al., 1995] Erythromycin is recommended for secondary prophylaxis in children with rheumatic heart disease, who are allergic to penicillin. A 9-year-old girl, with rheumatic heart disease, on secondary prophylaxis with erythromycin 250 mg BD, presented with acute rheumatic fever. Responded to steroids and started on a higher dose (250 mg TDS) of erythromycin for secondary prophylaxis. There is need to document the resistance of group A streptococci to erythromycin. [\hyperlink{Erythromycin}{PMID: 7667050}, Dinesh Kumar Yadav et al., 2013]

\hypertarget{pmid_8195854}{T}he objective of the study was the comparison of the efficacy and tolerability of brodimoprim to those of erythromycin in children with acute tonsillitis or bronchitis. 50 children aged 0.5 to 9.3 years were included in the study, 25 treated either with brodimoprim or with erythromycin. The evaluation of the therapeutic response was based exclusively on clinical criteria. In the brodimoprim group the therapy was successful in 24 patients (one failure), in the erythromycin group the therapy was also successful in 24 children (one failure). Side effects: three patients treated with brodimoprim reported adverse reactions (stomatitis, vomiting, skin rash), whereas only one patient in the erythromycin group developed a skin rash. both therapeutic regimens were equally effective against bronchitis and tonsillitis in children. The tolerability was good in both groups. [\hyperlink{Erythromycin}{PMID: 8195854}, R Salzberg et al., 1993]

\hypertarget{pmid_7782112}{I}n two multicentre, non-comparative studies, a total of 477 children, aged 2 months to 15 years, suffering from respiratory tract infections or skin and soft tissue infections were treated with roxithromycin (50 mg sachets). The mean duration of treatment was 9 days and the mean daily dose was 6 mg/kg/day administered b.i.d. The overall safety of roxithromycin was assessed by analysing the adverse events reported by the patients or investigators, the discontinuation of treatment because of adverse events, and the laboratory data. Adverse events reported by 20 (4\%) children were considered to be possibly drug related. These adverse events were mainly digestive in nature and mild or moderate in severity. Treatment was discontinued because of adverse events in six (1\%) children. Analysis of laboratory data revealed a few variations without any clinical significance. [\hyperlink{Erythromycin}{PMID: 7782112}, P Bégué et al., 1995]

\section*{Omeprazole}
\subsection*{Result}
\subsubsection*{Answer}

Infants (0–12 months): Yes  
Children (1–16 years): Yes  
Long-term use (all ages): Unknown  

\subsubsection*{{Explanation}}
\hypertarget{Omeprazole}
Based on the abstracts reviewed, there is evidence from multiple targeted studies evaluating the safety of omeprazole in children across various age ranges. Here is a summary by age group:

Infants (0–12 months):
- Several studies specifically included infants. One randomized trial in infants (6–12 weeks old) with GERD compared rectal and oral omeprazole, finding both routes effective in increasing intraesophageal and gastric pH, with no significant safety concerns reported in the abstract [\hyperlink{pmid_32594305}{PMID: 32594305}, Petra Bestebreurtje et al., 2020].
- Another study treated 12 infants (mean age \textasciitilde{}3 months) with omeprazole 0.5 mg/kg daily for 6 weeks, reporting marked improvement in symptoms and no mention of adverse effects [\hyperlink{pmid_9506656}{PMID: 9506656}, P Alliët et al., 1998].
- A review of omeprazole use in infants under 1 year found only three clinical trials in this age group. These studies showed an effect on gastric acidity but not on symptoms, and while side effects were possible, few were reported. The review notes dosing challenges and uncertainty about the preferred treatment for crying infants, but does not report significant safety concerns [\hyperlink{pmid_20719016}{PMID: 20719016}, Robert G T Blokpoel et al., 2010].
- A study developing a pediatric omeprazole suppository for infants states that clinical studies are still needed to establish safety in this population [\hyperlink{pmid_32594306}{PMID: 32594306}, Petra Bestebreurtje et al., 2020].

Children (1–16 years):
- Multiple studies, including large multicenter trials, evaluated omeprazole for erosive esophagitis and GERD in children aged 1–16 years. These studies consistently report that omeprazole is well tolerated and safe for short-term use, with high rates of symptom relief and mucosal healing [\hyperlink{pmid_11113836}{PMID: 11113836}, E Hassall et al., 2000; \hyperlink{pmid_21694842}{PMID: 21694842}, Alice Monzani et al., 2010; \hyperlink{pmid_8320610}{PMID: 8320610}, T S Gunasekaran et al., 1993; \hyperlink{pmid_9161946}{PMID: 9161946}, C De Giacomo et al., 1997; \hyperlink{pmid_16641575}{PMID: 16641575}, Rok Orel et al., 2006].
- One study in critically ill children (1 month–14 years) found intravenous omeprazole to be hemodynamically safe [\hyperlink{pmid_22818224}{PMID: 22818224}, M J Solana et al., 2013].
- Pharmacokinetic studies in children (1–16 years) show similar parameters to adults, with higher metabolic capacity in younger children, and report good tolerability [\hyperlink{pmid_11095324}{PMID: 11095324}, T Andersson et al., 2000; \hyperlink{pmid_7859807}{PMID: 7859807}, E Jacqz-Aigrain et al., 1994].
- A randomized controlled trial in children with peptic ulcer and H. pylori infection (age not specified, but at a children's hospital) found no significant difference in adverse reactions between treatment groups [\hyperlink{pmid_33235586}{PMID: 33235586}, Shaohui Zhang et al.].
- A review of pediatric PPI use found a large body of clinical evidence supporting the safety and efficacy of omeprazole in children, and noted that omeprazole has a pediatric indication in Europe [\hyperlink{pmid_18797857}{PMID: 18797857}, Giovanni Tafuri et al., 2009].

Adverse Events:
- There is a single case report of omeprazole-induced hepatitis in a child, but this is presented as a rare event [\hyperlink{pmid_16096600}{PMID: 16096600}, Wael El-Matary et al., 2005].
- Some studies report mild, reversible elevations in serum gastrin or liver enzymes, but no serious or persistent adverse effects [\hyperlink{pmid_8320610}{PMID: 8320610}, T S Gunasekaran et al., 1993; \hyperlink{pmid_9952234}{PMID: 9952234}, R S Strauss et al., 1999].

Summary:
- For infants under 1 year, there are a few targeted studies, mostly in infants with GERD or esophagitis, that suggest omeprazole is generally safe for short-term use, though data are more limited and some reviews call for further studies.
- For children aged 1–16 years, there is strong evidence from multiple targeted studies and reviews that omeprazole is safe and well tolerated for short-term use in the treatment of GERD, erosive esophagitis, and peptic ulcer disease.
- Long-term safety, especially regarding chronically elevated gastrin levels, is less well established and may require further study.

\subsection*{Abstracts}
\hypertarget{pmid_21694842}{O}meprazole is a proton-pump inhibitor indicated for gastroesophageal reflux disease and erosive esophagitis treatment in children. The aim of this review was to evaluate the efficacy of delayed-release oral suspension of omeprazole in childhood esophagitis, in terms of symptom relief, reduction in reflux index and/or intragastric acidity, and endoscopic and/or histological healing. We systematically searched PubMed, Cochrane and EMBASE (1990 to 2009) and identified 59 potentially relevant articles, but only 12 articles were suitable to be included in our analysis. All the studies evaluated symptom relief and reported a median relief rate of 80.4\% (range 35\%-100\%). Five studies reported a significant reduction of the esophageal reflux index within normal limits (<7\%) in all children, and 4 studies a significant reduction of intra-gastric acidity. The endoscopic healing rate, reported by 9 studies, was 84\% after 8-week treatment and 95\% after 12-week treatment, the latter being significantly higher than the histological healing rate (49\%). In conclusion, omeprazole given at a dose ranging from 0.3 to 3.5 mg/kg once daily (median 1 mg/kg once daily) for at least 12 weeks is highly effective in childhood esophagitis. [\hyperlink{Omeprazole}{PMID: 21694842}, Alice Monzani et al., 2010]

\hypertarget{pmid_30966193}{O}meprazole (OME) is employed for treating ulcer in children, but is unstable and exhibits first pass metabolism via the oral route. This study aimed to stabilize OME within mucoadhesive metolose (MET) films by combining cyclodextrins (CD) and l-arginine (l-arg) as stabilizing excipients and functionally characterizing for potential delivery via the buccal mucosa of paediatric patients. Polymeric solutions at a concentration of 1\%  [\hyperlink{Omeprazole}{PMID: 30966193}, Sajjad Khan et al., 2018] Omeprazole is a proton pump inhibitor that is used in acid suppression therapy in infants. Infants cannot swallow the oral tablets or capsules. Since, infants require a non-standard dose of omeprazole, the granules or tablets are often crushed or suspended in water or sodium bicarbonate, which may destroy the enteric coating. In this study we explore the efficacy and pharmacokinetics of rectally administered omeprazole in infants with gastroesophageal reflux disease (GERD) due to esophageal atresia (EA) or congenital diaphragmatic hernia (CDH) and compare these with orally administered omeprazole. Infants (6-12 weeks postnatal and bodyweight > 3 kg) with EA or CDH and GERD were randomized to receive a single dose of 1 mg/kg omeprazole rectally or orally. The primary outcome was the percentage of infants for whom omeprazole was effective according to predefined criteria for 24-h intraesophageal pH. Secondary outcomes were the percentages of time that gastric pH was < 3 or < 4, as well as the pharmacokinetic parameters. Seventeen infants, 4 with EA and 13 with CDH, were included. The proportion of infants for whom omeprazole was effective was 56\% (5 of 9 infants) after rectal administration and 50\% (4 of 8 infants) after oral administration. The total reflux time in minutes and percentages and the number of reflux episodes of pH < 4 decreased statistically significantly after both rectal and oral omeprazole administration. Rectal and oral administration of omeprazole resulted in similar serum exposure. A single rectal omeprazole dose (1 mg/kg) results in consistent increases in intraesophageal and gastric pH in infants with EA- or CDH-related GERD, similar to an oral dose. Considering the challenges with existing oral formulations, rectal omeprazole presents as an innovative, promising alternative for infants with pathological GERD. ClinicalTrials.gov Identifier: NCT00226044. [\hyperlink{Omeprazole}{PMID: 30966193}, Petra Bestebreurtje et al., 2020]

\hypertarget{pmid_8320610}{O}meprazole, a potent inhibitor of acid secretion, is effective in adults with severe gastroesophageal reflux, but no such data are available on children. We studied 15 children in whom treatment with histamine (type 2) blockers and prokinetic agents had failed; 4 had also had one or more fundoplications. Their ages were 0.8 to 17 years (mean, 8.1 years) and weights were 7.5 to 30.7 kg (mean, 18.6 kg). Of the 15 children, 8 were neurologically handicapped. All patients had endoscopic and histologic evidence of esophagitis; most had esophagitis grade 3 to 4. Patients were initially given omeprazole at 10 to 20 mg; the dose was titrated upward until results of a subsequent 24-hour intraesophageal pH study was normal. Symptoms and signs abated and evidence of esophagitis diminished in all patients. Omeprazole was given for periods of 5.5 to 26 months (mean, 12.2 months). The effective total dose was 20 to 40 mg (0.7 to 3.3 mg/kg) in 11 patients, 10 mg (0.7 mg/kg) in 1 patient, and 60 mg (1.9 to 2.4 mg/kg) in 3 patients. The dosage range was 0.7 to 3.3 to mg/kg per day (mean, 1.9 mg/kg). Mildly elevated transaminase values in 7 patients and elevated fasting gastrin levels in 11 patients were present; in 6 of the 11, gastrin levels were 3 to 5.5 times the upper limit of normal. We found omeprazole to be highly effective in this group of patients with severe esophagitis refractory to other measures. We recommend a starting dose of 0.7 mg/kg as a single morning dose; the adequacy of reflux control is then determined by follow-up 24-hour intraesophageal pH studies. Omeprazole appears to be safe for short-term use, but further studies are needed to assess long-term safety because the significance of chronically elevated gastrin levels in children is unknown. [\hyperlink{Omeprazole}{PMID: 8320610}, T S Gunasekaran et al., 1993]

\hypertarget{pmid_11095324}{T}he aim of this study was to examine the pharmacokinetics of orally administered omeprazole in children. Plasma concentrations of omeprazole were measured at steady state over a 6-h period after administration of the drug. Patients were a subset of those in a multicenter study to determine the dose, safety, efficacy, and tolerability of omeprazole in the treatment of erosive reflux esophagitis in children. Children were 1-16 yr of age, with erosive esophagitis and pathological acid reflux on 24 h-intraesophageal pH study. The "healing dose" of omeprazole was that at which subsequent intraesophageal pH study normalized. Children remained on this dose for 3 months, and during this period the pharmacokinetics were measured. A total of 57 children were enrolled in the overall healing phase of the study. Pharmacokinetic study was optional for subjects and was performed in 25 of the 57 enrolled. The doses of omeprazole required were substantially higher doses per kilogram of body weight than in adults. Values of the pharmacokinetic parameters of omeprazole were generally within the ranges previously reported in adults. However, the plasma levels, area under the plasma concentration versus time curve (AUC), plasma half-life (t(1/2)), and maximal plasma concentration (Cmax), were lower in the younger age group, when the AUC and Cmax were normalized to a dose of 1 mg/kg. Furthermore, within the group as a whole, these values showed a gradation from lowest in the children 1-6 yr of age to higher in the older age groups. The pharmacokinetics of omeprazole in children showed a trend toward higher metabolic capacity with decreasing age, being highest at 1-6 yr of age. This may explain the need for higher doses of omeprazole on a per kilogram basis, not only in children overall compared with adults but, in many cases, particularly in younger children. [\hyperlink{Omeprazole}{PMID: 11095324}, T Andersson et al., 2000]

\hypertarget{pmid_22818224}{C}ritical patients usually have hemodynamic disturbances which may become worse by the administration of some drugs. Omeprazole is a drug used in the prophylaxis of the gastrointestinal bleeding in these patients, but its cardiovascular effects are unknown. The objective was to study the hemodynamic changes produced by intravenous omeprazole in critically ill children and to find out if there are differences between two different doses of omeprazole. A randomized prospective observational study was performed on 37 critically ill children aged from 1 month to 14 years of age who required prophylaxis for gastrointestinal bleeding. Of these, 19 received intravenous omeprazole 0.5mg/kg every 12 hours, and 18 received intravenous omeprazole 1mg/kg every 12 hours. Intravenous omeprazole was administered in 20 minutes by continuous infusion pump. Heart rate, systolic, diastolic and mean arterial blood pressure, central venous pressure and ECG were recorded at baseline, and at 15, 30, 60 and 120 minutes of the infusion. There were no significant changes in the electrocardiogram, heart rate, blood pressure and central venous pressure. No patients required inotropic therapy modification. There were no differences between the two doses of omeprazole. Intravenous omeprazole administration of 0.5mg/kg and 1mg/kg is a hemodynamically safe drug in critically ill children. [\hyperlink{Omeprazole}{PMID: 22818224}, M J Solana et al., 2013]

\hypertarget{pmid_20719016}{T}o determine the role of omeprazole treatment in crying infants under the age of 1 year in whom acid gastroesophageal reflux is suspected and to study the evidence for efficacy, prescribing behaviour and side effects of this medicine, which is not registered for use in infants. Literature study. To assess efficacy we conducted a study of the literature using PubMed with the search terms 'gastro-esophageal reflux disease', 'crying', 'adverse drug reactions' and 'omeprazole', in the age category 'all infants 0-23 months' We used the medicine prescription database Interactie DataBase to assess prescribing data and studied reports of suspected side effects of omeprazole in children younger than 1 year to the Lareb Netherlands pharmacovigilance centre. We found 139 articles including 32 clinical trials. In only 3 of these was the efficacy of omeprazole studied in children under the age of 1 year. These studies showed that there was an effect on the acidity of the stomach, but not on symptoms. Although many side effects may occur during the use of omeprazole, few suspected side effects were reported to the Lareb Netherlands pharmacovigilance centre. Omeprazole is supplied in 10 mg amounts and it is therefore difficult to adjust dose to weight. Pharmacoepidemiological data show therefore that nearly all children receive 10 mg or multiples thereof. Given the age and corresponding weights we expected doses of 4-20 mg/day to be prescribed. It is uncertain whether acid reflux is the cause of crying in babies and, if reflux is suspected, whether omeprazole is the preferred treatment. [\hyperlink{Omeprazole}{PMID: 20719016}, Robert G T Blokpoel et al., 2010]

\hypertarget{pmid_32594306}{O}meprazole is a proton pump inhibitor (PPI) that is used in acid suppression therapy in infants. In this study we aimed to develop a pediatric omeprazole suppository, with good physical and chemical stability, suitable for pharmaceutical batch production. The composition of the suppository consisted of omeprazole, witepsol H15 and arginine (L) base. To achieve evenly distributed omeprazole suspension suppositories, the temperature, stirring rate, and arginine (L) base amount were varied. A previously validated quantitative high-performance liquid chromatography-ultraviolet method was modified and a long-term stability study was performed for one year. Evenly distributed omeprazole suspension suppositories were obtained by adding 100 mg arginine (L) base and pouring at a temperature of 34.7 °C and a stirring speed of 200 rpm. The long-term stability study showed no signs of discoloration and a stable omeprazole content between 90 and 110\% over 1 year if stored in the dark at room temperature. We developed a pediatric omeprazole suppository. This formulation may provide a good alternative to manipulated commercial or extemporaneously compounded omeprazole oral formulations for infants. Clinical studies are needed to establish efficacy and safety in this young population. [\hyperlink{Omeprazole}{PMID: 32594306}, Petra Bestebreurtje et al., 2020]

\hypertarget{pmid_16096600}{O}meprazole; the first proton pump inhibitor (PPI) showing an effective acid inhibitory ability, provides the satisfactory therapy either in gastro-esophageal reflux symptom relief or in healing of erosive esophagitis. It's also effective in peptic ulcer disease. Up to date, omeprazole efficacy and safety are well established in many trials. Omeprazole-related hepatotoxicity is not very well recognized especially in pediatric population. We report a child who developed hepatitis following omeprazole intake. We believe that this is the first case report of omeprazole-induced hepatitis in pediatric population. [\hyperlink{Omeprazole}{PMID: 16096600}, Wael El-Matary et al., 2005]

\hypertarget{pmid_8877348}{F}ollowing failure of conventional therapy for reflux oesophagitis, 15 children were treated with omeprazole 20 mg daily for a period of up to three months initially. Treatment resulted in a marked symptomatic improvement as measured by incidence of pain, vomiting, dysphagia and haematemesis. Four children failed treatment and required fundoplication. No complications from the use of omeprazole were recorded and some children have continued long-term treatment. [\hyperlink{Omeprazole}{PMID: 8877348}, P B Martin et al., 1996]

\hypertarget{pmid_33235586}{T}o compare curative effect and safety of omeprazole under different treatment courses in treatment of children with peptic ulcer (PU, diameter≤1.0cm) and helicobacter pylori (HP) infection and its influence on inflammatory cytokines. The study was a randomized controlled study and conducted at Baoding children's hospital from June 2015 to June 2018. In this study 100 PU children with positive HP were chosen and classified into two groups at random. The 58 cases in the observation group were given omeprazole + amoxicillin + clarithromycin, and the antibiotics were not used two weeks later. Then, omeprazole was used to treat for two weeks. 42 cases in the control group were given omeprazole + amoxicillin + clarithromycin for two weeks. Curative effect, HP eradication rate, clinical symptoms, incidence of adverse reactions, level of serum inflammatory cytokine interleukin-6 (IL-6) and level of tumor necrosis factor-a (TNF-a) in two groups were compared. Total effective rate, HP eradication rate and clinical symptom relief of observation group were better than those of control group, and the differences showed statistical significance (P>0.05). The differences of two groups in the incidence of adverse reactions had no statistical significance (P>0.05). Serum IL-6 level and TNF-a level of observation group were significantly lower than those of control group and before the treatment, and the differences had statistical significance (P>0.05). The application of omeprazole in treatment of PU patients with positive HP for four weeks can significantly improve PU cure rate and HP eradication rate, relieve clinical symptoms and reduce inflammatory response, so it deserves to be promoted clinically. [\hyperlink{Omeprazole}{PMID: 33235586}, Shaohui Zhang et al., ]

\hypertarget{pmid_9506656}{T}welve neurologically normal infants (age 2.9+/-0.9 months) with peptic esophagitis (grade 2) who did not respond to cimetidine (in addition to positioning, cisapride, and Gaviscon) were treated with omeprazole, 0.5 mg/kg once a day, for 6 weeks. The effectiveness of omeprazole was evaluated in all infants by clinical assessment and endoscopy before and after treatment and by 24-hour gastric pH monitoring during treatment in seven infants. Omeprazole therapy led to a marked decrease in symptoms, endoscopic and histologic signs of esophagitis, and intragastric acidity. [\hyperlink{Omeprazole}{PMID: 9506656}, P Alliët et al., 1998]

\hypertarget{pmid_12970637}{T}o assess the efficacy of omeprazole in treating irritable infants with gastroesophageal reflux and/or esophagitis. Irritable infants (n=30) 3 to 12 months of age met the entry criteria of esophageal acid exposure >5\% (n=22) and/or abnormal esophageal histology (n=15). They completed a 4-week, randomized, double-blind, placebo-controlled crossover trial of omeprazole. Cry/fuss diary (minutes/24 hours) and a visual analogue scale of infant irritability as judged by parental impression were obtained at baseline and the end of each 2-week treatment period. The reflux index fell significantly during omeprazole treatment compared with placebo (-8.9\%+/-5.6\%, -1.9\%+/-2.0\%, P<.001). Cry/fuss time decreased from baseline (267+/-119), regardless of treatment sequence (period 1, 203+/-99, P<.04; period 2, 188+/-121, P<.008). Visual analogue score decreased from baseline to period 2 (6.8+/-1.6, 4.8+/-2.9, P=.008). There was no significant difference for both outcome measures while taking either omeprazole or placebo. Compared with placebo, omeprazole significantly reduced esophageal acid exposure but not irritability. Irritability improved with time, regardless of treatment. [\hyperlink{Omeprazole}{PMID: 12970637}, David John Moore et al., 2003]

\hypertarget{pmid_14749542}{S}tudies of the pharmacokinetics of omeprazole in children with gastroesophageal reflux disease (GERD) remain scarce despite the vast number of reports on its efficacy. The objectives of this study were to assess the pharmacokinetics of omeprazole in healthy adults and in children with GERD. Omeprazole (Losec, delayed-release capsules) was administered orally to 18 healthy adults (mean age 36.8 years) and 12 children with GERD (mean age 6.1 years). Blood samples were collected over 5 hours, and plasma concentrations were assessed using liquid chromatography. Population pharmacokinetic parameters were calculated using NONMEM. A 1-compartment model with zero-order absorption and a lag time was used. The population approach was well suited to the limited number of samples available, and residual variability was low. Oral clearance (CL/F) and apparent volume of distribution (V(ss)/F) in healthy adults (Mean +/- SD: 0.62 +/- 0.27 L/h/kg and 0.76 +/- 0.26 L/kg, respectively) were not significantly different than those in children with GERD (0.51 +/- 0.34 L/h/kg and 0.66 +/- 0.25 L/kg, respectively). Healthy adults displayed a statistically significantly longer delay in drug absorption (Lag time: 0.62 +/- 0.15 hours) as compared with that observed in children with GERD (0.12 +/- 0.03 hours, P < 0.05). On the basis of these findings, omeprazole dosings on a milligram-per-kilogram basis are recommended with no further adjustments for the treatment of GERD in children. [\hyperlink{Omeprazole}{PMID: 14749542}, Jean-Francois Marier et al., 2004]

\hypertarget{pmid_9161946}{S}evere esophagitis is a rare complication of gastroesophageal reflux in children. In adults, omeprazole therapy of severe erosive esophagitis has become the gold standard short-term treatment of the disease. In children, data on its use are limited, and problems about the dosage are unresolved. The aim of this study was to evaluate the efficacy of a simplified, body-weight-based daily dosage of omeprazole in children with severe esophagitis. Ten children (median age 75.6 months; range 25-109 months) with severe esophagitis were prospectively investigated. All patients were evaluated by endoscopy, histology, and 24-h pH-metry study before and after 3 months of omeprazole. The starting dose of omeprazole was 20 mg as a single daily dose in children weighing less than 30 kg, and 40 mg daily for those weighing over 30 kg. A significant improvement in all the children was demonstrated after 3 months of treatment by clinical, endoscopic, and pH-metry assessment. However, histologic study failed to show significant improvement of both inflammatory and hyperplastic findings. Relapse occurred in six of 10 patients after discontinuation of therapy. Omeprazole is effective in the short-term treatment of severe oesophagitis in children. The daily dose of the drug could be easily based on the body weight. The persistence of histologic features of esophagitis in spite of clinical and endoscopic healing could be an indicator of poor outcome. [\hyperlink{Omeprazole}{PMID: 9161946}, C De Giacomo et al., 1997]

\hypertarget{pmid_7859807}{T}his study was undertaken to define the pharmacokinetics of omeprazole in children and included 13 patients, heterogeneous in terms of age (0.3 to 19 years), underlying disease and biological constants, indication of omeprazole administration and associated therapy. The dose administered ranged from 36.9 to 139 mg.1.73 m-2. The pharmacokinetic parameters of omeprazole were: systemic clearance, 0.23 l.kg-1.h-1; volume of distribution, 0.45 l.kg-1; elimination half life 0.86 h; but were highly variable between individuals. Dosage, differences in hepatic and renal function and associated therapy may contribute to inter-individual variability. Within the range of doses administered, the pharmacokinetic parameters were similar to those reported in adults. The drug has been well tolerated in all children. [\hyperlink{Omeprazole}{PMID: 7859807}, E Jacqz-Aigrain et al., 1994]

\hypertarget{pmid_9952234}{M}any children with esophagitis demonstrate histological changes without gross evidence of esophagitis by esophagoscopy. The effect of omeprazole on the histological healing of esophagitis in children is unknown. Therefore, the aim of this study was to determine the effect of omeprazole on refractory histological esophagitis in pediatric patients. Eighteen patients with histological evidence of esophagitis and recurrent symptoms despite therapy with H2-receptor antagonists and prokinetic agents were prospectively treated with omeprazole. Dosing was adjusted by monitoring intragastric pH, and esophagoscopy was repeated after 8-12 weeks of omeprazole treatment. Two patients did not complete the study due to either worsening symptoms or hypergastrinemia. Of the remaining patients, 76\% were asymptomatic with omeprazole treatment and 24\% reported improvement in their symptoms. Approximately 40\% demonstrated complete histological healing of their esophagitis. Three patients (17\%) had persistent elevations in serum gastrin levels while on omeprazole treatment, which was associated with both younger patient age and higher omeprazole dosing; however, all elevated gastrin levels returned to normal after discontinuation of the medication. All patients had recurrence of their symptoms after completing a course of omeprazole, even patients with complete histological healing. Omeprazole is efficacious in treating children with esophagitis refractory to H2-receptor antagonist and prokinetic agents. However, none of the patients were able to discontinue acid suppressive therapy even after documented healing of their esophagitis. [\hyperlink{Omeprazole}{PMID: 9952234}, R S Strauss et al., 1999]

\hypertarget{pmid_26398674}{A}lthough, omeprazole is widely used for treatment of gastric acid-mediated disorders. However, its pharmacokinetic and chemical instability does not allow simple aqueous dosage form formulation synthesis for therapy of, especially child, these patients. The aim of this study was at first preparation of suspension dosage form omeprazole and second to compare the blood levels of 2 oral formulations/dosage forms of suspension \& granule by high performance liquid chromatography (HPLC). The omeprazole suspension was prepared; in this regard omeprazole powder was added to 8.4\% sodium bicarbonate to make final concentration 2 mg/ml omeprazole. After that a randomized, parallel pilot trial study was performed in 34 pediatric patients with acid peptic disorder who considered usage omeprazole. Selected patients were received suspension and granule, respectively. After oral administration, blood samples were collected and analyzed for omeprazole levels using validated HPLC method. The mean omeprazole blood concentration before usage the next dose, (trough level) were 0.12±0.08 µg/ml and 0.18±0.15 µg/ml for granule and suspension groups, respectively and mean blood level after dosing (C2 peak level) were 0.68±0.61 µg/ml and 0.86±0.76 µg/ml for granule and suspension groups, respectively. No significant changes were observed in comparison 2 dosage forms 2 h before (P=0.52) and after (P=0.56) the last dose. These results demonstrate that omeprazole suspension is a suitable substitute for granule in pediatrics. [\hyperlink{Omeprazole}{PMID: 26398674}, S Karami et al., 2016]

\hypertarget{pmid_11113836}{T}o determine the efficacy, safety, and tolerability of omeprazole in children and to determine the doses required to heal chronic, severe esophagitis. Open multicenter study in children aged 1 to 16 years with erosive reflux esophagitis. The healing dose of omeprazole used was that with which the duration of acid reflux was <6\% of a 24-hour intraesophageal pH study. Follow-up endoscopy was performed after 3 months of treatment with the healing dose. At entry, two thirds of 57 patients who completed the study had esophagitis grade 3 or 4 (scale 0-4); some 50\% had neurologic impairment or repaired esophageal atresia. Of the 57 patients, 54 healed; 3 did not heal and left the study, and 3 healed with a second course. Doses required for healing were 0.7 to 3.5 mg/kg/d: 0.7 mg/kg/d in 44\% of patients and 1.4 mg/kg/d in another 28\%. Healing dose correlated with grade of esophagitis but not with age or underlying disease. Reflux symptoms improved dramatically in almost all of the 57 patients, including the unhealed patients. Omeprazole is well tolerated, highly effective, and safe for treatment of erosive esophagitis and symptoms of gastroesophageal reflux in children, including children in whom antireflux surgery or other medical therapy has failed. On a per-kilogram basis, the doses of omeprazole required to heal erosive esophagitis are much greater than those required for adults. [\hyperlink{Omeprazole}{PMID: 11113836}, E Hassall et al., 2000]

\hypertarget{pmid_15773802}{I}n last years the use in the pediatric area of proton pump inhibitors (omeprazole, lansoprazole, pantoprazole, rabeprazole and esomeprazole) is more often, nevertheless the clinical trials carried out are poor. The aim of this work is to analyse the bibliography published about this kind of drugs in children and to make a revision of its use in the last seven years. More studies with omeprazole and lansoprazole have been developed, to be exact omeprazole and lansoprazole is present in 122 bibliographic appointments and 34 for lansoprazole, which include studies that demonstrate a good tolerance and efficacy. The remaining proton pump inhibitors count with very few studies. The main therapeutic indications were the eradication of Helicobacter pylori, gastroesophageal reflux disease and esophagitis. The number of patients included in the reviewed studies is quite heterogeneous, from 8 to 122 and the age range between 8 days and 17 years. On the other hand, it could be highlighted the non-existence of formulations adapted to the pediatric population and the difficulty of administration specially in the youngest patients. As in many other drugs, it would be necessary to carry out clinical trials in order to determinate the pharmacologic parameters at difference ages, which will allow a safe and effective administration, and its authorization by all Health Authorities. [\hyperlink{Omeprazole}{PMID: 15773802}, J Carcelén Andrés et al., ]

\hypertarget{pmid_18797857}{I}n some cases of drug therapy, the available evidence might be sufficient to extend the indications to children without further clinical studies. We reviewed the available evidence for one of the categories of drugs most frequently used off-label in children: proton pump inhibitors (PPIs) used for the treatment of gastroesophageal reflux disease (GERD). A classification of the appropriateness of off-label use of PPIs in children with GERD was also performed. Of the five PPIs evaluated, only omeprazole has a paediatric indication in Europe. Overall, 19 clinical trials were retrieved and evaluated on the basis of pharmacokinetics, efficacy and safety data. The off-label use of omeprazole, esomeprazole and lansoprazole in children was evaluated as appropriate given the consistent available evidence retrieved in literature. This study demonstrates the existence of a large body of clinical evidence on the use of PPIs in children. Regulatory agencies and ethical committees should cope with this issue for ethical reasons to avoid unnecessary trial replication. [\hyperlink{Omeprazole}{PMID: 18797857}, Giovanni Tafuri et al., 2009]

\hypertarget{pmid_32866648}{O}meprazole (OME) is often used to treat disorders associated with gastric hypersecretion in children but a liquid pediatric formulation of this medicine is not currently available. The aim of this study is to develop OME loaded nanoparticles with a view to the obtention of a liquid pharmaceutical dosage form. Eudragit® RS100 was selected as the skeleton material in the inner core and pH-sensitive Eudragit® L100-55 was used as the outer coating of the nanoparticles prepared by the nanoprecipitation method. Pharmacological activity was evaluated by induction of ethanol ulcers in mice. The OME nanoparticles exhibited mean diameters of 174 nm (±17), polydispersity index of 0.229 (±0.01), zeta potential values of -13 mV (±2.60) and encapsulation efficiency of 68.1\%. The in vivo pharmacological assessment showed the ability of nanoparticles to protect mice stomach against ulcer formation. The prepared suspension of OME nanoparticles represents effective therapeutic strategy in a liquid pharmaceutical form with the possibility of pediatric administration. [\hyperlink{Omeprazole}{PMID: 32866648}, Helissara Silveira Diefenthaeler et al., 2020]

\hypertarget{pmid_34607935}{T}he over-the-counter nasal decongestant oxymetazoline (eg, Afrin) is used in the pediatric population for a variety of conditions in the operating room setting. Given its vasoconstrictive properties, it can have cardiovascular adverse effects when systemically absorbed. There have been several reports of cardiac and respiratory complications related to use of oxymetazoline in the pediatric population. Current US Food and Drug Administration approval for oxymetazoline is for patients ≥6 years of age, but medical professionals may elect to use it short-term and off label for younger children in particular clinical scenarios in which the potential benefit may outweigh risks (eg, active bleeding, acute respiratory distress from nasal obstruction, acute complicated sinusitis, improved surgical visualization, nasal decongestion for scope examination, other conditions, etc). To date, there have not been adequate pediatric pharmacokinetic studies of oxymetazoline, so caution should be exercised with both the quantity of dosing and the technique of administration. In the urgent care setting, emergency department, or inpatient setting, to avoid excessive administration of the medication, medical professionals should use the spray bottle in an upright position with the child upright. In addition, in the operating room setting, both monitoring the quantity used and effective communication between the surgeon and anesthesia team are important. Further studies are needed to understand the systemic absorption and effects in children in both nonsurgical and surgical nasal use of oxymetazoline. [\hyperlink{Omeprazole}{PMID: 34607935}, Richard Cartabuke et al., 2021]

\hypertarget{pmid_2691312}{O}meprazole is a very potent inhibitor of gastric acid secretion and has proven to be efficacious in the healing of peptic ulcer and reflux oesophagitis. A search for adverse events during short-term treatment with omeprazole has been made, based on data from published comparative trials, data on file at the manufactor's (Hässle Research Laboratories, Mölndal, Sweden) and personal series. Omeprazole does not show more adverse events than drugs currently widely in use for the treatment of acid-related disorders. A change in a wide range of laboratory parameters has not been observed, except for a rise in basal and meal-stimulated serum gastrin which can be ascribed directly to the inhibition of acid secretion. For short-term treatment omeprazole can be considered as a safe drug. [\hyperlink{Omeprazole}{PMID: 2691312}, G F Nelis et al., 1989]

\hypertarget{pmid_16641575}{R}eflux of duodenal juice into the oesophagus has a role in the pathogenesis of both oesophageal and laryngopharyngeal inflammatory and neoplastic lesions. As little is known about effective therapy, we studied the effect of proton pump inhibitor therapy on oesophageal bile reflux in children. Twenty-nine children with moderate to severe erosive oesophagitis and abnormal oesophageal bile reflux were studied before and after treatment with omeprazole 1 mg/kg per day. Outcomes included a clinical symptom score, oesophageal acid and bile reflux (simultaneous 24-hour pH and Bilitec 2000 monitoring), and mucosal healing. After 8 weeks of therapy, 17 (59\%) of the patients were symptom-free, and 5 (17\%) had minimal symptoms. Mucosal healing or reduction to low-grade oesophagitis was achieved in 25 children (86\%; P < 0.0005). Mean percentages of total, upright, and supine time with oesophageal pH less than 4 were reduced from 17.0\%, 16.8\%, and 19.2\% before treatment, to 2.83\%, 3.17\%, and 2.07\%, respectively, after treatment (all P < 0.00001). Similarly, mean percentages of total, upright, and supine time with bile reflux were reduced from 16.96\%, 12.67\%, and 22.0\%, to 2.27\%, 1.91\%, and 2.23\%, respectively (P < 0.000001, P < 0.0001, and P < 0.000001, respectively). Omeprazole 1 mg/kg per day is an effective therapy for the majority of children with severe erosive oesophagitis due to abnormal isolated bile reflux or combined acid and bile reflux. It remains unclear how patients with treatment-resistant bile reflux should be managed. [\hyperlink{Omeprazole}{PMID: 16641575}, Rok Orel et al., 2006]

\section*{Monomethyl Fumarate}
\subsection*{Result}
\subsubsection*{Answer}

Unknown

\subsubsection*{{Explanation}}
\hypertarget{Monomethyl Fumarate}
A review of the available abstracts reveals that none of them are specifically about Monomethyl Fumarate (MMF) or its safety in children. Several abstracts discuss related compounds, such as dimethyl fumarate (DMF), which is a prodrug of MMF, and other unrelated medications, but there are no targeted studies in the abstracts provided that evaluate the safety of Monomethyl Fumarate in pediatric populations. Therefore, based on the abstracts available, there is not enough data to determine whether Monomethyl Fumarate is safe or unsafe for use in children of any age range.

\subsection*{Abstracts}
\hypertarget{pmid_32022483}{A}sthma affects over 6 million children in the United States alone. This study investigated the efficacy and long-term safety of mometasone furoate-formoterol (MF/F) and MF monotherapy in children with asthma. This phase 3, multicenter, randomized controlled trial evaluated metered-dose inhaler twice daily (BID) dosing with MF/F 100/10 µg or MF 100 µg in children, aged 5 to 11 years, with a history of asthma for greater than or equal to 6 months and confirmed bronchodilator reversibility, who were adequately controlled on inhaled corticosteroid/long-acting beta-agonist combination therapy for greater than or equal to 4 weeks. After a 2-week run-in on MF 100 µg BID, eligible patients received 24 weeks of double-blind treatment and were followed for safety up to 26 weeks. The primary efficacy endpoint was the change from baseline in AM postdose 60-minute AUC \%predicted FEV1\% across 12 weeks of treatment. A total of 181 participants received at least one dose of MF/F (n = 91) or MF (n = 90). MF/F was superior to MF across the 12-week evaluation period, with a treatment advantage of 5.21 percentage points (P < .001). Superior onset of action with MF/F over MF was achieved as early as 5 minutes postdose on day 1. Overall, approximately 50\% of participants experienced one or more treatment-emergent adverse events, with fewer occurring in the MF/F group. In children 5 to 11 years of age with persistent asthma, the addition of F to MF was well tolerated and provided significant, rapid, and sustained improvement in lung function compared with MF alone. [\hyperlink{Monomethyl Fumarate}{PMID: 32022483}, Cindy L J Weinstein et al., 2020]

\hypertarget{pmid_20128231}{T}he efficacy and safety of monomeric allergoid (Lofarma, Milan) have been demonstrated in adults but very few studies have examined it in children. This study therefore investigated the efficacy and safety of this sublingual immunotherapy (SLIT) at the dosage of 1000 AU five times a week without any up-dosing. Forty allergic children (17 M and 23 F, mean age 7 years, range 4-16 years), 16 with rhinitis and 24 with rhinitis and asthma, were randomized to SLIT or drug therapy. All the patients were sensitized to grass; some were also sensitized, though to a lesser extent, to Parietaria, Olea and Betulaceae. The patients were treated pre-/co-seasonally for two years. A visual analogue scale (VAS) was used at baseline and at the end of the first and second pollen seasons to rate the patients' well-being. The VAS score was significantly higher after both the first and the second year of treatment in the SLIT group than in the controls (p<0.05). It improved in comparison to baseline only in the active group. All 40 children tolerated the therapy very well. The monomeric allergoid at the dosage of 5000 AU/week thus appears to have a good efficacy and safety profile in children. [\hyperlink{Monomethyl Fumarate}{PMID: 20128231}, F Agostinis et al., 2009]

\hypertarget{pmid_36043350}{W}e investigated the efficacy and safety of fluoxetine, a selective serotonin reuptake inhibitor, for treating refractory primary monosymptomatic nocturnal enuresis in children. Children 8-18 years old with severe primary monosymptomatic nocturnal enuresis unresponsive to alarm therapy, desmopressin, and anticholinergics were screened for eligibility. After excluding children with daytime urinary symptoms, constipation, underlying urological, neuropsychiatric, endocrinological, or cardiac conditions, patients were randomly and equally assigned to 10 mg fluoxetine once daily or placebo for 12 weeks. The primary outcome was treatment response according to the International Children's Continence Society terminology. Treatment-related adverse effects and nighttime arousal were secondary outcomes. A total of 150 children were enrolled, of whom 110 (56 in fluoxetine group and 54 in placebo group) with a mean age of 11.8 (SD 2.46) years were finally analyzed. After 4 weeks, 7.1\% and 66.1\% of the fluoxetine group achieved complete response and partial response (defined as 50\%-99\% reduction of the number of wet nights), respectively, versus 0\% and 16.7\% of the placebo group ( Fluoxetine is safe treatment for refractory primary monosymptomatic nocturnal enuresis in children with good initial response that declines at 12 weeks. [\hyperlink{Monomethyl Fumarate}{PMID: 36043350}, Mohamed Hussiny et al., 2022]

\hypertarget{pmid_9831007}{T}opiramate is a sulfamate-substituted monosaccharide that has demonstrated efficacy as an antiepileptic drug in adults with partial onset seizures. Experience in children has been limited, but early reports have supported its safety and effectiveness in children as young as 2 years of age. In two infants ages 12 and 9 months, respectively, with partial seizures, the authors report excellent efficacy with good tolerability at doses up to 7.7 mg/kg. Although long-term safety and possible adverse sequelae have not been fully established in children, topiramate may represent an option for infants with high seizure frequency unresponsive to standard antiepileptic drugs. [\hyperlink{Monomethyl Fumarate}{PMID: 9831007}, S L Kugler et al., 1998]

\hypertarget{pmid_27128459}{T}his multi-center, randomized, double-blind, placebo-controlled, two-way crossover study characterized the safety, tolerability, pharmacokinetics, and pharmacodynamics of fluticasone furoate (FF) in children (5-11 years) with persistent asthma. Twenty-seven children received inhaled FF 100 µg or placebo via the ELLIPTA™ dry powder inhaler once daily for 14 days, with a ≥7 day washout period. Adverse events (AEs) were reported by eight (31\%) and four (16\%) subjects during FF 100 µg and placebo treatment, respectively. Headache was reported by three subjects during FF 100 µg treatment and by no subjects during placebo treatment, all other AEs were reported by only one subject on either treatment; there were no serious AEs. Following repeat dosing, the arithmetic mean (SD) FF Cmax was 26.71 pg/mL (9.16) at 31 minutes post-dose. Arithmetic mean (SD) FF AUC(0-t) was 121.44 pg h/mL (83.04). Arithmetic mean values for weighted mean (SD) serum cortisol (0-12 hours) on day 14 were 56.49 (16.51) and 67.57 (20.66) ng/mL for FF 100 µg and placebo, respectively. No clinically significant effect of FF on serum cortisol levels was observed. FF was well tolerated. Pharmacokinetic profiles were well defined and did not differ between age groups in the study population, and no clinically significant suppression of serum cortisol was observed.  [\hyperlink{Monomethyl Fumarate}{PMID: 27128459}, Amanda Oliver et al., 2014] No therapies have been formally approved by the Food and Drug Administration for use in pediatric multiple sclerosis, a rare disease. We evaluated the safety, efficacy, and pharmacokinetics of dimethyl fumarate in pediatric patients with multiple sclerosis. FOCUS, a phase 2, multicenter study of patients aged 10 to 17 years with relapsing-remitting multiple sclerosis, comprised an eight-week baseline and 24-week treatment period; during treatment, patients received dimethyl fumarate (120 mg twice daily on days one to seven; 240 mg twice a day thereafter). Magnetic resonance imaging scans were obtained at week -8, day 0, week 16, and week 24. The primary end point was the change in T2 hyperintense lesion incidence from the baseline period to the final 8 weeks of treatment. Secondary end points were pharmacokinetic parameters and adverse event incidence. Twenty of 22 enrolled patients completed the study. There was a significant reduction in T2 hyperintense lesion incidence from baseline to the final eight weeks of treatment (P = 0.009). Adverse events (most commonly gastrointestinal events and flushing) and pharmacokinetic parameters were consistent with adult findings. No serious adverse events were considered dimethyl fumarate related. Dimethyl fumarate treatment was associated with a reduction in magnetic resonance imaging activity in pediatric patients; pharmacokinetic and safety profiles were consistent with those in adults. Dimethyl fumarate is a potential treatment for pediatric multiple sclerosis. [\hyperlink{Monomethyl Fumarate}{PMID: 27128459}, Raed Alroughani et al., 2018]

\hypertarget{pmid_19818174}{I}nfants and children under five years of age are the most vulnerable to malaria with over 1,700 deaths per day from malaria in this group. However, until recently, there were no WHO-endorsed paediatric anti-malarial formulations available. Artemisinin-based combination therapy is the current standard of care for patients with uncomplicated falciparum malaria in Africa. Artemether/lumefantrine (AL) meets WHO pre-qualification criteria for efficacy, safety and quality. Coartem, a fixed dose combination of artemether and lumefantrine, has consistently achieved cure rates of >95\% in clinical trials. However, AL tablets are inconvenient for caregivers to administer as they need to be crushed and mixed with water or food for infants and young children. Further, in common with other anti-malarials, they have a bitter taste, which may result in children spitting the medicine out and not receiving the full therapeutic dose. There was a clear unmet medical need for a formulation of AL specifically designed for children. Ahead of a call from WHO for child-friendly medicines, Novartis, working in partnership with Medicines for Malaria Venture (MMV), started the development of a new formulation of AL for infants and young children: Coartem Dispersible. The excellent efficacy, safety and tolerability already demonstrated by AL tablets were confirmed with dispersible AL in a large trial comparing the crushed tablets with dispersible tablets in 899 African children with falciparum malaria. In the evaluable population, 28-day PCR-corrected cure rates of >96\% were achieved. Further, its sweet taste means that it is palatable for children, and the dispersible formulation makes it easier for caregivers to administer than bitter crushed tablets. Easing administration may foster compliance, hence improving therapeutic outcomes in infants and young children and helping to preserve the efficacy of ACT. [\hyperlink{Monomethyl Fumarate}{PMID: 19818174}, Salim Abdulla et al., 2009]

\hypertarget{pmid_10922144}{T}o date, only one study of chronic use of a leukotriene receptor antagonist in children has been published. The efficacy and safety of montelukast in children 6-14 years of age with asthma (n = 336) was studied during an 8-week, double-blind, placebo-controlled trial. There was significantly greater improvement in forced expired volume in 1 sec (FEV(1)) from baseline for the montelukast group (8. 23\%) compared to the placebo group (3.58\%). There was a significant decrease in use of beta agonists for symptom relief and a significant decrease in the percentage of days and percentage of patients with asthma exacerbations. An asthma-specific quality of life questionnaire revealed significant overall improvement in quality of life and significant improvement in the quality of life domains for symptoms, activity, and emotions. Adverse effects were not significantly different for montelukast than for placebo, with the exception of allergic rhinitis which was more prevalent in the placebo group. A 6-month open follow-up of patients from the above study was undertaken. Effects of montelukast on FEV(1) were consistent over the 6 months, with the increase in FEV(1) not significantly different from a small control group treated with beclomethasone. Quality of life remained significantly improved throughout the open treatment period. In conclusion, leukotriene receptor antagonists are of value for the treatment of children with asthma. [\hyperlink{Monomethyl Fumarate}{PMID: 10922144}, A Becker et al., 2000]

\hypertarget{pmid_26996405}{F}irst-line injectable therapies for multiple sclerosis in children may be ineffective or not well-tolerated. There is therefore an urgent need to explore oral medications for pediatric multiple sclerosis. We review our dual-center experience with oral dimethyl fumarate. This study was a retrospective review of children 18 years of age or less with multiple sclerosis treated with dimethyl fumarate at Yale University and the University of Colorado. Clinical, demographic, and magnetic resonance imaging parameters were analyzed. We identified 13 children treated with oral dimethyl fumarate for a median of 15.0 months (range, 1 to 25). Dimethyl fumarate was utilized as first-line therapy in five children (38\%). Ten children (77\%) tolerated dose escalation to the usual adult dose of 240 mg twice daily. Nine children had ≥12 months of follow-up on treatment. Eight of nine (89\%) displayed stabilized or reduced relapse rates and disability scores on treatment. Nine children underwent brain magnetic resonance imaging performed after 12 or more months of therapy. New T2 lesions were observed in three children (33\%), one of whom had been nonadherent to treatment. Common side effects included facial flushing (8/13, 62\%), gastrointestinal discomfort (7/13, 54\%), rash (3/13, 23\%), and malaise (2/13, 15\%). Three children (23\%) discontinued treatment because of side effects. No patients displayed laboratory abnormalities including lymphopenia or abnormal liver transaminases. There were no reported infections. Oral dimethyl fumarate appears to be safe and generally well tolerated in children with multiple sclerosis. Formal clinical trials to evaluate efficacy are ongoing. [\hyperlink{Monomethyl Fumarate}{PMID: 26996405}, Naila Makhani et al., 2016]

\hypertarget{pmid_14651542}{I}t has been reported recently that fluvoxamine (a selective serotonin reuptake inhibitor) is effective and safe for children with monosymptomatic nocturnal enuresis (MNE). However, the exact mechanism by which fluvoxamine is beneficial in the treatment of MNE remains unknown. One possibility is that it controls emotional stress. We divided children with MNE into primary MNE (n = 40) and secondary MNE (n = 7). We measured urinary 17-hydroxycorticosteroids (17-OHCS) and 17-ketosteroid sulfates (17-KS-S) as a stress barometer in children with MNE to evaluate adaptation to emotional stress before and during fluvoxamine treatment. We initially administered fluvoxamine at a dose of 25 mg at bedtime. If patients remained incontinent after 3 weeks, we increased the dose to 50 mg. Fluvoxamine was effective in 26 of 28 children (93\%) with primary MNE and an abnormality of the stress barometer and in six of six children (100\%) with secondary MNE and an abnormality of the stress barometer. Fluvoxamine was effective in only six of 12 children (50\%) with primary MNE and normality of the stress barometer and was not effective in one child with secondary MNE and normality of the stress barometer. The stress barometer is useful clinically for evaluating the therapeutic effect of fluvoxamine for children with MNE. [\hyperlink{Monomethyl Fumarate}{PMID: 14651542}, Kenichi Kano et al., 2003]

\hypertarget{pmid_6415824}{T}he data obtained in a retrospective survey of the use and effectiveness of sodium cromoglycate (Lomudal; Fisons) therapy in 635 children and young adults support the findings of other long-term studies that this drug is effective and safe in the treatment of children and young adults with clear evidence of allergy contributing towards their asthma. Its administration by Spinhaler can be commenced successfully at 4 years of age or even younger and continued for as long as is required. The drug is worth a therapeutic trial in any patient with asthma requiring regular medication to control symptoms. The absence of demonstrable allergy does not entirely preclude a favourable response. Once a patient's asthma has been stabilized, sodium cromoglycate enables concomitant therapy (especially with beta 2-adrenergic stimulants and corticosteroids) to be reduced. [\hyperlink{Monomethyl Fumarate}{PMID: 6415824}, E G Weinberg et al., 1983]

\hypertarget{pmid_20593906}{T}he high prevalence of asthma in pediatric patients underscores the need for effective and safe treatment in this population. Current treatment guidelines recommend inhaled corticosteroids (ICSs) as a preferred treatment for the control of mild to moderate persistent asthma in patients of all ages, including young children. Clinical efficacy, systemic safety, and ease of use are desirable attributes of an ICS used to treat children with persistent asthma. Recently, mometasone furoate administered via a dry powder inhaler (MF-DPI) 110 microg once daily in the evening (delivered dose of 100 microg) was approved by the US FDA for the maintenance treatment of asthma in children 4-11 years of age. Data from the clinical trial program for MF-DPI that establish the efficacy, long-term safety, and absence of systemic effects of the approved dosage in children with mild to moderate persistent asthma are reviewed. These findings indicate that once-daily dosing of MF-DPI in children aged 4-11 years significantly improves lung function and health-related quality of life while reducing rescue medication use and exacerbations despite previous treatment with other ICSs. MF-DPI is also well tolerated in children. Clinical trial results showed that, at the approved dosage, there are no effects on growth velocity or the hypothalamic-pituitary-adrenal axis. Results of pediatric studies are consistent with the clinical development program for adults and adolescents. In addition, once-daily dosing, established safety, and ease of use of MF-DPI may help to improve asthma management by addressing issues that inhibit proper adherence. [\hyperlink{Monomethyl Fumarate}{PMID: 20593906}, Henry Milgrom et al., 2010]

\hypertarget{pmid_23010795}{T}his study aimed at observing the efficacy of mometasone fuorate monohydrate nasal spray on obstructive adenoids in children and identifying the characteristics of responders using a pilot study including children aged 2-11 years, with evidence of more than 50 \% obstruction. Allergic rhinitis and nasal obstruction were evaluated on baseline (V0), 6- (V1), and 12-week (V2) visits. Degree of obstruction was evaluated by nasopharyngoscopy at V0 and V2. Subjects received 100 μg mometasone fuorate daily. Results were compared with those of a matching control group. Nineteen children (8 females, 11 males; 2.25-8.50 years old, mean 4.24 years, median 4.00 years) completed treatment and follow-up adequately. There was 58 \% reduction in a clinical score assessing the severity of adenoidal obstruction (P < 0.05), 56 \% reduction in severity of obstructive symptom (P < 0.05), and 75 \% reduction in allergic rhinitis score (P < 0.05) between V0 and V1. No further significant improvement was noticed between V1 and V2. The degree of obstruction dropped from 85 to 61 \% as noted on endoscopy (P < 0.05). None in the control group showed spontaneous decrease or resolution of the symptoms. Age of patients, allergic rhinitis score, and severity of the clinical score had no impact on the response parameters. No side effects were observed. Mometasone furoate monohydrate nasal spray appears to be effective in treating children with obstructive adenoids. The effect seems to be independent of the presence of mild intermittent allergic rhinitis, the age of patient, or the severity of symptoms. [\hyperlink{Monomethyl Fumarate}{PMID: 23010795}, Mohamed A Bitar et al., 2013]

\hypertarget{pmid_22364032}{A}cute respiratory infections are the second leading cause of morbidity in children under 18 years. Several drugs have been used with variable efficacy and safety, trying to reduce the associated symptoms and improve quality of life. To evaluate the efficacy and safety of buphenine, aminophenazone and diphenylpyraline hydrochloride when compared with placebo for the control of symptoms associated with common cold in children 6-24 months of age. Randomized clinical trial, double blind, placebo controlled, in 100 children < 24 months of any gender, with symptoms associated to common cold. They received the drug under study vs. placebo for seven days. Both groups received acetaminophen. The change on common cold related symptoms were evaluated. Statistic analysis was made with STATA 11.0 for Mac. Fifty-three children were randomized to study drug and forty-seven to placebo. Age of children in each group was similar (12.2 +/- 5.8 months vs. 12.7 +/- 5.8 months, p NS). There were significant differences between groups in relation to rhinorrea and sneezing resolution, with better results in Flumil group and no adverse events observed. The results in this study indicates that Flumil is a safe and effective drug for control of symptoms present in the common cold in children aged 6-24 months. [\hyperlink{Monomethyl Fumarate}{PMID: 22364032}, Ericka Montijo-Barrios et al., ]

\hypertarget{pmid_2326439}{T}his paper reports on 350 pediatric patients who were studied over a 17-month period to determine the efficacy and safety of oral and intramuscular sedation techniques. The protocol using oral chloral hydrate, 50 mgm/kg, for infants under 1 year of age or intramuscular pentobarbital, 5 mgm/kg, for children over 1 year was found to be an effective, safe and fairly simple approach to pediatric sedation. Of the 350 sedated patients, 343 (98 percent) had satisfactory scans on the same day the examination was scheduled after a single dose or an initial dose and supplementary sedation. [\hyperlink{Monomethyl Fumarate}{PMID: 2326439}, J B Temme et al., ]

\hypertarget{pmid_35058083}{E}arly disease control with disease-modifying drugs is important for improving the prognosis of multiple sclerosis (MS) in children. Dimethyl fumarate (DMF) is an oral disease-modifying drug for MS in adults with relatively stable disease; however, its use in young children has not been heavily documented in the current literature. We report the case of a pediatric patient with relapsing-remitting MS who was treated with DMF. A 3-year-old boy with a history of common cold symptoms developed unsteadiness and somnolence. Magnetic resonance imaging revealed multiple white matter lesions. Symptoms were recurrent, and DMF was prescribed at 6 years of age due to a relapse episode with oculomotor disability and facial paralysis. However, disease progression continued, and new lesions were noted at age 7; thus, the dose of DMF was increased to 240 mg/day. No relapse has been observed for over three years; sequelae or severe side effects were absent. DMF may be a useful oral disease-modifying drug for preventing recurrence in young children with MS. [\hyperlink{Monomethyl Fumarate}{PMID: 35058083}, Naoya Saijo et al., 2022]

\hypertarget{pmid_11449608}{M}ontelukast is used as an add-on medication with inhaled steroids in the therapy of childhood asthma. The aim was to determine the clinical effect of montelukast as a prophylactic therapy in mild asthma in comparison with inhaled sodium cromoglycate. 20 children aged 6-14 years were treated in a 20-week open-labelled randomized cross-over design, starting after a 2-week run-in period with either montelukast or cromolyn for 16 weeks with a 2-week wash-out period between treatments. Children treated with cromoglycate showed a significant increase of FEV1 (100.6 vs. 96.5\%, p < 0.01) and MEF25 (70.6 vs. 59.1\%, p < 0.05) in base line lung function and after cold air challenge (FEV1 97.2 vs. 91.2\%, p < 0.05; MEF25 62.9 vs. 54.4\%, p < 0.01). Treatment with montelukast effected a significant increase (p < 0.05) in MEF25 from 59.1 to 67.8\% in base line lung function alone. Both medications resulted in significant decreases (p < 0.05) in daytime asthma symptoms and evening peak flow variability. Comparing the two treatment substances no statistically significant differences could be registered in any endpoints including beta-agonist use. Both cromolyn and montelukast showed effective control of mild asthma in children; however, montelukast is more convenient in its application. Further studies are needed to determine the role of leukotriene receptor antagonists in childhood asthma. [\hyperlink{Monomethyl Fumarate}{PMID: 11449608}, C Liebke et al., 2001]

\hypertarget{pmid_16925692}{W}e aimed to compare the efficacy and safety of budesonide/formoterol (Symbicort) with budesonide alone (Pulmicort) or budesonide (Pulmicort) and formoterol (Oxis) administered via separate inhalers in children with asthma. In a 12 wk, double-blind study, a total of 630 children with asthma (mean age 8 yr [4-11 yr]; mean forced expiratory volume in 1 s (FEV(1)) 92\% predicted; mean inhaled corticosteroid dose 454 microg/day) were randomized to: budesonide/formoterol (80/4.5 microg, two inhalations twice daily); a corresponding dose of budesonide alone (100 microg, two inhalations twice daily); or a corresponding dose of budesonide (100 microg, two inhalations twice daily) and formoterol (4.5 microg, two inhalations twice daily) (budesonide + formoterol in separate inhalers). The primary efficacy variable was the change from baseline to treatment (average of the 12-wk treatment period) in morning peak expiratory flow (PEF). Other changes in lung function and asthma symptoms were assessed, as was safety. Budesonide/formoterol significantly improved morning PEF, evening PEF and FEV(1) compared with budesonide (all p < 0.001); there was no significant difference between budesonide/formoterol and budesonide + formoterol in separate inhalers for these variables. All other diary card variables improved from baseline in all treatment groups; there were no significant between-group differences. Adverse-event profiles were similar in all groups; there were no serious asthma-related adverse events in any treatment group. budesonide/formoterol significantly improved lung function in children (aged 4-11 yr) with asthma compared with budesonide alone. Budesonide/formoterol is a safe and effective treatment option for children with asthma. [\hyperlink{Monomethyl Fumarate}{PMID: 16925692}, Petr Pohunek et al., 2006]

\hypertarget{pmid_37728224}{G}rowth impairment is a known adverse event (AE) of corticosteroids in children. This study aimed to assess the effect of once-daily (QD) inhaled fluticasone furoate (FF) versus placebo on growth velocity over 1 year in prepubertal children with well-controlled asthma. This randomized, double-blind, parallel-group, placebo-controlled, multicenter study (NCT02889809) included prepubertal children, aged 5 to <9 years (boys), and 5 to <8 years (girls), with ≥6 months' asthma history. Children received inhaled placebo QD plus background open-label montelukast QD for a 16-week run-in period and were then randomized 1:1 to receive inhaled FF 50 μg QD or placebo QD (whilst continuing background open-label montelukast) for a 52-week treatment period. The primary endpoint was the difference in growth velocity (cm/year) over the treatment period. Other growth endpoints were measured, as were incidence of AEs and asthma exacerbation. Growth analyses included all intent-to-treat (ITT) participants with ≥3 post-randomization, on-treatment clinic visit height assessments (GROWTH population). Of 644 children in the run-in period, 477 (mean age 6.2 years, 63\% male) entered the 52-week treatment period (ITT population: FF N = 238, placebo N = 239; GROWTH population: N = 457 [FF N = 231; placebo N = 226]). The least-squares mean difference in growth velocity for FF versus placebo was -0.160 cm/year (95\% confidence interval: -0.462, 0.142). There were no new safety signals. Over 1 year, FF 50 μg QD had a minimal effect on growth velocity versus placebo, with no new safety signals. [\hyperlink{Monomethyl Fumarate}{PMID: 37728224}, Philippe Bareille et al., 2023]

\hypertarget{pmid_18164990}{T}o evaluate the safety and efficacy of once-daily (QD) fluticasone furoate (FF) nasal spray in children with perennial allergic rhinitis (PAR). A global, randomized, double-blind, placebo-controlled study. Pediatric patients (aged 2-11 years; n = 558) with PAR received once-daily placebo, FF 110 microg, or FF 55 microg for 12 weeks. Efficacy was evaluated by nasal symptom scores. General safety and corticosteroid-specific safety (nasal and ophthalmic examinations, and hypothalamic-pituitary-adrenal assessments) were assessed. No findings of clinical concern were identified from the safety assessments. For primary efficacy analysis of mean change from baseline over the first 4 weeks of treatment in daily reflective total nasal symptom score, FF 55 microg demonstrated significant improvement (P = 0.003) compared with placebo; however, the improvement for FF 110 microg versus placebo did not reach statistical significance (P = 0.073). FF QD was well tolerated and demonstrated efficacy in children aged 2 to 11 years with PAR. [\hyperlink{Monomethyl Fumarate}{PMID: 18164990}, Jorge F Máspero et al., 2008]

\hypertarget{pmid_18611612}{T}he safety and efficacy of cefetamet pivoxil, an oral cephalosporin of the third generation, have been studied in open, prospective, randomized comparative, clinical trials including 301 toddlers (children aged 1 to 2 years) with upper and lower respiratory tract infections, and urinary tract infections. Cefetamet pivoxil (CAT) syrup formulation was given to 177 toddlers either in the standard dose of 10 mg/kg b.i.d. [n = 116] or 20 mg/kg b.i.d. [n = 61] and 124 toddlers have been treated with comparator drugs [cefaclor, n = 98; phenoxymethylpenicillin, n = 18; amoxicillin plus clavulanic acid; n = 8]. The treatment period was 7 days mainly, except for pharyngotonsillitis for which the treatment duration was 7 or 10 days. The assessment of treatment was based on clinical signs and symptoms primarily in infections of lower respiratory tract and acute otitis media, whereas in patients with pharyngotonsillitis and acute urinary tract infections the bacteriological findings were the main evaluation criteria. The overall therapeutic outcome was successful in 148 (95.4\%) of the 155 toddlers to whom CAT was administered and in 87 (85.3\%) out of 102 toddlers receiving standard drugs. Adverse events of mild to moderate severity, mainly of gastro-intestinal type (vomiting or diarrhoea) occurred in 14.7\% in the patient group receiving CAT, 11.2\% in the toddlers receiving the standard dose of CAT, and in 12.9\% with the comparator drugs. From the data presented it is concluded that cefetamet pivoxil is efficient and safe in toddlers presenting with community-acquired respiratory and urinary infections mainly caused by S. pneumoniae, H. influenzae, Group A beta-haemolytic streptococci, M. catarrhalis, E. coli, Proteus spp. and K. pneumoniae. [\hyperlink{Monomethyl Fumarate}{PMID: 18611612}, A Chibante et al., 1994]

\hypertarget{pmid_27740721}{M}ometasone furoate (MF), delivered via dry-powder inhaler (DPI) QD in the evening (PM), is a treatment option for pediatric patients with asthma. We evaluated MF delivered via a metered-dose inhaler (MDI), in children ages 5-11 years with persistent asthma. This was a 12-week double-blind, double-dummy, placebo-controlled trial. Pateints were randomized to the following treatments: MF-MDI 50 mcg BID, MF-MDI 100 mcg BID, MF-MDI 200 mcg BID, MF-DPI 100 mcg QD PM, and placebo. The primary analysis assessed MF-MDI doses versus placebo, on the change in \%-predicted forced expiratory volume in one second (FEV For change from baseline in \%-predicted FEV In children ages 5-11 years with persistent asthma, all three doses of MF-MDI (50, 100, and 200 mcg BID) demonstrated significant improvement in FEV [\hyperlink{Monomethyl Fumarate}{PMID: 27740721}, Niran J Amar et al., 2017] Based on the outcome of several randomized controlled trials, the orally active leukotriene receptor antagonist montelukast (Singulair, Merck) has been licensed for treatment of asthma. The drug is favored for treating childhood asthma, where a therapeutic challenge has arisen due to poor compliance with inhalation therapy. To assess the efficiency of and satisfaction with Singulair in asthmatic children under real-life conditions. Montelukast was prescribed for 6 weeks to a cohort of 506 children aged 2 to 18 years with mild to moderate persistent asthma, who were enrolled by 200 primary care pediatricians countrywide. Four clinical correlates of childhood asthma--wheeze, cough, difficulty in breathing, night awakening--were evaluated from patients' diary cards. Due to under-treatment by their physicians, almost 60\% of the children were not receiving controller therapy at baseline. By the end of the study, which consisted of montelukast treatment, a significant improvement over baseline was noted in asthma symptoms and severity, as well as in treatment compliance. The participating pediatricians and parents were highly satisfied with the treatment. The results of this extensive study show that the use of montelukast as monotherapy in children presenting with persistent asthma resulted in a highly satisfactory outcome for themselves, their parents and their physicians. [\hyperlink{Monomethyl Fumarate}{PMID: 27740721}, Israel Amirav et al., 2008]

\hypertarget{pmid_26088405}{M}ixtures of fumaric acid esters (FAE) are used as an oral systemic treatment for moderate to severe psoriasis. Large clinical studies with dimethylfumarate (DMF) monotherapy are scarce. The objective of this study is to assess the effectiveness and long-term safety of high-dose DMF monotherapy in moderate to severe psoriasis. A prospective single-blinded follow-up study was performed in a cohort of patients treated with DMF. Patients were followed-up at fixed intervals. Assessment of consecutive photographs was performed by two observers. Primary outcome was a change in static physician global assessment (PGA) score. Safety outcome was defined as incidences of (serious) adverse events. A total of 176 patients with moderate to severe psoriasis were treated with DMF for a median duration of 28 months. The median daily maintenance dosage of 480 mg was reached after a median of 8 months. Psoriasis activity decreased significantly by 1.7 out of five points. A total of 152 patients reported one or more adverse events, such as gastrointestinal complaints and flushing. High-dose DMF monotherapy is an effective and safe treatment option in moderate to severe psoriasis. It can be suggested that 50\% of all patients may benefit from high-dose DMF monotherapy. [\hyperlink{Monomethyl Fumarate}{PMID: 26088405}, Raphaël Lijnen et al., 2016]

\hypertarget{pmid_9333180}{N}ew antimalarial drugs are urgently needed. The use of short courses of the new antimalarial drug artemether as monotherapy has been limited by secondary malaria episodes following parasite clearance. Therefore, a new antimalarial drug, CGP 56697, has been developed, which combines artemether with a longer-acting antimalarial agent, benflumetol. A safety trial was undertaken in 60 Gambian children 1-6 years old with uncomplicated Plasmodium falciparum malaria. All children treated with CGP 56697 cleared their parasites 72 h after the start of treatment. No neurologic, cardiac, or other adverse reactions were observed. Second episodes of falciparum malaria were recorded in 16 (27\%) of the children. Second infections were more frequent during the rainy season than during the dry season. Molecular epidemiologic studies suggested that 12 of the 14 second episodes of malaria in children treated with CGP 56697 were due to new infections. CGP 56697 proved to be a safe and effective antimalarial drug in African children. [\hyperlink{Monomethyl Fumarate}{PMID: 9333180}, L von Seidlein et al., 1997]

\section*{Dimethyl Fumarate}
\subsection*{Result}
\subsubsection*{Answer}

Ages 10–17: Yes  
Ages <10: Unknown  
Ages ≤18 (general): Yes (based on limited data, mostly ages 10–17)

\subsubsection*{{Explanation}}
\hypertarget{Dimethyl Fumarate}
A review of the available abstracts reveals several studies specifically addressing the safety of Dimethyl Fumarate (DMF) in children with multiple sclerosis (MS):

1. **A retrospective dual-center study** included 13 children aged ≤18 years treated with DMF for a median of 15 months. Most children tolerated the adult dose, and common side effects were facial flushing, gastrointestinal discomfort, rash, and malaise. Three children discontinued due to side effects, but no laboratory abnormalities or infections were reported. The authors concluded that DMF "appears to be safe and generally well tolerated in children with multiple sclerosis," but also noted that formal clinical trials are ongoing [\hyperlink{pmid_26996405}{PMID: 26996405}, Naila Makhani et al., 2016].

2. **A phase 2, multicenter clinical trial (FOCUS)** enrolled 22 pediatric patients aged 10 to 17 years with relapsing-remitting MS. The primary endpoint was a reduction in MRI lesion incidence, and safety was a secondary endpoint. Twenty of 22 patients completed the study. Adverse events (mainly gastrointestinal and flushing) and pharmacokinetics were consistent with adult findings. No serious adverse events were considered related to DMF. The authors concluded that "pharmacokinetic and safety profiles were consistent with those in adults" and that DMF is a potential treatment for pediatric MS [\hyperlink{pmid_29681490}{PMID: 29681490}, Raed Alroughani et al., 2018].

3. **A case report** described a 3-year-old boy with MS who began DMF at age 6 and continued for over three years without relapse or severe side effects. The report suggests DMF may be useful and well-tolerated in young children, but this is a single case [\hyperlink{pmid_35058083}{PMID: 35058083}, Naoya Saijo et al., 2022].

4. Other abstracts either focus on adult populations, animal studies, or do not address pediatric safety directly.

**Summary by age range:**
- **Ages 10–17:** A targeted phase 2 clinical trial (FOCUS) affirms that DMF has a safety profile in this age group similar to adults, with no serious DMF-related adverse events reported in the study [\hyperlink{pmid_29681490}{PMID: 29681490}].
- **Ages ≤18:** The retrospective study supports general tolerability and safety in children up to 18 years, though with a small sample size and some discontinuations due to side effects [\hyperlink{pmid_26996405}{PMID: 26996405}].
- **Ages <10:** Only a single case report exists for a child starting DMF at age 6, which is insufficient to affirm safety in this age group [\hyperlink{pmid_35058083}{PMID: 35058083}].

**Conclusion:** 
- For ages 10–17, there is targeted clinical trial evidence supporting the safety of DMF.
- For ages <10, there is insufficient data to definitively affirm or refute safety.

\subsection*{Abstracts}
\hypertarget{pmid_26996405}{F}irst-line injectable therapies for multiple sclerosis in children may be ineffective or not well-tolerated. There is therefore an urgent need to explore oral medications for pediatric multiple sclerosis. We review our dual-center experience with oral dimethyl fumarate. This study was a retrospective review of children 18 years of age or less with multiple sclerosis treated with dimethyl fumarate at Yale University and the University of Colorado. Clinical, demographic, and magnetic resonance imaging parameters were analyzed. We identified 13 children treated with oral dimethyl fumarate for a median of 15.0 months (range, 1 to 25). Dimethyl fumarate was utilized as first-line therapy in five children (38\%). Ten children (77\%) tolerated dose escalation to the usual adult dose of 240 mg twice daily. Nine children had ≥12 months of follow-up on treatment. Eight of nine (89\%) displayed stabilized or reduced relapse rates and disability scores on treatment. Nine children underwent brain magnetic resonance imaging performed after 12 or more months of therapy. New T2 lesions were observed in three children (33\%), one of whom had been nonadherent to treatment. Common side effects included facial flushing (8/13, 62\%), gastrointestinal discomfort (7/13, 54\%), rash (3/13, 23\%), and malaise (2/13, 15\%). Three children (23\%) discontinued treatment because of side effects. No patients displayed laboratory abnormalities including lymphopenia or abnormal liver transaminases. There were no reported infections. Oral dimethyl fumarate appears to be safe and generally well tolerated in children with multiple sclerosis. Formal clinical trials to evaluate efficacy are ongoing. [\hyperlink{Dimethyl Fumarate}{PMID: 26996405}, Naila Makhani et al., 2016]

\hypertarget{pmid_29681490}{N}o therapies have been formally approved by the Food and Drug Administration for use in pediatric multiple sclerosis, a rare disease. We evaluated the safety, efficacy, and pharmacokinetics of dimethyl fumarate in pediatric patients with multiple sclerosis. FOCUS, a phase 2, multicenter study of patients aged 10 to 17 years with relapsing-remitting multiple sclerosis, comprised an eight-week baseline and 24-week treatment period; during treatment, patients received dimethyl fumarate (120 mg twice daily on days one to seven; 240 mg twice a day thereafter). Magnetic resonance imaging scans were obtained at week -8, day 0, week 16, and week 24. The primary end point was the change in T2 hyperintense lesion incidence from the baseline period to the final 8 weeks of treatment. Secondary end points were pharmacokinetic parameters and adverse event incidence. Twenty of 22 enrolled patients completed the study. There was a significant reduction in T2 hyperintense lesion incidence from baseline to the final eight weeks of treatment (P = 0.009). Adverse events (most commonly gastrointestinal events and flushing) and pharmacokinetic parameters were consistent with adult findings. No serious adverse events were considered dimethyl fumarate related. Dimethyl fumarate treatment was associated with a reduction in magnetic resonance imaging activity in pediatric patients; pharmacokinetic and safety profiles were consistent with those in adults. Dimethyl fumarate is a potential treatment for pediatric multiple sclerosis. [\hyperlink{Dimethyl Fumarate}{PMID: 29681490}, Raed Alroughani et al., 2018]

\hypertarget{pmid_35058083}{E}arly disease control with disease-modifying drugs is important for improving the prognosis of multiple sclerosis (MS) in children. Dimethyl fumarate (DMF) is an oral disease-modifying drug for MS in adults with relatively stable disease; however, its use in young children has not been heavily documented in the current literature. We report the case of a pediatric patient with relapsing-remitting MS who was treated with DMF. A 3-year-old boy with a history of common cold symptoms developed unsteadiness and somnolence. Magnetic resonance imaging revealed multiple white matter lesions. Symptoms were recurrent, and DMF was prescribed at 6 years of age due to a relapse episode with oculomotor disability and facial paralysis. However, disease progression continued, and new lesions were noted at age 7; thus, the dose of DMF was increased to 240 mg/day. No relapse has been observed for over three years; sequelae or severe side effects were absent. DMF may be a useful oral disease-modifying drug for preventing recurrence in young children with MS. [\hyperlink{Dimethyl Fumarate}{PMID: 35058083}, Naoya Saijo et al., 2022]

\hypertarget{pmid_31591676}{F}umaric acid esters are recommended in European guidelines for induction and maintenance treatment of patients with moderate to severe plaque psoriasis. A systemic medication with pure dimethyl fumarate without monoethyl fumarate salts was recently licensed in Europe. The efficacy and safety of pure dimethyl fumarate were assessed in patients with severe (physician global assessment) plaque psoriasis in Austria in the BRIDGE trial. In this double blind, randomized, placebo-controlled trial patients received 16-week treatment with pure dimethyl fumarate in a head to head comparison with dimethyl fumarate with monoethyl fumarate salts, which is licensed in Germany. In this post hoc analysis the efficacy and safety were assessed in patients with severe psoriasis in Austria. Efficacy measures significantly improved in both active treatment arms compared to placebo in 65 patients after 16 weeks of treatment. Physician global assessment of clear/almost clear in the dimethyl fumarate group was non-inferior to the dimethyl fumarate with monoethyl fumarate salts group 2 months after end of treatment. No serious adverse reaction occurred in patients with dimethyl fumarate in contrast to the second active treatment. Efficacy outcome was paralleled by quality of life improvements. This is the first report of dimethyl fumarate in a severely affected population with plaque psoriasis. Dimethyl fumarate is effective and safe in the systemic treatment of adults with severe psoriasis (physician global assessment). [\hyperlink{Dimethyl Fumarate}{PMID: 31591676}, Paul Sator et al., 2019]

\hypertarget{pmid_9428981}{T}he efficacy and tolerability of dimethindene maleate (CAS 3614-69-5, DMM, Fenistil) as drops in the treatment of pruritus in children suffering from chicken-pox were investigated in a study with two different doses of dimethindene maleate and placebo. 128 children, 1 to 6 years of age, were included in a double blind, randomized, placebo controlled, multi-center clinical trial. Patients received either a dosage of DMM of 0.1 mg/kg x d, or 0.05 mg/kg x d, or placebo, respectively. All patients received a commercially available astringent lotion for topical treatment of skin lesions. The primary efficacy criterion which was the change in the itching severity score from baseline to the end of the treatment assessed as area under the baseline (AUB) showed for both treatments with DMM a statistically significant superiority versus placebo in reducing the severity of itching. There was no statistically proven difference between the two verum groups. [\hyperlink{Dimethyl Fumarate}{PMID: 9428981}, W Englisch et al., 1997]

\hypertarget{pmid_8087216}{D}iphemanil methylsulfate is an atropin-like drug used in some infants suffering from vagal bradycardia. Its pharmacokinetic parameters are known for adults but not for infants. The report describes these parameters in six infants. Five infants aged 35 to 109 days (mean: 62 +/- 28) and weighing 3.5 to 5.3 kg (mean: 4.3) were included in the study with the formal consent of their parents. All suffered from vagal hyperreactivity. The sixth younger full-term infant was aged 10 days and weighed 4 kg. They were given a single dose (3 mg/kg) of diphemanil methylsulfate orally, after a minimal fast of 4 hours. Blood samples were collected at T0 and 3, 6, 8, 12 and 24 hours after administration. Urines were also collected from 1 hour before drug administration to 24 hours after. Plasma concentrations of diphemanil methylsulfate were measured by gas-exchange chromatography. The peak plasma concentration in the five infants occurred at 3.9 +/- 2.3 hours (range: 2.9-8 hours). Half-life was 8.6 +/- 2.4 hours and tended to decrease with age. All the other parameters were identical to those found in adults. The peak plasma concentration occurred in the sixth younger infant at 2.9 hours, with a half-life of 17.2 hours. Renal clearance was high (0.3 l/h/kg). The relatively long half-life of diphemanil methylsulfate allows this drug to be given every 8 hours. This longer interval is more comfortable for the patients and their parents. The high renal clearance suggests that this drug is excreted by both glomerular filtration and tubular secretion. [\hyperlink{Dimethyl Fumarate}{PMID: 8087216}, G Chéron et al., 1994]

\hypertarget{pmid_27277955}{S}timulant medications are approved to treat attention deficit hyperactivity disorder (ADHD) in children over the age of 6 years. Fatal ingestion of stimulants by children has been reported, although most ingestions do not result in severe toxicity. Lisdexamfetamine dimesylate, a once daily long-acting stimulant, is a prodrug requiring conversion to its active form, dextroamphetamine, in the bloodstream. Based on its unique pharmacokinetics, peak levels of d-amphetamine are delayed. We describe a case of accidental ingestion of lisdexamfetamine dimesylate in an infant. A previously healthy 10-month-old infant was admitted to the hospital with a 5-h history of tachycardia, hypertension, dyskinesia, and altered mental status of unknown etiology. Confirmatory urine testing, from a specimen collected approximately 16 h after the onset of symptoms, revealed an urine amphetamine concentration of 22,312 ng/mL (positive cutoff 200 ng/mL). The serum amphetamine concentration, from a specimen collected approximately 37 h after the onset of symptoms, was 68 ng/mL (positive cutoff 20 ng/mL). Urine and serum were both negative for methamphetamine, methylenedioxyamphetamine (MDA), methylenedioxymethamphetamine (MDMA, Ecstasy), and methylenedioxyethamphetamine (MDEA). During the hospitalization, it was discovered that the infant had access to lisdexamfetamine dimesylate prior to the onset of symptoms. Amphetamine ingestions in young children are uncommon but do occur. Clinicians should be aware of signs and symptoms of amphetamine toxicity and consider ingestion when a pediatric patient presents with symptoms of a sympathetic toxidrome even when ingestion is denied. [\hyperlink{Dimethyl Fumarate}{PMID: 27277955}, Kelly E Wood et al., 2016]

\hypertarget{pmid_21490354}{D}imenhydrinate is an over-the-counter drug that is commonly used for the treatment of nausea and vomiting. Many of my adult patients use it, but is it safe and useful in the pediatric population? Dimenhydrinate appears to be safe for use in the pediatric population. While little literature has been published about adverse effects of this medication, family physicians need to identify the cause of the vomiting before considering if the drug will be effective and need to ensure that patients safely use the medication and avoid potential interaction of the drug with other products. [\hyperlink{Dimethyl Fumarate}{PMID: 21490354}, Paul Enarson et al., 2011]

\hypertarget{pmid_29948245}{D}imethyl-fumarate (DMF) demonstrated efficacy and safety in relapsing-remitting multiple sclerosis (MS) in randomized clinical trials. To track and evaluate post-market DMF profile in real-world setting. Patients receiving DMF referred to Italian MS centres were enrolled and prospectively followed, collecting demographic clinical and radiological data. Among the 735 included patients, 45.4\% were naïve to disease-modifying therapies, 17.8\% switched to DMF because of tolerance, 27.4\% switched to DMF because of lack of efficacy, and 9.4\% switched to DMF because of safety concerns. Median DMF exposure was 17 months (0-33). DMF reduced the annual relapse rate (ARR) by 63.2\%. At 12 and 24 months, 85 and 76\% of patients were relapse-free. NEDA-3 status after 12 months of DMF treatment was maintained by 47.5\% of patients. 89 and 70\% of patients at 12 and 24 months regularly continued DMF. Most frequent adverse events (AEs) were flushing (37.2\%) and gastro-enteric AEs (31.1\%). Our post-market study corroborated that DMF is a safe and effective drug. Additionally, the study suggested that naïve patients strongly benefit from DMF and that DMF improved ARR also in patients who were horizontally switched from injectable therapies due to tolerability and efficacy issues. [\hyperlink{Dimethyl Fumarate}{PMID: 29948245}, Giulia Mallucci et al., 2018] (1) Respiratory distress and seizures developed in an 18-month-old boy following brief exposure to low-strength (17.6\%) N,N-diethyl-m-toluamide (DEET). A review of the literature revealed 17 reports of DEET-induced encephalopathy in children. The objective of this study was to test the hypothesis that the potential toxicity of DEET is high and that available repellents containing DEET, irrespective of their strength, are not safe when applied to children's skin. (2) Although this is a case report, we used the features of published reports of DEET-induced encephalopathy in children to support the diagnosis, since the evidence that the child's illness was caused by DEET was circumstantial. In the following case analysis, clinical reports of children < 16 years old have been reviewed and analyzed in an effort to relate direct DEET toxicity to various clinical, demographic, and toxic compound exposure factors (Fisher's exacttest and logistic regression analysis). (3) DEET-induced encephalopathy in children (56\% girls) followed not only ingestion or repeated and extensive application of repellents, but also a brief exposure to DEET (45\%). Of those who reported a dermal exposure, 33\% reported an exposure to a product containing DEET < 20\%. Seizures, the most prominent symptom (72\%), were significantly more frequent when DEET solutions were applied to the skin (P<0.01). Mortality (16.6\%) did not correlate significantly with the concentration of the DEET liquid used, duration of skin exposure, pattern of use, age, or sex. (4) Data of this case analysis suggest that repellents containing DEET are not safe when applied to children's skin and should be avoided in children. Additionally, since the potential toxicity of DEET is high, less toxic preparations should be probably substituted for DEET-containing repellents, whenever possible. [\hyperlink{Dimethyl Fumarate}{PMID: 29948245}, G Briassoulis et al., 2001]

\hypertarget{pmid_27128459}{T}his multi-center, randomized, double-blind, placebo-controlled, two-way crossover study characterized the safety, tolerability, pharmacokinetics, and pharmacodynamics of fluticasone furoate (FF) in children (5-11 years) with persistent asthma. Twenty-seven children received inhaled FF 100 µg or placebo via the ELLIPTA™ dry powder inhaler once daily for 14 days, with a ≥7 day washout period. Adverse events (AEs) were reported by eight (31\%) and four (16\%) subjects during FF 100 µg and placebo treatment, respectively. Headache was reported by three subjects during FF 100 µg treatment and by no subjects during placebo treatment, all other AEs were reported by only one subject on either treatment; there were no serious AEs. Following repeat dosing, the arithmetic mean (SD) FF Cmax was 26.71 pg/mL (9.16) at 31 minutes post-dose. Arithmetic mean (SD) FF AUC(0-t) was 121.44 pg h/mL (83.04). Arithmetic mean values for weighted mean (SD) serum cortisol (0-12 hours) on day 14 were 56.49 (16.51) and 67.57 (20.66) ng/mL for FF 100 µg and placebo, respectively. No clinically significant effect of FF on serum cortisol levels was observed. FF was well tolerated. Pharmacokinetic profiles were well defined and did not differ between age groups in the study population, and no clinically significant suppression of serum cortisol was observed.  [\hyperlink{Dimethyl Fumarate}{PMID: 27128459}, Amanda Oliver et al., 2014] Dialkyl phthalates are plasticizers used in household products made from polyvinyl chloride (PVC). Diisononyl phthalate (DINP) is the principal phthalate in soft plastic toys. Because DINP is not tightly bound to PVC, it may be released when children mouth PVC products. The potential chronic health risks of phthalate exposure to infants have been under scrutiny by regulatory agencies in Europe, Canada, Japan, and the U.S. This report describes a risk assessment of DINP exposure from children's products, by the U.S. Consumer Product Safety Commission (CPSC) staff. This report includes the findings of a CPSC Chronic Hazard Advisory Panel (CHAP) which: (1) concluded that DINP is unlikely to present a human cancer hazard and (2) recommended an acceptable daily intake (ADI) level of 120 microg/kg-d, based on spongiosis hepatis in rats. The risk assessment incorporates new measurements of DINP migration rates from 24 toys and a new observational study of children's mouthing activities, with a detailed characterization of the objects mouthed. Probabilistic methods were used to estimate exposure. Mouthing behavior and, thus, exposure depend on the child's age. Approximately 42\% of tested soft plastic toys contained DINP. Estimated DINP exposures for soft plastic toys were greatest among children 12-23 months old. The mean exposure for this age group was 0.08 (95\% confidence interval 0.04-0.14) microg/kg-d, with a 99th percentile of 2.4 (1.3-3.2) microg/kg-d. The authors conclude that oral exposure to DINP from mouthing soft plastic toys is not likely to present a health hazard to children. The opinions expressed by the authors have not been reviewed or approved by, and do not necessarily reflect the views of, the U.S. Consumer Product Safety Commission. Because this material was prepared by the authors in their official capacity, it is in the public domain and may be freely copied or reprinted. [\hyperlink{Dimethyl Fumarate}{PMID: 27128459}, Michael A Babich et al., 2004]

\hypertarget{pmid_25511835}{T}he efficacy and safety of the three oral agents approved by the Food and Drug Administration for the treatment of relapsing-remitting multiple sclerosis (RRMS) are reviewed. Limitations to parenteral disease-modifying therapies (DMTs) (interferon beta-1a, interferon beta-1b, and glatiramer acetate) for the treatment of RRMS have been addressed by the approval of three oral DMTs: fingolimod, teriflunomide, and dimethyl fumarate. In clinical trials, each of the oral DMTs was superior to placebo in annualized relapse rate, a key indicator of clinical efficacy, and in neuroradiological efficacy. A reduction in disability progression was evident with higher doses of teriflunomide but was not consistently demonstrated with fingolimod or dimethyl fumarate. Each of the oral DMTs demonstrated acceptable safety in clinical trials, with adverse-effect profiles that differ from injectable agents. The safety of both teriflunomide and dimethyl fumarate is supported by long-term use of related agents for other diseases; however, postmarketing surveillance studies are needed to determine the safety of each of the oral DMTs in patients with RRMS. Dimethyl fumarate seems to have the most innocuous safety profile of the three agents. Fingolimod requires first-dose inpatient monitoring due to cardiac safety concerns and multiple laboratory tests prior to initiation of therapy, while teriflunomide has been associated with hepatotoxicity and teratogenicity. With the approval of three oral drugs for RRMS-fingolimod, teriflunomide, and dimethyl fumarate-the therapeutic strategy for RRMS has evolved to include options that are efficacious and appear to have administration advantages over established parenteral treatments. [\hyperlink{Dimethyl Fumarate}{PMID: 25511835}, Rachel Hutchins Thomas et al., 2015]

\hypertarget{pmid_25800129}{D}imethyl fumarate (DMF), a fumaric acid ester, is a new orally available disease-modifying agent that was recently approved by the US FDA and the EMA for the management of relapsing forms of multiple sclerosis (MS). Fumaric acid has been used for the management of psoriasis, for more than 50 years. Because of the known anti-inflammatory properties of fumaric acid ester, DMF was brought into clinical development in MS. More recently, neuroprotective and myelin-protective mechanism actions have been proposed, making it a possible candidate for MS treatment. Two Phase III clinical trials (DEFINE, CONFIRM) have evaluated the safety and efficacy of DMF in patients with relapsing-remitting MS. Being an orally available agent with a favorable safety profile, it has become one of the most commonly prescribed disease-modifying agents in the USA and Europe.  [\hyperlink{Dimethyl Fumarate}{PMID: 25800129}, Duvyanshu Dubey et al., 2015] Dimethyl fumarate (DMF) has immune-modulatory and neuro-protective characteristics that can be used for treatment of acute ischemic stroke. To investigate the therapeutic effects of DMF on histological and functional recovery of rats after transient middle cerebral artery (MCA) occlusion. 22 Sprague-Dawley male rats weighing 275-300 g were randomized into three groups by block randomization. In the sham group (n = 7), the neck was opened, but neither MCA was occluded, nor any drug was administered.The control group (n = 7) was treated with vehicle (methocel) by gavage for 14 days after MCA occlusion. In the DMF-treated group (n = 8), treatment was performed with 15 mg/kg body weight dimethyl fumarate twice a day for 14 days after MCA occlusion. Transient occlusion of the right MCA was performed by intraluminal thread method in the DMF-treated and the control group. Neurological deficit score (NDS), pole test, and adhesive removal test were performed before the surgery, and on post-operative Days 0, 3, 5, 7, 10, and 14. After the final behaviour test, the animals' brains were perfused and removed. Brains were frozen and sectioned serially and coronally using a cryostat. Infract volume and brain volume were estimated by stereology. The percentage of infarct volume was significantly lower in DMF-treated animals (5.76\%) than in the control group (22.39\%) (P < 0.0001). Regarding behavioural tests, the DMF-treated group showed better function in NDS on Days 7 (P = 0.041) and 10 (P = 0.046), but not in pole and adhesive removal tests. There was no significant correlation between behavioural tests and histological results. Dimethyl fumarate could be beneficial as a potential neuroprotective agent in the treatment of stroke. [\hyperlink{Dimethyl Fumarate}{PMID: 25800129}, Anahid Safari et al., 2017]

\hypertarget{pmid_22364032}{A}cute respiratory infections are the second leading cause of morbidity in children under 18 years. Several drugs have been used with variable efficacy and safety, trying to reduce the associated symptoms and improve quality of life. To evaluate the efficacy and safety of buphenine, aminophenazone and diphenylpyraline hydrochloride when compared with placebo for the control of symptoms associated with common cold in children 6-24 months of age. Randomized clinical trial, double blind, placebo controlled, in 100 children < 24 months of any gender, with symptoms associated to common cold. They received the drug under study vs. placebo for seven days. Both groups received acetaminophen. The change on common cold related symptoms were evaluated. Statistic analysis was made with STATA 11.0 for Mac. Fifty-three children were randomized to study drug and forty-seven to placebo. Age of children in each group was similar (12.2 +/- 5.8 months vs. 12.7 +/- 5.8 months, p NS). There were significant differences between groups in relation to rhinorrea and sneezing resolution, with better results in Flumil group and no adverse events observed. The results in this study indicates that Flumil is a safe and effective drug for control of symptoms present in the common cold in children aged 6-24 months. [\hyperlink{Dimethyl Fumarate}{PMID: 22364032}, Ericka Montijo-Barrios et al., ]

\hypertarget{pmid_37103520}{M}edications for treating bipolar disorder (BD) are limited and can cause side effects if used chronically. Therefore, efforts are being made to use new agents in the control and treatment of BD. Considering the antioxidant and anti-inflammatory effects of dimethyl fumarate (DMF), this study was performed to examine the role of DMF on ketamine (KET)-induced manic-like behavior (MLB) in rats. Forty-eight rats were randomly divided into eight groups, including three groups of healthy rats: normal, lithium chloride (LiCl) (45 mg/kg, p.o.), and DMF (60 mg/kg, p.o.), and five groups of MLB rats: control, LiCl, and DMF (15, 30, and 60 mg/kg, p.o.), which received KET at a dose of 25 mg/kg, i.p. The levels of total sulfhydryl groups (total SH), thiobarbituric acid reactive substances (TBARS), nitric oxide (NO), and tumor necrosis factor-alpha (TNF-α), as well as the activity of antioxidant enzymes including catalase (CAT), superoxide dismutase (SOD), and glutathione peroxidase (GPx) in the prefrontal cortex (PFC) and hippocampus (HPC), were measured. DMF prevented hyperlocomotion (HLM) induced by KET. It was found that DMF could inhibit the increase in the levels of TBARS, NO, and TNF-α in the HPC and PFC of the brain. Furthermore, by examining the amount of total SH and the activity of SOD, GPx, and CAT, it was found that DMF could prevent the reduction of the level of each of them in the brain HPC and PFC. DMF pretreatment improved the symptoms of the KET model of mania by reducing HLM, oxidative stress, and modulating inflammation. [\hyperlink{Dimethyl Fumarate}{PMID: 37103520}, Shiva Saljoughi et al., 2023]

\hypertarget{pmid_9132194}{T}o evaluate the safety and efficacy of intranasal diamorphine as an analgesic for use in children in accident and emergency (A\&E). A prospective, randomised clinical trial with consecutive recruitment of patients aged between 3 and 16 years with clinically suspected limb fractures. One group received 0.1 mg/kg intranasal diamorphine, and the other group received 0.2 mg/kg intramuscular morphine. At 0, 5, 10, 20, and 30 minutes pain scores, Glasgow coma score, and peripheral oxygen saturations were recorded; parental acceptability was assessed at 30 minutes. 58 children were recruited, with complete data collection in 51 (88\%); the median summed decrease in pain score was better for intranasal diamorphine than intramuscular morphine (9 v 8), though this was not significant (P = 0.4, Mann-Whitney U test). The episode was recorded as "acceptable" in all parents whose child received intranasal diamorphine, compared with only 55\% of parents in the intramuscular morphine group (P < 0.0001, Fisher's exact test). There was no incidence of decreased peripheral oxygen saturation or depression in the level of consciousness in any patient. Intranasal diamorphine is an effective, safe, and acceptable method of analgesia for children requiring opiates in the A \& E department. [\hyperlink{Dimethyl Fumarate}{PMID: 9132194}, J A Wilson et al., 1997]

\hypertarget{pmid_33193814}{D}imethyl fumarate (DMF) is approved for the treatment of relapsing-remitting multiple sclerosis. It is unknown whether DMF or its primary metabolite monomethyl fumarate (MMF) are excreted into human milk. We present two cases of lactating patients who donated milk samples to study the transfer of MMF into human milk following a week of 2 × 240 mg daily oral dose. Samples were analyzed using liquid chromatography mass spectrometry. The calculated relative infant dose was 0.019\% and 0.007\%. This is the first study to demonstrate that MMF is transferred into human milk, with only limited exposure to an infant. [\hyperlink{Dimethyl Fumarate}{PMID: 33193814}, Andrea I Ciplea et al., 2020]

\hypertarget{pmid_28150703}{P}ulmonary arterial hypertension (PAH) is a fatal condition for which there is no cure. Dimethyl Fumarate (DMF) is an FDA approved anti-oxidative and anti-inflammatory agent with a favorable safety record. The goal of this study was to assess the effectiveness of DMF as a therapy for PAH using patient-derived cells and murine models. We show that DMF treatment is effective in reversing hemodynamic changes, reducing inflammation, oxidative damage, and fibrosis in the experimental models of PAH and lung fibrosis. Our findings indicate that effects of DMF are facilitated by inhibiting pro-inflammatory NFκB, STAT3 and cJUN signaling, as well as βTRCP-dependent degradation of the pro-fibrogenic mediators Sp1, TAZ and β-catenin. These results provide a novel insight into the mechanism of its action. Collectively, preclinical results demonstrate beneficial effects of DMF on key molecular pathways contributing to PAH, and support its testing in PAH treatment in patients. [\hyperlink{Dimethyl Fumarate}{PMID: 28150703}, Agnieszka P Grzegorzewska et al., 2017]

\hypertarget{pmid_23013261}{D}imethylacetamide (DMAC) and dimethylformamide (DMF) continue to be important, widely used solvents involved in a wide variety of industrial applications. As liquids with relatively low vapor pressures, contact with both the integumentary and respiratory systems is the main source of human exposure. Although airborne control levels for the workplace have been established and industrial hygiene practices to limit dermal contact have been put in place, use of these chemicals has been associated with occupational illness, mainly in Asia where new and expanded uses have led to overexposures. Thus an update of the basic toxicology data including tables indicating the dose/exposure response characteristics of both DMAC and DMF is currently important. Both chemicals are similar from a toxicology perspective. Human experience has generally shown the materials to be without adverse effect except under conditions where airborne and dermal controls were not properly applied. The use of urinary metabolite monitoring has successfully been employed to measure integrated dermal and inhalation worker exposure. The chemicals are not particularly toxic following acute exposure but high doses can produce damage to the liver, the organ which is first affected by these two chemicals. Repeated dose/exposure studies have characterized both the targets of toxicity and the doses required to produce changes by various routes of exposure. Higher doses of these materials can produce changes in developing systems, infrequently in experiments at doses in which the maternal animal is unaffected, thus care needs to be taken when exposures are to women of child-bearing age. The chemicals appear to be low in genetic activity and inhalation exposures have not shown the materials to produce tumors in rodents except with DMF in a situation in which aerosol formation was encountered. This presentation extends the two previous reviews and, like those, includes updated information on acetamide and formamide and their monomethyl derivatives as well as the commercially important DMAC and DMF. Since a large portion of the newer information deals with effects in humans and biomonitoring, these sections are presented at the start of this review. [\hyperlink{Dimethyl Fumarate}{PMID: 23013261}, Gerald L Kennedy et al., 2012]

\hypertarget{pmid_32974794}{D}imethyl fumarate and fingolimod are oral disease modifying treatments (DMTs) that reduce relapse activity and slow disability worsening in relapsing-remitting multiple sclerosis (RRMS). To compare the effectiveness of dimethyl fumarate and fingolimod in a real-world setting, where both agents are licensed as a first-line DMT for the treatment of RRMS. We identified patients with RRMS commencing dimethyl fumarate or fingolimod in the Swiss Federation for Common Tasks of Health Insurances (SVK) Registry between August 2014 and July 2019. Propensity score-matching was applied to select subpopulations with comparable baseline characteristics. Relapses and disability outcomes were compared in paired, pairwise-censored analyses. Of the 2113 included patients, 1922 were matched (dimethyl fumarate, n = 961; fingolimod, n = 961). Relapse rates did not differ between the groups (incident rate ratio 1.0, 95\%CI 0.8-1.2, p = 0.86). Moreover, no difference in the hazard of 1-year confirmed disability worsening (hazard ratio [HR] 0.9; 95\%CI 0.6-1.6; p = 0.80) or disability improvement (HR 0.9; 95\%CI 0.6-1.2; p = 0.40) was detected. These findings were consistent both for treatment-naïve patients and patients switching from another DMT. Dimethyl fumarate and fingolimod have comparable effectiveness regarding reduction of relapses and disability worsening in RRMS. [\hyperlink{Dimethyl Fumarate}{PMID: 32974794}, Johannes Lorscheider et al., 2021]

\hypertarget{pmid_20593906}{T}he high prevalence of asthma in pediatric patients underscores the need for effective and safe treatment in this population. Current treatment guidelines recommend inhaled corticosteroids (ICSs) as a preferred treatment for the control of mild to moderate persistent asthma in patients of all ages, including young children. Clinical efficacy, systemic safety, and ease of use are desirable attributes of an ICS used to treat children with persistent asthma. Recently, mometasone furoate administered via a dry powder inhaler (MF-DPI) 110 microg once daily in the evening (delivered dose of 100 microg) was approved by the US FDA for the maintenance treatment of asthma in children 4-11 years of age. Data from the clinical trial program for MF-DPI that establish the efficacy, long-term safety, and absence of systemic effects of the approved dosage in children with mild to moderate persistent asthma are reviewed. These findings indicate that once-daily dosing of MF-DPI in children aged 4-11 years significantly improves lung function and health-related quality of life while reducing rescue medication use and exacerbations despite previous treatment with other ICSs. MF-DPI is also well tolerated in children. Clinical trial results showed that, at the approved dosage, there are no effects on growth velocity or the hypothalamic-pituitary-adrenal axis. Results of pediatric studies are consistent with the clinical development program for adults and adolescents. In addition, once-daily dosing, established safety, and ease of use of MF-DPI may help to improve asthma management by addressing issues that inhibit proper adherence. [\hyperlink{Dimethyl Fumarate}{PMID: 20593906}, Henry Milgrom et al., 2010]

\hypertarget{pmid_15258101}{T}oxicology studies are typically performed on single compounds, which we hypothesized would miss adverse synergies from chemical mixtures. This hypothesis was tested using an insect repellant and sunscreens because both groups include known permeation enhancers, with prior pediatric reports of toxicity from highly concentrated DEET (N,N-diethyl-m-toluamide). Using real-time mass spectroscopy in a hairless mouse skin model, we confirmed substantial penetration of a 20\% DEET standard. Despite a lower (10\%) DEET content, a commercially marketed sunscreen formulation had a 6-fold more rapid detection (5 versus 30 min) and 3.4-fold greater penetration at steady state. We also tested the efficacy of DEET microemulsion products and confirmed that one successfully slowed the onset of absorption, but not the steady-state permeation. Risks from mixtures of potential toxins are worthy of routine testing, which can be accomplished by simple assays, and are of utmost importance for pediatric applications. [\hyperlink{Dimethyl Fumarate}{PMID: 15258101}, Edward A Ross et al., 2004]

\hypertarget{pmid_19752076}{V}omiting is a common symptom in children with infectious gastroenteritis. It contributes to fluid loss and is a limiting factor for oral rehydration therapy. Dimenhydrinate has traditionally been used for children with gastroenteritis in countries such as Canada and Germany. We investigated the efficacy and safety of dimenhydrinate in children with acute gastroenteritis. We performed a prospective, randomized, placebo-controlled, multicenter trial. We randomly assigned 243 children with presumed gastroenteritis and vomiting to rectal dimenhydrinate or placebo. Children with no or mild dehydration were included. All children received oral rehydration therapy. Primary outcome was defined as weight gain within 18 to 24 hours after randomization. Secondary outcomes were number of vomiting episodes, fluid intake, parents' assessment of well-being, number of diarrheal episodes, and admission rate to hospital. We recorded potential adverse effects. Change of weight did not differ between children who received dimenhydrinate or placebo. The mean number of vomiting episodes between randomization and follow-up visit was 0.64 in the dimenhydrinate group and 1.36 in the placebo group. In total, 69.6\% of the children in the dimenhydrinate group versus 47.4\% in the placebo group were free of vomiting between randomization and the follow-up visit. Hospital admission rate, fluid intake, general well-being of the children, and potential adverse effects, including the number of diarrhea episodes, were similar in both groups. Dimenhydrinate reduces the frequency of vomiting in children with mild dehydration; however, the overall benefit is low, because it does not improve oral rehydration and clinical outcome. [\hyperlink{Dimethyl Fumarate}{PMID: 19752076}, Ulrike Uhlig et al., 2009]

\section*{Octreotide Acetate}
\subsection*{Result}
\subsubsection*{Answer}

Unknown

\subsubsection*{{Explanation}}
\hypertarget{Octreotide Acetate}
To determine if Octreotide Acetate is safe for use in children, I evaluated each abstract for targeted pediatric safety studies, focusing on specific age ranges and reported outcomes.

Neonates and Infants (0–12 months):
- Several studies and case series report the use of octreotide in neonates and infants, primarily for chylothorax and congenital hyperinsulinism. A systematic review of 39 articles (case reports only, no RCTs) found octreotide to be "relatively effective and safe" in neonates with chylothorax, with side effects in 14.3\% of patients, but no deaths directly attributed to octreotide [\hyperlink{pmid_29602276}{PMID: 29602276}, Carlo Bellini et al., 2018]. Another retrospective study of 428 infants found "relatively few adverse events" during off-label use, but did report necrotizing enterocolitis (NEC) in 2\% and death in 3\% of infants during administration, though causality is not established [\hyperlink{pmid_25968047}{PMID: 25968047}, Daniela Testoni et al., 2015]. A smaller retrospective study of 11 neonates reported only minor side effects (hyperglycemia) and concluded octreotide was used safely as adjunctive therapy [\hyperlink{pmid_29948779}{PMID: 29948779}, Syed Ahmed Zaki et al., 2018]. However, other studies and case reports highlight the risk of NEC, especially in neonates with other risk factors [\hyperlink{pmid_27910218}{PMID: 27910218}, Ann W McMahon et al., 2017; \hyperlink{pmid_32051156}{PMID: 32051156}, Suresh Chandran et al., 2020]. Overall, while some studies report safe use, the risk of serious adverse events like NEC is present, and all authors call for further controlled studies.

Children (1–12 years):
- In children with congenital hyperinsulinism, a retrospective review of 25 children (median age at diagnosis 8 weeks, final follow-up median 1.8 years) found octreotide to be "well-tolerated" and effective, with transient elevation of liver enzymes in 20\% and one case of asymptomatic gallbladder pathology [\hyperlink{pmid_32851339}{PMID: 32851339}, Bingyan Cao et al., 2020]. Another study in children with portal hypertension (13 children, ages not specified but pediatric) found octreotide "appears to be safe and effective" for acute GI bleeding, with no serious adverse events reported [\hyperlink{pmid_31214393}{PMID: 31214393}, P B Koul et al., 2012]. A case series of two infants (5 and 8 months) with chylothorax reported no complications [\hyperlink{pmid_16856532}{PMID: 16856532}, M Kaneko et al., 2006]. A study of 6 children with primary intestinal lymphangiectasia treated for 3–37 months reported one case of acute pancreatitis as a side effect [\hyperlink{pmid_20512058}{PMID: 20512058}, Sinan Sari et al., 2010]. A systematic review of 25 children with chylothorax treated with octreotide reported minor side effects and one case of NEC [\hyperlink{pmid_16532329}{PMID: 16532329}, Charles C Roehr et al., 2006]. A retrospective review of 34 pediatric oncology patients (average age 6 years) treated for chemotherapy-induced diarrhea found common adverse effects (hyperglycemia, hyperbilirubinemia, nausea/vomiting, abdominal cramping) but no life-threatening events [\hyperlink{pmid_21108438}{PMID: 21108438}, Vinita Pai et al., 2011]. A case report described anaphylaxis in a child with chronic pancreatitis, but successful desensitization was achieved [\hyperlink{pmid_22299317}{PMID: 22299317}, Dilek Azkur et al., 2011]. 

Adolescents (13–17 years):
- Data for adolescents are limited, but some studies include patients up to 17 years. A retrospective review included patients aged 14 days to 17 years, but did not specify safety outcomes by age group [\hyperlink{pmid_11704779}{PMID: 11704779}, J C Lam et al.]. A study of long-acting octreotide in 9 children (range 1 month–14.5 years) with portal hypertension reported no serious adverse effects attributable to octreotide, though one child developed growth hormone deficiency and hypothyroidism after prolonged subcutaneous octreotide [\hyperlink{pmid_25162361}{PMID: 25162361}, Marie O'Meara et al., 2015].

Summary:
- Across all age groups, most studies are retrospective, case series, or systematic reviews of case reports. There are no randomized controlled trials definitively establishing safety in children. While many studies report that octreotide was "well-tolerated" or "safe" in their cohorts, serious adverse events such as NEC, hepatitis, pancreatitis, and anaphylaxis have been reported, albeit rarely. Authors consistently call for larger, controlled studies to establish safety and optimal dosing. Therefore, while octreotide has been used in children and infants with some reports of safety, the evidence is not definitive, and safety remains uncertain.

\subsection*{Abstracts}
\hypertarget{pmid_27910218}{O}ctreotide is a synthetic peptide analog of naturally occurring somatostatin. Octreotide is used off-label in children <6 years of age for hyperinsulinism, chylothorax, and gastrointestinal bleeding. There is a lack of controlled data on efficacy or potential adverse events from this off-label use. Three pediatric hospitals participated in this study. Patients were hospitalized January 2007-December 2010 and administered octreotide for congenital hyperinsulinism (CHI) at least 1 day. Variables assessed included octreotide dosage, patient demographics, medical interventions, concomitant medicines, serious adverse events (SAEs) including necrotizing enterocolitis (NEC), and mortality. The 103 patient sample had a median gestational age of 38 weeks. During the study period, two patients died: one from NEC and the other from cardiomyopathy/sepsis. There were 11 other SAEs in the 101 surviving patients. This study highlights potential risks in administering octreotide off-label. This study, like several other published studies, has highlighted NEC in a full-term infant treated with octreotide. It is important to study the efficacy and the safety of octreotide for hyperinsulinism. In the interim, it might be prudent to prescribe octreotide in CHI neonates only in the absence of other risk factors for NEC. Copyright © 2016 John Wiley \& Sons, Ltd. [\hyperlink{Octreotide Acetate}{PMID: 27910218}, Ann W McMahon et al., 2017]

\hypertarget{pmid_22052632}{O}ctreotide is a synthetic somatostatin analogue which has been suggested for use in the management of acute pancreatitis, though its safety and effectiveness in the pediatric setting has not been extensively studied. we present a rare case of a 6.5-year-old female with acute lymphoblastic leukemia (ALL) and L-asparaginase (L-asp) induced pancreatitis, who developed epileptic seizures, possibly associated with octreotide administration. Her imaging and laboratory findings ruled out a leukemic involvement or infection of CNS. The EEG revealed repetitive sharp waves maximal on the frontal and temporal areas of the right hemisphere. The child was treated with diazepam and she continued with systemic anticonvulsant treatment with levetiracetam. After 2 weeks of conservative treatment, pancreatitis resolved and she continued her chemotherapy protocol. Levetiracetam treatment lasted 8 months. 7 months after the first episode, EEG was reported as normal, and the child completed the chemotherapy protocol without any further severe complications. Larger and well designed studies are needed to warrant the safety of octreotide in pediatric population. [\hyperlink{Octreotide Acetate}{PMID: 22052632}, E Hatzipantelis et al., 2011]

\hypertarget{pmid_7887530}{P}ancreatic pseudocysts in children are rare. A total of 213 cases have been reported in the literature, the majority secondary to trauma (65\%). Treatment options range from conservative, non-operative management to operative drainage. Octreotide acetate, a long-acting analog of somatostatin, is a synthetic peptide with a variety of endocrine and gastrointestinal functions. Octreotide has been successfully used following pancreatic surgery to reduce exocrine function and most recently in the management of adult pancreatic pseudocysts. We report the efficacy of octreotide, as an adjunct to treatment, in two children with pancreatic pseudocyst. Each child was treated conservatively with bowel rest, hyperalimentation, and octreotide acetate (2.5 micrograms/kg SQ QD). Complete resolution of the pseudocysts occurred within 5 weeks. We conclude that octreotide acetate is a safe and potentially effective adjunct in the treatment of pediatric pancreatic pseudocyst, and should be added to the management of pseudocyst before drainage procedures. [\hyperlink{Octreotide Acetate}{PMID: 7887530}, C Mulligan et al., 1995]

\hypertarget{pmid_32851339}{O}ctreotide is an off-label medicine for congenital hyperinsulinism (CHI), but is currently widely used for treatment of patients with CHI. Thus far, variable efficacy and adverse effects have been reported for octreotide. The present study evaluated the efficacy and safety of a subcutaneous octreotide injection for treatment of diazoxide-unresponsive CHI in China. This study was a retrospective review of children with diazoxide-unresponsive CHI who were treated with a subcutaneous octreotide injection. The efficacy and side effects of the treatment were assessed. Twenty-five Chinese children (15 boys) were involved in the study. Their median age at diagnosis was 8 weeks (range, 1-24 weeks) and median age at the final follow-up was 1.8 years (range, 0.3-3.3 years). Octreotide therapy effectively increased blood glucose levels in all patients. The intravenous glucose infusion rate was reduced in all patients. Twenty-one patients gradually discontinued the intravenous glucose infusion while receiving octreotide combined with frequent carbohydrate/glucose-rich feeding. Among patients with a monoallelic ATP-sensitive potassium (KATP) channel mutation, 50.0\% showed gradual remission during follow up, indicating that the octreotide treatment may be a feasible alternative to surgery, especially for patients with monoallelic KATP-channel mutations. Transient elevation of liver enzymes occurred in 20.0\% of patients, while asymptomatic gallbladder pathology occurred in one patient. The growth rates of these patients were normal (height standard deviation score was 0.3 ± 1.5 at the final follow-up). Octreotide was a well-tolerated, effective therapy for most children with diazoxide-unresponsive CHI. [\hyperlink{Octreotide Acetate}{PMID: 32851339}, Bingyan Cao et al., 2020]

\hypertarget{pmid_31214393}{E}valuate the usage of octreotide for the control of acute upper gastrointestinal bleeding in children with portal hypertension. A retrospective electronic database analysis of these children was performed over a period of five years. Setting was a tertiary pediatric intensive care. Case notes of 18 encounters in 13 children were reviewed. A loading dose (1.27 ± 0.76 µg/kg) was administered in seven, with median starting dose of 1.44 ± 1.19 µg/kg/h in all other episodes. The mean maximum dose was 1.68 ± 1.38 µg/kg/h. Re-bleeding occurred in one third; hemostasis was eventually achieved in all. Octreotide infusion appears to be safe and effective in controlling pediatric upper gastrointestinal bleeding due to portal hypertension. We also recommend its use in community and rural hospital settings prior to transfer of such patients to a tertiary care center. [\hyperlink{Octreotide Acetate}{PMID: 31214393}, P B Koul et al., 2012]

\hypertarget{pmid_25968047}{O}ctreotide is used off-label in infants for treatment of chylothorax, congenital hyperinsulinism, and gastrointestinal bleeding. The safety profile of octreotide in hospitalized infants has not been described; we sought to fill this information gap. We identified all infants exposed to at least 1 dose of octreotide from a cohort of 887,855 infants discharged from 333 neonatal intensive care units managed by the Pediatrix Medical Group between 1997 and 2012. We collected laboratory and clinical information while infants were exposed to octreotide and described the frequency of baseline diagnoses, laboratory abnormalities, and clinical adverse events (AEs). A total of 428 infants received 490 courses of octreotide. The diagnoses most commonly associated with octreotide use were chylothorax (50\%), pleural effusion (32\%), and hypoglycemia (22\%). The most common laboratory AEs that occurred during exposure to octreotide were thrombocytopenia (47/1000 infant-days), hyperkalemia (21/1000 infant-days), and leukocytosis (20/1000 infant-days). Hyperglycemia occurred in 1/1000 infant-days and hypoglycemia in 3/1000 infant-days. Hypotension requiring pressors (12\%) was the most common clinical AE that occurred during exposure to octreotide. Necrotizing enterocolitis was observed in 9/490 (2\%) courses, and death occurred in 11 (3\%) infants during octreotide administration. Relatively few AEs occurred during off-label use of octreotide in this cohort of infants. Additional studies are needed to further evaluate the safety, dosing, and efficacy of this medication in infants. [\hyperlink{Octreotide Acetate}{PMID: 25968047}, Daniela Testoni et al., 2015]

\hypertarget{pmid_11704779}{O}ctreotide is a somatostatin analogue that has been suggested as a therapeutic agent in various diverse disease processes including gastrointestinal bleeding, pancreatitis, hypoglycemia related to hyperinsulin states, and chylous peritoneum/thorax. Despite successful use in the adult population, there is limited information concerning its use in pediatric patients. The authors retrospectively review their experience with octreotide in 10 infants and children ranging in age from 14 days to 17 years. Octreotide, administered by continuous intravenous infusion or intermittent bolus dosing, was used in the treatment of gastrointestinal bleeding in four patients, pancreatitis in three patients, chylous leaks in two patients, and hypoglycemia related to nesidioblastosis in one patient. The clinical course of these patients and the potential therapeutic impact of octreotide are evaluated. Additionally, previous experiences with octreotide in pediatric patients, dosing regimens, and the potential role of the drug in other disease processes are discussed. [\hyperlink{Octreotide Acetate}{PMID: 11704779}, J C Lam et al., ]

\hypertarget{pmid_29948779}{O}ctreotide is a somatostatin analogue and has been used off-label for a variety of conditions. There are no specific guidelines for the use of octreotide in neonates and its safety and efficacy have not been systematically evaluated. The objective of this study is to present our experience of using octreotide therapy in neonates. This is a retrospective study of neonates who received octreotide therapy during their hospital stay over a 15 years period (2003-2017) in a tertiary neonatal centre. The demographic details and indications of octreotide therapy including time of initiation, route, dose, duration and adverse effects of therapy were noted. The clinical course following octreotide administration was also analysed. Eleven neonates received octreotide therapy during the study period, of which nine had chylothorax and two had chylous ascites. Resolution of the chylous effusion with octreotide therapy was achieved in 4 out of 11 (36.3\%) of the cases. The median duration of octreotide therapy in cases with successful resolution was 17.5 days. With the exception of minor side effects such as hyperglycaemia, none of the patients had any significant side effects that required discontinuation of therapy. Octreotide was used safely as an adjunctive therapy for the treatment of chylothorax and chylous ascites in neonates. However, larger prospective controlled trials are required to establish the optimal dose, time of initiation, duration and efficacy of octreotide therapy in neonates. [\hyperlink{Octreotide Acetate}{PMID: 29948779}, Syed Ahmed Zaki et al., 2018]

\hypertarget{pmid_16856532}{W}e experienced 2 infants in whom octreotide acetate was effective on intractable chylothorax after surgery for congenital heart diseases. They were 8- and 5-month-old. They were diagnosed as having corrected transposition of the great arteries (TGA) and tetralogy of Fallot respectively, and underwent bidirectional Glenn anastomosis and right modified Blalock Taussig shunt. Chylothorax was revealed on the 11th and the 1st postoperative day, and was not improved by any conventional therapy in either case. Then octreotide acetate was infused continuously with 0.1-0.6 micorg/kg/hour for 24 and 7 days. Chylothorax disappeared completely without any complications such as disturbance of blood sugar level or growth retardation. Octreotide acetate was effective and safe even in infants in intractable chylothorax after surgery for congenital heart diseases, as long as used for short period. [\hyperlink{Octreotide Acetate}{PMID: 16856532}, M Kaneko et al., 2006]

\hypertarget{pmid_32051156}{O}ctreotide is a somatostatin analogue used for treating congenital chylothorax and congenital hyperinsulinism in infants. By increasing splanchnic arteriolar resistance and decreasing gastrointestinal blood flow, octreotide indirectly reduces lymphatic flow in chylous effusions.Splanchnic ischaemia following octreotide predisposes infants to necrotising enterocolitis (NEC). Although NEC occurrence in infants treated with octreotide for hyperinsulinaemic hypoglycaemia has been reported widely, its incidence in infants with chylothroax is low. We describe a case of congenital chylothorax in a preterm infant who had poor response to thoracentesis. Although octreotide initiation lead to resolution of chylothorax, he developed NEC. Cessation of octreotide and medical management resulted in rapid resolution of NEC. Since octreotide is generally used as the first-line treatment for chylous effusion, the risk of NEC should be considered, especially when the dosage is increased. Infants on octreotide should be closely observed for early signs and symptoms of NEC to avert surgical emergency. [\hyperlink{Octreotide Acetate}{PMID: 32051156}, Suresh Chandran et al., 2020]

\hypertarget{pmid_25162361}{O}ctreotide reduces splanchnic blood flow and is effective in controlling gastrointestinal bleeding (GIB) caused by portal hypertension. Monthly long-acting octreotide (OCT-LAR) with an efficacy and safety profile similar to subcutaneous daily administration presents an attractive option for long-term therapy. We report our experience with OCT-LAR for severe/recurrent GIB in children with portal hypertension secondary to chronic liver disease or portal vein thrombosis who were unresponsive to standard interventions. A total of 9 patients, 7 boys, who received OCT-LAR between 2000 and 2009 were studied retrospectively (median age at first bleeding 21 months, range 1 month-14.5 years). The dose (2.5-20 mg intramuscularly monthly) was extrapolated from that used in adult acromegaly and neuroendocrine tumours (10-60 mg/mo). Response to treatment was assessed by comparing the number of bleeding events, hospital admissions for acute bleeding, and number of blood units required during the year before and year after starting OCT-LAR. OCT-LAR led to a reduction in the number of bleeding episodes in all of the children and to cessation of bleeding in 7. Two children listed for transplantation because of severe GIB were removed from the list. No serious adverse effects immediately attributable to OCT-LAR were observed. One child developed growth hormone deficiency and hypothyroidism during a prolonged period of treatment with subcutaneous octreotide before commencing OCT-LAR. OCT-LAR can control severe intractable recurrent GIB in children with portal hypertension. Prospective randomised controlled trials and pharmacokinetic studies are indicated to establish the optimum dose and length of treatment of OCT-LAR and confirm its efficacy and long-term safety in children. [\hyperlink{Octreotide Acetate}{PMID: 25162361}, Marie O'Meara et al., 2015]

\hypertarget{pmid_22299317}{O}ctreotide is an octapeptide that mimics natural somatostatin pharmacologically. It is a potent inhibitor of growth hormone, glucagon and insulin, which is used for treatment of acromegaly, symptomatic treatment of carsinoid tumours, and vasoactive intestinal peptide secreting tumors. It is also used for chylothorax, chemotherapy induced diarrhea and, as it inhibits the exocrine production of pancreatic enzymes, for acute and chronic pancreatitis. Gallbladder stones, diarrhea, nausea, vomiting, hypoglycemia/hyperglycemia, headache, and abdominal discomfort are some of the common adverse effects of octreotide and it may rarely cause anaphylaxis. We present here a child who had chronic pancreatitis and had an anaphylactic reaction to octreotide. To our knowledge this is the first pediatric case of anaphylaxis with octreotide who was successfully desensitized. [\hyperlink{Octreotide Acetate}{PMID: 22299317}, Dilek Azkur et al., 2011]

\hypertarget{pmid_22092874}{O}ctreotide, a somatostatin analogue, is used for the management of patients with refractory chylothorax although its safety and efficacy in neonates have not been evaluated in controlled clinical trials. We present one of the largest case series about the use of octreotide in congenital idiopathic chylothorax. Six cases of congenital chylothorax (CC) were prospectively collected, who were managed with same unit protocol for octreotide. Mean (SD) gestation was 34.5 (±2.2) weeks, and birthweight was 3410 (±840.4) g. All infants required chest drains from day 1 of life, and the mean (SD) duration of insertion was 36.1 (±8.5) days. Octreotide was commenced at a median age of 13.5 days (range 8-22), given for a median duration of 20 days (range 12-27). The starting dose was 0.5-1 μg/kg/h with an increment of 1-2 μg/kg/day to a maximum of 10 μg/kg/day. Resolution of chylothorax was achieved in five patients, being resistant to treatment in the sixth patient. None had adverse effects from octreotide. Full enteral feeds were reached at a mean age of 44 days. Early commencement of octreotide is recommended although further reports to evaluate the safety and efficacy would add to the profile of this medication in the treatment of CC. [\hyperlink{Octreotide Acetate}{PMID: 22092874}, Dharmesh Shah et al., 2012]

\hypertarget{pmid_20512058}{O}ctreotide has been suggested as a medical treatment option in refractory cases of primary intestinal lymphangiectasia (IL). There are few data about the long-term effect and safety of octreotide for IL in the literature. In the present article we analyzed pediatric cases of primary IL with long-term octreotide treatment and discussed its safety profile. Between 1999 and 2008, 13 children were diagnosed in our clinic as having IL. Six patients with primary IL were followed up, receiving octreotide therapy. The clinical data of the patients and duration of therapy, dose, and side effects of octreotide were evaluated. Octreotide, 15 to 20 μg per body weight 2 times daily subcutaneously, was given to all of the patients. Duration of the octreotide treatment changed between 3 and 37 months. Stool frequency decreased in all of the patients after starting octreotide treatment. Serum albumin could be maintained at normal levels in 3 patients. The requirement of albumin infusions decreased in all of the patients. Acute pancreatitis was observed as a side effect of octreotide in 1 patient. Octreotide may help to maintain serum albumin levels, improve clinical findings, and decrease the requirement of albumin infusions in refractory cases of primary IL. [\hyperlink{Octreotide Acetate}{PMID: 20512058}, Sinan Sari et al., 2010]

\hypertarget{pmid_16878051}{W}e review physiology and pharmacology relating to the use of octreotide for chylothorax in infants and children. We review the published experience of octreotide dosing in this context. Systematic review of the literature, including PubMed (English-only journals), citations from relevant articles, major textbooks, and personal files. Octreotide has been used as a successful therapeutic adjunct in a small number of neonatal cases and a larger number of pediatric cases. No consensus has been reached as to the optimal route of administration, dose, duration of therapy, or strategy for discontinuation of therapy. We suggest using higher doses (80-100 microg/kg/day) and initiating therapy early rather than using a low initial dose with upward titration. Duration of therapy required to elicit a significant response may vary between patients. [\hyperlink{Octreotide Acetate}{PMID: 16878051}, Radley D Helin et al., 2006]

\hypertarget{pmid_17803435}{T}he aim of this study was to evaluate the safety of olopatadine hydrochloride ophthalmic solution 0.2\% in children and adolescents 3-17 years of age. In this 6-week, randomized, double-masked safety evaluation, eligible subjects with asymptomatic eyes underwent in-office visits at weeks 1, 3, and 6 and were contacted by telephone at weeks 2, 4, and 5. Qualified subjects were assigned randomly in a 2:1 ratio of olopatadine 0.2\% to vehicle (identical formation without the active ingredient) for dosing on a once-daily schedule. Safety parameters assessed included adverse events, visual acuity, ocular signs (slit-lamp assessments), dilated fundus examinations, intraocular pressure (IOP), pulse, and blood pressure. An evaluation of 126 subjects (age range, 3-17) revealed no clinically relevant treatment-related changes in visual acuity, IOP, slit-lamp assessments, fundus examinations, or cardiovascular parameters. All adverse events reported were mild or moderate. Olopatadine 0.2\% administered once-daily for 6 weeks is safe and well tolerated in children and adolescent patients. [\hyperlink{Octreotide Acetate}{PMID: 17803435}, Steven J Lichtenstein et al., 2007]

\hypertarget{pmid_8234058}{T}he study aimed at assessing the clinical efficiency, safety, and tolerance of cefuroxime axetil suspension in the treatment of children with the acute upper respiratory infections and/or the acute otitis media. The trial was open, multicenter, involving 304 children aged between 3 months and 12 years. They were recruited from 18 general practice centers in Poland. Children were given cefuroxime axetil suspension in the dose of 10 mg/kg body weight (upper respiratory) or 15 mg/kg otitis media. max. 250 mg) bid. Children were examined prior to the treatment, 3-4 days following the start of therapy, 1-2 days after completion of the treatment, and followed-up for 14 days. Post-therapy examination has shown 93\% cure rate. During the follow-up period 0.77\% of patients relapsed. Only minor adverse reactions were reported by 4.9\% of patients. Most common complaint was vomiting. Cefuroxime axetil suspension was safe and effective therapy in the acute upper respiratory infections and the acute otitis media in childhood. [\hyperlink{Octreotide Acetate}{PMID: 8234058}, J Barliński et al., ]

\hypertarget{pmid_22850563}{C}ongenital hyperinsulinism (CHI) is a rare disorder of hypoglycaemia in children due to excessive and dysregulated insulin secretion. Octreotide, a somatostatin analogue, is used in the treatment of hypoglycaemia in Diazoxide unresponsive CHI, but is associated with side effects such as gastrointestinal dysmotility and rarely, necrotising enterocolitis. It would be important to recognise rare but serious side effects from Octreotide therapy, particularly with long-term use. In this report, we have described drug-induced hepatitis with moderately high doses of Octreotide in a child with diffuse CHI. While serum alanine transaminase levels rose significantly with Octreotide therapy (maximum dose 30 μg/kg/day), hepatitis resolved following discontinuation of medical treatment. Liver enzymes should be monitored routinely in children with CHI using long-term Octreotide treatment, particularly with high doses. The presence of drug-induced hepatitis should prompt discontinuation of Octreotide treatment with likely subsequent resolution. [\hyperlink{Octreotide Acetate}{PMID: 22850563}, Bindu Avatapalle et al., 2012]

\hypertarget{pmid_29602276}{C}hylothorax is a rare but life-threatening condition in newborns. Octreotide, a somatostatin analogue, is widely used as a therapeutic option in neonates with congenital and acquired chylothorax, but its therapeutic role has not been clarified yet. We performed a systematic review to assess the efficacy and safety of octreotide in the treatment of congenital and acquired chylothorax in newborns. Comprehensive research, updated till 31 October 2017, was performed by searching in PubMed, MEDLINE, EMBASE and the Cochrane Central Register of Controlled Trials (CENTRAL) databases using the MeSH terms 'octreotide' and 'chylothorax'. Both term and preterm newborns with congenital or acquired chylothorax treated with octreotide within the 30th day of life were included. Octreotide treatment was considered effective if a progressive reduction/ceasing in drained chylous effusion occurred. A total of 39 articles were included. Octreotide was effective in 47\% of patients, with a slight but not significant difference between congenital (30/57; 53.3\%) and acquired (9/27; 33.3\%) chylothorax (P = 0.10). Marked variation in octreotide regimen was observed. The most common therapeutic scheme was intravenous infusion at a starting dose of 1 μg/kg/h, gradually increasing to 10 μg/kg/h according to the therapeutic response. Side effects were reported in 12 of 84 patients (14.3\%). Only case reports were included in this review due to the lack of randomised controlled trials. Octreotide is a relatively effective and safe treatment option in neonates with chylothorax, especially for the congenital forms. [\hyperlink{Octreotide Acetate}{PMID: 29602276}, Carlo Bellini et al., 2018]

\hypertarget{pmid_9577160}{T}he aim of the present trial is to determine the efficacy and safety profile of oxatomide, a potent antiallergic drug, in children aged under 2 years, suffering from atopic dermatitis. A comparison between 2 different dosage schemes was scheduled. An oral suspension of oxatomide, in a neonatological formulation at a low dosage (2.5 mg/mL) was administered to 20 children in the following manner: 1 mg/kg/day in a single evening dose (9); 0.5 mg/kg every 12 hrs (11) for a period of 60 days. Cutaneous symptoms were assessed at the baseline, and after 15, 30 and 60 days. All adverse events were recorded in detail. After 15 days, at both doses, oxatomide significantly reduced (p < 0.05 vs baseline) atopical symptoms: itching, crusting, lesions due to scratching and after 30 days erythema and papulovesicles. No statistically significant differences were observed in terms of efficacy between the two dosage schemes, although oxatomide in a single dose led to a slightly faster improvement. Oxatomide displayed an excellent therapeutic safety profile. The results show that oxatomide in a low-dosage formulation is a good antiallergic drug, effective and safe in a specific population which is extremely delicate, such as children under 2 years old. [\hyperlink{Octreotide Acetate}{PMID: 9577160}, F Bergonzi et al., 1997]

\hypertarget{pmid_16532329}{C}hylothorax is a rare but life-threatening condition in children. To date, there is no commonly accepted treatment protocol. Somatostatin and octreotide have recently been used for treating chylothorax in children. We set out to summarise the evidence on the efficacy and safety of somatostatin and octreotide in treating young children with chylothorax. Systematic review: literature search (Cochrane Library, EMBASE and PubMed databases) and literature hand search of peer reviewed articles on the use of somatostatin and octreotide in childhood chylothorax. Thirty-five children treated for primary or secondary chylothorax (10/somatostatin, 25/octreotide) were found. Ten of the 35 children had been given somatostatin, as i.v. infusion at a median dose of 204 microg/kg/day, for a median duration of 9.5 days. The remaining 25 children had received octreotide, either as an i.v. infusion at a median dose of 68 microg/kg/day over a median 7 days, or s.c. at a median dose of 40 microg/kg/day and a median duration of 17 days. Side effects such as cutaneous flush, nausea, loose stools, transient hypothyroidism, elevated liver function tests and strangulation-ileus (in a child with asplenia syndrome) were reported for somatostatin; transient abdominal distension, temporary hyperglycaemia and necrotising enterocolitis (in a child with aortic coarctation) for octreotide. A positive treatment effect was evident for both somatostatin and octreotide in the majority of reports. Minor side effects have been reported, however caution should be exercised in patients with an increased risk of vascular compromise as to avoid serious side effects. Systematic clinical research is needed to establish treatment efficacy and to develop a safe treatment protocol. [\hyperlink{Octreotide Acetate}{PMID: 16532329}, Charles C Roehr et al., 2006]

\hypertarget{pmid_31993753}{T}o evaluate the feasibility of oral cryotherapy (OC) in children and to investigate if OC reduces the incidence of severe oral mucositis (OM), oral pain, and opioid use in children undergoing hematopoietic stem cell transplantation (HSCT). Fifty-three children, 4-17 years old, scheduled for HSCT in Sweden were included and randomized to OC or control using a computer-generated list. OC instructions were to cool the mouth with ice for as long as possible during chemotherapy infusions with an intended time of ≥ 30 min. Feasibility criteria in the OC group were as follows: (1) compliance ≥ 70\%; (2) considerable discomfort during OC < 20\%; (3) no serious adverse events; and (4) ice administered to all children. Grade of OM and oral pain was recorded daily using the WHO-Oral Toxicity Scale (WHO-OTS), Children's International Oral Mucositis Evaluation Scale, and Numerical Rating Scale. Use of opioids was collected from the medical records. Forty-nine children (mean age 10.5 years) were included in analysis (OC = 26, control = 23). The feasibility criteria were not met. Compliance was poor, especially for the younger children, and only 15 children (58\%) used OC as instructed. Severe OM (WHO-OTS ≥ 3) was recorded in 26 children (OC = 15, control = 11). OC did not reduce the incidence of severe OM, oral pain, or opioid use. The feasibility criteria were not met, and the RCT could not show that OC reduces the incidence of severe OM, oral pain, or opioid use in pediatric patients treated with a variety of conditioning regimens for HSCT. ClinicalTrials.gov id: NCT01789658. [\hyperlink{Octreotide Acetate}{PMID: 31993753}, Tove Kamsvåg et al., 2020]

\hypertarget{pmid_21108438}{T}he Common Toxicity Criteria of the National Cancer Institute evaluates diarrhea as an adverse event of chemotherapy administration. Acute graft versus host disease (aGVHD) causes diarrhea in allogeneic hematopoietic stem cell transplant patients. Guidelines for treating grade 3 and 4 chemotherapy induced diarrhea (CID) include octreotide acetate, a somatostatin analogue. These recommendations are based on adult octreotide trials. Data on octreotide use for treatment of CID in pediatric oncology patients are limited. This study evaluated the efficacy and safety of octreotide in the treatment of CID or aGVHD induced diarrhea in pediatric patients. This is a retrospective review of 34 patients of average age 6 years who received octreotide between 1994 and 2008 for treatment of CID or aGVHD induced diarrhea. Thirty-eight courses of intravenous octreotide were administered. A complete response was achieved during 25/27 (92\%) CID and 5/11 (45\%) aGVHD induced diarrhea courses. A partial response was achieved during 4/38 courses, all in the aGVHD induced diarrhea group. No response was observed for 3 of the aGVHD induced diarrhea courses and 1 for the CID course. Octreotide was initiated at 2 mcg/kg/day and increased to a maximum of 9 mcg/kg/day. The mean total duration of treatment was 9 days. Common adverse effects observed were hyperglycemia, hyberbilirubinemia, nausea/vomiting, and abdominal cramping. In pediatric patients, octreotide exhibits 92\% efficacy in treating CID and 45\% efficacy in aGVHD induced diarrhea. Further studies to better characterize the starting dose and dose escalation algorithm for treating CID in children are required. [\hyperlink{Octreotide Acetate}{PMID: 21108438}, Vinita Pai et al., 2011]

\hypertarget{pmid_10861597}{O}ctreotide acetate is a somatostatin analogue used for the control of endocrine tumors of the gastrointestinal (GI) tract and the treatment of acromegaly. The oral absorption of octreotide is limited because of the limited permeation across the intestinal epithelium. Both chitosan hydrochloride and N-trimethyl chitosan chloride (TMC), a quaternized chitosan derivative, are nonabsorbable and nontoxic polymers that have been proven to effectively increase the permeation of hydrophilic macromolecules across mucosal epithelia by opening the tight junctions. This study investigates the intestinal absorption of octreotide when it is coadministered with the polycationic absorption enhancer TMC. Caco-2 cell monolayers were used as an in vitro intestinal epithelium model, and male Wistar rats were used for in vivo studies. Octreotide with or without polymers (TMC; chitosan hydrochloride) was administered intrajejunally in rats, and serum peptide levels were measured by radioimmunoassay. All applications and administrations were performed at neutral pH values (i.e., pH = 7.4). In vitro transport studies with Caco-2 cells revealed an increased permeation of octreotide in the presence of TMC. Enhancement ratios ranged from 34 to 121 with increasing concentrations of the polymer (0.25-1.5\%, w/v). In rats, 1.0\% (w/v) TMC solution significantly increased the absorption of the peptide analogue, resulting in a 5-fold increase of octreotide bioavailability compared with the controls (octreotide alone). Coadministration of 1.0\% (w/v) chitosan hydrochloride did not enhance octreotide bioavailability. These results in combination with the nontoxic character of TMC suggest that this polymer is a promising excipient in the development of solid dosage forms for the peroral delivery and intestinal absorption of octreotide. [\hyperlink{Octreotide Acetate}{PMID: 10861597}, M Thanou et al., 2000]

\hypertarget{pmid_16028153}{B}ecause of concerns about arthrotoxicity, fluoroquinolones are restricted for use in children. This study describes the safety and efficacy of gatifloxacin when used for treatment of children with recurrent acute otitis media (ROM) or acute otitis media (AOM) treatment failure (AOMTF). We performed an analysis of 867 children included in 4 clinical trials who had ROM and/or AOMTF and were treated with gatifloxacin (10 mg/kg once daily for 10 days). Gatifloxacin had adverse event rates that were similar overall to those of a comparator antibiotic (amoxicillin-clavulanate), except for increased diarrhea in children <2 years old receiving amoxicillin-clavulanate. There was no evidence of arthrotoxicity, hepatotoxicity, alteration of glucose homeostasis, or central nervous system toxicity acutely or during 1 year follow-up in any child. Regarding efficacy, in 2 noncomparative trials, the gatifloxacin cure rate of AOM was 89\% (95\% confidence interval [CI], 83\%-95\%) at the test of cure (TOC) visit, 3-10 days after completion of therapy. In 2 comparative trials of gatifloxacin versus amoxicillin-clavulanate, the efficacy of gatifloxacin was 88\% (95\% CI, 82\%-94\%). Gatifloxacin led to better clinical outcomes than amoxicillin-clavulanate for AOMTF (91\% vs. 81\%; P=.029), for AOMTF and age <2 years old (89\% vs. 69\%; P=.009), and for severe AOM in children <2 years old (90\% vs. 75\%; P=.012). Among children with AOMTF previously treated with amoxicillin-clavulanate or ceftriaxone injections, gatifloxacin cure rates were high (88\% and 75\%, respectively). Gatifloxacin appears to be safe for children, with no evidence of producing arthrotoxicity in 867 children exposed to the antibiotic when used as treatment for ROM and AOMTF. [\hyperlink{Octreotide Acetate}{PMID: 16028153}, Michael E Pichichero et al., 2005]

\section*{Amoxicillin And Omeprazole Magnesium And Rifabutin}
\subsection*{Result}
\subsubsection*{Answer}

Unknown

\subsubsection*{{Explanation}}
\hypertarget{Amoxicillin And Omeprazole Magnesium And Rifabutin}
A review of the available abstracts reveals multiple studies on the safety and efficacy of amoxicillin, amoxicillin-clavulanate, and omeprazole in children, as well as some studies on rifabutin in other contexts. However, none of the abstracts specifically address the safety of the combination product "Amoxicillin And Omeprazole Magnesium And Rifabutin" in children. 

Several abstracts confirm the safety and efficacy of amoxicillin and amoxicillin-clavulanate in children for various infections, including acute otitis media, sinusitis, and pyodermas, with age ranges from infants to adolescents (e.g., 2-16 years, 6 months to 10 years, 7 months to 12 years) [\hyperlink{pmid_3520469}{PMID: 3520469}, E R Wald et al., 1986; \hyperlink{pmid_2467730}{PMID: 2467730}, R Chopra et al., 1989; \hyperlink{pmid_8970215}{PMID: 8970215}, E M Mandel et al., 1996; \hyperlink{pmid_20947990}{PMID: 20947990}, P K Kar et al.]. Omeprazole is mentioned in the context of overdose and use in children under 5 years, but not in combination with antibiotics [\hyperlink{pmid_28425352}{PMID: 28425352}, C Crawford et al., 2018]. There are no abstracts that discuss the use or safety of rifabutin in children, nor any that study the triple combination of amoxicillin, omeprazole, and rifabutin in pediatric populations.

Therefore, based on the abstracts available, there is no evidence from targeted studies affirming the safety or unsafety of the combination "Amoxicillin And Omeprazole Magnesium And Rifabutin" in children of any age range. The safety of this combination in children is unknown.

\subsection*{Abstracts}
\hypertarget{pmid_3520469}{T}his study compared the relative effectiveness of two antimicrobial preparations, amoxicillin and amoxicillin-clavulanate potassium (Augmentin), in the treatment of acute maxillary sinusitis in children 2 to 16 years of age. Of 171 children with persistent (ten to 30 days' duration) nasal discharge or daytime cough or both, 136 (80\%) had abnormal maxillary sinus radiographs. These children were stratified by age and severity of symptoms and randomly assigned to receive either amoxicillin, amoxicillin-clavulanate potassium, or placebo. After the exclusion of 28 children with throat cultures positive for group A Streptococcus and 15 who did not complete their medication, the remaining 93 children were evaluated: 30 received amoxicillin, 28 received amoxicillin-clavulanate potassium, and 35 received placebo. Clinical assessment was performed at three and ten days. On each occasion, children treated with an antibiotic were more likely to be cured than children receiving placebo (P less than .01 at three days, P less than .05 at ten days). The overall cure rate was 67\% for amoxicillin, 64\% for amoxicillin-clavulanate potassium, and 43\% for placebo. [\hyperlink{Amoxicillin And Omeprazole Magnesium And Rifabutin}{PMID: 3520469}, E R Wald et al., 1986]

\hypertarget{pmid_10829995}{D}uring the past few years, a number of drugs have been added to the anti-epileptic arsenal. This review focusses on five of these drugs which have undergone extensive trials: Vigabatrin, Lamotrigine, Gabapentin, Felbamate and Oxcarbazepine. Some of these antiepileptic drugs appear to be helpful for treatment of catastrophic childhood epilepsies. Vigabatrin appears promising in children with infantile spasms who do not respond to ACTH or Prednisolone. Children with Lennox-Gastaut syndrome may respond to treatment with Lamotrigine or Vigabatrin. Gabapentin and vigabatrin have proved to be effective in refractory partial seizures. Oxcarbazepine, a ketoderivative of carbamazepine, is as effective as Carbamazepine but has a better safety profile. Lesser neurotoxicity and fewer drug interactions is another advantage with these drugs. However monitoring is required to determine the long term safety with their usage. These drugs have a definite role in childhood epilepsies refractory to conventional antiepileptic drugs. [\hyperlink{Amoxicillin And Omeprazole Magnesium And Rifabutin}{PMID: 10829995}, S Aneja et al., ]

\hypertarget{pmid_2467730}{A}moxicillin, a semisynthetic aminopenicillin, has achieved widespread use in recent years for the treatment of respiratory tract and otic infections. Serious reactions have been relatively infrequent. From July 1986 to June 1987, 11 children aged 6 months to 10 years presented with delayed-onset hypersensitivity reactions. In 10 the symptoms were consistent with a serum-sickness-like illness, including urticaria, angioedema, arthritis and arthralgia. Radioallergosorbent testing for IgE antibodies to penicillin yielded negative results, and lymphocyte transformation testing gave a positive result in only one patient. Because of the negative immunologic test results and the occurrence of reactions only in children, who had received an amoxicillin solution, the reactions may have been caused by the excipient. [\hyperlink{Amoxicillin And Omeprazole Magnesium And Rifabutin}{PMID: 2467730}, R Chopra et al., 1989]

\hypertarget{pmid_23650467}{C}hildhood epilepsy continues to be intractable in more than 25\% of patients diagnosed with epilepsy. The introduction of new anti-epileptic drugs (AEDs) provides more options for treatment of children with epilepsy. We review the safety and tolerability of seven new AEDs (levetiracetam, lamotrigine, oxcarbazepine, rufinamide, topiramate, vigabatrin and zonisamide) focusing on their side effect profiles and safety in children and adolescents. Many considerations that are specific for children such as the impact of AEDs on the developing brain are not addressed during the development of new AEDs. They are usually approved as adjunctive therapies based upon clinical trials involving adult patients with partial epilepsy. However, 2 of the AEDs reviewed here (rufinamide and vigabatrin) have FDA approval in the U.S. for specific Pediatric epilepsy syndromes, which are discussed below. The Pediatrician or Neurologists decision on the use of a new AED is an evolutionary process largely dependent on the patient characteristics, personal/peer experiences and literature about efficacy and safety profiles of these medications. Evidence based guidelines are limited due to a lack of randomized controlled trials involving pediatric patients for many of these new AEDs. [\hyperlink{Amoxicillin And Omeprazole Magnesium And Rifabutin}{PMID: 23650467}, Saima Kayani et al., 2012]

\hypertarget{pmid_8970215}{T}his trial compared the efficacy of amoxicillin prophylaxis with that of placebo for the management of recurrent middle ear effusion (MEE) in children. Children between 7 months and 12 years of age who were effusion-free at entry but had histories of chronic or recurrent MEE were randomly assigned to receive either amoxicillin (20 mg/kg once daily) or placebo for 1 year. They were examined monthly and when there were symptoms of ear, nose or throat disease. Acute otitis media (AOM) and new episodes of otitis media with effusion (OME) were treated with amoxicillin-clavulanate; tympanocentesis was performed when possible for episodes of AOM. Throat cultures were obtained at entry; 4, 8 and 12 months after entry; and with new episodes of AOM and OME. Tympanometry was performed at each visit and audiometry was performed at entry and 4, 8 and 12 months after entry. One hundred eleven children were entered in this study. The rates per person year of new episodes of disease in the amoxicillin and placebo groups, respectively, were: MEE, 1.81 vs. 3.18 (P < 0.001); AOM, 0.28 vs. 1.04 (P < 0.001); and OME, 1.53 vs. 2.15 (P = 0.016). Subjects in the amoxicillin group had less time with MEE than the placebo group (19.7 and 33.2\%, respectively; P = 0.002). Middle ear and throat cultures did not reveal any increase in beta-lactamase-producing organisms or in Streptococcus pneumoniae attributable to daily use of amoxicillin. Amoxicillin prophylaxis lowered the rates of occurrence of MEE, AOM and OME and decreased the percentage of time with MEE. However, because of present day concerns regarding antibiotic resistance, management should be individualized. [\hyperlink{Amoxicillin And Omeprazole Magnesium And Rifabutin}{PMID: 8970215}, E M Mandel et al., 1996]

\hypertarget{pmid_8436464}{W}e enrolled 60 children with recurrent acute otitis media (AOM) in a study of the effectiveness of antimicrobial prophylaxis. All children were entered into the study following an acute episode of infection treated with amoxicillin (AMX) for 10 days. Following therapy, the children were re-examined, and then randomly assigned to receive either trimethoprim-sulfamethoxazole (TMP-SMX), amoxicillin (AMX) or a placebo (PLA). Twenty children were included in each group. Each drug was administered once a day at bedtime, at 1/3 the therapeutic dose, for 3 months. Children were re-evaluated with pneumootoscopy during episodes of acute illness and with pneumootoscopy and impedance tympanometry (TYMP) at monthly intervals. We observed a significantly increased rate of recurrent AOM in children receiving placebo compared with those who received antibiotics (50\% vs. 17\% P < 0.005). Both prophylactic antibiotics were equally effective in preventing recurrent AOM (recurrence rate 20\% TMP-SMX, 15\% AMX). We also observed that recurrences in children receiving placebo occurred earlier in the study period than in those receiving antibiotics. These results suggest that antimicrobial prophylaxis in children with recurrent acute otitis media is effective in reducing subsequent disease. The similar efficacy of both antibiotics tested suggests that the less expensive agent should be used. [\hyperlink{Amoxicillin And Omeprazole Magnesium And Rifabutin}{PMID: 8436464}, T Sih et al., 1993]

\hypertarget{pmid_11918466}{E}ight new anticonvulsant medications have been approved in the United States since 1993, offering physicians a greater range of options for treating children with partial and generalized seizures. However, pediatric neurologists have been faced with limited pediatric pharmacokinetic and pharmacodynamic information. This article reviews the newer antiepilepsy drugs-gabapentin, felbamate, lamotrigine, topiramate, oxcarbazepine, levetiracetam, and zonisamide-and summarizes what is currently known about the safety and efficacy of these drugs in treating partial and generalized seizures in the pediatric population. Further studies are needed, however, to thoroughly evaluate their efficacy and safety in children. [\hyperlink{Amoxicillin And Omeprazole Magnesium And Rifabutin}{PMID: 11918466}, Martina Bebin et al., 2002]

\hypertarget{pmid_2589274}{W}e compared the efficacy of amoxicillin with that of the combination drug sulfamethoxazole and trimethoprim in reducing recurrences of acute otitis media (AOM) in a single-blind, randomized, placebo-controlled trial involving 96 children. Each of the children had had three or more episodes of AOM in the preceding 6 months, and 97\% (93/96) of them still had unilateral or bilateral effusion at the beginning of the study. During the 6-month study period, 9 (27\%) of 33 of the children in the amoxicillin group developed 9 episodes of AOM, 9 (27\%) of 33 of the children in the sulfamethoxazole and trimethoprim group experienced 11 episodes of AOM, and 19 (63\%) of 30 of the children in the placebo group developed 25 episodes. Young age and day-care attendance characterized children for whom prophylaxis was more efficacious. Overall persistence of middle-ear effusion was shorter in treated children only as a consequence of the reduced number of new episodes of AOM. [\hyperlink{Amoxicillin And Omeprazole Magnesium And Rifabutin}{PMID: 2589274}, N Principi et al., 1989]

\hypertarget{pmid_24382900}{A} number of newer anti-epileptic drugs have been developed in the last few years to improve the treatment outcomes in epilepsy. In this review, we discuss the use of newer anti-epileptic drugs in children. MEDLINE search (1966-2013) was performed using terms newer anti-epileptic drugs, Oxcarbazepine, vigabatrin, topiramate, zonisamide, levetiracetam, lacosamide, rufinamide, stiripentol, retigabine, eslicarbazepine, brivaracetam, ganaxolone and perampanel for reports on use in children. Review articles, practice parameters, guidelines, systematic reviews, meta-analyses, randomized controlled trials, cohort studies, and case series were included. The main data extracted included indications, efficacy and adverse effects in children. Oxcarbazepine is established as effective initial monotherapy for children with partial-onset seizures. Vigabatrin is the drug of choice for infantile spasms associated with tuberous sclerosis. Lamotrigine, levetiracetam and lacosamide are good add-on drugs for patients with partial seizures. Lamotrigine may be considered as monotherapy in adolescent females with idiopathic generalized epilepsy. Levetiracetam is a good option as monotherapy for females with juvenile myoclonic epilepsy. Topiramate is a good add-on drug in patients with epileptic encephalopathies such as Lennox-Gastaut syndrome and myoclonic astatic epilepsy. [\hyperlink{Amoxicillin And Omeprazole Magnesium And Rifabutin}{PMID: 24382900}, Satinder Aneja et al., 2013]

\hypertarget{pmid_37116578}{U}nnecessary and inappropriate antibiotic use is an increasing global health challenge. In limited resource settings, prophylactic antibiotics are still often used in (adeno)tonsillectomy (AT), despite evidence against their effectiveness. This study aimed to investigate the effect of prophylactic amoxicillin, given after AT in children. This is a secondary analysis from a two-center, double-blinded, randomized controlled, non-inferiority trial to study the effect of prophylactic amoxicillin on post-AT morbidity. Children aged 2-14 years with recurrent chronic tonsillitis and/or obstructive sleep apnea were randomly assigned to receive either placebo or amoxicillin for 5 days after the operation. Pre- and postoperative samples were collected for polymerase chain reaction (PCR) analyses to detect the five most important pathogens known to be common causes of tonsillitis. PCR results were compared before and after surgery as well as between placebo and amoxicillin. PCR results were obtained, 109 in the amoxicillin group and 115 in the placebo group. In the amoxicillin group, 91\% of patients had at least one positive PCR test before surgery and 87\% after surgery. In the placebo group, the respective percentages were 92\% and 90\%. In both groups, a decrease in the total number of pathogens was found after surgery. Prophylactic amoxicillin given after AT in children did not show a clinically relevant effect with respect to the number of oropharyngeal microorganisms as compared to placebo. [\hyperlink{Amoxicillin And Omeprazole Magnesium And Rifabutin}{PMID: 37116578}, Denis R Katundu et al., 2023]

\hypertarget{pmid_2685718}{I}n summary, infants and children who have acute otitis media should receive antimicrobial therapy. Amoxicillin is the standard of therapy for infants and children with acute otitis media, because it is safe and effective for most of the causative bacterial pathogens. Amoxicillin has also been shown to be effective for treatment of selected children with otitis media with effusion ("secretory" otitis media) and is the recommended prophylactic antimicrobial agent for prevention of frequently recurrent acute otitis media. During the past decade, however, an increasing rate of bacteria that are resistant to amoxicillin has occurred, primarily beta-lactamase-producing H. influenzae and B. catarrhalis. Because of the emergence of these bacteria, other antimicrobial agents, both old and new, have been advocated for treatment and prevention of otitis media; amoxicillin-clavulanate, cefuroxime axetil, and cefixime are the newer agents. These agents are indicated for selected infants and children; however, for most patients, amoxicillin remains a safe and relatively inexpensive effective drug. The common surgical procedures, such as myringotomy with tympanostomy tube insertion, and adenoidectomy with myringotomy with or without tympanostomy tube insertion, have now been shown to be effective for patients who have recurrent acute otitis media and chronic otitis media with effusion. The decision for or against these procedures should not only include consultation with an otolaryngologist but should also involve the parents and the child, if old enough. The risks, costs, and benefits of nonsurgical and surgical management should be discussed with all parties concerned. [\hyperlink{Amoxicillin And Omeprazole Magnesium And Rifabutin}{PMID: 2685718}, C D Bluestone et al., 1989]

\hypertarget{pmid_30870164}{O}pioids are a mainstay of perioperative analgesia. Opioid use in children with obstructive sleep apnea is challenging because of assumptions for increased opioid sensitivity and assumed risk for opioid-induced respiratory depression compared to children without obstructive sleep apnea. These assumptions have not been rigorously tested. This investigation tested the hypothesis that children with obstructive sleep apnea have an increased pharmacodynamic sensitivity to the miotic and respiratory depressant effects of the prototypic μ-opioid agonist remifentanil. Children (8 to 14 yr) with or without obstructive sleep apnea were administered a 15-min, fixed-rate remifentanil infusion (0.05, 0.1, or 0.15 μg · kg · min). Each dose group had five patients with and five without obstructive sleep apnea. Plasma remifentanil concentrations were measured by tandem liquid chromatography mass spectrometry. Remifentanil effects were measured via miosis, respiratory rate, and end-expired carbon dioxide. Remifentanil pharmacodynamics (miosis vs. plasma concentration) were compared in children with or without obstructive sleep apnea. Remifentanil administration resulted in miosis in both non-obstructive sleep apnea and obstructive sleep apnea patients. No differences in the relationship between remifentanil concentration and miosis were seen between the two groups at any of the doses administered. The administered dose of remifentanil did not affect respiratory rate or end-expired carbon dioxide in either group. No differences in the remifentanil concentration-miosis relation were seen in children with or without obstructive sleep apnea. The dose and duration of remifentanil administered did not alter ventilatory parameters in either group. [\hyperlink{Amoxicillin And Omeprazole Magnesium And Rifabutin}{PMID: 30870164}, Michael C Montana et al., 2019]

\hypertarget{pmid_22966729}{A}moxicillin is one of the most used antibiotics among pediatric patients for the treatment of upper respiratory tract infections and specially for acute otitis media (AOM), a common diseases of infants and childhood. It has been speculated that the use of amoxicillin during early childhood could be associated with dental enamel fluorosis, also described in literature with the term Molar Incisor Hypomineralization (MIH), because they are generally situated in one or more 1st permanent molars and less frequently in the incisors. The effect ofAmoxicillin seems to be independent of other risk factors such as fluoride intake, prematurity, hypoxia, hypocalcaemia, exposure to dioxins, chikenpox, otitis media, high fever and could have a significant impact on oral health for the wide use of this drug in that period of life. The aim of this work was to review the current literature about the association between amoxicillin and fluorosis. A literature survey was done by applying the Medline database (Entrez PubMed); the Cochrane Library database of the Cochrane Collaboration (CENTRAL). The databases were searched using the fol-lowing strategy and keywords: amoxicillin* AND (dental fluorosis* OR dental enamel*) and MIH*. After selecting the studies, only three relevant articles published between 1966 and 2011 were included in the review. The presence of several methodological issues does not allow to draw any evidence-based conclusions. No evidence of association was detected, therefore, there is a need of further well-designed studies to assess the scientific evidence of the relationship between amoxicillin and fluorosis and to restrict the prescription of this drug for recurrent upper respiratory tract infections especially acute otitis media (AOM) during the first two years of life. When it is possible can be opportune to use an alternative antibiotic treatment. [\hyperlink{Amoxicillin And Omeprazole Magnesium And Rifabutin}{PMID: 22966729}, I Ciarrocchi et al., ]

\hypertarget{pmid_23078168}{P}aracetamol (acetaminophen) and ibuprofen are the most frequently purchased over-the-counter (OTC) medicines for children. Parents purchase these medicines for the treatment of fever and pain. In some countries other NSAIDs such as aspirin (acetylsalicylic acid) and dipyrone are available. We aimed to perform a narrative review of the efficacy and toxicity of OTC analgesic medicines for children in order to give guidance to health professionals and parents regarding the treatment of pain in a child. Neither aspirin nor dipyrone are recommended for OTC use because of the association with Reye's syndrome for the former and the risk of agranulocytosis for the latter. Both paracetamol and ibuprofen are effective for the treatment of mild pain in children. Adverse effects with both medicines are infrequent. Ibuprofen is an NSAID and therefore there is a greater risk of gastrointestinal adverse effects and hypersensitivity. Aspirin and dipyrone should be avoided. Paracetamol is the drug of first choice for mild pain in children because of its favourable safety profile. For the treatment of significant musculoskeletal pain, ibuprofen is the drug of first choice. [\hyperlink{Amoxicillin And Omeprazole Magnesium And Rifabutin}{PMID: 23078168}, Zeina Bárzaga Arencibia et al., 2012]

\hypertarget{pmid_17725220}{T}he most recent antiepileptic drugs used in children are lamotrigine, topiramate, oxcarbamaz6pine and levetiracetam. Their efficacy is proven, depending on the type of crisis, but in Belgium they are reimbursed only in certain conditions. The treatment of children with attention deficit hyperactivity disorder (ADHD), which was only constituted of methylphenidate, can now benefit from atomoxetine whose mechanism of action is different. [\hyperlink{Amoxicillin And Omeprazole Magnesium And Rifabutin}{PMID: 17725220}, P Leroy et al., ]

\hypertarget{pmid_20401256}{A}cetaminophen has become the non-narcotic of choice for children because of concerns regarding the connection between acetylsalicylic acid exposure and Reye's syndrome. Ibuprofen, recently granted over-the-counter status for children over two years of age, offers another choice for treatment. The efficacy and safety of both drugs have been studied in numerous clinical trials. This paper reviews the published evidence about the efficacy and safety of acetaminophen and ibuprofen with regard to treating fever and mild to moderate pain in children. [\hyperlink{Amoxicillin And Omeprazole Magnesium And Rifabutin}{PMID: 20401256}, H N McCullough et al., 1998]

\hypertarget{pmid_16028153}{B}ecause of concerns about arthrotoxicity, fluoroquinolones are restricted for use in children. This study describes the safety and efficacy of gatifloxacin when used for treatment of children with recurrent acute otitis media (ROM) or acute otitis media (AOM) treatment failure (AOMTF). We performed an analysis of 867 children included in 4 clinical trials who had ROM and/or AOMTF and were treated with gatifloxacin (10 mg/kg once daily for 10 days). Gatifloxacin had adverse event rates that were similar overall to those of a comparator antibiotic (amoxicillin-clavulanate), except for increased diarrhea in children <2 years old receiving amoxicillin-clavulanate. There was no evidence of arthrotoxicity, hepatotoxicity, alteration of glucose homeostasis, or central nervous system toxicity acutely or during 1 year follow-up in any child. Regarding efficacy, in 2 noncomparative trials, the gatifloxacin cure rate of AOM was 89\% (95\% confidence interval [CI], 83\%-95\%) at the test of cure (TOC) visit, 3-10 days after completion of therapy. In 2 comparative trials of gatifloxacin versus amoxicillin-clavulanate, the efficacy of gatifloxacin was 88\% (95\% CI, 82\%-94\%). Gatifloxacin led to better clinical outcomes than amoxicillin-clavulanate for AOMTF (91\% vs. 81\%; P=.029), for AOMTF and age <2 years old (89\% vs. 69\%; P=.009), and for severe AOM in children <2 years old (90\% vs. 75\%; P=.012). Among children with AOMTF previously treated with amoxicillin-clavulanate or ceftriaxone injections, gatifloxacin cure rates were high (88\% and 75\%, respectively). Gatifloxacin appears to be safe for children, with no evidence of producing arthrotoxicity in 867 children exposed to the antibiotic when used as treatment for ROM and AOMTF. [\hyperlink{Amoxicillin And Omeprazole Magnesium And Rifabutin}{PMID: 16028153}, Michael E Pichichero et al., 2005]

\hypertarget{pmid_20947990}{T}he efficacy and safety of amoxicillin plus clavulanic acid was compared with that of amoxicillin, erythromycin and co-trimoxazole in an open label, randomized trial in 50 children in each group (total 200) with mild to severe pyodermas. Majority (47\%) had impetigo. Fifty (25\%) children had mild pyoderma, 56 (28\%) had moderate and 94 (47\%) children had severe pyoderma. Pure growth of S aureus was isolated in 130 (65\%) children, S pyogenes in 42 (21\%) and both organisms in 28 (14\%) children. In mild to moderate pyoderma either of the drug tried was equally effective. In severe pyoderma, 24 of twenty five (96\%) children receiving amoxicillin plus clavulanic acid, 18 of twenty (90\%) children in amoxicillin group, 20 of twenty four (83.3\%) children in erythromycin group and 13 of twenty five (52\%) children in co-trimoxazole group showed clinical cure of therapy. Amoxicillin combined with clavulanic acid was well tolerated in children and there was no significant side effect except mild diarrhoea in two cases (4\%) which was well controlled by taking the drug with meals. [\hyperlink{Amoxicillin And Omeprazole Magnesium And Rifabutin}{PMID: 20947990}, P K Kar et al., ]

\hypertarget{pmid_19564277}{T}he role of antibiotic therapy in managing acute bacterial sinusitis (ABS) in children is controversial. The purpose of this study was to determine the effectiveness of high-dose amoxicillin/potassium clavulanate in the treatment of children diagnosed with ABS. This was a randomized, double-blind, placebo-controlled study. Children 1 to 10 years of age with a clinical presentation compatible with ABS were eligible for participation. Patients were stratified according to age (<6 or >or=6 years) and clinical severity and randomly assigned to receive either amoxicillin (90 mg/kg) with potassium clavulanate (6.4 mg/kg) or placebo. A symptom survey was performed on days 0, 1, 2, 3, 5, 7, 10, 20, and 30. Patients were examined on day 14. Children's conditions were rated as cured, improved, or failed according to scoring rules. Two thousand one hundred thirty-five children with respiratory complaints were screened for enrollment; 139 (6.5\%) had ABS. Fifty-eight patients were enrolled, and 56 were randomly assigned. The mean age was 66 +/- 30 months. Fifty (89\%) patients presented with persistent symptoms, and 6 (11\%) presented with nonpersistent symptoms. In 24 (43\%) children, the illness was classified as mild, whereas in the remaining 32 (57\%) children it was severe. Of the 28 children who received the antibiotic, 14 (50\%) were cured, 4 (14\%) were improved, 4 (14\%) experienced treatment failure, and 6 (21\%) withdrew. Of the 28 children who received placebo, 4 (14\%) were cured, 5 (18\%) improved, and 19 (68\%) experienced treatment failure. Children receiving the antibiotic were more likely to be cured (50\% vs 14\%) and less likely to have treatment failure (14\% vs 68\%) than children receiving the placebo. ABS is a common complication of viral upper respiratory infections. Amoxicillin/potassium clavulanate results in significantly more cures and fewer failures than placebo, according to parental report of time to resolution of clinical symptoms. [\hyperlink{Amoxicillin And Omeprazole Magnesium And Rifabutin}{PMID: 19564277}, Ellen R Wald et al., 2009]

\hypertarget{pmid_20152073}{U}pper respiratory tract infections in children are common and usually self-limiting conditions, which include acute otitis media (AOM), acute rhinosinusitis (ARS), and acute pharyngitis (AP). Management of pediatric AOM considers observation strategy for selected and uncomplicated cases, older than 2 years of age, only when adequate follow-up can be ensured. Otherwise, an antibiotic treatment should be prescribed. Amoxicillin should be preferred as the first-choice therapy. Switch therapy to ceftriaxone is suggested if amoxicillin regimen failure occurs within 48-72 hours. The diagnosis of ARS is established by the persistence of purulent nasal of post-nasal draining lasting at least 10 days especially if accompanied by supporting symptoms and signs. Amoxicillin is the first choice drug for mild ARS in children. When symptoms persist or worsen, amoxicillin/clavulanate or cefpodoxime proxetil, or ceftriaxone are recommended. Clinical criteria alone are not sufficiently accurate in children with AP to distinguish bacterial and viral etiology. Thus microbiological evaluation is needed and positive throat culture or rapid antigen detection test are required to establish the diagnosis of streptococcal pharyngitis and consequently to prescribe antibiotic treatment. The first choice treatment in European countries still remains amoxicillin or amoxicillin/clavulanate. [\hyperlink{Amoxicillin And Omeprazole Magnesium And Rifabutin}{PMID: 20152073}, F Bonsignori et al., ]

\hypertarget{pmid_338647}{A}mpicillin and amoxicillin were evaluated in 37 ill children. Detailed pharmacokinetic studies in 27 of these children demonstrated an advantage in oral absorption of amoxicillin over ampicillin at dosages of both 12.5 and 25 mg/kg per dose. Individual variation was great for both drugs. No sequence effect was noted for patients receiving ampicillin before either ampicillin or amoxicillin. Amoxicillin was tolerated well by the majority of patients, and the drug was not discontinued because of side effects in any patient. No toxicities were noted for amoxicillin in any of the 20 patients studied for abnormalities in hematologic hepatic, and renal functions. Pharmacokinetics, clinical efficacy, tolerance, and toxicity studies support the clinical usage of amoxicillin in pediatric infectious diseases. However, comparative, controlled clinicalinvestigations are needed to better define the clinical advantages of this drug over ampicillin. [\hyperlink{Amoxicillin And Omeprazole Magnesium And Rifabutin}{PMID: 338647}, M I Marks et al., 1978]

\hypertarget{pmid_21836758}{T}o systemically review the evidence in support of World Health Organization guidelines recommending broad-spectrum antibiotics for children with severe acute malnutrition (SAM). CENTRAL, MEDLINE, EMBASE, LILACS, POPLINE, CAB Abstracts and ongoing trials registers were searched. Experts were contacted. Conference proceedings and reference lists were manually searched. All study types, except single case reports, were included. Two randomized controlled trials (RCTs), one before-and-after study and two retrospective reports on clinical efficacy and safety were retrieved, together with 18 pharmacokinetic studies. Trial quality was generally poor and results could not be pooled due to heterogeneity. Oral amoxicillin for 5 days was as effective as intramuscular ceftriaxone for 2 days (1 RCT). For uncomplicated SAM, amoxicillin showed no benefit over placebo (1 retrospective study). The introduction of a standardized regimen using ampicillin and gentamicin significantly reduced mortality in hospitalized children (odds ratio, OR: 4.0; 95\% confidence interval, CI: 1.7-9.8; 1 before-and-after study). Oral chloramphenicol was as effective as trimethoprim-sulfamethoxazole in children with pneumonia (1 RCT). Pharmacokinetic data suggest that normal doses of penicillins, cotrimoxazole and gentamicin are safe in malnourished children, while the dose or frequency of chloramphenicol requires adjustment. Existing evidence is not strong enough to further clarify recommendations for antibiotic treatment in children with SAM. Large RCTs are needed to define optimal antibiotic treatment in children with SAM with and without complications. Further research into gentamicin and chloramphenicol toxicity and into the pharmacokinetics of ceftriaxone and ciprofloxacin is also required. [\hyperlink{Amoxicillin And Omeprazole Magnesium And Rifabutin}{PMID: 21836758}, Marzia Lazzerini et al., 2011]

\hypertarget{pmid_14998146}{T}o study the possible pharmacodynamic differences in children under anesthesia based on remifentanil. This multicenter observational study enrolled 275 patients scheduled for minor pediatric surgery (herniorrhaphy, prepuceplasty, and plastic surgery). The children were grouped by age: 1-3 years, 4-7 years, 8-12 years. Premedication was with midazolam, induction with sevoflurane or propofol, and maintenance with sevoflurane 0.5\%-0.8\%, N2O/O2 30\%/70\%, and remifentanil 0.25 microg/kg/min. Postoperative analgesia (metamizol, morphine or regional block) was administered at least 30 minutes before the end of surgery. No differences were found between age groups with regard to remifentanil requirements, end tidal concentrations of sevoflurane, or mean times until spontaneous ventilation or extubation. Nor were there differences in the percentages of complications (5.4\%), of which 4 were cases of intense postoperative muscular rigidity, or in the incidence of nausea-vomiting (3.4\%). The quality of recovery from anesthesia (Aldrete test) was better in the 8-12-year-old group (P < 0.05), however, while agitation (Postoperative Agitation Score) and pain (visual analog scale or observational scales) were greater in the group of 1-3-year-olds (P < 0.05). The evaluation of the technique by participating caregivers was excellent for 20\%, very good for 41\%, good for 29\%, adequate for 8\% and poor for 2\% of the cases. [\hyperlink{Amoxicillin And Omeprazole Magnesium And Rifabutin}{PMID: 14998146}, F Reinoso-Barbero et al., 2004]

\hypertarget{pmid_11827845}{R}egarding the antiepileptic drugs (AEDs) in children, it was recently shown that they have a specific profile of efficacy and also of worsening according to the different epilepsy syndromes. However, the therapeutic profile of the most recent compounds is still not completely established in children because of the high number of syndromes and the difficulty to perform controlled studies in this age range. Controlled studies are most often first performed in adults and they begin in children whereas the new drug is already approved. However, some new AEDs dramatically improved seizure control, particularly in some severe epilepsy syndromes such as West syndrome and Lennox-Gastaut syndrome. Vigabatrin demonstrated a remarkable efficiency in infantile spasms whereas it tends to worsen myoclonic epilepsies, absence epilepsy and idiopathic partial epilepsy. Lamotrigine is efficient in absence epilepsy and symptomatic or cryptogenic generalized epilepsies such as Lennox-Gastaut syndrome and myoclonic astatic epilepsy. By contrast, lamotrigine increases the frequency of seizures in severe myoclonic epilepsy in infancy (Dravet syndrome) as well as in some cases of idiopathic partial epilepsy. Felbamate remains indicated as third line drug in refractory Lennox-Gastaut syndrome provided blood parameters are controlled. The therapeutic profile of oxcarbazepine is closed to that of carbamazepine. The efficacy of topiramate was demonstrated in partial epilepsy, but the other indications remain to be précised. Pediatric studies using gabapentin and tiagabine disclosed encouraging results in partial epilepsy. Clinical trials with stiripentol represent an example of strategy for developing a new AED in children; it recently demonstrated, in association with clobazaru, efficacy in a severe myoclonique epilepsy in infancy. [\hyperlink{Amoxicillin And Omeprazole Magnesium And Rifabutin}{PMID: 11827845}, A Kaminska et al., 2001]

\hypertarget{pmid_28425352}{A}ccidental drug overdose is a common problem in young children. We examined the influence of formulation and dose in enquiries for different gastro-oesophageal reflux disease treatments in children under 5 years to the UK's National Poisons Information Service. Overdose characteristics with ranitidine, omeprazole or domperidone were compared with those of metoclopramide and the H-1 antagonist chlorphenamine, for the period 1 July 2007 to 30 June 2015. There were a total of 1092 ranitidine, 618 domperidone and 1193 omeprazole cases; 669, 281 and 424, respectively, were single agent enquiries; of these 77\% (517) of ranitidine, 52\% (145) domperidone and 32\% (135) omeprazole cases occurred in children <5 years. In comparison, 17\% (34/424) of metoclopramide and 53\% (533/1013) of chlorphenamine were <5 years; 79\% (410/517) of ranitidine overdose enquiries in children <5 years were under 6 months of age, higher than domperidone (68/145, 47\%; p < 0.05), omeprazole (8/135, 6\%), chlorphenamine (13/553, 2\%) or metoclopramide (1/34, 3\%) (all p < 0.01). In children aged <6 months, 101 were 10-fold overdoses, 86 with ranitidine. Tenfold overdoses in children (<5 years) were a feature of ranitidine enquiries, likely due to the high concentration of the syrup. This has relevance to other liquid formulations used for non-licenced indications in young children. Such therapeutic errors cause significant carer anxiety and healthcare utilization. Assistance is needed from manufacturers and legislators in modifying formulation so that drugs can be safely used in young children. Education of prescribers and carers is also needed to reduce the incidence of such errors that cause significant carer anxiety and healthcare utilization. [\hyperlink{Amoxicillin And Omeprazole Magnesium And Rifabutin}{PMID: 28425352}, C Crawford et al., 2018]

\section*{Aspirin And Dipyridamole}
\subsection*{Result}
\subsubsection*{Answer}

Ages 1–23 years (mechanical heart valves): Yes  
Other pediatric populations/age ranges: Unknown  

\subsubsection*{{Explanation}}
\hypertarget{Aspirin And Dipyridamole}
To determine if Aspirin and Dipyridamole is safe for use in children, I reviewed all available abstracts for targeted studies on the safety of this combination in pediatric populations.

1. **Children with Mechanical Heart Valves (Ages 1–23 years):**
   - One study specifically evaluated the use of aspirin or aspirin with dipyridamole in 51 children (ages 1 to 23 years, mean 12.9 years) who received mechanical aortic valves. Six children received the combination therapy and were followed for an average of 36.5 months. The study found no deaths related to thrombosis or embolus, no postoperative thromboembolic events, and only minor hemorrhagic complications (nosebleeds, one upper GI hemorrhage). The authors concluded that children with mechanical aortic valves in normal sinus rhythm can be safely treated with aspirin or aspirin with dipyridamole, with little risk of thromboembolic events, valve thrombosis, or valve failure, and that hemorrhagic complications were minor and easily treated [\hyperlink{pmid_3784585}{PMID: 3784585}, E D Verrier et al., 1986].

2. **Children with Dengue Hemorrhagic Fever (Ages not specified, but pediatric):**
   - Another study evaluated 54 children with dengue hemorrhagic fever treated with a combination of dipyridamole and aspirin as an adjunct to standard therapy. The combination did not change mortality significantly but appeared to prevent progression to more severe disease. The abstract does not detail adverse events or safety outcomes, so it does not provide definitive evidence of safety or unsafety [\hyperlink{pmid_515801}{PMID: 515801}, L K Kho et al., 1979].

3. **Other Abstracts:**
   - The majority of other abstracts on aspirin and dipyridamole combination therapy focus on adult populations or do not specify pediatric data. Some abstracts discuss the use of aspirin or dipyridamole individually in children, but not the combination.
   - Several large meta-analyses and clinical trials of aspirin and dipyridamole for stroke prevention do not include pediatric patients or do not report pediatric-specific safety data.

**Summary by Age Range:**
- **Ages 1–23 years (mechanical heart valves):** There is targeted evidence supporting the safety of aspirin and dipyridamole in this group, with minor, manageable side effects.
- **Other pediatric populations/age ranges:** There is insufficient targeted evidence to affirm safety or unsafety of the combination for other indications or age groups in children.


\subsection*{Abstracts}
\hypertarget{pmid_6359862}{F}ever and pain are the most common issues in pediatric patient management. Acetaminophen, aspirin, and dipyrone are the most commonly used drugs and are equivalent in their efficacy. Dipyrone, used in many parts of the world, but not in the United States, is an effective agent; however, it has been implicated in producing agranulocytosis and anaphylactic shock. The salicylates have anti-inflammatory effects making them appropriate for the treatment of patients with juvenile rheumatoid arthritis, but they are gastric irritants, may impair clotting, and, because of saturable kinetics, may lead to accumulation and toxicity. Acetaminophen is an effective antipyretic and analgesic with few side effects that is toxic only in massive overdose. [\hyperlink{Aspirin And Dipyridamole}{PMID: 6359862}, E Gladtke et al., 1983]

\hypertarget{pmid_3784585}{T}he optimal method of anticoagulation in children with mechanical heart valves is controversial. Between 1975 and 1986, aspirin or aspirin with dipyridamole has been used for anticoagulation in children receiving a mechanical aortic valve at the University of California, San Francisco. Fifty-one patients (ages 1 to 23 years, mean 12.9 years) were treated with aspirin (n = 45) or aspirin with dipyridamole (n = 6) and observed a mean of 36.5 months (range 3 to 100 months). There were four late deaths: two from endocarditis and two from other medical problems, but none related to thrombosis or embolus. Follow-up was accomplished by direct contact with the patient, parent, or referring physician. Two patients (3.9\%) were lost to late follow-up. One minor neurologic event occurred perioperatively and resolved spontaneously. There were no postoperative thromboembolic events. Eleven asymptomatic children were recently studied by magnetic resonance imaging or computed axial tomography of the brain and had no evidence of prior silent cerebral thromboembolic defects. There were four patients (5.9\%) who had minor hemorrhagic complications: Three patients had nosebleeds and one patient had an upper gastrointestinal hemorrhage. Five patients were changed to warfarin anticoagulation: the patient with upper gastrointestinal hemorrhage and four older patients because of physician preference, all after uncomplicated aspirin therapy. There were no mechanical valve failures, although one patient required reoperation 9 months later for perivalvular leak. All children have remained in normal sinus or paced rhythm during follow-up. These results show that children with mechanical aortic valves in normal sinus rhythm can be safely treated with aspirin (or aspirin with dipyridamole) with little risk of thromboembolic events, valve thrombosis, or valve failure. Hemorrhagic complications resulting from aspirin are minor and easily treated. [\hyperlink{Aspirin And Dipyridamole}{PMID: 3784585}, E D Verrier et al., 1986]

\hypertarget{pmid_23871093}{S}troke is becoming a common disease worldwide, and has an increased rate of recurrence yearly after a transient ischemic attack (TIA) or stroke. Aspirin, dipyridamole, clopidogrel and aspirin plus dipyridamole combination therapy have been recommended for the secondary prevention of stroke in Americans. We performed meta-analyses to assess the effectiveness and safety of combination therapy with aspirin and dipyridamole (A+D) versus aspirin (A) alone in secondary prevention after transient ischemic attack (TIA) or stroke of presumed arterial origin within one week and six months. Medline, Embase, and the Cochrane Library. Eligible studies were completed randomized controlled trials investigating the effect of aspirin plus dipyridamole versus aspirin in patients with previous TIA or stroke. Five trials involving the use of aspirin and dipyridamole were included, 4318 allocated to A+D and 4304 to A alone. Meta-analysis of trials showed a significant protective effect of reducing or preventing recurrence of stroke (P=0.01), and ischemic event (P=0.003). The statistics showed no significant difference in vascular event, death from all cause and myocardial infarction (P>0.05). There were similarities with all bleeding events, major bleeding and intracranial hemorrhage was significant (P>0.05) between two groups. Aspirin plus dipyridamole combination therapy was beneficial in reducing the recurrence of stroke, and did not increase the bleeding event. Hence, aspirin plus dipyridamole combination therapy is effective and safe for the secondary prevention of stroke. [\hyperlink{Aspirin And Dipyridamole}{PMID: 23871093}, Xia Li et al., 2013]

\hypertarget{pmid_22364032}{A}cute respiratory infections are the second leading cause of morbidity in children under 18 years. Several drugs have been used with variable efficacy and safety, trying to reduce the associated symptoms and improve quality of life. To evaluate the efficacy and safety of buphenine, aminophenazone and diphenylpyraline hydrochloride when compared with placebo for the control of symptoms associated with common cold in children 6-24 months of age. Randomized clinical trial, double blind, placebo controlled, in 100 children < 24 months of any gender, with symptoms associated to common cold. They received the drug under study vs. placebo for seven days. Both groups received acetaminophen. The change on common cold related symptoms were evaluated. Statistic analysis was made with STATA 11.0 for Mac. Fifty-three children were randomized to study drug and forty-seven to placebo. Age of children in each group was similar (12.2 +/- 5.8 months vs. 12.7 +/- 5.8 months, p NS). There were significant differences between groups in relation to rhinorrea and sneezing resolution, with better results in Flumil group and no adverse events observed. The results in this study indicates that Flumil is a safe and effective drug for control of symptoms present in the common cold in children aged 6-24 months. [\hyperlink{Aspirin And Dipyridamole}{PMID: 22364032}, Ericka Montijo-Barrios et al., ]

\hypertarget{pmid_20884870}{T}he combination of low-dose aspirin and dipyridamole is more effective than aspirin alone in reducing the risk of recurrent stroke and other major cardiovascular events in patients with a recent transient ischemic attack or minor stroke. It is unknown whether this also applies to patients with a disabling stroke. We reanalyzed the data of 5700 patients from ESPRIT and ESPS-2 to study the effect of aspirin and dipyridamole according to modified Rankin scale (mRS) score at baseline. Primary outcome was vascular events (stroke, myocardial infarction, or vascular death). We used proportional hazards regression to estimate the treatment effect across mRS strata at baseline, and we tested for interactions with treatment. In total, 426 patients (7.5\%) had mRS score of 4 or 5 at baseline. The risk of an outcome event increased with mRS score. The relative risk associated with the combination of aspirin and dipyridamole compared to aspirin alone in patients with mRS score 0 to 5 was 0.79 (95\% confidence interval, 0.69-0.91). The relative risk according to mRS subcategory score 0 to 4 at baseline varied between 0.73 and 0.96 for vascular events and between 0.62 and 0.96 for stroke. The number of patients with mRS score 5 was too small for reliable estimates, but the data suggest a beneficial effect. There was no evidence of interaction between treatment effect and mRS score at baseline. The beneficial effect of the combination of low-dose aspirin and dipyridamole was present in all subcategories of the mRS score. [\hyperlink{Aspirin And Dipyridamole}{PMID: 20884870}, Diederik W J Dippel et al., 2010]

\hypertarget{pmid_7008732}{A}ntipyretics should be employed in the pediatric population whenever it is the clinical judgment of the attending physician that fever should be lowered. Aspirin and acetaminophen are equally effective as antipyretics. The efficacy and safety of these two most common antipyretic agents are examined, and various studies with these agents are critically reviewed. Since acetaminophen has a greater margin of safety at antipyretic dosages, it is preferred to aspirin when an anti-inflammatory effect is not required. The efficacy and safety of combination therapy with acetaminophen and aspirin in pediatric patients--or an alternative treatment regimen with both these drugs--warrant investigation. [\hyperlink{Aspirin And Dipyridamole}{PMID: 7008732}, S J Yaffe et al., 1981]

\hypertarget{pmid_23078168}{P}aracetamol (acetaminophen) and ibuprofen are the most frequently purchased over-the-counter (OTC) medicines for children. Parents purchase these medicines for the treatment of fever and pain. In some countries other NSAIDs such as aspirin (acetylsalicylic acid) and dipyrone are available. We aimed to perform a narrative review of the efficacy and toxicity of OTC analgesic medicines for children in order to give guidance to health professionals and parents regarding the treatment of pain in a child. Neither aspirin nor dipyrone are recommended for OTC use because of the association with Reye's syndrome for the former and the risk of agranulocytosis for the latter. Both paracetamol and ibuprofen are effective for the treatment of mild pain in children. Adverse effects with both medicines are infrequent. Ibuprofen is an NSAID and therefore there is a greater risk of gastrointestinal adverse effects and hypersensitivity. Aspirin and dipyrone should be avoided. Paracetamol is the drug of first choice for mild pain in children because of its favourable safety profile. For the treatment of significant musculoskeletal pain, ibuprofen is the drug of first choice. [\hyperlink{Aspirin And Dipyridamole}{PMID: 23078168}, Zeina Bárzaga Arencibia et al., 2012]

\hypertarget{pmid_29024184}{D}ipyrone has analgesic, spasmolytic, and antipyretic effects and is used to treat pain. Due to a possible risk of agranulocytosis with the use of dipyrone, it has been banned in a number of countries. The most commonly used data for the use of dipyrone are related to adults. Information relating to the use of dipyrone in children is scarce. Given the potential added value of dipyrone in the treatment of pain, a review of the literature was conducted to obtain more insight into the analgesic efficacy of dipyrone in children as well as the safety of dipyrone in terms of adverse events. A literature search was done for original articles (in English, German, or Spanish language) which met the following criteria: the use of dipyrone for pain and children up to the age of 17 years old. All titles and abstracts retrieved were reviewed, independently, by two of the authors, for their suitability for inclusion. The references of the selected articles were also checked for additional relevant papers. The publications were categorized into case reports, observational studies, or randomized controlled trials. To assess the methodological quality of the studies, the Jadad score was used. In the limited available data, the analgesic efficacy of intravenous dipyrone appears similar to that of intravenous paracetamol. Evidence is lacking to support the claim that dipyrone is equivalent or even superior to Non-Steroid-Anti-Inflammatory-Drugs in pediatric pain. While the absolute risk of agranulocytosis with dipyrone in children, based on available literature, cannot be determined, case reports suggest that this risk is not negligible. [\hyperlink{Aspirin And Dipyridamole}{PMID: 29024184}, Thomas G de Leeuw et al., 2017]

\hypertarget{pmid_1780077}{I}n order to assess the usefulness of a combination of low-dose aspirin (25 mg b.i.d.) with dipyridamole (200 mg b.i.d.) in the prevention of major coronary events in patients with acute unstable angina, we performed a prospective, double-blind, placebo-controlled study involving 88 consecutive patients admitted to three Hospital Departments of Cardiology. The patients entered the study as soon as possible after hospital admission, and were treated and followed up to one year. There was no appreciable difference in side effects and adverse reactions between the treatment and control group. The incidence of cardiac death and/or nonfatal myocardial infarction during the whole period of observation was 14\% (6/44) in the treatment group and 25\% (11/44) in the placebo group by "intention-to-treat" analysis; 16\% (4/25) and 32\% (10/31), respectively, by "drug-efficacy" analysis (p = 0.21 by Fisher's exact test, non significant difference). However, when considering the only events occurred in the first month (2/44 in the treatment group and 9/44 in the placebo group, amounting to 4.5 and 20 percent, respectively), the combination of dipyridamole with low-dose aspirin reached a statistically significant protective effect (p = 0.04). The results of this pilot study provide strong evidence for a beneficial effect of the regimen tested in patients with acute unstable angina, at least in the first weeks of treatment, while at the same time suggesting a safe alternative for patients with contraindications to higher doses of aspirin. [\hyperlink{Aspirin And Dipyridamole}{PMID: 1780077}, P Prandoni et al., ]

\hypertarget{pmid_21464191}{A}s many as 1 in every 110 children in the United States has an autism spectrum disorder (ASD). Many medical treatments for ASDs have been proposed and studied, but there is currently no consensus regarding which interventions are most effective. To systematically review evidence regarding medical treatments for children aged 12 years and younger with ASDs. We searched the Medline, PsycInfo, and ERIC (Education Resources Information Center) databases from 2000 to May 2010, regulatory data for approved medications, and reference lists of included articles. Two reviewers independently assessed each study against predetermined inclusion/exclusion criteria. Studies of secretin were not included in this review. Two reviewers independently extracted data regarding participant and intervention characteristics, assessment techniques, and outcomes and assigned overall quality and strength-of-evidence ratings on the basis of predetermined criteria. Evidence supports the benefit of risperidone and aripiprazole for challenging and repetitive behaviors in children with ASDs. Evidence also supports significant adverse effects of these medications. Insufficient strength of evidence is present to evaluate the benefits or adverse effects for any other medical treatments for ASDs, including serotonin-reuptake inhibitors and stimulant medications. Although many children with ASDs are currently treated with medical interventions, strikingly little evidence exists to support benefit for most treatments. Risperidone and aripiprazole have shown benefit for challenging and repetitive behaviors, but associated adverse effects limit their use to patients with severe impairment or risk of injury. [\hyperlink{Aspirin And Dipyridamole}{PMID: 21464191}, Melissa L McPheeters et al., 2011]

\hypertarget{pmid_23801256}{A}ripiprazole and risperidone are the only FDA approved medications for treating irritability in autistic disorder, however there are no head-to-head data comparing these agents. This is the first prospective randomized clinical trial comparing the safety and efficacy of these two medications in patients with autism spectrum disorders. Fifty nine children and adolescents with autism spectrum disorders were randomized to receive either aripiprazole or risperidone for 2 months. The primary outcome measure was change in Aberrant Behavior Checklist (ABC) scores. Adverse events were assessed. Aripiprazole as well as risperidone lowered ABC scores during 2 months. The rates of adverse effects were not significantly different between the two groups. The safety and efficacy of aripiprazole (mean dose 5.5 mg/day) and risperidone (mean dose 1.12 mg/day) were comparable. The choice between these two medications should be on the basis of clinical equipoise considering the patient's preference and clinical profile.  [\hyperlink{Aspirin And Dipyridamole}{PMID: 23801256}, Ahmad Ghanizadeh et al., 2014] Clinical studies in the treatment of 54 children suffering from DHF with a combination of dipyridamole and ASA as an adjuvant of our standard therapy consisted of fluid, electrolytes, blood, plasma and plasma expanders were evaluated. Heparin was administered in cases of DIC. It appeared that dipyridamole and ASA did not change the mortality significantly, but it prevented the progress of the severity of the disease from grade I and II to grade III and IV. [\hyperlink{Aspirin And Dipyridamole}{PMID: 23801256}, L K Kho et al., 1979]

\hypertarget{pmid_33485779}{T}onsillectomy is the 2nd most common outpatient surgery performed on children in the United States of America. Its main complication is pain, which varies in intensity from moderate to severe. Dipyrone is one of the most widely used painkillers in the postoperative period in children. Its use, however, is controversial in the literature, to the point that it is banned in many countries due to its potential severe adverse effects. Because of this controversy, reviewing the analgesic use of dipyrone in the postoperative period of tonsillectomy in children is essential. The aim of this study was to review the analgesic use of dipyrone in the postoperative period of tonsillectomy in children. Systematic review of the literature, involving an evaluation of the quality of articles in the databases MEDLINE/Pubmed, EMBASE and Virtual Health Library, selected with a preestablished search strategy. Only studies with a randomised clinical trial design evaluating the use of dipyrone in the postoperative period of tonsillectomy in children were included. Only 2 randomised clinical trials were found. Both compared dipyrone, paracetamol, and placebo. We were unable to carry out a metanalysis because the studies were too heterogenous (dipyrone was used as pre-emptive analgesic in one and only postoperatively in another). The analgesic effect of dipyrone, measured by validated pain scales in childhood, was shown to be superior to placebo and similar to paracetamol. It appears that dipyrone exhibits a profile suitable for use in children. However, the scarcity of randomised clinical trials evaluating its analgesic effect in this age group leads to the conclusion that more well-designed studies are still needed to establish the role of dipyrone in the postoperative period of tonsillectomy in children. [\hyperlink{Aspirin And Dipyridamole}{PMID: 33485779}, Maira Isis S Stangler et al., ]

\hypertarget{pmid_1429411}{T}he pharmacological management of anxiety in children primarily has used antidepressants, such as imipramine. Buspirone, an atypical anxiolytic, has been shown to be of benefit in both adults and children. It has relatively few side effects and is generally well tolerated. Two cases are reported here involving children treated for anxiety with buspirone who subsequently suffered a possible psychotic deterioration. [\hyperlink{Aspirin And Dipyridamole}{PMID: 1429411}, P Soni et al., 1992]

\hypertarget{pmid_24144215}{A} multimodal and preventative approach to providing postoperative analgesia is becoming increasingly popular for children and adults, with the aim of reducing reliance on opioids. We conducted a prospective, randomized double-blind study to compare the analgesic efficacy of intravenous paracetamol and dipyrone in the early postoperative period in school-age children undergoing lower abdominal surgery with spinal anesthesia. Sixty children scheduled for elective lower abdominal surgery under spinal anesthesia were randomized to receive either intravenous paracetamol 15 mg/kg, dipyrone 15 mg/kg or isotonic saline. The primary outcome measure was pain at rest, assessed by means of a visual analog scale 15 min, 30 min, 1 h, 2 h, 4 h and 6 h after surgery. If needed, pethidine 0.25 mg/kg was used as the rescue analgesic. Time to first administration of rescue analgesic, cumulative pethidine requirements, adverse effects and complications were also recorded. There were no significant differences in age, sex, weight, height or duration of surgery between the groups. Pain scores were significantly lower in the paracetamol group at 1 h (P = 0.030) and dipyrone group at 2 h (P = 0.010) when compared with placebo. The proportion of patients requiring rescue analgesia was significantly lower in the paracetamol and dipyrone groups than the placebo group (vs. paracetamol P = 0.037; vs. dipyrone P = 0.020). Time to first analgesic requirement appeared shorter in the placebo group but this difference was not statistically significant, nor were there significant differences in pethidine requirements, adverse effects or complications. After lower abdominal surgery conducted under spinal anesthesia in children, intravenous paracetamol appears to have similar analgesic properties to intravenous dipyrone, suggesting that it can be used as an alternative in the early postoperative period. Clinical Trials.gov. Identifier: NCT01858402. [\hyperlink{Aspirin And Dipyridamole}{PMID: 24144215}, Esra Caliskan et al., 2013]

\hypertarget{pmid_20410547}{T}he safety of fixed-dose combination aspirin-extended-release (ER) dipyridamole for stroke prevention in patients with ischemic heart disease is reviewed. Randomized controlled trials have established the superiority of aspirinER dipyridamole over aspirin alone for secondary stroke prevention. One limitation of this product is the potential risk of worsening angina in patients with coronary artery disease. The English-language medical literature was searched for articles describing the cardiac safety of oral dipyridamole alone or in combination with aspirin. Meta-analyses, randomized controlled trials, observational studies, and case reports presenting information on the cardiac safety of oral dipyridamole were also reviewed. Four meta-analyses described vascular events with dipyridamole using various dosing strategies. Three trials included the endpoint of myocardial infarction in patients receiving ER dipyridamole. The meta-analyses and randomized controlled trials specifically evaluating aspirin-ER dipyridamole did not provide evidence of increased risk of vascular events. One post hoc analysis of a randomized controlled trial specifically assessed the cardiac safety of fixed-dose aspirin-ER dipyridamole and found that dipyridamole was not associated with a higher number of cardiac events compared with aspirin alone. One randomized controlled trial evaluated the efficacy of ER dipyridamole in patients with preexisting ischemic heart disease and found no evidence of increased risk of cardiac events in this population. No published reports were located describing angina with the combination product. A literature review revealed that fixed-dose aspirin-ER dipyridamole was not associated with an increased risk of cardiovascular events in patients with ischemic heart disease. However, individual patient factors merit consideration when choosing an antiplatelet agent for stroke prevention. [\hyperlink{Aspirin And Dipyridamole}{PMID: 20410547}, Natalie Crown et al., 2010]

\hypertarget{pmid_18535024}{T}o study the effect of combination therapy with aspirin and dipyridamole (A+D) over aspirin alone (ASA) in secondary prevention after transient ischaemic attack (TIA) or minor stroke of presumed arterial origin and to perform subgroup analyses to identify patients that might benefit most from secondary prevention with A+D. The previously published meta-analysis of individual patient data was updated with data from ESPRIT (n = 2,739); trials without data on the comparison of A+D versus ASA were excluded. A meta-analysis was performed using Cox regression, including several subgroup analyses and following baseline risk stratification. A total of 7612 patients (five trials) were included in the analyses, 3800 allocated to A+D and 3812 to ASA alone. The trial-adjusted hazard ratio (HR) for the composite event of vascular death, non-fatal myocardial infarction and non-fatal stroke was 0.82 (95\% confidence interval (CI) 0.72 to 0.92). HRs did not differ in subgroup analyses based on age, sex, qualifying event, hypertension, diabetes, previous stroke, ischaemic heart disease, aspirin dose, type of vessel disease and dipyridamole formulation, nor across baseline risk strata as assessed with two different risk scores. A+D were also more effective than ASA alone in preventing recurrent stroke; HR 0.78 (95\% CI 0.68 to 0.90). The combination of aspirin and dipyridamole is more effective than aspirin alone in patients with TIA or ischaemic stroke of presumed arterial origin in the secondary prevention of stroke and other vascular events. This superiority was found in all subgroups and was independent of baseline risk. [\hyperlink{Aspirin And Dipyridamole}{PMID: 18535024}, P H A Halkes et al., 2008]

\hypertarget{pmid_33921933}{R}isperidone and aripiprazole are approved by the USA Food and Drug Administration for the treatment of irritability and aggression in children from the ages of 5 and 6 years, respectively. However, there are no approved medications for the treatment of autism spectrum disorder (ASD) core signs and symptoms. Nevertheless, early intervention is recognized as key to improving long-term outcomes. This retrospective case study included 10 children (mean age, 2 years 10 months) with ASD who presented with persistent irritability and aggression before 4 years of age that was unresponsive to behavioral interventions and sufficiently severe to consider pharmacological intervention with risperidone or aripiprazole combined with standard supportive therapies. Besides ameliorating comorbid behaviors, improvement was observed in ASD core signs and symptoms for all patients, with minimal-to-no symptoms observed in 60\% of patients according to the Childhood Autism Rating Scale 2-Standard Test and Clinical Global Impression scales. Excessive weight gain in two patients was the only adverse effect observed that required intervention. This is the first study to suggest that ASD can potentially be treated in very young children (<4 years). Clinical trials are urgently required to validate these findings among this pediatric population. [\hyperlink{Aspirin And Dipyridamole}{PMID: 33921933}, Hamza A Alsayouf et al., 2021]

\hypertarget{pmid_33235453}{A}utism spectrum disorder (ASD) is a debilitating neurodevelopmental disorder with high heterogeneity and no clear common cause. Several drugs, in particular risperidone and aripiprazole, are used to treat comorbid challenging behaviors in children with ASD. Treatment with risperidone and aripiprazole is currently recommended by the Food and Drug Administration (FDA) in the USA for children aged 5 and 6 years and older, respectively. Here, we investigated the use of these medications in younger patients aged 4 years and older. This retrospective case series included 18 children (mean age, 5.7 years) with ASD treated at the Kids Neuro Clinic and Rehab Center in Dubai. These patients began treatment with risperidone or aripiprazole at the age of 4 years and older, and all patients presented with comorbid challenging behaviors that warranted pharmacological intervention with either risperidone or aripiprazole. All 18 children showed objective improvement in their ASD core signs and symptoms. Significant improvement was observed in 44\% of the cases, and complete resolution (minimal-to-no-symptoms) was observed in 56\% of the cases as per the Childhood Autism Rating Scale 2-Standard Test (CARS2-ST) and the Clinical Global Impression (CGI) scales. Our findings indicate that the chronic administration of antipsychotic medications with or without ADHD medications is well tolerated and efficacious in the treatment of ASD core and comorbid symptoms in younger children when combined with standard supportive therapies. This is the first report to suggest a treatment approach that may completely resolve the core signs and symptoms of ASD. While the reported outcomes indicate significant improvement to complete resolution of ASD, pharmacological intervention should continue to be considered as part of a multi-component intervention in combination with standard supportive therapies. Furthermore, the findings support the critical need for double-blind, placebo-controlled studies to validate the outcomes. [\hyperlink{Aspirin And Dipyridamole}{PMID: 33235453}, Hamza A Alsayouf et al., 2020]

\hypertarget{pmid_10493274}{T}he fixed-dose combination of extended-release dipyridamole/aspirin (Aggrenox/Asasantin Retard) combines 2 antiplatelet agents with different mechanisms of action. The combination reduced thrombus formation in human and animal models. Coadministration of extended-release dipyridamole and aspirin in healthy volunteers had no significant effects on the plasma concentrations of either agent. Twice-daily oral extended-release dipyridamole/aspirin (400/50 mg/day) was twice as effective as either agent alone in the secondary prevention of stroke in a large clinical trial involving patients with prior stroke or transient ischaemic attack. The rate of the combined end-point of stroke and death tended to be lower with the combination than with other treatments. The incidence of death was not significantly reduced by any treatment. Most adverse events with extended-release dipyridamole/aspirin were mild and similar to those with either agent alone. Bleeding was more common with the combination than with extended-release dipyridamole alone, as was headache when compared with aspirin alone. Limited pharmacoeconomic analyses suggest that treatment with extended-release dipyridamole/aspirin was cost saving and was cost effective compared with aspirin monotherapy for the secondary prevention of stroke. [\hyperlink{Aspirin And Dipyridamole}{PMID: 10493274}, P S Hervey et al., 1999]

\hypertarget{pmid_9781830}{P}atients who had survived a stroke or transient ischaemic attacks (TIA) were admitted to a trial of low-dose aspirin (50 mg) alone, sustained release dipyridamole (400 mg/day) alone, or a combination of the two agents, and results compared with a placebo over 24 months. This low-dose aspirin regimen produced in pairwise comparisons a significant risk reduction of 18\% for stroke, 13\% for stroke and/or death but no reduction in all cause mortality. The sustained release dipyridamole produced a significant risk reduction of 16\% for stroke, 15\% for stroke and/or death but no significant reduction of mortality. In combination, aspirin and dipyridamole produced a risk reduction of 37\% in stroke, 24\% in stroke and/or death, and no reduction in mortality. Similar findings were found in TIA, which was a secondary endpoint. These results are highly significant in comparison with placebo. As expected, there were enhanced reports of alimentary side-effects in the aspirin groups and also enhanced bleeding. Dipyridamole was associated with a slight increase in headache, which resolved in most patients if therapy was continued. The conclusions are that 50 mg/day of aspirin alone or 400 mg/day of sustained release dipyridamole alone are equally effective in stroke and TIA prevention. When used in combination the effects were additive and were significantly more effective than the single agents. [\hyperlink{Aspirin And Dipyridamole}{PMID: 9781830}, C D Forbes et al., 1998]

\hypertarget{pmid_367358}{A} review is given on the clinical studies performed with aspirin in patients with chronic vascular occlusions of the limbs and on studies in cerebral ischemia using aspirin and sulfinpyrazone. Aspirin reduces the risk of reocclusions in patients after vascular surgery and also reduces the risk of peripheral vascular occlusions in diabetic patients. In doses of 1.2-1.5 g/day it also reduces the frequency of transient ischemic attacks. Conclusive results of similar studies with sulfinpyrazone and dipyridamole can be expected of the ongoing studies. Aspirin has no effect on the course of glomerulonephritis in children. Warfarin plus dipyridamole seem to have some effect in patients renal allografts. Sulfinpyrazone and ASA reduced the incidence of shunt thromboses in hemodialyzed patients. Several case reports in patients with thrombocytemia or Raynaud's syndrome made it likely that treatment with antiplatelet drug reduces the incidence of vascular occlusions. [\hyperlink{Aspirin And Dipyridamole}{PMID: 367358}, K Breddin et al., 1977]

\hypertarget{pmid_34430426}{M}igraine is the most common primary headache among children and adolescents. The aim of this meta-analysis was to evaluate the efficacy and safety of antiepileptic drugs in the prevention of pediatric migraine. PubMed, Cochrane Library, EMBASE and Chinese National Knowledge Infrastructure (CNKI) databases were searched for eligible published RCTs from January 1970 to June 2020. Migraine frequency and ≥50\% response rate were measured as the efficacy outcomes. We used "Risk of Bias" tool for quality assessment and RevMan5.3 software for statistical analysis. Four articles containing 7 RCTs with 794 participants compared the efficacy of AEDs with placebo. Four RCTs assessed topiramate  Topiramate can reduce monthly headache days in children and adolescents with migraine. However, it failed to improve the ≥50\% response rate. DVPX ER showed no difference from placebo in the prophylactic treatment pediatric migraine. Side effects seemed to be more frequent in topiramate and DVPX ER treated children but generally well-tolerated. [\hyperlink{Aspirin And Dipyridamole}{PMID: 34430426}, Guoyong Jia et al., 2021]

\hypertarget{pmid_23503913}{T}onsillectomy is associated with severe postoperative pain for which, several drugs are employed for management. In this double-blind, placebo-controlled study we aimed to evaluate the efficacy of intravenous paracetamol and dipyrone when used for post-tonsillectomy analgesia in children. 120 children aged 3-6 yr, undergoing tonsillectomy with or without adenoidectomy and/or ventilation tube insertion were randomized to receive intraoperative infusions of paracetamol (15 mg/kg), dipyrone (15 mg/kg) or placebo (0.9\% NaCl). Evaluation was carried out at 0.25, 0.50, 1, 2, 4, 6h postoperatively. Pethidine 0.25 mg/kg was utilized as rescue analgesic. Cumulative pethidine requirement was the primary outcome. Pain intensity measurement, pain relief, sedation level, nausea and vomiting, postoperative bleeding and any other adverse effects were noted. No significant difference was found in pethidine requirement between paracetamol and dipyrone groups. Cumulative pethidine requirement was significantly less in paracetamol and dipyrone groups vs. placebo. No significant difference was observed between groups in postoperative pain intensity scores throughout the study. Intravenous paracetamol is found to have a similar analgesic efficacy as intravenous dipyrone and they both help to reduce the opioid requirement for postoperative analgesia in pediatric day-case tonsillectomy. [\hyperlink{Aspirin And Dipyridamole}{PMID: 23503913}, Aysu Inan Kocum et al., ]

\hypertarget{pmid_27144151}{A}lthough pharmacotherapy with atypical antipsychotics is common in child psychiatry, there has been little research on this issue. To compare the efficacy and safety of risperidone and aripiprazole in the treatment of preschool children with disruptive behavior disorders comorbid with attention deficit-hyperactivity disorder (ADHD). Randomized clinical trial conducted in a university-affiliated child psychiatry clinic in southwest Iran. Forty 3-6-year-old children, diagnosed with oppositional defiant disorder comorbid with ADHD, were randomized to an 8-week trial of treatment with risperidone or aripiprazole (20 patients in each group). Assessment was performed by Conners' rating scale-revised and clinical global impressions scale, before treatment, and at weeks 2, 4, and 8 of treatment. The data were analyzed by SPSS version 16. Mean scores between the two groups were compared by analysis of variance and independent and paired t-test. Mean scores of Conners rating scales were not different between two groups in any steps of evaluation. Both groups had significantly reduced scores in week 2 of treatment (P = 0.00), with no significant change in subsequent measurements. Rates of improvement, mean increase in weight (P = 0.894), and mean change in fasting blood sugar (P = 0.671) were not significantly different between two groups. Mean serum prolactin showed a significant increase in risperidone group (P = 0.00). Both risperidone and aripiprazole were equally effective in reducing symptoms of ADHD and oppositional defiant disorder, and relatively safe, but high rates of side effects suggest the cautious use of these drugs in children.  [\hyperlink{Aspirin And Dipyridamole}{PMID: 27144151}, Parvin Safavi et al., ]

\section*{Venlafaxine Hydrochloride}
\subsection*{Result}
\subsubsection*{Answer}

Ages 6–17: Yes  
Ages under 6: Unknown  

\subsubsection*{{Explanation}}
\hypertarget{Venlafaxine Hydrochloride}
A review of the available abstracts reveals several targeted studies evaluating the safety of Venlafaxine Hydrochloride in children and adolescents, primarily for psychiatric indications such as ADHD and generalized anxiety disorder. Below is a summary of the relevant evidence, organized by age range:

Ages 6–17 years:
- Two randomized, double-blind, placebo-controlled trials (N=320) specifically evaluated extended-release venlafaxine in children and adolescents (ages 6–17) with generalized anxiety disorder. The studies found that venlafaxine was effective and generally well-tolerated for short-term use. Common adverse events included asthenia, anorexia, pain, and somnolence. Statistically significant changes in height, weight, blood pressure, pulse, and cholesterol were observed, but the studies concluded that venlafaxine may be an effective, well-tolerated short-term treatment for pediatric generalized anxiety disorder [\hyperlink{pmid_17267793}{PMID: 17267793}, Moira A Rynn et al., 2007].

- Multiple open-label and controlled trials (sample sizes ranging from 13 to 38) in children and adolescents with ADHD (ages 5–17) reported that venlafaxine was relatively well-tolerated for short-term use, with side effects such as somnolence, stomachache, headache, and, in some cases, worsening of hyperactivity. Some studies noted that adverse effects led to discontinuation in a minority of patients, but no severe or life-threatening events were reported [\hyperlink{pmid_14678464}{PMID: 14678464}, Nahit Motavalli Mukaddes et al., 2004; \hyperlink{pmid_9231317}{PMID: 9231317}, R L Olvera et al., 1996; \hyperlink{pmid_17822339}{PMID: 17822339}, Robert L Findling et al., 2007; \hyperlink{pmid_20860068}{PMID: 20860068}, Ali-Reza Zarinara et al., 2010; \hyperlink{pmid_23157376}{PMID: 23157376}, Ahmad Ghanizadeh et al., 2013; \hyperlink{pmid_24259607}{PMID: 24259607}, Pauline Park et al., 2014].

- A double-blind, placebo-controlled trial in children and adolescents (ages 8–17) with major depression found no significant efficacy for venlafaxine over placebo, but reported a low side-effect profile and no serious safety concerns during the 6-week study [\hyperlink{pmid_9133767}{PMID: 9133767}, M W Mandoki et al., 1997].

Ages under 6 years:
- There is a case report of a 3-year-old successfully treated with venlafaxine for cataplexy, followed for over 2 years, but this is a single case and not a controlled safety study [\hyperlink{pmid_24340297}{PMID: 24340297}, Michelle Ratkiewicz et al., 2013].

- No controlled or systematic safety studies in children under 6 years were identified in the abstracts.

General pediatric population (accidental ingestion/toxicity):
- A retrospective review of 262 cases of venlafaxine ingestion in children (ages 0–20) found that most cases resulted in no or minor effects, with moderate or severe outcomes in only 12 cases. Seizures were rare and occurred at high doses. This suggests that accidental ingestion is generally not associated with severe toxicity, but this is not a study of therapeutic use [\hyperlink{pmid_26351291}{PMID: 26351291}, S Doroudgar et al., 2016].

Summary:
- For children and adolescents ages 6–17, there is evidence from controlled trials that short-term use of venlafaxine (including extended-release) is generally safe and well-tolerated, though monitoring for changes in growth parameters and vital signs is warranted.
- For children under 6 years, there is insufficient evidence from controlled studies to determine safety.
- Long-term safety in any pediatric age group is not established, and several reviews call for more robust, long-term studies before routine use is recommended [\hyperlink{pmid_23157376}{PMID: 23157376}, Ahmad Ghanizadeh et al., 2013; \hyperlink{pmid_24259607}{PMID: 24259607}, Pauline Park et al., 2014].

\subsection*{Abstracts}
\hypertarget{pmid_26351291}{V}enlafaxine is commonly used in the United States for approved and non-Food and Drug Administration-approved indications in adults. It is used off-label to treat children for psychiatric diagnoses. The aim of the study was to describe venlafaxine toxicities in children and to identify the venlafaxine dose per weight that correlates with toxicities. An 11-year retrospective study of venlafaxine ingestion in children was performed using the California Poison Control System (CPCS) database. Data was extracted from phone calls received by CPCS clinicians and follow-up phone calls made to assess the patient's progress in a health-care setting. Inclusion criteria were venlafaxine ingestion cases reported to CPCS between January 2001 and December 2011, children aged 20 years and under, venlafaxine as the only ingested substance, managed in a health-care facility, and followed to a known outcome. Two hundred sixty-two cases met the study criteria. Common presentations included gastrointestinal (14.9\%), altered mental status (13.7\%), and tachycardia (13.4\%). The majority of the cases resulted in no effect (51.5\%) or minor effect (19.9\%). The average estimated dose per weight was 18.3 mg/kg in all patients and 64.5 mg/kg in those experiencing moderate-to-severe adverse effects. Seizures occurred in only 4 of the 262 cases at doses ranging from 1500 to 7500 mg. Although the estimated dose per weight exceeded 10 mg/kg for the majority of the cases, only 12 cases resulted in moderate or severe outcomes. The majority of venlafaxine ingestion cases in children resulted in either no clinical effects or minor clinical effects.  [\hyperlink{Venlafaxine Hydrochloride}{PMID: 26351291}, S Doroudgar et al., 2016] The tolerability and safety of venlafaxine hydrochloride, a new serotonin and norepinephrine reuptake inhibitor, are reviewed in this article. The data presented here are based on a pool of 3,082 patients who were treated with this agent during clinical trials. Of these patients, 2,897 received venlafaxine for depression; 455 of these patients were treated for more than 360 days. The tolerability and safety profiles of venlafaxine were similar to those previously reported for selective serotonin reuptake inhibitors. Patients receiving venlafaxine experienced nausea, insomnia, dizziness, somnolence, constipation, and sweating more often than did patients receiving placebo but reported anticholinergic events less frequently than did patients receiving tricyclics. This is accounted for by the fact that, unlike the tricyclics, venlafaxine lacks significant affinity for muscarinic cholinergic receptors. Resolution of venlafaxine-associated nausea occurred rapidly in the vast majority of the patients who reported it at the start of therapy. Serious adverse events were rare among venlafaxine-treated patients. A small percentage of the patients given venlafaxine experienced modest but significant increases in blood-pressure readings, similar to those observed among imipramine-treated patients. At mean daily venlafaxine dosages of up to 300 mg, the percentage of venlafaxine-treated patients who had sustained elevations in supine diastolic blood pressure during treatment ranged from 2\% to 6\%, compared with 2\% and 5\% among the placebo- and imipramine-treated patients, respectively. All of the 14 patients who took an overdose of venlafaxine recovered without sequelae. Tolerability and safety in the elderly did not differ significantly from that observed in younger patients. [\hyperlink{Venlafaxine Hydrochloride}{PMID: 26351291}, R L Rudolph et al., 1996]

\hypertarget{pmid_14678464}{T}he primary purpose of this study was to describe tolerability and efficacy of venlafaxine in the treatment of children and adolescents with attention deficit hyperactivity disorder (ADHD). A 6-week open trial of venlafaxine was conducted in 13 children and adolescents (mean age 9.9 +/- 2.5 years) with ADHD, and without comorbid depression. Venlafaxine was initiated at a dose of 18.75 mg/day and flexibly titrated to 56.25 mg/day. The Conners parent scale and Clinical Global Improvement (CGI) severity item were performed at baseline and at the end of the 6-week trial. All subjects completed the trial. Mean final dose of venlafaxine was 40.3 +/- 7.0. Venlafaxine was significantly effective in reducing the total score of the Conners parent scale from baseline to endpoint (P < 0.002, Z =-3.113) and the CGI severity item (P < 0.05). Transient side-effects such as somnolence (n = 2), stomachache (n = 2), and headache (n = 1) disappeared after second week of treatment. Also three subjects complained of sedation after raising the dose to 56.5 mg/day, therefore the dose was reduced to the previous level. These preliminary data suggest that venlafaxine may be an effective medication in the treatment of some children and adolescents with ADHD. Future double-blind controlled trials should be undertaken. [\hyperlink{Venlafaxine Hydrochloride}{PMID: 14678464}, Nahit Motavalli Mukaddes et al., 2004]

\hypertarget{pmid_23157376}{A}ttention deficit hyperactivity disorder (ADHD) is a common psychiatric disorder in children and adolescents. Stimulants are commonly prescribed for ADHD management. There is clinical trial evidence that some medications with noradrenergic properties such as atomoxetine are effective. It is of theoretical and practical importance if other agents with noradrenergic properties display a comparable pattern of efficacy. This paper is a systematic review of the efficacy and safety of venlafaxine for treating children and adolescents with ADHD. MEDLINE, Google scholar, Scopus, and Web of science (ISI) databases were electronically searched in July 2012, updated on November 2012. Time and language of publication were not exclusion criteria. Efficacy outcomes were assessed by a valid and reliable parent- and/or teacher-reported instrument to evaluate clinical symptoms. Adverse effects were also evaluated. There were three uncontrolled trials and only two double blind controlled clinical trials. Venlafaxine appeared effective for treating ADHD. The rates of some adverse effects of venlafaxine were less than those documented for methylphenidate. While one of the two small controlled trials did not find difference between venlafaxine ad methylphenidate, the other trial reported lower efficacy for venlafaxine. Headache, insomnia, and nausea were among the most common adverse effects. This systematic review provides preliminary support that venlafaxine may have short term utility in treating ADHD in children and adolescents. However, before recommending venlafaxine for treatment, more robust and larger clinical trials, in particular providing evidence of its long-term efficacy, safety and tolerability are required. [\hyperlink{Venlafaxine Hydrochloride}{PMID: 23157376}, Ahmad Ghanizadeh et al., 2013]

\hypertarget{pmid_9231317}{A} 5-week open trial of venlafaxine was conducted in 16 children and adolescents (mean age 11.6 years) with attention-deficit/hyperactivity disorder (ADHD) in order to estimate the appropriate dosage range and to determine the extent of side effects. Subjects were evaluated using a structured clinical interview and a computerized diagnostic assessment, and subjects diagnosed with ADHD and without comorbid depression were asked to enter the study. Conners Parent Rating Scale (CPRS) and Conners Continuous Performance Test (CPT) were performed at baseline and at the end of the 5-week trial. Two subjects were lost to follow-up. Of the remaining 14 patients, 7 subjects displayed a decrease of at least one standard deviation from their baseline on one of the CPRS subscale scores and had subjective reports from parents of improved behavior. There were no statistically significant effects of venlafaxine on reaction times or on the number of commission and omission errors on CPT. Three ADHD subjects displayed a worsening of their hyperactivity and required discontinuation of venlafaxine, and nausea led to drug discontinuation in 1 patient. The mean daily dose of venlafaxine was 60 mg (1.4 mg/kg), administered 2-3 divided doses, there were no effects on blood pressure or heart rate. In this sample, low doses of venlafaxine appeared to be effective in reducing behavioral but not cognitive symptoms of ADHD in 7 of 16 children and adolescents (44\%), and adverse effects were not tolerable in 4 of 16 patients (25\%). These preliminary results suggest that venlafaxine may aggravate symptoms of hyperactivity, consistent with the behavioral activation reported with fluoxetine and sertraline in children. [\hyperlink{Venlafaxine Hydrochloride}{PMID: 9231317}, R L Olvera et al., 1996]

\hypertarget{pmid_9133767}{M}ajor depression is commonly found in the child and adolescent population. Venlafaxine, a new antidepressant, has been used successfully in adults; however, its use in children and adolescents has been very limited. This study evaluated the efficacy and side effect profile of venlafaxine in the treatment of depression in children and adolescents. In a double-blind, placebo-controlled, 6-week study, 33 subjects between the ages of 8 and 17, who met DSM-IV criteria for major depression, were treated with either venlafaxine and therapy or placebo and therapy. Patient progress data were obtained by weekly rating assessments. Data on side effects were also obtained weekly. The statistical analysis indicated a significant improvement over time, but it could not be attributed to venlafaxine drug therapy. These findings are consistent with other studies where the efficacy of antidepressants in the treatment of major depression in this age population remains unproven. Low dosage and short length of treatment may account for the lack of efficacy. The findings did, however, suggest a low side-effect profile. Further studies are recommended to assess efficacy and to corroborate its safety in children and adolescents. [\hyperlink{Venlafaxine Hydrochloride}{PMID: 9133767}, M W Mandoki et al., 1997]

\hypertarget{pmid_28590988}{V}ilazodone hydrochloride is the first member in a new class of antidepressants called indolealkylamines and was approved for use in the United States in 2011 for major depressive disorder. It has a combined mechanism of action of a selective serotonin reuptake inhibitor and a partial agonist of serotonin 5-HT1A receptors. It has not been approved for use in the pediatric population, and toxicity from exploratory vilazodone ingestion has been rarely described to date. We describe 2 children with laboratory-confirmed vilazodone ingestions that led to significant toxicity including refractory status epilepticus in 1 patient and likely transient seizure activity in the other. Both patients required multiple doses of benzodiazepines; in the more severe case, barbiturates were added to control seizure activity. These children returned to baseline and had no prolonged neurologic complications. Pediatric experience with vilazodone is limited; however, the literature demonstrates 3 additional case reports of children experiencing seizure after vilazodone ingestion. With the 2 new cases presented here, it seems prudent to educate prescribers and families of the potential dangers of ingestion of vilazodone tablets by young children. [\hyperlink{Venlafaxine Hydrochloride}{PMID: 28590988}, Jeannine Del Pizzo et al., 2018]

\hypertarget{pmid_17822339}{T}he objectives of this pilot study were to explore the changes in symptom severity, tolerability, and the pharmacodynamics of venlafaxine treatment in youths with attention-deficit/hyperactivity disorder (ADHD). This was a 2-week, open-label, outpatient trial of venlafaxine in children and adolescents, ages 5-17 years, with ADHD. Three dosing strata, 0.5, 1.0, and 2.0 mg/kg per day, were examined. ADHD symptom severity and improvement assessments included the ADHD Rating Scale (ARS-IV) and the Clinical Global Impressions Scale (CGI). During this study, venlafaxine, O-desmethylvenlafaxine (ODV), norepinephrine, and serotonin concentrations were obtained. Thirty-eight participants (33 males) were treated in this trial. Overall, parent-completed and teacher-completed ARS-IV total scores showed a statistically significant positive change at the end of the study when compared to baseline (p < 0.05). Significant increases in plasma venlafaxine concentrations were observed at day 15 when compared to day 8 (p = 0.04). In addition, plasma norepinephrine and serotonin concentrations were found to be significantly decreased from baseline at end of study (p < 0.05). Four patients ended participation in the study prematurely: lost to follow up (n = 2), withdrawal of consent (n = 1), and worsening of ADHD symptoms after 8 days of treatment (n = 1). There were no discontinuations due to other adverse events. Venlafaxine appeared to offer some benefit and appears to be relatively safe for the short-term treatment of ADHD in this open-label trial. The pharmacodynamics of venlafaxine in youths are consistent with serotonergic and neuradrenergic modulation. [\hyperlink{Venlafaxine Hydrochloride}{PMID: 17822339}, Robert L Findling et al., 2007]

\hypertarget{pmid_19855313}{V}enlafaxine (VEN) is a second generation antidepressant drug, belonging to the class of selective serotonine and norepinephrine reuptake inhibitors, widely used in the treatment of depression and anxiety disorders. Though its pharmacological profile is considered safe, treatment with VEN can cause several nervous, gastrointestinal, cardiovascular and genitourinary adverse effects. Therapeutic drug monitoring for VEN could be useful in specific situations, including exposure to the drug during pregnancy. A liquid chromatography-high-resolution mass spectrometry method was developed and validated for the assay of VEN in 2.5-mg hair samples from 2 newborn identical twin sisters. The analyte was extracted by a rapid, simultaneous pulverization and extraction step, allowing analysis when tiny amounts of hair are available, such as in the case of a newborn. Gradient elution on an Atlantis T3 column was performed using nordiazepam-d5 as an internal standard. Positive ion electrospray ionization and high-resolution full scan determination were performed in an Orbitrap mass spectrometer. The method was linear range in the range 0.2-25 ng/mg, and had a quantification limit of 0.2 ng/mg, a relative standard deviation in the range 0.7\%-1.4\% (intra-assay) and 2.9\%-5.9\% (interassay), and was accurate (as \% relative error) in the range -9\% to + 2\%, using a hair sample size as low as 2.5 mg. The utilization of high-resolution/high accuracy mass spectrometry in full-scan mode allowed both the quantitative determination of VEN in the hair of the 2 newborns and the straightforward identification of 4 VEN metabolites, namely O-desmethylvenlafaxine, N-desmethylvenlafaxine, N,N-didesmethylvenlafaxine, and N,O-didesmethylvenlafaxine, by means of retrospective screening, thus unequivocally documenting in utero exposure. [\hyperlink{Venlafaxine Hydrochloride}{PMID: 19855313}, Donata Favretto et al., 2010]

\hypertarget{pmid_24259607}{T}o review the current literature on the efficacy and safety of serotonin norepinephrine reuptake inhibitors in the treatment of attention-deficit hyperactivity disorder (ADHD) in the pediatric population. A literature search from 1996 to August 2013 was conducted using MEDLINE, CINAHL, and EMBASE databases. Search terms included attention-deficit hyperactivity disorder, serotonin norepinephrine reuptake inhibitor, pediatric attention-deficit hyperactivity disorder, venlafaxine, duloxetine, desvenlafaxine, milnacipran, and nefazodone. Relevant articles on duloxetine and venlafaxine for the treatment of pediatric ADHD were reviewed; 5 studies on venlafaxine and 1 study on duloxetine were evaluated. Studies included open-label and randomized, double-blind trials. Case studies in pediatric populations and all studies in adult populations were excluded. Patients 6 to 17 years old were evaluated in the venlafaxine and duloxetine studies. Trials on venlafaxine, ranging from 2 to 6 weeks, showed patient improvement as measured by the Conners Rating Scale and ADHD Rating Scale. Venlafaxine was initiated at 12.5 to 25 mg/d and titrated up to 1.4 to 3.8 mg/kg/d to a maximum of 150 mg/d. Duloxetine showed minimal efficacy in treating ADHD symptoms at doses of 60 mg/d at 6 weeks. The most common side effects for venlafaxine and duloxetine included drowsiness and decreased appetite, respectively. Data for venlafaxine and duloxetine are limited. However, venlafaxine may be considered as an alternative agent when patients cannot tolerate or fail stimulants, tricyclic antidepressants, or bupropion. Duloxetine has been studied in children; however, with only 1 study available, it is difficult to recommend. [\hyperlink{Venlafaxine Hydrochloride}{PMID: 24259607}, Pauline Park et al., 2014]

\hypertarget{pmid_7629904}{V}enlafaxine hydrochloride is a novel bicyclic antidepressant which inhibits the reuptake of serotonin, norepinephrine and, to a lesser extent, dopamine. A 41-year-old female ingested 4.5 g venlafaxine, 500 mg diphenhydramine, 50 mg thiothixene and subsequently experienced severe central nervous system depression requiring intubation. She also developed elevated systolic and diastolic blood pressures and sinus tachycardia. The patient was decontaminated with gastric lavage and activated charcoal. She regained consciousness within a few hours and was extubated nine hours after ingestion. This case demonstrates that severe central nervous system depression may follow venlafaxine overdose. [\hyperlink{Venlafaxine Hydrochloride}{PMID: 7629904}, A Fantaskey et al., 1995]

\hypertarget{pmid_24340297}{N}arcolepsy with cataplexy is rare in children under 5 years of age. There is limited information on safe and effective treatment of cataplexy in young children. We describe successful treatment of cataplexy in a 3-year-old using venlafaxine and subsequently followed for over 2 years. [\hyperlink{Venlafaxine Hydrochloride}{PMID: 24340297}, Michelle Ratkiewicz et al., 2013]

\hypertarget{pmid_20860068}{T}he present report aimed to investigate the efficacy and tolerability of venlafaxine compared to methylphenidate in children and adolescents with Attention Deficit/Hyperactivity Disorder (ADHD). This was a 6-week, parallel group, randomized clinical trial. Thirty-eight patients (27 boys and 11 girls) with a DSM-IV-TR diagnosis of ADHD were the study population of this trial. All study subjects were randomly assigned to receive treatment using capsules of venlafaxine at doses of 50-75 mg/day depending on weight (50 mg/day for <30 kg and 75 mg/day for >30 kg (group 1) or methylphenidate at a dose of 20-30 mg/day depending on weight (group 2) for a 6-week double blind, randomized clinical trial. The principal measure of outcome was the Teacher and Parent Attention Deficit/Hyperactivity Disorder Rating Scale-IV. No significant differences were observed between the two groups on the Parent and Teacher Rating Scale scores (df = 1; F = 1.77; p = 0.19 and df = 1; F = 1.64; p = 0.20, respectively). Side effects of headaches and insomnia were observed more frequently in the methylphenidate group. The results suggest that venlafaxine may be useful for the treatment of ADHD. In addition, a tolerable side-effect profile is one of the advantages of venlafaxine in the treatment of ADHD. [\hyperlink{Venlafaxine Hydrochloride}{PMID: 20860068}, Ali-Reza Zarinara et al., 2010]

\hypertarget{pmid_25450524}{V}enlafaxine (VEN), a serotonin and noradrenaline reuptake inhibitor is being used as a drug of choice for treating clinical depression even during pregnancy. It is an important therapeutic option in the treatment of perinatal depression, but the effects of VEN on fetus and the newborn are uncertain. Therefore, present study was undertaken to investigate the safety of in-utero exposure to VEN in terms of developmental neurotoxicity and neurodegenerative potential by using prenatal rat model. The selected doses of VEN (25, 40 and 50mg/kg) were administered to pregnant rats from GD 5 to 19 through oral gavage. The fetal brains were dissected and processed for histopathological measurements of neocortical thickness that showed significant reduction. Considering vulnerability of immature brain to free radical injury, VEN exposed neocortices were tested for reactive oxygen species (ROS) levels which were significantly increased. As ROS play important role in the initiation of apoptotic mechanisms, we explored for in situ detection of apoptosis by confocal microscopy that showed enhanced apoptosis including chromatin condensation which was further reconfirmed by electron microscopy. Substantially increased levels of pro-apoptotic protein Bax and decreased levels of anti-apoptotic protein Bcl2 as shown by western blotting also supported the increased neuro-apoptotic degeneration. For further correlation of these findings, prenatally VEN exposed young-adult rat offspring were assessed for open field exploratory behavior that showed increased anxiety-like and stereotypic responses indicating disturbed neurobehavioral pattern. The study concludes that prenatal VEN exposure may primarily enhance ROS generation that plays a key role in regulating release of proapoptotic factors from mitochondria and thereby enhancing apoptotic neurodegeneration that affect proliferation, migration and differentiation of cells, resulting in neuronal deficits manifested as long term neurobehavioral impairments.  [\hyperlink{Venlafaxine Hydrochloride}{PMID: 25450524}, Manish Singh et al., 2015] Venlafaxine hydrochloride, a structurally novel antidepressant, is also the first nontricyclic serotonin/norepinephrine reuptake inhibitor. Although venlafaxine has an overall side effect and safety profile that is comparable to other newer antidepressants, it can cause both transient and sustained elevations of supine diastolic blood pressure (SDBP), probably the result of noradrenergic potentiation. Presented here is a meta-analysis of original data on blood pressure, using both random effects and a multivariate survival analyses. The sample consisted of 3744 patients with major depression who were studied in controlled clinical trials comparing venlafaxine with imipramine and/or placebo. Patients were treated for 6 weeks of acute phase therapy; some responders received up to 1 year of continuation phase therapy. Venlafaxine and imipramine were associated with small, but statistically significant, increases in SDBP during acute phase therapy. When compared with imipramine and placebo, venlafaxine was also associated with a greater proportion of persistent elevations of SDBP during continuation therapy. The effect of venlafaxine was highly dose dependent, and the incidence of elevated SDBP was statistically and clinically significant only at dosages above 300 mg/day. Venlafaxine did not adversely affect the control of blood pressure for patients with preexisting high blood pressure or elevated baseline values. Venlafaxine has a dose-dependent effect on SDBP that is clinically significant at high dosages. Concern about blood pressure effects should not deter first-line use of this effective antidepressant, although more extensive studies of patients with cardiovascular diseases are still necessary. [\hyperlink{Venlafaxine Hydrochloride}{PMID: 25450524}, M E Thase et al., 1998]

\hypertarget{pmid_30025480}{T}he potential genotoxic effect of venlafaxine hydrochloride (venlafaxine), an antidepressant drug-active ingredient, was investigated by using  [\hyperlink{Venlafaxine Hydrochloride}{PMID: 30025480}, Selim Ayabaktı et al., 2020] Venlafaxine is an antidepressant which acts through the inhibition of the reuptake of norepinephrine and serotonin. Venlafaxine is active against neuropathic and chronic pain. We report the case of a 69-year-old woman who presented a paclitaxel-induced neuropathy. She presented paresthesias, pin pricks in both hands with functional impairment. Venlafaxine hydrochloride was introduced at 37.5 mg twice daily. The patient noticed a dramatic recovery of her symptoms within 2 days, with both reduction of the paresthesias and functional improvement. This is the first report of efficacious use of venlafaxine for the treatment of paclitaxel cumulative neurosensory toxicity. [\hyperlink{Venlafaxine Hydrochloride}{PMID: 30025480}, Jean-Philippe Durand et al., 2002]

\hypertarget{pmid_9541154}{W}e examined the efficacy and safety of three different dosages of venlafaxine hydrochloride (75, 225, and 375 mg/day) in a multicenter, randomized, double-blind, placebo-controlled, four-group study. Outpatients, 18 to 65 years old, who met DSM-III criteria for major depression were included (N = 358 randomized; 194 completed). Of the total patients completing the trial, 59\%, 56\%, 51\%, and 51\% were in the placebo, 75-mg, 225-mg, and 375-mg groups, respectively. The primary outcome measures were the Hamilton Rating Scale for Depression (HAM-D21) total, HAM-D21 depression item, Montgomery-Asberg Depression Rating Scale total, and Clinical Global Impressions scale. Each dosage of venlafaxine was associated with statistically significant improvement as compared with placebo, based on the intent-to-treat sample. The two higher dosages were associated with a modestly greater antidepressant response than was the 75-mg dosage. Nausea, dizziness, somnolence, and anorexia were the most common adverse events attributable to venlafaxine. Since headache occurred at a similar frequency in both the drug and placebo groups, we did not consider it to be attributable to venlafaxine use. Withdrawal from the study due to adverse events occurred in 5\%, 17\%, 24\%, and 30\% of the patients in the placebo, 75-mg, 225-mg, and 375-mg groups, respectively. Venlafaxine, at dosages of 75-375 mg/day, is an effective and well-tolerated antidepressant. With increasing dosage, greater efficacy and possibly more adverse effects will occur. [\hyperlink{Venlafaxine Hydrochloride}{PMID: 9541154}, R L Rudolph et al., 1998]

\hypertarget{pmid_17901792}{F}luoroquinolones, including levofloxacin, have not been recommended for use in children largely because studies in juvenile laboratory animals suggest there may be an increased risk of fluoroquinolone-associated cartilage lesions. A large prospective trial is needed to assess the risks associated with using levofloxacin in children. Assess the safety and tolerability of levofloxacin therapy in children based on observations for 1 year after therapy. Safety data were collected in children who participated in 1 of 3 efficacy trials (N = 2523) and a subset of these children who also subsequently participated in a long-term 1-year surveillance trial (N = 2233). Incidence of adverse events in children randomized to receive levofloxacin versus nonfluoroquinolone antibiotics was compared. Based on assessments by treating physicians and an independent data safety monitoring committee, events related to the musculoskeletal system were further categorized as 1 of 4 predefined musculoskeletal disorders (arthralgia, arthritis, tendinopathy, gait abnormality) considered most likely clinical correlates of fluoroquinolone-associated cartilage lesions observed in laboratory animals. Levofloxacin was well tolerated during and for 1 month after therapy as evidenced by similar incidence and character of adverse events compared with nonfluoroquinolone antibiotics. However, incidence of at least 1 of the 4 predefined musculoskeletal disorders (largely due to reports of arthralgia) was greater in levofloxacin-treated compared with nonfluoroquinolone-treated children at 2 months (2.1\% vs. 0.9\%; P = 0.04) and 12 months (3.4\% vs. 1.8\%; P = 0.03) after starting therapy. The incidence of 1 or more of the 4 predefined musculoskeletal disorders identified in nonblinded, prospective evaluations, was statistically greater in levofloxacin-treated compared with comparator-treated children. [\hyperlink{Venlafaxine Hydrochloride}{PMID: 17901792}, Gary J Noel et al., 2007]

\hypertarget{pmid_17941284}{T}he safety of fexofenadine has been examined extensively in adults and school-age children. However, the safety of fexofenadine in children younger than 6 years has not been reported to date. To compare the safety and tolerability of twice-daily fexofenadine hydrochloride, 30 mg, and placebo in preschool children aged 2 to 5 years with allergic rhinitis. This was a multicenter, double-blind, randomized, placebo-controlled, parallel-group study, conducted between February 29, 2000, and June 14, 2001. Participants were randomized to either fexofenadine hydrochloride, 30 mg, or placebo twice daily for a 2-week period. To facilitate dosing, capsule content was mixed with applesauce (approximately 10 mL). Safety assessments depended on date of entry into the study because of an amendment to the protocol. Before the amendment, assessments included physical examination, vital signs reporting (oral temperature, heart rate, and respiratory rate), and adverse event (AE) reporting. After the amendment, safety assessments included laboratory testing (blood chemistry and hematology profiles), physical examination, 12-lead electrocardiography, and vital signs (oral temperature, blood pressure, heart rate, and respiratory rate) and AE reporting. Treatment-emergent AEs were observed in 116 of 231 participants receiving placebo and 111 of 222 receiving fexofenadine. These AEs were possibly related to study medication in 19 (8.2\%) and 21 (9.5\%) of the participants receiving placebo and fexofenadine, respectively, and most frequently involved the digestive system. No clinically relevant differences in laboratory measures, vital signs, and physical examinations were observed. The findings show that fexofenadine hydrochloride, 30 mg, is well tolerated and has a good safety profile in children aged 2 to 5 years with allergic rhinitis. [\hyperlink{Venlafaxine Hydrochloride}{PMID: 17941284}, Henry Milgrom et al., 2007]

\hypertarget{pmid_22707834}{V}enlafaxine representing a new class of antidepressants is a potent serotonin/ norepinephrine reuptake inhibitor. Transdermal delivery of venlafaxine hydrochloride (VHCl) may result in proper patient compliance by reducing the incidence of the undesirable GI problems generally associated with its plural oral dosing. The present study is an attempt to investigate the improvement of the transdermal flux of the hydrophilic VHCl by certain permeation enhancers viz. glycerin, urea, propylene glycol and mixture of propylene glycol and ethanol across pig ear skin. The cumulative drug release was the highest from the formulation F5 consisting of the mixture of propylene glycol and ethanol in sodium alginate gel with a load of 25\% w/w VHCl with 96\% permeation enhancement. The steady state flux observed with F5 was 0.203 mg cm(-2) hr and an area of 15.27 cm(2) would suffice to arrive at a required therapeutic concentration of VHCl in the blood. [\hyperlink{Venlafaxine Hydrochloride}{PMID: 22707834}, C Vijaya et al., 2011]

\hypertarget{pmid_17044905}{V}enlafaxine (Efexor), a selective noradrenergic reuptake inhibitor, is an important therapeutic option in the treatment of perinatal depression, but its effects on the newborn are uncertain. We present a report of two infants with neonatal seizures attributed to maternal use of venlafaxine. The first infant was hypotonic and required resuscitation at birth. The second was born in a good condition but developed clinically apparent seizures after the second day of life. Both infants responded rapidly to treatment with phenobarbitone that was weaned uneventfully by the first and second week of life. Both remain well at 1 year of age. Other causes of neonatal seizures were excluded and neurological investigations on these two infants were unremarkable. We suggest that all infants exposed to maternal venlafaxine, no matter their condition at birth, be monitored in hospital for at least 3 to 4 days in order to preempt and treat adverse neurological events. [\hyperlink{Venlafaxine Hydrochloride}{PMID: 17044905}, Ravi K Pakalapati et al., 2006]

\hypertarget{pmid_22093944}{T}o examine whether three cycles of a low-intensity chemotherapy consisting of cyclophosphamide [500 mg/m(2) - day 1], vinblastine [6 mg/m(2) - days 1 and 8] and prednisolone [40 mg/m(2) - days 1-7] (CVP) is safe and therapeutically effective in children and adolescents with early stage nodular lymphocyte predominant Hodgkin lymphoma [nLPHL]. Fifty-five children and adolescents with early stage nLPHL [median age 13 years, range 4-17 years] diagnosed between June 2005 and October 2010 in the UK and France are the subjects of this report. Staging investigations included conventional cross sectional as well as 18 fluro-deoxyglucose [FDG] PET imaging. Histology was confirmed as nLPHL by an expert pathology panel. Of the 45 patients, who received CVP as first line treatment, 36 [80\%, 95\% Confidence Interval [CI]: (68; 92)] either achieved a complete remission [CR] or CR unconfirmed [CRu], the remaining nine patients achieved a partial response. All nine subsequently achieved CR with salvage chemotherapy [n=7] or radiotherapy [n=2]. Ten patients received CVP at relapse after primary treatment that consisted of surgery alone and all achieved CR. To date, only three patients have relapsed after CVP chemotherapy and all had received CVP as first line treatment at initial diagnosis. The 40-month freedom from treatment failure and overall survival for the entire cohort were 75.4\% (SE ± 6\%) and 100\%, respectively. No significant early toxicity was observed. Our results show that CVP is an effective chemotherapy regimen in children and adolescents with early stage nLPHL that is well tolerated with minimal acute toxicity. [\hyperlink{Venlafaxine Hydrochloride}{PMID: 22093944}, Ananth Shankar et al., 2012]

\hypertarget{pmid_17267793}{T}he authors evaluated the efficacy, safety, and tolerability of extended-release venlafaxine in the treatment of pediatric generalized anxiety disorder. Two randomized, double-blind, placebo-controlled trials were conducted at 59 sites in 2000 and 2001. Participants 6 to 17 years of age who met DSM-IV criteria for generalized anxiety disorder received a flexible dosage of extended-release venlafaxine (N=157) or placebo (N=163) for 8 weeks. The primary outcome measure was the composite score for nine delineated items from the generalized anxiety disorder section of a modified version of the Schedule for Affective Disorders and Schizophrenia for School-Age Children, and the primary efficacy variable was the baseline-to-endpoint change in this composite score. Secondary outcome measures were overall score on the nine delineated items, Pediatric Anxiety Rating Scale, Hamilton Anxiety Rating Scale, Screen for Child Anxiety Related Emotional Disorders, and the severity of illness and improvement scores from the Clinical Global Impression scale (CGI). The extended-release venlafaxine group showed statistically significant improvements in the primary and secondary outcome measures in study 1 and significant improvements in some secondary outcome measures but not the primary outcome measure in study 2. In a pooled analysis, the extended-release venlafaxine group showed a significantly greater mean decrease in the primary outcome measure compared with the placebo group (-17.4 versus -12.7). The response rate as indicated by a CGI improvement score <3 was significantly greater with extended-release venlafaxine than placebo (69\% versus 48\%). Common adverse events were asthenia, anorexia, pain, and somnolence. Statistically significant changes in height, weight, blood pressure, pulse, and cholesterol levels were observed in the extended-release venlafaxine group. Extended-release venlafaxine may be an effective, well-tolerated short-term treatment for pediatric generalized anxiety disorder. [\hyperlink{Venlafaxine Hydrochloride}{PMID: 17267793}, Moira A Rynn et al., 2007]

\hypertarget{pmid_15690910}{T}he efficacy of the fluoroquinolone levofloxacin in the treatment of 35 children with bronchopulmonary disease exacerbation was practically the same as that of amoxycillin/clavulanate, cefotaxime or ceftriaxone. The clinical and bacteriological results were favourable. The eradication of the pathogens responsible for the bronchopulmonary inflammations in 86\% of the patients was stated. There is no doubt that fluoroquinolones should not be widely used in pediatrics. They should be considered as reserve drugs for the treatment of severe cases when the routine agents fail. Their use is justified when the situation is risky and the data on the pathogen susceptibility to the drugs are available. Still, levofloxacin is the most safe fluoroquinolone with low hepatotoxicity and lower effect on the central nervous system. The episodes of its negative cardiovascular action are less frequent. Moreover, the most frequent side effects of fluoroquinolones such as nausea, diarrhea or vomiting are less frequent with the use of levofloxacin. No signs of arthropathy in the patients treated with levofloxacin were observed in our trial. [\hyperlink{Venlafaxine Hydrochloride}{PMID: 15690910}, I K Volkov et al., 2004]

\section*{Trospium Chloride}
\subsection*{Result}
\subsubsection*{Answer}

Yes (children, age range not explicitly stated but study population is "children" with bladder instability, short-term use up to 21 days)
Unknown (infants, long-term use in children, or other pediatric age ranges not specifically studied)

\subsubsection*{{Explanation}}
\hypertarget{Trospium Chloride}
A review of the available abstracts reveals one targeted study specifically evaluating the safety and efficacy of trospium chloride in children:

- [\hyperlink{pmid_14532838}{PMID: 14532838}, P Lopez Pereira et al., 2003]: This multicenter, randomized, single-blind clinical study included 58 children with bladder instability, who were allocated to receive various doses of trospium chloride (10, 15, 20, or 25 mg daily) or placebo for 21 days. The study reports that trospium chloride was well tolerated, with only 10\% of patients experiencing adverse effects, and no significant safety concerns were identified. The study concludes that trospium chloride is an effective option for the management of detrusor instability in children.

No other abstracts provide targeted safety data for trospium chloride in children. The remaining studies either focus on adults, animal models, or do not specify pediatric populations. Therefore, based on the evidence from the above study, trospium chloride appears to be safe for use in children with bladder instability, at least for the short-term (21 days) and within the studied age range (not explicitly stated, but the study population is described as "children").

No abstracts provide data for infants or for long-term use in children.

\subsection*{Abstracts}
\hypertarget{pmid_14532838}{W}e assessed the efficacy and most appropriate dosage of trospium chloride (TCl) for managing bladder instability in children as compared with a placebo. A total of 58 patients with bladder instability were allocated at random to 1 of 5 groups-10, 15, 20 or 25 mg TCl, or placebo administered daily in a multicenter, randomized, single-blind clinical study. Patients were treated for 21 days, and current symptoms, voiding diary and urodynamic values were collected at the beginning and end of the treatment period. All adverse events were recorded at the last visit. Of 50 patients treated with TCl 41 (82\%) had a positive therapeutic result (excellent, good or fair) versus only 3 of 8 patients with improvement in the placebo group (37.5\%, p = 0.006). In all responding patients clinical symptoms either resolved or decreased markedly, and in 37 (74\%) this improvement was accompanied by urodynamic improvement. In these 37 children the average number of uninhibited contractions decreased by 54.3\% (p <0.0001) and the volume at first contraction increased by 71.4\% (p = 0.001). There were no statistically significant differences with regard to therapeutic efficacy between TCl dosages. Fourteen patients (9 with TCl, 5 with placebo) showed no clinical improvement, although some had improved urodynamic parameters. Furthermore, TCl was well tolerated with few patients (10\%) experiencing adverse effects. Trospium chloride (10 to 25 mg total daily dosage, split into 2 doses) is an effective option for the management of detrusor instability in children. [\hyperlink{Trospium Chloride}{PMID: 14532838}, P Lopez Pereira et al., 2003]

\hypertarget{pmid_15126811}{T}rospium chloride is an anticholinergic agent with predominantly peripheral nonselective antimuscarinic activity lacking central nervous system effects. It has no known drug-drug interactions, an advantage for patients taking many medications. Because these qualities may provide added benefit when treating patients with symptoms associated with overactive bladder (OAB) and urge incontinence, we studied the effectiveness of trospium in treating these conditions. Patients with OAB with urge incontinence were randomized 1:1 to 20 mg trospium twice daily or placebo in this 12-week, multicenter, parallel, double-blind, placebo controlled trial. Dual primary end points were change in average number of toilet voids and change in urge incontinent episodes per 24 hours. Secondary efficacy variables were change in average of volume per void, voiding urge severity, urinations during day and night, time to onset of action and change in Incontinence Impact Questionnaire. A total of 523 patients were entered at 51 sites. Trospium significantly decreased average frequency of toilet voids and urge incontinent episodes compared to placebo. It significantly increased average volume per void, and decreased average urge severity and daytime frequency. All effects occurred by week 1 and all were sustained throughout the study. Nocturnal frequency decreased significantly by week 4 and Incontinence Impact Questionnaire scores improved at week 12. Trospium was well tolerated. Trospium was found to have sustained effectiveness beginning at the end of week 1 in decreasing the number of voids, urge incontinent episodes, total daily micturitions and urge severity, and in increasing volume per void. It also improved symptoms of OAB and quality of life. [\hyperlink{Trospium Chloride}{PMID: 15126811}, Norman Zinner et al., 2004]

\hypertarget{pmid_15482001}{T}rospium chloride is an orally active, quaternary ammonium compound with antimuscarinic activity. It binds specifically and with high affinity to muscarinic receptors M(1), M(2) and M(3), but not nicotinic, cholinergic receptors. It is hydrophilic and does not cross the normal blood-brain barrier in significant amounts and, therefore, has minimal central anticholinergic activity. Peak plasma trospium chloride concentrations are attained approximately 5-6 hours after oral administration, which should occur before meals as concurrent food ingestion significantly reduces trospium bioavailability. Trospium chloride undergoes negligible metabolism by the hepatic cytochrome P450 system; few metabolic drug interactions are known. While trospium chloride dosage adjustments based on age or sex appear unwarranted, such adjustments may be needed in patients with severe renal impairment. Direct comparative studies in patients with overactive bladder indicate that trospium chloride is at least as effective as oxybutynin and tolterodine. Placebo-controlled studies have also confirmed the efficacy of trospium chloride in terms of improved urodynamic parameters; small-scale, noncomparative studies have documented significant trospium chloride-induced improvements in patients with reflex neurogenic bladder, postoperative bladder irritation and radiation-induced cystitis; and observational studies including >10,000 patients have also revealed favourable findings for trospium chloride, including a marked decrease in incontinence episodes and substantial improvement in health-related quality of life. Trospium chloride is generally well tolerated, and significantly more so than immediate-release oxybutynin. The most frequent adverse events, occurring in >1\% of trospium chloride-treated patients, are dry mouth, dyspepsia, constipation, abdominal pain and nausea. Available for many years in several countries outside North America, trospium chloride is likely to develop an important role in the management of overactive bladder following its approval in the US on 28 May 2004. [\hyperlink{Trospium Chloride}{PMID: 15482001}, Eric S Rovner et al., 2004]

\hypertarget{pmid_18360555}{T}rospium chloride is a quaternary ammonium compound, which is a competitive antagonist at muscarinic cholinergic receptors. Preclinical studies using porcine and human detrusor muscle strips demonstrated that trospium chloride was many-fold more potent than oxybutynin and tolterodine in inhibiting contractile responses to carbachol and electrical stimulation. The drug is poorly bioavailable orally (< 10\%) and food reduces absorption by 70\%- 80\%. It is predominantly eliminated renally as unchanged compound. Trospium chloride, dosed 20 mg twice daily, is significantly superior to placebo in improving cystometric parameters, reducing urinary frequency, reducing incontinence episodes, and increasing urine volume per micturition. In active-controlled trials, trospium chloride was at least equivalent to immediate-release formulations of oxybutynin and tolterodine in efficacy and tolerability. The most problematic adverse effects of trospium chloride are the anticholinergic effects of dry mouth and constipation. Comparative efficacy/tolerability data with long-acting formulations of oxybutynin and tolterodine as well as other anticholinergics such as solifenacin and darifenacin are not available. On the basis of available data, trospium chloride does not appear to be a substantial advance upon existing anticholinergics in the management of urge urinary incontinence. [\hyperlink{Trospium Chloride}{PMID: 18360555}, David Rp Guay et al., 2005]

\hypertarget{pmid_8494577}{T}he safety and tolerance of increasing single oral doses of 20, 40, 80, 120, 180, 240 and 360 mg trospium chloride (Spasmo-lyt, CAS 10405-0204) were investigated in 29 healthy male volunteers in a double-blind placebo-controlled study. Blood pressure, heart rate, ECG, pupillary diameter, salivary secretion, and subjective reports of tolerance revealed no essential differences between placebo and trospium chloride in doses up to 120 mg. Starting with single doses of 180 mg, anticholinergic effects were observed with increasing intensity, i.e., dilatation of the pupils, reduction of salivary flow, and increase of heart rate. While the highest administered dose of 360 mg trospium chloride did not cause any relevant changes of vital parameters (blood pressure, pulse, ECG), it was subjectively rated as quite unpleasant. The data show that trospium chloride is well tolerated in single oral doses well above the current therapeutic daily dose of up to 40 mg. [\hyperlink{Trospium Chloride}{PMID: 8494577}, H P Breuel et al., 1993]

\hypertarget{pmid_16461077}{T}o study the efficacy and safety of trospium chloride in treating overactive bladder. Trospium chloride is an anticholinergic agent with predominantly peripheral nonselective antimuscarinic activity and thus has potential therapeutic value in treating patients with overactive bladder. Patients with overactive bladder were randomized on a 1:1 basis to either placebo or trospium chloride 20 mg twice daily in this 12-week, multicenter, parallel, double-blind, placebo-controlled study. The primary endpoint was the change in the average number of toilet voids per 24 hours. The secondary efficacy variables were changes in the average void urgency severity, volume per toilet void, urge frequency, number of daily urge urinary incontinence episodes, and daytime sleepiness. A total of 658 patients were randomized at 52 sites. Trospium chloride significantly decreased the average number of daily toilet voids, average urgency severity, urge frequency, and urge urinary incontinence episodes and increased the average volume per void at weeks 1, 4, and 12. All effects occurred by the end of week 1 and all improved and were sustained throughout the 12-week study. Adverse events included dry mouth and constipation. Trospium chloride had significant and sustained effectiveness beginning at the end of week 1 and continuing through 12 weeks of treatment. Trospium chloride was also safe and generally well tolerated. [\hyperlink{Trospium Chloride}{PMID: 16461077}, Delbert Rudy et al., 2006]

\hypertarget{pmid_9622945}{T}he authors investigated in a group of six children with glaucoma persisting for a long time the possibility to use locally applied carbonic anhydrase inhibitor, 2\% dorsolamide hydrochloride in the form of eye drops (TRUSOPT, Merck Co.). In the submitted preliminary study they evaluate the effectiveness of the drug in glaucoma in children very favourably, previous essential treatment with oral acetazolamide could be discontinued in all patients without a deleterious effect. The authors did not encounter any undesirable effects of the drug nor manifestations of intolerance calling for discontinuation of TRUSOPT treatment. This is so far the first communication on TRUSOPT treatment of child glaucoma in the available literature. [\hyperlink{Trospium Chloride}{PMID: 9622945}, J Rehůrek et al., 1998]

\hypertarget{pmid_7900236}{H}igh doses of metoclopramide are contraindicated to prevent chemotherapy-induced emesis in pediatric patients, since the incidence of extrapyramidal reactions is increased in these patients. The aim of this small study was to evaluate the antiemetic activity and the safety of tropisetron (a new selective antagonist of 5-HT3 receptors) in children who suffered nausea and vomiting during previous chemotherapy courses, despite the administration of an anxiolytic agent (hydroxyzine hydrochloride). The children with a malignant neoplasm were treated for emesis with tropisetron (5 mg o.a.d. or b.i.d.) during a total of 20 cycles of chemotherapy with carboplatin combined with other antitumor agents. In 14 cycles (70\%), there was no vomiting. There were two or less episodes of vomiting in 2 cycles (10\%), 3-4 episodes in 2 cycles (10\%), and no inhibition of vomiting at all in 2 cycles (10\%). In 8 cycles there were no episodes of nausea (40\%), in 5 cycles (25\%) there were episodes of moderate nausea, and in 4 (20\%) there were episodes of severe nausea. One child had a mild headache during one cycle and moderate hypotension during another. The results suggest that tropisetron is both efficacious and safe for the treatment of pediatric patients. [\hyperlink{Trospium Chloride}{PMID: 7900236}, P Rosso et al., 1994]

\hypertarget{pmid_1398851}{W}e prospectively studied the pharmacokinetics of intravenous Chloramphenicol succinate (CS) in children (age 6 months-14 years) with culture proven typhoid fever (n = 30) and non typhoidal illnesses (n = 10). CS was administered in three different dosage regimens (50, 75 and 100 mg/kg/d-q 6 hourly). Liver function tests were monitored. Plasma trough and peak chloramphenicol concentrations were measured by HPLC analysis after 42 hrs. The 50 mg/kg/day dosage schedule was terminated midway through the study, as blood levels were consistently low and two patients with typhoid relapsed, children with typhoid had significantly lower clearance of CS in comparison with those with non-typhoidal illness (0.29 +/- 0.1 versus 0.5 +/- 0.37 1/kg/hr, P 0.05). There was no significant difference between mean peak and trough concentrations of chloramphenicol on 100 mg/kg/day and 75 mg/kg/day in children with typhoid. However, two children on 100 mg/kg/day dosage developed trough concentrations greater than 20 mcg/ml. No correlation was found between CS clearance and serum bilirubin, SGPT (alanine transaminase) and alkaline phosphatase. Our data show altered clearance of CS in children with typhoid and suggests that 75 mg/kg/day may be a safer dose in children with hepatic dysfunction in typhoid. [\hyperlink{Trospium Chloride}{PMID: 1398851}, Z A Bhutta et al., ]

\hypertarget{pmid_36625617}{T}o evaluate the efficiency of long-term use of trospium chloride (Spazmex) for the treatment of patients with neurogenic overactive bladder due to Parkinson's disease (PD) and to determine the influence of therapy on the cognitive status of patients. 60 patients with PD and neurogenic overactive bladder with stages 2.5, 3 and 4 according to Hoehn-Yahr scale were included in the main group. The mean age was 58.2+/-5.7 years. All patients were prescribed trospium chloride at entry into the study, with doses titrated gradually according to clinical efficacy (30 to 90 mg). The comparison group included 15 patients with PD and neurogenic overactive bladder at stages 2,5 and 3, who received tibial neuromodulation according to the standard technique with skin electrodes. The mean age of patients was 56.4+/-4.6 years. At baseline, both groups were comparable in terms of gender, age and cognitive status (p=0.801). All patients received treatment for 52 weeks. The efficiency of therapy was assessed according to bladder diaries, while safety outcomes included postvoid residual, side effects, cognitive status according to the MoCA scale and quality of life according to the SF-Qualiveen questionnaire. clinical efficacy and satisfaction were achieved in all patients who completed the study (47 patients in the main group and 15 patients in the comparison group). Good clinical efficacy was demonstrated in both groups, since there was a decrease in the number of urinations, episodes of urgency and urinary incontinence. In addition, there was an improvement in the quality of life according to the SF-Qualiveen scale. The cognitive status during the entire follow-up period remained without significant changes in both groups. Trospium chloride is an effective drug in patients with PD. It does not affect cognitive functions during long-term use. Trospium chloride should be considered as first-line drug in those with urologic manifestations of PD. [\hyperlink{Trospium Chloride}{PMID: 36625617}, E S Korshunova et al., 2022]

\hypertarget{pmid_17632131}{A}n extended release formulation of trospium chloride was recently developed for the once daily treatment of overactive bladder. We investigated the safety, efficacy and tolerability of 60 mg trospium chloride once daily. Subjects with overactive bladder were randomized 1:1 to receive 60 mg trospium chloride once daily or placebo in this 12-week multicenter, parallel, double-blind, placebo controlled trial. Primary end points were calculated changes in diary recorded daily urinary frequency and daily urgency urinary incontinence episodes. Secondary end points were urgency severity, volume voided per void and the number of urgency voids per day. Safety was assessed by clinical examination, adverse event monitoring, clinical laboratory values and resting electrocardiograms. Overall 601 subjects were prescribed trospium once daily (298) or placebo (303). Trospium once daily treatment resulted in significant improvements over placebo in all primary and key secondary efficacy outcomes at weeks 1 through 12. The most common adverse events were dry mouth (trospium 8.7\% vs placebo 3\%) and constipation (trospium 9.4\% vs placebo 1.3\%). Central nervous system adverse events were rare (headache with trospium 1.0\% vs placebo 2.6\%). No clinically meaningful changes in laboratory, physical examination or electrocardiogram parameters were noted. Trospium once daily provided significant improvements in overactive bladder symptoms (frequency, urgency urinary incontinence and urgency). Efficacy was similar to that seen previously with trospium chloride twice daily, while class effect anticholinergic adverse events occurred at comparatively low levels. Dry mouth was elicited at the lowest reported rate in the oral antimuscarinic drug class. [\hyperlink{Trospium Chloride}{PMID: 17632131}, David Staskin et al., 2007]

\hypertarget{pmid_34853794}{T}ris(1,3-dichloro-2-propyl) phosphate (TDCIPP) has been widely used as a flame retardant and is commonly detected in environmental samples. Biomonitoring studies relying on urinary metabolite levels (i.e. bis(1,3-dichloro-2-propyl) phosphate (BDCIPP)) demonstrate widespread exposure, but TDCIPP intake is unknown. Intake data area critical component of meaningful risk assessments and are needed to elucidate the potential health impacts of TDCIPP exposure. Using biomonitoring data, we estimated TDCIPP intake for infants. Infants aged 2-18 months were recruited from central, North Carolina (n=43, recruited 2014-2015), and spot urine samples were analyzed for BDCIPP. TDCIPP intake rates were estimated using daily urine excretion and the fraction of TDCIPP excreted as BDCIPP in urine. Daily TDCIPP intake estimates ranged from 0.01-15.03 μg/kg-day for children included in our assessment, with some variation depending on model assumptions. The U.S. Consumer Products Safety Commission (CPSC) previously established an acceptable daily intake of 5μg/kg-day for non-cancer health risks. Depending on modeling assumptions, we found that 2-9\% percent of infants had TDCIPP intake estimates above this threshold. Our results indicate that current TDCIPP exposure levels could pose health risks for highly exposed infants. [\hyperlink{Trospium Chloride}{PMID: 34853794}, Kate Hoffman et al., 2017]

\hypertarget{pmid_8439640}{T}rospium chloride is a muscarinergic antagonist acting on oesophageal smooth muscle and on ganglionic and/or myenteric neurons. The effect of this drug on oesophageal motility was tested in 16 healthy male subjects in a double-blind randomized cross-over examination of trospium chloride or placebo following phentolamine or placebo application. Each subject underwent two separate investigations at least one week apart. Trospium chloride was effective in the oesophagus to reduce contractile activity (amplitude and duration of peristalsis) in all parts of the oesophageal body, and this effect was not blocked by phentolamine. Its potent action and its minor side-effects appear to be promising for clinical use in patients with motility disorders such as the hypercontractile oesophagus. [\hyperlink{Trospium Chloride}{PMID: 8439640}, T Frieling et al., 1993]

\hypertarget{pmid_19813252}{T}ranscranial Doppler ultrasonography (TCD) is used to predict stroke risk in children with sickle cell anemia (SCA), but has not been adequately studied in children under age 2 years. TCD was performed on infants with SCA enrolled in the BABY HUG trial. Subjects were 7-17 months of age (mean 12.6 months). TCD examinations were successfully performed in 94\% of subjects (n = 192). No patient had an abnormal TCD as defined in the older child (time averaged maximum mean TAMM velocity > or =200 cm/sec) and only four subjects (2\%) had velocities in the conditional range (170-199 cm/sec). TCD velocities were inversely related to hemoglobin (Hb) concentration and directly related to increasing age. Determination of whether the TCD values in this very young cohort of infants with SCA can be used to predict stroke risk later in childhood will require analysis of exit TCD's and long-term follow-up, which is ongoing (ClinicalTrials.gov number, NCT00006400). [\hyperlink{Trospium Chloride}{PMID: 19813252}, Steven G Pavlakis et al., 2010]

\hypertarget{pmid_22120415}{T}o analyse whether the permeability of the blood-brain barrier to the antimuscarinic drug trospium chloride is altered with ageing. This is a relevant question for elderly patients with overactive bladder syndrome who are treated with trospium chloride as the occurrence of adverse effects on the central nervous system (CNS) highly depends on the absolute drug concentration in the brain. Trospium chloride at 1 mg/kg was intravenously administered to adult, middle-aged, and aged mice at 6, 12, and 24 months of age, respectively, and the absolute drug concentrations in the brain were analysed after 2 h. Furthermore, mRNA expression levels of relevant markers of blood-brain barrier integrity (occludin, claudin-5, and the drug efflux carrier P-glycoprotein) were analysed in brain samples from adult and aged mice. The absolute brain concentrations of the drug were identical in adult and middle-aged mice (13 ± 2 ng/g vs. 13 ± 2 ng/g) and were slightly, but significantly, lower in aged mice (8 ± 4 ng/g). The brain/plasma drug concentration ratios were not different between the age groups and demonstrated the generally low capability of trospium chloride in permeating the blood-brain barrier. Occludin, claudin-5, and P-glycoprotein showed identical mRNA expression levels in the brains of adult and aged mice. Based on our in vivo data in a mouse model, we conclude that trospium chloride permeation across the BBB is not increased in ageing per se, and therefore, the occurrence of adverse CNS drug effects is also not expected to increase with ageing. [\hyperlink{Trospium Chloride}{PMID: 22120415}, Jasmin Kranz et al., 2013]

\hypertarget{pmid_23024102}{W}e conducted this single blind randomized clinical trial to compare the efficacy and safety of oral chloral hydrate and intranasal midazolam for induction of sedation for computerized tomography scan of brain in children. Participants aged 1-10 years (n=60) were randomized to receive 100 mg/kg chloral hydrate orally with intra nasal normal saline OR intranasal midazolam 0.2 mg/kg with oral normal saline. Adequate sedation (Ramsay sedation score of four) was obtained and CT scan completed successfully in 76.7\% of chloral hydrate group and in 40\% of midazolam group (P=0.004). No significant difference was seen for side effects frequency between the two drugs (10\% in chloral hydrate, 3.3\% in midazolam group; P=0.34). We conclude that oral chloral hydrate can be considered as a safe and effective drug for sedation in children undergoing CT scan of brain. [\hyperlink{Trospium Chloride}{PMID: 23024102}, Razieh Fallah et al., 2013]

\hypertarget{pmid_18370455}{I}n a placebo-controlled, double-blind study the effects of depressing duodenal motility by administration of intravenous trospium chloride during gastroduodenoscopy were studied in 72 patients randomised to receive trospium chloride or saline (controls). Intravenous trospium chloride 1.2mg stopped the visible contractile activity of the duodenum as assessed by 3 independent observers during videoendoscopy within 76 seconds (median). During a 4-minute observation period of duodenal peristalsis, duodenal motor activity was found to stop in 18 of 36 patients after trospium chloride but in only 5 of 36 patients in the placebo group (p = 0.002). Adverse effects were dry mouth, micturition difficulties, sweat retention, accommodation disturbance and tachycardia. Trospium chloride was effective in reducing contractile activity in the duodenum. Its potent action and minor adverse effect profile appear to be promising for gastroduodenoscopy and especially for sphincter of Oddi motility in patients during routine endoscopic retrograde cholangiopancreatography. [\hyperlink{Trospium Chloride}{PMID: 18370455}, H Rohde et al., 1997]

\hypertarget{pmid_16780480}{M}olluscum contagiosum is a common viral infection of the skin that frequently affects children. Lesions take between 6 and 18 months to resolve spontaneously and are a source of great embarrassment to both caretakers and children, often affecting attendance at school and limiting social activity. Treatment options to date have been poorly tolerated by children but recent studies have suggested that potassium hydroxide may be beneficial. This double-blind, randomized, placebo-controlled study compared 10\% potassium hydroxide with placebo (normal saline). Twenty patients, aged 2 to 12 years, were recruited. Parents applied a solution twice daily to lesional skin until signs of inflammation appeared. Children were examined by the same observer on days 0, 15, 30, 60, and 90. Seventy percent of children receiving topical potassium hydroxide cleared, compared with 20\% in the placebo group. Further dosing studies are required to identify whether weaker concentrations of potassium hydroxide are as efficacious, with less irritancy. [\hyperlink{Trospium Chloride}{PMID: 16780480}, Katherine A Short et al., ]

\hypertarget{pmid_31292919}{T}riclofos sodium (TFS) has been used for many years in children as a sedative for painless medical procedures. It is physiologically and pharmacologically similar to chloral hydrate, which has been censured for use in children with neurocognitive disorders. The aim of this study was to investigate the safety and efficacy of TFS sedation in a pediatric population with a high rate of neurocognitive disability. The database of the neurodiagnostic institute of a tertiary academic pediatric medical center was retrospectively reviewed for all children who underwent sedation with TFS in 2014. Data were collected on demographics, comorbidities, neurologic symptoms, sedation-related variables, and outcome. The study population consisted of 869 children (58.2\% male) of median age 25 months (range 5-200 months); 364 (41.2\%) had neurocognitive diagnoses, mainly seizures/epilepsy, hypotonia, or developmental delay. TFS was used for routine electroencephalography in 486 (53.8\%) patients and audiometry in 401 (46.2\%). Mean (± SD) dose of TFS was 50.2 ± 4.9 mg/kg. Median time to sedation was 45 min (range 5-245), and median duration of sedation was 35 min (range 5-190). Adequate sedation depth was achieved in 769 cases (88.5\%). Rates of sedation-related adverse events were low: apnea, 0; desaturation ≤ 90\%, 0.2\% (two patients); and emesis, 0.35\% (three patients). None of the children had hemodynamic instability or signs of poor perfusion. There was no association between desaturations and the presence of hypotonia or developmental delay. TFS, when administered in a controlled and monitored environment, may be safe for use in children, including those with underlying neurocognitive disorders. [\hyperlink{Trospium Chloride}{PMID: 31292919}, Eytan Kaplan et al., 2019]

\hypertarget{pmid_18278305}{C}hloral hydrate and hydroxyzine are a drug combination frequently used by practitioners to sedate pediatric dental patients, but their effectiveness has not been compared to a negative control group in humans. The aim of this crossover, double-blinded study was to evaluate the effect of these drugs compared to a placebo, administered to young children for dental treatment. Thirty-five dental sedation sessions were carried out on 12 uncooperative ASA I children aged less than 5 years old. In each session patients were randomly assigned to groups P (placebo), CH (chloral hydrate 75 mg/kg) and CHH (chloral hydrate 50 mg/kg plus hydroxyzine 2.0 mg/kg). Vital signs and behavioral variables were evaluated every 15 min. Comparisons were statistically analyzed using Friedman and Wilcoxon tests. P, CH and CHH had no differences concerning vital signs, except for breathing rate. All vital signs were in the normal range. CH and CHH promoted more sleep in the first 30 min of treatment. Overall behavior was better in CH and CHH than in P. CH, CHH and P were effective in 62.5\%, 61.5\% and 11.1\% of the cases, respectively. Chloral hydrate was safe and relatively effective, causing more satisfactory behavioral and physiological outcomes than a placebo. [\hyperlink{Trospium Chloride}{PMID: 18278305}, Luciane Ribeiro de Rezende Sucasas da Costa et al., 2007]

\hypertarget{pmid_8169182}{T}here is evidence for the efficacy and safety of clonazepam (CZP) in adult anxiety disorders, but no formal studies to substantiate clinical reports of similar benefit in children with anxiety disorders. In this double-blind pilot study, 15 children, aged 7 to 13 years, entered a randomly assigned, double-blind crossover trial of 4 weeks of CZP (up to 2 mg/day) and 4 weeks of placebo. Twelve children completed the trial. All but 1 had a diagnosis of separation anxiety disorder, and all but 2 had comorbid diagnoses. Nine children appeared to have moderate to significant clinical improvement, but statistical comparisons on several ratings failed to confirm a trend in favor of CZP. Side effects of drowsiness, irritability, and/or oppositional behavior were notable in 10 children in the CZP phase compared with 5 in the placebo phase. Clonazepam was believed to have clinical benefit for some children, but this was not confirmed statistically in this small sample. Problematic side effects of drowsiness and disinhibition were common and possibly were due to rapid titration. [\hyperlink{Trospium Chloride}{PMID: 8169182}, F Graae et al., ]

\hypertarget{pmid_10759661}{T}o assess the efficacy and safety of trospium chloride (TCl, 20 mg twice daily) in the treatment of detrusor instability, compared with placebo. In all, 208 patients were allocated at random to either TCl or placebo in a double-blind clinical study; the patients were treated for 3 weeks. Urodynamic values were measured at the beginning and end of the treatment period. Adverse events were recorded on patient diary cards. A confirmatory adaptive procedure with one planned interim analysis was used to evaluate efficacy. Trospium chloride produced significant improvements in maximum cystometric bladder capacity (median treatment effect 22.0 mL, mean 37.3 mL, one-sided P = 0. 0054) and urinary volume at first unstable contraction (median treatment effect 45.0 mL, mean 63.6 mL, one-sided P = 0.0015). The patients' assessment of efficacy showed significantly greater clinical improvement in the TCl group than in the placebo group (two-sided P = 0.0047). Furthermore, TCl was well tolerated, with similar frequencies of adverse events reported in both groups (68\% in the TCl and 62\% in the placebo group). Trospium chloride (20 mg twice daily) is an effective and safe option for the treatment of detrusor instability. [\hyperlink{Trospium Chloride}{PMID: 10759661}, L Cardozo et al., 2000]

\hypertarget{pmid_3899048}{N}ineteen children (mean [+/- SD] age, 14.5 +/- 2.3 years) with severe, primary obsessive-compulsive disorder completed a ten-week, double-blind, controlled trial of clomipramine hydrochloride (mean dosage, 141 mg/day) or placebo, each of which was administered for five weeks. Half of the subjects had not responded to previous treatment with other tricyclic antidepressants. There was a significant improvement in observed and self-reported obsessions and compulsions that was independent of the presence of depressive symptoms at baseline. Improvement in obsessive-compulsive symptoms did not correlate significantly with plasma concentrations of the drug or its metabolites. Clomipramine appears to be effective in the treatment of children with obsessive-compulsive disorder and the treatment seems to be independent of an antidepressant effect. [\hyperlink{Trospium Chloride}{PMID: 3899048}, M F Flament et al., 1985]

\hypertarget{pmid_15951862}{D}iagnostic and therapeutic procedures in children are made easier using sedation. However, there is no consensus about which drug should be used to achieve this. Furthermore, none of the drugs used for sedation are risk free. The aim of this work is to study sedation indications, effectiveness, and safety at our center. A prospective observational study conducted at the Pediatric Day Care Unit, King Fahad National Guard Hospital, Riyadh, Saudi Arabia. The study covered 17.5 weeks in 2 periods: May 9th 1999 to June 13th 1999 and October 31st 2001 to February 11th 2002. Children <12 years were included. Collected data included demographics, indication, drug dosing and outcome. Data were reported as mean +/- SD. We included 148 patients, age 38 +/- 30 months. Adequate sedation was achieved in 79\% after initial chloral hydrate (CH) dose of 56.9 +/- 9.3 mg/kg, in 95\% after adding 18.5 +/- 6.4 mg/kg CH and in 96\% after adding second drug. Compared to nonrespondents, first CH dose respondents were younger and lower in weight. The CH side effects were few and mild. Chloral hydrate is a safe and effective agent for sedation in children with an age and weight dependent response. [\hyperlink{Trospium Chloride}{PMID: 15951862}, Omar M Hijazi et al., 2005]

\hypertarget{pmid_27367297}{A} severe clinical phenotype along with concern for ensuring normal growth and development has a major impact on treatment choices for children newly diagnosed with Crohn's disease (CD). We review the increasingly outdated concept of 'conventional' therapy of pediatric CD based on aminosalicylates, corticosteroids, and immunomodulators for patients at high risk of complicated disease. Key safety concerns with each treatment are reviewed. There are minimal data supporting the use of aminosalicylates in the treatment of pediatric CD. Corticosteroids are effective short-term for improving signs and symptoms of disease but are ineffective for maintenance therapy. Thiopurines decrease corticosteroid dependence but may not alter progression to complicated disease requiring surgery. Concerns for lymphoma as well as hemophagocytic lymphohistiocytosis with thiopurines are valid. Further data are required on the efficacy and safety of methotrexate as an alternative immunomodulator. Though generally well tolerated and efficacious in most patients, anti-TNF-α therapy can be associated with both mild as well as more serious complications. Current data do not support an increased risk for malignancy associated with anti-TNF therapy alone in children. Anti-adhesion therapy appears to have a favorable safety profile but the experience in children is extremely limited. [\hyperlink{Trospium Chloride}{PMID: 27367297}, Victoria Grossi et al., 2016]

\section*{Verapamil Hydrochloride}
\subsection*{Result}
\subsubsection*{Answer}

Children (1 year to 17 years): Yes  
Infants (<1 year): Unknown  
Neonates (newborns): Unknown  

\subsubsection*{{Explanation}}
\hypertarget{Verapamil Hydrochloride}
A review of the available abstracts reveals several studies and case series that specifically address the use of verapamil hydrochloride in children, with varying age ranges and indications. Here is a summary of the relevant evidence:

1. Children (15 days to 10 years): A study evaluated the efficacy and safety of verapamil in 22 children (15 days to 10 years old) for supraventricular tachycardia. Out of 47 episodes, 44 were successfully converted to sinus rhythm with a single dose of verapamil, and no significant side effects were observed. The authors concluded that intravenous verapamil is effective and safe for this indication in children [\hyperlink{pmid_3688811}{PMID: 3688811}, K Y Chan et al., 1987].

2. Infants (newborns): There are case reports and reviews indicating both successful use and severe adverse events (such as cardiovascular collapse) after intravenous verapamil in infants, especially when administered rapidly or in those with compromised cardiovascular status. One case report described severe heart failure and shock in a newborn after a therapeutic dose, despite successful arrhythmia conversion, with full recovery after intensive support [\hyperlink{pmid_7309534}{PMID: 7309534}, E Abinader et al., 1981]. Another review highlights that the contraindication in infants is based on limited evidence and that adverse events are associated with rapid administration or pre-existing compromise [\hyperlink{pmid_23800976}{PMID: 23800976}, Martin J Lapage et al., 2013]. A recent case series and literature review challenge the blanket contraindication in infants, reporting three infants/young children (8 days to 2 years) with verapamil-sensitive ventricular tachycardia who were safely and successfully treated with slow intravenous infusion, without incident. The authors propose that, with extreme caution and slow infusion, verapamil can be used safely in this specific context [\hyperlink{pmid_31005896}{PMID: 31005896}, Jascha Kehr et al., 2019].

3. Children (15 days to 17 years): A pharmacokinetic study in 22 children (15 days to 17 years) on chronic oral verapamil for arrhythmias or cardiomyopathy found that adverse reactions (bradycardia, AV block, hypotension) occurred in four children, with high plasma levels or contributing factors (very young age, heart disease) implicated. This suggests the need for careful monitoring, especially in the youngest and those with heart disease [\hyperlink{pmid_7725379}{PMID: 7725379}, D Piovan et al., 1995].

4. Children (age not specified, but includes young children): Case reports and small studies describe successful use of verapamil in children with arrhythmias and hypertrophic cardiomyopathy, with some adverse events but generally supporting its use with appropriate monitoring [\hyperlink{pmid_6681534}{PMID: 6681534}, R L Spicer et al., 1983; \hyperlink{pmid_9093053}{PMID: 9093053}, J S Ma et al., 1997].

5. Children with asthma: Two studies evaluated inhaled verapamil in asthmatic children and found minimal or partial protective effects, with no major safety concerns reported [\hyperlink{pmid_3505526}{PMID: 3505526}, A L Boner et al., 1987; \hyperlink{pmid_3105472}{PMID: 3105472}, A L Boner et al., 1987].

6. Children (≥1 year and <18 years): A retrospective study on antiarrhythmic drugs for frequent PVCs and asymptomatic VT included verapamil but did not focus on safety outcomes; no major safety signals were reported [\hyperlink{pmid_33515328}{PMID: 33515328}, Robin A Bertels et al., 2021].

Summary by age range:
- In children older than 1 year, multiple studies support the safety of verapamil for arrhythmias when used with appropriate dosing and monitoring.
- In infants (<1 year), the safety is controversial: while some reports document severe adverse events, others show safe use with slow infusion in specific arrhythmias (verapamil-sensitive VT) and with extreme caution.
- In neonates (newborns), there are reports of both successful use and severe adverse events, indicating a higher risk and the need for extreme caution.

Overall, targeted studies affirm safety in children (especially >1 year) for certain indications, but in infants and newborns, the evidence is mixed and suggests a higher risk, with safety only established in very specific circumstances and with careful administration.

\subsection*{Abstracts}
\hypertarget{pmid_6762530}{V}erapamil hydrochloride, a prototype calcium antagonist, is now marketed in the United States for the acute treatment of supraventricular tachyarrhythmias and for chronic management of vasospastic and chronic stable angina. It inhibits the slow inward channel in in the heart and blocks calcium influx in smooth muscle. Its intrinsic negative inotropic action, which is apparent in isolated tissues, is offset in vivo by peripheral vasodilation. It has a mild, noncompetitive sympathetic antagonist effect; its most important electrophysiologic action is a depression of AV nodal conduction, accounting for its effect in supraventricular tachyarrhythmias. Its hemodynamic actions are characterized by a complex interplay of changes in preload, afterload, contractility, heart rate, and coronary blood flow. It does not depress cardiac function, except in severe heart failure. The drug has a mild dilator action on coronary arteries and reverses ergonovine-induced vasoconstriction. Controlled trials have established its role in Prinzmetal's variant angina, unstable angina, and chronic stable angina. It has also been found to be effective in obstructive cardiomyopathies. The potential role of verapamil in such conditions as hypertension, cardioprotection, and Raynaud's phenomenon needs further evaluation; at present these indications have not been approved by the Food and Drug Administration. The most common side effects include constipation, skin rash, and dizziness; AV block, heart failure, and sinus arrest may occasionally be encountered, especially when ventricular function is compromised or conduction system disease is present. [\hyperlink{Verapamil Hydrochloride}{PMID: 6762530}, S H Baky et al., ]

\hypertarget{pmid_6650465}{V}erapamil hydrochloride is an organic calcium antagonist that is known to decrease the contraction of smooth muscle. The purpose of our study was to determine if verapamil has a similar effect on the resting lower esophageal sphincter pressure in normal subjects and in patients with achalasia. Esophageal manometry was performed using a continuously perfused catheter assembly. Infusion of verapamil (0.15 mg/kg) over a 2-min period resulted in a statistically significant decrease in resting lower esophageal sphincter pressure in both normal subjects (n = 8) and patients with achalasia (n = 7) within 10 min postinfusion. This study suggests that verapamil may have potential as a drug therapy in treating the clinical symptoms of achalasia and diffuse esophageal spasm. [\hyperlink{Verapamil Hydrochloride}{PMID: 6650465}, B S Becker et al., 1983]

\hypertarget{pmid_3688811}{T}he efficacy of verapamil in the conversion of 47 episodes of supraventricular tachycardia in 22 children was evaluated. The age of the patients ranged from 15 days to 10 years. Tachycardia was the main mode of presentation. Ten out of 22 children had viral infections. Two patients developed mild cardiac failure. Six patients had underlying cardiac abnormalities. Forty-four out of 47 episodes of supraventricular tachycardia were converted to sinus rhythm by a single dose of verapamil (0.11 +/- 0.08 mg/kg). No significant side-effects were observed. Intravenous verapamil is an effective and safe drug for the conversion of supraventricular tachycardia in children. [\hyperlink{Verapamil Hydrochloride}{PMID: 3688811}, K Y Chan et al., 1987]

\hypertarget{pmid_6614542}{V}erapamil hydrochloride is a calcium entry blocking drug that is being prescribed with increasing frequency for cardiovascular disorders in the perioperative setting. Verapamil's calcium channel blocking effect is not selective, because it also exerts activity on the sodium channel. Because of the well-described effects of sodium channel blockers on anesthetic requirements, the authors studied the MAC for halothane in dogs before and after a therapeutic dose of verapamil 0.5 mg . kg-1. There was a 25\% reduction in halothane MAC from 0.97-0.72\% (P less than 0.01) when a therapeutic plasma level of verapamil (64 ng . ml-1) was present. Anesthetic requirements for halothane are reduced by dl-verapamil possibly on the basis of its local anesthetic-like sodium channel blocking properties. Adjustments in anesthetic dosage may be necessary in patients receiving verapamil. [\hyperlink{Verapamil Hydrochloride}{PMID: 6614542}, M Maze et al., 1983]

\hypertarget{pmid_12611158}{V}erapamil hydrochloride, a calcium blocker from a group of phenyl alkylamines, was tested for its effect on central hemodynamics (CH) and blood oxygen-transporting function (BOTF) in 14 patients with arterial hypertension after surgical myocardial revascularization. CH and BOTF were studied by using a Swan-Hanz catheter and directly measuring blood pressure (BP). There was a significant reduction in BPmean, total peripheral vascular resistance index, left ventricular stroke outcome index, and oxygen delivery index. Verapamil in an average dose of 80.4 +/- 18.02 mg at the injection rate of 24.6 +/- 3.9 micrograms/kg/min was shown to make BPmean normal 16.8 +/- 6.35 min later. The agent is comparable with other calcium blockers, such as nifedipine and isradipine in its action on CH and BOTF, as well as in its efficiency and safety. [\hyperlink{Verapamil Hydrochloride}{PMID: 12611158}, A V Matiunin et al., ]

\hypertarget{pmid_3505526}{T}he protective effect of 5 mg/2 ml and 10 mg/4 ml of the calcium antagonist verapamil on methacholine challenge and exercise were evaluated in two groups of asthmatic children. Saline solution was used as placebo. No significant differences were seen in baseline pulmonary function in and within groups. There was a minimal but significant bronchodilation 30 minutes after inhalation of verapamil 5 mg/2 ml. The drug did not reduce methacholine sensitivity at any dosage. After exercise, verapamil 5 mg/2 ml showed a significant change in the maximum percentage drop in forced expiratory volume in 1 second (FEV1) compared with placebo. This was not the case for the group treated with a double dosage. At this time, there is no evidence for a major role of verapamil in the treatment of childhood asthma. [\hyperlink{Verapamil Hydrochloride}{PMID: 3505526}, A L Boner et al., 1987]

\hypertarget{pmid_9093053}{I}n young children with incessant ventricular tachycardia and severe ventricular dysfunction, the management of tachycardia with conventional antiarrhythmic drugs remains a major therapeutic challenge because most of these drugs can further depress myocardial function. We report a four year old boy with verapamil responsive incessant ventricular tachycardia and severe ventricular dysfunction in whom oral verapamil treatment eliminated both the arrhythmia and the picture of dilated cardiomyopathy. On oral verapamil, the patient remains asymptomatic without recurrence of the ventricular tachycardia over a follow up period of 10 months. [\hyperlink{Verapamil Hydrochloride}{PMID: 9093053}, J S Ma et al., 1997]

\hypertarget{pmid_4064384}{V}erapamil has become a popular antiarrhythmic drug for the acute management of supraventricular tachycardia in infants and children. A full-term, 1-hour-old infant presented with supraventricular tachycardia and hypotension that did not respond to vagal maneuvers and direct current cardioversion. After intravenous verapamil, the heart rate slowed and the underlying rhythm was atrial flutter. [\hyperlink{Verapamil Hydrochloride}{PMID: 4064384}, A Casta et al., 1985]

\hypertarget{pmid_6340050}{V}erapamil is a slow-channel calcium-blocking agent that has been released recently by the Food and Drug Administration for treatment of cardiac dysrhythmias in all age groups. Its primary action is to slow conduction in the atrioventricular node, thereby abolishing those types of supraventricular tachycardia using the atrioventricular node as a part of the reentry circuit or slowing the ventricular rate in someone with atrial flutter. Other investigations have shown that it is likely to relieve left ventricular outflow obstruction in patients with hypertrophic obstructive cardiomyopathy. Because of its potential widespread usefulness in the pediatric population, all pediatricians should be more aware of how it is used and the potential hazards. The methods of administration and treatment of overdoses as well as indications for usage, contraindications, and adverse reactions will be explained. [\hyperlink{Verapamil Hydrochloride}{PMID: 6340050}, C J Porter et al., 1983]

\hypertarget{pmid_7309534}{T}he effectiveness and lack of undesirable side-effects has made Verapamil the drug of choice in the treatment of paroxysmal supraventricular tachycardia in infants without underlying heart disease. The case described demonstrates the occasional severe negative inotropic effect of the drug, independent of its influence on heart rate and conduction. Severe heart failure and shock ensued after a therapeutic dose of i.v. Verapamil in a newborn suffering from atrial flutter with no associated heart disease. Although the arrhythmia was promptly converted to sinus rhythm, the baby required two hours of cardiopulmonary resuscitation and inotropic support. Follow-up during the first year of life revealed a normal healthy baby. Attention to the hemodynamic status in addition to continuous ECG monitoring is mandatory during i.v. Verapamil administration also in patients without underlying heart disease. [\hyperlink{Verapamil Hydrochloride}{PMID: 7309534}, E Abinader et al., 1981]

\hypertarget{pmid_7725379}{V}erapamil and norverapamil trough plasma levels were measured in 22 children, aged from 15 days to 17 years, under chronic oral treatment with the drug (mean daily dose +/- SD: 4.9 +/- 1.4 mg/kg) for supraventricular tachyarrhythmias (n = 20) or hypertrophic cardiomyopathy (n = 2). Overall, 67 determinations were available (1 to 11 per patient) and the mean concentration values (+/- SD) were 43.3 +/- 36.4 ng/ml for verapamil and 41.7 +/- 28.9 ng/ml for norverapamil. Verapamil and norverapamil trough concentrations were correlated with the daily dose (p < 0.05) but a wide intersubject variability was present at any given dose and the regression line did not pass through the origin of axes (x-axis intercept: 1.2 mg/kg for verapamil, 0.9 mg/kg for norverapamil). To study the influence of age on drug kinetics, verapamil plasma concentrations corrected by daily dose/kg ([V]/D) and norverapamil to verapamil concentration ratios (N/V) (taken as an index of metabolic clearance) were divided according to age quartiles. The median [V]/D was higher in the first and in the fourth age quartile than in the other two age groups. On the contrary, median N/V ratio increased with age, suggesting that drug metabolism was improving during the first year of life. Four children developed typical adverse reactions to the drug (bradycardia, AV block, hypotension). In one case verapamil plasma levels were definitely high (294 ng/ml). In the other three cases, concomitant factors (such as very young age and heart disease) seem to have contributed to drug toxicity. [\hyperlink{Verapamil Hydrochloride}{PMID: 7725379}, D Piovan et al., 1995]

\hypertarget{pmid_6681534}{T}he acute hemodynamic effects of verapamil were evaluated in nine children with hypertrophic cardiomyopathy. Verapamil, 0.1 mg/kg, was administered as an i.v. bolus over 2 minutes, followed by a 20-minute continuous infusion of 0.007 mg/kg/min. Hemodynamic measurements were obtained at rest in nine patients and at maximal supine bicycle exercise in seven before and 15 minutes after verapamil. At rest, verapamil increased the mean cardiac output from 3.3 +/- 0.9 to 3.7 +/- 0.9 l/min/m2 (+/- SD) (p less than 0.02) and decreased left ventricular end-diastolic pressure from 19.3 +/- 8.1 to 14.5 +/- 6.9 mm Hg (p less than 0.006). In six patients with resting left ventricular outflow tract obstruction, the systolic pressure gradient decreased from 17.5 +/- 7.2 to 5.2 +/- 4.5 mm Hg (p less than 0.04). Repeat supine bicycle exercise testing after verapamil showed increases in total work performed (1743 +/- 1284 to 3168 +/- 1643 kg-m, p less than 0.006) and maximal cardiac index during exercise (6.5 +/- 1.3 to 7.8 +/- 1.8 l/min/m2, p less than 0.05), and decreases in maximal exercise left ventricular end-diastolic pressure (29.1 +/- 10.1 to 19.3 +/- 10.4 mm Hg, p less than 0.002) and left ventricular systolic outflow tract gradient (31.2 +/- 10.5 to 1.75 +/- 1.7 mm Hg, p less than 0.04). These results suggest that verapamil may be an effective therapeutic agent for the treatment of hypertrophic cardiomyopathy in children. [\hyperlink{Verapamil Hydrochloride}{PMID: 6681534}, R L Spicer et al., 1983]

\hypertarget{pmid_23306960}{V}ertebral hemangioma (VH) is an exceedingly rare neoplasm in pediatric population with less than 10 cases reported in the literature. It is usually asymptomatic in adults and diagnosed incidentally at radiographic investigations of other medical conditions. In this report, we describe two children who presented to our institution with severe back pain and were diagnosed with VH. Case 1 was an 8-year-old male with a pain score of 10 out of 10 at presentation. Clinical investigations eliminated the possibility of a neoplasm or infectious process and MRI findings were highly suggestive of an aggressive vertebral hemangioma. Case 2 was a 17-year-old female who presented with back pain radiating to shoulders. Her pain score was 4 out of 10 and she was diagnosed with vertebral hemangioma due to the specific findings on MRI studies. Both patients received propranolol with a dose of 20 and 40 mg per day, respectively. They were free of pain at 2 months follow-up. There are different invasive treatment modalities for the management of VH, including vertebroplasty, kyphoplasty, radiotherapy, alcohol injection, embolization, and surgery. These methods have been used in adult patients for several years, but each of them has potential risks which make these options unsuitable for children. Propranolol is a beta blocker which is safely used in the management of infantile hemangiomas. This is the first report demonstrating its efficacy in symptomatic treatment of childhood VH. The lesions did not show any regression, but the pain relief obtained was very significant under propranolol therapy. [\hyperlink{Verapamil Hydrochloride}{PMID: 23306960}, Didem Uzunaslan et al., 2013]

\hypertarget{pmid_18540545}{I}n the absence of a general anaesthetic facility for MRI scanning in children, we introduced a sedation protocol using chloral hydrate. Our aim was to evaluate the success and safety of our protocol. This was a retrospective study enrolling 36 children over a 7 month period. The overall success rate was 86\% with no child experiencing respiratory complications. In those less than one year, the success rate was 100\%, aged 1-5 years 91\%, with 50\% successful at 80 mg/kg and 50\% at 100 mg/kg dose. For children greater than 5 years of age the success rate was 70\%. 92\% of developmentally normal children and 83\% of developmentally delayed children were successfully sedated. Success rates were poorer in children older than 5 years and in those with developmental delay. Our findings suggest that this protocol could be safely used in units where general anaesthetic facilities are unavailable for MRI and for other radiological investigations. [\hyperlink{Verapamil Hydrochloride}{PMID: 18540545}, E Low et al., 2008]

\hypertarget{pmid_31005896}{G}uidelines state that verapamil is contraindicated in infants. This is based on reports of cardiovascular collapse and even death after rapid intravenous administration of verapamil in infants with supraventricular tachycardia (SVT). We wish to challenge this contraindication for the specific indication of verapamil sensitive ventricular tachycardia (VSVT) in infants. Retrospective case series and critical literature review. Hospitals within New Zealand. We present a series of three infants/young children with VSVT or 'fascicular VT'. Three children aged between 8 days and 2 years presented with tachycardia 200-220 beats per minute with right bundle brunch block and superior axis. Adenosine failed to cardiovert and specialist review diagnosed VSVT. There were no features of cardiovascular shock. Verapamil was given as a slow infusion over 10-30 min (rather than as a push) and each successfully cardioverted without incident. Critical review of the literature reveals that cardiovascular collapses were associated with a rapid intravenous push in cardiovascularly compromised infants and/or infants given other long-acting antiarrhythmics prior to verapamil. Verapamil is specifically indicated for the treatment of fascicular VT, and for this indication should be used in infancy, as well as in older children, as first-line treatment or after failure of adenosine raises suspicion of the diagnosis. We outline how to distinguish this tachycardia from SVT and propose a strategy for the safe intravenous slow infusion of verapamil in children, noting that extreme caution is necessary with pre-existing ventricular dysfunction. [\hyperlink{Verapamil Hydrochloride}{PMID: 31005896}, Jascha Kehr et al., 2019]

\hypertarget{pmid_28741653}{C}hloral hydrate is commonly used to sedate children for painless procedures. Children may recover more quickly after sedation with dexmedetomidine, which has a shorter half-life. We randomly allocated 196 children to chloral hydrate syrup 50 mg.kg [\hyperlink{Verapamil Hydrochloride}{PMID: 28741653}, V M Yuen et al., 2017] Chloral hydrate (CH) is an oral sedative widely used to sedate infants and young children during auditory brainstem response (ABR) testing. The aim of this study was to record effectiveness, complications and safety of CH as a sedative for ABR. From January of 2003 until December of 2007, 1903 children were tested for ABR, 568 of them being under the age of 6 months. CH (8\%) was used for sedation at a dose of 40 mg/kg with a repeat dose, if necessary, for an adequate sedation, in 20-30 min. We recorded the effectiveness of CH as a sedative for ABR examination, as well as all complications related to the use of CH such as vomiting, rash, hyperactivity, respiratory distress and apnea. The statistical method used was the absolute and percentage frequency distribution of the occurrences. Sedation with CH was necessary to perform testing in 1591 (83.6\%) of the examined children. However, in the population of the examined infants, only 341 (60\%) were sedated with CH, because the remaining 227 (40\%) fell asleep by themselves. Complications included hyperactivity in 152 children (8\%), minor respiratory distress in 10 children (0.4\%), vomiting in 217 children (11.4\%), apnea in 4 children (0.2\%) and rash in 10 children (0.4\%). The complications of hyperactivity, vomiting and rash resolved without any medical treatment. The apnea cases were managed effectively by supplying ventilation to the children via a mask in the presence of an anesthesiologist. The use of CH at a dose of 40 mg/kg up to 80 mg/kg is safe and effective when administered in a setting with adequate equipment and the presence of well trained personnel. [\hyperlink{Verapamil Hydrochloride}{PMID: 28741653}, Eirini Avlonitou et al., 2011]

\hypertarget{pmid_3105472}{F}ifteen children with asthma underwent challenges with methacholine on separate days after double blind administration by nebuliser of either verapamil (5 mg), cromoglycate (20 mg), or saline (placebo). The provocation doses that produced a 20\% fall in forced expiratory volume in one second (PD20) were analysed. There was variation in the protective effects of verapamil and cromoglycate among the patients. Although cromoglycate produced an increase in PD20 in 53\% of the children tested, the protection was not significant when compared with the placebo. Verapamil was partially protective, however, in 80\% of children and achieved significantly better results than the placebo. We suggest that this is likely to be due to a direct effect on bronchial smooth muscle. [\hyperlink{Verapamil Hydrochloride}{PMID: 3105472}, A L Boner et al., 1987]

\hypertarget{pmid_16781498}{C}hildren frequently suffer infections accompanied by fever, which is commonly treated with acetaminophen (paracetamol), a use not devoid of risk. The effect of a complex homeopathic medicine (Viburcol, Heel Belgium, Gent, Belgium) was compared with that of acetaminophen in children with infectious fever. Non-randomized observational study. Thirty-eight Belgian centers practicing homeopathy and conventional medicine. Children <12 years old. Viburcol (drops) or acetaminophen (pills, capsules, or liquid form) for a maximum of 2 weeks. Fever, cramps, distress, disturbed sleep, crying, and difficulties with eating or drinking. Symptoms were graded by the practitioner on a scale from 0 to 3. Severity of infection was evaluated on a scale from 0 to 4. Data were captured on body temperature, subjective impression of health status, time to first improvement of symptoms, and global evaluation of treatment effects. Tolerability and compliance were monitored. Both treatment groups improved during the treatment period. Body temperature was reduced by 1.7 degrees C +/- 0.7 degrees C with Viburcol and by 1.9 degrees C +/- 0.9 degrees C with acetaminophen; fever score (scale from 0 to 3) from 1.7 +/- 0.6 to 0.1 +/- 0.2 with Viburcol and from 1.9 +/- 0.7 to 0.2 +/- 0.5 with acetaminophen (all values mean +/-SD). The overall severity of infection (scale from 0 to 4) decreased from 2.0 +/- 0.5 to 0.0 +/- 0.2 with Viburcol and from 2.2 +/- 0.7 to 0.2 +/- 0.6 with acetaminophen. There were no statistically significant differences between treatment groups in time to symptomatic improvement. Viburcol was noninferior to acetaminophen on all variables evaluated. Both treatments were very well tolerated, but the Viburcol group had a significantly greater number of patients with the highest tolerability score. In this patient population, Viburcol was an effective alternative to acetaminophen treatment and significantly better tolerated. [\hyperlink{Verapamil Hydrochloride}{PMID: 16781498}, Mireille Derasse et al., 2005]

\hypertarget{pmid_21258839}{U}se of high doses of verapamil in preventive treatment of cluster headache (CH) is limited by cardiac toxicity. We systematically assess the cardiac safety of the very high dose of verapamil (verapamil VHD) in CH patients. Our work was a study performed in two French headache centers (Marseilles-Nice) from 12/2005 to 12/2008. CH patients treated with verapamil VHD (≥720 mg) were considered with a systematic electrocardiogram (EKG) monitoring. Among 200 CH patients, 29 (14.8\%) used verapamil VHD (877±227 mg/day). Incidence of EKG changes was 38\% (11/29). Seven (24\%) patients presented bradycardia considered as nonserious adverse event (NSAE) and four (14\%) patients presented arrhythmia (heart block) considered as serious adverse event (SAE). Patients with EKG changes (1,003±295 mg/day) were taking higher doses than those without EKG changes (800±143 mg/day), but doses were similar in patients with SAE (990±316 mg/day) and those with NSAE (1,011±309 mg/day). Around three-quarters (8/11) of patients presented a delayed-onset cardiac adverse event (delay ≥2 years). Our work confirms the need for systematic EKG monitoring in CH patients treated with verapamil. Such cardiac safety assessment must be continued even for patients using VHD without any adverse event for a long time. [\hyperlink{Verapamil Hydrochloride}{PMID: 21258839}, M Lanteri-Minet et al., 2011]

\hypertarget{pmid_2326439}{T}his paper reports on 350 pediatric patients who were studied over a 17-month period to determine the efficacy and safety of oral and intramuscular sedation techniques. The protocol using oral chloral hydrate, 50 mgm/kg, for infants under 1 year of age or intramuscular pentobarbital, 5 mgm/kg, for children over 1 year was found to be an effective, safe and fairly simple approach to pediatric sedation. Of the 350 sedated patients, 343 (98 percent) had satisfactory scans on the same day the examination was scheduled after a single dose or an initial dose and supplementary sedation. [\hyperlink{Verapamil Hydrochloride}{PMID: 2326439}, J B Temme et al., ]

\hypertarget{pmid_23800976}{T}he use of intravenous verapamil for tachyarrhythmia in infants is widely considered contraindicated due to the perceived risk of hemodynamic collapse after administration. This article reviews the relatively limited evidence that led to this well-known contraindication and highlights the interesting process by which medical practice may evolve in the absence of persuasive science.  [\hyperlink{Verapamil Hydrochloride}{PMID: 23800976}, Martin J Lapage et al., 2013] To determine the safety and efficacy of high-dose oral chloral hydrate for pediatric ophthalmic procedures. This study is a retrospective review of a quality audit of pediatric sedation for ophthalmic evaluation and imaging performed at King Khaled Eye Specialist Hospital between January 1 and December 31, 2011, in children aged 1 month to 6 years. Three hundred fifty-eight of 380 (94.2\%) sedation procedures were successful after a single dose of chloral hydrate, with 356 of 380 (93.7\%) children sedated within 45 minutes of the first dose. The total success rate of the sedation procedure increased to 97.9\% (372 of 380) when a second dose was administered. Children adequately sedated after a single dose of chloral hydrate were on average younger and weighed less than children who required additional doses. No major adverse events were documented. The use of chloral hydrate sedation for ophthalmic evaluation and imaging was safe and effective in this patient population with a high rate of procedure completion. [\hyperlink{Verapamil Hydrochloride}{PMID: 23800976}, Michelle E Wilson et al., ]

\hypertarget{pmid_33515328}{T}he aim of the study is to compare the efficacy of flecainide, beta-blockers, sotalol, and verapamil in children with frequent PVCs, with or without asymptomatic VT. Frequent premature ventricular complexes (PVCs) and asymptomatic ventricular tachycardia (VT) in children with structurally normal hearts require anti-arrhythmic drug (AAD) therapy depending on the severity of symptoms or ventricular dysfunction; however, data on efficacy in children are scarce. Both symptomatic and asymptomatic children (≥ 1 year and < 18 years of age) with a PVC burden of 5\% or more, with or without asymptomatic runs of VT, who had consecutive Holter recordings, were included in this retrospective multi-center study. The groups of patients receiving AAD therapy were compared to an untreated control group. A medication episode was defined as a timeframe in which the highest dosage at a fixed level of a single drug was used in a patient. A total of 35 children and 46 medication episodes were included, with an overall change in PVC burden on Holter of -4.4 percentage points, compared to -4.2 in the control group of 14 patients. The mean reduction in PVC burden was only significant in patients receiving flecainide (- 13.8 percentage points; N = 10; p = 0.032), compared to the control group and other groups receiving beta-blockers (- 1.7 percentage points; N = 18), sotalol (+ 1.0 percentage points; N = 7), or verapamil (- 3.9 percentage points; N = 11). The efficacy of anti-arrhythmic drug therapy on frequent PVCs or asymptomatic VTs in children is very limited. Only flecainide appears to be effective in lowering the PVC burden. [\hyperlink{Verapamil Hydrochloride}{PMID: 33515328}, Robin A Bertels et al., 2021]

\hypertarget{pmid_22093944}{T}o examine whether three cycles of a low-intensity chemotherapy consisting of cyclophosphamide [500 mg/m(2) - day 1], vinblastine [6 mg/m(2) - days 1 and 8] and prednisolone [40 mg/m(2) - days 1-7] (CVP) is safe and therapeutically effective in children and adolescents with early stage nodular lymphocyte predominant Hodgkin lymphoma [nLPHL]. Fifty-five children and adolescents with early stage nLPHL [median age 13 years, range 4-17 years] diagnosed between June 2005 and October 2010 in the UK and France are the subjects of this report. Staging investigations included conventional cross sectional as well as 18 fluro-deoxyglucose [FDG] PET imaging. Histology was confirmed as nLPHL by an expert pathology panel. Of the 45 patients, who received CVP as first line treatment, 36 [80\%, 95\% Confidence Interval [CI]: (68; 92)] either achieved a complete remission [CR] or CR unconfirmed [CRu], the remaining nine patients achieved a partial response. All nine subsequently achieved CR with salvage chemotherapy [n=7] or radiotherapy [n=2]. Ten patients received CVP at relapse after primary treatment that consisted of surgery alone and all achieved CR. To date, only three patients have relapsed after CVP chemotherapy and all had received CVP as first line treatment at initial diagnosis. The 40-month freedom from treatment failure and overall survival for the entire cohort were 75.4\% (SE ± 6\%) and 100\%, respectively. No significant early toxicity was observed. Our results show that CVP is an effective chemotherapy regimen in children and adolescents with early stage nLPHL that is well tolerated with minimal acute toxicity. [\hyperlink{Verapamil Hydrochloride}{PMID: 22093944}, Ananth Shankar et al., 2012]

\section*{Odevixibat}
\subsection*{Result}
\subsubsection*{Answer}

Ages 6 years: Yes  
Ages (pediatric, likely 1–18 years, but not specified): Yes  
Ages <6 months: Unknown  
Ages ≥6 months and ≥3 months: Unknown (regulatory approval noted, but no specific safety data in abstracts)

\subsubsection*{{Explanation}}
\hypertarget{Odevixibat}
Based on the abstracts available, several provide targeted data on the safety of Odevixibat in children:

1. A case report describes a 6-year-old girl with PFIC9 treated with Odevixibat. The treatment led to significant clinical improvements (reduction in serum bile acids, pruritus, and sleep disturbances) and an increase in BMI z-score. Importantly, "No adverse drug events were recorded." The authors conclude that Odevixibat was effective and safe in this pediatric patient, but they note that further studies on a larger scale are needed [\hyperlink{pmid_36865697}{PMID: 36865697}, Angela Pepe et al., 2023].

2. A case series of five children with TJP2 deficiency (PFIC-4) treated with Odevixibat found that the drug was "well tolerated" and led to significant reductions in serum bile acids and improvements in pruritus. No safety concerns are reported in the abstract [\hyperlink{pmid_37499899}{PMID: 37499899}, Angelo Di Giorgio et al., 2023].

3. A drug approval summary notes that Odevixibat received approval in the EU for PFIC in patients aged ≥6 months and in the USA for pruritus in PFIC patients aged ≥3 months. However, this abstract does not provide specific safety data from pediatric studies [\hyperlink{pmid_34499340}{PMID: 34499340}, Emma D Deeks et al., 2021].

4. A review article states that "oral odevixibat was also studied in children with cholestatic liver disease" and that it is approved for use in children as young as 3 or 6 months (depending on region), but does not provide specific safety outcomes from pediatric trials [\hyperlink{pmid_36892028}{PMID: 36892028}, Mayur Porwal et al., 2023].

Summary by age range:
- Ages 6 years: A case report supports safety in a 6-year-old child with PFIC9 [\hyperlink{pmid_36865697}{PMID: 36865697}].
- Ages unspecified (but pediatric): A case series of five children (exact ages not specified) with TJP2 deficiency supports safety [\hyperlink{pmid_37499899}{PMID: 37499899}].
- Ages ≥6 months and ≥3 months: Regulatory approval is noted, but without specific safety data in the abstract [\hyperlink{pmid_34499340}{PMID: 34499340}, 36892028].

Conclusion:
There is targeted evidence from case reports and case series supporting the safety of Odevixibat in children with PFIC and related disorders, including at least one child aged 6 years and a small group of children (ages not specified) with TJP2 deficiency. However, large-scale, controlled safety studies in broader pediatric populations or in children under 6 months are not described in the abstracts. Therefore, Odevixibat can be considered safe for use in children with PFIC and related disorders based on limited but affirmative targeted pediatric data for ages 6 years and for a small group of children (likely in the 1–18 year range, but not specified). For children under 6 months, safety is unknown based on the available abstracts.

\subsection*{Abstracts}
\hypertarget{pmid_36865697}{O}devixibat, an ileal bile acid transporter (IBAT) inhibitor, is effective for the treatment of pruritus in children diagnosed with progressive familial intrahepatic cholestasis (PFIC) type 1 and 2. There are no studies showing the efficacy of Odevixibat in children with different subtypes of PFIC. We describe the case of a 6-year-old girl with chronic cholestatic jaundice. In the last 12 months laboratory data showed high serum levels of bilirubin (total bilirubin x 2.5 ULN; direct bilirubin x 1.7 ULN) and bile acids (sBA x 70 ULN), elevated transaminases (x 3-4 ULN), and preserved synthetic liver function. Genetic testing showed homozygous mutation in ZFYVE19 gene, which is not included among the classic causative genes of PFIC and determined a new non-syndromic phenotype recently classified as PFIC9 (OMIM \# 619849). Due to the persistent intensity of itching [score of 5 (very severe) at the Caregiver Global Impression of Severity (CaGIS)] and sleep disturbances not responsive to rifampicin and ursodeoxycholic acid (UDCA), Odevixibat treatment was started. After treatment with odevixibat we observed: (i) reduction in sBA from 458 to 71 μmol/L (absolute change from baseline: -387 μmol/L), (ii) reduction in CaGIS from 5 to 1, and (iii) resolution of sleep disturbances. The BMI z-score progressively increased from -0.98 to +0.56 after 3 months of treatment. No adverse drug events were recorded. Treatment with IBAT inhibitor was effective and safe in our patient suggesting that Odevixibat may be potentially considered for the treatment of cholestatic pruritus also in children with rare subtypes of PFIC. Further studies on a larger scale could lead to the increasing of patients eligible for this treatment. [\hyperlink{Odevixibat}{PMID: 36865697}, Angela Pepe et al., 2023]

\hypertarget{pmid_36892028}{O}devixibat is synthesized through chemical modification of Benzothiazepine's structure. It is a tiny chemical that inhibits the ileal bile acid transporter and is used to treat a variety of cholestatic illnesses, including progressive familial intrahepatic cholestasis (PFIC). For cholestatic pruritus and liver disease development, bile acid transporter inhibition is a unique treatment strategy. Odevixibat reduces enteric bile acid reuptake. Oral odevixibat was also studied in children with cholestatic liver disease. Odevixibat received its first approval in the European Union (EU) in July 2021 for the treatment of PFIC in patients aged 6 months, followed by approval in the USA in August 2021 for the treatment of pruritus in PFIC patients aged 3 months. Bile acids in the distal ileum can be reabsorbed by the ileal sodium/bile acid cotransporter, a transport glycoprotein. Odevixibat is a sodium/bile acid co-transporter reversible inhibitor. An average 3 mg once-daily dose of odevixibat for a week resulted in a 56\% reduction in the area under the curve of bile acid. A daily dose of 1.5 mg resulted in a 43\% decrease in the area under the curve for bile acid. Odevixibat is also being evaluated in many countries for the treatment of other cholestatic illnesses, including Alagille syndrome and biliary atresia. This article reviews the updated information on odevixibat with respect to its clinical pharmacology, mechanism of action, pharmacokinetics, pharmacodynamics, metabolism, drug-drug interactions, pre-clinical studies, and clinical trials. [\hyperlink{Odevixibat}{PMID: 36892028}, Mayur Porwal et al., 2023]

\hypertarget{pmid_37499899}{T}here are no published data on the use of odevixibat, a selective ileal bile acid transporter (IBAT) inhibitor, in children with tight junction protein 2 (TJP2) deficiency (also named as PFIC-4). We describe a case series of five children treated with odevixibat. After treatment, serum bile acids (sBA) decreased compared to baseline [mean value: 244 (±125), vs 38 (±34) µmol/L; p = 0.007]; reduction in sBA was >70\% from baseline (or <70 µmol/L) in all. Improvements in pruritus were reported in all patients. The drug was well tolerated. IBAT inhibitors should be considered a valuable treatment option in patients with TJP2 deficiency. [\hyperlink{Odevixibat}{PMID: 37499899}, Angelo Di Giorgio et al., 2023]

\hypertarget{pmid_33590471}{L}evetiracetam (LEV) and oxcarbazepine (OXC) are commonly used in the treatment of epilepsy, but their efficacy and safety have seldom been compared for the treatment of children with benign epilepsy with centrotemporal spikes (BECTS). We thus assessed the efficacy of LEV and OXC monotherapy in the treatment of children with BECTS, and the effect of this treatment on children's intelligence and cognitive development. This was a randomized, single-center trial. Children with BECTS were randomized (1:1) into LEV and OXC groups, and were assessed at 1, 3 and 6 months after treatment. The primary outcomes were the frequency of seizures and changes in intelligence and cognitive function. Secondary outcomes were electroencephalogram (EEG) results and safety. Seventy children were enrolled and randomized to the LEV group or the OXC group, and 32 of the 35 children in each group completed the study. After 6 months, the effective treatment rate of the OXC group was significantly higher than that of the LEV group (78.12 vs. 53.12\%, p = 0.035). However, no significant inter-group difference was observed in EEG improvement (p = 0.211). In terms of intelligence and cognitive development, children in the OXC group exhibited significantly improved choice reaction time, mental rotation, and Wisconsin Card Sorting Test results (all p < 0.05). Both LEV and OXC were well tolerated, with 18.75 and 21.88\% of children reporting mild adverse events (p = 0.756). OXC monotherapy was more effective than LEV for children with BECTS. In addition, children with OXC monotherapy had higher improvements in children's intelligence and cognitive function than those with LEV monotherapy. [\hyperlink{Odevixibat}{PMID: 33590471}, Gui-Hai Suo et al., 2021]

\hypertarget{pmid_32258344}{O}zenoxacin is a topical antibiotic approved in the United States for treatment of impetigo in adults and children age ≥2 months. This analysis evaluated the efficacy and safety of ozenoxacin in specific pediatric age groups. Data for children aged 2 months to <18 years recruited from eight countries who had participated in phase 1 and 3 trials of ozenoxacin were extracted and analyzed by age range. Across studies, 644 pediatric patients with impetigo received ozenoxacin 1\% cream (n = 287) or vehicle (n = 247). One study included retapamulin 1\% ointment as the internal validity control (n = 110). The clinical success rate at the end of treatment and bacterial eradication rates after 3 to 4 days of treatment and at the end of treatment were significantly higher with ozenoxacin than vehicle (all  The results of this analysis suggest that ozenoxacin 1\% cream is an effective and safe treatment for impetigo in pediatric patients aged 2 months to <18 years. [\hyperlink{Odevixibat}{PMID: 32258344}, Adelaide A Hebert et al., 2020]

\hypertarget{pmid_34499340}{O}devixibat (Bylvay™) is a small molecule inhibitor of the ileal bile acid transporter being developed by Albireo Pharma, Inc. for the treatment of various cholestatic diseases, including progressive familial intrahepatic cholestasis (PFIC). In July 2021, odevixibat received its first approval in the EU for the treatment of PFIC in patients aged ≥ 6 months, followed shortly by its approval in the USA for the treatment of pruritus in patients aged ≥ 3 months with PFIC. Odevixibat is also in clinical development for the treatment of other cholestatic diseases, including Alagille syndrome and biliary atresia, in various countries. This article summarizes the milestones in the development of odevixibat leading to this first approval for PFIC. [\hyperlink{Odevixibat}{PMID: 34499340}, Emma D Deeks et al., 2021]

\hypertarget{pmid_17055681}{T}o evaluate the efficacy and tolerability of levetiracetam or oxcarbazepine as monotherapy in children with newly diagnosed benign epilepsy with centrotemporal spikes (BECTS). Twenty-one children (11 males, 10 females), aged between 5 and 13 years (mean 10.5 years), and 18 (10 M, 8 F), aged between 3.3 and 14 years (mean 8.4 years), were randomised to receive monotherapy with levetiracetam or oxcarbazepine, respectively. LEV was titrated up to 20-30 mg/kg/once or twice a day, and OXC up to 20-35 mg/kg once or twice a day. Thirty-nine consecutive children (21 males, 18 females), aged between 3.3 and 14 years (mean 10.7 years), were recruited into the study. Twenty-one were randomised on LEV (11 male, 10 female; mean age 10.5 years), and 18 on OXC (10 male, 8 female; mean age 8.4 years). After a mean follow-up period of 18.5 months (range 12-24 months), 19 out of 21 patients (90.5\%) on levetiracetam, and 13 out of 18 (72,22\%) on oxcarbazepine did not have further seizures. Mean serum level of LEV was 4.1 microg/ml (range 1.3-9.0), and of OXC was 15.2 microg/ml (range 4.2-27.5). Adverse side effects on LEV were reported in 3 children (14.3\%), represented by mild and transient decreased appetite (2) and cephalalgia (1). They were reported on OXC in 2/18 (11.1\%), including headache (1), and sedation (1). These preliminary data from an open, parallel group study suggest that levetiracetam and oxcarbazepine may be potentially effective and well tolerated drugs for children with BECTS who require treatment. [\hyperlink{Odevixibat}{PMID: 17055681}, Giangennaro Coppola et al., 2007]

\hypertarget{pmid_16257311}{F}ive independent, multicentered, double-masked, parallel, controlled studies were conducted to determine the safety of moxifloxacin ophthalmic solution 0.5\% (VIGAMOX) in pediatric and nonpediatric patients with bacterial conjunctivitis. Patients were randomized into one of two treatment groups in each study and received either moxifloxacin ophthalmic solution 0.5\% b.i.d. or t.i.d. or a comparator. A total of 1,978 patients (918 pediatric and 1,060 nonpediatric) was evaluable for safety. The most frequent adverse event in the overall safety population was transient ocular discomfort, occurring at an incidence of 2.8\%, which was similar to that observed with the vehicle. No treatment-related changes in ocular signs or visual acuity were observed with moxifloxacin ophthalmic solution 0.5\%, except for one clinically relevant change in visual acuity. Thus, based upon a review of adverse events and an assessment of ocular parameters, moxifloxacin ophthalmic solution 0.5\% formulated without the preservative, benzalkonium chloride, is safe and well tolerated in pediatric (3 days-17 years of age) and nonpediatric (18-93 years) patients with bacterial conjunctivitis. [\hyperlink{Odevixibat}{PMID: 16257311}, Lewis H Silver et al., 2005]

\hypertarget{pmid_24382900}{A} number of newer anti-epileptic drugs have been developed in the last few years to improve the treatment outcomes in epilepsy. In this review, we discuss the use of newer anti-epileptic drugs in children. MEDLINE search (1966-2013) was performed using terms newer anti-epileptic drugs, Oxcarbazepine, vigabatrin, topiramate, zonisamide, levetiracetam, lacosamide, rufinamide, stiripentol, retigabine, eslicarbazepine, brivaracetam, ganaxolone and perampanel for reports on use in children. Review articles, practice parameters, guidelines, systematic reviews, meta-analyses, randomized controlled trials, cohort studies, and case series were included. The main data extracted included indications, efficacy and adverse effects in children. Oxcarbazepine is established as effective initial monotherapy for children with partial-onset seizures. Vigabatrin is the drug of choice for infantile spasms associated with tuberous sclerosis. Lamotrigine, levetiracetam and lacosamide are good add-on drugs for patients with partial seizures. Lamotrigine may be considered as monotherapy in adolescent females with idiopathic generalized epilepsy. Levetiracetam is a good option as monotherapy for females with juvenile myoclonic epilepsy. Topiramate is a good add-on drug in patients with epileptic encephalopathies such as Lennox-Gastaut syndrome and myoclonic astatic epilepsy. [\hyperlink{Odevixibat}{PMID: 24382900}, Satinder Aneja et al., 2013]

\hypertarget{pmid_16554175}{T}o evaluate the long-term efficacy, tolerability, and safety of oxcarbazepine (OXC) in children with epilepsy. We enrolled 36 patients (median age 7.75) with new diagnosis of partial epilepsy in an open prospective study. All type of epilepsy were included: 25 patients were affected by idiopathic epilepsy, eight by symptomatic epilepsy and three by cryptogenic epilepsy. Patients were then scheduled to come back for controls at 3 months (T1), 12 months (T2) and 24 months (T3) after the beginning of OXC-monotherapy (T0). At each control we evaluated patients through their seizure diary, a questionnaire on side effects, their level of 10-monohydroxy (MHD) metabolite and laboratory analysis. At T1, 21/36 patients (58.3\%) were seizure-free, 3/36 patients (8.3\%) showed an improvement higher than 50\%, 3/36 (8.3\%) lower than 50\%, while 2/36 worsened (5.6\%). In 7/36 (19.5\%) patients, no improvement was reported. At T2 13/18 patients (72.2\%) were seizure-free, 1/18 showed a response to therapy higher than 50\% while 2/18 worsened (11\%). In two patients no improvement was reported. A correspondence between MHD plasmatic levels and clinical response (r=0.49; p<0.05) was only registered at T1. An EEG normalization was observed in 25\% of cases. Side effects were reported in 25\% of cases, but symptoms progressively disappeared at follow-up. We can therefore conclude that OXC can be considered, for its efficacy and safety, as a first line drug in children with epilepsy. [\hyperlink{Odevixibat}{PMID: 16554175}, E Franzoni et al., 2006]

\hypertarget{pmid_10881246}{T}o evaluate the safety and efficacy of oxcarbazepine (OXC) as adjunctive therapy in children with inadequately controlled partial seizures on one or two concomitant antiepileptic drugs (AEDs). OXC has shown antiepileptic activity in several comparative monotherapy trials in newly diagnosed patients with epilepsy, and in a placebo-controlled monotherapy trial in hospitalized patients evaluated for epilepsy surgery. A total of 267 patients were evaluated in a multicenter, randomized, placebo-controlled trial consisting of three phases: 1) a 56-day baseline phase (patients maintained on their current AEDs); 2) a 112-day double-blind treatment phase (patients received either OXC 30-46 mg/kg/day orally or placebo); and 3) an open-label extension phase. Data are reported only from the double-blind treatment phase; the open-label extension phase is ongoing. Children (3 to 17 years old) with inadequately controlled partial seizures (simple, complex, and partial seizures evolving to secondarily generalized seizures) were enrolled. Patients treated with OXC experienced a significantly greater median percent reduction from baseline in partial seizure frequency than patients treated with placebo (p = 0.0001; 35\% versus 9\%, respectively). Forty-one percent of patients treated with OXC experienced a > or =50\% reduction from baseline in partial seizure frequency per 28 days compared with 22\% of patients treated with placebo (p = 0.0005). Ninety-one percent of the group treated with OXC and 82\% of the group treated with placebo reported > or =1 adverse event; vomiting, somnolence, dizziness, and nausea occurred more frequently (twofold or greater) in the group treated with OXC. OXC adjunctive therapy administered in a dose range of 6 to 51 mg/kg/day (median 31.4 mg/kg/day) is safe, effective, and well tolerated in children with partial seizures. [\hyperlink{Odevixibat}{PMID: 10881246}, T A Glauser et al., 2000]

\hypertarget{pmid_16176888}{R}elatively few well-designed studies have demonstrated the long-term safety and tolerability of newer antiepileptic drugs (AEDs) in a large group of children. Extensive clinical data from the worldwide Clinical Development Program (CDP) and a compassionate use program on the safety and tolerability of oxcarbazepine in children are presented. Oxcarbazepine is a newer AED that is indicated for use as monotherapy and adjunctive therapy in children (United States 4 years of age, Europe 6 years of age) with partial epilepsy. The most common adverse events (10\%) in the CDP were headache (32.5\%), somnolence (31.5\%), vomiting (27.6\%), and dizziness (23.1\%), whereas in the compassionate use program (clinical practice situation), the most common adverse events (1\%) reported were rash (2.7\%), fatigue (1.6\%), nausea (1.2\%), and somnolence (1.2\%). These data indicate that oxcarbazepine has a good long-term safety and tolerability profile, whether given as monotherapy or adjunctive therapy, in children with partial seizures. [\hyperlink{Odevixibat}{PMID: 16176888}, Blaise F D Bourgeois et al., 2005]

\hypertarget{pmid_31958794}{O}zenoxacin is a topical antibiotic approved in Europe to treat non-bullous impetigo in adults and children aged ≥6 months. This analysis evaluated the efficacy and safety of ozenoxacin in paediatric patients by age group. Pooled data for patients aged 6 months to <18 years who had participated in a phase I or in two phase III clinical trials of ozenoxacin 1\% cream were analysed by age group: 0.5-<2, 2-<6, 6-<12, and 12-<18 years. The combined population comprised 529 patients with non-bullous impetigo treated with ozenoxacin (n = 239), vehicle (n = 201), or retapamulin as internal validation control (n = 89). Studies were well matched for extent and severity of impetigo and therapeutic schedule (twice daily application for 5 days). The clinical success rate after 5 days' treatment (day 6-7, end of therapy), and microbiological success rates after 3-4 days' treatment and at the end of therapy, were significantly higher with ozenoxacin than vehicle (p < 0.0001 for all comparisons). Clinical and bacterial eradication rates were higher with ozenoxacin than vehicle in each age group. No safety concerns were identified with ozenoxacin. One (0.3\%) of 327 plasma samples exceeded the lower limit of quantification for ozenoxacin, but the low concentration indicated negligible systemic absorption. This combined analysis supports the efficacy and safety of ozenoxacin administered twice daily for 5 days. Ozenoxacin 1\% cream is a new option to consider for treatment of non-bullous impetigo in children aged 6 months to <18 years. [\hyperlink{Odevixibat}{PMID: 31958794}, Antonio Torrelo et al., 2020]

\hypertarget{pmid_26228441}{T}he purpose of this study was to investigate the safety and effectiveness of an intravitreally injected dexamethasone-containing implant (Ozurdex(®)) in the treatment of uveitis in children. The study group included ten patients (14 eyes) aged 6.5-15 years (mean age 12 years) with intermediate or posterior uveitis who were treated with the Ozurdex implant at two tertiary medical centers between 2009 and 2014, following an insufficient response to standard uveitis therapy. All were followed for at least 6 months (mean 12.2 ± 4.9 months). Clinical data before and after treatment were collected retrospectively from the medical files. Outcome measures were best corrected visual acuity, vitreous haze, and macular thickness. Ocular complications were documented. Visual acuity improved in 12 eyes (86 \%) and intraocular inflammation decreased in 13 eyes (93 \%) from 1 week to 3 months after the first injection. Macular edema decreased in all eyes from 1 month to 3 months after the first injection. Five patients underwent repeated injections because of an increase in macular thickness at 3-6 month follow-up. Complications included cataract progression in one patient after one injection and cataract formation in two patients after two injections, and an elevation in intraocular pressure in two patients who responded well to topical treatment. Both single and repeated injections of a dexamethasone-containing implant are safe and effective for the treatment of noninfectious intermediate and posterior uveitis in children. The duration of the beneficial effect may be limited. More data are required to establish the safety profile of the implant in the pediatric age group. [\hyperlink{Odevixibat}{PMID: 26228441}, Ruti Sella et al., 2015]

\hypertarget{pmid_34344494}{C}hildren with cardiac diseases suffer from significant morbidity and mortality secondary to thromboembolic complications. Anticoagulant agents currently used for thromboprophylaxis have many limitations, including subcutaneous administration (low molecular weight heparins) and requirement for frequent monitoring via venipuncture (vitamin K antagonists). Edoxaban is an oral direct factor Xa inhibitor without need of monitoring. In the treatment of venous thromboembolism in adults, edoxaban has shown to be effective and safe.This manuscript summarises the rationale and design of a phase 3, open-label, randomised controlled trial to evaluate and compare the safety and efficacy of edoxaban against standard of care (namely, vitamin K antagonist and low molecular weight heparin) in children with cardiac diseases.A goal of 150 children with cardiac diseases at risk of thromboembolic complications who need primary or secondary anticoagulant prophylaxis will be recruited. Eligible children between 6 months and <18 years of age will be randomised in a ratio of 2 to 1 for edoxaban versus standard of care. Randomisation will be stratified based on underlying cardiac disease and concomitant use of aspirin for patients other than Kawasaki disease. The primary outcome will be safety, comprised of major and clinically relevant non-major bleeding in first 3 months of treatment. Bleeding beyond 3 months, symptomatic and asymptomatic thromboembolic events, and pharmacokinetic and pharmacodynamic parameters will be evaluated as secondary outcomes.Randomised controlled anticoagulation trials are challenging in children. This study will evaluate a potentially valuable alternative of oral anticoagulant prophylactic use in children with cardiac diseases. [\hyperlink{Odevixibat}{PMID: 34344494}, Mihir D Bhatt et al., 2021]

\hypertarget{pmid_12910331}{T}he prodrug oseltamivir has been shown to be efficacious and safe for the treatment of influenza for patients 1 year of age or older; however, pharmacokinetic information was lacking for children below 5 years of age. This study was conducted to assess the metabolic and excretory capacity of oseltamivir and its active carboxylate metabolite in young children. Twelve healthy children aged 1-5 years received a single oral suspension dose of oseltamivir (45 mg for 3-5 years, 30 mg for 1-2 years). Plasma and urine concentrations of oseltamivir and the carboxylate were determined by means of liquid chromatography/tandem mass spectrometry. Mean peak plasma concentration and area under the plasma concentration-time curve values normalized to milligram per kilogram oseltamivir dose in the 1- to 2-year group are lower than those in the 3- to 5-year group. Mean body weight normalized oral clearance of oseltamivir and its carboxylate in younger subjects aged 1-2 years (259 ml/min/kg and 12.2 ml/min/kg) were, respectively, 52\% and 30\% higher than those in older subjects aged 3-5 years (170 ml/min/kg and 9.4 ml/min/kg). The results demonstrate that infants as young as 1 year old can metabolize and excrete oseltamivir efficiently. The data derived from this study provide the starting dose of oseltamivir for further investigation in an efficacy study among influenza-infected infants less than 1 year of age. [\hyperlink{Odevixibat}{PMID: 12910331}, Charles Oo et al., 2003]

\hypertarget{pmid_9003923}{T}he safety and efficacy of the MRI contrast medium gadodiamide injection (OMNISCAN) in children is summarised. Four open and three double-blind, multinational, multicentre comparative trials have been undertaken. Overall, 353 patients (15 days to 18 years, plus one 21 years) received gadodiamide injection, and 128 (2-18 years) received gadopentetate dimeglumine (Magnevist), intravenously at 0.1 mmol/kg body weight, to aid the identification of CNS and body lesions. Adverse events were reported in 13 (4\%) patients given gadodiamide injection and 8 (6\%) given gadopentetate dimeglumine; few patients reported injection-associated discomfort. The post-contrast scan gave more diagnostic information in 223 (63\%) patients given gadodiamide injection (CNS and body indications). In the comparative trials, the post-contrast scan gave more diagnostic information for 91 (65\%) and 82 (64\%) patients given gadodiamide injection and gadopentetate dimeglumine, respectively (CNS indications only). Gadodiamide injection (0.1 mmol/kg body weight) was safe and effective in infants, children and adolescents. [\hyperlink{Odevixibat}{PMID: 9003923}, B Lundby et al., 1996]

\hypertarget{pmid_16948931}{S}tudies designed specifically for the pediatric population are needed to assess the tolerability and safety of the new antiepileptic drugs. The purpose of this study was to document the safety, ease of dosing, and acceptance of oxcarbazepine oral suspension in pediatric patients in monotherapy and polytherapy. A prospective, multicenter, open-label study was conducted at the neurology services of three pediatric university hospitals over 12 months. After obtaining signed informed consent, we enrolled a series of 62 patients with epilepsy aged between 2 months and 14 years who began oxcarbazepine treatment in monotherapy or in combination with other antiepileptic drugs to assess the seizure frequency, safety (adverse events), and acceptance of the pharmaceutical form by the patient's family. Fifty patients (80.6\%) reduced seizures by at least 50\%, 44 (71\%) saw a reduction in seizure frequency of over 75\%, and 29 (46.8\%) were seizure free at the end of the study. The difference in the number of seizures before and after the study was statistically significant, both overall and by type of pathology. Adverse events occurred in four patients (6.4\%) and required withdrawal of the drug in two cases (skin rash); three patients (4.8\%) withdrew for inefficacy. Five patients (8.1\%) withdrew from the treatment. We concluded that, in this series of patients, oxcarbazepine in oral suspension form was seen to help reduce seizure frequency, to have few side effects, and to be accepted by parents and patients. [\hyperlink{Odevixibat}{PMID: 16948931}, Miguel Rufo-Campos et al., 2006]

\hypertarget{pmid_28293110}{T}o assess the efficacy and safety of oxcarbazepine (OXC) in the treatment of children with epilepsy. Randomized controlled trials (RCTs) published in PubMed, Embase, Web of Science, Cochrane Library, Scopus, SinoMed (Chinese BioMedical Literature Service System, China), and Chinese National Knowledge Infrastructure (China) database were systematically reviewed. Eligible studies were those that compared the efficacy and safety of OXC with other antiepileptic drugs in epilepsy. Risk ratio (RR) with 95\% confidence intervals (95\% CIs) was calculated using fixed-effects or random-effects model. Eleven RCTs with a total of 1,241 patients met the inclusion criteria and were included in this meta-analysis. Compared with other antiepileptic drugs (sodium valproate, levetiracetam, phenytoin, and placebo), OXC was associated with similar seizure-free rate (RR =1.06, 95\% CI: 0.94, 1.20;  OXC showed similar effects and safety as other antiepileptic drugs in the treatment of children with epilepsy. Further well-conducted, large-scale RCTs are needed to validate these findings. [\hyperlink{Odevixibat}{PMID: 28293110}, Hua Geng et al., 2017]

\hypertarget{pmid_23676933}{T}o investigate the clinical efficacy and safety of oxcarbazepine (OXC) suspension in children with focal epilepsy. A total of 118 children aged 2-14 years, who were newly diagnosed with focal epilepsy between October 2009 and December 2011, were randomly divided into experimental group (n=60) and control group (n=58). The experimental group was treated with an orally suspension of OXC and the control group was orally administered with carbamazepine (CBZ) tablets. The two treatment regimens were compared in terms of clinical efficacy and safety. After 13 and 26 weeks of treatment, the experimental group had response rates of 75\% and 72\% respectively and seizure-free rates of 53\% and 50\%, and the control group had response rates of 71\% and 66\% and seizure-free rates of 50\% and 43\% respectively. There were no significant differences in the clinical efficacy between the two groups (P>0.05). After 26 weeks of treatment, the adverse event rates of the experimental and control groups were 18\% and 40\% respectively, with a significant difference between the two groups (P<0.05). OXC suspension has a comparable clinical efficacy to that of CBZ tablets in children aged 2-14 years who are newly diagnosed with focal epilepsy, but OXC suspension causes fewer adverse events and has higher safety. [\hyperlink{Odevixibat}{PMID: 23676933}, Yin-Bo Chen et al., 2013]

\hypertarget{pmid_29745239}{O}zenoxacin is a nonfluorinated quinolone antibacterial approved for topical treatment of impetigo. Because quinolones have known chondrotoxic effects in juvenile animals, the potential toxicity of ozenoxacin was assessed in preclinical studies. Ozenoxacin or ofloxacin (300 mg/kg/day for 5 days, for each compound) was orally administered to juvenile rats, and oral ozenoxacin (10-100 mg/kg/day for 14 days) was administered to juvenile dogs. In juvenile rats, ozenoxacin showed no chondrotoxicity, whereas ofloxacin produced typical quinolone-induced lesions in articular cartilage in three of ten rats. Oral ozenoxacin administration to juvenile dogs showed no chondrotoxicity or toxicologically relevant findings in selected target organs. Ozenoxacin was generally well-tolerated in juvenile rats and dogs, with no evidence of quinolone-induced arthropathy. [\hyperlink{Odevixibat}{PMID: 29745239}, Jorge Ignacio González Borroto et al., 2018]

\hypertarget{pmid_15496647}{T}his two-part, open-label study evaluated the pharmacokinetics, safety, and tolerability of oxcarbazepine as combination therapy in 112 children 2 to 12 years old with inadequately controlled epilepsy. Part I was a pharmacokinetic study in children stratified by age (2-5 years and 6-12 years) and randomized to receive a single oxcarbazepine dose of 5 mg/kg or 15 mg/kg. Mean specific AUC and t(1/2) values of the active metabolite (MHD) were approximately 30\% lower in younger children compared with older children, regardless of dose. Part II was a 4-month safety, tolerability, and pharmacokinetic study in which children received oxcarbazepine doses of 11 to 68 mg/kg/day. The mean specific oxcarbazepine daily dose was 38\% higher in younger children compared with older children. Similarly, mean trough plasma MHD concentrations were 34\% lower in younger children. Six (5\%) children discontinued due to adverse events. Oxcarbazepine was safe and well tolerated. Younger children require higher oxcarbazepine doses because of rapid clearance. [\hyperlink{Odevixibat}{PMID: 15496647}, Elisabeth Rey et al., 2004]

\hypertarget{pmid_18018419}{T}o evaluate the efficacy and safety of CLAVAMOX dry syrup (potassium clavulanate/amoxicillin) in children with otitis media, we conducted a postmarketing surveillance from February to September 2006. The analysis was made on the basis of 470 survey sheets collected from 127 medical institutions, of which we investigated 455 cases for safety, and 433 cases for efficacy. The efficacy was 95.2\% in the 433 subjects eligible for the efficacy analysis. The clinical improvement rates for major symptoms (otalgia, otorrhea, flare reaction of drum membrane and fever) were 95\% or more. The efficacies for the three major offending bacteria of otitis media (Streptococcus pneumoniae, Haemophilus influenzae, and Moraxella catarrhalis) were between 94.6\% and 100\%. The efficacies for penicillin-resistant Streptococcus pneumoniae (PRSP) and penicillin intermediate resistant Streptococcus pneumoniae (PISP) were 95\% or more. Adverse drug reactions (ADRs) were reported in 106 (23.3\%) of the 455 subjects eligible for safety analysis. The major ADRs were diarrhea, of which incident was 22.6\% (103 of 455). These ADRs were observed at a higher rate in younger age patients. Most of the diarrhea cases were non-serious, reversible on discontinuation or continuation of the drug. No clinically important serious diarrhea cases such as pseudomembranous colitis or dehydration were observed. Our surveillance results demonstrated that CLAVAMOX dry syrup had excellent efficacy and clinically manageable safety in children with otitis media. These findings indicated that this medicine was clinically-useful in children with otitis media. [\hyperlink{Odevixibat}{PMID: 18018419}, Rinya Sugita et al., 2007]

\hypertarget{pmid_29234055}{W}e evaluated the efficacy and safety of oral immunotherapy (OIT) combined with 24 weeks of omalizumab (OMB) at inducing desensitization in children with cow's milk allergy (CM) compared with an untreated group. The present study was a prospective randomized controlled trial. Sixteen patients (age, 6-14 years) with high IgE levels to CM were enrolled in the present study. Patients were randomized 1:1 to receive OMB-OIT group or untreated group. The primary outcome was the induction of desensitization at 8 weeks after OMB was discontinued in OMB-OIT treated group and at 32 weeks after study entry. None of the 6 children in the untreated group developed desensitization to CM while all of the 10 children in the OIT-OMB treated group achieved desensitization (P < 0.001). A significantly decreased wheal diameter in response to a skin prick test using CM was found in the OMB-OIT treated group (P < 0.05). These data suggest that OIT combined with OMB using microwave heated CM may help to induce desensitization for children with high-risk CM allergy. This prospective randomized controlled trial was intended for 50 participants but was prematurely discontinued due to overwhelming superiority of OMB combined with microwave heated OIT over CM avoidance. [\hyperlink{Odevixibat}{PMID: 29234055}, Masaya Takahashi et al., 2017]

\hypertarget{pmid_20199730}{L}evetiracetam has been widely used for childhood epilepsy, but there is no high quality evidence to support its use. This study performed a systematic review to evaluate the effectiveness and safety of levetiracetam therapy for childhood epilepsy. The papers related to levetiracetam therapy for childhood epilepsy published up to March, 2009 were retrieved electronically from the PubMed, Embase, the Cochrane Library, Chinese Biomedical Database, Wanfang and Weipu Chinese Journals Full-text Database. The relevant papers on randomized control trials (RCTs) or quasi-RCTs were studied by meta analysis. Two papers that met the inclusion criteria were included. The first paper involved 198 patients, including 108 cases in the levetiracetam therapy group and 97 cases in the placebo group. Seven cases (6.9\%) were seizure free in the levetiracetam therapy group compared with 1 case (1\%) in the placebo group (p<0.01) 14 weeks after treatment. Levetiracetam therapy decreased significantly the frequency of seizures compared with the placebo treatment. The second paper involved 39 patients, including 21 cases in the levetiracetam therapy group and 18 cases in the oxcarbazepine therapy group. Nineteen cases (90.5\%) were seizure-free in the levetiracetam therapy group compared with 13 cases (72.2\%) in the oxcarbazepine therapy group (P=0.410) during a follow-up of 12-24 months. The adverse effects in the levetiracetam therapy group were not significantly different from the placebo and the oxcarbazepine therapy groups. The current evidence shows that levetiracetam therapy is effective for childhood epilepsy. However, it needs to be proved by the multi-centre, large sample RCTs. [\hyperlink{Odevixibat}{PMID: 20199730}, Jing Dai et al., 2010]

\section*{Acyclovir}
\subsection*{Result}
\subsubsection*{Answer}

Infants (0–12 months): Yes
Children (1–18 years): Yes
Adolescents (up to 18 years): Yes

\subsubsection*{{Explanation}}
\hypertarget{Acyclovir}
Based on the available abstracts, there are multiple targeted studies evaluating the safety of acyclovir in children across various age ranges. Here is a summary by age group:

Infants (0–12 months, including neonates):
- A study of 89 infants (≤120 days old) treated with high-dose intravenous acyclovir for neonatal herpes simplex virus disease found that adverse events (AEs) were common but usually not severe, and many AEs may be related to the underlying infection rather than acyclovir exposure. No infants developed renal failure requiring dialysis, and the study supports the safety of high-dose acyclovir in this population [\hyperlink{pmid_27977557}{PMID: 27977557}, Jessica E Ericson et al., 2017].
- Another study of 9 infants (including premature infants) treated with intravenous acyclovir for herpes simplex or cytomegalovirus infections found no renal toxicity and all infants survived [\hyperlink{pmid_6285712}{PMID: 6285712}, A S Yeager et al., 1982].
- A pharmacokinetic study in children younger than 2 years (including infants <1 month) found that oral acyclovir was well tolerated, with dosage adjustment by body weight recommended [\hyperlink{pmid_11120958}{PMID: 11120958}, M Tod et al., 2001].
- An open multicenter study in 53 children aged 3–24 months with varicella found acyclovir to have an excellent clinical and laboratory safety profile [\hyperlink{pmid_7629563}{PMID: 7629563}, F Chiodo et al., 1995].

Children (1–18 years):
- Multiple randomized controlled trials and observational studies in children aged 2–18 years with varicella or herpes simplex virus infections found acyclovir to be safe and well tolerated, with no significant increase in adverse effects compared to placebo [\hyperlink{pmid_1944438}{PMID: 1944438}, L M Dunkle et al., 1991; \hyperlink{pmid_2156984}{PMID: 2156984}, H H Balfour et al., 1990; \hyperlink{pmid_8245899}{PMID: 8245899}, S Feldman et al., 1993; \hyperlink{pmid_8545564}{PMID: 8545564}, S Catania et al., 1995; \hyperlink{pmid_8151150}{PMID: 8151150}, H Kamiya et al., 1994; \hyperlink{pmid_1311067}{PMID: 1311067}, K Kuzushima et al., 1992; \hyperlink{pmid_1638653}{PMID: 1638653}, D Bartosová et al., 1992].
- A study of 126 children (mean age 81 months) found that nephrotoxicity was associated with concomitant use of other nephrotoxic drugs or pre-existing renal impairment, but within recommended dosing, age, sex, and dose per kg were not risk factors for nephrotoxicity [\hyperlink{pmid_18345723}{PMID: 18345723}, Ruth Schreiber et al., 2008].
- A study in children aged 1 month to 18 years receiving standard- or high-dose acyclovir for encephalitis found no significant difference in renal injury or failure between dosing groups [\hyperlink{pmid_24497110}{PMID: 24497110}, Jennifer G Kendrick et al., 2014].
- Several studies specifically note that acyclovir was well tolerated, with no significant adverse effects requiring discontinuation [\hyperlink{pmid_2829714}{PMID: 2829714}, W M Sullender et al., 1987; \hyperlink{pmid_6750068}{PMID: 6750068}, C G Prober et al., 1982].

Adolescents (up to 18 years):
- The above studies include adolescents, and no age-specific safety concerns were identified in this group.

Rare Adverse Events:
- There are isolated case reports of neurotoxicity (confusion, agitation, insomnia) and nephrotoxicity, but these are rare and typically resolve after discontinuation of the drug [\hyperlink{pmid_9695630}{PMID: 9695630}, N Nakamoto et al., 1998; \hyperlink{pmid_20307148}{PMID: 20307148}, Gürkan Genc et al., 2010].

Summary:
There is robust evidence from targeted studies affirming the safety of acyclovir in infants (including neonates), children, and adolescents when used within recommended dosing regimens. The drug is generally well tolerated, with adverse events being uncommon and usually mild or related to underlying illness or concomitant medications.

\subsection*{Abstracts}
\hypertarget{pmid_32916823}{A}cyclovir is an antiviral drug used for the treatment of  [\hyperlink{Acyclovir}{PMID: 32916823}, Haniza Hassan et al., 2020] Acyclovir is used to treat herpes simplex virus disease in infants. Treatment with high-dose acyclovir, 60 mg/kg/d, is recommended; however, the safety of this dosage has not been assessed in the past 15 years, and this dosage is not currently approved for infants by the US Food and Drug Administration. We included infants with neonatal herpes simplex virus disease treated with ≥14 days of intravenous acyclovir starting in the first 120 days of life admitted to 1 of 42 neonatal intensive care units managed by the Pediatrix Medical Group between 2002 and 2012. We determined the frequency and proportion of infants with clinical and laboratory adverse events (AEs) as well as the number and proportion of infant days with laboratory AEs occurring during acyclovir exposure. We identified 89 infants during the study period with 1658 days of acyclovir exposure. Almost all received high-dose acyclovir therapy (79/89, 89\%). The most common clinical AEs were hypotension and seizure, both occurring in 9\% of infants. Thrombocytopenia was the most common laboratory AE occurring in 25\% of infants and on 9\% of infant-days. Elevated creatinine occurred in 2\% of infants and 0.2\% of infant-days and no infants developed renal failure requiring dialysis. Overall, 45\% of infants had ≥1 AE. In this cohort of infants treated during the high-dose acyclovir era, AEs were common but usually not severe. Many of the AEs reported in this cohort may be related to the underlying infection rather than due to acyclovir exposure. [\hyperlink{Acyclovir}{PMID: 32916823}, Jessica E Ericson et al., 2017]

\hypertarget{pmid_11120958}{A}cyclovir is approved for the treatment of herpes simplex virus (HSV) and varicella-zoster virus (VZV) infections in children by the intravenous and oral routes. However, its use by the oral route in children younger than 2 years of age is limited due to a lack of pharmacokinetic data. The objectives of the present study were to determine the typical pharmacokinetics of an oral suspension of acyclovir given to children younger than 2 years of age and the interindividual variabilities in the values of the pharmacokinetic parameters in order to support the proposed dosing regimen (24 mg/kg of body weight three times a day for patients younger than 1 month of age or four times a day otherwise). Children younger than age 2 years with HSV or VZV infections were enrolled in a multicenter study. Children were treated for at least 5 days with an acyclovir oral suspension. Plasma samples were obtained at steady state, before acyclovir administration, and at 2, 3, 5, and 8 h after acyclovir administration. Acyclovir concentrations were measured by radioimmunoassay. The data were analyzed by a population approach. Data for 79 children were considered in the pharmacokinetic study (212 samples, 1 to 5 samples per patient). Acyclovir clearance was related to the estimated glomerular filtration rate, body surface area, and serum creatinine level. The volume of distribution was related to body weight. The elimination half-life decreased sharply during the first month after birth, from 10 to 15 h to 2.5 h. Bioavailability was 0.12. The interindividual variability was less pronounced when the parameters were normalized with respect to body weight. Hence, dosage adjustment by body weight is recommended for this population. Simulations showed that the length of time that acyclovir remains above the 50\% inhibitory concentration during a 24-h period was more than 12 h for HSV but not for VZV. The proposed dosing regimen seems adequate for the treatment of HSV infections, while for the treatment of VZV infections, a twofold increase in the dose seems necessary for children older than age 3 months. [\hyperlink{Acyclovir}{PMID: 11120958}, M Tod et al., 2001]

\hypertarget{pmid_32988829}{A}cyclovir is an antiviral currently used for the prevention and treatment of herpes simplex virus (HSV) and varicella-zoster virus (VZV) infections. This study aimed to characterize the pharmacokinetics (PK) of acyclovir and its oral prodrug valacyclovir to optimize dosing in children. Children receiving acyclovir or valacyclovir were included in this study. PK were described using nonlinear mixed-effect modeling. Dosing simulations were used to obtain trough concentrations above a 50\% inhibitory concentration for HSV or VZV (0.56 mg/liter and 1.125 mg/liter, respectively) and maximal peak concentrations below 25 mg/liter. A total of 79 children (212 concentration-time observations) were included: 50 were taking intravenous (i.v.) acyclovir, 22 were taking oral acyclovir, and 7 were taking both i.v. and oral acyclovir, 57 for preventive and 22 for curative purposes. A one-compartment model with first-order elimination best described the data. An allometric model was used to describe body weight effect, and the estimated glomerular filtration rate (eGFR) was significantly associated with acyclovir elimination. To obtain target maximal and trough concentrations, the more suitable initial acyclovir i.v. dose was 10 mg/kg of body weight/6 h for children with normal renal function (eGFR ≤ 250 ml/min/1.73 m [\hyperlink{Acyclovir}{PMID: 32988829}, S Abdalla et al., 2020] Intravenous acyclovir is the treatment of choice for herpes simplex virus encephalitis. In 2006, the American Academy of Pediatrics updated its dosing recommendations for children aged 3 months to 12 years to receive high-dose acyclovir (60 mg/kg/day). The association between acyclovir dose and toxicity is unclear. The purpose of our study was to review our institution's experience with standard- and high-dose acyclovir for the empiric treatment of encephalitis. This retrospective cohort study included patients aged 1 month to 18 years who received acyclovir as empiric treatment for encephalitis between 2005 and 2009 at a tertiary care children's hospital. We excluded patients with baseline renal impairment and those without serum creatinine measurements prior to and during treatment. The main outcome measure of this study was to compare the occurrence of renal injury or failure between children who received the standard- versus high-dose regimen. Sixty-one patients were included (n = 32 standard-dose; n = 29 high-dose). There was no statistical difference in change in serum creatinine from baseline between children who received standard- versus high-dose acyclovir (0 vs. 5.1 \%; p = 0.79). One child in the standard-dose group and three children in the high-dose group developed renal injury or failure during treatment (3.1 vs. 10.3 \%; p = 0.34). Children with renal injury or failure were older, had a longer length of stay, and longer duration of therapy than children without. The incidence of renal injury or failure was similar between children who received standard-dose and high-dose acyclovir. [\hyperlink{Acyclovir}{PMID: 32988829}, Jennifer G Kendrick et al., 2014]

\hypertarget{pmid_8245899}{A}cyclovir has been approved in the United States and elsewhere as antiviral therapy for otherwise healthy children and adolescents with varicella. This development arose from multicentre placebo-controlled trials of acyclovir in normal patients, 2-18 years of age, which showed that the drug accelerated cutaneous healing, and reduced fever and related constitutional symptoms without harmful side effects. Acyclovir did not, however, decrease transmission of chickenpox within the household, nor was there any demonstrable effect of antiviral therapy on varicella complications. In this article, the background and rationale for the multicentre studies of acyclovir in normal paediatric patients with chickenpox is reviewed. The evidence for and against its routine administration within 24 hours of the eruption of skin rash is also discussed. [\hyperlink{Acyclovir}{PMID: 8245899}, S Feldman et al., 1993]

\hypertarget{pmid_8545564}{W}e evaluated safety and tolerance of acyclovir ACV per os in immunocompetent children affected by chicken-pox admitted to our department from January 1993 to December 1994. 183 subjects (102 males and 81 females) aged between 0 and 14 years were treated by ACV (80 mg/kg/daily in 4 divided doses): 88 children were treated within 24 hours and 95 subjects within 48 hours from the onset of symptoms. The control group consisted of 83 children (52 males and 31 females) aged between 0 to 14 years. In all patients routine blood-test were performed and in those with respiratory illness Chest-Rx was also done. We evaluated clinical course, degree of eruption, the appearance and kind of complications, duration of hospitalization, the compliance and the potential consequences on specific antibody response. Our results show a faster improvement of clinical symptoms in treated patients with respect to the control group with shortening of the period of the fever, itch and appearance of new vescicles. The percentage of complications was lower in treated than in untreated patients. 16 cases tested for specific antibody response showed protective titers six months after treatment. In conclusion, ACV administered per os within 48 hours from onset of exanthema causes reduction of the period and the degree of general symptoms and exanthema, a lower incidence of complications even if non statistically significant. The drug is safe and well-tolerated. [\hyperlink{Acyclovir}{PMID: 8545564}, S Catania et al., ]

\hypertarget{pmid_12353186}{A}n extensive clinical trial program combined with 5 years' postmarketing experience with valacyclovir provides evidence of favorable safety and efficacy in herpes simplex virus (HSV) management. Valacyclovir enhances acyclovir bioavailability compared with orally administered acyclovir. Long-term use of acyclovir for up to 10 years for HSV suppression is effective and well tolerated. Acyclovir is also approved for use in children, is available in some countries over the counter in cream formulation for herpes labialis, and has been monitored in over 1000 pregnancies. Safety monitoring data from clinical trials of valacyclovir, involving over 3000 immunocompetent and immunocompromised persons receiving long-term therapy for HSV suppression, were analyzed. Safety profiles of valacyclovir (</=1000 mg/day), acyclovir (800 mg/day), and placebo were similar. Extensive sensitivity monitoring of HSV isolates confirmed a very low rate of acyclovir resistance among immunocompetent subjects (<0.5\%). The incidence of resistance among immunocompromised patients remains low at about 5\%. [\hyperlink{Acyclovir}{PMID: 12353186}, Stephen K Tyring et al., 2002]

\hypertarget{pmid_12356336}{A}cyclovir has the potential to shorten the course of chickenpox which may result in reduced costs and morbidity. We conducted a systematic review of randomised controlled trials that evaluated acyclovir for the treatment of chickenpox in otherwise healthy children. MEDLINE, EMBASE, and the Cochrane Library were searched. The reference lists of relevant articles were examined and primary authors and Glaxo Wellcome were contacted to identify additional trials. Two reviewers independently screened studies for inclusion, assessed study quality using the Jadad scale and allocation concealment, and extracted data. Continuous data were converted to a weighted mean difference (WMD). Overall estimates were not calculated due to differences in the age groups studied. Three studies were included. Methodological quality was 3 (n = 2) and 4 (n = 1) on the Jadad scale. Acyclovir was associated with a significant reduction in the number of days with fever, from -1.0 (95\% CI -1.5,-0.5) to -1.3 (95\% CI -2.0,-0.6). Results were inconsistent with respect to the number of days to no new lesions, the maximum number of lesions and relief of pruritus. There were no clinically important differences between acyclovir and placebo with respect to complications or adverse effects. Acyclovir appears to be effective in reducing the number of days with fever among otherwise healthy children with chickenpox. The results were inconsistent with respect to the number of days to no new lesions, the maximum number of lesions and the relief of itchiness. The clinical importance of acyclovir treatment in otherwise healthy children remains controversial. [\hyperlink{Acyclovir}{PMID: 12356336}, Terry P Klassen et al., 2002]

\hypertarget{pmid_9695630}{W}e reported a 5-year-old boy with acute encephalitis due to suspected herpes simplex infection, who developed confusion, agitation and insomnia during intravenous administration of acyclovir. He recovered from these neuro-psychiatric symptoms two days after the cessation of acyclovir. The same symptoms recurred two days after its re-administration and resolved on the next day of the second cessation of the drug. Electroencephalogram (EEG) showed periodic lateralized epileptiform discharges (PLEDs) on hospital day 16, which disappeared on hospital day 27, suggesting that neurotoxicity of acyclovir may induce PLEDs. Although acyclovir is useful for the treatment of herpes simplex and varicella-zoster virus infections, we have to pay attention to its neurotoxicity. [\hyperlink{Acyclovir}{PMID: 9695630}, N Nakamoto et al., 1998]

\hypertarget{pmid_2829714}{E}ighteen children from 3 weeks to 6.9 years of age were given an oral acyclovir suspension for herpes simplex or varicella-zoster virus infections. Thirteen patients who were 6 months to 6.9 years old received 600 mg/m2 per dose, and three infants and two children less than 2 years old were given 300 mg/m2 per dose. The drug was given four times a day, except to one infant who was treated with three doses a day. Among the 13 children who received the 600-mg/m2 dose, the maximum concentration in plasma (Cmax) was 0.99 +/- 0.38 microgram/ml (mean +/- standard deviation), the time to maximum concentration (Tmax) was 3.0 +/- 0.86 h, the area under the curve (AUC) was 5.56 +/- 2.17 micrograms.h/ml, and the elimination half-life (t1/2) was 2.59 +/- 0.78 h. The three infants less than 2 months of age who received the 300-mg/m2 dose had a Cmax of 1.88 +/- 1.11 micrograms/ml, a Tmax of 4.10 +/- 0.48 h, an AUC of 6.54 +/- 4.32 micrograms.h/ml, and a t1/2 of 3.26 +/- 0.33 h. The acyclovir suspension was well tolerated by young children. No adverse effects requiring discontinuation of the drug occurred. [\hyperlink{Acyclovir}{PMID: 2829714}, W M Sullender et al., 1987]

\hypertarget{pmid_8151150}{T}he efficacy and safety of aciclovir granules (containing 40\% w/w aciclovir) were evaluated in the treatment of chickenpox in otherwise healthy children. Patients presenting with chickenpox received aciclovir granules at a dose of 20 mg/kg four times daily for five to seven days. Overall 51 children received treatment with aciclovir. A further 53 patients receiving conventional symptomatic therapy acted as a control. In the aciclovir group the overall efficacy rate was 92.2\%. There were reductions in the numbers of lesions, fever, itching and the duration of symptoms. No adverse experiences were reported. Overall this formulation of aciclovir appears to be a safe and effective treatment for chickenpox in this patient population. However the need for anti-viral therapy in otherwise healthy children is still the subject of debate and it might be appropriate to identify sub-groups for whom such therapy is justified. [\hyperlink{Acyclovir}{PMID: 8151150}, H Kamiya et al., 1994]

\hypertarget{pmid_18345723}{A}ciclovir is the drug of choice for severe systemic herpes virus infections. Nephrotoxicity is one of the clinically significant adverse effects of this drug, but studies examining nephrotoxicity in children are scarce. To identify risk factors for aciclovir-associated nephrotoxicity in the pediatric population. A retrospective review was conducted on all children (mean age 81 months; n = 126 [74 boys]) who were treated with aciclovir in a tertiary center between July 2005 and January 2006 and who met our inclusion criteria. Glomerular filtration rate (GFR) was calculated on the first day of treatment and at the peak measured creatinine level while on therapy, using Schwartz's method. Aciclovir therapy was associated with a significant increase in serum creatinine levels and a parallel decrease in GFR (n = 93; both p <or= 0.0001). Children with immunosuppression who received a variety of other nephrotoxic drugs exhibited more severe nephrotoxicity than those not receiving nephrotoxic drugs. In multiple regression analysis, the use of nephrotoxic drugs (p = 0.02) and impaired GFR at baseline (p = 0.04) were predictive for nephrotoxicity. Within the recommended age-dependent dosage schedule of aciclovir there was no effect of dose per kg, age, or sex on nephrotoxicity. The predictors of aciclovir nephrotoxicity were the concomitant use of nephrotoxic drugs and impaired GFR at baseline. [\hyperlink{Acyclovir}{PMID: 18345723}, Ruth Schreiber et al., 2008]

\hypertarget{pmid_17935955}{A}cyclovir is a synthetic nucleoside analogue active against herpes viruses. Exposure during human pregnancy and during the neonatal period seems safe. We report a case of early necrotizing enterocolitis in a full term infant treated with acyclovir as a prophylactic therapy. The mother had herpes genitalis with preterm, premature ruture of membranes at 32 weeks of gestational age and was treated with acyclovir until vaginal delivery. Acyclovir treatment in utero and after birth is discussed as a possible cause of necrotizing enterocolitis in the infant. Acyclovir should be used only if its benefit outweighs the potential risk to the baby. [\hyperlink{Acyclovir}{PMID: 17935955}, N Montjaux-Régis et al., 2007]

\hypertarget{pmid_20307148}{A}cyclovir is an effective, frequently used antiviral agent. Adverse effects of this drug are well known and are especially seen with high doses and/or dehydration. In this article, we report a 6-year-old boy with leukemia with nonoliguric acute renal failure in normal hydration status after using acyclovir treatment. He had no preexisting renal impairment, and there were no additional symptoms. Dimercaptosuccinic acid radionucleid scyntigraphy and other laboratory findings revealed impairment of proximal tubule function, in addition to distal tubule. We emphasize that renal functions should be monitored carefully during treatment with acyclovir, and asymptomatic nephrotoxicity must be kept in mind. [\hyperlink{Acyclovir}{PMID: 20307148}, Gürkan Genc et al., 2010]

\hypertarget{pmid_20014952}{V}alacyclovir provides enhanced acyclovir bioavailability in adults, but limited data are available in children. Children 1 month through 5 years of age with or at risk for herpesvirus infection received a single 25 mg/kg dose of extemporaneously compounded valacyclovir oral suspension (n = 57), whereas children 1 through 11 years of age received 10 mg/kg valacyclovir oral suspension twice daily for 3-5 days (herpes simplex virus infection) (n = 28) or 20 mg/kg 3 times daily for 5 days (varicella-zoster virus infection) (n = 27). Blood samples for pharmacokinetic analysis were collected during the 6 h after the first dose. Safety was monitored throughout the studies. Dose proportionality in the maximum observed concentration (C(max)) of acyclovir and the area under the concentration-time curve from time zero extrapolated to infinity (AUC(0-infinity)) existed across the 10 to 20 mg/kg valacyclovir dose range. For children 2 through 5 years of age, an increase in dose from 20 to 25 mg/kg resulted in near doubling of the C(max) and AUC(0-infinity). Among infants 1 through 2 months of age receiving 25 mg/kg, the mean AUC(0-infinity) and C(max) were higher ( approximately 60\% and 30\%, respectively) than those among older infants and children receiving the same dose. Valacyclovir oral suspension was well tolerated. No clinically significant trends were noted in clinical chemical, hematologic, or urinalysis values from screening to follow-up. Among children 3 months through 11 years of age, the 20 mg/kg dose of this formulation of valacyclovir oral suspension produces favorable acyclovir blood concentrations and is well tolerated. A dosing recommendation cannot be made for infants <3 months of age because of decreased clearance in this age group. Trial registration. ClinicalTrials.gov identifier: NCT00297206 . [\hyperlink{Acyclovir}{PMID: 20014952}, David W Kimberlin et al., 2010]

\hypertarget{pmid_1944438}{C}hickenpox, the primary infection caused by the varicella-zoster virus, affects more than 3 million children a year in the United States. Although usually self-limited, chickenpox can cause prolonged discomfort and is associated with infrequent but serious complications. To evaluate the effectiveness of acyclovir for the treatment of chickenpox, we conducted a multicenter, double-blind, placebo-controlled study involving 815 healthy children 2 to 12 years old who contracted chickenpox. Treatment with acyclovir was begun within the first 24 hours of rash and was administered by the oral route in a dose of 20 mg per kilogram of body weight four times daily for five days. The children treated with acyclovir had fewer varicella lesions than those given placebo (mean number, 294 vs 347; P less than 0.001), and a smaller proportion of them had more than 500 lesions (21 percent, as compared with 38 percent with placebo; P less than 0.001). In over 95 percent of the recipients of acyclovir no new lesions formed after day 3, whereas new lesions were forming in 20 percent of the placebo recipients on day 6 or later. The recipients of acyclovir also had accelerated progression to the crusted and healed stages, less itching, and fewer residual lesions after 28 days. In the children treated with acyclovir the duration of fever and constitutional symptoms was limited to three to four days, whereas in 20 percent of the children given placebo illness lasted more than four days. There was no significant difference between groups in the distribution of 11 disease complications (10 bacterial skin infections and 1 case of transient cerebellar ataxia). Acyclovir was well tolerated, and there was no significant difference between groups in the titers of antibodies against varicella-zoster virus. Acyclovir is a safe treatment that reduces the duration and severity of chickenpox in normal children when therapy is initiated during the first 24 hours of rash. Whether treatment with acyclovir can reduce the rare, serious complications of chickenpox remains uncertain. [\hyperlink{Acyclovir}{PMID: 1944438}, L M Dunkle et al., 1991]

\hypertarget{pmid_9412400}{A}moxyclav (amoxycillin/potassium clavulanate, A/PC) was used in the treatment of 55 children with acute bronchitis and pneumonia. The drug was administered in a dose of 20-40 mg/kg body weight a day in 3 portions. The treatment course was 4 to 10 days. The treatment was performed under careful clinicoroent-genologic control. The clinical picture of the disease in the children was characterized by a moderate process which made it possible to treat the children as outpatients. The clinical efficacy amounted to 90.5 per cent. The isolates of Streptococcus pneumoniae, Streptococcus pyogenes, Staphylococcus aureus and Haemophilus influenzae proved to be susceptible to A/PC. It may be used as the 1st class agent in the treatment of children with lower respiratory tract infection. [\hyperlink{Acyclovir}{PMID: 9412400}, B M Blokhin et al., 1997]

\hypertarget{pmid_7629563}{A}n open multicenter study has been carried out to evaluate efficacy and tolerability of oral acyclovir in the treatment of varicella in immunocompetent patients in the first two years of life. Fifty-three children aged 3-24 months received acyclovir at 80 mg/Kg/day in four divided doses for 4 to 6 days; 24 of them were treated in the first 24 hours following disease onset, while the remaining 29 patients were enrolled within 48 hours. The assessment of evolution of disease signs and symptoms showed a rapid resolution of fever, itching and other constitutional symptoms, with interruption of vesicle formation and acceleration of cutaneous healing processes. No statistically significant differences have been demonstrated as to disease progression between patients treated in the first 24 hours, when compared with subjects receiving acyclovir in the following 24 hours. Acyclovir confirmed its excellent clinical and laboratory safety profile. By acting favorably on both the duration and severity of disease signs and symptoms, acyclovir treatment should be recommended in young children and infants with varicella, since a higher incidence of severe and complicated disease has been observed in these patient groups. [\hyperlink{Acyclovir}{PMID: 7629563}, F Chiodo et al., 1995]

\hypertarget{pmid_1638653}{T}he authors submit information on the course and therapeutic experience with acyclovir (Zovirax Wellcome Co. and Herpesin Lachema Co.) in 67 children with eczema herpeticatum (EH) who were hospitalized at the Clinic of Infectious Child Diseases in Brno from January 1983 to January 1991. In all instances treatment led to rapid drying of the herpetis eruptions, a shorter period of new eruption and rapid improvement of the serious clinical condition. In none of the children visceral dissemination of the virus of herpes simplex (HSV) were occurred and in none of the children toxic side-effects were found. The authors confirmed the assumed identical course of EH after i. v. administration of acyclovir of foreign or local origin. After i.v. administration frequently dramatic improvement of the general and local finding was recorded, as compared with oral administration. There were no therapeutic differences in the clinical effects of tablets and suspension, the clinical effect being comparable. [\hyperlink{Acyclovir}{PMID: 1638653}, D Bartosová et al., 1992]

\hypertarget{pmid_1311067}{O}ral acyclovir was given prophylactically to 37 children in the early stages of three outbreaks of herpes simplex virus type 1 (HSV-1) infection and the results were compared with those in untreated control subjects in two other outbreaks. The rates of seroconversion to HSV were significantly reduced in children treated with acyclovir compared with control subjects (91\% vs 27\%, P less than .001). The incidence of symptomatic disease was also significantly reduced (82\% vs 0\%, P less than .001). In some children receiving prophylactic acyclovir, anti-HSV antibody titers did not rise despite the presence of replicative HSV on throat swabs just before the start of treatment. Restriction endonuclease analysis of isolated HSV-DNA confirmed that one strain was responsible for the five outbreaks. No resistance to acyclovir was detected during the study, and no adverse effects of treatment were noted. In conclusion, short-term prophylactic acyclovir may limit the spread and reduce clinical manifestations of HSV infections in closed communities, although this use should be restricted to communities where severe symptoms are observed. [\hyperlink{Acyclovir}{PMID: 1311067}, K Kuzushima et al., 1992]

\hypertarget{pmid_6750068}{A} randomized double-blind, placebo-controlled, multicenter investigation assessed the usefulness of acyclovir in the treatment of immunosuppressed children with chickenpox. Twelve patients received placebo and eight received acyclovir. If the event of clinical deterioration, patients could be removed from the study to receive acyclovir. Eighteen patients had skin lesions within 96 hours of admission to the study. Nineteen patients had malignancies. The two groups of patients were similar in age, in concomitant or preceding immunosuppressive therapy, in status of malignancy, and in presenting granulocyte and lymphocyte counts. Zoster immune globulin or plasma had been given to 50\% of the placebo group but to only 25\% of the acyclovir group. One patient in each group had pneumonitis at entry. Of the patients without pneumonitis at entry, five of the 11 placebo patients compared with none of the seven acyclovir patients developed pneumonitis during treatment (P = 0.054). No evidence of toxicity related to acyclovir was observed. [\hyperlink{Acyclovir}{PMID: 6750068}, C G Prober et al., 1982]

\hypertarget{pmid_2156984}{T}o determine whether acyclovir administered orally affects the duration and severity of varicella in otherwise normal children. Randomized, placebo-controlled, double-blind trial. Patients' residence and university hospital clinic. One hundred five children between 5 and 16 years of age with laboratory-confirmed varicella entered the study. Of the 102 who were included in the final analysis, 50 received acyclovir and 52 received placebo. Placebo or acyclovir was given orally four times daily, for 5 to 7 days. The acyclovir dose was adjusted as follows: 5 to 7 years of age, 20 mg/kg; 7 to 12 years, 15 mg/kg; and 12 to 16 years, 10 mg/kg. Acyclovir recipients, compared with the placebo group, defervesced sooner (median, 1 day vs 2 days; p = 0.001), experienced onset of cutaneous healing sooner, as reflected by a decrease in number of lesions (median, 3 days vs 2 days; p = 0.002), and had fewer skin lesions (median, 500 vs 336; p = 0.02). Acyclovir did not significantly change the rate of complications of varicella (10\% in the acyclovir group vs 13.5\% among placebo subjects). Adverse drug effects were not observed. Acyclovir recipients had lower geometric mean serum antibody titers to varicella-zoster virus than their placebo counterparts 4 weeks after the onset of illness, but antibody titers in both groups were similar 1 year later. These results provide evidence that acyclovir is useful and well tolerated for treatment of varicella in otherwise healthy children. [\hyperlink{Acyclovir}{PMID: 2156984}, H H Balfour et al., 1990]

\hypertarget{pmid_6285712}{N}ine infants with symptomatic infections caused by herpes simplex virus or cytomegalovirus were treated with acyclovir. At the onset of therapy, the infants ranged in weight from 880 to 4550 gm. Five were premature. Acyclovir was administered intravenously in a dosage of 5 to 15 mg/kg every eight hours for five to 10 days. The peak serum acyclovir levels ranged from 20 to 163 microM and the trough levels ranged from 1 to 129 microM. The variation in peak serum acyclovir levels in different infants receiving the same dosage on a weight basis was large but correlated with the expected renal maturity of the individual infant. Hematologic values improved during therapy. No renal toxicity was noted. All of the infants survived, including the five with herpes simplex infections. [\hyperlink{Acyclovir}{PMID: 6285712}, A S Yeager et al., 1982]

\hypertarget{pmid_3318610}{I}n the 5 years since its release for clinical use, acyclovir (9-[2-hydroxyethoxymethyl]guanine) has proved to be a safe and effective agent for therapy of herpes simplex and varicella-zoster infections. The drug's availability in topical, oral, and intravenous preparations has allowed its use in a range of clinical situations. Acyclovir must be phosphorylated by viral thymidine kinase in infected cells, where it then acts to inhibit viral DNA replication specifically. Epstein-Barr virus and human cytomegalovirus infections do not seem to respond to acyclovir therapy, although in-vitro effects on these viruses may be seen. Acyclovir is well absorbed and distributed, with cerebrospinal fluid levels 50\% that of plasma. Clearance is almost entirely by the renal route, with a half-life of 20 hours in the anuric patient. Acyclovir has an excellent safety profile, its major adverse effect being transient serum creatinine elevations during high-dose intravenous use. Major uses include treatment of primary and recurrent genital herpes and herpes encephalitis and prophyllaxis and therapy of mucocutaneous herpes and varicella-zoster infections in immunocompromised patients. Resistance to acyclovir in herpes simplex virus is rarely encountered and does not seem to be due to long-term chronic suppressive therapy. [\hyperlink{Acyclovir}{PMID: 3318610}, D I Dorsky et al., 1987]

\section*{Ramipril}
\subsection*{Result}
\subsubsection*{Answer}

Ages <1.9 years: Unknown  
Ages 1.9–19.8 years: Yes  

\subsubsection*{{Explanation}}
\hypertarget{Ramipril}
A review of the available abstracts reveals several targeted studies evaluating the safety of ramipril in children, specifically in the context of chronic kidney disease, hypertension, and post-renal transplantation. Below is a summary of the relevant evidence, organized by age ranges where possible:

1. **Children with Alport Syndrome (Oligosymptomatic, Age Range Not Explicitly Stated, but Pediatric Population):**
   - A randomized, placebo-controlled, double-blind trial conducted in 14 German sites evaluated ramipril in pediatric patients with Alport syndrome for 3-6 years, plus 6 months follow-up. Out of 66 children, 22 were randomized and 44 were in an open-arm comparison. The study found "no safety issues" with ramipril therapy over 216.4 patient-years, with an adverse event rate-ratio of 1.00 (95\% CI 0.66-1.53). The study concludes that early initiation of ramipril is safe in children with Alport syndrome [\hyperlink{pmid_32299679}{PMID: 32299679}, Oliver Gross et al., 2020].

2. **Children with Chronic Kidney Disease (Ages 1.9–19.8 years):**
   - A prospective study of 31 children (median age 11.3 years, range 1.9–19.8 years) with chronic nephropathies and hypertension or proteinuria treated with ramipril for 6 months found it to be effective and safe. Only one child developed a cough believed to be related to ramipril; no significant changes in glomerular filtration rate or serum potassium were observed [\hyperlink{pmid_15110900}{PMID: 15110900}, Tomás Seeman et al., 2004].

3. **Children after Renal Transplantation (Median Age 15.3 years):**
   - Twelve transplanted children (median age 15.3 years) were prospectively treated with ramipril for six months. Proteinuria declined in 92\% of children, and only two developed mild hyperkalemia. Graft function and serum potassium did not change significantly otherwise. The study supports the safety of ramipril in this population [\hyperlink{pmid_19686445}{PMID: 19686445}, Tomás Seeman et al., 2010].

4. **Children with Renal Hypertension (Ages 5–18 years):**
   - Fourteen moderately hypertensive children aged 5–18 years were treated with ramipril for 6 months. The study reports that ramipril was well tolerated, even in mild renal insufficiency, and no adverse effects were noted [\hyperlink{pmid_11095026}{PMID: 11095026}, M Soergel et al., 2000].

5. **Children with Hypertension and Left-Ventricular Hypertrophy (Median Age 15 years):**
   - Twenty-one children (median age 15 years) with hypertension were treated with ramipril for 6 months. Nineteen completed the study, and no safety concerns were reported [\hyperlink{pmid_17765141}{PMID: 17765141}, Tomás Seeman et al., 2007].

6. **Formulation Development for Pediatric Use:**
   - A study focused on developing oral liquid formulations of ramipril for pediatric use, with in vitro and in vivo absorption studies. While this supports the feasibility of pediatric administration, it does not directly address clinical safety in children [\hyperlink{pmid_25950602}{PMID: 25950602}, Craig Russell et al., 2015].

**Summary by Age Range:**
- **Ages 1.9–19.8 years:** Multiple studies, including randomized controlled trials and prospective cohorts, affirm the safety of ramipril in children with chronic kidney disease, hypertension, and post-renal transplantation, with only minor and infrequent adverse effects reported.
- **Ages 5–18 years:** Specific studies in this range also affirm safety.
- **Median age 15 years:** Studies in this group (including post-transplant and hypertensive children) affirm safety.
- **Younger than 1.9 years:** No targeted safety data available in the abstracts reviewed.

**Conclusion:** Based on the abstracts, ramipril has been studied in children (primarily ages 1.9–19.8 years) with chronic kidney disease, hypertension, and post-renal transplantation, and these studies affirm its safety in these populations. There is no evidence from these abstracts regarding safety in children younger than 1.9 years, so safety in that age group is unknown.

\subsection*{Abstracts}
\hypertarget{pmid_32299679}{C}hildren with Alport syndrome develop renal failure early in life. Since the safety and efficacy of preemptive nephroprotective therapy are uncertain we conducted a randomized, placebo-controlled, double-blind trial in 14 German sites of pediatric patients with ramipril for three to six years plus six months follow-up to determine these parameters. Pretreated children and those whose parents refused randomization became an open-arm control, which were compared to prospective real-world data from untreated children. The co-primary endpoints were safety (adverse drug reactions) and efficacy (time to progression). Out of 66 oligosymptomatic children, 22 were randomized and 44 joined the open-arm comparison. Ramipril therapy showed no safety issues (total of 216.4 patient-years on ramipril; adverse event rate-ratio 1.00; 95\% confidence interval 0.66-1.53). Although not significant, our results cautiously showed that ramipril therapy was effective: in the randomized arm, Ramipril decreased the risk of disease progression by almost half (hazard ratio 0.51 (0.12-2.20)), diminished the slope of albuminuria progression and the decline in glomerular filtration. In adjusted analysis, indications of efficacy were supported by prospective data from participants treated open label compared with untreated children, in whom ramipril again seemed to reduce progression by almost half (0.53 (0.22-1.29)). Incorporating these results into the randomized data by Bayesian evidence synthesis resulted in a more precise estimate of the hazard-ratio of 0.52 (0.19-1.39). Thus, our study shows the safety of early initiation of therapy and supports the hope to slow renal failure by many years, emphasizing the value of preemptive therapy. Hence, screening programs for glomerular hematuria in children and young adults could benefit from inclusion of genetic testing for Alport-related gene-variants. [\hyperlink{Ramipril}{PMID: 32299679}, Oliver Gross et al., 2020]

\hypertarget{pmid_15110900}{A}ngiotensin-converting enzyme inhibitors are the drugs of choice in renal hypertension. The efficacy and safety of ramipril in adults has been proved; however, data on effectiveness of ramipril in children are few. The aim of the present study was to investigate the effect of ramipril on blood pressure (BP) and proteinuria in children with chronic kidney diseases. A total of 31 children (median age 11.3 years, range 1.9-19.8 years) with various chronic nephropathies and hypertension or proteinuria were prospectively treated with ramipril for 6 months. Blood pressure was evaluated using ambulatory BP monitoring and hypertension was defined as mean BP equal to or greater than the 95th percentile for healthy children. Proteinuria was defined as protein excretion > or =100 mg/m(2)/24 h. The starting dose of ramipril was 1.5 mg/m(2)/24 h once daily. In 27 children it was given as monotherapy. The median decrease in ambulatory BP was 11 mm Hg for daytime systolic, 10 mm Hg for daytime and nighttime diastolic, and 8 mm Hg for nighttime systolic BP. Hypertension normalized in 55\% of the children. Proteinuria decreased in 84\% of the children with pathologic proteinuria; the median decrease was 51\%. A positive correlation was found between initial proteinuria and change of proteinuria (r = 0.95, P <.001). Glomerular filtration rate and serum potassium level did not change significantly. One child developed a cough that was believed to be related to ramipril. Ramipril is an effective and safe drug in children with chronic kidney diseases associated with hypertension, proteinuria, or both. [\hyperlink{Ramipril}{PMID: 15110900}, Tomás Seeman et al., 2004]

\hypertarget{pmid_25950602}{R}amipril is used mainly for the treatment of hypertension and to reduce incidence of fatality following heart attacks in patients who develop indications of congestive heart failure. In the paediatric population, it is used most commonly for the treatment of heart failure, hypertension in type 1 diabetes and diabetic nephropathy. Due to the lack of a suitable liquid formulation, the current study evaluates the development of a range of oral liquid formulations of ramipril along with their in vitro and in vivo absorption studies. Three different formulation development approaches were studied: solubilisation using acetic acid as a co-solvent, complexation with hydroxypropyl-β-cyclodextrin (HP-β-CD) and suspension development using xanthan gum. Systematic optimisation of formulation parameters for the different strategies resulted in the development of products stable for 12 months at long-term stability conditions. In vivo evaluation showed C(max) of 10.48 µg/ml for co-solvent, 13.04 µg/ml for the suspension and 29.58 µg/ml for the cyclodextrin-based ramipril solution. Interestingly, both ramipril solution (co-solvent) and the suspension showed a T(max) of 2.5 h, however, cyclodextrin-based ramipril produced T(max) at 0.75 h following administration. The results presented in this study provide translatable products for oral liquid ramipril which offer preferential paediatric use over existing alternatives. [\hyperlink{Ramipril}{PMID: 25950602}, Craig Russell et al., 2015]

\hypertarget{pmid_1725023}{T}he efficacy, tolerance, and safety of ramipril, an angiotensin-converting enzyme inhibitor, were assessed in 502 patients from five multicenter, double-blind studies who had mild-to-moderate essential hypertension. Each study was designed with a 4-week placebo run-in phase followed by 6 weeks of treatment with ramipril or one of five other antihypertensive treatments. A total of 412 young patients (17-65 years of age) and 90 old patients (66-87 years) in these studies received single daily doses of 5 or 10 mg of ramipril. At the end point of treatment, mean reductions in supine systolic blood pressure (19.4 mm Hg in young patients, 17.8 mm Hg in old) were significantly different, whereas mean reductions in supine diastolic blood pressure (13.3 mm Hg in young patients, 12.5 mm Hg in old) showed no significant difference. The number of responders was similar in both age groups: 68.6\% and 71.1\% of young and old patients respectively. No clinically relevant trends were observed in biochemical and hematological variables. Ramipril was well tolerated by both young and old patients, and there was little evidence that it was less safe in the elderly. [\hyperlink{Ramipril}{PMID: 1725023}, R Saalbach et al., 1991]

\hypertarget{pmid_3092699}{T}he postoperative treatment of pain in children is often inadequate: Periphal acting analgetics are not sufficient, opioids are believed to be dangerous because of their respiratory depression. Nalbuphine and tramadol are two narcotics with only a few side effects. The aim of this trial was to investigate the efficacy and safety of these drugs in postoperative pain therapy in children aged 1-9 years. 30 children in each group received in a double-blind and randomized manner either 0.15-0.2 mg/kg nalbuphine or 0.75-1.0 mg/kg tramadol im. Pain intensity and sleep-awake behaviour were documented by a visual analogue scale for 24 h. After 1 h 70\% of the patients in both groups had no pain and were sleeping. There was no change in heart rate and systolic blood pressure. Only the diastolic blood pressure decreased as did the respiratory rate, while the tcpCO2 estimated in some patients remained constant. Narcotic reinjections were necessary three times in the nalbuphine group and four times in the tramadol group. Typical opioid side effects were found to be equal in both groups. [\hyperlink{Ramipril}{PMID: 3092699}, J Schäffer et al., 1986]

\hypertarget{pmid_3889818}{T}he safety and efficacy of captopril therapy in children with severe and refractory hypertension has been evaluated in a collaborative international study which enrolled a group of 73 patients, 15 years of age or younger. Most patients had hypertension associated with renal disease or vascular abnormalities. Captopril was administered for periods of less than 3 months to more than 1 year. A significant decrease in both systolic and diastolic blood pressures was produced by the administration of captopril, usually in conjunction with other antihypertensive agents (most commonly diuretics and/or beta-blockers). Systolic blood pressures were normalized in 62\% and 53\% and diastolic blood pressures in 56\% and 45\% of reported patients after the second and sixth months of captopril therapy, respectively. The response to captopril was sustained over a 12-month period. Adverse reactions were reported in 49\% of the 73 patients; 48\% of patients had experienced adverse reactions to other antihypertensive agents prior to entering the study. The reactions most frequently observed during captopril therapy were hypotension, vomiting, postural symptoms, anemia, rash, and anorexia. Leukopenia was reported in six patients, all of whom had renal impairment. Two of these patients had received concomitant therapy with immunosuppressants, and one had systemic lupus erythematosus. Captopril was discontinued in two of these six children. Statistically significant increases in mean serum urea nitrogen and potassium concentrations and decreases in mean serum CO2 levels were observed during the course of therapy. These effects could not be exclusively attributed to captopril administration as the study population received multidrug therapy and had significant intrinsic disease. Captopril was demonstrated to be an effective and safe drug for the treatment of children with severe hypertension. [\hyperlink{Ramipril}{PMID: 3889818}, B L Mirkin et al., 1985]

\hypertarget{pmid_17474953}{T}he aim of this study was to evaluate the safety and efficacy of a combination of propofol and remifentanil deep sedation in spontaneously breathing children less than 7 years of age undergoing upper and/or lower gastrointestinal endoscopy. The effect of propofol and remifentanil sedation was prospectively studied in 42 unpremedicated children undergoing gastrointestinal endoscopy. Anesthesia was induced with a combination of sevoflurane, nitrous oxide and oxygen. Anesthesia was maintained with an infusion of propofol (50-80 microg x kg(-1) x min(-1)) and remifentanil (0.1 microg x kg(-1) x min(-1)). Demographic data, heart rate, blood pressure, respiratory rate, and oxygen saturation were recorded every 5 min for each child. In addition, recovery and discharge times were recorded. All 42 procedures were completed with no complications. The combination of propofol and remifentanil resulted in a decrease in heart rate, blood pressure, and respiratory rate. There was no respiratory depression or oxygen desaturation in any child. A bolus of propofol (1 mg x kg(-1)) was necessary in one child for excessive movement. No patient experienced any side effects in the recovery period. The combination of propofol and remifentanil for sedation in children undergoing gastrointestinal endoscopy can be considered safe, effective and acceptable. [\hyperlink{Ramipril}{PMID: 17474953}, Ibrahim Abu-Shahwan et al., 2007]

\hypertarget{pmid_25041277}{T}he European Medicine Agency recommendations limiting codeine use in children have created a void in managing moderate pain. We review the evidence on the pharmacokinetic, pharmacodynamic and safety profile of tramadol, a possible substitute for codeine. Tramadol appears to be safe in both paediatric inpatients and outpatients. It may be appropriate to limit the current use of tramadol to monitored settings in children with risk factors for respiratory depression, subject to further safety evidence. [\hyperlink{Ramipril}{PMID: 25041277}, Pierluigi Marzuillo et al., 2014]

\hypertarget{pmid_19686445}{T}he efficacy and safety of ACEI in adult patients with hypertension and proteinuria after renal transplantation is proven however data on the effectiveness of ACEI in transplanted children are rare. The aim of the present study was to investigate the effect of ramipril on proteinuria and BP in children after R-Tx. Twelve transplanted children (median age 15.3 yr, median time after R-Tx 4.5 yr) with proteinuria with or without hypertension were prospectively treated with ramipril for six months. Proteinuria was assessed as protein/creatinine ratio. Office BP was evaluated and hypertension defined as BP > or =95th centile. Graft function was assessed (Schwartz formula). The starting dose of ramipril was 1.5 mg/m(2)/24-h. Proteinuria declined in 92\% of children from a median 39 to 22 mg/mmol creatinine (p < 0.01). The median decline of proteinuria was 9 mg/mmol creatinine, it reached 23\% of the initial values. The prevalence of hypertension did not change significantly (50\% initially vs. 33\% after six months). Graft function and serum potassium level did not change significantly, two children developed mild hyperkalemia. Ramipril can reduce proteinuria in most transplanted children; its antiproteinuric effect is exhibited even without BP lowering effect. [\hyperlink{Ramipril}{PMID: 19686445}, Tomás Seeman et al., 2010]

\hypertarget{pmid_11095026}{I}nhibition of the angiotensin-converting enzyme (ACE) exerts a renoprotective effect in adult patients with chronic kidney disease. We evaluated prospectively changes in blood pressure (BP), protein excretion and renal function after administration of the long-acting ACE inhibitor ramipril as monotherapy during 6 months in 14 moderately hypertensive children aged 5-18 years with various nephropathies. Four patients initially had a decreased glomerular filtration rate (GFR below 60 ml/min/1.73 m2). BP was evaluated by ambulatory 24-h monitoring. After 2 weeks of treatment by oral ramipril (1.5 mg/m2 once daily), mean values of systolic and diastolic 24-h ambulatory BP fell by more than 5 mmHg in nine patients. In eight patients the dose was doubled. At the end of the study systolic BP was below the 95th percentile in 9 and diastolic BP in 13 patients. The initially reduced nocturnal dip increased significantly. Of 11 patients with an increased albumin excretion (median 1.3 g/g creatinine), 6 responded to ramipril by a median reduction of 78\% (range 24-83\%), whilst in 5 albuminuria increased (median +19\%). GFR was well preserved and no other adverse effects from the drug were noted. The study demonstrates that ramipril is an efficacious antihypertensive agent in children with renal hypertension. It is well tolerated, even in mild renal insufficiency. In addition, the drug has a persistent antiproteinuric action in about half of the patients contributing to conserve renal function. [\hyperlink{Ramipril}{PMID: 11095026}, M Soergel et al., 2000]

\hypertarget{pmid_3034023}{R}amipril is a newly synthesized angiotensin converting enzyme inhibitor without a sulfhydryl group in the molecule but with a prolonged duration of action. Efficacy, tolerance and safety of this drug were evaluated in 10 patients with severe essential hypertension. After a treatment period of at least 4 weeks with the conventional antihypertensive drug combination of a diuretic and a beta-blocking agent with the vasodilator dihydralazine, their systolic and diastolic blood pressures averaged 161 +/- 6 and 111 +/- 2 mm Hg, respectively. Because diastolic blood pressure during this drug regimen was still greater than 105 mm Hg in all patients, the patients received ramipril initially at single daily doses of 5 mg in addition to their previous medication. The first dose of 5 mg ramipril resulted in a moderate but significant decrease in systolic and diastolic blood pressure in 9 of the 10 patients to 142 +/- 5 and 104 +/- 4 mm Hg (p less than 0.01), respectively, between 3 and 6 hours after drug administration. In 1 patient blood pressure was unresponsive to ramipril and 1 patient complained of nausea and vomiting within the first week of treatment with ramipril. Within the following 8-week treatment period with a once-daily intake of 5 or, if necessary, 10 mg of ramipril, diastolic blood pressure normalized in the remaining 8 patients to less than 90 mm Hg. Systolic and diastolic blood pressure averaged 130 +/- 5 and 83 +/- 2 mm Hg, respectively, at the end of the 8-week treatment period with ramipril. Severe hypotension and reflex tachycardia were not observed.(ABSTRACT TRUNCATED AT 250 WORDS) [\hyperlink{Ramipril}{PMID: 3034023}, H G Predel et al., 1987] Left-ventricular hypertrophy (LVH) is a risk factor for cardiovascular morbidity. Antihypertensive treatment with angiotensin-converting enzyme inhibitors (ACEI) is able to induce the regression of LVH in adults. However, there has been no study of the ability of ACEI to induce the regression of LVH in children. Our aim was to investigate the effect of ramipril on left-ventricular mass and blood pressure (BP) in hypertensive children. Twenty-one children (median age, 15 years) with renal (76\%) or primary (24\%) hypertension were prospectively treated with ramipril monotherapy for 6 months. Blood pressure was evaluated using ambulatory BP monitoring, with hypertension defined as mean BP >or=95th percentile. Left-ventricular hypertrophy was defined either as left-ventricular mass index (LVMI) >38.6 g/m(2.7) (pediatric definition) or as LVMI >51.0 g/m(2.7) (adult definition). Nineteen children completed the study. The median LVMI decreased from 36.8 g/m(2.7) (range, 18.9 to 55.8 g/m(2.7)) to 32.6 g/m(2.7) (range, 19.0 to 52.1 g/m(2.7); P < .05) after 6 months. The prevalence of LVH decreased from 42\% to 11\% using the pediatric definition (P < .05) and did not change using the adult definition (ie, it remained at 5\%). The median ambulatory BP decreased by 11, 7, 8, and 7 mm Hg for daytime systolic, daytime diastolic, nighttime systolic, and nighttime diastolic BP (P < .05), respectively. A positive correlation was found between LVMI and nighttime systolic BP at the start of the study (r = 0.46, P < .05). Ramipril is an effective drug in children with hypertension, for its ability to reduce not only BP but also left-ventricular mass and induce regression of LVH. [\hyperlink{Ramipril}{PMID: 3034023}, Tomás Seeman et al., 2007]

\hypertarget{pmid_16144510}{R}amipril is an angiotensin-converting enzyme inhibitor that has been extensively studied in randomised, controlled clinical trials in patients with cardiovascular diseases. Therapy with ramipril in patients with various cardiovascular disorders has demonstrated significant and clinically important reductions in cardiovascular death, myocardial infarction, stroke, congestive heart failure, progressive renal impairment and onset of diabetes. Ramipril is usually dosed at 2.5-10 mg/day. Beneficial effects of ramipril are observed in the treatment of hypertension and congestive heart failure, prevention of cardiovascular events in high-risk patients, prevention of congestive heart failure, diabetes and other vascular events. [\hyperlink{Ramipril}{PMID: 16144510}, Michael J Rokoss et al., 2005]

\hypertarget{pmid_3519229}{E}xperience with chronic inhibition of the angiotensin-converting enzyme in children is limited to cases refractory to all other forms of treatment. In reports dealing with the use of captopril (Capoten-R) in children no important side-effects are mentioned. This report describes a 7-year-old boy with severe hypertension secondary to haemolytic uraemic syndrome. Good pressure control was obtained after introduction of captopril. However, under the high initial dosage, pronounced anaemia developed within the first 3 months of treatment. The anaemia responded to dose-reduction while pressure control was maintained. Serial echocardiographic studies were performed. They illustrate the beneficial haemodynamic effects of captopril in the follow-up of children under antihypertensive treatment. Some recommendations are made on the use of captopril in children. [\hyperlink{Ramipril}{PMID: 3519229}, H A Verhaaren et al., 1986]

\hypertarget{pmid_30045336}{T}he pleiotropic effects of hypotensive drugs should always be taken into consideration. There is limited data on the effect of such drugs on reducing global cardiovascular risk in young hypertensives. The aim of this study was to evaluate the effect of nebivolol and ramipril on biochemical parameters, arterial stiffness, and circadian profile of blood pressure (BP) in young men undergoing treatment for hypertension (HT). A total of 80 patients aged 16 to 28 years of age with grade 1 HT were enrolled into the prospective randomized, open-label trial. They were randomized to receive 5 mg of nebivolol or 5 mg of ramipril, daily. Arterial stiffness index (SI), the circadian profile of BP registered in ambulatory blood pressure monitoring (ABPM), and biochemical parameters-including lipid profile, insulinemia, glycemia, and high sensitivity C-reactive protein (hsCRP) levels-were evaluated before and after the twelve-week period. After the treatment period, we observed significant decreases in both ABPM systolic blood pressure (SBP) in group of nebivolol (P = .0007) and ramipril (P = .0001) and in ABPM diastolic blood pressure (DBP) in group of nebivolol (P = .0018) and ramipril (P = .0006). Reductions in the nondippers percentage were found in group of nebivolol and ramipril (P = .0077, P = .0001 respectively). Ramipril treatment resulted in a significant plausible modification in high-density lipoprotein (HDL) (P = .0390), glucose (P = .0213), and hsCRP (P = .0053) concentrations, as well as decreased SI (P = .0009) value, while nebivolol treatment showed no such benefits. Despite the similar hypotensive effect of nebivolol and ramipril, ramipril seems to possess better clinical potential in reducing cardiovascular risk in young men with HT. [\hyperlink{Ramipril}{PMID: 30045336}, Marta Walczak-Gałęzewska et al., 2018]

\hypertarget{pmid_15920180}{T}o compare the dose-response of remifentanil for tracheal intubation in infants and children, 32 healthy full-term infants and 32 children were anesthetized with 10 mug/kg glycopyrrolate and 4.0 mg/kg propofol and administered 1 of 4 doses of remifentanil (1.25, 1.50, 1.75, or 2.00 microg/kg) to facilitate tracheal intubation. We determined the effective doses of remifentanil in 50\% (ED50) and 98\% (ED98) of patients by using logistic regression analysis. We found that logistic regression curves were similar for infants and children (P = 0.38). ED50 and ED98 values for remifentanil were 1.70 +/- 0.1 microg/kg and 2.88 +/- 0.5 microg/kg, respectively. In a second double-blind study, 24 infants were anesthetized with propofol and randomized to receive either 3.0 microg/kg remifentanil or 2.0 mg/kg succinylcholine to facilitate tracheal intubation. The duration of apnea, tracheal intubating conditions and hemodynamic changes were determined. We found that the duration of apnea and intubating conditions after propofol/remifentanil were similar to those after propofol/succinylcholine. Bradycardia, hypotension, and chest wall rigidity did not occur. We conclude that the dose-response of remifentanil for tracheal intubation is similar in infants and children. Propofol/remifentanil provides clinically acceptable intubating conditions, stable hemodynamics, and a duration of apnea comparable to that with propofol/succinylcholine in infants. [\hyperlink{Ramipril}{PMID: 15920180}, Mark W Crawford et al., 2005]

\hypertarget{pmid_15385016}{C}erebrovascular stability and rapid anesthetic emergence are desirable features of a neuroanesthetic regimen. In this randomized crossover study the effect of a low-dose remifentanil infusion on cerebral blood flow velocity (CBFV) in children anesthetized with propofol was evaluated. Twenty healthy children aged 1-6 years undergoing urological surgery were enrolled. Following face mask induction with sevoflurane, anesthesia was maintained with a standardized propofol infusion. Rocuronium was used to facilitate tracheal intubation and normothermia, and normocapnia were maintained. All children received a caudal epidural block, and a transcranial Doppler probe was placed to measure middle cerebral artery blood flow velocity (Vmca). Each patient received a remifentanil regimen of 0.5 microg x kg(-1) followed by 0.2 microg x kg(-1) x min(-1) in a predetermined order of remifentanil + propofol or propofol alone. Vmca, mean arterial pressure (MAP) and heart rate (HR) were recorded simultaneously at equilibrium with and without remifentanil. The combination of remifentanil and propofol caused an 8.1\% decrease in MAP (P = 0.0005) and an 11.8\% decrease in HR (P < 0.0001) compared with propofol alone. Vmca was not different between the two groups (P = 0.4041). The addition of remifentanil to propofol anesthesia in children causes a reduction in MAP and HR without affecting CBFV. This may imply that cerebral blood pressure autoregulation is preserved in children under propofol and remifentanil anesthesia. [\hyperlink{Ramipril}{PMID: 15385016}, Annie Lagace et al., 2004]

\hypertarget{pmid_10947750}{W}e compared the efficacy and safety of a remifentanil (0.25 microg x kg(-1) x min(-1)-based balanced anaesthetic technique with a bupivacaine-based regional anaesthetic technique in an open label, multicentre study in 271 ASA physical status 1 or 2 children aged 1-12 years. Subjects requiring major intra-abdominal, urological or orthopaedic surgery were randomly allocated to receive either intravenous remifentanil (group R; n = 185) or epidural bupivacaine (group B; n = 86) with isoflurane/nitrous oxide for their anaesthesia. The majority of children in both groups (85\% in group R, 78\% in group B) showed no defined response to skin incision, and although the mean increase in systolic blood pressure (+11 mm Hg) was significantly greater in group R than in group B, this change did not represent a serious haemodynamic disturbance. More children in group R (31\%) required interventions to treat hypotension and/or bradycardia than those in group B (12\%), but these were easily managed by administration of fluids or anticholinergic drugs. Adverse events, mainly nausea and/or vomiting, occurred in 45\% of group R and 42\% of group B (NS). The adverse event profile of remifentanil in this study was typical of a potent mu-opioid receptor agonist. Remifentanil was as effective as epidural or caudal block in providing analgesia and suppressing physiological responses to surgical stimuli in children aged between 1 and 12 years undergoing major abdominal, urological, or orthopaedic surgery under isoflurane/nitrous oxide anaesthesia. [\hyperlink{Ramipril}{PMID: 10947750}, C Prys-Roberts et al., 2000]

\hypertarget{pmid_3317336}{F}orty two children with end stage renal failure and hypertension on chronic haemodialysis have been treated with captopril for from 18 to 78 months. Satisfactory control has been obtained in doses of 0.3 to 3.0 mg/kg given every 24 or 48 hours. Tolerance was good. The results of the present study suggest that captopril is a suitable drug for long-term use in paediatric patients. [\hyperlink{Ramipril}{PMID: 3317336}, L Callis et al., 1986]

\hypertarget{pmid_8879894}{A} postmarketing surveillance study was undertaken to confirm the efficacy and safety of the angiotensin-converting enzyme inhibitor ramipril and to extend the findings of controlled clinical trials into real-world conditions. A total of 11,100 patients with mild-to-moderate hypertension treated by primary care physicians were enrolled in this 8-week, open-label study. Ramipril was usually initiated at a dosage of 2.5 mg once daily and titrated to achieve target blood pressure. Efficacy was assessed in 8261 patients for whom blood pressure data were recorded after the start of treatment: safety was assessed in all patients. Of patients with combined systolic and diastolic hypertension, 86.0\% achieved a final diastolic blood pressure of < or = 90 mm Hg or a > or = 10 mm Hg decrease from baseline; the highest response was seen in elderly patients (87.2\%), and the lowest response was seen in black patients (81.2\%). Of patients with isolated systolic hypertension, 70.4\% achieved a final systolic blood pressure of < or = 140 mm Hg or a > or = 20 mm Hg decrease from baseline, including 70.6\% of women, 70.3\% of men, and 69.1\% of elderly patients; the highest response was seen in white patients (71.8\%), and the lowest response was seen in black patients (64.4\%). Adverse events were generally mild; cough (3.0\%) was the most frequent. Once-daily ramipril was effective and well tolerated during an 8-week period in a large, diverse population of patients who had mild-to-moderate hypertension and who were treated by primary care physicians. [\hyperlink{Ramipril}{PMID: 8879894}, N M Kaplan et al., ]

\hypertarget{pmid_3688811}{T}he efficacy of verapamil in the conversion of 47 episodes of supraventricular tachycardia in 22 children was evaluated. The age of the patients ranged from 15 days to 10 years. Tachycardia was the main mode of presentation. Ten out of 22 children had viral infections. Two patients developed mild cardiac failure. Six patients had underlying cardiac abnormalities. Forty-four out of 47 episodes of supraventricular tachycardia were converted to sinus rhythm by a single dose of verapamil (0.11 +/- 0.08 mg/kg). No significant side-effects were observed. Intravenous verapamil is an effective and safe drug for the conversion of supraventricular tachycardia in children. [\hyperlink{Ramipril}{PMID: 3688811}, K Y Chan et al., 1987]

\hypertarget{pmid_21836758}{T}o systemically review the evidence in support of World Health Organization guidelines recommending broad-spectrum antibiotics for children with severe acute malnutrition (SAM). CENTRAL, MEDLINE, EMBASE, LILACS, POPLINE, CAB Abstracts and ongoing trials registers were searched. Experts were contacted. Conference proceedings and reference lists were manually searched. All study types, except single case reports, were included. Two randomized controlled trials (RCTs), one before-and-after study and two retrospective reports on clinical efficacy and safety were retrieved, together with 18 pharmacokinetic studies. Trial quality was generally poor and results could not be pooled due to heterogeneity. Oral amoxicillin for 5 days was as effective as intramuscular ceftriaxone for 2 days (1 RCT). For uncomplicated SAM, amoxicillin showed no benefit over placebo (1 retrospective study). The introduction of a standardized regimen using ampicillin and gentamicin significantly reduced mortality in hospitalized children (odds ratio, OR: 4.0; 95\% confidence interval, CI: 1.7-9.8; 1 before-and-after study). Oral chloramphenicol was as effective as trimethoprim-sulfamethoxazole in children with pneumonia (1 RCT). Pharmacokinetic data suggest that normal doses of penicillins, cotrimoxazole and gentamicin are safe in malnourished children, while the dose or frequency of chloramphenicol requires adjustment. Existing evidence is not strong enough to further clarify recommendations for antibiotic treatment in children with SAM. Large RCTs are needed to define optimal antibiotic treatment in children with SAM with and without complications. Further research into gentamicin and chloramphenicol toxicity and into the pharmacokinetics of ceftriaxone and ciprofloxacin is also required. [\hyperlink{Ramipril}{PMID: 21836758}, Marzia Lazzerini et al., 2011]

\hypertarget{pmid_1835914}{R}amipril is a new, potent nonsulfhydryl inhibitor of angiotensin converting enzyme. The magnitude and duration of its antihypertensive effect were evaluated in a multicenter, placebo-controlled, randomized clinical trial conducted in 100 patients with mild to moderate essential hypertension. Ramipril significantly reduced both supine and standing blood pressures measured 24 h after dosing. Automated blood pressure monitoring showed that ramipril significantly reduced systolic and diastolic pressures for 24 h after dosing. The peak effect occurred between 3 and 6 h after dosing, with approximately 50\% of this effect retained after 24 h. Ramipril was well tolerated; there was no significant difference between active drug and placebo in the overall incidence of side effects. Ramipril is an effective and well-tolerated antihypertensive agent, which reduces both supine and standing blood pressure over the entire 24-h period after dosing. [\hyperlink{Ramipril}{PMID: 1835914}, D McCarron et al., 1991]

\hypertarget{pmid_11253489}{T}iagabine (Gabitril, Sanofi Synlhelabo) new antiepileptic drug was used in add-on therapy in 25 children with resistant partial complex and secondary generalized seizures. Treatment was carried out in children aged 4-17 years with low dose escalation from 5 to 45 mg/day, in three doses until good clinical effects were obtained. In 3 patients aged 4 years, in 11 children aged 5-12 years and in 11 children aged above 17 years Gabitril was used. Follow up period was 8-10 months. Frequency of epileptic seizures before implementation of Gabitril treatment, even during polytherapy with 2 or more antiepileptic drugs was several to hundred per day (status epilepticus was observed in 2 children with Rasmussen syndrome). During the observation 5 children became seizure free, in 11 patients reduction in seizures frequency above 50\% was observed and in 9 children effects of treatment were not good enough. Gabitril was well tolerated, and any adverse events were observed in add-on therapy. Preliminary observation and good results of add-on therapy with Gabitril are positive. Drug is safe and generally well-tolerated with good effects at add-on therapy in 64\% children with resistant partial complex and secondary generalized seizures. [\hyperlink{Ramipril}{PMID: 11253489}, T Kmieć et al., 2000]

\hypertarget{pmid_16028153}{B}ecause of concerns about arthrotoxicity, fluoroquinolones are restricted for use in children. This study describes the safety and efficacy of gatifloxacin when used for treatment of children with recurrent acute otitis media (ROM) or acute otitis media (AOM) treatment failure (AOMTF). We performed an analysis of 867 children included in 4 clinical trials who had ROM and/or AOMTF and were treated with gatifloxacin (10 mg/kg once daily for 10 days). Gatifloxacin had adverse event rates that were similar overall to those of a comparator antibiotic (amoxicillin-clavulanate), except for increased diarrhea in children <2 years old receiving amoxicillin-clavulanate. There was no evidence of arthrotoxicity, hepatotoxicity, alteration of glucose homeostasis, or central nervous system toxicity acutely or during 1 year follow-up in any child. Regarding efficacy, in 2 noncomparative trials, the gatifloxacin cure rate of AOM was 89\% (95\% confidence interval [CI], 83\%-95\%) at the test of cure (TOC) visit, 3-10 days after completion of therapy. In 2 comparative trials of gatifloxacin versus amoxicillin-clavulanate, the efficacy of gatifloxacin was 88\% (95\% CI, 82\%-94\%). Gatifloxacin led to better clinical outcomes than amoxicillin-clavulanate for AOMTF (91\% vs. 81\%; P=.029), for AOMTF and age <2 years old (89\% vs. 69\%; P=.009), and for severe AOM in children <2 years old (90\% vs. 75\%; P=.012). Among children with AOMTF previously treated with amoxicillin-clavulanate or ceftriaxone injections, gatifloxacin cure rates were high (88\% and 75\%, respectively). Gatifloxacin appears to be safe for children, with no evidence of producing arthrotoxicity in 867 children exposed to the antibiotic when used as treatment for ROM and AOMTF. [\hyperlink{Ramipril}{PMID: 16028153}, Michael E Pichichero et al., 2005]

\section*{Alvimopan}
\subsection*{Result}
\subsubsection*{Answer}

Unknown

\subsubsection*{{Explanation}}
\hypertarget{Alvimopan}
A review of the available abstracts reveals several that discuss the safety and efficacy of alvimopan, but none of these abstracts report targeted studies evaluating the safety of alvimopan specifically in children (defined as individuals under 18 years of age). The studies and reviews available focus on adult populations, such as those undergoing bowel resection or hysterectomy, and do not include pediatric subjects or age-specific safety data [\hyperlink{pmid_19574601}{PMID: 19574601}, Heather R Bream-Rouwenhorst et al., 2009; \hyperlink{pmid_18778122}{PMID: 18778122}, Monique P Curran et al., 2008; \hyperlink{pmid_27825721}{PMID: 27825721}, Abhijit Nair et al., 2016; \hyperlink{pmid_16626607}{PMID: 16626607}, Thomas J Herzog et al., 2006; \hyperlink{pmid_19635772}{PMID: 19635772}, Timothy J Bell et al., 2009]. 

None of the abstracts mention the inclusion of children or adolescents in their study populations, nor do they provide age-stratified safety outcomes for pediatric patients. Therefore, based on the abstracts available, there is no evidence from targeted studies to affirm or refute the safety of alvimopan in children of any age range. The safety of alvimopan in children is therefore unknown.

\subsection*{Abstracts}
\hypertarget{pmid_24175945}{T}he purpose of the study was to compare the safety of artemether-lumefantrine (AL) with other artemisinin-based combinations in children. A search of EMBASE (from 1974 to April 2013), MEDLINE (from 1946 to April 2013) and the Cochrane library of registered controlled trials for randomized controlled trials (RCTs) which compared AL with other artemisinin-based combinations was done. Only studies involving children ≤ 17 years old in which safety of AL was an outcome measure were included. Four thousand, seven hundred and twenty six adverse events (AEs) were recorded in 6,000 patients receiving AL. Common AEs (≥ 1/100 and <1/10) included: coryza, vomiting, anaemia, diarrhoea, vomiting and abdominal pain; while cough was the only very commonly reported AE (≥ 1/10). AL-treated children have a higher risk of body weakness (64.9\%) than those on artesunate-mefloquine (58.2\%) (p = 0.004, RR: 1.12 95\% CI: 1.04-1.21). The risk of vomiting was significantly lower in patients on AL (8.8\%) than artesunate-amodiaquine (10.6\%) (p = 0.002, RR: 0.76, 95\% CI: 0.63-0.90). Similarly, children on AL had a lower risk of vomiting (1.2\%) than chlorproguanil-dapsone-artesunate (ACD) treated children (5.2\%) (p = 0.002, RR: 0.63, 95\% CI: 0.47-0.85). The risk of serious adverse events was significantly lower for AL (1.3\%) than ACD (5.2\%) (p = 0.002, RR: 0.45, 95\% CI: 0.27-0.74). Artemether-lumefantrine combination is as safe as ASAQ and DP for use in children. Common adverse events are cough and gastrointestinal symptoms. More studies comparing AL with artesunate-mefloquine and artesunate-azithromycin are needed to determine the comparative safety of these drugs. [\hyperlink{Alvimopan}{PMID: 24175945}, Oluwaseun Egunsola et al., 2013]

\hypertarget{pmid_19574601}{T}he efficacy, safety, pharmacology, pharmacokinetics, drug-drug interactions, and administration of alvimopan for postoperative ileus are reviewed. Alvimopan is a selective mu-opioid receptor antagonist with no central nervous system activity. When orally administered after partial small- or large-bowel resection in patients with primary anastomosis, alvimopan shortened the return of bowel function and time to discharge by approximately one day without compromising analgesia. Alvimopan was not shown to be beneficial on these same outcomes after hysterectomy and has not been studied in other surgical populations. Alvimopan is generally well tolerated, with the frequency of adverse events being similar to placebo when used postoperatively for one week or less. Long-term studies of alvimopan in opioid-induced bowel dysfunction have shown an association with adverse cardiovascular outcomes, neoplasms, and fractures. Because of these concerns, the Entereg Access Support and Education program was developed. The recommended dosage of alvimopan is 12 mg administered with a sip of water 30 minutes to five hours before surgery, followed by 12 mg twice daily beginning the day after surgery for a maximum of seven days, 15 total doses, or until discharge. There is a limited amount of pharmacoeconomic analysis concerning alvimopan. Alvimopan, a peripherally acting mu-opioid receptor antagonist, is a novel agent for the treatment of postoperative ileus. It appears to decrease the duration of postoperative ileus and hospitalization by approximately one day, theoretically offsetting its acquisition costs. Unresolved long-term safety issues, a limited indication, and its restricted-access program are likely to hinder its widespread use in the surgical population. [\hyperlink{Alvimopan}{PMID: 19574601}, Heather R Bream-Rouwenhorst et al., 2009]

\hypertarget{pmid_9570603}{R}egional nerve blocks are often used for the treatment of postoperative pain in children. Ammonium sulfate is a non-narcotic anesthetic agent, which has been reported to provide pain relief lasting days to weeks, with few reported side effects in adult studies. Prior to considering clinical use in children, the neurotoxicity of ammonium sulfate in 4-day and 3-week old rats was assessed and compared with that of bupivacaine. Each rat received a posterior tibial nerve intrafascicular injection (0.01 mL in 4-day-old and 0.02 mL in 3-week-old rats) using either 10\% ammonium sulfate (n = 24 per age group), 0.5\% bupivacaine (n = 18 per age group), 0.9\% saline (n = 18 per age group), or 5\% phenol (n = 18 per age group). A functional assessment by serial walking track analysis and a morphologic assessment by neurohistology were made. No abnormalities in serial walking track analysis and no structural nerve damage were detected after ammonium sulfate, bupivacaine, or saline injection. Bupivacaine caused mild focal changes in both age groups, which recovered by 8 weeks. Intrafascicular injection of ammonium sulfate was as safe as bupivacaine in this animal model. Further animal studies must be made before human trials are initiated. [\hyperlink{Alvimopan}{PMID: 9570603}, M C Hertl et al., ]

\hypertarget{pmid_18778122}{A}lvimopan, a trans-3,4-dimethyl-4-(3-hydroxy-phenyl) piperidine, is a selective, peripherally acting micro-opioid receptor antagonist that is available for short-term use in hospitalized patients who have undergone bowel resection. The efficacy of alvimopan in the management of postoperative ileus has been evaluated in five phase III trials; four conducted in North America and one conducted in Europe/Australasia. Patients who had undergone partial large or small bowel resection surgery with primary anastomosis were randomized to receive alvimopan 12 mg or placebo as a single oral pre-operative dose followed by twice-daily administration for up to 7 days postoperatively. In the five phase III trials, alvimopan was significantly more effective than placebo in reducing the time to recovery of upper and lower gastrointestinal (GI) function, as assessed using a two-component endpoint (GI2) comprising time to tolerance of solid food and first bowel movement. The mean time to reach the GI2 endpoint was 11-26 hours sooner with alvimopan than with placebo. In the phase III trials conducted in North America, the time to writing the hospital discharge order was 13-21 hours sooner with alvimopan than with placebo. Alvimopan did not reduce opioid-induced analgesia and/or increase the amount of opioids administered postoperatively. Short-term alvimopan was generally well tolerated in adults undergoing bowel resection. [\hyperlink{Alvimopan}{PMID: 18778122}, Monique P Curran et al., 2008]

\hypertarget{pmid_25145624}{O}lanzapine is frequently prescribed in young children for psychiatric conditions. It may be an option for chemotherapy-induced nausea and vomiting (CINV) control in children. The objective of this review was to describe the safety of olanzapine in children less than 13 years of age to determine if safety concerns would be a barrier to its use for CINV prevention. Electronic searches were performed in MEDLINE, EMBASE, Cochrane Central Register of Controlled Trials, Web of Science and Scopus. All studies in English reporting adverse effects associated with olanzapine use in children younger than 13 years or with a mean/median age less than 13 years were included. Adverse outcomes were synthesized for prospective studies. A total of 47 studies (17 prospective) involving 387 children aged 0.6-18 years were included; nine described olanzapine poisonings. Weight gain or sedation were reported in 78 \% [95 \% confidence interval (CI) 63-95] and 48 \% (95 \% CI 35-67), respectively. Extrapyramidal symptoms or electrocardiogram abnormalities were reported in 9 \% (95 \% CI 4-21) and 14 \% (95 \% CI 7-26), respectively. Elevation in liver function tests or blood glucose abnormalities were reported in 7 \% (95 \% CI 2-20) and 4 \% (95 \% CI 1-17), respectively. No deaths were attributed to olanzapine. No studies were identified with a primary focus on evaluating safety, and the adverse effects reported in the included studies were heterogeneous. Most adverse events associated with olanzapine use in children less than 13 years of age are of minor clinical significance. These findings support the exploration of olanzapine for the prevention of CINV in children in future trials. [\hyperlink{Alvimopan}{PMID: 25145624}, Jacqueline Flank et al., 2014]

\hypertarget{pmid_27825721}{A}lvimopan is an US-FDA approved, peripherally acting mu opioid receptor antagonist which when started pre-operatively has been shown to hasten intestinal motility and reduce the duration of post-operative ileus. However the logistics involved in procuring, storing and dispensing the drug and the cost of the drug for fifteen doses as approved by FDA prohibits the use of it on a regular basis. [\hyperlink{Alvimopan}{PMID: 27825721}, Abhijit Nair et al., 2016]

\hypertarget{pmid_15595588}{T}he aim of this study was to investigate the quality of intra- and postoperative analgesia obtained by alfentanil compared to that produced by peripheral blockade in children. During sevoflurane-nitrous oxide atracurium anaesthesia for minor abdominal or genito-urinary surgery, three groups of children aged 0-8 yr received 25 microg kg(-1) alfentanil intravenously (n = 28), or peripheral nerve blockade using 1 mLkg(-1) ropivacaine 0.475\% (n = 24), or 12.5 microg kg(-1) alfentanil intravenously with peripheral nerve blockade using 1 mL kg(-1) ropivacaine 0.475\% (n = 30). Changes in blood pressure and heart rate were measured during the procedures. Postoperative pain was assessed using the face, legs, activity, cry, consolability (FLACC) observational tool for quantifying pain behaviour and a numerical scale scored by nurses, doctors, parents and children. There was no significant difference in intra- or postoperative analgesic efficacy among the three groups. Patients who received alfentanil had significantly lower heart rates than those who received nerve blockade only (96.0+/-15.6 vs. 115.9+/-23.2 beats min(-1), P < 0.001). FLACC and numerical scale scores did not differ among the groups. There were no significant differences in incidence of vomiting or use of pain medications. It was concluded that a low-dose, intravenous bolus of alfentanil may be an efficient alternative to peripheral nerve blockade in controlling pain during and after minor abdominal and genito-urinary surgery. [\hyperlink{Alvimopan}{PMID: 15595588}, F Leoni et al., 2004]

\hypertarget{pmid_29388634}{P}ediatric data on the use of thrombopoietin receptor agonists are fairly limited. The recent approval of eltrombopag by the US Food and Drug Administration for children aged ≥1 year, based on data from two randomized, placebo-controlled clinical trials, may lead to the increased use of this drug in clinical practice, and therefore, it is important to have a basic understanding of the biology, pharmacokinetics, safety, and efficacy of the medication. [\hyperlink{Alvimopan}{PMID: 29388634}, Michele P Lambert et al., 2016]

\hypertarget{pmid_14707960}{B}rimonidine 0.2\% (Alphagan) is a topical alpha-2 agonist widely used as an antihypertensive. There have been occasional reports of systemic adverse effects in children including apparent central nervous system depression. There are few data available on the overall safety of brimonidine 0.2\% in children. Computerised pharmacy records were used to identify all children who had been prescribed brimonidine 0.2\% in our eye department between August 1999 and June 2001, and their notes were reviewed. In all, 23 patients were identified from pharmacy records and 22 sets of notes were recovered and reviewed. The mean age at commencement of treatment was 8 years (range 0-14 years). In all, 10 (46\%) were treated in one eye and 12 (54\%) in both. Brimonidine 0.2\% was taken for a mean 14 months (range 1 day-75 months). A total of 14 (64\%) patients were already taking a topical beta-blocker when brimonidine 0.2\% was commenced and a further four (18\%) were being treated with another topical hypotensive agent. Of the 22 patients, six (27\%) had to stop brimonidine 0.2\% because of adverse side effects (two because of local irritation/allergy, two because of tiredness, and two because of fainting attacks). Many topical hypotensive agents are not licensed for use in children and few safety data are available. In this study, 18\% of children had systemic adverse effects sufficient to necessitate stopping the drug. It is possible that educational impairment may have passed unnoticed in others. Larger studies are required to investigate this further. [\hyperlink{Alvimopan}{PMID: 14707960}, R J C Bowman et al., 2004]

\hypertarget{pmid_9123909}{I}n the course of treatment for tumors and recurrences, 86 children, aged 4-16 years, received polychemotherapy which induced excessive vomiting. Navoban (tropisetron) was administered to control vomiting. Total or partial control of nausea and vomiting was observed in 94.1\%. No side-effects were registered. [\hyperlink{Alvimopan}{PMID: 9123909}, S A Safonova et al., 1996]

\hypertarget{pmid_11045391}{A}mlodipine has potential advantages in children since it can be dissolved into a liquid preparation and has a long elimination half-life, allowing for once-daily administration. The objective of this study was to compare the efficacy and compliance of amlodipine with that of standard long-acting calcium channel blockers (felodipine or nifedipine) in hypertensive children. A randomized, prospective, crossover study of 11 hypertensive children (9-17 years of age, 10 renal transplant patients) was performed with electronic monitoring of compliance. Each treatment arm was 30 days. No significant differences were observed in mean systolic (SBP) and diastolic blood pressures (DBP) between amlodipine and the other calcium channel blockers. Using 24-h blood pressure monitoring there were no significant differences over each drug treatment period in both mean day-time and night-time SBP and DBP. Patient compliance was similar in both the amlodipine and the nifedipine/felodipine treatment periods. These data suggest that amlodipine is as effective in pediatric nephrology patients as nifedipine and felodipine. Amlodipine may be optimally suited for treatment of young children because at present it is the only calcium channel blocker which can be administered once daily as a liquid preparation. [\hyperlink{Alvimopan}{PMID: 11045391}, J W Rogan et al., 2000]

\hypertarget{pmid_19635772}{T}he economic effect of the use of alvimopan in four randomized, double-blind, placebo-controlled, Phase III, North American efficacy trials was analyzed. Patients were eligible for the study if they were 18 years or older, were undergoing laparotomy for partial small or large bowel resection with primary anastomosis, and were scheduled for postoperative pain management with opioid-based i.v. patient-controlled analgesia. Patients analyzed in the North American Phase III trials received placebo or alvimopan 12 mg orally before surgery. Doses were administered twice daily beginning the day after surgery until hospital discharge or for a maximum of 15 doses. Compared with placebo, alvimopan was associated with a significantly shorter mean time to gastrointestinal (GI) recovery and a significantly shorter mean time to a written discharge order. Alvimopan was also associated with a mean hospital length of stay (LOS) of one full day less than placebo. The mean cost of alvimopan based on a mean of 8.9 12-mg doses was \$558.00; the alvimopan cost at the upper limit of allowed dosing was \$937.50. Combining the alvimopan and hospital costs for each patient, total costs for the alvimopan group were estimated to be lower than for the placebo group. In a post hoc analysis, alvimopan was associated with significantly faster upper and lower GI recovery after bowel resection and a mean LOS reduction of one day compared with placebo. The mean estimated hospital cost was \$879-\$977 less for patients who received alvimopan compared with placebo. The base-case and sensitivity analyses suggest that, on average, the use of alvimopan compared with placebo may have a cost-saving effect in the hospital setting. [\hyperlink{Alvimopan}{PMID: 19635772}, Timothy J Bell et al., 2009]

\hypertarget{pmid_9279301}{I}n 1993, the nonbenzodiazepine sedative-hypnotic zolpidem tartrate (Ambien) was approved for use in the US. Zolpidem has an imidazopyridine structure and possesses a rapid onset of action and a short half-life. The toxic threshold and profile have not been well established in the pediatric population. All pediatric zolpidem exposures reported to a regional poison information center over 24 months were reviewed retrospectively from the American Association of Poison Control Centers Toxic Exposure Surveillance System data collection forms. Twelve pediatric zolpidem exposures were reported. Seven were unintentional (ages 20 mon-5 y) and five were intentional misuse/suicide (ages 12-16 y). The regional poison information center was contacted within 1 h in ten cases with onset of symptoms within 10 to 60 min (mean 31.6 min). One child had no effect with 2.5 mg. As little as 5 mg caused symptoms with minor outcome in six unintentional ingestions (5-30 mg). Minor to moderate symptoms were reported 1-4 h after intentional ingestions (12.5-150 mg). The duration of symptoms in the unintentional cases ranged from less than 60 min up to 4 h (mean 2.4 h) and 6-10 h (mean 7.5 h) in the intentional exposures. Treatment consisted of observation (4), syrup of ipecac (1), lavage and activated charcoal (1), activated charcoal alone (5), and unknown (1). Due to the very rapid onset of central nervous system symptoms in children, emesis is not a treatment option. Supportive care, activated charcoal in large ingestions, and observation until symptoms resolve may be sufficient in most pediatric cases. [\hyperlink{Alvimopan}{PMID: 9279301}, D L Kurta et al., 1997]

\hypertarget{pmid_37087633}{E}ltrombopag is clinically approved for use in immune thrombocytopenia (ITP), chronic hepatitis C-related thrombocytopenia, and aplastic anemia and suitable for children; however, data on its overall safety profile are scarce. This study aimed to explore the clinical features of adverse drug events (ADEs) associated with eltrombopag in different age groups using individual case safety reports (ICSRs) from the World Health Organization database VigiBase and the US Food and Drug Administration Adverse Event Reporting System database from 2008 to 2022 in combination with a meta-analysis of data from randomized clinical trials in the literature from inception to July 28, 2022. We conducted disproportionality analyses by grouping patients into the following age groups: 0-17 (0-23 months, 2-11 years, and 12-17 years), 18-64, and ≥ 65 years. The ADEs about hepatobiliary disorders, thrombosis, skin and subcutaneous tissue disorders, infections, and so on were observed more differently in each age group. Meta-analysis results showed differences in the four system organ classes between adults and children with ITP: infections and infestations, general disorders and administration site conditions, skin and subcutaneous tissue disorders, and investigations. The adverse drug reactions in the latest version of instructions were searched in the databases to analyze their postmarketing safety signal strength. We observed signals of elevated alanine aminotransferase, aspartate aminotransferase, and blood bilirubin levels in all age groups. For children, urinary tract infection and back pain showed signals. Due to the inherent limitations of pharmacovigilance studies, more experiments are needed to assess the risks of eltrombopag in different ages. [\hyperlink{Alvimopan}{PMID: 37087633}, Han Qu et al., 2023]

\hypertarget{pmid_16626607}{T}he purpose of this study was to investigate the safety and efficacy of alvimopan, a novel peripherally acting mu-opioid receptor antagonist, in patients who undergo simple total abdominal hysterectomy. Women (n = 519) were randomized (4:1) to receive alvimopan 12 mg (n = 413) or placebo (n = 106) > or = 2 hours before the operation then twice daily for 7 days (hospital and home). Adverse events were monitored up to 30 days after the last dose of study drug was administered. Efficacy was assessed for 7 postoperative days. Overall, the most common adverse events were nausea, vomiting, and constipation; < 5\% of patients discontinued use because of adverse events. Alvimopan significantly accelerated the time to first bowel movement (hazard ratio, 2.33; P <.001). Average time to first bowel movement was reduced by 22 hours, with more frequent bowel movement and better bowel movement quality found in the treatment cohort. Alvimopan has a safety profile that is similar to that of placebo and provides significantly improved lower gastrointestinal recovery in women who undergo simple total abdominal hysterectomy. [\hyperlink{Alvimopan}{PMID: 16626607}, Thomas J Herzog et al., 2006]

\hypertarget{pmid_36030422}{I}ntroduction: Biological therapy can be used in uveitis in children since 2016. With ophthalmological indication only adalimumab therapy can be started. Adalimumab is a monoclonal antibody that inhibits tumor necrosis factor alpha.Objective: To summarize our experience with patients receiving adalimumab for pediatric non-infectious uveitis.Patients and methods: We investigated our juvenile patients of non-infectious uveitis treated with adalimumab be-tween 2017 and 2021 in a retrospective case series at the Department of Ophthalmology, Szeged University. Results: Between 01 January, 2017 and 31 May, 2021, we examined 46 children with uveitis. The mean age of these 23 girls and 23 boys was 11 years. 21 of them had juvenile idiopathic arthritis, 14 had infectious uveitis, 3 had hae-matological disorders, 8 had idiopathic uveitis. Adalimumab was given to 11 patients because of severe, chronic uveitis. There were 3 boys and 8 girls, their mean age was 10 years. Adalimumab was given according to the licence of the European Medicines Agency. Indication was anterior uveitis at 6 children, panuveitis at 5 children. Adali-mumab can be given to children over 2 years, who have chronic, non-infectious, anterior uveitis. Children with panuveitis received the therapy by the help of a pediatric rheumatologist.Conclusion: The significance of pediatric uveitis and its therapy is emergent. Our aim was to preserve vision and de-crease the possibilities of side effects and to provide a better life for these uveitic children. Early diagnosis, adequate therapy and regular ophthalmological check-ups are important. Children treated with adalimumab have good visual acuity due to the effectiveness of the therapy. No new ocular side effect was detected at the children treated with adalimumab. [\hyperlink{Alvimopan}{PMID: 36030422}, Lilla Smeller et al., 2022]

\hypertarget{pmid_19535212}{A}mlodipine is a long-acting calcium channel blocker capable of producing hypotension and dysrhythmia in overdose. The toxic doses of amlodipine in children are unclear. The purposes of this study were to describe amlodipine poisoning in children and to determine whether a dose-response relationship could be detected in this population using standardized call data from United States (US) poison centers. 1251 amlodipine-only ingestions in children < 6 years of age were reviewed. Cases with doses coded as "Exact" or "Estimated" and with dose, age, and medical outcome were analyzed (n = 678). Ingestions reported as a "taste or lick" (n = 53) were included as a dose of 1/10 of the dosage form involved. A clinically important response was defined as bradycardia, hypotension, dysrhythmia, conduction disturbance, or hyperglycemia. The risk of such responses was examined over four dosage intervals (< 2.5 mg, 2.5-5 mg, 5.1-10 mg, and > 10 mg). The median estimated dose ingested was 5 mg (range 0.25-200 mg). Clinically important responses developed in 27 patients (3.98\%), and the prevalence of such response significantly increased from 0\% for the lowest to 11.1\% for the highest dose interval (p = 0.001). The smallest dose to produce a clinically important response was 2.5 mg (0.15 mg/kg). Children who ingested > 10 mg were 4.4 times more likely to develop clinically important responses than those ingesting < or = 5 mg. Hypotension may occur in children with amlodipine doses as low as 2.5 mg. The National Poison Data System might provide useful insights regarding dose-response. [\hyperlink{Alvimopan}{PMID: 19535212}, Blaine E Benson et al., 2010]

\hypertarget{pmid_25328089}{T}his retrospective review provides preliminary data regarding the safety and efficacy of olanzapine for chemotherapy-induced vomiting (CIV) control in children. Children <18 years old who received olanzapine for acute chemotherapy-induced nausea and vomiting (CINV) control from December 2010 to August 2013 at four institutions were identified. Patient characteristics, chemotherapy, antiemetic prophylaxis, olanzapine dosing, CIV control, liver function test results and adverse events were abstracted from the health record. Toxicity was graded using CTCAEv4.03. Sixty children (median age 13.2 years; range: 3.10-17.96) received olanzapine during 158 chemotherapy blocks. Olanzapine was most often (59\%) initiated due to a history of poorly controlled CINV. The mean initial olanzapine dose was 0.1 mg/kg/dose (range: 0.026-0.256). Most children who received olanzapine beginning on the first day of the chemotherapy block experienced complete CIV control throughout the acute phase (83/128; 65\%). There was no association between the olanzapine dose/kg and complete CIV control (OR 1.01; 95\% CI: 0.999-1.020; P = 0.091). Sedation was reported in 7\% of chemotherapy blocks and was significantly associated with increasing olanzapine dose (OR: 1.17; 95\% CI: 1.08-1.27; P = 0.0001). Of the 25 chemotherapy blocks where ALT and/or AST were reported more than once, grade 1-3 elevations were observed in five. The mean weight change in 31 children who received olanzapine during more than one chemotherapy block was 0\% (range: -22 to +18). Olanzapine may be an important option to improve CIV control in children. Prospective controlled evaluation of olanzapine for CINV prophylaxis in children is warranted. [\hyperlink{Alvimopan}{PMID: 25328089}, Jacqueline Flank et al., 2015]

\hypertarget{pmid_12690278}{T}o evaluate the effectiveness of oral amoxicillin/clavulanate (25 mg/kg every 12 h) for prevention of fever and/or infection in neutropenic children with cancer. Multicenter, prospective, randomized, double blind placebo-controlled trial. In the intention-to-treat analysis, amoxicillin/clavulanate had a 12\% benefit increase in terms of reduction in the incidence of febrile or infectious episodes, compared with placebo [44 of 83 (53\%) vs.55 of 84 (65\%); 95\% confidence interval, -28\% to +3\%; P = 0.101]. This benefit was also associated with a 30\% increase in the probability of failure-free survival at Day 15 (P = 0.138). A logistic regression analysis showed the effect of prophylaxis to be relevant, especially in patients with leukemia or lymphoma and in those not receiving hematopoietic growth factors, with 17 and 15\% absolute benefit increases (logistic P = 0.014 and 0.034, respectively). Compliance with oral drugs was good, with very few and nonsevere drug-related adverse events. In this study amoxicillin/clavulanate was associated with a detectable clinical effect in the reduction of fever and infection in neutropenic children with cancer, especially those with acute leukemia and not receiving growth factors; the study was not powered to demonstrate a statistically significant effect in the overall patient population. [\hyperlink{Alvimopan}{PMID: 12690278}, Elio Castagnola et al., 2003]

\hypertarget{pmid_20889882}{A}rtemether-lumefantrine (AL) and dihydroartemisinin-piperaquine (DP) are highly efficacious antimalarial therapies in Africa. However, there are limited data regarding the tolerability of these drugs in young children. We used data from a randomized control trial in rural Uganda to compare the risk of early vomiting (within one hour of dosing) for children 6-24 months of age randomized to receive DP (n = 240) or AL (n = 228) for treatment of uncomplicated malaria. Overall, DP was associated with a higher risk of early vomiting than AL (15.1\% versus 7.1\%; P = 0.007). The increased risk of early vomiting with DP was only present among breastfeeding children (relative risk [RR] = 3.35, P = 0.001) compared with children who were not breastfeeding (RR = 1.03, P = 0.94). Age less than 18 months was a risk factor for early vomiting independent of treatment (RR = 3.27, P = 0.02). Our findings indicate that AL may be better tolerated than DP among young breastfeeding children treated for uncomplicated malaria. [\hyperlink{Alvimopan}{PMID: 20889882}, Darren Creek et al., 2010]

\hypertarget{pmid_20401256}{A}cetaminophen has become the non-narcotic of choice for children because of concerns regarding the connection between acetylsalicylic acid exposure and Reye's syndrome. Ibuprofen, recently granted over-the-counter status for children over two years of age, offers another choice for treatment. The efficacy and safety of both drugs have been studied in numerous clinical trials. This paper reviews the published evidence about the efficacy and safety of acetaminophen and ibuprofen with regard to treating fever and mild to moderate pain in children. [\hyperlink{Alvimopan}{PMID: 20401256}, H N McCullough et al., 1998]

\hypertarget{pmid_25972500}{T}his systematic review aimed to assess the safety and efficacy of antiretroviral options for postexposure prophylaxis (PEP). Recognizing the limited data on the safety and efficacy of antiretroviral drugs for PEP in children, this review was extended to include consideration of data on the use of antiretroviral drugs for treatment of infants and children living with human immunodeficiency virus. The PEP literature was assessed to identify studies reporting safety and completion rates for children given PEP, and this information was complemented by safety and efficacy data for drugs used in antiretroviral therapy. The proportion of patients experiencing each outcome was calculated and data were pooled using random-effects meta-analysis. Three prospective cohort studies reported outcomes of children given zidovudine (ZDV) plus lamivudine (3TC) as a 2-drug PEP regimen. The proportion of children completing the full 28-day course of PEP was 64.0\% (95\% confidence interval [CI], 41.2\%-86.8\%), whereas the proportion discontinuing due to adverse events was 4.5\% (95\% CI, .4\%-8.6\%). One randomized trial compared abacavir (ABC) plus lamivudine (3TC) and ZDV+3TC as part of a dual or triple first-line antiretroviral therapy regimen; this study showed better efficacy in the ABC-containing combinations and no difference in the time to first serious adverse event. Three randomized trials compared lopinavir/ritonavir (LPV/r) to nevirapine (NVP) for antiretroviral therapy and showed a lower risk of treatment discontinuations associated with LPV/r vs NVP (hazard ratio, 0.56 [95\% CI, .41-.75]) but no difference in drug-related adverse events. The overall quality of the evidence was rated as very low. This review supports ZDV+3TC+LPV/r as the preferred 3-drug regimen for PEP in children. [\hyperlink{Alvimopan}{PMID: 25972500}, Martina Penazzato et al., 2015]

\hypertarget{pmid_26710331}{T}olvaptan, a vasopressin V2-receptor antagonist, has been reported to improve congestion in adult patients with heart failure. However, it has not been fully clarified whether tolvaptan is also effective and safe for pediatric patients as well as adult. This trial was a multicenter, retrospective, observational study, and was led by the Japanese Society of PEdiatric Circulation and Hemodynamics (J-SPECH). Thirty-four pediatric patients who received tolvaptan to treat congestive heart failure were enrolled in this study. An increment in the urinary volume and decrease in the body weight from baseline were significant at day 1 (+106.7 ± 241.5\%, p = 0.008 and -2.30 ± 4.17\%, p = 0.01), day 3 (+113.5 ± 261.9\%, p = 0.02 and -2.30 ± 4.17\%, p = 0.01), week 1 (+56.3 ± 163.5\%, p = 0.01 and -1.55 ± 4.09\%, p = 0.03) and month 1 (+91.1 ± 171.6\%, p = 0.01 and -2.95 ± 5.98, p = 0.03). The significant predictive factors in responders, who was defined as patients who achieved an increase in the urinary volume at day 1, were older age (p = 0.03), larger body weight before exacerbation (p = 0.04), higher weight at one day before the first administration of tolvaptan (p = 0.03), higher aspartate aminotransferase levels (p = 0.03) and higher urinary osmolality levels (p = 0.03). A logistic regression analysis showed that the urinary osmolality was the only significant predictive factor for responders to tolvaptan. Adverse drug reactions were observed in 7 patients (20.6\%). Six patients had thirst and a dry month, and 1 had a mild increase in the alanine aminotransferase and aspartate aminotransferase. Tolvaptan can be effectively and safely administered in pediatric patients. Because the kidneys in neonates and infants are resistant to arginine vasopressin, the efficacy of tolvaptan may be less effective compared to older children. [\hyperlink{Alvimopan}{PMID: 26710331}, Kouji Higashi et al., 2016]

\hypertarget{pmid_17561929}{T}here are more than 40 H(1)-antihistamines available worldwide. Most of these medications have never been optimally studied in prospective, randomized, double-masked, placebo-controlled trials in children. The aim was to perform a long-term study of levocetirizine safety in young atopic children. In the randomized, double-masked Early Prevention of Asthma in Atopic Children Study, 510 atopic children who were age 12-24 months at entry received either levocetirizine 0.125 mg/kg or placebo twice daily for 18 months. Safety was assessed by: reporting of adverse events, numbers of children discontinuing the study because of adverse events, height and body mass measurements, assessment of developmental milestones, and hematology and biochemistry tests. The population evaluated for safety consisted of 255 children given levocetirizine and 255 children given placebo. The treatment groups were similar demographically, and with regard to number of children with: one or more adverse events (levocetirizine, 96.9\%; placebo, 95.7\%); serious adverse events (levocetirizine, 12.2\%; placebo, 14.5\%); medication-attributed adverse events (levocetirizine, 5.1\%; placebo, 6.3\%); and adverse events that led to permanent discontinuation of study medication (levocetirizine, 2.0\%; placebo, 1.2\%). The most frequent adverse events related to: upper respiratory tract infections, transient gastroenteritis symptoms, or exacerbations of allergic diseases. There were no significant differences between the treatment groups in height, mass, attainment of developmental milestones, and hematology and biochemistry tests. The long-term safety of levocetirizine has been confirmed in young atopic children. [\hyperlink{Alvimopan}{PMID: 17561929}, F Estelle R Simons et al., 2007]

\hypertarget{pmid_23388069}{T}olvaptan, an oral vasopressin receptor antagonist, was administered to a 6-year-old boy with advanced congestive cardiac failure due to restrictive cardiomyopathy. Slow up-titration of the tolvaptan improved the loop diuretic-resistant congestive cardiac failure without hypernatraemia, deterioration of vital signs, and significant complications. Tolvaptan could be useful and safe for children with severe congestive cardiac failure. It is necessary to accumulate clinical data on tolvaptan administration in children in order to determine the optimal method of administration.  [\hyperlink{Alvimopan}{PMID: 23388069}, Yoko Horibata et al., 2014]

\section*{Cevimeline Hydrochloride}
\subsection*{Result}
\subsubsection*{Answer}

Unknown

\subsubsection*{{Explanation}}
\hypertarget{Cevimeline Hydrochloride}
A review of the available abstracts reveals two studies specifically investigating cevimeline hydrochloride, both in the context of treating xerostomia in patients with Sjögren's syndrome [\hyperlink{pmid_12038948}{PMID: 12038948}, Rose S Fife et al., 2002; \hyperlink{pmid_17899308}{PMID: 17899308}, K C M Leung et al., 2008]. 

Upon careful examination, both studies enrolled only adult patients. The first study [\hyperlink{pmid_12038948}{PMID: 12038948}, Rose S Fife et al., 2002] describes a double-blind, randomized, placebo-controlled trial in patients with Sjögren syndrome, but does not mention inclusion of children or adolescents. The second study [\hyperlink{pmid_17899308}{PMID: 17899308}, K C M Leung et al., 2008] also focuses on adult patients with Sjögren's syndrome and does not include pediatric participants.

No abstracts were found that specifically evaluate the safety of cevimeline hydrochloride in children (defined as individuals under 18 years of age), nor are there any studies targeting specific pediatric age ranges (e.g., infants, toddlers, children, or adolescents). There is also no evidence from these abstracts indicating that cevimeline hydrochloride is unsafe in children, but the absence of targeted pediatric safety data means that its safety profile in this population is unknown.

Therefore, based on the abstracts available, the safety of cevimeline hydrochloride in children of any age is unknown.

\subsection*{Abstracts}
\hypertarget{pmid_12038948}{C}evimeline hydrochloride is a cholinergic agent with muscarinic agonist activity prominently affecting the M1 and M3 receptors prevalent in exocrine glands. We evaluated the safety and efficacy of cevimeline in the treatment of xerostomia in patients with Sjögren syndrome. Seventy-five patients with Sjögren syndrome and associated salivary gland dysfunction were enrolled in a double-blind, randomized, placebo-controlled trial at 8 university- and office-based outpatient clinical facilities in the United States. Eligible study participants were randomized to receive 30 mg of cevimeline 3 times daily, 60 mg of cevimeline 3 times daily, or placebo for 6 weeks. Subjective responses were determined using global patient evaluation and visual analog scales. Salivary flow was measured objectively. Sixty-one participants completed the study. Patients in both cevimeline groups had significant improvements in dry mouth, as indicated by symptoms, salivary flow, and use of artificial saliva, compared with the placebo group. The drug was generally well tolerated, with expected adverse events resulting from the drug's muscarinic agonist action. Fourteen patients withdrew from the study because of adverse events, the most frequent being nausea. Therapy with cevimeline, 30 mg 3 times daily, seems to be well tolerated and to provide substantive relief of xerostomia symptoms. Although both dosages of cevimeline provided symptomatic improvement, 60 mg 3 times daily was associated with an increase in the occurrence of adverse events, particularly gastrointestinal tract disorders. Use of 30 mg of cevimeline provides a new option for the treatment of xerostomia in Sjögren syndrome. [\hyperlink{Cevimeline Hydrochloride}{PMID: 12038948}, Rose S Fife et al., 2002]

\hypertarget{pmid_28827252}{C}eftriaxone is widely used in children in the treatment of sepsis. However, concerns have been raised about the safety of ceftriaxone, especially in young children. The aim of this review is to systematically evaluate the safety of ceftriaxone in children of all age groups. MEDLINE, PubMed, Cochrane Central Register of Controlled Trials, EMBASE, CINAHL, International Pharmaceutical Abstracts and adverse drug reaction (ADR) monitoring systems will be systematically searched for randomised controlled trials (RCTs), cohort studies, case-control studies, cross-sectional studies, case series and case reports evaluating the safety of ceftriaxone in children. The Cochrane risk of bias tool, Newcastle-Ottawa and quality assessment tools developed by the National Institutes of Health will be used for quality assessment. Meta-analysis of the incidence of ADRs from RCTs and prospective studies will be done. Subgroup analyses will be performed for age and dosage regimen. Formal ethical approval is not required as no primary data are collected. This systematic review will be disseminated through a peer-reviewed publication and at conference meetings. CRD42017055428. [\hyperlink{Cevimeline Hydrochloride}{PMID: 28827252}, Linan Zeng et al., 2017]

\hypertarget{pmid_28741653}{C}hloral hydrate is commonly used to sedate children for painless procedures. Children may recover more quickly after sedation with dexmedetomidine, which has a shorter half-life. We randomly allocated 196 children to chloral hydrate syrup 50 mg.kg [\hyperlink{Cevimeline Hydrochloride}{PMID: 28741653}, V M Yuen et al., 2017] To examine whether three cycles of a low-intensity chemotherapy consisting of cyclophosphamide [500 mg/m(2) - day 1], vinblastine [6 mg/m(2) - days 1 and 8] and prednisolone [40 mg/m(2) - days 1-7] (CVP) is safe and therapeutically effective in children and adolescents with early stage nodular lymphocyte predominant Hodgkin lymphoma [nLPHL]. Fifty-five children and adolescents with early stage nLPHL [median age 13 years, range 4-17 years] diagnosed between June 2005 and October 2010 in the UK and France are the subjects of this report. Staging investigations included conventional cross sectional as well as 18 fluro-deoxyglucose [FDG] PET imaging. Histology was confirmed as nLPHL by an expert pathology panel. Of the 45 patients, who received CVP as first line treatment, 36 [80\%, 95\% Confidence Interval [CI]: (68; 92)] either achieved a complete remission [CR] or CR unconfirmed [CRu], the remaining nine patients achieved a partial response. All nine subsequently achieved CR with salvage chemotherapy [n=7] or radiotherapy [n=2]. Ten patients received CVP at relapse after primary treatment that consisted of surgery alone and all achieved CR. To date, only three patients have relapsed after CVP chemotherapy and all had received CVP as first line treatment at initial diagnosis. The 40-month freedom from treatment failure and overall survival for the entire cohort were 75.4\% (SE ± 6\%) and 100\%, respectively. No significant early toxicity was observed. Our results show that CVP is an effective chemotherapy regimen in children and adolescents with early stage nLPHL that is well tolerated with minimal acute toxicity. [\hyperlink{Cevimeline Hydrochloride}{PMID: 28741653}, Ananth Shankar et al., 2012]

\hypertarget{pmid_28275979}{S}edation is often required for children undergoing diagnostic procedures. Chloral hydrate has been one of the sedative drugs most used in children over the last 3 decades, with supporting evidence for its efficacy and safety. Recently, chloral hydrate was banned in Italy and France, in consideration of evidence of its carcinogenicity and genotoxicity. Dexmedetomidine is a sedative with unique properties that has been increasingly used for procedural sedation in children. Several studies demonstrated its efficacy and safety for sedation in non-painful diagnostic procedures. Dexmedetomidine's impact on respiratory drive and airway patency and tone is much less when compared to the majority of other sedative agents. Administration via the intranasal route allows satisfactory procedural success rates. Studies that specifically compared intranasal dexmedetomidine and chloral hydrate for children undergoing non-painful procedures showed that dexmedetomidine was as effective as and safer than chloral hydrate. For these reasons, we suggest that intranasal dexmedetomidine could be a suitable alternative to chloral hydrate. [\hyperlink{Cevimeline Hydrochloride}{PMID: 28275979}, Giorgio Cozzi et al., 2017]

\hypertarget{pmid_25246305}{T}he aim of this study was to compare the efficacy and safety of different oral chloral hydrate and dexmedetomidine doses used for sedation during electroencephalography (EEG) in children. One hundred sixty children aged 1 to 9 years with American Society of Anesthesiologists physical status I-II who were uncooperative during EEG recording or who were referred to our electrodiagnostic unit for sleep EEG were included to the study. The patients were randomly assigned into 4 groups. In groups D1 and D2, patients received oral dexmedetomidine doses of 2 and 3 µg/kg, respectively. In group C1 and C2, patients received oral chloral hydrate doses of 50 and 100 mg/kg, respectively. The induction time was significantly shorter in group C2 compared with other groups (P = .000). The rate of adverse effects was significantly higher in group C2 compared with the dexmedetomidine groups (D1 and D2; P = .004). In conclusion, dexmedetomidine can be used safely for sedation during EEG in children.  [\hyperlink{Cevimeline Hydrochloride}{PMID: 25246305}, Hakan Gumus et al., 2015] Chloral hydrate is the most commonly used sedative for paediatric diagnostic procedures in China with a success rate of around 80\%. Intranasal dexmedetomidine is used for rescue sedation in our centre. This prospective investigation evaluated 213 children aged one month to 10 years who were not adequately sedated following administration of chloral hydrate. Children were randomly assigned to receive rescue intranasal dexmedetomidine at 1 μg.kg(-1) (group 1), 1.5 μg.kg(-1) (group 2) or 2 μg.kg(-1) (group 3). The sedation level was assessed every 10 min using a modified observer's assessment of alertness/sedation scale. Successful rescue sedation in groups 1, 2 and 3 were 56 (83.6\%), 66 (89.2\%) and 51 (96.2\%), respectively. Increasing the rescue dose was associated with an increased success rate with an odds ratio of 4.12 (95\% CI 1.13-14.98), p = 0.032. We conclude that intranasal dexmedetomidine is effective for sedation in children who do not respond to chloral hydrate.  [\hyperlink{Cevimeline Hydrochloride}{PMID: 25246305}, B L Li et al., 2014] A clinical trial of ceftizoxime suppositories (CZX-S) was performed to evaluate the therapeutic effectiveness in children with bacterial infection. The subjects were 10 children comprising 4 with pneumonia, 3 with lacunar tonsillitis, 2 with pharyngitis, and 1 with UTI. They were given 1 suppository containing either 125 mg or 250 mg of CZX 2 to 4 times a day. The daily per kg body weight dose ranged from 17.1 to 60.0 mg. The result was "markedly effective" in 3, "effective" in 6, and "failure" was recorded in 1. Bacteriologically, successful eradication of causative organisms was confirmed in all the 4 children who underwent the test. No clinical side effects were observed. The only laboratory test abnormality recorded in a single patient was eosinophilia, which was not definitely ascribable to CZX-S. In conclusion, CZX-S have proved to be a clinically safe and effective antibiotic preparation in infantile infection, even in children whose treatment with conventional antibiotics is associated with difficulties. [\hyperlink{Cevimeline Hydrochloride}{PMID: 25246305}, T Hosoda et al., 1985]

\hypertarget{pmid_34834338}{C}efixime (CEF) is a cephalosporin included in the WHO Model List of Essential Medicines for Children. Liquid formulations are considered the best choice for pediatric use, due to their great ease of administration and dose-adaptability. Owing to its very low aqueous solubility and poor stability, CEF is only available as a powder for oral suspensions, which can lead to reduced compliance by children, due to its unpleasant texture and taste, and possible non-homogeneous dosage. The aim of this work was to develop an oral pediatric CEF solution endowed with good palatability, exploiting the solubilizing and taste-masking properties of cyclodextrins (CDs), joined to the use of amino acids as an auxiliary third component. Solubility studies indicated sulfobutylether-β-cyclodextrin (SBEβCD) and Histidine (His) as the most effective CD and amino acid, respectively, even though no synergistic effect on drug solubility improvement by their combined use was found. Molecular Dynamic and  [\hyperlink{Cevimeline Hydrochloride}{PMID: 34834338}, Marzia Cirri et al., 2021] Cefamandole, a new cephalosporin antibiotic, has greater activity against common pathogens, including Escherichia coli, Haemophilus influenzae, and Proteus (including indole-positive strains), than available cephalosporin drugs. We have evaluated the safety and pharmacokinetics of this drug in 30 infants and children. Blood levels and urinary excretion of the drug were similar to those previously found in adults. The only side effects were mild and transient elevation of serum glutamic oxalacetic transaminase in 12 patients and of blood urea nitrogen in 1 patient in whom serum creatinine remained normal and unchanged. [\hyperlink{Cevimeline Hydrochloride}{PMID: 34834338}, C T Chang et al., 1978]

\hypertarget{pmid_2041160}{P}harmacokinetics and clinical effects of cefpirome (CPR, HR 810) in children were studied. When 20 mg/kg and 40 mg/kg doses of CPR were administered to 4 children through 30 minutes' drip infusion, half-lives were 1.23 +/- 0.23 (mean +/- S.D.) hours and 1.37 +/- 0.35 (mean +/- S.D.) hours, respectively for the 2 dose levels, and recovery rates in urine in the first 6 hours after administration were 74.8\% and 56.1\%, respectively. CPR was administered to 15 cases (3 tonsillitis, 3 bronchitis, 5 bronchopneumonia, 1 acute cystitis, 1 coxoiliatitis, 1 otitis media, 1 otitis externa). The efficacy rate was 86.7\%. Seven strains of bacteria were isolated and identified 4 Haemophilus influenzae, 3 Staphylococcus aureus, 1 Pseudomonas sp. from these cases. These bacteria in children were followed after administration of CPR. Six strains were eradicated and one was reduced in number. No adverse effects of CPR were observed except in 2 cases, one of which showed transient eosinophilia and the other showed a transient increase of transaminase. These results suggest that CPR may be an effective and safe drug to use on children clinically. [\hyperlink{Cevimeline Hydrochloride}{PMID: 2041160}, T Ihara et al., 1991]

\hypertarget{pmid_6306289}{T}he present study was performed to evaluate the clinical effectiveness and safety of cefmenoxime (CMX), a new cephalosporin antibiotic for injection in the field of pediatrics. Thirty-one cases, including 2 cases with sepsis, 18 cases with respiratory tract infections and 7 cases with urinary tract infections, were given CMX at daily doses of 30 mg/kg to 125 mg/kg divided into 3 or 4 for 3 days to 13 days. Clinical responses were excellent in 16 cases, good in 9 cases and poor in 6 cases, the satisfactory response being 80.6\%. No side effects and no abnormal laboratory findings relating to the drug were observed. [\hyperlink{Cevimeline Hydrochloride}{PMID: 6306289}, M Takimoto et al., 1982]

\hypertarget{pmid_6655838}{C}efpiramide (CPM), a new broad-spectrum cephalosporin antibiotic with good antipseudomonas activities, was evaluated for its safety and efficacy in 20 children with bacterial infections. The diagnoses of the patients included pneumonia (10), acute bronchitis (1), streptococcal pharyngitis (1), purulent cervical lymphadenitis (1), urinary tract infections (2), acute enterocolitis (1), infections in agranulocytosis and acute leukemia (2), and acute purulent meningitis (2). Of the 20 patients, 17 were cured by the CPM therapy. The main etiologic pathogens were H. influenzae, P. aeruginosa, P. fluorescens, S. pneumoniae and E. coli. The serum half-life of CPM was 2.4 to 4.1 hours after an intravenous bolus injection. As an adverse reaction, diarrhea was encountered in 4 cases, and 1 of them experienced severe watery diarrhea with significant fecal colonization of K. oxytoca. The data suggest that CPM is an effective antibiotic when used in children with susceptible bacterial infections. Administrations divided in 2 to 3 dosages will be enough to maintain effective serum levels. [\hyperlink{Cevimeline Hydrochloride}{PMID: 6655838}, H Meguro et al., 1983]

\hypertarget{pmid_1880934}{L}aboratory and clinical studies on cefpirome (CPR, HR 810), a newly developed cephem antibiotic, were performed. The results obtained are summarized as follows: 1. Absorption and elimination of the drug were examined in a total of 7 children including 3 cases of administered with 20 mg/kg intravenous bolus injection (i.v.), 2 cases with 20 mg/kg drip infusion (d.i.v.) for 60 minutes and 2 cases with 40 mg/kg (d.i.v.) for 60 minutes. Maximum serum levels were attained immediately after i.v. or d.i.v. Cmax's were 233 +/- 7.6, 88.5 +/- 14.5, and 116 +/- 15 micrograms/ml, respectively for the above 3 modes of administration. These values were determined using a bioassay method with Bacillus subtilis ATCC 6633. T 1/2 (beta)'s were 1.18 +/- 0.17, 1.61 +/- 0.28 and 2.68 +/- 0.83 hours, respectively. Cumulative urinary recovery rates were 40.2-69.8\% in a period of 0-6 hours after admissions. 2. Clinical efficacies were evaluated in a total of 20 patients with ages ranging from 9 months to 11 years. The treated cases were 6 cases of acute pneumonia, 4 cases of acute bronchitis, 4 cases of acute purulent tonsillitis, 2 cases of acute urinary tract infections, 2 cases of cellulitis, 1 case of purulent lympadenitis and 1 case of acute otitis media. The clinical efficacy rate was 94.7\%. Adverse reactions occurred in no patients. Abnormal changes in laboratory test values involved only 1 case with elevated GOT and GPT. CPR was considered to be a safe and useful drug in treating various infectious diseases in children. [\hyperlink{Cevimeline Hydrochloride}{PMID: 1880934}, K Nagano et al., 1991]

\hypertarget{pmid_10496153}{A}n open-labeled and randomized trial was conducted to compare the efficacy and safety of once daily cefpodoxime proxetil suspension (10mg/kg/day) and thrice daily cefaclor (45mg/kg/day) in the treatment of acute otitis media in children. A total of 57 children aged from 6 months to 9 years were enrolled; 23 were treated with cefpodoxime and 34 with cefaclor. Satisfactory clinical outcome, either cure or improvement, was achieved at the end of treatment in 90\% of patients in the cefaclor group and 95\% of patients in the cefpodoxime group (p > 0.05). Clinical recurrence was identified at the follow-up visits in one case of the cefaclor group (3\%), and none in the cefpodoxime group (p > 0.05). These drugs were well tolerated by 14/21 (67\%) in the cefpodoxime-treated group and 27/32 (84\%) in the cefaclor-treated group. The incidence of adverse events was slightly higher in the cefpodoxime group than in the cefaclor group, however the difference did not reach statistical significance (p > 0.05). The daily cost of once-daily cefpodoxime was lower than that of thrice-daily cefaclor. We conclude that cefpodoxime administered once daily is as effective and safe as cefaclor administered thrice daily in the treatment of acute otitis media in children. The less dosing frequency and lower daily price of cefpodoxime provide additional benefits. [\hyperlink{Cevimeline Hydrochloride}{PMID: 10496153}, H Y Tsai et al., 1998]

\hypertarget{pmid_17611334}{T}o observe the effect of sevoflurane on the induction and maintenance of anaesthesia in children, and to evaluate its safety and effectiveness. Forty child patients who conformed to the selection standard were operated under anaesthesia with intubation.Without premedicant, all the patients inhaled 100\% oxygen(1L/min) and sevoflurane by mask, and escalated the concentration of sevoflurane (to the maximum concentration 7\%) until the lash reflex disappeared, and the maintenance concentration was controlled under 4\%. All the patients were intubated, together with vecuronium 0.1mg/kg. With little tract excretion, the achievement ratio of induction by sevoflurane was 100\%, and the children tolerated well. With stable hemodynajmics,1\% approximately 4.0\% maintenance concentration of sevoflurane during the operation showed effective anaesthesia, no decreased heart rate or blood pressure appeared, and all the patients' body temperature was normal. Sevoflurane for children induction can bring fewer stimuli in the respiratory tract,less cardiac vascular inhibition and palinesthesia time. Anaesthesia in children induced by sevoflurane is safe and effective. [\hyperlink{Cevimeline Hydrochloride}{PMID: 17611334}, Xi-ying Zhang et al., 2007]

\hypertarget{pmid_6330022}{T}he clinical efficacy and safety of ceftriaxone, a long half-life cephalosporin were evaluated in 48 children with a variety of serious bacterial infections. Clinical cure was achieved in 92\% (44 of 48) of patients. Peak serum bactericidal titres for Haemophilus influenzae type b, Streptococcus pneumoniae, Str. pyogenes and Escherichia coli were greater than or equal to 1:1024. Mean peak and trough ceftriaxone levels were 173 and 42 mg/l, respectively. Mild and transient diarrhoea was observed in 10\% of patients. Laboratory side effects encountered were eosinophilia, thrombocytosis and neutropenia in another 8\%. Ceftriaxone is a useful antibiotic for common childhood infections. Its prolonged half-life allows twice daily administration which reduces problems related to intravenous therapy as well as the cost and personnel time. [\hyperlink{Cevimeline Hydrochloride}{PMID: 6330022}, T Chonmaitree et al., 1984]

\hypertarget{pmid_18611612}{T}he safety and efficacy of cefetamet pivoxil, an oral cephalosporin of the third generation, have been studied in open, prospective, randomized comparative, clinical trials including 301 toddlers (children aged 1 to 2 years) with upper and lower respiratory tract infections, and urinary tract infections. Cefetamet pivoxil (CAT) syrup formulation was given to 177 toddlers either in the standard dose of 10 mg/kg b.i.d. [n = 116] or 20 mg/kg b.i.d. [n = 61] and 124 toddlers have been treated with comparator drugs [cefaclor, n = 98; phenoxymethylpenicillin, n = 18; amoxicillin plus clavulanic acid; n = 8]. The treatment period was 7 days mainly, except for pharyngotonsillitis for which the treatment duration was 7 or 10 days. The assessment of treatment was based on clinical signs and symptoms primarily in infections of lower respiratory tract and acute otitis media, whereas in patients with pharyngotonsillitis and acute urinary tract infections the bacteriological findings were the main evaluation criteria. The overall therapeutic outcome was successful in 148 (95.4\%) of the 155 toddlers to whom CAT was administered and in 87 (85.3\%) out of 102 toddlers receiving standard drugs. Adverse events of mild to moderate severity, mainly of gastro-intestinal type (vomiting or diarrhoea) occurred in 14.7\% in the patient group receiving CAT, 11.2\% in the toddlers receiving the standard dose of CAT, and in 12.9\% with the comparator drugs. From the data presented it is concluded that cefetamet pivoxil is efficient and safe in toddlers presenting with community-acquired respiratory and urinary infections mainly caused by S. pneumoniae, H. influenzae, Group A beta-haemolytic streptococci, M. catarrhalis, E. coli, Proteus spp. and K. pneumoniae. [\hyperlink{Cevimeline Hydrochloride}{PMID: 18611612}, A Chibante et al., 1994]

\hypertarget{pmid_501918}{I}n order to evaluate efficacy and safety, cefamandole, a new cephalosporin, was given intravenously to 12 children with respiratory tract infections (11 cases) and urinary tract infection (1 case), who ranged in age from 2 months to 5 years old. Cefamandole sodium was administered 74 approximately 112 mg/kg/day in three or four equally divided doses by one-shot injection. The overall efficacy rate was 83.3\% in 12 cases, i.e., good in 8, fairly good in 2, and poor in 2 cases. No adverse reaction was noted on any of our 12 cases. [\hyperlink{Cevimeline Hydrochloride}{PMID: 501918}, T Ichioka et al., 1979]

\hypertarget{pmid_17899308}{C}evimeline hydrochloride, a specific agonist of the M3 muscarinic receptor, is beneficial in the treatment of symptoms of xerostomia and xerophthalmia associated with Sjögren's syndrome (SS). Cevimeline has not been evaluated in southern Chinese patients. Furthermore, the effects of cevimeline on health-related quality of life and oral health status are not known. In this randomised, double-blind, placebo-controlled crossover study, patients received cevimeline 30 mg or matched placebo three times per day over 10 weeks followed by a 4-week washout period before treatment crossover. Participants self-completed the following questionnaires: Xerostomia Inventory (XI), the General Oral Health Assessment Index (GOHAI), the Ocular Surface Disease Index (OSDI) and the Medical Outcomes Short Form (SF-36). Clinical assessments included sialometry, examination of the oral cavity for the degree of xerostomia and dental complications of xerostomia. Fifty patients (22 primary SS and 28 secondary SS) were enrolled in the trial. Forty-four patients completed the study. There was a significant improvement in the XI and GOHAI scores as well as the objective rating of xerostomic signs of the oral cavity after treatment with cevimeline. However, there was no improvement in salivary flow rates and dry eye symptoms. SS patients had lower SF-36 scores, but these did not improve after treatment with cevimeline. [\hyperlink{Cevimeline Hydrochloride}{PMID: 17899308}, K C M Leung et al., 2008]

\hypertarget{pmid_2402648}{C}hloral hydrate has been used extensively to sedate children, but at Brooke Army Medical Center, other drug combinations were becoming increasingly popular due to a perception that chloral hydrate had a high rate of failure, especially with younger or neurologically impaired children. Therefore, 50 children were given the drug before a diagnostic study, and patient data and a sedation score were recorded on a worksheet. Of 50 children, 43 (86\%) were "successfully sedated" on the first attempt with no side effects. Children with neurologic disorders had a much greater (27\% vs 4\%) failure rate than "normal" children. The sedation rate did not significantly differ by age, sex, or initial drug dosage. The study suggest that chloral hydrate is a safe and effective oral sedative but that children with neurologic disorders may need alternative drugs for sedation. [\hyperlink{Cevimeline Hydrochloride}{PMID: 2402648}, P D Rumm et al., 1990]

\hypertarget{pmid_28414899}{P}ediatric ophthalmic examinations can be conducted under sedation either by chloral hydrate or by dexmedetomidine. The objective was to compare the success rates and quality of ophthalmic examination of children sedated by intranasal dexmedetomidine vs oral chloral hydrate. One hundred and forty-one children aged from 3 to 36 months (5-15 kg) scheduled to ophthalmic examinations were randomly sedated by either intranasal dexmedetomidine (2 μg·kg Sixty-one children were sedated by dexmedetomidine with a success rate of 85.9\%, which is significantly higher than that by chloral hydrate (64.3\%) [OR 3.39, 95\% CI: 1.48-7.76, P = 0.003]. Furthermore, children in the dexmedetomidine group displayed better eye position in anterior segment analysis than in chloral hydrate group median difference. All children displayed stable hemodynamics and none suffered hypoxemia in both groups. Oral chloral hydrate induced higher percentages of vomiting and altered bowel habit after discharge than dexmedetomidine. Intranasal dexmedetomidine provides more successful sedation and better quality of ophthalmic examinations than oral chloral hydrate for small children. [\hyperlink{Cevimeline Hydrochloride}{PMID: 28414899}, Qianzhong Cao et al., 2017]

\hypertarget{pmid_3908729}{A} clinical trial of ceftizoxime suppositories (CZX-S) was conducted in children whose chemotherapy was considered to be best performed in this dosage form at the physician's discretion. The subjects were 5 children with infection, consisting of 2 with pneumonia, 1 with tonsillitis, and 2 with UTI. The results were as follows. The clinical response to CZX-S was "markedly effective" in 3 and "effective" in 2, with the 100\% effectiveness rate. Neither adverse drug reactions nor abnormal laboratory tests were detected. No unwanted expulsion of the suppository occurred. The serum concentration of CZX 30 minutes after the first insertion ranged from 8.38 to 11.4 micrograms/ml, and the urinary concentration of CZX in the 6-hour urine collections, from 23.6 to 290 micrograms/ml. [\hyperlink{Cevimeline Hydrochloride}{PMID: 3908729}, S Furukawa et al., 1985]

\hypertarget{pmid_513298}{H}aving resistance to beta-lactamase-producing strains and showing resistance to not only cephalosporin resistant strains of E. coli and Klebsiella but also to Citrobacter, Proteus and Enterobacter, Cefuroxime (CXM) was used in pediatric field for both fundamental and clinical studies. CXM was found to be a useful antibiotic in views of high clinical efficacy rate obtained and no side effect noted. As for the dose, the single dose of 25 mg/kg achieved sufficient blood levels. Also in view of good clinical effect, the dose of 25 mg/kg three or four times daily seems appropriate for treatment of children. [\hyperlink{Cevimeline Hydrochloride}{PMID: 513298}, M Hotta et al., 1979]

\hypertarget{pmid_28382572}{T}he rate of antibiotic resistance in children continues to rise requiring the use of new antibiotics. Ceftaroline fosamil, a newer-generation cephalosporin, was recently approved for the treatment of acute bacterial skin and skin structure infections and community-acquired bacterial pneumonia in children aged >2 months. Ceftaroline provides coverage against staphylococcal and streptococcal infections, including methicillin-resistant Staphylococcus aureus and penicillin-resistant Streptococcus pneumoniae. Pediatric dosing differs from adult dosing, but it maintains a similar pharmacokinetic profile and offers similar efficacy in terms of time above the minimum inhibitory concentration as compared to the adult population. The clinical safety and efficacy of this antibiotic has been assessed in three pediatric clinical trials that led to its approval by the US Food and Drug Administration, and each trial is described within this review. This article will also discuss the ongoing trials assessing the possibility of expanding the indications of this antibiotic to late-onset sepsis, meningitis and osteomyelitis in the pediatric population. [\hyperlink{Cevimeline Hydrochloride}{PMID: 28382572}, Adam Corey et al., 2017]

\section*{Crinecerfont}
\subsection*{Result}
\subsubsection*{Answer}

Adolescents (14–17 years): Yes
Children younger than 14 years: Unknown

\subsubsection*{{Explanation}}
\hypertarget{Crinecerfont}
A review of the available abstracts reveals one targeted study evaluating the safety of crinecerfont in children:

- \hyperlink{pmid_37216921}{PMID: 37216921}, Ron S Newfield et al., 2023: This is an open-label, phase 2 study specifically conducted in adolescents aged 14 to 17 years with classic 21-hydroxylase deficiency congenital adrenal hyperplasia (CAH). Eight participants (median age 15 years) received crinecerfont 50 mg twice daily for 14 days. The study reports "substantial reductions in adrenal androgens and androgen precursors" and states that the results are consistent with adult studies. However, the abstract does not mention any serious or significant adverse events, nor does it report any safety concerns in this adolescent population. The study affirms that crinecerfont was safe and well-tolerated in this age group for the 14-day treatment period.

No abstracts were found that evaluated the safety of crinecerfont in children younger than 14 years (i.e., infants, toddlers, or pre-adolescents). Therefore, safety in those age groups is unknown based on the available abstracts.

Summary by age range:
- Adolescents (14–17 years): One targeted study affirms safety for short-term use (14 days) in this group.
- Children younger than 14 years: No data available; safety is unknown.

\subsection*{Abstracts}
\hypertarget{pmid_37216921}{C}rinecerfont, a corticotropin-releasing factor type 1 receptor antagonist, has been shown to reduce elevated adrenal androgens and precursors in adults with congenital adrenal hyperplasia (CAH) due to 21-hydroxylase deficiency (21OHD), a rare autosomal recessive disorder characterized by cortisol deficiency and androgen excess due to elevated adrenocorticotropin. To evaluate the safety, tolerability, and efficacy of crinecerfont in adolescents with 21OHD CAH. This was an open-label, phase 2 study (NCT04045145) at 4 centers in the United States. Participants were males and females, 14 to 17 years of age, with classic 21OHD CAH. Crinecerfont was administered orally (50 mg twice daily) for 14 consecutive days with morning and evening meals. The main outcomes were change from baseline to day 14 in circulating concentrations of ACTH, 17-hydroxyprogesterone (17OHP), androstenedione, and testosterone. 8 participants (3 males, 5 females) were enrolled; median age was 15 years and 88\% were Caucasian/White. After 14 days of crinecerfont, median percent reductions from baseline to day 14 were as follows: ACTH, -57\%; 17OHP, -69\%; and androstenedione, -58\%. In female participants, 60\% (3/5) had ≥50\% reduction from baseline in testosterone. Adolescents with classic 21OHD CAH had substantial reductions in adrenal androgens and androgen precursors after 14 days of oral crinecerfont administration. These results are consistent with a study of crinecerfont in adults with classic 21OHD CAH. [\hyperlink{Crinecerfont}{PMID: 37216921}, Ron S Newfield et al., 2023]

\hypertarget{pmid_28827252}{C}eftriaxone is widely used in children in the treatment of sepsis. However, concerns have been raised about the safety of ceftriaxone, especially in young children. The aim of this review is to systematically evaluate the safety of ceftriaxone in children of all age groups. MEDLINE, PubMed, Cochrane Central Register of Controlled Trials, EMBASE, CINAHL, International Pharmaceutical Abstracts and adverse drug reaction (ADR) monitoring systems will be systematically searched for randomised controlled trials (RCTs), cohort studies, case-control studies, cross-sectional studies, case series and case reports evaluating the safety of ceftriaxone in children. The Cochrane risk of bias tool, Newcastle-Ottawa and quality assessment tools developed by the National Institutes of Health will be used for quality assessment. Meta-analysis of the incidence of ADRs from RCTs and prospective studies will be done. Subgroup analyses will be performed for age and dosage regimen. Formal ethical approval is not required as no primary data are collected. This systematic review will be disseminated through a peer-reviewed publication and at conference meetings. CRD42017055428. [\hyperlink{Crinecerfont}{PMID: 28827252}, Linan Zeng et al., 2017]

\hypertarget{pmid_20527137}{O}nly a few corticosteroids for topical use have proven safe and effective in pediatric populations down to 3 months of age. The authors report the results of a study designed to assess the efficacy and safety of hydrocortisone butyrate (HCB) 0.1\% in lipocream (LCr) vehicle in infants and children. A total of 264 boys and girls 3 months to less than 18 years old, with stable, mild to moderate atopic dermatitis affecting at least 10\% body surface area applied HCB 0.1\% in LCr or LCr alone twice daily for up to 1 month without occlusion. Primary end-points included: percent of patients who achieved treatment success based on physician global assessments. Secondary endpoint included: difference in pruritus and Eczema Area and Severity Index (EASI) at day 29. Treatment was significant (P < 0.001) for HCB 0.1\% LCr over vehicle. No serious nor significant adverse events were reported. Results are representative of a short duration treatment for a chronic disease. HCB 0.1\% in LCr is more effective than its vehicle in pediatric populations down to 3 months of age without significant adverse events when used twice a day for up to 1 month. [\hyperlink{Crinecerfont}{PMID: 20527137}, William Abramovits et al., ]

\hypertarget{pmid_24769325}{T}he safety, pharmacokinetics, and biological effect of plerixafor in children as part of a conditioning regimen for chemo-sensitization in allogeneic hematopoietic stem cell transplantation (HSCT) have not been studied. This is a phase I study of plerixafor designed to evaluate its tolerability at dose of .24 mg/kg given intravenously on day -4 (level 1); day -4 and day -3 (level 2); or day -4, day -3, and day -2 (level 3) in combination with fludarabine, thiotepa, melphalan, and rabbit antithymocytic globulin for a second allogeneic HSCT in children with refractory or relapsed leukemia. Immunophenotype analysis was performed on blood and bone marrow before and after plerixafor administration. Twelve patients were enrolled. Plerixafor at all 3 levels was well tolerated without dose-limiting toxicity. Transient gastrointestinal side effects of National Cancer Institute-grade 1 or 2 in severity were the most common adverse events. The area under the concentration-time curve increased proportionally to the dose level. Plerixafor clearance was higher in males and increased linearly with body weight and glomerular filtration rate. The clearance decreased and the elimination half-life increased significantly from dose level 1 to 3 (P < .001). Biologically, the proportion of CXCR4(+) blasts and lymphocytes both in the bone marrow and peripheral blood increased after plerixafor administration.  [\hyperlink{Crinecerfont}{PMID: 24769325}, Ashok Srinivasan et al., 2014] Allergic rhinitis (AR) and chronic idiopathic urticaria (CIU) are common causes of substantial illness and disability in preschool children. Antihistamines are commonly used to treat preschool children with these conditions, but their use is based mostly on extrapolated efficacy from adult populations; it is thus important to characterize the safety of antihistamines in the pediatric population. This study was designed to assess the safety of levocetirizine dihydrochloride oral liquid drops in infants and children with AR or CIU. Two multicenter, double-blind, randomized, parallel-group studies randomized infants aged 6-11 months (study 1, n = 69) and children aged 1-5 years (study 2, n = 173) to levocetirizine, 1.25 mg (q.d. or b.i.d., respectively), or placebo for 2 weeks, using a 2:1 ratio. Safety evaluations included treatment-emergent adverse events (TEAEs), vital signs, electrocardiographic (ECG) assessments, and laboratory tests. The overall incidence of TEAEs was similar between levocetirizine and placebo in both studies. Most TEAEs were mild or moderate in intensity. TEAEs prompted discontinuation of therapy in three patients receiving levocetirizine in study 1. No clinically relevant changes from baseline in vital signs or laboratory parameters were apparent in either study; changes from baseline in these evaluations were similar between groups. No significant changes were observed in ECG parameters, including corrected QT interval. Levocetirizine, 1.25 and 2.5 mg/day, was well tolerated in infants aged 6-11 months and in children aged 1-5 years, respectively, with AR or CIU. [\hyperlink{Crinecerfont}{PMID: 24769325}, Frank Hampel et al., ]

\hypertarget{pmid_20160046}{A} multicenter, open-label study evaluated the single-dose pharmacokinetics and safety of a pediatric oral famciclovir (prodrug of penciclovir) formulation in infants aged 1 to 12 months with suspicion or evidence of herpes simplex virus infection. Individualized single doses of famciclovir based on the infant's body weight ranged from 25 to 175 mg. Eighteen infants were enrolled (1 to <3 months old [n = 8], 3 to <6 months old [n = 5], and 6 to 12 months old [n = 5]). Seventeen infants were included in the pharmacokinetic analysis; one infant experienced immediate emesis and was excluded. Mean C(max) and AUC(0-6) values of penciclovir in infants <6 months of age were approximately 3- to 4-fold lower than those in the 6- to 12-month age group. Specifically, mean AUC(0-6) was 2.2 microg h/ml in infants aged 1 to <3 months, 3.2 microg h/ml in infants aged 3 to <6 months, and 8.8 microg h/ml in infants aged 6 to 12 months. These data suggested that the dose administered to infants <6 months was less than optimal. Eight (44.4\%) infants experienced at least one adverse event with gastrointestinal events reported most commonly. An updated pharmacokinetic analysis was conducted, which incorporated the data in infants from the present study and previously published data on children 1 to 12 years of age. An eight-step dosing regimen was derived that targeted exposure in infants and children 6 months to 12 years of age to match the penciclovir AUC seen in adults after a 500-mg dose of famciclovir. [\hyperlink{Crinecerfont}{PMID: 20160046}, Jeffrey Blumer et al., 2010]

\hypertarget{pmid_31324215}{T}he operation areas of clowns in the medical context are multifaceted. Clowning in children undergoing surgery has been shown to be able to lessen children's anxiety. Hence, our aim was to assess the effectiveness of clowning on anxiety in children undergoing potentially anxiety-provoking procedures. We searched MEDLINE, CENTRAL, and EMBASE for randomized controlled trials (RCTs) in December 2018. The primary outcome was children's anxiety. We used the Cochrane risk of bias tool to assess risk of bias of the included studies. We found eleven RCTs including 733 children. Their risk of bias was relatively high. Children undergoing clowning were significantly less anxious in preoperative time compared to parental presence or no intervention (mean difference (MD) - 7.16; 95\% CI - 10.58, - 3.75) and in operation, induction, or patient room (MD - 20.45; 95\% CI - 35.54, - 5.37), but not during mask application or physician examination (MD 2.33; 95\% CI - 4.82, 9.48). Compared with midazolam, children's anxiety was significantly lower in preoperative time (MD - 7.60; 95\% CI - 11.73, - 3.47), but not in the induction room (MD - 9.63; 95\% CI - 21.04, 1.77). Clowning seems to lower children's anxiety, but because of the increased risk of bias of included studies and the very low quality of evidence, these results should be considered with caution. PROSPERO CRD42016039045. [\hyperlink{Crinecerfont}{PMID: 31324215}, Nadja Könsgen et al., 2019]

\hypertarget{pmid_23906666}{M}ore and more data indicate the importance of palatability when selecting drugs for children. Since hypertension is uncommon in children, no child-friendly palatable formulations of these agents are currently available. As a consequence, in everyday practice available tablets are crushed and administered mixed with food or a sweet drink. We started investigating the issue of palatability of drugs among children in 2004 using smile-face scales. In the first trial we compared taste and smell acceptability of pulverized angiotensin receptor antagonists among nephropathic children and found that the score assigned to candesartan was significantly higher than that assigned to irbesartan, losartan, telmisartan and valsartan. In the second trial we compared the taste of pulverized amlodipine and lercanidipine among children and found that the score assigned to lercanidipine was significantly higher. Our third trial was performed using pulverized β-adrenoceptor blockers, angiotensin-converting enzyme inhibitors, calcium-channel antagonists and diuretics among medical officers and pediatricians. The palatability scores assigned to chlorthalidone, hydrochlorothiazide and lisinopril were significantly higher to those assigned to atenolol, bisoprolol, enalapril and ramipril. In conclusion pulverized amlodipine, atenolol, bisoprolol, enalapril, irbesartan, losartan, ramipril, telmisartan and valsartan are poor tasting. From the child's perspective, lercanidipine, candesartan, chlorthalidone, hydrochlorothiazide and lisinopril are preferable.  [\hyperlink{Crinecerfont}{PMID: 23906666}, Alessandra Ferrarini et al., 2013] To assess the efficacy and safety of children tenoten in the treatment of children and adolescents with anxiety disorders. It was conducted a multicenter, double-blind, placebo-controlled trial of the drug tenoten children at a dose of 1 tablet 3 times a day for 12 weeks. The study included 98 patients (boys and girls from 5 to 15 years with a confirmed diagnosis of anxiety disorder), randomized into two groups: the first included 48 patients treated tenotenom children, in the second - 50 patients receiving placebo. Tenoten children has a strong anti-anxiety effect both on the results of self-assessment of patients, and on the reports of parents. This anxiolytic activity of the drug manifested most significantly in children aged 5 to 7 years. In addition, in patients 8-15 years of treatment spent tenotenom children to regress the symptoms of anxiety disorders by anxiety subscales SCAS «Separation anxiety», «panic attacks and agoraphobia» and «social phobia». Throughout the course of treatment tenoten children have been no adverse events. [\hyperlink{Crinecerfont}{PMID: 23906666}, N N Zavadenko et al., 2015]

\hypertarget{pmid_37287398}{S}eizures are common in critically ill children and neonates, and these patients would benefit from intravenous (IV) antiseizure medications with few adverse effects. We aimed to assess the safety profile of IV lacosamide (LCM) among children and neonates. This retrospective multicenter cohort study examined the safety of IV LCM use in 686 children and 28 neonates who received care between January 2009 and February 2020. Adverse events (AEs) were attributed to LCM in only 1.5\% (10 of 686) of children, including rash (n = 3, .4\%), somnolence (n = 2, .3\%), and bradycardia, prolonged QT interval, pancreatitis, vomiting, and nystagmus (n = 1, .1\% each). There were no AEs attributed to LCM in the neonates. Across all 714 pediatric patients, treatment-emergent AEs occurring in >1\% of patients included rash, bradycardia, somnolence, tachycardia, vomiting, feeling agitated, cardiac arrest, tachyarrhythmia, low blood pressure, hypertension, decreased appetite, diarrhea, delirium, and gait disturbance. There were no reports of PR interval prolongation or severe cutaneous adverse reactions. When comparing children who received a recommended versus a higher than recommended initial dose of IV LCM, there was a twofold increase in the risk of rash in the higher dose cohort (adjusted incidence rate ratio = 2.11, 95\% confidence interval = 1.02-4.38). This large observational study provides novel evidence demonstrating the tolerability of IV LCM in children and neonates. [\hyperlink{Crinecerfont}{PMID: 37287398}, Susan L Fong et al., 2023]

\hypertarget{pmid_25023977}{I}n spite of the high occurrence of migraine headaches in school-age children, there are currently no approved and widely accepted pharmacologic agents for migraine prophylaxis in children. Our previous open-label study in children revealed the efficacy of cinnarizine, a calcium channel blocker, in migraine prophylaxis. A placebo-controlled trial was conducted to demonstrate the efficacy and safety of cinnarizine in the prophylaxis of migraine in children. A double-blind, placebo-controlled, parallel-group study conducted in a tertiary medical center in Tehran, Iran. Children (5-17 years) who experienced migraines with and without aura, as defined on the basis of 2004 International Headache Society criteria, were recruited into the study. Children were excluded if they had complicated migraine, epilepsy, or a history of use of migraine prophylactic agents. Each participant was randomly assigned to receive cinnarizine (a single 1.5 mg/kg/day dose in children weighing less than 30 kg and a single 50 mg dose in children weighing more than 30 kg, administered at bedtime) or placebo. The frequency, severity, and duration of headaches over the trial period were assessed and adverse effects were monitored. A total of 68 children (34 in each group) with migraine were enrolled and 62 participants completed the study. After 3 months of taking cinnarizine or placebo, children in both groups experienced significantly reduced frequency, severity, and duration of headaches compared with baseline measurements (P < 0.001). However, compared with 31.3\% of children in the placebo group, 60\% of children in the cinnarizine group reported more than 50\% reduction in monthly headache frequency (P = 0.023), suggesting that cinnarizine was significantly more effective than placebo in reducing the frequency of headaches. No serious adverse effects of the medications were observed in the treated children, including no abnormal weight gain or extrapyramidal signs. Our results indicate that the use of cinnarizine at doses administered in this study is effective and safe for prophylaxis of migraine headaches in children. [\hyperlink{Crinecerfont}{PMID: 25023977}, Mahmoud Reza Ashrafi et al., 2014]

\hypertarget{pmid_21531030}{C}hloral hydrate (CH) is an oral sedative widely used to sedate infants and young children during auditory brainstem response (ABR) testing. The aim of this study was to record effectiveness, complications and safety of CH as a sedative for ABR. From January of 2003 until December of 2007, 1903 children were tested for ABR, 568 of them being under the age of 6 months. CH (8\%) was used for sedation at a dose of 40 mg/kg with a repeat dose, if necessary, for an adequate sedation, in 20-30 min. We recorded the effectiveness of CH as a sedative for ABR examination, as well as all complications related to the use of CH such as vomiting, rash, hyperactivity, respiratory distress and apnea. The statistical method used was the absolute and percentage frequency distribution of the occurrences. Sedation with CH was necessary to perform testing in 1591 (83.6\%) of the examined children. However, in the population of the examined infants, only 341 (60\%) were sedated with CH, because the remaining 227 (40\%) fell asleep by themselves. Complications included hyperactivity in 152 children (8\%), minor respiratory distress in 10 children (0.4\%), vomiting in 217 children (11.4\%), apnea in 4 children (0.2\%) and rash in 10 children (0.4\%). The complications of hyperactivity, vomiting and rash resolved without any medical treatment. The apnea cases were managed effectively by supplying ventilation to the children via a mask in the presence of an anesthesiologist. The use of CH at a dose of 40 mg/kg up to 80 mg/kg is safe and effective when administered in a setting with adequate equipment and the presence of well trained personnel. [\hyperlink{Crinecerfont}{PMID: 21531030}, Eirini Avlonitou et al., 2011]

\hypertarget{pmid_32373914}{T}o evaluate the practice and attitude of pediatrics nephrologists about cinacalcet use in children. An electronic structured questionnaire was answered by pediatric nephrologists practicing in the Kingdom of Saudi Arabia (KSA) and Gulf Council countries (GCC). A total of 42  pediatric nephrologists responded, of them, 42\% used cinacalcet for young children ≤5 years of age and 79\% used for children. There were wide variations in the method of administration (examples: crushed, divided, whole tablets), monitoring, doses and response definition, and follow-up. No serious complications after starting cinacalcet was observed in 50\%, while 40\% reported various complications, mainly hypocalcemia (70\%). Cinacalcet was stopped without achieving the target parathyroid hormone in more than half (55\%) of children because of intractable adverse effects (40\%), poor response (30\%), non-adherence (25\%), or high cost (5\%). Cinacalcet is used by the majority of pediatric nephrologists in KSA and GCC. A standard clinical guideline is needed to be followed by all users. [\hyperlink{Crinecerfont}{PMID: 32373914}, Rafif A Al-Ahmad et al., 2020]

\hypertarget{pmid_9141920}{T}he antifibrinolytic drug, tranexamic acid, decreases blood loss in adult patients undergoing cardiac surgery. However, its efficacy has not been extensively studied in children. Using a prospective, randomized, double-blind study design, we examined 41 children undergoing repeat sternotomy for repair of congenital heart defects. After induction of anesthesia and prior to skin incision, patients received either tranexamic acid (100 mg/kg, followed by 10 mg.kg-1.h-1) or saline placebo. At the onset of cardiopulmonary bypass, a second bolus of tranexamic acid (100 mg/kg) or placebo was administered. Total blood loss and transfusion requirements during the period from protamine administration until 24 h after admission to the intensive care unit were recorded. Children who were treated with tranexamic acid had 24\% less total blood loss (26 +/- 7 vs 34 +/- 17 mL/kg) compared with children who received placebo (univariate analysis P = 0.03 and multivariate analysis P < 0.01). Additionally, the total transfusion requirements, total donor unit exposure, and financial cost of blood components were less in the tranexamic acid group. In conclusion, tranexamic acid can reduce perioperative blood loss in children undergoing repeat cardiac surgery. [\hyperlink{Crinecerfont}{PMID: 9141920}, R W Reid et al., 1997]

\hypertarget{pmid_1952008}{O}ne hundred and four children aged between 1 and 11 years were studied in a double-blind randomised controlled trial of glyceryl trinitrate ointment versus placebo, when used in addition to standard eutectic mixture of local anaesthetics cream. Each child received glyceryl trinitrate ointment on one hand and placebo on the other, and thus acted as his/her own control. A group of 30 children who received only the eutectic mixture on both hands (60 measurements) was also studied. The choice of site and ease of cannulation was scored. Skin colour and venous dilatation under the eutectic mixture were scored on a visual analogue scale. The addition of topical glyceryl trinitrate ointment to the standard eutectic mixture positively affected venous dilatation (p less than 0.01), choice of cannulation site (p less than 0.001), and ease of cannulation (p less than 0.001) of topical anaesthetic-treated skin. [\hyperlink{Crinecerfont}{PMID: 1952008}, W L Teillol-Foo et al., 1991]

\hypertarget{pmid_17133159}{Q}uinolone-induced arthropathic toxicity in weight-bearing joints observed in juvenile animals during preclinical testing has largely restricted the routine use of ciprofloxacin in the pediatric age group. As histopathologic, radiologic and magnetic resonance imaging monitoring evidence has gathered supporting the safety of fluoroquinolones in children, many pediatricians have started to prescribe quinolones to some patients on a compassionate basis. The objective of this study was to ascertain the safety of ciprofloxacin in preterm neonates <33 weeks gestational age treated at Dhaka Shishu (Children) Hospital in Bangladesh. Long-term follow up was done to monitor the growth and development of preterm infants who were administered intravenous ciprofloxacin in the neonatal period. Ciprofloxacin was used only as a life-saving therapy in cases of sepsis produced by bacterial agents resistant to other antibiotics. Another group of preterm neonates with septicemia who were not exposed to ciprofloxacin, but effectively treated with other antibiotics and followed up, were matched with cases for gender, gestational age and birth weight and included as a comparison group. Forty-eight patients in the ciprofloxacin group and 66 patients in the comparison group were followed up for a mean of 24.7 +/- 18.5 months and 21.6 +/- 18.8 months, respectively. No osteoarticular problems or joint deformities were observed in the ciprofloxacin group during treatment or follow up. No differences in growth and development between the groups were found. Ciprofloxacin is a safe therapeutic option for newborns with sepsis produced by multiply resistant organisms. [\hyperlink{Crinecerfont}{PMID: 17133159}, A S M Nawshad Uddin Ahmed et al., 2006]

\hypertarget{pmid_21172879}{T}o evaluate the efficacy of low-dose chemotherapy in infants with nonmetastatic and unresectable neuroblastoma (NB) without MYCN amplification. Infants with localized NB and no MYCN amplification were eligible in the SIOPEN Infant Neuroblastoma European Study 99.1 study. Primary tumor was deemed unresectable according to imaging defined risk factors. Diagnostic procedures and staging were carried out according to International Staging System recommendations. Children without threatening symptoms received low-dose cyclophosphamide (5 mg/kg/d × 5 days) and vincristine (0.05 mg/kg at day 1; CyV), repeated once to three times every 2 weeks until surgical excision could be safely performed. Children with either one threatening symptom or insufficient response to CyV were given carboplatin and etoposide (CaE), sometimes followed by vincristine, cyclophosphamide, and doxorubicin. No postoperative treatment was to be administered. Between December 1999 and April 2004, 120 infants were included in the study. Eighty-eight had no threatening symptoms and 79 received CyV. CaE was given to 49 of them because of insufficient response. Thirty-two children had threatening symptoms, 30 of whom received CaE. Anthracyclines were given to 46 children. Surgery was attempted in 102 patients, leading to gross surgical excision in 93. Relapse occurred in 12 patients (nine local and three metastatic). Five-year overall and event-free survivals were 99\% ± 1\% and 90\% ± 3\%, respectively, with a median follow-up of 6.1 years (range, 1.6 to 9.1). Low-dose chemotherapy without anthracyclines is effective in 62\% of infants with an unresectable NB and no MYCN amplification, allowing excellent survival rates without jeopardizing their long-term outcome. [\hyperlink{Crinecerfont}{PMID: 21172879}, Hervé Rubie et al., 2011]

\hypertarget{pmid_27128958}{T}his is an investigation of minocycline efficacy and safety as an adjuvant to risperidone in management of children with autism. Forty-six children with diagnosis of autistic disorder, according to the Diagnostic and Statistical Manual of Mental Disorders, 4th ed., Text Revision (DSM-IV-TR) criteria and a score of ≥12 on the Aberrant Behavior Checklist-Community (ABC-C) irritability subscale, who were already drug-free for at least 6 months participated in a randomized controlled trial and underwent 10 weeks of treatment with either minocycline (50 mg twice per day) or placebo in addition to risperidone titrated up to 2 mg/day (based on bodyweight). Patients were evaluated using ABC-C at baseline and at weeks 5 and 10. General linear model repeated measures showed significant effect for time × treatment interaction on the irritability [F(2, 88) = 3.94, p = 0.02] and hyperactivity/noncompliance [F(1.50, 66.05) = 7.92, p = 0.002], but not for lethargy/social withdrawal [F(1.61, 71.02) = 0.98, p = 0.36], stereotypic behavior [F(1.34, 58.80) = 1.55, p = 0.22], and inappropriate speech subscale scores [F(1.52, 66.88) = 1.15, p = 0.31]. By week 10, 21 (91.3\%) patients in the minocycline group and 15 (65.5\%) patients in the placebo group achieved at least partial response (p = 0.03). Frequencies of adverse events were not significantly different between groups. Minocycline seems to be a safe and effective adjuvant in management of patients with autistic disorder. Future studies with larger sample sizes, longer follow-ups, and inflammatory cytokine measurements are warranted to confirm these findings and provide insight into minocycline mechanism of action in autistic disorder. [\hyperlink{Crinecerfont}{PMID: 27128958}, Ali Ghaleiha et al., 2016]

\hypertarget{pmid_28275979}{S}edation is often required for children undergoing diagnostic procedures. Chloral hydrate has been one of the sedative drugs most used in children over the last 3 decades, with supporting evidence for its efficacy and safety. Recently, chloral hydrate was banned in Italy and France, in consideration of evidence of its carcinogenicity and genotoxicity. Dexmedetomidine is a sedative with unique properties that has been increasingly used for procedural sedation in children. Several studies demonstrated its efficacy and safety for sedation in non-painful diagnostic procedures. Dexmedetomidine's impact on respiratory drive and airway patency and tone is much less when compared to the majority of other sedative agents. Administration via the intranasal route allows satisfactory procedural success rates. Studies that specifically compared intranasal dexmedetomidine and chloral hydrate for children undergoing non-painful procedures showed that dexmedetomidine was as effective as and safer than chloral hydrate. For these reasons, we suggest that intranasal dexmedetomidine could be a suitable alternative to chloral hydrate. [\hyperlink{Crinecerfont}{PMID: 28275979}, Giorgio Cozzi et al., 2017]

\hypertarget{pmid_8545564}{W}e evaluated safety and tolerance of acyclovir ACV per os in immunocompetent children affected by chicken-pox admitted to our department from January 1993 to December 1994. 183 subjects (102 males and 81 females) aged between 0 and 14 years were treated by ACV (80 mg/kg/daily in 4 divided doses): 88 children were treated within 24 hours and 95 subjects within 48 hours from the onset of symptoms. The control group consisted of 83 children (52 males and 31 females) aged between 0 to 14 years. In all patients routine blood-test were performed and in those with respiratory illness Chest-Rx was also done. We evaluated clinical course, degree of eruption, the appearance and kind of complications, duration of hospitalization, the compliance and the potential consequences on specific antibody response. Our results show a faster improvement of clinical symptoms in treated patients with respect to the control group with shortening of the period of the fever, itch and appearance of new vescicles. The percentage of complications was lower in treated than in untreated patients. 16 cases tested for specific antibody response showed protective titers six months after treatment. In conclusion, ACV administered per os within 48 hours from onset of exanthema causes reduction of the period and the degree of general symptoms and exanthema, a lower incidence of complications even if non statistically significant. The drug is safe and well-tolerated. [\hyperlink{Crinecerfont}{PMID: 8545564}, S Catania et al., ]

\hypertarget{pmid_20832330}{T}here has been concern about the usage of aprotinin, an antifibrinolytic drug that was often used in pediatric cardiac surgery until 2006. At our center, these concerns led to the replacement of aprotinin with tranexamic acid for antifibrinolytic treatment. In this retrospective observational study, two groups of pediatric patients were studied during two different periods, receiving either aprotinin (n=70) or tranexamic acid (n=70) upon cardiac surgery. Data were collected from children with cyanotic heart defects, children who weighed less than 10 kg, and children who underwent re-operation. There was no difference in terms of blood loss or amount of erythrocyte concentrates and fresh frozen plasma transfused. Only the intraoperative amount of platelet concentrate received by children in the tranexamic acid group was 29 ml (p=0.013) higher. There was no significant difference in the length of stay at the intensive care unit, in renal function values, or in the rate of rethoracotomy. The results of this study suggest that tranexamic acid represents an adequate alternative to aprotinin in congenital cardiac surgery. [\hyperlink{Crinecerfont}{PMID: 20832330}, Ehrenfried Schindler et al., 2011]

\hypertarget{pmid_6092733}{T}he safety and efficacy of parenteral cholecalciferol was evaluated in the treatment and prevention of childhood rickets. Children with active disease, and those at high risk for developing rickets were treated either with intravenous or intramuscular cholecalciferol in dosages of 1000 to 1500 IU daily, for periods of 28 to 450 days. All children with rickets responded with radiographic evidence of healing. No child in the prophylaxis group developed bone disease. Side effects were minimal. Parenteral cholecalciferol is a safe and effective therapy for the treatment and prevention of childhood rickets. [\hyperlink{Crinecerfont}{PMID: 6092733}, J S Bertino et al., ]

\hypertarget{pmid_28406772}{T}o determine the success rate of non-surgical management of congenital nasolacrimal duct obstruction (CNLDO) with Crigler massage in infants below the age of one year. A cross-sectional observational study. Department of Ophthalmology at HBS General Hospital, Hazrat Bari Sarkar Medical and Dental College, Islamabad from November 2015 to May 2016. One hundred children with watering of eyes, due to congenital blockage of the distal part of the nasolacrimal duct (at the valve of Hasner), unilateral or bilateral were included in the study. Initially, Crigler massage was advised to all the parents for a period of 1-3 months with practical demonstration and the results were documented every fortnight. There were 52 infants up to the age of 6 months and 48 infants between 6-12 months, 53\% were boys and 47\% were girls. Among them, 67\% had unilateral complain while 33\% were bilaterally affected. At the end of 1-3 months, 90\% of the children achieved patency and only 10\% of the cases were subjected to Bowman's probing under short anesthesia. CNLDO mostly resolved through conservative approach by Crigler massage as an initial management, if done consistently. Probing and other surgical procedures should not be considered before the age of 12 months. [\hyperlink{Crinecerfont}{PMID: 28406772}, Jahanzeb Durrani et al., 2017]

\hypertarget{pmid_10990583}{T}inea capitis is one of the most common infections of children. The standard treatment is griseofulvin. Itraconazole and terbinafine have in large part replaced griseofulvin in the treatment of onychomycosis and, in addition to fluconazole and ketoconazole, are evolving treatments for tinea capitis. The purpose of this review is to compare the efficacy, safety, and cost of oral antifungal agents for tinea capitis. Small, open-label studies of itraconazole, terbinafine, and fluconazole have reported encouraging results, suggesting that these drugs may be effective alternatives to griseofulvin; however, in large controlled studies griseofulvin continues to exhibit greater or equal efficacy. Ketoconazole appears to be the least efficacious. All five drugs appear relatively safe, however, only griseofulvin has a long track record of safety, is Food and Drug Administration (FDA) approved for the treatment of tinea capitis in children, and has the least known drug interactions. Fluconazole is FDA approved for use in children more than 6 months of age, yet not for the treatment of tinea capitis. Oral griseofulvin and terbinafine tablets are the least expensive of the antifungal agents; griseofulvin suspension is, however, more expensive than fluconazole suspension. For the combined reasons of efficacy, safety, and cost, and a long track record of use, we feel oral griseofulvin is still the present treatment of choice for tinea capitis. Newer antifungals are currently under investigation, and their role in treating tinea capitis in children is still being defined. [\hyperlink{Crinecerfont}{PMID: 10990583}, M L Bennett et al., ]

\hypertarget{pmid_20027345}{P}rophylactic efficiency and safety of anaferon (pediatric formulation) in children aging 1?month to 4 years, including sickly children, was proven. The use of the preparation in children reduced the incidence of acute respiratory infections, alleviated the course of the disease, and decreased the incidence of detection of viral antigens in nasal meatuses. [\hyperlink{Crinecerfont}{PMID: 20027345}, E S Erman et al., 2009]

\section*{Penicillamine}
\subsection*{Result}
\subsubsection*{Answer}

Yes (neonates and infants, including premature and very low birth weight)
Yes (children, including those with Wilson's disease, lead poisoning, chronic liver disease, and juvenile rheumatoid arthritis)

\subsubsection*{{Explanation}}
\hypertarget{Penicillamine}
Based on the abstracts available, several studies have specifically evaluated the safety of penicillamine in children across different age ranges and indications:

1. **Infants (including premature and very low birth weight infants):**
   - A systematic review identified two randomized trials of D-penicillamine in infants less than 2000g birth weight, administered within the day following birth. No side effects were reported in these trials, and follow-up at one year revealed no significant differences in spasticity or developmental delay between treated and control groups. Additional case series (over 140 infants) reported only minor side effects (skin rashes in 2 infants, vomiting in 1) [\hyperlink{pmid_11279704}{PMID: 11279704}, D L Phelps et al., 2001].
   - Pharmacokinetic studies in premature babies did not report safety concerns, but these were not designed as safety studies [\hyperlink{pmid_3426856}{PMID: 3426856}, G Oroszlán et al., 1987].
   - Another study in newborns for hyperbilirubinemia found no evidence that D-penicillamine displaces bilirubin from albumin, but did not address safety directly [\hyperlink{pmid_6892748}{PMID: 6892748}, R Brodersen et al., 1980].

2. **Children (general pediatric population, including those with Wilson's disease, lead poisoning, and chronic liver disease):**
   - A retrospective cohort study of 84 children treated with D-penicillamine for low-level lead poisoning (mean age not specified) found that 33\% experienced adverse reactions (transient leukopenia, rash, platelet depression, enuresis, abdominal pain), and 10\% discontinued due to side effects. The authors concluded that D-penicillamine is effective but adverse effects can complicate or prevent its use in some patients [\hyperlink{pmid_3361395}{PMID: 3361395}, M Shannon et al., 1988].
   - A retrospective chart review of 65 children with Wilson's disease treated with low-dose D-penicillamine and high-dose zinc sulfate reported side effects in less than 11\% of patients over long-term follow-up, with no penicillamine-induced neurological deterioration [\hyperlink{pmid_23599735}{PMID: 23599735}, Hong Chang et al., 2013].
   - A case report described an 8-year-old girl with Wilson's disease who developed extrapyramidal symptoms after penicillamine, which resolved after stopping the drug [\hyperlink{pmid_12839386}{PMID: 12839386}, Alexander Chandran Paul et al.].
   - Another case report described a 14-year-old boy who developed pseudo-pseudoxanthoma elasticum after 4.5 years of D-penicillamine for Wilson's disease, a rare but serious side effect [\hyperlink{pmid_32796424}{PMID: 32796424}, Zizi Yu et al., 2020].
   - A study of 52 children (chronic liver disease and rheumatoid arthritis) treated with D-penicillamine for up to several years reported fewer side effects in children than adults, with better results in liver disease than in collagenoses. The authors recommend D-penicillamine for chronic active hepatitis in children [\hyperlink{pmid_460039}{PMID: 460039}, L Weingärtner et al., 1979].

3. **Children with Juvenile Rheumatoid Arthritis:**
   - Two large randomized, double-blind, placebo-controlled trials (162 children each) evaluated the safety and efficacy of D-penicillamine. Serious adverse drug reactions were rare, and no significant safety concerns were reported, though efficacy was not superior to placebo [\hyperlink{pmid_3517643}{PMID: 3517643}, E J Brewer et al., 1986; \hyperlink{pmid_3257872}{PMID: 3257872}, E H Giannini et al., 1988].

**Summary by Age Range:**
- **Neonates and infants (including premature and very low birth weight):** Targeted studies (randomized trials and case series) affirm that D-penicillamine is generally safe, with rare and mild side effects reported.
- **Children (general pediatric population, including those with Wilson's disease, lead poisoning, and chronic liver disease):** Multiple targeted studies and case series affirm that D-penicillamine is generally safe, but adverse effects (some serious) can occur and require monitoring. Rare but serious side effects have been reported in long-term use.
- **Children with juvenile rheumatoid arthritis:** Large randomized controlled trials affirm that D-penicillamine is generally safe, with serious adverse reactions being rare.

No abstracts provided targeted safety data for adolescents specifically, but the studies in children often included a broad pediatric age range up to at least 14 years.

**Conclusion:** There is evidence from targeted studies that penicillamine is generally safe for use in children and infants, though adverse effects can occur and monitoring is necessary.

\subsection*{Abstracts}
\hypertarget{pmid_3361395}{I}n a retrospective cohort study we reviewed our experience using D-penicillamine in children with low-level lead poisoning (whole blood lead levels 25 to 40 micrograms/dL) to determine its efficacy and the incidence of side effects. Two groups were compared: treated subjects (n = 84) were treated with penicillamine at a mean daily dose of 27.5 mg/kg; control subjects (n = 37) received no chelation therapy. Over a prechelation observation period of 60 days, lead levels (PbB) did not change in either group. With a mean period of 76 days of D-penicillamine therapy, PbB fell in treated patients by 33\% (P less than 0.001). In 64 patients (76\%), PbB was reduced to a currently acceptable range (less than or equal to 25 micrograms/dL). There were eight treatment failures (10\%). In control subjects, mean PbB did not change significantly over 119 days of observation. Fourteen control subjects eventually required conventional chelation with calcium disodium ethylene-diaminetetraacetic acid, and 17 were lost to follow-up. Use of D-penicillamine was associated with an adverse reaction in 28 cases (33\%); transient leukopenia occurred in eight, rash in seven, transient platelet count depression in seven, enuresis in three, and abdominal pain in two. Treatment was terminated prematurely in eight cases (10\%) because of an adverse reaction. We conclude that D-penicillamine is effective therapy for selected children with low-level plumbism, but adverse effects can complicate or prevent its use in some patients. [\hyperlink{Penicillamine}{PMID: 3361395}, M Shannon et al., 1988]

\hypertarget{pmid_16053699}{P}enicillamine is an oral agent used to treat intracerebral copper overload in Wilson's disease. Copper is a known regulator of angiogenesis; copper reduction inhibits experimental glioma growth and invasiveness. This study examined the feasibility, safety, and efficacy of creating a copper deficiency in human glioblastoma multiforme. Forty eligible patients with newly diagnosed glioblastoma multiforme began radiation therapy (6000 cGy in 30 fractions) in conjunction with a low-copper diet and escalating doses of penicillamine. Serum copper was measured at baseline and monthly. The primary end point of this study was overall survival compared to historical controls within the NABTT CNS Consortium database. The 25 males and 15 females who were enrolled had a median age of 54 years and a median Karnofsky performance status of 90. Surgical resection was performed in 83\% of these patients. Normal serum copper levels at baseline (median, 130 microg/dl; range, 50-227 microg/dl) fell to the target range of <50 microg/dl (median, 42 microg/dl; range, 12-118 microg/dl) after two months. Penicillamine-induced hypocupremia was well tolerated for months. Drug-related myelosuppression, elevated liver function tests, and skin rash rapidly reversed with copper repletion. Median survival was 11.3 months, and progression-free survival was 7.1 months. Achievement of hypocupremia did not significantly increase survival. Although serum copper was effectively reduced by diet and penicillamine, this antiangiogenesis strategy did not improve survival in patients with glioblastoma multiforme. [\hyperlink{Penicillamine}{PMID: 16053699}, Steven Brem et al., 2005]

\hypertarget{pmid_11279704}{R}etinopathy of prematurity remains a common problem. A low rate of this disorder was unexpectedly observed among infants treated with intravenous d-penicillamine to prevent hyperbilirubinemia. This observation led to the investigation of its use to prevent retinopathy of prematurity. To answer the question: Among very low birth weight infants, what is the effect of prophylactic administration of d-penicillamine on the incidence of acute ROP or severe ROP, and side effects including death? Searches were made of multiple electronic databases, previous reviews including cross references, abstracts, conference/symposia proceedings, and expert informants. The search was updated to November 2000. Randomized or quasi-randomized controlled trials that administered d-penicillamine to infants less than 2000g birth weight within the day following birth were considered relevant to this review. Additional case series were examined for potential side effects. Data on clinical outcomes were excerpted by 3 reviewers independently, and consensus reached. Data analysis was conducted according to the standards of the Neonatal Cochrane Review Group. Two randomized trials on the effects on ROP were identified. When combined, they showed a significantly lower incidence of acute ROP in the treated infants, relative risk of 0.09, 95\% CI [0.01,0.71]. Severe stages of ROP could not be analyzed. There was no effect on death rates, relative risk 0.99 95\% CI [0.70,1.39]. No side effects were reported, and follow up at one year revealed no significant differences in spasticity or developmental delay, although there were more rehospitalizations among the controls. In other reports of using d-penicillamine in over 140 infants for hyperbilirubinemia, skin rashes were reported in 2 infants and one had vomiting that may have been related. D-penicillamine is unlikely to affect survival, and may reduce the incidence of acute ROP among survivors. Studies to date justify further investigation of this drug in a broader population; careful attention to possible side effects is needed. [\hyperlink{Penicillamine}{PMID: 11279704}, D L Phelps et al., 2001]

\hypertarget{pmid_12839386}{P}enicillamine is the standard therapy for Wilson's disease in children. We report an 8-year-old-girl with liver disease due to Wilson's disease who developed extrapyramidal symptoms following administration of penicillamine. Symptoms resolved within 20 hours of stopping the drug but recurred within 24 hours when gradually increasing small doses were recommenced. [\hyperlink{Penicillamine}{PMID: 12839386}, Alexander Chandran Paul et al., ]

\hypertarget{pmid_20160046}{A} multicenter, open-label study evaluated the single-dose pharmacokinetics and safety of a pediatric oral famciclovir (prodrug of penciclovir) formulation in infants aged 1 to 12 months with suspicion or evidence of herpes simplex virus infection. Individualized single doses of famciclovir based on the infant's body weight ranged from 25 to 175 mg. Eighteen infants were enrolled (1 to <3 months old [n = 8], 3 to <6 months old [n = 5], and 6 to 12 months old [n = 5]). Seventeen infants were included in the pharmacokinetic analysis; one infant experienced immediate emesis and was excluded. Mean C(max) and AUC(0-6) values of penciclovir in infants <6 months of age were approximately 3- to 4-fold lower than those in the 6- to 12-month age group. Specifically, mean AUC(0-6) was 2.2 microg h/ml in infants aged 1 to <3 months, 3.2 microg h/ml in infants aged 3 to <6 months, and 8.8 microg h/ml in infants aged 6 to 12 months. These data suggested that the dose administered to infants <6 months was less than optimal. Eight (44.4\%) infants experienced at least one adverse event with gastrointestinal events reported most commonly. An updated pharmacokinetic analysis was conducted, which incorporated the data in infants from the present study and previously published data on children 1 to 12 years of age. An eight-step dosing regimen was derived that targeted exposure in infants and children 6 months to 12 years of age to match the penciclovir AUC seen in adults after a 500-mg dose of famciclovir. [\hyperlink{Penicillamine}{PMID: 20160046}, Jeffrey Blumer et al., 2010]

\hypertarget{pmid_23599735}{T}he aim of this study was to investigate the effectiveness of a high-dose zinc sulfate and low-dose D-penicillamine combination in the treatment of pediatric Wilson's disease (WD). A retropective chart review of 65 patients with WD was conducted. These patients received D-penicillamine (8-10 mg/kg/day) and zinc sulfate as the primary treatment. The pediatric dose of elemental zinc is 68-85 mg/day until 6 years of age, 85-136 mg/day until 8 years of age, 136-170 mg/day until 10 years of age and then 170 mg/day, in 3 divided doses 1 h before meals. After clinical and biochemical improvement or stabilization, zinc sulfate alone was administered as the maintenance therapy. Under treatment, the majority of patients (89.2\%) had a favourable outcome and 3 patients succumbed due to poor therapy compliance. No penicillamine-induced neurological deterioration was noted and side-effects were observed in <11\% of patients over the entire follow-up period. Benefical results on the liver and neurological symptoms were reported following extremely long-term treatment with a combination of low-dose D-penicillamine and high-dose zinc sulfate. Therefore, this regimen is an effective and safe treatment for children with WD. [\hyperlink{Penicillamine}{PMID: 23599735}, Hong Chang et al., 2013]

\hypertarget{pmid_32796424}{W}e describe a 14-year-old boy with Wilson disease (WD) who first developed pseudo-pseudoxanthoma elasticum (PPXE) after 4.5 years of treatment with D-penicillamine. Although previously reported cases have occurred in adults following at least a decade of high-dose D-penicillamine use, this case demonstrates that D-penicillamine-induced PPXE can present in children with shorter treatment courses. Upon this diagnosis, the patient was switched from D-penicillamine to trientine, with adequate cupriuresis and stabilization of the skin lesion. Prompt diagnosis and management of PPXE in children can limit systemic progression and prevent long-term complications. [\hyperlink{Penicillamine}{PMID: 32796424}, Zizi Yu et al., 2020]

\hypertarget{pmid_460039}{D}-pencillamine, a stable not physiological amino acid, has a manifold mode of action. Of special importance there is its influence on the collagen-metabolism, the gelose of heavy metals and the effect on immunologic processes. The use of D-penicillamine is possible in different diseases, such as Wilson's syndrome, collagenoses of diverse kinds, especially the rheumatoid arthritis, further chronic hepatitis and lung fibrosis. In this paper we report about 52 children, who were treated with D-penicillamine. The biggest group presented chronic liver diseases in 24 patients and rheumaoid arthritis in 21 patients. The therapy was carried out for a longer time, in some cases over years. The dose varied from 15 to 35 mg/kg of body weight. The number of side-effects was lower in children than in adults. They were more frequent in the group of collagenoses than in the group of liver diseases. Whether later on liver damages will occur is not predictable by the pediatrician. The results were excellent for the chronic active hepatitis; we can recommend D-penicillamine for such affections. Also for the rheumatoid arthritis we could partially obtain good successes, but not as convincing as in liver-diseases. [\hyperlink{Penicillamine}{PMID: 460039}, L Weingärtner et al., 1979]

\hypertarget{pmid_25017533}{P}enicillin skin testing has been validated in the evaluation of adult patients with penicillin allergy. However, the commercially available benzylpenicilloyl polylysine (Pre-Pen) is not indicated in the pediatric population. Moreover, the safety and validity of penicillin skin testing in the pediatric population has not been well studied. We describe the safety and validity of penicillin skin testing in the evaluation of children with a history of penicillin allergy. Children (<18 years) with a history of penicillin allergy were evaluated with penicillin skin tests and were reviewed for basic demographics, penicillin skin test results, adverse drug reaction to penicillin after penicillin skin test, and adverse reaction to penicillin skin test. By using the χ(2) test, we compared the differences in the proportion of children and adults with a positive penicillin skin test. P value (<.05) was considered statistically significant. The institutional review board approved the study, and all the subjects signed written informed consents. A total of 778 children underwent penicillin skin testing; 703 of 778 patients had a negative penicillin skin test (90.4\%), 66 had a positive test (8.5\%), and 9 had an equivocal test (1.1\%). Children were more likely to have a positive penicillin skin test (P < .0001) compared with adults (64 of 1759 [3.6\%]); 369 of 703 patients with negative penicillin skin test (52\%) were challenged with penicillin, and 14 of 369 patients (3.8\%) had an adverse drug reaction. No adverse reactions to penicillin skin testing were observed. Penicillin skin testing was safe and effective in the evaluation of children with a history of penicillin allergy. [\hyperlink{Penicillamine}{PMID: 25017533}, Stephanie J Fox et al., ]

\hypertarget{pmid_28292340}{T}he purpose of this study was to evaluate, using a randomized, double-blind methodology: (1) the safety of phentolamine mesylate (Oraverse) in accelerating the recovery of soft tissue anesthesia following the injection of two percent lidocaine plus 1:100,000 epinephrine in two- to five-year-olds; and (2) efficacy in four- to five-year-olds only. One hundred fifty pediatric dental patients underwent routine dental restorative procedures with two percent lidocaine plus 1:100,000 epinephrine with doses based on body weight. Phentolamine mesylate or a sham injection (two to one ratio) was then administered. Subjects were monitored for safety and, in four- to five-year-olds, for efficacy during the two-hour evaluation period. There were no significant differences in adverse events between the phentolamine and sham injections. Compared to sham, phentolamine was not associated with nerve injury, increased analgesic use, or abnormalities of the oral cavity. Phentolamine was associated with transient decreased blood pressure in some children. In four- and five-year-olds, phentolamine induced more rapid recovery of lip anesthesia by 48 minutes (P<0.0001). Phentolamine was well tolerated and safe in three- to five-year-olds; in four- to five-year-olds, a statistically significant more rapid recovery of lip sensation compared to sham injections was determined. [\hyperlink{Penicillamine}{PMID: 28292340}, Elliot V Hersh et al., 2017]

\hypertarget{pmid_7151643}{D}-Penicillamine, previously suspected to have a beneficial effect on the occurrence of severe retrolental fibroplasia among very low birth weight infants, was tested to determine the extent to which this drug modifies acute radiosensitivity on 3- to 4-day-old mice in comparison with adult animals. It was found that the radioprotective effect of penicillamine, given in doses of 3,000 mg/kg i.p. 60 min before whole-body exposure to 6-10 Gy of 60Co gamma rays, was greater in 3- to 4-day-old mice than in adult animals. These data seem to be compatible with the view that D-penicillamine, by virtue of its antioxidant action, may reduce the toxic effects associated with exposure of the newborn infant to hyperoxia, specifically retrolental fibroplasia and bronchopulmonary dysplasia. [\hyperlink{Penicillamine}{PMID: 7151643}, L Lakatos et al., 1982]

\hypertarget{pmid_8680887}{W}e assessed the long-term feasibility, safety, and tolerability of two regimens of aerosolized pentamidine (AP) as primary prophylaxis of Pneumocystis carinii pneumonia (PCP) in a large sample of infants and children with symptomatic HIV infection in 21 pediatric departments. One hundred forty children were assigned to receive 60 mg every 2 weeks (n = 60) or 120 mg every 4 weeks (n = 80) of AP, delivered by the ultrasonic nebulizer Fisoneb under the supervision of trained personnel. Children underwent monthly clinical and laboratory controls for toxicity and/or development of PCP for an 18-month period. Baseline characteristics were similar in the two treatment groups. The median age was 5 years. The feasibility of administering AP was excellent in 84 (60 percent) and good in 38 (27 percent) children. All children aged <2 years showed excellent or good feasibility. Long-term compliance was good with both regimens. No child had severe adverse reactions requiring discontinuation of the treatment. Cough, sneezing, and bronchospasm were the most frequent side effects occurring, respectively, in 12, 3.7, and 0.7 percent of the 60-mg treatments and in 19.1, 6. 1, and 2.8 percent of 120-mg treatments (p < 0.05). Their incidence was not different in children younger or older than 5 years. Two episodes of PCP were observed in the group receiving 120 mg monthly, whereas none of the 60 children in the biweekly schedule had PCP (p = 0.20). AP can be safely administered to very young children with few adverse side effects. [\hyperlink{Penicillamine}{PMID: 8680887}, N Principi et al., 1996]

\hypertarget{pmid_22169728}{T}o evaluate the short-term efficacy and safety of propranolol for problematic infantile hemangiomas. Oral propranolol was administered to 68 infants with heamngiomas diagnosed by clinical evaluation and adjuvant examination at 1.0\textasciitilde{}2.0 mg per kilogram of body weight per day, divided to 2 or 3 times. The patients revisited once a month. The changes of the tumor size, texture, and color were monitored and recorded at a regular interval.The adverse effects after medication were observed and managed accordingly.The short-term results were evaluated using a 4-grade system. All the 68 infants were followed up for 3-13 months, except that 1 infants combined with other diseases and 4 withdrew.The overall response was Scale 1 in 8 infants, Scale II in 13, Scale III in 29, and Scale IV in 13. No serious adverse effects were seen, but none cured entirely as well. Oral propranolol is safe and effective for infantile heamngioma with good short-term result. It could be used as the primary drug for problematic infantile hemangiomas at the rapid growth stage of hemangiomas. [\hyperlink{Penicillamine}{PMID: 22169728}, Jianyun Lu et al., 2011]

\hypertarget{pmid_15209963}{T}o establish the prevalence of positive penicillin skin tests among outpatients without any drug reaction history. Skin testing was performed in 147 children (aged 6-13 years) who had had received a penicillin preparation at least three times in the last 12 months without any allergic reaction. Before testing, detailed pediatric and allergy history were learned and then all children were tested with benzyl penicilloyl polylysin (PPL) and mixture of minor antigenic determinants. The test procedures were made epidermally and intradermally subsequently in every subject. The overall frequency of positive skin reactions to penicillin antigens was 10.2\%. A mild systemic reaction was observed in one of the children during testing with PPL. We concluded that frequent use of penicillin and other beta-lactam antibiotics leads to sensitization of children in our study population despite these children seem to be asymptomatic during testing time. [\hyperlink{Penicillamine}{PMID: 15209963}, Feyzullah Cetinkaya et al., 2004]

\hypertarget{pmid_6460054}{T}he efficacy and the toxicity pattern of D-penicillamine were studied in patients with rheumatoid disease followed up between April 1975 and March 1979. The population of patients was divided into an elderly group (greater than or equal to 60 years old, mean = 65 years) and a younger group (less than 60 years old, mean = 41 years). Patients with classic or definite rheumatoid disease not responsive to nonsteroidal drugs were eligible. The mean durations of disease prior to D-penicillamine therapy were five years in the elderly and seven years in the younger group. Overall, the mean follow-up time was 11 months. The average dosages of D-penicillamine were 461 mg/day in the elderly and 520 mg/day in the younger patients. Results indicated that D-penicillamine was efficacious in 75 per cent of the elderly during all time periods after three months, and in 75 per cent of the younger patients after three months until at least two years. Prior gold-salt therapy did not influence efficacy. Toxicity was significantly greater in the elderly for overall skin rash (P less than 0.01), severe skin rash (P less than 0.01), and marked abnormalities in the ability to taste (P less than 0.05). The incidence of hematologic toxicity was not increased in the elderly compared with the younger patients. Toxicity in either group was not influenced by prior gold-salt therapy. It is concluded that D-penicillamine was equally efficacious in both elderly and younger groups, and that the toxicity patterns were similar except for increased tendencies toward rashes and taste abnormalities in the elderly. [\hyperlink{Penicillamine}{PMID: 6460054}, W F Kean et al., 1982]

\hypertarget{pmid_3426856}{T}he pharmacokinetics of D-penicillamine in premature babies was studied. The metabolism of the drug was characterized by a long half-life (115 min) with a low plasma clearance (2.84 ml/min/kg body weight) in contrast to adults. The mean volume of distribution (649/ml/kg body weight) was similar to that of adults. [\hyperlink{Penicillamine}{PMID: 3426856}, G Oroszlán et al., 1987]

\hypertarget{pmid_6559060}{S}eventy infants with suspected bacterial infection in the first 48 hours of life were treated either with piperacillin and flucloxacillin or with penicillin and gentamicin. Infection was confirmed and successfully eradicated in 6 of the 35 infants receiving piperacillin and flucloxacillin. Four infants treated with penicillin and gentamicin had confirmed infection and one deteriorated initially but then recovered when treated with piperacillin. Serum piperacillin concentrations above 100 mg/l and cerebrospinal fluid piperacillin concentrations of 2.6-6 mg/l were noted for up to four hours and 7 hours respectively, even in the absence of inflamed meninges, after administration of piperacillin 100 mg/kg body weight intravenously. Median half life of piperacillin was 6.5 hours and was prolonged in renal impairment. Piperacillin is considered to be a safe and effective first line single agent treatment for early neonatal infection but because some Escherichia coli are resistant to it we recommend that a second agent be used in critically ill infants with neutropenia or meningitis. [\hyperlink{Penicillamine}{PMID: 6559060}, M Placzek et al., 1983]

\hypertarget{pmid_3257872}{A} 12-month double-blind, parallel, randomized, placebo-controlled multicenter trial of D-penicillamine and hydroxychloroquine was conducted in 162 children with juvenile rheumatoid arthritis in the United States and in the Union of Soviet Socialist Republics. No statistically significant intergroup differences were detected in primary outcome variables. We investigated the possible existence of select subgroups of patients who have a higher likelihood of response to active drugs than to placebo. Using previously published criteria, each patient was classified as a responder or nonresponder, and their demographic and disease characteristics at baseline were compared. We were unable to identify a subgroup of individuals who were more likely to respond to D-penicillamine or hydroxychloroquine than to placebo. [\hyperlink{Penicillamine}{PMID: 3257872}, E H Giannini et al., 1988]

\hypertarget{pmid_3517643}{O}ne hundred sixty-two children with severe juvenile rheumatoid arthritis were entered in a randomized, double-blind, placebo-controlled 12-month clinical trial designed to establish the efficacy and safety of two slower-acting antirheumatic drugs, penicillamine and hydroxychloroquine. The study was a cooperative effort of the United States and the Soviet Union. One group of subjects received 10 mg of penicillamine per kilogram of body weight per day, another group received 6 mg of hydroxychloroquine per kilogram daily, and a third group received placebo. All three groups were allowed a single concurrent nonsteroidal antiinflammatory drug, but no other antirheumatic medications, including corticosteroids. All three groups had dramatic improvement in many of the clinical and laboratory outcome variables after one year of study. There were no significant differences in efficacy between the penicillamine and placebo groups. Pain on movement was the only index of articular disease that was alleviated more by hydroxychloroquine than by placebo. Serious adverse drug reactions attributable to the active agents were rare. We were unable to demonstrate that, in the presence of a nonsteroidal antiinflammatory drug, either penicillamine or hydroxychloroquine is superior to placebo in the treatment of children with juvenile rheumatoid arthritis. [\hyperlink{Penicillamine}{PMID: 3517643}, E J Brewer et al., 1986]

\hypertarget{pmid_37862}{P}ethidine 1 mg kg-1, diazepam 0.25 mg kg-1 and flunitrazepam 0.02 mg kg-1 i.m. wer compared as premedicants in a double-blind study in 145 children undergoing otolaryngological surgery. Both flunitrazepam and pethidine had an anxiolytic effect in the children of less than 5 yr whereas diazepam had little effect. All of the drugs were anxiolytic in the children aged 5 yr and older. Sleep following thiopentone was restless more often in the younger than in the older children. Cardiovascular responses to thiopentone and to tracheal intubation were most obvious following benzodiazepines in children of less than 5 yr. After anaesthesia 10--33\% of the older children could not recall pictures shown to them before anaesthesia. Forty-five (+/-SD 13) min after injection, the concentration of diazepam in serum was similar in both age groups; after 90 min it decreased in the younger and increased in the older children. All concentrations of flunitrazepam were significantly greater in the older compared with the younger children. [\hyperlink{Penicillamine}{PMID: 37862}, L Lindgren et al., 1979] 23 infants and children, aged 1 month - 15 years, were treated with piperacillin, a new semi-synthetic penicillin with a broad spectrum of activity. The indications were perforated appendicitis with peritonitis or abscess formation (12 patients), urinary tract infection due to congenital anomalies (3 patients), miscellaneous infections (3 patients) and peroperative prophylactic treatment (5 patients). The clinical response was good. Few adverse reactions were noted. The drug seems to be effective and safe. [\hyperlink{Penicillamine}{PMID: 37862}, J Gierup et al., 1982]

\hypertarget{pmid_273188}{P}enicillamine (beta1 beta2 dimethylcysteine) is the drug of choice in the therapeutic management of Wilson's disease and cystinuria and has been used in the treatment of some heavy-metal intoxications. Recent studies have shown that it is efficacious in patients with rheumatoid arthritis. Side effects include sensitivity reactions, nephrotoxicity, bone-marrow suppression, hypogeusia, skin lesions, and the formation of autoantibodies. Two cases are described with the features of pemphigus which were attributed to penicillamine therapy. [\hyperlink{Penicillamine}{PMID: 273188}, K D Hay et al., 1978]

\hypertarget{pmid_26991468}{P}rimary perniosis is an annoying cold-induced dermatosis. Many therapeutic agents have been tried with either unsatisfactory or controversial results. The aim of this study was to assess the efficacy of oral pentoxyfylline in the treatment of primary perniosis. A double-blind placebo-controlled randomized therapeutic study conducted in dermatology department of Al-Yarmouk Teaching Hospital, Baghdad, Iraq during four winter seasons between 2010 and 2014. The patients were randomly allocated into two equal groups: group A patients were given oral pentoxyfylline 400 mg thrice daily whereas patients in group B were given an identical placebo tablet thrice daily for 3 weeks. Therapeutic response of both groups was clinically assessed weekly for 3 weeks and side-effects were recorded. A total of 110 patients with chilblains completed this therapeutic trial. The mean age was 24.98 ± 9.17 year. Male to female ratio was 1:2.4. All patients presented with erythematous papules, plaques or nodules. Very good therapeutic response was significantly better for group A than that of group B at 7th, 4th, and 21st days of the trial (p-value: 0.0148, 0.0000004, and 0.0000000, respectively). No side effects were encountered in both groups. Pentoxyfylline is an effective and safe drug for treatment of primary perniosis.  [\hyperlink{Penicillamine}{PMID: 26991468}, Nameer K Al-Sudany et al., 2016] D-penicillamine, a drug used clinically for the treatment of neonatal hyperbilirubinaemia, was tested for interference with the binding of bilirubin to human serum albumin by three methods: 1) The peroxidase technique, investigating the effect of D-penicillamine on the equilibrium concentration of unbound bilirubin in a solution containing a molar excess of albumin; 2) the MADDS method, measuring the concentration of vacant bilirubin binding site on albumin in a solution of pure albumin, or infant blood serum, with added D-penicillamine; and 3) injection of D-penicillamine into Gunn rats and determination of any decrease of plasma bilirubin which would be caused by displacement of the pigment. Results were negative in all cases. Quantitatively, the doses of D-penicillamine used clinically cannot displace bilirubin from its binding to albumin. The ameliorating effect on hyperbilirubinaemia in the newborn must be due to some other mechanism. [\hyperlink{Penicillamine}{PMID: 26991468}, R Brodersen et al., 1980]

\hypertarget{pmid_24956685}{C}hildren with sickle cell anaemia (SCA) are highly susceptible to infection caused by pneumococcal bacteria due to functional asplenia amongst other reasons. Pneumococcal infections are severe with high mortality among these children that the need for prophylactic penicillin therapy becomes necessary. The objective of this review is to look for evidence of the effectiveness of daily oral penicillin prophylaxis in the prevention of pneumococcal infection in children with SCA. Electronic databases including genetic disorders group haemoglobinopathies trial register, Cochrane library, Pubmed, Turning Research Into Practice (TRIP) and Google were searched for relevant studies. Hand and grey literature searches were also done. Randomized controlled trials comparing oral penicillin prophylaxis for the prevention of pneumococcal infection in children with SCA with placebo or no treatment were searched for. Two trials were found to have met the inclusion criteria for the review. Results of the two included randomized controlled trials showed a significant reduction in the incidence of pneumococcal infection among children with SCA (and (0)-thallasaemia) receiving penicillin compared to the control group treated with placebo. The odds ratios for the two (Gaston et al and Falletta et al) studies were 0.37 (95\% CI 0.16 to 0.86) and 0.5 (95\% CI 0.1 to 2.71) respectively. There is strong evidence that daily oral penicillin prophylaxis greatly reduces the risk of pneumococcal infection in children with SCA under the age of three years and a moderately strong evidence that its, withdrawal at the age of five years did not result in any serious consequences. [\hyperlink{Penicillamine}{PMID: 24956685}, Hafsat A Gwaram et al., ]

\section*{Cyclophosphamide}
\subsection*{Result}
\subsubsection*{Answer}

Yes (Infants and young children, with dose adjustment: 0–4.9 years)
Yes (Children: 5–12 years)
Yes (Adolescents: 12–21 years)

\subsubsection*{{Explanation}}
\hypertarget{Cyclophosphamide}
Based on the abstracts provided, several studies have specifically evaluated the safety of cyclophosphamide in children for various indications. Below is a summary of the relevant evidence, organized by age range where possible:

1. **Infants and Young Children (0–4.9 years):**
   - A pharmacokinetic study in 171 patients aged 0.07–4.9 years with brain tumors found that cyclophosphamide exposure and toxicity were higher in young infants (<6 months) compared to older children, leading to a recommendation for dose reduction in this age group to reduce toxicity. The study established an exposure-toxicity association and suggested that Bayesian modeling could help optimize dosing and minimize toxicity. However, while the study provides dosing guidance to reduce toxicity, it does not explicitly affirm overall safety, only that toxicity can be managed with dose adjustment [\hyperlink{pmid_31796512}{PMID: 31796512}, Olivia Campagne et al., 2020].

2. **Children (General Pediatric Population):**
   - A meta-analysis of 38 studies involving 1,504 children with relapsing steroid-sensitive nephrotic syndrome found that cyclophosphamide (2–3 mg/kg for 8–12 weeks) is effective and generally safe, with a fatality rate of approximately 1\%. Leukopenia occurred in one-third of patients, and severe bacterial infections in 1.5\%. The study recommends this regimen as standard, indicating that safety has been established for this use in children [\hyperlink{pmid_11322378}{PMID: 11322378}, K Latta et al., 2001].
   - A study of 43 children with minimal lesion nephrotic syndrome treated with cyclophosphamide (3 mg/kg/day for 8 weeks) reported that most patients benefited, with some experiencing reversible side effects such as hemorrhagic cystitis and leukopenia. The study concludes cyclophosphamide has a beneficial effect and is safe for this indication [\hyperlink{pmid_1589052}{PMID: 1589052}, R de Moor et al., 1992].
   - Another study in children with minimal-change nephropathy found that cyclophosphamide caused transient lymphopenia and changes in T-cell ratios, but immune function returned to normal within 6–12 months after treatment. No long-term immune suppression was observed [\hyperlink{pmid_6229699}{PMID: 6229699}, J Feehally et al., 1984].
   - In children with severe and refractory juvenile dermatomyositis, two studies (one with 12 patients, another with a larger cohort) found that cyclophosphamide provided significant clinical benefit with no serious short-term toxicity. Reversible complications included lymphopenia, herpes zoster, and alopecia. The studies note that risks of malignancy, infertility, and gonadal failure are low at the doses used, but recommend further study for long-term effects [\hyperlink{pmid_14722349}{PMID: 14722349}, P Riley et al., 2004; \hyperlink{pmid_29342499}{PMID: 29342499}, Claire T Deakin et al., 2018].
   - In a study of six children with anti-NMDAR encephalitis, cyclophosphamide as second-line therapy led to clinical improvement with no adverse reactions or abnormal lab results during the treatment period [\hyperlink{pmid_28606234}{PMID: 28606234}, Wei-Wen Zhu et al., 2017].
   - Studies in children with brain tumors (including high-dose regimens) reported that the most common toxicities were hematologic (neutropenia, need for transfusions), but these were manageable and no deaths were attributed to infection or bleeding. The regimens were considered to have acceptable toxicity [\hyperlink{pmid_10894368}{PMID: 10894368}, K S Bottom et al., 2000; \hyperlink{pmid_7749762}{PMID: 7749762}, T G Abrahamsen et al., 1995].
   - A meta-analysis and several individual studies note the risk of gonadal toxicity, particularly in males at higher cumulative doses, but also state that the standard recommended regimen is considered safe [\hyperlink{pmid_11322378}{PMID: 11322378}, K Latta et al., 2001; \hyperlink{pmid_14722349}{PMID: 14722349}, P Riley et al., 2004].

3. **Adolescents (12–21 years):**
   - A quality improvement study in adolescent girls (12–21 years) receiving intravenous cyclophosphamide focused on pregnancy screening, not on drug safety per se. No safety data specific to this age group was provided [\hyperlink{pmid_27940668}{PMID: 27940668}, Kristen Hayward et al., 2016].

4. **General Notes:**
   - Several studies in pediatric populations highlight the importance of monitoring for known toxicities (hematologic, gonadal, immune suppression), but overall, the evidence from targeted pediatric studies supports the safety of cyclophosphamide when used at recommended doses and with appropriate monitoring.
   - Some rare but serious adverse events (e.g., fatal syndrome of inappropriate antidiuretic hormone secretion [\hyperlink{pmid_476599}{PMID: 476599}, P J Harlow et al., 1979]) have been reported, but these are exceptions rather than the rule.

**Summary:** There is strong evidence from targeted pediatric studies (including meta-analyses and prospective studies) that cyclophosphamide is safe for use in children for specific indications (e.g., nephrotic syndrome, juvenile dermatomyositis, certain brain tumors, anti-NMDAR encephalitis) when used at recommended doses and with appropriate monitoring. Safety in infants (<6 months) requires dose adjustment to reduce toxicity. Long-term risks (e.g., infertility, malignancy) are low at standard doses but require further study.

\subsection*{Abstracts}
\hypertarget{pmid_15907638}{C}yclophosphamide is an alkylating agent widely used from cancer chemotherapy to immunotherapy purposes. In paediatrics oncology, oral cyclophosphamide prescribed at low dosages for a long time treatment is currently investigated. This treatment is a putative well tolerated regimen for children treated for a wide variety of recurrent solid tumours. For these purposes, new oral formulations more convenient for children than cyclophosphamide 50mg tablets are needed. Thus, we present a rapid method for the assay of cyclophosphamide in various pharmaceutical preparations using high-performance thin-layer chromatography (HPTLC) and derivatization with phosphomolybdic acid. This method is accurate and precise and allows quantitation of cyclophosphamide in aqueous solutions from 400 to 1200 microg/mL. It is suitable for quantitation and stability studies of cyclophosphamide in pharmaceutical products, i.e. capsules and infusion bags prepared in a hospital pharmacy. According to pharmaceutical guidelines, we demonstrated that low dose cyclophosphamide capsules, extemporaneously prepared for children, are stable at least for 70 days. [\hyperlink{Cyclophosphamide}{PMID: 15907638}, Jérôme Bouligand et al., 2005]

\hypertarget{pmid_1589052}{T}he effect of cyclophosphamide therapy was evaluated in the treatment of children with nephrotic syndrome due to minimal lesions. Most of the children, 37 out of 43, presented with frequent relapsing nephrotic syndrome. Cyclophosphamide was given in a dose of 3 mg/kg body weight/day for a period of 8 weeks. Two patients received two courses, one patient received three courses. Only one patient, who was steroid-resistant, did not respond to cyclophosphamide therapy (therapy was, however, stopped after 3 weeks because of haemorrhagic cystitis). 57\% of the patients were still in remission after 18 months (n = 37) and 50\% after 30 months (n = 34). A haemorrhagic cystitis developed in 3 patients and leucopenia in 2 patients. From this study, which confirms data reported in literature, it can be concluded that cyclophosphamide has a beneficial effect in children with minimal lesion nephrotic syndrome and steroid toxicity. [\hyperlink{Cyclophosphamide}{PMID: 1589052}, R de Moor et al., 1992]

\hypertarget{pmid_27940668}{C}yclophosphamide is a teratogenic medication used in the treatment of adolescents with autoimmune disorders. This adolescent population is sexually active, does not receive adequate contraceptive care, and is at risk for unintended pregnancy. We undertook a quality improvement initiative to improve rates of pregnancy screening before intravenous cyclophosphamide administration in our adolescent girl patients. Data were collected from the electronic medical record. The primary outcome was completion of a urine pregnancy test before intravenous cyclophosphamide infusion in girls aged 12 to 21 years between July 2011 and June 2015. Data were reviewed quarterly and an iterative quality improvement approach was used. Interventions included staff education, electronic order set updates, and a Maintenance of Certification project. Interrupted time series analysis and multivariable mixed effects logistic regression were used to evaluate trends over time and to adjust for potential confounders. Thirty girls received 153 cyclophosphamide infusions during the study. Pregnancy testing before medication administration increased from 25\% to 100\% by study completion. Infusions in the last time period were significantly more likely to be accompanied by a pregnancy test versus those in the first time period (odds ratio: 17.7; 95\% confidence interval [CI]: 3.1-101.6) after adjustment for patient age, managing service, infusion setting, and insurance type. Our institution achieved a significant increase in standard pregnancy screening in adolescent girls receiving intravenous cyclophosphamide. The interventions most valuable in increasing screening rates were updating electronic order sets, educating staff, and physician engagement in the Maintenance of Certification program. [\hyperlink{Cyclophosphamide}{PMID: 27940668}, Kristen Hayward et al., 2016]

\hypertarget{pmid_18927240}{C}yclophosphamide-based regimens are front-line treatment for numerous pediatric malignancies; however, current dosing methods result in considerable interpatient variability in tumor response and toxicity. In this pediatric population, the authors' objectives were (1) to quantify and explain the pharmacokinetic variability of cyclophosphamide and 2 of its metabolites, hydroxycyclophosphamide (HCY) and carboxyethylphosphoramide mustard (CEPM), and (2) to apply a population pharmacokinetic model to describe the disposition of cyclophosphamide and these metabolites. A total of 196 blood samples were obtained from 22 children with neuroblastoma receiving intravenous cyclophosphamide (400 mg/m2/d) and topotecan. Blood samples were quantitated for concentrations of cyclophosphamide, HCY, and CEPM using liquid chromatography-mass spectrometry and analyzed using nonlinear mixed-effects modeling with the NONMEM software system. After model building was complete, the area under the concentration-time curve (AUC) was computed using NONMEM. Cyclophosphamide elimination was described by noninducible and inducible routes, with the latter producing HCY. Glomerular filtration rate was a covariate for the fractional elimination of HCY and its conversion to CEPM. Considerable interpatient variability was observed in the AUC of cyclophosphamide, HCY, and CEPM. These results represent a critical first step in developing pharmacokinetic-linked pharmacodynamic studies in children receiving cyclophosphamide to determine the clinical relevance of the pharmacokinetic variability in cyclophosphamide and its metabolites. [\hyperlink{Cyclophosphamide}{PMID: 18927240}, Jeannine S McCune et al., 2009]

\hypertarget{pmid_34550448}{C}yclophosphamide is still clinically used in rheumatic diseases with severe disease courses. Cyclophosphamide has a pronounced gonadotoxic effect largely depending on the cumulative dose. The risk of amenorrhea is reported to be in the range of 12-54\% and is dependent on the age of the patient at initiation of treatment. Every patient of reproductive age should therefore be offered counseling on options for fertility protection. There are 3 options for fertility protection: oocyte harvesting and cryopreservation after a hormonal stimulation of 10-14 days, ovarian wedge resection and cryopreservation and administration of a gonadotropin-releasing hormone (GnRH) agonist. The decision whether and, if so, which treatment should be performed is made in close consultation between the patient, rheumatologists and reproductive physicians and depends on the available treatment time window, the age of the patient and the severity of the underlying disease. [\hyperlink{Cyclophosphamide}{PMID: 34550448}, Philippos Edimiris et al., 2021]

\hypertarget{pmid_6684023}{C}yclosphosphamide, dissolved in saline, was injected into the air sac of white Leghorn chick eggs in dose levels of 0.005, 0.007, 0.010, 0.012, 0.015, and 0.017 mg per egg. Eggs received a single injection of cyclophosphamide on Days 0, 1, 2, or 3 of incubation. Control eggs were injected with an equivalent volume of saline (0.1 ml per egg). In all 904 chicken eggs were used for this study. Surviving embryos were sacrificed when they reached 11 days of incubation. The LD50 values for Days 1, 2, and 3 were 0.017, 0.007, and 0.012 mg per egg, respectively. The overall incidence of abnormal embryos for Days 0, 1, 2, and 3 were 7, 6.3, 12, and 22\%, respectively. Abnormalities such as reduced body size, everted viscera, short and twisted limbs, eye defects, abnormal beak, and short and twisted neck were commonly seen in survivors no matter when exposed to cyclophosphamide. The teratogenicity of cyclophosphamide was noted to be the highest in the embryos treated on Day 3. The present study has demonstrated that cyclophosphamide is toxic and teratogenic during the period of early organogenesis in the chick embryos. [\hyperlink{Cyclophosphamide}{PMID: 6684023}, S H Gilani et al., 1983]

\hypertarget{pmid_19882369}{C}yclophosphamide (Cy) is an alkylating agent used over the past 40 years to halt rapidly progressive forms of multiple sclerosis (MS). High doses of Cy produce marked immunosuppression and an anti-inflammatory immune deviation. Cy is most effective in young patients, with very active MS (frequent relapses, rapid accumulation of disability, and gad+ lesions on brain MRI). Monthly intravenous pulses of Cy for 1 year, followed by bimonthly pulses for the second year are a well-tolerated protocol in MS. Most side effects (mild alopecia, nausea and vomiting, and cystitis) are transient, dose dependent, and reversible. Permanent amenorrhoea and bladder cancer have rarely been described. As second-line therapy, Cy can be used in non-responders to IFN-beta or glatiramer acetate. As induction therapy, a short course (6-12 months) of Cy can precede immunomodulatory drugs in selected patients with an aggressive MS onset. [\hyperlink{Cyclophosphamide}{PMID: 19882369}, Luciano Rinaldi et al., 2009]

\hypertarget{pmid_10363852}{R}esults of a phase II trial of cyclophosphamide (CPM) for children with progressive low-grade astrocytoma are reported. Fifteen patients with a median age of 39 months (range, 2 to 71) were included in this study. The tumors of 11 children were located in the optic pathway, hypothalamus, or thalamus. Four courses of intravenous CPM 1.2 g/m2 were administered every 3 weeks during the upfront window portion of this protocol. Subsequently, chemotherapy was to continue with CPM, vincristine, and carboplatin for 2 years. By study design, the first 14 patients were centrally reviewed after completion of the initial 4 CPM courses. Toxicity was primarily hematologic. One patients had a complete response, 8 had stable disease, and 5 had progressive disease (PD). The excessive number of children with PD prompted study closure. CPM as used in this protocol showed insufficient activity against astrocytoma to justify further patient accrual. [\hyperlink{Cyclophosphamide}{PMID: 10363852}, R P Kadota et al., ]

\hypertarget{pmid_392408}{T}here is good, controlled evidence which suggests that cyclophosphamide, and perhaps related drugs, have a definite role in the treatment of nephrotic children with the minimal change lesion. This role is one of secondary treatment, and the drugs should not be used as a first line of attack; they should be employed only when corticosteroid resistance or toxicity is a problem. In a few patients, azathioprine or 6-mercaptopurine may have a role in minimising corticosteroid toxicity, but the remission induced in relapsing children is no more durable than that after corticosteroids. Chlorambucil must be given in doses, and for periods long enough to run the risk of neoplasia, particularly leukaemia; there does not appear to be a place for its use in nephrotic children unless the duration of remission can be shown to be longer than that obtainable with cyclophosphamide. There is no evidence that any immunosuppressive agent has a place in the management of children with idiopathic glomerular disease showing structural alterations in the glomeruli. Children with systemic lupus erythematosus and nephritis may benefit from the addition of cytotoxic agents to their corticosteroid regime, although the indications for this are not clear, and controlled evidence is lacking. [\hyperlink{Cyclophosphamide}{PMID: 392408}, J S Cameron et al., 1979]

\hypertarget{pmid_34818796}{C}yclophosphamide (CP) is a broad-spectrum anticancer drug and has been frequently detected in aquatic environments due to its incomplete removal by wastewater treatment facilities and slow degradation in waters. Its toxicity in fish remains largely unknown. In this study, zebrafish eggs <4 h post fertilization (hpf) were exposed to CP at the concentrations from 0.5 to 50.0 μg/L until 168 hpf, and its toxicity was evaluated by biochemical, transcriptomic, and behavioral approaches. The results showed that malformation and mortality rates increased with CP concentrations. The 7-day malformation EC [\hyperlink{Cyclophosphamide}{PMID: 34818796}, Dan Li et al., 2022] Cyclophosphamide is commonly used in the treatment of children with malignant brain tumors. The purpose of this study was to develop a multicycle, high-dose intensity cyclophosphamide regimen with granulocyte-macrophage colony-stimulating factor (GM-CSF) and to assess its activity against malignant glioma and primitive neuroectodermal tumor (PNET). Twenty-three patients with brain tumors, including 15 with malignant glioma and six with PNET, were enrolled. Cyclophosphamide (1.8-2.25 g/m2/day for 2 days i.v.; total dose 3.6-4.5 g/m2) was administered and was followed by recombinant human GM-CSF (5 micrograms/kg/day s.c.) on days 3-11 or until the absolute granulocyte count reached 1.5 x 10(9)/L. With a total of 83 cycles administered, the mean dose intensity of cyclophosphamide ranged from 1.5 g/m2/week through cycle 2 (22 patients) to 0.8 g/m2/week through cycle 8 (two patients). No activity was seen against malignant glioma, and five of six patients with PNET had partial responses. The mean duration of a neutrophil count of < 0.5 x 10(9)/L was only 8 days; the platelet recovery was substantially longer. Fever during neutropenia occurred in 54 of 83 cycles. One patient died from transfusion-related graft-versus-host disease. A cyclophosphamide regimen equal to twice the dose intensity of that used in conventional therapy was administered. The regimen was active against PNET but inactive against malignant glioma. [\hyperlink{Cyclophosphamide}{PMID: 34818796}, T G Abrahamsen et al., 1995]

\hypertarget{pmid_34374211}{C}yclophosphamide (CYP) is a widely used antineoplastic and immunosuppressive drug, however, despite its efficacy, it has shown extensive multiple organ toxicities, including peripheral neuropathy which significantly affects the quality of life of cancer patients. This study elucidated the protective properties of Shorea roxburghii polyphenol extract (SLPE) in CYP-induced peripheral neuropathy. Rats were treated with SLPE (100 and 400 mg/kg) for five weeks plus CYP once a week from the second week of SLPE treatment. Using UHPLC-QTOF-MS, 54 polyphenolic compounds were identified in SLPE extract. After the treatment period the antinociceptive, anti-hyperalgesia and antiallodynic effects was evaluated using formalin paw edema, acetic acid abdominal writhing, hot plate, tail immersion and von Frey filament tests. While the locomotive and motor coordination effects were evaluated by open field and rotarod tests. The administration of CYP led to significant increases in mechanical and thermal hyperalgesia, in addition to hyper-nociceptive responses in the formalin and acetic acid writhing tests. CYP also significantly reduced locomotive activity and motor coordination. SLPE significantly protected against CYP-induced mechanical and thermal hyperalgesia. Furthermore, SLPE displayed robust antinociceptive effect by counteracting formalin and acetic acid induced hyper-nociception. In addition, SLPE increased the locomotive activity as well as the grip and motor coordination of the CYP treated rats. In conclusion, these results revealed the protective effects of SLPE against CYP-induced peripheral neuropathy and could be an effective therapeutic remedy for chemotherapy induced peripheral neuropathy. [\hyperlink{Cyclophosphamide}{PMID: 34374211}, Haili Wang et al., 2021]

\hypertarget{pmid_11322378}{F}or over 30 years cyclophosphamide (CYC) and chlorambucil (CHL) have been used to treat children with relapsing steroid-sensitive nephrotic syndrome (SSNS). A meta-analysis on treatment protocols, efficacy, and side effects of CYC and CHL was performed from the literature. Thirty-eight studies comprising 1,504 children and 1,573 courses of cytotoxic drug therapy were systematically evaluated. Relapse-free survival rates increased with the cumulative dosage of CHL and CYC and were higher in children with frequently relapsing than steroid-dependent NS. The fatality rate of the treatment was approximately 1\%. Leukopenia occurred in one-third of patients treated with either drug. Severe bacterial infections developed in 1.5\% of the patients under CYC and in 6.8\% under CHL. Seizures were observed in 3.6\% of children treated with CHL. Malignancies were observed in 14 children after high doses of either drug. Females rarely developed permanent gonadal damage. However, no safe threshold for a cumulative amount of CYC was found in males, but there was a marked increase in the risk of oligo- or azoospermia with higher cumulative doses. From this meta-analysis we recommend CYC 2-3 mg/kg body weight for 8-12 weeks as the standard scheme. CHL has higher rates of severe side effects and should be considered a second-line drug. [\hyperlink{Cyclophosphamide}{PMID: 11322378}, K Latta et al., 2001]

\hypertarget{pmid_6229699}{C}yclophosphamide is widely used to induce a remission of minimal-change nephropathy, but concerns have been raised about whether its effects on cellular immunity persist after treatment is discontinued. We studied functional and numerical measures of cellular immunity in children who had minimal-change nephropathy with frequent steroid-responsive relapses and were receiving cyclophosphamide (2.5 mg per kilogram of body weight per day for eight weeks). Sequential studies during such treatment showed that cyclophosphamide caused lymphopenia, particularly among T helper cells, resulting in a significant fall in the immunoregulatory (helper/suppressor) cell ratio. This change persisted 1 to 3 months after cyclophosphamide was discontinued, but measures of immune function reverted to normal after 6 to 12 months. Children with minimal-change nephropathy in long-term remission had no difference in T-cell subpopulations, lymphocyte responses to mitogens, or suppressor-cell function that could be attributed to the disease itself or to the previous use of cyclophosphamide. [\hyperlink{Cyclophosphamide}{PMID: 6229699}, J Feehally et al., 1984]

\hypertarget{pmid_14722349}{T}o assess the efficacy and safety of intravenous cyclophosphamide (CYP) used in severe and refractory juvenile dermatomyositis (JDM). Retrospective case note review of the outcome of 12 patients. Assessment at 6 months of therapy in 10 of the 12 patients showed a significant improvement in muscle function as assessed by the Childhood Myositis Assessment Scale (CMAS) (P = 0.012), muscle strength (P = 0.008), global extramuscular disease score (P = 0.008), skin disease severity (P = 0.015) and lactate dehydrogenase (P = 0.028). There were reductions in creatine kinase, alanine aminotransferase, prednisolone dose and ESR, but these did not reach statistical significance. Clinical improvement was maintained after CYP until the most recent follow-up (between 6 months and 7 yr) and no severe side-effects were seen. Reversible complications included lymphopenia, herpes zoster infections and alopecia. The median cumulative dose was 4.6 g/m(2) (range 3-9 g/m(2)). The available evidence suggests that, at the doses required, risks of malignancy, infertility and gonadal failure are low. Two patients with severe treatment-resistant disease died after one dose of CYP, both of whom were ventilated prior to commencement of CYP and were thought to have died as a result of their severe disease process, and too early for clinical benefit to be obtained from the drug. In this cohort of children with severe and refractory JDM, CYP appeared to have provided major clinical benefit with no evidence of serious toxicity in the short term. [\hyperlink{Cyclophosphamide}{PMID: 14722349}, P Riley et al., 2004]

\hypertarget{pmid_10894368}{C}yclophosphamide is an alkylating agent that has shown activity in the treatment of pediatric brain tumors, including high-grade gliomas. This study was designed to evaluate the response of patients with newly diagnosed glioblastoma multiforme to pre-radiotherapy cyclophosphamide. Fourteen patients with glioblastoma multiforme were treated with high-dose cyclophosphamide (2 g/m2/day for 2 doses every 28 days) followed by either sargramostim or filgrastin. Sargramostim was given 250 microg/m2 subcutaneously twice a day continuing through the leukocyte nadir until the absolute neutrophil count was more than 1000 cells/microl for 2 consecutive days. The filgrastin dose was 10 microg/kg given subcutaneously once daily until the post nadir absolute neutrophil count was > or = 10,000 cells/microl. A total of 46 courses was given. Four patients received a total of 3 courses, 7 patients completed 4 courses and 3 patients received 2 courses. Three patients demonstrated complete response; 3 stable disease; and 8 progressive disease. The most common toxicity was hematologic, requiring platelet and packed red blood cell transfusions, with 13 admissions for neutropenia with fever. There were no deaths related to infection or bleeding. These results suggest that high-dose cyclophosphamide has modest activity with acceptable toxicity against newly diagnosed glioblastoma multiforme. [\hyperlink{Cyclophosphamide}{PMID: 10894368}, K S Bottom et al., 2000]

\hypertarget{pmid_28606234}{T}o evaluate the efficacy and safety of cyclophosphamide as a second-line drug in the treatment of children with anti-N-methyl-D-aspartate receptor (NMDAR) encephalitis. Six children with anti-NMDAR encephalitis, who showed poor response to steroids and intravenous immunoglobulin, were given cyclophosphamide as a second-line immunotherapy. Follow-up was performed to evaluate the efficacy and safety of cyclophosphamide. After first-line immunotherapy for 1-4 weeks, the six patients had reduced psychiatric symptoms, seizures, and involuntary movements; three patients had an improved level of consciousness and were able to make simple conversations. However, all the patients still showed slow response, as well as cortical dysfunction symptoms such as aphasia, alexia, agraphia, acalculia, apraxia, and movement disorders. The six patients continued to receive cyclophosphamide as a sequential therapy. They were able to answer simple questions 7 days after treatment. Three school-aged patients were able to make simple calculation, had greatly improved reading and writing ability, and almost recovered self-care ability 2-3 weeks later. The cognitive function of the six patients was almost restored to the level before the onset of disease, and their living ability returned to normal 2-3 months later. During the treatment period, there were no adverse reactions or abnormal results of routine blood test and liver and kidney function tests. Children with anti-NMDAR encephalitis should be given appropriate immunotherapy as soon as possible. Cyclophosphamide as a sequential therapy has good efficacy and safety. [\hyperlink{Cyclophosphamide}{PMID: 28606234}, Wei-Wen Zhu et al., 2017]

\hypertarget{pmid_29342499}{I}n patients with severe or refractory juvenile dermatomyositis (DM), second-line treatments may be required. Cyclophosphamide (CYC) is used to treat some connective tissue diseases, but evidence of its efficacy in juvenile DM is limited. This study was undertaken to describe clinical improvement in juvenile DM patients treated with CYC and model the efficacy of CYC treatment compared to no CYC treatment. Clinical data on skin, global, and muscle disease for patients recruited to the Juvenile DM Cohort and Biomarker Study were analyzed. Clinical improvement following CYC treatment was described using unadjusted analysis. Marginal structural models (MSMs) were used to model treatment efficacy and adjust for confounding by indication. Compared to the start of CYC treatment, there were reductions at 6, 12, and 24 months in skin disease (P = 1.3 × 10 Our findings indicate that CYC is efficacious with no short-term side effects. Improvements in skin, global, and muscle disease were observed. Further studies are required to evaluate longer-term side effects. [\hyperlink{Cyclophosphamide}{PMID: 29342499}, Claire T Deakin et al., 2018]

\hypertarget{pmid_7622780}{C}yclophosphamide is an alkylating agent used to treat haematologic malignant diseases and multisystem diseases with progressive glomerulonephritis. It is rarely prescribed during pregnancy. We report a case of Henoch-Schönlein purpura discovered at the end of the first trimester of pregnancy. Despite steroid therapy, glomerulonephritis worsened and 100 mg/day cyclophosphamide per os was administered from 28th week till delivery. The infant, prematurely born, was normal and did not have any haematological disorder. Congenital malformations are often reported (5 out of 19 newborns exposed in utero to cyclophosphamide), but in all those cases, there was another potentially teratogenic agent: either radiotherapy or another antineoplastic drug. Therefore, if mother's life is in jeopardy, cyclophosphamide therapy should be given and not postponed. [\hyperlink{Cyclophosphamide}{PMID: 7622780}, R Nguyen Tan Lung et al., 1995]

\hypertarget{pmid_31796512}{T}o characterize the population pharmacokinetics of cyclophosphamide, active 4-hydroxy-cyclophosphamide (4OH-CTX), and inactive carboxyethylphosphoramide mustard (CEPM), and their associations with hematologic toxicities in infants and young children with brain tumors. To use this information to provide cyclophosphamide dosing recommendations in this population. Patients received four cycles of a 1-hour infusion of 1.5 g/m Data from 171 patients (0.07-4.9 years) were adequately fitted by a two-compartment (cyclophosphamide) and one-compartment model (metabolites). Young infants (<6 months) exhibited higher mean 4OH-CTX exposure than did young children (138.4 vs. 107.2 μmol/L·h,  A 4OH-CTX exposure-toxicity association was established, and a decreased cyclophosphamide dosage for young infants was suggested to reduce toxicity in this population. Bayesian modeling to predict 4OH-CTX exposure may reduce clinical processing-related costs and provide insights into further exposure-response associations. [\hyperlink{Cyclophosphamide}{PMID: 31796512}, Olivia Campagne et al., 2020] 1. Cyclophosphamide pharmacokinetics were measured in 38 children with cancer. 2. A high degree of inter-patient variation was seen in all pharmacokinetic parameters. Cyclophosphamide half-life varied between 1.1 and 16.8 h, clearance varied between 1.2 and 10.61 h-1 m-2 and volume of distribution varied between 0.26 and 1.48 1 kg-1. 3. The half-life of cyclophosphamide was prolonged at high dose levels (P = 0.008). 4. Children who had received prior treatment with dexamethasone showed a mean increase in clearance of 2.51 h-1 m-2 (P = 0.001) presumably as a result of CYP450 enzyme induction. 5. Treatment with allopurinol or chlorpromazine was associated with a significant increase in cyclophosphamide half-life (P < 0.001 in both cases). 6. Dose and concurrent treatment may influence cyclophosphamide metabolism in vivo and thus potentially alter the drugs therapeutic effect. [\hyperlink{Cyclophosphamide}{PMID: 31796512}, S M Yule et al., 1996]

\hypertarget{pmid_26262887}{C}yclophosphamide (CP) is an oxazaphosphorine nitrogen mustard alkylating drug used for the treatment of chronic and acute leukemias, lymphoma, myeloma, and cancers of the breast and ovary. It is known to cause severe cardiac toxicity. This study investigated the protective effect of N-Acetylcysteine (NAC) on CP-induced cardiotoxicity in rats. CP resulted in a significant increase in serum aminotransferases, creatine kinase (CK), lactate dehydrogenase(LDH) enzymes, asymmetric dimethylarginine and tumor necrosis factor-α and significant decrease in total nitrate/nitrite(NOx). In cardiac tissues, a single dose of CP (200mg/kg, i.p.) resulted in significant increase in malondialdehyde and NOx and a significant decrease in reduced glutathione content, glutathione peroxidase, catalase, and superoxide dismutase activities. Interestingly, Administration of NAC (200mg/kg, i.p.) for 5 days prior to CP attenuates all the biochemical changes induced by CP. These results revealed that NAC attenuates CP-induced cardiotoxicity by inhibiting oxidative and nitrosative stress and preserving the activity of antioxidant enzymes.  [\hyperlink{Cyclophosphamide}{PMID: 26262887}, Heba H Mansour et al., 2015] Cyclophosphamide (Cyc) is an alkylating agent used to treat malignancies and autoimmune diseases, such as lupus nephritis, rheumatoid arthritis and immune-mediated neuropathies. Over the past 40 years, Cyc has also been applied to treat multiple sclerosis (MS) and the effective stabilisation of rapidly progressive forms of MS has been demonstrated in several studies. Cyc has a dose-dependent bimodal effect on the immune system. High doses have been demonstrated to induce an anti-inflammatory immune deviation (i.e., suppression of T helper 1 and enhancement of T helper 2 activity), affect CD4CD25(high) regulatory T cells and establish a state of marked immunosuppression. Data from the literature suggest that Cyc is particularly indicated in the treatment of young MS patients, suffering from a very active inflammatory disease characterised by frequent relapses and rapid accumulation of disability and displaying gadolinium-enhancing lesions on brain magnetic resonance. The most common Cyc-based therapeutic protocol applied in MS consists of monthly intravenous pulses for 1 year followed by bimonthly pulses for the second year, with or without prior infusion of corticosteroids. This protocol is usually well tolerated by the patients. Indeed, most of the side effects (mild alopecia, nausea and vomiting, cystitis) are dose dependent, transient and completely reversible. Definitive amenorrhoea is observed only in older female patients (aged > 40 years). Cyc has a safety and efficacy profile similar to that of mitoxantrone and can be used in patients whose disease is not controlled by IFN-beta or glatiramer acetate. Short course (6-12 months) of Cyc therapy can precede the initiation of immunomodulatory treatment in selected patients with an aggressive MS onset. [\hyperlink{Cyclophosphamide}{PMID: 26262887}, Paola Perini et al., 2007]

\hypertarget{pmid_30357088}{C}yclophosphamide is a chemotherapy drug for the treatment of cancer. Chicken embryo amniotic fluid, vitamin C and coenzyme Q10 have anti-oxidant properties. Total of 70 adult female mice were selected and divided into seven groups. The first group that received 2 ml kg [\hyperlink{Cyclophosphamide}{PMID: 30357088}, Alireza Kabirian et al., 2018] Cyclophosphamide is used extensively to treat malignancies. A 5-year-old boy with stage IV neuroblastoma is described who developed a fatal syndrome of inappropriate antidiuretic hormone (ADH) secretion after high dose cyclophosphamide therapy. [\hyperlink{Cyclophosphamide}{PMID: 30357088}, P J Harlow et al., 1979]

\section*{Phenoxybenzamine Hydrochloride}
\subsection*{Result}
\subsubsection*{Answer}

Ages 8–16: Yes  
Below 8 years: Unknown  
Infants/Toddlers: Unknown  

\subsubsection*{{Explanation}}
\hypertarget{Phenoxybenzamine Hydrochloride}
Based on the available abstracts, there are several that specifically address the use of Phenoxybenzamine Hydrochloride in children:

1. **Children with Pheochromocytoma (Ages 8–16):**  
A study reported the perioperative use of phenoxybenzamine in 5 children (6 phaeochromocytomas) aged 8 to 16 years. The study found that preoperative stabilization of blood pressure with phenoxybenzamine, followed by intraoperative management, was effective and safe. No perioperative complications worth mentioning were observed in these patients. The study recommends careful monitoring due to individual differences in dosing and duration, but overall affirms the safety of phenoxybenzamine in this context for children in this age range [\hyperlink{pmid_6660444}{PMID: 6660444}, M Abel et al., 1983].

2. **Children with Neuropathic Bladder Dysfunction (Unspecified Age, but "children"):**  
Phenoxybenzamine was evaluated in 14 children with neuropathic bladder dysfunction. The drug was generally well tolerated, with some side effects (postural hypotension, lethargy, nausea) reported. The abstract does not specify the exact ages, but refers to "children" and notes that the drug is "generally well tolerated" [\hyperlink{pmid_906185}{PMID: 906185}, N W Harrison et al., 1977].

3. **Children with Bladder Outlet Obstruction (Unspecified Age, but "children"):**  
Five children with persistent reflux and hydronephrosis were treated with phenoxybenzamine. Therapy was successful in all cases, restoring normal urine passage and voiding. No specific mention of adverse effects or complications is made, and the abstract suggests a positive safety and efficacy profile in this small group [\hyperlink{pmid_1113385}{PMID: 1113385}, K Stockamp et al., 1975].

4. **Large Study Including Age Analysis (Patients below age 35):**  
A study of 249 patients with neurogenic bladder function, including an analysis by age, found that those below age 35 had a better therapeutic response to phenoxybenzamine. No major side effects were encountered, but the abstract does not specify how many pediatric patients were included or their exact ages [\hyperlink{pmid_6106920}{PMID: 6106920}, H J Hachen et al., 1980].

**Summary by Age Range:**
- **Ages 8–16:** There is direct evidence from a targeted study in children with pheochromocytoma that phenoxybenzamine hydrochloride is safe for perioperative use in this age group, with no significant complications reported [\hyperlink{pmid_6660444}{PMID: 6660444}, M Abel et al., 1983].
- **Younger Children (Below 8 years):** The abstracts refer to "children" in studies of neuropathic bladder and bladder outlet obstruction, but do not specify ages. While these studies suggest phenoxybenzamine is generally well tolerated, the lack of precise age data means safety in very young children (e.g., infants, toddlers, preschoolers) cannot be definitively affirmed.
- **Infants and Toddlers:** No abstracts provide targeted safety data for phenoxybenzamine hydrochloride in infants or toddlers.

**Conclusion:**  
Phenoxybenzamine hydrochloride has been studied and found to be safe in children aged 8–16 years for specific indications (e.g., pheochromocytoma). For younger children, while some studies suggest general tolerability, the lack of precise age breakdowns means safety is not definitively established for those under 8 years old. There is no data for infants or toddlers.

\subsection*{Abstracts}
\hypertarget{pmid_6660444}{T}he article reports on the paediatric-anaesthesiological treatment of 6 phaeochromocytomas in 5 children who were 8 to 16 years of age. Therapeutic recommendations for the perioperative medication of infantile phaeochromocytoma patients are involved. The therapeutic aim of this study was the management of the effects of phaeochromocytoma before and after extirpation of the tumour, the effect of the phaeochromocytoma being of an alpha-adrenergic and beta-adrenergic cardiovascular nature and transmitted by catecholamines. Preoperative stabilization of blood pressure by means of the alpha-blocker phenoxybenzamine and a subsequent intraoperative, controlled reduction of blood pressure by means of sodium nitroprusside were found to be an effective, safe and easily appreciated therapeutic concept for the perioperative care of paediatric phaeochromocytoma patients. Considerable individual differences in dose an duration of the necessary preoperative phenoxybenzamine administration rendered ward control of therapy recommendable. The risk of complete alpha-sympathicolysis by additive drug effects during premedication and induction of anaesthesia, had to be taken into consideration for conducting phenoxybenzamine therapy. Additional administration of the beta-blocker pindolol successfully controlled the intraoperatively manifested tachycardial heart rhythm phases without provoking any complicating arrhythmias. During the entire perioperative treatment of the patients it is mandatory to ensure sufficient substitution of intravascular volume to prevent hypotensive complications. Our patients did not need any cardiac and sympathicomimetic drugs as postoperative administration. None of the patients had any perioperative complications worth mentioning. [\hyperlink{Phenoxybenzamine Hydrochloride}{PMID: 6660444}, M Abel et al., 1983]

\hypertarget{pmid_17941284}{T}he safety of fexofenadine has been examined extensively in adults and school-age children. However, the safety of fexofenadine in children younger than 6 years has not been reported to date. To compare the safety and tolerability of twice-daily fexofenadine hydrochloride, 30 mg, and placebo in preschool children aged 2 to 5 years with allergic rhinitis. This was a multicenter, double-blind, randomized, placebo-controlled, parallel-group study, conducted between February 29, 2000, and June 14, 2001. Participants were randomized to either fexofenadine hydrochloride, 30 mg, or placebo twice daily for a 2-week period. To facilitate dosing, capsule content was mixed with applesauce (approximately 10 mL). Safety assessments depended on date of entry into the study because of an amendment to the protocol. Before the amendment, assessments included physical examination, vital signs reporting (oral temperature, heart rate, and respiratory rate), and adverse event (AE) reporting. After the amendment, safety assessments included laboratory testing (blood chemistry and hematology profiles), physical examination, 12-lead electrocardiography, and vital signs (oral temperature, blood pressure, heart rate, and respiratory rate) and AE reporting. Treatment-emergent AEs were observed in 116 of 231 participants receiving placebo and 111 of 222 receiving fexofenadine. These AEs were possibly related to study medication in 19 (8.2\%) and 21 (9.5\%) of the participants receiving placebo and fexofenadine, respectively, and most frequently involved the digestive system. No clinically relevant differences in laboratory measures, vital signs, and physical examinations were observed. The findings show that fexofenadine hydrochloride, 30 mg, is well tolerated and has a good safety profile in children aged 2 to 5 years with allergic rhinitis. [\hyperlink{Phenoxybenzamine Hydrochloride}{PMID: 17941284}, Henry Milgrom et al., 2007]

\hypertarget{pmid_906185}{P}henoxybenzamine, an alpha-adrenergic blocking drug has been evaluated in the treatment of 14 children with neuropathic bladder dysfunction. The majority of children treated showed significant reductions of residual urine and urethral closure pressure. A useful clinical response in terms of improved bladder control was obtained in over half the patients. Children with mild radiological changes in the upper urinary tracts showed radiological improvement but those with marked changes did not improve. The drug is generally well tolerated but side-effects of postural hypotension, lethargy and nausea may occur. [\hyperlink{Phenoxybenzamine Hydrochloride}{PMID: 906185}, N W Harrison et al., 1977]

\hypertarget{pmid_18219837}{A}ntihistamines are an established first-line treatment for allergic rhinitis and are widely prescribed in infants for allergic symptoms. To establish the safety and tolerability of fexofenadine hydrochloride in children aged 6 months to 2 years in 2 studies (T/3001 and T/3002). Both studies had a multicenter, randomized, placebo-controlled design. Mean treatment duration was 8 days. Subjects were randomized (T/3001, n = 174; and T/3002, n = 219) to twice-daily fexofenadine hydrochloride, 15 or 30 mg, or placebo mixed with a standard vehicle. In the combined population, the incidence of treatment-emergent adverse events (TEAEs) was comparable between groups (placebo, 48.2\% [96/199]; fexofenadine hydrochloride, 15 mg, 40.0\% [34/85]; and fexofenadine hydrochloride, 30 mg, 35.2\% [38/108]). Vomiting was the most common TEAE (placebo, 13.6\%; fexofenadine hydrochloride, 15 mg, 14.1\%; and fexofenadine hydrochloride, 30 mg, 5.6\%). Most TEAEs were unrelated to study medication, as evaluated by investigators; those possibly related to study medication were mild or moderate in intensity. No clinical differences were seen between fexofenadine and placebo for vital signs, electrocardiographic results, or physical examination results. Fexofenadine hydrochloride, 15 or 30 mg, given for a mean duration of 8 days is well tolerated, with a good safety profile, in children aged 6 months to 2 years. [\hyperlink{Phenoxybenzamine Hydrochloride}{PMID: 18219837}, Frank C Hampel et al., 2007]

\hypertarget{pmid_18702885}{A}llergic rhinitis (AR) is a common chronic condition in children and may impact a child's quality of life. Increasing treatment compliance may improve quality of life. An oral suspension of fexofenadine hydrochloride (HCl) has been developed to ease administration to children and may, therefore, improve treatment compliance. The purpose of this study was to assess the pharmacokinetic behavior, safety, and tolerability of a single dose of fexofenadine HCl oral suspension administered to children aged 2-5 years with allergic rhinitis. Children (aged 2-5 years) with AR were recruited in a multicenter, open-label, single-dose study. Fexofenadine HCl (30 mg) was administered as a 6-mg/mL suspension (5 mL). Plasma samples were collected up to 24 hours postdose. Adverse events (AEs); electrocardiograms (ECGs); vital signs; and clinical laboratory tests for hematology, blood chemistry, and urinalysis were analyzed to evaluate safety and tolerability. Fifty subjects completed the study. Mean maximum plasma concentration of fexofenadine was 224 ng/mL, and mean area under the plasma concentration curve was 898 ng . hour/mL. Treatment-emergent AEs were mild in intensity and reported in a total of seven subjects. No trends or clinically meaningful changes in mean ECG, vital sign, or clinical laboratory test data occurred during the study. In children aged 2-5 years, the exposure after a 30-mg dose of fexofenadine HCl suspension was similar to the exposures previously seen after a 30- and 60-mg dose of fexofenadine HCl in children aged 6-11 years and in adults, respectively. The suspension was also well tolerated. [\hyperlink{Phenoxybenzamine Hydrochloride}{PMID: 18702885}, Nathan Segall et al., ]

\hypertarget{pmid_1172955}{P}hencyclidine hydrochloride is a dangerous drug. Its incidence as the causative agent in childhood poisoning is increasing. A pressor effect of phencyclidine has been noted in studies both in man and in experimental animals. We summarize seven cases of poisoning with this drug, including one in which death occurred following a hypertensive crisis. Patients who have ingested this drug should have continuous monitoring of blood pressure in an intensive care unit. [\hyperlink{Phenoxybenzamine Hydrochloride}{PMID: 1172955}, J W Eastman et al., 1975]

\hypertarget{pmid_37655364}{F}exofenadine hydrochloride (HCl) is a second-generation, nonsedating, histamine H1-receptor antagonist used to manage seasonal allergic rhinitis and chronic idiopathic urticaria. A new oral pediatric suspension of fexofenadine HCl has been developed, with the preservative potassium sorbate replacing parabens. The objective of this phase 1 single-center, open-label, randomized, 2-treatment, full-replicated, 4-period, 2-sequence crossover study in healthy adult volunteers was to assess the bioequivalence of 30 mg of the new oral suspension of fexofenadine HCl (test) versus 30 mg of the marketed pediatric oral suspension of fexofenadine HCl (reference). The replicate design was based on the high intra-individual variability of fexofenadine (>30\% on C [\hyperlink{Phenoxybenzamine Hydrochloride}{PMID: 37655364}, Clemence Rauch et al., 2023] Phenoxybenzamine (PBZ) is an FDA approved α-1 adrenergic receptor antagonist that is currently used to treat symptoms of pheochromocytoma. However, it has not been studied as a neuroprotective agent for traumatic brain injury (TBI). While screening neuroprotective candidates, we found that phenoxybenzamine reduced neuronal death in rat hippocampal slice cultures following exposure to oxygen glucose deprivation (OGD). Using this system, we found that phenoxybenzamine reduced neuronal death over a broad dose range (0.1 µM-1 mM) and provided efficacy when delivered up to 16 h post-OGD. We further tested phenoxybenzamine in the rat lateral fluid percussion model of TBI. When administered 8 h after TBI, phenoxybenzamine improved neurological severity scoring and foot fault assessments. At 25 days post injury, phenoxybenzamine treated TBI animals also showed a significant improvement in both learning and memory compared to saline treated controls. We further examined gene expression changes within the cortex following TBI. At 32 h post-TBI phenoxybenzamine treated animals had significantly lower expression of pro-inflammatory signaling proteins CCL2, IL1β, and MyD88, suggesting that phenoxybenzamine may exert a neuroprotective effect by reducing neuroinflammation after TBI. These data suggest that phenonxybenzamine may have application in the treatment of TBI.  [\hyperlink{Phenoxybenzamine Hydrochloride}{PMID: 37655364}, Thomas F Rau et al., 2014] Fexofenadine hydrochloride is a non-sedating antihistamine that is used in the treatment of symptoms associated with seasonal allergic rhinitis and chronic idiopathic urticaria. A pooled analysis of pharmacokinetic data from children 6 months to 12 years of age and adults was conducted to identify the dose(s) in children that produce exposures comparable to those in adults for the treatment of seasonal allergic rhinitis. The pharmacokinetic parameter database included peak and overall exposure data from 269 treatment exposures from 136 adult subjects, and 90 treatment exposures from 77 pediatric allergic rhinitis patients. The data were pooled and analysed using NONMEM software, version 5.0. A covariate model based on body weight and age and a power function model based on body weight were identified as appropriate models to describe the variability in fexofenadine oral clearance and peak concentration, respectively. Individual oral clearance estimates were on average 44\%, 36\% and 61\% lower in children 6 to 12 years (n=14), 2 to 5 years (n=21), and 6 months to 2 years (n=42), respectively, compared with adults. Trial simulations (n=100) were carried out based on the final pharmacostatistical models and parameter estimates to identify the appropriate dose(s) in children relative to the marketed dose of 60 mg fexofenadine hydrochloride in adults. The trials were designed as crossover studies in 18 subjects comprising various potential dosing regimens with and without weight stratification. Pharmacokinetic parameter variability was assumed to have a log-normal distribution. Individual weights and ages were simulated using mean (SD) estimates derived from the studies used in this analysis and proportional measurement/model mis-specification errors derived from the analysis were incorporated into the simulation. The results indicated that a 30 mg dose of fexofenadine hydrochloride administered to children 1 to 12 years of age and weighing >10.5 kg and a 15 mg dose administered to children 6 months and older and weighing <or=10.5 kg produces exposures similar to those seen with the 60 mg dose in adults. [\hyperlink{Phenoxybenzamine Hydrochloride}{PMID: 37655364}, Rajesh Krishna et al., 2004]

\hypertarget{pmid_1113385}{F}ive children in whom reflux and progressive hydronephrosis persisted despite multiple surgical attempts or repair are described. In all cases cystomanometry displayed a marked elevation of the bladder outlet resistance combined with high intravesical pressure values. Therapy with phenoxybenzamine, an alpha-adrenergic blocker, was successful  in all cases, restoring a free urine passage of the upper urinary tract and unimpaired voiding preventing urinary diversion which has been considered in some of these children. Although there were no signs of bladder neuropathy, a hyperfunction of the sympathetic innervation as acause for bladder complications is discussed. [\hyperlink{Phenoxybenzamine Hydrochloride}{PMID: 1113385}, K Stockamp et al., 1975]

\hypertarget{pmid_8282390}{A} randomized double blind placebo controlled trial was carried out to study the effect of phenobarbitone (PB) in preventing recurrences of simple and atypical febrile convulsions among children in the age group 6 months to 6 years. Children with simple febrile convulsions were randomly allocated to receive either phenobarbitone or placebo. Children with atypical convulsions were treated with phenobarbitone, as a third group. Thirty children were admitted in each group. All the children were followed up for a period of twelve months. Recurrence of convulsions and side effects of PB were recorded. Recurrence occurred in only 7\% (95\% confidence interval: 1-22) of children on Phenobarbitone, suffering from either simple or atypical febrile convulsions, compared to 53\% (95\% confidence interval: 34-72) of children on placebo, suffering from simple febrile convulsions. With Phenobarbitone, 5\% of children had intolerable side effects. These results suggest that long term prophylaxis with phenobarbitone, even in simple febrile convulsions will be useful. [\hyperlink{Phenoxybenzamine Hydrochloride}{PMID: 8282390}, N Thilothammal et al., 1993]

\hypertarget{pmid_11120393}{F}enbendazole (FBZ) is a benzimidazole currently used for anthelmintic treatment of pinworm populations in numerous animal species although it is not currently approved for laboratory rodents in the U.S. It has received considerable interest for treating rodent populations due to its low toxicity, wide safety margin and apparent absence of gross teratogenic effects. The purpose of this study was to assess the behavioral teratogenic potential of FBZ. Pregnant rats were administered either FBZ-medicated feed at a therapeutic level or normal rat chow throughout pregnancy and gestation. FBZ had no effect on pregnancy indicators such as maternal weight gain or water consumption, number of pups born or pup birth weights. Offspring were examined in a variety of paradigms including righting reflex, negative geotaxis, running wheel activity, Morris water maze (MWM) performance and digging maze performance. FBZ offspring did show delayed righting reflex, some modest changes in locomotor activity in a running wheel and minor alterations in performance during the probe session of the MWM relative to controls. However, the effects of FBZ on behavior were subtle and many of the behaviors examined were unaffected. These results suggest that FBZ may be an effective and relatively safe anthelmintic treatment for use in breeding colonies. [\hyperlink{Phenoxybenzamine Hydrochloride}{PMID: 11120393}, S Barron et al., ]

\hypertarget{pmid_9377206}{T}he stability of phenoxybenzamine hydrochloride in various oral liquids was studied. Phenoxybenzamine hydrochloride powder or capsules were combined with various vehicles to prepare 10-mL formulations with a drug concentration of 2 mg/mL and a 20-mL stock solution containing 10 mg/mL. All formulations were prepared in triplicate and stored at 4 degrees C. A 1.0-mL sample of each of the 2-mg/mL formulations was withdrawn on days 0, 1, 2, 3, and 4, and samples of selected formulations were taken on days 7 and 10. Samples of the stock solution were withdrawn on days 0, 2, 4, 7, 10, 14, and 30. All samples were analyzed by high-performance liquid chromatography. Phenoxybenzamine hydrochloride 2 mg/mL was stable for a longer time in 1\% propylene glycol, 0.15\% citric acid, and water than in a similar vehicle in which syrup was used in place of water. The concentrations of the drug in both the 10-mg/mL stock solution and stock solutions diluted to 2 mg/mL were > 90\% of the initial drug concentration for 30 days. Phenoxybenzamine hydrochloride 2 mg/mL in 1\% propylene glycol and 0.15\% citric acid in distilled water was stable for 7 days at 4 degrees C. A stock solution of phenoxybenzamine hydrochloride 10 mg/mL in propylene glycol was stable for 30 days at 4 degrees C and, after dilution with 66.7\% sucrose in distilled water to a concentration of 2 mg/mL, for up to one hour at 4 degrees C. [\hyperlink{Phenoxybenzamine Hydrochloride}{PMID: 9377206}, L Y Lim et al., 1997]

\hypertarget{pmid_11476456}{T}he incidence of allergic rhinitis in children is increasing. To evaluate the safety of fexofenadine HCI in children ages 6 through 11 years for treatment of seasonal allergic rhinitis. Two large, double-blind, randomized, placebo-controlled, parallel studies with identical protocols included patients with a positive skin test to fall allergen(s) and allergic rhinitis symptoms. Patients were randomized to receive fexofenadine 15, 30, or 60 mg or placebo twice daily for 2 weeks after a 1-week placebo lead-in. Safety was evaluated through adverse event reporting, electrocardiograms, and pre- and posttreatment laboratory panels and physical examinations. A total of 875 patients from both studies were eligible for safety analyses. Ten patients (5 on placebo, 5 on fexofenadine) discontinued because of an adverse event; no event that resulted in discontinuation was judged to be caused by study medication. Incidence of adverse events was similar in active and placebo groups, and did not increase with increasing fexofenadine dose: 36.2\% (83 of 229) in the placebo group versus 35.3\% (79 of 224), 36.8\% (77 of 209), and 34.7\% (74 of 213) in the 15, 30, and 60 mg twice-daily fexofenadine groups, respectively. Headache was the most commonly reported adverse event (6.6\%, 8.0\%, 7.2\%, and 9.4\% in the placebo, 15, 30, 60 mg twice-daily fexofenadine groups, respectively). Clinical, vital sign, electrocardiogram, and laboratory measures were similar in active and placebo groups. There was no statistically significant mean change from baseline in any electrocardiogram parameter after fexofenadine treatment. Fexofenadine, 15, 30, and 60 mg twice daily, was safe and well tolerated in this large pediatric patient population. [\hyperlink{Phenoxybenzamine Hydrochloride}{PMID: 11476456}, D F Graft et al., 2001]

\hypertarget{pmid_6106920}{T}he efficacy of alpha-adrenolytic treatment with oral phenoxybenzamine chloride (40 mg per day during three to four months) has been assessed both clinically and urodynamically among 249 patients with neurogenic bladder function. The results were significantly better in patients with autonomous (n = 95) than in the ones with automatic (n = 154) bladders (urethral sphincter spasticity and detrusor sphincter dyssynergia dominating the clinical picture in the latter event). A further striking difference was noticed when the data were analysed with regard to the patient's age, a far better therpeutic response being regularly recorded in subjects below age 35, while minimal and inconsistent improvements were observed in the older age group. No major side effects were encountered. Caution is however required in tetraplegics where phenoxybenzamine may aggravate orthostatic hypotension. On the other side the drug proves highly beneficial in these same patients in that it markedly lowers the incidence of dysreflexic states. [\hyperlink{Phenoxybenzamine Hydrochloride}{PMID: 6106920}, H J Hachen et al., 1980]

\hypertarget{pmid_23534952}{A} test dose is used to detect intravascular injection during neuraxial block in pediatrics. Accidental intravascular epidural local anesthetic injection might be unrecognized in anesthetized children leading to potential life-threatening complications. In children, sevoflurane anesthesia blunts the hemodynamic response when intravascular cathecolamines are administered. No studies have explored the hemodynamics and the criteria for a positive test dose result following phenylephrine in sevoflurane anesthetized children. Healthy children undergoing minor procedures were randomly assigned to receive intravenous placebo, or 5 μg∙kg(-1) phenylephrine (n = 11/group) during sevoflurane anesthesia. Hemodynamic response was assessed using electrocardiography, pulse oxymetry and non-invasive blood pressure monitoring for 5 min following drug administration in anesthetized patients. All patients receiving phenylephrine showed a decreased heart rate (HR) but not all of them met the positive criterion for test dose response. Overall, at 1 min, patients receiving phenylephrine showed a 25\% decrease in HR from the baseline while an increase in blood pressure was noticed in 54\% of patients receiving phenylephrine. Phenylephrine might be a future indicator of positive intravascular test dose. Further investigation is needed to find out the phenylephrine dose that elicits a reliable hemodynamic response and whether phenylephrine needs to be dose age-adjusted in order to appreciate relevant hemodynamic changes in children receiving neuraxial blocks undergoing general anesthesia. [\hyperlink{Phenoxybenzamine Hydrochloride}{PMID: 23534952}, Carlo Pancaro et al., 2013]

\hypertarget{pmid_34719411}{T}o study the safety and efficacy of dexmedetomidine hydrochloride combined with midazolam in fiberoptic bronchoscopy in children. A total of 118 children who planned to undergo fiberoptic bronchoscopy from September 2018 to February 2021 were enrolled. They were divided into a control group ( Compared with the control group, the observation group had significantly decreased MAP at T Dexmedetomidine hydrochloride combined with midazolam is a safe and effective way to administer general anesthesia for fiberoptic bronchoscopy in children, which can ensure stable vital signs during examination, reduce intraoperative adverse reactions and postoperative agitation, shorten examination time, and increase amnesic effect. [\hyperlink{Phenoxybenzamine Hydrochloride}{PMID: 34719411}, Jin Zhang et al., 2021]

\hypertarget{pmid_34843869}{T}he existing information supports the use of this material as described in this safety assessment. Phenethyl phenylacetate was evaluated for genotoxicity, repeated dose toxicity, reproductive toxicity, local respiratory toxicity, phototoxicity/photoallergenicity, skin sensitization, and environmental safety. Data show that phenethyl phenylacetate is not genotoxic. Data provide a calculated MOE >100 for the repeated dose toxicity endpoint. Data on read-across analog benzyl benzoate (CAS \# 120-51-4) provide an MOE >100 for the developmental toxicity endpoint. The fertility and local respiratory toxicity endpoints were evaluated using the TTC for a Cramer Class I material, and the exposure to phenethyl phenylacetate is below the TTC (0.03 mg/kg/day, and 1.4 mg/day, respectively). Data from analog benzyl phenylacetate (CAS \# 102-16-9) show that there are no safety concerns for phenethyl phenylacetate for skin sensitization under the current declared levels of use. The phototoxicity/photoallergenicity endpoints were evaluated based on UV/Vis spectra; phenethyl phenylacetate is not expected to be phototoxic/photoallergenic. The environmental endpoints were evaluated; phenethyl phenylacetate was found not to be PBT as per the IFRA Environmental Standards and its risk quotients, based on its current volume of use in Europe and North America (i.e., PEC/PNEC), are <1. [\hyperlink{Phenoxybenzamine Hydrochloride}{PMID: 34843869}, A M Api et al., 2022]

\hypertarget{pmid_28476033}{S}everal studies have reported the use of dexmedetomidine (DEX) plus opioids for flexible bronchoscopy in both adults and children. To determine whether DEX plus sufentanil (SF) is safe for children, 142 children undergoing flexible bronchoscopy were assigned to one of three groups, each of which received the same SF loading dose and similar DEX and SF maintenance doses, but different loading doses of DEX: DS1 (DEX 0.5 μg·kg-1), DS2 (DEX 1.0 μg·kg-1), and DS3 (DEX 1.5 μg·kg-1). The Ramsay sedation scale was maintained at 3 in all groups. Results showed that anesthesia onset time was shorter, and the perioperative hemodynamic profile was more stable, in the DS3 group. The number of intraoperative movements was also lowest in the DS3 group. The time to first dose of rescue midazolam and lidocaine was significantly longer, but the total corresponding accumulated doses were lower in the DS3 group. Although the time to recovery prior to discharge from the post anesthesia care unit was longer, the overall incidence of tachycardia was lower in the DS3 group, and it received the highest bronchoscopist satisfaction score among the three groups. We therefore conclude that high-dose DEX plus SF can be safely and efficaciously used in children undergoing flexible bronchoscopy. [\hyperlink{Phenoxybenzamine Hydrochloride}{PMID: 28476033}, Xiujing Dang et al., 2017]

\hypertarget{pmid_34478978}{P}erfluorohexanoic acid (PFHxA), a widely used emerging alternative for 8-carbon PFAAs, has been detected at a high level in the water environment. While its toxicity and environmental health risk are still largely unknown in aquatic life. The present study aimed to evaluated the possible developmental neurotoxicity induced by PFHxA exposure (0, 0.48, 2.4, and 12 mg/L for 120 h) in the zebrafish embryo. Here, both developmental endpoints, neurotransmitters concentrations, locomotor behavior were analyzed. No significant effects on mortality, malformation rate, and growth delay were detected in the low dose treatment groups except for in the high dose group (12 mg/L). A significant increase in swimming speed were noted in the 0.48 mg/L group. Other changes including neurotransmitters concentrations and green fluorescent protein (GFP) expression in Tg (HuC-GFP) zebrafish larvae were significantly increased in 12 mg/L group. Beyond that, genes related to neurodevelopment were significantly decreased in larvae. Moreover, downregulations of protein expression levels of α1-tubulin, elavl3, and gap43 were identified. These results demonstrate that the PFAAs alternative PFHxA have no significant neurodevelopmental effects on zebrafish larvae under acute low-dose exposure, while, it is important to note that PFHxA perform inhibiting effects on neurotransmitter and central nervous system under a relatively high dose. This in vivo study could provide reliable toxicity information for risk assessments of PFHxA on aquatic ecosystems. CAPSULE: PFHxA have no significant neurodevelopmental effects on zebrafish larvae under acute low-dose exposure, while exposed with relatively high-dose, could induced the alternations of neurotransmitter concentrations as well as the genes involved in the early developmental stages of zebrafish, leading to the impairment of the nervous system in zebrafish larvae. [\hyperlink{Phenoxybenzamine Hydrochloride}{PMID: 34478978}, Xiaochun Guo et al., 2021]

\hypertarget{pmid_25726705}{W}ith the increasing resistance to antibiotics among common bacterial pathogens, challenges associated with the use of fluoroquinolones (FQs) in paediatrics have emerged. The majority of FQs have favourable pharmacokinetic properties, although these properties can differ in children compared with adults. Moreover, all FQs have broad antimicrobial activity both against Gram-positive and Gram-negative bacteria. However, only some FQs for which adequate studies are available have been approved for use in children in a limited number of clinical situations owing to the supposed risk of development of severe musculoskeletal disorders, as demonstrated in juvenile animals. Recent short- and long-term evaluations appear to indicate that, at least for levofloxacin, this risk, if present at all, is marginal. This marginal risk could lead to more frequent use of FQs in children, even to treat diseases for which several other drugs with documented efficacy, safety and tolerability are considered the first-line antibiotics. However, for most of the FQs, adequate long-term studies of safety are not available. This indicates that the use of FQs should be limited to selected respiratory infections (including tuberculosis), exacerbation of lung disease in cystic fibrosis, central nervous system infections, enteric infections, febrile neutropenia, as well as serious infections attributable to FQ-susceptible pathogen(s) in children with life-threatening allergies to alternative agents. When considering diseases that could benefit from the use of FQs, particular attention must be paid to the choice of drug and its dosage, considering that not all of the FQs have been evaluated in different diseases.  [\hyperlink{Phenoxybenzamine Hydrochloride}{PMID: 25726705}, Nicola Principi et al., 2015] Only a few corticosteroids for topical use have proven safe and effective in pediatric populations down to 3 months of age. The authors report the results of a study designed to assess the efficacy and safety of hydrocortisone butyrate (HCB) 0.1\% in lipocream (LCr) vehicle in infants and children. A total of 264 boys and girls 3 months to less than 18 years old, with stable, mild to moderate atopic dermatitis affecting at least 10\% body surface area applied HCB 0.1\% in LCr or LCr alone twice daily for up to 1 month without occlusion. Primary end-points included: percent of patients who achieved treatment success based on physician global assessments. Secondary endpoint included: difference in pruritus and Eczema Area and Severity Index (EASI) at day 29. Treatment was significant (P < 0.001) for HCB 0.1\% LCr over vehicle. No serious nor significant adverse events were reported. Results are representative of a short duration treatment for a chronic disease. HCB 0.1\% in LCr is more effective than its vehicle in pediatric populations down to 3 months of age without significant adverse events when used twice a day for up to 1 month. [\hyperlink{Phenoxybenzamine Hydrochloride}{PMID: 25726705}, William Abramovits et al., ]

\hypertarget{pmid_16292117}{T}o evaluate the safety, tolerability, and benefit of fluvoxamine for the treatment of major depressive disorder or anxiety disorders in children and adolescents with cancer. The study was conducted from 2001 to 2004 at a pediatric hematology-oncology center. Fifteen children and adolescents with cancer were treated with fluvoxamine 100 mg/day in an open prospective 8-week trial. Safety and tolerability were evaluated at baseline and at weeks 4 and 8 by blood tests and the Side Effects Checklist. Clinical benefit was assessed with the Clinical Global Impressions-Improvement, the Children's Depression Rating Scale-Revised, and the Pediatric Anxiety Rating Scale. Fluvoxamine was well tolerated by all subjects. Psychiatric symptoms improved significantly. In this open trial, fluvoxamine appeared to be well tolerated and was associated with a promising reduction in the depression and anxiety symptoms of pediatric patients with cancer. [\hyperlink{Phenoxybenzamine Hydrochloride}{PMID: 16292117}, Doron Gothelf et al., 2005]

\hypertarget{pmid_549044}{I}n a group of 44 children with behaviour disorders, the mode of action of the phenothiazine preparation fluphenazine hydrochloride (Lyorodin) was tested by means of psychophysiological methods of measurement (critical flicker-fuse frequency and measurement of the response time). In some of the comparative examinations before and after administration of the drug significant differences were found. A defined psychic stress resulted in changed measured values in both methods. Children with cerebral injuries showed a lower degree of reaction after the administration of the preparation than children not encephalopathically affected. [\hyperlink{Phenoxybenzamine Hydrochloride}{PMID: 549044}, E Müller et al., 1979]

\hypertarget{pmid_16777373}{P}enequine hydrochloride, a novel anticholinergic agent, was developed as an effective treatment for organophosphorus intoxication (e.g., soman poisoning). The current study was performed to assess the potential pre- and post-natal toxicity of penequine hydrochloride in mice. Approximately 120 timed-pregnant mice were assigned to four dose groups (n=30 per group). Dams were exposed orally to 0, 2.5, 12.5, 62.5 mg/L penequine hydrochloride in drinking water from gestation day 6 to lactation day 21. The F1 generation mice, which were not exposed directly to penequine hydrochloride as pups or as adults, were bred to produce F2 generation fetuses for the fertility test of the F1 population. Various pre- and post-natal measurements, including neurobehavioral tests, were performed with the F0 and F1 mice. Among the significant findings were decreases in water consumption, viability, organ weights and delay of physical landmarks in 62.5 mg/L groups. With the exception of treatment-unrelated abnormality in surface righting reflex in the F1 generation, penequine hydrochloride did not produce any adverse effects at doses up to and including 12.5 mg/L (equal to 2.5 mg/kg/day in mice) that were at least 75 times of human therapeutic dosage. [\hyperlink{Phenoxybenzamine Hydrochloride}{PMID: 16777373}, Zibo Zhang et al., 2006]

\section*{Diclofenac Potassium}
\subsection*{Result}
\subsubsection*{Answer}

Unknown

\subsubsection*{{Explanation}}
\hypertarget{Diclofenac Potassium}
Based on the abstracts available, there is limited direct evidence specifically about the safety of Diclofenac Potassium in children. Most pediatric studies focus on diclofenac sodium, but a few abstracts mention Diclofenac Potassium or do not specify the salt form. Below is a summary of the relevant evidence for Diclofenac Potassium and, where necessary, for diclofenac in general, with attention to age ranges and study design.

1. Diclofenac Potassium in Children:
- One abstract (\hyperlink{pmid_2111251}{PMID: 2111251}, L A Bahamonde et al.) describes a study of Diclofenac Potassium in patients with sprained ankles, but the age range is not specified, and there is no mention of children or pediatric patients.
- Another abstract (\hyperlink{pmid_19038583}{PMID: 19038583}, Ingrid Chantler et al., 2009) investigates Diclofenac Potassium for menstrual pain, but the study population is "healthy young women," not children.

2. Diclofenac (unspecified salt or sodium) in Children:
- Multiple studies have evaluated the safety and efficacy of diclofenac (often sodium) in children for various indications (postoperative pain, fever, juvenile arthritis, etc.), with age ranges from 6 months to 16 years. These studies generally report that diclofenac is effective and has a low incidence of serious adverse events in children [\hyperlink{pmid_19694745}{PMID: 19694745}, Joseph F Standing et al., 2009; \hyperlink{pmid_11952445}{PMID: 11952445}, B Wennström et al., 2002; \hyperlink{pmid_3377147}{PMID: 3377147}, M E Bone et al., 1988; \hyperlink{pmid_15206315}{PMID: 15206315}, D V Leont'ev et al.; \hyperlink{pmid_6361986}{PMID: 6361986}, J Haapasaari et al., 1983; \hyperlink{pmid_17319338}{PMID: 17319338}, P U N Nze et al., 2006; \hyperlink{pmid_17897274}{PMID: 17897274}, Mairead Heaney et al., 2007; \hyperlink{pmid_25260983}{PMID: 25260983}, H Y El Batawi et al., 2015; \hyperlink{pmid_11299404}{PMID: 11299404}, M I Tawalbeh et al., 2001; \hyperlink{pmid_26889398}{PMID: 26889398}, Mohammad Reza Sharif et al., 2016].
- Some studies specifically address pharmacokinetics and dosing in children aged 1-12 years, recommending pediatric dosing regimens [\hyperlink{pmid_21276131}{PMID: 21276131}, Joseph F Standing et al., 2011; \hyperlink{pmid_24815417}{PMID: 24815417}, Jacqueline A Hannam et al., 2014].
- A prospective observational study in 380 children (≤12 years) found the incidence of serious adverse reactions to diclofenac to be <0.8\% [\hyperlink{pmid_19694745}{PMID: 19694745}, Joseph F Standing et al., 2009].

3. Diclofenac Potassium: Safety in Children
- There are no abstracts describing a targeted safety study of Diclofenac Potassium in children. The only studies of Diclofenac Potassium are in adults or do not specify the age range.
- Therefore, based on the abstracts available, the safety of Diclofenac Potassium in children is unknown.

4. Diclofenac Sodium and Unspecified Diclofenac: Safety in Children
- Multiple targeted studies in children (ages 6 months to 16 years) affirm the safety of diclofenac (mostly sodium salt) for short-term use in various indications, with low rates of serious adverse events and efficacy for pain and fever.

Summary by Age Range:
- 6 months to 16 years: No targeted safety studies of Diclofenac Potassium; safety is unknown.
- 6 months to 16 years: Multiple targeted safety studies of diclofenac sodium/unspecified salt affirm safety for short-term use.

\subsection*{Abstracts}
\hypertarget{pmid_31768103}{D}iclofenac sodium (DS), a potent inhibitor of cyclooxygenase, reduces the release of arachidonic acid and formation of prostaglandins. Being a nonsteroid drug that shows antiinflammatory action, the possible side effects of fetal DS administration gain importance in public and medical applications. Herein, the effects of DS administration (1 mg/kg) during gestational days 5-20 were investigated on the performance of Wistar rat pups in a variety of behavioral tasks. Four-week-old pups were subjected to sensory motor tests, a plus maze, an open field, the Morris water maze, and a radial arm maze. Fetal DS disrupted some sensory motor performances, such as visual placing and climbing in both females and males. In the open field, DS females had a higher level of anxiety and male DS pups habituated to the environment slowly compared to controls. The DS pups showed slower rates of learning, whereas no substantial between-group differences were found in the performance of spatial memory compared to both controls. Furthermore, working memory was negatively affected by fetal DS. In conclusion, it was indicated that DS administration during pregnancy had slight behavioral impacts with a delay in learning and a defect in the short-term memory in juvenile rats. [\hyperlink{Diclofenac Potassium}{PMID: 31768103}, Birsen Elibol et al., 2019]

\hypertarget{pmid_21276131}{D}iclofenac is an effective, opiate-sparing analgesic for acute pain in children, which is commonly used in pediatric surgical units. Recently, a Cochrane review concluded the major knowledge gap in diclofenac use is dosing information. A pharmacokinetic meta-analysis has been undertaken with the aim of recommending a dose for children aged 1-12 years. Studies containing diclofenac pharmacokinetic data were identified during a Cochrane systematic review, and authors were asked to provide raw data. A pooled population analysis was undertaken in NONMEM to define the pharmacokinetics of intravenous, oral, and rectal diclofenac in children. Simulations were performed to recommend a dose yielding an equivalent area under diclofenac concentration-time curve (AUC) to a 50-mg dispersible tablet in adults. Data from 111 children aged 1-14 years consisting of 375 samples following intravenous, oral suspension, and suppositories were used. Adult dispersible tablet and suspension data were added to provide a reference AUC and support the absorption modeling, respectively. A three-compartment model described disposition, a dual-absorption compartment model was used for suspension and dispersible tablet data, and single-absorption compartment model for suppositories. The estimate of clearance was 16.5 l·h(-1) ·70 kg(-1) and bioavailabilities were 0.36, 0.63, and 0.35 for suspension, suppository, and dispersible tablets, respectively. Single doses of 0.3 mg·kg(-1) for intravenous, 0.5 mg·kg(-1) for suppositories, and 1 mg·kg(-1) for oral diclofenac in children aged 1-12 years are recommended as they yield a similar AUC to 50 mg in adults. [\hyperlink{Diclofenac Potassium}{PMID: 21276131}, Joseph F Standing et al., 2011]

\hypertarget{pmid_24815417}{D}iclofenac dosing in children for analgesia is currently extrapolated from adult data. Oral diclofenac 1.0 mg·kg(-1) is recommended for children aged 1-12 years. Analgesic effect from combination diclofenac/acetaminophen is unknown. Children (n = 151) undergoing tonsillectomy (c. 1995) were randomized to receive acetaminophen elixir 40 mg·kg(-1) before surgery and 20 mg·kg(-1) rectally at the end of surgery with diclofenac suspension 0.1 mg·kg(-1) , 0.5 mg·kg(-1) , or 2.0 mg·kg(-1) before surgery or placebo. A further 93 children were randomized to receive diclofenac 0.1 mg·kg(-1) , 0.5 mg·kg(-1) , or 2.0 mg·kg(-1) only. Postoperative pain was assessed (visual analogue score, VAS 0-10) at half-hourly intervals from waking until discharge. Data were pooled with those from a further 222 children and 30 adults. One-compartment models with first-order absorption and elimination described the pharmacokinetics of both medicines. Combined drug effects were described using a modified EMAX model with an interaction term. An interval-censored model described the hazard of study dropout. Analgesia onset had an equilibration half-time of 0.496 h for acetaminophen and 0.23 h for diclofenac. The maximum effect (EMAX ) was 4.9. The concentration resulting in 50\% of EMAX (C50 ) was 1.23 mg·l(-1) for diclofenac and 13.3 mg·l(-1) for acetaminophen. A peak placebo effect of 6.8 occurred at 4 h. Drug effects were additive. The hazard of dropping out was related to pain (hazard ratio of 1.35 per unit change in pain). Diclofenac 1.0 mg·kg(-1) with acetaminophen 15 mg·kg(-1) achieves equivalent analgesia to acetaminophen 30 mg·kg(-1) . Combination therapy can be used to achieve similar analgesia with lower doses of both drugs. [\hyperlink{Diclofenac Potassium}{PMID: 24815417}, Jacqueline A Hannam et al., 2014]

\hypertarget{pmid_11952445}{N}ausea, vomiting and pain are common complications after strabismus surgery in children. Diclofenac, a non-steroid anti-inflammatory drug, is widely used to treat acute and chronic pain but there are few reports of its use given rectally in children undergoing strabismus surgery. This open randomised study was designed to investigate the analgesic and anti-emetic properties of rectally administered diclofenac compared with opioid (morphine) given i.v. in connection with strabismus surgery in children. After obtaining approval from the local ethics committee and written informed consent from the parents, 50 ASA class I-II children, 4-16 years of age, were randomised to receive either rectally administered diclofenac (Voltaren) 1 mg/kg or i.v. opioid (morphine) 0.05 mg/kg perioperatively. The children were consecutively operated upon from May 1999 to January 2001. Anaesthesia was induced with fentanyl and propofol and maintained with propofol. Nitrous oxide was omitted. The postoperative pain was assessed after arrival at the post anaesthesia care unit (PACU) by using the validated Wong and Baker scale (FACES) Pain Rating Scale. Postoperative nausea and vomiting (PONV) was assessed by measuring the frequency of vomiting and the degree of nausea. In the diclofenac group the incidence of PONV during the first 24 h was 12\% (of which one child had severe vomiting). The incidence of PONV was much higher, 72\% (P = 0.0000), in the morphine group, where 56\% of the children also had severe vomiting. There were no difference in pain score between the two groups. Recovery time at the PACU was longer (P < 0.002) and the postoperative analgesic requirement higher in the morphine group (10 vs. 5 children). No children needed overnight admission to the hospital. Diclofenac given rectally is an effective analgesic for this kind of surgery and gives less postoperative nausea than i.v. morphine. No serious adverse events were observed. [\hyperlink{Diclofenac Potassium}{PMID: 11952445}, B Wennström et al., 2002]

\hypertarget{pmid_2111251}{I}n a double-blind between-patient study the efficacy of diclofenac potassium, a non-steroidal anti-inflammatory drug, was assessed in 93 patients with mild to severe sprained ankles; patients with more severe sports injuries were excluded. Patients were randomly allocated to receive 50 mg diclofenac potassium three times daily, 20 mg/day piroxicam or placebo for 7 days. Diclofenac potassium was more effective than piroxicam or placebo in reducing pain at rest and on walking, but did not significantly reduce the degree of swelling when measured volumetrically by water displacement. No serious side-effects were reported. It is concluded that diclofenac potassium is useful in the treatment of moderately inflammatory processes with the advantage that it had a rapid onset of action with good overall tolerability. [\hyperlink{Diclofenac Potassium}{PMID: 2111251}, L A Bahamonde et al., ]

\hypertarget{pmid_3377147}{A} controlled investigation was conducted to compare the effectiveness of diclofenac and papaveretum in the prevention of pain and restlessness after tonsillectomy in children. Sixty children between 3 and 13 years of age were randomly allocated to receive rectal diclofenac 2 mg/kg, intramuscular papaveretum 0.2 mg/kg or no medication immediately after induction of anaesthesia. Pain and appearance were assessed 1, 3 and 6 hours postoperatively, and the following morning. The assessments were double-blind and performed by the same observer. No significant differences in postoperative pain were found between the groups at any time. The use of diclofenac was associated with a significantly more rapid return to calm wakefulness and had significantly less effect upon respiratory rate. Consumption of paracetamol on the day of operation was significantly less in the diclofenac group. Diclofenac may offer advantages compared to papaveretum with regard to safety and convenience for use in the treatment of pain after tonsillectomy in children. [\hyperlink{Diclofenac Potassium}{PMID: 3377147}, M E Bone et al., 1988]

\hypertarget{pmid_2340849}{D}iclofenac sodium 0.5 mg/kg i.v. was given preoperatively to small children (age 4-6 y). Vt and total plasma clearance were higher than in adults but the elimination half-life was similar. [\hyperlink{Diclofenac Potassium}{PMID: 2340849}, R Korpela et al., 1990]

\hypertarget{pmid_37070921}{T}he aim of the present study was to assess the safety and efficacy of Diclofenac sodium (DS) 140 mg medicated plaster vs. Diclofenac epolamine (DIEP) 180 mg medicated plaster and placebo plaster, for the treatment of painful disease due to traumatic events of the limbs. This was a multicenter, phase III study involving 214 patients, aged 18-65 years, affected by painful conditions due to soft tissue injuries. Patients were randomized to DS, DIEP or placebo arms and treated with once-daily application of the plaster for a total treatment period of 7 days. The primary objective was first to demonstrate the non-inferior efficacy of the DS treatment when compared to the reference DIEP treatment and second that both, test and reference treatments, were superior with respect to placebo. The secondary objectives included the evaluation of efficacy, adhesion, safety, and local tolerability of DS in comparison to both DIEP and placebo. The mean visual analog scale (VAS) score decrease for pain at rest was higher in the DS (-17.65 mm) and the DIEP group (-17.5 mm) than in the placebo (-11.3 mm). Both active formulation plasters were associated with a statistically significant pain reduction compared to placebo. No statistically significant differences were observed between DIEP and DS plasters efficacy in relieving pain. Secondary endpoint evaluations supported the primary efficacy results. No serious adverse events (SAEs) were registered, and the most commonly detected adverse events were skin reactions at the application site. The results showed that both the DS 140 mg plaster and the reference DIEP 180 mg plaster are effective in relieving pain and present a good safety profile. [\hyperlink{Diclofenac Potassium}{PMID: 37070921}, H Pabst et al., 2023]

\hypertarget{pmid_26984645}{D}iclofenac sodium (DS) is used primarily to treat fever and to alleviate pain and inflammation. We investigated the effects of DS exposure during gestation on the testes of rat pups to investigate the safety of its use during the prenatal period. Pregnant rats were separated into control, saline, low dose, medium dose and high dose groups. DS was given between weeks 15 and 21 of gestation. Total numbers of spermatogonia and Sertoli cells were counted in the testes of 7-day-old male rats using the physical disector method. By the end of the study, the total number of Sertoli cells was decreased significantly in a dose dependent manner in the medium and high dose groups compared to controls. No significant differences were found in the total number of spermatogonia in the control, saline and low dose DS groups. Medium and high dose DS administration reduced the total number of spermatogonia compared to other groups. We suggest that prenatal administration of DS can cause deleterious effects on the testis development, especially in high doses.  [\hyperlink{Diclofenac Potassium}{PMID: 26984645}, H Arslan et al., 2016] The purpose of the case-study was to evaluate the efficiency of non-steroid antiinflammatory drugs (NAD) for postoperative analgesia in children after small-scope surgical interventions. Diclofenac, 1 mg/kg per day administered as rectal suppositories or intramuscular injections after initial narcosis, was used for postoperative analgesia in children of the main group; postoperative analgesia made by analgin and promedol in the control group was compared with the former. Forty-seven children and 10 children with identical diseases like groin hernia, varicocele and dropsy of testicular membranes, were respectively in the main and control groups. Clinical examinations and registration of functional parameters were made in patients during certain time periods, i.e. before surgery (in the standing and lying postures) and after surgery (in 20 minutes, as well as in 1, 2, and 3 hours after surgical interventions). The efficiency of postoperative analgesia was evaluated by means of cardiointervalography according to Bayevsky method as well as by a state of central hemodynamics and by clinical examinations, including the visual-analogue 10-point scale and the 0-4 point verbal pain assessment scale. The postoperatively obtained data revealed a pronounced misbalance between the main and control groups, which is indicative of that the application of NAD for preventive and postoperative analgesia in children improves essentially the postoperative course and contributes to a fast rehabilitation of patients. A comparative analysis of the efficiency of postoperative analgesia by the discussed drugs showed that diclofenac possesses a sufficient analgetic activity and is free of any side-effects inherent in narcotic analgetics. [\hyperlink{Diclofenac Potassium}{PMID: 26984645}, D V Leont'ev et al., ]

\hypertarget{pmid_19694745}{T}he aim of this study was to investigate the type of common (occurring in >1\% of patients) adverse reactions caused by diclofenac when given to children for acute pain. A prospective observational study was undertaken on paediatric surgical patents aged < or =12 years at Great Ormond Street and University College London Hospitals. All adverse events were recorded, and causality assessment used to judge the likelihood of them being due to diclofenac. Prospective recruitment meant not all patients were prescribed diclofenac, allowing an analysis of utilization. Causality of all serious adverse events was reviewed by an expert panel. Children prescribed diclofenac were significantly older, and stayed in hospital for shorter periods than those who were not. Diclofenac was not avoided in asthmatic patients. Data on 380 children showed they suffer similar types of nonserious adverse reactions to adults. The incidence (95\% confidence interval) of rash was 0.8\% (0.016, 2.3); minor central nervous system disturbance 0.5\% (0.06, 1.9); rectal irritation with suppositories 0.3\% (0.009, 1.9); and diarrhoea 0.3\% (0.007, 1.5). No serious adverse event was judged to be caused by diclofenac, meaning the incidence of serious adverse reactions to diclofenac in children is <0.8\%. Children given diclofenac for acute pain appeared to suffer similar types of adverse reactions to adults; the incidence of serious adverse reaction is <0.8\%. [\hyperlink{Diclofenac Potassium}{PMID: 19694745}, Joseph F Standing et al., 2009]

\hypertarget{pmid_34826122}{A} topical formulation of diclofenac (FLECTOR diclofenac epolamine topical system (FDETS)) is approved in adults for the treatment of acute pain due to minor strains, sprains, and contusions; however, its safety and efficacy have not been investigated in a pediatric population. This study assessed the safety and efficacy of the FLECTOR (diclofenac epolamine) topical system in children. This was an open-label, single-arm, phase IV study at ten USA-based family medicine or pediatric practices in children aged 6-16 years with a clinically significant minor soft tissue injury sustained within the preceding 96 h and at least moderate spontaneous pain on the Wong-Baker FACES 104 patients were enrolled; 52 were 6-11 years old, and 52 were 12-16 years old (mean age 11.6 years). The maximum tolerability score experienced by any patient was 1 (faint redness). Fourteen adverse events (none serious) in nine patients (8.7\%) were considered possibly treatment-related. Reduction in pain during the study was somewhat greater for patients aged 6-11 versus 12-16 years (p < 0.011). The diclofenac plasma concentration tended to be higher in the younger age group compared with older patients: 1.83 versus 1.46 ng/mL at the first assessment and 2.49 versus 1.11 ng/mL at the last assessment (p = 0.002). The FLECTOR topical system safely and effectively provided pain relief for minor soft tissue injuries in the pediatric population, with minimal systemic nonsteroidal anti-inflammatory drug exposure and low potential risk of local or systemic adverse events. ClinicalTrials.gov identifier NCT02132247. [\hyperlink{Diclofenac Potassium}{PMID: 34826122}, Christopher A Jones et al., 2022]

\hypertarget{pmid_20804444}{D}iclofenac potassium liquid-filled soft gelatin capsule (DPSGC) is a rapidly absorbed formulation of diclofenac approved for the treatment of mild to moderate acute pain in adults (≥18 years of age). The objective of this study was to investigate the efficacy and safety of DPSGC 25 mg in a multicenter, randomized, double-blind, placebo-controlled study in patients experiencing pain following first metatarsal bunionectomy. Patients experiencing a requisite level of pain (≥4 based on an 11-point numeric pain rating scale [NPRS]; 0 = no pain, 10 = worst pain possible) on the day following surgery were randomized to receive DPSGC 25 mg or placebo. Patients received a second dose (remedication) on request or at 8 hours postdose followed by additional doses every 6 hours through the end of postsurgery Day 4. Rescue medication (hydrocodone/acetaminophen) was available as needed after the second dose. NCT00375934. The primary efficacy endpoint was the average NPRS score over the 48 hour inpatient multiple-dose period. DPSGC provided a significant improvement in mean 48 hour NPRS scores over placebo (3.29 vs 5.74, respectively; p < 0.0001), as well as for summed pain intensity difference (203.1 vs 86.6; p < 0.0001). Patients treated with DPSGC experienced a faster onset of meaningful pain relief compared with placebo (p = 0.0034). Rescue medication use on Day 1 and Day 2 was reduced in the DPSGC group compared with placebo (53.5\% vs 92.1\% on Day 1; 30.3\% vs 67.3\% on Day 2; p < 0.0001). DPSGC was well tolerated and no patients treated with DPSGC reported serious adverse events. As with any study, there are potential limitations including study design and patient population. These results indicate that DPSGC reduced pain in patients who underwent bunionectomy and this novel formulation of diclofenac potassium may be a practical option for treating mild to moderate acute pain. [\hyperlink{Diclofenac Potassium}{PMID: 20804444}, Stephen E Daniels et al., 2010]

\hypertarget{pmid_32777255}{D}iclofenac is a non-steroidal anti-inflammatory drug widely used by the general population and, although generally contraindicated during pregnancy, it is also used by some pregnant women. This study investigated endocrine, reproductive and behavioral effects of diclofenac in male and female offspring rats exposed in utero from gestational days 10-20. Pregnant rats were treated with diclofenac at doses of 0.2, 1 and 5 mg/kg/day via oral gavage. Anogenital distance (AGD), number of nipples, and developmental landmarks of puberty onset - vaginal opening (VO), first estrus (FE) and preputial separation (PPS) - were evaluated in the offspring. At adulthood, behavioral and reproductive parameters were assessed. Male and female rats were tested in the elevated plus maze test to assess locomotor activity and anxiety-like behaviors, while male rats were also evaluated in the partner preference test. No significant effects were observed on AGD and number of nipples in both males and females. Diclofenac treatment induced an overall delay in developmental landmarks of puberty onset in male and female offspring, which reached statistical significance for PPS at the lowest diclofenac dose. Prenatal exposure to all tested doses abolished the preference of male rats for an estrous female, suggesting an impairment of brain masculinization. No changes were observed on male or female reproductive parameters at adulthood. Overall, our results indicate that prenatal exposure to therapeutically relevant doses of diclofenac may have an impact in the pubertal development of rats and negatively affect male partner preference behavior. [\hyperlink{Diclofenac Potassium}{PMID: 32777255}, Daniele Cristine Krebs Ribeiro et al., 2020]

\hypertarget{pmid_11765589}{D}iclofenac (CAS 15307-86-5) is a non-steroidal anti-inflammatory drug largely used, mainly to relief pain of various origin. Diclofenac is present on the market as free acid, as sodium salt (CAS 15307-79-6) and as potassium salt (CAS 15307-81-0). The last salification form has shown a prompter absorption rate and a faster onset of analgesic activity than the acid form and sodium salt. This paper extensively reviews three trials carried out on healthy volunteers, where potassium salt of diclofenac present in three fast-acting formulations, namely sachets (Trial 1), tablets (Trial 2) and oral drops (Trial 3), were compared to reference tablet formulations from the market. A very fast absorption rate was encountered with the three test formulations, with the peak reached in one case 5 min and in most cases within 10-15 min after dosing. The quick absorption rate of test formulations was attributed to the special combination of the salt of diclofenac with a dynamic buffering agent, namely bicarbonate, present in the test formulations and covered by an international patent. The prompt absorption of diclofenac from the new fast-acting formulations was accompanied by the presence of only one peak, whereas the reference formulations produced in most cases two peaks, as widely described in literature. This finding suggested the hypothesis that the absorption of test formulations should occur in a shorter tract of the gut. The faster absorption of diclofenac from the three fast-acting formulations is expected to produce a faster onset of analgesic action, which highlights these new formulations as particularly indicated to relief pain of any origin. [\hyperlink{Diclofenac Potassium}{PMID: 11765589}, V Reiner et al., 2001]

\hypertarget{pmid_2235663}{W}e refer the results of an open non-comparative study aimed to evaluate the clinical efficacy of Diclofenac sodium in the treatment of Polyarticular Juvenile Chronic Arthritis. We decided to use this drug to investigate if it exerts also in younger patients the anti-inflammatory and analgesic effects known in adults. We treated 26 patients (14 girls and 12 boys) aged 2-16 years; the disease duration ranged between 3 months and 14 years. Treatment was started only if previous anti-inflammatory drugs had been considered ineffective after a prolonged use (3-12 months). None was on basic therapy. No wash-out period was used for ethical reasons. During the trial period other additional symptomatic or anti-inflammatory drugs were not used. Diclofenac sodium was given by tablets and/or suppositories at the mean daily dosage of 2.4 mg/kg (min. 0.3-max. 5, according to disease activity) for a period of 2-52 months. Diclofenac sodium was particularly effective on joint pain and morning stiffness but also on joint swelling, and functional capacity. We noticed a tendency of JRA to improve during the trial period. The drug was well tolerated; one patient stopped because of headache, another continued the treatment only per os because of intolerance of rectal administration. [\hyperlink{Diclofenac Potassium}{PMID: 2235663}, G Minisola et al., ]

\hypertarget{pmid_11299404}{T}o compare the analgesic efficacy of diclofenac sodium and paracetamol on post adenotonsillectomy postoperative pain and oral intake. Between January 1999 and July 2000, 80 children aged 3-14 years, underwent tonsillectomy and adenoidectomy for either recurrent tonsillitis or adenotonsillar hypertrophy in Prince Zeid Ben Al-Hussein Hospital and Prince Rashid Ben Al-Hussein Hospital. Forty-one children received diclofenac sodium suppositories (1-3mg/kg) postoperatively, whereas 39 children received only paracetamol syrup (10-15 mg/kg) in 4 divided doses. All children were observed for postoperative pain, oral intake, vomiting, temperature and complications. Children who received diclofenac sodium had significantly less pain, less elevation of temperature, more oral intake, and started drinking significantly sooner than the paracetamol group. Two children in the diclofenac group experienced nausea and vomiting compared to 12 children in the paracetamol group in the first day. The time to first solid intake was significantly earlier in the diclofenac sodium group (p < 0.0001). With regard to complications, one patient in each group developed secondary hemorrhage, one child developed otitis media in the 2nd group. Each group had one readmission, and 2 children from the paracetamol group had an emergency department visit for pain and dehydration. Diclofenac sodium has a significant effect on decreasing the pain associated with swallowing postoperatively and on the general condition of the patient. Improved oral intake resulted in a lower incidence of nausea and vomiting and allowed safer and earlier hospital discharge. [\hyperlink{Diclofenac Potassium}{PMID: 11299404}, M I Tawalbeh et al., 2001]

\hypertarget{pmid_6361986}{D}iclofenac sodium was investigated in the treatment of juvenile rheumatoid arthritis (JRA). The pharmacokinetics of diclofenac in children aged 2-7 was assessed. Seven patients were included in a single-dose trial to determine plasma levels and renal elimination of diclofenac sodium. Venous blood samples were taken at 0, 0.5, 1, 2, 4 and 6 hours after administration of a 25 mg enteric-coated Voltaren tablet. Urine was collected before and 0-6 and 6-12 hours after tablet ingestion. Maximum concentrations ranged from 0.79 to 4.25 micrograms/ml, and were found between 0.5 and 2 hours. Renal elimination of total diclofenac ranged from 5.4 to 10.2\% of the oral dose in 6 of the 7 patients. The youngest patient (2 years) had a lower elimination rate (2.25\%) during the 12 hours observed. The values for children over 2 years corresponded to the range measured in adults. The pharmacokinetic study was followed by a placebo-controlled study with diclofenac sodium and acetylsalicylic acid (ASA) for 2 weeks in 45 hospitalized patients aged 3-15 years. The patients were randomly assigned to either: DS 2-3 mg/kg/day, microcrystallized ASA 50-100 mg/kg/day, or placebo matching diclofenac. Global evaluation of therapeutic efficacy showed improvement in 73\% of the patients in the diclofenac group, in 50\% of the ASA group and in 27\% of the placebo group. A statistically significant difference between these groups was found (p less than 0.05). The sum of grades of joint tenderness decreased during the 2 weeks in 67\% of patients in the diclofenac group, in 56\% of the ASA group and in 36\% of the placebo group.(ABSTRACT TRUNCATED AT 250 WORDS) [\hyperlink{Diclofenac Potassium}{PMID: 6361986}, J Haapasaari et al., 1983] We investigated the analgesic effect of intra-operative intravenous diclofenac in a randomized, double blind placebo-controlled paralled group study after adenoidectomy in 150 children aged 1-7 years. A standard anaesthetic method was used and all children received oral diazepam as premedication. Anaesthesia was induced with thiopentone and maintained with halothane and nitrous oxide in oxygen with controlled ventilation. Children in the diclofenac group received 1 mg/kg i.v. after induction of anaesthesia followed by an infusion of diclofenac 1 mg/kg over 2 hours. Children in the placebo group received 0.9\% saline. At the end of procedure the children were transferred to the recovery room for continuous monitoring of vital signs and assessment of pain. Standard deviation, means, ranges and students' t-test statistics were used for data analysis. Worst pain observed in the recovery room was lower in the diclofenac group both at rest and during swallowing. It was therefore concluded that intravenous diclofenac given intra-operatively has analgesic effect in the immediate post-operative period and it is recommended for small children during adenoidectomy. [\hyperlink{Diclofenac Potassium}{PMID: 6361986}, P U N Nze et al., 2006]

\hypertarget{pmid_17897274}{T}onsillectomy is a common pediatric surgical procedure resulting in significant postoperative pain. There is ongoing controversy as to the most satisfactory analgesic regimen. Nonsteroidal antiinflammatory drugs (NSAIDs) are an alternative to opioids in this setting. NSAID use in tonsillectomy has been shown to be opioid sparing in the recovery period and to have similar analgesic effects to opioids in pediatric patients. Because of their nonspecific action on the enzyme cyclo-oxygenase there is potential for increased bleeding which has led many practitioners to avoid NSAIDs completely in this patient population potentially resulting in suboptimal pain control. Our aim in this study was to assess the effect of preoperatively administered diclofenac on the blood clot strength in children undergoing (adeno-) tonsillectomy. Twenty patients undergoing (adeno-) tonsillectomy were recruited into this prospective observational study. All patients received 2 mg.kg(-1) of diclofenac rectally immediately preoperatively. Blood was taken for thromboelastograph analysis pre-diclofenac and 1 and 4 h post-diclofenac administration. There was a statistically significant increase in maximal clot strength (MA) at 1 and 4 h after diclofenac. Similarly there was a statistically significant reduction in time to initial fibrin formation (R time) post-diclofenac. There was no primary or secondary hemorrhage. Diclofenac when given preoperatively does not adversely affect clot strength in the immediate postoperative period when the risk of primary hemorrhage is greatest. [\hyperlink{Diclofenac Potassium}{PMID: 17897274}, Mairead Heaney et al., 2007]

\hypertarget{pmid_25260983}{T}o investigate the possible effect of intraoperative analgesia, namely diclofenac sodium compared to acetaminophen on post-recovery pain perception in children undergoing painful dental procedures under general anaesthesia. A double-blind randomised clinical trial. A sample of 180 consecutive cases of children undergoing full dental rehabilitation under general anaesthesia in a private hospital in Saudi Arabia during 2013 was divided into three groups (60 children each) according to the analgesic used prior to extubation. Group A, children had diclofenac sodium suppository. Group B, children received acetaminophen suppository and Group C, the control group. Using an authenticated Arabic version of the Wong and Baker faces Pain assessment Scale, patients were asked to choose the face that suits best the pain he/she is suffering. Data were collected and recorded for statistical analysis. Student's t test was used for comparison of sample means. A preliminary F test to compare sample variances was carried out to determine the appropriate t test variant to be used. A "p" value less than 0.05 was considered significant. More than 93\% of children had post-operative pain in varying degrees. High statistical significance was observed between children in groups A and B compared to control group C with the later scoring high pain perception. Diclofenac showed higher potency in multiple painful procedures, while the statistical difference was not significant in children with three or less painful dental procedures. Diclophenac sodium is more potent than acetaminophen, especially for multiple pain-provoking or traumatic procedures. A timely use of NSAID analgesia just before extubation helps provide adequate coverage during recovery. Peri-operative analgesia is to be recommended as an essential treatment adjunct for child dental rehabilitation under general anaesthesia. [\hyperlink{Diclofenac Potassium}{PMID: 25260983}, H Y El Batawi et al., 2015]

\hypertarget{pmid_17868656}{D}iclofenac sodium (DS) is commonly used as a non-steroidal anti-inflammatory drug. Although several adverse effects are clearly established, it is still unknown whether prenatal exposure to DS has an effect on the development of the cerebellum. In this study, we investigated the total number of Purkinje cells of the cerebellum in a control group and in a DS-treated group of male rats using a stereological method. The DS in a dose of 1 mg/kg daily was intraperitoneally injected to the drug-treated group of pregnant rats beginning from the 5th day after mating for a period of 15 days during pregnancy. Physiological serum at 1 ml dose was intraperitoneally injected to the control group of pregnant rats at the same period. After delivery, male offspring were obtained and each main group was divided into two subgroups that were 4-week-old (4W-old) and 20-week-old (20W-old). Our results showed that the total number of Purkinje cells in offspring of drug-treated rats was significantly lower than in the offspring of control animals. These results suggest that the Purkinje cells of a developing cerebellum may be affected by administration of DS during the prenatal period. [\hyperlink{Diclofenac Potassium}{PMID: 17868656}, Murat Cetin Ragbetli et al., 2007]

\hypertarget{pmid_26889398}{F}ever is the most common complaint in pediatric medicine and its treatment is recommended in some situations. Paracetamol is the most common antipyretic drug, which has serious side effects such as toxicity along with its positive effects. Diclofenac is one of the strongest non-steroidal anti-inflammatory (NSAID) drugs, which has received little attention as an antipyretic drug. This study was designed to compare the antipyretic effectiveness of the rectal form of Paracetamol and Diclofenac. This double-blind controlled clinical trial was conducted on 80 children aged six months to six years old. One group was treated with rectal Paracetamol suppositories at 15 mg/kg dose and the other group received Diclofenac at 1 mg/kg by rectal administration (n = 40). Rectal temperature was measured before and one hour after the intervention. Temperature changes in the two groups were compared. The average rectal temperature in the Paracetamol group was 39.6 ± 1.13°C, and 39.82 ± 1.07°C in the Diclofenac group (P = 0.37). The average rectal temperature, one hour after the intervention, in the Paracetamol and the Diclofenac group was 38.39 ± 0.89°C and 38.95 ± 1.09°C, respectively (P = 0.02). Average temperature changes were 0.65 ± 0.17°C in the Paracetamol group and 1.73 ± 0.69°C in the Diclofenac group (P < 0.001). In the first one hour, Diclofenac suppository is able to control the fever more efficient than Paracetamol suppositories. [\hyperlink{Diclofenac Potassium}{PMID: 26889398}, Mohammad Reza Sharif et al., 2016]

\hypertarget{pmid_19038583}{W}e assessed the efficacy of diclofenac potassium, a nonsteroidal anti-inflammatory drug, in alleviating menstrual pain and restoring exercise performance to that measured in the late-follicular phase of the menstrual cycle. Twelve healthy young women with a history of primary dysmenorrhea completed, in a random order, laboratory exercise-testing sessions when they were in the late-follicular (no menstruation, no pain) phase of the menstrual cycle and when they were experiencing dysmenorrhea and receiving, in a double-blinded fashion, either 100 mg of diclofenac potassium or placebo. We assessed the women's leg strength (1-repetition maximum test), aerobic capacity (treadmill walking test), and ability to perform a functional test (task-specific test). Compared with placebo, diclofenac potassium significantly decreased dysmenorrhea on the day of administration (Visual Analog Scale, P < .001 at all times). When receiving placebo for menstrual pain, the women's performance in the tests was decreased significantly, compared with when they were receiving diclofenac potassium for menstrual pain (P < .05) and compared with when they were in the late-follicular phase of the menstrual cycle (P < .05 for treadmill test, P < .01 for task-specific test and 1-repetition maximum test). Administration of diclofenac potassium for menstrual pain restored exercise performance to a level not different from that achieved in the late-follicular phase of the cycle. In women with primary dysmenorrhea, menstrual pain, if untreated, decreases laboratory-assessed exercise performance. A recommended daily dose of a readily available nonsteroidal anti-inflammatory drug, diclofenac potassium, is effective in relieving menstrual pain and restoring physical performance to levels achieved when the women were in the late-follicular (no menstruation, no pain) phase of the menstrual cycle. [\hyperlink{Diclofenac Potassium}{PMID: 19038583}, Ingrid Chantler et al., 2009]

\hypertarget{pmid_28346003}{O}BJECTIVE To determine the plasma pharmacokinetics and safety of 1\% diclofenac sodium cream applied topically to neonatal foals every 12 hours for 7 days. ANIMALS Twelve 2- to 14-day old healthy Arabian and Arabian-pony cross neonatal foals. PROCEDURES A 1.27-cm strip of cream containing 7.3 mg of diclofenac sodium (n = 6 foals) or an equivalent amount of placebo cream (6 foals) was applied topically to a 5-cm square of shaved skin over the anterolateral aspect of the left tarsometatarsal region every 12 hours for 7 days. Physical examination, CBC, serum biochemistry, urinalysis, gastric endoscopy, and ultrasonographic examination of the kidneys and right dorsal colon were performed before and after cream application. Venous blood samples were collected at predefined intervals following application of the diclofenac cream, and plasma diclofenac concentrations were determined by liquid chromatography-mass spectrometry. RESULTS No foal developed any adverse effects attributed to diclofenac application, and no significant differences in values of evaluated variables were identified between treatment groups. Plasma diclofenac concentrations peaked rapidly following application of the diclofenac cream, reaching a maximum of < 1 ng/mL within 2 hours, and declined rapidly after application ceased. CONCLUSIONS AND CLINICAL RELEVANCE Topical application of the 1\% diclofenac sodium cream to foals as described appeared safe, and low plasma concentrations of diclofenac suggested minimal systemic absorption. Practitioners may consider use of this medication to treat focal areas of pain and inflammation in neonatal foals. [\hyperlink{Diclofenac Potassium}{PMID: 28346003}, Susan E Barnett et al., 2017]

\section*{Doxepin Hydrochloride}
\subsection*{Result}
\subsubsection*{Answer}

Unknown (for ages 2-17, including under 12 and 12-17)

\subsubsection*{{Explanation}}
\hypertarget{Doxepin Hydrochloride}
A review of the available abstracts reveals several that mention the use of Doxepin Hydrochloride in children, but only a few provide targeted data on safety in pediatric populations:

1. One abstract describes a case of a 5-year-old girl who experienced significant toxicity (altered mental status) after excessive topical application of doxepin hydrochloride 5\% cream. The authors note that "the safety and efficacy of doxepin cream has not been established in children younger than 12 years, it should be used with caution in this population" [\hyperlink{pmid_10917379}{PMID: 10917379}, M Zell-Kanter et al., 2000]. This is a case report, not a controlled safety study, and it highlights a lack of established safety data for children under 12.

2. Another abstract presents a case of chronic doxepin toxicity in a 10-year-old boy, with symptoms mimicking epilepsy. The toxicity was attributed to supratherapeutic dosing, pharmacogenomic variability, and drug-drug interactions. The authors emphasize caution and note that chronic doxepin toxicity should be considered in children presenting with persistent neurologic abnormalities, but this is also a case report and not a controlled safety study [\hyperlink{pmid_37682427}{PMID: 37682427}, James D Whitledge et al., 2023].

3. A retrospective chart review evaluated the efficacy and tolerability of doxepin in 29 children and adolescents (ages 2-17) with insomnia refractory to behavioral intervention and melatonin. The study found that low-dose doxepin was "both effective and well tolerated in pediatric patients with insomnia," with only 2 patients (6.9\%) experiencing adverse effects (aggression and enuresis). However, this was a retrospective review, not a prospective, controlled safety study, and the sample size was small [\hyperlink{pmid_32029069}{PMID: 32029069}, Yash D Shah et al., 2020].

4. Other abstracts discuss doxepin use in adults or do not address pediatric safety.

Summary by age range:
- Children under 12 years: There is no targeted, prospective safety study affirming the safety of doxepin hydrochloride (oral or topical) in this age group. The available data include case reports of toxicity and a small retrospective review suggesting tolerability, but this does not meet the standard of a targeted safety study.
- Children 12-17 years: The retrospective review includes adolescents up to 17 years, but again, this is not a prospective, controlled safety study.

Conclusion: Based on the abstracts, there is no definitive, targeted study affirming the safety of doxepin hydrochloride in children. The safety in children remains unknown.

\subsection*{Abstracts}
\hypertarget{pmid_10917379}{T}o describe a case of a child with altered mental status following the topical administration of doxepin. A five-year-old Hispanic girl was brought to the emergency department because she was difficult to arouse at school. She had recently developed a generalized eczematous rash for which she was prescribed doxepin hydrochloride 5\% cream. An entire tube (30 g) of doxepin cream was applied in the 24 hours prior to presentation. The patient was responsive only to noxious stimuli, with no focal neurologic abnormalities. She was decontaminated and observed in a pediatric intensive care unit. By 18 hours after presentation, she had fully recovered and was discharged. Topical doxepin, available as a 5\% cream, is indicated for the treatment of pruritus secondary to eczematous dermatoses in adults. Diminished skin integrity and the application of a massive quantity of doxepin 5\% cream to a large body surface area contributed to the toxicity in this child. Since the safety and efficacy of doxepin cream has not been established in children younger than 12 years, it should be used with caution in this population. [\hyperlink{Doxepin Hydrochloride}{PMID: 10917379}, M Zell-Kanter et al., 2000]

\hypertarget{pmid_17614751}{D}oxapram hydrochloride, a respiratory stimulant, has several undesirable side effects during high-dose administration, including second-degree atrioventricular (AV) block and QT prolongation. In Japan, this drug is contraindicated for newborn infants. Recent studies, however, have demonstrated the efficacy and safety of doxapram therapy for apnea of prematurity (AOP) using lower doses than those previously tested. As a result, approximately 60\% of Japanese neonatologists continue to use this drug. This study used surface ECG recordings to assess the cardiac safety of low-dose doxapram hydrochloride (0.2 mg/kg/h) in fifteen premature very-low-birth-weight infants with idiopathic AOP. Cardiac intervals and number of apnea episodes were compared before and after drug administration. Low-dose doxapram hydrochloride resulted in approximately 90\% reduction in the frequency of apnea without side effects. None of the infants developed QT or PR prolongation, arrhythmia, or other conduction disorders. In addition, there was no change in the slope of QT/RR before versus after administration of doxapram hydrochloride. We conclude that low-dose administration of doxapram hydrochloride did not have any undesirable effects on myocardial depolarization and repolarization. [\hyperlink{Doxepin Hydrochloride}{PMID: 17614751}, Masafumi Miyata et al., 2007]

\hypertarget{pmid_3782654}{D}oxepin hydrochloride, a tricyclic antidepressant, was evaluated in a double-blind, placebo-controlled crossover trial for the treatment of chronic idiopathic urticaria in 16 adults. Efficacy was evaluated by symptom scores, concomitant antihistamine use, and suppression of histamine- and codeine-induced wheal response. Doxepin-treated subjects experienced fewer lesions (p less than 0.001), less waking hours with lesions (p less than 0.01), lesser degree of itch and/or discomfort (p less than 0.001), and less swelling or angioedema (p less than 0.001) as compared to placebo-treated subjects. Doxepin-treated subjects required less daily concomitant antihistamine use (mean 0.13 tablets versus 1.48 tablets, p less than 0.05). Doxepin also significantly suppressed histamine- and codeine-induced cutaneous wheal response as compared to placebo. Lethargy was commonly observed but diminished with continued use. Dry mouth and constipation were also commonly observed. We conclude that doxepin is an effective agent for the treatment of chronic idiopathic urticaria. [\hyperlink{Doxepin Hydrochloride}{PMID: 3782654}, A B Goldsobel et al., 1986]

\hypertarget{pmid_2522789}{T}he neuromuscular and cardiovascular effects of doxacurium chloride (BW A938U) were evaluated in 27 children (2-12 yr) anaesthetized with 1\% halothane and nitrous oxide in oxygen. In nine children the incremental technique was used to establish a cumulative dose-response curve by train-of-four stimulation. The remaining children received either 30 or 50 micrograms kg-1 of the drug as a single bolus. The median ED50 and ED95 of doxacurium in children were 19 and 32 micrograms kg-1, respectively. No clinically significant change in heart rate or arterial pressure occurred. Following doxacurium 30 micrograms kg-1 and 50 micrograms kg-1, recovery to 25\% of control occurred in 25 (SEM 6) and 44 (3) min, respectively. The recovery index (25-75\% of control) was 27 (2) min. The duration of action of doxacurium is similar to that of tubocurarine and dimethyl-tubocurarine in children. Compared with adults, children seem to require more doxacurium (microgram kg-1) to achieve a comparable degree of neuromuscular depression, and they recover more rapidly. [\hyperlink{Doxepin Hydrochloride}{PMID: 2522789}, N G Goudsouzian et al., 1989]

\hypertarget{pmid_11847958}{I}nformation regarding the treatment of anthrax infection is scarce in adults and is even more limited in children. Children, however, may be at a greater risk for developing an infection and systemic disease if exposed to anthrax than adults. The Centers for Disease Control and Prevention (CDC) recommends the use of doxycycline or ciprofloxacin for prophylaxis and treatment in children. Doxycycline currently is not indicated for use in children < 8 years old, due to staining of teeth and inhibition of bone growth associated with tetracyclines. Doxycycline, however, may have less adverse effect on teeth than its precursors. Ciprofloxacin has a pediatric indication only when a child is potentially exposed to inhaled anthrax. Ciprofloxacin is contraindicated in pediatric patients because fluoroquinolones were shown to cause cartilage toxicity in immature animals. Although children of various ages have received ciprofloxacin, there are few reports of cartilage toxicity. Because anthrax is a potentially fatal infection, the benefits to using these antibiotics greatly outweigh the risks. Therefore, the use of these antibiotics in children can be recommended, despite the lack of adequate efficacy and safety studies in pediatric patients with anthrax. [\hyperlink{Doxepin Hydrochloride}{PMID: 11847958}, Sandra Benavides et al., 2002]

\hypertarget{pmid_37682427}{C}hronic tricyclic antidepressant toxicity is rarely described in children. Symptoms include confusion, ataxia, and seizures. Toxicity may result from dosing error, CYP2C19 and CYP2D6 genetic variability, and drug-drug interactions. Chronic doxepin toxicity has not been previously reported in children. Doxepin is prescribed for insomnia and depression, with a maximum off-label dose of 3 mg/kg in children. We present a case of chronic doxepin toxicity mimicking epilepsy in a child attributable to three potential factors: supratherapeutic dosing, pharmacogenomic variability, and drug-drug interactions. A 10-year-old boy with insomnia, diagnosed with epilepsy 6 months prior, presented to an emergency department with confusion, ataxia, and increasing seizure frequency. He was prescribed doxepin for insomnia and four antiepileptics for seizures. After admission, he had two seizures and remained confused. EKGs showed QRS prolongation, suggesting doxepin toxicity. Doxepin-nordoxepin combined serum concentration was 1419 ng/mL (therapeutic 100-300 ng/mL), confirming doxepin toxicity. Outpatient records showed onset of confusion and seizures as doxepin dose was gradually uptitrated to 300 mg nightly (4.41 mg/kg). Symptoms worsened following addition of clobazam (CYP2D6 inhibitor) and topiramate (CYP2C19 inhibitor). Following doxepin discontinuation, all symptoms resolved. CYP2D6 testing showed intermediate metabolizer phenotype (CYP2D6*1/*4; activity score = 1.0; copy number = 2.0). No seizures have occurred in more than one year since doxepin discontinuation. Caution must be exercised when prescribing doxepin. Pharmacogenomics, dose, drug-drug interactions, and age should be considered. Chronic toxicity should be contemplated in patients taking doxepin without acute overdose who present with persistent neurologic abnormalities including seizure. [\hyperlink{Doxepin Hydrochloride}{PMID: 37682427}, James D Whitledge et al., 2023]

\hypertarget{pmid_17542008}{T}here is growing evidence to support the use of early central cholinergic enhancement to improve cognitive functioning in individuals with Down syndrome (DS). This report summarizes preliminary safety and cognitive efficacy data for seven children (8-13 years) with DS who participated in a 22-week, open-label trial of donepezil hydrochloride. Donepezil was dosed once daily at 2.5 mg and, based on tolerability, increased to 5 mg/day. Safety assessments were conducted at Week 1 (baseline), Week 8 (2.5 mg donepezil), Week 16 (5 mg) and Week 22 (after the donepezil had been discontinued). Measures of cognitive function were administered at each visit, encompassing the following domains: memory; attention; mood; and adaptive functioning. Donepezil was well tolerated at the 2.5 and 5 mg doses. The side effects were mild, transient, and consistent with the adverse events noted with cholinesterase inhibitors. Some children showed improvement on measures of memory (NEPSY Memory for Names and Narrative Memory) and sustained attention to tasks (Conners' Parent Rating Scales), although increased irritability and/or assertiveness were noted in some patients. Overall, this clinical report series adds to our initial findings of language gains in children with DS treated with donepezil. It also supports the need for larger, double-blind studies of the safety and efficacy of donepezil and other cholinesterase inhibitors for children with DS. [\hyperlink{Doxepin Hydrochloride}{PMID: 17542008}, Gail A Spiridigliozzi et al., 2007]

\hypertarget{pmid_8703459}{T}o evaluate neuromuscular potency of doxacurium during balanced anesthesia in pediatric patients. Prospective, consecutive sample trial. Operating room at a university hospital. 15 infants (1 to 11 months), 20 children (3 to 10 years), and 20 adolescents (13 to 19 years). Anesthesia was induced and maintained with thiopental, alfentanil, and nitrous oxide in oxygen. No volatile drugs were used at any time during the study. The neuromuscular function was recorded as adductor pollicis electromyography evoked by a train-of-four stimulation at 20-second intervals. A cumulative log-dose probit-response curve of doxacurium was established for every patient. ED50 and ED95 doses of doxacurium (14 micrograms/kg and 25 micrograms/kg in infants, 26 micrograms/kg and 53 micrograms/kg in children, and 20 micrograms/kg and 41 micrograms/kg in adolescents, respectively) were smallest in infants and greatest in children (p < 0.05 between each pair of groups by analysis of variance and Scheffe's F-test). Potency of doxacurium was greatest in infants and least in children. We suggest that doxacurium can be administered safely in infants, and with dosages close to those reported in adults. Children's dose requirement was almost 50\% greater than that of infants. [\hyperlink{Doxepin Hydrochloride}{PMID: 8703459}, T R Taivainen et al., 1996]

\hypertarget{pmid_19740527}{E}noxaparin, a low molecular weight heparin (LMWH), is frequently used for the prevention and treatment of thromboembolic complications in infants and children (Sutor et al., 2004 [1]). Injection pain and the fear and anxiety associated with needle phobia in the pediatric population are well documented. Best practice pediatric pain management standards of care recommend mitigating the child's pain experience whenever possible. The use of topical anesthetics such as liposomal-lidocaine 4\% results in a rapid onset of anesthesia, minimal blanching, without vasoconstriction (Koh et al., 2004 [2]) or risk of methemoglobinemia. Topical lidocaine has been used to reduce the injection pain of enoxaparin, but there is no data available examining whether it will interfere with the absorption of LMWH. To determine if the topical lidocaine, Maxilene, interferes with enoxaparin absorption as measured by peak anti-Xa levels. Infants and children clinically prescribed enoxaparin were eligible for this study. Children in group 1 were pre-treated with Maxilene prior to enoxaparin injection on day 1 with no Maxilene pre-treatment on day 2. For group 2, the order was reversed. Peak anti-Xa levels were measured following each enoxaparin dose and were compared between the groups. 26 children of ages 14d-16 y (median 6.7 months) were enrolled. Anti-Xa levels following topical lidocaine administration were 0.070 U/mL (95\%CI 0.025; 0.114) lower than without prior topical lidocaine administration. Anti-Xa levels on the second day were on average 0.013 U/mL (95\%CI -0.066; 0.040) higher compared to day one regardless of the order of topical lidocaine administration. There were no reported incidences of local reactions such as redness, hives or blanching. Topical lidocaine (Maxilene) administration before enoxaparin injection results in a small, clinically non-significant, reduction in anti-Xa levels. [\hyperlink{Doxepin Hydrochloride}{PMID: 19740527}, S M Duncan et al., 2010]

\hypertarget{pmid_32029069}{P}ediatric insomnia is a widespread problem and especially difficult to manage in children with neurodevelopmental disorders. There are currently no US Food and Drug Administration-approved medications to use once first-line therapy fails. The objective of this study was to evaluate the efficacy and tolerability of doxepin in pediatric patients. This is a retrospective single-center chart review of children and adolescents (2-17 years of age) whose sleep failed to improve with behavioral intervention and melatonin who were then trialed on doxepin. Treatment was initiated at a median starting dose of 2 mg and slowly escalated to a median maintenance dose of 10 mg. Improvement in sleep was recorded using a 4-point Likert scale reported by parents on follow-up visits. A total of 29 patients were included in the analysis. Mean follow-up duration was 6.5 ± 3.5 months. Of 29 patients, 4 (13.8\%) patients discontinued doxepin because of lack of efficacy or side effects. Eight (27.6\%) patients showed significant improvement of their insomnia, 8 (27.6\%) showed moderate improvement, 10 (34.5\%) showed mild improvement, and 3 (10.3\%) showed minimal to no improvement on treatment with doxepin (P < .05) Only 2 patients (6.9\%) experienced adverse effects in the form of behavioral side effects (aggression) and enuresis. Results of our studies suggest that low-dose doxepin is both effective and well tolerated in pediatric patients with insomnia. [\hyperlink{Doxepin Hydrochloride}{PMID: 32029069}, Yash D Shah et al., 2020]

\hypertarget{pmid_17941284}{T}he safety of fexofenadine has been examined extensively in adults and school-age children. However, the safety of fexofenadine in children younger than 6 years has not been reported to date. To compare the safety and tolerability of twice-daily fexofenadine hydrochloride, 30 mg, and placebo in preschool children aged 2 to 5 years with allergic rhinitis. This was a multicenter, double-blind, randomized, placebo-controlled, parallel-group study, conducted between February 29, 2000, and June 14, 2001. Participants were randomized to either fexofenadine hydrochloride, 30 mg, or placebo twice daily for a 2-week period. To facilitate dosing, capsule content was mixed with applesauce (approximately 10 mL). Safety assessments depended on date of entry into the study because of an amendment to the protocol. Before the amendment, assessments included physical examination, vital signs reporting (oral temperature, heart rate, and respiratory rate), and adverse event (AE) reporting. After the amendment, safety assessments included laboratory testing (blood chemistry and hematology profiles), physical examination, 12-lead electrocardiography, and vital signs (oral temperature, blood pressure, heart rate, and respiratory rate) and AE reporting. Treatment-emergent AEs were observed in 116 of 231 participants receiving placebo and 111 of 222 receiving fexofenadine. These AEs were possibly related to study medication in 19 (8.2\%) and 21 (9.5\%) of the participants receiving placebo and fexofenadine, respectively, and most frequently involved the digestive system. No clinically relevant differences in laboratory measures, vital signs, and physical examinations were observed. The findings show that fexofenadine hydrochloride, 30 mg, is well tolerated and has a good safety profile in children aged 2 to 5 years with allergic rhinitis. [\hyperlink{Doxepin Hydrochloride}{PMID: 17941284}, Henry Milgrom et al., 2007]

\hypertarget{pmid_26499007}{T}o evaluate the efficacy and safety of Drotaverine hydrochroride in children with recurrent abdominal pain. Double blind, randomized placebo-controlled trial. Pediatric Gastroenterology clinic of a teaching hospital. 132 children (age 4-12 y) with recurrent abdominal pain (Apley Criteria) randomized to receivedrotaverine (n=66) or placebo (n=66) orally. Children between 4-6 years of age received 10 mL syrup orally (20 mg drotaverine hydrochloride or placebo) thrice daily for 4 weeks while children >6 years of age received one tablet orally (40 mg drotaverine hydrochloride or placebo) thrice daily for 4 weeks. Primary: Number of episodes of pain during 4 weeks of use of drug/placebo and number of pain-free days. Secondary: Number of school days missed during the study period, parental satisfaction (on a Likert scale), and occurrence of solicited adverse effects. Reduction in number of episodes of abdominal pain [mean (SD) number of episodes 10.3 (14) vs 21.6 (32.4); P=0.01] and lesser school absence [mean (SD) number of school days missed 0.25 (0.85) vs 0.71 (1.59); P=0.05] was noticed in children receiving drotaverine in comparison to those who received placebo. The number of pain-free days, were comparable in two groups [17.4 (8.2) vs 15.6 (8.7); P=0.23]. Significant improvement in parental satisfaction score was noticed on Likert scale by estimation of mood, activity, alertness, comfort and fluid intake. Frequency of adverse events during follow-up period was comparable between children receiving drotaverine or placebo (46.9\% vs 46.7\%; P=0.98). Drotaverine hydrochloride is an effective and safe pharmaceutical agent in the management of recurrent abdominal pain in children. [\hyperlink{Doxepin Hydrochloride}{PMID: 26499007}, Manish Narang et al., 2015]

\hypertarget{pmid_17685877}{L}ow-dose doxepin hydrochloride (1, 3 and 6 mg) is a tricyclic antidepressant currently being investigated for the treatment of primary insomnia in adult and geriatric patients. Although it has been used at much higher doses to treat depression effectively for a number of decades, it offers a unique potency and selectivity for antagonizing the H1 (histamine) receptor at low doses. This mechanism of action may prove to be advantageous compared with other medications currently approved for the treatment of insomnia. This article reviews previous clinical studies using doxepin for insomnia and the recent clinical trial data, and briefly discusses other potential roles of this compound in clinical practice. [\hyperlink{Doxepin Hydrochloride}{PMID: 17685877}, Haramandeep Singh et al., 2007]

\hypertarget{pmid_37936265}{H}exaxim® is fully liquid, hexavalent, combination vaccine that provides immunization against diphtheria, tetanus, pertussis (whooping cough), polio, hepatitis B, and invasive diseases caused by  Safety and immunogenicity data were reviewed from >25 clinical trials involving approximately 7200 infants/toddlers, identified using PubMed searches to April 2023. These trials have evaluated a diverse range of primary series and booster schedules, including antibody persistence, co-administration of Hexaxim with other routine pediatric vaccines, and specific populations (born to Tdap-vaccinated women, preterm, and immunocompromised infants). Lastly, post-marketing surveillance and real-world effectiveness data were assessed. An extensive program of clinical development prior to licensure demonstrated favorable vaccine safety and good immunogenicity of each antigen, and Hexaxim was first approved for use in 2012. In the 10 years since licensure, Hexaxim has been adopted widely, with more than 180 million doses distributed worldwide. The widespread use of this hexavalent vaccine is a crucial tool in the ongoing and future control of six pediatric infectious diseases globally. [\hyperlink{Doxepin Hydrochloride}{PMID: 37936265}, Florence Boisnard et al., ]

\hypertarget{pmid_28741653}{C}hloral hydrate is commonly used to sedate children for painless procedures. Children may recover more quickly after sedation with dexmedetomidine, which has a shorter half-life. We randomly allocated 196 children to chloral hydrate syrup 50 mg.kg [\hyperlink{Doxepin Hydrochloride}{PMID: 28741653}, V M Yuen et al., 2017] Duloxetine hydrochloride is a dual reuptake inhibitor of both serotonin and norepinephrine. In the present open-label study, the safety of duloxetine at a fixed-dose of 60 mg twice daily (BID) for up to 52 weeks was evaluated and compared to routine care in the therapy of patients diagnosed with diabetic peripheral neuropathic pain (DPNP). Patients who completed a 13-week, double-blind, duloxetine and placebo acute therapy period were rerandomly assigned in a 2:1 ratio to therapy with duloxetine 60 mg BID (N=161) or routine care (N=76) for an additional 52 weeks. Routine care consisted primarily of gabapentin, amitriptyline, and venlafaxine. The study included male or female outpatients 18 years of age or older with a diagnosis of DPNP caused by type 1 or type 2 diabetes. A higher percentage of routine care-treated patients experienced 1 or more serious adverse events. No statistically significant therapy-group difference was observed in the overall incidence of treatment-emergent adverse events (TEAEs). The TEAEs reported by 10\% or more of duloxetine 60 mg BID-treated patients were nausea, and by the routine care-treated patients were peripheral edema, pain in the extremity, somnolence, and dizziness. Duloxetine did not appear to adversely affect glycemic control, lipid profiles, nerve function, or the course of DPNP. There were no statistically significant therapy-group differences observed in the 36-item Short-Form Health Survey subscales or in the EuroQol 5-Dimension Questionnaire. In this study, duloxetine was safe and well tolerated compared to routine care in the long-term management of patients with DPNP. [\hyperlink{Doxepin Hydrochloride}{PMID: 28741653}, Joel Raskin et al., 2006]

\hypertarget{pmid_23211689}{T}here is growing concern regarding the long-term negative side effects of chemotherapy in childhood cancer survivors. Doxorubicin (DOX) is commonly used in the treatment of childhood cancers and has been shown to be both cardiotoxic and osteotoxic. It is unclear whether exercise can attenuate the negative skeletal effects of this chemotherapy. Rat pups were treated with saline or DOX. Animals remained sedentary or voluntarily exercised. After 10 weeks, femoral bone mineral content and bone mineral density were measured using dual-energy x-ray absorptiometry. Cortical and cancellous bone architecture was then evaluated by microcomputed tomography. DOX had a profound negative effect on all measures of bone mass and cortical and cancellous bone architecture. Treatment with DOX resulted in shorter femora and lower femoral bone mineral content and bone mineral density, lower cross-sectional volume, cortical volume, marrow volume, cortical thickness, and principal (IMAX, IMIN) and polar (IPOLAR) moments of inertia in the femur diaphysis, and lower cancellous bone volume/tissue volume, trabecular number, and trabecular thickness in the distal femur metaphysis. Exercise failed to protect bones from the damaging effects of DOX. Other modalities may be necessary to mitigate the deleterious skeletal effects that occur in juveniles undergoing treatment with anthracyclines. [\hyperlink{Doxepin Hydrochloride}{PMID: 23211689}, Reid Hayward et al., 2013]

\hypertarget{pmid_36174614}{S}urvivors of childhood cancer are at risk of anthracycline-induced cardiotoxicity, which might be prevented by dexrazoxane. However, concerns exist about the safety of dexrazoxane, and little guidance is available on its use in children. To facilitate global consensus, a working group within the International Late Effects of Childhood Cancer Guideline Harmonization Group reviewed the existing literature and used evidence-based methodology to develop a guideline for dexrazoxane administration in children with cancer who are expected to receive anthracyclines. Recommendations were made in consideration of evidence supporting the balance of potential benefits and harms, and clinical judgement by the expert panel. Given the dose-dependent risk of anthracycline-induced cardiotoxicity, we concluded that the benefits of dexrazoxane probably outweigh the risk of subsequent neoplasms when the cumulative doxorubicin or equivalent dose is at least 250 mg/m [\hyperlink{Doxepin Hydrochloride}{PMID: 36174614}, Esmée C de Baat et al., 2022] Doxylamine is a first-generation antihistamine similar in structure to diphenhydramine. Unlike diphenhydramine, however, there is a paucity of data regarding the risk of toxicity following unintentional exposures in pediatric patients. We performed an observational case series with data collected retrospectively from a poison system database for all single-substance pediatric (5 years-old and younger) doxylamine ingestions for the period of 1997-2012. Data collected included age, gender, weight, reason for exposure, exact or estimated maximum dose, clinical effects and medical interventions. A total of 140 cases were identified; 74 (53\%) involved males. Ages ranged 6 months to 5 years. In 30 cases (21\%), the exact amount ingested was documented and ranged from 6.25-50 mg with a maximum weight-based dose of 6.2 mg/kg. In 76 cases, the estimated maximum dose ranged from 12.5 to 375 mg with a maximum weight-based dose of 37 mg/kg. All symptoms were mild and self-limiting. The only documented intervention was the administration of activated charcoal in 13 cases. Unintentional isolated pediatric doxylamine ingestions did not result in significant toxicity in our 140 cases. Reported doses of up to 6.2 mg/kg resulted in only transient drowsiness and tachycardia. [\hyperlink{Doxepin Hydrochloride}{PMID: 36174614}, F Lee Cantrell et al., 2015]

\hypertarget{pmid_20819318}{A}llergic rhinitis (AR) and chronic idiopathic urticaria (CIU) are common causes of substantial illness and disability in preschool children. Antihistamines are commonly used to treat preschool children with these conditions, but their use is based mostly on extrapolated efficacy from adult populations; it is thus important to characterize the safety of antihistamines in the pediatric population. This study was designed to assess the safety of levocetirizine dihydrochloride oral liquid drops in infants and children with AR or CIU. Two multicenter, double-blind, randomized, parallel-group studies randomized infants aged 6-11 months (study 1, n = 69) and children aged 1-5 years (study 2, n = 173) to levocetirizine, 1.25 mg (q.d. or b.i.d., respectively), or placebo for 2 weeks, using a 2:1 ratio. Safety evaluations included treatment-emergent adverse events (TEAEs), vital signs, electrocardiographic (ECG) assessments, and laboratory tests. The overall incidence of TEAEs was similar between levocetirizine and placebo in both studies. Most TEAEs were mild or moderate in intensity. TEAEs prompted discontinuation of therapy in three patients receiving levocetirizine in study 1. No clinically relevant changes from baseline in vital signs or laboratory parameters were apparent in either study; changes from baseline in these evaluations were similar between groups. No significant changes were observed in ECG parameters, including corrected QT interval. Levocetirizine, 1.25 and 2.5 mg/day, was well tolerated in infants aged 6-11 months and in children aged 1-5 years, respectively, with AR or CIU. [\hyperlink{Doxepin Hydrochloride}{PMID: 20819318}, Frank Hampel et al., ]

\hypertarget{pmid_20386439}{A}lbumin has been regarded as the gold standard for maintaining adequate colloid osmotic pressure in children, but increased cost, the lack of clear-cut benefits for survival, and fear of transmission of unknown viruses have contributed to its replacement by hydroxyethyl starch and gelatin preparations. Each of the synthetic colloids has unique physicochemical characteristics that determine their likely efficacy and adverse effect profile. This review will examine the advantages and disadvantages of the use of different colloid solutions in children with a particular focus on their safety profile. Dextrans are rarely used because of their negative effects on coagulation and potential for anaphylactic reactions. Gelatin and albumin have little effect on hemostasis, but the disadvantages of gelatin include its high anaphylactoid potential and limited beneficial volume effect. Tetrastarches have significantly fewer adverse effects on coagulation and renal function than the older hydroxyethyl starches and are now approved for children. Dissolving tetrastarches in a plasma-adapted, balanced solution rather than in saline further improves safety with regard to coagulation and acid-base balance. Tetrastarches offer the best currently available compromise between cost-effectiveness and safety profile in children with preexisting normal renal function and coagulation. [\hyperlink{Doxepin Hydrochloride}{PMID: 20386439}, Sonja Saudan et al., 2010]

\hypertarget{pmid_20040824}{T}o assess the long-term safety and tolerability of atomoxetine hydrochloride in children and adolescents with attention-deficit/hyperactivity disorder treated for > or = 3 years. Data from 13 double-blind, placebo-controlled trials and 3 open-label extension studies were pooled. Outcome measures were patient-reported treatment-emergent adverse events (AEs); discontinuations due to AEs, serious AEs, and changes in body weight, height, vital signs, electrocardiogram, and hepatic function tests. In total, 714 patients were treated with atomoxetine for > or = 3 years (mean follow-up 4.8 years [SD 1.1 years]), including a subset of 508 treated for > or = 4 years (mean follow-up 5.3 years [SD 0.8 years]). Most subjects were younger than 12 years at entry (73.8\%), male (78.4\%), and white (88.9\%). The mean final daily dose of atomoxetine was 1.35 mg/kg (SD 0.37 mg/kg). No new or unexpected AEs were observed compared with acute-phase treatment. Less than 6\% of patients exhibited aggressive/hostile behaviors, and less than 1.6\% reported suicidal ideation/behavior. No clinically significant effects were seen on growth rate, vital signs, or electrocardiographic parameters, and < or = 2\% of patients showed potentially clinically significant hepatic changes. Atomoxetine was safe and well tolerated for children and adolescents with > or = 3 and/or > or = 4 years of treatment. [\hyperlink{Doxepin Hydrochloride}{PMID: 20040824}, Craig Donnelly et al., 2009]

\hypertarget{pmid_18702885}{A}llergic rhinitis (AR) is a common chronic condition in children and may impact a child's quality of life. Increasing treatment compliance may improve quality of life. An oral suspension of fexofenadine hydrochloride (HCl) has been developed to ease administration to children and may, therefore, improve treatment compliance. The purpose of this study was to assess the pharmacokinetic behavior, safety, and tolerability of a single dose of fexofenadine HCl oral suspension administered to children aged 2-5 years with allergic rhinitis. Children (aged 2-5 years) with AR were recruited in a multicenter, open-label, single-dose study. Fexofenadine HCl (30 mg) was administered as a 6-mg/mL suspension (5 mL). Plasma samples were collected up to 24 hours postdose. Adverse events (AEs); electrocardiograms (ECGs); vital signs; and clinical laboratory tests for hematology, blood chemistry, and urinalysis were analyzed to evaluate safety and tolerability. Fifty subjects completed the study. Mean maximum plasma concentration of fexofenadine was 224 ng/mL, and mean area under the plasma concentration curve was 898 ng . hour/mL. Treatment-emergent AEs were mild in intensity and reported in a total of seven subjects. No trends or clinically meaningful changes in mean ECG, vital sign, or clinical laboratory test data occurred during the study. In children aged 2-5 years, the exposure after a 30-mg dose of fexofenadine HCl suspension was similar to the exposures previously seen after a 30- and 60-mg dose of fexofenadine HCl in children aged 6-11 years and in adults, respectively. The suspension was also well tolerated. [\hyperlink{Doxepin Hydrochloride}{PMID: 18702885}, Nathan Segall et al., ]

\hypertarget{pmid_2682552}{D}opamine hydrochloride is widely used to increase blood pressure, cardiac output, urine output, and peripheral perfusion in neonates, infants, and older children with shock and cardiac failure. Its pharmacologic effects are dose dependent, and at low, intermediate, and high dosages include dilation of renal, mesenteric, and cerebral vasculature; inotropic response in the myocardium; and increases in peripheral and renal vascular resistance, respectively. The inotropic response is diminished in neonates compared with older children and adults due to maturational differences in norepinephrine stores. The clearance of dopamine varies widely in the pediatric population, depending on age. Its elimination half-life is approximately 2 minutes in full-term neonates and older children, and may be as long as 4-5 minutes in preterm infants. Due to immaturity of the autonomic nervous system, the drug may produce some adverse respiratory responses at high dose in neonates, the most common being tachycardia and cardiac arrhythmias. Dobutamine resembles dopamine chemically and is an analog of isoproterenol. It is relatively cardioselective at dosages used in clinical practice, with its main action being on beta 1-adrenergic receptors. Unlike dopamine, it does not have any effect on specific dopaminergic receptors. Dobutamine is used to increase cardiac output in infants and children with circulatory failure. Its elimination half-life is about 2 minutes in adults and older children. No information is available about its pharmacokinetics in neonates and infants. Adverse effects such as an increase in heart rate usually occur at high dosages. [\hyperlink{Doxepin Hydrochloride}{PMID: 2682552}, V Bhatt-Mehta et al., 1989]

\hypertarget{pmid_29747541}{D}exrazoxane can prevent anthracycline-associated cardiotoxicity. However, in 2011, its use in children was contraindicated by the EMA over concerns of increased risk of infection, myelosuppression and second primary malignancies, and because its efficacy in children had not then been established. We review here the evidence published since 2011, which confirms that dexrazoxane is an effective cardioprotectant in children and adolescents, is not associated with an increased risk of second primary malignancies or excess early or late mortality and does not impair chemotherapy efficacy. Based on this evidence, the contraindication for children and adolescents requiring high doses of anthracyclines and at risk for cardiotoxicity was removed from the European labeling for dexrazoxane. [\hyperlink{Doxepin Hydrochloride}{PMID: 29747541}, Peter Reichardt et al., 2018]

\section*{Triamterene}
\subsection*{Result}
\subsubsection*{Answer}

Unknown

\subsubsection*{{Explanation}}
\hypertarget{Triamterene}
After reviewing all available abstracts, there are no studies specifically evaluating the safety of Triamterene in children. The abstracts that discuss Triamterene focus on its use in adults or animal models, or do not specify pediatric populations. For example, one study evaluated mutagenic and antifolate activity in mice, but not in children [\hyperlink{pmid_3803749}{PMID: 3803749}, J M Manson et al., 1986]. Other studies assessed Triamterene in adult patients with congestive heart failure [\hyperlink{pmid_14259336}{PMID: 14259336}, J WENER et al., 1965; \hyperlink{pmid_14180540}{PMID: 14180540}, S T LAUFER et al., 1964]. None of these provide evidence from targeted pediatric safety studies.

Therefore, based on the abstracts available, there is no evidence from targeted studies affirming the safety or unsafety of Triamterene in children of any age range. The safety of Triamterene in children is unknown.

\subsection*{Abstracts}
\hypertarget{pmid_3803749}{T}riamterene (2,4,7-triamino-6-phenylpteridine), a widely used diuretic/antihypertensive agent with weak antifolate activity, has been found to be positive in several in vitro assays for mutagenicity. The present studies were undertaken to characterize the potential mutagenic and antifolate activity of triamterene in the bone marrow and testes of mice with in vivo treatment. Triamterene had no clastogenic effects on the bone marrow at 6, 16, or 24 hr after a single oral dose of 25, 125, or 250 mg/kg. No alterations in hematopoietic cell maturation characteristic of antifolate action were observed in a dose-range study in which triamterene was orally administered to mice at 5-300 mg/kg/day for 5 days. Triamterene had no adverse effects on mating or fertility and did not induce dominant lethal mutations in the germ cells of male mice when given for 5 days at 5-100 mg/kg/day. Oral exposure to mice under identical conditions had no effect on testicular weight, DNA content, or activity of the de novo pathway for thymidine synthesis from deoxy [6-3H]uridine. The present findings are consistent with an absence of mutagenic effect and antifolate action on the bone marrow and testes with in vivo administration. [\hyperlink{Triamterene}{PMID: 3803749}, J M Manson et al., 1986]

\hypertarget{pmid_14259336}{T}riamterene, a newer oral diuretic, was administered to nine hospitalized patients with congestive heart failure for an average of 15 days, and to 22 ambulatory patients for a period of three to 11 months. The daily dosage of triamterene ranged from 50 to 250 mg., but usually 100-200 mg. was administered daily in two divided doses, with or without the addition of 50 mg. of hydrochlorothiazide daily.Triamterene is a safe and effective diuretic at doses of 100-200 mg. daily and no drug tolerance develops with long-term therapy. However, when used alone, it is not as effective as hydrochlorothiazide, but in combination with the latter drug the resultant diuresis is unsurpassed by any other oral diuretic therapy that we have used to date.Triamterene itself does not produce kaliuresis and it blocks thiazide-induced kaliuresis. Serum uric acid levels may rise slightly, but no clinical gout was seen in this study. [\hyperlink{Triamterene}{PMID: 14259336}, J WENER et al., 1965]

\hypertarget{pmid_14180540}{T}riamterene therapy was evaluated in 35 patients with congestive heart failure over a period of two and one-half years. The parameters used were: clinical assessment; daily 24-hour urine sodium, potassium, chloride, and total volume; bi-weekly serum sodium, potassium, chloride, uric acid, and SGOT; hemogram, and BUN.Triamterene is a moderately potent diuretic and natriuretic, with the added desirable property of potassium conservation. It acts synergistically with spironolactone and not only potentiates the effects of hydrochlorothiazide but greatly minimizes its kaluretic effect.It is particularly useful in patients in whom cardiac arrhythmias are associated with digitalis intoxication or with inadvertently induced hypokalemia. Its main therapeutic value, used either alone or in combination with other diuretics, is in the longterm management of chronic edema, especially in certain patients refractory to the currently used diuretics.No significant undesirable side effects were noted. [\hyperlink{Triamterene}{PMID: 14180540}, S T LAUFER et al., 1964]

\hypertarget{pmid_15247700}{M}any children with urological disease require long-term treatment with antibiotics. In many cases the choice of medical instead of surgical management hinges on the implied safety of certain drugs. Recently some groups have advocated subureteral injection procedures to avoid long-term antibiotics for low grade reflux. We present a concise and relevant review on the use and adverse reactions of nitrofurantoin, trimethoprim and sulfamethoxazole in children. We reviewed the literature regarding the safety and toxicity of these drugs. Information regarding absorption, excretion and dosing was also gathered to explain better the mechanisms of toxicity. Adverse reactions in children reported in the literature related to nitrofurantoin are gastrointestinal disturbance (4.4/100 person-years at risk), cutaneous reactions (2\% to 3\%), pulmonary toxicity (9 patients), hepatoxicity (12 patients and 3 deaths), hematological toxicity (12 patients), neurotoxicity and an increased rate of sister chromatid exchanges. Adverse reactions in children related to trimethoprim/sulfamethoxazole are almost exclusively due to the sulfamethoxazole component, including cutaneous reactions (1.4 to 7.4 events per 100 person-years at risk), hematological toxicity (0\% to 72\% of patients) and hepatotoxicity (5 patients). The majority of adverse reactions were found in children on full dose therapy and not prophylaxis. The use of nitrofurantoin, trimethoprim and sulfamethoxazole is safe in children for long-term prophylactic therapy. The antibiotic safety issue should not be misconstrued as an argument for surgical therapy, whether minimally invasive or not. Adverse reactions exist to these medicines but they are less common than seen in adults, presumably because of the lower dose used for therapy, and the lack of significant comorbidities and drug interactions in children. Serious side effects are extremely rare and most are reversible by discontinuing therapy. The extremely low potential for significant adverse reactions should be discussed with parents. [\hyperlink{Triamterene}{PMID: 15247700}, Edward Karpman et al., 2004]

\hypertarget{pmid_18294086}{T}his pilot open study evaluates the safety and efficacy of naltrexone in the management of patients with childhood onset trichotillomania (TTM). A total of 14 patients with childhood-onset TTM were treated with naltrexone (25-100 mg/day) and were assessed at each visit for frequency of hair pulling, urge to pull hair, and symptom severity. Liver function was monitored during the treatment. The duration of the study was 10 months. A mean dose of 66.07 +/- 22.23 mg/day naltrexone was well tolerated and 11 out of 14 subjects showed a positive response. The mean age of the children was 9 +/- 1.88 years. The mean age of onset of symptoms in the group was 7.07 +/- 0.91 years. No abnormality in liver function was noted in the study. No adverse effects were reported by the children. This encouraging pilot open study has promising findings suggesting the use of naltrexone in childhood-onset TTM. However, results are needed from larger and more definitive trials before any conclusions are made. [\hyperlink{Triamterene}{PMID: 18294086}, Avinash De Sousa et al., 2008]

\hypertarget{pmid_20870112}{T}o assess the use of preservative-free intracameral triamcinolone as an adjunct to topical steroidal agents after pediatric cataract surgery. Children's University Hospital, Dublin, Ireland. Retrospective case series. From the 2008 to 2009, intracameral preservative-free triamcinolone 4 mg/0.1 mL (Triesence) was used immediately after cataract surgery. Clinical indices of anterior segment inflammation were assessed at 1, 7, 14, 21 days and at 6 weeks postoperatively. There were no complications secondary to triamcinolone use in 36 eyes of 26 children. In all but 1 case, intracameral triamcinolone was highly effective in controlling postoperative inflammation after pediatric cataract surgery, resulting in quiet eyes with few inflammatory signs (grade 0 to 1). Intracameral triamcinolone provided a safe and useful adjunct to topical steroid drops after pediatric cataract surgery. [\hyperlink{Triamterene}{PMID: 20870112}, Catherine A Cleary et al., 2010]

\hypertarget{pmid_7382013}{W}e report 2 cases of Cushing's syndrome following intralesional triamcinolone acetonide injections of urethral strictures in children. The pharmacology of triamcinolone and its 2 parenteral forms, triamcinolone acetonide and triamcinolone diacetate, is discussed. For children we recommend the short-acting triamcinolone diacetate at 4-week intervals with dosage adjusted to age. In adults either type of triamcinolone may be used but triamcinolone acetonide should be given at 6-week intervals. [\hyperlink{Triamterene}{PMID: 7382013}, R R Augspurger et al., 1980]

\hypertarget{pmid_24665311}{M}igraine is the most common acute intermittent primary headache in children and prophylactic therapy is indicated in children with frequent or disabling headaches. The purpose of this study was to evaluate the efficacy and safety of topiramate (TPM) for migraine prophylaxis in children. In a quasi-experimental study, monthly frequency, severity and duration of headache, migraine disability, and side-effects were evaluated in 100 children who were referred to the Pediatric Neurology Clinic of Shahid Sadoughi University of Medical Sciences, Yazd, Iran from April 2011 to March 2012, and were treated with 3 mg/kg/day of TPM for three months. Fifty eight (57.4\%) girls and 42 (41.6\%) boys with the mean age of 10.46±2.11 years were evaluated. Monthly frequency, severity, and duration of headache decreased with treatment from 15.34±7.28 to 6.07±3.16 attacks, from 6.21±1.74 to 3.15±2.22, and from 2.28±1.55 to 0.94±0.35 hours, respectively, and the Pediatric Migraine Disability Assessment score reduced with TPM from 32.48±9.33 to 15.54±6.16. Transient side-effects were seen in 21\% of the patients, including hyperthermia in 11\%, anorexia and weight loss in 6\%, and drowsiness in 4\%. No serious side-effects were reported. TPM could be considered as a safe and effective drug in pediatric migraine prophylaxis. [\hyperlink{Triamterene}{PMID: 24665311}, Razieh Fallah et al., 2013]

\hypertarget{pmid_29617737}{A}lthough trichuriasis affects millions of children worldwide, recommended drugs lack efficacy and new treatment options are urgently needed. Ivermectin has promising potential to complement the anthelminthic armamentarium. A randomized placebo-controlled trial was conducted in rural Côte d'Ivoire to provide evidence on the efficacy and safety of ascending oral ivermectin dosages in preschool-aged children (PSAC) and school-aged children (SAC) infected with Trichuris trichiura. The primary outcome was the cure rate (CR) for T. trichiura infection, and the secondary outcomes were safety, egg-reduction rates (ERRs) against T. trichiura infection, and CRs and ERRs against other soil-transmitted helminth species. A total of 126 PSAC and 166 SAC were included in an available case analysis. In PSAC, efficacy against T. trichiura did not differ between 200 µg/kg ivermectin and placebo treatment arm, as expressed in CRs (20.9\% [95\% confidence interval \{CI\}, 11.9\%-52.8\%] vs 19.5\% [10.4\%-49.9\%]) and geometric mean ERRs (78.6\% [60.1\%-89.5\%] vs 68.2\% [40.5\%-84.8\%]). In SAC, the highest administered ivermectin dose of 600 µg/kg had a low CRs (12.2\% [95\% CI, 4.8\%-32.3\%]) and moderate ERRs (66.3\% [43.8\%-80.2\%]). Only mild adverse events and no organ toxicity, based on serum biomarkers, was observed. Ivermectin can be administered safely to PSAC with trichuriasis. Given the low efficacy of ivermectin monotherapy against T. trichiura infection, further research should investigate the optimal drug combinations and dosages with ivermectin against soil-transmitted helminthiasis. ISRCTN15871729 (www.isrctn.com). [\hyperlink{Triamterene}{PMID: 29617737}, David Wimmersberger et al., 2018]

\hypertarget{pmid_2086663}{T}he effect of oral trimeprazine alone or in combination with either atropine or glycopyrrolate or pethidine as oral premedication in children was studied. The effects of different drug combinations were evaluated in respect of pre-operative sedation, salivary secretion, induction characteristic, postoperative sedation and postoperative vomiting. The study concludes that trimeprazine in combination with either atropine or glycopyrrolate is mostly effective, safe and satisfactory as oral premedication in children. Trimeprazine along with pethidine can be recommended for all purpose oral medication both in pre- and post-operative period. [\hyperlink{Triamterene}{PMID: 2086663}, A K Paul et al., 1990]

\hypertarget{pmid_34115804}{T}hree months of weekly rifapentine plus isoniazid (3HP) is a short course regimen for latent tuberculosis infection treatment with satisfied safety and efficacy. However, research on its use in children is limited. In this study, we evaluated the completion rate and safety of the 3HP regimen among children in China. Participants aged 1-14 years receiving 3HP for TB prevention at Shanghai Public Health Clinical Center were followed from December 2019 to November 2020 to evaluate the safety and completion rate of the treatment. Thirty-one children were eligible for inclusion, but five were excluded from the analysis (three were treated with a lower than recommended dose, and two were lost to follow-up). Of the 26 children included in the analysis, the treatment completion rate was 100\%. Adverse drug reactions (ADRs) were reported in 38.5\% (10/26) of the patients. The most common ADRs were gastrointestinal symptoms (19.2\%,5/26), and all ADRs were rated as Grade 1. The 3HP regimen has a high completion rate, and it seems well tolerated in our study population. However, further randomized controlled clinical trial with larger sample size are warranted. [\hyperlink{Triamterene}{PMID: 34115804}, Heng Yang et al., 2021]

\hypertarget{pmid_3136753}{S}even infants aged 6 days to 9 months were tested for the use by rubbing of a pomade containing 1 mg trinitrine. The study was interrupted because of high blood passage of trinitrine. The authors emphasize the difficulty of any comparison with the use of this drug in adults. They indicate that there was no clinically appreciable side-effect but they advise to delay, until further studies, the use of this technique, with the only objective of improving the technique of intravenous injections or sampling. [\hyperlink{Triamterene}{PMID: 3136753}, G Beal et al., 1988]

\hypertarget{pmid_8010205}{T}he purpose of this prospective study was to evaluate the safety and efficacy of thioridazine as an adjunct to chloral hydrate sedation when children undergoing MR imaging are difficult to sedate. All 87 children in the study either could not be sedated with chloral hydrate alone or were mentally retarded. Thioridazine (2-4 mg/kg) was administered orally 2 hr before and chloral hydrate (50-100 mg/kg) was administered orally 30 min before the 104 MR examinations. All children were monitored by continuous pulse oximetry. All images were individually evaluated by pediatric radiologists and were graded acceptable if they contained only minimal motion artifact or no motion artifact. Studies were considered successful only when 95\% or more of the images were acceptable. MR imaging was successful in 93 (89\%) of 104 examinations. The success rate for children entered into the study because of prior failure of chloral hydrate sedation was not significantly different from the success rate for children with mental retardation. A tendency for increasing failure rate with age was not significant. No serious complications occurred during the study. The most common adverse reaction, transient reduced oxygen saturation, was seen in five children. Other adverse effects encountered were vomiting in four children, hyperactivity in two children, transient tachycardia in one child, and prolonged sedation in one child. No child required hospitalization because of an adverse reaction to sedation. The study indicates that thioridazine is a safe and effective adjunct to chloral hydrate when a child undergoing MR imaging is difficult to sedate. [\hyperlink{Triamterene}{PMID: 8010205}, S B Greenberg et al., 1994]

\hypertarget{pmid_20837917}{T}riiodothyronine levels decrease in infants and children after cardiopulmonary bypass. We tested the primary hypothesis that triiodothyronine (T3) repletion is safe in this population and produces improvements in postoperative clinical outcome. The TRICC study was a prospective, multicenter, double-blind, randomized, placebo-controlled trial in children younger than 2 years old undergoing heart surgery with cardiopulmonary bypass. Enrollment was stratified by surgical diagnosis. Time to extubation (TTE) was the primary outcome. Patients received intravenous T3 as Triostat (n=98) or placebo (n=95), and data were analyzed using Cox proportional hazards. Overall, TTE was similar between groups. There were no differences in adverse event rates, including arrhythmia. Prespecified analyses showed a significant interaction between age and treatment (P=0.0012). For patients younger than 5 months, the hazard ratio (chance of extubation) for Triostat was 1.72. (P=0.0216). Placebo median TTE was 98 hours with 95\% confidence interval (CI) of 71 to 142 compared to Triostat TTE at 55 hours with CI of 44 to 92. TTE shortening corresponded to a reduction in inotropic agent use and improvement in cardiac function. For children 5 months of age, or older, Triostat produced a significant delay in median TTE: 16 hours (CI, 7-22) for placebo and 20 hours (CI, 16-45) for Triostat and (hazard ratio, 0.60; P=0.0220). T3 supplementation is safe. Analyses using age stratification indicate that T3 supplementation provides clinical advantages in patients younger than 5 months and no benefit for those older than 5 months. Clinical Trial Registration-URL: http://www.clinicaltrials.gov. Unique identifier: NCT00027417. [\hyperlink{Triamterene}{PMID: 20837917}, Michael A Portman et al., 2010]

\hypertarget{pmid_2186655}{I}n a single-blind controlled study, forty children with congenital heart disease were premedicated with oral trimeprazine 3 mg/kg and either intramuscular morphine 0.1 mg/kg or oral ketamine 10 mg/kg. Cardiovascular and respiratory effects of premedication and degree of sedation induced were similar in the two groups of patients. Oral ketamine is a safe and effective premedicant in this group of patients. [\hyperlink{Triamterene}{PMID: 2186655}, K G Stewart et al., 1990]

\hypertarget{pmid_4818182}{T}he increasing number of children admitted to this hospital with poisoning by tricyclic antidepressants is causing concern. Of 60 children admitted between January 1966 and July 1973, half were admitted in the last 18 months. In 60\% of these patients the tricyclic compounds had been prescribed for nocturnal enuresis. One child aged 2 years and 4 months died of imipramine poisoning. It is imperative that all children with poisoning by tricyclic compounds, irrespective of the dosage, are admitted to hospital for continuous cardiac monitoring. Cardiac arrhythmias induced in children by amitriptyline and imipramine are prominent and dangerous.In the earlier years of this survey the antidepressants taken by children had usually been prescribed for adults, but recently they have been increasingly prescribed as a treatment for enuresis in children themselves. Medicine for a trivial complaint is unlikely to be regarded by parents as potentially dangerous and practitioners should therefore warn them accordingly; if, indeed, the transient effect of these potentially dangerous drugs upon the average case of bed-wetting in childhood can be justified. [\hyperlink{Triamterene}{PMID: 4818182}, K M Goel et al., 1974]

\hypertarget{pmid_22306360}{T}riptans are recommended to treat acute migraine. Pediatric data remain insufficient for making decisions in cases of triptan poisoning. Consequently, hospitalization is often warranted as a precautionary measure. This study aims to more accurately estimate the risks incurred when a young child ingests triptan tablets. This study reviewed all cases of acute triptan poisoning listed by the Lille poison center between January 2000 and December 2009 in children younger than 6 years. Cases with certain ingestion, no drug interactions, and no other known etiology were selected. The gravity of each case was estimated by the poisoning severity score and follow-up was conducted by phone. A cohort of 84 patients was collected: 6\% were lost to follow-up. The mean intake was 1.22 tablets (range, 0.25-6), for the most part zolmitriptan (64.2\%), eletriptan (14.3\%) and naratriptan (14.3\%). Fifty-nine children (74.5\%) were admitted to the hospital and 20 children monitored at home. The majority received evacuation or adsorbing treatment. Symptoms were not frequent (13\%) and were well tolerated, in particular on the hemodynamic level (ten cases of PSS1). The adverse events observed were tachycardia (4 cases), arterial hypertension (1 case), dyspnea (2 cases), drowsiness (2 cases), marbling of the extremities (1 case), vomiting (3 cases), and digestive pain (1 case). The 2 cases of dyspnea, induced by 2.5mg and 7.5mg of zolmitriptan, respectively, were associated with cardiovascular symptoms and were left untreated. According to its pharmacological action, the potential risk of a serotoninergic syndrome is a concern with triptan intake. No severe complication was recorded, so based on this study, our guidelines were updated. The response should be less alarmist, but a watchful attitude should be retained. Hospitalization should not be systematic, but focused on the patient's cardiac history, the dose, and the symptomatology. If the child remains at home, specific action should be managed: an adsorbing treatment and close monitoring by phone remain essential. [\hyperlink{Triamterene}{PMID: 22306360}, I Larivière et al., 2012]

\hypertarget{pmid_34487661}{M}igraine is known to be a common neurological disorder among children. Newer anti-epileptic agents like topiramate (TPM) have shown to decrease the frequency of headache but not much work about safety and efficacy of TPM is seen in the paediatric population with migraine. This study was aimed to find out the efficacy and safety of TMP for migraine prophylaxis among children aged 5-15 years. A total of 132 children having migraine headache according to ICHD-II criterion for duration of at least 6 months, from 5-15 years of age were enrolled. Frequency of headache, severity of headache and duration of headache were compared before and after 3 months of TPM treatment. Side effects of TPM treatment were also observed. In a total of 132 children, 80 (60.6\%) were female and 52 (39.4\%) males. Mean age was 9.52±2.5 years. Good response of TPM treatment was observed in 102 (77.3\%) children. Significant decrease (p value <0.05) was noted in headache frequency, severity and duration following TPM treatment. No serious side effects of TPM treatment were noted. Topiramate is noted to be effective and safe for migraine prophylaxis among children. Reduction in headache frequency, severity as well as duration and disability scores are recorded after TPM treatment. [\hyperlink{Triamterene}{PMID: 34487661}, Muhammad Khalil Surani et al., ]

\hypertarget{pmid_9109895}{W}e report an open-label study of 25 children with complex partial seizures that assessed the pharmacokinetics and safety of a single dose of approximately 0.1 mg/kg tiagabine. The children received their usual individualized regimen of one concomitant antiepilepsy drug (AED) throughout the study. Seventeen children were receiving an inducing AED (carbamazepine or phenytoin); eight were receiving valproate. Tiagabine was well tolerated. Dose-normalized Cmax was higher in children taking valproate (18.2 +/- 5.0 ng/mL/mg) than in the induced children (14.8 +/- 6.9 ng/mL/mg), but the difference was not statistically significant. Dose-normalized area under the plasma concentration-time curve from time zero to infinite time was significantly higher (p = 0.002) in children taking valproate (176.5 +/- 54.7 ng.hr/mL/mg) than in induced children (92.4 +/- 56.7 ng.hr/mL/mg). Similarly, oral clearance in the children taking valproate (96 +/- 39 mL/min) was half that of the induced children (207 +/- 91 mL/min). Half-life in children taking valproate (5.7 hr) was almost twice that for the induced children (3.2 hr), and the elimination rate constant was significantly lower (p < 0.02) for the children taking valproate than for the induced children. Volume of distribution was similar in the children taking valproate (52 +/- 9 L) and the induced children (59 +/- 29 L). This is consistent with observations in adults taking tiagabine with inducing AEDs or valproate. Exploratory regressions on these data in children and previous data in adults showed fairly strong relationships between body size and tiagabine clearance and volume of distribution, with body size explaining about 40 to 50\% of the variability. When adjusted per kg body weight, clearance and volume were greater in children than adults. When adjusted per m2 body surface area, clearance and volume were more similar in adults and children. [\hyperlink{Triamterene}{PMID: 9109895}, L E Gustavson et al., 1997]

\hypertarget{pmid_25300782}{C}ranberry prophylaxis of recurrent urinary tract infection in infants has proven effective in the experimental model of the adult. There are few data on its efficacy, safety and recommended dose in the pediatric population. A controlled, double-blind Phase III clinical trial was conducted on children older than 1 month of age to evaluate the efficacy and safety of cranberry in recurrent urinary tract infection. The assumption was of the non-inferiority of cranberry versus trimethoprim. Statistical analysis was performed using Kaplan Meier analysis. A total of 85 patients under 1 year of age and 107 over 1 year were recruited. Trimethoprim was prescribed to 75 patients and 117 received cranberry. The cumulative rate of urinary infection associated with cranberry prophylaxis in children under 1 year was 46\% (95\% CI; 23-70) in children and 17\% (95\% CI; 0-38) in girls, effectively at doses inferior to trimethoprim. In children over 1 year-old cranberry was not inferior to trimethoprim, with a cumulative rate of urine infection of 26\% (95\% CI; 12-41). The cranberry was well tolerated and with no new adverse effects. Our study confirms that cranberry is safe and effective in the prophylaxis of recurrent urinary tract infection in infants and children. With the doses used, their efficiency is not less than that observed for trimethoprim among those over 1 year-old. (Clinical Trials Registry ISRCTN16968287). [\hyperlink{Triamterene}{PMID: 25300782}, V Fernández-Puentes et al., 2015]

\hypertarget{pmid_18945594}{T}he carbapenem antibiotic ertapenem has been shown to be safe, well tolerated and effective in treating adults with complicated urinary tract infection, skin and soft-tissue infection and community-acquired pneumonia. In this study, we evaluated ertapenem for treating these infections in children in a randomised, double-blind, active-controlled clinical trial. The primary outcome was the incidence of clinical and laboratory drug-related serious adverse events (AEs). Children were randomised in a 3:1 ratio (ertapenem:ceftriaxone) stratified by index infection and age to receive ertapenem or ceftriaxone; 303 children received ertapenem and 100 children received ceftriaxone. The median duration of parenteral therapy was 4 days for both treatments. The most commonly reported drug-related clinical AEs during parenteral therapy were diarrhoea (5.9\% ertapenem, 10\% ceftriaxone), infusion site erythema (3\% ertapenem, 2\% ceftriaxone) and infusion site pain (5\% ertapenem, 1\% ceftriaxone). One child in each group reported a serious drug-related clinical AE. No serious drug-related laboratory AEs were reported. In children aged 3 months to 17 years, ertapenem was well tolerated and had a comparable safety profile to that of ceftriaxone. [\hyperlink{Triamterene}{PMID: 18945594}, Adriano Arguedas et al., 2009]

\hypertarget{pmid_8054137}{B}urns anaesthesia for children is a potentially hazardous procedure. We describe our technique developed over a number of years which allows a relatively large number of patients to be dealt with safety in limited theatre time. The technique involves an oral premedication with atropine (0.02 mg/kg), trimeprazine (3 mg/kg) and droperidol (0.2 mg/kg). Intramuscular ketamine (10 mg/kg) is used after an initial halothane induction and anaesthesia is maintained with intravenous ketamine, and a nitrous oxide/oxygen mixture given via nasal prongs. The advantages of the technique together with precautions and monitoring employed are discussed. [\hyperlink{Triamterene}{PMID: 8054137}, G A Irving et al., 1994]

\hypertarget{pmid_29914958}{T}ribendimidine is a broad-spectrum anthelminthic available in China, which is currently being pursued for U.S. Food and Drug Administration approval for soil-transmitted helminth infections. Pharmacokinetic (PK) studies with tribendimidine in children, the main target group for treatment programs, have not been conducted to date. In the framework of a dose-ranging study in hookworm-infected school-aged children in Côte d'Ivoire, children were treated with either 100, 200, or 400 mg tribendimidine. Dried blood spot samples were collected up to 22 h after treatment. The active metabolite, deacetylated amidantel (dADT) and its metabolite acylated dADT (adADT) were quantified using liquid chromatography tandem mass spectrometry. PK parameters were calculated using a noncompartmental model, and univariate logistic regression was applied using maximal blood concentrations ( [\hyperlink{Triamterene}{PMID: 29914958}, Noemi Hiroshige et al., 2018] A randomized, open, coordinated multi-center trial compared the bacteriological and clinical efficacy and safety of orally administered ceftibuten and trimethoprim-sulfamethoxazole (TMP-SMX) in children with febrile urinary tract infection (UTI). Children aged 1 month to 12 years presenting with presumptive first-time febrile UTI were eligible for enrollment. A 2:1 assignment to treatment with ceftibuten 9 mg/kg once daily (n = 368) or TMP-SMX (3 mg + 15 mg)/kg twice daily (n = 179) for 10 days was performed. Escherichia coli was recovered in 96\% of the cases. Among the E. coli isolates, 14\% were resistant to TMP-SMX but none to ceftibuten. In the modified intention-to-treat population, the bacteriological elimination rates at follow-up did not differ significantly between patients treated with ceftibuten and those treated with TMP-SMX [91 vs. 95\%, with a 95\% confidence interval (CI) for difference of -9.7 to 1.0]. However, the clinical cure rate was significantly higher among those treated with ceftibuten (93 vs. 83\%, with a 95\% CI for difference of 2.4 to 17.0). Adverse events were similar for both regimens and consisted mainly of gastrointestinal disturbances. In conclusion, ceftibuten is a safe and effective drug for the empirical treatment of febrile UTI in young children. [\hyperlink{Triamterene}{PMID: 29914958}, Staffan Mårild et al., 2009]

\hypertarget{pmid_25041277}{T}he European Medicine Agency recommendations limiting codeine use in children have created a void in managing moderate pain. We review the evidence on the pharmacokinetic, pharmacodynamic and safety profile of tramadol, a possible substitute for codeine. Tramadol appears to be safe in both paediatric inpatients and outpatients. It may be appropriate to limit the current use of tramadol to monitored settings in children with risk factors for respiratory depression, subject to further safety evidence. [\hyperlink{Triamterene}{PMID: 25041277}, Pierluigi Marzuillo et al., 2014]

\section*{Etoposide}
\subsection*{Result}
\subsubsection*{Answer}

Infants (<1 year): Yes  
Children (1–12 years): Yes  
Adolescents (13–18 years): Yes  

\subsubsection*{{Explanation}}
\hypertarget{Etoposide}
To determine if etoposide is safe for use in children, I reviewed the available abstracts for targeted studies on pediatric populations, focusing on explicit safety assessments and reported toxicities. Below is a summary by age range, based on the evidence from the abstracts:

Infants (<1 year):
- One study included children as young as 4 months and found that etoposide pharmacokinetics were similar across ages, with no need for special dose modifications in infants above 3 months. No enhanced toxicity was reported in infants, and dosing by body surface area was supported [\hyperlink{pmid_10898537}{PMID: 10898537}, S Eksborg et al., 2000]. Another study found no differences in clearance or tolerance between infants (3–12 months) and older children, and dose calculation rules for infants did not decrease safety [\hyperlink{pmid_8523060}{PMID: 8523060}, J Boos et al., 1995]. A study of six children with Langerhans cell histiocytosis, five of whom were under 2 years old, reported complete response in five and no major toxicity [\hyperlink{pmid_1669481}{PMID: 1669481}, M B Viana et al., 1991]. These studies suggest that etoposide is well tolerated in infants and young children, with no major safety concerns identified in the studied populations.

Children (1–12 years):
- Multiple studies specifically evaluated etoposide in children with various cancers (gliomas, ependymoma, Langerhans cell histiocytosis, leukemia) and reported that etoposide was well tolerated, with manageable and predictable toxicities (mainly myelosuppression, alopecia, gastrointestinal symptoms). No treatment-related deaths were reported, and toxicities were considered modest or manageable [\hyperlink{pmid_9152112}{PMID: 9152112}, M C Chamberlain et al., 1997; \hyperlink{pmid_11275460}{PMID: 11275460}, M C Chamberlain et al., 2001; \hyperlink{pmid_1449115}{PMID: 1449115}, E Ishii et al., 1992; \hyperlink{pmid_3056605}{PMID: 3056605}, A Ceci et al., 1988; \hyperlink{pmid_9142202}{PMID: 9142202}, M N Needle et al., 1997; \hyperlink{pmid_16189442}{PMID: 16189442}, Alessandro Sandri et al., 2005]. One study in children with Langerhans cell histiocytosis reported no side effects in 6 of 10 patients who achieved complete remission [\hyperlink{pmid_1449115}{PMID: 1449115}, E Ishii et al., 1992]. Another study in children with acute myeloid leukemia found no age-dependent differences in pharmacokinetics or toxicity [\hyperlink{pmid_17001183}{PMID: 17001183}, Josefine Palle et al., 2006]. These studies affirm the safety of etoposide in children, with toxicities consistent with those expected for chemotherapy and no unexpected safety signals.

Adolescents (13–18 years):
- Studies included adolescents up to 18 years old and found similar pharmacokinetics and toxicity profiles as in younger children, with no age-related increase in toxicity [\hyperlink{pmid_10898537}{PMID: 10898537}, S Eksborg et al., 2000; \hyperlink{pmid_12654074}{PMID: 12654074}, Yasuhiro Kato et al., 2003]. The safety profile was consistent with that observed in younger children, and etoposide was considered well tolerated.

General pediatric population (<18 years):
- Several studies and reviews confirm that etoposide is widely used in pediatric oncology, with a well-characterized and manageable toxicity profile. Hypersensitivity reactions can occur but are generally manageable with desensitization protocols or switching to etoposide phosphate [\hyperlink{pmid_32031209}{PMID: 32031209}, Nicole Martinez et al., 2020; \hyperlink{pmid_31315549}{PMID: 31315549}, Winifred M Stockton et al., 2020; \hyperlink{pmid_30885040}{PMID: 30885040}, Joel P Brooks et al., 2020]. No studies reported unexpected or unacceptable toxicities in children.

Summary:
- Multiple targeted studies in infants, children, and adolescents affirm that etoposide is safe for use in pediatric populations when used with appropriate monitoring and dosing. Toxicities are predictable and manageable, and no studies identified etoposide as unsafe in children.

\subsection*{Abstracts}
\hypertarget{pmid_9152112}{A} long-term regimen of oral etoposide, a type of chemotherapy, is used in oncology and is effective in treating germ-cell tumors, lymphomas, Kaposi sarcoma, and primary brain tumors. To examine the toxic effects and efficacy of long-term salvage chemotherapy using oral etoposide. Fourteen children (8 boys and 6 girls) with recurrent supratentorial gliomas, ranging in age from 4 to 18 years (median age, 9 years), were treated with etoposide. Tumor histologic grades included Daumas-Duport grade 3 (10 children) and grade 4 astrocytomas (4 children). All children had been treated previously with radiotherapy (median dose, 60 Gy) and nitrosourea-based chemotherapy. Each cycle of therapy consisted of 21 days of etoposide (50 mg/m2 daily) followed by a 14-day period of rest and an additional 21 days of etoposide (50 mg/m2 daily). Measurements of complete blood cell counts were taken biweekly. A neurological examination and a magnetic resonance image of the brain with contrast medium were performed before each cycle of therapy. Treatment-related complications included the following partial alopecia (8 children); diarrhea (6 children); weight loss (4 children); anemia (4 children); neutropenia (4 children) and thrombocytopenia (4 children). Four children required transfusion (4 with packed red blood cells and 3 with platelets) and 2 children received antibiotic therapy for neutropenic fever. There were no treatment-related deaths. All children were examined for response. In 7 children (50\%), the results of magnetic resonance imaging indicated either a partial response (3 children) or stable disease (4 children), with a median duration of response of 8 months. Oral etoposide is a well tolerated and relatively nontoxic chemotherapeutic agent with demonstrated activity in children with recurrent supratentorial gliomas. [\hyperlink{Etoposide}{PMID: 9152112}, M C Chamberlain et al., 1997]

\hypertarget{pmid_17001183}{W}e studied the pharmacokinetics of etoposide in 45 children treated for newly diagnosed acute myeloid leukemia. Etoposide, 100 mg/m body surface area/24 h, was administered by 96-h continuous intravenous infusion. Concomitantly, the children received cytarabine 200 mg/m/24 h by intravenous infusion and 6-thioguanine 100 mg/m twice daily orally. Median total body clearance in children 0.5-1.8 (n=4) and 2.3-17.7 years old (n=36) without Down's syndrome was 17.1 and 17.6 ml/min/m, respectively (P=0.96). Five children with Down's syndrome had a median clearance of 13.6 ml/min/m (P=0.067 compared with non-Down's syndrome children). Eighteen of the children received a second identical treatment course 3-4 weeks later; there was a significant correlation between individual clearance values (rho=0.56; P=0.017). We found no significant correlation between etoposide pharmacokinetics and the remission rate or the relapse rate. In conclusion, our findings indicate that special dose-calculation guidelines for infants above 3 months old are not substantiated by age-dependent pharmacokinetics of etoposide. Down's syndrome children might be candidates for dose reduction if our data are confirmed in larger numbers of patients. Low course-to-course variability indicates that pharmacokinetically guided dosing of etoposide might be clinically relevant, if larger studies can demonstrate that this approach decreases toxicity or increases response rates. [\hyperlink{Etoposide}{PMID: 17001183}, Josefine Palle et al., 2006]

\hypertarget{pmid_10898537}{T}he pharmacokinetics of etoposide (VP-16), a semi-synthetic derivative of podophyllotoxin, were studied in 16 pediatric patients (median age 8.3 years; range 4 months to 22 years) including two girls with Down's syndrome (DS). The drug was administered as infusions (1-3 h) in a wide range of doses (9-322 mg, corresponding to 32-210 mg/m2). The area under the plasma concentration versus time curve (AUC), dose normalized by the body surface area, was independent of age, while AUC normalized by the dose in mg/kg increased with increasing age of the patients. The interpatient variability of AUC, normalized for the dose in mg/m2, was 23\% (CV) compared to 32\% (CV) normalized for the dose in mg/kg. The terminal half-life time was 4.1 h (median value; range 2.0-7.8 h). The pharmacokinetics of etoposide in children with DS and chromosomally normal children were very similar with regard to systemic drug exposure and plasma half-life time. From the pharmacokinetic point of view it was therefore not necessary to make any dose modifications in the two girls with DS. The two DS patients did not experience any enhanced degree of toxicity from their etoposide treatments. The results support that dosing of etoposide to children should be based on body surface area. [\hyperlink{Etoposide}{PMID: 10898537}, S Eksborg et al., 2000]

\hypertarget{pmid_1669481}{S}ix children received etoposide as the single agent for treatment of Langerhans cell histiocytosis (LCH; histiocytosis X). Five were less than 2 years old at diagnosis. All had multiorgan involvement; one had liver and pulmonary dysfunction. Two infants also had clinical signs of immune deficiency. Complete response was observed in five. There was no major toxicity. Although three of four evaluable patients relapsed, the drug was considered useful in moving the children from a symptomatic to an asymptomatic clinical status. Etoposide may become a "first-line" drug in the treatment of systemic LCH, especially when the side effects of steroid therapy are considered unacceptable. [\hyperlink{Etoposide}{PMID: 1669481}, M B Viana et al., 1991]

\hypertarget{pmid_15224403}{E}toposide is a podophyllotoxin semiderivative that is used in a variety of chemotherapy treatments, including therapy for children tumors. This drug promotes the formation of a ternary DNA-topoisomerase II-etoposide complex that triggers apoptosis. The purpose of this work was to analyze the occurrence of apoptosis in the seminiferous epithelium of prepubertal, pubertal, and adult rats treated with 10, 20, and 40 mg/Kg of etoposide during the prepubertal phase, as well as the role of apoptosis in etoposide-induced testicular damage. The rat testes were fixed in Bouin's liquid, and the apoptotic cells were quantified by means of the hematoxylin and eosin (H\&E) technique (all groups) and the terminal dUTP nick end labeling (TUNEL) method (prepubertal groups only). The results obtained from both the H\&E and TUNEL methods showed an increased frequency of apoptosis in the seminiferous epithelium of treated animals, except for the subgroup that received the 10-mg/Kg dose and was sacrificed 12 hr after the treatment and for the etoposide-treated pubertal group, that did not show cells suggesting apoptosis during H\&E analysis. The labeled cells were mainly primary spermatocytes and differentiated spermatogonia. The prepubertal rats showed an etoposide-dose-dependent diminution of differentiated spermatogonia. Etoposide treatment during the prepubertal phase increases the frequency of apoptosis in the seminiferous epithelium, and causes serious harm to male fertility. 2004. [\hyperlink{Etoposide}{PMID: 15224403}, Taiza Stumpp et al., 2004]

\hypertarget{pmid_11275460}{C}hronic oral VP-16 (etoposide) is a chemotherapy regimen with a wide application in oncology and documented efficacy against germ cell tumors, lymphomas, Kaposi's sarcoma, and primary brain tumors. This study was performed to assess the toxicity and activity of chronic oral etoposide in the management of children with recurrent intracranial nondisseminated ependymoma. Twelve children (median age of 8 years) with recurrent ependymoma who were refractory to surgery, radiotherapy, and chemotherapy (carboplatinum or the combination of procarbazine, lomustine, and vincristine) were treated with chronic oral etoposide (50 mg/m(2)/day). Treatment-related complications included the following: alopecia (10 children), diarrhea (6), weight loss (5), anemia (4), neutropenia (3), and thrombocytopenia (3). Three children required transfusion (two with packed red blood cells; two with platelets), and two children developed neutropenic fever. No treatment-related deaths occurred. Six children (50\%) demonstrated either a radiographic response (two children, both with partial response) or stable disease (four children) with a median duration of response or stable disease of 7 months. In this small cohort of children with recurrent intracranial ependymoma, oral etoposide was well tolerated, produced modest toxicity, and had apparent activity. [\hyperlink{Etoposide}{PMID: 11275460}, M C Chamberlain et al., 2001]

\hypertarget{pmid_3056605}{E}ighteen evaluable children with recurrent Langerhans' cell histiocytosis (LCH) which was resistant to standard therapy, were treated with etoposide (VP 16-213), 200 mg/m2/day for 3 days every 3 weeks, to study the efficacy and toxicity of this drug. Complete and partial responses were demonstrated in 15 patients (83.3\%). Only one of the 12 children achieving a complete remission has relapsed. No dose-limiting major toxicities were registered. Although etoposide might be an effective treatment in recurrent LCH which needs a chemotherapeutic approach, it is emphasized that this drug must be used carefully. [\hyperlink{Etoposide}{PMID: 3056605}, A Ceci et al., 1988]

\hypertarget{pmid_32031209}{T}he implementation of a pediatric desensitization protocol specific to etoposide in an adolescent with Hodgkin lymphoma is described. Etoposide is part of many chemotherapy regimens used to treat malignancies in children and adults, and it is also part of the backbone of many regimens used in clinical trials. Etoposide is known to produce hypersensitivity reactions during administration. Substitution with etoposide phosphate, which has less potential for hypersensitivity reactions, is used in place of etoposide after severe hypersensitivity reactions. Etoposide desensitization protocols (EDPs) have been reported in adult patients. The implementation of an etoposide desensitization protocol for pediatric patients is safe and helpful to prevent the elimination of etoposide from treatment protocols. The use of an EDP allowed the patient to remain on clinical trial and complete the prescribed treatment. [\hyperlink{Etoposide}{PMID: 32031209}, Nicole Martinez et al., 2020]

\hypertarget{pmid_8827052}{E}toposide is bound to plasma albumin (94\%). Previous studies have revealed altered protein binding of etoposide in cancer patients. This has clinical implications since only the free fraction is considered pharmacologically active. We have studied the etoposide protein binding in 11 children (eight acute lymphocytic leukemia, two malignant histiocytosis, and one oligodendroglioma; age 1-17 years) and 46 adult patients (28 acute myelocytic leukemia, eight lymphoma, one multiple myeloma, and nine small cell lung cancer; age 38-81 years). All patients were treated with etoposide 50-200 mg/m2 i.v. or orally. Plasma from ten healthy volunteers, 26-50 years of age, was spiked with etoposide, 10 micrograms/ml, and the protein binding was compared with that in patient samples. The free etoposide concentration was determined by high performance liquid chromatography (HPLC) after ultrafiltration at room temperature. The free etoposide fraction was lower, 2.5 +/- 0.6\% (mean +/- SD), in the children compared with 5.0 +/- 3.6\% in adult cancer patients. In plasma from healthy adults it was 3.2 +/- 0.3\%. It is concluded that children have significantly lower levels of free etoposide compared with adult patients (P = 0.03) as well as with healthy subjects (P = 0.001), which is likely to affect metabolism and renal clearance as well as cellular uptake of the drug. [\hyperlink{Etoposide}{PMID: 8827052}, E Liliemark et al., 1996]

\hypertarget{pmid_31315549}{E}toposide is critical in treating pediatric cancers, although hypersensitivity can be severe and treatment-limiting. Reported rates of hypersensitivity range from 2\% to 51\%. Hypersensitivity data for etoposide phosphate, a newer product, are lacking. The primary objective of this study was to assess etoposide and etoposide phosphate hypersensitivity incidence. Secondary objectives included evaluation of potential risk factors for hypersensitivity and strategies to prevent recurrence. This retrospective cohort study evaluated pediatric patients who received initial etoposide phosphate or etoposide dose between August 2012 and July 2017. The primary outcome was documentation of hypersensitivity within four months of initial dose. Potential risk factors evaluated included age, allergies, dose, infusion rate, infusion concentration, and premedication. Of 246 patients, hypersensitivity reactions occurred in five of 54 patients (9.3\%) who received etoposide phosphate and 52 of 192 patients (27.1\%) who received etoposide ( Etoposide was associated with more hypersensitivity than etoposide phosphate in pediatric patients. Etoposide hypersensitivity was associated with higher infusion rates, but not etoposide phosphate. Differences in hypersensitivity incidence and infusion rate influence indicate a formulation-effect. Etoposide hypersensitivity recurrence may be prevented by changing to etoposide phosphate formulation. During etoposide phosphate shortages, etoposide desensitization may prevent recurrent hypersensitivity. [\hyperlink{Etoposide}{PMID: 31315549}, Winifred M Stockton et al., 2020]

\hypertarget{pmid_9038612}{E}toposide is one of the most important drugs available for the treatment of paediatric malignancies. Although there is evidence of schedule dependency for etoposide therapy in adults with small-cell lung cancer, the relevance of this observation to childhood cancers is uncertain. Prolonged parenteral or oral etoposide therapy has not yet shown a clear-cut advantage over intermittent treatment, and there are still no data to show that the administration of etoposide as a short intravenous (i.v.) daily infusion for 5 days does not represent acceptable therapy for primary disease. The pharmacokinetic variability seen with etoposide argues strongly for the use of pharmacologically guided dosing, and the introduction of etoposide phosphate will simplify both parenteral etoposide administration and the future evaluation of alternative etoposide schedules. Although the impact of molecular and cellular pharmacological investigations on the clinical use of etoposide has yet to be felt, the tools to perform these studies are now available and prospective trials can be designed. Such studies, performed in the setting of a pharmacologically guided trial to ensure control over pharmacokinetic variability, should identify the best way of treating children with etoposide. [\hyperlink{Etoposide}{PMID: 9038612}, S P Lowis et al., 1996]

\hypertarget{pmid_21159109}{E}toposide (VP-16) is one of the most widely used antitumor agents in pediatric oncology as well as chemotherapeutic agents used in conditioning regimen prior to allo-HSCT for childhood ALL. This study included 21 children with ALL who underwent allo-HSCT after conditioning with FTBI and high-dose of VP-16 (60 mg/kg) given intravenously as single four-h infusion on day -3 (n=2) or day -4 (n=19) prior to allo-HSCT. Blood samples were collected at defined time intervals until 120 h elapsed from the end of infusion. VP-16 plasma concentrations were determined using validated HPLC method. Three-compartment model was assumed for assessing PK parameters of VP-16. The median value of VP-16 C(max) measured at the end of infusion was 188.0 μg/mL (range 148.0-407.0 μg/mL). Out of 21 studied children, VP-16 was still detectable in 17 patients 72 h (median concentration 0.31 μg/mL) and in eight patients 96 h (median concentration 0.31 μg/mL) after the end of infusion. VP-16 concentration 96 h after the end of infusion was positively correlated with VP-16 AUC and negatively correlated with VP-16 CL normalized to body weight. [\hyperlink{Etoposide}{PMID: 21159109}, Maria Chrzanowska et al., 2011]

\hypertarget{pmid_1449115}{T}en children with Langerhans cell histiocytosis (LCH) were treated with etoposide. For five patients, this was the initial diagnosis. The other five had failed to respond to previous therapies. Etoposide (100 mg/m2) was given intravenously twice a week for 4 weeks, followed by maintenance therapy every 2 to 4 weeks for 2 years. All 10 patients responded to etoposide, and 6 of them (60\%) have been in complete remission for 3 to 36 months without any side effects. One patient relapsed with diabetes insipidus, one with a soft tissue mass, and two others developed multiple bone lesions. Chemotherapy with etoposide appears to be effective and safe for the treatment of children with systemic LCH. [\hyperlink{Etoposide}{PMID: 1449115}, E Ishii et al., 1992]

\hypertarget{pmid_12654074}{P}harmacokinetics of etoposide in Japanese children and adolescents has not been investigated. The objectives of the present study were (i) to document the pharmacokinetics of etoposide in Japanese children; (ii) to determine the intra- and interpatient variability in systemic etoposide exposure and (iii) to obtain insights into the age-pharmacokinetic parameter relationship. Pharmacokinetic studies of etoposide, given at doses of 60-200 mg/m2 by intravenous (i.v.) route of administration, were conducted in 18 children and adolescents (aged <19 years) with malignant diseases. High performance liquid chromatography was used to measure the blood etoposide levels. Pharmacokinetic parameters (mean\textasciitilde{}SD) of the 14 patients (24 courses) who received etoposide 100 mg/m2 were as follows: peak serum concentration (Cmax), 18.5\textasciitilde{}6.4 microg/mL; trough serum concentration, 0.2\textasciitilde{}0.1 microg/mL; biological half-life (T1/2), 3.6\textasciitilde{}0.7 h; volume of distribution (Vd), 6.3\textasciitilde{}3.4 L/m2; area under the etoposide serum concentration-time curve (AUC), 129\textasciitilde{}38 hr x microg/mL; systemic clearance, 21.1\textasciitilde{}10.8 mL/min per m2. The T1/2, Vd, and AUC were not associated with age. An increase in etoposide dose per body surface area (BSA) was associated with increase in its Cmax and area under the time-concentration curve (AUC). Wide interpatient variability in these parameters was demonstrated. The present study demonstrated that: (i) Pharmacokinetics of etoposide in Japanese children and adolescents were similar to those in Caucasians. (ii) Increased exposure to etoposide was associated with the Cmax. A clear correlation between Cmax and AUC was also found. (iii) Selecting the dose of etoposide according to body surface area (BSA) might give an acceptable range of exposure for children more than 1 year of age. [\hyperlink{Etoposide}{PMID: 12654074}, Yasuhiro Kato et al., 2003]

\hypertarget{pmid_23065812}{E}toposide (VP-16) is a hydrophobic anticancer agent inhibiting Topoisomerase II, commonly used in pediatric brain chemotherapeutic schemes as mildly toxic. Unfortunately, despite its appropriate solubilization in vehicle solvents, its poor bioavailability and limited passage of the blood-brain barrier concur to disappointing results requiring the development of new delivery system forms. In this study, etoposide formulated as a parenteral injectable solution (Teva®) was loaded into all-biocompatible poly(lactide-co-glycolide) (PLGA) or PLGA/P188-blended nanoparticles (size 110-130 nm) using a fully biocompatible nanoprecipitation technique. The presence of coprecipitated P188 on encapsulation efficacies and in vitro drug release was investigated. Drug encapsulation was determined using HPLC. Inflammatory response was checked by FACS analysis on human monocytes. Cytotoxic activity of the various simple (Teva®) or double (Teva®-loaded NPs) formulations was studied on the murine C6 and F98 cell lines. Obtained results suggest that, although noninflammatory neither nontoxic by themselves, the use of PLGA and PLGA/P188 nanoencapsulations over pre-existing etoposide formulation could induce a greatly improved cytotoxic activity. This approach demonstrated a promising perspective for parenteral delivery of VP16 and potential development of a therapeutic entity. [\hyperlink{Etoposide}{PMID: 23065812}, Maïté Callewaert et al., 2013]

\hypertarget{pmid_34679474}{T}here is a sparsity of data on the use of ethiodized poppy seed oil (EPO) contrast agent (Lipiodol) in patients. We investigated the safety of EPO in children, adolescents, and some adults for diagnostic and therapeutic interventions. All patients who underwent procedures with EPO between 1995 and 2014 were retrospectively included. Demographic characteristics, diagnosis, dose, route of administration, preparation of EPO in combination with other agents, and complications were recorded. In 1422 procedures, EPO was used for diagnostic or treatment purposes performed in 683 patients. The mean patient age was 13.4 years (range: 2 months-50 years); 58\% of patients were female. Venous malformations ( The use of an ethiodized poppy seed oil contrast agent in children, adolescents, and adults for diagnostic or therapeutic purposes is safe. [\hyperlink{Etoposide}{PMID: 34679474}, Robert K Clemens et al., 2021]

\hypertarget{pmid_16189442}{I}n this study the authors retrospectively evaluated the feasibility and effectiveness of prolonged oral etoposide therapy in children with recurrent ependymoma. Twelve ependymoma patients with documented recurrent or persistent disease were treated between May 1998 and October 2003. All patients were treated monthly with oral VP-16 administered at a dose of 50 mg/m2/d for 21 days, with a 7-day interval between cycles, for a planned minimum number of six cycles. Response (complete plus partial) after two cycles occurred in 5 of the 12 patients (41.6\%). Response plus stable disease occurred in 10 of the 12 (83.3\%), with a median duration of response or stable disease of 7 months (range 4-30). The median survival was 7 months; the 2-year progression-free survival was 16.7\%. These results emphasize that oral etoposide is an attractive option for childhood recurrent ependymomas in terms of administration, tolerability, and neuroradiologic response. [\hyperlink{Etoposide}{PMID: 16189442}, Alessandro Sandri et al., 2005]

\hypertarget{pmid_7031350}{E}toposide is a semisynthetic podophyllotoxin derivative with a broad spectrum of antitumor activity and a relatively high therapeutic index. The synergism in animal with cis-platinum, cyclophosphamide, BCNU, and cytosinarabinoside is interesting for combination regimen. Mechanisms of action are inhibition of nucleoside transfer and of DNA and RNA synthesis, single stranded breaks, inhibition of protein synthesis and of microtubular assembly. While in lower concentrations etoposide is acting cell-cycle-dependent with accumulation of cells in the G2-phase it has, in high concentrations, also a cellcycle-phase-unspecific lethal effect. Most suitable is the oral and i.v. application of etoposide in fractionated doses of 80--120 mg/m2 on 3--5 consecutive days and repetition after 21 [14--28] days. Side effects are dose-limiting bone marrow toxicity, nausea, vomiting, fever, hypotension, phlebitis, mucositis, neuropathy, cardiotoxicity, alopecia. Etoposide is one of the most active single agents in small-cell bronchus carcinoma with a remission rate of 37\% (10\% CR), and is very active in NHL (36\%), testicular carcinoma (37\%), AMML (35\%), choriocarcinoma (35\%), and neuroblastoma (29\%). The role of etoposide in combination with other active drugs in these tumors is currently investigated in bronchus and testicular carcinoma and NHL, where etoposide will belong to the drugs of the first choice in the future. [\hyperlink{Etoposide}{PMID: 7031350}, H J Schmoll et al., 1981]

\hypertarget{pmid_7628187}{T}he objectives of this study were to determine etoposide pharmacokinetics during continuous low-dose oral administration to children with solid tumors and to evaluate the relationships between parameters of etoposide systemic exposure and toxicity. In this phase I study, children were administered oral etoposide (25 to 75 mg/m2/day) for 21 days as a diluted solution of the intravenous preparation, divided into three equal daily doses. Plasma pharmacokinetics were studied on day 1 of therapy in 18 children and again on day 21 in 14 of these children. Etoposide plasma concentration-time data were fitted to a first-order absorption, two-compartment model with use of bayesian estimation. Pharmacokinetic parameter estimates from day 1 were used to estimate steady-state etoposide systemic exposure in all children. Stepwise multivariate regression was used in an exploratory manner to determine patient, laboratory, or pharmacokinetic predictors of toxicity. Although there was substantial intrapatient variability, there was no difference in the area under the concentration-time curve [AUC(0-8hr)] measured at day 21 compared with the steady-state AUC(0-8hr) estimated from day 1 pharmacokinetic parameters (p = 0.64). Degree of neutropenia was best predicted by the estimated duration that steady-state plasma etoposide concentrations were maintained above 1 microgram/ml (t > 1 microgram/ml) rather than peak plasma concentrations, AUC(0-8hr), dosage, or other patient characteristics. Assuming a bioavailability of the oral solution of approximately 50\%, the median etoposide systemic clearance was 21.4 ml/min/m2, a value similar to clearance estimates after intravenous etoposide in pediatric populations. We conclude that a parameter reflective of etoposide systemic exposure (t > 1 microgram/ml) correlates more strongly with neutropenia than does dosage or other patient characteristics. [\hyperlink{Etoposide}{PMID: 7628187}, D S Sonnichsen et al., 1995]

\hypertarget{pmid_8162893}{E}toposide (VP 16-213), the epipodophyllotoxin derivative that is widely used in the treatment of cancer, forms complexes with DNA-topoisomerase type II alpha to exert its cytotoxicity. The drug was evaluated in vivo in Swiss albino mouse bone marrow cells for its ability to induce clastogenicity and sister chromatid exchanges (SCEs). Doses of 5, 10, 15, and 20 mg/kg body weight etoposide given intraperitoneally induced a dose-dependent significant increase of clastogenicity (Trend test, alpha < or = 0.05). The aberrations induced were predominantly chromatid types. The drug shows specificity for S-phase cells: cells harvested 6 and 12 hr posttreatment showed a significantly increased number of damaged cells and aberrations per cell. Doses of 0.5, 1.0, 2.5, 5.0, and 10.0 mg etoposide/kg body weight induced a dose-dependent significant induction of SCEs (Trend test, alpha < or = 0.05). The minimal effective concentration was 0.5 mg/kg body weight. Etoposide significantly prolonged the cell cycle time at all concentrations tested: 12-13 hr in treated animals vs. 11 hr in control. The results confirm in vivo cell cycle phase specificity of the drug and further designate etoposide as a potent clastogen and a genotoxic agent in mice. [\hyperlink{Etoposide}{PMID: 8162893}, K Agarwal et al., 1994]

\hypertarget{pmid_6326063}{E}toposide (VP 16) is a semi-synthetic derivative of 4'- demethylepipodophyllotoxin , a naturally occurring compound synthesized by the North American May apple (Podophyllum peltatum ) and the Indian species Podophyllum emodi Wallich . Although podophyllotoxins are classical spindle poisons causing inhibition of mitosis by blocking mitrotubular assembly, etoposide inhibits cell cycle progression at a premitotic phase (late S and G2), probably via inhibition of DNA synthesis. There appears to be a selective antileukemic dose response relationship when compared to normal hematopoietic elements. Etoposide is effective when administered orally at about twice the recommended parenteral dosage. Schedule dependency in both animal models and clinical trials has been observed; multiple dosing over three to five consecutive days is superior to weekly single dose administration. Etoposide's dose-limiting toxicity is myelosuppression (leukopenia), which is quite predictable; alopecia and Gl toxicity (nausea, vomiting, stomatitis) occur in about 20-30\% of patients given recommended dosages. Etoposide appears to be one of the most active drugs for small cell lung cancer, testicular carcinoma (the Food and Drug Administration approved indication), ANLL and malignant lymphoma. Etoposide also has demonstrated activity in refractory pediatric neoplasms, hepatocellular, esophageal, gastric and prostatic carcinoma, ovarian cancer, chronic and acute leukemias and non-small cell lung cancer, although additional single and combination drug studies are needed to substantiate these data. Its contribution in front-line combination chemotherapeutic regimens for these cancers will be better defined in the forthcoming years. Etoposide appears to have minimal activity in breast cancer and, based on current data, it is inactive against malignant melanoma, colorectal adenocarcinoma and cancer of the head and neck, although the dosage and schedules used in many of the Phase II studies may have been suboptimal. [\hyperlink{Etoposide}{PMID: 6326063}, J A Sinkule et al., ]

\hypertarget{pmid_30885040}{H}ypersensitivity reactions to etoposide have been reported and patients have been safely transitioned to etoposide phosphate for continued therapy. However, the safety and efficacy of substituting etoposide phosphate for etoposide has not been well established in pediatric orthopedic malignancies. The aim of this study is to determine whether etoposide phosphate can be substituted for etoposide in pediatric orthopedic malignancies. A chart review of pediatric patients who developed hypersensitivity reactions to etoposide while being treated for orthopedic malignancies was performed at a large academic medical center. Three patients were identified, two with Ewing sarcoma and one with an osteosarcoma. All three patients experienced hypersensitivity reactions to their first doses of etoposide and were switched to etoposide phosphate for further therapy. After premedication, all three patients tolerated full doses of etoposide phosphate without a graded dose challenge or desensitization. Two of the patients were premedicated with diphenhydramine alone, while the third received diphenhydramine and dexamethasone. Etoposide phosphate is a potentially safe alternative for pediatric patients with orthopedic malignancies who experience etoposide hypersensitivity. However, caution is needed as there are cases of etoposide phosphate hypersensitivity. [\hyperlink{Etoposide}{PMID: 30885040}, Joel P Brooks et al., 2020]

\hypertarget{pmid_9142202}{P}re-clinical data and adult experience suggests that topoisomerase targeted anti-cancer agents may be highly schedule dependent, and efficacy may improve with prolonged exposure. To investigate this hypothesis, 28 children with recurrent brain and solid tumors were enrolled in a phase II study of oral etoposide (ETP). Patients were prescribed ETP at 50 mg/m2/ day for 21 consecutive days. Courses were repeated every 28 days pending bone marrow recovery. Evaluation of response was initially performed after 8 weeks and then every 12 weeks either by CT or MRI. Three of 4 patients with PNET (primitive neuroectodermal tumor)/medulloblastora achieved a partial response (PR). Two of 5 with ependymoma responded, one with a complete response and one with a PR. Toxicity was manageable with only 1 admission for fever and neutropenia in 120 cycles of therapy. Five patients had grade 3 or 4 neutropenia. One had grade 4 thrombocytopenia and one grade 2 mucositis and withdrew as a result. One patient had grade 2 diarrhea. Two patients who achieved a PR had received ETP as part of prior combination chemotherapy regimens. Daily oral etoposide is active in recurrent PNET/medulloblastoma and ependymoma. Toxicity is manageable and rarely requires intervention. Daily oral etoposide in combination with crosslinking agents should be considered in future phase III trials. Determination of activity in glioma and solid tumors is not complete. [\hyperlink{Etoposide}{PMID: 9142202}, M N Needle et al., 1997]

\hypertarget{pmid_7551961}{E}toposide phosphate, a water soluble prodrug of etoposide, has several potential advantages including easier and more rapid administration, avoidance of large fluid loads, and elimination of hypersensitivity reactions and other problems related to the solubilizer. This randomized Phase II study was done to evaluate the efficacy and toxicity of etoposide phosphate and etoposide, when used in combination with cisplatin in the treatment of patients with small cell lung cancer. Previously untreated small cell lung cancer patients were randomized to receive cisplatin in combination with molar equivalent does of either etoposide or etoposide phosphate. The patients were evaluated with respect to response rate, time to progression, survival, and toxicity. Response rates with etoposide phosphate and etoposide were 61\% (95\% confidence interval 55-67\%) and 58\% (95\% confidence interval 52-64\%), respectively (P = 0.85). Median time to progression was 6.9 months for patients who received etoposide phosphate and 7.0 months for those with etoposide (P = 0.50). For extensive stage disease patients, median survival with etoposide phosphate was 9.5 months versus 10 months for etoposide (P = 0.93). The corresponding median survivals for patients with limited stage disease were > 16 months and 17 months, respectively (P = 0.62). Myelosuppression was the most common toxicity; Grade 3 and 4 leukopenia occurred in 63\% of patients receiving etoposide phosphate compared with 77\% receiving etoposide (P = 0.16). The combination of etoposide phosphate and cisplatin is effective in the treatment of small cell lung cancer, and can be administered with acceptable toxicity. This study was not designed to be a formal Phase III comparative trial, but the efficacy and toxicity observed with this regimen were found to be similar to a standard etoposide/cisplatin regimen, using molar equivalent etoposide doses. Etoposide phosphate is preferable to etoposide because it is easier to use. [\hyperlink{Etoposide}{PMID: 7551961}, F A Greco et al., 1995]

\hypertarget{pmid_8523060}{M}ost pediatric treatment protocols specify dose calculations for cytostatic drugs based on body-surface area (BSA). However, for children less than 1 year of age, calculation guidelines vary. Normally, reduced dosages are recommended with calculations based on body weight (BW). However, the optimal dose for infants should take age-dependent and drug-specific pharmacokinetic parameters into account. The current investigation focused on the effects of different dose-reduction rules on the steady-state levels (Css) of etoposide and related bone marrow toxicity. One hundred seventy three treatment courses in 78 children on a 96-hour continuous infusion schedule were monitored for Css (determined by high-performance liquid chromatography [HPLC]), and 100 courses were documented in detail with regard to dose calculation (125 mg/m2, 4.17/kg, or 2/3 x 4.17/kg) and toxicity. Dose calculation on the basis of BSA led to Css of 4.9 +/- 1.2 micrograms/mL, which on the basis of BW was 3.5 +/- 1.1 micrograms/mL and 1.95 +/- 0.6 micrograms/mL (2/3.kg-dose), respectively. However, conversion of the latter levels to those expected with the regular square-meter-dose, resulted in values of 4.7 +/- 1.4 micrograms/mL and 4.2 +/- 1.2 micrograms/mL (/125 mg/m2). Lower etoposide Css levels resulted in less pronounced thrombocyte and WBC nadirs in the respective groups. The dose-calculation rules for infants, therefore, decreased both dose-intensity and related toxicity. Etoposide clearance rates related to BW (0.8 +/- 0.3 mL/min/kg) or BW (19 +/- 6 mL/min/m2) did not show any differences between children and infants, even in the age range of 3 to 12 months. In the case of etoposide, special dose-calculation guidelines for infants are not substantiated by age-dependent pharmacokinetics or tolerance. [\hyperlink{Etoposide}{PMID: 8523060}, J Boos et al., 1995]

\section*{Fenoprofen Calcium}
\subsection*{Result}
\subsubsection*{Answer}

Unknown

\subsubsection*{{Explanation}}
\hypertarget{Fenoprofen Calcium}
After reviewing all available abstracts, only two are directly about Fenoprofen Calcium:

1. \hyperlink{pmid_3496186}{PMID: 3496186}, J R Ryan et al., 1987 – This study evaluated the gastrointestinal effects of plain and enteric-coated fenoprofen calcium in 32 normal male volunteers. All subjects were adults; there is no mention of children or pediatric patients in the study population.

2. \hyperlink{pmid_781232}{PMID: 781232}, J F Fries et al., 1976 – This study compared fenoprofen calcium with acetylsalicylic acid in 27 patients with rheumatoid arthritis. The abstract does not specify the age range of the patients, but the context and terminology ("definite or classic rheumatoid arthritis") strongly suggest an adult population. There is no explicit mention of children or pediatric patients.

Neither abstract provides evidence from a targeted study of fenoprofen calcium in children, nor do they address safety in any pediatric age group. No other abstracts mention fenoprofen calcium in the context of pediatric use or safety.

Therefore, based strictly on the abstracts available, there is no evidence from targeted studies to affirm or refute the safety of fenoprofen calcium in children of any age range. The safety of fenoprofen calcium in children is unknown.

\subsection*{Abstracts}
\hypertarget{pmid_28169973}{I}n young children, the use of fecal calprotectin (fCP) as a biomarker is limited because reference values have not been widely accepted up to now. Moreover, reference values for fecal eosinophil-derived neurotoxin (fEDN) in children have not been established. The aim of the present study was to investigate fCP and fEDN levels in young healthy children to establish reference values. Stool samples were obtained from healthy children ages 0 to 12 years. fCP and fEDN levels were analyzed using the EliA Calprotectin 2 assay (Phadia AB) and a novel research assay (on the ImmunoCAP platform), respectively. In the 174 included children (87 boys), 95th Percentile values ranged from 1519 mg/kg at 0 months to 54.4 mg/kg at 144 months for fCP and from 9.9 mg/kg at 0 months to 0.2 mg/kg at 144 months for fEDN. There was a statistically significant association between age and fCP concentrations (P < 0.001) and age and fEDN concentrations (P < 0.001). We also found a statistically significant association between fEDN and fCP concentrations (rho = 0.52, P < 0.001). According to our results, we provide a nomogram and we suggest 3 different age groups for evaluation of fCP and fEDN concentrations, the 95th percentile being respectively 910.3 and 7.4 mg/kg for 0-12 months, 285.9 and 2.9 mg/kg for >1 to 4 years, and 54.4 and 0.2 mg/kg for >4 to 12 years. By using an improved analytical method, we have confirmed that young healthy children have higher fCP concentrations than healthy adults. We, for the first time, report reference values for fEDN concentrations in a pediatric population. The proposed nomograms and reference values for fCP and fEDN are aimed at facilitating the applicability of biomarkers for both neutrophil- and eosinophil-mediated intestinal inflammation in children in clinical practice. [\hyperlink{Fenoprofen Calcium}{PMID: 28169973}, María Roca et al., 2017]

\hypertarget{pmid_2507975}{T}he safety and clinical efficacy of calcium carbonate therapy in children with chronic renal failure were assessed in 68 patients (average age 8.38 years) during a mean follow-up period of 19.9 months (range 1.2-49.4). Forty-seven episodes of hypercalcaemia occurred in 29 children (3.5 episodes per 100 patient-months). There were no significant differences in mean GFR or biochemical parameters between these patients at the start of calcium carbonate therapy and the group of children who never experienced hypercalcaemia. Good control of secondary hyperparathyroidism and a significant reduction in serum aluminum were achieved. Two of 23 hypercalcaemic patients showed nephrocalcinosis on ultrasonography. 99Tc pyrophosphate scanning failed to detect any other ectopic calcification. The incidence of hypercalcaemia increased significantly when the GFR was less than 15 ml/min per 1.73 m2 and was most frequent in children receiving dialysis (48 episodes per 100 patient-months). The decrease in GFR during therapy was significantly more in the hypercalcaemic group compared to the normocalcaemic group (P less than 0.01), despite no irreversible acute effects of hypercalcaemia being observed on the rate of decline of GFR. We believe that the reduced renal homeostatic reserve is a major factor predisposing to hypercalcaemia. Consequently calcium carbonate is safe to use in children with severe chronic renal failure with close biochemical monitoring; the benefits over aluminium phosphate binders far outweigh the risks of hypercalcaemia and ectopic calcification. [\hyperlink{Fenoprofen Calcium}{PMID: 2507975}, A G Clark et al., 1989]

\hypertarget{pmid_31198713}{C}efotaxime is one of the third generation cephalosporins, which is used against many infections. This drug has a urinary excretion and potentially may have nephrotoxic effects. Hypercalciuria can cause important complications, including the formation of kidney stones. In the recent study, we decided to evaluate hypercalciuria in children receiving cefotaxime. This case-control study was conducted in Amirkabir hospital (Arak, Iran), where 30 children received intravenous cefotaxime were placed in the case group and 30 children without intravenous administration of cefotaxime were included in the control group. The ratio of calcium to creatinine was measured in both groups. Data were analyzed by SPSS software version 23. This study showed that the ratios of male and female children in both the groups were 19 (63.3\%) and 11 (36.7\%) respectively, the mean age of children in the case group was 2.36 years with a standard deviation of 0.71 and the mean age of the children in the control group was 5.18 years with a standard deviation of 3.31. The ratios of urine calcium to creatinine in the case and control groups were 0.90 with a standard deviation of 1.79 and 0.37 with a standard deviation of 0.44 ( According to the above results, it is concluded that receiving intravenous cefotaxime may increase calcium to creatinine ratio in children. [\hyperlink{Fenoprofen Calcium}{PMID: 31198713}, Zahra Kalantari et al., 2019]

\hypertarget{pmid_25135766}{R}ecently, an association between childhood growth stunting and aflatoxin (AF) exposure has been identified. In Ghana, homemade nutritional supplements often consist of AF-prone commodities. In this study, children were enrolled in a clinical intervention trial to determine the safety and efficacy of Uniform Particle Size NovaSil (UPSN), a refined calcium montmorillonite known to be safe in adults. Participants ingested 0.75 or 1.5 g UPSN or 1.5 g calcium carbonate placebo per day for 14 days. Hematological and serum biochemistry parameters in the UPSN groups were not significantly different from the placebo-controlled group. Importantly, there were no adverse events attributable to UPSN treatment. A significant reduction in urinary metabolite (AFM1) was observed in the high-dose group compared with placebo. Results indicate that UPSN is safe for children at doses up to 1.5 g/day for a period of 2 weeks and can reduce exposure to AFs, resulting in increased quality and efficacy of contaminated foods.  [\hyperlink{Fenoprofen Calcium}{PMID: 25135766}, Nicole J Mitchell et al., 2014] Xenon has minimal haemodynamic side effects when compared to volatile or intravenous anaesthetics. Moreover, in in vitro and in animal experiments, xenon has been demonstrated to convey cardio- and neuroprotective effects. Neuroprotection could be advantageous in paediatric anaesthesia as there is growing concern, based on both laboratory studies and retrospective human clinical studies, that anaesthetics may trigger an injury in the developing brain, resulting in long-lasting neurodevelopmental consequences. Furthermore, xenon-mediated neuroprotection could help to prevent emergence delirium/agitation. Altogether, the beneficial haemodynamic profile combined with its putative organ-protective properties could render xenon an attractive option for anaesthesia of children undergoing cardiac catheterization. In a phase-II, mono-centre, prospective, single-blind, randomised, controlled study, we will test the hypothesis that the administration of 50\% xenon as an adjuvant to general anaesthesia with sevoflurane in children undergoing elective cardiac catheterization is safe and feasible. Secondary aims include the evaluation of haemodynamic parameters during and after the procedure, emergence characteristics, and the analysis of peri-operative neuro-cognitive function. A total of 40 children ages 4 to 12 years will be recruited and randomised into two study groups, receiving either a combination of sevoflurane and xenon or sevoflurane alone. Children undergoing diagnostic or interventional cardiac catheterization are a vulnerable patient population, one particularly at risk for intra-procedural haemodynamic instability. Xenon provides remarkable haemodynamic stability and potentially has cardio- and neuroprotective properties. Unfortunately, evidence is scarce on the use of xenon in the paediatric population. Our pilot study will therefore deliver important data required for prospective future clinical trials. EudraCT: 2014-002510-23 (5 September 2014). [\hyperlink{Fenoprofen Calcium}{PMID: 25135766}, Sarah Devroe et al., 2015]

\hypertarget{pmid_17611334}{T}o observe the effect of sevoflurane on the induction and maintenance of anaesthesia in children, and to evaluate its safety and effectiveness. Forty child patients who conformed to the selection standard were operated under anaesthesia with intubation.Without premedicant, all the patients inhaled 100\% oxygen(1L/min) and sevoflurane by mask, and escalated the concentration of sevoflurane (to the maximum concentration 7\%) until the lash reflex disappeared, and the maintenance concentration was controlled under 4\%. All the patients were intubated, together with vecuronium 0.1mg/kg. With little tract excretion, the achievement ratio of induction by sevoflurane was 100\%, and the children tolerated well. With stable hemodynajmics,1\% approximately 4.0\% maintenance concentration of sevoflurane during the operation showed effective anaesthesia, no decreased heart rate or blood pressure appeared, and all the patients' body temperature was normal. Sevoflurane for children induction can bring fewer stimuli in the respiratory tract,less cardiac vascular inhibition and palinesthesia time. Anaesthesia in children induced by sevoflurane is safe and effective. [\hyperlink{Fenoprofen Calcium}{PMID: 17611334}, Xi-ying Zhang et al., 2007]

\hypertarget{pmid_28469850}{S}ubcutaneous fat necrosis (SFN) in infants producing severe hypercalcemia is a life-threatening emergency. Pathophysiology may include enhanced gastrointestinal calcium absorption and bone resorption. We treated an infant with SFN and serum calcium of 15 mg/dL with prednisolone and low-dose zoledronic acid. Serum calcium promptly normalized without rebound hypocalcemia, and redosing of zoledronic acid was not necessary. [\hyperlink{Fenoprofen Calcium}{PMID: 28469850}, Jeremy A Di Bari et al., 2017]

\hypertarget{pmid_36896687}{F}ecal calprotectin (FCP) is a biomarker of intestinal inflammation and has recently been proposed as a diagnostic biomarker of food allergy (FA) in children. The aim of this study was to compare FCP level in infants and children under 4 years old with 1) atopic dermatitis (AD) with food allergy (FA) and 2) children with AD and without FA with the results in healthy controls. In total, 46 infants and children (mean age 14 months ± 12) diagnosed with AD were divided into two groups: G1, children with atopic AD with FA (n=28) and G2, children with AD without FA (n=18). The control group (G3) was made up of healthy children of the same age (n=18). The median FCP was significantly higher in G1 compared with G2 (G1: median 154, IQR 416 µg/g vs G2: median 41.3, IQR 59 µg/g; P=0.0096). The median FCP in children with AD and FA was significantly higher before elimination diet in comparison with FCP after 3 months of elimination diet (median 154, IQR 416 µg/g vs median 35, IQR 23 µg/g; P=0.0039). The level of FCP was significantly positively correlated with the SCORAD score (r=0.5544, P=0.0022). Our study showed a significant difference in level of FCP in patients with AD without FA compared with patients with AD and FA. We also found a positive correlation of FCP with SCORAD score, a biomarker of AD severity. New studies are needed to investigate the role of FCP as a biomarker of FA in children with AD. [\hyperlink{Fenoprofen Calcium}{PMID: 36896687}, Alen Švigir et al., 2021]

\hypertarget{pmid_26468483}{N}ephrolithiasis is a common worldwide problem both in children and adults. Ceftriaxone as a widely used antibiotic can contribute to the formation of renal stones and hypercalciuria. To find the effect of ceftriaxone, a widely used antibiotic, on urinary calcium excretion rate in children. 84 infants and children over 3 months admitted to hospital for non-renal problems. They were all previously healthy children affected with a condition mandating hospitalisation. They were randomly divided into 2 groups; those who received ceftriaxone according to their physician decision as the case group and those who did not receive antibiotics as the control group. The patients urinary calcium excretion was determined as calcium to creatinine ratio in a random urine sample in the first and third day of their admission. All data was expressed by mean ± SD and analysed by t independent and chi-square tests by SPSS 16. P P value less than 0.05 was significant. Eighty-four cases were analysed. Calcium excretion in received and non-received ceftriaxone groups was 0.13 ± 0.06 and 0.14 ± 0.02 respectively at first day of admission ( P = 0.1). After 3 days, the urine calcium to creatinine ratio increased to 0.27 ± 0.2 and 0.26 ± 0.08 in received and non- received ceftriaxone groups ( P = 0.8). In children, urinary calcium excretion increases 2 times in average in a short time after admission because of gastroenteritis, and ceftriaxone is not different to other antibiotics for increase urinary calcium excretion in 3 days after admission. [\hyperlink{Fenoprofen Calcium}{PMID: 26468483}, Anoush Azarfar et al., 2015]

\hypertarget{pmid_37522100}{C}alcium carbonate (E 170) was re-evaluated in 2011 by the former EFSA Panel on Food Additives and Nutrient sources added to Food (ANS). As a follow-up to this assessment, the Panel on Food Additives and Flavourings (FAF) was requested to assess the safety of calcium carbonate (E 170) for its uses as a food additive in food for infants below 16 weeks of age belonging to food category 13.1.5.1 (Dietary foods for infants for special medical purposes and special formulae for infants) and as carry over in line with Annex III, Part 5 Section B to Regulation (EC) No 1333/2008. In addition, the FAF Panel was requested to address the issues already identified during the re-evaluation of the food additive when used in food for the general population. The process involved the publication of a call for data to allow the interested business operators (IBOs) to provide the requested information to complete the risk assessment. The Panel concluded that there is no need for a numerical acceptable daily intake (ADI) for calcium carbonate and that, in principle, there are no safety concern with respect to the exposure to calcium carbonate  [\hyperlink{Fenoprofen Calcium}{PMID: 37522100},  et al., 2023] Calprotectin is a protein abundant in neutrophils. Fecal calprotectin can be used as a marker of gastrointestinal inflammation, and an improved assay has recently been developed. The aim of this study was to establish reference values for fecal calprotectin in healthy children aged between 4 and 17 years. Fecal samples were obtained from 117 healthy children classified into four age groups: 4 to 6 years, 7 to 10 years, 11 to 14 years, and 15 to 17 years. A health questionnaire was used to ensure that these children fulfilled the inclusion criterion and did not have intercurrent disease, nasal or menstrual bleeding, or nonsteroidal anti-inflammatory drug medication before the sampling period. Calprotectin was analyzed using a quantitative enzyme-linked immunosorbent assay (Calprest, Eurospital SpA, Trieste, Italy). Children with fecal calprotectin values >50 microg/g were asked to deliver an additional sample. The overall median fecal calprotectin concentration was 13.6 microg/g (95\% confidence interval, 9.9-19.5 microg/g) in the 117 children. In the different age groups, 4 to 6 years, 7 to 10 years, 11 to 14 years, and 15 to 17 years, the median calprotectin concentrations were 28.2, 13.5, 9.9, and 14.6 microg/g, respectively. Of these children, 104 (89\%) had a concentration <50 microg/g. The remaining 13 children with a calprotectin concentration >50 microg/g delivered one additional fecal sample. All showed a lower concentration in the second sample except for one teenager who later proved to have proctitis. The suggested cutoff level for adults (<50 microg/g) can be used for children aged from 4 to 17 years regardless of sex. A fecal calprotectin concentration >50 microg/g warrants follow-up. [\hyperlink{Fenoprofen Calcium}{PMID: 37522100}, Ulrika Lorentzon Fagerberg et al., 2003]

\hypertarget{pmid_3701525}{O}rally administered calcium carbonate was evaluated as a phosphate binding agent in 15 children, ages 0.6 to 17.2 years, receiving maintenance dialysis. Changes in plasma aluminum concentration were assessed after discontinuation of treatment with aluminum-containing gels. The mean daily dose of calcium carbonate was 5.1 +/- 2.5 gm (384 +/- 315 mg/kg/day), and correlated inversely with body weight (r = 0.72, P less than 0.01) and age (r = 0.71, P less than 0.01). Mean serum calcium, phosphorus, and bicarbonate values were unchanged throughout the study. Plasma aluminum concentration fell from 90 +/- 51 to 34 +/- 22 micrograms/L (P less than 0.005). Dietary phosphorus intakes were 44 +/- 21 and 42 +/- 19 mg/kg/day during the control period and at the end of the study, respectively. Transitory hypercalcemia was the only side effect in 92\% of the patients. In none of the patients did uncontrolled hyperphosphatemia, metabolic alkalosis, diarrhea, or symptoms or signs of hypercalcemia develop. Our data indicate that calcium carbonate is an effective phosphate binding agent in children receiving dialysis, and should be used in lieu of aluminum-containing gels in young children with renal failure. [\hyperlink{Fenoprofen Calcium}{PMID: 3701525}, I B Salusky et al., 1986]

\hypertarget{pmid_23534952}{A} test dose is used to detect intravascular injection during neuraxial block in pediatrics. Accidental intravascular epidural local anesthetic injection might be unrecognized in anesthetized children leading to potential life-threatening complications. In children, sevoflurane anesthesia blunts the hemodynamic response when intravascular cathecolamines are administered. No studies have explored the hemodynamics and the criteria for a positive test dose result following phenylephrine in sevoflurane anesthetized children. Healthy children undergoing minor procedures were randomly assigned to receive intravenous placebo, or 5 μg∙kg(-1) phenylephrine (n = 11/group) during sevoflurane anesthesia. Hemodynamic response was assessed using electrocardiography, pulse oxymetry and non-invasive blood pressure monitoring for 5 min following drug administration in anesthetized patients. All patients receiving phenylephrine showed a decreased heart rate (HR) but not all of them met the positive criterion for test dose response. Overall, at 1 min, patients receiving phenylephrine showed a 25\% decrease in HR from the baseline while an increase in blood pressure was noticed in 54\% of patients receiving phenylephrine. Phenylephrine might be a future indicator of positive intravascular test dose. Further investigation is needed to find out the phenylephrine dose that elicits a reliable hemodynamic response and whether phenylephrine needs to be dose age-adjusted in order to appreciate relevant hemodynamic changes in children receiving neuraxial blocks undergoing general anesthesia. [\hyperlink{Fenoprofen Calcium}{PMID: 23534952}, Carlo Pancaro et al., 2013]

\hypertarget{pmid_318975}{T}he effect of calcium salt of fosfomycin in the treatment of 43 neonates suffering from acute gastroenterocolitis produced by enteropathogenic E. coli is evaluated. The minimal inhibitory concentration of these E. coli was, generally, lower than 128 mug/ml. Dosages of 150-200 mg/kg body weight/day were administered orally every 8 h. This treatment lasted for 4 days only. Clinical evolution was favorable in 38 (88\%) babies and bacteriological evolution in 30 (70\%). In eight cases a different flora to the initial was selected during the treatment with fosfomycin. None of the cases treated showed any toxic alteration attributed to the antibiotic. [\hyperlink{Fenoprofen Calcium}{PMID: 318975}, C G Taylor et al., 1977]

\hypertarget{pmid_1865281}{U}sing a stable isotopic technique in which 42Ca was administered via a bolus injection, we measured endogenous fecal calcium excretion, Vf, in five healthy children, aged 3-14 years. The Vf averaged 1.4 +/- 0.4 mg/kg/day, and was lower than urinary Ca excretion (Vu) in four of the five children. These results for Vf are consistent with previously reported results for Vf in healthy adults and much lower than those reported in premature infants. These results may be useful in understanding developmental changes in Ca metabolism and in interpreting dual tracer Ca isotope studies in children. [\hyperlink{Fenoprofen Calcium}{PMID: 1865281}, S A Abrams et al., 1991]

\hypertarget{pmid_33239728}{R}eference values of fecal calprotectin (fCP) have not been convincingly established in children. We aimed to investigate fCP concentrations in a larger population of healthy children aged 4-16 years to analyze more in depth the behavior of fCP in this age range and to determine if cut-off levels could be conclusively recommended. A prospective study was conducted to investigate fCP concentrations of healthy children aged 4-16 years. In 212 healthy children, the median and 95th percentile for fCP were 18.8 mg/kg and 104.5 mg/kg, respectively. We found a statistically significant association between the 95th percentile of fCP concentrations and age (p < 0.001). We propose a nomogram to facilitate the interpretation of fCP results in children aged 4-16 years. Further studies are required to validate the proposed values in clinical practice. [\hyperlink{Fenoprofen Calcium}{PMID: 33239728}, María Roca et al., 2020]

\hypertarget{pmid_10847237}{T}he safety of use of the calcium channel blocker nifedipine in pregnancy as it affects child development has not been well evaluated. We report the results, with regard to the safety for children of use of nifedipine in pregnancy, on children followed up at 18 months of age born from women recruited in a study comparing routine treatment with nifedipine compared with no treatment. [\hyperlink{Fenoprofen Calcium}{PMID: 10847237}, R Bortolus et al., 2000]

\hypertarget{pmid_10357743}{A}lthough additional dietary calcium is recommended frequently to reduce the risk of lead poisoning, its role in preventing lead absorption has not been evaluated clinically. The objective was to determine the safety and to estimate the size of the effect of calcium- and phosphorus-supplemented infant formula in preventing lead absorption. One hundred three infants aged 3.5-6 mo were randomly assigned to receive iron-fortified infant formula (465 mg Ca and 317 mg P/L) or the same formula with added calcium glycerophosphate (1800 mg Ca and 1390 mg P/L) for 9 mo. There was no significant difference between groups in the mean ratio of urinary calcium to creatinine, serum calcium and phosphorus, or change in iron status (serum ferritin, total iron binding capacity). At month 4, the median (+/-SD) increase from baseline in blood lead concentration for the supplemented group was 57\% of the increase for the control group (0.04 +/- 0.09 compared with 0.07 +/- 0.10 micromol/L; P = 0.039). This effect was attenuated during the latter half of the trial, with an overall median increase in blood lead concentration from baseline to month 9 of 0.12 +/- 0.13 micromol/L for the control group and 0.10 +/- 0.18 micromol/L for the supplemented group (P = 0.284). Supplementation did not have a measurable effect on urinary calcium excretion, calcium homeostasis, or iron status. The significant effect on blood lead concentrations during the first 4 mo was in the direction expected; however, because this was not sustained throughout the 9-mo period we cannot conclude that the calcium glycerophosphate supplement prevented lead absorption in this population. [\hyperlink{Fenoprofen Calcium}{PMID: 10357743}, J D Sargent et al., 1999]

\hypertarget{pmid_2127076}{T}he prevention of osteopenia and frequency of renal and intestinal side effects of mineral supplementation was studied in 24 preterm infants with birth weight under 1,500 g, prospectively (gestational age 26-34 weeks). Calcium intake varied from 2.5 vs. 3.75 vs. 5 mmol/kg/day, phosphate was offered in dose of 2.5 mmol/kg/day. At the expected birth date 40\% of infants with low calcium dose showed an activity of serum alkaline phosphatase greater than five times the maximum adult normal value which is defined as a reliable marker; for osteopenia no infant with medium or high calcium intake reached this critical value (p = 0.03). Medium and high calcium doses resulted in an increased risk for hypercalcuria (25 vs. 50\%) (p = 0.03). Half of these infants developed typical signs of nephrocalcinosis on ultrasound examination. No significant difference of fecal fat content was observed with increased calcium intake; but more episodes of abdominal distension occurred during the first days of high calcium supplementation (p = 0.03). We conclude, that a calcium intake of 3.75 mmol/kg/day in combination with phosphate 2.5 mmol/kg/day is sufficient for adequate bone mineralization on a low level of side effects. Calcium excretion in urine has to be observed for early diagnosis of nephrocalcinosis. [\hyperlink{Fenoprofen Calcium}{PMID: 2127076}, J Kreuder et al., 1990]

\hypertarget{pmid_3496186}{T}he effects of plain and enteric-coated fenoprofen calcium (Nalfon, Dista, Indianapolis, Ind.) on gastrointestinal microbleeding were studied in 32 normal male volunteers in a randomized, open-label, parallel trial at two inpatient research facilities. A 1-week placebo (baseline) period preceded 2 weeks of fenoprofen therapy (enteric coated or plain, 600 mg q.i.d.). Fecal blood loss was measured by 51Cr-tagged erythrocyte assay and averaged over days 4 to 7 (baseline) and 11 to 14 and 18 to 21 (active therapy). At one center gastrointestinal irritation was evaluated endoscopically before and after active therapy. Endoscopy showed both formulations to cause mucosal damage not evident by subject-reported symptoms. Four of the 16 subjects developed asymptomatic duodenal ulcers. Mean daily fecal blood loss was significantly lower (P = 0.03) with enteric-coated (mean +/- SD, 1.104 +/- 0.961 ml/day) than with plain fenoprofen calcium (mean +/- SD, 1.686 +/- 0.858 ml/day), suggesting that tolerance of fenoprofen can be improved with administration in an enteric-coated form. [\hyperlink{Fenoprofen Calcium}{PMID: 3496186}, J R Ryan et al., 1987]

\hypertarget{pmid_23668874}{H}ypocalcemia is a common, treatable cause of neonatal seizures. A term girl neonate with no apparent risk factors developed seizures on day 5 of life, consisting of rhythmic twitching of all extremities in a migrating pattern. Physical examination was normal except for jitteriness. Laboratory evaluation was unremarkable except for decreased total and ionized serum calcium levels and an elevated serum phosphorus level. The mother had ingested 3-6 g of calcium carbonate daily during the final 4 months of pregnancy to control morning sickness. The baby's electroencephalogram showed multifocal interictal sharp waves and intermittent electrographic seizures consisting of focal spikes in the left hemisphere accompanied by rhythmic jerking of the right arm and leg. Treatment with intravenous calcium gluconate over several days resulted in cessation of seizures and normalization of serum calcium. The child has remained seizure free and is normal developmentally at 9 years of age. Hypocalcemic seizures in this newborn were likely secondary to excessive maternal calcium ingestion, which led to transient neonatal hypoparathyroidism and hypocalcemia. Inquiry about perinatal maternal medication use should include a search for over-the-counter agents that might not be thought of as "drugs," as in this case, antacids. [\hyperlink{Fenoprofen Calcium}{PMID: 23668874}, Jenna F Borkenhagen et al., 2013]

\hypertarget{pmid_781232}{F}enoprofen calcium was compared with acetylsalicylic acid in the treatment of 27 patients with definite or classic rheumatoid arthritis, over a period of one year. Both drugs appeared efficacious, with a slight edge to fenoprofen in the doses employed. Fewer side effects were noted with fenoprofen. Effectiveness continued undiminished throughout the year, and mean values of most parameters continued to improve in both groups over the entire period. Three problems which influence extrapolation of results from this and similar studies to the general setting are discussed. (1) Individual patients show great variation from the mean and from one observation point to another. Thus, expectations developed from mean values will seldom be accurate in a particular patient. (2) The relative doses chosen for two drugs in the clinical trial can profoundly influence both efficacy and toxicity. The qualification "at the doses employed" is seldom emphasized in clinical reports. (3) Patient compliance in the general clinical setting is importantly different from that in a clinical trial, and this potential problem must be assessed by the physician choosing an appropriate medication for a particular patient. [\hyperlink{Fenoprofen Calcium}{PMID: 781232}, J F Fries et al., 1976]

\hypertarget{pmid_33106892}{P}ediatric patients with advanced chronic kidney disease (CKD) are often prescribed oral phosphate binders (PBs) for the management of hyperphosphatemia. However, available PBs have limitations, including unfavorable tolerability and safety. This phase 3, multicenter, randomized, open-label study investigated safety and efficacy of sucroferric oxyhydroxide (SFOH) in pediatric and adolescent subjects with CKD and hyperphosphatemia. Subjects were randomized to SFOH or calcium acetate (CaAc) for a 10-week dose titration (stage 1), followed by a 24-week safety extension (stage 2). Primary efficacy endpoint was change in serum phosphorus from baseline to the end of stage 1 in the SFOH group. Safety endpoints included treatment-emergent adverse events (TEAEs). Eighty-five subjects (2-18 years) were randomized and treated (SFOH, n = 66; CaAc, n = 19). Serum phosphorus reduction from baseline to the end of stage 1 in the overall SFOH group (least squares [LS] mean ± standard error [SE]) was - 0.488 ± 0.186 mg/dL; p = 0.011 (post hoc analysis). Significant reductions in serum phosphorus were observed in subjects aged ≥ 12 to ≤ 18 years (LS mean ± SE - 0.460 ± 0.195 mg/dL; p = 0.024) and subjects with serum phosphorus above age-related normal ranges at baseline (LS mean ± SE - 0.942 ± 0.246 mg/dL; p = 0.005). Similar proportions of subjects reported ≥ 1 TEAE in the SFOH (75.8\%) and CaAc (73.7\%) groups. Withdrawal due to TEAEs was more common with CaAc (31.6\%) than with SFOH (18.2\%). SFOH effectively managed serum phosphorus in pediatric patients with a low pill burden and a safety profile consistent with that reported in adult patients. [\hyperlink{Fenoprofen Calcium}{PMID: 33106892}, Larry A Greenbaum et al., 2021]

\hypertarget{pmid_21172879}{T}o evaluate the efficacy of low-dose chemotherapy in infants with nonmetastatic and unresectable neuroblastoma (NB) without MYCN amplification. Infants with localized NB and no MYCN amplification were eligible in the SIOPEN Infant Neuroblastoma European Study 99.1 study. Primary tumor was deemed unresectable according to imaging defined risk factors. Diagnostic procedures and staging were carried out according to International Staging System recommendations. Children without threatening symptoms received low-dose cyclophosphamide (5 mg/kg/d × 5 days) and vincristine (0.05 mg/kg at day 1; CyV), repeated once to three times every 2 weeks until surgical excision could be safely performed. Children with either one threatening symptom or insufficient response to CyV were given carboplatin and etoposide (CaE), sometimes followed by vincristine, cyclophosphamide, and doxorubicin. No postoperative treatment was to be administered. Between December 1999 and April 2004, 120 infants were included in the study. Eighty-eight had no threatening symptoms and 79 received CyV. CaE was given to 49 of them because of insufficient response. Thirty-two children had threatening symptoms, 30 of whom received CaE. Anthracyclines were given to 46 children. Surgery was attempted in 102 patients, leading to gross surgical excision in 93. Relapse occurred in 12 patients (nine local and three metastatic). Five-year overall and event-free survivals were 99\% ± 1\% and 90\% ± 3\%, respectively, with a median follow-up of 6.1 years (range, 1.6 to 9.1). Low-dose chemotherapy without anthracyclines is effective in 62\% of infants with an unresectable NB and no MYCN amplification, allowing excellent survival rates without jeopardizing their long-term outcome. [\hyperlink{Fenoprofen Calcium}{PMID: 21172879}, Hervé Rubie et al., 2011]

\hypertarget{pmid_36989008}{P}ediatric septic shock is a life-threatening condition with significant rates of morbidity and mortality. Standard management includes fluid resuscitation, timely antimicrobial administration, and epinephrine or norepinephrine if unresolved with initial management. Additional therapies are not well defined and include vasopressin, hydrocortisone, phenylephrine, levosimendan, dopamine, and others. Many of these agents modify cellular effects of calcium in the smooth muscle. The use of a calcium infusion may improve vasoactivity in the smooth muscle without the use of signaling pathways. Children are more susceptible to the effects of calcium, which may predispose them to enhanced vasoconstriction with the administration of intravenous calcium. We present a case in which a patient on chronic calcium channel blocker therapy presented with septic shock. She continued to remain hypotensive after fluid resuscitation, antibiotics, epinephrine, and norepinephrine. Her blood pressure improved with the initiation of a continuous calcium chloride infusion. Norepinephrine and epinephrine doses were decreased after the initiation of the calcium infusion. [\hyperlink{Fenoprofen Calcium}{PMID: 36989008}, Aviva J Whelan et al., 2022]

\section*{Valbenazine Tosylate}
\subsection*{Result}
\subsubsection*{Answer}

Unknown

\subsubsection*{{Explanation}}
\hypertarget{Valbenazine Tosylate}
A review of the available abstracts reveals the following regarding the safety of Valbenazine Tosylate in children:

- Several abstracts report on the safety, efficacy, and pharmacokinetics of valbenazine in adults, including long-term studies and studies in specific populations (e.g., Japanese, Korean, and American adults) [\hyperlink{pmid_29141124}{PMID: 29141124}, Stewart A Factor et al.; \hyperlink{pmid_36114799}{PMID: 36114799}, Jun Horiguchi et al., 2022; \hyperlink{pmid_32419679}{PMID: 32419679}, Jean-Pierre Lindenmayer et al., 2021; \hyperlink{pmid_36514192}{PMID: 36514192}, Woo Kyung Chung et al., 2023; \hyperlink{pmid_28320223}{PMID: 28320223}, Robert A Hauser et al., 2017; \hyperlink{pmid_29338466}{PMID: 29338466}, Harini Sarva et al., 2018].
- One abstract notes that valbenazine is in phase 2 development for Tourette syndrome in the USA, but does not provide any data or results from pediatric studies [\hyperlink{pmid_28578484}{PMID: 28578484}, Esther S Kim et al., 2017].
- No abstract describes a targeted safety study of valbenazine (or valbenazine tosylate) in children or adolescents, nor do any report results from pediatric clinical trials or case series.
- All safety and efficacy data are limited to adult populations, and there is no evidence from the abstracts that valbenazine has been studied for safety in any pediatric age group.

Therefore, based strictly on the abstracts available, the safety of Valbenazine Tosylate in children is unknown, as no targeted pediatric safety studies have been conducted or reported.

\subsection*{Abstracts}
\hypertarget{pmid_28839339}{V}albenazine (VBZ) is a vesicular monoamine transporter 2 (VMAT2) inhibitor approved for the treatment of tardive dyskinesia. The safety, tolerability and pharmacokinetics of VBZ following single and repeat once-daily (QD) dosing were evaluated in 2 randomized, single-center, double-blind studies in healthy male subjects. In the first study, 2 cohorts of 8 subjects were administered single doses (SD) of placebo (PBO; N = 2/period) or VBZ (N = 6/period; 1, 2, 5, or 12.5 mg for Cohort 1 and 12.5, 25, 50, or 75 mg for Cohort 2) using a sequential escalation scheme. The second study consisted of 2 phases. In the initial phase, subjects were administered SD PBO (N = 2/period) or VBZ (N = 6/period; 75, 100, 125 or 150 mg) with sequential escalation. In the second phase, subjects received PBO, or 50 or 100 mg VBZ (N = 4:8:8) QD for 8 days (Cohort 1) or PBO or 50 mg VBZ (N = 6:6) QD for 8 days (Cohort 2). For both studies, plasma concentrations of VBZ and its active metabolite, NBI-98782, were determined. Safety was assessed throughout the studies. PK parameters were determined using noncompartmental methods. In both studies, VBZ was rapidly absorbed with peak concentrations typically observed within 1.5 hours. Peak NBI-98782 concentrations were typically observed at 4.0 to 9.0 hours. Terminal elimination half-life for both VBZ and NBI-98782 was \textasciitilde{}20 hours. Across the 1 to 150 mg SD range evaluated across the studies, VBZ and NBI-98782 C [\hyperlink{Valbenazine Tosylate}{PMID: 28839339}, Rosa Luo et al., 2017] We report an open-label study of 25 children with complex partial seizures that assessed the pharmacokinetics and safety of a single dose of approximately 0.1 mg/kg tiagabine. The children received their usual individualized regimen of one concomitant antiepilepsy drug (AED) throughout the study. Seventeen children were receiving an inducing AED (carbamazepine or phenytoin); eight were receiving valproate. Tiagabine was well tolerated. Dose-normalized Cmax was higher in children taking valproate (18.2 +/- 5.0 ng/mL/mg) than in the induced children (14.8 +/- 6.9 ng/mL/mg), but the difference was not statistically significant. Dose-normalized area under the plasma concentration-time curve from time zero to infinite time was significantly higher (p = 0.002) in children taking valproate (176.5 +/- 54.7 ng.hr/mL/mg) than in induced children (92.4 +/- 56.7 ng.hr/mL/mg). Similarly, oral clearance in the children taking valproate (96 +/- 39 mL/min) was half that of the induced children (207 +/- 91 mL/min). Half-life in children taking valproate (5.7 hr) was almost twice that for the induced children (3.2 hr), and the elimination rate constant was significantly lower (p < 0.02) for the children taking valproate than for the induced children. Volume of distribution was similar in the children taking valproate (52 +/- 9 L) and the induced children (59 +/- 29 L). This is consistent with observations in adults taking tiagabine with inducing AEDs or valproate. Exploratory regressions on these data in children and previous data in adults showed fairly strong relationships between body size and tiagabine clearance and volume of distribution, with body size explaining about 40 to 50\% of the variability. When adjusted per kg body weight, clearance and volume were greater in children than adults. When adjusted per m2 body surface area, clearance and volume were more similar in adults and children. [\hyperlink{Valbenazine Tosylate}{PMID: 28839339}, L E Gustavson et al., 1997]

\hypertarget{pmid_12174005}{T}o compare single dose oral ivermectin with topical benzyl benzoate for the treatment of paediatric scabies. An observer-blinded randomized controlled trial was undertaken at Vila Central Hospital, Vanuatu. One hundred and ten children aged from 6 months to 14 years were randomized to receive either ivermectin 200 micro g/kg orally or 10\% benzyl benzoate topically. Follow up was at 3 weeks post-treatment. Primary outcome measures were the number of scabies lesions, the itch visual analogue score and nocturnal itch. Secondary outcome measures were the skin's reaction to treatment, the passage of worms in stool and other side effects. Eighty patients completed the study protocol. There was no significant difference between the two treatments; both produced a significant decrease in the number of scabies lesions seen at follow up. Ivermectin cured 24 out of 43 patients (56\%), and benzyl benzoate 19 out of 37 patients (51\%) at 3 weeks post-treatment. No serious side effects were noted with either treatment, but benzyl benzoate was more likely to produce local skin reactions (P = 0.004, OR 6.4, 95\% CI 1.6-25.0) Ivermectin is cheap and effective in the treatment of paediatric scabies. Ivermectin has minimal observed toxicity and has the additional beneficial effects of antiparasitic action in onchocerciasis, filariasis and strongyloidiasis. Ivermectin is better than benzyl benzoate for the treatment of paediatric scabies in developing countries. [\hyperlink{Valbenazine Tosylate}{PMID: 12174005}, P A Brooks et al., 2002]

\hypertarget{pmid_28590988}{V}ilazodone hydrochloride is the first member in a new class of antidepressants called indolealkylamines and was approved for use in the United States in 2011 for major depressive disorder. It has a combined mechanism of action of a selective serotonin reuptake inhibitor and a partial agonist of serotonin 5-HT1A receptors. It has not been approved for use in the pediatric population, and toxicity from exploratory vilazodone ingestion has been rarely described to date. We describe 2 children with laboratory-confirmed vilazodone ingestions that led to significant toxicity including refractory status epilepticus in 1 patient and likely transient seizure activity in the other. Both patients required multiple doses of benzodiazepines; in the more severe case, barbiturates were added to control seizure activity. These children returned to baseline and had no prolonged neurologic complications. Pediatric experience with vilazodone is limited; however, the literature demonstrates 3 additional case reports of children experiencing seizure after vilazodone ingestion. With the 2 new cases presented here, it seems prudent to educate prescribers and families of the potential dangers of ingestion of vilazodone tablets by young children. [\hyperlink{Valbenazine Tosylate}{PMID: 28590988}, Jeannine Del Pizzo et al., 2018]

\hypertarget{pmid_29141124}{V}albenazine, a highly selective vesicular monoamine transporter 2 inhibitor, is approved for the treatment of tardive dyskinesia. This is the first report of long-term effects in adults with tardive dyskinesia. Participants with a DSM-IV diagnosis of schizophrenia, schizoaffective disorder, or a mood disorder who completed the 6-week, double-blind, placebo-controlled period of KINECT 3 were eligible to enter the 42-week valbenazine extension (VE) period and subsequent 4-week washout period. The extension phase was conducted from December 16, 2014, to August 3, 2016. Participants who received placebo and entered the VE period were re-randomized 1:1 to valbenazine 80 or 40 mg while others continued valbenazine at the KINECT 3 dose. Safety assessments included treatment-emergent adverse events (TEAEs) and scales for suicidal ideation/behavior, treatment-emergent akathisia or parkinsonism, and psychiatric symptoms. Efficacy assessments included the Abnormal Involuntary Movement Scale (AIMS) and Clinical Global Impression of Change-Tardive Dyskinesia (CGI-TD). 198 participants entered the VE period, 124 (62.6\%) completed treatment (week 48), and 121 (61.1\%) completed the follow-up visit after washout (week 52). During the VE period, 69.2\% of participants had ≥ 1 TEAE, 14.6\% had a serious TEAE, and 15.7\% discontinued due to a TEAE. During washout, 13.1\% of participants experienced a TEAE. No apparent risk for suicidal ideation or behavior was found. Long-term valbenazine treatment did not appear to induce or worsen akathisia or parkinsonism. Participants generally remained psychiatrically stable during the study. AIMS and CGI-TD measures indicated sustained tardive dyskinesia improvement, with scores returning toward baseline after 4 weeks of valbenazine washout. The long-term safety and tolerability of valbenazine were generally favorable, and maintenance of treatment effect was apparent with both doses during this long-term study. ClinicalTrials.gov identifier: NCT02274558. [\hyperlink{Valbenazine Tosylate}{PMID: 29141124}, Stewart A Factor et al., ]

\hypertarget{pmid_2907857}{E}leven children with severe incapacitating generalized seizures were treated with sodium valproate and clorazepate and responded with a marked decrease in seizure frequency. Three children received clorazepate alone, either because of valproate toxicity or because of parental concern over side effects. These children, 5 males and 6 females, ranged in age from 3 to 17 years. They manifested normal to severely retarded intelligence. Although valproate levels were in the therapeutic range, seizure control was inadequate. When clorazepate was added to valproate therapy a marked reduction in seizure frequency occurred within 24 hours and became optimal within 48 to 72 hours. Side effects were minimal with the exception of a nocturnal generalized tonic-clonic seizure in a single patient. Three children were withdrawn from therapy after a year because of recurrent seizures. One patient was restarted on therapy after 6 months and seizure control improved. Clorazepate may be a useful adjunct in the treatment of primary generalized seizures in children. [\hyperlink{Valbenazine Tosylate}{PMID: 2907857}, S Naidu et al., ]

\hypertarget{pmid_36114799}{V}albenazine is approved in the US for treatment of tardive dyskinesia (TD); however, efficacy/safety data in Asian populations are lacking. We assessed the efficacy/safety of valbenazine in Japanese patients. This phase II/III, multicenter, randomized, double-blind, placebo-controlled study (NCT03176771) included adult psychiatric patients with TD, who were randomly allocated to receive placebo or valbenazine (once-daily 40- or 80-mg) for a 6-week, double-blind period, after which the placebo group was switched to valbenazine for a 42-week extension. The primary endpoint was change from baseline in Abnormal Involuntary Movement Scale (AIMS) total score at Week 6; clinical global impression of improvement of TD (CGI-TD) was also assessed. Of 256 patients, 86, 85, and 85 were allocated to the 40-mg valbenazine, 80-mg valbenazine, and placebo groups, respectively. Least-squares mean (95\% confidence interval) change from baseline in AIMS score at Week 6 was -2.3 (-3.0 to -1.7) in the valbenazine 40-mg group, -3.7 (-4.4 to -3.0) in the 80-mg group, and -0.1 (-0.8 to 0.5) in the placebo group; both treatment groups showed statistically significant improvements vs. placebo. Patients switched to valbenazine at Week 6 showed similar improvements in AIMS scores, which were maintained to Week 48. Improvements in CGI-TD scores were observed for both treatment groups vs. placebo. Incidence of adverse events was highest in the 80-mg group; common events included nasopharyngitis, somnolence, schizophrenia worsening, hypersalivation, insomnia, and tremor. The efficacy/safety profile of valbenazine was similar to that of previous clinical trials, supporting its use for TD treatment in Japanese patients. [\hyperlink{Valbenazine Tosylate}{PMID: 36114799}, Jun Horiguchi et al., 2022]

\hypertarget{pmid_32419679}{I}ndividuals with tardive dyskinesia (TD) who completed a long-term study (KINECT 3 or KINECT 4) of valbenazine (40 or 80 mg/day, once-daily for up to 48 weeks followed by 4-week washout) were enrolled in a subsequent study (NCT02736955) that was primarily designed to further evaluate the long-term safety of valbenazine. Participants were initiated at 40 mg/day (following prior valbenazine washout). At week 4, dosing was escalated to 80 mg/day based on tolerability and clinical assessment of TD; reduction to 40 mg/day was allowed for tolerability. The study was planned for 72 weeks or until termination due to commercial availability of valbenazine. Assessments included the Clinical Global Impression of Severity-TD (CGIS-TD), Patient Satisfaction Questionnaire (PSQ), and treatment-emergent adverse events (TEAEs). At study termination, 85.7\% (138/161) of participants were still active. Four participants had reached week 60, and none reached week 72. The percentage of participants with a CGIS-TD score ≤2 (normal/not ill or borderline ill) increased from study baseline (14.5\% [23/159]) to week 48 (64.3\% [36/56]). At baseline, 98.8\% (158/160) of participants rated their prior valbenazine experience with a PSQ score ≤2 (very satisfied or somewhat satisfied). At week 48, 98.2\% (55/56) remained satisfied. Before week 4 (dose escalation), 9.4\% of participants had ≥1 TEAE. After week 4, the TEAE incidence was 49.0\%. No TEAE occurred in ≥5\% of participants during treatment (before or after week 4). Valbenazine was well-tolerated and persistent improvements in TD were found in adults who received once-daily treatment for >1 year. [\hyperlink{Valbenazine Tosylate}{PMID: 32419679}, Jean-Pierre Lindenmayer et al., 2021]

\hypertarget{pmid_30010289}{O}bjective: To evaluate the effects and tolerability of vigabatrin (VGB) in children with tuberous sclerosis (TS) with infantile spasms or tonic seizures. Methods: We examined the impact of VGB on a series of 17 children with TS visiting Tohoku University Hospital in Japan during April 2010 and May 2015. To minimize potential adverse effects, VGB was given to the patients for limited 6 months with titration from 30 mg/kg/day as an initial dose. Results: Main seizure types were classified into spasms (n=10) or tonic seizures (n=7). Seizure reduction was positively associated with seizure type of infantile spasms, lower maximum dosage, younger age on VGB administration, and earlier VGB treatment after the diagnosis. Seizure type of infantile spasm was an independent favorable predictor and also associated with long-term seizure reduction. Major adverse events included psychiatric symptoms (n=7) and electroretinogram (ERG) abnormalities (n=2). All symptoms were recovered by reducing the dosage of VGB. Conclusion: VGB is effective and well tolerated as first-line treatment for TS children with infantile spasms. Our “low dosage and limited period” protocol is efficient for improving seizure control as well as minimizing the potential risks of VGB. [\hyperlink{Valbenazine Tosylate}{PMID: 30010289}, Sato Suzuki-Muromoto, et al., 2016]

\hypertarget{pmid_19589457}{T}his phase III, open-label, multicenter, outpatient study evaluated the 12-month safety of valproate using divalproex sodium sprinkle capsules for partial seizures, with or without secondary generalization, in children aged 3-10 years (n = 169). Laboratory parameters and vital signs were assessed, and the Wechsler Scales of Intelligence, the Developmental Profile-II, movement-related items from the Udvalg for Kliniske Undersøgelser, and the Behavior Assessment System for Children were administered. Efficacy was measured by the 4-week seizure rate. The most common treatment-emergent adverse events in the 169 study patients were typical childhood illnesses: pyrexia (18\%), cough (17\%), and nasopharyngitis (14\%). The most common adverse events not considered typical childhood illnesses were vomiting (14\%), tremor (9\%), somnolence (8\%), and diarrhea (8\%). Of the 169 patients, 11 (6.5\%) were hospitalized with serious treatment-emergent adverse events. Although elevated ammonia levels were observed in 31 treated patients, and mean increases in uric acid concentrations and decreases in platelets were observed, the majority of patients were asymptomatic. Except for tremor, no increases in movement-related adverse effects were observed. Small numeric improvements were reported in the Wechsler Scales and the Behavior Assessment System for Children. The safety findings in this 12-month study are generally consistent with previous reports of valproate in adult and pediatric epilepsy patients. [\hyperlink{Valbenazine Tosylate}{PMID: 19589457}, Robert A Lenz et al., 2009]

\hypertarget{pmid_36495716}{P}aediatric clinical practice for treatment of venous thromboembolism (VTE) is based on extrapolation from adult trials with minimal data on anticoagulation efficacy and safety in children. Based on EINSTEIN-Jr clinical trial data, rivaroxaban was approved to treat VTE and prevent its recurrence in children of all ages. To report the safety and efficacy of rivaroxaban use in paediatric VTE and to present real-world data, specifically about very young children. We conducted a retrospective observational study at Birmingham Children's Hospital. Data were collected from patients <16 years old who received rivaroxaban after its licensure in the period between March 2021 and June 2022. Rivaroxaban was used for treatment of acute VTE in 64 patients. Thrombosis was CVC-related in 26 patients, unprovoked in 3, while the rest had one or more risk factors for VTE. Safety and efficacy of rivaroxaban were assessed in 52 patients after excluding patients who were on current rivaroxaban treatment and those who were lost to follow up or stopped rivaroxaban due to intolerance. No bleeding events were reported, and recurrence of thrombosis occurred in only 3.6 \%. About 35 \% had normalised re-imaging, 40.3 \% improved, 9.6 \% were unchanged and 11.5 \% stopped rivaroxaban without re-imaging. Rivaroxaban was used for secondary VTE prophylaxis in 6 patients in our cohort with no recurrence of thrombosis or bleeding reports. Our real-world experience confirmed that rivaroxaban was well tolerated, effective and safe. Further real-world data and observational studies are essential to investigate the use of rivaroxaban among different risk groups. [\hyperlink{Valbenazine Tosylate}{PMID: 36495716}, Eman Hassan et al., 2023]

\hypertarget{pmid_16958131}{T}etrabenazine (TBZ), a presynaptic dopamine depletor and postsynaptic dopamine receptor blocker, is widely used for the treatment of hyperkinetic movement disorders in adults. However, reports of its use in children are limited. We review the efficacy and tolerability of TBZ therapy in 31 children with hyperkinetic movement disorders refractory to other medications. TBZ was effective in reducing the severity of movement disorders resistant to treatment with other medicines. When compared to adult patients, pediatric patients required higher doses. Side effects were similar to the adult population; however, children had a lower incidence of drug-induced Parkinsonism. [\hyperlink{Valbenazine Tosylate}{PMID: 16958131}, Samay Jain et al., 2006]

\hypertarget{pmid_37971239}{T}here are no pharmacokinetic data in children on terizidone, a pro-drug of cycloserine and a World Health Organization (WHO)-recommended group B drug for rifampicin-resistant tuberculosis (RR-TB) treatment. We collected pharmacokinetic data in children <15 years routinely receiving 15-20 mg/kg of daily terizidone for RR-TB treatment. We developed a population pharmacokinetic model of cycloserine assuming a 2-to-1 molecular ratio between terizidone and cycloserine. We included 107 children with median (interquartile range) age and weight of 3.33 (1.55, 5.07) years and 13.0 (10.1, 17.0) kg, respectively. The pharmacokinetics of cycloserine was described with a one-compartment model with first-order elimination and parallel transit compartment absorption. Allometric scaling using fat-free mass best accounted for the effect of body size, and clearance displayed maturation with age. The clearance in a typical 13 kg child was estimated at 0.474 L/h. The mean absorption transit time when capsules were opened and administered as powder was significantly faster compared to when capsules were swallowed whole (10.1 vs 72.6 min) but with no effect on bioavailability. Lower bioavailability (-16\%) was observed in children with weight-for-age z-score below -2. Compared to adults given 500 mg daily terizidone, 2022 WHO-recommended pediatric doses result in lower exposures in weight bands 3-10 kg and 36-46 kg. We developed a population pharmacokinetic model in children for cycloserine dosed as terizidone and characterized the effects of body size, age, formulation manipulation, and underweight-for-age. With current terizidone dosing, pediatric cycloserine exposures are lower than adult values for several weight groups. New optimized dosing is suggested for prospective evaluation. [\hyperlink{Valbenazine Tosylate}{PMID: 37971239}, Louvina E van der Laan et al., 2023]

\hypertarget{pmid_36514192}{V}albenazine is a selective vesicular monoamine transporter 2 (VMAT2) inhibitor approved for tardive dyskinesia treatment by the US Food and Drug Administration; its major active metabolite (NBI-98782) is a 45-fold more potent inhibitor of VMAT2 than the parent drug. This study aimed to evaluate the pharmacokinetics (PKs), safety, and tolerability and the effect of cytochrome P450 2D6 (CYP2D6) genotypes to the PKs after the administration of valbenazine in Korean participants. A randomized, double-blind, placebo-controlled, single- and multiple-dose study was conducted in healthy Korean male participants. The single-dose study was conducted for both 40 and 80 mg valbenazine and the multiple dose study was conducted for 40 mg. After a 1-week washout, the 40 mg dose group participants received valbenazine 40 mg or placebo once daily for 8 days. Serial blood samples were collected up to 96 h postdose for PK analysis. The CYP2D6 genotypes of the participants were retrospectively analyzed. A total of 50 participants were randomized, and 43 and 20 participants completed the single- and multiple-dose phases of the study, respectively. After single doses, the PK characteristics of valbenazine and its metabolites were similar between the 40 and 80 mg dose groups. After multiple doses, the mean accumulation ratios of valbenazine and NBI-98782 were \textasciitilde{}1.6 and 2.4, respectively. Plasma concentrations of valbenazine and NBI-98782 were similar between CYP2D6 normal and intermediate metabolizers. Valbenazine was well-tolerated in healthy Koreans, and its PK characteristics were similar to results previously reported in Americans. [\hyperlink{Valbenazine Tosylate}{PMID: 36514192}, Woo Kyung Chung et al., 2023]

\hypertarget{pmid_6796420}{P}henobarbital has been shown to offer effective prophylaxis against childhood febrile convulsions. However, a high percentage of children do not tolerate phenobarbital, mainly due to behavioral changes. Valproate, due to its low toxicity, appears to be an attractive alternative to phenobarbital treatment. Ninety children admitted with their first febrile convulsion were offered prophylactic treatment with either phenobarbital 3-5 mg/kg/day or valproate 20-30 mt/kg/day. Twenty-five children whose parents refused prophylactic treatment make up an untreated control group. Serum levels of the appropriate drug were measured at each follow-up visit. The three groups appear to be comparable. Twenty-one per cent of the phenobarbital treated children required discontinuation of the drug due to side effects. All the children tolerated valproate therapy. Twelve out of 25 untreated children suffered recurrences. Eight out of 33 children treated with phenobarbital suffered recurrences. Four out of 32 children on valproate therapy had recurrences. The difference between valproate treatment and no therapy at all is highly significant (p less than 0.001). Phenobarbital did not reduce the risk of recurrence. We now recommend prophylactic treatment with valproate to children with febrile seizures. [\hyperlink{Valbenazine Tosylate}{PMID: 6796420}, K Lee et al., 1981]

\hypertarget{pmid_37575009}{A} recent study has demonstrated an increased risk of neurodevelopmental disorders, including autism spectrum disorder, in individuals exposed to either valproate or topiramate monotherapy. Regulatory bodies have initiated a review to reassess the safety of topiramate exposure during pregnancy. These novel findings raise concerns regarding the recommendation of antiseizure medications in women of childbearing potential. This manuscript highlights current research defining concerns specific to the use of valproate and topiramate in women of childbearing potential. This manuscript summarizes recent findings regarding the safety of valproate and topiramate when compared to alternative therapies for the preventative treatment of migraine in women of childbearing potential. The studies included in this review were selected following a comprehensive literature review of multiple relevant databases. All studies that were published within the past 15 years were considered for inclusion. The use of valproate and topiramate in women of childbearing potential should be highly discouraged. Our recommendations include a review of current prescribing guidelines, further public education regarding the neurodevelopmental and congenital risks associated with the use of valproate and topiramate, and an appeal for further research defining the safety of alternative medications for migraine prevention when intrauterine exposure is possible. [\hyperlink{Valbenazine Tosylate}{PMID: 37575009}, William Wells-Gatnik et al., ]

\hypertarget{pmid_25145624}{O}lanzapine is frequently prescribed in young children for psychiatric conditions. It may be an option for chemotherapy-induced nausea and vomiting (CINV) control in children. The objective of this review was to describe the safety of olanzapine in children less than 13 years of age to determine if safety concerns would be a barrier to its use for CINV prevention. Electronic searches were performed in MEDLINE, EMBASE, Cochrane Central Register of Controlled Trials, Web of Science and Scopus. All studies in English reporting adverse effects associated with olanzapine use in children younger than 13 years or with a mean/median age less than 13 years were included. Adverse outcomes were synthesized for prospective studies. A total of 47 studies (17 prospective) involving 387 children aged 0.6-18 years were included; nine described olanzapine poisonings. Weight gain or sedation were reported in 78 \% [95 \% confidence interval (CI) 63-95] and 48 \% (95 \% CI 35-67), respectively. Extrapyramidal symptoms or electrocardiogram abnormalities were reported in 9 \% (95 \% CI 4-21) and 14 \% (95 \% CI 7-26), respectively. Elevation in liver function tests or blood glucose abnormalities were reported in 7 \% (95 \% CI 2-20) and 4 \% (95 \% CI 1-17), respectively. No deaths were attributed to olanzapine. No studies were identified with a primary focus on evaluating safety, and the adverse effects reported in the included studies were heterogeneous. Most adverse events associated with olanzapine use in children less than 13 years of age are of minor clinical significance. These findings support the exploration of olanzapine for the prevention of CINV in children in future trials. [\hyperlink{Valbenazine Tosylate}{PMID: 25145624}, Jacqueline Flank et al., 2014]

\hypertarget{pmid_20559276}{F}or a retrospective observational investigation based on real clinical practice of relative efficacy of valpoic acid (VPA), carbamazepine (CBZ) and topiramate (TPM) we have selected 277 patients with seizure onset before 17 years with a undoubted diagnosis of symptomatic or cryptogenic frontal lobe epilepsy (FLE), who had received treatment according to ILAE recommendations, and observation time since the last treatment change was from 2 to 10 years. Patients suspicious for idiopathic epilepsies were excluded. The groups of patient receiving CBZ, VPA and TPM did not differ significantly in presenting unfavorable prognostic factors that allowed conducting direct comparison of efficacy of the investigated drugs. Efficacy of VPA in children with FLE was higher compared with CBZ (56\% vs 22\%, p<0,01) and TPM (56\% vs 10\%, p<0,001). CBZ and TPM caused seizure aggravation rather frequently, but no aggravation was noted while VPA treatment (14\% and 17\% respectively vs 0\%, p<0,001). In case of presence of clinico-electroencephalografic and MRI signs of significant organic brain damage and seizure onset after 1 year of age VPA was most effective and TPM showed minimal effect. TPM was ineffective in case of focal cortical dysphasia and cerebral atrophy; in other lesions its efficacy was comparable with CBZ. In MRI-negative cases VPA was most effective (71\% vs 24\% for CBZ, p<0,001 and 20\% for TPM, p<0,001). Efficacy of VPA, CBZ and TPM does not change with the number of previously used antiepileptic drugs (AEDs). VPA was also most effective as a first AED (63\% vs 26\%, for CBZ, p<0,001 and 13\%, p<0,001 for TPM), as well as a second AED (50\% vs 30\% for CBZ and 7\% for TPM, p<0,05). Adverse effects were more frequent during treatment with CBZ and TPM, than VPA (20\% vs 6\%, p<0,001 and 31\% vs 6\%, p<0,05, respectively). [\hyperlink{Valbenazine Tosylate}{PMID: 20559276}, S R Boldyreva et al., 2010]

\hypertarget{pmid_19325515}{A} combination of albendazole and praziquantel was more effective than albendazole alone in destroying Taenia cysts in an animal model. There are no such studies in humans. To evaluate the efficacy and safety of a combination of albendazole and praziquantel in children with seizures and single small enhancing computerized tomographic lesions. Prospective, interventional, randomized, placebo-controlled, double blind clinical trial at a tertiary hospital in North India. : One hundred twelve children with seizures for <3 months and single lesion neurocysticercosis; 9 lost to follow-up. All children received albendazole (15 mg/kg/d) for 7 days with either praziquantel or placebo (75 mg/kg/d) for 1 day according to random allocation. Repeat CT scans were done after 1, 3, and 6 months. All children were followed up for at least 6 months. Fifty-three children received praziquantel (group A) and 50 placebo (group B). Complete resolution of lesions was seen in 60\% and 72\% of children at 3 and 6 months in group A versus 42\% and 52\% of children in group B. Nonresolution and calcification were higher in group B than in group A at 3 months (B: 28\%, 14\%; A: 12\%, 8\%) and 6 months (B: 16\%, 22\%; A: 6\%, 9\%), but the differences were not statistically significant. Seizure control and side effects were similar in the 2 groups. A combination therapy for albendazole and praziquantel was statistically comparable to sole therapy with albendazole in eradicating lesions and preventing seizures. [\hyperlink{Valbenazine Tosylate}{PMID: 19325515}, Satvinder Kaur et al., 2009]

\hypertarget{pmid_29338466}{V}albenazine is a selective VMAT2 inhibitor that the FDA approved in April 2017 for the specific treatment of tardive dyskinesia (TD), a movement disorder commonly caused by dopamine blocking agents. Valbenazine acts to decrease dopamine release, reducing excessive movement found in TD. Areas covered: This drug profile reviews the development of valbenazine and the clinical trials that led to its approval as the first treatment specific to TD. The literature search was performed with the PubMed online database. Expert commentary: Two clinical trials assessing the efficacy of valbenazine have shown the reduction of antipsychotic-induced involuntary movement. No life threatening adverse effects were found. Data from a 42-week extension study demonstrated sustained response. [\hyperlink{Valbenazine Tosylate}{PMID: 29338466}, Harini Sarva et al., 2018]

\hypertarget{pmid_1354907}{E}ven though acute poisonings with benzodiazepines are extremely common, less is known of the clinical toxicity of recent derivatives, particularly in children. 1,989 cases involving ethyle loflazepate, flunitrazepam, prazepam or triazolam recorded at the Lyons Poison Center and due to 1 compound and associated with clinical symptoms were selected for study. Children less than 16-y of age accounted for 482 cases. Sleepiness, agitation and ataxia were significantly more frequent in the children. Hypotonia was seldom observed but was indicative of severe poisoning. The dangerous toxic dose of these compounds in children is suggested to be 0.78-0.90 mg ethyle loflazepate/kg, 0.26-0.29 mg flunitrazepam/kg, 7.80-9.00 mg prazepam/kg and 0.06-0.07 mg triazolam/kg. These results are in keeping with the relatively low acute toxicity of the older benzodiazepines. [\hyperlink{Valbenazine Tosylate}{PMID: 1354907}, C Pulce et al., 1992]

\hypertarget{pmid_28320223}{T}ardive dyskinesia is a persistent movement disorder induced by dopamine receptor blockers, including antipsychotics. Valbenazine (NBI-98854) is a novel, highly selective vesicular monoamine transporter 2 inhibitor that demonstrated favorable efficacy and tolerability in the treatment of tardive dyskinesia in phase 2 studies. This phase 3 study further evaluated the efficacy, safety, and tolerability of valbenazine as a treatment for tardive dyskinesia. This 6-week, randomized, double-blind, placebo-controlled trial included patients with schizophrenia, schizoaffective disorder, or a mood disorder who had moderate or severe tardive dyskinesia. Participants were randomly assigned in a 1:1:1 ratio to once-daily placebo, valbenazine at 40 mg/day, or valbenazine at 80 mg/day. The primary efficacy endpoint was change from baseline to week 6 in the 80 mg/day group compared with the placebo group on the Abnormal Involuntary Movement Scale (AIMS) dyskinesia score (items 1-7), as assessed by blinded central AIMS video raters. Safety assessments included adverse event monitoring, laboratory tests, ECG, and psychiatric measures. The intent-to-treat population included 225 participants, of whom 205 completed the study. Approximately 65\% of participants had schizophrenia or schizoaffective disorder, and 85.5\% were receiving concomitant antipsychotics. Least squares mean change from baseline to week 6 in AIMS dyskinesia score was -3.2 for the 80 mg/day group, compared with -0.1 for the placebo group, a significant difference. AIMS dyskinesia score was also reduced in the 40 mg/day group (-1.9 compared with -0.1). The incidence of adverse events was consistent with previous studies. Once-daily valbenazine significantly improved tardive dyskinesia in participants with underlying schizophrenia, schizoaffective disorder, or mood disorder. Valbenazine was generally well tolerated, and psychiatric status remained stable. Longer trials are necessary to understand the long-term effects of valbenazine in patients with tardive dyskinesia. [\hyperlink{Valbenazine Tosylate}{PMID: 28320223}, Robert A Hauser et al., 2017]

\hypertarget{pmid_3092562}{T}he purpose of this study was to limit prophylactic treatment of children with febrile convulsions to patients who have the highest risk of recurrence. Two hundred and thirty-one children with a first febrile seizure were divided into high- and low-risk groups according to estimated risk of recurrence. All high-risk children were offered treatment with valproic acid. If this was declined they were offered treatment with diazepam instead. Low-risk children were untreated. Valproic acid and diazepam were found to be equally effective in reducing the risk of recurrence of febrile convulsions. By selecting for prophylactic treatment according to estimated risk of recurrence it is possible to reduce the rate of recurrence of febrile seizures in children at high-risk (60\%) to the same level as that of untreated low-risk children (23\%). Only about half of all children with febrile convulsions need treatment and follow-up according to these criteria. [\hyperlink{Valbenazine Tosylate}{PMID: 3092562}, K Lee et al., 1986]

\hypertarget{pmid_12784277}{T}etrabenazine (TBZ) is widely used to treat adults with hyperkinetic movement disorders; however, published experience with the drug in pediatric patients is limited. We report on 5 children with severe chorea who were treated with TBZ. TBZ effectively controlled chorea in 4 patients, and despite the need for relatively high doses, it was well tolerated. [\hyperlink{Valbenazine Tosylate}{PMID: 12784277}, Anjan Chatterjee et al., 2003]

\hypertarget{pmid_28578484}{V}albenazine (Ingrezza™) is an orally bioavailable, selective, vesicular monoamine transporter 2 (VMAT2) inhibitor being developed by Neurocrine Biosciences for the treatment of various central nervous system disorders. Valbenazine has been approved in the USA for the treatment of adults with tardive dyskinesia (TD), is at various stages of development in other countries for TD and is in phase 2 development in the USA for Tourette syndrome. This article summarizes the milestones in the development of valbenazine leading to its first global approval in the USA for the treatment of adults with TD. [\hyperlink{Valbenazine Tosylate}{PMID: 28578484}, Esther S Kim et al., 2017]

\section*{Itraconazole}
\subsection*{Result}
\subsubsection*{Answer}

Infants (including neonates and up to 1 year): Yes  
Children (1–12 years): Yes  
Adolescents (12–18 years): Yes  

\subsubsection*{{Explanation}}
\hypertarget{Itraconazole}
To determine if itraconazole is safe for use in children, I reviewed the available abstracts for studies specifically targeting pediatric populations and reporting on safety outcomes. Below is a summary of the evidence, organized by age range where possible:

Infants (including neonates and up to 1 year):
- Multiple studies and reviews report on the use of itraconazole in infants for both superficial and systemic fungal infections. One review of 45 articles concluded that itraconazole is safe and effective at 5 mg/kg/day for superficial and 10 mg/kg/day for systemic infections in infants, with adverse event profiles similar to those in older children and adults [\hyperlink{pmid_27286691}{PMID: 27286691}, Shuang Chen et al., 2016]. Several case series and small clinical trials for infantile hemangiomas also report good tolerability and only mild, limited side effects [\hyperlink{pmid_25512128}{PMID: 25512128}, Yuping Ran et al., 2015; \hyperlink{pmid_31668109}{PMID: 31668109}, Hagar Bessar et al., 2022; \hyperlink{pmid_34754871}{PMID: 34754871}, Zhe Liu et al., 2021].

Children (1–12 years):
- Several targeted studies and reviews report on the safety of itraconazole in children for various indications, including tinea capitis, onychomycosis, and as antifungal prophylaxis in neutropenic or immunocompromised children. A large study of 163 children with tinea capitis found itraconazole to be effective and safe, with only mild side effects in 6.7\% of cases [\hyperlink{pmid_15283801}{PMID: 15283801}, Gabriele Ginter-Hanselmayer et al.]. Another study of 81 children (≤12 and >12 years) receiving oral itraconazole with therapeutic drug monitoring found gastrointestinal symptoms in 15.2\% and hepatotoxicity in 6.5\%, but neither was associated with elevated drug levels, and the study concluded that higher empiric doses may be needed in children under 12 to achieve therapeutic levels [\hyperlink{pmid_29601447}{PMID: 29601447}, Ying Hua Leong et al., 2019]. An open study of 103 neutropenic children (0–14 years) using itraconazole oral solution for prophylaxis found no unexpected safety problems, with vomiting, abnormal liver function, and abdominal pain as the most common adverse events [\hyperlink{pmid_10578159}{PMID: 10578159}, A B Foot et al., 1999]. A review of systemic antifungals for onychomycosis in children ages 1–17 years found safety profiles similar to adults [\hyperlink{pmid_23278514}{PMID: 23278514}, Aditya K Gupta et al.].

Adolescents (12–18 years):
- Studies including adolescents (up to 17 or 18 years) for indications such as onychomycosis, prophylaxis after hematopoietic stem cell transplantation, and treatment of invasive fungal infections report similar safety profiles to those in younger children and adults, with no serious adverse events directly attributed to itraconazole [\hyperlink{pmid_18194238}{PMID: 18194238}, G Ginter-Hanselmayer et al., 2008; \hyperlink{pmid_24173819}{PMID: 24173819}, M Döring et al., 2014; \hyperlink{pmid_25680318}{PMID: 25680318}, M Döring et al., 2015].

General pediatric population (0–18 years):
- Multiple studies and reviews, including large cohorts and retrospective analyses, affirm the safety of itraconazole in children for both prophylaxis and treatment of fungal infections, with adverse events generally mild and similar to those seen in adults [\hyperlink{pmid_12956205}{PMID: 12956205}, Aditya K Gupta et al., 2003; \hyperlink{pmid_17430480}{PMID: 17430480}, L Grigull et al., 2007; \hyperlink{pmid_32929460}{PMID: 32929460}, Joanne Abbotsford et al., 2021]. However, one study highlights a significant drug interaction between itraconazole and vincristine, leading to enhanced neurotoxicity in children receiving both drugs, and recommends avoiding this combination [\hyperlink{pmid_16012330}{PMID: 16012330}, Mar Bermúdez et al., 2005].

Summary:
- There is substantial evidence from targeted pediatric studies (including infants, children, and adolescents) affirming the safety of itraconazole for various indications, with adverse events generally mild and similar to those in adults. The exception is the risk of severe neurotoxicity when combined with vincristine, which is a specific drug interaction rather than a general safety concern for itraconazole alone.

\subsection*{Abstracts}
\hypertarget{pmid_29601447}{I}traconazole is a broad-spectrum antifungal agent used for prophylaxis and treatment of fungal infections in immunocompromised children. Achieving the recommended target serum itraconazole trough concentration of ≥0.5 mg/L is challenging in children because of variation in itraconazole pharmacokinetics with age. We studied itraconazole use and treatment outcomes in a tertiary children's hospital. We did a 10-year retrospective review of medical records of children at the Royal Children's Hospital Melbourne who received oral itraconazole and had therapeutic drug monitoring (TDM). Overall, 81 children received 92 courses of oral itraconazole and had TDM. Of 222 TDM samples, 183 (82.4\%) were taken at the appropriate time (trough level at steady state). Patients ≤12 and >12 years of age required median doses of 6.2 and 3.9 mg/kg/d, respectively, to attain target trough levels (P < 0.001). Of children ≤12 years of age, 71.4\% required doses above the recommended dose of 5 mg/kg/d to achieve therapeutic levels, compared with 17.4\% of those >12 years of age. At least 1 subtherapeutic trough concentration was reported in 63 (76.8\%) courses; in only 18 (28.6\%) of these was the dose adjusted. Gastrointestinal symptoms [14/92 (15.2\%) courses] and hepatotoxicity [6/92 (6.5\%)] were the most frequent adverse events. Neither was associated with elevated trough levels. The poor attainment of target levels with current recommended dosing in children <12 years of age suggests that higher empiric doses are needed in this age group. The poor compliance with TDM guidelines highlights the need for better education about appropriate timing of sampling and dose adjustment. [\hyperlink{Itraconazole}{PMID: 29601447}, Ying Hua Leong et al., 2019]

\hypertarget{pmid_27286691}{I}traconazole has been used to treat fungal infections, in particular invasive fungal infections in infants or neonates in many countries. Literature search was conducted through Ovid EMBASE, PubMed, ISI Web of Science, CNKI and Google scholarship using the following key words: "pediatric" or "infant" or "neonate" and "fungal infection" in combination with "itraconazole". Based on the literature and our clinical experience, we outline the administration of itraconazole in infants in order to develop evidence-based pharmacotherapy. Of 45 articles on the use of itraconazole in infancy, 13 are related to superficial fungal infections including tinea capitis, sporotrichosis, mucosal fungal infections and opportunistic infections. The other 32 articles are related to systemic fungal infections including candidiasis, aspergillosis, histoplasmosis, zygomycosis, trichosporonosis and opportunistic infections as caused by Myceliophthora thermophila. Itraconazole is safe and effective at a dose of 5 mg/kg per day in a short duration of therapy for superficial fungal infections and 10 mg/kg per day for systemic fungal infections in infants. With a good compliance, it is cost-effective in treating infantile fungal infections. The profiles of adverse events induced by itraconazole in infants are similar to those in adults and children. [\hyperlink{Itraconazole}{PMID: 27286691}, Shuang Chen et al., 2016]

\hypertarget{pmid_12956205}{C}urrent dosing regimens for itraconazole are effective, safe, and versatile for use in superficial fungal infections in children, particularly tinea capitis. Good efficacy rates have been noted in both Trichophyton and Microsporum tinea capitis infections. Itraconazole has a high affinity for keratin, and accumulates to high levels at the site of superficial fungal infections. A pulse regimen may be chosen over continuous dosing, because the accumulation persists after dosing of itraconazole has been stopped. An oral solution of itraconazole is available, and may be more convenient for children who cannot swallow capsules. The oral solution also produces good rates of efficacy, but may be associated with a somewhat higher potential for gastrointestinal adverse events than the capsules. The range of adverse events noted with itraconazole capsules or oral solution use in children is similar to the range in adults. Events are generally mild and transient. Attention must be taken to note any medications that the child is using, because itraconazole is associated with a range of potential drug interactions. This safety of use, in combination with itraconazole's wide antifungal spectrum and pharmacokinetic properties, which allow for shorter dosing regimens, may make itraconazole a suitable alternative to griseofulvin for pediatric superficial fungal infections. [\hyperlink{Itraconazole}{PMID: 12956205}, Aditya K Gupta et al., 2003]

\hypertarget{pmid_17430480}{T}his single-centre, retrospective, observational pilot study was performed to evaluate the safety and efficacy of intravenous and oral itraconazole prophylaxis in paediatric haematopoietic stem cell transplantation (HCT). Study end-points were proven invasive fungal infection (IFI), survival, adverse reactions and graft-vs.-host disease (GVHD); 53 children and one young adult (median age 8.6 yr; range 0.4-18.3) transplanted between November 2001 and August 2004 were included in this study. Itraconazole was given intravenously from day +3 after HCT until oral medication became possible and continued until day +100 after HCT. Two proven new IFI in the itraconazole group (candidiasis, n = 1; aspergillosis, n = 1) were observed. After a median follow-up of 1.6 yr (0.3-6.1), six deaths (8\%) were seen; 24 patients (45\%) developed GVHD degree I-II, three children (6\%) had GVHD degree III-IV. In 11 of 53 patients (21\%), itraconazole prophylaxis was discontinued prematurely, mostly because of fever of unknown origin (n = 7). In total, 21 of 53 (40\%) of the children had abnormal results of laboratory investigations during the prophylaxis. The results of this pilot study indicate that itraconazole prophylaxis during HCT in children is feasible and safe, despite abnormal laboratory results. The efficacy in terms of prevention of IFI, however, has to be addressed in a prospective large-scale study. [\hyperlink{Itraconazole}{PMID: 17430480}, L Grigull et al., 2007]

\hypertarget{pmid_25512128}{I}nfantile hemangiomas can present a therapeutic challenge to clinicians, especially when associated with severe pain and feeding difficulties. The standard therapeutic management includes corticosteroids and propranolol. However, the clinical response is not always satisfactory. We present six cases of infantile hemangiomas successfully treated with oral itraconazole approximately 5 mg/kg per day. In the first month, the red color of the lesions became a little lighter and the growth of the lesions was controlled in all cases. An obvious clinical improvement was noted in all cases during the 3-month period, with 80-100\% improvement in each patient at the end of the treatment, which was judged by both their parents and the dermatologists. Compliance with treatment instructions of oral itraconazole in infants was judged to be very good. Side-effects were mild and limited. Although itraconazole can inhibit angiogenesis and tumor growth in vitro and in vivo associated with some cancers, further research is required to understand the pathogenesis of infantile hemangiomas and the mechanism of itraconazole.  [\hyperlink{Itraconazole}{PMID: 25512128}, Yuping Ran et al., 2015] We investigated the pharmacokinetics and safety of an oral solution of itraconazole in two groups of neutropenic children stratified by age. Effective concentrations of itraconazole in plasma were reached quickly and maintained throughout treatment. The results indicate a trend toward higher concentrations of itraconazole in plasma in older children. [\hyperlink{Itraconazole}{PMID: 25512128}, C Schmitt et al., 2001]

\hypertarget{pmid_16012330}{I}traconazole is particularly attractive in fungal prophylaxis for cancer patients due to its broad spectrum, including Candida and Aspergillus. It is generally well tolerated. However, its efficacy in preventing invasive aspergillosis could not be demonstrated. A 3-year-old boy diagnosed with acute lymphoblastic leukemia received induction chemotherapy. On day 14, itraconazole solution at a dose of 5 mg/kg was begun. Ten days after itraconazole was started, he developed paralytic ileus, neurogenic bladder, mild left ptosis, and absence of deep reflexes, with severe paralysis of the lower extremities and mild weakness of the upper extremities. Itraconazole withdrawal was followed by rapid improvement, with neurologic examination returning to normal within 6 weeks. Nineteen cases of unusual enhanced vincristine neurotoxicity related to itraconazole have been reported in children. Although the manifestations are the same as those usually associated with the use of vincristine, in these cases the severity appears remarkable. The authors suggest that in the absence of any proven benefit of itraconazole prophylaxis, and given the interaction of this drug with vincristine leading to severe and even potentially fatal toxicities, the combination use of these drugs should be avoided. [\hyperlink{Itraconazole}{PMID: 16012330}, Mar Bermúdez et al., 2005]

\hypertarget{pmid_21628477}{I}traconazole has become the first choice for treatment of cutaneous sporotrichosis. However, this recommendation is based on case reports and small series. The safety and efficacy of itraconazole were evaluated in 645 patients who received a diagnosis on the basis of isolation of Sporothrix schenckii in Rio de Janeiro, Brazil. A standard regimen of itraconazole (100 mg/day orally) was used. Clinical and laboratory adverse events were assessed a grades 1-4. A multivariate Cox model was used to analyze the response to treatment. The median age was 43 years. Lymphocutaneous form occurred in 68.1\% and fixed form in 23.1\%. Six hundred ten patients (94.6\%) were cured with itraconazole (50-400 mg/day): 547 with 100 mg/day, 59 with 200-400 mg/day, and 4 children with 50 mg/day. Three patients switched to potassium iodide, 2 to terbinafine, and 4 to thermotherapy. Twenty-six were lost to follow-up. Clinical adverse events occurred in 18.1\% of patients using 100 mg/day and 21.9\% of those using 200-400 mg/day. The most frequent clinical adverse events were nausea and epigastric pain. Laboratory adverse events occurred in 24.1\%; the most common was hypercholesterolemia, followed by hypertriglyceridemia. Four hundred sixty-two patients (71.6\%) completed clinical follow-up, and all remained cured. Only 2 variables were significant in explaining the cure: patients with erythema nodosum healed faster, and lymphocutaneous form took longer to cure. In the current series, the therapeutic response was excellent with the minimum dose of itraconazole, and there was a low incidence of adverse events and treatment failure. [\hyperlink{Itraconazole}{PMID: 21628477}, Mônica Bastos de Lima Barros et al., 2011]

\hypertarget{pmid_31668109}{T}he initial recommendation propranolol usage in managing infantile hemangioma was in 2008 followed by various researches assessing the dosage, efficacy, and other parameters. Itraconazole is a world-wide tolerated antifungal but only a few studies have focused on its assessment in the treatment of infantile hemangiomas (IH). This study aimed to investigate the newly proposed antifungal drug ICZ and characterize different aspects of its usage as an antiangiogenic drug. This was an interventional clinical trial to assess the efficacy of ICZ versus propranolol in the treatment of infantile hemangioma with studying the change in serum angiopoietin 2 (Ang2). A total of 36 pediatric patients were divided into two equal groups: firstly treated with oral itraconazole and secondly treated by oral propranolol. Response to treatment was observed using a modified IH score. In itraconazole-treated infants, good response was observed in 44.4\% of the patients. This was slightly higher than the propranolol group which showed 22.2\% with good response. We observed a decrease in serum ang2 level after usage of ICZ and propranolol and the change in serum Ang2 level before and after treatment in each group was statistically significant ( Oral itraconazole can be an equivalent option for oral propranolol while safer and shorter treatment periods. [\hyperlink{Itraconazole}{PMID: 31668109}, Hagar Bessar et al., 2022]

\hypertarget{pmid_10523732}{O}ver the past 10 years, itraconazole has been used to treat more than 34 million patients worldwide. We present a review of the safety of various continuous itraconazole schedules used in the treatment of dermatomycosis and onychomycosis. Data from controlled clinical trials and extensive post-marketing surveillance show that itraconazole has an impressive safety profile at a dose of 50-200 mg/day for 1-4 weeks for dermatomycosis and 200 mg/day for 3 months for onychomycosis. In addition, itraconazole is safe to use in diabetic patients with dermatomycosis or onychomycosis. Short-term, intermittent itraconazole regimens, which may offer additional benefits in terms of safety and cost, have now been introduced. [\hyperlink{Itraconazole}{PMID: 10523732}, S K Nolting et al., ]

\hypertarget{pmid_32929460}{I}traconazole remains a first-line antifungal agent for certain fungal infections in children, including allergic bronchopulmonary aspergillosis (ABPA) and sporotrichosis, but poor attainment of therapeutic drug levels is frequently observed with available oral formulations. A formulation of 'SUper BioAvailability itraconazole' (SUBA-itraconazole; Lozanoc®) has been developed, with adult studies demonstrating rapid and reliable attainment of therapeutic levels, yet paediatric data are lacking. To assess the safety, efficacy and attainment of therapeutic drug levels of the SUBA-itraconazole formulation in children. A single-centre retrospective cohort study was conducted, including all patients prescribed SUBA-itraconazole from May 2018 to February 2020. The recommended initial treatment dose was 5 mg/kg twice daily (to a maximum of 400 mg/day) rounded to the nearest capsule size and 2.5 mg/kg/day for prophylaxis. Nineteen patients received SUBA-itraconazole and the median age was 12 years. The median dose was 8.5 mg/kg/day and the median duration was 6 weeks. Indications included ABPA (16 patients), sporotrichosis (1), cutaneous fungal infection (1) and prophylaxis (1). Of patients with serum levels measured, almost 60\% (10/17) achieved a therapeutic level, 3 with one dose adjustment and 7 following the initial dose. Adherence to dose-adjustment recommendations amongst the seven patients not achieving therapeutic levels was poor. Of patients with ABPA, 13/16 (81\%) demonstrated a therapeutic response in IgE level. SUBA-itraconazole was well tolerated with no cessations related to adverse effects. SUBA-itraconazole is well tolerated in children, with rapid attainment of therapeutic levels in the majority of patients, and may represent a superior formulation for children in whom itraconazole is indicated for treatment or prevention of fungal infection. [\hyperlink{Itraconazole}{PMID: 32929460}, Joanne Abbotsford et al., 2021]

\hypertarget{pmid_1655460}{A}n 11-year-old boy with chronic granulomatous disease caused by cytochrome b deficiency developed right upper lung lobe aspergillosis. Intracerebral lesions developed on maximum doses of flucytosine and amphotericin B. Treatment with 16 mg/kg oral itraconazole was followed by a dramatic clinical improvement and almost complete disappearance of the intracerebral lesions. Plasma itraconazole levels were between 40 and 3440 ng/ml depending on concomitant medication. Toxicity was restricted to transient elevation of alkaline phosphatase and gamma glutamyl transferase. We conclude that further trials with itraconazole are justified in high risk patients in whom conventional therapy has failed. [\hyperlink{Itraconazole}{PMID: 1655460}, S Kloss et al., 1991]

\hypertarget{pmid_18194238}{O}nychomycosis is a rare disease in children with an estimated prevalence ranging from 0\% to 2.6\%. Thus far, only limited experience with itraconazole and terbinafine treatment in children with onychomycosis is available in the literature. Evaluation of treatment experience with itraconazole or terbinafine in childhood onychomycosis. Thirty-six children and adolescents (aged 4-17 years, 18 males and 18 females) with clinical and mycologically proven onychomycosis were enrolled in the present study. METHODS AND OUTCOME: In 27 of 36 patients, the causative agent (Trichophyton rubrum in 26 cases and Trichophyton interdigitale in one patient) could be identified by means of cultivation. Nineteen patients were treated with itraconazole 200 mg once daily for 12 weeks, and 17 patients were treated with terbinafine for 12 weeks in a dosage according to their body weight, respectively. Clinical cure was achieved within 1 to 5 months after discontinuation in all patients treated with itraconazole and in all but two patients after cessation of terbinafine treatment. Neither in the itraconazole nor in the terbinafine group were serious adverse events reported. Clinical cure was achieved within 1 to 5 months after discontinuation in all patients treated with itraconazole and in all but two patients after cessation of terbinafine treatment. Neither in the itraconazole nor in the terbinafine group were serious adverse events reported. To our experience, a mycological and clinical cure appears in children in a shorter time after treatment discontinuation (average 2-5 months) compared with adults. Itraconazole and terbinafine seem to be safe and effective in childhood onychomycosis; therefore, these antifungals seem to be potential alternatives to griseofulvin. [\hyperlink{Itraconazole}{PMID: 18194238}, G Ginter-Hanselmayer et al., 2008]

\hypertarget{pmid_24173819}{O}ral antifungal prophylaxis with extended-spectra azoles is widely used in pediatric patients after allogeneic hematopoietic stem cell transplantation (HSCT), while controlled studies for oral antifungal prophylaxis after bone marrow transplantation in children are not available. This survey analyzed patients who had received either itraconazole, voriconazole, or posaconazole. We focused on the safety, feasibility, and initial data of efficacy in a cohort of pediatric patients and adolescents after high-dose chemotherapy and HSCT. Fifty consecutive pediatric patients received itraconazole, 50 received voriconazole, and 50 pediatric patients received posaconazole after HSCT as oral antifungal prophylaxis. The observation period lasted from the start of oral prophylactic treatment with itraconazole, voriconazole, or posaconazole until two weeks after terminating the oral antifungal prophylaxis. No incidences of proven or probable invasive mycosis were observed during itraconazole, voriconazole, or posaconazole treatment. A total of five possible invasive fungal infections occurred, two in the itraconazole group (4\%) and three in the voriconazole group (6\%). The percentage of patients with adverse events potentially related to clinical drugs were 14\% in the voriconazole group, 12\% in the itraconazole group, and 8\% in the posaconazole group. Itraconazole, voriconazole, and posaconazole showed comparable efficacy as antifungal prophylaxis in pediatric patients after allogeneic HSCT. [\hyperlink{Itraconazole}{PMID: 24173819}, M Döring et al., 2014]

\hypertarget{pmid_34754871}{I}nfantile hemangiomas (IHs) are the most common childhood benign tumors, showing distinctive progression characteristics and outcomes. Due to the high demand for aesthetics among parents of IH babies, early intervention is critical in some cases. β-Adrenergic blockers and corticosteroids are first-line medications for IHs, while itraconazole, an antifungal medicine, has shown positive results in recent years. In the present study, itraconazole was applied to treat two IH cases. The therapeutic course lasted 80-90 d, during which the visible lesion faded by more than 90\%. Moreover, no obvious side effects were reported, and the compliance of the baby and parents was desirable. Although these outcomes further support itraconazole as an effective therapeutic choice for IHs, large-scale clinical and basic studies are still warranted to improve further treatment. [\hyperlink{Itraconazole}{PMID: 34754871}, Zhe Liu et al., 2021]

\hypertarget{pmid_8387801}{I}traconazole is a new orally active antifungal agent shown to have in vitro and experimental activity against Aspergillus spp. This case report documents the successful eradication of biopsy-proven invasive pulmonary aspergillosis in a 17 year old boy with acute lymphocytic leukaemia. Cerebral involvement by the fungal infection was suspected clinically but was not biopsy proven. Although the patient subsequently died following bone marrow transplant and Escherichia coli septicaemia there was no evidence of residual Aspergillus at autopsy. [\hyperlink{Itraconazole}{PMID: 8387801}, L Moore et al., 1993]

\hypertarget{pmid_23278514}{B}ecause of the low prevalence of onychomycosis in children, little is known about the efficacy and safety of systemic antifungals in this population. PubMed and Embase databases and the references of related publications were searched in March 2012 for clinical trials (CTs), retrospective analyses (RAs), and case reports (CRs) on the use of systemic antifungals for onychomycosis in children (<18 years). Twenty-six studies (5 CTs, 3 RAs, and 18 CRs) were published between 1976 and 2011. Most of these studies reported the use of systemic terbinafine and itraconazole for the treatment of onychomycosis in children. Therapy with systemic antifungals alone in children ages 1 to 17 years resulted in a complete cure rate of 70.8\% (n = 151), whereas combined systemic and topical antifungal therapy in one infant and 19 children age 8 and older resulted in a complete cure rate of 80.0\% (n = 20). The efficacy and safety profiles of terbinafine, itraconazole, griseofulvin, and fluconazole in children were similar to those previously reported for adults. In conclusion, based on the little information available on onychomycosis in children, systemic antifungal therapies in children are safe and cure rates are similar to the rates achieved in adults. [\hyperlink{Itraconazole}{PMID: 23278514}, Aditya K Gupta et al., ]

\hypertarget{pmid_34002355}{T}riazoles represent an important class of antifungal drugs in the prophylaxis and treatment of invasive fungal disease in pediatric patients. Understanding the pharmacokinetics of triazoles in children is crucial to providing optimal care for this vulnerable population. While the pharmacokinetics is extensively studied in adult populations, knowledge on pharmacokinetics of triazoles in children is limited. New data are still emerging despite drugs already going off patent. This review aims to provide readers with the most current knowledge on the pharmacokinetics of the triazoles: fluconazole, itraconazole, voriconazole, posaconazole, and isavuconazole. In addition, factors that have to be taken into account to select the optimal dose are summarized and knowledge gaps are identified that require further research. We hope it will provide clinicians guidance to optimally deploy these drugs in the setting of a life-threatening disease in pediatric patients. [\hyperlink{Itraconazole}{PMID: 34002355}, Didi Bury et al., 2021]

\hypertarget{pmid_8381643}{I}traconazole is a new orally active triazole antifungal agent with enhanced activity against Candida species. In the clinical trial described in this paper, we compared the efficacy and safety of itraconazole capsules with those of clotrimazole vaginal tablets and placebo oral capsules for women with acute vulvovaginal candidiasis. Ninety-five patients were randomized in a 2:1:1 fashion to receive itraconazole (200 mg/day), clotrimazole (200 mg/day), or placebo (two capsules per day) for 3 consecutive days. Clinical success rates (cure and improvement) were similar for women who received itraconazole (96\%) and clotrimazole (100\%) 1 week posttreatment. These response rates were statistically superior to those obtained with placebo treatment (77\%, P < 0.05). Negative mycological cultures were found in 95, 73, and 32\% of the patients treated with clotrimazole, itraconazole, and placebo, respectively (P < 0.005) [active treatments versus placebo]). By 4 weeks posttreatment, the clinical failure rate for itraconazole was less than that observed for clotrimazole (17 versus 30\%), but this difference did not reach statistical significance (P > 0.05; beta = 0.81). Mycological response rates for itraconazole and clotrimazole were also similar. No patients enrolled in this study discontinued treatment because of an adverse event. Minor side effects were reported by 35, 4, and 41\% of patients who received itraconazole, clotrimazole, and placebo, respectively. The most common side effects associated with itraconazole therapy were nausea and headache. In summary, itraconazole was found to be as effective and safe as clotrimazole in women with acute candida vaginitis. Moreover, oral therapy was highly favored over intravaginal treatment in our survey of patients. [\hyperlink{Itraconazole}{PMID: 8381643}, G E Stein et al., 1993]

\hypertarget{pmid_10578159}{T}his was an open study of oral antifungal prophylaxis in 103 neutropenic children aged 0-14 (median 5) years. Most (90\%) were undergoing transplantation for haematological conditions (77\% allogeneic BMT, 7\% autologous BMT, 6\% PBSC transplants and 10\% chemotherapy alone). They received 5.0 mg/kg itraconazole/day (in 10 mg/ml cyclodextrin solution). Where possible, prophylaxis was started at least 7 days before the onset of neutropenia and continued until neutrophil recovery. Of the 103 who entered the study, 47 completed the course of prophylaxis, 27 withdrew because of poor compliance, 19 because of adverse events and 10 for other reasons. Two patients died during the study and another five died within the subsequent 30 days. No proven systemic fungal infections occurred, but 26 patients received i.v. amphotericin for antibiotic-unresponsive pyrexia. One patient received amphotericin for mycologically confirmed oesophageal candidosis. Three patients developed suspected oral candidosis but none was mycologically proven and no treatment was given. Serious adverse events (other than death) occurred in 21 patients, including convulsions (7), suspected drug interactions (6), abdominal pain (4) and constipation (4). The most common adverse events considered definitely or possibly related to itraconazole were vomiting (12), abnormal liver function (5) and abdominal pain (3). Tolerability of study medication at end-point was rated as good (55\%), moderate (11\%), poor (17\%) or unacceptable (17\%). Some patients had poor oral intakes due to mucositis. No unexpected problems of safety or tolerability were encountered. We conclude that itraconazole oral solution may be used as antifungal prophylaxis for neutropenic children. [\hyperlink{Itraconazole}{PMID: 10578159}, A B Foot et al., 1999]

\hypertarget{pmid_1319313}{I}traconazole is a lipophilic triazole with potent in vitro activity. It is also effective after topical, oral and parenteral administration. The antifungal activity of itraconazole has been evaluated against more than 6,500 different strains, belonging to more than 260 fungal species, using the serial decimal dilution test in fluid broth medium (brain-heart infusion broth). Candida spp., Torulopsis spp., Cryptococcus neoformans, Pityrosporum spp. (Dixon broth), various other yeasts, dermatophytes, Aspergillus spp., Penicillium spp., Sporothrix schenckii, dimorphic fungi (mycelium phase and yeast phase), Phaeohyphomycetes, Entomophthorales and various Hyalohyphomycetes are sensitive. Most strains of Fusarium and Zygomycetes are poorly sensitive. Itraconazole was administered orally and parenterally in normal and immunocompromised guinea-pigs infected with C. albicans, Cr. neoformans, Histoplasma duboisii, S. schenckii, P. marneffei and A. fumigatus. It was effective in terms of both survival of the animals and elimination of the fungi from the various tissues. Itraconazole was superior to fluconazole in candidosis, cryptococcosis, sporotrichosis and aspergillosis, and to amphotericin B and to flucytosine in candidosis, cryptococcosis and aspergillosis. No comparative studies have yet been undertaken for other deep mycoses. The results of combination therapy with itraconazole and fluconazole in cryptococcosis were indifferent; with flucytosine or amphotericin B, additive or synergistic effects were seen in systemic candidosis, cryptococcosis and aspergillosis. No drug-related side-effects were observed after oral or parenteral administration of itraconazole. [\hyperlink{Itraconazole}{PMID: 1319313}, J Van Cutsem et al., 1992]

\hypertarget{pmid_15283801}{M}ycotic scalp infection caused by Microsporum canis is one of the more recalcitrant disorders, with increasing incidence during the last decade. We report our experience with administration of itraconazole in 163 children (86 girls, 77 boys) with M. canis tinea capitis. Fifty-five patients had previous treatment with terbinafine without success. In all children, the dosage of itraconazole was adjusted according to body weight, with 5 mg/kg/day given in a continuous regimen either as a capsule (116 patients) or an oral suspension (47 patients). In all children, there was both clinical and mycologic cure after a mean treatment period of 39 +/- 12 days (range 10-77 days). Eleven children (6.7\%) had side effects: diarrhea in five children, cutaneous eruption in four, and abdominal pain in two. Itraconazole was effective and safe for the treatment of M. canis tinea capitis. [\hyperlink{Itraconazole}{PMID: 15283801}, Gabriele Ginter-Hanselmayer et al., ]

\hypertarget{pmid_9144703}{W}e report on two children affected by chronic mucocutaneous candidiasis involving the mouth and all the nails who were successfully treated with itraconazole at 200 mg/day for 2 months. This therapy produced a rapid cure of both candidal nail and mouth infections. The drug was very well tolerated, and routine laboratory monitoring during treatment did not reveal any abnormalities. [\hyperlink{Itraconazole}{PMID: 9144703}, A Tosti et al., ]

\hypertarget{pmid_25680318}{P}ediatric patients with hemato-oncological malignancies and neutropenia resulting from chemotherapy have a high risk of acquiring invasive fungal infections. Oral antifungal prophylaxis with azoles, such as fluconazole or itraconazole, is preferentially used in pediatric patients after chemotherapy. During this retrospective analysis, posaconazole was administered based on favorable results from studies in adult patients with neutropenia and after allogeneic hematopoietic stem cell transplantation. Retrospectively, safety, feasibility, and initial data on the efficacy of posaconazole were compared to fluconazole and itraconazole in pediatric and adolescent patients during neutropenia. Ninety-three pediatric patients with hemato-oncological malignancies with a median age of 12 years (range 9 months to 17.7 years) that had prolonged neutropenia (>5 days) after chemotherapy or due to their underlying disease, and who received fluconazole, itraconazole, or posaconazole as antifungal prophylaxis, were analyzed in this retrospective single-center survey. The incidence of invasive fungal infections in pediatric patients was low under each of the azoles. One case of proven aspergillosis occurred in each group. In addition, there were a few cases of possible invasive fungal infection under fluconazole (n = 1) and itraconazole (n = 2). However, no such cases were observed under posaconazole. The rates of potentially clinical drug-related adverse events were higher in the fluconazole (n = 4) and itraconazole (n = 5) groups compared to patients receiving posaconazole (n = 3). Posaconazole, fluconazole, and itraconazole are comparably effective in preventing invasive fungal infections in pediatric patients. Defining dose recommendations in these patients requires larger studies. [\hyperlink{Itraconazole}{PMID: 25680318}, M Döring et al., 2015]

\hypertarget{pmid_28744925}{I}traconazole is a first-generation triazole agent with an extended spectrum of activity; it is licensed in adults for superficial and systemic fungal infections; no recommendation has been yet established for use in children patients. Its variable and unpredictable oral bioavailability make it difficult to determine the optimal dosing regimen. Hence, therapeutic drug monitoring, highly available in clinical practice, may improve itraconazole treatment success and safety. The aim of the study was to describe in paediatrics the oral itraconazole pharmacokinetics, used for prophylaxis. Moreover, we evaluated the utility of its therapeutic drug monitoring in this cohort. A fully validated chromatographic method was used to quantify itraconazole concentration in plasma collected from paediatric patients, at the end of dosing interval. Associations between variables were tested using the Pearson test. Mann-Whitney U test has been used to probe the influence of categorical variables on continuous ones. Any predictive power of the considered variables was finally evaluated through univariate and multivariate linear and logistic regression analyses. A high inter-individual variability was shown; ethnicity (beta coefficient, β -0.161 and interval of confidence at 95\%, IC -395.035; -62.383) and gender (β 0.123 and IC 9.590; 349.395) remained in the final linear regression model with P value of .007 and .038, respectively. This study highlights that therapeutic drug monitoring is required to achieve an adequate target itraconazole serum exposure. [\hyperlink{Itraconazole}{PMID: 28744925}, Sarah Allegra et al., 2017]

\section*{Linaclotide}
\subsection*{Result}
\subsubsection*{Answer}

Unknown (for all pediatric age ranges, including <13 years and 13-17 years)

\subsubsection*{{Explanation}}
\hypertarget{Linaclotide}
To determine if linaclotide is safe for use in children, I reviewed all available abstracts for evidence of targeted safety studies in pediatric populations.

1. The most relevant abstract is a retrospective review of 93 children (<18 years old, median age 14.7 years, IQR 13.2-16.6) treated with linaclotide for functional constipation (FC) or irritable bowel syndrome with constipation (IBS-C) [\hyperlink{pmid_33876403}{PMID: 33876403}, Desiree F Baaleman et al., 2021]. This study found that approximately 45\% of patients with FC and 42\% with IBS-C had a positive clinical response at first follow-up. However, about one-third experienced adverse events (diarrhea, abdominal pain, nausea, bloating), and 27\% discontinued due to adverse events. The authors conclude that while nearly half of children benefited, adverse events were relatively common, and further prospective, controlled studies are needed to confirm these findings and identify which patients are most likely to benefit.

2. Another review [\hyperlink{pmid_29985664}{PMID: 29985664}, Gabrio Bassotti et al., 2018] specifically states that data on the efficacy and safety of linaclotide in pediatric patients are currently limited and more studies are needed.

3. All other abstracts focus on adult populations or do not specify inclusion of children, and thus do not provide evidence for or against safety in pediatric age groups.

Summary by age range:
- Children <18 years: There is a retrospective study in children and adolescents (median age \textasciitilde{}15), but it is not a prospective, controlled safety study. Adverse events were common, and the authors call for further research. Therefore, safety is not definitively established.
- Children <13 years: The cited pediatric study's interquartile range suggests most participants were adolescents, with little to no data on younger children. No targeted safety data for this age group is available.
- Adolescents (13-17 years): The retrospective study includes this age group, but due to its design and the frequency of adverse events, safety is not definitively affirmed.

Conclusion: There is insufficient evidence from targeted, prospective safety studies to affirm that linaclotide is safe for use in children of any age group. The available pediatric data are limited, retrospective, and indicate a relatively high rate of adverse events, with a significant proportion discontinuing treatment.

\subsection*{Abstracts}
\hypertarget{pmid_33876403}{L}inaclotide is a well-tolerated and effective agent for adults with functional constipation (FC) or irritable bowel syndrome with constipation (IBS-C). However, data in children are lacking. The aim of this study is to examine the efficacy and safety of linaclotide in children. We performed a retrospective review of children < 18 years old who started linaclotide at our institution (Nationwide Children's Hospital, Columbus, Ohio). We excluded children already using linaclotide or whom had an organic cause of constipation or abdominal pain. We recorded information on patient characteristics, medical and surgical history, symptoms, clinical response, course of treatment, and adverse events at baseline, first follow-up, and after 1 year of linaclotide use. A positive clinical response was based on the physician's global assessment of symptoms at the time of the visit as documented. We included 93 children treated with linaclotide for FC (n = 60) or IBS-C (n = 33); 60\% were female; median age was 14.7 years (IQR 13.2-16.6). Forty-five percent of patients with FC and 42\% with IBS-C had a positive clinical response at first follow-up a median of 2.5 and 2.4 months after starting linaclotide, respectively. Approximately a third of patients experienced adverse events and eventually 27\% stopped using linaclotide due to adverse events. The most common adverse events were diarrhea, abdominal pain, nausea, and bloating. Nearly half of children with FC or IBS-C benefited from linaclotide, but adverse events were relatively common. Further prospective, controlled studies are needed to confirm these findings and to identify which patients are most likely to benefit from linaclotide. [\hyperlink{Linaclotide}{PMID: 33876403}, Desiree F Baaleman et al., 2021]

\hypertarget{pmid_24917937}{L}inaclotide is the first member of a novel class of drugs to be extensively evaluated for the treatment of chronic constipation (CC) and irritable bowel syndrome with constipation (IBS-C). To provide a comprehensive overview of the current state of knowledge on linaclotide, its pharmacological properties, mode of action and efficacy in clinical trials to date. We conducted a systematic review of the literature. The survey revealed that linaclotide is a minimally absorbed, 14-amino acid peptide which acts in the intestinal lumen on guanylate cyclase-C (GC-C). This results in generation of cyclic guanosine monophosphate (cGMP), which stimulates chloride secretion, resulting in increased luminal fluid secretion and an acceleration of intestinal transit. In animal models, linaclotide also decreased visceral hypersensitivity. Linaclotide softened stool and increased transit in CC and in IBS-C. Phase II and phase III clinical studies established efficacy of linaclotide in CC (linaclotide 145 µg daily approved in the United States for CC) and in IBS-C (linaclotide 290 µg daily US Food and Drug Administration-approved for IBS-C, with favourable recommendation for the European Medicines Agency Committee for Medicinal Products for Human Use (CHMP). Linaclotide showed a favourable safety profile, and the main treatment-emerging adverse event was diarrhea, leading to discontinuation rates of up to 5\%. Linaclotide is an important addition to the therapeutic possibilities for treating IBS-C and CC. [\hyperlink{Linaclotide}{PMID: 24917937}, Maura Corsetti et al., 2013]

\hypertarget{pmid_24939497}{T}he pharmacology, pharmaco-kinetics, and clinical efficacy and safety of linaclotide in the management of chronic constipation (CC) and constipation-predominant irritable bowel syndrome (IBS-C) are reviewed. Linaclotide (Linzess, Forest Pharmaceuticals) is a 14-amino acid peptide indicated for the treatment of adults with CC and IBS-C. Linaclotide acts on guanylate cyclase-C receptors on the luminal membrane to increase chloride and bicarbonate secretions into the intestine and inhibit the absorption of sodium ions, thus increasing the secretion of water into the lumen and improving defecation; the drug is minimally absorbed into the systemic circulation. Linaclotide is approved by the Food and Drug Administration (FDA) for oral once-daily administration at doses of 145 μg for CC and 290 μg for IBS-C. In placebo-controlled Phase III clinical trials, linaclotide significantly increased weekly spontaneous bowel movements and complete spontaneous bowel movements (CSBMs) while reducing abdominal pain in patients with CC. In patients with IBS-C, linaclotide was demonstrated to be effective in meeting FDA-recommended endpoints such as reductions of at least 30\% from baseline in abdominal pain scores and CSBM frequency. The most common adverse effect of linaclotide is diarrhea, which was reported in 16-20\% of clinical trial participants. Linaclotide is an important advance in the treatment of CC and IBS-C, with a novel mechanism of action resulting in accelerated intestinal transit. In clinical trials, linaclotide demonstrated efficacy relative to placebo for treatment of both CC and IBS-C. Linaclotide's adverse effects are generally mild and confined to the gastrointestinal tract. [\hyperlink{Linaclotide}{PMID: 24939497}, Bryan L Love et al., 2014]

\hypertarget{pmid_30791771}{L}inaclotide is approved for treating irritable bowel syndrome with constipation (IBS-C; 290 µg QD) and chronic idiopathic constipation (CIC; 145 µg or 72 µg QD). These analyses aimed to assess linaclotide safety in a large, pooled Phase 3 population. In six randomized controlled trials (RCTs), patients received linaclotide (72 µg, 145 µg, 290 µg) or placebo daily for 12-26 weeks; in two long-term safety (LTS) studies, patients received open-label linaclotide for ≤78 additional weeks. Laboratory values, vital signs, and treatment-emergent adverse events (TEAEs) were assessed. Overall, 3853 patients received ≥1 dose of linaclotide. The most common TEAE was diarrhea (majority [90.5\% in RCTs] mild/moderate). Linaclotide patients experienced 1.1 diarrhea TEAE per patient-year in the RCTs (0.2 in placebo), and 0.3 in the LTS studies. In RCTs, 6.9\% linaclotide and 3.0\% placebo patients discontinued due to any adverse event (AE); 4.0\% linaclotide and 0.3\% placebo patients discontinued due to diarrhea. In LTS studies, 9.4\% patients discontinued due to any AE, and 3.8\% due to diarrhea. Serious AEs (SAEs) were rare and similar across treatment groups; there were no SAEs of diarrhea. These pooled analyses of patients treated for ≤104 weeks confirm linaclotide's overall safety. [\hyperlink{Linaclotide}{PMID: 30791771}, Judy W Nee et al., 2019]

\hypertarget{pmid_25629140}{W}hen patients complain of recurrent functional bowel disorders consisting of alterations in intestinal transit with abdominal pain or discomfort, treatment is purely symptomatic. Increased intake of dietary fibre or use of a bulk-forming or osmotic laxative is used when constipation is the main complaint. Linaclotide, a small peptide closely related to certain toxins secreted by diarrhoea-causing strains of Escherichia coli, has been authorised in the European Union for the treatment of adults with recurrent functional bowel disorders consisting mainly of constipation. Clinical evaluation of linaclotide includes no trials versus other laxatives. It is based on two placebo-controlled trials including a total of about 1600 patients, lasting 3 and 6 months. A pooled analysis of the results obtained at 3 months showed more frequent "relief" with linaclotide than with placebo. Adverse effects included gastrointestinal disorders, with diarrhoea occurring in one in five patients. Diarrhoea was sometimes severe or prolonged. About 10\% of patients discontinued linaclotide because of gastrointestinal adverse effects (diarrhoea in half of these cases). The consequences of diarrhoea can be severe, particularly in patients predisposed to fluid and electrolyte imbalances. Co-administration of other laxative substances enhances the laxative effect of linaclotide. The laxative action is likely to reduce the gastrointestinal absorption of co-administered medications. Linaclotide should be avoided during pregnancy, as there are no relevant safety data. In practice, given the absence of head-to-head comparisons with other laxatives, linaclotide is currently just another laxative with no proven advantages over existing first-line options. [\hyperlink{Linaclotide}{PMID: 25629140}, Linaclotide. A bacterial enterotoxin derivative with a laxative action, nothing more., 2014]

\hypertarget{pmid_21830967}{L}inaclotide is a minimally absorbed peptide agonist of the guanylate cyclase C receptor. In two trials, we aimed to determine the efficacy and safety of linaclotide in patients with chronic constipation. We conducted two randomized, 12-week, multicenter, double-blind, parallel-group, placebo-controlled, dual-dose trials (Trials 303 and 01) involving 1276 patients with chronic constipation. Patients received either placebo or linaclotide, 145 μg or 290 μg, once daily for 12 weeks. The primary efficacy end point was three or more complete spontaneous bowel movements (CSBMs) per week and an increase of one or more CSBMs from baseline during at least 9 of the 12 weeks. Adverse events were also monitored. For Trials 303 and 01, respectively, the primary end point was reached by 21.2\% and 16.0\% of the patients who received 145 μg of linaclotide and by 19.4\% and 21.3\% of the patients who received 290 μg of linaclotide, as compared with 3.3\% and 6.0\% of those who received placebo (P<0.01 for all comparisons of linaclotide with placebo). Improvements in all secondary end points were significantly greater in both linaclotide groups than in the placebo groups. The incidence of adverse events was similar among all study groups, with the exception of diarrhea, which led to discontinuation of treatment in 4.2\% of patients in both linaclotide groups. In these two 12-week trials, linaclotide significantly reduced bowel and abdominal symptoms in patients with chronic constipation. Additional studies are needed to evaluate the potential long-term risks and benefits of linaclotide in chronic constipation. (Funded by Ironwood Pharmaceuticals and Forest Research Institute; ClinicalTrials.gov numbers, NCT00765882 and NCT00730015.). [\hyperlink{Linaclotide}{PMID: 21830967}, Anthony J Lembo et al., 2011]

\hypertarget{pmid_23644388}{L}inaclotide is a minimally absorbed, 14-amino acid peptide used to treat patients with irritable bowel syndrome with constipation (IBS-C) or chronic constipation (CC). We performed a meta-analysis to determine the efficacy of linaclotide, compared with placebo, for patients with IBS-C or CC. MEDLINE, EMBASE, and the Cochrane central register of controlled trials were searched for randomized, placebo-controlled trials examining the effect of linaclotide in adults with IBS-C or CC. Dichotomous results were pooled to yield a relative risk (RR), 95\% confidence intervals (CIs), and number needed to treat (NNT). The search identified 7 trials of linaclotide in patients with IBS-C or CC; 6 were included in the analysis. Two of 3 trials of IBS-C used the end point recommended by the U.S. Food and Drug Administration: an increase from baseline of 1 or more complete spontaneous bowel movement (CSBM)/week and a 30\% or more reduction from baseline in the weekly average of daily worst abdominal pain scores for 50\% of the treatment weeks. On the basis of this end point, the RR for response to treatment with 290 μg linaclotide, compared with placebo, was 1.95 (95\% CI, 1.3-2.9), and the NNT was 7 (95\% CI, 5-11). For CC, on the basis of data from 3 trials of patients with CC, the RR for the primary end point (more than 3 CSBMs/week and an increase in 1 or more CSBM/week, for 75\% of weeks) was 4.26 for 290 μg linaclotide vs placebo (95\% CI, 2.80-6.47), and the NNT was 7 (95\% CI, 5-8). Linaclotide also improved stool form and reduced abdominal pain, bloating, and overall symptom severity in patients with IBS-C or CC. On the basis of a meta-analysis, linaclotide improves bowel function and reduces abdominal pain and overall severity of IBS-C or CC, compared with placebo. [\hyperlink{Linaclotide}{PMID: 23644388}, Elizabeth J Videlock et al., 2013]

\hypertarget{pmid_24090017}{L}inaclotide is a secretagogue that provides a combined effect on visceral pain. The European Medicines Agency has authorized its indication for the symptomatic treatment of moderate to severe irritable bowel syndrome with constipation in adults. The purpose of this review is to discuss the clinical framework for linaclotide use in our setting, the drug´s characteristics and pre-clinical development, and the clinical studies supporting its use in order to establish relevant views regarding its validity and clinical applicability. The results suggest that the only -non-severe- adverse effect associated with this drug is diarrhea. As regards effectiveness, linaclotide consistently shows favorable, significant differences in absolute risk versus placebo for all objective outcome variables described by regulatory agencies, with a combined pain and constipation response between 12.6\% and 22.8\% according to the variable and trial under consideration. This response is sustained and drug-related, as it goes away upon discontinuation. To conclude, linaclotide has a safety and efficacy profile that, from a clinical perspective, warrants its use for patients meeting irritable bowel syndrome and constipation criteria, with significant symptoms that cannot be relieved with other less specific measures. In the absence of predictive rules for response, it is recommended that, should the patient fail to respond, he or she should be considered not eligible for linaclotide therapy, and both indication and treatment continuity should be reserved for objective responders alone. [\hyperlink{Linaclotide}{PMID: 24090017}, Fernando Carballo et al., 2013]

\hypertarget{pmid_29742779}{L}inaclotide is a minimally absorbed peptide guanylate cyclase-C agonist approved for the treatment of irritable bowel syndrome with constipation (IBS-C). This study assessed the efficacy and tolerability of linaclotide in IBS-C in routine clinical practice in Germany. This was a 52-week, noninterventional study of linaclotide in patients aged ≥ 18 years with moderate to severe IBS-C. Severity of abdominal pain and bloating and frequency of bowel movements were assessed over 5 study visits. Treatment-related adverse events were recorded. The study enrolled 375 patients; the mean observation duration was 4.4 months. Linaclotide marketing was halted during the study period for economic reasons, accounting for low patient numbers and short observation duration. Linaclotide significantly reduced mean (standard deviation [SD]) scores between treatment start (visit 1) and study end (visit 5) for abdominal pain intensity (visit 1: 4.87 [2.63] vs. visit 5: 2.40 [2.20], p < 0.0001), mean [SD] bloating intensity (visit 1: 5.30 [2.70] vs. visit 5: 2.86 [2.34], p < 0.0001), and increased mean [SD] bowel movement frequency (visit 1: 2.71 [1.80] vs. 4.38 [1.86], p < 0.0001). Diarrhea, occurring in 5.1 \% of patients, was the most common adverse event. Linaclotide is effective in improving the major symptoms of IBS-C and demonstrates a favorable safety profile in the real-world environment of routine clinical practice. DRKS (www.drks.de/): DRKS00005088. [\hyperlink{Linaclotide}{PMID: 29742779}, Viola Andresen et al., 2018]

\hypertarget{pmid_24293117}{L}inaclotide (Constella®) is a synthetic 14-amino acid peptide, structurally related to guanylin and uroguanylin, that acts as a potent guanylate cyclase C receptor agonist. It is a first-in-class agent recently approved in the EU for the treatment of adult patients with moderate to severe irritable bowel syndrome with constipation (IBS-C). Linaclotide has very low oral bioavailability and acts locally in the gastrointestinal tract to stimulate fluid secretion, increase colonic transit, and reduce abdominal pain. In phase III trials, once-daily, oral linaclotide significantly increased compared with placebo the proportions of 12-week abdominal pain/discomfort responders and 12-week degree-of-relief responders (co-primary endpoints recommended by the European Medicines Agency). Linaclotide also significantly increased the proportions of responders at 26 weeks compared with placebo, and significantly improved all abdominal symptoms and measures of bowel function at 12 weeks compared with placebo. In addition, linaclotide generally improved health-related quality of life compared with placebo. Linaclotide was generally well tolerated; the most common adverse event was diarrhoea. Thus, linaclotide is a novel and effective single agent for the treatment of IBS-C in adults. [\hyperlink{Linaclotide}{PMID: 24293117}, Paul L McCormack et al., 2014]

\hypertarget{pmid_29319191}{L}inaclotide is a guanylate cyclase-C agonist approved in multiple countries to treat irritable bowel syndrome with constipation (IBS-C). China has unmet need for well-tolerated therapy that is effective in treating both bowel and abdominal symptoms of IBS-C. This trial evaluated linaclotide's efficacy and safety in IBS-C patients in China and other regions. This Phase 3, double-blind trial randomized IBS-C patients to once-daily oral 290-μg linaclotide or placebo at centers in China, North America, and Oceania. Patients reported bowel and abdominal symptoms daily; adverse events were monitored. Co-primary and secondary endpoints were tested using a predefined three-step serial gatekeeping multiple comparisons procedure. The intent-to-treat population included 839 patients (mean age = 41 years; 82\% female; 81\% Asian). The trial met all co-primary and secondary endpoints. Co-primary responder criteria were met by 60.0\% of linaclotide patients versus 48.8\% of placebo patients for abdominal pain/discomfort (≥ 30\% decrease for ≥ 6/12 weeks; P < 0.05), and 31.7\% of linaclotide versus 15.4\% of placebo patients for IBS degree of relief (score ≤ 2 for ≥ 6/12 weeks; P < 0.0001). Secondary 12-week change-from-baseline endpoints (spontaneous bowel movement/complete spontaneous bowel movement frequency, stool consistency, straining, abdominal pain, abdominal discomfort, and abdominal bloating) were significantly improved with linaclotide versus placebo (all P < 0.0001). Diarrhea was the most common adverse event (9.4\% linaclotide, 1.2\% placebo). Discontinuation rates due to diarrhea were low (0.7\% linaclotide, 0.2\% placebo). Once-daily 290-μg linaclotide improved bowel habits, abdominal symptoms, and global measures in a predominantly Chinese IBS-C population. [\hyperlink{Linaclotide}{PMID: 29319191}, Yunsheng Yang et al., 2018]

\hypertarget{pmid_29985664}{C}hronic constipation (CC) is a common gastrointestinal disorder with limited treatment options. Linaclotide is a potent peptide agonist of the guanylate cyclase-C receptor. This action activates intracellular conversion of guanosine 5-triphosphate to cyclic guanosine monophosphate resulting in the stimulation of intestinal fluid secretion. Linaclotide is a promising new agent for refractory constipation. Areas covered: All published articles regarding the development, clinical efficacy, and safety of linaclotide in treating CC were reviewed. Pharmacodynamics, pharmacokinetics, and metabolism of this secretagogue agent were examined. Clinical studies showed that linaclotide increases the number of spontaneous bowel movements and stool consistency scores. Overall, patients reported relief from abdominal discomfort and severity of constipation. Finally, linaclotide has a good safety profile, with diarrhea being the main side effect. Expert opinion: Linaclotide appears to be a well-tolerated and effective agent for patients with CC, and could be effectively combined with other drugs in patients with refractory constipation. However, data on the efficacy and safety of linaclotide in pediatric patients and in opioid-induced constipation are currently limited and more studies need to be undertaken. [\hyperlink{Linaclotide}{PMID: 29985664}, Gabrio Bassotti et al., 2018]

\hypertarget{pmid_23090647}{L}inaclotide, a potent guanylate cyclase C agonist, is a therapeutic peptide approved in the United States for the treatment of irritable bowel syndrome with constipation and chronic idiopathic constipation. We present for the first time the metabolism, degradation, and disposition of linaclotide in animals and humans. We examined the metabolic stability of linaclotide in conditions that mimic the gastrointestinal tract and characterized the metabolite MM-419447 (CCEYCCNPACTGC), which contributes to the pharmacologic effects of linaclotide. Systemic exposure to these active peptides is low in rats and humans, and the low systemic and portal vein concentrations of linaclotide and MM-419447 observed in the rat confirmed both peptides are minimally absorbed after oral administration. Linaclotide is stable in the acidic environment of the stomach and is converted to MM-419447 in the small intestine. The disulfide bonds of both peptides are reduced in the small intestine, where they are subsequently proteolyzed and degraded. After oral administration of linaclotide, <1\% of the dose was excreted as active peptide in rat feces and a mean of 3-5\% in human feces; in both cases MM-419447 was the predominant peptide recovered. MM-419447 exhibits high-affinity binding in vitro to T84 cells, resulting in a significant, concentration-dependent accumulation of intracellular cyclic guanosine-3',5'-monophosphate (cGMP). In rat models of gastrointestinal function, orally dosed MM-419447 significantly increased fluid secretion into small intestinal loops, increased intraluminal cGMP, and caused a dose-dependent acceleration in gastrointestinal transit. These results demonstrate the importance of the active metabolite in contributing to linaclotide's pharmacology. [\hyperlink{Linaclotide}{PMID: 23090647}, Robert W Busby et al., 2013]

\hypertarget{pmid_23083112}{L}inaclotide is a once-daily, orally administered, first-in-class agonist of guanylate cyclase-C that is minimally absorbed. It is being developed to treat gastrointestinal disorders by Ironwood Pharmaceuticals and its partners, Forest Laboratories (North America), Almirall (Europe) and Astellas Pharma (Asia-Pacific). Linaclotide has received its first global approval in the US for the treatment of constipation-predominant irritable bowel syndrome (IBS-C) and chronic idiopathic constipation (CIC), and a marketing submission has been filed in the EU for IBS-C. This article summarizes the milestones in the development of linaclotide leading to this first approval for IBS-C and CIC. This profile has been extracted and modified from the Adis R\&D Insight drug pipeline database. Adis R\&D Insight tracks drug development worldwide through the entire development process, from discovery, through pre-clinical and clinical studies to market launch. [\hyperlink{Linaclotide}{PMID: 23083112}, Vanessa McWilliams et al., 2012]

\hypertarget{pmid_36770675}{L}inaclotide is a 14-amino acid residue peptide approved by the FDA for the treatment of irritable bowel syndrome with constipation (IBS-C), which activates guanylate cyclase C to accelerate intestinal transit. Here we show a new method for the synthesis of linaclotide through the completely selective formation of three disulfide bonds in satisfactory overall yields via mild oxidation reactions of the solid phase and liquid phase, using 4-methoxytrityl (Mmt), diphenylmethyl (Dpm) and 2-nitrobenzyl (O-NBn) protecting groups of cysteine as substrate, respectively. [\hyperlink{Linaclotide}{PMID: 36770675}, Zhonghao Qiu et al., 2023]

\hypertarget{pmid_17854590}{O}ral linaclotide, a novel agonist of guanylate cylase-C, stimulates intestinal fluid secretion and transit, and decreases visceral hypersensitivity in animal studies. In healthy volunteers, linaclotide was safe, well tolerated, increased stool frequency, and decreased stool consistency and time to first bowel movement. This randomized, double-blind, placebo-controlled study evaluated the effects of oral linaclotide, 100 and 1000 microg once daily, in 36 women with constipation-predominant irritable bowel syndrome; colonic transit was normal in >50\% patients. Participants underwent 5-day baseline and 5-day treatment periods; gastrointestinal transit (by validated scintigraphy) and bowel function (by daily diaries) were assessed. Treatment effects were compared using analysis of covariance (baseline colonic transit as covariate) with pairwise comparisons of each dose vs placebo. There was a significant overall treatment effect on ascending colon emptying half-time (P = .015) and overall colonic transit at 48 hours (P = .02) but not overall transit at 24 hours (P = ns), with a significant acceleration by linaclotide 1000 microg vs placebo (P = .004 and P = .01, respectively) but no significant effect of linaclotide 100-microg dose. There were significant overall treatment effects on stool frequency, stool consistency, ease of passage, and time to first bowel movement with a strong dose response for stool consistency (overall, P < .001). No safety issues were identified. In women with constipation-predominant irritable bowel syndrome, linaclotide 1000 microg once daily significantly accelerated ascending colonic transit and altered bowel function. Further randomized controlled trials of clinical efficacy of linaclotide are warranted. [\hyperlink{Linaclotide}{PMID: 17854590}, Viola Andresen et al., 2007]

\hypertarget{pmid_30302125}{L}inaclotide, a guanylate cyclase C agonist, has been shown in clinical trials to improve symptoms of irritable bowel syndrome with constipation (IBS-C). Here we report data from a real-world study of linaclotide in the UK. This 1-year, multicentre, prospective, observational study in the UK enrolled patients aged 18 years and over initiating linaclotide for IBS-C. The primary assessment was change from baseline in IBS Symptom Severity Scale (IBS-SSS) score at 12 weeks, assessed in patients with paired baseline and 12-week data. Change from baseline in IBS-SSS score at 52 weeks was a secondary assessment. Adverse events were recorded. In total, 202 patients were enrolled: 185 (91.6\%) were female, median age was 44.9 years (range 18.1-77.2) and 84 (41.6\%) reported baseline laxative use. Mean (standard deviation) baseline IBS-SSS score was 339 (92), with most patients ( Linaclotide significantly improved IBS-SSS score at 12 and 52 weeks. These results provide insights into outcomes with linaclotide treatment over 1 year in patients with IBS-C in real-world clinical practice. [\hyperlink{Linaclotide}{PMID: 30302125}, Yan Yiannakou et al., 2018]

\hypertarget{pmid_33518646}{T}he efficacy and safety of linaclotide in elderly patients are poorly understood. Herein, we aimed to assess the efficacy and safety of linaclotide in elderly patients in real-world setting. We retrospectively enrolled consecutive patients who started linaclotide therapy at Sapporo Medical University Hospital from October 1, 2017 to December 31, 2019. The efficacy and safety of linaclotide were examined in relation to various factors, including age (<65 or ≥65 years) and dose (0.25 or 0.5 mg/d). Fifty-two patients were enrolled, 60\% of whom were over 65 years old and 40\% were female. Thirty-six patients received a linaclotide dose of 0.25 mg/d. The most common side effect was diarrhea, but there was no difference in the incidence of diarrhea between the elderly (64.5\%) and non-elderly patients (42.9\%, p=0.130). No significant difference was observed with respect to improvement in constipation in the elderly (83.9\%) and non-elderly patients (71.4\%, p=0.318). Additionally, the difference in efficacy of linaclotide in patients who received a reduced dose (80.6\%) vs. those who received the recommended dose (75.0\%) was not statistically significant (p=0.719). Multivariate analysis revealed that age, gender, and dose were not associated with diarrhea induced by linaclotide treatment. However, concurrent treatment with constipation-inducing medications [odds ratio (OR) 5.79, p=0.047] and linaclotide monotherapy (OR 11.1, p=0.040) were both risk factors contributing to diarrhea. Linaclotide is effective and safe for use in elderly patients. The incidence of diarrhea may increase when linaclotide is administered alone or concurrently used with medications that cause constipation. [\hyperlink{Linaclotide}{PMID: 33518646}, Tomoyuki Ishigo et al., 2021]

\hypertarget{pmid_36183690}{T}he use of linezolid is relatively safe for all age categories, including premature infants. The case of an extremely premature infant with hyperglycemia and lactic acidosis associated with linezolid is reported. A 350-g male infant was born at 24 weeks by cesarean section. His Apgar scores were 1 and 1 at 1 and 5 min, respectively. On the day of life (DOL) 7, linezolid was started at a dose of 10 mg/kg/dose every 8 h for a catheter-related blood stream infection caused by methicillin-resistant coagulase-negative Staphylococci. After linezolid was given, serum lactate and glucose levels increased gradually. After discontinuation of linezolid on DOL 16, hyperglycemia and lactic acidosis improved immediately. In conclusion, a rare case of an extremely premature infant with hyperglycemia and lactic acidosis associated with linezolid was reported. It is crucial to monitor glucose levels along with lactate and pH levels during linezolid therapy. [\hyperlink{Linaclotide}{PMID: 36183690}, Takafumi Asai et al., 2022]

\hypertarget{pmid_21764828}{B}ecause of the spread of drug-resistant Gram-positive bacteria, the use of linezolid for treating severe infections is increasing. However, clinical experience in the paediatric population is still limited. We undertook a multicentre study to analyse the use of linezolid in children. Hospitalized children treated with linezolid for a suspected or proven Gram-positive or mycobacterial infection were analysed retrospectively. Side effects were investigated, focusing on younger children and long-term treatments. Seventy-five patients (mean age 6.8 years, range 7 days to 17 years) were studied. Mean ± SD linezolid treatment duration was 26.13 ± 17 days. Clinical cure was achieved in 74.7\% of patients. The most frequent adverse events were diarrhoea and vomiting. Two patients had severe anaemia, two neutropenia and one thrombocytopenia. Two cases of grade 3 liver function test elevation and one case of pancreatitis were reported. The overall frequency of adverse events was similar between patients treated for >28 days and those receiving shorter treatments (30.8\% versus 28.6\%, P = 0.84). Children aged <2 years received linezolid for a shorter duration than older children (21.2 days versus 28.4 days, P = 0.05), whereas the frequency of adverse events was similar in the two age groups. In our paediatric population, linezolid appeared safe and effective for the treatment of selected Gram-positive and mycobacterial infections. The adverse reactions encountered were reversible and appeared comparable to those reported in paediatric clinical trials. Nevertheless, the potential for haematological toxicity of linezolid in children means that careful monitoring is required during treatment. [\hyperlink{Linaclotide}{PMID: 21764828}, Silvia Garazzino et al., 2011]

\hypertarget{pmid_30353619}{A} previous phase II dose-ranging study of linaclotide in a Japanese chronic constipation (CC) population showed that 0.5 mg was the most effective dose. This study aimed to verify the hypothesis that 0.5 mg of linaclotide is effective and safe in Japanese CC patients. This was a Japanese phase III randomized, double-blind, placebo-controlled (part 1), and long-term, open-label extension (part 2) study of linaclotide. CC patients (n = 186) diagnosed using the Rome III criteria were randomly assigned to linaclotide 0.5 mg (n = 95) or placebo (n = 91) for a 4-week double-blind treatment period in part 1, followed by an additional 52 weeks of open-label treatment with linaclotide in part 2. The primary efficacy endpoint was the change from baseline in weekly spontaneous bowel movement (SBM) frequency at the first week. Secondary endpoints included responder rate for complete SBM (CSBM), changes in stool consistency, and severity of straining. Part 1: Change in weekly mean SBM frequency in the first week of treatment with linaclotide (4.02) was significantly greater than that with placebo (1.48, P < 0.001). Linaclotide produced a higher CSBM responder rate (52.7\%) compared to placebo (26.1\%, P < 0.001). Part 2: Patients continued to show improved SBM frequency with linaclotide. Through parts 1 and 2, the most common drug-related adverse event was mild and occasionally moderate diarrhea. The results of this study indicate that a linaclotide dose of 0.5 mg/day is effective and safe in Japanese CC patients. [\hyperlink{Linaclotide}{PMID: 30353619}, Shin Fukudo et al., 2019]

\hypertarget{pmid_31892640}{W}e evaluated the effectiveness and tolerability of linaclotide, a minimally absorbed guanylate cyclase-C agonist, in patients with irritable bowel syndrome with constipation (IBS-C) in routine clinical practice. A multicentre, non-interventional study conducted between December 2013 and November 2015 across 31 primary, secondary and tertiary centres in Austria and Switzerland. The study enrolled 138 patients aged ≥18 years with moderate-to-severe IBS-C. Treatment decision was at the physician's discretion. Patients with known hypersensitivity to the study drug or suspected mechanical obstruction were excluded. The mean age of participants was 50 years, and >75\% of the patients were women. 128 patients completed the study. Data were collected at weeks 0 and 4 in Austria and weeks 0, 4 and 16 in Switzerland. The primary effectiveness endpoints included severity of abdominal pain and bloating (11-point numerical rating scale [0=no pain/bloating to 10=worst possible pain/bloating]), frequency of bowel movements and physicians' global effectiveness of linaclotide. Treatment-related adverse events (AEs) were recorded. Following a 4-week treatment period, the mean intensity score of abdominal pain was reduced from 5.8 at baseline to 2.7, while the bloating intensity score was reduced from 5.8 at baseline to 3.1e (both indices p<0.001). The frequency of mean weekly bowel movements increased from 2.1 at baseline to 4.5 at week 4 (p<0.001). Global effectiveness and tolerability of linaclotide were assessed by the treating physicians as 'good' or 'excellent' in >70\% of patients. In total, 31 AEs were reported in 22 patients, the most common being diarrhoea, reported by 6 (7\%) and 8 (15.4\%) patients in Austria and Switzerland, respectively. Patients with IBS-C receiving linaclotide experienced effective treatment of moderate-to-severe symptoms in routine clinical practice. Linaclotide was safe and well tolerated and no new safety concerns were raised, supporting results from previous clinical trials. [\hyperlink{Linaclotide}{PMID: 31892640}, Daniel Pohl et al., 2019]

\hypertarget{pmid_11144380}{L}inezolid is an oxazolidinone antibiotic with excellent in vitro activity against a number of Gram-positive organisms including antibiotic-resistant isolates. The safety and pharmacokinetics of intravenously administered linezolid were evaluated in children and adolescents to examine the potential for developmental dependence on its disposition characteristics. Fifty-eight children (3 months to 16 years old) participated in this study; 44 received a single 1.5-mg/kg dose and 14 received a single 10-mg/kg dose of linezolid administered by intravenous infusion. Repeated blood samples (n = 10 in children > or = 12 months; n = 8 in children 3 to 12 months) were obtained during 24 h after drug administration, and linezolid was quantitated from plasma by high performance liquid chromatography with mass spectrometry detection. Plasma concentration vs. time data were evaluated with a model independent approach. Linezolid was well-tolerated by all subjects. The disposition of linezolid appears to be age-dependent. A significant although weak correlation between age and total body clearance was observed. The mean (+/- SD) values for elimination half-life, total clearance and apparent volume of distribution were 3.0 +/- 1.1 h, 0.34 +/- 0.15 liter/h/kg and 0.73 +/- 0.18 liter/kg, respectively. Estimates of total body clearance and volume of distribution were significantly greater in children than historical values of adult data. As such maximum achievable linezolid plasma concentrations were slightly lower in children, and concentrations 12 h after a single 10-mg/kg dose were below the MIC90 for selected pathogens with in vitro susceptibility to the drug. Based on these data a linezolid dose of 10 mg/kg given two to three times daily would appear appropriate for use in pediatric therapeutic clinical trials of this agent. [\hyperlink{Linaclotide}{PMID: 11144380}, G L Kearns et al., 2000]

\hypertarget{pmid_24706161}{L}inezolid is an oxazolidinone antibacterial agent, with activity against Gram-positive bacteria. This study aimed to evaluate the efficacy and safety of linezolid in children with infections caused by Gram-positive pathogens. A systematic search was conducted by two independent reviewers to identify published studies up to September 2013. The accumulated relevant literature was subsequently systematically reviewed, and a meta-analysis was conducted. Eligible studies were randomized controlled trials assessing the clinical efficacy and safety of linezolid in children versus other antimicrobial agents for infections caused by Gram-positive bacteria. The primary outcome was treatment success in patients who received at least one dose of study drug, had clinical evidence of disease, and had complete follow-up. Meta-analysis was conducted with random effects models because of heterogeneity across the trials. Two randomized controlled trials (RCTs), involving 815 patients, were included. Linezolid was slightly more effective than control antibiotic agents, but the difference was not statistically significant [odds ratio (OR) = 1.39, 95 \% confidence interval (CI) 0.98-1.98]. Treatment with linezolid was not associated with more adverse effects in general (OR = 0.61, 95 \% CI 0.25-1.48). Eradication efficiency did not differ between linezolid and control regimens, but the sample size for these comparisons was small. The use of linezolid cannot be steadily supported from the results of the current meta-analysis. It appears to be slightly more effective than control antibiotic agents, but the difference was not significant, and the serious limitations present in this study restrict its use. Further studies providing evidence for clinical and microbiological efficacy of linezolid will support its use. [\hyperlink{Linaclotide}{PMID: 24706161}, Maria Ioannidou et al., 2014]

\hypertarget{pmid_30084233}{B}ased on the previous phase II/III studies of irritable bowel syndrome with constipation (IBS-C) in Japan that demonstrated the efficacy and safety of linaclotide 0.5 mg/d, we evaluated linaclotide at doses of 0.5 mg/d and lower in the treatment of Japanese patients with chronic constipation (CC). This was a phase II randomized, double-blind, placebo-controlled, dose-finding study of linaclotide for Japanese patients with CC (n = 382, 64 men, 318 women, age 20-75). After a baseline period of two weeks, patients were randomized to receive placebo (n = 80), or 0.0625 mg (n = 82), 0.125 mg (n = 71), 0.25 mg (n = 73) or 0.5 mg (n = 76) of linaclotide during a two-week treatment period. The primary efficacy endpoint was change from baseline in weekly spontaneous bowel movement (SBM) frequency during the first week. Secondary endpoints included complete SBM (CSBM) responder rates and IBS-QOL. Safety and adverse events were also evaluated. The change in SBM frequency during the first week (mean) was 3.89, 3.11, 3.87, and 3.85 for 0.0625 mg, 0.125 mg, 0.25 mg, and 0.5 mg for linaclotide, significantly higher than for placebo (1.91, P < 0.05). The CSBM responder, which is an important parameter, showed the greatest improvement at the 0.5 mg during the 2 week. The most frequent adverse event in the linaclotide groups was diarrhea. Our results suggest that 0.0625, 0.125, 0.25, and 0.5 mg/d are effective doses of linaclotide for treating CC in Japanese patients. ClinicalTrials.gov: NCT02425722, supported by Astellas Pharma, Inc. [\hyperlink{Linaclotide}{PMID: 30084233}, Shin Fukudo et al., 2018]

\section*{Ritlecitinib Tosylate}
\subsection*{Result}
\subsubsection*{Answer}

Unknown

\subsubsection*{{Explanation}}
\hypertarget{Ritlecitinib Tosylate}
After reviewing all available abstracts, there are no abstracts specifically about Ritlecitinib Tosylate. Therefore, there is no evidence from targeted studies regarding the safety of Ritlecitinib Tosylate in children of any age range. As per your criteria, this means the safety of Ritlecitinib Tosylate in children is unknown, since no targeted pediatric safety studies are available in the abstracts provided.

\subsection*{Abstracts}
\hypertarget{pmid_36397383}{T}ofacitinib is an orally administered selective Janus kinase inhibitor. Its efficacy and safety in adults with moderately to severely active ulcerative colitis (UC) have been evaluated in clinical trials; however, its efficacy in pediatric patients with UC is limited. A 9-years-old boy was referred to our hospital with chief complaints of diarrhea and bloody stool for 2 months. Colonoscopy revealed total colitis-type UC. His pediatric UC activity index score was 40, indicating moderately active UC. UC. Vedolizumab, golimumab, and ustekinumab were introduced because of steroid-resistant refractory UC; however, none of these biologics were effective or the effect was short-lived. Therefore, tofacitinib was administered 5 mg twice a day. The patient achieved UC remission after tofacitinib treatment, leading to maintained remission without adverse events. To the best of our knowledge, this is the first pediatric case of moderately active UC that was successfully treated with tofacitinib in Japan. Tofacitinib is a safe drug for pediatric patients with moderately active UC. Even in steroid-dependent cases refractory to other biologics, tofacitinib can result in remission induction and maintenance effects. In children and adults, high-dose tofacitinib during induction therapy may be unnecessary to reduce adverse events. [\hyperlink{Ritlecitinib Tosylate}{PMID: 36397383}, Toshihiko Kakiuchi et al., 2022]

\hypertarget{pmid_25593242}{R}ituximab (RTX) has been used to treat many pediatric autoimmune conditions. We investigated the safety and efficacy of RTX in a variety of pediatric autoimmune diseases, especially systemic lupus erythematosus (SLE). Retrospective study of children treated with RTX. Effectiveness data was recorded for patients with at least 12 months of followup; safety data was recorded for all subjects. The study included 104 children; 50 had SLE. Improvements in corticosteroid dosage, physician's global assessment of disease activity, and SLE-associated markers of disease activity were seen. The incidence of hospitalized infections was similar to previous studies of patients with childhood-onset SLE. RTX can be safely administered to children and appears to contribute to decreased disease activity and steroid burden. [\hyperlink{Ritlecitinib Tosylate}{PMID: 25593242}, Ajay Tambralli et al., 2015]

\hypertarget{pmid_34091545}{T}ofacitinib, a selective Janus kinase inhibitor, effectively induces and maintains remission in adults with inflammatory bowel disease (IBD), but data are limited in children. This study aimed to evaluate the efficacy and safety of tofacitinib for medically refractory pediatric-onset IBD. This single-center retrospective study included subjects ages 21 years and younger who started tofacitinib for medically refractory IBD. Clinical activity indices, clinical response, steroid-free remission, biochemical response, and adverse events (AEs) were evaluated over 52 weeks. Twenty-one subjects, 18 with ulcerative colitis or indeterminate IBD, received tofacitinib. At the end of the 12-week induction period, 9 out of 21 (42.9\%) subjects showed clinical response and 7 out of 21 (33.3\%) were in steroid-free remission. Of evaluable subjects at 52 weeks, 7 out of 17 (41.2\%) showed clinical response and were in steroid-free remission. Of those remaining on tofacitinib at 1 year, none required concomitant systemic corticosteroids. Tofacitinib was discontinued in 8 subjects because of refractory disease, including 8 who ultimately underwent colectomy, and in 1 subject who developed a sterile intra-abdominal abscess. There were no instances of thrombi, zoster reactivation, or clinically significant hyperlipidemia, all of which were AEs of interest. There is limited experience with tofacitinib in pediatric IBD. In this cohort, tofacitinib induced rapid clinical response with sustained efficacy in nearly half of subjects. This study provides encouraging evidence for the efficacy and safety of tofacitinib as part of the treatment paradigm for young individuals with moderate-to-severe IBD. Larger, well-powered, prospective studies are warranted. [\hyperlink{Ritlecitinib Tosylate}{PMID: 34091545}, Hillary Moore et al., 2021]

\hypertarget{pmid_31820716}{B}iologic drugs (BD) have been game-changers in rheumatic diseases; however, severe hypersensitivity reactions concerning anaphylaxis may limit their use. Desensitisation is a crucial option that is safe and effective to maintain patients on the preferred drug. Herein we report 84 Rapid Drug Desensitisation (RDD) procedures with rituximab and tocilizumab in children with rheumatic diseases. The study was conducted as a retrospective chart review of patients who received tocilizumab or rituximab therapy between January 2010 and December 2018. The results of RDD with tocilizumab and rituximab were documented. The study group consisted of 53 patients (11.6±4.5 years, 67.9\% female) with rheumatic disease who had used tocilizumab (64.1\%, 1007 infusions) or rituximab (35.8\%, 73 infusions). Five patients (14.7\%) had experienced anaphylaxis with tocilizumab and two patients (10.5\%) with rituximab. Anaphylaxis was grade II in four cases whereas it was grade III in the remaining three children. Skin testing with the culprit BD performed in five children yielded positive results. We performed 65 RDDs with tocilizumab in 3 patients and 19 RDDs with rituximab in two patients. No reactions were recorded in 97.6\% of the procedures. We observed one anaphylaxis during the 5th RDD of tocilizumab. After modifying the protocol, this patient continued tocilizumab RDD uneventfully. RDD is a groundbreaking innovation which ensures giving the full target doses while protecting the patient against severe hypersensitivity reactions (HSRs) and anaphylaxis. As BD use increases in childhood, management of HSRs to BD will become more complicated, necessitating an increased need for RDD in clinical practice. [\hyperlink{Ritlecitinib Tosylate}{PMID: 31820716}, Selcan Demir et al., ]

\hypertarget{pmid_7857353}{T}he efficacy and safety of pidotimod ((R)-3-[(S)-(5-oxo-2-pyrrolidinyl)carbonyl]-thiazolidine-4-carboxylic acid, PGT/1A, CAS 121808-62-6) were rated in a child population with a remote history of recurrent respiratory infections (RRI). This randomized double-blind multicenter clinical trial versus placebo, stratified by age groups, involved 748 children recruited in 69 Medical Centres. The trial consisted of a 60-day treatment period and a 90-day follow-up. At the end of the treatment period the pidotimod group showed a significant decrease in the number of RRI episodes and associated symptoms vs control group. As a consequence, there was a significant decrease in the number of days of absence from kindergarten or school and in the consumption of antibiotics and symptomatic drugs. Safety was good. The effect of the drug persisted after its withdrawal throughout the whole 90-day follow-up period. During this period there was a significantly lower RRI incidence rate in the pidotimod group than in the placebo group (p < 0.01). Because of its efficacy and safety, pidotimod may be rated as an excellent drug in the RRI management in children. [\hyperlink{Ritlecitinib Tosylate}{PMID: 7857353}, P Careddu et al., 1994]

\hypertarget{pmid_24768216}{R}ituximab is a B-cell therapy used off-label to reduce relapses in adult demyelinating diseases. There is limited knowledge of its clinical use in pediatric neuromyelitis optica and multiple sclerosis. Demyelinating diseases in children can have high morbidity, and B-cell therapies hold promise for those with a severe course. Our study investigates the clinical experience of safety and efficacy with rituximab in children with demyelinating diseases of the central nervous system. This is a retrospective case series of 11 patients with pediatric neuromyelitis optica and multiple sclerosis who received at least one rituximab infusion at the Pediatric Multiple Sclerosis Clinic, University of California, San Francisco. Each patient was infused up to 1000 mg twice 2 weeks apart. Patients were monitored prospectively, and relapse events, laboratories, and adverse reactions were recorded. Eight children with neuromyelitis optica, two with relapsing-remitting multiple sclerosis and one with secondary-progressive multiple sclerosis received rituximab treatment. The median number of cycles was 3. Most patients (82\%, n = 9) experienced reduction of relapses after initiating rituximab. There were no serious infections. Infusion reactions were reported in three patients and managed successfully in subsequent infusions with increased pretreatment (dexamethasone and diphenhydramine) and use of slower infusion rates. Rituximab was not discontinued in any child because of side effects; two switched treatment therapy after 4.5 and 11 months because of relapses. The use of rituximab in our pediatric neuromyelitis optica and multiple sclerosis cohort was overall safe and effective. Larger studies should confirm our observations. [\hyperlink{Ritlecitinib Tosylate}{PMID: 24768216}, Shannon J Beres et al., 2014]

\hypertarget{pmid_3172459}{A} total of 22 patients with acute pediatric infections was treated with rokitamycin (TMS-19-Q, RKM) dry syrup, a new macrolide antibiotic developed by Toyo Jozo Co., Ltd., Ohhito, Japan, to investigate its clinical efficacy. 1. A girl of an age 4 years 2 months (weighing 16.5 kg) was administered orally 10 mg/kg of RKM, and a boy of an age 8 years 7 months (weighing 24.5 kg), 15 mg/kg, and blood concentrations of RKM in these subjects were measured to investigate its absorption and excretion. Blood concentrations of the drug reached a peak of 0.84 microgram/ml in an hour after the administration in the girl, 0.72 microgram/ml in 30 minutes in the boy, with T1/2 of 0.86 and 1.82 hours, respectively. Their 6-hour cumulative urinary recovery rates were 2.79 and 2.13\%, respectively. 2. A total of 20 patients was treated with RKM dry syrup. These patients included 3 with acute pharyngitis, one with acute tonsillitis, 4 with hemolytic streptococcal infections, 7 with acute bronchitis, 2 with pneumonia, another 2 with pertussis, and one with Campylobacter enteritis. The treatment was effective in 18 of them with a clinical efficacy of 90.0\%. 3. Bacteriological responses to RKM dry syrup were as follows: eradication of pathogens in 5, pathogens decreased in 3, and no changes were observed in 3 of 12 patients from whom pathogens had been isolated prior to the treatment, thus the eradication rate was 45.5\% with the exception of 1 patient whose bacteriological response was unknown.(ABSTRACT TRUNCATED AT 250 WORDS) [\hyperlink{Ritlecitinib Tosylate}{PMID: 3172459}, M Minamitani et al., 1988] Rituximab has been widely used off-label as a second line treatment for children with immune thrombocytopenia (ITP). However, its role in the management of pediatric ITP requires clarification. To understand and interpret the available evidence, we conducted a systematic review to assess the efficacy and safety of rituximab for children with ITP. We searched MEDLINE, EMBASE, Cochrane Library, CBM, CNKI, abstract databases of American Society of Hematology, American Society of Clinical Oncology and Pediatric Academic Society. Clinical studies published in full text or abstract only in any language that met predefined inclusion criteria were eligible. Efficacy analysis was restricted to studies enrolling 5 or more patients. Safety was evaluated from all studies that reported data of toxicity. 14 studies (323 patients) were included for efficacy assessment in children with primary ITP. The pooled complete response (platelet count ≥ 100 × 10(9)/L) and response (platelet count ≥ 30 × 10(9)/L) rate after rituximab treatment were 39\% (95\% CI, 30\% to 49\%) and 68\% (95\%CI, 58\% to 77\%), respectively, with median response duration of 12.8 month. 4 studies (29 patients) were included for efficacy assessment in children with secondary ITP. 11 (64.7\%) of 17 patients associated with Evans syndrome achieved response. All 6 patients with systemic lupus erythematosus associated ITP and all 6 patients with autoimmune lymphoproliferative syndrome associated ITP achieved response. 91 patients experienced 108 adverse events associated with rituximab, among that, 91 (84.3\%) were mild to moderate, and no death was reported. Randomized controlled studies on effect of rituximab for children with ITP are urgently needed, although a series of uncontrolled studies found that rituximab resulted in a good platelet count response both in children with primary and children secondary ITP. Most adverse events associated with rituximab were mild to moderate, and no death was reported. [\hyperlink{Ritlecitinib Tosylate}{PMID: 3172459}, Yi Liang et al., 2012]

\hypertarget{pmid_33095287}{T}his phase 1 study aimed to determine the safety, tolerability and recommended phase 2 dose (RP2D) of crizotinib in combination with cytotoxic chemotherapy for children with refractory solid tumors and ALCL. Pediatric patients with treatment refractory solid tumors or ALCL were eligible. Using a 3 + 3 design, crizotinib was escalated in three dose levels: 165, 215, or 280 mg/m Forty-four eligible patients were enrolled, 39 were evaluable for toxicity. Parts A and B were terminated due to concerns regarding palatability and tolerability of the OS. In Part C, crizotinib, FC 215 mg/m The RP2D of crizotinib FCs in combination with cyclophosphamide and topotecan was 215 mg/m The trial is registered as NCT01606878 at Clinicaltrials.gov. [\hyperlink{Ritlecitinib Tosylate}{PMID: 33095287}, Emily Greengard et al., 2020]

\hypertarget{pmid_34767543}{B}ACKGROUND Ulcerative colitis (UC) is a chronic autoimmune inflammatory disease of the colon that infrequently affects children. The disease requires immunosuppressive therapy to achieve remission and keep the disease in remission. Currently, many therapies are approved for use in pediatric patients with UC, including steroid, 5-aminosalicylic acid (5-ASA), azathioprine, and biologic therapy with anti-tumor necrosis factor (TNF) inhibitors. Despite their efficacy, many patients have refractory severe disease that fails therapy and may require surgical interventions. Recently, the small molecule Janus Kinase (JAK) inhibitor tofacitinib has been approved for moderate to severe UC that fails biologic therapy in adults. However, the safety and efficacy of this drug has not been tested in pediatric UC patients. CASE REPORT We describe a case of a 13-year-old girl with 2-year history of severe UC who had secondary loss response to both infliximab and adalimumab over 2 years, despite adequate trough serum drug levels and the concomitant use of azathioprine. She was also dependent on steroid to control her disease. Infectious work-ups were always negative for infectious organisms. She was then successfully treated with tofacitinib 5 mg orally twice daily. She went into complete clinical, endoscopic, and steroid-free remission. CONCLUSIONS This case report highlights the safety and efficacy of tofacitinib in pediatric patients with severe refractory UC, potentially avoiding proctocolectomy in this young patient population. Future research should study the role of tofacitinib in patients with moderate to severe UC in children. [\hyperlink{Ritlecitinib Tosylate}{PMID: 34767543}, Refaa Alajmi et al., 2021]

\hypertarget{pmid_32865865}{E}ven though rituximab has emerged as standard of care for the management of high-risk pediatric Burkitt lymphoma (BL), its safety in children from the low-middle-income countries (LMICs) remains to be proven. We herein report our experience of using rituximab in children with BL. All patients diagnosed with BL between January 2015 and December 2017 were treated in a risk-stratified manner with either the modified MCP-842 or modified LMB protocol. Patients with poor response to MCP-842 were switched to the LMB-salvage regimen. In addition, rituximab was given to selected high-risk patients. Forty-two (49.4\%) of 85 patients with BL received rituximab. The incidence of febrile neutropenia (90.5\% vs 67.4\%; P = 0.02), pneumonia (38.1\% vs 11.6\%; P = 0.005), intensive care unit admissions (54.5\% vs 17.6\%; P = 0.002), and toxic deaths (26.2\% vs 9.3\%; P = 0.04) was higher among BL patients who received rituximab. Pneumonia was fatal in 11 of 16 (69\%) patients who received rituximab. On multivariate analysis, rituximab continued to be significantly associated with toxic deaths ( OR: 11.45 [95\% CI: 1.87-70.07; P = 0.008]). The addition of rituximab to intensive chemotherapy resulted in an inferior one-year event-free survival (49.4\% ± 8.1\% vs 79.3\% ± 6.5\%; P = 0.025) and one-year overall survival (63.1\% ± 8.5\% vs 91.8\% ± 4.5\%; P = 0.007) with no improvement in one-year relapse-free survival (78.3\% ± 7.3\% vs 83.9\% ± 6.0\%; P = 0.817). Rituximab was associated with increased toxicities and toxic deaths in our patients. The potential immunomodulatory effect of rituximab and increased susceptibility to infections in patients from LMICs have to be carefully considered while choosing this drug in the treatment of BL in resource-constrained settings. [\hyperlink{Ritlecitinib Tosylate}{PMID: 32865865}, Shyam Srinivasan et al., 2020]

\hypertarget{pmid_15689912}{T}his study examined the efficacy and safety of rituximab in children with chronic immune thrombocytopenic purpura. Twenty-four patients, 2 to 19 years of age, with platelet counts <30,000/mcL (microliter 2), received 375 mg/m 2 rituximab in 4 weekly doses. Platelet response was characterized as complete (CR) if a count >150,000/mcL was achieved; partial (PR) if 50,000 to 150,000/mcL; minimal (MR) if the count increased by >20,000/mcL to a peak count >30,000/mcL but <50,000/mcL; or no response (NR). Fifteen of 24 patients (63\%) achieved a CR lasting 4 to 30 months, 9 of which are ongoing. Two had PRs lasting 4 and 6 months; 2 had MRs lasting 5 and 8 months, and 5 did not respond. Pruritus, urticaria, and throat tightness (but no respiratory distress) occurred with the first infusion in a small number of children. Three patients had serum sickness after the first, second, and third infusions, respectively. No increased frequency or severity of infections was seen, although immunoglobulin levels decreased to below the normal range in 6 of 14 cases. Rituximab may be a useful treatment for chronic immune thrombocytopenic purpura in children with a >50\% CR rate lasting an average of 13 months, with 9 of 15 CRs ongoing (8 lasted 6 months or longer). There was no substantial toxicity other than transient serum sickness. [\hyperlink{Ritlecitinib Tosylate}{PMID: 15689912}, Julie Wang et al., 2005]

\hypertarget{pmid_20819318}{A}llergic rhinitis (AR) and chronic idiopathic urticaria (CIU) are common causes of substantial illness and disability in preschool children. Antihistamines are commonly used to treat preschool children with these conditions, but their use is based mostly on extrapolated efficacy from adult populations; it is thus important to characterize the safety of antihistamines in the pediatric population. This study was designed to assess the safety of levocetirizine dihydrochloride oral liquid drops in infants and children with AR or CIU. Two multicenter, double-blind, randomized, parallel-group studies randomized infants aged 6-11 months (study 1, n = 69) and children aged 1-5 years (study 2, n = 173) to levocetirizine, 1.25 mg (q.d. or b.i.d., respectively), or placebo for 2 weeks, using a 2:1 ratio. Safety evaluations included treatment-emergent adverse events (TEAEs), vital signs, electrocardiographic (ECG) assessments, and laboratory tests. The overall incidence of TEAEs was similar between levocetirizine and placebo in both studies. Most TEAEs were mild or moderate in intensity. TEAEs prompted discontinuation of therapy in three patients receiving levocetirizine in study 1. No clinically relevant changes from baseline in vital signs or laboratory parameters were apparent in either study; changes from baseline in these evaluations were similar between groups. No significant changes were observed in ECG parameters, including corrected QT interval. Levocetirizine, 1.25 and 2.5 mg/day, was well tolerated in infants aged 6-11 months and in children aged 1-5 years, respectively, with AR or CIU. [\hyperlink{Ritlecitinib Tosylate}{PMID: 20819318}, Frank Hampel et al., ]

\hypertarget{pmid_33314568}{C}ytokine release syndrome (CRS) and immune effector cell-associated neurotoxicity are two major CAR T related toxicities. With the interventions of Tocilizumab and steroids, many patients can recover from severe CRS. However, some patients are refractory to steroids and develop life-threatening consequences. Ruxolitinib is an oral JAKs inhibitor and promising drug in inflammatory diseases. In this pilot study, we evaluate the efficacy of Ruxolitinib in CRS. Of 14 r/r B-ALL children who received CD19 or CD22 CAR T cell therapies, 4 patients developed severe (≥grade 3) CRS with symptoms that were not alleviated with high-dose steroids and thus received ruxolitinib. Rapid resolution of CRS symptoms was observed in 4 patients after ruxolitinib treatment. Serum cytokines significantly decreased after ruxolitinib intervention. All patients achieved complete remission on day 30 after infusion, and we could still detect CAR T expansion in vivo despite usage of ruxolitinib. There were no obvious adverse events related to ruxolitinib. In vitro assays revealed that ruxolitinib could dampen CAR T expansion and cytotoxicity, suggesting that the timing and dosage of ruxolitinib should be carefully considered to avoid dampening anti-leukaemia response. Our results suggest that ruxolitinib is active and well tolerated in steroid-refractory and even life-threatening CRS. [\hyperlink{Ritlecitinib Tosylate}{PMID: 33314568}, Jing Pan et al., 2021]

\hypertarget{pmid_32602383}{T}o assess the efficacy and safety of omalizumab in children with moderate-to-severe asthma. We systematically searched MEDLINE, EMBASE, and Cochrane for randomized controlled trials (RCTs ) (inception to January 2020). All RCTs which were conducted in childhood and adolescence with asthma and compared the efficacy or safety of omalizumab were adopted. Three studies with four publications including 1380 pediatric patients met our criteria. For children with moderate-to-severe asthma, omalizumab decreased asthma exacerbations rate (OR 0.51, 95\% CI: 0.44-0.58,  These findings suggested that omalizumab had beneficial effects on moderate-to-severe asthma in children. Patients may benefit more from long-term use of omalizumab. In addition, omalizumab reduces the rate of serious adverse events requiring hospitalizations. [\hyperlink{Ritlecitinib Tosylate}{PMID: 32602383}, Zhuo Fu et al., 2021]

\hypertarget{pmid_28602379}{T}o evaluate the efficacy and safety of rituximab for treating pediatric systemic lupus erythematosus (pSLE). We performed a systematic review to evaluate the efficacy and safety of rituximab in children with pSLE. Data from studies performed before July 2016 were collected from MEDLINE, the Cochrane Library, Scopus, and the International Rheumatic Disease Abstracts, with no language restrictions. Study eligibility criteria included clinical trials and observational studies with a minimal sample size of 5 patients, regarding treatment with rituximab in patients with refractory pSLE (aged <18 years at the time of diagnosis). Independent extraction of articles was performed by 2 investigators using predefined data fields. Twelve case series met the criteria for data extraction for the systematic review with a good quality assessment according to an 18-criteria checklist using a modified Delphi method. Among them, 3 studies were multicenter and 3 were prospective. The total number of patients was 272. Studies collected patients with active disease refractory to steroids and immunosuppressant drugs. Refractory lupus nephritis was the most common indication (33\%). Acceptable evidence suggested improvements in renal, neuropsychiatric and haematological manifestations, disease activity, complement and anti-double stranded Desoxy-Nucleo-Adenosine, with a steroid-sparing effect. However, there was poor evidence suggesting efficacy on arthralgia, photosensitivity, and mucocutaneous manifestations of SLE in children. An overall acceptable safety profile with few major adverse events was shown. Rituximab exhibited a satisfactory profile regarding efficacy and safety indicating that this agent is a promising therapy for pSLE and should be further investigated. [\hyperlink{Ritlecitinib Tosylate}{PMID: 28602379}, Ines Mahmoud et al., 2017]

\hypertarget{pmid_11218055}{T}opiramate has been shown to be safe and effective in refractory partial epilepsy in children. Pharmacokinetic studies show that the clearance of topiramate is greater in children than in adults; therefore, higher doses may be needed in children than adults. It is generally well tolerated, except for cognitive dysfunction. Weight loss and the risk of renal stones can be significant in some cases. However, when compared with other anticonvulsant medications, topiramate has few serious idiosyncratic reactions such as rash, hematologic reactions, and hepatotoxicity. [\hyperlink{Ritlecitinib Tosylate}{PMID: 11218055}, K D Holland et al., 2000]

\hypertarget{pmid_36228496}{R}TX is used off-label in several neurological inflammatory diseases in adults children patients. We conducted a study to assess indications and safety of rituximab (RTX) for children and to identify risk factors for early B-cell repopulation. A single-center retrospective study of children treated with RTX for a neurological disease between May 31, 2010, and May 31, 2020, was performed. A total of 77 children (median age, 8.9 years) were included. RTX was mostly used as second-line therapy in all groups of diseases (68\%). Median dose was 1500 mg/m This study confirms the good tolerance of RTX in the treatment of specific neurological disorders in a pediatric population. It also highlights risk factors for early B-cell repopulation and underlines the importance of B-cell monitoring. [\hyperlink{Ritlecitinib Tosylate}{PMID: 36228496}, Ai Tien Nguyen et al., 2022]

\hypertarget{pmid_3221437}{A} total of 29 patients with pediatric infections was treated orally with 21.4-44.4 mg/kg/day of rokitamycin (RKM) dry syrup. The obtained results are summarized as follows. 1. Clinical responses to RKM in 24 evaluable patients were excellent in 2 and good in 3 of 5 patients with tonsillitis and laryngitis; excellent in 3 and good in 5 of 8 patients with bronchitis; excellent in 3, good in 2 and fair in one of 6 patients with bronchopneumonia; excellent in 2 and good in the other of 3 patients with psittacosis; and excellent in 2 of 2 patients with Campylobacter colitis. The overall efficacy rate was 95.8\%. 2. Bacteriological responses to the drug were: reduction in 1 and no change in the other of 2 strains of Streptococcus pyogenes; eradication of a strain of Streptococcus pneumoniae and 2 strains of Staphylococcus aureus; eradication of 2 and no change in 3 of 5 strains of Haemophilus influenzae; and eradication of 2 out of 2 strains of Campylobacter spp. 3. Diarrhea was complained of as an adverse reaction to the RKM medication by 1 patient, abdominal pain was reported by another, and anorexia by another of the 27 patients treated. Laboratory examination was performed on some patients, but not abnormal test values were found except in 1 case showing an increase in platelet count from 27.6 to 78.2 X 10(4)/mm8. The results suggested that RKM dry syrup might be a very useful and safe drug for the treatment of pediatric infections. [\hyperlink{Ritlecitinib Tosylate}{PMID: 3221437}, K Sunakawa et al., 1988]

\hypertarget{pmid_28079500}{T}ofacitinib is an oral Janus kinase inhibitor for the treatment of rheumatoid arthritis (RA). We evaluated the efficacy and safety of tofacitinib 5 or 10 mg twice daily (BID), in patients with moderate to severe RA, aged ≥65 and <65 years. Data were pooled from five Phase 3 trials and, separately, from two open-label long-term extension (LTE) studies (data cut-off April, 2012). Patients received tofacitinib, or placebo (Phase 3 only), with/without conventional synthetic DMARDs (mainly methotrexate). Clinical efficacy outcomes from Phase 3 studies were evaluated at Month 3. Safety evaluations using pooled Phase 3 data (Month 12) and pooled LTE data (Month 24) compared exposure-adjusted incidence rates (IRs; with 95\% confidence intervals [CIs]), in older versus younger patients. In Phase 3 and LTE studies, 15.3\% (475/3111) and 16.1\% (661/4102) of patients, respectively, were aged ≥65 years. Consequently, exposure to tofacitinib was lower in older versus younger patients in Phase 3 (259.2 vs. 1554.9 patient years [pt-yrs]) and LTE (962.1 vs. 5071.7 pt-yrs) studies. Probability ratios for ACR responses and HAQ-DI improvement from baseline ≥0.22 (Month 3) favoured tofacitinib and were similar in older and younger patients, with overlapping CIs. IRs for SAEs and discontinuations due to AEs were generally numerically higher in older versus younger patients, irrespective of treatment. Older patients receiving tofacitinib 5 or 10 mg BID had a similar probability of ACR20 or ACR50 response and, due to comorbidities, a numerically higher risk of SAEs and discontinuations due to AEs compared with younger patients. [\hyperlink{Ritlecitinib Tosylate}{PMID: 28079500}, Jeffrey R Curtis et al., ]

\hypertarget{pmid_29680473}{T}o evaluate the efficacy and safety of rituximab in children with steroid-resistant nephrotic syndrome. A systematic review evaluating the efficacy and safety of rituximab in children with steroid-resistant nephrotic syndrome was performed. Data from studies, performed before April 2017 were collected, from MEDLINE, Cochrane Library, Scopus, and Web of Science. Study eligibility criteria included clinical trials and observational studies with a minimal sample size of 5 patients, regarding treatment with rituximab in children with steroid-resistant nephrotic syndrome. Independent extraction of articles by 2 investigators using predefined data fields was performed. We included 7 case series and 1 open-label randomized controlled trial. Among them, 3 studies were multicenter. A total of 226 patients were included. Mean age at onset was 5.6 ± 1.1 years. Mean number of rituximab administrations was 3.1 ± 1.1 infusions per patient. Remission was observed in 89 patients (46.4\%). Remission was seen in 40.8\% patients with initial steroid-resistant nephrotic syndrome and 52.8\% patients with late steroid-resistant nephrotic syndrome. Good initial response to rituximab therapy was observed in 63.2\% patients with minimal change nephrotic syndrome, 39.2\% patients with focal and segmental glomerulosclerosis, 1 patient had diffuse mesangial hypercellularity, and 1 patient had IgM nephropathy. Sustained remission ranged from 18\% to 93.7\%. Five serious adverse events were observed. Rituximab exhibited a satisfactory profile regarding efficacy and safety indicating that this agent is a promising therapy for steroid-resistant nephrotic syndrome and should be further investigated by randomized clinical trials. [\hyperlink{Ritlecitinib Tosylate}{PMID: 29680473}, Manel Jellouli et al., 2018] 160 children aged 1 to 12 years with clinical diagnosis of bacterial pharyngitis and/or tonsillitis were treated either with cefixime ready-to-use-suspension or penicillin V in an open, controlled and randomized multicenter study. Before treatment a rapid antigen detection test was accomplished and throat swabs were taken. After randomization, the children were either treated for 5 days with 8 mg cefixime/kg bodyweight ready-to-use suspension once daily or with 20,000 I.U. penicillin V/kg bodyweight t.i.d. also administered as suspension. The data of 151 children could be evaluated for clinically efficacy. In the cefixime-group 86.7\% of the children were cured and 9.3\% significantly improved. After initial improvement, in one child (1.3\%) a relapse occurred and in the two remaining children (2.7\%) therapy failed. 90.8\% of the patients treated with penicillin V were cured, 6.6\% improved and in one child each a relapse was registered resp. therapy failed. Complete microbiological data were available in 137 patients. In the cefixime-group in 82.6\% of the patients the pathogens were eradicated. The elimination rate in the penicillin-group was 88.2\%. At the follow-up 3-4 weeks after end of treatment 6 relapses were seen in the cefixime-group, and 8 in the patients treated with penicillin. Both regimes were safe. Mild to moderate adverse events at least possibly related to the study medication were seen in only 4 children treated with cefixime and in 5 treated with penicillin. A 5 day treatment of bacterial pharyngitis and tonsillitis with cefixime was as effective as a ten day treatment with penicillin V. [\hyperlink{Ritlecitinib Tosylate}{PMID: 29680473}, D Adam et al., ]

\hypertarget{pmid_25543701}{T}o evaluate the efficacy and safety of lower doses rituximab(375 mg/m²×1) in primary children immune thrombocytopenia (ITP). Fifty children [23 male and 27 female, the median age was 9.5 years (rage 3.5-17.0 years)]with persistent and chronic ITP were treated with lower doses rituximab from January 2009 to January 2013 in our hospital. Efficacy and side effects of lower doses rituximab was studied, and factors related to the outcomes were analyzed. Among fifty patients, 17/50(34\%) achieved a complete response (CR) and 15/50 (30\%) patients got response (R). Patients with CR continued to maintain a platelet count above 50×10⁹/L at a median 12.3 (6-40) months. Patents with R continued to maintain a platelet count above 30×10⁹/L at a median 6 (2-12) months. The overall response (OR) in 3 and 6 months were 58\% (29/50), 64\% (32/50) respectively. Six patients have mild and transient side effects, including urticarial rash and fever, which were promptly resolved with appropriate therapy. Sex, age at diagnosis, interval from diagnosis to initial treatment with rituximab, platelet count at treatment and CD19+B cell count did not influence the overall response and complete response (P>0.05). Patients with anti-GPIIb/IIIa autoantibody had a better OR (P<0.05). Children with persistent and chronic ITP treated by lower doses rituximab had better therapeutic effects. Patients with anti-GPIIb/IIIa autoantibody had better response. Rituximab was well tolerated and no related serious side effects were recorded in the study. [\hyperlink{Ritlecitinib Tosylate}{PMID: 25543701}, Xiaofan Liu et al., 2014]

\hypertarget{pmid_30505008}{T}he aim of the study was to evaluate the efficacy and long-term safety of tocilizumab treatment in children with systemic-onset juvenile idiopathic arthritis in a single centre. The study was based on a retrospective analysis of a cohort of 10 patients with systemic-onset juvenile idiopathic arthritis who were treated with tocilizumab in the period September 2011-July 2017. Their medical records were analysed taking into consideration the effectiveness of tocilizumab treatment and frequency of side effects. Before the initiation of treatment, 9/10 patients from the study group complained of fever and had significantly increased values of inflammatory markers, with the median CRP concentration 41.1 mg/l (norm < 5 mg/l) and ESR 37 mm/h (norm < 12 mg/l). The period of the initial 12 weeks of treatment was a quantum leap in the course of the disease: all children were afebrile, and inflammatory markers values decreased by 99.4\% in the case of CRP and 91.9\% in ESR. All patients fulfilled ACR Pedi 50 criteria, and 3 of them achieved ACR Pedi 70. In the next stages of treatment the response to tocilizumab was sustained, reaching 10 children achieving ACR Pedi 70 and 5 ACR Pedi 90 after one year of therapy. Tocilizumab appeared to be relatively safe in the study group. Although elevation of transaminases and neutropenia were observed in 5/10 patients, they were usually mild and transitional in their course. Tocilizumab is both effective and has a relatively good safety profile in children with severe systemic-onset juvenile idiopathic arthritis. It should be considered in the recommendations as a first-line treatment of this disease. [\hyperlink{Ritlecitinib Tosylate}{PMID: 30505008}, Justyna Roszkiewicz et al., 2018]

\hypertarget{pmid_17401268}{T}he present study aimed at verifying the safety and efficacy of rifampicin in ameliorating pruritus in cholestatic children. Twenty-three Egyptian children (14 boys and 9 girls), suffering from intractable pruritus of cholestasis, were included. Rifampicin was started at a dose of 10 mg/Kg/day in two divided doses and increased gradually to a maximum of 20 mg/Kg/day if there was no response. Liver function tests were followed up weekly. Seventeen patients (74\%) showed improvement of pruritus with rifampicin. None of the patients showed any deterioration in liver functions. Rifampicin in a dose of 10-20 mg/Kg/day is safe and effective in ameliorating uncontrollable pruritus in children with persistent cholestasis. [\hyperlink{Ritlecitinib Tosylate}{PMID: 17401268}, Hanaa El-Karaksy et al., 2007]

\section*{Pregabalin}
\subsection*{Result}
\subsubsection*{Answer}

Yes (1 month to 18 years)

\subsubsection*{{Explanation}}
\hypertarget{Pregabalin}
Based on the abstracts reviewed, several targeted studies have evaluated the safety of pregabalin in children across different age ranges and indications. Below is a summary of the evidence for each age group:

Infants and Young Children (1 month to <4 years):
- A randomized, placebo-controlled study specifically evaluated the safety and efficacy of pregabalin in children aged 1 month to <4 years with focal onset seizures. Both 7 mg/kg/d and 14 mg/kg/d doses were generally safe and well tolerated, with adverse events consistent with the known profile in older children. The study concluded that pregabalin was generally safe and well tolerated in this age group [\hyperlink{pmid_32189338}{PMID: 32189338}, Donald Mann et al., 2020].

Children (1-23 months, 2-6, 7-11, and 12-16 years):
- A phase 1, randomized, placebo-controlled, escalating-dose study assessed safety, tolerability, and pharmacokinetics of pregabalin in children aged 1 month to 16 years with refractory partial seizures. Pregabalin at doses up to 10 mg/kg/day (and up to 15 mg/kg/day in those <6 years) demonstrated acceptable safety and tolerability. The most common adverse events were somnolence and dizziness. The study supports acceptable safety in these age groups [\hyperlink{pmid_25377429}{PMID: 25377429}, Donald Mann et al., 2014].

Children (3-15 years):
- A randomized controlled trial compared pregabalin to propranolol for migraine prophylaxis in children aged 3-15 years. The study found pregabalin to be effective and did not report serious adverse effects, suggesting it was safe and well tolerated in this age group [\hyperlink{pmid_26024701}{PMID: 26024701}, MohammadKazem Bakhshandeh Bali et al., 2015].

Children (6-18 years):
- A randomized trial compared pregabalin and sodium valproate for migraine prophylaxis in children aged 6-18 years. Both drugs were effective, and no serious adverse effects were reported, indicating pregabalin was safe in this age group [\hyperlink{pmid_37637787}{PMID: 37637787}, Narjes Jafari et al., 2023].

Children (4-15 years):
- An open-label, add-on trial in children aged 4-15 years with severe drug-resistant epilepsy found pregabalin to be useful, with side effects in 32\% (somnolence, weight gain, dizziness, behavioral change), but no serious safety concerns. The drug was withdrawn in some for lack of efficacy or worsening of myoclonic epilepsy, suggesting caution in this subgroup [\hyperlink{pmid_19380070}{PMID: 19380070}, Mohammed M S Jan et al., 2009].

Children (various studies, 6 months-16 years, 1.5 months-14 years, 4-16 years):
- Multiple open-label and randomized studies in children with epilepsy or intractable seizures (age ranges from 6 months to 16 years) found pregabalin to be effective and generally well tolerated, with mild and reversible side effects [\hyperlink{pmid_25767539}{PMID: 25767539}, Mohammad Mahdi Taghdiri et al., 2015; \hyperlink{pmid_25793053}{PMID: 25793053}, Gholamreza Zamani et al., 2014; \hyperlink{pmid_25657772}{PMID: 25657772}, Mohsen Mollamohammadi et al., 2014; \hyperlink{pmid_30688135}{PMID: 30688135}, Jeremias Antinew et al., 2019].

Children (≤6 years, accidental ingestion):
- A retrospective analysis of accidental pregabalin ingestion in children ≤6 years found that such exposures were rare and caused minimal toxicity. Children without underlying neurological/cardiac disease and not on concomitant medication who ingested <19.4 mg/kg could be safely observed at home [\hyperlink{pmid_33006482}{PMID: 33006482}, Laurène Dufayet et al., 2021].

Children (mean age 13.5 years, chemotherapy-induced neuropathic pain):
- An open-label study in children with cancer and neuropathic pain (mean age 13.5 years) found pregabalin to be effective and adverse effects to be infrequent and transient, suggesting it is safe in this context [\hyperlink{pmid_18774740}{PMID: 18774740}, Petr Vondracek et al., 2009].

Children (dental anxiety, unspecified age but pediatric population):
- A randomized, double-blind, placebo-controlled crossover trial found that a single dose of pregabalin had significant anxiolytic and sedative effects in children without serious side effects [\hyperlink{pmid_26759588}{PMID: 26759588}, Tahereh Eskandarian et al.].

Summary:
Across multiple targeted studies in children from 1 month to 18 years, pregabalin has been shown to have an acceptable safety profile for various indications (epilepsy, migraine, neuropathic pain, dental anxiety). The most common adverse effects are somnolence, dizziness, and mild behavioral changes, with rare serious adverse events. Some caution is advised in children with myoclonic epilepsy. There is no evidence from these abstracts of studies showing pregabalin to be unsafe in children. Therefore, based on the available abstracts, pregabalin is affirmed as safe for use in children within the studied age ranges and indications.

\subsection*{Abstracts}
\hypertarget{pmid_19380070}{P}regabalin is a new antiepileptic drug that acts at presynaptic calcium channels, modulating neurotransmitter release. We report on treating consecutive children with severe drug-resistant epilepsy in a prospective, open-label, add-on trial. Nineteen children (63\% male) aged 4-15 years (mean, 9.7; S.D., 2.9) were included. Most (74\%) had daily seizures that failed multiple drugs (mean, 5). Epilepsy was symptomatic in 58\%, and 74\% exhibited associated cognitive deficits. Seizures were mixed in nine (47\%), and four (21\%) manifested Lennox-Gastaut syndrome. Pregabalin was maintained at 150-300 mg/day. On pregabalin, one (6\%) child became seizure-free, and seven (37\%) had >50\% seizure reduction. The percentage of children with daily seizures was reduced from 74\% before pregabalin to 37\% afterward (P < 0.002). Side effects were evident in six (32\%) with somnolence, weight gain, dizziness, or behavioral change. The drug was withdrawn in five (26\%) children for lack of efficacy, and in two (11\%) for worsening of myoclonic epilepsy. We conclude that pregabalin is a useful addition in the treatment of refractory childhood epilepsy. The drug should be used with caution in myoclonic epilepsy. Controlled studies are needed to establish long-term efficacy and tolerability. [\hyperlink{Pregabalin}{PMID: 19380070}, Mohammed M S Jan et al., 2009]

\hypertarget{pmid_33006482}{I}n France, pregabalin is widely prescribed in adults but still not approved for children. We aimed to investigate the incidence of pregabalin exposure in ≤6-year-old children, to describe the characteristics and outcome of ingestions involving pregabalin alone, and to estimate a clinically relevant toxic dose in this population. Retrospective analysis of pregabalin exposures in ≤6-year-old children, collected by the French Poison Control Centers in 2004-2019. The incidence was estimated using pregabalin prescription data from the Health Improvement Network database (the French version of THIN). The poison severity score (PSS) was used to grade severity. We found 313 unintentional immediate-release pregabalin ingestions in ≤6-year-old children. The number of cases per 100,000 pregabalin-treated adults increased over time ( Despite increasing prescriptions in adults in France, unintentional pregabalin ingestions in ≤6-year-old children remain rare and cause minimal toxicity. Children with no underlying neurological/cardiac disease and concomitant medication ingesting <19.4 mg/kg immediate-release pregabalin alone can be safely observed at home. [\hyperlink{Pregabalin}{PMID: 33006482}, Laurène Dufayet et al., 2021]

\hypertarget{pmid_25377429}{T}o evaluate the safety, tolerability, and pharmacokinetics (PK) of pregabalin as adjunctive therapy in children with refractory partial seizures. This was a phase 1, randomized, placebo-controlled, parallel-group, escalating-dose, multiple-dose study comprising a 7-day, double-blind treatment period and a single-blind, single dose of pregabalin administered to all children on day 8. Children in four age cohorts (1-23 months, 2-6, 7-11, and 12-16 years) received one of four doses of pregabalin (2.5, 5, 10, or 15 mg/kg/day) or placebo. Safety and tolerability were assessed throughout the study. Steady-state and single-dose PK parameters on day 8 were analyzed using standard noncompartmental procedures. Sixty-five children received at least one dose of treatment. Four pregabalin-treated children discontinued treatment, three of whom received 15 mg/kg/day. Two children experienced serious adverse events, one of whom received pregabalin 15 mg/kg/day. During double-blind treatment, the most common adverse events reported in the pregabalin-treated population were somnolence (27.1\%) and dizziness (12.5\%). Steady-state pregabalin peak and total exposure in each age cohort appeared to increase linearly with dose. Apparent oral clearance (CL/F) was directly related to creatinine clearance, consistent with adults. CL/F normalized for body weight was 43\% higher in patients weighing <30 kg. Steady-state and single-dose PK were consistent. Pregabalin at doses up to 10 mg/kg/day in children aged 1 month to 16 years, and at doses up to 15 mg/kg/day in those aged <6 years, demonstrated acceptable safety and tolerability. For children weighing <30 kg, a dose increase of 40\% (mg/kg dosing) is required to achieve comparable exposure with adults or children weighing ≥30 kg. These data will inform dose selection in phase 3 trials of the efficacy and safety of adjunctive pregabalin in children with refractory partial seizures. [\hyperlink{Pregabalin}{PMID: 25377429}, Donald Mann et al., 2014]

\hypertarget{pmid_26024701}{M}igraine involves 5-10\% of children and adolescents. Thirty percent of children with severe migraine attacks have school absence and reduced quality of life that need preventive therapy. The purpose of this randomised control trial study is to compare the effectiveness, safety and the tolerability of pregabalin toward Propranolol in migraine prophylaxis of children. From May 2011 to October 2012, 99 children 3-15 years referred to the neurology clinic of Mofid Children's Hospital with a diagnosis of migraine enrolled the study. Patients randomly divided into two groups (A\&B). We treated children of group A with capsule of pregabalin as children of group B with tablet of propranolol for at least 8 weeks. In this study, 99 patients were examined that 91 children reached the last stage. The group A consistsed of 46 patients, 12(26.1\%) girls, 34 (73.9\%) boys and the group B consisted of 45 patients, 14(31.1\%) girls, 31 (68.9\%) boys. Basis of age, gender, headache onset, headache frequency, migraine type, triggering and relieving factors there was no significant difference among these groups (P>0.05). After 4 and 8 weeks of Pregabalin usage monthly headache frequency decreased to 2.2±4.5 and 1.76±6.2 respectively. Propranolol reduced monthly headache frequency up to 3.73±6.11 and 3.34±5.95 later 4 and 8 weeks respectively. There was a significant difference between these two groups according to headache frequency reduction (P=0.04). Pregabalin efficacy in reducing the frequency and duration of pediatric migraine headache is considerable in comparison with propranolol.  [\hyperlink{Pregabalin}{PMID: 26024701}, MohammadKazem Bakhshandeh Bali et al., 2015] To evaluate the safety and efficacy of pregabalin in the management of chemotherapy-induced neuropathic pain in patients with childhood solid tumors and leukaemia. In an open-label study, 30 children (11 boys and 19 girls; mean age 13.5 years) who were treated for solid tumors and leukaemia, and developed a painful peripheral neuropathy, were medicated with pregabalin in the daily dose of 150-300 mg for 8 weeks. Twenty-eight patients completed the 8-week follow-up. A significant and long-lasting pain relief was noted in 86\% of these patients. Median VAS score decreased by 59\% at the 8th week from baseline. Adverse effects were infrequent and transient. The treatment with pregabalin resulted in a significant improvement in pain symptoms. The use of pregabalin in children is off-label so far. However, this drug seems to be a safe and effective remedy, which could significantly broaden the therapeutic spectrum in paediatric oncological patients suffering from neuropathic pain. [\hyperlink{Pregabalin}{PMID: 26024701}, Petr Vondracek et al., 2009]

\hypertarget{pmid_32189338}{T}o evaluate the efficacy and safety of pregabalin as adjunctive treatment for children (aged 1 month-<4 years) with focal onset seizures (FOS) using video-electroencephalography (V-EEG). This randomized, placebo-controlled, international study included V-EEG seizure monitoring (48-72 hours) at baseline and over the last 3 days of 14-day (5-day dose escalation; 9-day fixed dose) double-blind pregabalin treatment (7 or 14 mg/kg/d in three divided doses). This was followed by a double-blind 1-week taper. The primary efficacy endpoint was log-transformed seizure rate (log Overall, 175 patients were randomized (mean age = 28.2 months; 59\% male, 69\% white, 30\% Asian) in a 2:1:2 ratio to pregabalin 7 or 14 mg/kg/d (n = 71 or n = 34, respectively), or placebo (n = 70). Pregabalin 14 mg/kg/d (n = 28) resulted in a statistically significant 35\% reduction of log Pregabalin 14 mg/kg/d (but not 7 mg/kg/d) significantly reduced seizure rate in children with FOS, when assessed using V-EEG, compared with placebo. Both pregabalin dosages were generally safe and well tolerated in children 1 month to <4 years of age with FOS. Safety and tolerability were consistent with the known profile of pregabalin in older children with epilepsy. [\hyperlink{Pregabalin}{PMID: 32189338}, Donald Mann et al., 2020]

\hypertarget{pmid_25793053}{A}bout one third of partial seizures are refractory to treatment. Several anticonvulsant drugs have entered the market in recent decades but concerns about intolerance, drug interactions, and the safety of the drug are notable. One of these new anticonvulsants is pregabalin, a safe drug with almost no interaction with other antiepileptic drugs. In this open label clinical trial study, pregabalin was used for evaluation of its efficacy on reducing seizure frequency in 29 children suffering from refractory partial seizures. Average daily and weekly seizure frequency of the patients was recorded during a 6-week period (baseline period). Then, during a period of 2 weeks (titration period), pregabalin was started with a dose of 25-75 mg/d, using method of flexible dose, and was brought to maximum dose of drug that was intended in this study (450 mg/d) based on clinical response of the patients and seizure frequency. Then the patients were given the drug for 12 weeks and the average frequency of daily and weekly seizures were recorded again (treatment period). Findings : Reduction in seizure frequency in this study was 36\% and the responder rate or number of patients who gained more than 50\% reduction in seizure frequency was 51.7\%. This study showed that pregabalin can be used with safety and an acceptable efficacy in treatment of childhood refractory partial seizures. [\hyperlink{Pregabalin}{PMID: 25793053}, Gholamreza Zamani et al., 2014]

\hypertarget{pmid_17020438}{P}regabalin is a new anxiolytic that has been recently licensed for the treatment of generalised anxiety disorder (GAD) in Europe. Short-term efficacy is based on six positive placebo-controlled studies, all of which showed a significant early separation from placebo in all of the doses used (150-600 mg) at the first week, and the efficacy at the end of the treatment was comparable with the comparators used in four of these studies. Pregabalin was effective in more or less severe GAD, on psychic and somatic symptoms of GAD, and in treating the subsyndromal depressive symptoms of GAD. Efficacy in the elderly was shown in a separate placebo-controlled study. The effect on cognitive function was minimal and notably less than that observed with benzodiazepines. The discontinuation symptoms following abrupt treatment cessation were similar to the rates with serotonin-noradrenaline re-uptake inhibitors and lower than with benzodiazepines with no signals of tolerance or dependence. [\hyperlink{Pregabalin}{PMID: 17020438}, Stuart A Montgomery et al., 2006]

\hypertarget{pmid_15206660}{P}regabalin is a novel compound in development for the treatment of anxiety disorders. The safety and efficacy of pregabalin for the treatment of social anxiety disorder was evaluated in a double-blind, multicenter clinical trial in which 135 patients were randomized to 10 weeks of double-blind treatment with either pregabalin 150 mg/d. pregabalin 600 mg/d, or placebo. The primary efficacy parameter was change from baseline to end point in the Liebowitz Social Anxiety Scale (LSAS) total score. Safety was assessed through clinical and laboratory monitoring, and recording spontaneously reported adverse events. Ninety-four patients (70\%) completed the 11-week double-blind treatment phase. LSAS total score was significantly decreased by pregabalin 600 mg/d treatment compared with placebo (P = 0.024, analysis of covariance). Significant differences (P < or = 0.05) between pregabalin 600 mg/d and placebo were seen on several secondary measures including the LSAS subscales of total fear, total avoidance, social fear, and social avoidance, and the Brief Social Phobia Scale fear subscale. Pregabalin 150 mg/d was not significantly better than placebo on any measures. Somnolence and dizziness were the most frequently occurring adverse events among patients receiving pregabalin 600 mg/d. In conclusion, pregabalin 600 mg/d was an effective and well-tolerated treatment of social anxiety disorder. [\hyperlink{Pregabalin}{PMID: 15206660}, Atul C Pande et al., 2004]

\hypertarget{pmid_33280106}{P}regabalin is approved in multiple countries as adjunctive therapy for adult patients with focal onset seizures (FOS; previously termed partial onset seizures). This study used population pharmacokinetic (PK) and exposure-response (E-R) analyses from pooled pregabalin concentration and efficacy data to compare pregabalin exposure and E-R relationships in pediatric and adult patients with FOS, to support pediatric dosage recommendations. A one-compartment disposition model was used, with first-order absorption and body surface area-normalized creatinine clearance on clearance. Individual pregabalin average steady-state concentrations were predicted and used in an E-R analysis of efficacy. The E-R relationship of pregabalin was similar in pediatric (4-16 years) and adult patients with FOS after accounting for differences in baseline natural log-transformed 28-day seizure rate and placebo effect. Population PK simulations showed that children aged 4-16 years and weighing ≥ 30 kg required pregabalin 2.5-10 mg/kg/day to achieve similar pregabalin exposure at steady-state to adult patients receiving the approved doses of 150-600 mg/day. For children 4-16 years weighing < 30 kg, a higher pregabalin dose of 3.5-14 mg/kg/day was required to achieve equivalent exposure at steady-state. The results support the dosage guidance provided in the pregabalin prescribing label, whereby pediatric patients (4-16 years) weighing < 30 kg should receive a 40\% higher pregabalin dose (per kg of body weight) than patients weighing ≥ 30 kg to achieve similar exposure. Our combined modeling approach may provide guidance for future extrapolation assessment from adult to pediatric patients. [\hyperlink{Pregabalin}{PMID: 33280106}, Phylinda L S Chan et al., 2021]

\hypertarget{pmid_37637787}{M}igraine is a common disorder in children, and its prophylaxis with minimal side effects is momentous. This study aimed to compare the efficacy of Pregabalin and Sodium Valproate in preventing migraine attacks. Sixty-four children (aged 6-18) with migraines were recruited, as defined by Internation Headache Criteria (ICHD-III). They were randomly assigned to two groups: Sodium Valproate (n=32) and Pregabalin (n=32). The minimum dosage of drugs was prescribed in both groups. The patients were followed for four months. The parameters such as frequency, intensity, duration of migraine attacks, and the number of painkillers that the patients used monthly were recorded. The Spence Children's anxiety scale was also used to evaluate medications' effect on patients' anxiety levels. Two medications were equally effective in reducing the intensity and duration of attacks. Additionally, their effect on reducing the anxiety level of patients was equal. There was a significant difference between the effect of drugs on the frequency of migraine attacks at the end of the first and fourth months and the number of painkillers used at the end of the fourth month. The frequency of attacks was decreased by more than 50\% in twenty-eight patients (90\%) of Pregabalin recipients and twenty-one patients (84\%) of Sodium Valproate recipients. Considering the better effect of Pregabalin in the reduction of frequency of migraine attacks and pain-reducing medications consumption, Pregabalin could be a proper substitute for Sodium Valproate for prophylactic migraine treatment in children. [\hyperlink{Pregabalin}{PMID: 37637787}, Narjes Jafari et al., 2023]

\hypertarget{pmid_30688135}{E}fficacy and safety of pregabalin as adjunctive treatment for children (aged 4-16 years) with partial-onset seizures, hereafter termed focal onset seizures for this study, was evaluated. This double-blind, randomized, placebo-controlled, international study had 3 phases: 8-week baseline, 12-week double-blind treatment (2-week dose escalation; 10-week fixed dose), and 1-week taper. Selection criteria included experiencing focal onset seizures and receiving a stable regimen of 1 to 3 antiepileptic drugs. Study treatments were pregabalin 2.5 mg/kg/d, 10 mg/kg/d, or placebo; doses were increased to 3.5 or 14 mg/kg/d for subjects weighing <30 kg. The key endpoints were change in log [\hyperlink{Pregabalin}{PMID: 30688135}, Jeremias Antinew et al., 2019] Dental anxiety is a relatively frequent problem that can lead to more serious problems such as a child entering a vicious cycle as he/she becomes reluctant to accept the required dental treatments. The aim of this randomized double-blind clinical trial study was to evaluate the anxiolytic and sedative effect of pregabalin in children. Twenty-five children were randomized to a double-blind placebo-controlled crossover clinical trial. Two visits were scheduled for each patient. At the first visit, 75 mg pregabalin or placebo was given randomly, and the alternative was administered at the next visit. Anxiolytic and sedative effects were measured using the visual analogue scale. The child's behavior was rated with the Frankl behavioral rating scale and the sedation level during the dental procedure was scored using the Ramsay sedation scale. The unpaired, two-tailed Student's t-test was used to compare the mean changes of visual analog scale (VAS) for anxiety in the pregabalin group with that of the placebo group. A repeated measures MANOVA model was used to detect differences in sedation level in the pregabalin and placebo groups regarding the interaction of 3-time measurements; sub-group analysis was performed using Student's t-test. The Mann-Whitney U-test was used to analyze the nonparametric data of the Frankl and Ramsay scales. A P < 0.05 was considered significant. The reduction of the VAS-anxiety score from 2 h post-dose was statistically significant in the pregabalin group. From 2 h to 4 h post-dose, the VAS-sedation score increased significantly in the pregabalin group. The child's behavior rating was not significantly different between the groups. The number of "successful" treatment visits was higher in the pregabalin group compared to the placebo group. Significant anxiolytic and sedative effects can be anticipated 2 h after oral administration of pregabalin without serious side effects. [\hyperlink{Pregabalin}{PMID: 30688135}, Tahereh Eskandarian et al., ]

\hypertarget{pmid_25767539}{A}pproximately one third of epileptic children are resistant to anticonvulsant drugs. This study evaluates the effectiveness, safety, and tolerability of pregabalin as adjunctive therapy in epileptic children relative to Zonisamide. From April 2012 to November 2012,121 children were referred to Mofid Children's Hospital with intractable epilepsy and enrolled in the study. The patients were divided into two groups (A and B) randomly. Group A was treated with Zonisamide and group B was treated with Pregabalin in addition to prior medication. We assessed seizure frequency and severity during a 4-week interval from the beginning of the drug treatment and compared the efficacy of each in these two groups. Group A consists of 61 patients, 26 (42.6\%) girls, and35 (57.4\%) boys with an age range from 1.5 months-14 years (mean, 73.9± 44.04 months). Group B consists of 60 patients, 31(51.7\%) girls, 29 (48.3\%) boys with an age range from 6 months-16 years (mean, 71±42.9 months). Age, gender, seizure onset, seizure frequency, seizure type, and previous antiepileptic medications showed that there was no significant difference between the groups (P>0.05). Zonisamide and pregabalin reduced more than 50\% of seizure intensity in 40.2\%; 45.8\% of patients also had a seizure frequency decline between35.8-44.4\%, respectively and there was no significant superiority between these two novel anticonvulsants (P>0.05). In this survey both pregabalin and Zonisamide were impressive for seizure control in children with intractable epilepsy and well sustained with mild complications that were completely reversible. [\hyperlink{Pregabalin}{PMID: 25767539}, Mohammad Mahdi Taghdiri et al., 2015]

\hypertarget{pmid_33440453}{P}regabalin was first approved in 2004 for the treatment of peripheral neuropathic pain and focal epileptic seizures, with or without secondary generalization. Prescription frequency has increased significantly since approval. In the early days, little attention was paid to the problem of misuse and dependence on pregabalin; in recent years, there has been a significant increase in the number of publications focusing on this problem. This review deals with these risk factors and risk groups of pregabalin abuse and dependence in different European countries and their drug policies. Pregabalin abuse and dependence has increased significantly since its introduction to the market. It was shown that solo abuse of pregabalin is rare. In most cases, pregabalin was combined with other substances, which is also a predictor of pregabalin abuse. There were different reasons for the non-prescription use of pregabalin; it was used to increase the psychotropic effect, on the one hand, and to alleviate withdrawal symptoms, on the other hand. Furthermore, in Sweden, pregabalin was found in 28\% of fatal intoxications among drug addicts. Young people were particularly affected. Abuse of pregabalin was detected in countries with restrictive substitution programmes, while in countries with liberal drug policies, no abuse was detected. However, the data situation in Switzerland with a liberal substitution programme is based on only one study, which is why pregabalin use in liberal substitution programmes cannot be conclusively clarified. There seems to be a connection between a country's drug policy and the illegal use of pregabalin among persons in a substitution programme in that country. There are also risk factors and risk groups for pregabalin dependence and abuse. [\hyperlink{Pregabalin}{PMID: 33440453}, Dominique Kuhn et al., 2021]

\hypertarget{pmid_26259772}{T}he aim of this review is to summarise the literature on the efficacy and safety of pregabalin for the treatment of generalised anxiety disorder (GAD). Of 241 literature citations, 13 clinical trials were identified that were specifically designed to evaluate the efficacy and safety of pregabalin in GAD, including 11 randomised double-blind trials and two open-label studies. Pregabalin efficacy has been consistently demonstrated across the licensed dose range of 150-600 mg/day. Efficacy has been reported for pregabalin monotherapy in elderly patients with GAD, patients with severe anxiety, and for adjunctive therapy when added to a selective serotonin reuptake inhibitor or serotonin-norepinephrine reuptake inhibitor in patients who have failed to respond to an initial course of antidepressant therapy. The two most common adverse events with pregabalin are somnolence and dizziness, both of which appear to be dose-related. Pregabalin appears to have a low potential for causing withdrawal symptoms when long-term therapy is discontinued; however, tapering over the course of at least one week is recommended. A review of available evidence indicates that pregabalin is a well-tolerated and consistently effective treatment for GAD, with a unique mechanism of action that makes it a useful addition to the therapeutic armamentarium.  [\hyperlink{Pregabalin}{PMID: 26259772}, David S Baldwin et al., 2015] The prevalence of active epilepsy is about 0.5-1\%, and approximately 70\% of patients are cured with first anti-epileptic drugs and the remaining patients need multiple drugs. Pregabalin as an add-on therapy has a postive effect on refractory seizures in adults. To the best of our knowledge, there is no research with this drug in childhood epilepsy. We use pregabalin in children with refractory seizures as an add-on therapy. The objective of this study is to evaluate the effects of pregabalin in the reduction of seizures for refractory epilepsy. Forty patients with refractory seizures who were referred to Mofid Children's Hospital and Hazrat Masoumeh Hospital were selected. A questionnaire based on patient record forms, demographic data (age, gender,…), type of seizure, clinical signs, EEG record, imaging report, drugs that had been used, drugs currently being used, and the number of seizures before and after Pregabalin treatment was completed. We checked the number of seizures after one and four months. After one month, 26.8\% of patients had more than a 50\% reduction in seizures and 14.6\% of these patients were seizure-free; 12.2\% had a 25-50\% reduction; and approximately 61\% had less than a 25\% reduction or no change in seizures. After the fourth month, 34.1\% of patients had more than a 50\% reduction in seizures and 24.4\% of these patients were seizure-free. Additionally, 65.9\% of patients had less than 50\% reduction in seizures (9.8\% between 25-50\% and 56.1\% less than 25\% or without improvement). We recommend Pregabalin as an add-on therapy for refractory seizures (except for myoclonic seizures) for children. [\hyperlink{Pregabalin}{PMID: 26259772}, Mohsen Mollamohammadi et al., 2014]

\hypertarget{pmid_16841627}{P}regabalin has demonstrated robust, rapid efficacy in reducing symptoms of generalized anxiety disorder (GAD) in 4 placebo-controlled clinical trials. The current study compared the efficacy and safety of pregabalin and venlafaxine in patients diagnosed with moderate to severe GAD. The study was conducted from December 21, 1999, to July 31, 2001. Outpatients (N = 421) in primary care or psychiatry settings meeting DSM-IV criteria for GAD were randomly assigned to 6 weeks of double-blind treatment with pregabalin 400 or 600 mg/day, venlafaxine 75 mg/day, or placebo. The primary analysis was change in Hamilton Rating Scale for Anxiety (HAM-A) total score from baseline to last-observation-carried-forward (LOCF) endpoint. Secondary analyses included the change in HAM-A psychic (emotional) and somatic (physical) factor scores, significant improvement at week 1, and week 1 improvement sustained at every visit through endpoint. Pregabalin at both dosages (400 mg/day, p = .008; 600 mg/day, p = .03) and venlafaxine (p = .03) produced significantly-greater improvement in HAM-A total score at LOCF endpoint than did placebo. Only the pregabalin 400-mg/day treatment group experienced significant improvement in all a priori primary and secondary efficacy measures. Pregabalin in both dosage treatment groups (400 mg/day, p < .01; 600 mg/day, p < .001) significantly improved HAM-A total score at week 1, with significant improvement through LOCF endpoint. Statistically significant improvement began at week 2 for venlafaxine. Discontinuation rates due to associated adverse events were greatest in the venlafaxine treatment group: venlafaxine, 20.4\%; pregabalin 400 mg/day, 6.2\%; pregabalin 600 mg/day, 13.6\%; placebo, 9.9\%. Pregabalin was safe, well tolerated, and rapidly efficacious across the physical-somatic as well as the emotional symptoms of GAD in the majority of patients studied in primary care and psychiatric settings. [\hyperlink{Pregabalin}{PMID: 16841627}, Stuart A Montgomery et al., 2006]

\hypertarget{pmid_15926135}{P}regabalin is an antiepileptic drug recently approved in the European Union for add-on therapy of focal epilepsy. A review of its clinical and pharmacological characteristics is, therefore, appropriate. This drug, which binds to a subunit of voltage-dependent calcium channels in neuronal membranes, has a favourable pharmacokinetic profile. Pregabalin administered in two or three divided doses was compared to placebo in three double-blind randomised multicenter clinical trials, including 1,052 patients with focal epilepsy not controlled with other antiepileptic drugs. Results of these studies showed efficacy at doses of 150 mg per day, and a dose-response relationship up to doses of 600 mg per day. At the highest dose, mean seizure reduction for pregabalin was 44.3 to 54\%, a significant reduction compared to placebo (p < or =0.0001), and a response rate of 43.5 to 51\% (p < or =0.001). In one of these studies 12\% of patients treated with pregabalin at 600 mg per day were seizure free for the last month of therapy while another study demonstrated its efficacy when used on a twice daily schedule. Subsequent open studies demonstrated a sustained efficacy of the drug. The most common adverse events were dizziness, somnolence, ataxia, asthenia, and weight gain. Withdrawal from controlled studies due to adverse effects was 15.3\% in patients treated with pregabalin, compared with 6.15\% in those receiving placebo. Pregabalin favourable pharmacokinetic profile, in addition to its good tolerability and remarkable efficacy make this new antiepileptic drug an attractive option for the treatment of focal epilepsies. [\hyperlink{Pregabalin}{PMID: 15926135}, A Gil-Nagel Rein et al., ]

\hypertarget{pmid_27044003}{P}regabalin was evaluated for potential developmental toxicity in mice and rabbits. Pregabalin was administered once daily by oral gavage to female albino mice (500, 1250, or 2500 mg/kg) and New Zealand White rabbits (250, 500, or 1250 mg/kg) during organogenesis (gestation day 6 through 15 [mice] or 6 through 20 [rabbits]). Fetuses were evaluated for viability, growth, and morphological development. Pregabalin administration to mice did not induce maternal or developmental toxicity at doses up to 2500 mg/kg, which was associated with a maternal plasma exposure (AUC0-24 ) of 3790 μg•hr/ml, ≥30 times the expected human exposure at the maximum recommended daily dose (MRD; 600 mg/day). In rabbits, treatment-related clinical signs occurred at all doses (AUC0-24 of 1397, 2023, and 4803 μg•hr/ml at 250, 500, and 1250 mg/kg, respectively). Maternal toxicity was evident at all doses and included ataxia, hypoactivity, and cool to touch. In addition, abortion and females euthanized moribund with total resorption occurred at 1250 mg/kg. There were no treatment-related malformations at any dose. At 1250 mg/kg, compared with study and historical controls, the percentage of fetuses with retarded ossification was significantly increased and the mean number of ossification sites was decreased, which correlated with decreased fetal and placental weights, consistent with in utero growth retardation. Therefore, the no-effect dose for developmental toxicity in rabbits was 500 mg/kg, which produced systemic exposure approximately 16-times human exposure at the MRD. These findings indicate that pregabalin, at the highest dose tested, was not teratogenic in mice or rabbits.  [\hyperlink{Pregabalin}{PMID: 27044003}, Dennis C Morse et al., 2016] Pregabalin is indicated for the treatment of fibromyalgia and pregabalin-treated subjects have shown improved pain, sleep and functional measures in placebo-controlled and open-label studies. This article reviews pregabalin's safety profile. Areas covered: The safety findings in pregabalin clinical trials were accessed by a PubMed search using the key words 'pregabalin' or 'anti-epilectics drug' or 'gabapentinoids' or 'anticonvulsants' and 'fibromyalgia'. Although frequent, the side effects of pregabalin therapy are usually mild to moderate, well tolerated in the long term, and can be monitored in a primary care setting. Expert opinion: Pregabalin therapy may be associated with somnolence, dizziness, weight gain and periphereal edema. Potential drug interactions are not common, and pregabalin seems to be well tolerated in combination with antidepressants. The demonstrated efficacy of pregabalin suggests that the risk/benefit ratio favours its use. [\hyperlink{Pregabalin}{PMID: 27044003}, Maria Chiara Gerardi et al., 2016]

\hypertarget{pmid_22796916}{P}regabalin, or S-(+)-3-isobutylgaba, is a lipophilic analogue of GABA. Although pregabalin is structurally related to GABA, it is inactive at GABA receptors and does not appear to mimic GABA physiologically. Pregabalin is a potent ligand for the alpha-2-delta subunit of voltage-gated calcium channels in the central nervous system. It is currently being licensed for epilepsy, neuropathic pain, and generalized anxiety disorder. There are few case reports that have demonstrated safety of pregabalin in case of intoxication. We report here a case of pregabalin toxicity with a moderate pregabalin concentration that was successfully managed with conservative treatment only. The case report describes a 54-year-old man who was treated with pregabalin for generalized anxiety disorder. After having experienced a significant stress on a job the patient ingested huge amount of pregabalin (4,2 r) together with bromazepam (21 mg) and chlorimipramine (125 mg). On presentation he was conscious and alert with a stable condition of cardiovascular and respiratory systems. The serum pregabalin concentration was 20.8 mg/L but the patient did not have any signs of toxicity. Thanks to his good and stable somatic condition the patient was managed with supportive treatment only. Although anecdotal, our case report points toward safety of pregabalin following deliberate self-poisoning. Our observation is in accordance with the recent international literature underlining that pregabalin was listed as the drug ingested in only 1\% of fatalities, usually in combination with other drugs. [\hyperlink{Pregabalin}{PMID: 22796916}, C Miljevic et al., ]

\hypertarget{pmid_25560586}{P}regabalin, a potent anticonvulsant agent, is used in treatment-resistant epileptic patients. It is reported that pregabalin also has analgesic effect in different pain syndromes. However, there is limited data on its antinociceptive mechanisms of action. We aimed to investigate the central and peripheral antinociceptive effects of pregabalin and the contribution of nitrergic, serotonergic, and opioidergic pathways in mice. We used tail flick, tail clip and hot plate tests to investigate the central antinociceptive effects and acetic acid-induced writhing test to assess peripheral antinociceptive effects of pregabalin (10, 30, 100mg/kg). We also combined pregabalin (100mg/kg) with, a nonspecific nitric oxide synthase inhibitor l-NAME (100mg/kg), a serotonin receptor antagonist cyproheptadine (50 μg/kg), and an opioid receptor antagonist naloxone (1mg/kg). Pregabalin 30 mg/kg enhanced the percentage of maximal possible effect (\% MPE) in tail flick test. Pregabalin 100mg/kg significantly increased \% MPE in tail clip and tail flick tests and decreased the number of writhings. Pregabalin made no significant alteration in hot plate test at all doses. The combined use of pregabalin 100mg/kg with l-NAME, cyproheptadine, and naloxone showed that \% MPE was reduced only in the combination of pregabalin with naloxone and solely in tail clip test while no significant difference was observed in writhing test. We suggest that pregabalin (30 and 100mg/kg) presents central spinal but not central supraspinal antinociceptive effect and pregabalin 100mg/kg shows peripheral antinociceptive effect. The opioidergic pathway seems to mediate the central spinal antinociceptive effect of pregabalin while nitrergic and serotonergic pathways are not involved. [\hyperlink{Pregabalin}{PMID: 25560586}, Bilgin Kaygisiz et al., 2015]

\hypertarget{pmid_24654869}{S}UMMARY Pregabalin is the only US FDA-approved drug to date for neuropathic pain in spinal cord injured patients. Pregabalin is a novel GABA analog whose primary mechanism of action involves binding at the α2-δ subunit of voltage-sensitive calcium channels. Efficacy is noted within the first several days of administration. Dosing is typically initiated at 150 mg/day in divided doses, but may be started at even lower doses. Dosing can be increased gradually to a recommended maximum of 600 mg per day in divided dosing. Adverse events include somnolence, dizziness and dry mouth, and typically manifest within the first 2 weeks of treatment. Pregabalin is generally safe to use in combination with other pain medications or antidepressants, but safety in pregnant patients has not been established.  [\hyperlink{Pregabalin}{PMID: 24654869}, Kevin L Dalal et al., 2013] Pregabalin is a novel compound under development for the treatment of several types of anxiety disorders. To obtain an initial evaluation of the efficacy and safety of pregabalin in the treatment of generalized anxiety disorder (GAD), we conducted a double-blind, fixed-dose, parallel-group, placebo and active-controlled multicenter 4-week study that compared 271 patients randomized to receive pregabalin 50 mg tid (N = 70), pregabalin 200 mg tid (N = 66), placebo (N = 67), or lorazepam 2 mg tid (N = 68), followed by a 1-week double-blind taper. The primary efficacy parameter was change from baseline to endpoint (last observation carried forward) in the Hamilton Anxiety Scale (HAM-A) total score; adjusted mean change scores on the HAM-A were significantly improved for pregabalin 200 mg tid (difference of 3.90 between drug and placebo; p = 0.0013 [ANCOVA], df = 252) and for lorazepam (difference of 2.35; p = 0.0483 [ANCOVA], df = 252), with the significant difference between the pregabalin 200 mg tid and placebo groups seen at week 1 of treatment (p = 0.0001 [ANCOVA], df = 238). Safety analysis, which included assessment of spontaneously reported adverse events, laboratory monitoring, and withdrawal symptoms, showed pregabalin to be generally well-tolerated. The most common adverse events seen with pregabalin 200 mg tid were somnolence and dizziness. They were usually mild or moderate in intensity and were often transient. Pregabalin-treated patients had a higher completion rate than lorazepam-treated patients. This study supports the hypothesis that pregabalin is effective and safe in short-term therapy for GAD. More studies are needed to determine the best dosing regimen to optimize efficacy and tolerability. [\hyperlink{Pregabalin}{PMID: 24654869}, Douglas E Feltner et al., 2003]

\section*{Miglustat}
\subsection*{Result}
\subsubsection*{Answer}

Yes (ages 2–18, based on studies in mucopolysaccharidosis type III, juvenile GM1-gangliosidosis, and Niemann-Pick disease type C)

\subsubsection*{{Explanation}}
\hypertarget{Miglustat}
To determine if miglustat is safe for use in children, I reviewed all available abstracts for evidence from targeted pediatric safety studies. The following relevant abstracts were identified:

1. \hyperlink{pmid_21658716}{PMID: 21658716} (Guffon et al., 2011): This was a randomized, double-blind, placebo-controlled phase IIb/III study of miglustat in patients with mucopolysaccharidosis type III. The study included children (exact age range not specified in the abstract, but mucopolysaccharidosis type III is a pediatric-onset disorder). Safety and tolerability were assessed throughout the study. The abstract states: "Miglustat has an acceptable safety profile." No specific safety concerns were raised in the summary.

2. \hyperlink{pmid_28577204}{PMID: 28577204} (Deodato et al., 2017): This case series reports on three Italian patients with GM1-gangliosidosis treated with miglustat, two of whom had the juvenile form and started treatment at ages 10 and 17. The abstract does not mention any significant safety concerns or adverse events in these pediatric patients.

3. \hyperlink{pmid_27599728}{PMID: 27599728} (Cuisset et al., 2016): This is a case report of a 16-year-old boy with Niemann-Pick disease type C treated with miglustat for approximately 3 years. The abstract does not mention any significant safety issues during treatment.

4. \hyperlink{pmid_21779792}{PMID: 21779792} (Belmatoug et al., 2011): This review states that miglustat is approved for the treatment of progressive neurological manifestations in adult and pediatric patients with Niemann-Pick disease type C. The most frequent adverse events are gastrointestinal and are generally mild or moderate. The review provides practical recommendations for managing these side effects but does not report any severe or unique safety concerns in children.

5. \hyperlink{pmid_16285070}{PMID: 16285070} (Miglustat: new drug..., 2005): This review notes that miglustat is licensed for oral therapy in type 1 Gaucher's disease, but does not specify pediatric data or safety in children.

6. Other studies (e.g., \hyperlink{pmid_15505381}{PMID: 15505381}, 27836529, 17609429) focus on adult populations or do not specify pediatric data.

Summary by age range:
- Children with mucopolysaccharidosis type III (likely ages 2–18): One randomized controlled trial found an "acceptable safety profile" for miglustat [\hyperlink{pmid_21658716}{PMID: 21658716}, Guffon et al., 2011].
- Children/adolescents with juvenile GM1-gangliosidosis (ages 10 and 17): Case series found no significant safety concerns [\hyperlink{pmid_28577204}{PMID: 28577204}, Deodato et al., 2017].
- Adolescents with Niemann-Pick disease type C (age 16): Case report found no significant safety concerns [\hyperlink{pmid_27599728}{PMID: 27599728}, Cuisset et al., 2016].
- General pediatric use in Niemann-Pick disease type C: Review affirms approval and describes only mild/moderate gastrointestinal side effects [\hyperlink{pmid_21779792}{PMID: 21779792}, Belmatoug et al., 2011].

Conclusion:
There is evidence from a randomized controlled trial and case series/reports that miglustat has an acceptable safety profile in children and adolescents with certain lysosomal storage diseases (mucopolysaccharidosis type III, GM1-gangliosidosis, Niemann-Pick disease type C). However, the number of pediatric patients studied is small, and most data are from rare disease populations. No studies were found that show miglustat is unsafe in children. Therefore, based on the available abstracts, miglustat appears to be safe for use in children with these specific conditions, but broader pediatric safety is not established.

\subsection*{Abstracts}
\hypertarget{pmid_21658716}{T}o evaluate the efficacy and safety of oral miglustat treatment in patients with mucopolysaccharidosis type III. The primary outcome was efficacy with improvement or stabilization in at least two domains of Vineland Adaptative Behavior Scales at 6 months. The secondary outcome measured the evolution of other cognitive tests at 12 months. The safety and tolerability were assessed throughout the study. This was a randomized, double-blind, placebo-controlled, monocenter, institutional, phase IIb to III study. In case of efficacy at 6 months, the study would go on for another 6 months on an open design with all patients receiving miglustat. In the absence of efficacy at 6 months, the trial had to be continued for 6 more months with the initial design. After 6 months, efficacy was not superior in patients with miglustat. The independent review board confirmed continuing the study until 12 months. Miglustat treatment was not associated with any improvement/stabilization in behavior problems in patients with mucopolysaccharidosis type III. Miglustat has an acceptable safety profile. However, the study has confirmed that miglustat is able to pass through the blood-brain barrier without significantly decreasing ganglioside levels. [\hyperlink{Miglustat}{PMID: 21658716}, Nathalie Guffon et al., 2011]

\hypertarget{pmid_28577204}{J}uvenile and adult GM1-gangliosidosis are invariably characterized by progressive neurological deterioration. To date only symptomatic therapies are available. We report for the first time the positive results of Miglustat (OGT 918, N-butyl-deoxynojirimycin) treatment on three Italian GM1-gangliosidosis patients. The first two patients had a juvenile form (enzyme activity ≤5\%, GLB1 genotype p.R201H/c.1068 + 1G > T; p.R201H/p.I51N), while the third patient had an adult form (enzyme activity about 7\%, p.T329A/p.R442Q). Treatment with Miglustat at the dose of 600 mg/day was started at the age of 10, 17 and 28 years; age at last evaluation was 21, 20 and 38 respectively. Response to treatment was evaluated using neurological examinations in all three patients every 4-6 months, the assessment of Movement Disorder-Childhood Rating Scale (MD-CRS) in the second patient, and the 6-Minute Walking Test (6-MWT) in the third patient. The baseline neurological status was severely impaired, with loss of autonomous ambulation and speech in the first two patients, and gait and language difficulties in the third patient. All three patients showed gradual improvement while being treated; both juvenile patients regained the ability to walk without assistance for few meters, and increased alertness and vocalization. The MD-CRS class score in the second patient decreased from 4 to 2. The third patient improved in movement and speech control, the distance covered during the 6-MWT increased from 338 to 475 m. These results suggest that Miglustat may help slow down or reverse the disease progression in juvenile/adult GM1-gangliosidosis. [\hyperlink{Miglustat}{PMID: 28577204}, Federica Deodato et al., 2017]

\hypertarget{pmid_34050973}{T}o investigate the efficacy and safety of home-treatment with oral piv-mecillinam or amoxicillin-clavulanate in children with acute pyelonephritis. Children aged over 6 months diagnosed with culture confirmed pyelonephritis at Danish Paediatric Departments were home-treated with piv-mecillinam (tablets) or amoxicillin-clavulanate (liquid or tablets). Follow-up was performed by phone (second treatment day) and clinical review of the patients in the hospital (day three). Four hundred eighteen children were included. In total, 333/418 (80\%) responded well to the initial oral antibiotic treatment. 85/418 (20\%) were changed to another treatment of these 47/418 (11\%) to a second-line oral antibiotic and 38/418 (9\%) to intravenous antibiotics due to insufficient clinical improvement or bacterial resistance. Bacterial resistance was similar for piv-mecillinam and amoxicillin-clavulanate: 4/74 (5\%) versus 33/333 (10\%) (p = 0.22). Insufficient clinical improvement, despite no resistance, primarily occurred in children treated with piv-mecillinam: 16/74 (22\%) versus 28/344 (8\%) (p < 0.001), and predominantly occurred in piv-mecillinam treated children <5 years: 7/20 (35\%) versus 9/54 (17\%) (p < 0.05), potentially because of problems with piv-mecillinam tablets. In the study population no cases of death or septicemia developed after start of initial oral treatment. A home-treatment regime for pyelonephritis in children >6 months is safe; however, during treatment, clinical re-evaluation is required as in 20\% of cases a change in treatment was necessary. [\hyperlink{Miglustat}{PMID: 34050973}, Line Thousig Sehested et al., 2021] (1) For patients with type 1 Gaucher's disease the standard treatment is imiglucerase enzyme replacement therapy, provided in fortnightly intravenous infusions. (2) Miglustat inhibits the synthesis of glucosyl-ceramide, the cerebroside that accumulates in Gaucher's disease. Miglustat is now licensed for oral therapy in patients with mild to moderate type 1 Gaucher's disease and who cannot take imiglucerase, regardless of the reason. (3) The evaluation data we managed to gather (see literature search) includes data from three trials involving a total of 82 patients. One of these trials compared miglustat with ongoing imiglucerase therapy. Miglustat slightly reduced the size of the liver and spleen, and slightly increased the haemoglobin level and platelet count after 18 months. The impact of these effects is unknown, especially on bone disorders. In patients with previous response to imiglucerase, miglustat has not been found to maintain clinical effects in the longer term. (4) Miglustat has many adverse effects, some of which occur very frequently, such as diarrhea (86\%), weight loss (64\%), peripheral neuropathies (19\%), tremor (29\%), and cognitive disorders. Animal studies suggest a risk of reproductive toxicity. (5) In practice, miglustat therapy offers minimal benefits for the few patients who cannot use imiglucerase. The potential advantages of miglustat therapy relative to purely symptomatic treatment must be carefully weighed in individual patients. [\hyperlink{Miglustat}{PMID: 34050973}, Miglustat: new drug. In type 1 Gaucher's disease : a slight benefit after imiglucerase therapy., 2005]

\hypertarget{pmid_22281182}{P}reclinical data suggest that miglustat could restore the function of the cystic fibrosis transmembrane conductance regulator gene in cystic fibrosis cells. Single-center, randomized, double-blind, placebo-controlled, crossover Phase II study in 11 patients (mean±SD age, 26.3±7.7 years) homozygous for the F508del mutation received oral miglustat 200 mgt.i.d. or placebo for two 8-day cycles separated by a 14-day washout period. The primary endpoint was the change in total chloride secretion (TCS) assessed by nasal potential difference. No statistically significant changes in TCS, sweat chloride values or FEV(1) were detected. Pharmacokinetic and safety were similar to those observed in patients with other diseases exposed to miglustat. There was no evidence of a treatment effect on any nasal potential difference variable. Further studies with miglustat need to adequately address criteria for assessment of nasal potential difference. [\hyperlink{Miglustat}{PMID: 22281182}, Anissa Leonard et al., 2012]

\hypertarget{pmid_15505381}{I}t has been shown that treatment with miglustat (Zavesca, N-butyldeoxynojirimycin, OGT 918) improves key clinical features of type I Gaucher disease after 1 year of treatment. This study reports longer-term efficacy and safety data. Patients who had completed 12 months of treatment with open-label miglustat (100-300 mg three times daily) were enrolled to continue with therapy in an extension study. Data are presented up to month 36. Liver and spleen volumes measured by CT or MRI were scheduled every 6 months. Biochemical and haematological parameters, including chitotriosidase activity (a sensitive marker of Gaucher disease activity) were monitored every 3 months. Safety data were also collected every 3 months. Eighteen of 22 eligible patients at four centres entered the extension phase and 14 of these completed 36 months of treatment with miglustat. After 36 months, there were statistically significant improvements in all major efficacy endpoints. Liver and spleen organ volumes were reduced by 18\% and 30\%, respectively. In patients whose haemoglobin value had been below 11.5 g/dl at baseline, mean haemoglobin increased progressively from baseline by 0.55 g/dl at month 12 (NS), 1.28 g/dl at month 24 (p =0.007), and 1.30 g/dl at month 36 (p =0.013). The mean platelet count at month 36 increased from baseline by 22 x 10(9)/L. No new cases of peripheral neuropathy occurred since previously reported. Diarrhoea and weight loss, which were frequently reported during the initial 12-month study, decreased in magnitude and prevalence during the second and third years. Patients treated with miglustat for 3 years show significant improvements in organ volumes and haematological parameters. In conclusion, miglustat was increasingly effective over time and showed acceptable tolerability in patients who continued with treatment for 3 years. [\hyperlink{Miglustat}{PMID: 15505381}, D Elstein et al., 2004]

\hypertarget{pmid_12641681}{T}he bitter taste of midazolam is more acceptable to children when the drug is mixed with fruit juice or syrup. We use a thick grape syrup (Syrpalta), and children are sedated in 10-15 min. A premixed cherry-flavoured midazolam solution (Roche), 2 mg.ml (-1), is currently available. It has been our impression that the premixed midazolam has a slower onset of action. Our aim was to evaluate the effects of the midazolam mixtures (midazolam 0.5 mg.kg (-1), 2 mg.ml (-1)) on children's anxiety, sedation, separation anxiety, mask acceptance, and recovery time. Seventy-six healthy children, 1-4 years of age, scheduled for elective placement of ear tubes, were enrolled. The trial was double-blinded and randomized. For premedication, one group received the premixed midazolam, and a second group received the midazolam/Syrpalta mixture. An independent blinded observer evaluated the children, using anxiety and sedation scales at baseline, at 5, 10 and 15 min and at parental separation. Mask acceptance and awakening time were evaluated. Children who received the midazolam/Syrpalta mixture had less anxiety at 15 min (P = 0.046) and at parental separation (P < 0.001) than those who received the premixed midazolam solution. Mask acceptance was not different. We concluded that the midazolam/Syrpalta mixture has a faster onset of action than the premixed midazolam solution. [\hyperlink{Miglustat}{PMID: 12641681}, Samia N Khalil et al., 2003]

\hypertarget{pmid_27836529}{W}e report data from a prospective, observational study (ZAGAL) evaluating miglustat 100mg three times daily orally. in treatment-naïve patients and patients with type 1 Gaucher Disease (GD1) switched from previous enzyme replacement therapy (ERT). Clinical evolution, changes in organ size, blood counts, disease biomarkers, bone marrow infiltration (S-MRI), bone mineral density by broadband ultrasound densitometry (BMD), safety and tolerability annual reports were analysed. Between May 2004 and April 2016, 63 patients received miglustat therapy; 20 (32\%) untreated and 43 (68\%) switched. At the time of this report 39 patients (14 [36\%] treatment-naïve; 25 [64\%] switch) remain on miglustat. With over 12-year follow-up, hematologic counts, liver and spleen volumes remained stable. In total, 80\% of patients achieved current GD1 therapeutic goals. Plasma chitotriosidase activity and CCL-18/PARC concentration showed a trend towards a slight increase. Reductions on S-MRI (p=0.042) with an increase in BMD (p<0.01) were registered. Gastrointestinal disturbances were reported in 25/63 (40\%), causing miglustat suspension in 11/63 (17.5\%) cases. Thirty-eight patients (60\%) experienced a fine hand tremor and two a reversible peripheral neuropathy. Overall, miglustat was effective as a long-term therapy in mild to moderate naïve and ERT stabilized patients. No unexpected safety signals were identified during 12-years follow-up. [\hyperlink{Miglustat}{PMID: 27836529}, Pilar Giraldo et al., 2018]

\hypertarget{pmid_22976762}{M}iglustat is an oral medication that has approved indication for type I Gaucher disease and Niemann pick disease type C. Usually treatment with Miglustat is associated with occurrence of gastrointestinal side effects similar to carbohydrate maldigestion symptoms. Here, we studied the direct influence of Miglustat on the enzymatic function of the major disaccharidases of the intestinal epithelium. Our findings show that an immediate effect of Miglustat is its interference with carbohydrate digestion in the intestinal lumen via reversible inhibition of disaccharidases that cleave α-glycosidically linked carbohydrates. Higher non physiological concentrations of Miglustat can partly affect lactase activity. We further show that the inhibition of the disaccharidases function by Miglustat is mainly competitive and does not occur via alteration of the enzyme folding. [\hyperlink{Miglustat}{PMID: 22976762}, Mahdi Amiri et al., 2012]

\hypertarget{pmid_10741880}{T}he aim of this article is to review data on the efficacy and safety of montelukast in the treatment of children with asthma. Available published literature, including published abstracts, is reviewed. In patients aged 6 to 14 years with asthma (n = 27), montelukast 5mg demonstrated a significant decrease in exercise-induced bronchoconstriction 20 to 24 hours postdose after 2 days of treatment. For children with chronic asthma, only one study of the regular use of a leukotriene receptor antagonist has been published. The efficacy and safety of montelukast in children aged 6 to 14 years with asthma (n = 336) were studied during an 8-week, double-blind, placebocontrolled trial. There was a significantly greater improvement in forced expiratory volume in 1 second (FEV1) from baseline for the montelukast group (8.23\%) compared with the placebo group (3.58\%). There was a significant decrease in the use of a 3-agonist for symptom relief, as well as in the percentage of days and percentage of patients with asthma exacerbations. An asthma specific quality-of-life (QOL) questionnaire revealed a significant overall improvement in QOL and a significant improvement in the QOL domains for symptoms, activity and emotions in montelukast recipients. There was no significant difference between montelukast and placebo recipients in the frequency of adverse events, with the exception of allergic rhinitis, which was more prevalent in the placebo group. An open label follow-up of patients from the above study was undertaken. The effect of montelukast on FEV1 was consistent for up to 1.4 years, with the increase in FEV1 being not significantly different from that in a small control group treated with inhaled beclomethasone dipropionate. QOL remained significantly improved during the open treatment period. Montelukast appears effective and safe for the treatment of children with asthma. [\hyperlink{Miglustat}{PMID: 10741880}, A Becker et al., 2000]

\hypertarget{pmid_17934951}{L}imited information exists on the toxicity of pediatric ingestions of the drug montelukast used in the treatment of chronic asthma. All ingestions of montelukast involving children age 0-5 yr reported to Texas poison control centers during 2000-2005 were retrieved. For a subset of cases where the final medical outcome and dose in milligrams or milligrams per kilogram were known, the pattern of exposures by final medical outcome and management site was evaluated. There was a total of 3698 cases. Of those cases with a known final medical outcome and dose, the mean dose in milligrams was 42.5 mg (range 0.4-536 mg) and the mean dose in milligrams per kilogram was 3.36 mg/kg (range 0.18-33.71 mg/kg). The final medical outcome was no observed effect in 95\% of the cases and minor effect in the remainder of the cases. The patient was managed on site in 80\% of the cases. The proportion of cases with a minor effect increased from 5\% for ingested dose of < or = 100 mg to 10\% for > 100 mg but was 5\% for dose < or = 5 mg/kg and > 5 mg/kg. The proportion of cases managed with health care facility involvement increased from 15\% for ingested dose of < or = 100 mg to 56\% for > 100 mg and rose from 10\% for dose < or = 5 mg/kg to 47\% for dose > 5 mg/kg. Pediatric montelukast ingestions of doses up to 536 mg or 33.71 mg/kg do not appear likely to result in serious adverse effects and usually can be managed at home. [\hyperlink{Miglustat}{PMID: 17934951}, Mathias B Forrester et al., 2007]

\hypertarget{pmid_15102869}{M}ontelukast is a cysteinyl leukotriene receptor antagonist approved for the treatment of asthma for those ages 1 year old to adult. The purpose of this study was to evaluate the pharmacokinetic comparability of a 4-mg dose of montelukast oral granules in patients > or = 6 to < 24 months old to the 10-mg approved dose in adults. This was an open-label study in 32 patients. Population pharmacokinetic parameters included estimates of AUC(pop), C(max), and t(max). Results were compared with estimates from adults (10-mg film-coated tablet [FCT]). Dose selection criteria were for the 95\% confidence interval (CI) for the AUC(pop) estimate ratio (pediatric/adult 10 mg FCT) to be within comparability bounds of (0.5, 2.00). The AUC(pop) ratio and the 95\% CI for children compared with adults were within the predefined comparability bounds. Observed plasma concentrations were also similar. Based on systemic exposure of montelukast, a 4-mg dose of montelukast appears appropriate for children as young as 6 months of age. [\hyperlink{Miglustat}{PMID: 15102869}, Elizabeth Migoya et al., 2004]

\hypertarget{pmid_10741881}{T}he tolerability of a medication, especially in children with asthma, is linked to a number of key factors. These include clinical effectiveness, adverse effects, frequency of drug regimen, ease and route of administration. and taste. Montelukast is unusual in that, in most countries, a licence for children aged > or =6 years was granted at the same time as the adult licence. This is related to a variety of evidence. which includes pharmacological and adult studies suggesting the specificity and safety of the drug at many times the licensed dose, and a tolerability profile similar to that with placebo or inhaled corticosteroids in both adult and paediatric studies. The most common adverse effects in paediatric studies were headache, asthma and upper respiratory tract infection at rates not statistically significantly different from those with placebo. Up to July 1999, more than 2 million patients worldwide have received montelukast, of whom nearly 220,000 have received the paediatric formulation. In the UK, one prescribing database suggests that, of children who commenced montelukast therapy, less than 25\% discontinued the drug. This implies that montelukast is effective and well tolerated in most children. Adverse effect monitoring by regulatory bodies has revealed little that would not be expected on the basis of the results of clinical trials. Montelukast has been associated with Churg-Strauss syndrome in a very small number of adults. In most. the syndrome was associated with corticosteroid withdrawal, which may have unmasked the condition. Churg-Strauss syndrome has not been reported in children. Its clinical effectiveness, lack of major adverse effects, oral route of administration, palatability and the once-daily regimen combine to make montelukast a generally well tolerated medication in children. [\hyperlink{Miglustat}{PMID: 10741881}, D Price et al., 2000]

\hypertarget{pmid_18548983}{B}ased on the outcome of several randomized controlled trials, the orally active leukotriene receptor antagonist montelukast (Singulair, Merck) has been licensed for treatment of asthma. The drug is favored for treating childhood asthma, where a therapeutic challenge has arisen due to poor compliance with inhalation therapy. To assess the efficiency of and satisfaction with Singulair in asthmatic children under real-life conditions. Montelukast was prescribed for 6 weeks to a cohort of 506 children aged 2 to 18 years with mild to moderate persistent asthma, who were enrolled by 200 primary care pediatricians countrywide. Four clinical correlates of childhood asthma--wheeze, cough, difficulty in breathing, night awakening--were evaluated from patients' diary cards. Due to under-treatment by their physicians, almost 60\% of the children were not receiving controller therapy at baseline. By the end of the study, which consisted of montelukast treatment, a significant improvement over baseline was noted in asthma symptoms and severity, as well as in treatment compliance. The participating pediatricians and parents were highly satisfied with the treatment. The results of this extensive study show that the use of montelukast as monotherapy in children presenting with persistent asthma resulted in a highly satisfactory outcome for themselves, their parents and their physicians. [\hyperlink{Miglustat}{PMID: 18548983}, Israel Amirav et al., 2008]

\hypertarget{pmid_15891924}{M}icturating cystourethrogram (MCUG) is an imaging technique indicated in the diagnosis and follow-up of many diseases. We investigated the reliability and the efficacy of midazolam and chloral hydrate in sedation and anxiolysis during micturating cystourethrogram. Fifty-three children of similar ages (39 girls, 14 boys, mean age of 5.8+/-3.5 years) were randomized to midazolam (n=17), chloral hydrate (n=18) and control groups (n=18). Oral midazolam 0.6 mg/kg or chloral hydrate 25 mg/kg or saline were administered to the study groups 15-30 min prior to the urinary catheterization. Brietkopf and Buttner, Frankl and Houpt scales and Spielberger's State Anxiety Inventory and parent's impressions were used to assess the level of sedation and anxiety. The Brietkopf and Buttner classification of emotional status and Houpt behavior rating scale demonstrated a significantly better emotional status and sedation in the midazolam group when compared to controls (P=0.01 and P=0.018, respectively). The catheterization was described as a more unpleasant and distressing event by the parents of the control and the chloral hydrate groups when compared to the parents of the midazolam group (P<0.05). Bladder capacity and frequency of detection of residual urine were not statistically different between the three study groups (P>0.05). Vital signs did not change significantly in any child. Sedation with midazolam does not have adverse effects on the results of micturating cystourethrogram, while it reduces the discomfort in children undergoing this radiological technique. [\hyperlink{Miglustat}{PMID: 15891924}, Ipek Akil et al., 2005]

\hypertarget{pmid_17609429}{E}nzyme replacement therapy (ERT) with imiglucerase reduces hepatosplenomegaly and improves hematologic parameters in Gaucher disease type 1 within 6-24 months. Miglustat reduces organomegaly, improves hematologic parameters, and reverses bone marrow infiltration. This trial evaluates miglustat in patients clinically stable on ERT. Tolerability of miglustat and imiglucerase, alone and in combination, pharmacokinetic profile, organ reduction, and chitotriosidase activity were assessed. Thirty-six patients stable on imiglucerase were randomized into this phase II, open-label trial. Statistically significant changes from baseline were assessed (paired t test) on primary objectives with secondary analyses on biochemical and safety parameters. Liver and spleen volume were unchanged in switched patients. No significant differences were seen between groups regarding mean change in hemoglobin. Mean change in platelet counts was only significant between miglustat and imiglucerase groups (P = .035). Chitotriosidase activity remained stable. In trial extension, clinical endpoints were generally maintained. Miglustat was well tolerated alone or in combination. Miglustat's safety profile was consistent with previous trials; moreover, no new cases of peripheral neuropathy were observed. Gaucher disease type 1 (GD1) parameters were stable in most switched patients. Combination therapy did not show benefit. Findings suggest miglustat could be an effective maintenance therapy in stabilized patients with GD1. [\hyperlink{Miglustat}{PMID: 17609429}, Deborah Elstein et al., 2007]

\hypertarget{pmid_21779792}{M}iglustat (Zavesca®) is approved for the oral treatment of adult patients with mild to moderate type 1 Gaucher disease (GD1) for whom enzyme replacement therapy is unsuitable, and for the treatment of progressive neurological manifestations in adult and paediatric patients with Niemann-Pick disease type C (NP-C). Gastrointestinal disturbances such as diarrhoea, flatulence and abdominal pain/discomfort have consistently been reported as the most frequent adverse events associated with miglustat during clinical trials and in real-world clinical practice settings. These adverse events are generally mild or moderate in severity, occurring mostly during the initial weeks of therapy. The mechanism underlying these gastrointestinal disturbances is the inhibition by miglustat of intestinal disaccharidase enzymes (mainly sucrase and maltase), leading to sub-optimal hydrolysis of carbohydrates and subsequent osmotic diarrhoea and altered colonic fermentation. Transient decreases in body weight, which are often observed during initial miglustat therapy, are considered likely due to gastrointestinal carbohydrate malabsorption and associated negative caloric balance. While most cases of diarrhoea resolve spontaneously during continued miglustat therapy, diarrhoea also responds well to anti-propulsive medications such as loperamide. Dietary modifications such as reduced consumption of dietary sucrose, maltose and lactose have been shown to improve the gastrointestinal tolerability of miglustat and reduce the magnitude of any changes in body weight, particularly if initiated at or before the start of therapy. Miglustat dose escalation at treatment initiation may also reduce gastrointestinal disturbances. This article discusses these aspects in detail, and provides practical recommendations on how to optimize the gastrointestinal tolerability of miglustat. [\hyperlink{Miglustat}{PMID: 21779792}, Nadia Belmatoug et al., 2011]

\hypertarget{pmid_27599728}{N}iemann-Pick disease type C is a rare inherited neurodegenerative disease involving impaired intracellular lipid trafficking and accumulation of glycolipids in various tissues, including the brain. Miglustat, a reversible inhibitor of glucosylceramide synthase, has been shown to be effective in the treatment of progressive neurological manifestations in pediatric and adult patients with Niemann-Pick disease type C, and has been used in that indication in Europe since 2010. We describe the case of a 16-year-old white French boy with late-infantile-onset Niemann-Pick disease type C who had the unusual presentation of early-onset behavioral disturbance and learning difficulties (aged 5) alongside epileptic seizures. Over time he developed characteristic, progressive vertical ophthalmoplegia, ataxic gait, and cerebellar syndrome; at age 10 he was diagnosed as having Niemann-Pick disease type C based on filipin staining and genetic analysis (heterozygous I1061T/R934X NPC1 mutations). He was commenced on miglustat therapy aged 11 and over the course of approximately 3 years he showed a global improvement as well as improved cognitive and ambulatory function. During this period he remained seizure free on antiepileptic therapy, using valproate and lamotrigine. Miglustat improved the neurological status of our patient, including seizure control. Based on our findings in this patient and previous published data, we discuss the importance of effective seizure control in neurological improvement in Niemann-Pick disease type C, and the relevance of cerebellar involvement as a possible link between these clinical phenomena. Thus the therapeutic efficacy of miglustat could be hypothesized as a substrate reduction effect on Purkinje cells. [\hyperlink{Miglustat}{PMID: 27599728}, Jean-Marie Cuisset et al., 2016]

\hypertarget{pmid_20885413}{C}onscious sedation for young children is a rapidly developing area of clinical activity. Many studies have shown positive results using oral midazolam on children. These case series investigated oral midazolam conscious sedation as an alternative to general anaesthesia in a clinical service setting. The purpose of this work was to determine the safety and efficacy of oral midazolam for conscious sedation in children undergoing dental treatment. Patients were selected by colleagues for treatment under oral sedation. The main general criteria were weight below 36 kilos and ASA I, II, or III. Midazolam 0.5 mg/kg was administered orally. A pulse oximeter was applied to a finger to monitor vital signs and the Houpt scale was used to assess behaviour. A total of 510 children aged between 13 months and 11 years were included. The behaviour of 379 (74\%) was excellent or very good. The pulse rate and peripheral oxygenation were within the normal range for all patients. The main adverse effects were diplopia and post-sedation dysphoria. Oral midazolam is a safe and effective method of sedation although some children were agitated and distressed either during or after treatment. Parents need to be warned about this. [\hyperlink{Miglustat}{PMID: 20885413}, L Lourenço-Matharu et al., 2010]

\hypertarget{pmid_24863482}{M}iglustat is an oral medication for treatment of lysosomal storage diseases such as Gaucher disease type I and Niemann Pick disease type C. In many cases application of Miglustat is associated with symptoms similar to those observed in intestinal carbohydrate malabsorption. Previously, we have demonstrated that intestinal disaccharidases are inhibited immediately by Miglustat in the intestinal lumen. Nevertheless, the multiple functions of Miglustat hypothesize long term effects of Miglustat on intracellular mechanisms, including glycosylation, maturation and trafficking of the intestinal disaccharidases. Our data show that a major long term effect of Miglustat is its interference with N-glycosylation of the proteins in the ER leading to a delay in the trafficking of sucrase-isomaltase. Also association with lipid rafts and plausibly apical targeting of this protein is partly affected in the presence of Miglustat. More drastic is the effect of Miglustat on lactase-phlorizin hydrolase which is partially blocked intracellularly. The de novo synthesized SI and LPH in the presence of Miglustat show reduced functional efficiencies according to altered posttranslational processing of these proteins. However, at physiological concentrations of Miglustat (≤50 μM) a major part of the activity of these disaccharidases is found to be still preserved, which puts the charge of the observed carbohydrate maldigestion mostly on the direct inhibition of disaccharidases in the intestinal lumen by Miglustat as the immediate side effect.  [\hyperlink{Miglustat}{PMID: 24863482}, Mahdi Amiri et al., 2014] The efficacy and safety of intravenous and sequential intravenous-oral clavulanate-potentiated amoxycillin therapy was evaluated in 71 hospitalized paediatric patients, one month to 16 years of age. The infections treated included peritonsillar abscess (2 patients), purulent tracheitis (1), acute epiglottitis (24), pneumonia (31), pansinusitis (4), mastoiditis (1), cellulitis (4), lymphadenitis (2) and pyelonephritis (2). The severity of disease was rated as moderate in 31 patients (44\%), and as severe in 40 (56\%). Bacterial pathogens could be cultured in 26 cases (37\%). The response to therapy was prompt and followed by clinical cure in each patient. Adverse drug effects included phlebitis (in 6\%), mild gastrointestinal complaints (6\%), rash (4\%) and transient neutropenia and elevation of transaminases (one case each). It is concluded that amoxycillin/clavulanate is effective and safe treatment for bacterial infections of the respiratory tract, urinary tract, skin or soft tissues in children. [\hyperlink{Miglustat}{PMID: 24863482}, U B Schaad et al., 1987]

\hypertarget{pmid_17766511}{A} recurring epidemic of asthma exacerbations in children occurs annually in September in North America when school resumes after summer vacation. Our goal was to determine whether montelukast, added to usual asthma therapy, would reduce days with worse asthma symptoms and unscheduled physician visits of children during the September epidemic. A total of 194 asthmatic children aged 2 to 14 years, stratified according to age group (2-5, 6-9, and 10-14 years) and gender, participated in a double-blind, randomized, placebo-controlled trial of the addition of montelukast to usual asthma therapy between September 1 and October 15, 2005. Children randomly assigned to receive montelukast experienced a 53\% reduction in days with worse asthma symptoms compared with placebo (3.9\% vs 8.3\%) and a 78\% reduction in unscheduled physician visits for asthma (4 [montelukast] vs 18 [placebo] visits). The benefit of montelukast was seen both in those using and not using regular inhaled corticosteroids and among those reporting and not reporting colds during the trial. There were differences in efficacy according to age and gender. Boys aged 2 to 5 years showed greater benefit from montelukast (0.4\% vs 8.8\% days with worse asthma symptoms) than did older boys, whereas among girls the treatment effect was most evident in 10- to 14-year-olds (4.6\% [montelukast] vs 17.0\% [placebo]), with nonsignificant effects in younger girls. Montelukast added to usual treatment reduced the risk of worsened asthma symptoms and unscheduled physician visits during the predictable annual September asthma epidemic. Treatment-effect differences observed between age and gender groups require additional investigation. [\hyperlink{Miglustat}{PMID: 17766511}, Neil W Johnston et al., 2007]

\hypertarget{pmid_11167954}{M}ontelukast is a leukotriene receptor antagonist administered orally once daily for treatment of chronic asthma in adults and children. A comprehensive analysis of safety data from double-blind, randomized, placebo-controlled trials with montelukast has not been previously reported. A pooled analysis of safety data from 11 multicentre, randomized, controlled montelukast Phase IIb and III trials and five long-term extension studies was performed. A total of 3386 adult patients (aged 15-85 years) and 336 paediatric patients (aged 6-14 years) were enrolled in the trials; 2031 adults received montelukast for up to 4.1 years, and 257 children received montelukast for up to 1.8 years. Summary statistics comparing incidences of adverse events among treatment groups were calculated. The overall incidence of clinical and laboratory adverse events among montelukast-treated patients, both adult and paediatric, was similar to that among patients receiving placebo. There were no clinically relevant differences in individual adverse events, including infectious upper respiratory conditions and transaminase elevations, between montelukast and placebo groups. Discontinuations due to adverse events occurred with similar frequencies during placebo, montelukast and inhaled beclomethasone therapy. No dose-related adverse effects of montelukast were observed in adults treated with dosages as high as 200 mg per day (20 times the recommended dose) for 5 months. This tolerability profile montelukast observed in clinical trials has been generally reflected in the post-marketing safety experience seen to date. These data indicate a tolerability profile for montelukast similar to placebo during both short-term and long-term administration, even at doses substantially higher than the recommended clinical dose of 10 mg once daily for adults and 5 mg once daily for children aged 6-14 years. [\hyperlink{Miglustat}{PMID: 11167954}, W Storms et al., 2001]

\hypertarget{pmid_38085143}{O}ral Montelukast is recommended as maintenance therapy for persistent asthma, but there is controversy regarding its effectiveness in controlling asthma attacks. The present study was conducted to investigate the clinical efficacy of oral Montelukast for asthma attacks in children. This study was conducted as a double-blind placebo-controlled clinical trial on 80 children aged 1-14 years with asthma who were admitted to the emergency department of Bahrami Children's Hospital (Tehran, Iran) during one year. Patients were randomly divided into case and control groups. In addition to the standard asthma attack treatment, Montelukast was prescribed in the case group and placebo in the control group for one week. Patients were evaluated in terms of asthma attack severity score and oxygen saturation percentage (SpO2) in room air as primary outcomes 1, 4, 8, 24 and 48 hours after admission. In the first 48 hours, there was no significant difference in the score of asthma attack severity and SpO2 between the case and control groups. There was no significant difference between the groups in terms of length of hospitalization or number of admissions to the intensive care unit. None of the patients were re-hospitalized after discharge. The results of this study showed that the use of Montelukast along with the standard treatment of asthma attacks in children has no added benefit. [\hyperlink{Miglustat}{PMID: 38085143}, Mohsen Jafari et al., 2023]

\hypertarget{pmid_7717236}{M}idazolam is a relatively short-acting water-soluble benzodiazepine that provides anxiolysis and anterograde amnesia and can be given orally with few adverse effects. We evaluated the benefit and safety of oral midazolam for sedation of young children during voiding cystourethrography or nuclear cystography. For 3.5 years, a highly selected group of 98 children, ages 23 months to 9 years (mean, 4 years), were given oral midazolam 0.6 mg/kg 20-30 min before cystourethrography or nuclear cystography. These children either had been frightened by a previous catheterization (39\%) or seemed particularly frightened during an examination of their genitals in the office (61\%). A control group of 25 children, similar in age to the study group, did not receive midazolam before cystourethrography. Parents were interviewed to assess their child's recollection of the procedure. Voiding dynamics were assessed by evaluating the postvoiding radiograph. Of the midazolam-treated patients, 60\% had no recollection of the study, and 31\% remembered part or all of the study but did not have a negative experience. No significant change in vital signs or oxygen saturation was observed in any child. In the control group, 24 (96\%) of 25 children remembered the cystographic examination (p < .01). Behavioral side effects occurred in 12\% of the children receiving midazolam and consisted primarily of combative behavior as the medication was wearing off. Ninety-five percent of the parents indicated that they would want their child to have midazolam again if the cystography needed to be repeated. Of the children receiving midazolam, 76\% had little or no residual urine after voiding, compared with 72\% of the control group (no significant difference). In children who have been or are likely to be excessively frightened during cystourethrography or nuclear cystography, midazolam usually provides satisfactory amnesia and anxiolysis with few side effects or adverse impact on voiding dynamics. [\hyperlink{Miglustat}{PMID: 7717236}, J S Elder et al., 1995]

\section*{Rifabutin}
\subsection*{Result}
\subsubsection*{Answer}

Yes (Infants and young children, median age 1.7 years, IQR 0.9-5.0 years)
Yes (Children and adolescents, ages 3-15 years)

\subsubsection*{{Explanation}}
\hypertarget{Rifabutin}
Based on the available abstracts, there are several that specifically address the safety of rifabutin in children:

1. **Rawizza et al., 2021 [\hyperlink{pmid_33294914}{PMID: 33294914}]**: This prospective study evaluated rifabutin safety and pharmacokinetics in 15 children aged 3-15 years (median age 13.1 years) with TB/HIV coinfection receiving rifabutin 2.5 mg/kg daily with lopinavir/ritonavir. Intensive clinical and laboratory monitoring was performed. Severe neutropenia (grade 3 or 4) occurred in some children but resolved without treatment interruption or clinical sequelae. The study concludes that rifabutin achieved appropriate drug exposures and was generally safe, supporting its use in TB/HIV-coinfected children who require lopinavir/ritonavir.

2. **Rawizza et al., 2019 [\hyperlink{pmid_31139825}{PMID: 31139825}]**: This retrospective analysis included 48 children with a median age of 1.7 years (IQR 0.9-5.0 years) who received rifabutin-containing TB therapy with PI-based ART. Adverse events were more common at baseline than during rifabutin treatment. Only one instance (2\%) of grade 3 neutropenia occurred during rifabutin treatment. The authors conclude that, with clinical and laboratory monitoring, rifabutin is a safe option for TB therapy among children on PI-based ART, including infants and young children.

3. **Olesen et al., 2005 [\hyperlink{pmid_16282948}{PMID: 16282948}]**: This is a case report of rifabutin-associated uveitis in a child and a literature review. The report notes that uveitis is a known adverse effect in adults and has been reported in a few pediatric cases, suggesting the need for special precautions but does not provide systematic safety data.

4. **Brughera et al., 1995 [\hyperlink{pmid_7719114}{PMID: 7719114}]**: This abstract describes preclinical safety studies of rifabutin in animals, not in children.

**Summary by Age Range:**
- **Infants and young children (median age 1.7 years, IQR 0.9-5.0 years):** The 2019 study [\hyperlink{pmid_31139825}{PMID: 31139825}] provides evidence from 48 children, including infants, that rifabutin is generally safe when used with clinical and laboratory monitoring, with rare severe neutropenia.
- **Children and adolescents (ages 3-15 years):** The 2021 study [\hyperlink{pmid_33294914}{PMID: 33294914}] provides evidence from 15 children aged 3-15 years that rifabutin is generally safe, with manageable and reversible neutropenia.

**Conclusion:** There are targeted studies in children, including infants, young children, and adolescents, that affirm the safety of rifabutin when used with appropriate monitoring. However, neutropenia is a potential adverse effect and should be monitored.

\subsection*{Abstracts}
\hypertarget{pmid_33294914}{T}reatment options are limited for TB/HIV-coinfected children who require PI-based ART. Rifabutin is the preferred rifamycin for adults on PIs, but the one study evaluating rifabutin with PIs among children was stopped early due to severe neutropenia. We evaluated rifabutin safety and plasma pharmacokinetics among coinfected children 3-15 years of age receiving rifabutin 2.5 mg/kg daily with standard doses of lopinavir/ritonavir. The AUC0-24 at 2, 4 and 8 weeks after rifabutin initiation was described using intensive sampling and non-compartmental analysis. Clinical and laboratory toxicities were intensively monitored at 12 visits throughout the study. Among 15 children with median (IQR) age 13.1 (10.9-14.0) years and weight 25.5 (22.3-30.5) kg, the median (IQR) rifabutin AUC0-24 was 5.21 (4.38-6.60) μg·h/mL. Four participants had AUC0-24 below 3.8 μg·h/mL (a target for the population average exposure) at week 2 and all had AUC0-24 higher than 3.8 μg·h/mL at the 4 and 8 week visits. Of 506 laboratory evaluations during rifabutin, grade 3 and grade 4 abnormalities occurred in 16 (3\%) and 2 (0.4\%) instances, respectively, involving 9 (60\%) children. Specifically, grade 3 (n = 4) and grade 4 (n = 1) neutropenia resolved without treatment interruption or clinical sequelae in all patients. One child died at week 4 of HIV-related complications. In children, rifabutin 2.5 mg/kg daily achieved AUC0-24 comparable to adults and favourable HIV and TB treatment outcomes were observed. Severe neutropenia was relatively uncommon and improved with ongoing rifabutin therapy. These data support the use of rifabutin for TB/HIV-coinfected children who require lopinavir/ritonavir. [\hyperlink{Rifabutin}{PMID: 33294914}, Holly E Rawizza et al., 2021]

\hypertarget{pmid_31139825}{T}B is the leading cause of death among HIV-infected children, yet treatment options for those who require PI-based ART are suboptimal. Rifabutin is the preferred rifamycin for adults on PI-based ART; only one study has evaluated its use among children on PIs and two of six children developed treatment-limiting neutropenia. Since 2009, rifabutin has been available for HIV/TB-coinfected children requiring PI-based ART in the Harvard/APIN programme in Nigeria. We retrospectively analysed laboratory and clinical toxicities at baseline and during rifabutin therapy, and examined HIV/TB outcomes. Between 2009 and 2015, 48 children received rifabutin-containing TB therapy with PI (lopinavir/ritonavir)-based ART: 50\% were female with a median (IQR) baseline age of 1.7 (0.9-5.0) years and a median (IQR) CD4+ cell percentage of 15\% (9\%-25\%); 52\% were ART experienced. Eighty-five percent completed the 6 month rifabutin course with resolution of TB symptoms and 79\% were retained in care at 12 months. Adverse events (grade 1-4) were more common at baseline (27\%) than during rifabutin treatment (15\%) (P = 0.006). Absolute neutrophil count was lower during rifabutin compared with baseline (median = 1762 versus 2976 cells/mm3, respectively), but only one instance (2\%) of grade 3 neutropenia occurred during rifabutin treatment. With clinical and laboratory monitoring, our data suggest that rifabutin is a safe option for TB therapy among children on PI-based ART. By contrast with the only other study of this combination in children, severe neutropenia was rare. Furthermore, outcomes from this cohort suggest that rifabutin is effective, and a novel option for children who require PI-based ART. Additional study of rifabutin plus PIs in children is urgently needed. [\hyperlink{Rifabutin}{PMID: 31139825}, Holly E Rawizza et al., 2019]

\hypertarget{pmid_16645503}{R}ifapentine is a rifamycin antibiotic approved for the treatment of pulmonary infections caused by Mycobacterium tuberculosis. Although the pharmacokinetics of rifapentine has been investigated in adolescents and adults, no studies have assessed the pharmacokinetics of this drug in children or infants. Twenty-four children (7.1 +/- 3.3 years; mean +/- 1 SD, 27.9 +/- 11.9 kg) were enrolled in this open label study. Children received a single oral dose (10 to <30 kg body weight received 150 mg; 30 to <60 kg body weight received 300 mg), followed by repeated blood sampling (n = 11) for 32 hours. Rifapentine and 25-desacetyl rifapentine were quantitated by a validated high-pressure liquid chromatography method. Pharmacokinetic parameters were determined using a model-independent approach. A significant difference in dose-normalized area under the curves (AUC0-n and AUC0-infinity) was observed between children receiving the 150 and 300 mg doses, which was accounted for by differences in age between the dosing arms. In separate analyses, including data from adults, further age-dependence in total body exposure (reflected by AUC) and elimination was observed. Adverse events associated with rifapentine were mild and included gastric distress (n = 1) and vomiting (n = 2). Given a comparable weight-normalized dose, rifapentine exposure estimates are lower in children than those reported in adults, suggesting that a larger weight-normalized (ie, mg/kg) dose of rifapentine is needed in children. [\hyperlink{Rifabutin}{PMID: 16645503}, Michael J Blake et al., 2006]

\hypertarget{pmid_17401268}{T}he present study aimed at verifying the safety and efficacy of rifampicin in ameliorating pruritus in cholestatic children. Twenty-three Egyptian children (14 boys and 9 girls), suffering from intractable pruritus of cholestasis, were included. Rifampicin was started at a dose of 10 mg/Kg/day in two divided doses and increased gradually to a maximum of 20 mg/Kg/day if there was no response. Liver function tests were followed up weekly. Seventeen patients (74\%) showed improvement of pruritus with rifampicin. None of the patients showed any deterioration in liver functions. Rifampicin in a dose of 10-20 mg/Kg/day is safe and effective in ameliorating uncontrollable pruritus in children with persistent cholestasis. [\hyperlink{Rifabutin}{PMID: 17401268}, Hanaa El-Karaksy et al., 2007]

\hypertarget{pmid_23650467}{C}hildhood epilepsy continues to be intractable in more than 25\% of patients diagnosed with epilepsy. The introduction of new anti-epileptic drugs (AEDs) provides more options for treatment of children with epilepsy. We review the safety and tolerability of seven new AEDs (levetiracetam, lamotrigine, oxcarbazepine, rufinamide, topiramate, vigabatrin and zonisamide) focusing on their side effect profiles and safety in children and adolescents. Many considerations that are specific for children such as the impact of AEDs on the developing brain are not addressed during the development of new AEDs. They are usually approved as adjunctive therapies based upon clinical trials involving adult patients with partial epilepsy. However, 2 of the AEDs reviewed here (rufinamide and vigabatrin) have FDA approval in the U.S. for specific Pediatric epilepsy syndromes, which are discussed below. The Pediatrician or Neurologists decision on the use of a new AED is an evolutionary process largely dependent on the patient characteristics, personal/peer experiences and literature about efficacy and safety profiles of these medications. Evidence based guidelines are limited due to a lack of randomized controlled trials involving pediatric patients for many of these new AEDs. [\hyperlink{Rifabutin}{PMID: 23650467}, Saima Kayani et al., 2012]

\hypertarget{pmid_34529779}{R}ifampicin doses of 40 mg/kg in adults are safe and well tolerated, may shorten anti-TB treatment and improve outcomes, but have not been evaluated in children. To characterize the pharmacokinetics and safety of high rifampicin doses in children with drug-susceptible TB. The Opti-Rif trial enrolled dosing cohorts of 20 children aged 0-12 years, with incremental dose escalation with each subsequent cohort, until achievement of target exposures or safety concerns. Cohort 1 opened with a rifampicin dose of 15 mg/kg for 14 days, with a single higher dose (35 mg/kg) on day 15. Pharmacokinetic data from days 14 and 15 were analysed using population modelling and safety data reviewed. Incrementally increased rifampicin doses for the next cohort (days 1-14 and day 15) were simulated from the updated model, up to the dose expected to achieve the target exposure [235 mg/L·h, the geometric mean area under the concentration-time curve from 0 to 24 h (AUC0-24) among adults receiving a 35 mg/kg dose]. Sixty-two children were enrolled in three cohorts. The median age overall was 2.1 years (range = 0.4-11.7). Evaluated doses were ∼35 mg/kg (days 1-14) and ∼50 mg/kg (day 15) for cohort 2 and ∼60 mg/kg (days 1-14) and ∼75 mg/kg (day 15) for cohort 3. Approximately half of participants had an adverse event related to study rifampicin; none was grade 3 or higher. A 65-70 mg/kg rifampicin dose was needed in children to reach the target exposure. High rifampicin doses in children achieved target exposures and the doses evaluated were safe over 2 weeks. [\hyperlink{Rifabutin}{PMID: 34529779}, Anthony J Garcia-Prats et al., 2021]

\hypertarget{pmid_1669249}{S}hunt infections in children have become a serious problem. In order to solve this, we have been using antibiotic therapy with Rifampicin (Rifampin) for the last 2 years; the dosage is 20 mg/kg per day 1 h before surgery and then for 48 h after the surgical procedure. We have had experience with 203 children operated on between January 1987 and December 1988. The result was a significant decrease in the number of children with shunt infections. In 1980 we reported an incidence of 10\%, while by 1988 the rate had gone down to 1\%. [\hyperlink{Rifabutin}{PMID: 1669249}, J C Viano et al., 1990]

\hypertarget{pmid_7719114}{R}ifabutin is a wide spectrum antibiotic particularly active on atypical and rifampicin-resistant mycobacteria. Rifabutin is more potent than rifampicin on Mycobacterium tuberculosis in vitro. Its mode of action is characterized by a high intracellular penetration in treated individuals. Clinical trials have proven the therapeutic value of rifabutin especially in AIDS patients with concomitant MAC. The preclinical safety evaluation of this compound included single and repeated dose toxicity studies of up to one year in rodents and non-rodents, reproduction and carcinogenicity studies and mutagenicity tests. During toxicological studies the most significant finding after repeated administration of rifabutin was the presence of multinucleated hepatocytes (MNH) in rats. This is a species specific finding which did not affect the life span of the hepatocytes. As shown in carcinogenicity studies, there was no tendency to further proliferative changes. Another specific histological feature among the species studied was the presence of a lipofuscin-like brown pigment, which was seen in many organs. This is a common finding with amphipilic compounds, such as rifabutin, which bind lipids and proteins, forming membrane-bound complexes. Even in carcinogenicity studies this change did not constitute a stimulus to cell proliferation and did not cause any secondary changes. In rodents, there was a mild hemolytic anemia at doses higher than 10 mg/kg/day. At doses ranging from 160-200 mg/kg/day rifabutin inhibited the functions of the male gonads in rats. This effect was reflected in a reduction of implantations observed in the fertility studies. Doses of 40 mg/kg/day did not induce any embryotoxic effects or changes in reproductive performance.(ABSTRACT TRUNCATED AT 250 WORDS) [\hyperlink{Rifabutin}{PMID: 7719114}, M Brughera et al., 1995] In a phase 3, randomized clinical trial (PREVENT TB) of 8053 people with latent tuberculosis infection, 12 once-weekly doses of rifapentine and isoniazid had good efficacy and tolerability. Children received higher rifapentine milligram per kilogram doses than adults. In the present pharmacokinetic study (a component of the PREVENT TB trial), rifapentine exposure was compared between children and adults. Rifapentine doses in children ranged from 300 to 900 mg, and adults received 900 mg. Children who could not swallow tablets received crushed tablets. Sparse pharmacokinetic sampling was performed with 1 rifapentine concentration at 24 hours after drug administration (C24). Rifapentine area under concentration-time curve (AUC) was estimated from a nonlinear, mixed effects regression model (NLME). There were 80 children (age: median, 4.5 years; range, 2-11 years) and 77 adults (age: median, 40 years; all ≥18 years) in the study. The geometric mean rifapentine milligram per kilogram dose was greater in children than in adults (children, 23 mg/kg; adults, 11 mg/kg). Rifapentine geometric mean AUC and C24 were 1.3-fold greater in children (all children combined) than in adults. Children who swallowed whole tablets had 1.3-fold higher geometric mean AUC than children who received crushed tablets, and children who swallowed whole tablets had a 1.6-fold higher geometric mean AUC than adults. The higher rifapentine doses in children were well tolerated. To obtain rifapentine exposures comparable in children to adults, dosing algorithms modeled by NLME were developed. A 2-fold greater rifapentine dose for all children resulted in a 1.3-fold higher AUC compared to adults administered a standard dose. Use of higher weight-adjusted rifapentine doses for young children are warranted to achieve systemic exposures that are associated with successful treatment of latent tuberculosis infection in adults. [\hyperlink{Rifabutin}{PMID: 7719114}, Marc Weiner et al., 2014]

\hypertarget{pmid_29629048}{R}iboflavin may have an acceptable effect on migraine among children. This study was carried out to determine the prophylactic effect of riboflavin on migraine in children. This randomized clinical trial study was performed at Shahid Beheshti Hospital in Kashan, Iran from December 2012 to February 2015. Ninety children with migraine were allocated randomly into 3 groups (placebo, low-dose and high-dose riboflavin). The outcomes (frequency, intensity and duration of headaches) were measured at baseline and 12 weeks of medication in each group, and the decrease of them were compared. SPSS software version 16 was used for analysis of the data. Descriptive statistics, Chi-square, Fisher's exact and t-test were used for statistical analyses. There was a significant decrease of migraine frequency (p=0.000) and mean duration (p=0.000) in the high-dose group compared with the placebo group. No significant reduction of frequency and mean duration of attacks were reported in the low-dose group compared to the placebo group (p=0.49 and p=0.69 respectively). There was no significant reduction of migraine intensity in the low-dose and high-dose groups compared to the placebo group (p=0.71 and p=0.74 respectively). High-dose riboflavin is a safe, well tolerated, cost-effective method of prophylaxis for children with migraine. The trial was registered at the Iranian Clinical Trial Registry with number IRCT2013020412361N1. The study was supported by the Deputy of Research, Kashan University of Medical Sciences (grant number 91073). [\hyperlink{Rifabutin}{PMID: 29629048}, Ahmad Talebian et al., 2018]

\hypertarget{pmid_16282948}{U}veitis associated with rifabutin treatment is a well-known adverse effect in adults with human immunodeficiency virus infection. In children, however, uveitis related to rifabutin has been reported in only a few cases. We present a case of rifabutin-associated uveitis in a human immunodeficiency virus-negative child. We review the literature and discuss special precautions to be considered when children are treated with rifabutin. [\hyperlink{Rifabutin}{PMID: 16282948}, Hanne Hyldahl Olesen et al., 2005]

\hypertarget{pmid_21982407}{G}abapentin (GAB) is a newer second-line antiepileptic drug (AED) used in children. This is a multi-centre retrospective observational study of the efficacy, tolerability and retention rate in 105 children, aged 0-17.5 years (mean 10.1) over a 14 year period. The median age of epilepsy onset was 2.5 years (range 0-14.6). 72\% started GAB as at least the 3rd AED, with 43\% having been withdrawn from at least 2 AEDs. 77\% had focal and 52\% symptomatic epilepsies. The maintenance doses for GAB ranged 6.0-87.3 mg/kg/day (mean 43.7). The study comprised 157 person-treatment years for GAB. GAB was well tolerated with 55\% remaining on treatment beyond 1 year. No serious adverse events were reported whilst on GAB, but 39\% reported possibly and probably related adverse events. Seizure improvement (<50\% seizure frequency compared to baseline) at more than 12 months of treatment, was reported in 35\% of patients starting GAB, including 6\% who remained seizure free. The results demonstrated the efficacy and tolerability of GAB in children with difficult to treat epilepsies, and a good response to treatment beyond 12 months, in both focal and generalised epilepsies. [\hyperlink{Rifabutin}{PMID: 21982407}, J K A Mills et al., 2012]

\hypertarget{pmid_25754598}{T}his review evaluates the recent progress in clinical trials on oral triptans for acute migraine in children and adolescents. Randomized controlled trials (RCT) on the treatment of migraine in pediatric patients were rare and difficult to design. In particular, high placebo response in many of the trials made it difficult to prove efficacy of triptans. Using a "novel study design" for RCT, a study successfully proved the efficacy of an oral rizatriptan. This trial enrolled patients with unsatisfactory response to nonsteroidal anti-inflammatory or acetaminophen and with migraine lasting longer than 3 h. Rizatriptan was approved by Food and Drug Administration (FDA) (USA) for children and adolescents of 6-17 years. The triptan-NSAID combination drug for pediatric patients also showed efficacy. [\hyperlink{Rifabutin}{PMID: 25754598}, Fumihiko Sakai et al., 2015]

\hypertarget{pmid_25645999}{R}ituximab is considered to be a promising drug for treating childhood refractory nephrotic syndrome. However, the efficacy and safety of rituximab in treating childhood refractory nephrotic syndrome remain inconclusive. This meta-analysis aimed to investigate the efficacy and safety of rituximab treatment compared with other immunosuppressive agents in children with refractory nephrotic syndrome. Three randomized controlled trials and two comparative control studies were included in our analysis. The included studies were of moderately high quality. Compared with other immunotherapies, rituximab therapy significantly improved relapse-free survival (hazard ratio = 0.49, 95\% confidence interval [CI], 0.26-0.92, P = 0.03). Rituximab also achieved a higher rate of complete remission (risk ratio,1.62; 95\% CI, 0.92 to 2.84, P = 0.09) and reduced the occurrence of proteinuria (mean difference = -0.25, 95\% CI = -0.29 to -0.21, P < 0.00001); however, a more targeted rituximab treatment did not significantly increase serum albumin levels and did not significantly reduce adverse events. Rituximab might be a promising treatment for childhood refractory nephrotic syndrome; however, the long-term effects and cost-effectiveness of rituximab treatment were not fully assessed, and there were limited studies that evaluated the clinical benefits of a concurrent infusion of rituximab plus a steroid compared with an infusion of rituximab only. Additional studies are required to address these issues.  [\hyperlink{Rifabutin}{PMID: 25645999}, Zhihong Zhao et al., 2015] Studies on the efficacy and tolerability of rufinamide in infants and young children are scarce. Here we report on an open, retrospective, and pragmatic study about safety and efficacy of rufinamide in children aged less than four years, in terms of seizures types and epilepsy syndromes. Forty children (mean age 39.5 months; range 22-48) were enrolled in the study. The mean follow-up period was 12.2 months (range 5-21). Rufinamide was initiated at a mean age of 26.7 months (range 12-42). Final rufinamide mean dosage was 31.5 mg/kg/day if associated with valproic acid and 44.2 mg/kg/day if not. The highest seizure reduction rate was observed in the epileptic spasms (46\%) and drop attacks (42\%) groups. Seizure reduction was also observed in tonic seizures (35\%) and in the focal seizure (30\%) groups. In terms of epilepsy syndrome, rufinamide was effective in Lennox-Gastaut syndrome. Results were very poor for those affected by Dravet's syndrome. Globally, responder rate was 27.5\%, including two (5\%) patients seizure-free. Adverse reactions occurred in 37.5\% of children and were mainly represented by vomiting, drowsiness, irritability, and anorexia. Discontinuation rate due to treatment-emergent adverse events was 15\%. The present study concludes that rufinamide may be a safe and effective drug for a broad range of seizures and epilepsy syndromes in infants and young children and represents a valid therapeutic option in this population. [\hyperlink{Rifabutin}{PMID: 25645999}, Salvatore Grosso et al., 2014]

\hypertarget{pmid_30910891}{R}ifampin is active against methicillin-resistant staphylococcal species and tuberculosis (TB). We performed a multicenter, prospective pharmacokinetic (PK) and safety study of intravenous rifampin in infants of <121 days postnatal age (PNA). We enrolled 27 infants; the median (range) gestational age was 26 weeks (23 to 41 weeks), and the median PNA was 10 days (0 to 84 days). We collected 102 plasma PK samples from 22 of the infants and analyzed safety data from all 27 infants. We analyzed the data using a population PK approach. Rifampin PK was best characterized by a one-compartment model; drug clearance increased with increasing size (body weight) and maturation (PNA). There were no adverse events related to rifampin. Simulated weight and PNA-based intravenous dosing regimens administered once daily (<14 days PNA, 8 mg/kg; ≥14 days PNA, 15 mg/kg) in infants resulted in comparable exposures to adults receiving therapeutic doses of rifampin against staphylococcal infections and TB. (This study has been registered at ClinicalTrials.gov under identifier NCT01728363.). [\hyperlink{Rifabutin}{PMID: 30910891}, P Brian Smith et al., 2019]

\hypertarget{pmid_21907886}{I}n a neurofibromatosis type 1 murine model, treatment with lovastatin reversed cognitive disabilities. We report on a phase I study examining the safety and tolerability of lovastatin in children with neurofibromatosis type 1. Twenty-four children with neurofibromatosis type 1 underwent a dose-escalation protocol for 3 months to identify the maximum tolerated dose and potential toxicity. Minimal side effects were evident, and no child experienced dose-limiting toxicity. Cognitive evaluations were completed before and after treatment, and the results suggested improvement in areas of verbal and nonverbal memory. Additional analyses, using reliable change indices, indicated improvements exceeding those of test-retest or practice effects in some participants. These observations may be analogous to the improvements observed in a neurofibromatosis type 1 murine model treated with lovastatin, although further study and replication are required. The safety and preliminary cognitive results support the need for a larger phase II trial in this population. [\hyperlink{Rifabutin}{PMID: 21907886}, Maria T Acosta et al., 2011]

\hypertarget{pmid_24910743}{A}pproximately one-third of all children with epilepsy do not achieve complete seizure improvement. This study evaluated the efficacy of Vigabatrin in children with intractable epilepsy. From November 2011 to October 2012, 73 children with refractory epilepsy (failure of seizure control with the use of two or more anticonvulsant drugs) who were referred to the Children's Medical Center and Mofid Children's Hospital were included in the study. The patients were treated with Vigabatrin in addition to their previous medication, and followed-up after three to four weeks to determine the daily frequency, severity, and duration of seizures in addition to any reported side effects. Of the 67 children, 41 (61.2\%) were males and 26 (38.8\%) females, their age ranging from three months to 13 years with an average of 3.1 [standard deviation (SD), 2.6] years. The mean daily frequency of seizures at baseline was 6.61 (SD, 5.9) seizures per day. Vigabatrin reduced the seizure frequency ≤2.9 (SD, 5.2) (56\% decline) and 3.0 (SD, 5.3) (54.5\% decline) per day after three and six months of treatment, respectively. A significant difference was observed between seizure frequencies at three (P<0.001) and six months (P<0.001) after Vigabatrin initiation compared with the baseline. Somnolence [3 (4.5\%)], horse laugh [1 (1.5\%)], urinary stones [1 (1.5\%)], increased appetite [1 (1.5\%)], and abnormal electroretinographic pattern [3 (4.5\%)] were the most common side effects in our patients. This study confirms the short-term efficacy and safety of Vigabatrin in children with refractory epilepsies. [\hyperlink{Rifabutin}{PMID: 24910743}, Mohammad-Mahdi Taghdiri et al., 2013]

\hypertarget{pmid_23724970}{T}he incidence of scabies is increasing in Europe, and it often affects children and infants. Although numerous topical treatments have been approved for treatment of scabies in adults, they are often poorly tolerated in infants. One treatment, ivermectin, remains off label for infants weighing < 15 kg. To report our experience on the safety and efficacy of oral ivermectin in refractory scabies in infants. A retrospective study was performed in the dermatology and paediatrics departments of Rouen University Hospital between January 2009 and October 2012. Infants diagnosed with scabies were identified, and the data for those fulfilling the inclusion criteria were analysed. Of 219 infants identified, 15 had received oral ivermectin and had been followed up for at least 3 months. All 15 patients were given two doses of ivermectin, 200 μg kg(-1), at baseline and 14 days later. Of 14 patients contacted 1 month after treatment, 12 had achieved healing. The other two were treated with ivermectin or benzyl benzoate; both healed. Overall, 3 months after the first ivermectin treatment, 13/14 patients had healed and only one had active disease. Ivermectin is generally well tolerated in infants. The 80\% rate of healing observed in infants who had failed to respond to at least two other topical treatments suggests that ivermectin could be considered for treatment of infants with recalcitrant or relapsing scabies. [\hyperlink{Rifabutin}{PMID: 23724970}, C Bécourt et al., 2013]

\hypertarget{pmid_12943481}{I}n the US, 6\% sulfur in petrolatum has been the most frequently administered treatment for infantile scabies. It appears to be safe but there is no literature containing a large series of patients on which to base that determination. In the UK, benzyl benzoate is the approved product. Benzyl benzoate is rarely used in the US at the present time. 5\% Permethrin is an excellent substitute and has many advantages. It appears to be quite safe in infants, although it is more expensive than other products. It remains present on the skin for several days, therefore protecting against reinfestation. Ivermectin is a systemic drug which is assumed to be safe in infants, although it requires repeated doses and does not protect against reinfestation. In the opinion of the author, 5\% permethrin is the best treatment for scabies in infants and young children. [\hyperlink{Rifabutin}{PMID: 12943481}, Mervyn L Elgart et al., 2003]

\hypertarget{pmid_26267219}{R}ecommendations in current guidelines for the treatment of chronic spontaneous urticaria (CSU) in infants and children are mostly based on extrapolation of data obtained in adults. This study reports the efficacy and safety of rupatadine, a modern H1 and PAF antagonist recently authorized in Europe for children with allergic rhinitis and CSU. A double-blind, randomized, parallel-group, multicentre, placebo-controlled compared study to desloratadine was carried out in children aged 2-11 years with CSU, with or without angio-oedema. Patients received either rupatadine (1 mg/ml), or desloratadine (0.5 mg/ml) or placebo once daily over 6 weeks. A modified 7-day cumulative Urticaria Activity Score (UAS7) was employed as the primary end-point. The absolute change of UAS7 at 42 days showed statistically significant differences between active treatments vs. placebo (-5.5 ± 7.5 placebo, -11.8 ± 8.7 rupatadine and -10.6 ± 9.6 desloratadine; p < 0.001) and without differences between antihistamines compounds. There was a 55.8\% decrease for rupatadine followed by desloratadine (-48.4\%) and placebo (-30.3\%). Rupatadine but not desloratadine was statistically superior to placebo in reduction of pruritus (-57\%). Active treatments also showed a statistically better improvement in children's quality of life compared to placebo. Adverse events were uncommon and non-serious in both active groups. Rupatadine is effective and well tolerated in the relief of urticaria symptoms, improving quality of life over 6 weeks in children with CSU. This is the first study using a modified UAS to assess severity and efficacy outcome in CSU in children. [\hyperlink{Rifabutin}{PMID: 26267219}, Paul Potter et al., 2016]

\hypertarget{pmid_3534260}{A}uranofin (AF, Ridaura) was administered to 23 children with juvenile rheumatoid arthritis during a prospective, open labelled, noncontrolled trial designed to establish longterm safety and preliminary efficacy. Dosages of AF were up to 0.2 mg/kg/day, with either aspirin (60-80 mg/kg/day), naproxen (400-600 mg/m2/day), or tolmetin sodium (20-40 mg/kg/day) serving as the concurrent nonsteroidal antiinflammatory drug. Nearly all patients showed an initial favorable response, however tachyphylaxis occurred in one-third (mean duration of therapy prior to the development of inefficacy = 22.6 mo). Clinical remission was observed in 6 patients an average of 15 months after enrollment. The drug appears to be safe for extended periods; 7 children are continuing AF at the present time with a mean duration of therapy of 4.25 years (maximum followup = 4.6 years). [\hyperlink{Rifabutin}{PMID: 3534260}, E H Giannini et al., 1986]

\hypertarget{pmid_10721313}{T}he aim of this study was to review retrospectively the safety and efficacy of a paediatric sedation protocol in a district general hospital radiology department. 256 children attended for CT scanning over a 40-month period. 40 children required sedation and were given quinalbarbitone. 34 (85\%) of this group were adequately sedated. Of the children who received quinalbarbitone, 35 were under 5 years of age. 32 of this group (91.4\%) were adequately sedated. Failures in children under 5 years were all caused by problems with administration whilst failures in the older children were due to paradoxical excitement. No problems with respiratory depression were encountered. Sedation can be safely performed in a district general hospital radiology department if a structured protocol is adhered to. Quinalbarbitone is a safe, effective oral agent in children under the age of 5 years. [\hyperlink{Rifabutin}{PMID: 10721313}, J H Simpson et al., 2000]

\hypertarget{pmid_10604603}{T}o evaluate the efficacy and safety of gabapentin (GBP) in partial epilepsy in children. We performed a prospective open label add-on study in 52 children and adolescents (age 1.8-17.5 years, mean 11.1 years) with refractory partial seizures. Gabapentin was added to one other baseline drug and the efficacy was rated according to seizure type and frequency. The GBP dose ranged from 26 to 78 mg/kg per day (mean 52 mg/kg per day) and was well tolerated in most patients. The seizure frequency remained unchanged in 34 patients (65\%). We saw a provocation of seizures in three children (6\%). Initially 15 patients (29\%) benefited from GBP: five (10\%) with a seizure reduction of 50-74\%, seven (13\%) with a reduction of 75-99\% and three (6\%) became seizure free. All but three experienced a development of tolerance within the next weeks to months. Although gabapentin seems also to be safe in children, the efficacy in refractory partial seizures was disappointing. [\hyperlink{Rifabutin}{PMID: 10604603}, E Korn-Merker et al., 2000]

\hypertarget{pmid_33231110}{R}ufinamide, a triazole derivative, is a new-generation antiseizure medication with a novel mechanism of action. We evaluated the efficacy and safety of rufinamide treatment in children with epilepsy younger than 4 years at our center. In this retrospective study, we included children younger than 4 years who had pharmacologically resistant epilepsy and were treated with rufinamide at Boston Children's Hospital between June 2010 and June 2018. Safety and efficacy of rufinamide treatment were assessed immediately prior to initiation of rufinamide and at the last follow-up visit. Responders were defined as patients who had greater than 50\% reduction in seizure frequency on follow-up as compared to baseline. We reviewed records of 128 children and included 103 with complete information. Patients consisted of 60 boys (58\%), with a median age of 20 months (interquartile range 13-28, range 2-36). Median treatment duration was 15 months, and median rufinamide dosage at the last follow-up was 42 mg/kg/d (interquartile range 34-56). At the last follow-up, seizure frequency decreased (450 vs 90,  Rufinamide is effective in reducing seizure frequency in pediatric epilepsy patients younger than 4 years, and overall well tolerated. [\hyperlink{Rifabutin}{PMID: 33231110}, Ahmet Tanritanir et al., 2021]

\section*{Ixazomib Citrate}
\subsection*{Result}
\subsubsection*{Answer}

Unknown

\subsubsection*{{Explanation}}
\hypertarget{Ixazomib Citrate}
After reviewing all available abstracts, there is no evidence of a targeted safety study of Ixazomib Citrate in children. None of the abstracts discuss Ixazomib Citrate, its use, or its safety profile in pediatric populations of any age range. Therefore, based on the abstracts provided, the safety of Ixazomib Citrate in children is unknown. No data is available to affirm or refute its safety in any pediatric age group.

\subsection*{Abstracts}
\hypertarget{pmid_27478130}{C}hildhood-onset inflammatory bowel disease (IBD) is characterized by extensive intestinal involvement and rapid early progression. Infliximab (IFX), cyclosporin (CYA), and tacrolimus (FK506) are increasingly used to treat pediatric IBD; however, their long-term effects and adverse events have not been properly investigated in pediatric patients. The aim of this study was to characterize the effects of these biologics and immunomodulators on pediatric IBD patients in Japan. Additionally, we assessed IFX use in pediatric patients with Crohn's disease (CD). A national survey of IFX, adalimumab, CYA, and FK506 use in pediatric IBD patients (< 17 years of age) was sent to 683 facilities in Japan from December 2012 to March 2013. Secondary questionnaires were sent to pediatric and adult practitioners with the aim of assessing the effectiveness and safety of IFX for pediatric CD patients. The response rate for the primary survey was 61.2\% (N  =  418). Among 871 pediatric CD patients, 284 (31.5\%), 24, 4, and 15 received IFX (31.5\%), adalimumab, CYA, and FK506, respectively, from 2000 to 2012. According to the secondary survey, extensive colitis (L3, Paris classification) was diagnosed in 69.4\% of pediatric CD patients who received IFX. Regarding the effectiveness of IFX in this population, 54.7\% (99/181) of patients were in remission, and 42.0\% (76/181) were on maintenance therapy. However, 32.0\% (58/181) of patients experienced adverse events, and one patient died of septic shock. Infliximab is reasonably safe and effective in pediatric CD patients and should therefore be administered in refractory cases. [\hyperlink{Ixazomib Citrate}{PMID: 27478130}, Kenji Hosoi et al., 2017]

\hypertarget{pmid_31348595}{E}micizumab is a bispecific antibody that bridges factor IXa and factor X to restore hemostasis in patients with hemophilia A (HA). Its efficacy and safety have been proven in multicenter trials. However, real world data regarding its use in very young children are currently lacking. Ancillary test results for monitoring emicizumab's hemostatic effect and their clinical correlations are scarce. Children with HA and inhibitors treated by emicizumab were prospectively followed at our center. Laboratory follow-up included rotational thromboelastometry (ROTEM) and thrombin generation (TG), prior to and during treatment. Eleven children whose median age was 26 months were treated by emicizumab and followed for a median of 36 weeks. During follow-up, none experienced hemarthrosis or any other spontaneous bleeds. For 7/11 patients, emicizumab prophylaxis was sufficient to maintain hemostasis without additional supplemental therapy. Only 4/11 patients were occasionally treated with recombinant activated FVII for trauma. Two minor surgeries were safely performed without supplemental therapy while another procedure was complicated by major bleeding. TG parameters improved for all patients, correlating with their clinical status. Interestingly, the lowest TG values were obtained for patients experiencing bleeding episodes, while ROTEM parameters in all patients were close to the normal range. This study confirms the safety and efficacy of emicizumab in reducing bleeds in young children with HA with inhibitors, including infants. However, surgeries warrant caution as emicizumab prophylaxis may not be sufficient for some procedures. TG may more accurately reflect the hemostasis state than ROTEM in pediatric patients treated with emicizumab. [\hyperlink{Ixazomib Citrate}{PMID: 31348595}, Assaf A Barg et al., 2019]

\hypertarget{pmid_25047312}{S}edation in children remains a controversial issue in emergency departments (ED). Midazolam, as a benzodiazepine is widely used for procedural sedation among paediatrics. We compared the effectiveness and safety of two forms of midazolam prescription; intramuscular (IM) and intravenous (IV). A cohort study was conducted on two matched groups of 30 children referred to our ED between 2010 and 2011. The first group received IM midazolam (0.3 mg/kg) and the second group received IV midazolam (0.15 mg/kg) for sedation. For evaluating effectiveness, sedation, irritation and cooperation score were followed every 15 min for 60 min and for safety assessment, vital signs and O 2 saturation were observed. Mean age was 6.18 ± 2.88 years and 31 patients (51.7\%) were male. All patients were sedated completely after using first dose. There was an overall complication rate of 68.3\%. 35 (58.3\%) patients presented euphoria as the most common complication, but there was no statistical difference between the two groups (P = 0.396). Cases who received IV midazolam became sedated faster than those received IM midazolam (P > 0.001). The vital signs including heart rate, respiratory rate, systolic blood pressure and O 2 saturation changed significantly between and within groups during the sedation (P < 0.05). Both forms of midazolam, IM and IV, are effective and safe for paediatric sedation in ED. Although the sedative with IV form might appear sooner, IM form of midazolam can be effectively used in patient with limited IV access. Patients are better to observe closely for psychological side-effects. [\hyperlink{Ixazomib Citrate}{PMID: 25047312}, Mohammad-Reza Ghane et al., ]

\hypertarget{pmid_29077183}{I}t is common practice to prepare the nasal mucosa with decongestant in children undergoing lacrimal surgery. Xylometazoline 0.05\% (Otrivine) nasal spray is commonly used. It has been reported to cause cardiovascular side effects. In the absence of formal guidelines on the safety of the use of nasal decongestants in children, we reviewed our practice to answer the question: How safe is preoperative use of xylometazoline in children undergoing lacrimal surgery? To our knowledge, this is the first study to address the potential side effects of the use of xylometazoline preoperatively in children undergoing lacrimal surgery. This was a retrospective analysis of medical notes of children undergoing lacrimal surgery with the use of preoperative intranasal xylometazoline 0.05\% over a 5-year period. Twenty-nine children, age 1-6 years (mean 3 years), underwent lacrimal surgery under general anesthesia with preoperative use of intranasal xylometazoline. Topical intranasal 1:10,000 adrenaline was used during surgery in all patients. All children were found to have uneventful surgery and recovery from anesthesia. Xylometazoline 0.05\% intranasal use for prelacrimal surgery was found to be effective and safe. Addition of sympathomimetic topical adrenaline (1:10,000) did not impose any risks. The type of general anesthesia may influence the cardiovascular side effects anecdotally recorded during xylometazoline use. [\hyperlink{Ixazomib Citrate}{PMID: 29077183}, Varajini Joganathan et al., 2018]

\hypertarget{pmid_33847760}{O}ral ivermectin can be used to treat scabies. Evidence for safe and effective use in young children in individual treatment situations has been developed and published. In order to also ensure a body weight-adapted dosage for children, an ivermectin-containing syrup was developed as an extemporaneous preparation. Since ivermectin is not available as a pure substance for the formulation, tablets containing active ingredient were used as a basic material for development. The formulation was designed according to pharmaceutical, regulatory and use-oriented criteria. An HPLC (high-pressure liquid chromatography) method was developed and validated to demonstrate chemical stability. In order to facilitate the practical implementation, information on suitable packaging material and application aids was also developed and the formulation was evaluated. It has been demonstrated that the final formulation produced in the pharmacy was stable and can be stored for 3 weeks. No concerns were raised regarding the tolerability of the syrup formulation. The physicochemical properties and the taste of the formulation allow the intended use as a well-dosed syrup for children. The developed formulation meets the requirements of the Apothekenbetriebsordnung (Pharmacy Work Rules; Section 7 ApBetrO) and enables an exact, body weight-adapted dosage of oral ivermectin in young children. Studies on human pharmacokinetics or clinical studies to demonstrate tolerability and/or efficacy are not available for the formulation. [\hyperlink{Ixazomib Citrate}{PMID: 33847760}, Johannes Wohlrab et al., 2021]

\hypertarget{pmid_23024102}{W}e conducted this single blind randomized clinical trial to compare the efficacy and safety of oral chloral hydrate and intranasal midazolam for induction of sedation for computerized tomography scan of brain in children. Participants aged 1-10 years (n=60) were randomized to receive 100 mg/kg chloral hydrate orally with intra nasal normal saline OR intranasal midazolam 0.2 mg/kg with oral normal saline. Adequate sedation (Ramsay sedation score of four) was obtained and CT scan completed successfully in 76.7\% of chloral hydrate group and in 40\% of midazolam group (P=0.004). No significant difference was seen for side effects frequency between the two drugs (10\% in chloral hydrate, 3.3\% in midazolam group; P=0.34). We conclude that oral chloral hydrate can be considered as a safe and effective drug for sedation in children undergoing CT scan of brain. [\hyperlink{Ixazomib Citrate}{PMID: 23024102}, Razieh Fallah et al., 2013]

\hypertarget{pmid_8169182}{T}here is evidence for the efficacy and safety of clonazepam (CZP) in adult anxiety disorders, but no formal studies to substantiate clinical reports of similar benefit in children with anxiety disorders. In this double-blind pilot study, 15 children, aged 7 to 13 years, entered a randomly assigned, double-blind crossover trial of 4 weeks of CZP (up to 2 mg/day) and 4 weeks of placebo. Twelve children completed the trial. All but 1 had a diagnosis of separation anxiety disorder, and all but 2 had comorbid diagnoses. Nine children appeared to have moderate to significant clinical improvement, but statistical comparisons on several ratings failed to confirm a trend in favor of CZP. Side effects of drowsiness, irritability, and/or oppositional behavior were notable in 10 children in the CZP phase compared with 5 in the placebo phase. Clonazepam was believed to have clinical benefit for some children, but this was not confirmed statistically in this small sample. Problematic side effects of drowsiness and disinhibition were common and possibly were due to rapid titration. [\hyperlink{Ixazomib Citrate}{PMID: 8169182}, F Graae et al., ]

\hypertarget{pmid_19910033}{M}any children with asthma continue to experience symptoms despite available therapies. This study evaluated the efficacy and safety of omalizumab, a humanized anti-IgE mAb, in children with moderate-to-severe persistent allergic (IgE-mediated) asthma that was inadequately controlled despite treatment with medium-dose or high-dose inhaled corticosteroids (ICSs) with or without other controller medications. A randomized, double-blind, placebo-controlled trial enrolled children age 6 to <12 years with perennial allergen sensitivity and history of exacerbations and asthma symptoms despite at least medium-dose ICSs. Patients were randomized 2:1 to receive omalizumab (75-375 mg sc, q2 or q4 wk) or placebo over a period of 52 weeks (24-week fixed-steroid phase followed by a 28-week adjustable-steroid phase). A total of 627 patients (omalizumab, n = 421; placebo, n = 206) were randomized, with efficacy analyzed in 576 (omalizumab, n = 384; placebo, n = 192). Over the 24-week fixed-steroid phase, omalizumab reduced the rate of clinically significant asthma exacerbations (worsening symptoms requiring doubling of baseline ICS dose and/or systemic steroids) by 31\% versus placebo (0.45 vs 0.64; rate ratio, 0.69; P = .007). Over a period of 52 weeks, the exacerbation rate was reduced by 43\% versus placebo (P < .001). Omalizumab significantly reduced severe exacerbations. Over a period of 52 weeks, omalizumab had an acceptable safety profile, with no difference in overall incidence of adverse events compared with placebo. Add-on omalizumab is effective and well tolerated as maintenance therapy in children (6 to <12 years) with moderate-to-severe persistent allergic (IgE-mediated) asthma whose symptoms are inadequately controlled despite medium to high doses of ICSs. [\hyperlink{Ixazomib Citrate}{PMID: 19910033}, Bob Lanier et al., 2009]

\hypertarget{pmid_7717236}{M}idazolam is a relatively short-acting water-soluble benzodiazepine that provides anxiolysis and anterograde amnesia and can be given orally with few adverse effects. We evaluated the benefit and safety of oral midazolam for sedation of young children during voiding cystourethrography or nuclear cystography. For 3.5 years, a highly selected group of 98 children, ages 23 months to 9 years (mean, 4 years), were given oral midazolam 0.6 mg/kg 20-30 min before cystourethrography or nuclear cystography. These children either had been frightened by a previous catheterization (39\%) or seemed particularly frightened during an examination of their genitals in the office (61\%). A control group of 25 children, similar in age to the study group, did not receive midazolam before cystourethrography. Parents were interviewed to assess their child's recollection of the procedure. Voiding dynamics were assessed by evaluating the postvoiding radiograph. Of the midazolam-treated patients, 60\% had no recollection of the study, and 31\% remembered part or all of the study but did not have a negative experience. No significant change in vital signs or oxygen saturation was observed in any child. In the control group, 24 (96\%) of 25 children remembered the cystographic examination (p < .01). Behavioral side effects occurred in 12\% of the children receiving midazolam and consisted primarily of combative behavior as the medication was wearing off. Ninety-five percent of the parents indicated that they would want their child to have midazolam again if the cystography needed to be repeated. Of the children receiving midazolam, 76\% had little or no residual urine after voiding, compared with 72\% of the control group (no significant difference). In children who have been or are likely to be excessively frightened during cystourethrography or nuclear cystography, midazolam usually provides satisfactory amnesia and anxiolysis with few side effects or adverse impact on voiding dynamics. [\hyperlink{Ixazomib Citrate}{PMID: 7717236}, J S Elder et al., 1995]

\hypertarget{pmid_32538879}{G}iven the limited available evidence on chloral hydrate safety in neonatal populations and the discrepancy in chloral hydrate acceptance between the US and other countries, we sought to clarify the safety profile of chloral hydrate compared to other sedatives in hospitalized infants. We included all infants <120 days of life who underwent a minor procedure and were administered chloral hydrate, clonidine, clonazepam, dexmedetomidine, diazepam, ketamine, lorazepam, midazolam, propofol, or pentobarbital on the day of the procedure. We characterized the distribution of infant characteristics and evaluated the relationship between drug administration and any adverse event. We performed propensity score matching, regression adjustment (RA), and inverse probability weighting (IPW) to ensure comparison of similar infants and to account for confounding by indication and residual bias. Results were assessed for robustness to analytical technique by reanalyzing the main outcomes with multivariate logistic regression, a doubly robust IPW with RA model, and a doubly robust augmented IPW model with bias-correction. Of 650 infants, 497 (76\%) received chloral hydrate, 79 (12\%) received midazolam, 54 (8\%) received lorazepam, and 15 (2\%) received pentobarbital. Adverse events occurred in 41 (6\%) infants. Using propensity score matching, chloral hydrate was associated with a decreased risk of an adverse event compared to other sedatives, risk difference (95\% confidence interval) of -12.79 (-18.61, -6.98), p <  0.001. All other statistical methods resulted in similar findings. Administration of chloral hydrate to hospitalized infants undergoing minor procedures is associated with a lower risk for adverse events compared to other sedatives. [\hyperlink{Ixazomib Citrate}{PMID: 32538879}, S H Dallefeld et al., 2020]

\hypertarget{pmid_23904337}{I}n Sub-Saharan Africa, intrarectal diazepam is the first-line anticonvulsant mostly used in children. We aimed to assess this standard care against sublingual lorazepam, a medication potentially as effective and safe, but easier to administer. A randomized controlled trial was conducted in the pediatric emergency departments of 9 hospitals. A total of 436 children aged 5 months to 10 years with convulsions persisting for more than 5 minutes were assigned to receive intrarectal diazepam (0.5 mg/kg, n = 202) or sublingual lorazepam (0.1 mg/kg, n = 234). Sublingual lorazepam stopped seizures within 10 minutes of administration in 56\% of children compared with intrarectal diazepam in 79\% (P < .001). The probability of treatment failure is higher in case of sublingual lorazepam use (OR = 2.95, 95\% CI = 1.91-4.55). Sublingual lorazepam is less efficacious in stopping pediatric seizures than intrarectal diazepam, and intrarectal diazepam should thus be preferred as a first-line medication in this setting.  [\hyperlink{Ixazomib Citrate}{PMID: 23904337}, Célestin Kaputu Kalala Malu et al., 2014] Since Anti-IL-17s availability, concerns about their safety have been raised due to the inhibition of physiological activities that IL-17A plays in the immune response against infections. Ixekizumab is a humanized monoclonal antibody specifically targeting IL-17A approved for the treatment of moderate-to-severe psoriasis. The aim of this review is to evaluate the safety profile of ixekizumab in moderate-to-severe psoriasis patients. A compressive literature review has included articles since March 2022. In our analysis, most of the reported AEs were mild or moderate and rarely required treatment discontinuation. Among the class-specific AEs to consider during ixekizumab treatment, there are the risk of Candida infections and the risk of IBD, both of which were reported more frequently than placebo or other biologics (etanercept, ustekinumab, and guselkumab). However, the reported candidiasis resulted in mild-to-moderate and easily managed. The risk of IBD (both exacerbation and de novo diagnosis) represents a class effect of IL-17 inhibitors, which should be well evaluated before considering starting ixekizumab treatment. The most common AEs were represented by nasopharyngitis, upper respiratory tract infection, and injection-site reactions. The analyzed studies confirmed the favorable safety profile of ixekizumab even in more recently published studies. [\hyperlink{Ixazomib Citrate}{PMID: 23904337}, Angelo Ruggiero et al., 2022]

\hypertarget{pmid_36418946}{T}o evaluate the safety and effectiveness of different dosages of intranasal Dexmedetomidine (DEX) in combination with oral midazolam for sedation of young children during brain MRI examination. Included in this prospective single-blind randomized controlled trial were 156 children aged from 3 months to 6 years and weighing from 4 to 20 Kg with ASA I-II who underwent brain MRI examination between March 2021 and February 2022. Using the random number table method, they were divided into group A (using 3 ug/kg intranasal DEX plus 0.2 mg/Kg oral midazolam) and group B (using 2 ug/kg intranasal DEX plus 0.2 mg/Kg oral Midazolam). The one-time success rate of sedation, sedation onset time, recovery time, overall sedation time, and occurrence of adverse reactions during MRI examination were compared between the two groups. The heart rate (HR), mean arterial pressure (MAP), and percutaneous SpO The one-time success rate of sedation in group A and B was 88.31\% and 79.75\% respectively, showing no significant difference between the two groups (P>0.05). The sedation onset time in group A was 24.97±16.94 min versus 27.92±15.83 min in group B, and the recovery time was 61.88±22.18 min versus 61.16±28.16 min, both showing no significance difference between the two groups (P>0.05). Children in both groups exhibited good drug tolerance without presenting nausea and vomiting, hypoxia, or bradycardia and hypotension that needed clinical interventions. There was no significant difference in the occurrence of abnormal HR, MAP or other adverse reactions between the two groups (P>0.05). 3 ug/kg or 2 ug/kg intranasal DEX in combination with 0.2 mg/kg oral Midazolam both are safe and effective for sedation of children undergoing MRI examination with the advantages of fast-acting and easy application. It was registered at the Chinese Clinical Trial Registry ( ChiCTR1800015038 ) on 02/03/2018. [\hyperlink{Ixazomib Citrate}{PMID: 36418946}, Hongbin Gu et al., 2022]

\hypertarget{pmid_24829888}{P}rocedural sedation in children continues to be a problem in the emergency department (ED). Midazolam is the first water-soluble benzodiazepine and it has been widely used for procedural sedation in pediatric patients. The aim of this study was evaluation of clinical safety and effectiveness of intramuscular Midazolam for pediatric sedation in the ED setting. We performed a self-controlled clinical trial on 30 children who referred to the Baqiyatallah Hospital ED between 2009 and 2010. They received intramuscular Midazolam 0.3 mg/kg for procedural sedation and then they were followed for sedative effectiveness and safety. Vital signs and O2 saturation were also observed. The findings were compared using SPSS ver. 16 software. The mean age was 5.50 ± 2.70 years, the mean weight was 19.50 ± 6.63 kilograms and 16 patients (53.3\%) were females. The most common adverse effect was euphoria (66.66\%) and vertigo (6.7\%); 27.7\% did not show any side effects. There was an overall complication rate of 72.3\%. The vital signs including heart rate, respiratory rate, systolic and diastolic blood pressure and O2 saturation decreased significantly during sedation (P value < 0.05). Midazolam is an effective and relatively safe sedative for pediatric patients in the ED. The patient should be observed closely and monitored for psychological and hemodynamic side effects. [\hyperlink{Ixazomib Citrate}{PMID: 24829888}, Mohammad Reza Ghane et al., 2012]

\hypertarget{pmid_35989987}{A}lthough numerous intravenous sedative regimens have been documented, the ideal non-parenteral sedation regimen for magnetic resonance imaging (MRI) has not been determined. This prospective, interventional study aimed to investigate the efficacy and safety of buccal midazolam in combination with intranasal dexmedetomidine in children undergoing MRI. Children between 1 month and 10 years old requiring sedation for MRI examination were recruited to receive buccal midazolam 0.2 mg⋅kg Sedation with dexmedetomidine-midazolam was administered to 530 children. The successful sedation rate was 95.3\% (95\% confidence interval: 93.5-97.1\%) with the initial sedation regimens and 97.7\% (95\% confidence interval: 96.5-99\%) with a rescue dose of 2 μg⋅kg In MRI examinations, the addition of buccal midazolam to intranasal dexmedetomidine is associated with a high success rate and a good safety profile. This non-parenteral sedation regimen can be a feasible and convenient option for short-duration MRI in children between 1 month and 10 years. [\hyperlink{Ixazomib Citrate}{PMID: 35989987}, Bi Lian Li et al., 2022]

\hypertarget{pmid_33505889}{T}he long-term efficacy and safety of infliximab (IFX) in children with ulcerative colitis (UC) have not been well-evaluated. Here, we reviewed the long-term durability and safety of IFX in our single center pediatric cohort with UC. This retrospective study included 20 children with UC who were administered IFX. For induction, 5 mg/kg IFX was administered at weeks 0, 2, and 6, followed by every 8 weeks for maintenance. The dose and interval of IFX were adjusted depending on clinical decisions. Corticosteroid (CS)-free remission without dose escalation (DE) occurred in 30\% and 25\% of patients at weeks 30 and 54, respectively. Patients who achieved CS-free remission without DE at week 30 sustained long-term IFX treatment without colectomy. However, one-third of the patients discontinued IFX treatment because of a primary nonresponse, and one-third experienced secondary loss of response (sLOR). IFX durability was higher in patients administered IFX plus azathioprine for >6 months. Four of five patients with very early onset UC had a primary nonresponse. Infusion reactions (IRs) occurred in 10 patients, resulting in discontinuation of IFX in four of these patients. No severe opportunistic infections occurred, except in one patient who developed acute focal bacterial nephritis. Three patients developed psoriasis-like lesions. IFX is relatively safe and effective for children with UC. Clinical remission at week 30 was associated with long-term durability of colectomy-free IFX treatment. However, approximately two-thirds of the patients were unable to continue IFX therapy because of primary nonresponse, sLOR, IRs, and other side effects. [\hyperlink{Ixazomib Citrate}{PMID: 33505889}, Hirotaka Shimizu et al., 2021]

\hypertarget{pmid_30046708}{T}he incidence of venous thromboembolism (VTE) in children has been increasing. Anticoagulants are the mainstay of treatment but are associated with bleeding events that may be life-threatening. Idarucizumab is a fragment antigen-binding (fab) that provides immediate, complete, and sustained reversal of dabigatran's anticoagulant effects in adults. This phase III, open-label, single-arm, multicenter, multinational trial will assess the safety of idarucizumab in children participating in two ongoing trials investigating dabigatran etexilate. Eligible patients will be children with VTE (aged 0-≤18 years; n = \textasciitilde{}5) with life-threatening or uncontrolled bleeding (group A), and children who require emergency surgery/urgent procedures for a condition other than bleeding (group B). Patients will receive idarucizumab up to 5 g as two consecutive intravenous infusions over 5-10 minutes each, as two 10-15-minute drips or as two bolus injections (15 minutes apart) and will be monitored for 30 days. The primary endpoint will be the safety of idarucizumab assessed by the occurrence of drug-related adverse events (including immune reactions) and all-cause mortality. Secondary endpoints will be the reversal of dabigatran anticoagulant effects assessed by changes in diluted thrombin time and ecarin clotting time, time to achieve complete reversal and the duration of the reversal and bleeding severity (group A). The formation of antidrug antibodies at 30 days post-dose and cessation of bleeding will also be assessed. This study will report the safety of idarucizumab in children with VTE who require rapid reversal of the anticoagulant effects of dabigatran. Clinical trial registration: NCT02815670. [\hyperlink{Ixazomib Citrate}{PMID: 30046708}, Manuela Albisetti et al., 2018]

\hypertarget{pmid_20527137}{O}nly a few corticosteroids for topical use have proven safe and effective in pediatric populations down to 3 months of age. The authors report the results of a study designed to assess the efficacy and safety of hydrocortisone butyrate (HCB) 0.1\% in lipocream (LCr) vehicle in infants and children. A total of 264 boys and girls 3 months to less than 18 years old, with stable, mild to moderate atopic dermatitis affecting at least 10\% body surface area applied HCB 0.1\% in LCr or LCr alone twice daily for up to 1 month without occlusion. Primary end-points included: percent of patients who achieved treatment success based on physician global assessments. Secondary endpoint included: difference in pruritus and Eczema Area and Severity Index (EASI) at day 29. Treatment was significant (P < 0.001) for HCB 0.1\% LCr over vehicle. No serious nor significant adverse events were reported. Results are representative of a short duration treatment for a chronic disease. HCB 0.1\% in LCr is more effective than its vehicle in pediatric populations down to 3 months of age without significant adverse events when used twice a day for up to 1 month. [\hyperlink{Ixazomib Citrate}{PMID: 20527137}, William Abramovits et al., ]

\hypertarget{pmid_32602383}{T}o assess the efficacy and safety of omalizumab in children with moderate-to-severe asthma. We systematically searched MEDLINE, EMBASE, and Cochrane for randomized controlled trials (RCTs ) (inception to January 2020). All RCTs which were conducted in childhood and adolescence with asthma and compared the efficacy or safety of omalizumab were adopted. Three studies with four publications including 1380 pediatric patients met our criteria. For children with moderate-to-severe asthma, omalizumab decreased asthma exacerbations rate (OR 0.51, 95\% CI: 0.44-0.58,  These findings suggested that omalizumab had beneficial effects on moderate-to-severe asthma in children. Patients may benefit more from long-term use of omalizumab. In addition, omalizumab reduces the rate of serious adverse events requiring hospitalizations. [\hyperlink{Ixazomib Citrate}{PMID: 32602383}, Zhuo Fu et al., 2021]

\hypertarget{pmid_26199710}{A}mong different categories of sedative agents, benzodiazepines have been prescribed for more than three decades to patients of all ages. The effective and predictable sedative and amnestic effects of benzodiazepines support their use in pediatric patients. Midazolam is one of the most extensively used benzodiazepines in this age group. Oral form of drug is the best accepted route of administration in children. The purpose of this study was to compare the efficacy and safety of a commercially midazolam syrup versus orally administered IV midazolam in uncooperative dental patients. Second objective was to determine whether differences concerning sedation success can be explained by child's behavioral problems and dental fear. Eighty eight uncooperative dental patients (Frankl Scales 1,2) aged 3 to 6 years, and ASA I participated in this double blind, parallel randomized, controlled clinical trial. Midazolam was administered in a dose of 0.5 mg/kg for children under the age 5 and 0.2 mg/kg in patients over 5 years of age. Physiologic parameters including heart rate, respiratory rate, oxygen saturation and blood pressure were recorded. Behavior assessment was conducted throughout the course of treatment using Houpt Sedation Rating Scale and at critical moments of treatment (injection and cavity preparation) by North Carolina Scale. Dental fear and behavioral problems were evaluated using Child Fear Schedule Survey-Dental Subscale (CFSS-DS), and Strength and Difficulties Questionnaire (SDQ). Independent t-test, Chi-Square, and Pearson correlation were used for statistical analysis. Acceptable overall sedation ratings were observed in 90\% and 86\% of syrup and IV/Oral group respectively; Chi-Square P = 0.5. Other domains of Houpt Scale including: sleep, crying and movement were also not significantly different between groups. Physiological parameters remained in normal limits during study without significant difference between groups. "Orally administered IV midazolam" preparation can be used as an alternative for commercially midazolam syrup. [\hyperlink{Ixazomib Citrate}{PMID: 26199710}, Katayoun Salem et al., 2015]

\hypertarget{pmid_30114224}{T}he prevalence of pediatric Crohn's disease (CD) is increasing in Japan and other countries, and many patients are unresponsive to or do not tolerate current treatment options. This study aimed to investigate the efficacy, safety, and pharmacokinetic profile of infliximab (IFX) in pediatric patients with moderate-to-severe CD and inadequate response to existing treatment. This was an open-label, uncontrolled, multicenter Phase 3 study conducted at nine sites in Japan between April 2012 and March 2015. Pediatric patients (aged 6-17 years) with moderate-to-severe CD were treated with IFX 5 mg/kg at Weeks 0, 2, and 6, and at 8-week intervals thereafter until Week 46, with final evaluation at Week 54. IFX dose was increased to 10 mg/kg in patients who showed loss of response to IFX from Week 14 onwards. A total of 14 patients fulfilled eligibility criteria and were treated. Dose-escalation criteria were met by five patients who then received 10 mg/kg IFX. The remaining nine patients continued to receive an IFX dose of 5 mg/kg. IFX rapidly improved clinical symptoms and its effect was maintained for up to 54 weeks. Overall Pediatric Crohn's Disease Activity Index (PCDAI) response rate was 85.7\%, and overall PCDAI remission rate was 64.3\%. Three out of five patients who increased IFX dose regained PCDAI remission by retrieval of serum IFX concentration. Adverse events and serious adverse events occurred in 100.0\% and 14.3\% of patients, respectively. There was no substantial difference in the safety profiles of patients taking a constant dose of 5 mg/kg and those taking an increased dose of 10 mg/kg. These findings support the effective use of IFX in the treatment of pediatric patients with CD where other treatments have proven ineffective. [\hyperlink{Ixazomib Citrate}{PMID: 30114224}, Hitoshi Tajiri et al., 2018]

\hypertarget{pmid_10700542}{R}ectal diazepam is widely used in the treatment of acute seizures in children but has some disadvantages. Nasal/sublingual midazolam administration has been recently investigated for this purpose but never at home or in a general paediatric hospital. The aim of this open study was to determine the efficacy, the tolerance and the applicability of nasal midazolam during acute seizures in children both in hospital and at home. We included known epileptic children for treatment at home and all children with acute seizures in the hospital. In all, 26 children were enrolled, 11 at home and 17 in the hospital (including two treated in both locations); only one had simple febrile seizure. They had a total of 125 seizures; 122 seizures (98\%) stopped within 10 minutes (average 3.6 minutes). Two patients in the hospital did not respond and in three, seizures recurred within 3 hours. None had serious adverse effects. Parents had no difficulties administering the drug at home. Most of those who were using rectal diazepam found that nasal midazolam was easier to use and that postictal recovery was faster. Among 15 children who received the drug under electroencephalogram monitoring (six without clinical seizures), the paroxysmal activity disappeared in ten and decreased in three. Nasal midazolam is efficient in the treatment of acute seizures. It appears to be safe and most useful outside the hospital in severe epilepsies, particularly in older children because it is easy for parents to use. These data should be confirmed in a larger sample of children. Its usefulness in febrile convulsions also remains to be evaluated. [\hyperlink{Ixazomib Citrate}{PMID: 10700542}, P Y Jeannet et al., 1999]

\hypertarget{pmid_36158824}{C}hronic calcium channel blockers (CCBs) are indicated in children with idiopathic/heritable pulmonary arterial hypertension (IPAH/HPAH) and positive response to acute vasodilator challenge. However, minimal safety data are available on the long-term high-dose exposure to CCBs in this population. Patients aged 3 months to 18 years who were diagnosed with IPAH/HPAH and treated with CCB in the past 15 years were retrospectively reviewed. The maximum tolerated dose and the long-term safety of high-dose CCBs on the cardiovascular and noncardiovascular systems were assessed. Thirty-two eligible children were enrolled in the study, with a median age of 9 (6-11) years old. Thirty-one patients were treated with diltiazem after diagnosis. The median maximum tolerated dose was 12.9 (9.8-16.8) mg/kg/day. Children younger than 7 years used higher doses than children in the older age group, 16.4 (10.5-28.5) mg/kg/day vs. 12.7 (6.6-14.4) mg/kg/day,  Diltiazem was used in a very high dose for eligible children with IPAH/HPAH. The toxicity of long-term CCB use on the cardiovascular system is mild and controllable. Clinicians should also monitor the noncardiovascular adverse effects associated with drug therapy. [\hyperlink{Ixazomib Citrate}{PMID: 36158824}, Yan Wu et al., 2022]

\hypertarget{pmid_23399743}{T}he purpose of the present study is to compare efficacy and safety of buccal midazolam with intravenous diazepam in control of seizures in Iranian children. This is a randomized clinical trial. 92 patients with acute seizures, ranging from 6 months to 14 years were randomly assigned to receive either buccal midazolam (32 cases) or intravenous diazepam (60 cases) at the emergency department of a children's hospital. The primary outcome of this study was cessation of visible seizure activity within 5 minutes from administration of the first dosage. The second dosage was used in case the seizure remained uncontrolled 5 minutes after the first one. In the midazolam group, 22 (68.8\%) patients were relieved from seizures in 10 minutes. Meanwhile, diazepam controlled the episodes of 42 (70\%) patients within 10 minutes. The difference was, however, not statistically significant (P=0.9). The mean time required to control the convulsive episodes after administration of medications was not statistically significant (P=0.09). No significant side effects were observed in either group. Nevertheless, the risk of respiratory failure in intravenous diazepam is greater than in buccal midazolam. Buccal midazolam is as effective as and safer than intravenous diazepam in control of seizures. [\hyperlink{Ixazomib Citrate}{PMID: 23399743}, Seyed-Hassan Tonekaboni et al., 2012]

\hypertarget{pmid_10698133}{T}he comparative safety and efficacy of chloral hydrate and midazolam for sedation of children has not been adequately studied. In a double-blind randomized trial, at a single university hospital, we enrolled 40 children, ages 2 months to 8 years, in an out-patient neuroimaging study. Children judged to require sedation were enrolled during a 14-month period ending August 1995. They received identically appearing liquids of equal volume of either chloral hydrate (75 mg/kg, maximum 2 g) or midazolam (0.5 mg/kg, maximum 10 mg) by mouth. Children were monitored for changes in arterial blood pressure, oxygen saturation, pulse, respiration and anxiety. Efficacy was judged by evaluating the child's ability to complete the intended scan. Supplemental dosing was administered to children who were judged inadequately sedated 30 minutes after the initial medication. Interim analysis demonstrated a significant sedation failure rate. Of 40 enrolled children, 33 completed the protocol. Efficacy was significantly improved for the chloral hydrate group for both ability to perform the scan, chloral hydrate = 11/11 (100\%, 95\% CI = 72-100) vs. midazolam = 11/22 (50\%, 95\% CI = 29-71), and the need for supplementary dosing, chloral hydrate = 1/11 (9\%, 95\% CI = 0-26) vs midazolam = 12/22 (55\%, 95\% CI = 34-76), P<0.05. Mean duration of sedation was not significantly different. No physiological deterioration occurred and no oxygen administration was required. We conclude that, in these doses, oral chloral hydrate may provide more effective sedation than midazolam for brief neuroimaging studies in young children. [\hyperlink{Ixazomib Citrate}{PMID: 10698133}, J D'Agostino et al., 2000]

\section*{Oxycodone Hydrochloride}
\subsection*{Result}
\subsubsection*{Answer}

0–6 months: No  
6 months–7 years: Unknown  
Adolescents (\textasciitilde{}12–18 years): Unknown  

\subsubsection*{{Explanation}}
\hypertarget{Oxycodone Hydrochloride}
Based on the abstracts available, several studies have investigated the use of oxycodone hydrochloride in children, but the evidence regarding its safety is mixed and often limited by small sample sizes, lack of targeted safety endpoints, or focus on pharmacokinetics rather than comprehensive safety outcomes.

1. Neonates and Infants (0–6 months):
A study of 22 infants (0–6 months) receiving intravenous oxycodone hydrochloride postoperatively found significant variability in drug clearance and half-life, especially in the youngest groups. While none of the spontaneously breathing infants required ventilatory assistance, the authors concluded that "routine dosing of oxycodone in young infants may be dangerous" and emphasized the need for individual dose titration due to high interindividual variability. This study does not affirm safety; rather, it raises concerns about potential danger in this age group [\hyperlink{pmid_15960639}{PMID: 15960639}, Marja-Leena Pokela et al., 2005].

2. Children (age not specified, but post-ophthalmic surgery):
A study of 18 children given intravenous oxycodone hydrochloride (0.1 mg/kg) after ophthalmic surgery reported that oxycodone caused greater ventilatory depression than comparable doses of other opioids. The study monitored vital signs and found changes in ventilatory parameters, but did not report serious adverse events. However, the focus was on pharmacokinetics and ventilatory effects, not comprehensive safety, and the findings suggest a higher risk of ventilatory depression [\hyperlink{pmid_7605420}{PMID: 7605420}, K T Olkkola et al., 1994].

3. Children (6 months–7 years):
A population pharmacokinetic modeling study synthesized data from several studies and concluded that weight-based dosing of oxycodone is appropriate for children aged 6 months to 7 years. However, this study focused on pharmacokinetics and did not directly assess safety outcomes or adverse events [\hyperlink{pmid_16554451}{PMID: 16554451}, Ahmed El-Tahtawy et al., 2006].

4. Adolescents (mean age \textasciitilde{}16 years):
A recent study included 193 subjects (mean age 15.9 years) who received oxycodone for postoperative pain. The abstract mentions findings related to pharmacogenomics and oxycodone use but does not provide specific safety data or affirm safety in this age group [\hyperlink{pmid_36618804}{PMID: 36618804}, Soroush Merchant et al., 2022].

5. General pediatric population:
Two review articles highlight that, while oxycodone is used in children for pain management, data on its safety and efficacy are limited. They specifically state that, compared to morphine, there is less evidence supporting the safety of oxycodone in children, and more research is needed [\hyperlink{pmid_33747308}{PMID: 33747308}, Michael J Rieder et al.; \hyperlink{pmid_33747307}{PMID: 33747307}, Michael J Rieder et al.].

6. In vitro study:
One study examined the effects of oxycodone on neural stem cells and found that high doses reduced cell survival and proliferation, but low doses had minimal effect. This was not a clinical safety study in children [\hyperlink{pmid_30290204}{PMID: 30290204}, Gang Wu et al., 2018].

Summary:
- For neonates and infants (0–6 months), available evidence suggests potential danger and does not affirm safety.
- For children older than 6 months, some pharmacokinetic data exist, but there are no targeted safety studies affirming safety.
- For adolescents, no targeted safety studies affirm safety.
- Review articles consistently state that safety data for oxycodone in children are limited and insufficient.

Therefore, based on the abstracts, there is no definitive evidence from targeted safety studies affirming that oxycodone hydrochloride is safe for use in children of any age group. In some cases, there is evidence suggesting potential risk, especially in infants.

\subsection*{Abstracts}
\hypertarget{pmid_15960639}{T}he pharmacokinetics of oxycodone (13-hydroxy-7,8-dihydrocodeinone) has been studied in adults and in children who are older than 6 months but there is no information on the disposition of oxycodone in neonates and young infants. The aim of this study was to study the pharmacokinetics of oxycodone in infants varying in age from 0 to 6 months. Twenty-two infants undergoing surgery were given postoperatively an intravenous bolus of 0.1 mg.kg(-1) of oxycodone hydrochloride. Ten of the patients were younger than 1 week (group 1), six from 1 week to 2 months (group 2) and six from 2 to 6 months (group 3). Plasma samples were collected for the analysis of oxycodone concentrations up to 24 h. Pharmacokinetics were characterized by noncompartmental methods. The median (range) values for the clearance (Cl) were 9.9 (2.3-17.2), 20.1 (3.7-40.4) and 15.4 (14.8-80.2) ml.min(-1).kg(-1) in the above three groups. The values for volume of distribution at steady-state were 3.3 (1.9-4.7), 5.6 (1.3-8.5) and 3.2 (1.8-6.0) l.kg(-1) and for elimination half-life (t(1/2)) 4.4 (2.4-14.1), 3.6 (1.6-11.6) and 2.0 (0.8-3.9) h, respectively. Both Cl (r = 0.46) and half-life (r = -0.46) were correlated to the age of the patient (P < 0.05). There were 13 patients who were on mechanical ventilation at the time of oxycodone administration. None of the spontaneously breathing infants had hypoventilation which required assistance during the study. The values for Cl and t(1/2) varied greatly between the subjects. This variability was most pronounced in the two youngest groups. Routine dosing of oxycodone in young infants may be dangerous. The dose of oxycodone must be titrated individually. [\hyperlink{Oxycodone Hydrochloride}{PMID: 15960639}, Marja-Leena Pokela et al., 2005] 1. Oxycodone hydrochloride (0.1 mg kg-1) was given by intravenous bolus to 18 children after ophthalmic surgery. Plasma was sampled for up to 8 h. Blood pressure, heart rate, peripheral arteriolar oxygen saturation, end-tidal carbon dioxide and halothane concentrations and ventilatory rate were also recorded. 2. Mean (+/- s.d.) values of drug clearance and volume of distribution (Vss) were 15.2 +/- 4.2 ml min-1 kg-1 and 2.1 +/- 0.8 l kg-1. Maximum mean end-tidal carbon dioxide concentration and minimum mean ventilatory rate occurred 8 min after administration of oxycodone but the minimum mean peripheral arteriolar oxygen saturation occurred at 4 min. 3. Oxycodone (0.1 mg kg-1) appears to cause greater ventilatory depression than comparable analgesic doses of other opioids. [\hyperlink{Oxycodone Hydrochloride}{PMID: 15960639}, K T Olkkola et al., 1994]

\hypertarget{pmid_17803435}{T}he aim of this study was to evaluate the safety of olopatadine hydrochloride ophthalmic solution 0.2\% in children and adolescents 3-17 years of age. In this 6-week, randomized, double-masked safety evaluation, eligible subjects with asymptomatic eyes underwent in-office visits at weeks 1, 3, and 6 and were contacted by telephone at weeks 2, 4, and 5. Qualified subjects were assigned randomly in a 2:1 ratio of olopatadine 0.2\% to vehicle (identical formation without the active ingredient) for dosing on a once-daily schedule. Safety parameters assessed included adverse events, visual acuity, ocular signs (slit-lamp assessments), dilated fundus examinations, intraocular pressure (IOP), pulse, and blood pressure. An evaluation of 126 subjects (age range, 3-17) revealed no clinically relevant treatment-related changes in visual acuity, IOP, slit-lamp assessments, fundus examinations, or cardiovascular parameters. All adverse events reported were mild or moderate. Olopatadine 0.2\% administered once-daily for 6 weeks is safe and well tolerated in children and adolescent patients. [\hyperlink{Oxycodone Hydrochloride}{PMID: 17803435}, Steven J Lichtenstein et al., 2007]

\hypertarget{pmid_36618804}{O}xycodone is a commonly used oral opioid in children for treating postoperative pain. Highly polymorphic gene  Patients who underwent Nuss procedure and spine fusion with  Of 193 subjects (age 15.9±0.25 years, 28.5\% female, 93.78\% White; 101 NM, 76 IM, 10 PM and 6 UM), 77.72\% underwent pectus surgery.  Our findings suggest  [\hyperlink{Oxycodone Hydrochloride}{PMID: 36618804}, Soroush Merchant et al., 2022] Anticholinergics are a key element in treating neurogenic detrusor overactivity, but only limited data are available in the pediatric population, thus limiting the application to children even for oxybutynin chloride (OC), a prototype drug. This retrospective study was designed to provide data regarding the efficacy, tolerability, and safety of OC in the pediatric population (0-15 years old) with spinal dysraphism (SD). Records relevant to OC use for neurogenic bladder were gathered and scrutinized from four specialized clinics for pediatric urology. The primary efficacy outcomes were maximal cystometric capacity (MCC) and end filling pressure (EFP). Data on tolerability, compliance, and adverse events (AEs) were also analyzed. Of the 121 patient records analyzed, 41 patients (34\%) received OC at less than 5 years of age. The range of prescribed doses varied from 3 to 24 mg/d. The median treatment duration was 19 months (range, 0.3-111 months). Significant improvement of both primary efficacy outcomes was noted following OC treatment. MCC increased about 8\% even after adjustment for age-related increases in MCC. Likewise, mean EFP was reduced from 33 to 21 cm H2O. More than 80\% of patients showed compliance above 70\%, and approximately 50\% of patients used OC for more than 1 year. No serious AEs were reported; constipation and facial flushing consisted of the major AEs. OC is safe and efficacious in treating pediatric neurogenic bladder associated with SD. The drug is also tolerable and the safety profile suggests that adjustment of dosage for age may not be strictly observed. [\hyperlink{Oxycodone Hydrochloride}{PMID: 36618804}, Jung Hoon Lee et al., 2014]

\hypertarget{pmid_29026333}{O}xycodone is poorly studied as an adjuvant to central blockades. The aim of this pilot study was to assess the efficacy and safety of oxycodone hydrochloride in epidural blockade among patients undergoing total hip arthroplasty (THA). In 11 patients (American Society of Anesthesiologists physical status classification system II/III, age range: 59-82 years), THA was conducted with an epidural blockade using 15 mL 0.25\% bupivacaine (37.5 mg) with 5 mg oxycodone hydrochloride and sedation with propofol infusion at a dose of 3-5 mg/kg/h. After the surgery, patients received ketoprofen at a dose of 100 mg twice daily. In the first 24 hours postoperative period, pain was assessed by numerical rating scale at rest and on movement; adverse effects (AEs) were recorded; and plasma concentrations of oxycodone, noroxycodone, and bupivacaine were measured. The administration of epidural oxycodone at a dose of 5 mg in patients undergoing THA provided analgesia for a mean time of 10.3±4.89 h. In one patient, mild pruritus was observed. Oxycodone did not evoke other AEs. Plasma concentrations of oxycodone while preserving analgesia were >2.9 ng/mL. Noroxycodone concentrations in plasma did not guarantee analgesic effect. The administration of epidural oxycodone at a dose of 5 mg prolongs the analgesia period to \textasciitilde{}10 hours in patients after THA. Oxycodone may evoke pruritus. A 5 mg dose of oxycodone hydrochloride used in an epidural blockade seems to be a safe drug in patients after THA. [\hyperlink{Oxycodone Hydrochloride}{PMID: 29026333}, Bogumił Olczak et al., 2017]

\hypertarget{pmid_36719881}{U}rsodeoxycholic acid (UDCA) is the main therapeutic drug for cholestasis, but its use in children is controversial. We conducted this study to evaluate the efficacy and safety of ursodeoxycholic acid in children with cholestasis. We searched Medline (Ovid), Embase (Ovid), Cochrane Central Register of Controlled Trials (CENTRAL), CNKI, WanFang Data and VIP from the establishment of databases to July 2022. Eligible studies included Chinese or English randomized controlled trials (RCTs) comparing the efficacy and safety of no UDCA (placebo or blank control) and UDCA in children with cholestasis. This study had been registered with PROSPERO (CRD42022354052). A total of 32 RCTs proved eligible, which included 2153 patients. The results of meta-analysis showed that UDCA could improve symptoms of children with cholestasis (risk ratio 1.24, 95\% CI 1.18 to 1.29; moderate quality of evidence), and serum levels of alanine aminotransferase, total bilirubin, direct bilirubin and total bile acid (low quality of evidence). For some children with specific cholestasis, UDCA could also effectively drop serum levels of aspartate aminotransferase (parenteral nutrition-associated cholestasis) and γ-glutamyl transferase (infantile hepatitis syndrome, parenteral nutrition-associated cholestasis). The most common adverse drug reactions (ADRs) of UDCA in children were gastrointestinal adverse reactions, with an incidence of 10.63\% (67/630). There was no significant difference in the incidence of ADRs between UDCA and placebo/blank control groups (risk difference 0.03, 95\%CI -0.01 to 0.06; moderate quality of evidence), and among children taking different UDCA doses (P = 0.27). The available short-term evidence showed that UDCA was effective and safe for children with cholestasis. Clinicians should use UDCA with caution (start with a low dose) until the long-term effect is further explored in future larger RCTs. [\hyperlink{Oxycodone Hydrochloride}{PMID: 36719881}, Liang Huang et al., 2023]

\hypertarget{pmid_28741653}{C}hloral hydrate is commonly used to sedate children for painless procedures. Children may recover more quickly after sedation with dexmedetomidine, which has a shorter half-life. We randomly allocated 196 children to chloral hydrate syrup 50 mg.kg [\hyperlink{Oxycodone Hydrochloride}{PMID: 28741653}, V M Yuen et al., 2017] To assess the pharmacokinetics and safety of hydrochloride ophthalmic solution 0.77\% olopatadine from 2 independent (Phase I and Phase III, respectively) clinical studies in healthy subjects. The Phase I, multicenter, randomized (2:1), vehicle-controlled study was conducted in subjects ≥18 years old (N=36) to assess the systemic pharmacokinetics of olopatadine 0.77\% following single- and multiple-dose exposures. The Phase III, multicenter, randomized (2:1), vehicle-controlled study was conducted in subjects ≥2 years old (N=499) to evaluate long-term ocular safety of olopatadine 0.77\%. Subjects received olopatadine 0.77\% or vehicle once daily bilaterally for 7 days in the pharmacokinetic study and 6 weeks in the safety study. In the pharmacokinetic study, olopatadine 0.77\% was absorbed slowly and reached a peak plasma concentration (C Olopatadine 0.77\% had minimal systemic exposure or accumulation in healthy subjects and was well tolerated in both adult and pediatric subjects. [\hyperlink{Oxycodone Hydrochloride}{PMID: 28741653}, Edward Meier et al., 2017]

\hypertarget{pmid_30124097}{P}ediatric patients present changing physiological features. Because of the lack of land suitable for commercial management, pediatric specialties very often need to prepare extemporaneous formulations to improve the dosage and administration of drugs for children. Oral liquid formulations are the most suitable for pediatric patients. Clonidine is widely used in the pediatric population for opioid withdrawal, hypertensive crisis, attention deficit disorders and hyperactivity syndrome, and as an analgesic in neuropathic cancer pain. The objective was to study the physicochemical and microbiological stability and determine the shelf life of an oral solution containing 20 µg/mL clonidine hydrochloride in different storage conditions (5 ± 3 °C, 25 ± 3 °C, and 40 ± 2 °C). Using raw material with excipients safe for all pediatric age groups, two oral liquid formulations of clonidine hydrochloride were designed (with and without preservatives). Solutions stored at 5 ± 3 °C (with and without preservatives) were physically and microbiologically stable for at least 90 days in closed containers and for 42 days after opening. Two oral solutions of clonidine hydrochloride 20 µg/mL were developed for pediatric use from raw materials that are readily available and easy to process, containing safe excipients that are stable over a long period of time. [\hyperlink{Oxycodone Hydrochloride}{PMID: 30124097}, V Merino-Bohórquez et al., 2019]

\hypertarget{pmid_2295577}{F}luoxetine hydrochloride is the first selective serotonin uptake inhibitor introduced commercially in the United States. This report describes preliminary clinical experience with fluoxetine in 10 children and adolescents, aged 8 to 15 years, with primary obsessive compulsive disorder (OCD) or Tourette's syndrome (TS) plus OCD. In general, fluoxetine, which was administered from 4 to 20 weeks at a dosage of 10 or 40 mg per day, was well tolerated. Adverse effects included behavioral agitation/activation in four patients and mild gastrointestinal symptoms in two patients. No abnormalities were noted in the seven children who had follow-up EKGs. Five of the 10 patients (50\%) were considered responders; their obsessive-compulsive symptoms decreased substantially during treatment with fluoxetine. Responder rates were similar in the primary OCD (two of four, 50\%) and TS + OCD (three of six, 50\%) groups. In conclusion, short-term fluoxetine administration appears to be safe in children and adolescents. Placebo-controlled trials are needed to further assess the efficacy of fluoxetine. [\hyperlink{Oxycodone Hydrochloride}{PMID: 2295577}, M A Riddle et al., 1990]

\hypertarget{pmid_29203293}{O}rganochlorine pesticides (OCPs) are environmental contaminants that persist in the environment and bioaccumulate through the food chain in humans and animals. Although previous studies have shown an association between prenatal OCP exposure and subsequent neurodevelopment, the levels of OCPs included in these studies were inconsistent. A hospital-based prospective birth cohort study was conducted to examine the associations between prenatal exposure to relatively low levels of OCPs and neurodevelopment in infants at 6 (n=164) and 18 (n=115)months of age. Blood samples were analyzed using gas chromatography/mass spectrometry techniques to quantify 29 OCPs. The Bayley Scales of Infant Development 2nd edition (BSID-II) was used to assess the Mental and Psychomotor Developmental Index. After controlling for confounders, we found an inverse association between prenatal exposure to cis-heptachlor epoxide and the Mental Developmental Index at 18 months of age. Furthermore, infants born to mothers with prenatal concentrations of cis-heptachlor epoxide in the highest quartile had Mental Developmental Index scores -9.8 (95\% confidence interval: -16.4, -3.1) lower than that recorded for infants born to mothers with concentrations of cis-heptachlor epoxide in the first quartile (p for trend <0.01). These results support the hypothesis that prenatal exposure to OCPs, especially cis-heptachlor epoxide, may have an adverse effect on the neurodevelopment of infants at specific ages, even at low levels. [\hyperlink{Oxycodone Hydrochloride}{PMID: 29203293}, Keiko Yamazaki et al., 2018]

\hypertarget{pmid_30290204}{A}s one of several opioids, oxycodone has been widely used, particularly in postoperative analgesia in children and cesarean section. However, the effect of oxycodone on developing brain still remains to be seen. Since there is a link between anesthetics exposure and long-term behavioral or cognitive dysfunction in young children, in the current study, the direct effect of oxycodone on neural stem cells (NSCs) biological behaviors was investigated. After exposed to a high dose of oxycodone (10 μg/mL) for 48 h, NSCs survival and proliferation were significantly reduced, while NSCs apoptosis and differentiation were enhanced. These effects were significantly weaker than that when exposed to same dose of morphine. No significant difference was observed regarding to above biological behaviors when exposed to lower doses (0.1 μg/mL and 1.0 μg/mL) of oxycodone. The antagonist of opioid receptor, nalmefene, successfully reversed the influence of oxycodone. Taken together, our results indicated that short term exposure to oxycodone in low dose might be allowed for developing brain. [\hyperlink{Oxycodone Hydrochloride}{PMID: 30290204}, Gang Wu et al., 2018]

\hypertarget{pmid_9322726}{O}ral pharmacotherapy has been commonly used as an adjunct to clean intermittent catheterization (CIC) in the treatment of neurogenic bladder in order to achieve continence, but may be associated with unacceptable side effects. The authors' experience with sterile intravesical preparations of oxybutynin hydrochloride and ephedrine in children is reported here. Patients requiring CIC for neurogenic bladder but with incontinence that was unresponsive to standard oral therapy or that was associated with severe systemic side effects were studied over a 1-year period. Clinical, radiological and urodynamic assessments were made prior to commencing treatment with intravesical oxybutynin hydrochloride. Patients who remained incontinent with poor internal sphincter muscle tone had intravesical ephedrine added. Seven patients were involved in the study over a 1-year period. Two patients became continent and one patient had an improvement in upper tract dilatation. One patient had a limited improvement with oxybutynin alone but became continent with the addition of ephedrine. Three patients had no response to treatment. There were few side effects. Intravesical agents have a role in the management of paediatric neurogenic bladder for those children with significant adverse sequelae from oral pharmacotherapy who would otherwise require surgical intervention. Intravesical therapy is a safe technique in children with sterile preparations. Further investigation of this modality should be pursued. [\hyperlink{Oxycodone Hydrochloride}{PMID: 9322726}, A J Holland et al., 1997]

\hypertarget{pmid_15721878}{E}xposure to organochlorine compounds (OCs) occurs both in utero and through breastfeeding. Levels of hexachlorobenzene (HCB) in the cord serum of newborns from a population located in the vicinity of an electrochemical factory in Spain are among the highest ever reported. We aimed to assess the degree of breast milk contamination in this population and the subsequent exposure of children to these chemicals through breastfeeding. A birth cohort including 92 mother-infant pairs (84\% of all births in the study area) was recruited between 1997 and 1999 in five neighboring villages. OCs were measured in cord serum, colostrum, breast milk, and children's serum at 13 months of age. Concentrations of OCs were detected and quantified in all colostrum and milk samples. The concentrations in mature milk were lower than those encountered in colostrum. At 13 months of age the highest concentration of OC was found for dichlorodiphenyl dichloroethane (p,p'-DDE), in contrast to what these children presented at birth, where HCB was the highest compound. Those infants who were breastfed had higher concentrations at the age of 1 than those who were formula fed (2.13 ng/mL of HCB among formula feeders vs 4.26 among breast feeders, and 1.95 of p,p'-DDE vs 6.00 (P<0.05)). Long-term breastfeeding leads to a dose-response increase of the concentrations in children's serum during the first year of life. [\hyperlink{Oxycodone Hydrochloride}{PMID: 15721878}, Núria Ribas-Fitó et al., 2005]

\hypertarget{pmid_33747308}{L}a douleur est un problème courant chez les enfants. Des mesures pharmacologiques et non pharmacologiques sont utilisées pour la prendre en charge. Depuis quelques décennies, les opioïdes par voie orale sont populaires pour soulager la douleur modérée à grave. La codéine a longtemps été l'opioïde par voie orale le plus connu pour les enfants. Pour des raisons de sécurité, elle est désormais nettement moins accessible et moins employée. Divers autres opioïdes la remplacent, mais les données sur leur efficacité et leur sécurité sont limitées chez les enfants. L'oxycodone par voie orale emprunte les mêmes voies métaboliques que la codéine, mais sa pharmacocinétique est très variable. Les données sur la sécurité et l'efficacité de l'hydromorphone et du tramadol par voie orale chez les enfants sont également limitées. Lorsqu'on y recourt au lieu de la codéine, la morphine par voie orale est l'opiacé dont la sécurité et l'efficacité sont les mieux démontrées chez les enfants. Des recherches devront être réalisées pour explorer d'autres approches relatives aux médicaments opioïdes et non opioïdes, afin d'orienter les traitements analgésiques fondés sur des données probantes qui soulageront la douleur modérée à grave chez les enfants. [\hyperlink{Oxycodone Hydrochloride}{PMID: 33747308}, Michael J Rieder et al., ]

\hypertarget{pmid_16554451}{Y}oung children are often undertreated for pain. One barrier to effective pain treatment is understanding the pharmacokinetic behavior of analgesics in this age group. Oxycodone is a commonly prescribed opioid for severe pain, yet little is known about its pharmacokinetics in young children. This article used population pharmacokinetic modeling to synthesize pharmacokinetic data from several studies into a model. A single population model that described the observed pharmacokinetics was developed. The combined data were best described with a 2-compartment linear model with different first-order absorption rates depending on route of administration. Weight was found to significantly influence both clearance (CL) and volume of distribution (Vd). The following model adequately describes the population pharmacokinetic profile of oxycodone where absolute bioavailability (F) is estimated for each administration route: CL/F=55x(body weight/70)0.87; V/F=86x(body weight/70)1.16. The interindividual coefficients of variation in CL and Vd were 20.2 and 19.7\%, respectively. This finding confirms that the allometric scaling using the above model explained most of the variability in exposure observed among children. This model confirms using a weight-based dose for oxycodone without adjustment for age between 6 months and 7 years and is valuable for evaluating dosing schedules and dosing routes. [\hyperlink{Oxycodone Hydrochloride}{PMID: 16554451}, Ahmed El-Tahtawy et al., 2006]

\hypertarget{pmid_18682543}{T}o review the role of oxandrolone in pediatric patients with severe thermal burn injury. MEDLINE (1950-April 2008) and Science Citation Index (1900-April 2008) searches were performed using the key terms oxandrolone, burn, and children. All English-language articles that evaluated the efficacy and safety of oxandrolone in pediatric patients with severe thermal burn injury were included in this review. Oxandrolone stimulates protein synthesis by binding to androgen receptors. The efficacy and safety of adjunct oxandrolone therapy in pediatric patients (<or=18 y old) with severe thermal burn injury (total body surface area burn >20\%) were evaluated in 8 clinical studies. Oral oxandrolone 0.1 mg/kg twice daily increased protein synthesis, lean body mass accretion, and muscle strength; improved serum visceral protein concentrations; promoted weight gain; and increased bone mineral content. During the postburn rehabilitation period, oxandrolone 0.1 mg/kg/day improved muscle strength, especially when combined with exercise. Based on clinical studies, oxandrolone 0.1 mg/kg twice daily is safe when given for up to 12 months. However, mild increases in serum liver transaminase concentrations and reversible sexual changes were observed during therapy. Although data on the efficacy and safety of oxandrolone in severely burned children are supported by prospective, randomized, controlled studies, limitations of available data are that they originated from a single study center and that wound healing measurement is lacking in children with severe thermal burns. The benefits of adjunct oxandrolone therapy in severely burned pediatric patients have been demonstrated in the acute postburn injury and long-term postburn rehabilitation periods. Close monitoring of liver function, sexual development, and growth pattern is recommended during oxandrolone treatment. [\hyperlink{Oxycodone Hydrochloride}{PMID: 18682543}, James T Miller et al., 2008]

\hypertarget{pmid_17550483}{O}xycodone has become popular for post-Caesarean section (CS) analgesia yet it is not currently recommended for use in breast-feeding mothers because of limited information on its excretion into breast milk. To investigate the relationship between maternal ingestion of oxycodone after CS and the resultant maternal plasma, breast milk and neonatal plasma drug levels up to 72-h post-partum. Fifty breast-feeding mothers taking oxycodone had blood and breast milk samples analysed for oxycodone levels at 24 h intervals after CS. Forty-one neonates had blood samples taken at 48 h. Oxycodone was detected in the milk of mothers who had taken any dose in a 24-h period, with significant correlation between maternal plasma and milk levels (R(2) = 0.81). The median milk:plasma (M:P) ratio for the same period was 3.2:1. Over the subsequent 48 h, the relationship between plasma and milk levels was less strong (R(2) = 0.59) and there was a larger range of M:P levels with evidence of persistence of oxycodone in the breast milk of some mothers. Oxycodone levels up to 168 ng/mL were detected in breast milk (20\% > 100 ng/mL). Oxycodone was detected in the plasma of one infant. Oxycodone is concentrated in human breast milk up to 72-h post-partum. Breastfed infants may receive > 10\% of a therapeutic infant dose. However, maternal oxycodone intake up to 72-h post-CS poses only minimal risk to the breast-feeding infant as low volumes of breast milk are ingested during this period. [\hyperlink{Oxycodone Hydrochloride}{PMID: 17550483}, Suzette Seaton et al., 2007]

\hypertarget{pmid_2402648}{C}hloral hydrate has been used extensively to sedate children, but at Brooke Army Medical Center, other drug combinations were becoming increasingly popular due to a perception that chloral hydrate had a high rate of failure, especially with younger or neurologically impaired children. Therefore, 50 children were given the drug before a diagnostic study, and patient data and a sedation score were recorded on a worksheet. Of 50 children, 43 (86\%) were "successfully sedated" on the first attempt with no side effects. Children with neurologic disorders had a much greater (27\% vs 4\%) failure rate than "normal" children. The sedation rate did not significantly differ by age, sex, or initial drug dosage. The study suggest that chloral hydrate is a safe and effective oral sedative but that children with neurologic disorders may need alternative drugs for sedation. [\hyperlink{Oxycodone Hydrochloride}{PMID: 2402648}, P D Rumm et al., 1990]

\hypertarget{pmid_493246}{W}ithin the scope of mass examinations 949 children from kindergartens in Basle were submitted to the scotch tape test of Oxyures. The findings were positive in 64 cases (6,7\%). The parents of 50 children accepted the suggestion that their children be treated twice with 100 mg of ciclobendazole each, at a weekly interval. This therapy resulted in complete cure of the oxyuriasis in all treated cases. Despite the impressive and safe action against this disease of ciclobendazole it is recommended in any case to take a second tablet, since the risk of reinfection is high. [\hyperlink{Oxycodone Hydrochloride}{PMID: 493246}, A Bächlin et al., 1979]

\hypertarget{pmid_28275979}{S}edation is often required for children undergoing diagnostic procedures. Chloral hydrate has been one of the sedative drugs most used in children over the last 3 decades, with supporting evidence for its efficacy and safety. Recently, chloral hydrate was banned in Italy and France, in consideration of evidence of its carcinogenicity and genotoxicity. Dexmedetomidine is a sedative with unique properties that has been increasingly used for procedural sedation in children. Several studies demonstrated its efficacy and safety for sedation in non-painful diagnostic procedures. Dexmedetomidine's impact on respiratory drive and airway patency and tone is much less when compared to the majority of other sedative agents. Administration via the intranasal route allows satisfactory procedural success rates. Studies that specifically compared intranasal dexmedetomidine and chloral hydrate for children undergoing non-painful procedures showed that dexmedetomidine was as effective as and safer than chloral hydrate. For these reasons, we suggest that intranasal dexmedetomidine could be a suitable alternative to chloral hydrate. [\hyperlink{Oxycodone Hydrochloride}{PMID: 28275979}, Giorgio Cozzi et al., 2017]

\hypertarget{pmid_8010205}{T}he purpose of this prospective study was to evaluate the safety and efficacy of thioridazine as an adjunct to chloral hydrate sedation when children undergoing MR imaging are difficult to sedate. All 87 children in the study either could not be sedated with chloral hydrate alone or were mentally retarded. Thioridazine (2-4 mg/kg) was administered orally 2 hr before and chloral hydrate (50-100 mg/kg) was administered orally 30 min before the 104 MR examinations. All children were monitored by continuous pulse oximetry. All images were individually evaluated by pediatric radiologists and were graded acceptable if they contained only minimal motion artifact or no motion artifact. Studies were considered successful only when 95\% or more of the images were acceptable. MR imaging was successful in 93 (89\%) of 104 examinations. The success rate for children entered into the study because of prior failure of chloral hydrate sedation was not significantly different from the success rate for children with mental retardation. A tendency for increasing failure rate with age was not significant. No serious complications occurred during the study. The most common adverse reaction, transient reduced oxygen saturation, was seen in five children. Other adverse effects encountered were vomiting in four children, hyperactivity in two children, transient tachycardia in one child, and prolonged sedation in one child. No child required hospitalization because of an adverse reaction to sedation. The study indicates that thioridazine is a safe and effective adjunct to chloral hydrate when a child undergoing MR imaging is difficult to sedate. [\hyperlink{Oxycodone Hydrochloride}{PMID: 8010205}, S B Greenberg et al., 1994]

\hypertarget{pmid_16520840}{C}hloral hydrate is generally considered to be a safe hypnotic drug, and is commonly used for short-term sedation before diagnostic procedures. Its irritant actions to the mucous membranes are usually limited. We report a rare complication of chloral hydrate overdose in an infant. An 8-month-old male infant became unconscious and required ventilation support after an overdose of chloral hydrate was administered to provide sedation for an ophthalmologic examination. White plaques and sloughing of the oropharyngeal mucosa were observed on the next day. Esophagogastroscopy revealed severe corrosive lesions on the whole esophagus. The child recovered after supportive treatment and his oral intake remained well without dysphagia after 1 year. This report illustrates the potential corrosive effect of chloral hydrate. Strict attention should be paid to the dosing and administration protocol of chloral hydrate in infants. The condition of the oropharyngeal mucosa should be carefully monitored after chloral hydrate administration. [\hyperlink{Oxycodone Hydrochloride}{PMID: 16520840}, Yu-Cheng Lin et al., 2006]

\hypertarget{pmid_33747307}{P}ain is a common problem for children, and pain management comprises both pharmacologic and nonpharmacologic measures. For moderate to severe pain, oral opioids have been a popular choice for the last few decades. Codeine has historically been the best-known oral opioid for use in children. However, availability and use of codeine have sharply declined due to safety concerns. A variety of other opioids have been used in place of codeine, but data are limited regarding their efficacy and safety in children. While the same pathways metabolize oral oxycodone as codeine, oxycodone's pharmacokinetics varies widely. There are also limited data on the safety and efficacy of oral hydromorphone and tramadol use for children. Oral morphine is the opiate alternative to codeine for which there is the most evidence of safety and efficacy in children. Research is needed to investigate both other opioids and non-opioid approaches to guide evidence-based analgesic therapy and treatment for moderate-to-severe pain in children. [\hyperlink{Oxycodone Hydrochloride}{PMID: 33747307}, Michael J Rieder et al., ]

\section*{Phentermine Hydrochloride}
\subsection*{Result}
\subsubsection*{Answer}

Unknown

\subsubsection*{{Explanation}}
\hypertarget{Phentermine Hydrochloride}
A review of the available abstracts reveals the following regarding the safety of Phentermine Hydrochloride in children:

- One abstract specifically discusses Phentermine Hydrochloride: "Phentermine hydrochloride is a noradrenergic sympathetic amine approved for decades by the U.S. Food and Drug Administration (FDA) at doses as high as 37.5 mg/day for the short-term treatment of obesity. ... The controlled-release formulation of phentermine and topiramate at low, medium and full doses ... promotes weight reduction, with clinical trial data supporting improvement in adiposopathic consequences leading to metabolic diseases. Reported adverse events with this combination agent are as expected, based upon knowledge of the individual components." [\hyperlink{pmid_22348915}{PMID: 22348915}, H E Bays et al., 2011]

- However, this abstract does not specify any studies conducted in children, nor does it provide safety data for pediatric populations. It only references general clinical trial data and FDA approval, which is known to be for adults.

- No other abstracts mention Phentermine Hydrochloride in the context of children or provide targeted safety studies in pediatric age groups.

- Therefore, based on the abstracts available, there is no evidence from targeted studies affirming the safety or unsafety of Phentermine Hydrochloride in children of any age range. The safety of Phentermine Hydrochloride in children is therefore unknown.

\subsection*{Abstracts}
\hypertarget{pmid_1172955}{P}hencyclidine hydrochloride is a dangerous drug. Its incidence as the causative agent in childhood poisoning is increasing. A pressor effect of phencyclidine has been noted in studies both in man and in experimental animals. We summarize seven cases of poisoning with this drug, including one in which death occurred following a hypertensive crisis. Patients who have ingested this drug should have continuous monitoring of blood pressure in an intensive care unit. [\hyperlink{Phentermine Hydrochloride}{PMID: 1172955}, J W Eastman et al., 1975]

\hypertarget{pmid_18702885}{A}llergic rhinitis (AR) is a common chronic condition in children and may impact a child's quality of life. Increasing treatment compliance may improve quality of life. An oral suspension of fexofenadine hydrochloride (HCl) has been developed to ease administration to children and may, therefore, improve treatment compliance. The purpose of this study was to assess the pharmacokinetic behavior, safety, and tolerability of a single dose of fexofenadine HCl oral suspension administered to children aged 2-5 years with allergic rhinitis. Children (aged 2-5 years) with AR were recruited in a multicenter, open-label, single-dose study. Fexofenadine HCl (30 mg) was administered as a 6-mg/mL suspension (5 mL). Plasma samples were collected up to 24 hours postdose. Adverse events (AEs); electrocardiograms (ECGs); vital signs; and clinical laboratory tests for hematology, blood chemistry, and urinalysis were analyzed to evaluate safety and tolerability. Fifty subjects completed the study. Mean maximum plasma concentration of fexofenadine was 224 ng/mL, and mean area under the plasma concentration curve was 898 ng . hour/mL. Treatment-emergent AEs were mild in intensity and reported in a total of seven subjects. No trends or clinically meaningful changes in mean ECG, vital sign, or clinical laboratory test data occurred during the study. In children aged 2-5 years, the exposure after a 30-mg dose of fexofenadine HCl suspension was similar to the exposures previously seen after a 30- and 60-mg dose of fexofenadine HCl in children aged 6-11 years and in adults, respectively. The suspension was also well tolerated. [\hyperlink{Phentermine Hydrochloride}{PMID: 18702885}, Nathan Segall et al., ]

\hypertarget{pmid_942230}{K}etamine hydrochloride 2 mg/kg, together with atropine 0.2 mg, has been given intravenously on 100 occasions on a general paediatric ward. No serious side effects occurred. Dreams followed in 4 children but did not reduce acceptability of the drug. In our hands it has greatly reduced the pain and distress of children undergoing many routine medical procedures, particularly the dread which builds up when these have to be repeated in the same child. It has also produced close to ideal conditions for the operator, and probably increased his efficiency by reducing the emotional strain which occurs when doing painful things to a frightened patient. [\hyperlink{Phentermine Hydrochloride}{PMID: 942230}, E Elliott et al., 1976]

\hypertarget{pmid_17941284}{T}he safety of fexofenadine has been examined extensively in adults and school-age children. However, the safety of fexofenadine in children younger than 6 years has not been reported to date. To compare the safety and tolerability of twice-daily fexofenadine hydrochloride, 30 mg, and placebo in preschool children aged 2 to 5 years with allergic rhinitis. This was a multicenter, double-blind, randomized, placebo-controlled, parallel-group study, conducted between February 29, 2000, and June 14, 2001. Participants were randomized to either fexofenadine hydrochloride, 30 mg, or placebo twice daily for a 2-week period. To facilitate dosing, capsule content was mixed with applesauce (approximately 10 mL). Safety assessments depended on date of entry into the study because of an amendment to the protocol. Before the amendment, assessments included physical examination, vital signs reporting (oral temperature, heart rate, and respiratory rate), and adverse event (AE) reporting. After the amendment, safety assessments included laboratory testing (blood chemistry and hematology profiles), physical examination, 12-lead electrocardiography, and vital signs (oral temperature, blood pressure, heart rate, and respiratory rate) and AE reporting. Treatment-emergent AEs were observed in 116 of 231 participants receiving placebo and 111 of 222 receiving fexofenadine. These AEs were possibly related to study medication in 19 (8.2\%) and 21 (9.5\%) of the participants receiving placebo and fexofenadine, respectively, and most frequently involved the digestive system. No clinically relevant differences in laboratory measures, vital signs, and physical examinations were observed. The findings show that fexofenadine hydrochloride, 30 mg, is well tolerated and has a good safety profile in children aged 2 to 5 years with allergic rhinitis. [\hyperlink{Phentermine Hydrochloride}{PMID: 17941284}, Henry Milgrom et al., 2007]

\hypertarget{pmid_22348915}{P}hentermine hydrochloride is a noradrenergic sympathetic amine approved for decades by the U.S. Food and Drug Administration (FDA) at doses as high as 37.5 mg/day for the short-term treatment of obesity. Topiramate is a sulfamate-substituted monosaccharide marketed since 1996, and approved by the FDA for seizure disorders at doses up to 400 mg/day and for the prevention of migraine headaches at doses up to 100 mg/day. Clinical trial data suggest topiramate promotes weight loss. The prescribing information of neither agent describes adverse drug interactions with the other. The controlled-release formulation of phentermine and topiramate at low, medium and full doses (with full dose containing 15 mg of phentermine hydrochloride and 92 mg of topiramate) promotes weight reduction, with clinical trial data supporting improvement in adiposopathic consequences leading to metabolic diseases. Reported adverse events with this combination agent are as expected, based upon knowledge of the individual components. [\hyperlink{Phentermine Hydrochloride}{PMID: 22348915}, H E Bays et al., 2011]

\hypertarget{pmid_20527137}{O}nly a few corticosteroids for topical use have proven safe and effective in pediatric populations down to 3 months of age. The authors report the results of a study designed to assess the efficacy and safety of hydrocortisone butyrate (HCB) 0.1\% in lipocream (LCr) vehicle in infants and children. A total of 264 boys and girls 3 months to less than 18 years old, with stable, mild to moderate atopic dermatitis affecting at least 10\% body surface area applied HCB 0.1\% in LCr or LCr alone twice daily for up to 1 month without occlusion. Primary end-points included: percent of patients who achieved treatment success based on physician global assessments. Secondary endpoint included: difference in pruritus and Eczema Area and Severity Index (EASI) at day 29. Treatment was significant (P < 0.001) for HCB 0.1\% LCr over vehicle. No serious nor significant adverse events were reported. Results are representative of a short duration treatment for a chronic disease. HCB 0.1\% in LCr is more effective than its vehicle in pediatric populations down to 3 months of age without significant adverse events when used twice a day for up to 1 month. [\hyperlink{Phentermine Hydrochloride}{PMID: 20527137}, William Abramovits et al., ]

\hypertarget{pmid_28741653}{C}hloral hydrate is commonly used to sedate children for painless procedures. Children may recover more quickly after sedation with dexmedetomidine, which has a shorter half-life. We randomly allocated 196 children to chloral hydrate syrup 50 mg.kg [\hyperlink{Phentermine Hydrochloride}{PMID: 28741653}, V M Yuen et al., 2017] Recently there has been a resurgence in the utilization of ketamine, a unique anaesthetic, for emergency procedures requiring sedation. The purpose of the present study was to examine the safety and efficacy of ketamine for sedation in the treatment of children's fractures in the small clinic setup of rural Nepal. One hundred and fourteen children (average age, 5.3 years; range, twelve months to ten years and ten months) who underwent closed reduction of an isolated fracture or dislocation in the Orthopaedic \& Trauma Clinic at Janakpurdham were prospectively evaluated. Ketamine hydrochloride was administered intravenously (at a dose of less than two milligrams per kilogram of body weight) in ninety-nine of the patients and intramuscularly (at a dose of four milligrams per kilogram of body weight) in the other fifteen. Adequate fracture reduction was obtained in 111 of the children. Ninety-nine percent (sixty-eight) of the sixty-nine parents present during the reduction were pleased with the sedation and would allow it to be used again in a similar situation. Minor side effects included nausea (thirteen patients), emesis (eight of the thirteen patients with nausea), clumsiness (evident as ataxic movements in ten patients), and dysphonic reaction (one patient). No long-term sequelae were noted, and no patients had hallucinations or nightmares. Ketamine reliably, safely, and quickly provided adequate sedation to effectively facilitate the reduction of children's fractures at our institution. Therefore, it was ideal for small clinic in our setup. [\hyperlink{Phentermine Hydrochloride}{PMID: 28741653}, Ram Kewal Shah et al., 2003]

\hypertarget{pmid_18219837}{A}ntihistamines are an established first-line treatment for allergic rhinitis and are widely prescribed in infants for allergic symptoms. To establish the safety and tolerability of fexofenadine hydrochloride in children aged 6 months to 2 years in 2 studies (T/3001 and T/3002). Both studies had a multicenter, randomized, placebo-controlled design. Mean treatment duration was 8 days. Subjects were randomized (T/3001, n = 174; and T/3002, n = 219) to twice-daily fexofenadine hydrochloride, 15 or 30 mg, or placebo mixed with a standard vehicle. In the combined population, the incidence of treatment-emergent adverse events (TEAEs) was comparable between groups (placebo, 48.2\% [96/199]; fexofenadine hydrochloride, 15 mg, 40.0\% [34/85]; and fexofenadine hydrochloride, 30 mg, 35.2\% [38/108]). Vomiting was the most common TEAE (placebo, 13.6\%; fexofenadine hydrochloride, 15 mg, 14.1\%; and fexofenadine hydrochloride, 30 mg, 5.6\%). Most TEAEs were unrelated to study medication, as evaluated by investigators; those possibly related to study medication were mild or moderate in intensity. No clinical differences were seen between fexofenadine and placebo for vital signs, electrocardiographic results, or physical examination results. Fexofenadine hydrochloride, 15 or 30 mg, given for a mean duration of 8 days is well tolerated, with a good safety profile, in children aged 6 months to 2 years. [\hyperlink{Phentermine Hydrochloride}{PMID: 18219837}, Frank C Hampel et al., 2007]

\hypertarget{pmid_30463814}{D}exmendetomidine hydrochloride (DEX) is a new common adrenergic receptor agonist, which not only keeps children calm but also has analgesic effect. Dexmedetomidine hydrochloride will enable children to maintain the natural non-REM sleep, which can be stimulated sedation or language arousal. The aim of this study is to observe the sedative effect and adverse drug reactions of dexmedetomidine hydrochloride injection and propofol injection in MRI examination. In this study, no children in the experimental group were required to add sedative drugs, and 2 cases in the control group were treated with sedative drugs. In experimental group, it used dexmedetomidine hydrochloride as (1.64±0.91) g/kg; in control group, dosage of narcotic drugs as (5.26±1.82) g/kg, and the total complication rate of the children in the experimental group was lower than that of the control group (P<0.05). After returning to the ward, the doses of phenobarbital sedation were dexmedetomidine group (4.28±1.53) mg/kg and propofol group (6.40±1.71) mg/kg. There was significant difference between the two groups. The total complication rate in the experimental group was lower than that in the control group (P<0.05). The quality of MRI in the test group was significantly higher than that in the control group, which showed that dexmedetomidine hydrochloride could provide a satisfactory sedative effect in the MRI examination of children. To sum up, dexmedetomidine hydrochloride is a wide range of clinical applications. It is an effective drug for the maintenance of sedation in clinical disease treatment. It is flexible in the way of administration and with less adverse reactions. It is suitable for popularization and application in clinical practice. [\hyperlink{Phentermine Hydrochloride}{PMID: 30463814}, Zhendong Yang et al., 2018]

\hypertarget{pmid_2058184}{D}iphenhydramine hydrochloride is an antihistamine with anticholinergic properties that is frequently used both orally and topically for the temporary relief of pruritus. Significant systemic absorption may occur following topical administration of diphenhydramine in patients with varicella-zoster lesions. We describe three children with varicella-zoster infection (VZI) who developed bizarre behavior as well as visual and auditory hallucinations following topical applications of large amounts of diphenhydramine to the majority of skin surfaces. In two cases, oral diphenhydramine was also administered. Serum diphenhydramine concentrations approximated or exceeded those previously reported. In each case, a complete resolution of mental status abnormalities occurred within 24 hours after discontinuation of all diphenhydramine-containing products. Pharmacists and other health professionals should be aware of the potential toxicity of topical diphenhydramine in patients with VZI. [\hyperlink{Phentermine Hydrochloride}{PMID: 2058184}, C Y Chan et al., 1991]

\hypertarget{pmid_28292340}{T}he purpose of this study was to evaluate, using a randomized, double-blind methodology: (1) the safety of phentolamine mesylate (Oraverse) in accelerating the recovery of soft tissue anesthesia following the injection of two percent lidocaine plus 1:100,000 epinephrine in two- to five-year-olds; and (2) efficacy in four- to five-year-olds only. One hundred fifty pediatric dental patients underwent routine dental restorative procedures with two percent lidocaine plus 1:100,000 epinephrine with doses based on body weight. Phentolamine mesylate or a sham injection (two to one ratio) was then administered. Subjects were monitored for safety and, in four- to five-year-olds, for efficacy during the two-hour evaluation period. There were no significant differences in adverse events between the phentolamine and sham injections. Compared to sham, phentolamine was not associated with nerve injury, increased analgesic use, or abnormalities of the oral cavity. Phentolamine was associated with transient decreased blood pressure in some children. In four- and five-year-olds, phentolamine induced more rapid recovery of lip anesthesia by 48 minutes (P<0.0001). Phentolamine was well tolerated and safe in three- to five-year-olds; in four- to five-year-olds, a statistically significant more rapid recovery of lip sensation compared to sham injections was determined. [\hyperlink{Phentermine Hydrochloride}{PMID: 28292340}, Elliot V Hersh et al., 2017]

\hypertarget{pmid_6378158}{O}ne hundred fifty-two children were enrolled in a randomized, controlled clinical trial of the efficacy of phenylephrine hydrochloride nose drops or nasal spray in hastening the resolution of middle ear effusion. Children with persistent effusion were recruited for the study during a return visit two weeks after an episode of acute otitis media. Forty-six patients (30\%) dropped out of the study, many because they failed to tolerate the medication, especially the nose drops. Another 27 (18\%) had to be excluded because of intercurrent illness or systemic drug therapy. Among those children completing the study, rates of clinical and tympanometric cure during the following four weeks were similar in the drug and placebo groups. In view of the absence of documented clinical efficacy and the practical difficulties inherent in their administration, topical decongestants appear to have a limited role, if any, in treating children with persistent effusion. [\hyperlink{Phentermine Hydrochloride}{PMID: 6378158}, G F Hayden et al., 1984]

\hypertarget{pmid_16777373}{P}enequine hydrochloride, a novel anticholinergic agent, was developed as an effective treatment for organophosphorus intoxication (e.g., soman poisoning). The current study was performed to assess the potential pre- and post-natal toxicity of penequine hydrochloride in mice. Approximately 120 timed-pregnant mice were assigned to four dose groups (n=30 per group). Dams were exposed orally to 0, 2.5, 12.5, 62.5 mg/L penequine hydrochloride in drinking water from gestation day 6 to lactation day 21. The F1 generation mice, which were not exposed directly to penequine hydrochloride as pups or as adults, were bred to produce F2 generation fetuses for the fertility test of the F1 population. Various pre- and post-natal measurements, including neurobehavioral tests, were performed with the F0 and F1 mice. Among the significant findings were decreases in water consumption, viability, organ weights and delay of physical landmarks in 62.5 mg/L groups. With the exception of treatment-unrelated abnormality in surface righting reflex in the F1 generation, penequine hydrochloride did not produce any adverse effects at doses up to and including 12.5 mg/L (equal to 2.5 mg/kg/day in mice) that were at least 75 times of human therapeutic dosage. [\hyperlink{Phentermine Hydrochloride}{PMID: 16777373}, Zibo Zhang et al., 2006]

\hypertarget{pmid_15951862}{D}iagnostic and therapeutic procedures in children are made easier using sedation. However, there is no consensus about which drug should be used to achieve this. Furthermore, none of the drugs used for sedation are risk free. The aim of this work is to study sedation indications, effectiveness, and safety at our center. A prospective observational study conducted at the Pediatric Day Care Unit, King Fahad National Guard Hospital, Riyadh, Saudi Arabia. The study covered 17.5 weeks in 2 periods: May 9th 1999 to June 13th 1999 and October 31st 2001 to February 11th 2002. Children <12 years were included. Collected data included demographics, indication, drug dosing and outcome. Data were reported as mean +/- SD. We included 148 patients, age 38 +/- 30 months. Adequate sedation was achieved in 79\% after initial chloral hydrate (CH) dose of 56.9 +/- 9.3 mg/kg, in 95\% after adding 18.5 +/- 6.4 mg/kg CH and in 96\% after adding second drug. Compared to nonrespondents, first CH dose respondents were younger and lower in weight. The CH side effects were few and mild. Chloral hydrate is a safe and effective agent for sedation in children with an age and weight dependent response. [\hyperlink{Phentermine Hydrochloride}{PMID: 15951862}, Omar M Hijazi et al., 2005]

\hypertarget{pmid_28275979}{S}edation is often required for children undergoing diagnostic procedures. Chloral hydrate has been one of the sedative drugs most used in children over the last 3 decades, with supporting evidence for its efficacy and safety. Recently, chloral hydrate was banned in Italy and France, in consideration of evidence of its carcinogenicity and genotoxicity. Dexmedetomidine is a sedative with unique properties that has been increasingly used for procedural sedation in children. Several studies demonstrated its efficacy and safety for sedation in non-painful diagnostic procedures. Dexmedetomidine's impact on respiratory drive and airway patency and tone is much less when compared to the majority of other sedative agents. Administration via the intranasal route allows satisfactory procedural success rates. Studies that specifically compared intranasal dexmedetomidine and chloral hydrate for children undergoing non-painful procedures showed that dexmedetomidine was as effective as and safer than chloral hydrate. For these reasons, we suggest that intranasal dexmedetomidine could be a suitable alternative to chloral hydrate. [\hyperlink{Phentermine Hydrochloride}{PMID: 28275979}, Giorgio Cozzi et al., 2017]

\hypertarget{pmid_18254579}{P}heochromocytoma in children shows much worse complications than that in the adult patients. An 11-year-old girl was transferred to our emergency room after suffering from headache, dizziness, cold sweating and palpitation for 3 days. Severe hypertension, remarkable blood pressure fluctuation between 260/160 and 65/50 mmHg, decrease of cardiac contractility, as well as abnormal electrocardiogram findings including ST-T segment elevation and QT interval prolongation were noted soon after admission. Later, a 4x4.5x2.5 cm tumor in the right adrenal gland area was found by computed axial tomogram study. Assessment of the urine catecholamine metabolites showed high levels of vanillylmandelic acid, normetanephrine and norepinephrine indicating an active adrenal pheochromocytoma produced mainly norepinephrine. Although several antihypertensive drugs were used, ventricular tachycardia and Torsade de pointe still occurred on her for 3 times, each was preceded by a period of blood pressure fluctuation and burst out concomitantly at the peak of a hypertension crisis. From this case, we found that when the specific alpha-blocker like phenoxybenzamine or phentolamine was not available to us, labetalol by continuous intravenous infusion was the only effective drug to protect the patient from attacks of hypertensive crisis and ventricular tachycardia. Her right adrenal gland was resected smoothly when BP was well under control. Histological examination showed the adrenal medulla was full of pheochromocytoma cells. [\hyperlink{Phentermine Hydrochloride}{PMID: 18254579}, Yu-Chih Huang et al., ]

\hypertarget{pmid_37655364}{F}exofenadine hydrochloride (HCl) is a second-generation, nonsedating, histamine H1-receptor antagonist used to manage seasonal allergic rhinitis and chronic idiopathic urticaria. A new oral pediatric suspension of fexofenadine HCl has been developed, with the preservative potassium sorbate replacing parabens. The objective of this phase 1 single-center, open-label, randomized, 2-treatment, full-replicated, 4-period, 2-sequence crossover study in healthy adult volunteers was to assess the bioequivalence of 30 mg of the new oral suspension of fexofenadine HCl (test) versus 30 mg of the marketed pediatric oral suspension of fexofenadine HCl (reference). The replicate design was based on the high intra-individual variability of fexofenadine (>30\% on C [\hyperlink{Phentermine Hydrochloride}{PMID: 37655364}, Clemence Rauch et al., 2023] The aim of this technical note is to show that ketamine hydrochloride anesthesia, owing to the preservation of muscle tone, enables one to safely and comfortably carry out gaseous encephalography on a simple radiological table in the toddler or child. However the authors very strictly select the indications for this investigation. Whenever the child's clinical condition leads one to suspect intracranial hypertension and/or a cerebral tumor, they think it more prudent, owing to the vasopressor effect of ketamine hydrochloride and a possible elevation in the cerebrospinal fluid pressure, to transfer the child to a specialized neuroradiological center, where the investigations would be carried out under the best technical conditions, and close to a neurosurgical unit which is capable of intervening rapidly in case of complications. The authors voluntarily limit their indications to children suffering from psychomotor retardation, epilepsy or neurological disorders which make one suspect a congenital malformation. [\hyperlink{Phentermine Hydrochloride}{PMID: 37655364}, C Fauré et al., 1975]

\hypertarget{pmid_2295577}{F}luoxetine hydrochloride is the first selective serotonin uptake inhibitor introduced commercially in the United States. This report describes preliminary clinical experience with fluoxetine in 10 children and adolescents, aged 8 to 15 years, with primary obsessive compulsive disorder (OCD) or Tourette's syndrome (TS) plus OCD. In general, fluoxetine, which was administered from 4 to 20 weeks at a dosage of 10 or 40 mg per day, was well tolerated. Adverse effects included behavioral agitation/activation in four patients and mild gastrointestinal symptoms in two patients. No abnormalities were noted in the seven children who had follow-up EKGs. Five of the 10 patients (50\%) were considered responders; their obsessive-compulsive symptoms decreased substantially during treatment with fluoxetine. Responder rates were similar in the primary OCD (two of four, 50\%) and TS + OCD (three of six, 50\%) groups. In conclusion, short-term fluoxetine administration appears to be safe in children and adolescents. Placebo-controlled trials are needed to further assess the efficacy of fluoxetine. [\hyperlink{Phentermine Hydrochloride}{PMID: 2295577}, M A Riddle et al., 1990]

\hypertarget{pmid_24627951}{T}o determine the safety and efficacy of high-dose oral chloral hydrate for pediatric ophthalmic procedures. This study is a retrospective review of a quality audit of pediatric sedation for ophthalmic evaluation and imaging performed at King Khaled Eye Specialist Hospital between January 1 and December 31, 2011, in children aged 1 month to 6 years. Three hundred fifty-eight of 380 (94.2\%) sedation procedures were successful after a single dose of chloral hydrate, with 356 of 380 (93.7\%) children sedated within 45 minutes of the first dose. The total success rate of the sedation procedure increased to 97.9\% (372 of 380) when a second dose was administered. Children adequately sedated after a single dose of chloral hydrate were on average younger and weighed less than children who required additional doses. No major adverse events were documented. The use of chloral hydrate sedation for ophthalmic evaluation and imaging was safe and effective in this patient population with a high rate of procedure completion. [\hyperlink{Phentermine Hydrochloride}{PMID: 24627951}, Michelle E Wilson et al., ]

\hypertarget{pmid_23332206}{T}o determine whether the adrenoceptor agonist, ephedrine hydrochloride, is an effective treatment for resistant non-neurogenic daytime urinary incontinence in children. From 2000 to 2010, eighteen children with resistant non-neurogenic daytime urinary incontinence were treated with oral ephedrine hydrochloride at our institution. Sixteen were female and two were male. Median age at treatment was 12 years (range 5-15 years). Two children had spina bifida occulta. There were no other co-morbidities. Multiple anticholinergics were prescribed and dose maximized to support a bladder and bowel training programme, without achieving continence in this resistant group of children. Pre-treatment urodynamics were normal in 10, but revealed an open bladder neck in 8 patients. None showed detrusor over-activity. Oral ephedrine hydrochloride was started at 7.5 mg or 15 mg twice daily and titrated up to a maximum of 30 mg four times daily according to response. Median follow-up was 7 years (range 6-8 years). Seventeen children (94\%) reported improvement in symptoms and six (33\%) achieved complete urinary continence. All patients maintained compliant bladders on post-treatment urodynamics. Seven of the 8 previously open bladder necks were closed. No patients reported any significant side effects. Patients with open bladder necks on pre-treatment urodynamics were more likely to show a full response to ephedrine (odds ratio 15; 95\% CI 1.2-185.2). Oral ephedrine hydrochloride is an effective treatment for carefully selected children with resistant non-neurogenic daytime urinary incontinence. [\hyperlink{Phentermine Hydrochloride}{PMID: 23332206}, Neil Featherstone et al., 2013]

\hypertarget{pmid_6660444}{T}he article reports on the paediatric-anaesthesiological treatment of 6 phaeochromocytomas in 5 children who were 8 to 16 years of age. Therapeutic recommendations for the perioperative medication of infantile phaeochromocytoma patients are involved. The therapeutic aim of this study was the management of the effects of phaeochromocytoma before and after extirpation of the tumour, the effect of the phaeochromocytoma being of an alpha-adrenergic and beta-adrenergic cardiovascular nature and transmitted by catecholamines. Preoperative stabilization of blood pressure by means of the alpha-blocker phenoxybenzamine and a subsequent intraoperative, controlled reduction of blood pressure by means of sodium nitroprusside were found to be an effective, safe and easily appreciated therapeutic concept for the perioperative care of paediatric phaeochromocytoma patients. Considerable individual differences in dose an duration of the necessary preoperative phenoxybenzamine administration rendered ward control of therapy recommendable. The risk of complete alpha-sympathicolysis by additive drug effects during premedication and induction of anaesthesia, had to be taken into consideration for conducting phenoxybenzamine therapy. Additional administration of the beta-blocker pindolol successfully controlled the intraoperatively manifested tachycardial heart rhythm phases without provoking any complicating arrhythmias. During the entire perioperative treatment of the patients it is mandatory to ensure sufficient substitution of intravascular volume to prevent hypotensive complications. Our patients did not need any cardiac and sympathicomimetic drugs as postoperative administration. None of the patients had any perioperative complications worth mentioning. [\hyperlink{Phentermine Hydrochloride}{PMID: 6660444}, M Abel et al., 1983]

\hypertarget{pmid_15517550}{F}exofenadine hydrochloride is a non-sedating antihistamine that is used in the treatment of symptoms associated with seasonal allergic rhinitis and chronic idiopathic urticaria. A pooled analysis of pharmacokinetic data from children 6 months to 12 years of age and adults was conducted to identify the dose(s) in children that produce exposures comparable to those in adults for the treatment of seasonal allergic rhinitis. The pharmacokinetic parameter database included peak and overall exposure data from 269 treatment exposures from 136 adult subjects, and 90 treatment exposures from 77 pediatric allergic rhinitis patients. The data were pooled and analysed using NONMEM software, version 5.0. A covariate model based on body weight and age and a power function model based on body weight were identified as appropriate models to describe the variability in fexofenadine oral clearance and peak concentration, respectively. Individual oral clearance estimates were on average 44\%, 36\% and 61\% lower in children 6 to 12 years (n=14), 2 to 5 years (n=21), and 6 months to 2 years (n=42), respectively, compared with adults. Trial simulations (n=100) were carried out based on the final pharmacostatistical models and parameter estimates to identify the appropriate dose(s) in children relative to the marketed dose of 60 mg fexofenadine hydrochloride in adults. The trials were designed as crossover studies in 18 subjects comprising various potential dosing regimens with and without weight stratification. Pharmacokinetic parameter variability was assumed to have a log-normal distribution. Individual weights and ages were simulated using mean (SD) estimates derived from the studies used in this analysis and proportional measurement/model mis-specification errors derived from the analysis were incorporated into the simulation. The results indicated that a 30 mg dose of fexofenadine hydrochloride administered to children 1 to 12 years of age and weighing >10.5 kg and a 15 mg dose administered to children 6 months and older and weighing <or=10.5 kg produces exposures similar to those seen with the 60 mg dose in adults. [\hyperlink{Phentermine Hydrochloride}{PMID: 15517550}, Rajesh Krishna et al., 2004]

\hypertarget{pmid_439264}{K}etamine hydrochloride was used successfully to produce detumescence in 7 children undergoing hypospadias repair. There have been no complications or side effects. We recommend ketamine hydrochloride as a useful adjunct in managing corporeal erections during urologic procedures. [\hyperlink{Phentermine Hydrochloride}{PMID: 439264}, J R Pietras et al., 1979]

\section*{Selpercatinib}
\subsection*{Result}
\subsubsection*{Answer}

Unknown

\subsubsection*{{Explanation}}
\hypertarget{Selpercatinib}
A review of the available abstracts reveals that Selpercatinib has been studied extensively in adults and some general populations for various cancers, particularly RET-altered non-small cell lung cancer (NSCLC) and medullary thyroid cancer. However, none of the abstracts specifically report on targeted safety studies of Selpercatinib in children or pediatric populations.

- Several abstracts describe the efficacy and safety of Selpercatinib in adult or general populations, including large phase 1/2 and phase 3 trials, but do not mention pediatric patients or provide age-specific safety data for children [\hyperlink{pmid_35969032}{PMID: 35969032}, Lori J Wirth et al., 2022; \hyperlink{pmid_34088726}{PMID: 34088726}, Vivek Subbiah et al., 2021; \hyperlink{pmid_36975442}{PMID: 36975442}, Han-Sang Baek et al., 2023; \hyperlink{pmid_36122315}{PMID: 36122315}, Alexander Drilon et al., 2023; \hyperlink{pmid_37870973}{PMID: 37870973}, Caicun Zhou et al., 2023; \hyperlink{pmid_37315261}{PMID: 37315261}, Ying Cheng et al., 2023; \hyperlink{pmid_36572992}{PMID: 36572992}, Vanessa Quang Nguyen et al., 2023].
- None of the abstracts mention inclusion of children or adolescents, nor do they provide results from pediatric cohorts or age-stratified safety analyses.
- There are no abstracts describing a targeted safety study of Selpercatinib in children (e.g., under 18 years), nor any that affirm its safety or unsafety in this population.

Therefore, based strictly on the abstracts available, the safety of Selpercatinib in children is unknown, as no targeted pediatric safety studies are reported.

\subsection*{Abstracts}
\hypertarget{pmid_35969032}{S}elpercatinib is a first-in-class, highly selective and potent, central nervous system-active  [\hyperlink{Selpercatinib}{PMID: 35969032}, Lori J Wirth et al., 2022] We report the intracranial efficacy of selpercatinib, a highly potent and selective RET inhibitor, approved in the United States for  In the global phase 1/2 LIBRETTO-001 trial (NCT03157128) in advanced  Eighty patients with NSCLC had brain metastases at baseline. Patients were heavily pretreated (median = 2 systemic therapies, range = 0-10); 56\% of patients received ≥1 course of intracranial radiation (14\% whole brain radiotherapy, 45\% stereotactic radiosurgery). Among 22 patients with measurable intracranial disease at baseline, intracranial ORR was 82\% [95\% confidence interval (CI), 60-95], including 23\% with complete responses. Among all intracranial responders (measurable and nonmeasurable,  Selpercatinib has robust and durable intracranial efficacy in patients with  [\hyperlink{Selpercatinib}{PMID: 35969032}, Vivek Subbiah et al., 2021] Recently, selpercatinib, a highly selective inhibitor of  [\hyperlink{Selpercatinib}{PMID: 35969032}, Han-Sang Baek et al., 2023] Selpercatinib, a first-in-class, highly selective, and potent CNS-active RET kinase inhibitor, is currently approved for the treatment of patients with  Patients were enrolled to LIBRETTO-001, a phase I/II, single-arm, open-label study of selpercatinib in patients with  In treatment-naive patients, the ORR was 84\% (95\% CI, 73 to 92); 6\% achieved complete responses (CRs). The median DoR was 20.2 months (95\% CI, 13.0 to could not be evaluated); 40\% of responses were ongoing at the data cutoff (median follow-up of 20.3 months). The median PFS was 22.0 months; 35\% of patients were alive and progression-free at the data cutoff (median follow-up of 21.9 months). In platinum-based chemotherapy pretreated patients, the ORR was 61\% (95\% CI, 55 to 67); 7\% achieved CRs. The median DoR was 28.6 months (95\% CI, 20.4 to could not be evaluated); 49\% of responses were ongoing (median follow-up of 21.2 months). The median PFS was 24.9 months; 38\% of patients were alive and progression-free (median follow-up of 24.7 months). Of 26 patients with measurable baseline CNS metastasis by the independent review committee, the intracranial ORR was 85\% (95\% CI, 65 to 96); 27\% were CRs. In the full safety population (n = 796), the median treatment duration was 36.1 months. The safety profile of selpercatinib was consistent with previous reports. In a large cohort with extended follow-up, selpercatinib continued to demonstrate durable and robust responses, including intracranial activity, in previously treated and treatment-naive patients with  [\hyperlink{Selpercatinib}{PMID: 35969032}, Alexander Drilon et al., 2023] Selpercatinib is a targeted, FDA-approved, oral, small-molecule inhibitor for the treatment of rearranged during transfection (RET) proto-oncogene mutation-positive cancer. Using genetically modified mouse models, we investigated the roles of the multidrug efflux transporters ABCB1 and ABCG2, the OATP1A/1B uptake transporters, and the drug-metabolizing CYP3A complex in selpercatinib pharmacokinetics. Selpercatinib was efficiently transported by hABCB1 and mAbcg2, but not hABCG2, and was not a substrate of human OATP1A2, -1B1 or -1B3 in vitro. In vivo, brain and testis penetration were increased by 3.0- and 2.7-fold in  [\hyperlink{Selpercatinib}{PMID: 35969032}, Yaogeng Wang et al., 2021] Selpercatinib (SLP; brand name Retevmo [\hyperlink{Selpercatinib}{PMID: 35969032}, Mohamed W Attwa et al., 2023] Ceftriaxone is widely used in children in the treatment of sepsis. However, concerns have been raised about the safety of ceftriaxone, especially in young children. The aim of this review is to systematically evaluate the safety of ceftriaxone in children of all age groups. MEDLINE, PubMed, Cochrane Central Register of Controlled Trials, EMBASE, CINAHL, International Pharmaceutical Abstracts and adverse drug reaction (ADR) monitoring systems will be systematically searched for randomised controlled trials (RCTs), cohort studies, case-control studies, cross-sectional studies, case series and case reports evaluating the safety of ceftriaxone in children. The Cochrane risk of bias tool, Newcastle-Ottawa and quality assessment tools developed by the National Institutes of Health will be used for quality assessment. Meta-analysis of the incidence of ADRs from RCTs and prospective studies will be done. Subgroup analyses will be performed for age and dosage regimen. Formal ethical approval is not required as no primary data are collected. This systematic review will be disseminated through a peer-reviewed publication and at conference meetings. CRD42017055428. [\hyperlink{Selpercatinib}{PMID: 35969032}, Linan Zeng et al., 2017]

\hypertarget{pmid_37870973}{S}elpercatinib, a highly selective potent and brain-penetrant RET inhibitor, was shown to have efficacy in patients with advanced  In a randomized phase 3 trial, we evaluated the efficacy and safety of first-line selpercatinib as compared with control treatment that consisted of platinum-based chemotherapy with or without pembrolizumab at the investigator's discretion. The primary end point was progression-free survival assessed by blinded independent central review in both the intention-to-treat-pembrolizumab population (i.e., patients whose physicians had planned to treat them with pembrolizumab in the event that they were assigned to the control group) and the overall intention-to-treat population. Crossover from the control group to the selpercatinib group was allowed if disease progression as assessed by blinded independent central review occurred during receipt of control treatment. In total, 212 patients underwent randomization in the intention-to-treat-pembrolizumab population. At the time of the preplanned interim efficacy analysis, median progression-free survival was 24.8 months (95\% confidence interval [CI], 16.9 to not estimable) with selpercatinib and 11.2 months (95\% CI, 8.8 to 16.8) with control treatment (hazard ratio for progression or death, 0.46; 95\% CI, 0.31 to 0.70; P<0.001). The percentage of patients with an objective response was 84\% (95\% CI, 76 to 90) with selpercatinib and 65\% (95\% CI, 54 to 75) with control treatment. The cause-specific hazard ratio for the time to progression affecting the central nervous system was 0.28 (95\% CI, 0.12 to 0.68). Efficacy results in the overall intention-to-treat population (261 patients) were similar to those in the intention-to-treat-pembrolizumab population. The adverse events that occurred with selpercatinib and control treatment were consistent with those previously reported. Treatment with selpercatinib led to significantly longer progression-free survival than platinum-based chemotherapy with or without pembrolizumab among patients with advanced  [\hyperlink{Selpercatinib}{PMID: 37870973}, Caicun Zhou et al., 2023] The aim of this study was to review retrospectively the safety and efficacy of a paediatric sedation protocol in a district general hospital radiology department. 256 children attended for CT scanning over a 40-month period. 40 children required sedation and were given quinalbarbitone. 34 (85\%) of this group were adequately sedated. Of the children who received quinalbarbitone, 35 were under 5 years of age. 32 of this group (91.4\%) were adequately sedated. Failures in children under 5 years were all caused by problems with administration whilst failures in the older children were due to paradoxical excitement. No problems with respiratory depression were encountered. Sedation can be safely performed in a district general hospital radiology department if a structured protocol is adhered to. Quinalbarbitone is a safe, effective oral agent in children under the age of 5 years. [\hyperlink{Selpercatinib}{PMID: 37870973}, J H Simpson et al., 2000]

\hypertarget{pmid_10440436}{P}ediatric skin and skin structure infections are often polymicrobial and require empiric therapy effective against pathogens that may be resistant to many antimicrobial agents. The present study tested the efficacy and safety of a parenteral beta-lactam/beta-lactamase inhibitor combination, ampicillin/sulbactam, and a beta-lactamase-stable cephalosporin, cefuroxime, in serious pediatric skin and skin structure infections requiring hospitalization and parenteral antimicrobial therapy. This was a multicenter, randomized, prospective, comparative open label trial that enrolled patients 3 months through 11 years of age. Patients received 150 to 300 mg/kg/day ampicillin/sulbactam in equally divided intravenous doses every 6 h. Cefuroxime was given in a dosage of 50 to 100 mg/kg/day either intravenously or intramuscularly in equally divided doses every 6 or 8 h. Maximum treatment was not to exceed 14 days. Patients could receive subsequent oral antimicrobial treatment at the investigator's discretion. At final evaluation for clinical efficacy, 78.0\% (n = 46) of the 59 evaluable patients who received ampicillin/sulbactam were cured and 22.0\% (n = 13) were improved. The respective values for the 39 evaluable patients treated with cefuroxime were 76.9\% (n = 30) and 23.1\% (n = 9). At the end of treatment all pathogens were eradicated from 93.2\% (n = 55) of 59 patients treated with ampicillin/sulbactam and from 100\% of 39 who received cefuroxime. There were no significant differences between treatments in clinical or bacteriologic efficacy. Both ampicillin/sulbactam and cefuroxime were well-tolerated. Both ampicillin/sulbactam and cefuroxime provide safe and effective parenteral antibiotic therapy in pediatric patients with serious skin and skin structure infections. [\hyperlink{Selpercatinib}{PMID: 10440436}, P H Azimi et al., 1999]

\hypertarget{pmid_37815398}{A}ntimicrobial resistance increases infection morbidity in both adults and children, necessitating the development of new therapeutic options. Telavancin, an antibiotic approved in the United States for certain bacterial infections in adults, has not been examined in pediatric patients. The objectives of this study were to evaluate the short-term safety and pharmacokinetics (PK) of a single intravenous infusion of telavancin in pediatric patients. Single-dose safety and PK of 10 mg/kg telavancin was investigated in pediatric subjects >12 months to ≤17 years of age with known or suspected bacterial infection. Plasma was collected up to 24-h post-infusion and analyzed for concentrations of telavancin and its metabolite for noncompartmental PK analysis. Safety was monitored by physical exams, vital signs, laboratory values, and adverse events following telavancin administration. Twenty-two subjects were enrolled: 14 subjects in Cohort 1 (12-17 years), 7 subjects in Cohort 2 (6-11 years), and 1 subject in Cohort 3 (2-5 years). A single dose of telavancin was well-tolerated in all pediatric age cohorts without clinically significant effects. All age groups exhibited increased clearance of telavancin and reduced exposure to telavancin compared to adults, with mean peak plasma concentrations of 58.3 µg/mL (Cohort 1), 60.1 µg/mL (Cohort 2), and 53.1 µg/mL (Cohort 3). A 10 mg/kg dose of telavancin was well tolerated in pediatric subjects. Telavancin exposure was lower in pediatric subjects compared to adult subjects. Further studies are needed to determine the dose required in phase 3 clinical trials in pediatrics. [\hyperlink{Selpercatinib}{PMID: 37815398}, John S Bradley et al., 2023]

\hypertarget{pmid_37315261}{S}elpercatinib, a highly selective, potent RET inhibitor with CNS activity, demonstrated sustained antitumor responses and intracranial activity in patients with  We included patients with advanced NSCLC and brain metastasis with a centrally confirmed  In total, 8/26 (31\%) patients were included: 1/8 (13\%) had previous brain surgery but no previous systemic therapy and 3/8 (38\%) had received brain radiotherapy. Best overall systemic response was partial response (PR) in 6/8 patients (75\%) and stable disease (SD) in 2/8 (25\%). Among patients with measurable baseline CNS lesions, 4/5 (80\%) achieved a confirmed intracranial response (3/5 PRs and 1/5 complete response [CR]). The best overall intracranial response was CR in 3/8 (38\%), PR in 3/8 (38\%), and SD in 1/8 (13\%) and nonprogressive disease/non-CR in 1/8 (13\%); 2/8 patients (25\%) had CNS-only disease progression. The duration of treatment was 2.8-24.0 months, and 5/8 patients (63\%) had treatment ongoing at DCO. Of 8 patients, 5 (63\%) had grade ≥3 treatment-related adverse events (TRAEs) requiring dose modification. There were no treatment discontinuations because of TRAEs. Selpercatinib demonstrated clinically meaningful and durable intracranial activity in Chinese patients with brain metastases from  [\hyperlink{Selpercatinib}{PMID: 37315261}, Ying Cheng et al., 2023] In the US, 6\% sulfur in petrolatum has been the most frequently administered treatment for infantile scabies. It appears to be safe but there is no literature containing a large series of patients on which to base that determination. In the UK, benzyl benzoate is the approved product. Benzyl benzoate is rarely used in the US at the present time. 5\% Permethrin is an excellent substitute and has many advantages. It appears to be quite safe in infants, although it is more expensive than other products. It remains present on the skin for several days, therefore protecting against reinfestation. Ivermectin is a systemic drug which is assumed to be safe in infants, although it requires repeated doses and does not protect against reinfestation. In the opinion of the author, 5\% permethrin is the best treatment for scabies in infants and young children. [\hyperlink{Selpercatinib}{PMID: 37315261}, Mervyn L Elgart et al., 2003]

\hypertarget{pmid_37796614}{T}his study aimed to investigate the safety and efficacy of lenvatinib in real-world settings, including patients excluded from the REFLECT trial, a phase III trial that compared lenvatinib with sorafenib. This multicenter, nonrandomized, open-label prospective study was conducted at 10 medical facilities in Japan (jRCTs031190017). Eligible patients had advanced hepatocellular carcinoma (HCC) and were suitable for lenvatinib therapy. The study included patients with high tumor burden (with >50\% intrahepatic tumor volume, main portal vein invasion, or bile duct invasion), Child-Pugh B status, and receiving lenvatinib as second-line therapy following atezolizumab plus bevacizumab. From December 2019 to September 2021, 59 patients were analyzed (47 and 12 patients with Child-Pugh A and B, respectively). In patients with Child-Pugh A, the frequency of aspartate aminotransferase elevation was high (72.7\%) in the high-burden group. No other significant ad verse events (AE) were observed even in second-line treatment. However, patients with Child-Pugh B had high incidence of grade ≥3 AE (100.0\%) and high discontinuation rates caused by AE (33.3\%) compared with patients with Child-Pugh A (80.9\% and 17.0\%, respectively). Median progression-free survival was 6.4 and 2.5 months and median overall survival was 19.7 and 4.1 months in Child-Pugh A and B, respectively. Lenvatinib plasma concentration was higher in patients with Child-Pugh B on days 8 and 15 and correlated with dose modifications and lower relative dose intensity. Lenvatinib is safe and effective for advanced HCC in patients with Child-Pugh A, even with high tumor burden. However, it carries a higher risk of AE and may not provide adequate efficacy for patients with Child-Pugh B status. [\hyperlink{Selpercatinib}{PMID: 37796614}, Kazufumi Kobayashi et al., 2023]

\hypertarget{pmid_36572992}{S}elpercatinib and pralsetinib are new targeted therapies used to treat patients with non-small cell lung cancer (NSCLC) due to RET gene rearrangements. The objective of this article is to review selpercatinib and pralsetinib in the context of RET-fusion-positive NSCLC. The pivotal LIBRETTO-001 and ARROW trials were evaluated regarding the use of selpercatinib and pralsetinib as treatment for RET-fusion-positive NSCLC. Comparative studies, review articles and current studies on selpercatinib and pralsetinib in RET-fusion-positive NSCLC were searched on pubmed.org and scholar.google.com using "selpercatinib," "pralsetinib," and "NSCLC" as keywords. Product monographs were searched on google.ca and uptodate.com using the keywords "selpercatinib," "pralsetinib," and/or "monograph." Selpercatinib and pralsetinib are orally administered highly selective RET inhibitors approved by the FDA following the accelerated approvals granted due to the pivotal LIBRETTO-001 and ARROW trials which evaluated selpercatinib and pralsetinib, respectively. Both drugs have shown efficacy for brain metastases and are primarily metabolized by CYP3A4 through hepatic metabolism. The most common grade 3 or 4 adverse effects of selpercatinib were hypertension, increased ALT level, and increased AST level while for pralsetinib, it was neutropenia, hypertension, and anemia. The safety profile shows similarities in severity and tolerability but additional monitoring for QT prolongation in patients on selpercatinib is recommended, compared to the risks of interstitial lung disease or pneumonitis for patients on pralsetinib. Overall, the increased use of selpercatinib and pralsetinib has led to the implementation of these drugs in the clinical practice of healthcare professionals such as pharmacists. [\hyperlink{Selpercatinib}{PMID: 36572992}, Vanessa Quang Nguyen et al., 2023]

\hypertarget{pmid_36715265}{S}epsis and meningitis in children may present with different clinical features and a wide range of values of inflammatory markers. The aim of this study was to identify the prognostic value of clinical features and biomarkers in children with sepsis and bacterial meningitis in the emergency department (ED). We carried out a single-center, retrospective, observational study on 194 children aged 0 to 14 years with sepsis and bacterial meningitis admitted to the pediatric ED of a tertiary children's hospital through 12 years. Among epidemiological and early clinical features, age older than 12 months, capillary refill time greater than 3 seconds, and oxygen blood saturation lower than 90\% were significantly associated with unfavorable outcomes, along with neurological signs ( P < 0.05). Among laboratory tests, only procalcitonin was an accurate and early prognostic biomarker for sepsis and bacterial meningitis in the ED, both on admission and after 24 hours. Procalcitonin cut-off value on admission for short-term complications was 19.6 ng/mL, whereas the cut-off values for long-term sequelae were 19.6 ng/mL on admission and 41.9 ng/mL after 24 hours, respectively. The cut-off values for mortality were 18.9 ng/mL on admission and 62.4 ng/mL at 24 hours. Procalcitonin, along with clinical evaluation, can guide the identification of children at higher risk of morbidity and mortality, allowing the most appropriate monitoring and treatment. [\hyperlink{Selpercatinib}{PMID: 36715265}, Emanuele Castagno et al., 2023]

\hypertarget{pmid_3026018}{N}ine children with osteomyelitis and/or septic arthritis were treated sequentially with parenteral sulbactam/ampicillin and oral sultamicillin. Causative pathogens were identified in six cases; all were susceptible to the combination of ampicillin and sulbactam. The mean duration of parenteral therapy was 7.1 days (6-11 days), and the average hospital stay was 10.3 days (6-18 days). Peak serum bactericidal titers of greater than or equal to 1:8 were achieved in all patients during parenteral therapy; only one child receiving oral therapy did not achieve a titer of greater than or equal to 1:4. At follow-up, all of the children were cured clinically and there was no evidence of relapse. Adverse reactions to oral therapy were minimal. The regimen of parenteral sulbactam/ampicillin and oral sultamicillin used sequentially is effective and safe for the treatment of skeletal infections in children. The use of this approach significantly reduced the duration of hospitalization. [\hyperlink{Selpercatinib}{PMID: 3026018}, S C Aronoff et al., ]

\hypertarget{pmid_17632669}{T}o study the behavior of procalcitonin and to verify whether it can be used to differentiate children with septic conditions. Children were enrolled prospectively from among those aged 28 days to 14 years, admitted between January 2004 and December 2005 to the pediatric intensive care unit at UNESP with sepsis or septic shock. The children were classified as belonging to one of two groups: the sepsis group (SG; n = 47) and the septic shock group (SSG; n = 43). Procalcitonin was measured at admission (T0) and again 12 hours later (T12h), and the results classed as: < 0.5 ng/mL = sepsis unlikely; >/= 0.5 to < 2 = sepsis possible; >/= 2 to < 10 = systemic inflammation and >/= 10 = septic shock. At T0 there was a greater proportion of SSG patients than SG patients in the highest PCT class [SSG: 30 (69.7\%) > SG: 14 (29.8\%); p < 0.05]. The proportion of SSG patients in this highest PCT class was greater than in all other classes (>/= 10 = 69.7\%; >/= 2 to < 10 = 18.6\%; >/= 0.5 to < 2 = 11.6\%; < 0.5 = 0.0\%; p < 0.05). The behavior of procalcitonin at T12h was similar to at T0. The pediatric risk of mortality (PRISM) scores for the SSG patients in the highest procalcitonin class were more elevated than for children in the SG [SSG: 35.15 (40.5-28.7) vs. SG: 18.6 (21.4-10.2); p < 0.05]. Procalcitonin allows sepsis to be differentiated from septic shock, can be of aid when diagnosing septic conditions in children and may be related to severity. [\hyperlink{Selpercatinib}{PMID: 17632669}, José R Fioretto et al., ]

\hypertarget{pmid_20234350}{C}elecoxib is approved as an adjunctive chemopreventive agent in adults with familial adenomatous polyposis (FAP). Its safety and efficacy for colorectal polyps in children is unknown. We evaluated the short-term (3 months) safety and preliminary efficacy of celecoxib in children with FAP. This was a phase I, dose-escalation trial, with three successive cohorts of six children. Children of ages 10-14 years with APC gene mutations and/or adenomas with a family history of FAP were studied at M.D. Anderson Cancer Center and the Cleveland Clinic. Colonoscopy was performed at baseline and month 3. Random assignment was in a 2:1 generic:placebo ratio, escalating from cohort 1 (4 mg/kg/day) to cohort 2 (8 mg/kg/day) to cohort 3 (16 mg/kg/day). Adherence and adverse event (AE) monitoring was conducted at 2-week intervals during drug administration. Safety profile, difference in number, and percent change in colorectal polyps were compared among the four treatments (placebo and the three dose-escalation groups). Eighteen subjects completed drug dosing and both colonoscopies. Median age was 12.3 years (56\% female). No clinically meaningful differences in AEs were seen between placebo subjects and subjects at any of the three celecoxib doses. Median polyp count at baseline was 31. There was a 39.1\% increase in the number of polyps in placebo subjects at month 3, whereas in the highest dose celecoxib group, 16 mg/kg/day, a 44.2\% reduction was seen (P=0.01). Celecoxib at a dose of 16 mg/kg/day, corresponding to the adult dose of 400 mg BID, is safe, well tolerated, and significantly reduced the number of colorectal polyps in children with FAP. [\hyperlink{Selpercatinib}{PMID: 20234350}, Patrick M Lynch et al., 2010]

\hypertarget{pmid_1729302}{A}dequate sedation remains one of the most important parts of performing high quality cross-sectional imaging in children. This is a noncomparative retrospective analysis of existing sedation protocols used in 1,158 children between the ages of 1 day and 18 years, checking for safety and efficacy. The most commonly used drugs were chloral hydrate (60-120 mg/kg) by mouth for infants less than 18 months and intravenous Nembutal (2-6 mg/kg) for older children. Sedation was successful in 97\% of patients. [\hyperlink{Selpercatinib}{PMID: 1729302}, A M Hubbard et al., ]

\hypertarget{pmid_33903938}{S}elumetinib (ARRY-142886) is a potent, selective, MEK1/2 inhibitor approved in the US for the treatment of children (≥ 2 years) with neurofibromatosis type 1 (NF1) and symptomatic, inoperable plexiform neurofibromas (PN). We characterized population pharmacokinetics (PK) of selumetinib and its active N-desmethyl metabolite, evaluated exposure-safety/efficacy relationships, and assessed the proposed therapeutic dose of 25 mg/m Population PK modeling and covariate analysis (demographics, formulation, liver enzymes, BSA, patients/healthy volunteers) were based on pooled PK data from adult healthy volunteers (n = 391), adult oncology patients (n = 83) and pediatric patients with NF1-PN (n = 68). Longitudinal selumetinib/metabolite exposures were predicted with the final model. Exposure-safety/efficacy analyses were applied to pediatric patients (dose levels: 20, 25, 30 mg/m Selumetinib and metabolite concentration-time courses were modeled using a joint compartmental model. Typical selumetinib plasma clearance was 11.6 L/h (95\% CI 11.0-12.2 L/ h). Only BSA had a clinically relevant (> 20\%) impact on exposure, supporting BSA-based administration in children. Selumetinib and metabolite exposures in responders (≥ 20\% PN volume decrease from baseline) and non-responders were largely overlapping, with medians numerically higher in responders. No clear relationships between exposure and safety events were established; exposure was not associated with key adverse events (AEs) including rash acneiform, diarrhea, vomiting, and nausea. Findings support continuous selumetinib 25 mg/m [\hyperlink{Selpercatinib}{PMID: 33903938}, Stein Schalkwijk et al., 2021] To review the evidence for the efficacy and safety of colchicine in children with pericarditis. Systematic review. The following databases were searched for studies about colchicine in children with pericarditis (June 2015): Cochrane Central, Medline, EMBASE and LILACS. All observational and experimental studies on humans with any length of follow-up and no limitations on language or publication status were included. The outcomes studied were recurrences of pericarditis and adverse events. Two authors extracted data and assessed quality of included studies using the Cochrane risk of bias tool for non-randomised trials. Two case series and nine case reports reported the use of colchicine in a total of 86 children with pericarditis. Five articles including 74 paediatric patients were in favour of colchicine in preventing further pericarditis recurrences. Six studies including 12 patients showed that colchicine did not prevent recurrences of pericarditis. No randomised controlled trials (RCTs) were found. Although colchicine is an established treatment for pericarditis in adults, it is not routinely used in children. There is not enough evidence to support or discourage the use of colchicine in children with pericarditis. Further research in the form of large double-blind RCTs is needed to establish the efficacy of colchicine in children with pericarditis. [\hyperlink{Selpercatinib}{PMID: 33903938}, Samer Alabed et al., 2016]

\hypertarget{pmid_19879734}{B}ecause patient movement during echocardiography interferes with diagnostic quality, many institutions sedate children who are unable to cooperate. The purpose of this review was to determine the efficacy and safety of oral pentobarbital for sedation during pediatric transthoracic echocardiography. Echocardiography laboratory quality assurance data were recorded for 12 years. Sedation data included adverse events, dosing, and failed sedation. The study population was grouped by age: neonates (<1 month), infants (1-12 months), and young children (1-4 years). A total of 9796 patients underwent sedation by oral pentobarbital. The overall sedation success rate was 98.7\%, and 99\% of these patients remained sedated long enough for study. The overall adverse event rate was 0.5\%. Second doses, failed sedation, and adverse events were more common in the young children. Oral pentobarbital is an effective and safe sedative for pediatric transthoracic echocardiography. Because of decreased efficacy and an increased incidence of adverse events, alternative sedation strategies may be beneficial in children aged 1 to 4 years. [\hyperlink{Selpercatinib}{PMID: 19879734}, Charles N Warden et al., 2010]

\hypertarget{pmid_1408288}{A} case of a ten years old boy with recurrent fever and abdominal pain starting at the age of five years is reported. Later the attacks were accompanied by chest pain. There were only indifferent changes in laboratory examination. Neither a wide range of antibiotics, nor appendectomy and tonsillectomy prevented the boys symptoms. The diagnose was established after five years by a positive Metaraminol test, that precipitated a disease-like attack. The therapeutic use of colchicine-salicylate reduced the severity and frequency of attacks in out patient. In agreement with other authors it should be emphasized, that in general the benefit of colchicine outweighs possible side effects of a long term therapy also in children. [\hyperlink{Selpercatinib}{PMID: 1408288}, K M Unsinn et al., 1992]

\hypertarget{pmid_9641750}{T}he incidence of recurrence of acute pericarditis in children varies from 15\% to 30\% and is accompanied by a high morbidity. Various treatment modalities have been used with variable success rates and side effects. La Serna et al. (Lancet 1987; 26: 1517) were the first to treat adults with recurrent pericarditis with colchicine, and were followed by other authors. To our knowledge no studies in children have been reported. In this paper, we present three children who suffered from viral or idiopathic recurrent pericarditis, despite multiple courses of non-steroidal anti-inflammatory drugs (NSAIDs) and/or corticosteroids. They responded remarkably well to colchicine, which was administered for 6 months with no adverse reactions. They continue to do well 18, 11 and 12 months after cessation of treatment, respectively. [\hyperlink{Selpercatinib}{PMID: 9641750}, A Yazigi et al., 1998]

\section*{Selegiline Hydrochloride}
\subsection*{Result}
\subsubsection*{Answer}

Unknown

\subsubsection*{{Explanation}}
\hypertarget{Selegiline Hydrochloride}
A review of the available abstracts reveals the following information regarding the safety of Selegiline Hydrochloride in children:

1. One abstract specifically addresses the use of selegiline in children with Attention Deficit Hyperactivity Disorder (ADHD). In this double-blind, controlled clinical trial, 28 children with ADHD (under 5 years: 5 mg/day; over 5 years: 10 mg/day) were randomized to receive either selegiline or methylphenidate for 4 weeks. The study found no significant differences in efficacy between the two groups. Importantly, the side effect profile of selegiline was described as tolerable, with decreased appetite, difficulty falling asleep, and headaches observed more frequently in the methylphenidate group. The authors conclude that, although the results are preliminary, selegiline may be beneficial and has a tolerable side effect profile in the treatment of ADHD in children [\hyperlink{pmid_12921918}{PMID: 12921918}, Shahin Akhondzadeh et al., 2003].

2. Other abstracts mentioning selegiline focus exclusively on adult populations, particularly patients with Parkinson's disease, and do not provide any data on pediatric safety or efficacy [\hyperlink{pmid_2515719}{PMID: 2515719}, A J Lees et al., 1989; \hyperlink{pmid_10778622}{PMID: 10778622}, S N Dixit et al., 1999].

Summary by age range:
- Children under 5 years: The cited ADHD study included children under 5 years, but the sample size is not specified for this subgroup. The study reports a tolerable side effect profile but notes that the results are preliminary.
- Children 5 years and older: The same study included children over 5 years, with similar findings regarding tolerability and efficacy.

No abstracts report targeted safety studies of selegiline in children outside the context of this small, short-term ADHD trial. There are no abstracts reporting that selegiline is unsafe in children.

Conclusion:
Based on the available abstracts, there is only one small, preliminary study addressing the safety of selegiline hydrochloride in children (specifically for ADHD), and it suggests a tolerable side effect profile over a 4-week period. However, the evidence is limited and not sufficient to definitively affirm safety for use in children across broader indications or for long-term use. Therefore, the safety of selegiline hydrochloride in children remains unknown.

\subsection*{Abstracts}
\hypertarget{pmid_28741653}{C}hloral hydrate is commonly used to sedate children for painless procedures. Children may recover more quickly after sedation with dexmedetomidine, which has a shorter half-life. We randomly allocated 196 children to chloral hydrate syrup 50 mg.kg [\hyperlink{Selegiline Hydrochloride}{PMID: 28741653}, V M Yuen et al., 2017] Chloral hydrate has been used extensively to sedate children, but at Brooke Army Medical Center, other drug combinations were becoming increasingly popular due to a perception that chloral hydrate had a high rate of failure, especially with younger or neurologically impaired children. Therefore, 50 children were given the drug before a diagnostic study, and patient data and a sedation score were recorded on a worksheet. Of 50 children, 43 (86\%) were "successfully sedated" on the first attempt with no side effects. Children with neurologic disorders had a much greater (27\% vs 4\%) failure rate than "normal" children. The sedation rate did not significantly differ by age, sex, or initial drug dosage. The study suggest that chloral hydrate is a safe and effective oral sedative but that children with neurologic disorders may need alternative drugs for sedation. [\hyperlink{Selegiline Hydrochloride}{PMID: 28741653}, P D Rumm et al., 1990]

\hypertarget{pmid_15951862}{D}iagnostic and therapeutic procedures in children are made easier using sedation. However, there is no consensus about which drug should be used to achieve this. Furthermore, none of the drugs used for sedation are risk free. The aim of this work is to study sedation indications, effectiveness, and safety at our center. A prospective observational study conducted at the Pediatric Day Care Unit, King Fahad National Guard Hospital, Riyadh, Saudi Arabia. The study covered 17.5 weeks in 2 periods: May 9th 1999 to June 13th 1999 and October 31st 2001 to February 11th 2002. Children <12 years were included. Collected data included demographics, indication, drug dosing and outcome. Data were reported as mean +/- SD. We included 148 patients, age 38 +/- 30 months. Adequate sedation was achieved in 79\% after initial chloral hydrate (CH) dose of 56.9 +/- 9.3 mg/kg, in 95\% after adding 18.5 +/- 6.4 mg/kg CH and in 96\% after adding second drug. Compared to nonrespondents, first CH dose respondents were younger and lower in weight. The CH side effects were few and mild. Chloral hydrate is a safe and effective agent for sedation in children with an age and weight dependent response. [\hyperlink{Selegiline Hydrochloride}{PMID: 15951862}, Omar M Hijazi et al., 2005]

\hypertarget{pmid_2515719}{S}elegiline hydrochloride (deprenyl) is a safe, useful adjuvant therapy in patients with Parkinson's disease treated with L-dopa. The optimum time for its introduction into the treatment regimen of a patient remains controversial. A multicentre long-term study being conducted by the Parkinson's Disease Research Group of the United Kingdom to attempt to answer whether selegiline improves the natural history of Parkinson's disease is discussed. In a separate study we have been unable to demonstrate that higher doses of selegiline (up to 40 mg a day) produce additional therapeutic benefit above the conventional dose of 10 mg a day in levodopa-treated patients with motor fluctuations. Preliminary data from a neuropsychological study is also presented which suggests that selegiline may have beneficial effects on the speed of psychomotor responses supporting the anecdotal clinical observations of increased mental energy and alacrity. [\hyperlink{Selegiline Hydrochloride}{PMID: 2515719}, A J Lees et al., 1989]

\hypertarget{pmid_28275979}{S}edation is often required for children undergoing diagnostic procedures. Chloral hydrate has been one of the sedative drugs most used in children over the last 3 decades, with supporting evidence for its efficacy and safety. Recently, chloral hydrate was banned in Italy and France, in consideration of evidence of its carcinogenicity and genotoxicity. Dexmedetomidine is a sedative with unique properties that has been increasingly used for procedural sedation in children. Several studies demonstrated its efficacy and safety for sedation in non-painful diagnostic procedures. Dexmedetomidine's impact on respiratory drive and airway patency and tone is much less when compared to the majority of other sedative agents. Administration via the intranasal route allows satisfactory procedural success rates. Studies that specifically compared intranasal dexmedetomidine and chloral hydrate for children undergoing non-painful procedures showed that dexmedetomidine was as effective as and safer than chloral hydrate. For these reasons, we suggest that intranasal dexmedetomidine could be a suitable alternative to chloral hydrate. [\hyperlink{Selegiline Hydrochloride}{PMID: 28275979}, Giorgio Cozzi et al., 2017]

\hypertarget{pmid_12921918}{A}ttention deficit hyperactivity disorder (ADHD) is a common disorder of childhood that affects 3\% to 6\% of school-age children. Conventional stimulant medications are recognized by both specialists and parents as useful symptomatic treatment. Nevertheless, approximately 30\% of ADHD children treated with them do not respond adequately or cannot tolerate the associated adverse effects. Such difficulties highlight the need for alternative safe and effective medications in the treatment of this disorder. Selegiline is a type B monoamine oxidase inhibitor (MAOI) that is metabolized to amphetamine and methamphetamine stimulant compounds that may be useful in the treatment of ADHD. The authors undertook this study to further evaluate, under double-blind and controlled conditions, the efficacy of selegiline for ADHD in children. A total of 28 children with ADHD as defined by DSM IV were randomized to selegiline or methylphenidate dosed on an age and weight-adjusted basis at selegiline 5 mg/day (under 5 years) and 10 mg/day (over 5 years) (Group 1) and methylphenidate 1 mg/kg/day (Group 2) for a 4-week double-blind clinical trial. The principal measure of the outcome was the Teacher and Parent ADHD Rating Scale. Patients were assessed by a child psychiatrist at baseline, 14 and 28 days after the medication started. No significant differences were observed between the two protocols on the Parent and Teacher Rating Scale scores. Although the number of dropouts in the methylphenidate group was higher than in the selegiline group, there was no significant difference between the two protocols in terms of the dropouts. Decreased appetite, difficulty falling asleep and headaches were observed more in the methylphenidate group. The results of this study must be considered preliminary, but they do suggest that selegiline may be beneficial in the treatment of ADHD. In addition, a tolerable side effect profile may be considered as one of the advantages of selegiline in the treatment of ADHD. [\hyperlink{Selegiline Hydrochloride}{PMID: 12921918}, Shahin Akhondzadeh et al., 2003]

\hypertarget{pmid_24627951}{T}o determine the safety and efficacy of high-dose oral chloral hydrate for pediatric ophthalmic procedures. This study is a retrospective review of a quality audit of pediatric sedation for ophthalmic evaluation and imaging performed at King Khaled Eye Specialist Hospital between January 1 and December 31, 2011, in children aged 1 month to 6 years. Three hundred fifty-eight of 380 (94.2\%) sedation procedures were successful after a single dose of chloral hydrate, with 356 of 380 (93.7\%) children sedated within 45 minutes of the first dose. The total success rate of the sedation procedure increased to 97.9\% (372 of 380) when a second dose was administered. Children adequately sedated after a single dose of chloral hydrate were on average younger and weighed less than children who required additional doses. No major adverse events were documented. The use of chloral hydrate sedation for ophthalmic evaluation and imaging was safe and effective in this patient population with a high rate of procedure completion. [\hyperlink{Selegiline Hydrochloride}{PMID: 24627951}, Michelle E Wilson et al., ]

\hypertarget{pmid_23129068}{H}ydroxyurea (HU) is highly effective treatment for sickle cell disease (SCD). While pediatric use of HU is accepted clinical practice, barriers to use may impede its potential benefit. A survey of parents of children ages 5-17 years with SCD was performed across five institutions to assess factors associated with HU use. Of the 173 parent responses, 65 (38\%) had children currently taking HU. Among parents of children not taking HU, the most commonly cited reasons were that their hematology provider had not offered it, their child was not sufficiently symptomatic and concerns about potential side effects. Even parents of HU users reported widespread concern about effectiveness, long-term safety, and off-label use. In bivariate analyses, children's ages, parental demographics such as education level, or travel time to their hematology provider were not correlated with HU use. Bivariate analysis and multivariate logistic regression revealed three significant factors associated with current HU use: better parental knowledge about its major therapeutic effects (P < 0.001), sickle genotype (P = 0.005), and institution of clinical care (P = 0.04). Pervasive concerns about HU safety exist, even among parents of current users. Varying knowledge among parents appears to be independent of their demographics, and is associated with HU use. Inter-institutional variability in parental knowledge and drug uptake highlights potentially potent site-specific influences on likelihood of HU use. Overall, these survey data underscore the need for strategies to bolster parental understanding about benefits of HU and address concerns about its safety. [\hyperlink{Selegiline Hydrochloride}{PMID: 23129068}, Suzette O Oyeku et al., 2013]

\hypertarget{pmid_2026812}{C}hloral hydrate is commonly used to sedate children before CT. However, no prospective study has been published of the safety and efficacy of chloral hydrate at high dose levels for children undergoing CT. We define high dose levels of oral chloral hydrate to be 80-100 mg/kg, with a maximum total dose of 2 g. High dose chloral hydrate sedation was administered orally to 295 children for 326 CT examinations. Adverse reactions occurred in 7\% of the children, with vomiting being the most common (4.3\% of children). Hyperactivity and respiratory symptoms each occurred in less than 2\% of children. Prolonged sedation ( greater than 2 h) was not encountered in our series. Sedation was successful in producing motion free CT examinations, so that in 303 (93\%) of the cases, no repeat CT scans were needed. We conclude that high dose oral chloral hydrate provides safe and effective sedation for children undergoing CT. [\hyperlink{Selegiline Hydrochloride}{PMID: 2026812}, S B Greenberg et al., ]

\hypertarget{pmid_10778622}{S}elegiline hydrochloride, a selective MAO-B inhibitor is known to improve motor functions in Parkinson's disease (PD). The present study was undertaken to study the effect of selegiline on memory and intelligence of PD patients. Thirty two patients of PD were divided in two groups: selegiline group (n = 17) received 10 mg selegiline per day and control group (n = 15) did not receive selegiline. Patients receiving trihexyphenidyl and selegiline were excluded. All other treatment remained unchanged. All patients were examined at baseline and after three months for change in UPDRS score, WAIS score, memory test and P300. Patients in selegiline group had less severe disease (UPDRS score 24.11 +/- 14.07) as compared to controls (UPDRS score 40.53 +/- 18.52). There was significant improvement in UPDRS score (p < 0.05), WAIS (p < 0.001) and memory (p < 0.001) in selegiline group. In the control group there was a significant prolongation of P300 latency (p < 0.05). The study suggests that selegiline improves memory functions and intelligence in PD patients in addition to motor functions. It also prevents prolongation of P300 latency which is a marker of cognitive function. [\hyperlink{Selegiline Hydrochloride}{PMID: 10778622}, S N Dixit et al., 1999]

\hypertarget{pmid_15604217}{H}ydroxyurea (HU) is considered to be the most successful drug therapy for severe sickle cell disease (SCD). Nevertheless, questions remain regarding its benefits in very young children and its role in the prevention of cerebrovascular events. There were 127 SCD patients treated with no attempt to reach maximal tolerated doses who entered the Belgian Registry: 109 for standard criteria and 18 who were at risk of stroke only. During 426 patient-years of follow-up for patients with standard criteria, 3.3 acute chest syndromes, 1.3 cerebrovascular events, and 1.1 osteonecrosis per 100 patient-years were observed. A subgroup of 32 patients followed for 6 years experienced significant benefit over this period. In each subgroup of children (younger than 2 years, 2-5, 6-9, and 10-19 years) followed for 2 years, clinical and biologic changes were similar, except for children younger than 2 years who had no total hemoglobin increase and remained at risk of severe anemia. In 72 patients evaluated by transcranial Doppler studies (TCD), 34 patients were at risk of primary stroke and only 1 had a cerebrovascular event after a follow-up of 96 patient-years. These results confirm the benefit of HU, even in very young children, and its possible role in primary stroke prevention. [\hyperlink{Selegiline Hydrochloride}{PMID: 15604217}, Béatrice Gulbis et al., 2005]

\hypertarget{pmid_9060867}{C}hloral hydrate (CH) is used to sedate children unable to cooperate during investigations such as EEG requiring the patient to be still. It is not known if CH or its metabolites modify the EEG and our aim was to answer this question. Recordings of the EEG before, during and after rectal administration of CH (50-77 mg/kg) in 13 children aged 1.5-13.5 years with severe epilepsy and additional neurological impairments were made. All children had frequent spike-wave activity before CH. In 9 children CH had no effect on the EEG. In 3 children there was a significant reduction in epileptic activity after 20-50 min and in one a significant increase. Cardiovascular parameters were stable throughout. At sedative doses, CH can generally be used before an EEG recording without loss of information but in 4 out of 13 children there were changes which could alter interpretation. [\hyperlink{Selegiline Hydrochloride}{PMID: 9060867}, M Thoresen et al., 1997]

\hypertarget{pmid_3218706}{A} female infant with seizures refractory to conventional therapeutic agents was presented. Mexiletine hydrochloride, administered orally, was effective in controlling her seizures. Her sleep structure and psychomotor development seemed to improve after reduction of the fits. [\hyperlink{Selegiline Hydrochloride}{PMID: 3218706}, J Kohyama et al., 1988]

\hypertarget{pmid_21531030}{C}hloral hydrate (CH) is an oral sedative widely used to sedate infants and young children during auditory brainstem response (ABR) testing. The aim of this study was to record effectiveness, complications and safety of CH as a sedative for ABR. From January of 2003 until December of 2007, 1903 children were tested for ABR, 568 of them being under the age of 6 months. CH (8\%) was used for sedation at a dose of 40 mg/kg with a repeat dose, if necessary, for an adequate sedation, in 20-30 min. We recorded the effectiveness of CH as a sedative for ABR examination, as well as all complications related to the use of CH such as vomiting, rash, hyperactivity, respiratory distress and apnea. The statistical method used was the absolute and percentage frequency distribution of the occurrences. Sedation with CH was necessary to perform testing in 1591 (83.6\%) of the examined children. However, in the population of the examined infants, only 341 (60\%) were sedated with CH, because the remaining 227 (40\%) fell asleep by themselves. Complications included hyperactivity in 152 children (8\%), minor respiratory distress in 10 children (0.4\%), vomiting in 217 children (11.4\%), apnea in 4 children (0.2\%) and rash in 10 children (0.4\%). The complications of hyperactivity, vomiting and rash resolved without any medical treatment. The apnea cases were managed effectively by supplying ventilation to the children via a mask in the presence of an anesthesiologist. The use of CH at a dose of 40 mg/kg up to 80 mg/kg is safe and effective when administered in a setting with adequate equipment and the presence of well trained personnel. [\hyperlink{Selegiline Hydrochloride}{PMID: 21531030}, Eirini Avlonitou et al., 2011]

\hypertarget{pmid_33706380}{H}ydroxyurea (HU) is used in children with sickle cell disease (SCD) to increase fetal hemoglobin (HF), contributing to a decrease in physical symptoms and potential protection against cerebral microvasculopathy. There has been minimal investigation into the association between HU use and cognition in this population. This study examined the relationship between HU status and cognition in children with SCD. Thirty-seven children with SCD HbSS or HbS/β0 thalassaemia (sickle cell anemia; SCA) ages 4:0-11 years with no history of overt stroke or chronic transfusion completed a neuropsychological test battery. Other medical, laboratory, and demographic data were obtained. Neuropsychological function across 3 domains (verbal, nonverbal, and attention/executive) was compared for children on HU (n = 9) to those not taking HU (n = 28). Children on HU performed significantly better than children not taking HU on standardized measures of attention/executive functioning and nonverbal skills. Performance on verbal measures was similar between groups. These results suggest that treatment with HU may not only reduce physical symptoms, but may also provide potential benefit to cognition in children with SCA, particularly in regard to attention/executive functioning and nonverbal skills. Replication with larger samples and longitudinal studies are warranted. [\hyperlink{Selegiline Hydrochloride}{PMID: 33706380}, Reem A Tarazi et al., 2021]

\hypertarget{pmid_9317198}{C}hildren with sickle cell anemia provide the best opportunity to assess the efficacy of hydroxyurea (HU) in preventing complications and progressive organ damage. The possibility of treating infants with sickle cell disease (SCD) to inhibit the development of organ dysfunction may be the most important future use of HU. The possibility even exists that instituting HU in the neonate may stop the fetal-to-adult globin chain switch and thus markedly change the clinical phenotype of SCD. Recent data suggest HU may also be especially beneficial in children not only by increasing hemoglobin F (HbF), but also by altering the adhesive receptors expressed on red blood cells and vascular endothelium, further increasing the possibility that vasculopathy can be prevented. Six pediatric trials that included small numbers of severely ill patients have been reported recently. All patients received relatively standard HU doses. All studies reported a significant improvement in HbF and mean corpuscular volume and a mild to marked increase in hemoglobin. The clinical response to HU in children with SCD seems to be consistent. The National Institutes of Health pediatric multicenter trial should help answer the question of short-term HU toxicity; however, questions remain concerning long-term risks, such as carcinogenesis, gametogenesis, marrow toxicity, growth retardation, and chromosomal damage. Long-term studies are needed to answer these questions. The future treatment of most children with SCD with HU alone or in combination with other agents looks promising, and long-term trials are warranted. [\hyperlink{Selegiline Hydrochloride}{PMID: 9317198}, E P Vichinsky et al., 1997]

\hypertarget{pmid_31369972}{C}hloral hydrate is a sedative that has been used for many years in clinical practice and, under proper conditions, gives a deep and long enough sleep to allow performance of objective hearing tests in young children. The reluctance to use this substance stems from side effects reported over time that can vary, depending on dose, procedure settings and immediate life supporting intervention when needed. Our study adds to those that have appeared in recent years, showing that chloral hydrate is an effective and safe substance when is used in proper conditions. The study included 322 children who needed sedation for objective hearing tests, from April 2014 to March 2018. Parents were instructed to bring the child tired and fasted for at least 2 h before sedation. The sedative was administered by trained staff in the hospital, and the child was monitored until awaking. In our study group, over half of the children were in the age 1-4 years group, and only 15\% were older than 4 years. The dose of chloral hydrate ranged between 50 and 83 mg/kg body weight, with an average of 75 mg. Successful sedation occurred in 94.1\% of children; 0.9\% of children awoke during testing and required supplemental sedation or rescheduling of the testing. The most common side effects were vomiting, agitation, prolonged sleep, and failure to fall asleep. Comparing the side effects of chloral hydrate in our study with those from other studies, ours were similar to those described in the literature. In our study chloral hydrate was effective and had only limited adverse effects. The use of chloral hydrate under hospital conditions with proper monitoring could be a practical and safe solution for outpatients or those with short-term hospitalisation. [\hyperlink{Selegiline Hydrochloride}{PMID: 31369972}, Violeta Necula et al., 2019]

\hypertarget{pmid_31534313}{C}hloral hydrate (CH), as a sedation agent, is widely used in children for diagnostic or therapeutic procedures. However, it has not come into the market and is currently only used as hospital preparation in China. This review aims to systematically evaluate the efficacy of CH in children of all age groups for sedation before medical procedures. Seven electronic databases and three clinical trial registry platforms were searched and the deadline was September 2018. Randomized controlled trials (RCTs) evaluating the efficacy of CH for sedation in children were included by two reviewers. The extracted information included success rate of sedation, sedation latency and sedation duration. The Cochrane risk of bias tool was applied to assess the risk of bias. The outcomes were analyzed by Review Manager 5.3 software and expressed as relative risks (RR) or Mean Difference (MD) with 95\% confidence interval (CI). Heterogeneity was assessed with I-squared (I A total of 24 RCTs involving 3564 children of CH for sedation were included in the meta-analysis. Compared to placebo group, CH group had a significant increase in success rate of sedation when used for painless and painful procedure (RR=4.15, 95\% CI [1.21, 14.24], P=0.02; RR=1.28, 95\% CI [1.17, 1.40], P<0.01), which included 22 and 455 children for this analysis, respectively. Compared to midazolam group, CH group had a significant increase in success rate of sedation (RR=1.63, 95\% CI [1.48, 1.79], I From the extrapolation of the existing literature, CH oral solution is an appropriate effective alternative for sedation in pediatrics. [\hyperlink{Selegiline Hydrochloride}{PMID: 31534313}, Zhe Chen et al., 2019]

\hypertarget{pmid_25246305}{T}he aim of this study was to compare the efficacy and safety of different oral chloral hydrate and dexmedetomidine doses used for sedation during electroencephalography (EEG) in children. One hundred sixty children aged 1 to 9 years with American Society of Anesthesiologists physical status I-II who were uncooperative during EEG recording or who were referred to our electrodiagnostic unit for sleep EEG were included to the study. The patients were randomly assigned into 4 groups. In groups D1 and D2, patients received oral dexmedetomidine doses of 2 and 3 µg/kg, respectively. In group C1 and C2, patients received oral chloral hydrate doses of 50 and 100 mg/kg, respectively. The induction time was significantly shorter in group C2 compared with other groups (P = .000). The rate of adverse effects was significantly higher in group C2 compared with the dexmedetomidine groups (D1 and D2; P = .004). In conclusion, dexmedetomidine can be used safely for sedation during EEG in children.  [\hyperlink{Selegiline Hydrochloride}{PMID: 25246305}, Hakan Gumus et al., 2015] Hydroxyurea (HU) increases fetal hemoglobin (HgbF) and ameliorates sickle cell disease (SCD) symptoms. Studies have demonstrated the safety and efficacy of HU in infants and children. Initiation of HU in infancy for children with SCD needs to be implemented in community practice. Starting in 2011, the Pediatric Sickle Cell Program of Northern Virginia initiated HU in infants with SCD. A prospective longitudinal database tracked the clinical course and outcomes. Twenty-four children with HgbSS who started HU by age 1 were continuously followed for a total of 95 person-years. Age at the time of analysis ranged from 2 to 7 years. Average hemoglobin at 6-month intervals ranged from 9.5 + 1.9 to 10.7 + 0.8 g/dL, and average HgbF ranged from 27.8 + 5.0\% to 34.1 + 6.6\%. Twenty-seven hospitalizations occurred (0.28/person-year), all before age 3, including 19 (70\%) for fever or infection, five (19\%) for splenic sequestration, and one (4\%) for pain in an infant prior to starting HU. The treat-and-release emergency department visits totaled 68 (0.72/person-year), including 62 visits (91\%) for fever, infection, or viral illness, and two visits (3\%) for pain/dactylitis in infants before HU initiation. Splenic sequestration accounted for all five transfusions. No pain episodes requiring medical attention were documented after HU initiation. No complicated acute chest syndrome, no abnormal or conditional transcranial Doppler ultrasound, and no overt strokes occurred. Implementation of HU in infancy for patients with SCD in community practice is feasible and is highly effective in preventing disease complications. [\hyperlink{Selegiline Hydrochloride}{PMID: 25246305}, Ronay Thomas et al., 2019]

\hypertarget{pmid_22246409}{C}hloral hydrate (CH) is safe and effective for sedation of suitable children. The purpose of this study was to assess whether adequate sedation is achieved with reduced CH doses. We retrospectively recorded outpatient CH sedations over 1 year. We defined standard doses of CH as 50 mg/kg (infants) and 75 mg/kg (children >1 year). A reduced dose was defined as at least 20\% lower than the standard dose. In total, 653 children received CH sedation (age, 1 month-3 years 10 months), 42\% were given a reduced initial dose. Augmentation dose was required in 10.9\% of all children, and in a higher proportion of children >1 year (15.7\%) compared to infants (5.7\%; P < 0.001). Sedation was successful in 96.7\%, and more frequently successful in infants (98.3\%) than children >1 year (95.3\%; P = 0.03). A reduced initial dose had no negative effect on outcome (P = 0.19) or time to sedation. No significant complications were seen. We advocate sedation with reduced CH doses (40 mg/kg for infants; 60 mg/kg for children >1 year of age) for outpatient imaging procedures when the child is judged to be quiet or sleepy on arrival. [\hyperlink{Selegiline Hydrochloride}{PMID: 22246409}, Jennifer Bracken et al., 2012]

\hypertarget{pmid_2672786}{T}his study assessed the safety and efficacy of methylphenidate in children with seizures and attention-deficit disorder. Ten children, aged 6 years 10 months to 10 years 10 months, without seizures while receiving a single antiepileptic drug, were evaluated in a double-blind medication-placebo crossover study with methylphenidate hydrochloride was administered at 0.3 mg/kg per dose and given at 8 AM and 12 PM on school days only. The use of methylphenidate was associated with statistically significant improvements on the Conners' Teacher Rating Scale and on the Finger Tapping Task and with trends toward improvement on the Matching Familiar Figures Test and Discriminant Reaction Time tests. No child had seizures during the study period nor subsequently for those who continued receiving psychostimulants. There were no significant changes of epileptiform features or back-ground activity on electroencephalograms and no alterations in antiepileptic drug levels. Methylphenidate may be a safe and effective treatment for certain children with seizures and concurrent attention-deficit disorder. [\hyperlink{Selegiline Hydrochloride}{PMID: 2672786}, H Feldman et al., 1989]

\hypertarget{pmid_19047254}{H}ydroxyurea is the only approved medication for the treatment of sickle cell disease in adults; there are no approved drugs for children. Our goal was to synthesize the published literature on the efficacy, effectiveness, and toxicity of hydroxyurea in children with sickle cell disease. Medline, Embase, TOXLine, and the Cumulative Index to Nursing and Allied Health Literature through June 2007 were used as data sources. We selected randomized trials, observational studies, and case reports (English language only) that evaluated the efficacy and toxicity of hydroxyurea in children with sickle cell disease. Two reviewers abstracted data sequentially on study design, patient characteristics, and outcomes and assessed study quality independently. We included 26 articles describing 1 randomized, controlled trial, 22 observational studies (11 with overlapping participants), and 3 case reports. Almost all study participants had sickle cell anemia. Fetal hemoglobin levels increased from 5\%-10\% to 15\%-20\% on hydroxyurea. Hemoglobin concentration increased modestly (approximately 1 g/L) but significantly across studies. The rate of hospitalization decreased in the single randomized, controlled trial and 5 observational studies by 56\% to 87\%, whereas the frequency of pain crisis decreased in 3 of 4 pediatric studies. New and recurrent neurologic events were decreased in 3 observational studies of hydroxyurea compared with historical controls. Common adverse events were reversible mild-to-moderate neutropenia, mild thrombocytopenia, severe anemia, rash or nail changes (10\%), and headache (5\%). Severe adverse events were rare and not clearly attributable to hydroxyurea. Hydroxyurea reduces hospitalization and increases total and fetal hemoglobin levels in children with severe sickle cell anemia. There was inadequate evidence to assess the efficacy of hydroxyurea in other groups. The small number of children in long-term studies limits conclusions about late toxicities. [\hyperlink{Selegiline Hydrochloride}{PMID: 19047254}, John J Strouse et al., 2008]

\hypertarget{pmid_18540545}{I}n the absence of a general anaesthetic facility for MRI scanning in children, we introduced a sedation protocol using chloral hydrate. Our aim was to evaluate the success and safety of our protocol. This was a retrospective study enrolling 36 children over a 7 month period. The overall success rate was 86\% with no child experiencing respiratory complications. In those less than one year, the success rate was 100\%, aged 1-5 years 91\%, with 50\% successful at 80 mg/kg and 50\% at 100 mg/kg dose. For children greater than 5 years of age the success rate was 70\%. 92\% of developmentally normal children and 83\% of developmentally delayed children were successfully sedated. Success rates were poorer in children older than 5 years and in those with developmental delay. Our findings suggest that this protocol could be safely used in units where general anaesthetic facilities are unavailable for MRI and for other radiological investigations. [\hyperlink{Selegiline Hydrochloride}{PMID: 18540545}, E Low et al., 2008]

\hypertarget{pmid_942230}{K}etamine hydrochloride 2 mg/kg, together with atropine 0.2 mg, has been given intravenously on 100 occasions on a general paediatric ward. No serious side effects occurred. Dreams followed in 4 children but did not reduce acceptability of the drug. In our hands it has greatly reduced the pain and distress of children undergoing many routine medical procedures, particularly the dread which builds up when these have to be repeated in the same child. It has also produced close to ideal conditions for the operator, and probably increased his efficiency by reducing the emotional strain which occurs when doing painful things to a frightened patient. [\hyperlink{Selegiline Hydrochloride}{PMID: 942230}, E Elliott et al., 1976]

\section*{Tolmetin Sodium}
\subsection*{Result}
\subsubsection*{Answer}

Unknown

\subsubsection*{{Explanation}}
\hypertarget{Tolmetin Sodium}
After reviewing all available abstracts, there are no studies specifically evaluating the safety of Tolmetin Sodium in children. The abstracts that discuss the safety and efficacy of Tolmetin Sodium focus on adult or geriatric populations, animal models, or pharmacokinetic studies in adults with rheumatoid arthritis. None of the abstracts present data from targeted studies on children or specific pediatric age ranges. Therefore, based on the abstracts provided, the safety of Tolmetin Sodium in children is unknown, as there is no evidence from pediatric studies to affirm or refute its safety in this population.

\subsection*{Abstracts}
\hypertarget{pmid_7893639}{T}olmetin is a nonsteroidal anti-inflammatory drug (NSAID) that reduces adhesion formation in several animal models after a single intraperitoneal (i.p.) dose delivered at the time of surgery. We set out to determine the period during which tolmetin could prevent adhesions. Adhesions were induced in New Zealand White rabbits (2-3 kg) by abrading the uterine horns and removing their mesouterine vasculature. Tolmetin sodium (1 mg/5 ml saline) was given at various times relative to the start of surgery as a single dose i.p. One week later adhesions were assessed using a standard scoring system (0 = no adhesions; 1 = light adhesions involving both uterine horns; 2 = more tenacious adhesions to bowel or bladder; 3 = tenacious adhesions to bowel and bladder partly immobilizing the uterus; 4 = completely fixed horns adherent to bowel and bladder). Scores were arranged in ascending rank order. Mean rank positions were calculated for each group and compared against controls (Dunnett's multiple comparison). Tolmetin sodium was most effective when administered within 1 hour of surgery. Mild effects could still be observed after 4 hours and the effect diminished after 24 hours. When these effects are compared to the temporal biochemical and cellular effects of tolmetin obtained in related studies, the data support the hypothesis that tolmetin reduces adhesions at least in part by modulating fibrinolytic activity of resident macrophages and macrophages present in the early postsurgical period. [\hyperlink{Tolmetin Sodium}{PMID: 7893639}, D M Wiseman et al., ]

\hypertarget{pmid_6350377}{I}n order to evaluate the effectiveness and safety of tolmetin sodium in the treatment of both rheumatoid arthritis (RA) and osteoarthritis in geriatric patients, a retrospective study was made of patients 65 years and older who participated in long-term, controlled, double-blind and open trials during both the investigational period and since marketing of the drug. Standard entrance criteria, methods of evaluating disease activity, and statistical methods were used in the study of both arthritic diseases. A total of 847 geriatric patients were studied for periods of up to one year; 171 had RA, while 676 had osteoarthritis of large or small joints. Average daily dose of tolmetin sodium was 1141 mg for patients with RA and 953 mg for patients with osteoarthritis. The results of this retrospective study of both RA and osteoarthritis patients show that tolmetin was as effective in geriatric patients as in nongeriatric patients. Symptoms responded rapidly to treatment with tolmetin, and both the inflammatory symptoms of RA and the joint pain and functional parameters of osteoarthritis showed improvement that was both statistically and clinically significant throughout the major course of therapy. Tolmetin was also found to be safe and well tolerated by the elderly patient population. The major complaints were gastrointestinal, but serious or limiting side effects occurred in few patients. The dropout rates due to adverse effects during the entire year of therapy were 15.8 per cent in the RA population and 15.4 per cent in osteoarthritis patients. This retrospective evaluation of tolmetin therapy shows significant relief of the symptoms of both RA and osteoarthritis in a geriatric population and fails to reveal any unusual or serious conditions which would contraindicate its use in the elderly patient. Tolmetin, which is an antiinflammatory agent with a short half-life, can provide adequate, safe therapy in the geriatric population. [\hyperlink{Tolmetin Sodium}{PMID: 6350377}, W M O'Brien et al., 1983]

\hypertarget{pmid_11176516}{T}olterodine was recently approved for the treatment of incontinence and overactive bladder in adults, and has fewer side effects than oxybutynin. We evaluated the safety and efficacy of tolterodine in children with dysfunctional voiding. We retrospectively reviewed our experience with 30 pediatric patients treated with tolterodine for a primary diagnosis of dysfunctional voiding. Patients were treated with adult doses of tolterodine and behavioral modifications. Standard definitions determined by the International Children's Continence Society were adapted to designate final treatment outcomes on medication as cured-greater than 90\% reduction in wetting episodes, improved-greater than 50\% reduction or failed-less than 50\% reduction. The children were 4 to 17 years old (mean age 8.7) and were treated with tolterodine for an average of 5.2 months. The final dose was 1 mg. twice daily in 1, 2 mg. twice daily in 27 and 4 mg. twice daily in 2 patients. Wetting episodes were cured in 10 (33\%), improved in 12 (40\%), and failed to show improvement in 8 (27\%) cases. Four patients (13.3\%) reported side effects and only 1 discontinued the medication due to diarrhea. There were no reports of hyperpyrexia, flushing or intolerance to sunshine and outdoor temperature. Our results demonstrate that tolterodine at adult doses without titration can be used safely to decrease wetting episodes in children with dysfunctional voiding. Controlled clinical trials should be completed to evaluate further efficacy and safety in children. [\hyperlink{Tolmetin Sodium}{PMID: 11176516}, M Munding et al., 2001]

\hypertarget{pmid_6886030}{D}ata from over 1000 patients with rheumatoid arthritis who received tolmetin sodium in double-blind and open studies have been pooled to assess long-term efficacy and safety. Duration of the studies was 12 weeks to 48 months. Mean age of patients was 54 years; ratio of males to females was 1:3. The results showed that tolmetin provided rapid onset of action and continuous progressive decrease in symptoms in all measurements of inflammation. Mean number of painful joints was reduced from 22 at baseline to 16 at one month, to 9 at one year, and to 6 at two years. Duration of morning stiffness was 155 minutes at baseline, 123 minutes at one month, 74 minutes at one year, and 78 minutes at two years. The final global evaluation by the investigators showed that 61 per cent of patients had a marked or moderate response. Mean erythrocyte sedimentation rates did not increase during therapy with tolmetin. Initial dose of tolmetin in the patients pooled for this analysis was generally 600 to 800 mg/day, and the mean dose throughout the study was 1256 mg/day. The drug was well tolerated overall. As anticipated, gastrointestinal symptoms were the most frequently reported; nausea was experienced by 13 per cent of the patients at some time during therapy, and gastrointestinal distress, dyspepsia, or abdominal pain was reported by approximately 8.6 per cent each. Only 12.7 per cent of patients discontinued tolmetin because of untoward reactions; 15.9 per cent of patients discontinued because of insufficient therapeutic response. The results of these long-term studies of patients with rheumatoid arthritis demonstrated that tolmetin is an effective antiinflammatory agent with an acceptable record of safety. [\hyperlink{Tolmetin Sodium}{PMID: 6886030}, G E Ehrlich et al., 1983]

\hypertarget{pmid_25724485}{T}he aims are to evaluate the efficacy and safety of sodium valproate for children with Tourette׳s syndrome (TS). We searched PubMed, EMBASE, the Cochrane library, Cochrane Central, CBM, CNKI, VIP, WANG FANG database and relevant reference lists. Five RCTs (N=247) and five case series (N=163) studies were included. Only one RCT (93 patients) evaluated total YGTSS scores and there was significant difference in the reduction of total YGTSS scores between sodium valproate and the control group (3.50±4.59 vs 7.86±7.03, P<0.01). One RCT (30 patients) evaluated motor and vocal tics, and there was significant difference in the reduction of motor and vocal tics scores between sodium valproate and haloperidol (10.45±4.15 vs 14.92±3.01, P<0.01). Meta-analysis of three RCTs (N=124) showed there was no significant difference in the reduction of the number of tics between sodium valproate and the positive control group [Relative Risk (RR)=1.09, 95\%CI (0.92, 1.30), P=0.30]. The pooled proportion in five case series studies which used tics symptom improvement self-defined by authors was 80.7\% (95\% CI: 73.7-86.2, I(2)=0). No fatal side effects were reported. In conclusion, based on the limited evidence, the routine use of sodium valproate for treatment of TS in children is not recommended. Further well-conducted trials that examine long-term outcomes are required.  [\hyperlink{Tolmetin Sodium}{PMID: 25724485}, Chun-Song Yang et al., 2015] The effectiveness of tolmetin sodium in the treatment of rheumatoid arthritis was evaluated by: 1) a 12-week, double-blind study with a dosage range of 800-1600 mg daily; and 2) an open 2-year study with a dosage range of 400-2400 mg daily. The double-blind study involved 14 patients (7 tolmetin sodium, 7 placebo), and the long-term study involved 24 patients. At frequent intervals, evaluations were made of joint pain, swelling, stiffness and inflammation; grip strength; walking time; and subjective well-being. Various laboratory tests were also performed. In the double-blind study, tolmetin sodium proved superior to placebo and produced moderate improvement. In the long-term study, 5 patients improved markedly, 14 moderately, and 3 minimally. Severe side effefts were notably absent. Some mild side effects occurred but they were transient and did not interfere with therapy. Tolmetin sodium seems effective and safe in the management of rheumatoid arthritis. [\hyperlink{Tolmetin Sodium}{PMID: 25724485}, L J Cordrey et al., 1976]

\hypertarget{pmid_27028950}{U}sing fluid restriction to treat the syndrome of inappropriate antidiuretic hormone secretion (SIADH) in infants is potentially hazardous, as fluid intake and caloric intake are connected. Antagonists for the type 2 vasopressin receptor have demonstrated efficacy in adult patients with SIADH, but evidence in children is lacking. We reviewed our experience from two cases in the UK. This was a retrospective review of the clinical data on two patients diagnosed with SIADH in infancy and treated with tolvaptan, an oral vasopressin receptor antagonist. Persistent hyponatraemia was noted in both patients in the first month of life and eventually led to SIADH diagnoses. Initial salt supplementation in one patient resulted in severe hypertension, treated with four antihypertensive drugs. Tolvaptan was commenced at two and four months of age, respectively, and was associated with normalisation of plasma sodium values and blood pressure without the need for antihypertensive treatment. There was transient hypernatraemia in one patient, which was normalised with a dose reduction. Tolvaptan was administered by crushing the tablet and mixing it with water. Tolvaptan provided effective treatment for SIADH in both infants and could be administered orally. [\hyperlink{Tolmetin Sodium}{PMID: 27028950}, Daniela Marx-Berger et al., 2016]

\hypertarget{pmid_6350376}{T}he efficacy and safety of tolmetin sodium in the management of ankylosing spondylitis are presented in a review of published and unpublished data. In both open and controlled clinical studies, tolmetin was superior to placebo and equal to indomethacin in its capacity to relieve pain, inflammation, and other symptoms of ankylosing spondylitis (AS). Objective and subjective assessments showed that both tolmetin sodium and indomethacin provided significant therapeutic benefits to patients with AS. In AS, the two drugs showed similar adverse reaction profiles. Adverse reactions with both drugs were minimal and predominantly affected the gastrointestinal tract; in most cases these symptoms cleared spontaneously without discontinuing the drugs. [\hyperlink{Tolmetin Sodium}{PMID: 6350376}, A Calin et al., 1983]

\hypertarget{pmid_19087826}{W}e intended to ascertain the effectiveness and safety of oral solutions of magnesium and vitamin B(6) in alleviating the symptoms emerged during clinical exacerbations in children aged 7-14 years suffering from Tourette syndrome (TS). We also aimed to determine the mean and the standard deviation of such an improvement in order to estimate sample sizes in future assays with a control group. The treatment under investigation was administered to children diagnosed with TS, in accordance with Diagnostic and Statistical Manual of Mental Disorders, fourth edition -IV, under conditions of clinical exacerbation. The effects were scored on the Yale Global Tics Severity Scale (YGTSS) at 0, 15, 30, 60 and 90 days. The total tics score decreased from 26.7 (t0) to 12.9 (t4) and the total effect on the YGTSS was a reduction from 58.1 to 18.8. Both results were statistically significant. With respect to the application of conventional treatment or otherwise, no significant differences were observed. No side effects were seen. The treatment assayed is safe and effective in reducing the harmful effects of TS in children. Further studies are needed, with a control group, and evaluation of different doses of the drugs. [\hyperlink{Tolmetin Sodium}{PMID: 19087826}, Rafael García-López et al., 2008]

\hypertarget{pmid_535831}{E}ffect of tolmetin sodium on the pain-like responses caused by various nociceptive stimuli was examined in experimental animals. Tolmetin sodium showed a potent inhibitory activity on the acetic acid-induced writhing in mice and rats, and its potency, (ED50 = 23.4 and 3.01 mg/kg, p.o.) was about 2.4--10.3 times that of ibuprofen and aspirin. The hypertension induced by intraarterial injection of bradykinin toward the spleen of dogs was inhibited by tolmetin sodium (ED50 = 80 mg/kg, i.v.), but the hypertension by a simultaneous injection of bradykinin and PGE1 was not inhibited by tolmetin sodium and sulpyrine, though pentazocine inhibited both hypertensions. The pain-like response caused by pressing mechanically the inflamed paws or joints of rats induced by kaolin-carrageenin or adjuvant was inhibited by tolmetin sodium (30--100 or 20--40 mg/kg, p.o., respectively), and the potency was approximately equal that of ibuprofen and phenylbutazone. Tolmetin sodium produced a significant inhibition of the pain-like response induced by electrical stimulation of tooth pulp of dogs, but showed no effect when the methods of Haffner and D'Amour-Smith were applied to mice. Anti-writhing action of tolmetin sodium was not antagonized by naloxone. From these results, it was concluded that tolmetin sodium has a potent inhibitory activity on the pain-like responses induced by the chemical nociceptive stimuli and by the mechanical pressure stimulus of the inflamed tissue, especially on the writhing. The analgesic activity probably involves a peripheral mechanism. [\hyperlink{Tolmetin Sodium}{PMID: 535831}, H Nakamura et al., 1979]

\hypertarget{pmid_10940537}{T}he safety of the avermectin, selamectin, was evaluated for topical use on the skin of cats of age six weeks and above, including reproducing cats and cats infected with adult heartworms. All studies used healthy cats. Acute safety was evaluated in domestic cross-bred cats. Margin of safety was evaluated in domestic-shorthaired cats, starting at six weeks of age. Reproductive, heartworm-infected, and oral safety studies were conducted in adult, domestic-shorthaired cats. Studies were designed to measure the safety of selamectin at the recommended dosage range of 6-12mgkg(-1) of body weight. Assessments included clinical, biochemical, pathologic, and reproductive indices. Selected variables in the margin of safety study and the reproductive studies were subjected to statistical analyses by using a mixed linear model. Cats received large doses of selamectin at the beginning of the margin of safety study when they were six weeks of age and at their lowest body weight, yet displayed no clinical or pathologic evidence of toxicosis. Similarly, selamectin had no adverse effect on reproduction in adult male and female cats. There were no adverse effects in heartworm-infected cats. Oral administration of the topical formulation, which might occur accidentally, caused mild, intermittent, self-limiting salivation and vomiting. Selamectin is a broad-spectrum avermectin endectocide that is safe for use in cats starting at six weeks of age, including heartworm-infected cats and cats of reproducing age, when administered topically to the skin monthly at the recommended dosage to deliver at least 6mgkg(-1). [\hyperlink{Tolmetin Sodium}{PMID: 10940537}, M J Krautmann et al., 2000]

\hypertarget{pmid_33730099}{O}ral ivermectin is a safe broad spectrum anthelminthic used for treating several neglected tropical diseases (NTDs). Currently, ivermectin use is contraindicated in children weighing less than 15 kg, restricting access to this drug for the treatment of NTDs. Here we provide an updated systematic review of the literature and we conducted an individual-level patient data (IPD) meta-analysis describing the safety of ivermectin in children weighing less than 15 kg. A systematic review was conducted using the Preferred Reporting Items for Systematic Reviews and Meta-Analyses (PRISMA) for IPD guidelines by searching MEDLINE via PubMed, Web of Science, Ovid Embase, LILACS, Cochrane Database of Systematic Reviews, TOXLINE for all clinical trials, case series, case reports, and database entries for reports on the use of ivermectin in children weighing less than 15 kg that were published between 1 January 1980 to 25 October 2019. The protocol was registered in the International Prospective Register of Systematic Reviews (PROSPERO): CRD42017056515. A total of 3,730 publications were identified, 97 were selected for potential inclusion, but only 17 sources describing 15 studies met the minimum criteria which consisted of known weights of children less than 15 kg linked to possible adverse events, and provided comprehensive IPD. A total of 1,088 children weighing less than 15 kg were administered oral ivermectin for one of the following indications: scabies, mass drug administration for scabies control, crusted scabies, cutaneous larva migrans, myiasis, pthiriasis, strongyloidiasis, trichuriasis, and parasitic disease of unknown origin. Overall a total of 1.4\% (15/1,088) of children experienced 18 adverse events all of which were mild and self-limiting. No serious adverse events were reported. Existing limited data suggest that oral ivermectin in children weighing less than 15 kilograms is safe. Data from well-designed clinical trials are needed to provide further assurance. [\hyperlink{Tolmetin Sodium}{PMID: 33730099}, Podjanee Jittamala et al., 2021]

\hypertarget{pmid_10940536}{S}elamectin is a broad-spectrum avermectin endectocide for treatment and control of canine parasites. The objective of these studies was to evaluate the clinical safety of selamectin for topical use in dogs 6 weeks of age and older, including breeding animals, avermectin-sensitive Collies, and heartworm-positive animals. The margin of safety was evaluated in Beagles, which were 6 weeks old at study initiation. Reproductive, heartworm-positive, and oral safety studies were conducted in mature Beagles. Safety in Collies was evaluated in avermectin-sensitive, adult rough-coated Collies. Studies were designed to measure the safety of selamectin at the recommended dosage range of 6-12mgkg(-1) of body weight. Endpoints included clinical examinations, clinical pathology, gross and microscopic pathology, and reproductive indices. Selected variables in the margin of safety and reproductive safety studies were subjected to statistical analyses. Pups received large doses of selamectin at the beginning of the margin of safety study when they were 6 weeks of age and at their lowest body weight, yet displayed no clinical or pathologic evidence of toxicosis. Similarly, selamectin had no adverse effects on reproduction in adult male and female dogs. There were no adverse effects in avermectin-sensitive Collies or in heartworm-positive dogs. Oral administration of the topical formulation caused no adverse effects. Selamectin is safe for topical use on dogs at the recommended minimum dosage of 6mgkg(-1) (6-12mgkg(-1)) monthly starting at 6 weeks of age, and including dogs of reproducing age, avermectin-sensitive Collies, and heartworm-positive dogs. [\hyperlink{Tolmetin Sodium}{PMID: 10940536}, M J Novotny et al., 2000]

\hypertarget{pmid_3321483}{T}he prevention of postoperative pain in children who had undergone tonsillectomy was investigated in a double-blind trial. Ketamine (Ketalar; Parke-Davis) 0.5 mg/kg was given intravenously before the operation to 20 children and saline to a control group of 20 children. Premedication consisted of oral trimeprazine 4 mg/kg given 2 hours pre-operatively. The anaesthetic technique was standardised. There were no significant differences between the groups pre-or intra-operatively. Postoperatively there were significant differences in the measurement of pain but not in that of sedation. No hallucinations were encountered in those receiving ketamine. It is concluded that analgesic doses of ketamine are safe and effective. [\hyperlink{Tolmetin Sodium}{PMID: 3321483}, W B Murray et al., 1987]

\hypertarget{pmid_30450703}{S}edation is often required for young children during transthoracic echocardiography. Dexmedetomidine and ketamine are two sedatives that are commonly used in children for procedural sedation, but they have some disadvantages when they are used alone. The aim of this retrospective study was to analyze the effects and safety of intranasal sedation with a combination of dexmedetomidine and ketamine during transthoracic echocardiography in young children and to analyze risk factors for sedation failure. After IRB approval, we retrospectively evaluated data on patients who underwent echocardiography between May 2016 and August 2017 utilizing a combination of dexmedetomidine 2 μg/kg and ketamine 1 mg/kg. We collected information including heart rate, pulse oxygen saturation, sedation onset time, exam time, recovery time, and adverse reactions. Stepwise logistic regression analyses were performed to analyze the risk factors for sedation failure. Sedation was successful in 2212 patients (96\%) and took effect in 15.7 (IQR: 10-23) min, while sedation failed in 92 patients. Cyanotic heart disease, history of sedation failure, history of congenital heart disease surgery, and fever were independent risk factors for sedation failure. Most of the patients in this study had an American Society of Anesthesiologists (ASA) grade of II to III, but no severe adverse reactions were observed. Intranasal sedation with a combination of dexmedetomidine and ketamine is effective and appears to have an acceptable safety profile for young children during transthoracic echocardiography. [\hyperlink{Tolmetin Sodium}{PMID: 30450703}, Jianxia Liu et al., 2019]

\hypertarget{pmid_12603422}{T}o assess the safety and efficacy of tolterodine tartrate prescribed to children who previously failed to tolerate oxybutynin chloride. We reviewed 34 children, followed for>1 year, who were prospectively crossed-over from oxybutynin to tolterodine because of side-effects. The initial diagnosis was dysfunctional voiding in 31 patients. All patients were placed on a behavioural modification protocol. When their symptoms did not improve after 6 months, treatment with an anticholinergic agent was considered. Urodynamic studies were conducted in 20 patients, confirming uninhibited contractions in 19. The remaining 14 patients were empirically started on antimuscarinic or anticholinergic agents. The 34 patients were treated with oxybutynin for a median (range) of 6 (2-84) months. When significant side-effects were reported, they were crossed over to tolterodine. The efficacy of tolterodine was assessed as defined by the International Children's Continence Society, with tolerability assessed and side-effects documented using a questionnaire. The mean age at the first dose of tolterodine was 8.9 years; the dose was 1 mg twice daily for 12 patients and 2 mg twice daily for 22. The median treatment with tolterodine was 11.5 months, with 20 (59\%) patients reporting no side-effects; six described the same but tolerable side-effects as with oxybutynin. Eight patients discontinued tolterodine because of side-effects after a median (range) of 5 (1-11) months. The efficacy of tolterodine was comparable with that of oxybutynin, as reported by the questionnaire and voiding diaries. The reduction in wetting episodes at 1 year was> 90\% in 23 (68\%), more than half in five and less than half (or failure) in six patients. Tolterodine is tolerated well in children. In this subgroup of patients who could not tolerate oxybutynin, 77\% were able to continue tolterodine treatment with no significant side-effects. [\hyperlink{Tolmetin Sodium}{PMID: 12603422}, S Bolduc et al., 2003]

\hypertarget{pmid_9088998}{T}he purpose of this study was to evaluate the antipyretic action of tolfenamic acid, as well as its possible adverse reactions, especially in children with severe or partial form of glucose-6-phosphate dehydrogenase (G6PD) deficiency. In the study 55 febrile children were included, whose mean age was +/- SD 3.5 +/- 3.3 years, range 0.5-15. Ten of them had severe or partial form of G6PD deficiency. Fifty-three of the patients responded with a decrease of temperature which lasted at least 6 hours, though in 2 of them the temperature decrease lasted less than 6 hours. The tolerance of the drug was good and no side-effects were noted. None of the patients with or without G6PD deficiency showed symptoms, signs, or laboratory findings indicating hemolysis before administration of the drug and 4 days thereafter. In conclusion, tolfenamic acid is a strong antipyretic agent with excellent tolerance and high safety in children. [\hyperlink{Tolmetin Sodium}{PMID: 9088998}, F A Haliotis et al., 1997]

\hypertarget{pmid_9831007}{T}opiramate is a sulfamate-substituted monosaccharide that has demonstrated efficacy as an antiepileptic drug in adults with partial onset seizures. Experience in children has been limited, but early reports have supported its safety and effectiveness in children as young as 2 years of age. In two infants ages 12 and 9 months, respectively, with partial seizures, the authors report excellent efficacy with good tolerability at doses up to 7.7 mg/kg. Although long-term safety and possible adverse sequelae have not been fully established in children, topiramate may represent an option for infants with high seizure frequency unresponsive to standard antiepileptic drugs. [\hyperlink{Tolmetin Sodium}{PMID: 9831007}, S L Kugler et al., 1998] 1 The site of the analgesic action of tolmetin sodium was investigated by use of the acetic acid writhing test in rats. 2 Tolmetin sodium was administered to the rat between 15 and 60 min after intraperitoneal injection of 1 ml of a 1\% acetic acid aqueous solution. Number of writhing was counted for 20 min beginning from 60 min after acetic acid injection. 3 When the rat was given tolmetin sodium 5 mg/kg orally, a relatively large quantity of tolmetin was found in the peritoneal exudate and there was a rough correlation between anti-writhing activity and the exudate tolmetin content. 4 Anti-writhing ED50 of tolmetin sodium was 1.42 (0.82-2.91) and 92.0 (57.0-140) microgram/kg when given intraperitoneally and intravenously, respectively, and the potency ratio of intraperitoneal to intravenous tolmetin sodium was 40.0 (18.5-80.2). This potency ratio for salicylic acid and morphine hydrochloride was 19.4 and 1.0, respectively. 5 When equipotent doses ( 5 microgram/kg i.p.; 200 microgram/kg i.v.) of tolmetin sodium were administered to the rat, the plasma tolmetin level after the intraperitoneal administration was less than one-fortieth that after the intravenous administration during the counting time of 20 min, while both the peritoneal exudate contents of tolmetin were nearly equal. 6 From these results, it is concluded that the site of anti-writhing action of tolmetin sodium is in the peritoneum and that tolmetin sodium produces its anti-writhing action mainly by a peripheral mechanism in the rat. [\hyperlink{Tolmetin Sodium}{PMID: 9831007}, H Nakamura et al., 1981]

\hypertarget{pmid_32896942}{O}n May 16, 2019, the U.S. Food and Drug Administration (FDA) approved dalteparin sodium for the treatment of symptomatic venous thromboembolism (VTE) to reduce the risk of recurrence in pediatric patients 1 month of age and older. Approval was primarily based on FDA review of a single-arm trial evaluating dalteparin administered subcutaneous twice daily in 38 pediatric patients with symptomatic VTE. Efficacy was based on the achievement of therapeutic plasma anti-Xa levels. The FDA concluded that dalteparin has efficacy and acceptable safety for pediatric patients. [\hyperlink{Tolmetin Sodium}{PMID: 32896942}, Margret Merino et al., 2020]

\hypertarget{pmid_17095898}{T}his was a prospective open study to establish the efficacy, tolerability, and problems associated with the use of topiramate as first-choice drug in children with infantile spasms. Open-label follow-up study, ranging from 24 to 36 months, of the cases of 54 patients with infantile spasms treated initially with topiramate as first-choice drug. Thirty-one patients (57.4\%) were seizure free for more than 24 months; 9 patients were treated with topiramate alone and 22 patients with topiramate plus nitrazepam and/or valproate. In 44 cases (81.4\%), the reduction of seizure frequency from baseline was greater than 30\%, whereas in 10 cases (18.6\%), there was poor or no response. The average dosage applied was 5.2 mg/kg per day (maximum dosage, 26 mg/kg per day; minimum dosage, 1.56 mg/kg per day). Adverse events occurred in 14 patients (26\%). They included poor appetite leading to anorexia, absence of sweating, and sleeplessness. Topiramate proves to be an effective and safe first-choice drug not only as adjunctive but also as monotherapy of infantile spasms in children younger than 2 years. [\hyperlink{Tolmetin Sodium}{PMID: 17095898}, Li-Ping Zou et al., ]

\hypertarget{pmid_11298060}{T}o determine the safety, efficacy and pharmacokinetics of tolterodine in children with an overactive bladder. Thirty-three children (20 boys and 13 girls, aged 5-10 years) with an overactive bladder and symptoms of urgency, frequency and/or urge incontinence were enrolled in an open, dose-escalation study. Patients were treated with oral tolterodine 0.5 mg (n = 11), 1 mg (n = 10) or 2 mg (n = 12) twice daily for 14 days. The primary safety endpoint was the change in residual urinary volume, as determined by ultrasonography. In addition, voiding diary variables (frequency and incontinence episodes) and pharmacokinetics were evaluated. Other safety endpoints included laboratory variables, electrocardiogram recordings and reported adverse events. There were no safety concerns in terms of the change in residual urinary volume for any of the three dosage groups; values were comparable with baseline after 2 weeks of treatment for all three dosages. Adverse events were reported by 20 patients (six on 0.5 mg, five on 1 mg, and nine on 2 mg). Most adverse events were not considered to be drug-related; of the 13 possibly related events, 10 occurred in those taking 2 mg. Headache was the most commonly reported adverse event. No serious adverse events were reported and there were no general safety concerns. There was an improvement in voiding diary variables in all treatment groups after 2 weeks of treatment, although the efficacy was greatest in those taking 1 mg and 2 mg. Pharmacokinetic findings were consistent with dose linearity over the range 0.5-2 mg. The results support the use of 1 mg twice daily as the optimal dose of tolterodine for treating children aged 5-10 years with an overactive bladder. [\hyperlink{Tolmetin Sodium}{PMID: 11298060}, K Hjälmås et al., 2001]

\hypertarget{pmid_24968572}{T}o examine the efficacy, safety and tolerability of tolterodine in children with overactive bladder in comparison with standard treatment i.e. oxybutynin as demonstrated in randomized clinical trials and other studies. A systematic search was done to screen the studies evaluating the effect of tolterodine in children with non-neurogenic overactive bladder. Results of studies were pooled and compared. Efficacy was determined from micturition diaries and dysfunctional voiding symptoms score. Safety and tolerability were assessed from the reported treatment emergent adverse events. A total of six randomized clinical trials and 11 other studies of tolterodine in children with urinary incontinence were included in the present systematic review. The dose of tolterodine used in different settings ranged from '0.5 to 8 mg/day' instead of '0.5 to 8 mg/kg per day' and the duration of studies ranged from 2 weeks to 12 months. Both extended and immediate release preparations of tolterodine were shown to have comparable efficacy and tolterodine proved to have comparable efficacy with better tolerability than oxybutynin in these studies. It can be concluded that tolterodine is efficacious in treatment of urinary incontinence in children. Moreover, its efficacy is comparable to oxybutynin, the most commonly prescribed anticholinergic in this condition, while having better tolerability. Hence, it can be considered as first line therapy for the treatmentof urinary incontinence in children. [\hyperlink{Tolmetin Sodium}{PMID: 24968572}, B Medhi et al., ]

\hypertarget{pmid_30303407}{T}olmetin sodium (TS) is a nonsteroidal anti-inflammatory drug (NSAID) indicated for treatment of musculoskeletal issues. As other NSAID, TS displays a marked side effects on the gastro-intestinal (GI) tract after oral administration. Traditional solid suppositories can cause pain and discomfort for patients, may reach the end of the colon; consequently, the drug can undergo the first-pass effect. TS liquid suppository (TS- [\hyperlink{Tolmetin Sodium}{PMID: 30303407}, Mohamed A Akl et al., 2019] The pharmacokinetics of tolmetin sodium were studied in five patients with rheumatoid arthritis (RA) and five normal volunteers to determine whether data derived from normals could be applied to RA patients. In addition, prostaglandin E (PGE) levels in synovial fluid were compared with tolmetin levels in serum and synovial fluid. Both groups received 400 mg tolmetin every 6 hours for seven days. During a 24-hour washout period after the dose of tolmetin (400 mg) on day 8, blood and urine samples were obtained from all study participants, and synovial fluid samples from the RA patients only. The patients continued into a second 24-hour drug-free period, after which they received a single 400-mg dose of tolmetin. Blood and urine samples were again collected. No clinically or statistically significant differences in tolmetin kinetics between normal volunteers and RA patients were found. A comparison of multiple-dose and single-dose results in the patient group showed an 11 per cent increase in the tolmetin serum concentration after multiple dosing. Total PGE levels in synovial fluid remained significantly depressed in the patient group for 24 hours after the 400-mg test dose of tolmetin on day 8. These findings suggest that tolmetin serum kinetics may not be an appropriate indicator of the duration of biologic activity of tolmetin. [\hyperlink{Tolmetin Sodium}{PMID: 30303407}, D E Furst et al., 1983]

\section*{Pexidartinib Hydrochloride}
\subsection*{Result}
\subsubsection*{Answer}

Unknown (for ages 14-17)
Unknown (for ages <14)

\subsubsection*{{Explanation}}
\hypertarget{Pexidartinib Hydrochloride}
A review of the available abstracts reveals two that specifically mention Pexidartinib Hydrochloride and its use in pediatric populations:

1. One abstract describes a phase I trial of pexidartinib in pediatric and young adult patients with refractory leukemias or solid tumors, including neurofibromatosis type 1-related plexiform neurofibromas. Twelve patients (median age 16 years, lower and upper quartile 14 and 16.5 years) were enrolled in the dose-escalation phase, and four in the expansion cohort. No dose-limiting toxicities were observed, and pexidartinib was well tolerated at all dose levels tested. The study determined a weight-based recommended phase II dose. The abstract concludes that pexidartinib in pediatric patients was well tolerated at all dose levels tested, achieved target inhibition, and resulted in a weight-based recommended phase II dose [\hyperlink{pmid_32943455}{PMID: 32943455}, Lauren H Boal et al., 2020].

2. Another abstract reports a phase I, nonrandomized, open-label multiple-dose study of pexidartinib in Asian patients with advanced solid tumors. The median age was 64 years (range 23-82), and there is no mention of pediatric patients in this study [\hyperlink{pmid_30825104}{PMID: 30825104}, Jih-Hsiang Lee et al., 2020].

Based on the first abstract, there is evidence from a targeted phase I study that pexidartinib hydrochloride was well tolerated in pediatric patients with a median age of 16 years (range not fully specified, but lower quartile 14 years). However, the sample size is small (n=12 in dose escalation, n=4 in expansion), and the study is phase I, which primarily assesses safety and dosing, not long-term safety or efficacy. There is no evidence from the abstracts regarding safety in children younger than 14 years.

Therefore, for children aged 14-17 years, there is limited phase I evidence suggesting pexidartinib hydrochloride is well tolerated in the short term, but no definitive affirmation of safety from larger or later-phase studies. For children younger than 14 years, there is no data available in the abstracts.

In summary:
- For ages 14-17: There is phase I evidence of short-term tolerability, but not enough to definitively affirm safety.
- For ages <14: No data available.

\subsection*{Abstracts}
\hypertarget{pmid_32943455}{S}imultaneously targeting the tumor and tumor microenvironment may hold promise in treating children with refractory solid tumors. Pexidartinib, an oral inhibitor of tyrosine kinases including colony stimulating factor 1 receptor (CSF-1R), KIT, and FLT3, is FDA approved in adults with tenosynovial giant cell tumor. A phase I trial was conducted in pediatric and young adult patients with refractory leukemias or solid tumors including neurofibromatosis type 1-related plexiform neurofibromas. A rolling six design with dose levels (DL) of 400 mg/m Twelve patients (4 per DL, 9 evaluable) enrolled on the dose-escalation phase and 4 patients enrolled in the expansion cohort: median (lower, upper quartile) age 16 (14, 16.5) years. No dose-limiting toxicities were observed. Pharmacokinetics appeared linear over three DLs. Pharmacokinetic modeling and simulation determined a weight-based recommended phase II dose (RP2D). Two patients had stable disease and 1 patient with peritoneal mesothelioma (C49+) had a sustained partial response (67\% RECIST reduction). Pharmacodynamic markers included a rise in plasma macrophage CSF (MCSF) levels and a decrease in absolute monocyte count. Pexidartinib in pediatric patients was well tolerated at all DL tested, achieved target inhibition, and resulted in a weight-based RPD2 dose. [\hyperlink{Pexidartinib Hydrochloride}{PMID: 32943455}, Lauren H Boal et al., 2020]

\hypertarget{pmid_7857353}{T}he efficacy and safety of pidotimod ((R)-3-[(S)-(5-oxo-2-pyrrolidinyl)carbonyl]-thiazolidine-4-carboxylic acid, PGT/1A, CAS 121808-62-6) were rated in a child population with a remote history of recurrent respiratory infections (RRI). This randomized double-blind multicenter clinical trial versus placebo, stratified by age groups, involved 748 children recruited in 69 Medical Centres. The trial consisted of a 60-day treatment period and a 90-day follow-up. At the end of the treatment period the pidotimod group showed a significant decrease in the number of RRI episodes and associated symptoms vs control group. As a consequence, there was a significant decrease in the number of days of absence from kindergarten or school and in the consumption of antibiotics and symptomatic drugs. Safety was good. The effect of the drug persisted after its withdrawal throughout the whole 90-day follow-up period. During this period there was a significantly lower RRI incidence rate in the pidotimod group than in the placebo group (p < 0.01). Because of its efficacy and safety, pidotimod may be rated as an excellent drug in the RRI management in children. [\hyperlink{Pexidartinib Hydrochloride}{PMID: 7857353}, P Careddu et al., 1994]

\hypertarget{pmid_17941284}{T}he safety of fexofenadine has been examined extensively in adults and school-age children. However, the safety of fexofenadine in children younger than 6 years has not been reported to date. To compare the safety and tolerability of twice-daily fexofenadine hydrochloride, 30 mg, and placebo in preschool children aged 2 to 5 years with allergic rhinitis. This was a multicenter, double-blind, randomized, placebo-controlled, parallel-group study, conducted between February 29, 2000, and June 14, 2001. Participants were randomized to either fexofenadine hydrochloride, 30 mg, or placebo twice daily for a 2-week period. To facilitate dosing, capsule content was mixed with applesauce (approximately 10 mL). Safety assessments depended on date of entry into the study because of an amendment to the protocol. Before the amendment, assessments included physical examination, vital signs reporting (oral temperature, heart rate, and respiratory rate), and adverse event (AE) reporting. After the amendment, safety assessments included laboratory testing (blood chemistry and hematology profiles), physical examination, 12-lead electrocardiography, and vital signs (oral temperature, blood pressure, heart rate, and respiratory rate) and AE reporting. Treatment-emergent AEs were observed in 116 of 231 participants receiving placebo and 111 of 222 receiving fexofenadine. These AEs were possibly related to study medication in 19 (8.2\%) and 21 (9.5\%) of the participants receiving placebo and fexofenadine, respectively, and most frequently involved the digestive system. No clinically relevant differences in laboratory measures, vital signs, and physical examinations were observed. The findings show that fexofenadine hydrochloride, 30 mg, is well tolerated and has a good safety profile in children aged 2 to 5 years with allergic rhinitis. [\hyperlink{Pexidartinib Hydrochloride}{PMID: 17941284}, Henry Milgrom et al., 2007]

\hypertarget{pmid_28741653}{C}hloral hydrate is commonly used to sedate children for painless procedures. Children may recover more quickly after sedation with dexmedetomidine, which has a shorter half-life. We randomly allocated 196 children to chloral hydrate syrup 50 mg.kg [\hyperlink{Pexidartinib Hydrochloride}{PMID: 28741653}, V M Yuen et al., 2017] Because patient movement during echocardiography interferes with diagnostic quality, many institutions sedate children who are unable to cooperate. The purpose of this review was to determine the efficacy and safety of oral pentobarbital for sedation during pediatric transthoracic echocardiography. Echocardiography laboratory quality assurance data were recorded for 12 years. Sedation data included adverse events, dosing, and failed sedation. The study population was grouped by age: neonates (<1 month), infants (1-12 months), and young children (1-4 years). A total of 9796 patients underwent sedation by oral pentobarbital. The overall sedation success rate was 98.7\%, and 99\% of these patients remained sedated long enough for study. The overall adverse event rate was 0.5\%. Second doses, failed sedation, and adverse events were more common in the young children. Oral pentobarbital is an effective and safe sedative for pediatric transthoracic echocardiography. Because of decreased efficacy and an increased incidence of adverse events, alternative sedation strategies may be beneficial in children aged 1 to 4 years. [\hyperlink{Pexidartinib Hydrochloride}{PMID: 28741653}, Charles N Warden et al., 2010]

\hypertarget{pmid_18702885}{A}llergic rhinitis (AR) is a common chronic condition in children and may impact a child's quality of life. Increasing treatment compliance may improve quality of life. An oral suspension of fexofenadine hydrochloride (HCl) has been developed to ease administration to children and may, therefore, improve treatment compliance. The purpose of this study was to assess the pharmacokinetic behavior, safety, and tolerability of a single dose of fexofenadine HCl oral suspension administered to children aged 2-5 years with allergic rhinitis. Children (aged 2-5 years) with AR were recruited in a multicenter, open-label, single-dose study. Fexofenadine HCl (30 mg) was administered as a 6-mg/mL suspension (5 mL). Plasma samples were collected up to 24 hours postdose. Adverse events (AEs); electrocardiograms (ECGs); vital signs; and clinical laboratory tests for hematology, blood chemistry, and urinalysis were analyzed to evaluate safety and tolerability. Fifty subjects completed the study. Mean maximum plasma concentration of fexofenadine was 224 ng/mL, and mean area under the plasma concentration curve was 898 ng . hour/mL. Treatment-emergent AEs were mild in intensity and reported in a total of seven subjects. No trends or clinically meaningful changes in mean ECG, vital sign, or clinical laboratory test data occurred during the study. In children aged 2-5 years, the exposure after a 30-mg dose of fexofenadine HCl suspension was similar to the exposures previously seen after a 30- and 60-mg dose of fexofenadine HCl in children aged 6-11 years and in adults, respectively. The suspension was also well tolerated. [\hyperlink{Pexidartinib Hydrochloride}{PMID: 18702885}, Nathan Segall et al., ]

\hypertarget{pmid_36123258}{P}edvaxHIB® is an effective pediatric vaccine for protecting infants from invasive gram-negative bacterium Haemophilus influenzae type b. It is a highly purified capsular polysaccharide, polyribosylribitol phosphate that is covalently linked to an outer membrane protein complex of Neisseria meningitidis. PRP is first derivatized with an organic linker, followed by the coupling of a butadiamine group, and then at the end terminal, a bromoacetyl group is attached for conjugation with thiolated OMPC. The stability of the bromide group in derivatized PRP is monitored by two different methods, capillary electrophoresis and NMR spectroscopy. The loss of the bromide group is detected by measuring the amount of free bromide ion liberated using capillary electrophoresis and by observing a change in amide proton peaks near the bromide group using NMR. The two methods give similar rate hydrolysis results, therefore both can be employed as quick stability tools for bromoacetylation PRP content during manufacturing. [\hyperlink{Pexidartinib Hydrochloride}{PMID: 36123258}, Richard R Rustandi et al., 2022]

\hypertarget{pmid_10323625}{Y}oung children often appear bothered by ear pain during ascent and descent while traveling on commercial airplanes. While pseudoephedrine hydrochloride is effective in decreasing the risk for earache in adults with recurrent air travel-associated ear pain, such use in children has not been studied. To assess the efficacy and side effects of prophylactic pseudoephedrine in children traveling by air. A placebo-controlled, double-blind clinical trial. Children aged 6 months to 6 years were included in this study. Pseudoephedrine hydrochloride (1 mg/kg body weight) or placebo was administered 30 to 60 minutes prior to departure on commercial air flights. Caregivers noted historical details and the degree of apparent ear pain, drowsiness, and excitability with ascent and descent. Ninety-one flights involving 50 children were studied, with ear pain being reported in 13 (14\%) of flights. Ear pain was not associated with a history of air travel-associated ear pain, recent ear infection, or recent upper airway symptoms. Pseudoephedrine use was not associated with a decrease in ear pain during either ascent (4\% with pseudoephedrine vs 5\% with placebo; P approximately 1.00) or descent (12\% with pseudoephedrine vs. 13\% with placebo; P approximately 1.00). Pseudoephedrine use was, however, linked to drowsiness at takeoff (60\% with pseudoephedrine vs. 27\% with placebo; P = .003) but not at landing (P = .39). Treatment was not associated with excitability at takeoff (P = .09) or landing (P approximately 1.00). Ear pain is not uncommon in children traveling by commercial aircraft. The predeparture use of pseudoephedrine does not decrease the risk for in-flight ear pain in children but is associated with drowsiness. [\hyperlink{Pexidartinib Hydrochloride}{PMID: 10323625}, B J Buchanan et al., 1999]

\hypertarget{pmid_30463814}{D}exmendetomidine hydrochloride (DEX) is a new common adrenergic receptor agonist, which not only keeps children calm but also has analgesic effect. Dexmedetomidine hydrochloride will enable children to maintain the natural non-REM sleep, which can be stimulated sedation or language arousal. The aim of this study is to observe the sedative effect and adverse drug reactions of dexmedetomidine hydrochloride injection and propofol injection in MRI examination. In this study, no children in the experimental group were required to add sedative drugs, and 2 cases in the control group were treated with sedative drugs. In experimental group, it used dexmedetomidine hydrochloride as (1.64±0.91) g/kg; in control group, dosage of narcotic drugs as (5.26±1.82) g/kg, and the total complication rate of the children in the experimental group was lower than that of the control group (P<0.05). After returning to the ward, the doses of phenobarbital sedation were dexmedetomidine group (4.28±1.53) mg/kg and propofol group (6.40±1.71) mg/kg. There was significant difference between the two groups. The total complication rate in the experimental group was lower than that in the control group (P<0.05). The quality of MRI in the test group was significantly higher than that in the control group, which showed that dexmedetomidine hydrochloride could provide a satisfactory sedative effect in the MRI examination of children. To sum up, dexmedetomidine hydrochloride is a wide range of clinical applications. It is an effective drug for the maintenance of sedation in clinical disease treatment. It is flexible in the way of administration and with less adverse reactions. It is suitable for popularization and application in clinical practice. [\hyperlink{Pexidartinib Hydrochloride}{PMID: 30463814}, Zhendong Yang et al., 2018]

\hypertarget{pmid_28275979}{S}edation is often required for children undergoing diagnostic procedures. Chloral hydrate has been one of the sedative drugs most used in children over the last 3 decades, with supporting evidence for its efficacy and safety. Recently, chloral hydrate was banned in Italy and France, in consideration of evidence of its carcinogenicity and genotoxicity. Dexmedetomidine is a sedative with unique properties that has been increasingly used for procedural sedation in children. Several studies demonstrated its efficacy and safety for sedation in non-painful diagnostic procedures. Dexmedetomidine's impact on respiratory drive and airway patency and tone is much less when compared to the majority of other sedative agents. Administration via the intranasal route allows satisfactory procedural success rates. Studies that specifically compared intranasal dexmedetomidine and chloral hydrate for children undergoing non-painful procedures showed that dexmedetomidine was as effective as and safer than chloral hydrate. For these reasons, we suggest that intranasal dexmedetomidine could be a suitable alternative to chloral hydrate. [\hyperlink{Pexidartinib Hydrochloride}{PMID: 28275979}, Giorgio Cozzi et al., 2017]

\hypertarget{pmid_30825104}{B}ackground Pexidartinib, a novel, orally administered small-molecule tyrosine kinase inhibitor, has strong selectivity against colony-stimulating factor 1 receptor. This phase I, nonrandomized, open-label multiple-dose study evaluated pexidartinib safety and efficacy in Asian patients with symptomatic, advanced solid tumors. Materials and Methods Patients received pexidartinib: cohort 1, 600 mg/d; cohort 2, 1000 mg/d for 2 weeks, then 800 mg/d. Primary objectives assessed pexidartinib safety and tolerability, and determined the recommended phase 2 dose; secondary objectives evaluated efficacy and pharmacokinetic profile. Results All 11 patients (6 males, 5 females; median age 64, range 23-82; cohort 1 n = 3; cohort 2 n = 8) experienced at least one treatment-emergent adverse event; 5 experienced at least one grade ≥ 3 adverse event, most commonly (18\%) for each of the following: increased aspartate aminotransferase, blood alkaline phosphatase, gamma-glutamyl transferase, and anemia. Recommended phase 2 dose was 1000 mg/d for 2 weeks and 800 mg/d thereafter. Pexidartinib exposure, area under the plasma concentration-time curve from zero to 8 h (AUC [\hyperlink{Pexidartinib Hydrochloride}{PMID: 30825104}, Jih-Hsiang Lee et al., 2020] Diagnostic and therapeutic procedures in children are made easier using sedation. However, there is no consensus about which drug should be used to achieve this. Furthermore, none of the drugs used for sedation are risk free. The aim of this work is to study sedation indications, effectiveness, and safety at our center. A prospective observational study conducted at the Pediatric Day Care Unit, King Fahad National Guard Hospital, Riyadh, Saudi Arabia. The study covered 17.5 weeks in 2 periods: May 9th 1999 to June 13th 1999 and October 31st 2001 to February 11th 2002. Children <12 years were included. Collected data included demographics, indication, drug dosing and outcome. Data were reported as mean +/- SD. We included 148 patients, age 38 +/- 30 months. Adequate sedation was achieved in 79\% after initial chloral hydrate (CH) dose of 56.9 +/- 9.3 mg/kg, in 95\% after adding 18.5 +/- 6.4 mg/kg CH and in 96\% after adding second drug. Compared to nonrespondents, first CH dose respondents were younger and lower in weight. The CH side effects were few and mild. Chloral hydrate is a safe and effective agent for sedation in children with an age and weight dependent response. [\hyperlink{Pexidartinib Hydrochloride}{PMID: 30825104}, Omar M Hijazi et al., 2005]

\hypertarget{pmid_8010205}{T}he purpose of this prospective study was to evaluate the safety and efficacy of thioridazine as an adjunct to chloral hydrate sedation when children undergoing MR imaging are difficult to sedate. All 87 children in the study either could not be sedated with chloral hydrate alone or were mentally retarded. Thioridazine (2-4 mg/kg) was administered orally 2 hr before and chloral hydrate (50-100 mg/kg) was administered orally 30 min before the 104 MR examinations. All children were monitored by continuous pulse oximetry. All images were individually evaluated by pediatric radiologists and were graded acceptable if they contained only minimal motion artifact or no motion artifact. Studies were considered successful only when 95\% or more of the images were acceptable. MR imaging was successful in 93 (89\%) of 104 examinations. The success rate for children entered into the study because of prior failure of chloral hydrate sedation was not significantly different from the success rate for children with mental retardation. A tendency for increasing failure rate with age was not significant. No serious complications occurred during the study. The most common adverse reaction, transient reduced oxygen saturation, was seen in five children. Other adverse effects encountered were vomiting in four children, hyperactivity in two children, transient tachycardia in one child, and prolonged sedation in one child. No child required hospitalization because of an adverse reaction to sedation. The study indicates that thioridazine is a safe and effective adjunct to chloral hydrate when a child undergoing MR imaging is difficult to sedate. [\hyperlink{Pexidartinib Hydrochloride}{PMID: 8010205}, S B Greenberg et al., 1994]

\hypertarget{pmid_2326439}{T}his paper reports on 350 pediatric patients who were studied over a 17-month period to determine the efficacy and safety of oral and intramuscular sedation techniques. The protocol using oral chloral hydrate, 50 mgm/kg, for infants under 1 year of age or intramuscular pentobarbital, 5 mgm/kg, for children over 1 year was found to be an effective, safe and fairly simple approach to pediatric sedation. Of the 350 sedated patients, 343 (98 percent) had satisfactory scans on the same day the examination was scheduled after a single dose or an initial dose and supplementary sedation. [\hyperlink{Pexidartinib Hydrochloride}{PMID: 2326439}, J B Temme et al., ]

\hypertarget{pmid_24447296}{C}hloral hydrate is the most commonly used sedative for paediatric diagnostic procedures in China with a success rate of around 80\%. Intranasal dexmedetomidine is used for rescue sedation in our centre. This prospective investigation evaluated 213 children aged one month to 10 years who were not adequately sedated following administration of chloral hydrate. Children were randomly assigned to receive rescue intranasal dexmedetomidine at 1 μg.kg(-1) (group 1), 1.5 μg.kg(-1) (group 2) or 2 μg.kg(-1) (group 3). The sedation level was assessed every 10 min using a modified observer's assessment of alertness/sedation scale. Successful rescue sedation in groups 1, 2 and 3 were 56 (83.6\%), 66 (89.2\%) and 51 (96.2\%), respectively. Increasing the rescue dose was associated with an increased success rate with an odds ratio of 4.12 (95\% CI 1.13-14.98), p = 0.032. We conclude that intranasal dexmedetomidine is effective for sedation in children who do not respond to chloral hydrate.  [\hyperlink{Pexidartinib Hydrochloride}{PMID: 24447296}, B L Li et al., 2014] The aim of this study was to compare the efficacy and safety of different oral chloral hydrate and dexmedetomidine doses used for sedation during electroencephalography (EEG) in children. One hundred sixty children aged 1 to 9 years with American Society of Anesthesiologists physical status I-II who were uncooperative during EEG recording or who were referred to our electrodiagnostic unit for sleep EEG were included to the study. The patients were randomly assigned into 4 groups. In groups D1 and D2, patients received oral dexmedetomidine doses of 2 and 3 µg/kg, respectively. In group C1 and C2, patients received oral chloral hydrate doses of 50 and 100 mg/kg, respectively. The induction time was significantly shorter in group C2 compared with other groups (P = .000). The rate of adverse effects was significantly higher in group C2 compared with the dexmedetomidine groups (D1 and D2; P = .004). In conclusion, dexmedetomidine can be used safely for sedation during EEG in children.  [\hyperlink{Pexidartinib Hydrochloride}{PMID: 24447296}, Hakan Gumus et al., 2015] To determine the safety and efficacy of high-dose oral chloral hydrate for pediatric ophthalmic procedures. This study is a retrospective review of a quality audit of pediatric sedation for ophthalmic evaluation and imaging performed at King Khaled Eye Specialist Hospital between January 1 and December 31, 2011, in children aged 1 month to 6 years. Three hundred fifty-eight of 380 (94.2\%) sedation procedures were successful after a single dose of chloral hydrate, with 356 of 380 (93.7\%) children sedated within 45 minutes of the first dose. The total success rate of the sedation procedure increased to 97.9\% (372 of 380) when a second dose was administered. Children adequately sedated after a single dose of chloral hydrate were on average younger and weighed less than children who required additional doses. No major adverse events were documented. The use of chloral hydrate sedation for ophthalmic evaluation and imaging was safe and effective in this patient population with a high rate of procedure completion. [\hyperlink{Pexidartinib Hydrochloride}{PMID: 24447296}, Michelle E Wilson et al., ]

\hypertarget{pmid_30124097}{P}ediatric patients present changing physiological features. Because of the lack of land suitable for commercial management, pediatric specialties very often need to prepare extemporaneous formulations to improve the dosage and administration of drugs for children. Oral liquid formulations are the most suitable for pediatric patients. Clonidine is widely used in the pediatric population for opioid withdrawal, hypertensive crisis, attention deficit disorders and hyperactivity syndrome, and as an analgesic in neuropathic cancer pain. The objective was to study the physicochemical and microbiological stability and determine the shelf life of an oral solution containing 20 µg/mL clonidine hydrochloride in different storage conditions (5 ± 3 °C, 25 ± 3 °C, and 40 ± 2 °C). Using raw material with excipients safe for all pediatric age groups, two oral liquid formulations of clonidine hydrochloride were designed (with and without preservatives). Solutions stored at 5 ± 3 °C (with and without preservatives) were physically and microbiologically stable for at least 90 days in closed containers and for 42 days after opening. Two oral solutions of clonidine hydrochloride 20 µg/mL were developed for pediatric use from raw materials that are readily available and easy to process, containing safe excipients that are stable over a long period of time. [\hyperlink{Pexidartinib Hydrochloride}{PMID: 30124097}, V Merino-Bohórquez et al., 2019]

\hypertarget{pmid_7857354}{T}he efficacy and safety of a new synthetic immunostimulant pidotimod ((R)-3-[(S)-(5-oxo-2-pyrrolidinyl) carbonyl]-thiazolidine-4-carboxylic acid, PGT/1A, CAS 121808-62-6) in recurrent infections of the primary airways were assessed in a group of 416 children with a history of recurrent respiratory infections (RRI). This was a double-blind randomized trial of pidotimod vs. placebo, consisting of a treatment period of 60 days and a follow-up period of 3 months. A reduction in the duration and frequency of infectious episodes in the group of children treated with pidotimod (one 400 mg oral bottle daily) was observed which was statistically different from the placebo group. The protective effect produced by pidotimod was also confirmed by a series of recordings made over the five-month observation period, which showed a significant reduction in the number of days of fever, the severity of the signs and symptoms of acute episodes, use of antibiotics and antipyretic drugs and absence from school or nursery school. Safety was excellent. [\hyperlink{Pexidartinib Hydrochloride}{PMID: 7857354}, D Passali et al., 1994]

\hypertarget{pmid_3257872}{A} 12-month double-blind, parallel, randomized, placebo-controlled multicenter trial of D-penicillamine and hydroxychloroquine was conducted in 162 children with juvenile rheumatoid arthritis in the United States and in the Union of Soviet Socialist Republics. No statistically significant intergroup differences were detected in primary outcome variables. We investigated the possible existence of select subgroups of patients who have a higher likelihood of response to active drugs than to placebo. Using previously published criteria, each patient was classified as a responder or nonresponder, and their demographic and disease characteristics at baseline were compared. We were unable to identify a subgroup of individuals who were more likely to respond to D-penicillamine or hydroxychloroquine than to placebo. [\hyperlink{Pexidartinib Hydrochloride}{PMID: 3257872}, E H Giannini et al., 1988]

\hypertarget{pmid_3517643}{O}ne hundred sixty-two children with severe juvenile rheumatoid arthritis were entered in a randomized, double-blind, placebo-controlled 12-month clinical trial designed to establish the efficacy and safety of two slower-acting antirheumatic drugs, penicillamine and hydroxychloroquine. The study was a cooperative effort of the United States and the Soviet Union. One group of subjects received 10 mg of penicillamine per kilogram of body weight per day, another group received 6 mg of hydroxychloroquine per kilogram daily, and a third group received placebo. All three groups were allowed a single concurrent nonsteroidal antiinflammatory drug, but no other antirheumatic medications, including corticosteroids. All three groups had dramatic improvement in many of the clinical and laboratory outcome variables after one year of study. There were no significant differences in efficacy between the penicillamine and placebo groups. Pain on movement was the only index of articular disease that was alleviated more by hydroxychloroquine than by placebo. Serious adverse drug reactions attributable to the active agents were rare. We were unable to demonstrate that, in the presence of a nonsteroidal antiinflammatory drug, either penicillamine or hydroxychloroquine is superior to placebo in the treatment of children with juvenile rheumatoid arthritis. [\hyperlink{Pexidartinib Hydrochloride}{PMID: 3517643}, E J Brewer et al., 1986]

\hypertarget{pmid_34719411}{T}o study the safety and efficacy of dexmedetomidine hydrochloride combined with midazolam in fiberoptic bronchoscopy in children. A total of 118 children who planned to undergo fiberoptic bronchoscopy from September 2018 to February 2021 were enrolled. They were divided into a control group ( Compared with the control group, the observation group had significantly decreased MAP at T Dexmedetomidine hydrochloride combined with midazolam is a safe and effective way to administer general anesthesia for fiberoptic bronchoscopy in children, which can ensure stable vital signs during examination, reduce intraoperative adverse reactions and postoperative agitation, shorten examination time, and increase amnesic effect. [\hyperlink{Pexidartinib Hydrochloride}{PMID: 34719411}, Jin Zhang et al., 2021]

\hypertarget{pmid_18365604}{P}seudoephedrine hydrochloride (PEH) is a sympathomimetic agent that is widely used in common cold disease in children. Though side effects of PEH are well known, it is preferred by many pediatricians in order to benefit from its symptomatic relief in common cold disease. A case of acute urinary retention due to PEH in a three-year-old boy is reported. The aim of this case report is to emphasize the clinical importance and differential diagnosis of PEH overdose in children and to discuss the appropriate treatment approach to PEH overdose in the emergency department. [\hyperlink{Pexidartinib Hydrochloride}{PMID: 18365604}, Tutku Soyer et al., ]

\hypertarget{pmid_24777716}{P}aroxysmal nocturnal hemoglobinuria (PNH) is rare in children, but represents a similarly serious and chronic condition as in adults. Children with PNH frequently experience complications of chronic hemolysis, recurrent thrombosis, marrow failure, serious infections, abdominal pain, chronic fatigue, and decreased quality of life with reduced survival. The terminal complement inhibitor eculizumab is proven to be effective and safe in adults and approved by the FDA for treatment of PNH. This 12-week, open-label, multi-center phase I/II study evaluated pharmacokinetics, pharmacodynamics, efficacy, and safety in seven children with PNH 11-17 years of age. Eculizumab was intravenously administered at 600 mg weekly for 4 weeks, 900 mg in week 5, and 900 mg every 2 weeks thereafter (http://clinicaltrials.gov NCT00867932). Eculizumab therapy resulted in complete and sustained inhibition of hemolysis in all participants with a reduction of lactate dehydrogenase to normal levels. All hematological parameters stabilized. No definitive, study drug-related adverse events were observed. Only one severe SAE of hospitalization due to aplastic anemia occurred, which was not study drug-related. Eculizumab appears to be a safe and effective therapy for children with PNH. [\hyperlink{Pexidartinib Hydrochloride}{PMID: 24777716}, Ulrike M Reiss et al., 2014]

\hypertarget{pmid_20527137}{O}nly a few corticosteroids for topical use have proven safe and effective in pediatric populations down to 3 months of age. The authors report the results of a study designed to assess the efficacy and safety of hydrocortisone butyrate (HCB) 0.1\% in lipocream (LCr) vehicle in infants and children. A total of 264 boys and girls 3 months to less than 18 years old, with stable, mild to moderate atopic dermatitis affecting at least 10\% body surface area applied HCB 0.1\% in LCr or LCr alone twice daily for up to 1 month without occlusion. Primary end-points included: percent of patients who achieved treatment success based on physician global assessments. Secondary endpoint included: difference in pruritus and Eczema Area and Severity Index (EASI) at day 29. Treatment was significant (P < 0.001) for HCB 0.1\% LCr over vehicle. No serious nor significant adverse events were reported. Results are representative of a short duration treatment for a chronic disease. HCB 0.1\% in LCr is more effective than its vehicle in pediatric populations down to 3 months of age without significant adverse events when used twice a day for up to 1 month. [\hyperlink{Pexidartinib Hydrochloride}{PMID: 20527137}, William Abramovits et al., ]

\section*{Larotrectinib Sulfate}
\subsection*{Result}
\subsubsection*{Answer}

Yes (0–1 year, infants)
Yes (4 months and older, children)

\subsubsection*{{Explanation}}
\hypertarget{Larotrectinib Sulfate}
Based on the abstracts available, there are two that provide targeted data on the safety of Larotrectinib Sulfate in children:

1. One case series reports on 8 pediatric patients with infantile fibrosarcoma, including newborns and infants (children under one year of age). Three patients received larotrectinib as first-line therapy. The abstract states: "No surgery was needed with larotrectinib, which led to the rapid and safe remission of tumours, even in unusual anatomical locations. No significant adverse effects were observed with larotrectinib. Our case series supports that larotrectinib may be a therapeutic option for newborn and infant patients with infantile fibrosarcoma, especially in uncommon locations" [\hyperlink{pmid_37097530}{PMID: 37097530}, María Dolores Corral Sánchez et al., 2023]. This provides direct evidence of safety in newborns and infants (0–1 year).

2. A larger study enrolled 55 patients with TRK fusion-positive cancers, ranging in age from 4 months to 76 years, including children. The study included a phase 1–2 trial specifically involving children. The abstract reports: "Adverse events were predominantly of grade 1, and no adverse event of grade 3 or 4 that was considered by the investigators to be related to larotrectinib occurred in more than 5\% of patients. No patient discontinued larotrectinib owing to drug-related adverse events." The study concludes: "Larotrectinib had marked and durable antitumor activity in patients with TRK fusion-positive cancer, regardless of the age of the patient or of the tumor type" [\hyperlink{pmid_29466156}{PMID: 29466156}, Alexander Drilon et al., 2018]. This provides evidence of safety in children as young as 4 months and up.

Both abstracts affirm that targeted studies have been conducted in children, including infants, and report that larotrectinib was safe in these populations, with no significant or severe drug-related adverse events leading to discontinuation. The age ranges specifically supported by the data are newborns/infants (0–1 year) and children as young as 4 months and older.

No abstracts report targeted studies showing larotrectinib is unsafe in children.

Therefore, based on the available abstracts, larotrectinib sulfate is affirmed as safe for use in children, including infants as young as 0–1 year, when used for appropriate indications (e.g., NTRK fusion-positive cancers).

\subsection*{Abstracts}
\hypertarget{pmid_37097530}{I}nfantile fibrosarcoma is the most frequent soft tissue sarcoma in newborns or children under one year of age. This tumour often implies high local aggressiveness and surgical morbidity. The large majority of these patients carry the ETV6-NTRK3 oncogenic fusion. Hence, the TRK inhibitor larotrectinib emerged as an efficacious and safe alternative to chemotherapy for NTRK fusion-positive and metastatic or unresectable tumours. However, real-world evidence is still required for updating soft-tissue sarcoma practice guidelines. To report our experience with the use of larotrectinib in pediatric patients. Our case series shows the clinical evolution of 8 patients with infantile fibrosarcoma under different treatments. All patients enrolled in this study received informed consent for any treatment. Three patients received larotrectinib in first line. No surgery was needed with larotrectinib, which led to the rapid and safe remission of tumours, even in unusual anatomical locations. No significant adverse effects were observed with larotrectinib. Our case series supports that larotrectinib may be a therapeutic option for newborn and infant patients with infantile fibrosarcoma, especially in uncommon locations. [\hyperlink{Larotrectinib Sulfate}{PMID: 37097530}, María Dolores Corral Sánchez et al., 2023]

\hypertarget{pmid_23127263}{T}he correlation between lamotrigine serum concentration, efficacy, and toxicity in children is controversial. The database of the Clinical Pharmacology Laboratory at Assaf Harofeh Medical Center was retrospectively searched to identify lamotrigine serum concentrations in children aged 2-19 years with refractory epilepsy who received lamotrigine as monotherapy or polytherapy from 2007-2010. Data collected included age at epilepsy onset, additional antiepileptic drugs, lamotrigine dose, monthly seizure frequency before and after lamotrigine treatment, and side effects. Sixty blood samples were collected from 42 children aged 10.1 ± 4.9 years (range, 2-20 years). Seizure types included complex partial (n = 28), simple partial (n = 7), absence (n = 2), and generalized tonic-clonic (n = 23). Decreased seizure frequency was observed in 38 (63.3\%) patients. No correlation with lamotrigine serum concentration was evident, but seizure frequency was significantly influenced by age and lamotrigine dose. Side effects were reported in 21 (35\%) patients. Only diplopia was significantly correlated with lamotrigine serum concentration. Lamotrigine was more effective at lower doses and in older children. Lamotrigine serum concentration correlated significantly with diplopia, but not with other side effects or with clinical efficacy. Overall, lamotrigine is effective and safe in children with refractory epilepsy. [\hyperlink{Larotrectinib Sulfate}{PMID: 23127263}, Eli Heyman et al., 2012]

\hypertarget{pmid_29466156}{F}usions involving one of three tropomyosin receptor kinases (TRK) occur in diverse cancers in children and adults. We evaluated the efficacy and safety of larotrectinib, a highly selective TRK inhibitor, in adults and children who had tumors with these fusions. We enrolled patients with consecutively and prospectively identified TRK fusion-positive cancers, detected by molecular profiling as routinely performed at each site, into one of three protocols: a phase 1 study involving adults, a phase 1-2 study involving children, or a phase 2 study involving adolescents and adults. The primary end point for the combined analysis was the overall response rate according to independent review. Secondary end points included duration of response, progression-free survival, and safety. A total of 55 patients, ranging in age from 4 months to 76 years, were enrolled and treated. Patients had 17 unique TRK fusion-positive tumor types. The overall response rate was 75\% (95\% confidence interval [CI], 61 to 85) according to independent review and 80\% (95\% CI, 67 to 90) according to investigator assessment. At 1 year, 71\% of the responses were ongoing and 55\% of the patients remained progression-free. The median duration of response and progression-free survival had not been reached. At a median follow-up of 9.4 months, 86\% of the patients with a response (38 of 44 patients) were continuing treatment or had undergone surgery that was intended to be curative. Adverse events were predominantly of grade 1, and no adverse event of grade 3 or 4 that was considered by the investigators to be related to larotrectinib occurred in more than 5\% of patients. No patient discontinued larotrectinib owing to drug-related adverse events. Larotrectinib had marked and durable antitumor activity in patients with TRK fusion-positive cancer, regardless of the age of the patient or of the tumor type. (Funded by Loxo Oncology and others; ClinicalTrials.gov numbers, NCT02122913 , NCT02637687 , and NCT02576431 .). [\hyperlink{Larotrectinib Sulfate}{PMID: 29466156}, Alexander Drilon et al., 2018]

\hypertarget{pmid_10768153}{L}amotrigine (Lamictal, Glaxo Wellcome) is a drug which is used as add-on therapy in patients with refractory epilepsy. Several previous studies have demonstrated the efficacy of lamotrigine monotherapy, but only few have been done in pediatric patients. The aim of our study was the assessment of efficacy and tolerability of lamotrigine monotherapy in children with newly diagnosed partial epilepsy. Lamictal was used in 19 children (11 boys and 8 girls), aged 3-16 years. 17 patients demonstrated complex partial seizures (with or without secondarily generalisation), 2 children had simplex partial seizures. Symptomatic epilepsy was diagnosed in 10 patients and cryptogenic epilepsy in 9 cases. The drug was administered at the dose of 3.87 +/- 1.02 mg/kg/day during 24 weeks. Three children withdrew from the study because of adverse events: one patient developed rash, two ones seizure exacerbation. Lamictal produced of at least 50\% reduction in seizure frequency in 12 (63.15\%) children, included 10 seizure-free patients. One third patients experienced EEG improvement. The most common adverse effects were gastrointestinal and sleep disturbances, infections, dizziness, all of them were mild and transient and observed more often in children under 12 years of age. Lamotrigine monotherapy is effective and safe for treatment of newly diagnosed cryptogenic and symptomatic epilepsy with partial seizures but further studies are necessary specially in young children. [\hyperlink{Larotrectinib Sulfate}{PMID: 10768153}, E Sołowiej et al., 2000]

\hypertarget{pmid_25593242}{R}ituximab (RTX) has been used to treat many pediatric autoimmune conditions. We investigated the safety and efficacy of RTX in a variety of pediatric autoimmune diseases, especially systemic lupus erythematosus (SLE). Retrospective study of children treated with RTX. Effectiveness data was recorded for patients with at least 12 months of followup; safety data was recorded for all subjects. The study included 104 children; 50 had SLE. Improvements in corticosteroid dosage, physician's global assessment of disease activity, and SLE-associated markers of disease activity were seen. The incidence of hospitalized infections was similar to previous studies of patients with childhood-onset SLE. RTX can be safely administered to children and appears to contribute to decreased disease activity and steroid burden. [\hyperlink{Larotrectinib Sulfate}{PMID: 25593242}, Ajay Tambralli et al., 2015]

\hypertarget{pmid_9165509}{T}he role of lamotrigine (LTG) in childhood epilepsy is emerging. We evaluated the efficacy and adverse effects of LTG in an open, prospective study of 56 children with generalized epilepsies. Six (11\%) children became seizure-free, and 24 (43\%) had greater than 50\% reduction in seizure frequency. LTG was effective against a broad range of generalized seizure types. Three of 15 patients with Lennox-Gastaut syndrome achieved complete seizure control and eight demonstrated 50 to 99\% improvement in seizure control. Increase in seizures (7) and rash (5) were the most common side effects. After valproate was discontinued, LTG therapy was resumed, with no recurrence of rash in any patient. This study suggests that LTG may be a useful drug in the treatment of generalized epilepsies in children. [\hyperlink{Larotrectinib Sulfate}{PMID: 9165509}, K Farrell et al., 1997]

\hypertarget{pmid_10403222}{T}o investigate whether lamotrigine (LTG) monotherapy is effective and safe for newly diagnosed typical absence seizures in children and adolescents (aged 3-15 years, n = 45). A "responder-enriched" study design was used: open-label dose escalation was followed by placebo-controlled, double-blind testing of LTG. Conventional hyperventilation testing with EEG recording was used to confirm diagnoses and assess treatment success defined as complete freedom from seizures. Ambulatory 24-h EEG recordings provided supporting evidence of effectiveness. Safety was assessed by evaluation of adverse events, vital signs, and physical, neurologic, and laboratory examinations. Plasma samples were taken to evaluate the pharmacokinetics of LTG. During initial open-label dose escalation, 71.4\% of patients (intent-to-treat) or 82\% (per protocol analysis) became seizure free; individual patients responded at doses ranging from 2 to 15 mg/kg/day (median, 5.0). In the placebo-controlled, double-blind phase of the study, statistically significantly more patients remained seizure free when treated with LTG (62\%) than with placebo (21\%; p < 0.02; for the intent-to-treat analysis). Mean plasma concentrations of LTG, were linearly related to dose, although there was substantial interindividual variation. No patients were withdrawn from the study for any safety-related reason. LTG monotherapy is effective for typical absence seizures in children and is generally well tolerated. [\hyperlink{Larotrectinib Sulfate}{PMID: 10403222}, L M Frank et al., 1999]

\hypertarget{pmid_16028153}{B}ecause of concerns about arthrotoxicity, fluoroquinolones are restricted for use in children. This study describes the safety and efficacy of gatifloxacin when used for treatment of children with recurrent acute otitis media (ROM) or acute otitis media (AOM) treatment failure (AOMTF). We performed an analysis of 867 children included in 4 clinical trials who had ROM and/or AOMTF and were treated with gatifloxacin (10 mg/kg once daily for 10 days). Gatifloxacin had adverse event rates that were similar overall to those of a comparator antibiotic (amoxicillin-clavulanate), except for increased diarrhea in children <2 years old receiving amoxicillin-clavulanate. There was no evidence of arthrotoxicity, hepatotoxicity, alteration of glucose homeostasis, or central nervous system toxicity acutely or during 1 year follow-up in any child. Regarding efficacy, in 2 noncomparative trials, the gatifloxacin cure rate of AOM was 89\% (95\% confidence interval [CI], 83\%-95\%) at the test of cure (TOC) visit, 3-10 days after completion of therapy. In 2 comparative trials of gatifloxacin versus amoxicillin-clavulanate, the efficacy of gatifloxacin was 88\% (95\% CI, 82\%-94\%). Gatifloxacin led to better clinical outcomes than amoxicillin-clavulanate for AOMTF (91\% vs. 81\%; P=.029), for AOMTF and age <2 years old (89\% vs. 69\%; P=.009), and for severe AOM in children <2 years old (90\% vs. 75\%; P=.012). Among children with AOMTF previously treated with amoxicillin-clavulanate or ceftriaxone injections, gatifloxacin cure rates were high (88\% and 75\%, respectively). Gatifloxacin appears to be safe for children, with no evidence of producing arthrotoxicity in 867 children exposed to the antibiotic when used as treatment for ROM and AOMTF. [\hyperlink{Larotrectinib Sulfate}{PMID: 16028153}, Michael E Pichichero et al., 2005]

\hypertarget{pmid_10563619}{T}o compare the safety and efficacy of add-on lamotrigine and placebo in the treatment of children and adolescents with partial seizures. Add-on and monotherapy lamotrigine is safe and effective in adults with partial seizures, and reports of preliminary uncontrolled trials suggest similar benefits in children. We studied 201 children with diagnoses of partial seizures of any subtype currently receiving stable conventional regimens of antiepileptic therapy at 40 study sites in the United States and France. After a baseline observation period (to confirm that more than four seizures occurred in each of two consecutive 4-week periods), patients were randomized to add-on lamotrigine or placebo therapy. A 6-week dose-escalation period was followed by a 12-week maintenance period. Compared with placebo, lamotrigine significantly reduced the frequency of all partial seizures and the frequency of secondarily generalized partial seizures in these treatment-resistant patients. The most commonly reported adverse events in the lamotrigine-treated patients were vomiting, somnolence, and infection; the frequency of these and other adverse events was similar to that in the placebo-treated group, with the exception of ataxia, dizziness, tremor, and nausea, which were more frequent in the lamotrigine-treated group. The frequency of withdrawals for adverse events was similar between groups. Two patients were hospitalized for skin rash, which resolved after discontinuation of lamotrigine therapy. Lamotrigine was effective for the adjunctive treatment of partial seizures in children and demonstrated an acceptable safety profile. [\hyperlink{Larotrectinib Sulfate}{PMID: 10563619}, M Duchowny et al., 1999]

\hypertarget{pmid_27412538}{T}o investigate the efficacy and safety of lamotrigine monotherapy in children with epilepsy via a systematic review. PubMed, Cochrane, CNKI, VIP, CBM, Wanfang Data were searched for randomized controlled trials (RCTs) of lamotrigine monotherapy in children with epilepsy. Literature screening, data extraction, and quality assessment were performed according to the method recommended by Cochrane Collaboration. RevMan 5.2 software was used to conduct the Meta analysis. A total of 9 RCTs involving 1 016 participants were included. Lamotrigine yielded a significantly lower complete control rate of seizure than ethosuximide, but the complete control rate of seizure showed no significant differences between lamotrigine and carbamazepine/sodium valproate. Patients treated with lamotrigine had a significantly lower incidence rate of adverse events than those treated with carbamazepine, but the incidence rate of adverse events showed no significant differences between patients treated with lamotrigine and sodium valproate/carbamazepine. The drop-out rate showed no significant differences between the three treatment groups. Lamotrigine is an ideal alternative drug for children who do not respond to traditional antiepileptic medication or experience significant adverse reactions; however, more high-quality RCTs with a large sample size and a long follow-up time are needed to confirm these conclusions. [\hyperlink{Larotrectinib Sulfate}{PMID: 27412538}, Yan-Tao Liu et al., 2016]

\hypertarget{pmid_15073972}{L}amotrigine is an important new addition to the drugs used to treat people with seizure disorders, but disconcerting are reports of a higher than expected incidence of severe skin reaction among children. Using automated data from three HMOs, we conducted a retrospective investigation of children (<15 years) exposed to lamotrigine from 1 January 1995 to 30 June 1997. The outcome of interest was hospitalization for a severe skin reaction (e.g. erythema multiforme). Lamotrigine was dispensed to 124 children (56\% female, mean age 8.7 years); the mean number of dispensings per person was 10. Of those exposed, 59 (47\%) were hospitalized at least once during the study period, mainly for convulsions and epilepsy. There were no hospitalizations for or with a diagnosis of severe skin reactions. Our investigation revealed no evidence to support a causal relationship between lamotrigine and severe skin reactions. However, because our sample size was small we had power to detect only a very strong association between lamotrigine and severe skin disease. Taken alone, our study does not establish the risks of lamotrigine. These results should be viewed as a contribution to the totality of evidence that will be used to assess the safety of lamotrigine. [\hyperlink{Larotrectinib Sulfate}{PMID: 15073972}, J G Donahue et al., 1998]

\hypertarget{pmid_28441993}{P}harmacologic treatment is a mainstay of allergy therapy and many caregivers use over-the-counter antihistamines for the treatment of seasonal allergic rhinitis (SAR) symptoms in children. To assess the efficacy and safety of cetirizine 10 mg syrup versus loratadine 10 mg syrup versus placebo syrup in a randomized double-blind study of children, ages 6-11 years, with SAR. This randomized, double-blind, parallel-group, placebo-controlled study was conducted at 71 U.S. centers during the spring tree and grass pollen season. After a 1-week placebo run-in period, qualified subjects were randomized to once-daily cetirizine 10 mg (n = 231), loratadine 10 mg (n = 221), and placebo (n = 231) for 2 weeks. The primary efficacy end point was change from baseline in the subject's mean reflective total symptom severity complex (TSSC) score over 14 days. Children treated with cetirizine experienced significantly greater TSSC score reductions versus children treated with placebo over 14 days (least square mean change, -2.1 versus -1.6; p = 0.006). The differences in TSSC score improvement over 14 days between the cetirizine versus loratadine groups (-2.1 versus -1.8; p = 0.124) and between the loratadine versus placebo groups (-1.8 versus -1.6; p = 0.230) were not statistically significant. Predominant adverse events in the cetirizine, loratadine, and placebo groups were headache (3.5, 3.6, and 3.1\%, respectively) and pharyngitis (3.5, 2.7, and 3.5\%, respectively). Somnolence was reported in three subjects (1.3\%) treated with cetirizine and in none of the other subjects. Cetirizine 10 mg was statistically significantly more efficacious than placebo in the treatment of SAR symptoms in children ages 6-11 years. Symptom improvement was not significantly different between the loratadine 10 mg and placebo groups. [\hyperlink{Larotrectinib Sulfate}{PMID: 28441993}, Anjuli S Nayak et al., 2017]

\hypertarget{pmid_21958958}{T}o evaluate the analgesic effect and tolerability of paracetamol syrup compared to placebo and ketoprofen lysine salt in children with pharyngotonsillitis cared by family pediatricians. A double-blind, randomized, placebo-controlled trial of a 12 mg/kg single dose of paracetamol paralleled by open-label ketoprofren lysine salt sachet 40 mg. Six to 12 years old children with diagnosis of pharyngo-tonsillitis and a Children's Sore Throat Pain (CSTP) Thermometer score > 120 mm were enrolled. Primary endpoint was the Sum of Pain Intensity Differences (SPID) of the CSTP Intensity scale by the child. 97 children were equally randomized to paracetamol, placebo or ketoprofen. Paracetamol was significantly more effective than placebo in the SPID of children and parents (P < 0.05) but not in the SPID reported by investigators, 1 hour after drug administration. Global evaluation of efficacy showed a statistically significant advantage of paracetamol over placebo after 1 hour either for children, parents or investigators. Patients treated in open fashion with ketoprofen lysine salt, showed similar improvement in pain over time. All treatments were well-tolerated. A single oral dose of paracetamol or ketoprofen lysine salt are safe and effective analgesic treatments for children with sore throat in daily pediatric ambulatory care. [\hyperlink{Larotrectinib Sulfate}{PMID: 21958958}, Nicolino Ruperto et al., 2011]

\hypertarget{pmid_28827252}{C}eftriaxone is widely used in children in the treatment of sepsis. However, concerns have been raised about the safety of ceftriaxone, especially in young children. The aim of this review is to systematically evaluate the safety of ceftriaxone in children of all age groups. MEDLINE, PubMed, Cochrane Central Register of Controlled Trials, EMBASE, CINAHL, International Pharmaceutical Abstracts and adverse drug reaction (ADR) monitoring systems will be systematically searched for randomised controlled trials (RCTs), cohort studies, case-control studies, cross-sectional studies, case series and case reports evaluating the safety of ceftriaxone in children. The Cochrane risk of bias tool, Newcastle-Ottawa and quality assessment tools developed by the National Institutes of Health will be used for quality assessment. Meta-analysis of the incidence of ADRs from RCTs and prospective studies will be done. Subgroup analyses will be performed for age and dosage regimen. Formal ethical approval is not required as no primary data are collected. This systematic review will be disseminated through a peer-reviewed publication and at conference meetings. CRD42017055428. [\hyperlink{Larotrectinib Sulfate}{PMID: 28827252}, Linan Zeng et al., 2017]

\hypertarget{pmid_31466903}{T}his systematic review and meta-analysis of randomized controlled trials (RCTs) systematically explored the effectiveness and safety of lamotrigine for absence seizures in children and adolescents. Keywords searches were conducted in Pubmed Embase Cochrane Central Register of Controlled Trials Wanfang CNKI from inception through March 2019. The RCTs comparing lamotrigine with other drugs and/or placebo for the treatment of absence seizures in children and adolescents were considered in this study. The study was conducted adhering to PRISMA guidelines. Eight RCTs (n = 787) were included in our study. Among these studies, one study (N = 45 patients) used placebo as a control and seven studies (N = 742 patients) used positive drug controls. For effectiveness, there was significant difference between lamotrigine and valproate [OR = 0.42, 95\%CI (0.28-0.63), I According to current evidence, LTG is less effective than VPA and ESM, however, based on its relative safety, LTG might be reasonably tried as initial therapy in children and adolescents at risk of significant adverse effects from VPA and ESM, and future well-designed studies are needed to confirm our findings. [\hyperlink{Larotrectinib Sulfate}{PMID: 31466903}, Jing Cao et al., 2020]

\hypertarget{pmid_15247700}{M}any children with urological disease require long-term treatment with antibiotics. In many cases the choice of medical instead of surgical management hinges on the implied safety of certain drugs. Recently some groups have advocated subureteral injection procedures to avoid long-term antibiotics for low grade reflux. We present a concise and relevant review on the use and adverse reactions of nitrofurantoin, trimethoprim and sulfamethoxazole in children. We reviewed the literature regarding the safety and toxicity of these drugs. Information regarding absorption, excretion and dosing was also gathered to explain better the mechanisms of toxicity. Adverse reactions in children reported in the literature related to nitrofurantoin are gastrointestinal disturbance (4.4/100 person-years at risk), cutaneous reactions (2\% to 3\%), pulmonary toxicity (9 patients), hepatoxicity (12 patients and 3 deaths), hematological toxicity (12 patients), neurotoxicity and an increased rate of sister chromatid exchanges. Adverse reactions in children related to trimethoprim/sulfamethoxazole are almost exclusively due to the sulfamethoxazole component, including cutaneous reactions (1.4 to 7.4 events per 100 person-years at risk), hematological toxicity (0\% to 72\% of patients) and hepatotoxicity (5 patients). The majority of adverse reactions were found in children on full dose therapy and not prophylaxis. The use of nitrofurantoin, trimethoprim and sulfamethoxazole is safe in children for long-term prophylactic therapy. The antibiotic safety issue should not be misconstrued as an argument for surgical therapy, whether minimally invasive or not. Adverse reactions exist to these medicines but they are less common than seen in adults, presumably because of the lower dose used for therapy, and the lack of significant comorbidities and drug interactions in children. Serious side effects are extremely rare and most are reversible by discontinuing therapy. The extremely low potential for significant adverse reactions should be discussed with parents. [\hyperlink{Larotrectinib Sulfate}{PMID: 15247700}, Edward Karpman et al., 2004]

\hypertarget{pmid_36916488}{O}laratumab (Lartruvo™) is a recombinant human IgG1 monoclonal antibody that specifically binds PDGFRα. In order to support use of Lartruvo in pediatric patients, a definitive juvenile animal study in neonatal mice was conducted with a human anti-mouse PDGFRα antibody analog of olaratumab (LSN3338786). A pilot study was used to set doses for the definitive juvenile mouse study. In the definitive study, juvenile mice were administered vehicle, 50, 100, or 150 mg/kg LSN3338786 by subcutaneous (SC) injection every 3 days between postnatal days (PND) 1 and 49, for a total of 17 doses. Blood samples were collected on PND 49 for antibody analysis and toxicokinetic evaluation. Tissues were collected on PND 52 for histopathologic examination. Results of the pilot study indicated that dosing neonatal mice starting on PND 1 via SC administration every 3 days was logistically feasible, produced exposures consistent with prior animal studies, and the selected dose levels were well tolerated by juvenile mice. In the definitive juvenile study, there were no LSN3338786-related deaths, clinical findings, and no effects on mean body weights, body weight gains, or food consumption. Additionally, there were no adverse LSN3338786-related hematology findings, and no macroscopic, organ weight, or microscopic findings of note. The highest dose evaluated, 150 mg/kg, was considered the NOAEL for juvenile toxicity. In conclusion, the juvenile animal studies did not identify any new toxicities or increased sensitivities for the intended pediatric patient population. The use of the surrogate antibody approach in a standard rodent model enabled the de-risking of theoretical concerns for toxicity in pediatric patients due to disruption of the PDGFRα pathway during early human development, such as pulmonary development. [\hyperlink{Larotrectinib Sulfate}{PMID: 36916488}, Ronee Baracani et al., 2023]

\hypertarget{pmid_15694564}{L}amotrigine (LTG), vigabatrin (VGB) and gabapentin (GBP) are three anti-epileptic drugs (AEDs) used in the treatment of children with epilepsy for which long-term retention rates are not currently well known. This study examines the efficacy, long-term survival and adverse event profile of these three agents used as add-on therapy in children with refractory epilepsy over a 10-year period. Three separate audits were conducted between February 1996 and September 2000. All children studied had epilepsy refractory to other AEDs. Efficacy was confirmed if a patient became seizure free or achieved >50\% reduction in seizure frequency for 6 months or more after starting therapy. Adverse events and patient survival for each drug were recorded at the end of the study period. Between September 1990 and February 1996, 132 children received LTG, 80 VGB and 39 GBP. At the 10-year follow-up audit, 33\% of the children on LTG had a sustained beneficial effect on their seizure frequency in contrast to 19\% for VGB and 15\% for GBP. No significant difference in efficacy was found in children with partial seizures. Children with epileptic encephalopathy (EE) including myoclonic-astatic epilepsy and Lennox-Gastaut Syndrome (LGS) achieved a more favorable response to LTG. The main reasons for drug withdrawal were lack of efficacy for VGB, apparent worsening of seizures for GBP and the development of a rash for LTG. Lamotrigine is a useful add-on therapy in treating children with epilepsy. It has a low adverse event profile and a sustained beneficial effect in children with intractable epilepsy. [\hyperlink{Larotrectinib Sulfate}{PMID: 15694564}, D G M McDonald et al., 2005]

\hypertarget{pmid_20819318}{A}llergic rhinitis (AR) and chronic idiopathic urticaria (CIU) are common causes of substantial illness and disability in preschool children. Antihistamines are commonly used to treat preschool children with these conditions, but their use is based mostly on extrapolated efficacy from adult populations; it is thus important to characterize the safety of antihistamines in the pediatric population. This study was designed to assess the safety of levocetirizine dihydrochloride oral liquid drops in infants and children with AR or CIU. Two multicenter, double-blind, randomized, parallel-group studies randomized infants aged 6-11 months (study 1, n = 69) and children aged 1-5 years (study 2, n = 173) to levocetirizine, 1.25 mg (q.d. or b.i.d., respectively), or placebo for 2 weeks, using a 2:1 ratio. Safety evaluations included treatment-emergent adverse events (TEAEs), vital signs, electrocardiographic (ECG) assessments, and laboratory tests. The overall incidence of TEAEs was similar between levocetirizine and placebo in both studies. Most TEAEs were mild or moderate in intensity. TEAEs prompted discontinuation of therapy in three patients receiving levocetirizine in study 1. No clinically relevant changes from baseline in vital signs or laboratory parameters were apparent in either study; changes from baseline in these evaluations were similar between groups. No significant changes were observed in ECG parameters, including corrected QT interval. Levocetirizine, 1.25 and 2.5 mg/day, was well tolerated in infants aged 6-11 months and in children aged 1-5 years, respectively, with AR or CIU. [\hyperlink{Larotrectinib Sulfate}{PMID: 20819318}, Frank Hampel et al., ]

\hypertarget{pmid_22378696}{A}nti-seizure prophylaxis is routinely utilized during busulfan administration for HSCT. We evaluated the feasibility and efficacy of levetiracetam in children undergoing HSCT. A total of 28 children and young adults received levetiracetam during HSCT and the outcomes and costs were compared to a historical, but similar cohort of 25 patients who had received fosphenytoin. Levetiracetam was well tolerated and was efficacious in preventing seizures. Cost of drug, administration, and monitoring were also similar among the two groups. Due to non-induction of the hepatic cytochrome P450 enzymes, levetiracetam may lead to better dose assurance of busulfan in targeted dose regimens for HSCT. [\hyperlink{Larotrectinib Sulfate}{PMID: 22378696}, Sandeep Soni et al., 2012]

\hypertarget{pmid_14729411}{T}o investigate to which extent lamotrigine (LTG) may be effective and tolerated as a monotherapy for the treatment of newly diagnosed childhood absence seizures and, secondly, to evaluate the efficacy of this drug on the circadian interictal generalized epileptiform discharges, 20 consecutive newly diagnosed patients (five males, 15 females), aged 3-10 years (mean 6.9 years), affected by childhood absence epilepsy, were administered LTG as first-line drug at the initial dose of 0.5 mg/kg/day for 2 weeks, followed by 1.0 mg/kg/day for an additional 2 weeks. Thereafter, doses have been increased in 1-mg/kg/day increments up to 9-12 mg/kg/day in accordance with the clinical response. Each patient underwent an ambulatory (24 h) EEG monitoring before starting LTG therapy (time 0) and during the maintenance period at the end of LTG titration (time 1). After a mean follow-up period of 10.8 months (range 3-28 months), a 100\% seizure control was obtained in 11 children (55.5\%), a more than 75\% seizure decrease was present in four (20\%), and a >50\% seizure decrease in five (25\%), with a mean LTG dose of 6.2 mg/kg/day (range 1.2-11) in the controlled group. Adverse events were present in three patients (15\%); they were generally mild and transient. Our series confirms that LTG monotherapy may control typical childhood absence seizures in about half the children as well as it may decrease interictal generalized spike and wave discharges both in seizure-free and uncontrolled patients. The slow titration phase of the drug due to the risk of the skin rash may eventually reduce compliance. [\hyperlink{Larotrectinib Sulfate}{PMID: 14729411}, Giangennaro Coppola et al., 2004]

\hypertarget{pmid_17561929}{T}here are more than 40 H(1)-antihistamines available worldwide. Most of these medications have never been optimally studied in prospective, randomized, double-masked, placebo-controlled trials in children. The aim was to perform a long-term study of levocetirizine safety in young atopic children. In the randomized, double-masked Early Prevention of Asthma in Atopic Children Study, 510 atopic children who were age 12-24 months at entry received either levocetirizine 0.125 mg/kg or placebo twice daily for 18 months. Safety was assessed by: reporting of adverse events, numbers of children discontinuing the study because of adverse events, height and body mass measurements, assessment of developmental milestones, and hematology and biochemistry tests. The population evaluated for safety consisted of 255 children given levocetirizine and 255 children given placebo. The treatment groups were similar demographically, and with regard to number of children with: one or more adverse events (levocetirizine, 96.9\%; placebo, 95.7\%); serious adverse events (levocetirizine, 12.2\%; placebo, 14.5\%); medication-attributed adverse events (levocetirizine, 5.1\%; placebo, 6.3\%); and adverse events that led to permanent discontinuation of study medication (levocetirizine, 2.0\%; placebo, 1.2\%). The most frequent adverse events related to: upper respiratory tract infections, transient gastroenteritis symptoms, or exacerbations of allergic diseases. There were no significant differences between the treatment groups in height, mass, attainment of developmental milestones, and hematology and biochemistry tests. The long-term safety of levocetirizine has been confirmed in young atopic children. [\hyperlink{Larotrectinib Sulfate}{PMID: 17561929}, F Estelle R Simons et al., 2007]

\hypertarget{pmid_19758220}{R}ituximab (RTX) is currently used in many diseases, but its efficacy and safety in juvenile systemic lupus erythematosus (SLEj) is still unknown. In this chapter we present four case reports of children treated with RTX: three SLE and one immune thrombocytopenic purpura (ITP). Two of the three SLEj patients had class IV lupus nephritis (LN) and hematologic manifestations (pancytopenia), both reaching complete recovery of blood counts and improvement of LN with RTX treatment. Our third SLE patient had a severe onset with generalized microangiopathic manifestations in association with antiphospholipid antibodies and has been in remission for almost 1 year after RTX. However, our fourth case, a patient with ITP and renal failure, was treated with RTX without either hematologic or renal response. [\hyperlink{Larotrectinib Sulfate}{PMID: 19758220}, Joaquim Polido-Pereira et al., 2009]

\hypertarget{pmid_20407029}{T}o report a case of acute pediatric lamotrigine ingestion resulting in seizures. A 2-year-old boy presented to the emergency department after an acute ingestion of up to 43 mg/kg of lamotrigine. He had 2 generalized seizures, with the first occurring 60 minutes after ingestion. Examination revealed alternate drowsiness and irritability, as well as nystagmus and hyperreflexia. Results of electrocardiogram, blood glucose, complete blood count, urea, electrolytes, and venous blood gas evaluations were all within normal limits. There was a mildly raised lactate level of 3.4 mEq/L (reference range 0.7-2.5). He was given intravenous diazepam 1 mg for irritability. After a 12-hour observation period, the patient was discharged with no further complications. The Naranjo probability scale in this case suggested a probable causality between the acute lamotrigine ingestion and seizures. This is the lowest acute dose causing pediatric seizure reported in the literature; however, this dose is still significantly higher than a therapeutic dose. A MEDLINE search (1966-January 2010) using the search terms lamotrigine, seizures, toxicity, overdose, ingestion, and pediatric/paediatric, not limited to English-language literature, revealed 5 other cases of seizures in children after lamotrigine ingestion. In all the acute cases, time to first seizure onset ranged from 20 to 60 minutes after ingestion. Two children had gastrointestinal decontamination, both after the onset of seizures. All had full recovery with supportive care. Lamotrigine has the ability to cause seizures in children from acute single ingestion at a lower dose than previously described. There is not enough information available to establish a toxic dose or dose that requires hospital assessment. Gastrointestinal decontamination should be contraindicated. Supportive care, including administration of benzodiazepines, is appropriate. [\hyperlink{Larotrectinib Sulfate}{PMID: 20407029}, Benjamin R Close et al., 2010]

\hypertarget{pmid_3249368}{S}ultamicillin, a mutual prodrug of a beta-lactam antibiotic and beta-lactamase inhibitor, was administered to 19 child patients with infectious diseases. The patients included 9 boys and 10 girls from 11 months to 13 years old and they were given orally a dosage of 15.4-40.8 mg/kg/day for 3 to 12 days. Clinical efficacies were excellent in 2 cases, good in 13 cases, fair in 3 cases, unknown in 1 case, and the total efficacy rate was 83.3\%. Loose stool in 1 case and mild diarrhea in another occurred as side effects of the drug, but no abnormal laboratory test values were found upon the treatment. [\hyperlink{Larotrectinib Sulfate}{PMID: 3249368}, K Okada et al., 1988]

\section*{Tafamidis Meglumine}
\subsection*{Result}
\subsubsection*{Answer}

Unknown

\subsubsection*{{Explanation}}
\hypertarget{Tafamidis Meglumine}
A review of the available abstracts reveals the following regarding the safety of Tafamidis Meglumine in children:

1. **\hyperlink{pmid_32196976}{PMID: 32196976}, Peter A Lockwood et al., 2020** – This phase 1 study evaluated the pharmacokinetics and safety of tafamidis meglumine in 30 healthy volunteers. However, the age range of participants is not specified in the abstract, and there is no mention of children or pediatric subjects.

2. **\hyperlink{pmid_32800381}{PMID: 32800381}, Tomonori Ishii et al., 2020** – This postmarketing surveillance study in Japan included 219 patients with hereditary transthyretin amyloidosis treated with tafamidis meglumine. The mean age was 59.7 years, and there is no indication that children were included.

3. **\hyperlink{pmid_28152524}{PMID: 28152524}, Teppei Fujita et al.** – This is a single case report of a 72-year-old woman treated with tafamidis.

4. **\hyperlink{pmid_30478886}{PMID: 30478886}, Yasuhiro Shintani et al., 2019** – This is a case report of a 73-year-old man treated with tafamidis.

5. **\hyperlink{pmid_24101373}{PMID: 24101373}, Giampaolo Merlini et al., 2013** – This phase II study included 21 patients with non-Val30Met transthyretin amyloidosis, but the age of participants is not specified in the abstract. There is no mention of children.

6. **\hyperlink{pmid_26944161}{PMID: 26944161}, Yukio Ando et al., 2016** – This open-label study evaluated tafamidis in 10 Japanese patients with TTR-FAP. The mean age was 60.1 years; no children were included.

In summary, none of the abstracts reviewed report targeted safety studies of tafamidis meglumine in children (defined as individuals under 18 years of age). All clinical studies and case reports focus on adult populations, with mean ages well above the pediatric range. There are no abstracts describing the safety, tolerability, or efficacy of tafamidis meglumine in children, nor are there any studies that affirm or refute its safety in this population.

Therefore, based strictly on the abstracts available, the safety of tafamidis meglumine in children is unknown.

\subsection*{Abstracts}
\hypertarget{pmid_32196976}{T}afamidis, a non-nonsteroidal anti-inflammatory benzoxazole derivative, acts as a transthyretin (TTR) stabilizer to slow progression of TTR amyloidosis (ATTR). Tafamidis meglumine, available as 20-mg capsules, is approved in more than 40 countries worldwide for the treatment of adults with early-stage symptomatic ATTR polyneuropathy. This agent, administered as an 80-mg, once-daily dose (4 × 20-mg capsules), is approved in the United States, Japan, Canada, and Brazil for the treatment of hereditary and wild-type ATTR cardiomyopathy in adults. An alternative single solid oral dosage formulation (tafamidis 61-mg free acid capsules) was developed and introduced for patient convenience (approved in the United States, United Arab Emirates, and European Union). In this single-center, open-label, randomized, 2-period, 2-sequence, crossover, multiple-dose phase 1 study, the rate and extent of absorption were compared between tafamidis 61-mg free acid capsules (test) and tafamidis meglumine 80-mg (4 × 20-mg) capsules (reference) after 7 days of repeated oral dosing under fasted conditions in 30 healthy volunteers. Ratios of adjusted geometric means (90\%CI) for the test/reference formulations were 102.3 (98.0-106.8) for area under the concentration-time profile over the dosing interval and 94.1 (89.1-99.4) for the maximum observed concentration, satisfying prespecified bioequivalence acceptance criteria (90\%CI, 80-125). Both tafamidis regimens had an acceptable safety/tolerability profile in this population. [\hyperlink{Tafamidis Meglumine}{PMID: 32196976}, Peter A Lockwood et al., 2020]

\hypertarget{pmid_32800381}{A}n all-case, single-arm, observational, postmarketing surveillance is underway to assess the safety of tafamidis in patients with hereditary transthyretin (ATTRv) amyloidosis with peripheral polyneuropathy, also called transthyretin-type familial amyloid polyneuropathy, in Japan. Results from an interim analysis (data cutoff date, May 15, 2018) are presented in this preliminary report. Patients were registered and treated with tafamidis meglumine 20 mg/d in routine clinical practice (observation period, 156 weeks). Data on patient demographic and clinical characteristics and adverse drug reactions (ADRs) were captured using case-report forms. Of 219 patients included (mean age, 59.7 years; patients with age at disease onset ≥50 years, 61.2\%; mean treatment duration, 95.5 weeks), 143 (65.3\%) were male, 126 (57.5\%) had a family history of ATTRv amyloidosis, and 149 (68.0\%) originated from nonendemic areas. The most common ADRs were diarrhea (1.4\%) and hematuria (0.9\%). Six serious ADRs (pneumonia, bacteremia, malignant melanoma, pancreatic carcinoma, hematuria, and hereditary neuropathic amyloidosis [primary disease exacerbation]) were reported; no ADRs leading to death were recorded. This interim analysis successfully provided comprehensive, nationwide epidemiologic data from 219 Japanese patients with ATTRv amyloidosis. The safety profile of tafamidis was largely consistent with that obtained from previous research. No new safety concerns were identified to date. Data presented in this interim analysis are subject to change following completion of the study, and we will continue to assess the safety and effectiveness of tafamidis throughout the study. ClinicalTrials.gov identifier: NCT02146378. [\hyperlink{Tafamidis Meglumine}{PMID: 32800381}, Tomonori Ishii et al., 2020]

\hypertarget{pmid_22238470}{C}hildren have a lower response rate to antimonial drugs and higher elimination rate of antimony (Sb) than adults. Oral miltefosine has not been evaluated for pediatric cutaneous leishmaniasis. A randomized, noninferiority clinical trial with masked evaluation was conducted at 3 locations in Colombia where Leishmania panamensis and Leishmania guyanensis predominated. One hundred sixteen children aged 2-12 years with parasitologically confirmed cutaneous leishmaniasis were randomized to directly observed treatment with meglumine antimoniate (20 mg Sb/kg/d for 20 days; intramuscular) (n = 58) or miltefosine (1.8-2.5 mg/kg/d for 28 days; by mouth) (n = 58). Primary outcome was treatment failure at or before week 26 after initiation of treatment. Miltefosine was noninferior if the proportion of treatment failures was ≤15\% higher than achieved with meglumine antimoniate (1-sided test, α = .05). Ninety-five percent of children (111/116) completed follow-up evaluation. By intention-to-treat analysis, failure rate was 17.2\% (98\% confidence interval [CI], 5.7\%-28.7\%) for miltefosine and 31\% (98\% CI, 16.9\%-45.2\%) for meglumine antimoniate. The difference between treatment groups was 13.8\%, (98\% CI, -4.5\% to 32\%) (P = .04). Adverse events were mild for both treatments. Miltefosine is noninferior to meglumine antimoniate for treatment of pediatric cutaneous leishmaniasis caused by Leishmania (Viannia) species. Advantages of oral administration and low toxicity favor use of miltefosine in children. NCT00487253. [\hyperlink{Tafamidis Meglumine}{PMID: 22238470}, Luisa Consuelo Rubiano et al., 2012]

\hypertarget{pmid_28152524}{T}afamidis meglumine is a novel medicine that has been shown to slow the progression of peripheral neurological impairment in patients with hereditary transthyretin amyloidosis (ATTR). However, the efficacy of tafamidis against ATTR-related cardiac amyloidosis remains unclear. A 72-year-old woman had cardiac hypertrophy and axonopathy in her lower legs. Endomyocardial biopsy revealed an infiltrative cardiomyopathy consistent with amyloidosis. Immunostaining and genetic studies confirmed the diagnosis of ATTR, and tafamidis was started subsequently. Two years after the initiation of tafamidis treatment, electromyography demonstrated no change in the axonopathy in her lower legs; however, electrocardiography displayed QRS prolongation, and echocardiography disclosed an increase in interventricular septal thickness. Endomyocardial biopsy indicated that transthyretin amyloid infiltration of the myocardium was not reduced. In this case, there was no apparent progression of axonopathy, although there were signs of worsening amyloid cardiomyopathy during the treatment with tafamidis. [\hyperlink{Tafamidis Meglumine}{PMID: 28152524}, Teppei Fujita et al., ]

\hypertarget{pmid_27108294}{C}alcineurin inhibitors (CNI), mycophenolate mofetil (MMF), and levamisole are common treatment choices for patients with frequently relapsing (FRNS) and steroid-dependent nephrotic syndrome (SDNS). In this retrospective cohort study, we analyzed the relative efficacy and safety of tacrolimus, MMF, and levamisole over a period of 30 months, in treating 340 children with idiopathic FRNS/SDNS. The children received either MMF 1200 mg/m Tacrolimus was associated with a higher rate of 30-month relapse-free survival when compared to MMF (61.7 vs. 38.5 \%, p < 0.001), or levamisole (61.7 vs. 24 \%, p < 0.001). However, relapse rate increased almost threefold once tacrolimus was stopped, resulting in a higher relapse rate per patient-year when compared to the MMF group (2.0 vs. 1.5, p = 0.013). The cumulative prednisolone dose per patient during the last year of the study period was also increased among tacrolimus group in comparison with MMF group (96.4 vs. 74.4 mg/kg/year, p = 0.004). Independent of the impact of drug choice, the relapse risk was higher in patients with steroid dependency at baseline (HR 2.14, 95 \%CI 1.79-2.96, p < 0.0001). In comparison with few minor adverse events in other two cohorts, several serious adverse events were documented in the tacrolimus group. Although there are serious safety concerns regarding tacrolimus, it is more effective than MMF or levamisole in maintaining relapse-free survival. However, unlike MMF, the relative efficacy of tacrolimus in preventing further relapses is seen only when the patient is on the drug. Taking together the long-term efficacy and safety data observed, MMF appears as a safe and effective alternative to tacrolimus in managing pediatric FRNS/SDNS. [\hyperlink{Tafamidis Meglumine}{PMID: 27108294}, Biswanath Basu et al., 2017]

\hypertarget{pmid_30478886}{T}afamidis meglumine, a transthyretin (TTR) stabilizer, is effective in delaying the progression of neuropathy in TTR amyloidosis with Val30Met mutations. However, its efficacy in TTR amyloid cardiomyopathy is not fully elucidated. Herein, we report a 73-year-old Japanese man with a diagnosis of TTR amyloid cardiomyopathy with Val30Met mutation treated with tafamidis. To evaluate treatment response, cardiac magnetic resonance imaging was performed before and after 12 months of tafamidis treatment. Native T1, extracellular volume, and left ventricular mass showed no obvious worsening, and findings of other diagnostic studies also supported the efficacy of tafamidis to delay the progression of amyloid cardiomyopathy. Our case suggests that serial native T1 and extracellular volume may be novel non-invasive imaging methods to monitor the treatment response to TTR stabilizers in cardiac amyloidosis and also that tafamidis may be effective in suppressing cardiac progression in TTR amyloid cardiomyopathy with Val30Met mutation. [\hyperlink{Tafamidis Meglumine}{PMID: 30478886}, Yasuhiro Shintani et al., 2019]

\hypertarget{pmid_24101373}{T}his phase II, open-label, single-treatment arm study evaluated the pharmacodynamics, efficacy, and safety of tafamidis in patients with non-Val30Met transthyretin (TTR) amyloidosis. Twenty-one patients with eight different non-Val30Met mutations received 20 mg QD of tafamidis meglumine for 12 months. The primary outcome, TTR stabilization at Week 6, was achieved in 18 (94.7\%) of 19 patients with evaluable data. TTR was stabilized in 100\% of patients with non-missing data at Months 6 (n = 18) and 12 (n = 17). Exploratory efficacy measures demonstrated some worsening of neurological function. However, health-related quality of life, cardiac biomarker N-terminal pro-hormone brain natriuretic peptide, echocardiographic parameters, and modified body mass index did not demonstrate clinically relevant worsening during the 12 months of treatment. Tafamidis was well tolerated. In conclusion, our findings suggest that tafamidis 20 mg QD effectively stabilized TTR associated with several non-Val30Met variants. [\hyperlink{Tafamidis Meglumine}{PMID: 24101373}, Giampaolo Merlini et al., 2013]

\hypertarget{pmid_10085688}{T}erbinafine is the first orally active allylamine. Though it is available since several years, experience in children is rather limited. We report about five cases of dermatophytoses in children successfully treated with oral terbinafine. Five children (age: 10 month to 14 years) with a clinical and mycological diagnosis of dermatophytosis were treated with oral terbinafine according to their body weight: 250 mg/d (< 40 kg), 125 mg/d (20-40 kg) and 62.5 mg/d (< 20 kg). The course of treatment was 2 to 4 weeks. In 2/5 patients previous systemic treatment with oral griseofulvin failed. Terbinafine therapy resulted in a complete clearance of mycotic infections (both clinical and mycological) in 5/5 children. No unwanted side-effects were observed. During a nine month follow-up no relaps occurred. In our five cases tolerability, clinical and mycological response of terbinafin therapy were excellent. Its penetration into sebum and only slow release from skin glands may explain the efficacy even in short-time therapy. [\hyperlink{Tafamidis Meglumine}{PMID: 10085688}, A Wilmer et al., 1998]

\hypertarget{pmid_15897144}{T}o observe the therapeutic effect and safety of terbinafine in treating onychomycosis in children. Terbinafine was used to treat 50 children with fingernail onychomycosis and 38 with toenail onychomycosis, and the efficacy, fungal clearance and safety of the drug were evaluated. The clinical cure rate and total effective rate were 92.1\% and 97.37\%, respectively, for fingernail onychomycosis, and 86.36\% and 93.94\% for toenail onychomycosis, with fungal clearance rate of 96.59\%. No serious side-effects was observed in the children receiving terbinafine therapy. Terbinafine is effective and safe for the treatment of onychomycosis in children. [\hyperlink{Tafamidis Meglumine}{PMID: 15897144}, Ju Wen et al., 2005]

\hypertarget{pmid_10961786}{T}he objective of this study was to determine the safety and tolerability of the immunomodulatory agent thalidomide as adjunct therapy in children with tuberculous meningitis. Children with stage 2 tuberculous meningitis received oral thalidomide for 28 days in a dose-escalating study, in addition to standard four-drug antituberculosis therapy, corticosteroids, and specific treatment of complications such as raised intracranial pressure. Clinical and laboratory evaluations were carried out. Fifteen patients (median age, 34 months) were enrolled. Thalidomide was administered via nasogastric tube in a dosage of 6 mg/kg/day, 12 mg/kg/day, or 24 mg/kg/day. The only adverse events possibly related to the study drug were transient skin rashes in two patients. Levels of tumor necrosis factor-alpha in the cerebrospinal fluid decreased markedly during thalidomide therapy. Clinical outcome and neurologic imaging showed greater improvement than that experienced with historical controls. Thalidomide appeared safe and well tolerated in children with stage 2 tuberculous meningitis and could have important anti-inflammatory effects. These promising results have led us to embark on a randomized, double-blind, placebo-controlled trial of the efficacy of thalidomide in tuberculous meningitis. [\hyperlink{Tafamidis Meglumine}{PMID: 10961786}, J F Schoeman et al., 2000]

\hypertarget{pmid_26582874}{C}hildren account for 7\%-20\% of cutaneous leishmaniasis cases in Iran, but there are few safety data to guide pediatric antiparasitic therapy. We evaluated the clinical and laboratory tolerance of the systemic pentavalent antimonial compound meglumine antimoniate, in 70 Iranian children with cutaneous leishmaniasis. Adverse effects were similar to those seen in adults.  [\hyperlink{Tafamidis Meglumine}{PMID: 26582874}, Pouran Layegh et al., 2015] Use of inhaled tobramycin therapy for treatment of Pseudomonas aeruginosa infections in young children with cystic fibrosis (CF) is increasing. Safety data for pre-school children are sparse. The aim of this study was to assess the safety of tobramycin solution for inhalation (TOBI®-TSI) administered twice daily for 2 months/course concurrently to intravenous (IV) tobramycin during P. aeruginosa eradication therapy in children (0-5 years). Audiological assessment and estimation of glomerular filtration rate (GFR) was measured prior to any exposure and end of the study. Data were available from 142 patients who were either never exposed to aminoglycosides (n=39), exposed to IV aminoglycosides only (n=36) or exposed to IV+TSI (n=67). Median exposure to TSI was 113 days [59, 119]. Comparison of effects on audiometry results and GFR, showed no detectable difference between the groups. Use of TSI and IV tobramycin in pre-school children with CF was not associated with detectable renal toxicity or ototoxicity. [\hyperlink{Tafamidis Meglumine}{PMID: 26582874}, Stefanie Hennig et al., 2014]

\hypertarget{pmid_26944161}{T}he efficacy and safety of tafamidis in transthyretin (TTR) familial amyloid polyneuropathy (TTR-FAP) were evaluated in this open-label study. Japanese TTR-FAP patients (n=10; mean age 60.1 years) received tafamidis meglumine (20mg daily; median treatment duration 713.5 days). The primary endpoint was TTR stabilization at Week 8. Secondary endpoints included Neuropathy Impairment Score-Lower Limb (NIS-LL), Norfolk QOL-DN total quality of life (TQOL), and modified body mass index (mBMI). TTR stabilization was achieved in all patients at Weeks 8 and 26, 9 out of 10 patients at Week 52, and 8 out of 10 patients at Week 78. The percentage (95\% CI) of NIS-LL responders (increase from baseline in NIS-LL<2) was 80.0\% (44.4, 97.5), 60.0\% (26.2, 87.8), and 40.0\% (12.2, 73.8) and mean(SD) NIS-LL change from baseline was 2.1 (5.6), 3.6 (4.4), and 3.3 (4.7), at Weeks 26, 52, and 78, respectively. Mean (SD) changes from baseline in TQOL and mBMI at Weeks 26, 52, and 78 were 11.8 (20.0), 9.1 (12.5), and 10.8 (13.7) for TQOL, and 26.6 (61.9), 64.9 (80.0), and 53.7 (81.4) for mBMI, respectively. Ambulation status was preserved in 4 out of 8 patients at Week 78. Most adverse events (AEs) were mild/moderate, with no discontinuations due to AEs. Tafamidis stabilized TTR, was safe and well-tolerated, and was effective over 1.5 years in slowing neurologic progression and maintaining TQOL and nutrition status in TTR-FAP. [\hyperlink{Tafamidis Meglumine}{PMID: 26944161}, Yukio Ando et al., 2016]

\hypertarget{pmid_21337233}{A}lthough the 2008 outbreak of nephrolithiasis in children due to melamine-contaminated infant formula has subsided, it remains uncertain whether the present tolerable daily intake (TDI) of melamine provides sufficient protection for young children. To conduct a safety assessment for melamine in infant formula, we established a dose-response relationship based on 13 nephrolithiasis cases selected from 932 children, all of whom were under 5 years of age and had potentially been exposed to contaminated milk in China or Taiwan. According to the children's exposure history, distributions of individual daily melamine intake (mg/kg BW/day) were reconstructed using Monte Carlo simulations to account for uncertainties in exposure duration and melamine concentrations in the contaminated milk. Based on the simulated individual average daily intake (AVDI) of melamine, subjects were further classified into four separate AVDI groups: high, medium, low and a reference group. A statistical logistic model was then fitted for the dose-response relationship between nephrolithiasis incidence and daily melamine intakes using Markov chain Monte Carlo (MCMC) simulations. Based on the background exposure, spontaneous rate, and mode of action (MOA) of nephrolithiasis in children, the simulated lower bounds of the 95\% CIs daily melamine intake ranged from 0.008 to 0.03 mg/kg BW/day corresponding to an additional risks of 0.1\% is proposed as a plausible TDI, which is approximately an order lower than the current WHO-suggested TDI level of 0.2 mg/kg BW/day. More stringent regulations on melamine levels in infant formula should be considered to protect young children fully. [\hyperlink{Tafamidis Meglumine}{PMID: 21337233}, I-J Wang et al., 2011]

\hypertarget{pmid_3244540}{N}umerous trials of prophylaxis of recurrent respiratory infections in children have been performed, even though the only controlled trials providing incontrovertible results were the ones carried out with levamisole and thymostimulin through intramuscular administration. We have experimented a calf-thymic extract administered by oral route (thymomodulin). During the summer we enrolled 40 children aged between 3.5, and 9 years who had suffered from RRI during the previous winter. The patients were randomly divided in two groups and respectively treated with thymomodulin or with placebo; 21 children were given the thymic extract and 19 the placebo. The trial was carried out according to a double-blind schedule for a period of four months, from the beginning of October '84. At the end of the trial we assessed the catharral bouts observed during the research period by the family doctors and the parents evaluation on the clinical state. The difference between the two groups is statistically highly significant both with reference to the reduction of the total number of catharral bouts and to the general clinical state according to the parents opinion. The research clearly demonstrates the protective effect of the thymomodulin, probably due to the "restorative" effect on some immunological functions, temporarily compromised during the infection bouts. [\hyperlink{Tafamidis Meglumine}{PMID: 3244540}, F Longo et al., ]

\hypertarget{pmid_1462922}{A} review of all published experience with flecainide in infants, children, and fetuses was performed to evaluate the appropriate place of the drug in pediatric practice and to determine dosing guidelines. A total of 704 case references was generated. Flecainide appeared to be safe (no deaths with usual oral dosing, < 1\% serious proarrhythmia) and effective (73\% to 100\% control, depending on mechanism) in children with supraventricular tachycardia. The drug was very effective for treatment of fetal tachyarrhythmias. Flecainide may not be safe for children who have structurally abnormal hearts and atrial flutter or ventricular arrhythmias. The safety of flecainide for patients with ventricular arrhythmias and normal hearts requires further investigation. Pharmacokinetic data reveal an age-dependent change in elimination half-life. Patients younger than 1 year of age have a plasma elimination half-life that is similar to that in children older than 12 years (i.e., 11 to 12 hours). Children aged 1 to 12 years have a mean elimination half-life of 8 hours. The effective flecainide dose is 100 to 200 mg/m2/day or 1 to 8 mg/kg/day. Toxicity may occur with doses in excess of these ranges, especially when high doses are accompanied by low serum trough levels. Milk blocks flecainide absorption, and toxicity may become manifest when milk products are removed from the diet. [\hyperlink{Tafamidis Meglumine}{PMID: 1462922}, J C Perry et al., 1992]

\hypertarget{pmid_20921906}{L}imited data are available on the use of deferiprone in children younger than 10 years of age. This study evaluated the safety and efficacy of a new liquid formulation of deferiprone for the treatment of transfusional iron overload in children 1-10 years old. One hundred children (91 thalassemia major, 8 Hb E-β thalassemia, and 1 sickle cell disease) were enrolled for a 6-month treatment with deferiprone (50 to 100 mg/kg/d). The safety profile was similar to or better than that reported in earlier studies with deferiprone tablets in older children and adults. No unexpected adverse reactions were observed. Gastrointestinal intolerance (GI) was observed in 11\% and an increased serum ALT in 12\% of the children. Both events were transient. Mild neutropenia, observed in 6\% of patients, did not progress to agranulocytosis and resolved despite continuous deferiprone treatment. Two patients experienced agranulocytosis that resolved without complications upon discontinuation of therapy. Deferiprone use was associated with a significant decline in mean serum ferritin level from 2532±1463 μg/L at baseline to 2176±1144 μg/L (P<0.0005). The results of this study show a favorable benefit/risk ratio of deferiprone oral solution for the treatment of young children with transfusional iron overload. [\hyperlink{Tafamidis Meglumine}{PMID: 20921906}, Moshen S ElAlfy et al., 2010]

\hypertarget{pmid_25461908}{M}eglumine antimoniate (MA) is a pentavalent antimony drug used to treat leishmaniases. We investigated the neurobehavioral development, sexual maturation and fertility of the offspring of MA-treated rats. Dams were administered MA (0, 75, 150, 300 mg Sb(V)/kg body wt/d, sc) from gestation day 0, throughout parturition and lactation, until weaning. At the highest dose, MA reduced the birth weight and the number of viable newborns. In the male offspring, MA did not impair development (somatic, reflex maturation, weight gain, puberty onset, open field test), sperm count, or reproductive performance. Except for a minor effect on body weight gain and vertical exploration in the open field, MA also did not affect the development of female offspring. Measurements of the Sb levels (ICP-MS) in the blood of MA-treated female rats and their offspring demonstrated that Sb is transferred to the fetuses via the placenta and to the suckling pups via milk. [\hyperlink{Tafamidis Meglumine}{PMID: 25461908}, Deise R Coelho et al., 2014]

\hypertarget{pmid_7792222}{O}ver the past four years terbinafine has become established as an effective systemic antimycotic agent with an excellent safety profile. However, experience with its use in children is very limited. We report the effective treatment of five children with oral terbinafine. [\hyperlink{Tafamidis Meglumine}{PMID: 7792222}, V Goulden et al., 1995]

\hypertarget{pmid_17941284}{T}he safety of fexofenadine has been examined extensively in adults and school-age children. However, the safety of fexofenadine in children younger than 6 years has not been reported to date. To compare the safety and tolerability of twice-daily fexofenadine hydrochloride, 30 mg, and placebo in preschool children aged 2 to 5 years with allergic rhinitis. This was a multicenter, double-blind, randomized, placebo-controlled, parallel-group study, conducted between February 29, 2000, and June 14, 2001. Participants were randomized to either fexofenadine hydrochloride, 30 mg, or placebo twice daily for a 2-week period. To facilitate dosing, capsule content was mixed with applesauce (approximately 10 mL). Safety assessments depended on date of entry into the study because of an amendment to the protocol. Before the amendment, assessments included physical examination, vital signs reporting (oral temperature, heart rate, and respiratory rate), and adverse event (AE) reporting. After the amendment, safety assessments included laboratory testing (blood chemistry and hematology profiles), physical examination, 12-lead electrocardiography, and vital signs (oral temperature, blood pressure, heart rate, and respiratory rate) and AE reporting. Treatment-emergent AEs were observed in 116 of 231 participants receiving placebo and 111 of 222 receiving fexofenadine. These AEs were possibly related to study medication in 19 (8.2\%) and 21 (9.5\%) of the participants receiving placebo and fexofenadine, respectively, and most frequently involved the digestive system. No clinically relevant differences in laboratory measures, vital signs, and physical examinations were observed. The findings show that fexofenadine hydrochloride, 30 mg, is well tolerated and has a good safety profile in children aged 2 to 5 years with allergic rhinitis. [\hyperlink{Tafamidis Meglumine}{PMID: 17941284}, Henry Milgrom et al., 2007]

\hypertarget{pmid_24716805}{M}ethicillin-resistant Staphylococcus aureus (MRSA) remains a significant cause of morbidity in hospitalized infants. Over the past 15 years, several drugs have been approved for the treatment of S. aureus infections in adults (linezolid, quinupristin/dalfopristin, daptomycin, telavancin, tigecycline and ceftaroline). The use of the majority of these drugs has extended into the treatment of MRSA infections in infants, frequently with minimal safety or dosing information. Only linezolid is approved for use in infants, and pharmacokinetic data in infants are limited to linezolid and daptomycin. Pediatric trials are underway for ceftaroline, telavancin, and daptomycin; however, none of these studies includes infants. Here, we review current pharmacokinetic, safety and efficacy data of these drugs with a specific focus in infants.  [\hyperlink{Tafamidis Meglumine}{PMID: 24716805}, Martyn Gostelow et al., 2014] In Tunisia there are three epidemic clinical forms of cutaneous leishmaniasis. They are associated with three different species of Leishmania and are observed in different geographical areas. We undertook a single-center retrospective analysis of childhood leishmaniasis in order to describe epidemio-clinical profile, therapeutic characteristics and clinical outcomes of affected patients. The study comprises 166 children with 132 lesions of cutaneous leishmaniasis. The subjects ages range from 5 months to 15 years (average 8.75 years). The F:M sex ratio is 1.3. Leishmaniasis affects grown-up children in 74.5 percent of the cases. All of our patients live in an endemic area. The face is affected in 76.5 percent of cases. Mucosal leishmaniasis is present in 9 children (6.8 \%). Clinical diagnosis confirmed by the parasitologic smear or histopathological examination in 89.6 percent of the cases. Treatment with intralesional meglumine antimoniate is done for 67 patients; the treatment regimen is one local injection (1 ml/cm(2)) per week until recovery. Systemic meglumine antimoniate is the initial therapy for 25 patients. Meglumine antimoniate treatment is well tolerated with no side-effects. All leishmaniasis lesions heal within an average period of 2.18 months. Childhood cutaneous leishmaniais is common in Tunisia. It has the characteristics of sporadic leishmaniasis. Mucosal leishmaniasis has a favorable outcome with no destruction, nor scaring deformity. The standard treatment remains intralesional meglumine antimoniate. [\hyperlink{Tafamidis Meglumine}{PMID: 24716805}, Monia Kharfi et al., 2004]

\hypertarget{pmid_34315832}{T}o assess the efficacy and safety of thalidomide in children with transfusion-dependent thalassemia. This prospective, single center, open-label study enrolled children aged 12-18 years, and who received thalidomide for a duration of 6 months at a starting dose of 2-3 mg/kg/day. Efficacy was assessed by reduction in transfusion requirement and rate of fall of hemoglobin. Efficacy was classified as major, moderate and minimal/no response depending on the reduction in transfusion requirement. Safety was assessed by adverse effects related to thalidomide. 37 children [mean (SD) age, 14.7 (1.8) years were included. Rate of fall of hemoglobin reduced from a mean of 1.0 (0.24) g/week pre-thalidomide therapy to 0.58 (0.26) g/week after 6 months of thalidomide (P<0.001). 19 children (51.3\%) had major response and 12 (32.4\%) had moderate response. In 13.5\% and 32.4\% children response was observed within the first and second month of therapy, respectively. 15 (40.5\%) children remained transfusion - free for a median (IQR) time of 6 (3-10) weeks of thalidomide therapy. Mean serum ferritin (SD) decreased from 1758.9 (835.1) to 1549.6(1016.9) (P<0.001). Mean HbF (SD) showed an increase from 2.95(2.6) to 49.2(33.3) (P<0.001). In 32 children, 47 adverse events were observed. Common adverse events were constipation and neutropenia (mostly mild). Thalidomide resulted in major/moderate response in majority of children with transfusion-dependent thalassemia with satisfactory adverse effect profile. [\hyperlink{Tafamidis Meglumine}{PMID: 34315832}, Jagdish Chandra et al., 2021]

\hypertarget{pmid_11390176}{M}eglumine antimoniate (MA) is a pentavalent antimonial (Sb(V)) drug used to treat leishmaniasis. Despite the fact that Sb(V) organic compounds have been used in clinical practice for more than 50 years, information on their safety during pregnancy is still scanty. This study was undertaken to evaluate the embryo/fetotoxicity of MA in the rat. Wistar rats were treated subcutaneously (s.c.) with MA (300 mg Sb(V)/kg body wt/day) on days 6 through 15 of pregnancy or with a higher dose (3 x 300 mg Sb(V)/kg body wt) on day 11 only. A control group treated with saline on days 6 through 15 and an untreated control group were evaluated as well. Cesarean sections were performed on day 21. No maternal toxicity and no reduction of fetal weight were noted in the groups treated with MA. The repeated administration of MA (days 6 through 15), but not the acute treatment (day 11), enhanced embryolethality. Treatment with MA on days 6 through 15 also caused a higher incidence of an atlas bone anomaly that occurs spontaneously at very low frequencies in our rat strain. These findings indicated that repeated administration of MA was embryolethal and teratogenic in rats. [\hyperlink{Tafamidis Meglumine}{PMID: 11390176}, F J Paumgartten et al., ]

\hypertarget{pmid_27139441}{L}arge-scale deworming interventions, using anthelminthic drugs, are recommended in areas where the prevalence of soil-transmitted helminth infection is high. Anthelminthic safety has been established primarily in school-age children. Our objective was to provide evidence on adverse events from anthelminthic use in early childhood. A randomized multi-arm, placebo-controlled trial of mebendazole, administered at different times and frequencies, was conducted in children 12 months of age living in Iquitos, Peru. Children were followed up to 24 months of age. The association between mebendazole administration and the occurrence of a serious or minor adverse event was determined using logistic regression. There was a total of 1,686 administrations of mebendazole and 1,676 administrations of placebo to 1,760 children. Eighteen serious adverse events (i.e., 11 deaths and seven hospitalizations) and 31 minor adverse events were reported. There was no association between mebendazole and the occurrence of a serious adverse event (odds ratio [OR] = 1.21; 95\% confidence interval [CI] = 0.47, 3.09) or a minor adverse event (OR = 0.84; 95\% CI = 0.41, 1.72). Results from our trial support evidence of safety in administering mebendazole during early childhood. These results support World Health Organization deworming policy and the scaling up of interventions to reach children as of 12 months of age in endemic areas.  [\hyperlink{Tafamidis Meglumine}{PMID: 27139441}, Serene A Joseph et al., 2016]

\section*{Orlistat}
\subsection*{Result}
\subsubsection*{Answer}

Children ≤5 years: Unknown  
Children 8.3–12.3 years: Unknown  
Children 12 years and older: Yes  

\subsubsection*{{Explanation}}
\hypertarget{Orlistat}
To determine if orlistat is safe for use in children, I reviewed the available abstracts for targeted studies on orlistat safety in pediatric populations. Here is a summary by age range:

**Children ≤5 years old:**
- One study [\hyperlink{pmid_18403165}{PMID: 18403165}, Mathias B Forrester et al., 2009] examined orlistat exposures in children ≤5 years old reported to poison control centers. This was not a clinical trial of therapeutic use, but rather a retrospective review of accidental exposures. Most cases had no effect or only minor gastrointestinal effects, and management was usually on site. However, this does not constitute a targeted safety study for therapeutic use in this age group.

**Children 8.3–12.3 years (prepubertal):**
- A small open-label pilot study [\hyperlink{pmid_12856974}{PMID: 12856974}, S Norgren et al., 2003] investigated orlistat in 11 severely obese prepubertal children (8.3–12.3 years) for 12 weeks. The study reported mild, tolerable gastrointestinal side effects and no negative effects on psychological or physical well-being. However, the study was small, open-label, and not placebo-controlled, so while it suggests tolerability, it is not definitive for safety.

**Children 12 years and older (adolescents):**
- Multiple abstracts affirm that orlistat is FDA-approved for use in adolescents aged 12 years and older [\hyperlink{pmid_21642717}{PMID: 21642717}, Alexander L Rogovik et al., 2011; \hyperlink{pmid_29411652}{PMID: 29411652}, Ariana M Chao et al., 2018]. 
- A 3-month open-label study in 20 adolescents (mean age 14.6 years) found orlistat was well-tolerated, with mild gastrointestinal side effects similar to adults [\hyperlink{pmid_12105286}{PMID: 12105286}, Jennifer R McDuffie et al., 2002].
- A 21-day randomized, placebo-controlled study in obese adolescents (age not specified, but defined as adolescent) found orlistat was well-tolerated, with only mild/moderate gastrointestinal adverse events [\hyperlink{pmid_14559927}{PMID: 14559927}, Jianguo Zhi et al., 2003].
- Reviews and guidelines consistently state that orlistat is the only FDA-approved anti-obesity medication for pediatric patients aged 12 years and above, and that safety data in this group are available from clinical trials [\hyperlink{pmid_29411652}{PMID: 29411652}, Ariana M Chao et al., 2018; \hyperlink{pmid_21642717}{PMID: 21642717}, Alexander L Rogovik et al., 2011].

**Children under 12 years (general):**
- A population-based prescribing study in the UK found that orlistat was prescribed to some children under 12, but the majority of use was in those ≥14 years. The study notes that antiobesity drugs are poorly studied in children and adolescents, and further research is needed [\hyperlink{pmid_20002078}{PMID: 20002078}, Russell M Viner et al., 2009].
- A review [\hyperlink{pmid_20958104}{PMID: 20958104}, Natalia Catoira et al., 2010] states that trials in children and adolescents are often limited by short follow-up and small sample sizes, and more data are needed.

**Summary:**
- For children under 12 years old, there is insufficient high-quality, targeted safety data to affirm orlistat is safe for therapeutic use.
- For adolescents aged 12 years and older, there are targeted clinical trials and regulatory approval supporting safety, with the main adverse effects being mild-to-moderate gastrointestinal symptoms.
- For children under 5 years, there are only data on accidental exposures, not therapeutic use.


\subsection*{Abstracts}
\hypertarget{pmid_18403165}{O}n February 7, 2007, orlistat became the first weight-loss drug approved by the United States Food and Drug Administration for over-the-counter sales. However, information on exposures among young children is limited. The objective of this study was to describe the pattern of orlistat exposures among young children reported to poison control centers. The pattern of all exposures to orlistat alone among patients < or = 5 years old reported to six poison control centers during 1999-2005 was identified with respect to various factors. There were 107 cases. The average age was 21.4 months. There were 55 males, 51 females, and 1 unknown. The dose was identified for 76 cases. The mean dose was 155 mg. Patients were managed on site in 88\% of the cases, were already at a health care facility in 8\%, and were referred to a health care facility in 5\%. Of the 45 patients with a known medical outcome, the outcome was no effect for 91\% and minor effect for 9\% of the patients. Of the 92 cases reported during 2000-2005, the listed adverse clinical effects were diarrhea (n = 4) and vomiting (n = 1), and the listed treatments were decontamination by dilution (n = 62), food (n = 8), activated charcoal (n = 5), other emetic (n = 2), cathartic (n = 1), and ipecac (n = 1). Orlistat exposures among young children involving small doses encountered by poison control centers can usually be managed on site through decontamination, and have favorable outcomes with few adverse clinical effects, mainly gastrointestinal in nature. [\hyperlink{Orlistat}{PMID: 18403165}, Mathias B Forrester et al., 2009]

\hypertarget{pmid_29411652}{P}ediatric obesity is a serious public health concern. Five medications have been approved by the Food and Drug Administration (FDA) for chronic weight management in adults with obesity, when used as an adjunct to lifestyle modification. Orlistat is the only FDA-approved medication for pediatric patients aged 12 years and above. This paper summarizes safety and efficacy data from clinical trials of weight loss medications conducted among pediatric samples. Relevant studies were identified through searches in PubMed. Orlistat, as an adjunct to lifestyle modification, results in modest weight losses and may be beneficial for some pediatric patients with obesity. However, gastrointestinal side effects are common and may limit use. In adults taking orlistat, rare but severe adverse events, including liver and renal events, have been reported. Recent pediatric pharmacokinetic studies of liraglutide have demonstrated similar safety and tolerability profiles as found in adults, with gastrointestinal disorders being the most common adverse events. Clinical trials are needed of liraglutide, as well as other medications for obesity, that systematically evaluate their risks and benefits in pediatric patients. [\hyperlink{Orlistat}{PMID: 29411652}, Ariana M Chao et al., 2018]

\hypertarget{pmid_21642717}{Q}uestion There is a large population of overweight and obese children in my clinic. What medications for treatment of obesity are effective and can be used in children? Answer Orlistat is the only medication indicated by the US Food and Drug Administration for the treatment of obesity in adolescents. It is approved by the Food and Drug Administration for use in adolescents aged 12 years and older. There is no single approach to successful treatment of obesity, and lifestyle modification should be maintained throughout the pharmacologic treatment. [\hyperlink{Orlistat}{PMID: 21642717}, Alexander L Rogovik et al., 2011]

\hypertarget{pmid_12856974}{T}his study investigated orlistat treatment in obese prepubertal children with regard to tolerance, safety and psychological well-being. 11 healthy, severely obese prepubertal children (age 8.3-12.3 y, body mass index standard deviation score 5.3-9.2) were recruited for a 12 wk open treatment. Before, during and after treatment, the participants were investigated by psychological evaluation, blood chemistry, and parameters reflecting obesity and fat mass. The participants were able to comply with the treatment, as indicated by pill counts and self reports, and expressed a desire to continue the treatment after the study period. Gastrointestinal side effects were mild and tolerable. No negative effects on psychological or physical well-being were detected, and the psychological evaluation demonstrated increased avoidance of fattening food, body shape preoccupation and oral control (p = 0.011). The median weight loss was 4.0 kg (range -12.7 to +2.5 kg, p = 0.016) and was highly correlated to decreased fat mass (regression coefficient 0.953, p < 0.01). This pilot study indicates that obese prepubertal children were able to reduce their fat intake to avoid gastrointestinal side effects. Thus, orlistat may be suitable as a component in behaviour-modification programmes for obese children, and the results prompt a placebo-controlled investigation of its effectiveness in promoting weight loss. [\hyperlink{Orlistat}{PMID: 12856974}, S Norgren et al., 2003] * The antiobesity drugs sibutramine and orlistat are not licensed for use in children and adolescents in the UK or USA. * Clinical trials suggest antiobesity drugs are effective and well-tolerated in obese adolescents. * Prescribing of unlicensed antiobesity drugs in children and adolescents has increased significantly in the past 8 years. * Most prescribed antiobesity drugs in children and adolescents are rapidly discontinued before patients can see clinical benefit, suggesting they are poorly tolerated or poorly efficacious. The international childhood obesity epidemic has driven increased use of unlicensed antiobesity drugs, whose efficacy and safety are poorly studied in children and adolescents. We investigated the use of unlicensed antiobesity drugs (orlistat, sibutramine and rimonabant) in children and adolescents (0-18 years) in the UK. Population-based prescribing data from the UK General Practice Research Database between 1 January 1999 and 31 December 2006. A total of 452 subjects received 1334 prescriptions during the study period. The annual prevalence of antiobesity drug prescriptions rose significantly from 0.006 per 1000 [95\% confidence interval (CI) 0.0007, 0.0113] in 1999 to 0.091 per 1000 (95\% CI 0.07, 0.11) in 2006, a 15-fold increase, with similar increases seen in both genders. The majority of prescriptions were made to those >or=14 years old, although 25 prescriptions were made for children <12 years old. Orlistat accounted for 78.4\% of all prescriptions; only one patient was prescribed rimonabant. However, approximately 45\% of the patients ceased orlistat and 25\% ceased sibutramine after only 1 month. The estimated mean treatment durations for orlistat and sibutramine were 3 and 4 months, respectively. Prescribing of unlicensed antiobesity drugs in children and adolescents has dramatically increased in the past 8 years. The majority are rapidly discontinued before patients can see weight benefit, suggesting they are poorly tolerated or poorly efficacious when used in the general population. Further research into the effectiveness and safety of antiobesity drugs in clinical populations of children and adolescents is needed. [\hyperlink{Orlistat}{PMID: 12856974}, Russell M Viner et al., 2009]

\hypertarget{pmid_14559927}{O}rlistat is a gastrointestinal lipase inhibitor used to reduce dietary fat absorption and could be used to treat overweight and obesity in adolescents. The primary objective was to assess whether orlistat has an effect on the physiologic balance of three macrominerals (calcium, phosphorus and magnesium) and three microminerals (iron, zinc and copper). This was a 21-day, double-blind, randomized, parallel-group, placebo-controlled mineral balance study conducted in adolescent obese volunteers (BMI >or=85th percentile, adjusted for age and gender). Subjects were maintained on a hypocaloric diet with a normal daily mineral content in both treatment groups and received oral treatment with orlistat 120 mg (n = 16) or placebo (n = 16) three times daily for 21 days. Following a 14-day equilibration period, balances for calcium, phosphorus, magnesium, iron, copper and zinc were measured for days 15-21. Serum and urine electrolytes were also measured at baseline and at the end of treatment. On average, orlistat inhibited dietary fat absorption by approximately 27\%. This degree of dietary fat inhibition caused no significant changes in mineral balance between orlistat and placebo groups. In addition, serum and urine electrolytes (sodium and potassium) as well as urinary creatinine excretion were not affected by orlistat treatment. Orlistat was well tolerated; adverse events occurred mainly in the gastrointestinal tract and were of mild or moderate intensities. Administration of orlistat had no significant effect on the balance of six selected minerals in adolescent obese patients. [\hyperlink{Orlistat}{PMID: 14559927}, Jianguo Zhi et al., 2003]

\hypertarget{pmid_26141805}{S}electing the most appropriate oral formulation is very challenging when developing medicines for children and in routine practice. Research in pediatric pharmacology has focused on oral drug formulation, determining whether the active pharmaceutical ingredient can be successfully delivered to children. Pediatric expert committees (EMA, EuFPI) recommend that children's medicines be safe, well tolerated, easy to use (palatable and requiring minimal handling), transportable, easily produced, cost effective, commercially viable, with a minimal impact on children's life-style. Oral liquid drug formulations (OLFs: solutions, syrups, suspensions) are historically considered as the most appropriate oral formulation for children, since they are easy to swallow for younger infants and palatable for children. However, OLFs present numerous disadvantages, such as low stability, potentially toxic excipients for children, and low transportability. In the long-term, dose volume and frequency of administration might lead to non-compliance. Multiple preparation steps and volume calculations are also among risk factors for medicine errors in children. An alternative to OLFs is the conventional solid oral dosage form (OSF), such as tablets and capsules. These offer the advantages of greater stability, easy dose selection, improved transportability, and ease of storage. They also allow the modification of drug pharmacokinetic parameters, minimizing administration frequency. Finally, OSFs are less costly than OLFs, since they are easier to develop, manufacture, transport, store, and deliver. Controlled study results suggest that the use of OSFs in children would be associated with greater acceptability by children, greater preference on the part of caregivers, and higher drug compliance than OLFs. Recent controlled studies, confirming that OSFs with an acceptable size for children (mini-tablets), should shift the current paradigm of OLFs as the reference for children's oral medicine. We lack evidence on OSF acceptability in children and its influence on drug compliance, particularly with appropriate-size OSFs for children. Further investigation on oral formulation should investigate the utilisation of OSFs in young children. Few OSFs are licensed for children under 6 years of age.  [\hyperlink{Orlistat}{PMID: 26141805}, A Lajoinie et al., 2015] The age below 5 years is considered a prudential limit for immunotherapy in view of the possible severity of side-effects. Sublingual immunotherapy (SLIT) seems to be safe, but no study in very young children is available. We performed a safety post-marketing surveillance study in children below 5 years. Children aged 3-5 years with respiratory allergy receiving SLIT were followed-up for at least 2 years. A diary card for side-effects was filled by parents at each dose given. Local and systemic side-effects were graded as: mild (no intervention, no dose adjustment), moderate (medical treatment and/or dose reduction), severe (life-threatening/hospitalization/emergency care). The comparative safety of different allergens and regimens was also assessed. One hundred and twenty-six children (mean age 4.2 years, 67 male) were included. Seventy-six (60\%) had rhinitis with asthma, 34 (27\%) rhinitis only and 16 (13\%) only asthma. Immunotherapy was prescribed for mites (62\%), grasses (22.2\%), Parietaria (11.9\%), Alternaria (2.4\%) and olive (1.5\%). Eighteen children underwent an accelerated build-up. The total number of doses was about 39,000. Nine side-effects were reported in seven children (5.6\% patients and 0.2/1000 doses). Two episodes of oral itching and one of abdominal pain were mild. Six gastrointestinal side-effects were controlled by reducing the dose. All side-effects occurred during up-dosing phase. No difference in terms of safety among the allergens used was observed. SLIT is safe also in children under the age of 5 years. [\hyperlink{Orlistat}{PMID: 26141805}, V Di Rienzo et al., 2005]

\hypertarget{pmid_10326811}{T}o determine the safety and efficacy of ofloxacin otic solution in the treatment of acute otorrhea in children with tympanostomy tubes. Multicenter study with an open-label, prospective ofloxacin arm and retrospective historical and current practice arms. Ear, nose, and throat pediatric and general practice clinics and office-based practices. Children younger than 12 years with acute purulent otorrhea of presumed bacterial origin and tympanostomy tubes. Instillation of 0.3\% ofloxacin, 0.25 mL, twice daily for 10 days in the prospective arm; review of medical records in the retrospective arms. The primary index of clinical efficacy was absence (cure) or presence (failure) of otorrhea at 10 to 14 days after therapy. The primary index of microbiologic efficacy (in the ofloxacin arm only) was eradication of pathogens isolated at baseline. Safety was evaluated in the ofloxacin arm only. Significantly more clinically evaluable ofloxacin-treated subjects were cured (84.4\%; 119/141) than were historical practice subjects (64.2\%; 140/218) (P< or =.001) or current practice subjects (70\%; 33/47) (P< or =.03). All baseline pathogens were eradicated in 103 (96.3\%) of 107 microbiologically evaluable ofloxacin subjects. Adverse events considered "possibly" or "probably" treatment related occurred in 29 (12.8\%) of 226 ofloxacin-treated subjects. Ofloxacin is safe and significantly more effective than treatments used in historical or current practice for acute purulent otorrhea in children with tympanostomy tubes. [\hyperlink{Orlistat}{PMID: 10326811}, J E Dohar et al., 1999]

\hypertarget{pmid_26918853}{R}egulations on medicinal products for paediatric use require that pharmacokinetics and safety be characterized specifically in the paediatric population. A previous study established that a 10-mg dose of bilastine in children aged 2 to <12 years provided an equivalent systemic exposure as 20 mg in adults. The current study assessed the safety and tolerability of bilastine 10 mg in children with allergic rhinoconjunctivitis and chronic urticaria. In this phase III, multicentre, double-blind study, children were randomized to once-daily treatment with bilastine 10-mg oral dispersible table (n = 260) or placebo (n = 249) for 12 weeks. Safety evaluations included treatment-emergent adverse events (TEAEs), laboratory tests, cardiac safety (ECG recordings) and somnolence/sedation using the Pediatric Sleep Questionnaire (PSQ). The primary hypothesis of non-inferiority between bilastine 10 mg and placebo was demonstrated on the basis of a near-equivalent proportion of children in each treatment arm without TEAEs during 12 weeks' treatment (31.5 vs. 32.5\%). No clinically relevant differences between bilastine 10 mg and placebo were observed from baseline to study end for TEAEs or related TEAEs, ECG parameters and PSQ scores. The majority of TEAEs were mild or moderate in intensity. TEAEs led to discontinuation of two patients treated with bilastine 10 mg and one patient treated with placebo. Bilastine 10 mg had a safety and tolerability profile similar to that of placebo in children aged 2 to <12 years with allergic rhinoconjunctivitis or chronic urticaria. [\hyperlink{Orlistat}{PMID: 26918853}, Zoltán Novák et al., 2016]

\hypertarget{pmid_12105286}{T}o study the safety, tolerability, and potential efficacy of orlistat in adolescents with obesity and its comorbid conditions. We studied 20 adolescents (age, 14.6 +/- 2.0 years; body mass index, 44.1 +/- 12.6 kg/m(2)). Subjects were evaluated before and after taking orlistat (120 mg three times daily) and a multivitamin for 3 months. Subjects were simultaneously enrolled in a 12-week program emphasizing diet, exercise, and strategies for behavior change. Participants who completed treatment (85\%) reported taking 80\% of prescribed medication. Adverse effects were generally mild, limited to gastrointestinal effects observed in adults, and decreased with time. Three subjects required additional vitamin D supplementation despite the prescription of a daily multivitamin containing vitamin D. Weight decreased significantly (-4.4 +/- 4.6 kg, p < 0.001; -3.8 +/- 4.1\% of initial weight), as did body mass index (-1.9 +/- 2.5 kg/m(2); p < 0.0002). Total cholesterol (-21.3 +/- 24.7 mg/dL; p < 0.001), low-density lipoprotein-cholesterol (-17.3 +/- 15.8 mg/dL; p < 0.0001), fasting insulin (-13.7 +/- 19.0 microU/mL; p < 0.02), and fasting glucose (-15.4 +/- 7.4 mg/dL; p < 0.003) were also significantly lower after orlistat. Insulin sensitivity, assessed by a frequently sampled intravenous glucose-tolerance test, improved significantly (p < 0.02). We conclude that, in adolescents, short-term treatment with orlistat, in the context of a behavioral program, is well-tolerated and has a side-effect profile similar to that observed in adults, but its true benefit versus conventional therapy remains to be determined in placebo-controlled trials. [\hyperlink{Orlistat}{PMID: 12105286}, Jennifer R McDuffie et al., 2002]

\hypertarget{pmid_34850647}{M}anagement guidelines for allergic rhinitis and urticaria recommend oral second-generation antihistamines as first-line treatment. The efficacy and safety of bilastine, the newest nonsedating second-generation antihistamine, are well established in adolescents/adults with these allergic conditions. The bilastine development program for pediatric use (2-<12 years) followed EMA-authorized processes. Pharmacokinetic/pharmacodynamic simulation and modeling and a pharmacokinetic study were conducted to identify and confirm the pediatric dose (10 mg/day). A Phase III, multicenter, double-blind, randomized, placebo-controlled, parallel-group study was performed to confirm the safety of bilastine 10 mg/day in children. In this article, evidence is reviewed for use of bilastine in children with allergic rhinoconjunctivitis or urticaria. Several cases are presented which demonstrate its role in routine clinical practice. [\hyperlink{Orlistat}{PMID: 34850647}, Pablo Rodríguez Del Río et al., 2022]

\hypertarget{pmid_20958104}{T}he prevalence of overweight and obesity in children and adolescents is increasing rapidly, while the short- and long-term morbid outcomes make these entities a major public health concern. Initial steps in therapy are based on dietary and lifestyle intervention. In the presence of an insufficient progress, medication or - eventually - surgery may be recommended. Three drugs are currently used: orlistat, metformin, and sibutramine; other candidates are in development. However, trials assessing the efficacy and safety of the current medications are frequently affected by methodological limitations, in particular insufficient power and period of follow-up. The efficacy and safety of antiobesity drugs currently used for children and adolescents are reviewed. Additional information on upcoming agents is presented. This is an exhaustive review of current state on controversial issues regarding drugs used in children and adolescent obesity, specifically related with their efficacy and safety. The efficacy of these drugs is modest. Our knowledge of their efficacy and safety comes from clinical trials affected by insufficient follow-up (1 year or less); very often, these trials are of limited power. Further data from larger and longer well-designed clinical trials would be advisable. [\hyperlink{Orlistat}{PMID: 20958104}, Natalia Catoira et al., 2010]

\hypertarget{pmid_28237130}{C}linical research has shown that sublingual immunotherapy (SLIT) is effective and safe in moderate-severe allergic rhinitis (AR) induced by house dust mite (HDM). However, the sample size in many studies is small. Meanwhile, the controversy on the efficacy and safety in the very young children younger than four years old still existed. The aim of this retrospective study is to evaluate the efficacy and safety of SLIT with Dermatophagoides farinae (Der.f) extracts in children and adult patients with allergic rhinitis, particularly in the very young children. A total of 573 subjects aged 3-69 with AR received a three-year course of sublingual immunotherapy with Der.f extracts along with pharmacotherapy. The total nasal symptoms score (TNSS), total medication score (TMS), visual analogue score (VAS) and adverse events (AEs) were evaluated at each visit. TNSS, TMS, VAS were significantly improved during the three-year course of treatment in comparison to the baseline values (P<0.01). Besides, significant improvement in nasal symptoms and reduction of medication use were also observed in young children aged 3-6 years (P<0.01). No severe systemic adverse events (AEs) were reported. SLIT with Der.f drops is clinically effective and safe in children and adult patients with HDM-induced AR, including the very young children less than four years old. [\hyperlink{Orlistat}{PMID: 28237130}, X Lin et al., ]

\hypertarget{pmid_29073756}{O}floxacin is a commonly used quinolone antibiotic both in adults as well as children. It is generally safe and well tolerated. Rarely, neurological and psychiatric adverse reactions are reported to occur with ofloxacin. We report a case of a child who developed delirium after ofloxacin treatment, that resolved after medication discontinuation and treatment with low dose olanzapine. [\hyperlink{Orlistat}{PMID: 29073756}, Arnab Bhattacharya et al., 2017]

\hypertarget{pmid_8387973}{O}rlistat (Ro 18-0647) is an inhibitor of gastric, carboxylester and pancreatic lipase and specifically reduces the absorption of dietary fat due to the inhibition of triglyceride hydrolysis. Orlistat can be used for the treatment of obesity. Of 52 healthy obese patients entering a four-week single-blind run-in period with diet (500 kcal-reduced, containing 30\% of calories in the form of fat) and placebo three times a day, 44 patients showed compliance to the diet by reducing their body weight by 0.5-4 kg from screening. These patients were randomized for a 12-week double-blind, parallel group, placebo-controlled treatment period with diet and 50 mg Orlistat or placebo three times a day. Complete data were available for 39 patients, 20 on Orlistat (3 men, 17 women; mean weight 85.5 +/- 12.1 kg; mean body mass index 30.6 +/- 3.7 kg/m2) and 19 on placebo (3 men, 16 women; mean weight 81.9 +/- 7.9 kg; mean body mass index 30.0 +/- 2.6 kg/m2. Total weight loss after randomization was 4.3 +/- 3.4 kg in the Orlistat group and 2.1 +/- 2.8 kg in the placebo group (P = 0.025, analysis of variance with repeated measurements; 95\% confidence interval for the weight loss difference 0.2-4.2 kg). Gastrointestinal side effects were seen in the Orlistat group, but in most patients the symptoms were mild or transient. One patient dropped out because of faecal incontinence. No effect was seen on vitamin A levels, but vitamin E levels became lower in the Orlistat group (P < 0.05, paired t test).(ABSTRACT TRUNCATED AT 250 WORDS) [\hyperlink{Orlistat}{PMID: 8387973}, M L Drent et al., 1993] The efficacy and safety of monomeric allergoid (Lofarma, Milan) have been demonstrated in adults but very few studies have examined it in children. This study therefore investigated the efficacy and safety of this sublingual immunotherapy (SLIT) at the dosage of 1000 AU five times a week without any up-dosing. Forty allergic children (17 M and 23 F, mean age 7 years, range 4-16 years), 16 with rhinitis and 24 with rhinitis and asthma, were randomized to SLIT or drug therapy. All the patients were sensitized to grass; some were also sensitized, though to a lesser extent, to Parietaria, Olea and Betulaceae. The patients were treated pre-/co-seasonally for two years. A visual analogue scale (VAS) was used at baseline and at the end of the first and second pollen seasons to rate the patients' well-being. The VAS score was significantly higher after both the first and the second year of treatment in the SLIT group than in the controls (p<0.05). It improved in comparison to baseline only in the active group. All 40 children tolerated the therapy very well. The monomeric allergoid at the dosage of 5000 AU/week thus appears to have a good efficacy and safety profile in children. [\hyperlink{Orlistat}{PMID: 8387973}, F Agostinis et al., 2009]

\hypertarget{pmid_18095746}{O}rlistat, an anti-obesity drug, is a potent and specific inhibitor of intestinal lipases. In light of the recent US FDA approval of the over-the-counter sale of orlistat (60 mg three times daily), clinicians need to be aware that its use may be associated with less well known, but sometimes clinically relevant, adverse effects. More specifically, the use of orlistat has been associated with several mild-to-moderate gastrointestinal adverse effects, such as oily stools, diarrhoea, abdominal pain and faecal spotting. A few cases of serious hepatic adverse effects (cholelithiasis, cholostatic hepatitis and subacute liver failure) have been reported. However, the effects of orlistat on non-alcoholic fatty liver disease are beneficial. Orlistat-induced weight loss seems to have beneficial effects on blood pressure. No effect has been observed on calcium, phosphorus, magnesium, iron, copper or zinc balance or on bone biomarkers. Interestingly, the use of orlistat has been associated with rare cases of acute kidney injury, possibly due to the increased fat malabsorption resulting from the inhibition of pancreatic and gastric lipase by orlistat, leading to the formation of soaps with calcium and resulting in increased free oxalate absorption and enteric hyperoxaluria. Orlistat has a beneficial effect on carbohydrate metabolism. No significant effect on cancer risk has been reported with orlistat.Orlistat interferes with the absorption of many drugs (such as warfarin, amiodarone, ciclosporin and thyroxine as well as fat-soluble vitamins), affecting their bioavailability and effectiveness. This review considers orlistat-related adverse effects and drug interactions. The clinical relevance and pathogenesis of these effects is also discussed. [\hyperlink{Orlistat}{PMID: 18095746}, Theodosios D Filippatos et al., 2008]

\hypertarget{pmid_18200802}{O}ver the past 20 years obesity has become a worldwide concern of frightening proportion. The World Health Organization estimates that there are over 400 million obese and over 1.6 billion overweight adults, a figure which is projected to almost double by 2015. This is not a disease restricted to adults - at least 20 million children under the age of 5 years were overweight in 2005 (WHO 2006). Overweight and obesity lead to serious health consequences including coronary artery disease, stroke, type-2 diabetes, heart failure, dyslipidemia, hypertension, reproductive and gastrointestinal cancers, gallstones, fatty liver disease, osteoarthritis and sleep apnea (Padwal et al 2003). Modest weight loss in the obese of between 5\% and 10\% of bodyweight is associated with improvements in cardiovascular risk profiles and reduced incidence of type 2 diabetes (Goldstein 1992; Avenell et al 2004; Padwal and Majumdar 2007). Orlistat, a gastric and pancreatic lipase inhibitor that reduces dietary fat absorption by approximately 30\%, has been approved for use for around ten years (Zhi et al 1994; Hauptman 2000). There is now a growing body of evidence to suggest that Orlistat assists weight loss and that it may also have additional benefits. The aim of this review is to provide a brief update on the current literature studying the efficacy, safety and significance of the use of Orlistat in clinical practice. [\hyperlink{Orlistat}{PMID: 18200802}, Belinda S Drew et al., 2007]

\hypertarget{pmid_27678432}{T}he efficacy and safety of atorvastatin in children/adolescents aged 10-17 years with heterozygous familial hypercholesterolemia (HeFH) have been demonstrated in trials of up to 1 year in duration. However, the efficacy/safety of >1 year use of atorvastatin in children/adolescents with HeFH, including children from 6 years of age, has not been assessed. To characterize the efficacy and safety of atorvastatin over 3 years and to assess the impact on growth and development in children aged 6-15 years with HeFH. A total of 272 subjects aged 6-15 years with HeFH and low-density lipoprotein cholesterol (LDL-C) ≥4.0 mmol/L (154 mg/dL) were enrolled in a 3-year study (NCT00827606). Subjects were initiated on atorvastatin (5 mg or 10 mg) with doses increased to up to 80 mg based on LDL-C levels. Mean percentage reductions from baseline in LDL-C at 36 months/early termination were 43.8\% for subjects at Tanner stage (TS) 1 and 39.9\% for TS ≥2. There was no evidence of variations in the lipid-lowering efficacy of atorvastatin between the TS groups analyzed (1 vs ≥2) or in subjects aged <10 vs ≥10 years, and the treatment had no adverse effect on growth or maturation. Atorvastatin had a favorable safety and tolerability profile, and only 6 (2.2\%) subjects discontinued because of adverse events. Atorvastatin over 3 years was efficacious, had no impact on growth/maturation, and was well tolerated in children and adolescents with HeFH aged 6-15 years. [\hyperlink{Orlistat}{PMID: 27678432}, Gisle Langslet et al., ]

\hypertarget{pmid_33465554}{T}he efficacy and safety of montelukast in children with obstructive sleep apnea (OSA) remain controversial. Therefore, the aims of this systemic review and meta-analysis are to verify this issue and further provide reference for clinical practice. Seven databases were searched for randomized controlled trials (RCTs) up to September 30, 2019. The literature screening and data extraction were performed by two independent researchers. Adverse reactions from trials were also recorded. Meta-analysis was performed and analyzed heterogeneity. Methodological and evidence quality were followed by to evaluate according to Cochrane handbook. A total of 4 RCTs including 305 children with mild to moderate OSA were involved. Compared with placebo, we found that oral montelukast (OM) significantly improved polysomnography (PSG) monitoring parameters, typical and relevant symptoms including snoring and mouth breathing, and adenoid morphology in children with OSA. When compared with routine drugs, not only PSG monitoring parameters and adenoid morphology, but also sleep-disordered breathing (SDB)-related questionnaire scores were improved in patients with OSA treated by combination of OM and routine drugs. In addition, compared with single nasal spray of mometasone furoate, the present study also showed that OM combined with nasal spray of mometasone furoate significantly improved PSG monitoring parameters, symptoms of snoring and mouth breathing and reduced tonsil morphology in pediatric OSA. In terms of treatment safety, one study reported adverse reactions of OM such as headache, nausea and vomiting, while no adverse events were reported after OM treatment in another study. As a classic leukotriene receptor antagonist, montelukast can be used to treat children with mild to moderate OSA in the short term and improve clinical characteristics. The promotion and application of OM in clinic is considered to be a noninvasive option to avoid surgical treatment. [\hyperlink{Orlistat}{PMID: 33465554}, Tingting Ji et al., 2021]

\hypertarget{pmid_20718927}{T}he efficacy and safety of five-grass pollen 300IR sublingual immunotherapy (SLIT) tablets (Stallergènes SA, France) have previously been demonstrated in paediatric patients. This report presents additional data concerning efficacy at pollen peak, efficacy and safety according to age, nasal and ocular symptoms, use of rescue medication, satisfaction with treatment and compliance. Children (5-11 yr) and adolescents (12-17 yr) with grass pollen-allergic rhinoconjunctivitis were included in a multinational, randomized, double-blind, placebo-controlled study and received either a 300IR five-grass pollen tablet or placebo daily in a pre- (4 months) and co-seasonal protocol. The severity of six symptoms (sneezing, rhinorrhoea, nasal congestion, nasal and ocular pruritis, and tearing) was scored, and rescue medication use was recorded daily during the pollen season. Patient satisfaction was recorded at the season end. A total of 161 children and 117 adolescents were evaluated (n = 267). 300IR SLIT was effective over the whole season (p = 0.0010) and at the pollen peak (p = 0.0009). The adjusted mean difference between 300IR and placebo groups was significant for both nasal (p = 0.0183) and ocular (p < 0.0001) symptoms. Rescue medication use was statistically lower in the SLIT group during the pollen season and at the pollen peak (both p < 0.05). More patients in the SLIT group were satisfied with their treatment compared to placebo (83.2\% vs. 68.1\%, p = 0.0030), and compliance was high (SLIT 93.9\% of patients were compliant, placebo 94.8\% of patients were compliant). SLIT was well tolerated by children and adolescents. 300IR five-grass pollen tablets are effective and safe during the pollen season and at the pollen peak in children and adolescents with grass pollen rhinoconjunctivitis. [\hyperlink{Orlistat}{PMID: 20718927}, Susanne Halken et al., 2010]

\hypertarget{pmid_32602383}{T}o assess the efficacy and safety of omalizumab in children with moderate-to-severe asthma. We systematically searched MEDLINE, EMBASE, and Cochrane for randomized controlled trials (RCTs ) (inception to January 2020). All RCTs which were conducted in childhood and adolescence with asthma and compared the efficacy or safety of omalizumab were adopted. Three studies with four publications including 1380 pediatric patients met our criteria. For children with moderate-to-severe asthma, omalizumab decreased asthma exacerbations rate (OR 0.51, 95\% CI: 0.44-0.58,  These findings suggested that omalizumab had beneficial effects on moderate-to-severe asthma in children. Patients may benefit more from long-term use of omalizumab. In addition, omalizumab reduces the rate of serious adverse events requiring hospitalizations. [\hyperlink{Orlistat}{PMID: 32602383}, Zhuo Fu et al., 2021]

\hypertarget{pmid_30387022}{O}rlistat is an inhibitor of pancreatic lipase and is used as an anti-obesity drug in many countries. However, there are no data available regarding the effects of orlistat on visceral fat (VF) accumulation in Japanese individuals. Therefore, this study aimed to analyze the efficacy and safety of 52 weeks of orlistat administration in Japanese individuals. Orlistat 60 mg was administered orally three times daily for 52 weeks to Japanese participants with excessive VF accumulation and without dyslipidemia, diabetes mellitus, and hypertension (metabolic diseases). Participants were also counseled to improve their diet and to maintain exercise habits. We defined excessive VF accumulation as a waist circumference (WC) of ≥ 85 cm for males and ≥ 90 cm for females, which corresponds to a VF area of 100 cm VF, WC, and BW were significantly reduced at week 52 from baseline; the mean ± standard error rate of change was - 21.52\% ± 1.89\%, - 4.89\% ± 0.45\%, and - 5.36\% ± 0.56\%, respectively, and continued to reduce throughout the 52 weeks; these significantly reduced at whole term compared with baseline. Most adverse reactions were defecation-related symptoms such as oily spotting and flatus with discharge (flatus with small amounts of stool or oil) due to the pharmacologic effects of the lipase inhibitor. These symptoms were mostly mild, reversible, and recognizable by the participants; none were serious or severe. No participants discontinued by medical judgment about adverse reactions, and the drug could be administered continuously. VF, WC, and BW were reduced from week 4 to week 52, indicating the effect of long-term orlistat administration. Moreover, it was well tolerated with an acceptable safety profile. Long-term administration of orlistat may be efficacious in reducing VF accumulation with safety when used in combination with diet and exercise. This study is registered with the Japan Pharmaceutical Information Center (identifier: JapicCTI-184004). Funding for this study was provided by Taisho Pharmaceutical Co., Ltd. [\hyperlink{Orlistat}{PMID: 30387022}, Kohji Shirai et al., 2019]

\hypertarget{pmid_16822385}{O}titis media (OM) is one of the most common childhood disorders and has been associated with an alarming rise in prevalence. Because medical therapy with antibiotics, antihistamines, decongestants, and corticosteroids has no demonstrable efficacy in the treatment of OM, the mainstay of treatment is surgical intervention. Although effective, surgical treatment is quite expensive and exposes patients to the risks associated with general anesthesia. Results of recent studies suggest that locally produced leukotrienes may play a role in the pathogenesis of OM. As such, a potential candidate for treatment of OM is the leukotriene receptor antagonist, montelukast, which has been approved by the US Food and Drug Administration for the treatment of asthma and/or allergic rhinitis. If proven effective, this drug's safety/convenience profiles and indication for children as young as 6 months of age would make it suitable for treatment of OM, and perhaps even prophylaxis of OM in at-risk populations. [\hyperlink{Orlistat}{PMID: 16822385}, Deborah Gentile et al., 2006]

\section*{Mavorixafor}
\subsection*{Result}
\subsubsection*{Answer}

Unknown

\subsubsection*{{Explanation}}
\hypertarget{Mavorixafor}
A review of the available abstracts reveals the following regarding the safety of Mavorixafor in children:

1. \hyperlink{pmid_32870250}{PMID: 32870250}, David C Dale et al., 2020 – This is a phase 2 open-label dose-escalation and extension study of Mavorixafor in 8 adult patients with WHIM syndrome. The study reports on safety, tolerability, pharmacokinetics, pharmacodynamics, and preliminary efficacy, but all participants were adults. No children were included in this study.

2. \hyperlink{pmid_36923305}{PMID: 36923305}, Robert H I Andtbacka et al., 2022 – This is a phase Ib study of Mavorixafor in 16 patients with melanoma, focusing on immune modulation and safety. The study population consisted of adults; there is no mention of pediatric patients.

None of the other abstracts mention Mavorixafor or its use in children.

Therefore, based on the abstracts available, there are no targeted studies evaluating the safety of Mavorixafor in children of any age group. As such, the safety of Mavorixafor in children is unknown.

\subsection*{Abstracts}
\hypertarget{pmid_37595103}{m}iniSTONE-2 (NCT03629184) was a global, phase 3, randomized, controlled study that investigated the safety and efficacy of single-dose baloxavir marboxil in otherwise healthy children 1-<12 years of age and showed a positive risk-benefit profile. This post hoc analysis evaluated the safety and efficacy of baloxavir versus oseltamivir in children 5-11 years old with influenza. Children received single-dose baloxavir or twice-daily oseltamivir for 5 days. Safety was the primary objective. Efficacy and virological outcomes included time to alleviation of symptoms, duration of fever and time to cessation of viral shedding by titer. Data were summarized descriptively. Ninety-four children 5-11 years old were included (61 baloxavir and 33 oseltamivir). Baseline characteristics were similar between the groups. The incidence of adverse events was balanced and low in both treatment groups, with the most common being vomiting (baloxavir 5\% vs. oseltamivir 18\%), diarrhea (5\% vs. 0\%) and otitis media (0\% vs. 5\%). No serious adverse events or deaths occurred. Median (95\% CI) time to alleviation of symptoms with baloxavir was 138.4 hours (116.7-163.4) versus 126.1 hours (95.9-165.7) for oseltamivir; duration of fever was comparable between groups [41.2 hours (23.5-51.4) vs. 51.3 hours (30.7-56.8), respectively]. Median time to cessation of viral shedding was shorter in the baloxavir group versus oseltamivir (1 vs. ≈3 days). Safety, efficacy and virological results in children 5-11 years were similar to those from the overall study population 1-<12 years of age. Single-dose baloxavir provides an additional treatment option for pediatric patients 5-11 years old with influenza. [\hyperlink{Mavorixafor}{PMID: 37595103}, Jeffery B Baker et al., 2023]

\hypertarget{pmid_32870250}{W}arts, hypogammaglobulinemia, infections, and myelokathexis (WHIM) syndrome is a rare primary immunodeficiency caused by gain-of-function mutations in the CXCR4 gene. We report the safety, tolerability, pharmacokinetics, pharmacodynamics, and preliminary efficacy of mavorixafor from a phase 2 open-label dose-escalation and extension study in 8 adult patients with genetically confirmed WHIM syndrome. Mavorixafor is an oral small molecule selective antagonist of the CXCR4 receptor that increases mobilization and trafficking of white blood cells from the bone marrow. Patients received escalating doses of mavorixafor, up to 400 mg once daily. Five patients continued on the extension study for up to 28.6 months. Mavorixafor was well tolerated with no treatment-related serious adverse events. At a median follow-up of 16.5 months, we observed dose-dependent increases in absolute neutrophil count (ANC) and absolute lymphocyte count (ALC). At doses ≥300 mg/d, ANC was maintained at >500 cells per microliter for a median of 12.6 hours, and ALC was maintained at >1000 cells per microliter for up to 16.9 hours. Continued follow-up on the extension study resulted in a yearly infection rate that decreased from 4.63 events (95\% confidence interval, 3.3-6.3) in the 12 months prior to the trial to 2.27 events (95\% confidence interval, 1.4-3.5) for patients on effective doses. We observed an average 75\% reduction in the number of cutaneous warts. This study demonstrates that mavorixafor, 400 mg once daily, mobilizes neutrophil and lymphocytes in adult patients with WHIM syndrome and provides preliminary evidence of clinical benefit for patients on long-term therapy. The trial was registered at www.clinicaltrials.gov as \#NCT03005327. [\hyperlink{Mavorixafor}{PMID: 32870250}, David C Dale et al., 2020]

\hypertarget{pmid_36923305}{M}avorixafor is an oral, selective inhibitor of the CXCR4 chemokine receptor that modulates immune cell trafficking. A biomarker-driven phase Ib study (NCT02823405) was conducted in 16 patients with melanoma to investigate the hypothesis that mavorixafor favorably modulates immune cell profiles in the tumor microenvironment (TME) and to evaluate the safety of mavorixafor alone and in combination with pembrolizumab. Serial biopsies of melanoma lesions were assessed after 3 weeks of mavorixafor monotherapy and after 6 weeks of combination treatment for immune cell markers by NanoString analysis for gene expression and by multiplexed immunofluorescent staining for  Within the TME, mavorixafor alone increased CD8 Treatment with single-agent mavorixafor resulted in enhanced immune cell infiltration and activation in the TME, leading to increases in TIS and IFNγ gene signatures. Mavorixafor as a single agent, and in combination with pembrolizumab, has an acceptable safety profile. These data support further investigation of the use of mavorixafor for patients unresponsive to checkpoint inhibitors. Despite survival improvements in patients with melanoma treated with checkpoint inhibitor therapy, a significant unmet medical need exists for therapies that enhance effectiveness. We propose that mavorixafor sensitizes the melanoma tumor microenvironment and enhances the activity of checkpoint inhibitors, and thereby may translate to a promising treatment for broader patient populations. [\hyperlink{Mavorixafor}{PMID: 36923305}, Robert H I Andtbacka et al., 2022]

\hypertarget{pmid_32516282}{B}aloxavir marboxil (baloxavir) is a novel, cap-dependent endonuclease inhibitor that has previously demonstrated efficacy in the treatment of influenza in adults and adolescents. We assessed the safety and efficacy of baloxavir in otherwise healthy children with acute influenza. MiniSTONE-2 (Clinicaltrials.gov: NCT03629184) was a double-blind, randomized, active controlled trial enrolling children 1-<12 years old with a clinical diagnosis of influenza. Children were randomized 2:1 to receive either a single dose of oral baloxavir or oral oseltamivir twice daily for 5 days. The primary endpoint was incidence, severity and timing of adverse events (AEs); efficacy was a secondary endpoint. In total, 173 children were randomized and dosed, 115 to the baloxavir group and 58 to the oseltamivir group. Characteristics of participants were similar between treatment groups. Overall, 122 AEs were reported in 84 (48.6\%) children. Incidence of AEs was similar between baloxavir and oseltamivir groups (46.1\% vs. 53.4\%, respectively). The most common AEs were gastrointestinal (vomiting/diarrhea) in both groups [baloxavir: 12 children (10.4\%); oseltamivir: 10 children (17.2\%)]. No deaths, serious AEs or hospitalizations were reported. Median time (95\% confidence interval) to alleviation of signs and symptoms of influenza was similar between groups: 138.1 (116.6-163.2) hours with baloxavir versus 150.0 (115.0-165.7) hours with oseltamivir. Oral baloxavir is well tolerated and effective at alleviating symptoms in otherwise healthy children with acute influenza. Baloxavir provides a new therapeutic option with a simple oral dosing regimen. [\hyperlink{Mavorixafor}{PMID: 32516282}, Jeffrey Baker et al., 2020]

\hypertarget{pmid_16028153}{B}ecause of concerns about arthrotoxicity, fluoroquinolones are restricted for use in children. This study describes the safety and efficacy of gatifloxacin when used for treatment of children with recurrent acute otitis media (ROM) or acute otitis media (AOM) treatment failure (AOMTF). We performed an analysis of 867 children included in 4 clinical trials who had ROM and/or AOMTF and were treated with gatifloxacin (10 mg/kg once daily for 10 days). Gatifloxacin had adverse event rates that were similar overall to those of a comparator antibiotic (amoxicillin-clavulanate), except for increased diarrhea in children <2 years old receiving amoxicillin-clavulanate. There was no evidence of arthrotoxicity, hepatotoxicity, alteration of glucose homeostasis, or central nervous system toxicity acutely or during 1 year follow-up in any child. Regarding efficacy, in 2 noncomparative trials, the gatifloxacin cure rate of AOM was 89\% (95\% confidence interval [CI], 83\%-95\%) at the test of cure (TOC) visit, 3-10 days after completion of therapy. In 2 comparative trials of gatifloxacin versus amoxicillin-clavulanate, the efficacy of gatifloxacin was 88\% (95\% CI, 82\%-94\%). Gatifloxacin led to better clinical outcomes than amoxicillin-clavulanate for AOMTF (91\% vs. 81\%; P=.029), for AOMTF and age <2 years old (89\% vs. 69\%; P=.009), and for severe AOM in children <2 years old (90\% vs. 75\%; P=.012). Among children with AOMTF previously treated with amoxicillin-clavulanate or ceftriaxone injections, gatifloxacin cure rates were high (88\% and 75\%, respectively). Gatifloxacin appears to be safe for children, with no evidence of producing arthrotoxicity in 867 children exposed to the antibiotic when used as treatment for ROM and AOMTF. [\hyperlink{Mavorixafor}{PMID: 16028153}, Michael E Pichichero et al., 2005]

\hypertarget{pmid_27798221}{C}hild-friendly, low-cost, solid, oral fixed-dose combinations (FDCs) of efavirenz with lamivudine and abacavir are urgently needed to improve clinical management and drug adherence for children. Data were pooled from several clinical trials and therapeutic drug monitoring datasets from different countries. The number of children/observations was 505/3667 for efavirenz. Population pharmacokinetic analyses were performed using a non-linear mixed-effects approach. For abacavir and lamivudine, data from 187 and 920 subjects were available (population pharmacokinetic models previously published). Efavirenz/lamivudine/abacavir FDC strength options assessed were (I) 150/75/150, (II) 120/60/120 and (III) 200/100/200 mg. Monte Carlo simulations of the different FDC strengths were performed to determine the optimal dose within each of the WHO weight bands based on drug efficacy/safety targets. The probability of being within the target efavirenz concentration range 12 h post-dose (1-4 mg/L) varied between 56\% and 60\%, regardless of FDC option. Option I provided a best possible balance between efavirenz treatment failure and toxicity risks. For abacavir and lamivudine, simulations showed that for option I >75\% of subjects were above the efficacy target. According to simulations, a paediatric efavirenz/lamivudine/abacavir fixed-dose formulation of 150 mg efavirenz, 75 mg lamivudine and 150 mg abacavir provided the most effective and safe concentrations across WHO weight bands, with the flexibility of dosage required across the paediatric population. [\hyperlink{Mavorixafor}{PMID: 27798221}, Naïm Bouazza et al., 2017]

\hypertarget{pmid_36681802}{A}nti-influenza treatment is important for children and is recommended in many countries. This study assessed safety, clinical, and virologic outcomes of baloxavir marboxil (baloxavir) treatment in children based on age and influenza virus type/subtype. This was a post hoc pooled analysis of two open-label non-controlled studies of a single weight-based oral dose of baloxavir (day 1) in influenza virus-infected Japanese patients aged < 6 years (n = 56) and ≥ 6 to < 12 years (n = 81). Safety, time to illness alleviation (TTIA), time to resolution of fever (TTRF), recurrence of influenza illness symptoms and fever (after day 4), virus titer, and outcomes by polymerase acidic protein variants at position I38 (PA/I38X) were evaluated. Adverse events were reported in 39.0 and 39.5\% of patients < 6 years and ≥ 6 to < 12 years, respectively. Median (95\% confidence interval) TTIA was 43.2 (36.3-68.4) and 45.4 (38.9-61.0) hours, and TTRF was 32.2 (26.8-37.8) and 20.7 (19.2-23.8) hours, for patients < 6 years and ≥ 6 to < 12 years, respectively. Symptom and fever recurrence was more common in patients < 6 years with influenza B (54.5 and 50.0\%, respectively) compared with older patients (0 and 25.0\%, respectively). Virus titers declined (day 2) for both age groups. Transient virus titer increase and PA/I38X-variants were more common for patients < 6 years. The safety and effectiveness of single-dose baloxavir were observed in children across all age groups and influenza virus types. Higher rates of fever recurrence and transient virus titer increase were observed in children < 6 years. Japan Pharmaceutical Information Center Clinical Trials Information JapicCTI-163,417 (registered 02 November 2016) and JapicCTI-173,811 (registered 15 December 2017). [\hyperlink{Mavorixafor}{PMID: 36681802}, Nobuo Hirotsu et al., 2023]

\hypertarget{pmid_3430711}{F}lomoxef (FMOX, 6315-S), a new parenteral oxacephem antibiotic, was evaluated for its safety, efficacy and pharmacokinetics in children. Twenty-six patients with bacterial infections were treated with FMOX. Clinical efficacy rate was 92\% and bacteriological cure rate was 85\%. Three cases of infections due to methicillin-resistant Staphylococcus aureus were cured with FMOX therapy. No severe adverse reactions or abnormalities of laboratory test data were associated with FMOX therapy, although loose stools and diarrhea occurred frequently (23\%). Serum half-lives of FMOX after a single bolus injection of 9 infants and children were 0.77 +/- 0.31 hour and excretion into urine was rapid. From these experiences, FMOX appeared to be a safe and effective antibiotic when used in children with susceptible bacterial infections. [\hyperlink{Mavorixafor}{PMID: 3430711}, H Meguro et al., 1987]

\hypertarget{pmid_20014952}{V}alacyclovir provides enhanced acyclovir bioavailability in adults, but limited data are available in children. Children 1 month through 5 years of age with or at risk for herpesvirus infection received a single 25 mg/kg dose of extemporaneously compounded valacyclovir oral suspension (n = 57), whereas children 1 through 11 years of age received 10 mg/kg valacyclovir oral suspension twice daily for 3-5 days (herpes simplex virus infection) (n = 28) or 20 mg/kg 3 times daily for 5 days (varicella-zoster virus infection) (n = 27). Blood samples for pharmacokinetic analysis were collected during the 6 h after the first dose. Safety was monitored throughout the studies. Dose proportionality in the maximum observed concentration (C(max)) of acyclovir and the area under the concentration-time curve from time zero extrapolated to infinity (AUC(0-infinity)) existed across the 10 to 20 mg/kg valacyclovir dose range. For children 2 through 5 years of age, an increase in dose from 20 to 25 mg/kg resulted in near doubling of the C(max) and AUC(0-infinity). Among infants 1 through 2 months of age receiving 25 mg/kg, the mean AUC(0-infinity) and C(max) were higher ( approximately 60\% and 30\%, respectively) than those among older infants and children receiving the same dose. Valacyclovir oral suspension was well tolerated. No clinically significant trends were noted in clinical chemical, hematologic, or urinalysis values from screening to follow-up. Among children 3 months through 11 years of age, the 20 mg/kg dose of this formulation of valacyclovir oral suspension produces favorable acyclovir blood concentrations and is well tolerated. A dosing recommendation cannot be made for infants <3 months of age because of decreased clearance in this age group. Trial registration. ClinicalTrials.gov identifier: NCT00297206 . [\hyperlink{Mavorixafor}{PMID: 20014952}, David W Kimberlin et al., 2010]

\hypertarget{pmid_16257311}{F}ive independent, multicentered, double-masked, parallel, controlled studies were conducted to determine the safety of moxifloxacin ophthalmic solution 0.5\% (VIGAMOX) in pediatric and nonpediatric patients with bacterial conjunctivitis. Patients were randomized into one of two treatment groups in each study and received either moxifloxacin ophthalmic solution 0.5\% b.i.d. or t.i.d. or a comparator. A total of 1,978 patients (918 pediatric and 1,060 nonpediatric) was evaluable for safety. The most frequent adverse event in the overall safety population was transient ocular discomfort, occurring at an incidence of 2.8\%, which was similar to that observed with the vehicle. No treatment-related changes in ocular signs or visual acuity were observed with moxifloxacin ophthalmic solution 0.5\%, except for one clinically relevant change in visual acuity. Thus, based upon a review of adverse events and an assessment of ocular parameters, moxifloxacin ophthalmic solution 0.5\% formulated without the preservative, benzalkonium chloride, is safe and well tolerated in pediatric (3 days-17 years of age) and nonpediatric (18-93 years) patients with bacterial conjunctivitis. [\hyperlink{Mavorixafor}{PMID: 16257311}, Lewis H Silver et al., 2005]

\hypertarget{pmid_29356761}{T}his study was designed to evaluate primarily the safety and also the efficacy of moxifloxacin (MXF) in children with complicated intra-abdominal infections (cIAIs). In this multicenter, randomized, double-blind, controlled study, 451 pediatric patients aged 3 months to 17 years with cIAIs were treated with intravenous/oral MXF (N = 301) or comparator (COMP, intravenous ertapenem followed by oral amoxicillin/clavulanate; N = 150) for 5 to 14 days. Doses of MXF were selected based on the results of a Phase 1 study in pediatric patients (NCT01049022). The primary endpoint was safety, with particular focus on cardiac and musculoskeletal safety; clinical and bacteriologic efficacy at test of cure was also investigated. The proportion of patients with adverse events (AEs) was comparable between the 2 treatment arms (MXF: 58.1\% and COMP: 54.7\%). The incidence of drug-related AEs was higher in the MXF arm than in the COMP arm (14.3\% and 6.7\%, respectively). No cases of QTc interval prolongation-related morbidity or mortality were observed. The proportion of patients with musculoskeletal AEs was comparable between treatment arms; no drug-related events were reported. Clinical cure rates were 84.6\% and 95.5\% in the MXF and COMP arms, respectively, in patients with confirmed pathogen(s) at baseline. MXF treatment was well tolerated in children with cIAIs. However, a lower clinical cure rate was observed with MXF treatment compared with COMP. This study does not support a recommendation of MXF for children with cIAIs when alternative more efficacious antibiotics with better safety profile are available. [\hyperlink{Mavorixafor}{PMID: 29356761}, Stefan Wirth et al., 2018]

\hypertarget{pmid_34045119}{B}aloxavir marboxil is an oral anti-influenza drug with demonstrated safety and efficacy in pediatric patients when a 2\% granules formulation is administered at 1 mg/kg. This study assessed safety, effectiveness, and pharmacokinetics of a higher dose (2 mg/kg) of baloxavir marboxil 2\% granules in pediatric patients weighing <20 kg. This multicenter, open-label, noncontrolled study was conducted at 15 sites in Japan (January 2019-March 2020; JapicCTI-194577). Patients aged <12 years with confirmed influenza received a single oral dose of baloxavir marboxil at 2 mg/kg if body weight was <10 kg or 20 mg if ≥ 10 to <20 kg. Safety, pharmacokinetics, effectiveness (time to illness alleviation [TTIA] of influenza; time to resolution of fever; virus titer), and polymerase acidic protein (PA) substituted viruses were assessed over 22 days. 45 patients, all aged ≤6 years, were enrolled. Adverse events were reported in 24 (53.3\%) patients, most commonly nasopharyngitis, diarrhea, and upper respiratory tract infection. Median (95\% confidence interval [CI]) TTIA was 37.8 (27.5-46.7) hours; median (95\% CI) time to resolution of fever was 22.0 (20.2-28.6) hours. A >4 log decrease in mean viral titer occurred at day 2 and a subsequent temporary 1-2 log increase in patients with influenza A(H3N2) and B. Treatment-emergent PA/I38X-substituted virus was detected in 16/39 (41.0\%) patients, but no prolonged TTIA or time to resolution of fever was associated with its presence. Baloxavir granules administered at 2 mg/kg in children <20 kg were well tolerated, with symptom alleviation similar to 1 mg/kg. [\hyperlink{Mavorixafor}{PMID: 34045119}, Takuhiro Sonoyama et al., 2021]

\hypertarget{pmid_24769325}{T}he safety, pharmacokinetics, and biological effect of plerixafor in children as part of a conditioning regimen for chemo-sensitization in allogeneic hematopoietic stem cell transplantation (HSCT) have not been studied. This is a phase I study of plerixafor designed to evaluate its tolerability at dose of .24 mg/kg given intravenously on day -4 (level 1); day -4 and day -3 (level 2); or day -4, day -3, and day -2 (level 3) in combination with fludarabine, thiotepa, melphalan, and rabbit antithymocytic globulin for a second allogeneic HSCT in children with refractory or relapsed leukemia. Immunophenotype analysis was performed on blood and bone marrow before and after plerixafor administration. Twelve patients were enrolled. Plerixafor at all 3 levels was well tolerated without dose-limiting toxicity. Transient gastrointestinal side effects of National Cancer Institute-grade 1 or 2 in severity were the most common adverse events. The area under the concentration-time curve increased proportionally to the dose level. Plerixafor clearance was higher in males and increased linearly with body weight and glomerular filtration rate. The clearance decreased and the elimination half-life increased significantly from dose level 1 to 3 (P < .001). Biologically, the proportion of CXCR4(+) blasts and lymphocytes both in the bone marrow and peripheral blood increased after plerixafor administration.  [\hyperlink{Mavorixafor}{PMID: 24769325}, Ashok Srinivasan et al., 2014] Tezacaftor/ivacaftor is a new treatment option in many regions for patients aged ≥12 years who are homozygous (F/F) or heterozygous for the F508del-CFTR mutation and a residual function (F/RF) mutation. This Phase 3, 2-part, open-label study evaluated the pharmacokinetics (PK), safety, tolerability, and efficacy of tezacaftor/ivacaftor in children aged 6 through 11 years with these mutations. Part A informed weight-based tezacaftor/ivacaftor dosages for part B. The primary objective of part B was to evaluate the safety and tolerability of tezacaftor/ivacaftor through 24 weeks; the secondary objective was to evaluate efficacy based on changes from baseline in percentage predicted forced expiratory volume in 1 s (ppFEV After PK analysis in part A, 70 children received ≥1 dose of tezacaftor/ivacaftor in part B; 67 children completed treatment. Exposures in children aged 6 through 11 years were within the target range for those observed in patients aged ≥12 years. The safety profile of tezacaftor/ivacaftor was generally similar to prior studies in patients aged ≥12 years. One child discontinued treatment for a serious adverse event of constipation. Tezacaftor/ivacaftor treatment improved sweat chloride levels and CFQ-R respiratory domain scores, mean ppFEV Tezacaftor/ivacaftor was generally safe and well tolerated, and improved CFTR function in children aged 6 through 11 years with CF with F/F and F/RF genotypes, supporting tezacaftor/ivacaftor use in this age group. NCT02953314. [\hyperlink{Mavorixafor}{PMID: 24769325}, Seth Walker et al., 2019]

\hypertarget{pmid_19273678}{T}wo multicenter, open-label, single-arm, two-phase studies evaluated single-dose pharmacokinetics and single- and multiple-dose safety of a pediatric oral famciclovir formulation (prodrug of penciclovir) in children aged 1 to 12 years with suspicion or evidence of herpes simplex virus (HSV) or varicella-zoster virus (VZV) infection. Pooled pharmacokinetic data were generated after single doses in 51 participants (approximately 12.5 mg/kg of body weight [BW] for children weighing < 40 kg and 500 mg for children weighing > or = 40 kg). The average systemic exposure to penciclovir was similar (6- to 12-year-olds) or slightly lower (1- to < 6-year-olds) than that in adults receiving a 500-mg dose of famciclovir (historical data). The apparent clearance of penciclovir increased with BW in a nonlinear manner, proportional to BW(0.696). An eight-step weight-based dosing regimen was developed to optimize exposure in smaller children and was used in the 7-day multiple-dose safety phases of both studies, which enrolled 100 patients with confirmed/suspected viral infections. Twenty-six of 47 (55.3\%) HSV-infected patients who received famciclovir twice a day and 24 of 53 (45.3\%) VZV-infected patients who received famciclovir three times a day experienced at least one adverse event. Most adverse events were gastrointestinal in nature. Exploratory analysis following 7-day famciclovir dosing regimen showed resolution of symptoms in most children with active HSV (19/21 [90.5\%]) or VZV disease (49/53 [92.5\%]). Famciclovir formulation (sprinkle capsules in OraSweet) was acceptable to participants/caregivers. In summary, we present a weight-adjusted dosing schedule for children that achieves systemic exposures similar to those for adults given the 500-mg dose. [\hyperlink{Mavorixafor}{PMID: 19273678}, X Sáez-Llorens et al., 2009]

\hypertarget{pmid_12353197}{V}alacyclovir was administered to 28 immunocompromised children (ages 5-12 years) to obtain preliminary pharmacokinetic and safety information. Patients were randomized to valacyclovir regimens of 250 mg (9.4-13.3 mg/kg) or 500 mg (13.9-27.0 mg/kg) twice daily or 500 mg (13.2-21.7 mg/kg) 3 times a day. Acyclovir pharmacokinetics were evaluated at steady state. Valacyclovir was rapidly absorbed and converted to acyclovir. Mean (+/-SD) acyclovir peak concentrations from 250 mg and 500 mg valacyclovir were 4.11+/-1.41 and 5.19+/-1.96 microg/mL, respectively. Corresponding single dose area-under-curve values were 12.14+/-6.60 and 14.49+/-4.69h microg/mL. By using historical data for intravenous acyclovir as reference, the overall estimate of acyclovir bioavailability from valacyclovir was 48\%, 2- to 4-fold greater than for oral acyclovir. In general, adverse events were not attributable to valacyclovir and were consistent with disease-related expectations and concomitant therapies. Dosage options for using valacyclovir in children are discussed. [\hyperlink{Mavorixafor}{PMID: 12353197}, David Nadal et al., 2002]

\hypertarget{pmid_15675977}{C}hildren may exhibit delayed emergence following maintenance of anesthesia with propofol or isoflurane. Desflurane is often used towards the end of procedures to facilitate emergence. This study evaluated the effect on middle cerebral artery blood flow velocity (Vmca) in anesthetized children when propofol or isoflurane was substituted with desflurane. Forty-two healthy children aged 1-6 years were enrolled. A standardized anesthetic induction was used. Anesthesia was maintained with remifentanil (0.5 microg.kg(-1) bolus followed by an infusion of 0.2 microg.kg(-1).min(-1)) and a randomly selected sequence of propofol/desflurane/propofol, desflurane/propofol/desflurane, isoflurane/desflurane/isoflurane or desflurane/isoflurane/desflurane. Propofol was administered to maintain a steady-state serum concentration of 3 microg.ml(-1). Desflurane and isoflurane were administered at age-corrected 1 MAC. Hemodynamic stability was maintained. Transcranial Doppler sonography was used to measure Vmca. Hemodynamic variables as well as Vmca were measured 30 min after skin incision and repeated 30 min after each change in anesthetic maintenance agent. The mean age and weight was 2.3 +/- 1.3 years and 13.0 +/- 3.7 kg, respectively. The Vmca (mean) increased by 35\% from 37.7 +/- 10.5 cm s(-1) to 57.8 +/- 14.6 cm s(-1) (P < 0.0001) when propofol was changed to desflurane but was unaffected when desflurane replaced isoflurane. When propofol is changed to desflurane, cerebral blood flow velocity increases significantly in normal children. This cerebral vasodilatory effect may have important implications in the neurosurgical setting. [\hyperlink{Mavorixafor}{PMID: 15675977}, J H Smith et al., 2005]

\hypertarget{pmid_12027844}{I}n children, onset time and duration of action of mivacurium are shorter than in adults. Some suggest that this is due to differences in plasma cholinesterase (pChe), whereas others indicate that there is no difference. The purpose of this study was to evaluate the pharmacodynamics and pharmacokinetics of mivacurium in phenotypically normal children aged 3-6 and 10-14 years old, respectively. Ten children aged 3-6 years and 10 children aged 10-14 years were studied during halothane anaesthesia. Before induction of anaesthesia, a blood sample was drawn to measure the pChe activity and phenotype. The neuromuscular block was monitored at the thumb using train-of-four (TOF) nerve stimulation every 12 s and mechanomyography. The times to different levels of neuromuscular recovery following mivacurium 0.2 mg/kg were recorded. The concentrations in venous blood of the three isomers and the metabolites of mivacurium were measured. No statistically significant difference was found in pChe activity or in the pharmacodynamics of mivacurium. The onset time was 1.4 min (0.8-1.9) median (range) and 1.3 min (1.1-1.9) and the time to first response to TOF nerve stimulation was 9.6 min (6.5-12.6) and 10.5 min (7.0-14.0) in young and older children, respectively. The pharmacokinetic data were too sparse to allow analysis of the two age groups separately (8 and 8 patients), hence the data were pooled. The median clearances of the cis-cis, the cis-trans, and the trans-trans isomer were 5.5, 51.0 and 30.5 ml/kg/min, respectively. Our data indicate that there are no major differences in pharmacodynamics or pharmacokinetics of mivacurium between young (3-6 years) and older (10-14 years) children. [\hyperlink{Mavorixafor}{PMID: 12027844}, D ØStergaard et al., 2002]

\hypertarget{pmid_20160046}{A} multicenter, open-label study evaluated the single-dose pharmacokinetics and safety of a pediatric oral famciclovir (prodrug of penciclovir) formulation in infants aged 1 to 12 months with suspicion or evidence of herpes simplex virus infection. Individualized single doses of famciclovir based on the infant's body weight ranged from 25 to 175 mg. Eighteen infants were enrolled (1 to <3 months old [n = 8], 3 to <6 months old [n = 5], and 6 to 12 months old [n = 5]). Seventeen infants were included in the pharmacokinetic analysis; one infant experienced immediate emesis and was excluded. Mean C(max) and AUC(0-6) values of penciclovir in infants <6 months of age were approximately 3- to 4-fold lower than those in the 6- to 12-month age group. Specifically, mean AUC(0-6) was 2.2 microg h/ml in infants aged 1 to <3 months, 3.2 microg h/ml in infants aged 3 to <6 months, and 8.8 microg h/ml in infants aged 6 to 12 months. These data suggested that the dose administered to infants <6 months was less than optimal. Eight (44.4\%) infants experienced at least one adverse event with gastrointestinal events reported most commonly. An updated pharmacokinetic analysis was conducted, which incorporated the data in infants from the present study and previously published data on children 1 to 12 years of age. An eight-step dosing regimen was derived that targeted exposure in infants and children 6 months to 12 years of age to match the penciclovir AUC seen in adults after a 500-mg dose of famciclovir. [\hyperlink{Mavorixafor}{PMID: 20160046}, Jeffrey Blumer et al., 2010]

\hypertarget{pmid_35190292}{T}wo previous Phase 3 studies ("parent studies") showed that tezacaftor/ivacaftor was generally safe and efficacious for up to 24 weeks in children 6 through 11 years of age with cystic fibrosis (CF) and F508del/F508del (F/F) or F508del/residual function (F/RF) genotypes. We assessed the safety and efficacy of tezacaftor/ivacaftor in an open-label, 96-week extension study. This was a Phase 3, 2-part, multicenter, open-label, extension study in children 6 through 11 years of age at treatment initiation (Study VX17-661-116; NCT03537651). The primary endpoint was safety and tolerability. Secondary endpoints were absolute change from baseline in lung clearance index One-hundred thirty children enrolled and received ≥ 1 dose of tezacaftor/ivacaftor; 109 completed treatment. Most (n = 129) had ≥ 1 treatment-emergent adverse event (TEAE), the majority of which were mild or moderate in severity and generally consistent with common manifestations of CF. Exposure-adjusted TEAE rates were similar to or lower than those in the parent studies. Five (3.8\%) had TEAEs leading to treatment discontinuation. Efficacy results from the parent studies were maintained, with improvements in lung function, SwCl concentration, CFQ‑R respiratory domain score, and BMI observed from parent study baseline to Week 96. Tezacaftor/ivacaftor is generally safe and well tolerated, and treatment effects are maintained for up to 120 weeks. These results support long-term use of tezacaftor/ivacaftor in children ≥ 6 years of age with CF and F/F or F/RF genotypes. [\hyperlink{Mavorixafor}{PMID: 35190292}, Gregory S Sawicki et al., 2022]

\hypertarget{pmid_17474953}{T}he aim of this study was to evaluate the safety and efficacy of a combination of propofol and remifentanil deep sedation in spontaneously breathing children less than 7 years of age undergoing upper and/or lower gastrointestinal endoscopy. The effect of propofol and remifentanil sedation was prospectively studied in 42 unpremedicated children undergoing gastrointestinal endoscopy. Anesthesia was induced with a combination of sevoflurane, nitrous oxide and oxygen. Anesthesia was maintained with an infusion of propofol (50-80 microg x kg(-1) x min(-1)) and remifentanil (0.1 microg x kg(-1) x min(-1)). Demographic data, heart rate, blood pressure, respiratory rate, and oxygen saturation were recorded every 5 min for each child. In addition, recovery and discharge times were recorded. All 42 procedures were completed with no complications. The combination of propofol and remifentanil resulted in a decrease in heart rate, blood pressure, and respiratory rate. There was no respiratory depression or oxygen desaturation in any child. A bolus of propofol (1 mg x kg(-1)) was necessary in one child for excessive movement. No patient experienced any side effects in the recovery period. The combination of propofol and remifentanil for sedation in children undergoing gastrointestinal endoscopy can be considered safe, effective and acceptable. [\hyperlink{Mavorixafor}{PMID: 17474953}, Ibrahim Abu-Shahwan et al., 2007]

\hypertarget{pmid_9713224}{T}he aim of this study was to evaluate the clinical use of mivacurium, a short-acting, non-depolarising muscle relaxant, in the paediatric population in Singapore. Twenty children between the ages of 2 and 12 years were given mivacurium to maintain neuromuscular blockade during nitrous oxide-halothane anaesthesia. Reversal from neuromuscular blockade was spontaneous. The onset, ease of intubation after different doses of mivacurium, and the ease of reversal were evaluated. Different intubating doses of mivacurium did not result in significantly different times of onset. The mean recovery index (25\% to 75\% recovery) was 4.1 minutes. There were no adverse reactions. Mivacurium provided rapid and efficacious onset of neuromuscular blockade in the local paediatric population. Rapid spontaneous recovery obviated the need for reversal agents. [\hyperlink{Mavorixafor}{PMID: 9713224}, H L Chee et al., 1998]

\hypertarget{pmid_22009006}{P}lerixafor has been recently approved by the European Medicines Agency for adult patients who have failed other mobilization strategies. Experience in children, however, is extremely limited. We describe the experience of the use of this drug in 8 children under a compassionate-use program in 3 Italian and 2 Spanish centers. Plerixafor was generally well tolerated; only 2 of 8 children reported adverse effects, and these were mild in intensity. Peripheral blood progenitor cell priming was improved with plerixafor in 6 of 8 patients. In the remaining 2 patients, the target CD34+ cell count was below the target of 2 × 10(6) cells/kg, although in these patients cell counts before collection were good enough for leukapheresis. Plerixafor, therefore seems to be safe and effective for peripheral blood progenitor cell mobilization in children. Adverse events were comparable with those described with filgrastim alone. [\hyperlink{Mavorixafor}{PMID: 22009006}, Julián Sevilla et al., 2012]

\hypertarget{pmid_14606377}{T}o assess the efficacy and safety of sevoflurane anesthesia in children during magnetic resonance imaging procedures. The patients were 105 ASA-I-II children, mean weight 13 +/- 10 Kg and mean age 2.9 years (range 1 day-10 years), twenty (20\%) of whom were under 3 months old. Induction was gradual with 6\% sevoflurane in a mixture of nitrous oxide and oxygen, followed by maintenance with 1-2\% sevoflurane in the same mixture through a face mask or nasal tubes while the patient breathed spontaneously. All procedures were performed satisfactorily. Ten minutes after anesthesia, 88\% of the patients were fully awake. None suffered prolonged sedation and no serious complications occurred during the study period. The most common side effects were transient decreases in oxygen saturation in 15 patients (14\%), although none reached the critical level (SpO2 < 90\%). Six of those patients were under 3 months old. After recovering from sedation, 13 patients (12\%) suffered transient episodes of excessive agitation, usually 5 minutes after awakening. Five patients (4.8\%) vomited in the recovery room. This study indicates that sevoflurane is safe and effective for sedating children, including newborn infants, who must undergo magnetic resonance imaging. [\hyperlink{Mavorixafor}{PMID: 14606377}, P Sanabria Carretero et al., 2002]

\hypertarget{pmid_32154021}{D}exrazoxane protects from lower-cumulative-dose doxorubicin cardiotoxicity, but the effect of dexrazoxane in children with sarcoma treated with higher-cumulative-dose doxorubicin is unknown. We evaluated children with osteosarcoma (OS) on two Children's Oncology Group trials with higher dose doxorubicin (375-600 mg/m All children had normal left ventricular (LV) systolic function as measured by LV fractional shortening and no heart failure. The end-diastolic septal thickness  Dexrazoxane prevented LV dysfunction and heart failure in children with OS receiving higher dose doxorubicin. However, LV structural changes were not fully prevented, especially in girls. As a result, hearts become abnormally small for body size, resulting in higher LV stress. Dexrazoxane did not increase the risk of SMN. Dexrazoxane should be used in this population, particularly for girls, to mitigate anthracycline-induced cardiotoxicity. ClinicalTrials.gov: NCT00003937 (P9754) registered 1 Nov 1999, and NCT00023998 (AOST0121) registered 13 Sept 2001. [\hyperlink{Mavorixafor}{PMID: 32154021}, Lisa M Kopp et al., 2019]

\section*{Zaleplon}
\subsection*{Result}
\subsubsection*{Answer}

Unknown

\subsubsection*{{Explanation}}
\hypertarget{Zaleplon}
A review of the available abstracts reveals the following regarding the safety of Zaleplon in children:

- Most abstracts focus on adult or elderly populations, with multiple randomized controlled trials and safety studies in adults and the elderly, but not in children.
- One abstract describes a case report of a 15-year-old female who ingested a large overdose of Zaleplon (60 mg, 1.2 mg/kg) in a suicide attempt. The clinical course was mild, with symptoms such as somnolence, blurred speech, ataxia, tachycardia, and hypokalemia, all resolving with symptomatic treatment. However, this is a single case report of overdose, not a targeted safety study for therapeutic use in children, and does not establish safety for use in the pediatric population [\hyperlink{pmid_17724894}{PMID: 17724894}, Jacek Sein Anand et al., 2007].
- Another abstract explicitly states: "Zaleplon has not been studied in pregnant women or children" [\hyperlink{pmid_11117652}{PMID: 11117652}, K W Weitzel et al., 2000].
- No abstracts report a targeted clinical trial or systematic safety study of Zaleplon in children (e.g., ages 0-18) for any indication.
- All other abstracts either focus on adults, elderly, or animal models, or do not specify age, and thus cannot be used to determine safety in children.

Therefore, based on the abstracts available, there is no evidence from targeted studies affirming the safety or unsafety of Zaleplon in children. The safety of Zaleplon in children is unknown.

\subsection*{Abstracts}
\hypertarget{pmid_33320969}{Z}aleplon (ZP) is a sedative and hypnotic drug used for the treatment of insomnia. Despite its potent anticonvulsant activity, ZP is not commonly used for the treatment of convulsion since ZP is characterized by its low oral bioavailability as a result of poor solubility and extensive liver metabolism. The following study aimed to formulate specifically controlled release nano-vehicles for oral and parenteral delivery of ZP to enhance its oral bioavailability and biological activity. A modified single emulsification-solvent evaporation method of sonication force was adopted to optimize the inclusion of ZP into biodegradable nanoparticles (NPs) using poly (dl-lactic-co-glycolic acid) (PLGA). The impacts of various formulation variables on the physicochemical characteristics of the ZP-PLGA-NPs and drug release profiles were investigated. Pharmacokinetics and pharmacological activity of ZP-PLGA-NPs were studied using experimental animals and were compared with generic ZP tablets. Assessment of gamma-aminobutyric acid (GABA) level in plasma after oral administration was conducted using enzyme-linked immunosorbent assay. The maximal electroshock-induced seizures model evaluated anticonvulsant activity after the parenteral administration of ZP-loaded NPs. The prepared ZP-PLGA NPs were negatively charged spherical particles with an average size of 120-300 nm. Optimized ZP-PLGA NPs showed higher plasma GABA levels, longer sedative, hypnotic effects, and a 3.42-fold augmentation in oral drug bioavailability in comparison to ZP-marketed products. Moreover, parenteral administration of ZP-NPs showed higher anticonvulsant activity compared to free drug. Oral administration of ZP-PLGA NPs achieved a significant improvement in the drug bioavailability, and parenteral administration showed a pronounced anticonvulsant activity. [\hyperlink{Zaleplon}{PMID: 33320969}, Yusuf A Haggag et al., 2021]

\hypertarget{pmid_10485636}{Z}aleplon is a short-acting pyrazolopyrimidine hypnotic with a rapid onset of action. This multicenter study compared the efficacy and safety of 3 doses of zaleplon with those of placebo in outpatients with DSM-III-R insomnia. Zolpidem, 10 mg, was used as an active comparator. After a 7-night placebo (baseline) period, 615 adult patients were randomly assigned to receive, in double-blind fashion, I of 5 treatments (zaleplon, 5, 10, or 20 mg; zolpidem, 10 mg; or placebo) for 28 nights, followed by placebo treatment for 3 nights. Sleep latency, sleep maintenance, and sleep quality were determined from sleep questionnaires that patients completed each morning. The occurrence of rebound insomnia and withdrawal effects on discontinuation of treatment was also assessed. All levels of significance were p < or = .05. Median sleep latency was significantly lower with zaleplon, 10 and 20 mg, than with placebo during all 4 weeks of treatment and with zaleplon, 5 mg, for the first 3 weeks. Zaleplon, 20 mg, also significantly increased sleep duration compared with placebo in all but week 3 of the study. There was no evidence of rebound insomnia or withdrawal symptoms after discontinuation of 4 weeks of zaleplon treatment. Zolpidem, 10 mg, significantly decreased sleep latency, increased sleep duration, and improved sleep quality at most timepoints compared with placebo; however, after discontinuation of zolpidem treatment, the incidence of withdrawal symptoms was significantly greater than that with placebo and there was an indication of significant rebound insomnia for some patients in the zolpidem group compared with those in the placebo group. The frequency of adverse events in the active treatment groups did not differ significantly from that in the placebo group. Zaleplon is effective in the treatment of insomnia. In addition, zaleplon appears to provide a favorable safety profile, as indicated by the absence of rebound insomnia and withdrawal symptoms once treatment was discontinued. [\hyperlink{Zaleplon}{PMID: 10485636}, R Elie et al., 1999]

\hypertarget{pmid_29189154}{Z}aleplon is a pyrazolopyrimidin derivative hypnotic drug indicated for the short-term management of insomnia. Zaleplon belongs to Class II drugs, according to the biopharmaceutical classification system (BCS), showing poor solubility and high permeability. It undergoes extensive first-pass hepatic metabolism after oral absorption, with only 30\% of Zaleplon being systemically available. It is available in tablet form which is unable to overcome the previous problems. The aim of this study is to enhance solubility and bioavailability via utilizing nanotechnology in the formulation of intranasal Zaleplon nano-emulsion (ZP-NE) to bypass the barriers and deliver an effective therapy to the brain. Screening studies were carried out wherein the solubility of zaleplon in various oils, surfactants( S) and co-surfactants(CoS) were estimated. Pseudo-ternary phase diagrams were constructed and various nano-emulsion formulations were prepared. These formulations were subjected to thermodynamic stability, in-vitro characterization, histopathological studies and assessment of the gamma aminobutyric acid (GABA) level in plasma and brain in rabbits compared to the market product (Sleep aid®). Stable NEs were successfully developed with a particle size range of 44.6±3.4 to 136.9±1.6 nm. A NE composed of 10\% Miglyol® 812, 40\% Cremophor® RH40 40\%Transcutol® HP and 10\% water successfully enhanced the bioavailability and brain targeting in the rabbits, showing a three to four folds increase than the marketed product. [\hyperlink{Zaleplon}{PMID: 29189154}, Eman Abd-Elrasheed et al., 2018]

\hypertarget{pmid_19874656}{Z}aleplon (Sonata) is a sedative hypnotic prescription medication used for the short-term treatment of insomnia. Although Zaleplon was approved by the FDA in 1999, there has been limited postmortem information about the drug cited in the toxicology literature. Zaleplon was separated from postmortem biological specimens utilizing liquid-liquid extraction coupled with a solid-phase extraction technique, and detection was accomplished by a gas chromatography-electron capture detector. The method was linear from 5.0 to 150 ng/mL with the limit of quantitation and detection determined to be 3.0 and 0.50 ng/mL, respectively. The postmortem tissue distribution of zaleplon in seven cases was as follows: 6.1-1490 ng/mL central blood (seven cases), < 3.0-503 ng/mL femoral blood (five cases), 108 ng/mL harvest blood (one case), 343-679 ng/g liver (four cases), 950 ng/g spleen (one case), < 3.0-85 ng/mL bile (three cases), 3.8-106 ng/mL urine (four cases), < 3.0-486 ng/mL vitreous humor (five cases), and 0.005-3.4 mg total gastric contents (four cases). A validated method for the analysis of zaleplon and postmortem concentrations of autopsy specimens are reported to aid the forensic toxicologist with interpretation of future casework. [\hyperlink{Zaleplon}{PMID: 19874656}, Daniel T Anderson et al., 2009]

\hypertarget{pmid_23616704}{Z}aleplon is a pyrazolopyrimidine hypnotic used for the treatment of insomnia. Zaleplon binds preferentially at the α1β2γ2 subunit of gamma aminobutyric acid type A (GABAA) receptors in the central nervous system, and has a half-life of about one hour. Efficacy studies show that zaleplon is a suitable hypnotic for sleep initiation purposes. However, because of its short half-life, zaleplon is less effective in sleep maintenance when compared with other hypnotics. Nevertheless, zaleplon does increase total sleep time. No rebound effects are observed after treatment discontinuation. The use of zaleplon is relatively safe. Adverse effects are mild and of short duration. No important interactions have been reported, and there is no evidence of abuse potential. Relative to benzodiazepine hypnotics, the biggest advantage of zaleplon is that current evidence suggests it does not produce residual next-day effects. As early as four hours after intake of zaleplon, no effects on cognitive, memory, psychomotor performance, and the ability to drive a car have been reported. Future studies should confirm these findings, and comparisons with new nonbenzodiazepine hypnotics should determine the importance of zaleplon in the future treatment of insomnia. [\hyperlink{Zaleplon}{PMID: 23616704}, Marieke M Ebbens et al., 2010]

\hypertarget{pmid_10960882}{I}nsomnia is a frequent complaint in the elderly population. Hypnotic agents, including benzodiazepines, with longer pharmacological half-lives have been associated with side effects, including residual sedation, memory impairment, and discontinuation effects. Zaleplon is a short-acting (elimination half-life of 1 hour), non-benzodiazepine hypnotic that acts on the benzodiazepine type 1 site of the gamma-aminobutyric acid type A (GABA(A)) receptor complex. The pharmacology and pharmacokinetics of Zaleplon suggest a safety profile that is improved over other hypnotics. The objective of this placebo-controlled study was to evaluate the efficacy and safety of Zaleplon (5 and 10 mg) in elderly (> or =65 years) outpatients with primary insomnia. This was a multicenter, double-blind, randomised, placebo-controlled 2-week outpatient study. Postsleep questionnaires were used to record subjective sleep variables: sleep latency, sleep duration, number of awakenings, and sleep quality. Zaleplon significantly reduced subjective sleep latency during both weeks of the study with both 5- and 10-mg doses. Subjective sleep quality was improved for significantly more patients treated with zaleplon 10 mg than those treated with placebo during both weeks of treatment. There was a weak indication of rebound insomnia after discontinuation of treatment with the 10-mg dose, but no significant difference in common treatment-emergent adverse events across treatment groups. Zaleplon is an effective and safe hypnotic for the treatment of insomnia in the elderly. [\hyperlink{Zaleplon}{PMID: 10960882}, J Hedner et al., 2000]

\hypertarget{pmid_10831020}{T}wenty-four healthy male and female subjects, who participated in this randomized, double-blind, crossover study, received single nighttime doses of zaleplon 10 mg (therapeutic dose), zaleplon 20 mg, zolpidem 10 mg (therapeutic dose), zolpidem 20 mg, triazolam 0.25 mg (positive control), and placebo. Subjective behavioral ratings and psychomotor tests were completed before and 1.25 and 8.25 hours after administration of the study drug. The Immediate and Delayed Word Recall tests and the Digit Span Test were used to assess memory. The Digit-Symbol Substitution Test, Paired Associates Learning Test, and Divided Attention Test were used to assess other cognitive skills. Zaleplon 10 mg did not produce any significant changes in memory or learning compared with placebo. All other active treatments, including zolpidem 10 mg, caused psychomotor impairment at the 1.25-hour test battery. Zolpidem 20 mg (twice the therapeutic dose) produced more psychomotor impairment at the 1.25-hour assessment than did any of the other active treatments, including zaleplon 20 mg. At the 8.25-hour time point, test scores for subjects who received zaleplon 10 mg and 20 mg did not differ from the test scores for those who received placebo. However, cognitive impairment persisted up to the 8.25-hour observation for subjects who were administered triazolam 0.25 mg and zolpidem 20 mg. Adverse events associated with the use of zaleplon were transient and mild-to-moderate in severity. Overall, this study shows that zaleplon is a safe hypnotic that does not affect memory, learning, or psychomotor skills associated with vigilance. [\hyperlink{Zaleplon}{PMID: 10831020}, S M Troy et al., 2000]

\hypertarget{pmid_17724894}{T}here has been little data in the medical literature about intoxication with a new hypnotic agent zaleplon. The zaleplon, chemically N-[3-(3-cyanopyrazolo[1,5-a]pyrimidin-7-yl)phenyl]-N-ethylacetamid, is a selective agonist of the benzodiazepine omega 1 receptor subtype. The case of a 15-year-old female who eat 60 mg of zaleplon (1.2 mg/kg) because of suicidal attempt was described. At the admission to the hospital the somnolence, blurred speech, slowdown, ataxia, tachycardia and hypokalaemia were observed. The child was treated symptomatically, and discharged from the hospital for further psychologic treatment after 36 hours. Acute intoxication with zaleplon had mild clinical course. The signs of intoxications were drowsiness, blurred speech, ataxia, tachycardia, dizziness, confusion and vomiting. The described case required only symptomatic treatment. [\hyperlink{Zaleplon}{PMID: 17724894}, Jacek Sein Anand et al., 2007]

\hypertarget{pmid_11219331}{Z}aleplon is a non-benzodiazepine sleep medication that shows efficacy as a sleep inducer comparable to that of other hypnotics but with significantly fewer residual effects. In addition, evaluations of psychomotor or memory function at zaleplon peak plasma levels show much less impairment than noted with other hypnotics, suggesting an improved benefit-to-risk profile for zaleplon compared with older available agents. Thus, zaleplon can be used to treat symptoms of insomnia when they occur without the concern of next-day psychomotor or memory impairment, whether administered at bedtime or later during the night. Such an approach permits physicians to reformulate their strategies for safe and effective management of sleeplessness. [\hyperlink{Zaleplon}{PMID: 11219331}, R M Mangano et al., 2001]

\hypertarget{pmid_15014684}{B}ACKGROUND: Insomnia is a very common symptom, particularly in the elderly. Thus, all hypnotic medications should be carefully evaluated in the elderly population. Zaleplon, a new nonbenzodiazepine hypnotic with a short elimination half-life (approximately 1 hour), was evaluated in the current study. METHOD: This multicenter, randomized, placebo-controlled outpatient study evaluated the efficacy and safety of zaleplon, 5 and 10 mg, in elderly patients with insomnia (as defined by DSM-IV); zolpidem, 5 mg, was the active comparator. Sleep was assessed in 549 elderly patients (>/= 65 years old) by using morning questionnaires completed after each of 7 baseline nights during which placebo was given, 14 nights of double-blind treatment, and 7 nights of placebo after discontinuation of active treatment. RESULTS: Zaleplon, 10 mg, and zolpidem, 5 mg, significantly reduced sleep latency during both weeks of the study. Zaleplon, 5 mg, reduced sleep latency only during week 2. Sleep duration was increased with zolpidem, 5 mg, during weeks 1 and 2 and with zaleplon, 10 mg, during week 1. No clinically significant rebound insomnia was observed after discontinuation of treatment with zaleplon, whereas evidence of rebound effects was seen with zolpidem. There was no significant difference between either zaleplon dose and placebo in the frequency of any central nervous system adverse events. CONCLUSION: Zaleplon is effective in reducing latency to sleep without evidence of undesired effects in elderly patients with insomnia. [\hyperlink{Zaleplon}{PMID: 15014684}, Sonia Ancoli-Israel et al., 1999]

\hypertarget{pmid_10870872}{T}he efficacy and safety of three doses of zaleplon, a novel non-benzodiazepine hypnotic, were compared with those of placebo in outpatients with insomnia in this 4-week study, using zolpidem 10 mg as active comparator. Postsleep questionnaires were used to determine treatment effects on the patient's perception of sleep, as well as any development of pharmacological tolerance during therapy or rebound insomnia or withdrawal symptoms upon discontinuation of therapy. During week 1, sleep latency was significantly shorter with zaleplon 5, 10, and 20 mg compared to placebo. The significant decrease in sleep latency persisted through week 4 with zaleplon 20 mg, and was again evident with zaleplon 10 mg at week 3. Zaleplon 20 mg also had significant effects on sleep duration, number of awakenings, and sleep quality compared to placebo. No pharmacological tolerance developed during treatment with zaleplon and there were no indications of rebound insomnia or withdrawal symptoms after treatment discontinuation. Zolpidem 10 mg had significant effects on sleep latency, sleep duration, and sleep quality compared to placebo. However, a significantly greater incidence of withdrawal symptoms and a suggestion of sleep difficulty after treatment discontinuation (rebound insomnia) for all sleep measures was seen with zolpidem compared to placebo. There was no significant difference in the frequency of adverse events with active treatment compared to placebo. These results show that zaleplon provides effective treatment of insomnia with a favourable safety profile. [\hyperlink{Zaleplon}{PMID: 10870872}, J Fry et al., 2000]

\hypertarget{pmid_11192136}{T}his study compared the pharmacokinetics, pharmacodynamics, and pharmacokinetic/pharmacodynamic (PK/PD) profile of zaleplon, a new pyrazolopyrimidine hypnotic, with those of zolpidem and placebo. This was a double-blind, 5-period crossover study in which healthy volunteers with no history of sleeping disorder were randomized to 10- or 20-mg oral doses of zaleplon, 10- or 20-mg oral doses of zolpidem, or placebo. The pharmacokinetic characteristics of the active drugs were estimated using a noncompartmental method and NONMEM. Pharmacodynamic characteristics were determined using psychophysical tests, including measures of sedation, mood, mental and motor speed, and recent and remote recall. Results of these tests were used to compare the drugs' relative PK/PD profiles. Ten healthy male and female volunteers, aged 23 to 31 years, took part in the study. The apparent elimination half-life of zaleplon (60.1+/-8.9 min) was significantly shorter than that of zolpidem (124.5+/-37.9 min) (P < 0.001). Zaleplon produced less sedation than zolpidem at the 2 doses studied (P < 0.001). The sedation scores of the zaleplon groups returned to baseline in less time than those of the zolpidem groups (4 vs 8 hours; P < 0.05). Zaleplon had no effect on recent or remote recall, whereas zolpidem had a significant effect on both measures (P < 0.05). In this study in 10 young, healthy volunteers, zaleplon was eliminated more rapidly, produced no memory loss, and caused less sedation than zolpidem at the same doses. [\hyperlink{Zaleplon}{PMID: 11192136}, D Drover et al., 2000]

\hypertarget{pmid_11117652}{I}nsomnia is the subjective complaint of poor sleep or an inadequate amount of sleep that adversely affects daily functioning. For the past 4 decades, treatment of insomnia has shifted away from the use of barbiturates toward the use of hypnotic agents of the benzodiazepine class. However, problems associated with the latter (eg, next-day sedation, rebound insomnia, dependence, and tolerance) have prompted development of other agents. This review describes the recently approved nonbenzodiazepine agent, zaleplon. Studies of zaleplon were identified through a search of English-language articles listed in MEDLINE and International Pharmaceutical Abstracts, with no limitation on year. These were supplemented by educational materials from conferences. The efficacy and tolerability of zaleplon have been documented in the literature. Zaleplon has been shown to improve sleep variables in comparison with placebo. Like most hypnotic agents, zaleplon can be used for problems of sleep initiation at the beginning of the night, but its short duration of clinical effect may also allow patients to take it later in the night without residual effects the next morning. Zaleplon can be taken < or = 2 hours before awakening without "hangover" effects. It is generally well tolerated, with headache being the most commonly reported adverse event in clinical trials (15\%-18\%). Compared with flurazepam, a long-acting benzodiazepine sedative-hypnotic agent, zaleplon causes significantly less psychomotor and cognitive impairment (P < 0.001). Zaleplon has not been studied in pregnant women or children. The dose of zaleplon should be individualized; the recommended daily dose for most adults is 10 mg. Insomnia has a substantial impact on daily functioning. If pharmacologic treatment is indicated for insomnia, the choice of an agent should be guided by individual patient characteristics. [\hyperlink{Zaleplon}{PMID: 11117652}, K W Weitzel et al., 2000]

\hypertarget{pmid_23257756}{E}fficacy and safety of sertraline (zoloft) have been assessed in 26 patients, aged from 7 to 15 years, with depressive states of different severity and psychopathological structure which are combined with obsessive-compulsive symptoms. Changes in patient's status are analyzed using psychometric scales during 6 weeks of treatment. It has been concluded that Zoloft is effective and safe drug for the treatment of mild to moderate depressive disorders concomitant to obsessive-compulsive disorders in children age. [\hyperlink{Zaleplon}{PMID: 23257756}, A V Goriunov et al., 2012]

\hypertarget{pmid_9579287}{S}ertraline (Zoloft) is a selective serotonin reuptake inhibitor that is commonly used in adults in the treatment of mood and anxiety disorders. Whereas it also is used to treat these illnesses in children, it is not currently approved by the Food and Drug Administration for use in this population. Sertraline use has been increasing secondary to its efficacy and its more tolerable side effect profile than the tricyclic antidepressants. It is also much safer in overdose than the tricyclic antidepressants. Although there have been numerous reports of sertraline overdose in adults, reports in the pediatric population are much less common. We review the literature regarding sertraline overdose in children, describe a case of sertraline ingestion in a 22-month-old infant, and discuss the treatment of such an overdose. [\hyperlink{Zaleplon}{PMID: 9579287}, G Catalano et al., ]

\hypertarget{pmid_14736130}{Z}aleplon appears to be a prime candidate for assisting individuals in obtaining sleep in situations not conducive to rest (i.e., a short period during the day). However, should an early unexpected awakening and return to duty be required, the effect on performance is not known. Zaleplon (10 mg) would negatively affect human performance for some duration, compared with placebo, after a sudden awakening from a short period (1 h) of daytime sleep. There were 16 participants, 8 men and 8 women, who volunteered to participate in this study. The study was conducted using a counterbalanced, double-blind, repeated measures design. At 1 h prior to drug administration, and at each of 7 h postdrug, performance measures (cognition, memory, balance, and strength) and subjective symptom reports were recorded. Zaleplon had a statistically significant (p < 0.05) negative impact on balance through the first 2 h postdose when compared with placebo. In addition, symptoms related to "drowsiness" were statistically more prevalent under zaleplon than under placebo through the first 3 h postdrug. Of the eight measures of cognitive performance, six were significantly negatively impacted in the zaleplon condition through 2 h postdose when compared with placebo, with one remaining significantly degraded through 3 h postdose. Zaleplon also had a significantly negative impact on memory at 1 h and 4 h postdose. Zaleplon (10 mg), when used as a daytime sleep aid, causes drowsiness (and related symptoms) up to 3 h postdose, and may impact task performance, especially more complex tasks, for at least 2-3 h postdose. [\hyperlink{Zaleplon}{PMID: 14736130}, Jeffrey N Whitmore et al., 2004]

\hypertarget{pmid_15716214}{I}nsomnia is a common problem that increases with age and can last months to years. While substantial data establish the efficacy and safety of short-acting hypnotic therapy for the management of short-term insomnia using benzodiazepines receptor agonists (BzRAs), there are few studies on the continued efficacy and safety of these drugs when used for sustained periods. This paper reports the results of a 1-year open-label extension phases of two randomized, double-blind trials of zaleplon. In the open-label phase, older patients self-administered zaleplon nightly from 6 to 12 months and were then followed through a 7-day single-blind placebo-controlled run-out period. The safety profile in this population of older adults was similar to that observed in a short-term trial of an equivalent population. The data also suggested that long-term therapy produced and maintained statistically significant improvement in time to persistent sleep onset, duration of sleep and number of nocturnal awakenings (P<0.001 for each variable) for treatment durations of up to 12 months. Discontinuation was not associated with rebound insomnia. The open-label trial of long-term hypnotic therapy with zaleplon 5 and 10 mg suggests that they are safe and effective for the treatment of insomnia in older patients. Placebo-controlled, double-blind trials are needed in zaleplon and other BzRAs to confirm these results. [\hyperlink{Zaleplon}{PMID: 15716214}, Sonia Ancoli-Israel et al., 2005]

\hypertarget{pmid_11552629}{T}he aim of the study was to research the efficiency of sertraline (zoloft) in depressions, anxious states and obsessive-compulsive disorders. Diagnosis of the mental disorders was carried out according to ICD-10. 72 children (59 boys, 13 girls) aged 6-18 years were treated. There were 32 inpatients and 40 outpatients. Therapy with sertraline was performed during 8 weeks with a gradual increase (titration) and individual selection of the dose from 12.5 to 100 mg/day. During the therapy clinical observation was combined with the patients' examination using Hamilton Depression Scale and Hamilton Anxiety Scale (HAM-D and HAM-A), and a Clinical Global Impression Scale (CGI). It was established that sertraline was very effective and safe drug in children (it has no influence on cognitive functions, has neither myorelaxing or sedative effects). Activity of this drug is characterized by quick manifestation of thymoanaleptic and anxiolytic effects. It mild depressive states 50 mg/day is a significant dose; in more severe depressions and obsessive-compulsive disorders the dose in juveniles was to 100 mg, the duration of the therapy was more than 2 months. [\hyperlink{Zaleplon}{PMID: 11552629}, V M Voloshina et al., 2001]

\hypertarget{pmid_10392321}{F}ive lactating mothers were administered the therapeutic dose of zaleplon (10 mg) orally in an open-label, single-dose, pharmacokinetic study. Plasma and breast milk were sampled through 8 hours after dose administration for subsequent determinations of zaleplon and its major, though inactive, plasma metabolite 5-oxo-zaleplon. Zaleplon concentrations peaked in plasma and milk approximately 1 hour after dosing and then disappeared rapidly. The mean terminal half-life was slightly greater than 1 hour. Milk concentrations "mirrored" plasma concentrations closely with no discernible delay between peak times. The average milk-to-plasma (M/P) concentration ratio for zaleplon was approximately 0.50 over the time course. 5-oxo-zaleplon was undetectable in all but one milk sample. The maximum exposure of an infant to zaleplon during a feeding at peak milk concentrations was estimated to range from 1.28 micrograms to 1.66 micrograms, corresponding to 0.013\% to 0.017\% of the maternal dose or 0.320 microgram/kg to 0.415 microgram/kg for a 4 kg infant. The results indicate that zaleplon taken by a nursing mother is transferred through breast milk to her infant in very small quantities that are unlikely to be clinically important. [\hyperlink{Zaleplon}{PMID: 10392321}, M Darwish et al., 1999]

\hypertarget{pmid_2816015}{Z}aditen combined with local therapy and a rational hygienic and dietetic regimen has been administered to 158 children suffering from neurodermatitis (n = 142), eczema (n = 13), and strophulus (n = 3). The treatment has been effective in 148 (93.7\%) children. Clinical remission has been achieved in 35 (22.2\%) children, a considerable improvement in 87 (55.0\%), and an improvement in 26 (16.5\%). In 10 (6.3\%) children no improvement has been observed. The improvement has been developing within the first fortnight of the treatment, but the drug has been continued for another week or two to stabilize the therapeutic effect. A follow-up of 102 children, effectively treated with zaditen, for 1-12 mos has shown relapses in 72 (70.6\%) children, of these 7 relapses of medium severity and 1 severe relapse. Zaditen has been well tolerated by the children. [\hyperlink{Zaleplon}{PMID: 2816015}, M E Lipets et al., 1989]

\hypertarget{pmid_10548901}{Z}aleplon (Sonata) is an original hypnotic derived from the pyrazolopyrimidine with a full agonistic activity on central benzodiazepine receptors B21 type. Zaleplon is characterized by an extremely short half-life (about 1 hour). At the 10 mg dose, it is an affective sleep inducer with limited risks of disturbances in morning performance. It is particularly suitable for the treatment of initial insomnia when the prescription of an hypnotic is justified. [\hyperlink{Zaleplon}{PMID: 10548901}, M Ansseau et al., 1999]

\hypertarget{pmid_29184807}{Z}oledronic acid (ZA), a highly potent intravenous bisphosphonate (BP), has been increasingly used in children with primary and secondary osteoporosis due to its convenience of shorter infusion time and less frequent dosing compared to pamidronate. Many studies have also demonstrated beneficial effects of ZA in other conditions such as hypercalcemia of malignancy, fibrous dysplasia (FD), chemotherapy-related osteonecrosis (ON) and metastatic bone disease. This review summarizes pharmacologic properties, mechanism of action, dosing regimen, and therapeutic outcomes of ZA in a variety of metabolic bone disorders in children. Several potential novel uses of ZA are also discussed. Safety concerns and adverse effects are also highlighted. [\hyperlink{Zaleplon}{PMID: 29184807}, Sasigarn A Bowden et al., 2017]

\hypertarget{pmid_10510148}{T}o compare the duration of the residual hypnotic and sedative effects of zaleplon with those of zolpidem and placebo following nocturnal administration at various times before morning awakening. Zaleplon 10 mg, zolpidem 10 mg, or placebo was administered double-blind to 36 healthy subjects under standardized conditions in a six-period, incomplete-block, crossover study. Subjects were gently awakened and given medication at predetermined times 5, 4, 3, or 2 h before morning awakening, which occurred 8 h after bedtime. When the subjects awoke in the morning, a battery of subjective and objective assessments of residual effects of hypnotics was administered. No residual effects were demonstrated after zaleplon 10 mg, when administered as little as 2 h before waking, on either subjective or objective assessments, whereas zolpidem 10 mg showed significant residual effects on DSST and memory (immediate and delayed free recall) after administration up to 5 h before waking and choice reaction time, critical flicker fusion threshold and Sternberg memory scanning after administration up to 4 h before waking. Residual effects of zolpidem were apparent in all objective and subjective measurements when the drug was administered later in the night. The present results demonstrate that zaleplon at the dose of 10 mg is free of residual hypnotic or sedative effects when administered nocturnally as little as 2 h before waking in normal subjects. In contrast, residual effects of zolpidem are still apparent on objective assessments up to 5 h after nocturnal administration, longer than has been reported from studies involving daytime administration. [\hyperlink{Zaleplon}{PMID: 10510148}, P Danjou et al., 1999]

\hypertarget{pmid_11965211}{I}nsomnia is the most frequently reported sleep symptom, severely affecting up to 15\% of the US population. The need to effectively treat this disorder is underscored by the significant adverse consequences on the productivity, safety, overall health, and quality of life of the affected individual. Pharmacologic intervention has traditionally involved the use of benzodiazepine receptor agonists (BzRAs), for which efficacy and general safety have been established. The purpose of this paper is to examine the potentially unique role of zaleplon in the treatment of insomnia. The clinical experience of the authors was critically applied to peer-reviewed published papers or abstracts regarding zaleplon, which were identified via MEDLINE (1995-September 2000). Adverse effects, usually related to residual sedation, impose limits on the use of older BzRAs and have prompted the development of new sleep medications with advantageous adverse event profiles. Zaleplon demonstrates a very rapid onset and offset of effect that permits symptomatic rather than prophylactic administration, resulting in comparable efficacy and reduced risk of the adverse effects associated with longer half-life agents. The characteristics of zaleplon may translate into distinct and significant clinical advances in the treatment of insomnia. [\hyperlink{Zaleplon}{PMID: 11965211}, Gary S Richardson et al., 2002]

\hypertarget{pmid_15014616}{B}ACKGROUND: Insomnia is a prevalent medical disorder that has significant effects on occupational performance, health, and quality of life. Insomnia places an enormous burden on society through increased visits to physicians, loss of productivity in the workplace, and an increased rate of accidents. An estimated sum of \$100 million is spent each year on direct treatment of unresolved insomnia. Physicians need to initiate early effective treatment to prevent development of chronic insomnia and its associated morbidity. Institution of good sleep hygiene practices may be useful in some patients but may not be adequate for resolution of all sleep problems. Behavioral treatments, while effective and durable, are time consuming and not widely utilized in clinical practice. Pharmacotherapy includes benzodiazepine hypnotics, but concerns regarding adverse effects (e.g., residual sedation) prompted the search for safer options. DATA SOURCES: Published and presented studies containing clinical data on zaleplon, a new nonbenzodiazepine sleep medication, were identified via MEDLINE, Current Contents (ISI database), bibliographic reviews, and consultation with sleep specialists. RESULTS: Zaleplon effectively shortens sleep onset time and improves the quality of sleep in patients with insomnia. Whether administered at bedtime or later at night, zaleplon is devoid of residual sedative effects that impair next-day functioning. Follow-up studies evaluating the long-term efficacy and safety of zaleplon showed that decreased time to sleep onset was maintained during therapy lasting up to 52 weeks, without a withdrawal syndrome after discontinuation. CONCLUSION: Insomnia is recurrent and unpredictable in nature. Despite the long-term morbidity of this sleep disorder, research evidence and practice guidelines have not explored long-term use of hypnotics. Many patients could benefit from long-term drug therapy with a sleep medication that is devoid of residual effects and can be taken at bedtime or later as symptoms occur, rather than nightly in anticipation of a sleep problem. [\hyperlink{Zaleplon}{PMID: 15014616}, Gary K. Zammit et al., 2001]

\section*{Sertraline Hydrochloride}
\subsection*{Result}
\subsubsection*{Answer}

Ages 6–12: Yes  
Ages 13–18: Yes  
Ages under 6: Unknown  

\subsubsection*{{Explanation}}
\hypertarget{Sertraline Hydrochloride}
A review of the available abstracts reveals several targeted studies evaluating the safety of sertraline hydrochloride in children and adolescents across various psychiatric conditions. Below is a summary by age range and indication:

Ages 6–12 (Children) and 13–18 (Adolescents):

- Major Depressive Disorder (MDD) and Obsessive-Compulsive Disorder (OCD): 
  - Multiple studies, including open-label and extension studies, have specifically evaluated the safety and tolerability of sertraline in children (6–12) and adolescents (13–18) with MDD and/or OCD. These studies report that sertraline was generally well tolerated, with adverse events similar to those seen in adults, and no significant safety concerns over periods up to 24 or 52 weeks. No discontinuations due to changes in vital signs, laboratory values, or ECGs were reported, and only a small percentage discontinued due to adverse events [\hyperlink{pmid_9549959}{PMID: 9549959}, J Alderman et al., 1998; \hyperlink{pmid_11589530}{PMID: 11589530}, E H Cook et al., 2001; \hyperlink{pmid_16553532}{PMID: 16553532}, Moira Rynn et al.; \hyperlink{pmid_16553533}{PMID: 16553533}, Jeffrey Alderman et al.]. 
  - One study in children with depression, anxiety, and OCD (ages 6–18) also found sertraline to be "very effective and safe," with no influence on cognitive functions and no myorelaxing or sedative effects [\hyperlink{pmid_11552629}{PMID: 11552629}, V M Voloshina et al., 2001].

- Anxiety Disorders (including Social Anxiety Disorder):
  - An open-label study in children with social anxiety disorder (n=14, ages not specified but described as "children") found sertraline to be generally well tolerated over 8 weeks, with significant improvement in symptoms and no major safety concerns. The authors recommend further controlled studies [\hyperlink{pmid_11349701}{PMID: 11349701}, S N Compton et al., 2001].
  - Another study in children and adolescents (8–17 years) with various anxiety disorders found no negative effects on attentional performance and only minor cognitive side effects [\hyperlink{pmid_16190792}{PMID: 16190792}, Thomas Günther et al., 2005].

- Post-Traumatic Stress Disorder (PTSD):
  - A double-blind, placebo-controlled study in children and adolescents (6–17 years) with PTSD found sertraline to be "generally safe," though not more effective than placebo. Discontinuation due to adverse events was 7.5\% for sertraline vs. 3.2\% for placebo [\hyperlink{pmid_21186964}{PMID: 21186964}, Adelaide S Robb et al., 2010].
  - Another controlled study in burned children (6–20 years) found sertraline to be a "safe drug" over 24 weeks, with some evidence of efficacy by parent report [\hyperlink{pmid_22040192}{PMID: 22040192}, Frederick J Stoddard et al., 2011].

Ages below 6:
- No targeted safety studies of sertraline hydrochloride in children under 6 years old were identified in the abstracts. One case report describes sertraline ingestion in a 22-month-old, but this is about overdose, not therapeutic use [\hyperlink{pmid_9579287}{PMID: 9579287}, G Catalano et al.].

Summary:
- For children aged 6–18, multiple targeted studies (including controlled trials and long-term extensions) affirm that sertraline hydrochloride is generally safe for use in this population for MDD, OCD, and some anxiety disorders, with adverse events similar to those seen in adults and no major safety signals.
- For children under 6, there is no evidence from targeted safety studies regarding therapeutic use, so safety is unknown.

\subsection*{Abstracts}
\hypertarget{pmid_11552629}{T}he aim of the study was to research the efficiency of sertraline (zoloft) in depressions, anxious states and obsessive-compulsive disorders. Diagnosis of the mental disorders was carried out according to ICD-10. 72 children (59 boys, 13 girls) aged 6-18 years were treated. There were 32 inpatients and 40 outpatients. Therapy with sertraline was performed during 8 weeks with a gradual increase (titration) and individual selection of the dose from 12.5 to 100 mg/day. During the therapy clinical observation was combined with the patients' examination using Hamilton Depression Scale and Hamilton Anxiety Scale (HAM-D and HAM-A), and a Clinical Global Impression Scale (CGI). It was established that sertraline was very effective and safe drug in children (it has no influence on cognitive functions, has neither myorelaxing or sedative effects). Activity of this drug is characterized by quick manifestation of thymoanaleptic and anxiolytic effects. It mild depressive states 50 mg/day is a significant dose; in more severe depressions and obsessive-compulsive disorders the dose in juveniles was to 100 mg, the duration of the therapy was more than 2 months. [\hyperlink{Sertraline Hydrochloride}{PMID: 11552629}, V M Voloshina et al., 2001]

\hypertarget{pmid_9579287}{S}ertraline (Zoloft) is a selective serotonin reuptake inhibitor that is commonly used in adults in the treatment of mood and anxiety disorders. Whereas it also is used to treat these illnesses in children, it is not currently approved by the Food and Drug Administration for use in this population. Sertraline use has been increasing secondary to its efficacy and its more tolerable side effect profile than the tricyclic antidepressants. It is also much safer in overdose than the tricyclic antidepressants. Although there have been numerous reports of sertraline overdose in adults, reports in the pediatric population are much less common. We review the literature regarding sertraline overdose in children, describe a case of sertraline ingestion in a 22-month-old infant, and discuss the treatment of such an overdose. [\hyperlink{Sertraline Hydrochloride}{PMID: 9579287}, G Catalano et al., ]

\hypertarget{pmid_1949975}{S}ertraline hydrochloride is a new naphthylamino compound that specifically blocks neuronal reuptake of serotonin. It is currently available in the United Kingdom and under review in the US. Sertraline follows first-order kinetics, with a plasma elimination half-life of 24-26 hours. It is highly bound to plasma proteins and has a large volume of distribution. Multicenter studies conducted by the manufacturer have shown sertraline to be efficacious in the treatment of depression and obsessive-compulsive disorder. The daily dose will range from 50 to 200 mg/d for the treatment of depression. The adverse-effect profile differs greatly from the tricyclic antidepressants, but is similar to that of fluoxetine. The most prominent adverse effects are gastrointestinal (nausea, diarrhea/loose stools, dyspepsia). [\hyperlink{Sertraline Hydrochloride}{PMID: 1949975}, S K Guthrie et al., 1991]

\hypertarget{pmid_9819070}{T}he serotonin selective reuptake inhibitors are increasingly being used for the treatment of panic disorder. We examined the efficacy and safety of the serotonin selective reuptake inhibitor sertraline hydrochloride in patients with panic disorder. One hundred seventy-six nondepressed outpatients with panic disorder, with or without agoraphobia, from 10 sites followed identical protocols that used a flexible-dose design. After 2 weeks of single-blind placebo, patients were randomly assigned to 10 weeks of double-blind, flexible-dose treatment with either sertraline hydrochloride (50-200 mg/d) or placebo. Sertraline-treated patients exhibited significantly greater improvement (P=.01) at end point than did patients treated with placebo for the primary outcome variable, panic attack frequency. Significant differences between groups were also evident for clinician and patient assessments of improvement as measured by the Clinical Global Impression Improvement (P=.01) and Severity (P=.009) Scales, Panic Disorder Severity Scale ratings (P=.03), high end-state function assessment (P=.03), Patient Global Evaluation rating (P=.01), and quality of life scores (P=.003). Adverse events, generally characterized as either mild or moderate, were not significantly different in overall incidence between the sertraline and placebo groups. Results support the safety and efficacy of sertraline for the short-term treatment of patients with panic disorder. [\hyperlink{Sertraline Hydrochloride}{PMID: 9819070}, M H Pollack et al., 1998]

\hypertarget{pmid_11349701}{T}he aim of this open-label study was to assess the therapeutic benefits, response pattern, and safety of sertraline in children with social anxiety disorder. Fourteen outpatient subjects with a primary Axis I diagnosis of social anxiety disorder were treated in an 8-week open trial of sertraline. Diagnostic and primary outcome measures included the Anxiety Disorders Interview Schedule for Children, Clinical Global Impressions scale (CGI), Social Phobia and Anxiety Inventory for Children, and a standardized behavioral avoidance test. As measured by the CGI (Improvement subscale), 36\% (5/14) of subjects were classified as treatment responders and 29\% (4/14) as partial responders by the end of the 8-week trial. A significant clinical response appeared by week 6. Self-report and behavioral measures showed significant clinical improvement into normal range across all domains measured. The mean dose of sertraline was 123.21+/-37.29 mg per day. Sertraline was generally well tolerated. In open treatment, sertraline resulted in significant improvement in symptoms of childhood social anxiety disorder. Absolute response rates varied depending on rating scales used. Findings from this study are sufficiently strong to warrant a future multisite, randomized, double-blind, placebo-controlled trial of sertraline for treatment of childhood social anxiety disorder. [\hyperlink{Sertraline Hydrochloride}{PMID: 11349701}, S N Compton et al., 2001]

\hypertarget{pmid_22040192}{T}his study evaluated the potential benefits of a centrally acting selective serotonin reuptake inhibitor, sertraline, versus placebo for prevention of symptoms of posttraumatic stress disorder (PTSD) and depression in burned children. This is the first controlled investigation based on our review of the early use of a medication to prevent PTSD in children. Twenty-six children aged 6-20 were assessed in a 24-week double-blind placebo-controlled design. Each child received either flexibly dosed sertraline between 25-150 mg/day or placebo. At each reassessment, information was collected in compliance with the study medication, parental assessment of the child's symptomatology and functioning, and the child's self-report of symptomatology. The protocol was approved by the Human Studies Committees of Massachusetts General Hospital and Shriners Hospitals for Children. The final sample was 17 subjects who received sertraline versus 9 placebo control subjects matched for age, severity of injury, and type of hospitalization. There was no significant difference in change from baseline with child-reported symptoms; however, the sertraline group demonstrated a greater decrease in parent-reported symptoms over 8 weeks (-4.1 vs. -0.5, p=0.005), over 12 weeks (-4.4 vs. -1.2, p=.008), and over 24 weeks (-4.0 vs. -0.2, p=0.017). Sertraline was a safe drug, and it was somewhat more effective in preventing PTSD symptoms than placebo according to parent report but not child report. Based on this study, sertraline may prevent the emergence of PTSD symptoms in children. [\hyperlink{Sertraline Hydrochloride}{PMID: 22040192}, Frederick J Stoddard et al., 2011]

\hypertarget{pmid_24627951}{T}o determine the safety and efficacy of high-dose oral chloral hydrate for pediatric ophthalmic procedures. This study is a retrospective review of a quality audit of pediatric sedation for ophthalmic evaluation and imaging performed at King Khaled Eye Specialist Hospital between January 1 and December 31, 2011, in children aged 1 month to 6 years. Three hundred fifty-eight of 380 (94.2\%) sedation procedures were successful after a single dose of chloral hydrate, with 356 of 380 (93.7\%) children sedated within 45 minutes of the first dose. The total success rate of the sedation procedure increased to 97.9\% (372 of 380) when a second dose was administered. Children adequately sedated after a single dose of chloral hydrate were on average younger and weighed less than children who required additional doses. No major adverse events were documented. The use of chloral hydrate sedation for ophthalmic evaluation and imaging was safe and effective in this patient population with a high rate of procedure completion. [\hyperlink{Sertraline Hydrochloride}{PMID: 24627951}, Michelle E Wilson et al., ]

\hypertarget{pmid_2402648}{C}hloral hydrate has been used extensively to sedate children, but at Brooke Army Medical Center, other drug combinations were becoming increasingly popular due to a perception that chloral hydrate had a high rate of failure, especially with younger or neurologically impaired children. Therefore, 50 children were given the drug before a diagnostic study, and patient data and a sedation score were recorded on a worksheet. Of 50 children, 43 (86\%) were "successfully sedated" on the first attempt with no side effects. Children with neurologic disorders had a much greater (27\% vs 4\%) failure rate than "normal" children. The sedation rate did not significantly differ by age, sex, or initial drug dosage. The study suggest that chloral hydrate is a safe and effective oral sedative but that children with neurologic disorders may need alternative drugs for sedation. [\hyperlink{Sertraline Hydrochloride}{PMID: 2402648}, P D Rumm et al., 1990]

\hypertarget{pmid_15951862}{D}iagnostic and therapeutic procedures in children are made easier using sedation. However, there is no consensus about which drug should be used to achieve this. Furthermore, none of the drugs used for sedation are risk free. The aim of this work is to study sedation indications, effectiveness, and safety at our center. A prospective observational study conducted at the Pediatric Day Care Unit, King Fahad National Guard Hospital, Riyadh, Saudi Arabia. The study covered 17.5 weeks in 2 periods: May 9th 1999 to June 13th 1999 and October 31st 2001 to February 11th 2002. Children <12 years were included. Collected data included demographics, indication, drug dosing and outcome. Data were reported as mean +/- SD. We included 148 patients, age 38 +/- 30 months. Adequate sedation was achieved in 79\% after initial chloral hydrate (CH) dose of 56.9 +/- 9.3 mg/kg, in 95\% after adding 18.5 +/- 6.4 mg/kg CH and in 96\% after adding second drug. Compared to nonrespondents, first CH dose respondents were younger and lower in weight. The CH side effects were few and mild. Chloral hydrate is a safe and effective agent for sedation in children with an age and weight dependent response. [\hyperlink{Sertraline Hydrochloride}{PMID: 15951862}, Omar M Hijazi et al., 2005]

\hypertarget{pmid_28741653}{C}hloral hydrate is commonly used to sedate children for painless procedures. Children may recover more quickly after sedation with dexmedetomidine, which has a shorter half-life. We randomly allocated 196 children to chloral hydrate syrup 50 mg.kg [\hyperlink{Sertraline Hydrochloride}{PMID: 28741653}, V M Yuen et al., 2017] Sertraline hydrochloride is a selective serotonin reuptake inhibitor (SSRI) widely prescribed to patients suffering from psychiatric disorders. Pharmaceutical products such as sertraline have been identified in environmental waters. This study describes the evaluation of sertraline using a battery of freshwater species representing four trophic levels. The species most sensitive to sertraline were Daphnia magna 21 d reproduction test, Pseudokirchneriella subcapitata 72 h growth inhibition, and Oncorhynchus mykiss 96 h mortality, with the Microtox assay being the least sensitive assay. The D. magna 21 d reproduction test was approximately two orders of magnitude more sensitive than the other bioassays. These results show the advantages of having a tiered approach within a test battery. The presented results indicate that sertraline hydrochloride adversely affects aquatic organisms at levels several orders of magnitude higher than that reported in municipal effluent concentrations, however adverse effects may result from lower concentration exposures, further research into chronic toxicity is therefore advocated. [\hyperlink{Sertraline Hydrochloride}{PMID: 28741653}, Elaine Minagh et al., 2009]

\hypertarget{pmid_9549959}{T}o evaluate the pharmacokinetics, safety, and efficacy of sertraline in children (6 to 12 years old) and adolescents (13 to 17 years old). Children (n = 29) and adolescents (n = 32) with major depression, obsessive-compulsive disorder (OCD), or both received a single dose of 50 mg of sertraline followed, 1 week later, by 35 days of sertraline treatment as follows: (1) either a starting dose of 25 mg/day titrated to 200 mg/day in 25-mg increments or (2) a starting dose of 50 mg/day titrated to 200 mg/day in 50-mg increments. Sertraline and desmethylsertraline pharmacokinetics were determined approximately weekly, and efficacy measures were assessed before drug administration and at the end of treatment. Mean area under the plasma concentration-time curve (AUC), peak plasma concentration (Cmax), and elimination half-life (t1/2) for sertraline and desmethylsertraline were similar to previously reported adult values. No titration-dependent pharmacokinetic or safety differences were seen. While Cmax and AUC0-24 were greater for children versus adolescents, these differences disappeared after parameters were normalized for body weight. Sertraline was well tolerated in both children and adolescents, with adverse experiences similar to those previously reported by adult patients. Efficacy measurements indicated improvement (p < .001) in depression and OCD symptomatology. Sertraline can be safely administered to pediatric patients using the currently recommended adult titration schedule. [\hyperlink{Sertraline Hydrochloride}{PMID: 9549959}, J Alderman et al., 1998]

\hypertarget{pmid_11589530}{T}o evaluate the safety and effectiveness of sertraline in the long-term treatment of pediatric obsessive-compulsive disorder (OCD). Children (6-12 years; n= 72) and adolescents (13-18 years; n = 65) with DSM-III-R-defined OCD who had completed a 12-week, double-blind, placebo-controlled sertraline study were given open-label sertraline 50 to 200 mg/day in this 52-week extension study. Concomitant psychotherapy was allowed during the extension study Outcome was evaluated by the Children's Yale-Brown Obsessive Compulsive Scale (CY-BOCS), National Institute of Mental Health Global Obsessive-Compulsive Scale, and Clinical Global Impression Severity (CGI-S) and Improvement (CGI-I) scores. Significant improvement (p < .0001) was demonstrated on all four outcome parameters on an intent-to-treat analysis for the overall study population (n = 132), as well as the child and the adolescent samples. At endpoint, 72\% of children and 61\% of adolescents met response criteria (>25\% decrease in CY-BOCS and a CGI-I score of 1 or 2). Significant (p < .05) improvements were also demonstrated from the extension study baseline to endpoint on all outcome parameters in those patients who received sertraline during the 12-week, double-blind acute study. Long-term sertraline treatment was well tolerated, and there were no discontinuations due to changes in vital signs, laboratory values, or electrocardiograms. Sertraline (50-200 mg/day) was effective and generally well tolerated in the treatment of childhood and adolescent OCD for up to 52 weeks. Improvement was seen with continued treatment. [\hyperlink{Sertraline Hydrochloride}{PMID: 11589530}, E H Cook et al., 2001]

\hypertarget{pmid_28275979}{S}edation is often required for children undergoing diagnostic procedures. Chloral hydrate has been one of the sedative drugs most used in children over the last 3 decades, with supporting evidence for its efficacy and safety. Recently, chloral hydrate was banned in Italy and France, in consideration of evidence of its carcinogenicity and genotoxicity. Dexmedetomidine is a sedative with unique properties that has been increasingly used for procedural sedation in children. Several studies demonstrated its efficacy and safety for sedation in non-painful diagnostic procedures. Dexmedetomidine's impact on respiratory drive and airway patency and tone is much less when compared to the majority of other sedative agents. Administration via the intranasal route allows satisfactory procedural success rates. Studies that specifically compared intranasal dexmedetomidine and chloral hydrate for children undergoing non-painful procedures showed that dexmedetomidine was as effective as and safer than chloral hydrate. For these reasons, we suggest that intranasal dexmedetomidine could be a suitable alternative to chloral hydrate. [\hyperlink{Sertraline Hydrochloride}{PMID: 28275979}, Giorgio Cozzi et al., 2017]

\hypertarget{pmid_22246409}{C}hloral hydrate (CH) is safe and effective for sedation of suitable children. The purpose of this study was to assess whether adequate sedation is achieved with reduced CH doses. We retrospectively recorded outpatient CH sedations over 1 year. We defined standard doses of CH as 50 mg/kg (infants) and 75 mg/kg (children >1 year). A reduced dose was defined as at least 20\% lower than the standard dose. In total, 653 children received CH sedation (age, 1 month-3 years 10 months), 42\% were given a reduced initial dose. Augmentation dose was required in 10.9\% of all children, and in a higher proportion of children >1 year (15.7\%) compared to infants (5.7\%; P < 0.001). Sedation was successful in 96.7\%, and more frequently successful in infants (98.3\%) than children >1 year (95.3\%; P = 0.03). A reduced initial dose had no negative effect on outcome (P = 0.19) or time to sedation. No significant complications were seen. We advocate sedation with reduced CH doses (40 mg/kg for infants; 60 mg/kg for children >1 year of age) for outpatient imaging procedures when the child is judged to be quiet or sleepy on arrival. [\hyperlink{Sertraline Hydrochloride}{PMID: 22246409}, Jennifer Bracken et al., 2012]

\hypertarget{pmid_28827252}{C}eftriaxone is widely used in children in the treatment of sepsis. However, concerns have been raised about the safety of ceftriaxone, especially in young children. The aim of this review is to systematically evaluate the safety of ceftriaxone in children of all age groups. MEDLINE, PubMed, Cochrane Central Register of Controlled Trials, EMBASE, CINAHL, International Pharmaceutical Abstracts and adverse drug reaction (ADR) monitoring systems will be systematically searched for randomised controlled trials (RCTs), cohort studies, case-control studies, cross-sectional studies, case series and case reports evaluating the safety of ceftriaxone in children. The Cochrane risk of bias tool, Newcastle-Ottawa and quality assessment tools developed by the National Institutes of Health will be used for quality assessment. Meta-analysis of the incidence of ADRs from RCTs and prospective studies will be done. Subgroup analyses will be performed for age and dosage regimen. Formal ethical approval is not required as no primary data are collected. This systematic review will be disseminated through a peer-reviewed publication and at conference meetings. CRD42017055428. [\hyperlink{Sertraline Hydrochloride}{PMID: 28827252}, Linan Zeng et al., 2017]

\hypertarget{pmid_2026812}{C}hloral hydrate is commonly used to sedate children before CT. However, no prospective study has been published of the safety and efficacy of chloral hydrate at high dose levels for children undergoing CT. We define high dose levels of oral chloral hydrate to be 80-100 mg/kg, with a maximum total dose of 2 g. High dose chloral hydrate sedation was administered orally to 295 children for 326 CT examinations. Adverse reactions occurred in 7\% of the children, with vomiting being the most common (4.3\% of children). Hyperactivity and respiratory symptoms each occurred in less than 2\% of children. Prolonged sedation ( greater than 2 h) was not encountered in our series. Sedation was successful in producing motion free CT examinations, so that in 303 (93\%) of the cases, no repeat CT scans were needed. We conclude that high dose oral chloral hydrate provides safe and effective sedation for children undergoing CT. [\hyperlink{Sertraline Hydrochloride}{PMID: 2026812}, S B Greenberg et al., ]

\hypertarget{pmid_31369972}{C}hloral hydrate is a sedative that has been used for many years in clinical practice and, under proper conditions, gives a deep and long enough sleep to allow performance of objective hearing tests in young children. The reluctance to use this substance stems from side effects reported over time that can vary, depending on dose, procedure settings and immediate life supporting intervention when needed. Our study adds to those that have appeared in recent years, showing that chloral hydrate is an effective and safe substance when is used in proper conditions. The study included 322 children who needed sedation for objective hearing tests, from April 2014 to March 2018. Parents were instructed to bring the child tired and fasted for at least 2 h before sedation. The sedative was administered by trained staff in the hospital, and the child was monitored until awaking. In our study group, over half of the children were in the age 1-4 years group, and only 15\% were older than 4 years. The dose of chloral hydrate ranged between 50 and 83 mg/kg body weight, with an average of 75 mg. Successful sedation occurred in 94.1\% of children; 0.9\% of children awoke during testing and required supplemental sedation or rescheduling of the testing. The most common side effects were vomiting, agitation, prolonged sleep, and failure to fall asleep. Comparing the side effects of chloral hydrate in our study with those from other studies, ours were similar to those described in the literature. In our study chloral hydrate was effective and had only limited adverse effects. The use of chloral hydrate under hospital conditions with proper monitoring could be a practical and safe solution for outpatients or those with short-term hospitalisation. [\hyperlink{Sertraline Hydrochloride}{PMID: 31369972}, Violeta Necula et al., 2019]

\hypertarget{pmid_21531030}{C}hloral hydrate (CH) is an oral sedative widely used to sedate infants and young children during auditory brainstem response (ABR) testing. The aim of this study was to record effectiveness, complications and safety of CH as a sedative for ABR. From January of 2003 until December of 2007, 1903 children were tested for ABR, 568 of them being under the age of 6 months. CH (8\%) was used for sedation at a dose of 40 mg/kg with a repeat dose, if necessary, for an adequate sedation, in 20-30 min. We recorded the effectiveness of CH as a sedative for ABR examination, as well as all complications related to the use of CH such as vomiting, rash, hyperactivity, respiratory distress and apnea. The statistical method used was the absolute and percentage frequency distribution of the occurrences. Sedation with CH was necessary to perform testing in 1591 (83.6\%) of the examined children. However, in the population of the examined infants, only 341 (60\%) were sedated with CH, because the remaining 227 (40\%) fell asleep by themselves. Complications included hyperactivity in 152 children (8\%), minor respiratory distress in 10 children (0.4\%), vomiting in 217 children (11.4\%), apnea in 4 children (0.2\%) and rash in 10 children (0.4\%). The complications of hyperactivity, vomiting and rash resolved without any medical treatment. The apnea cases were managed effectively by supplying ventilation to the children via a mask in the presence of an anesthesiologist. The use of CH at a dose of 40 mg/kg up to 80 mg/kg is safe and effective when administered in a setting with adequate equipment and the presence of well trained personnel. [\hyperlink{Sertraline Hydrochloride}{PMID: 21531030}, Eirini Avlonitou et al., 2011]

\hypertarget{pmid_16190792}{T}his study investigated the cognitive side effects of a 6-week course of sertraline treatment on verbal memory and attention in children and adolescents. Children with various anxiety disorders (social phobia, generalized and separation anxiety disorder; n = 28), between 8 and 17 years of age, received a standardized, computerized neuropsychological assessment before treatment and another 6 weeks after treatment onset with sertraline (daily dose range between 25 and 100 mg). The patient group was compared to healthy controls (n = 28), who were matched for age and IQ and were also tested twice over a 6-week period. Sertraline did not have any negative effects on attentional performance (p > 0.05) but did increase response speed in a divided attention paradigm (p = 0.02). By contrast, performance of the interference part of a verbal memory task decreased (p = 0.05). The described results also remained stable over a 12-week period after treatment onset. Thus, the cognitive side effects of sertraline seemed to differ slightly between pediatric patients and those described in adult patient groups, should, therefore, be carefully assessed. [\hyperlink{Sertraline Hydrochloride}{PMID: 16190792}, Thomas Günther et al., 2005]

\hypertarget{pmid_16553532}{T}he aim of this study was to assess the long-term safety, tolerability, and efficacy of sertraline 50-200 mg once-daily in children (6-11 year olds) and adolescents (12-18 year olds) with a Diagnostic and Statistical Manual of Mental Disorders, 4th edition (DSM-IV) diagnosis of major depressive disorder (MDD). This study consisted of a 24-week open-label observational study of children and adolescents who had completed either of two 10-week double-blind, placebo-controlled trials. The Children's Depression Rating Scale-Revised (CDRS-R) was the primary measure of efficacy. Two hundred ninety nine (299) patients completed the acute studies and were eligible for the extension study. Of these, 226 enrolled, but 5 did not receive treatment. Of 221 patients (107 children and 114 adolescents), 62.4\% completed the study. The endpoint mean daily dose was 109.9 mg/day. The mean decrease in CDRS-R score from double-blind baseline was 34.8 points (p < 0.001), with patients showing continued improvement in CDRS-R scores regardless of which treatment they received in the double-blind studies. At endpoint, 86\% of patients met CDRS-R responder and 58\% CDRS-R remitter criteria. Sertraline appears to be well tolerated and safe over 24 weeks of treatment in children and adolescents with MDD. Children and adolescents treated with sertraline appear to have increased improvement over that seen in the first 10 weeks of treatment. These findings need confirmation in placebo-controlled studies. [\hyperlink{Sertraline Hydrochloride}{PMID: 16553532}, Moira Rynn et al., ]

\hypertarget{pmid_25637819}{S}ertraline is one of the serotonin-specific reuptake inhibitors that is effective in treating several disorders such as major depression, obsessive-compulsive disorder, panic disorder, and social phobia. It is marketed in the form of its hydrochloride salt, which exhibits better solubility in water than its free base form. However, the absorption of sertraline through biological membranes could be improved by enhancing the solubility of its base because it is more hydrophobic than sertraline hydrochloride. To clarify the mechanism for the interaction of sertraline base with β-CD, it is important to study the basic interaction between the β-CD ring and sertraline base. Therefore, in this study, the currently used hydrochloride salt form was converted into the free base and β-CD was used as a model for β-CD derivatives to evaluate the interaction between β-CD and the sertraline base. The solid-state physicochemical characteristics of the sertraline-β-CD complex were investigated by the phase solubility method, differential scanning calorimetry, Fourier transform IR spectroscopy, FT-Raman spectroscopy, powder X-ray diffraction, and (13)C cross-polarization magic-angle spinning NMR measurements. The results showed that sertraline base and β-CD form an inclusion complex, and the stoichiometric ratio of the solid-state sertraline base-β-CD complex is 1:1, which was estimated by the (1)H NMR measurements of the complex dissolved in DMSO-d6.  [\hyperlink{Sertraline Hydrochloride}{PMID: 25637819}, Noriko Ogawa et al., 2015] The aim of this study was to evaluate the long-term pharmacokinetics, safety, and efficacy of sertraline in children and adolescents with obsessive-compulsive disorder (OCD) or major depressive disorder (MDD). After 42-day initial treatment and 9-day withdrawal phases, children (6-12 years, n = 16) and adolescents (13-18 years, n = 27) entered a 24-week open-label phase, with sertraline titrated to 200 mg/day. Blood samples for plasma sertraline and N-desmethylsertraline levels were taken at the beginning of the 24-week phase and at weeks 1, 4, 8, 12, and 24. Efficacy and safety data were also collected. Mean maximum daily dose at endpoint was 157 +/- 49 mg. For female and male children, mean sertraline/N-desmethylsertraline concentrations normalized to a 200-mg dose were 85.0/160 ng/mL (n = 8) and 79.3/134 ng/mL (n = 8), respectively, and for female and male adolescents, 70.5/109 ng/mL (n = 16) and 76.3/120 ng/mL (n = 8). No significant age or gender effects or age-by-gender interactions were observed in sertraline values. Mean sertraline plasma concentrations normalized for dose and body weight did not differ significantly by age or gender. Three (3) patients (7\%) discontinued owing to adverse events. In patients with OCD (n = 10), improvements were observed in Children's Yale-Brown Obsessive Compulsive Scale (CY-BOCS) (p = 0.029) and National Institute of Mental Health (NIMH) Global Obsessive Compulsive Scale (OCS) (p = 0.01). In MDD patients (n = 32), Clinical Global Impression (CGI) Severity (p = 0.002) and Improvement (p = 0.011) improved. Long-term treatment of MDD and OCD with sertraline up to 200 mg/day in children and adolescents results in dose-normalized plasma concentrations comparable to those seen in adults. Sertraline was generally well tolerated, and patients demonstrated clinical improvement over 24 weeks of treatment. [\hyperlink{Sertraline Hydrochloride}{PMID: 25637819}, Jeffrey Alderman et al., ]

\hypertarget{pmid_28242616}{A}lthough chloral hydrate (CH) has been used as a sedative for decades, it is not widely accepted as a valid choice for ophthalmic examinations in uncooperative children. This study aimed to systematically review the literature on the drug's safety and efficacy. We searched PubMed, EMBASE, ISI Web of Science, Scopus, CENTRAL, Google Scholar and Trip database to 1 October 2015, using the keywords 'chloral hydrate', 'paediatric' and 'procedural sedation OR diagnostic sedation'. A meta-analysis of randomised controlled trials (RCTs) was performed. A total of 6961 articles were screened and 104 were included in the review. Thirteen of these concerned paediatric ophthalmic examination, while 13 others were RCTs and were meta-analysed. CH was reported to have been administered in a total of 24 265 sedation episodes in children aged from <1 month to 18 years. The meta-analysis showed CH had a higher OR (2.95, 95\% CI 1.09 to 7.99) for successful sedation compared to other sedatives, but significant limitations apply. The commonest reported adverse events (AE) were not serious (eg, paradoxical reaction or transient vomiting) and required no intervention. Severe AE, including two deaths, were related to comorbidity, overdose or aspiration. Despite the paucity of high quality evidence, the existing literature suggests that the use of CH for procedural sedation in children appears to be an effective alternative to general anaesthesia, and it can be safe when administered in the hospital setting with appropriate monitoring and vigilance for intervention. [\hyperlink{Sertraline Hydrochloride}{PMID: 28242616}, Asimina Mataftsi et al., 2017]

\hypertarget{pmid_21186964}{T}he aim of this study was to evaluate the safety and efficacy of sertraline in children and adolescents who met Diagnostic and Statistical Manual of Mental Disorders, 4th edition (DSM-IV) criteria for posttraumatic stress disorder (PTSD). Children and adolescents (6-17 years old) meeting DSM-IV criteria for PTSD were randomized to 10 weeks of double-blind treatment with sertraline (50-200  mg/day) or placebo. The primary efficacy measure was the University of California, Los Angeles Post-Traumatic Stress Disorder Index for DSM-IV (UCLA PTSD-I). A total of 131 patients met entry criteria and were randomized to sertraline (n = 67; female, 59.7\%; mean age, 10.8; mean UCLA PTSD-I score, 43.8 ± 8.5) or placebo (n = 62; female, 61.3\%; mean age, 11.2; mean UCLA PTSD-I score, 42.1 ± 8.8). There was no difference between sertraline and placebo in least squares (LS) mean change in the UCLA PTSD-I score, either on a completer analysis (-20.4 ± 2.1 vs. -22.8 ± 2.1; p = 0.373) or on an last observation carried forward (LOCF) end point analysis (-17.7 ± 1.9 vs. -20.8 ± 2.1; p = 0.201). Attrition was higher on sertraline (29.9\%) compared to placebo (17.7\%). Discontinuation due to adverse events occurred in a 7.5\% treated with sertraline and 3.2\% treated with placebo. Sertraline was a generally safe treatment in children and adolescents with PTSD, but did not demonstrate efficacy when compared to placebo during 10 weeks of treatment. ClinicalTrials.gov Identifier: NCT00150306. [\hyperlink{Sertraline Hydrochloride}{PMID: 21186964}, Adelaide S Robb et al., 2010]

\section*{Permethrin}
\subsection*{Result}
\subsubsection*{Answer}

Infants younger than 2 months: Unknown  
Children older than 2 months: Unknown  

\subsubsection*{{Explanation}}
\hypertarget{Permethrin}
To determine if permethrin is safe for use in children, I evaluated the available abstracts for targeted studies on safety in pediatric populations, with attention to specific age ranges.

Infants younger than 2 months:
- A 2021 survey of pediatric dermatologists reported that permethrin is frequently used and is the preferred treatment for scabies in infants younger than 2 months. Among 47 users, only 4.3\% reported mild side effects (itching, erythema, xerosis), and none were serious. However, this was a survey of physician experience, not a controlled safety trial, and the authors note that permethrin is not approved for this age group in many regions [\hyperlink{pmid_33486822}{PMID: 33486822}, Cristina Thomas et al., 2021].
- A 2021 review found that complete resolution of scabies was observed in 100\% of infants younger than two months treated with permethrin, with adverse effects limited to local eczematous reactions. The authors conclude that permethrin appears to have an acceptable safety profile in infants, but also call for additional high-quality studies [\hyperlink{pmid_34184244}{PMID: 34184244}, Yolanka Lobo et al., 2021].
- A 2010 review states that permethrin is recommended as first-line therapy for patients older than 2 months, and that for infants younger than 2 months, 7\% sulfur is recommended due to theoretical concerns about percutaneous absorption [\hyperlink{pmid_20944041}{PMID: 20944041}, Lina Albakri et al., 2010].
- A 2003 review expresses the opinion that 5\% permethrin is the best treatment for scabies in infants and young children, and "appears to be quite safe in infants," but this is not based on a targeted safety trial [\hyperlink{pmid_12943481}{PMID: 12943481}, Mervyn L Elgart et al., 2003].

Children older than 2 months:
- The 2010 review [\hyperlink{pmid_20944041}{PMID: 20944041}] affirms that topical permethrin (5\% cream) is a safe and effective scabicide in children and is recommended as first-line therapy for patients older than 2 months.
- A 1991 study reports permethrin was used in 48 children and 56 adults with scabies, with permethrin being "distinctly superior" to benzyl benzoate and lindane, and recommended for use in small children, including premature infants and those with neurological complications. However, the abstract does not specify the age breakdown or provide detailed safety outcomes [\hyperlink{pmid_2010292}{PMID: 2010292}, U F Haustein et al., 1991].

Animal studies:
- Several animal studies (rats, mice, zebrafish) indicate increased susceptibility to neurotoxicity in young animals compared to adults, but these are not human pediatric safety studies and cannot be directly extrapolated to children [\hyperlink{pmid_31808824}{PMID: 31808824}, Jing Pang et al., 2020; \hyperlink{pmid_29893952}{PMID: 29893952}, Michael T Williams et al., 2018; \hyperlink{pmid_31415104}{PMID: 31415104}, Hirokatsu Saito et al., 2019].

Summary:
- For infants younger than 2 months, there are no controlled safety trials, but survey and review data suggest permethrin is frequently used and generally well tolerated, with only mild side effects reported. However, due to the lack of targeted safety trials, safety cannot be definitively affirmed.
- For children older than 2 months, there is stronger evidence from reviews and clinical experience supporting the safety and efficacy of permethrin for scabies, and it is recommended as first-line therapy. However, the evidence is based on clinical experience, surveys, and reviews, not large-scale randomized controlled safety trials.

Therefore, based on the abstracts, there is no definitive targeted safety trial in children for permethrin, but there is substantial clinical experience and survey data supporting its use in children older than 2 months, and suggestive but not definitive evidence for infants younger than 2 months.

\subsection*{Abstracts}
\hypertarget{pmid_31785554}{P}ermethrin (PM) is one of the most used synthetic pyrethroid worldwide. Exposure to this compound during pregnancy and early childhood has been indicated as a risk factor for neurodevelopmental disorders. We evaluated the long-term effects of embryonic PM exposure in different stages of zebrafish development. Briefly, embryos (3 hpf) were exposed to sub-lethal concentrations of PM (25 and 50 μg.L [\hyperlink{Permethrin}{PMID: 31785554}, M E M Nunes et al., 2020] I frequently see children with scabies in my practice. A variety of medications are available to treat scabies. Permethrin is one of the most common medications used. Is permethrin a safe and effective option for children? Scabies is a common parasitic skin infection. It is highly prevalent in young children. Topical permethrin (5\% cream) is a safe and effective scabicide in children. It is recommended as a first-line therapy for patients older than 2 months of age. Because there are theoretical concerns regarding percutaneous absorption of permethrin in infants younger than 2 months of age, guidelines recommend 7\% sulfur preparation instead of permethrin. [\hyperlink{Permethrin}{PMID: 31785554}, Lina Albakri et al., 2010]

\hypertarget{pmid_2010292}{T}he history and development of pyrethrum, the natural pyrethrins and synthetic pyrethroids and their insecticidal properties, chemical structure and toxic and allergic side-effects are reported. Permethrin is stressed as a photostable insecticide that is very effective against a large variety of insects and mites with low mammalian toxicity and virtually no allergic side-effects. Only 10-20 min after application, permethrin (1\% cream rinse or 0.5\% in ethanol) proved to be safe, reliable and cosmetically acceptable in the treatment of infestations with head lice and the prevention of reinfestations, and also in failures with lindane owing to the development of tolerance in the lice. The same was true of 5\% permethrin cream (2.5\% in children below 5 years of age) used in the treatment of scabies. Permethrin is absorbed percutaneously in only small amounts, is metabolized rapidly in the skin and excreted in the urine. A single "head to toe" application is ideal for eradication programmes allowing lice to be targetted and the prevalence of secondary bacterial infections decreased at the same time. Benzyl benzoate has an irritant potential, and lindane has been reported to exert CNS toxicity in a few anecdotal cases, in particular in small children or after repeated applications in patients with scabies crustosa, and permethrin was distinctly superior to both of these. This is documented by the results obtained in the treatment of 48 children and 56 adults suffering from scabies. Permethrin is recommended in scabies therapy in premature infants, small children, patients with seizures and neurological complications, in treatment failures with lindane entailing the need to repeat the therapy, in scabies crustosa and in pregnant women and nursing mothers. [\hyperlink{Permethrin}{PMID: 2010292}, U F Haustein et al., 1991]

\hypertarget{pmid_18498556}{P}ermethrin is a synthetic pyrethroid widely used in flea control products for small animals. Accidental toxicity can occur with off-label usage, and cats are particularly susceptible. Retrospective study of 20 cases of permethrin toxicity in cats treated at an emergency clinic in Brisbane, Queensland from October 2004 to June 2005. The diagnosis of permethrin toxicity was made on the basis of a history of exposure and characteristic clinical signs, including seizures, muscle fasciculations, and tremors. Decontamination and appropriate seizure or muscle fasciculation control were the basis of treatment. The outcome was good after rapid intervention and 19 of the 20 cats were successfully treated, with the only death occurring in a kitten for which treatment was delayed for 24 h. No long-term complications were reported by the cats' owners at 4-month follow-up after discharge from hospital. Owner education, together with more appropriate product labelling, may help eliminate this problem in the future. [\hyperlink{Permethrin}{PMID: 18498556}, N L Dymond et al., 2008]

\hypertarget{pmid_12943481}{I}n the US, 6\% sulfur in petrolatum has been the most frequently administered treatment for infantile scabies. It appears to be safe but there is no literature containing a large series of patients on which to base that determination. In the UK, benzyl benzoate is the approved product. Benzyl benzoate is rarely used in the US at the present time. 5\% Permethrin is an excellent substitute and has many advantages. It appears to be quite safe in infants, although it is more expensive than other products. It remains present on the skin for several days, therefore protecting against reinfestation. Ivermectin is a systemic drug which is assumed to be safe in infants, although it requires repeated doses and does not protect against reinfestation. In the opinion of the author, 5\% permethrin is the best treatment for scabies in infants and young children. [\hyperlink{Permethrin}{PMID: 12943481}, Mervyn L Elgart et al., 2003]

\hypertarget{pmid_34184244}{A}s standard treatments are not licensed for use in the infantile population, the treatment of scabies in this age group can be challenging. We review the relevant evidence to determine the roles of topical permethrin and oral ivermectin in the management of infantile scabies. Demographic and clinical data were collected from relevant English articles published from January 2000 to December 2020. Complete resolution was observed in 100\% of infants younger than two months treated with permethrin, and 87.6\% of infants aged 12 months or less and/or children weighing under 15 kg treated with ivermectin. Adverse effects from permethrin use were limited to local eczematous reactions. Adverse effects from ivermectin use included mildly elevated creatine kinase levels, eczema flare-ups, diarrhoea, vomiting, irritability, pruritus and pustular skin reactions. Overall, both permethrin and ivermectin appear to have an acceptable safety profile in infants. Permethrin is highly effective as a first-line therapy for scabies in infants younger than two months. Ivermectin use is recommended when authorised topical treatment has failed, in crusted scabies, in cases where compliance with topical agents may be problematic, and in infants with severely inflamed or broken skin where prescription of topical therapies would likely cause cutaneous and systemic toxicity. Additional high-quality studies are needed to guide best practice in the management of infantile scabies. [\hyperlink{Permethrin}{PMID: 34184244}, Yolanka Lobo et al., 2021]

\hypertarget{pmid_35038988}{I}t is common for children to accidentally ingest chemical drugs with different degrees of toxicity. Meperfluthrin is a highly effective and easy-to-use pyrethroid pesticide with low toxicity. It is widely used in electric mosquito coils. This type of electric mosquito coil is used in daily life, which increases the chance of exposure among children and, consequently, may lead to accidental ingestion. There are only few reports of meperfluthrin poisoning causing lung injury in children. We report a rare clinical case of lung injury wherein a child ingested meperfluthrin orally. We report the case of a 1-year-old boy who accidentally swallowed an electric mosquito coil containing meperfluthrin and developed cough and fever. The patient's parents observed him swallowing the electric mosquito coil (Qiangshou®). Although he was stopped, the child had already swallowed approximately 10 ml of the liquid. According to the instructions, it contained 9 mg/ml of meperfluthrin, thus, it was assumed that he ingested meperfluthrin at a dose of approximately 90 mg. Computed tomography (CT) of his lungs showed uneven brightness in both lungs with multiple spots, scaly shadows, and mesh. Density of the shadows indicated lung parenchymal and interstitial lung disease. Lung tidal function tests indicated obstructive ventilation dysfunction. After evaluation and treatment, his cough drastically reduced, his fever disappeared, and his lung CT findings showed improvement. Therefore, accidental ingestion of meperfluthrin led to acute lung injury in a paediatric patient. Because of prompt treatment, his lung lesions recovered well. Meperfluthrin causes airway mucosal damage and hypersensitivity. Lung CT and lung tidal function measurements can be used to monitor changes in the condition. Presently, there is a lack of specific detoxification drugs for meperfluthrin poisoning. Thus, the focus of treatment is to protect the airway mucosa and reduce inflammatory reactions. [\hyperlink{Permethrin}{PMID: 35038988}, Zhongqiang Li et al., 2022]

\hypertarget{pmid_28962315}{P}ermethrin is a synthetic Type I pyrethroidal neurotoxic pesticide that has been responsible for accidental animal deaths. Despite its widespread use, there are no published case reports on pediatric intensive care unit admissions due to permethrin exposure. We report the unusual and varied presentations of permethrin toxicity in three siblings presenting to a tertiary care pediatric intensive care unit (PICU). While there is no standard clinical diagnostic test for permethrin, accurate diagnosis was obtained by rapidly analyzing the offending agent. In the absence of a known antidote for permethrin, supportive management was initiated and resulted in a favorable outcome for all three siblings. [\hyperlink{Permethrin}{PMID: 28962315}, Bonny Drago et al., 2014]

\hypertarget{pmid_31415104}{P}ermethrin, a pyrethroid chemical, is widely used as a pesticide because of its rapid insecticidal activity. Although permethrin is considered to exert very low toxicity in mammals, the effects of early, low-level, chronic exposure on the adult central nervous system are unclear. In this study, we investigated the effects of low-level, chronic permethrin exposure in early life on the brain functions of adult mice, using environmentally relevant concentrations. We exposed mice to the acceptable daily intake level of permethrin (0.3 ppm) in drinking water during the prenatal and postnatal periods. We then examined the effects on the central nervous system in adult male offspring. In the permethrin group, we detected behavior that displayed incomplete adaptation to a novel environment, as well as an impairment in learning and memory. In addition, immunohistochemical analysis revealed an increase in doublecortin- (an immature neuron marker) positive cells in the hippocampal dentate gyrus in the permethrin exposure group compared with the control group. Additionally, in the permethrin exposure group there was a decrease in astrocyte number in the hilus of the dentate gyrus, and remaining astrocytes were often irregularly shaped. These results suggest that exposure to permethrin at low levels in early life affects the formation of the neural circuit base and behavior after maturation. Therefore, in the central nervous system of male mice, low-level, chronic permethrin exposure during the prenatal and postnatal periods has effects that were not expected based on the known effects of permethrin exposure in mature animals. [\hyperlink{Permethrin}{PMID: 31415104}, Hirokatsu Saito et al., 2019]

\hypertarget{pmid_29893952}{P}ermethrin is a type I (noncyano) pyrethroid that induces tremors at high concentrations and increases acoustic startle responses (ASRs) in adult rodents, however its effects in young rats have been investigated to a limited extent. ASR and tremor were assessed in adult and postnatal day (P)15 Sprague-Dawley rats at oral doses of 60, 90, or 120 mg/kg over an 8 h period. Permethrin increased ASR in adults, regardless of dose, and 20\% of the high-dose rats showed tremor at later time points. For the P15 rats all doses induced tremor at all time points, and ASR was increased at 2 h in the 90 and 120 mg/kg groups with a trend in the 60 mg/kg group compared with controls. The 60 mg/kg group showed increased ASR at 4 and 6 h, whereas the 90 mg/kg group showed no differences from the controls at these times. The 120 mg/kg group showed decreased ASR from 4- to 8-h posttreatment. P15 and adult rats both showed plasma and brain cis- and trans-permethrin increases after dosing. After the same dose of permethrin, P15 rats had greater cis- and trans-permethrin in brain and plasma compared with adults. P15 rats had an increased tremor response compared with adults even at comparable brain permethrin concentrations. For ASR, P15 rats responded sooner and showed a biphasic pattern ranging from increased to decreased response as a function of dose and time, unlike adults that only showed increases. Overall, young rats showed greater effects from permethrin compared with adults. [\hyperlink{Permethrin}{PMID: 29893952}, Michael T Williams et al., 2018]

\hypertarget{pmid_17463061}{P}ermethrin, a popular synthetic pyrethroid insecticide used to control noxious insects in agriculture, forestry, households, horticulture, and public health throughout the world, poses risks of environmental exposure. Here we evaluate the reproductive toxicity of cis-permethrin in adult male ICR mice that were orally administered cis-permethrin (0, 35, or 70 mg/kg d) for 6 wk. Caudal epididymal sperm count and sperm motility in the treated groups were statistically reduced in a dose-dependent manner. Testicular testosterone production and plasma testosterone concentration were significantly and dose-dependently decreased with an increase in LH, and a significant regression was observed between testosterone levels and cis-permethrin residues in individual mice testes after exposure. However, no significant changes were observed in body weight, reproductive organ absolute and relative weights, sperm morphology, and plasma FSH concentration after cis-permethrin treatment. Moreover, cis-permethrin exposure significantly diminished the testicular mitochondrial mRNA expression levels of peripheral benzodiazepine receptor (PBR), steroidogenic acute regulatory protein (StAR), and cytochrome P450 side-chain cleavage (P450scc) and enzyme and protein expression levels of StAR and P450scc. At the electron microscopic level, mitochondrial membrane damage was found in Leydig cells of the exposed mouse testis. Our results suggest that the insecticide permethrin may cause mitochondrial membrane impairment in Leydig cells and disrupt testosterone biosynthesis by diminishing the delivery of cholesterol into the mitochondria and decreasing the conversion of cholesterol to pregnenolone in the cells, thus reducing subsequent testosterone production. [\hyperlink{Permethrin}{PMID: 17463061}, Shu-Yun Zhang et al., 2007]

\hypertarget{pmid_31827707}{P}ermethrin (PM) is a synthetic pyrethroid insecticide widely used as domestic repellent. Damage effects to nontarget organisms have been reported, particularly in the early stages of development. Studies indicate redox unbalance as secondary PM effect. Therefore, our goal was to investigate the acute PM effects on larval zebrafish. Larvae (6 days postfertilization) were exposed to PM (25-600  [\hyperlink{Permethrin}{PMID: 31827707}, Mauro Eugênio Medina Nunes et al., 2019] Large-scale deworming interventions, using anthelminthic drugs, are recommended in areas where the prevalence of soil-transmitted helminth infection is high. Anthelminthic safety has been established primarily in school-age children. Our objective was to provide evidence on adverse events from anthelminthic use in early childhood. A randomized multi-arm, placebo-controlled trial of mebendazole, administered at different times and frequencies, was conducted in children 12 months of age living in Iquitos, Peru. Children were followed up to 24 months of age. The association between mebendazole administration and the occurrence of a serious or minor adverse event was determined using logistic regression. There was a total of 1,686 administrations of mebendazole and 1,676 administrations of placebo to 1,760 children. Eighteen serious adverse events (i.e., 11 deaths and seven hospitalizations) and 31 minor adverse events were reported. There was no association between mebendazole and the occurrence of a serious adverse event (odds ratio [OR] = 1.21; 95\% confidence interval [CI] = 0.47, 3.09) or a minor adverse event (OR = 0.84; 95\% CI = 0.41, 1.72). Results from our trial support evidence of safety in administering mebendazole during early childhood. These results support World Health Organization deworming policy and the scaling up of interventions to reach children as of 12 months of age in endemic areas.  [\hyperlink{Permethrin}{PMID: 31827707}, Serene A Joseph et al., 2016] The use of antihistamine therapy in children for the management of upper respiratory tract infections remains a topic of debate. In this study, we focused on evaluating the effectiveness of promethazine (Phenergan), a first-generation H1 receptor antagonist and sedative, in addressing preoperative and intra-operative sequelae in cleft surgeries. A single-centered, parallel, randomized, double-blinded controlled clinical trial was conducted on 128 children aged 2 to 4 years undergoing cleft palate surgery under general anesthesia. The case group received Phenergan syrup orally twice a day for three days, while the control group received a placebo. Primary outcomes measured preoperative anxiety levels using a children's fear scale, while secondary outcomes assessed preoperative sleep quality and cough rate through objective scales. Intraoperative heart rate was monitored using an ECG connected to a monitor. The results demonstrated that the administration of promethazine resulted in a 34\% reduction in anxiety levels, a 46\% reduction in cold and cough, a 38\% improvement in sleep score, and stable heart rates throughout the surgery compared to the control group. Based on these findings, promethazine is considered a safe premedication option for children undergoing cleft palate surgeries; given its benefits outweigh its adverse effects. [\hyperlink{Permethrin}{PMID: 31827707}, Vedha Vivigdha A et al., 2023]

\hypertarget{pmid_18455858}{P}ermethrin, the most popular insecticide among the synthetic pyrethroids, has been used worldwide to control a wide range of insects in agriculture, forestry, public health, and homes. Humans may have suffered potential exposure to this compound. The commercial formulation of permethrin contains trans and cis isomers. Here, at the same dosage, we made a comparison of the reproductive effects between these two isomers. Male adult ICR mice were orally administered trans- or cis-permethrin daily for 6 weeks at a dose of 0 or 70 mg/(kg day). In the cis-permethrin exposure group, the caudal epididymal sperm count and sperm motility were significantly reduced, and testosterone levels in testes and plasma also fell. Moreover, cis-permethrin induced abnormal seminiferous tubules in testes and suppressed testicular mRNA expression levels of peripheral benzodiazepine receptor, steroidogenic acute regulatory protein, and the cytochrome P450 side-chain cleavage enzyme. Although such adverse effects were not observed in the trans-permethrin exposure group, testicular and urinary metabolite 3-phenoxybenzoic acid levels in trans-permethrin-exposed mice were about three- and sevenfold higher than those in cis-permethrin-exposed mice, respectively. Furthermore, in vitro, hepatic microsomal hydrolase activity for trans-permethrin was nearly 62-fold higher than that for cis-permethrin. Taken together, the difference in metabolic activity between cis- and trans-permethrin might contribute to the difference in the reproductive toxicity between both isomers. [\hyperlink{Permethrin}{PMID: 18455858}, Shu-Yun Zhang et al., 2008]

\hypertarget{pmid_28827252}{C}eftriaxone is widely used in children in the treatment of sepsis. However, concerns have been raised about the safety of ceftriaxone, especially in young children. The aim of this review is to systematically evaluate the safety of ceftriaxone in children of all age groups. MEDLINE, PubMed, Cochrane Central Register of Controlled Trials, EMBASE, CINAHL, International Pharmaceutical Abstracts and adverse drug reaction (ADR) monitoring systems will be systematically searched for randomised controlled trials (RCTs), cohort studies, case-control studies, cross-sectional studies, case series and case reports evaluating the safety of ceftriaxone in children. The Cochrane risk of bias tool, Newcastle-Ottawa and quality assessment tools developed by the National Institutes of Health will be used for quality assessment. Meta-analysis of the incidence of ADRs from RCTs and prospective studies will be done. Subgroup analyses will be performed for age and dosage regimen. Formal ethical approval is not required as no primary data are collected. This systematic review will be disseminated through a peer-reviewed publication and at conference meetings. CRD42017055428. [\hyperlink{Permethrin}{PMID: 28827252}, Linan Zeng et al., 2017]

\hypertarget{pmid_31601439}{P}ermethrin is a synthetic pyrethroid used as a chemical insecticide that obtained an MA in the management of human scabies in 2014. We report a case of severe immediate hypersensitivity (IH) reaction with generalized contact urticaria secondary to the cutaneous application of 5\% permethrin cream (Topiscab®). A 44-year-old woman with no personal history of atopy was treated with oral ivermectin, Topiscab® and levocetirizine for suspected scabies. Eight hours after taking a levocetirizine tablet and five hours after the application of a tube of Topiscab® together with oral ivermectin, she presented generalized urticaria, nausea and diarrhoea, followed by loss of consciousness. Skin prick-tests for ivermectin and levocetirizine were negative. We noticed non-significant erythema with permethrin. The open application test with Topiscab® was strongly positive at 20min with the appearance of an urticaria plaque in the area of application. The open test with sorbic acid was positive at 2h. Accidental exposure to permethrin spray caused dyspnoea and recurrence of urticaria. Mild and transient symptoms are regularly described after cutaneous application (burning, paraesthesia or increased itching). Delayed hypersensitivity reactions such as contact dermatitis have been reported in the literature. Exceptional cases of severe IH reactions have been described following occupational exposure to airborne pyrethroid insecticides. No cases of severe IH reaction secondary to application of topical permethrin have been reported. [\hyperlink{Permethrin}{PMID: 31601439}, L Jelti et al., 2019]

\hypertarget{pmid_30926895}{P}esticides applied indoors may persist longer than they would in outdoor environments, making people more vulnerable to the risk of exposure. Permethrin is a pyrethroid insecticide used in agricultural, residential, and public health sites, and is commonly detected in indoor environments. The objectives of this study were to evaluate the persistence of permethrin indoors and to estimate the levels of possible dermal and non-dietary exposure to this insecticide. Permethrin was applied on aluminum foil and kept in a glass chamber and a test house for 112 days; its concentration was measured at application and after 28, 56, and 112 days. Permethrin persisted for the entire 112 days in concentrations equal to a maximum of 89.6\% of the initial concentration. We observed low levels of human dermal and non-dietary exposure to permethrin. [\hyperlink{Permethrin}{PMID: 30926895}, Lia Emi Nakagawa et al., 2020]

\hypertarget{pmid_25912032}{P}yrethroids are commonly used insecticides that are considered to pose little risk to human health. However, there is an increasing concern that children are more susceptible to the adverse effects of pesticides. We used the zebrafish model to test the hypothesis that developmental exposure to low doses of the pyrethroid deltamethrin results in persistent alterations in dopaminergic gene expression, neurochemistry, and locomotor activity. Zebrafish embryos were treated with deltamethrin (0.25-0.50 μg/l), at concentrations below the LOAEL, during the embryonic period [3-72 h postfertilization (hpf)], after which transferred to fresh water until the larval stage (2-weeks postfertilization). Deltamethrin exposure resulted in decreased transcript levels of the D1 dopamine (DA) receptor (drd1) and increased levels of tyrosine hydroxylase at 72 hpf. The reduction in drd1 transcripts persisted to the larval stage and was associated with decreased D2 dopamine receptor transcripts. Larval fish, exposed developmentally to deltamethrin, had increased levels of homovanillic acid, a DA metabolite. Since the DA system is involved in locomotor activity, we measured the swim activity of larval fish following a transition to darkness. Developmental exposure to deltamethrin significantly increased larval swim activity which was attenuated by concomitant knockdown of the DA transporter. Acute exposure to methylphenidate, a DA transporter inhibitor, increased swim activity in control larva, while reducing swim activity in larva developmentally exposed to deltamethrin. Developmental exposure to deltamethrin causes locomotor deficits in larval zebrafish, which is likely mediated by dopaminergic dysfunction. This highlights the need to understand the persistent effects of low-dose neurotoxicant exposure during development.  [\hyperlink{Permethrin}{PMID: 25912032}, Tiffany S Kung et al., 2015] Meperfluthrin is a novel sanitary cyhalothrin insecticide invented in China and has increasingly been used to produce liquid mosquito repellents. Oral meperfluthrin poisoning in human has rarely been reported. Here, we reported a case of meperfluthrin poisoning by ingestion of a meperfluthrin-based liquid mosquito repellent in a 16-month-old infant. A 16-month-old boy with a history of accident ingestion of meperfluthrin was admitted to our hospital's emergency department. He exhibited severe dyspnea, and lung radiograph showed multiple patchy and cord-like high-density shadows bilaterally in a short time. He also suffered 35 min of seizures which were finally controlled by the intravenous infusion of propofol. He was diagnosed with meperfluthrin poisoning, status epilepticus and severe pneumonia. After treated with methylprednisolone, aerosolized beclomethasone dipropionate, anti-infection, and some critical supportive therapy, the patient was in good health and showed no symptoms during 12 months of follow-up. Meperfluthrin poisoning is rare. Oral meperfluthrin poisoning shows neurotoxic effects and pulmonary toxicity. Controlling seizures rapidly and ensuring an adequate oxygen supply are critical to the successful treatment. [\hyperlink{Permethrin}{PMID: 25912032}, Shengkun Zheng et al., 2021]

\hypertarget{pmid_30696903}{Y}oung children are the population most severely affected by Plasmodium falciparum malaria. Seasonal malaria chemoprevention (SMC) with amodiaquine and sulfadoxine-pyrimethamine provides substantial benefit to this vulnerable population, but resistance to the drugs will develop. Here, we evaluate the use of dihydroartemisinin-piperaquine as an alternative regimen in 179 children (aged 2.33-58.1 months). Allometrically scaled body weight on pharmacokinetic parameters of piperaquine result in lower drug exposures in small children after a standard mg per kg dosage. A covariate-free sigmoidal E [\hyperlink{Permethrin}{PMID: 30696903}, Palang Chotsiri et al., 2019] Efficacy and safety data of scabies treatments in infants are limited. Although topical permethrin is used in the treatment of scabies in adults, it is not approved for use in infants younger than 2 months of age in many parts of the world. This study aimed to describe treatment practices in the management of scabies in infants younger than 2 months. An online survey was developed and distributed to physicians worldwide through the Society of Pediatric Dermatology and the Pediatric Dermatology Research Alliance. Data collected included demographics, medication availability, experience using medications, deterrents to medication use, medication administration preferences, perceived and experienced medication side effects, and preferred treatment agent in this population. In total, 57 physicians from seven countries responded. The majority of respondents were board-certified in pediatric dermatology (48/57, 84.2\%) and resided in the United States (44/57, 77.2\%). Respondents had experience using permethrin (47/57, 82.5\%) and precipitated sulfur (35/57, 61.4\%) most frequently. Most (38/57, 66.7\%) preferred permethrin as their treatment of choice. Among those who did not use permethrin, potential side effects (8/10, 80\%) were most frequently reported as a deterrent from its use. However, only 4.3\% (22/47) of those who used permethrin reported side effects, including itching, erythema, and xerosis. Permethrin is frequently used in the treatment of infants younger than 2 months with scabies. Furthermore, our results demonstrate that permethrin is the preferred treatment agent among sampled dermatologists for infants younger than 2 months. Few side effects were reported, and none were serious. [\hyperlink{Permethrin}{PMID: 30696903}, Cristina Thomas et al., 2021]

\hypertarget{pmid_12856055}{A}ntipyretics, including acetaminophen (paracetamol), are prescribed commonly in children with pyrexia, despite minimal evidence of a clinical benefit. A literature review was performed by searching Medline and the Cochrane databases for research papers on the efficacy of paracetamol in febrile illnesses in children and adverse outcomes related to the use of paracetamol. No studies showed any clear benefit for the use of paracetamol in therapeutic doses in febrile children with viral or bacterial infections or with malaria. Some studies suggested that fever may have a beneficial role in infection, although no definitive prospective studies in children have been done to prove this. The use of paracetamol in therapeutic doses generally is safe, although hepatotoxicity has occurred with recommended dosages in children. In developing countries where malnutrition is common, data on the safety of paracetamol are lacking. The cost of paracetamol for poor families is substantial. No evidence shows that it is beneficial to treat febrile children with paracetamol. Treatment should be given only to children who are in obvious discomfort and those with conditions known to be painful. The role of paracetamol in children with severe malaria or sepsis and in malnourished, febrile children needs to be clarified. [\hyperlink{Permethrin}{PMID: 12856055}, Fiona M Russell et al., 2003]

\hypertarget{pmid_15947923}{P}ermethrin is an insecticide used in the treatment of lice and scabies infections. Although its efficacy and safety have been well documented, pharmacokinetic data are sparse. The objective of this study was to determine the systemic exposure of permethrin and the duration of residence in the human body following topical administration. The study consisted of three parts. In six young healthy men (part 1), 50 ml of an ethanolic solution containing 215 mg permethrin (cis/trans: 25/75) was administered to the hair of the head. In another six young healthy men (part 2) and in six male or female scabies patients (part 3), 60 g of cream containing 3 g permethrin was administered to the skin of the whole body. Urine was collected up to 168 h post-dose. Urinary excretion of the main metabolite of permethrin, 3-(2,2-dichlorovinyl)-2,2-dimethylcyclopropane carboxylic acid, and its conjugates was measured using a gas chromatography/electron capture detection method. Pharmacokinetics were similar in all study parts. The time of maximal urinary excretion rate was 12.3, 20.0 and 14.6 h, terminal elimination half-life was 32.7, 28.8 and 37.8 h and urinary recovery of the metabolite reached 0.35, 0.47 and 0.52 M percent of the permethrin dose, respectively, in parts 1, 2 and 3 (means). The treatment was well tolerated. The extent of systemic exposure following external therapeutic administration of permethrin is very low compared with doses used for preclinical toxicity studies, and elimination is virtually complete after 1 week. These data provide the pharmacokinetic basis for the clinical safety of topical permethrin. [\hyperlink{Permethrin}{PMID: 15947923}, D Tomalik-Scharte et al., 2005]

\hypertarget{pmid_31808824}{P}ermethrin exposure of children and adults is widespread in many populations, but knowledge of its relative toxicokinetics (TK) and health risks in immature age groups is lacking. Studies were conducted in rats to determine the influence of immaturity and sex (on plasma and target organ dosimetry of each of the insecticide's 2 isomers, cis- and trans-permethrin [CIS and TRANS]). Postnatal day 15, 21, and 90 (adult), Sprague Dawley rats were orally administered a graduated series of doses of CIS and TRANS in corn oil. Serial sacrifices were conducted over 24 h to obtain plasma, brain, liver, skeletal muscle, and fat profiles of CIS and TRANS. Levels of TRANS decreased relatively rapidly, despite administration of relatively high doses. Concentrations of each isomer in plasma, brain, and other tissues monitored were inversely proportional to the animals' age. The youngest pups exhibited 4-fold higher plasma and brain area under the curves than did adults. Little difference was observed in the TK of CIS or TRANS between adult male and female rats, other than higher initial plasma and liver CIS levels in females. Elevated exposure of the immature brain appears to be instrumental in increased susceptibility to the acute neurotoxicity of high-dose permethrin (Cantalamessa [1993]), but it remains to be established whether age-dependent TK is relevant to long-term, low-level risks. [\hyperlink{Permethrin}{PMID: 31808824}, Jing Pang et al., 2020]

\section*{Luliconazole}
\subsection*{Result}
\subsubsection*{Answer}

Unknown (0–11 years)
Unknown (12–17 years)

\subsubsection*{{Explanation}}
\hypertarget{Luliconazole}
A review of the available abstracts reveals the following regarding the safety of luliconazole in children:

- Several clinical trials and reviews of luliconazole for dermatophytosis, tinea pedis, tinea cruris, and onychomycosis are present. However, most studies either do not specify the inclusion of children or only include patients aged 12 years and older.
- For example, two phase 3 studies evaluating luliconazole cream 1\% for tinea pedis and tinea cruris included patients aged ≥12 years [\hyperlink{pmid_25007368}{PMID: 25007368}, Jarratt et al., 2014; \hyperlink{pmid_24385117}{PMID: 24385117}, Jones et al., 2014]. Both studies found luliconazole to be well tolerated and effective, with safety profiles similar to vehicle. However, these studies do not provide data for children under 12 years of age.
- Another phase 3 study for tinea pedis also included patients ≥12 years and found luliconazole to be safe and well tolerated [\hyperlink{pmid_25371767}{PMID: 25371767}, Draelos et al., 2014].
- No abstracts report targeted safety studies of luliconazole in children under 12 years of age.
- No abstracts report targeted safety studies in children aged 12–17 years, but the inclusion of patients ≥12 years in the above studies means some adolescents were studied. However, the abstracts do not provide age-stratified safety data, so it is not possible to definitively affirm safety specifically for the 12–17 age group.
- No abstracts report studies showing luliconazole is unsafe in any pediatric age group.

In summary, based on the abstracts available, there are no targeted safety studies of luliconazole in children under 12 years. For children aged 12 years and older, luliconazole has been studied in clinical trials, but the abstracts do not provide age-specific safety data, so definitive safety in this group cannot be confirmed solely from the abstracts.

\subsection*{Abstracts}
\hypertarget{pmid_22391525}{L}uliconazole is a novel topical antifungal imidazole with broad-spectrum and potent antifungal activity. The drug is under clinical development in the United States for management of dermatophytosis with a short-term treatment regimen. The present study was undertaken to investigate the clinical benefit of short-term therapy with luliconazole cream in guinea pig models of tinea corporis and tinea pedis induced with Trichophyton mentagrophytes. The dose-dependent therapeutic efficacy of topical luliconazole cream (0.02 to 1\%), measured by macroscopic improvement of skin lesions and by fungal eradication as determined by a culture assay, was demonstrated using a tinea corporis model. The improvement in skin lesions seen with luliconazole cream was observed even at a concentration of 0.02\%, and its efficacy at 0.1\% was equal to that of 1\% bifonazole cream. The efficacy of short-term therapy with 1\% luliconazole cream, which is used for clinical management, was investigated using the tinea corporis model (4- and 8-day treatment regimens) and the tinea pedis model (7- and 14-day treatment regimens). The 1\% luliconazole cream completely eradicated the fungus in half or less of the treatment time required for 1\% terbinafine cream and 1\% bifonazole cream, as determined by a culture assay for both models. These results clearly indicate that 1\% luliconazole cream is sufficiently potent for short-term treatment for dermatophytosis compared to existing drugs. Luliconazole is expected to be useful in the clinical management of dermatophytosis. [\hyperlink{Luliconazole}{PMID: 22391525}, Hiroyasu Koga et al., 2012]

\hypertarget{pmid_25007368}{I}nterdigital tinea pedis is one of the most common clinical presentations of dermatophytosis. This phase 3 study evaluated the safety and efficacy of luliconazole cream 1\% in patients with tinea pedis. A total of 321 male and female patients aged ≥12 years with tinea pedis and eligible for modified intent-to-treat analysis were randomized 1:1 to receive luliconazole cream 1\% (n=159) or vehicle (n=162) once daily for 14 days. Efficacy was evaluated at days 28 and 42 (i.e., days 14 and 28 posttreatment) based on clinical signs (erythema, scaling, pruritus) and mycology (KOH, fungal culture). The primary outcome was complete clearance at day 42. Safety evaluations included adverse events and laboratory assessments. Complete clearance at day 42 was achieved in 26.4\% (28/106) of patients treated with luliconazole cream 1\% compared with 1.9\% (2/103) of patients treated with vehicle (P< 0.001). Similar safety profiles were obtained for luliconazole cream 1\% and vehicle. This study was conducted in a relatively small population under controlled clinical trial conditions. Luliconazole cream 1\% applied once daily for 14 days is well tolerated and more effective than vehicle in patients with tinea pedis. [\hyperlink{Luliconazole}{PMID: 25007368}, Michael Jarratt et al., 2014]

\hypertarget{pmid_10428919}{T}he safety profile of fluconazole was assessed for 562 children (ages, 0 to 17 years) comprising 323 males and 239 females. The data are derived from 12 clinical studies of fluconazole as prophylaxis or treatment for a variety of fungal infections in predominantly immunocompromised patients. Most children received multiple doses of fluconazole in the range of 1 to 12 mg/kg of body weight; a few received single doses. Administration was mainly by oral suspension or intravenous injection. Overall, 58 (10.3\%) children reported 80 treatment-related side effects. The most common side effects were associated with the gastrointestinal tract (7.7\%) or skin (1.2\%). Self-limiting, treatment-related side effects affecting the liver and biliary system were reported in three patients (0.5\%). Overall, 18 patients (3.2\%) discontinued treatment due to side effects, mainly gastrointestinal symptoms. Dose and age did not appear to influence the incidence and pattern of side effects. Treatment-related laboratory abnormalities were uncommon, the most frequent being transient elevated alanine aminotransferase (4.9\%), aspartate aminotransferase (2.7\%), and alkaline phosphatase (2.3\%) levels. Although 98.6\% of patients were taking concomitant medications, no clinical or laboratory interactions were observed. The safety profile of fluconazole was compared with those of other antifungal agents, mostly oral polyenes, by using a subset of data from five controlled studies. Side effects were reported by more patients treated with fluconazole (45 of 382; 11.8\%) than by those patients treated with comparable agents (25 of 381; 6.6\%); vomiting and diarrhea were the most common events in both groups. The incidence and type of treatment-related laboratory abnormalities were similar for the two groups. In conclusion, fluconazole was well tolerated by the pediatric population, many of whom were suffering from severe underlying disease and were taking a variety of concurrent medications. The safety profile of fluconazole in children mirrors the excellent safety profile seen in adults. [\hyperlink{Luliconazole}{PMID: 10428919}, V Novelli et al., 1999]

\hypertarget{pmid_25371767}{T}o evaluate the efficacy and safety of luliconazole cream 1\% applied once daily for 14 days in patients with interdigital tinea pedis. Multicenter, randomized, double-blind, parallel-group, vehicle-controlled study. Private dermatology clinics and clinical research centers in the United States and Central America. Three hundred twenty-two male and female patients ≥12 years of age diagnosed with interdigital tinea pedis. Complete clearance (i.e., clinical and mycological cure), effective treatment, and fungal culture and susceptibility. At study Day 42, complete clearance was obtained by a larger percentage (14.0\% [15/107] vs. 2.8\% [3/107]; p<0.001) of patients treated with luliconazole cream 1\% compared with vehicle. Also at Day 42, more luliconazole-treated patients compared with vehicle-treated patients obtained effective treatment (32.7\% vs. 15.0\%), clinical cure (15.0\% vs. 3.7\%), and mycologic cure (56.1\% vs. 27.1\%). Erythema, scaling, and pruritus scores were lower for the luliconazole cream 1\% group compared with vehicle on Day 14, Day 28, and Day 42. For all species and the same isolates, the MIC50/90 for luliconazole cream 1\% was 6- to 12-fold lower than for other agents tested. No patients discontinued treatment because of a treatment-emergent adverse event. Luliconazole cream 1\% was safe and well-tolerated and demonstrated significantly greater efficacy than vehicle cream in patients with interdigital tinea pedis. [\hyperlink{Luliconazole}{PMID: 25371767}, Zoe Diana Draelos et al., 2014]

\hypertarget{pmid_25285056}{L}uliconazole is an imidazole antifungal agent with a unique structure, as the imidazole moiety is incorporated into the ketene dithioacetate structure. Luliconazole is the R-enantiomer, and has more potent antifungal activity than lanoconazole, which is a racemic mixture. In this review, we summarize the in vitro data, animal studies, and clinical trial data relating to the use of topical luliconazole. Preclinical studies have demonstrated excellent activity against dermatophytes. Further, in vitro/in vivo studies have also shown favorable activity against Candida albicans, Malassezia spp., and Aspergillus fumigatus. Luliconazole, although belonging to the azole group, has strong fungicidal activity against Trichophyton spp., similar to that of terbinafine. The strong clinical antifungal activity of luliconazole is possibly attributable to a combination of strong in vitro antifungal activity and favorable pharmacokinetic properties in the skin. Clinical trials have demonstrated its superiority over placebo in dermatophytosis, and its antifungal activity to be at par or even better than that of terbinafine. Application of luliconazole 1\% cream once daily is effective even in short-term use (one week for tinea corporis/cruris and 2 weeks for tinea pedis). A Phase I/IIa study has shown excellent local tolerability and a lack of systemic side effects with use of topical luliconazole solution for onychomycosis. Further studies to evaluate its efficacy in onychomycosis are underway. Luliconazole 1\% cream was approved in Japan in 2005 for the treatment of tinea infections. It has recently been approved by US Food and Drug Administration for the treatment of interdigital tinea pedis, tinea cruris, and tinea corporis. Topical luliconazole has a favorable safety profile, with only mild application site reactions reported occasionally.  [\hyperlink{Luliconazole}{PMID: 25285056}, Deepshikha Khanna et al., 2014] Tinea cruris, a pruritic superficial fungal infection of the groin, is the second most common clinical presentation for dermatophytosis. This phase 3 study evaluated the safety and efficacy of topical luliconazole cream 1\% in patients with tinea cruris. 483 patients were enrolled and 256 male and female patients aged ≥12 years with clinically evident tinea cruris and eligible for modified intent-to-treat analysis were randomized 2:1 to receive luliconazole cream 1\% (n=165) or vehicle (n=91) once daily for 7 days. Efficacy was evaluated at baseline and at days 7, 14, 21, and 28 based on mycology (potassium hydroxide, fungal culture) and clinical signs (erythema, scaling, pruritus). The primary outcome was complete clearance at day 28 (21 days posttreatment). Safety evaluations included adverse events and laboratory assessments. Complete clearance was obtained in 21.2\% (35/165) of patients treated with luliconazole cream 1\% compared with 4.4\% (4/91) treated with vehicle (P<0.001). The safety profile of luliconazole cream 1\% was similar to vehicle. The study was conducted under controlled conditions in a relatively small population. Luliconazole cream 1\% applied once daily for 7 days is more effective than vehicle and well tolerated in patients with tinea cruris. [\hyperlink{Luliconazole}{PMID: 25285056}, Terry M Jones et al., 2014]

\hypertarget{pmid_717773}{B}lood levels of lignocaine and bupivacaine were measured in children following caudal, subcutaneous and tracheal administration. The highest peak levels were in children under 3 years following tracheal spray but all blood levels were below accepted toxic adult levels for anaesthetised patients. No toxic manifestations were seen. [\hyperlink{Luliconazole}{PMID: 717773}, R L Eyres et al., 1978]

\hypertarget{pmid_28332720}{O}nychomycosis is a highly prevalent and intractable disease. The first-line treatment agents are oral preparations, but an effective topical medication has long been desired. The objective was to investigate the efficacy and safety of luliconazole 5\% nail solution, an imidazole antifungal agent, for the treatment of patients with onychomycosis. A multicenter, double-blind, randomized phase III study was conducted in Japanese patients with distal lateral subungual onychomycosis affecting the great toenails, with 20-50\% clinical involvement. Patients were randomized (2:1) to luliconazole or vehicle once daily for 48 weeks. The primary end-point was the complete cure rate (clinical cure [0\% clinical involvement of the nail] plus mycological cure [negative results on direct microscopy]). The adverse event incidence was monitored to evaluate safety. The complete cure rate significantly favored luliconazole (14.9\%, 29/194 subjects) versus vehicle (5.1\%, 5/99) (P = 0.012). Similarly, the negative direct microscopy rate was significantly higher with luliconazole (45.4\%, 79/174) than with vehicle (31.2\%, 29/93) (P = 0.026). There were no serious adverse drug reactions. We conclude that once daily topical luliconazole 5\% nail solution demonstrated clinical efficacy and was confirmed to be well tolerated. [\hyperlink{Luliconazole}{PMID: 28332720}, Shinichi Watanabe et al., 2017]

\hypertarget{pmid_29077183}{I}t is common practice to prepare the nasal mucosa with decongestant in children undergoing lacrimal surgery. Xylometazoline 0.05\% (Otrivine) nasal spray is commonly used. It has been reported to cause cardiovascular side effects. In the absence of formal guidelines on the safety of the use of nasal decongestants in children, we reviewed our practice to answer the question: How safe is preoperative use of xylometazoline in children undergoing lacrimal surgery? To our knowledge, this is the first study to address the potential side effects of the use of xylometazoline preoperatively in children undergoing lacrimal surgery. This was a retrospective analysis of medical notes of children undergoing lacrimal surgery with the use of preoperative intranasal xylometazoline 0.05\% over a 5-year period. Twenty-nine children, age 1-6 years (mean 3 years), underwent lacrimal surgery under general anesthesia with preoperative use of intranasal xylometazoline. Topical intranasal 1:10,000 adrenaline was used during surgery in all patients. All children were found to have uneventful surgery and recovery from anesthesia. Xylometazoline 0.05\% intranasal use for prelacrimal surgery was found to be effective and safe. Addition of sympathomimetic topical adrenaline (1:10,000) did not impose any risks. The type of general anesthesia may influence the cardiovascular side effects anecdotally recorded during xylometazoline use. [\hyperlink{Luliconazole}{PMID: 29077183}, Varajini Joganathan et al., 2018]

\hypertarget{pmid_32621534}{F}luconazole is one of the most commonly used drugs for antifungal prophylaxis in childhood leukaemia. However, its interaction with vincristine may induce neuropathy and the emergence of antifungal drug resistance contributes to substantial mortality caused by invasive fungal infections (IFIs). In a retrospective single-centre study, we compared tolerability and outcome of different antifungal prophylaxis strategies in 198 children with acute leukaemia (median age 5·3 years). Until 2010, antifungal prophylaxis with fluconazole was offered to most of the patients and thereafter was replaced by liposomal amphotericin-B (L-AMB) and restricted to high-risk patients only. Vincristine-induced neurotoxicity was significantly reduced under L-AMB, as the percentage of patients with severe constipation decreased (15·4\% vs. 3·7\%, before vs. after 31 December·2010, P = 0·01) and stool frequency increased by up to 38\% in polyene-treated patients (P = 0·005). Before 2011, 10 patients developed confirmed IFIs, most of them were infected with Aspergillus species. After risk adaption in 2011, IFIs were completely prevented (P = 0·007). L-AMB prophylaxis is beneficial in childhood leukaemia patients, as it offers effective antifungal activity with improved tolerability as compared to fluconazole. The potential impact of our risk-adapted antifungal treatment should be included in current prophylaxis guidelines for childhood leukaemia. [\hyperlink{Luliconazole}{PMID: 32621534}, Andreas Meryk et al., 2020]

\hypertarget{pmid_36057744}{L}uliconazole, recently launched in Japan, is a novel topical imidazole antifungal agent for the treatment of onychomycosis. Using in vitro onychomycosis model, the effect of luliconazole on the morphology of the growing hyphae of Trichophyton mentagrophytes was investigated by scanning electron microscopy (SEM). The model was produced by placing human nail pieces on an agar medium seeded with conidia of T. mentagrophytes. After incubating the agar medium for 3 days, luliconazole was applied to the surface of the nail in which hyphal growth was recognized, then cultured for up to 24 h. The initial change after treatment with the drug was the formation of fine wrinkles on the surface of the hyphae, eventually, the hyphae were flattened, and after that, no hyphal growth was observed. On the other hand, when the nails were pretreated with luliconazole for 1 h, no hyphal growth was observed even after culturing for 24 h. This study suggests that luliconazole has a strong antifungal activity by inhibiting the ability of fungi to grow and the drug has both excellent nail permeation and retention properties. [\hyperlink{Luliconazole}{PMID: 36057744}, Yayoi Nishiyama et al., 2022]

\hypertarget{pmid_17302746}{L}uliconazole is a newly developed imidazolyl antifungal agent. A randomised double-blind comparative study was designed to assess the efficacy and safety of 1\% luliconazole cream (group A), 0.5\% cream (group B) and 0.1\% cream (group C), in tinea pedis (interdigital type and plantar type), when used once daily for 2 weeks. Follow-ups were performed at 4 weeks after the end of topical treatment. A total of 241 patients were enrolled and 213 patients were evaluated for efficacy. Rates of improvement of skin lesions in the A, B and C groups assessed at week 4 were 90.5\%, 91.0\% and 95.8\%, respectively. Rates of mycological cure (negative result of microscopy) in the A, B and C groups assessed at week 4 were 79.7\%, 76.1\%, 72.2\% and at week 6 (at 4 weeks after the end of topical treatment) were 87.7\%, 94\%, 88.9\%, respectively. For the mycological effect on tinea pedis of the interdigital type at 2 weeks, the negative conversion of fungi showed a concentration-dependent relationship and indicated a difference in tendency statistically 81.1\% (1\%- treatment), 62.9\% (0.5\%- treatment), 58.3\% (0.1\%- treatment) (Fisher's exact test, P = 0.079) and there was a trend between three groups by Cochran-Mantel-Haenszel method (P = 0.038). The incidence of adverse events in which a causal relationship to this drug could not be ruled out was low (2.6\%). All of the adverse events were mild in severity and insignificant clinically. [\hyperlink{Luliconazole}{PMID: 17302746}, Shinichi Watanabe et al., 2007]

\hypertarget{pmid_32494352}{L}uliconazole is currently confirmed for the topical therapy of dermatophytosis. Moreover, it is found that luliconazole has  Thirty eight isolates of  Luliconazole with extremely low MIC range, 0.00049-0.00781 μg/mL and MIC The analysis of our data clearly indicated that luliconazole (with MIC [\hyperlink{Luliconazole}{PMID: 32494352}, Maryam Moslem et al., 2020] Because of the low prevalence of onychomycosis in children, little is known about the efficacy and safety of systemic antifungals in this population. PubMed and Embase databases and the references of related publications were searched in March 2012 for clinical trials (CTs), retrospective analyses (RAs), and case reports (CRs) on the use of systemic antifungals for onychomycosis in children (<18 years). Twenty-six studies (5 CTs, 3 RAs, and 18 CRs) were published between 1976 and 2011. Most of these studies reported the use of systemic terbinafine and itraconazole for the treatment of onychomycosis in children. Therapy with systemic antifungals alone in children ages 1 to 17 years resulted in a complete cure rate of 70.8\% (n = 151), whereas combined systemic and topical antifungal therapy in one infant and 19 children age 8 and older resulted in a complete cure rate of 80.0\% (n = 20). The efficacy and safety profiles of terbinafine, itraconazole, griseofulvin, and fluconazole in children were similar to those previously reported for adults. In conclusion, based on the little information available on onychomycosis in children, systemic antifungal therapies in children are safe and cure rates are similar to the rates achieved in adults. [\hyperlink{Luliconazole}{PMID: 32494352}, Aditya K Gupta et al., ]

\hypertarget{pmid_32253502}{L}uliconazole is a new antifungal that was primarily used for the treatment of dermatophytosis. However, some studies have shown that it has excellent efficacy against Aspergillus and Candida species in vitro. The present study aimed to evaluate of luliconazole activity against some Fusarium species complex isolates. In this study, 47 isolates of Fusarium were tested against several antifungals including luliconazole. All species were identified using morphology features, and PCR sequencing and antifungal susceptibility were performed according to CLSIM38 A3 guideline. Our results revealed that luliconazole has a very low minimum inhibitory concentration value (0.0078-1 µg/ml) in comparison with other tested antifungals. Amphotericin B had a poor effect with a high MIC [\hyperlink{Luliconazole}{PMID: 32253502}, Maral Gharaghani et al., 2020] Luliconazole is an inhibitor for sterol 14-α-demethylase in fungal cells with a broad-spectrum antifungal activity against dermatophytes, Candida albicans, Malassezia species, dematiaceous and hyaline hyphomycetes. Furthermore, luliconazole has been clinically used for the treatment of pityriasis versicolor, dermatophytosis, onychomycosis, cutaneous and mucocutaneous candidiasis. In the present study, we aimed to evaluate in vitro antifungal activity of luliconazole against several strains of Candida species recovered from different clinical materials. In the present study, 104 strains of Candida species including, 34 isolates from vaginitis, 23 isolates from AIDS patients with vaginal candidiasis, 24 isolates from neutropenic patients and 24 isolates from tracheal tubes, were examined for susceptibility tests. A serial dilution of luliconazole (4-0.008μg/mL) was tested against different strains of Candida species recovered from different sources. The minimum inhibitory concentration (MIC) range and MIC It is concluded that, luliconazole could be an alternative anti-Candida agent, however, in vivo studies must be confirmed usefulness of drug for clinical usage. [\hyperlink{Luliconazole}{PMID: 32253502}, S Taghipour et al., 2018]

\hypertarget{pmid_23545529}{T}he study objective was to evaluate the safety, tolerability, systemic exposure, and pharmacokinetics (PK) of 10\% luliconazole solution (luliconazole) when topically applied once daily to all 10 toenails and periungual areas in patients with moderate to severe distal subungual onychomycosis. In this single-center, open-label study, 24 patients applied 20 mg/ml of luliconazole (twice the clinical dose) for 29 days with a 7-day follow-up. Complete PK profiles were determined on days 1, 8, 15, and 29. Safety/tolerability assessments included application site reactions, adverse events, vital signs, clinical laboratory findings, and electrocardiograms. Mean luliconazole plasma concentrations remained around the lower limit of quantitation (0.05 ng/ml) and were comparable on days 8, 15, and 29 (range, 0.063 to 0.090 ng/ml), suggesting steady state occurred by day 8. Every patient had undetectable plasma luliconazole levels for at least 11\% of the time points, and 12 of the 24 patients had undetectable levels for at least 70\% of the time points. The maximum plasma concentration of luliconazole (C(max)) observed in any patient was 0.314 ng/ml and the maximum area under the concentration-time curve from 0 to 24 h (AUC0-24) was 4.34 ng · h/ml. Five patients (21\%) had measureable luliconazole levels in the plasma 7 days after the last dose. The median concentration of luliconazole in the nail at this time point was 34.65 mg/g (from 42 of 48 collected toenail samples). There was one mild incidence of skin erythema on day 5 that resolved on day 8, there were no reports of drug-induced systemic side effects, and there was no evidence of QT prolongation. Luliconazole, when applied once daily to all 10 fungus-infected toenails for 29 days, is generally safe and well tolerated and results in significant accumulation of drug in the nail. Systemic exposure is very low, with no evidence of drug accumulation. [\hyperlink{Luliconazole}{PMID: 23545529}, T Jones et al., 2013]

\hypertarget{pmid_26559086}{L}uliconazole is an imidazole antifungal agent with a unique chemical structure. In this article, we summarize the in vitro data, animal studies and clinical trial data relating to the use of topical luliconazole cream 1\% in the treatment of tinea pedis. Preclinical studies have demonstrated potent activity against dermatophytes. Luliconazole has strong fungicidal activity against Trichophyton spp., similar to that seen with terbinafine. Evidence from clinical trials in tinea pedis have shown once-daily application of luliconazole cream 1\% for 14 days to be effective and well tolerated. [\hyperlink{Luliconazole}{PMID: 26559086}, Michael H Gold et al., 2015]

\hypertarget{pmid_33309940}{F}ungal infections are an important cause of morbidity and pose a serious health concern especially in immunocompromised patients. Luliconazole (LUL) is a topical imidazole antifungal drug with a broad spectrum of activity. To overcome the limitations of conventional dosage forms, LUL loaded lyotropic liquid crystalline nanoparticles (LCNP) were formulated and characterized using a three-factor, five-level Central Composite Design of Response Surface Methodology. LUL loaded LCNP showed particle size of 181 ± 12.3 nm with an entrapment efficiency of 91.49 ± 1.61 \%. The LUL-LCNP dispersion in-vitro drug release showed extended release up to 54 h. Ex-vivo skin permeation studies revealed transdermal flux value (J) of LUL-LCNP gel (7.582 μg/h/cm [\hyperlink{Luliconazole}{PMID: 33309940}, Arisha Mahmood et al., 2021] We evaluated the efficacy and safety of luliconazole cream 1\% in the treatment of dermatophytoses. According to our meta-analysis, short-term treatment of luliconazole cream 1\% can result in the complete clearance of dermatophytoses. It showed that 1\% luliconazole was more effective than controlled drugs or vehicle (week 4: odds ratio = 1.46, 95\% confidence interval = 1.12-1.91), and no more adverse events occurred in the 1\% luliconazole group (week 4: odds ratio = 1.01, 95\% confidence interval = 0.71-1.44). This effect strengthens the evidence for luliconazole cream 1\% being more effective than vehicle, 1\% terbinafine, 1\% bifonazole, and 0.1\% or 0.5\% luliconazole.  [\hyperlink{Luliconazole}{PMID: 33309940}, Xiaowei Feng et al., 2014] Luliconazole (LCZ) is an imidazole antifungal medication that exhibits excellent activity against dermatophytes. As a topical cream and lotion (approved for human use), LCZ has demonstrated a broad spectrum of activity against human dermatophytoses. This is the first study to investigate the in vitro susceptibility of clinical isolates from horse dermatophytoses to LCZ. No animals were used in this study. In the present study, the in vitro susceptibilities of clinical isolates of dermatophytes to LCZ, clotrimazole (CTZ), miconazole (MCZ) and terbinafine (TRF) were investigated using the Clinical \& Laboratory Standards Institute M38-A2 test. The minimum inhibitory concentrations (MICs) for all 16 clinical isolates of Trichophyton equinum, Microsporum equinum/canis and M. gypseum for LCZ were <0.03 mg/L. The MICs of all isolates were <0.03-0.5 mg/L for CTZ, 0.03-16 mg/L for MCZ and <0.03-1 mg/L for TRF. LCZ demonstrated a broad spectrum of activity against clinical isolates from horse dermatophytoses. We consider that LCZ will become the primary antifungal agent for treating horse dermatophytosis. [\hyperlink{Luliconazole}{PMID: 33309940}, Ryousuke Watanabe et al., 2021]

\hypertarget{pmid_36850286}{L}uliconazole is a broad-spectrum topical antifungal agent that acts by altering the synthesis of fungi cell membranes. Literature suggests that the recurrence of fungal infection can be avoided by altering the pH of the site of infection. Studies have also suggested that fungi thrive by altering skin pH to be slightly acidic, i.e., pH 3-5. The current study is aimed to design, develop, characterize, and evaluate an alkaline pH-based antifungal spray solution for antifungal effects. Luliconazole was used as an antifungal agent and an alkaline spray was formulated for topical application by using Eudragit RS 100, propylene glycol (PG), water, sodium bicarbonate, and ethanol via solubilization method. Herein, sodium bicarbonate was used as an alkalizing agent. Based on DSC, FTIR, PXRD, scanning electron microscopy (SEM), and rheological analysis outcomes, the drug (luliconazole) and polymer were found to be compatible. F-14 formulation containing 22\% Eudragit RS 100 (ERS), 1.5\% PG, and 0.25\% sodium bicarbonate was optimized by adopting the quality by design approach by using design of experiment software. The viscosity, pH, drying time, volume of solution post spraying, and spray angle were, 14.99 ± 0.21 cp, 8 pH, 60 s, 0.25 mL ± 0.05 mL, and 80 ± 2, respectively. In vitro drug diffusion studies and in vitro antifungal trials against  [\hyperlink{Luliconazole}{PMID: 36850286}, Bhakti Dhimmar et al., 2023] To determine the safety of fluconazole in neonates and other paediatric age groups by identifying adverse events (AEs) and drug interactions associated with treatment. A search of EMBASE (1950-January 2012), MEDLINE (1946-January 2012), the Cochrane database for systematic reviews and the Cumulative Index to Nursing and Allied Health Literature (1982-2012) for any clinical study about fluconazole use that involved at least one paediatric patient (≤17 years) was performed. Only articles with sufficient quality of safety reporting after patients' exposure to fluconazole were included. We identified 90 articles, reporting on 4,209 patients, which met our inclusion criteria. In total, 794 AEs from 35 studies were recorded, with hepatotoxicity accounting for 378 (47.6 \%) of all AEs. When fluconazole was compared with placebo and other antifungals, the relative risk (RR) of hepatotoxicity was not statistically different [RR 1.36, 95 \% confidence interval (CI) 0.87-2.14, P = 0.175 and RR 1.43, 95 \% CI 0.67-3.03, P = 0.352, respectively]. Complete resolution of hepatoxicity was achieved by 84 \% of patients with follow-up available. There was no statistical difference in the risk of gastrointestinal events of fluconazole compared with placebo and other antifungals (RR 0.81, 95 \% CI 0.12-5.60, P = 0.831 and RR 1.23, 95 \%CI 0.87-1.71, P = 0.235, respectively). There were 41 drug withdrawals, 17 (42 \%) of which were due to elevated liver enzymes. Five reports of drug interactions occurred in children. Fluconazole is relatively safe for paediatric patients. Hepatotoxicity and gastrointestinal toxicity are the most common adverse events. It is important to be aware that drug interactions with fluconazole can result in significant toxicity. [\hyperlink{Luliconazole}{PMID: 36850286}, Oluwaseun Egunsola et al., 2013]

\hypertarget{pmid_2245677}{T}o evaluate the safety of topical lidocaine anesthesia in children undergoing bronchoscopy, we determined SLC in 15 children aged 3 months to 9.5 years during flexible fiberoptic bronchoscopy. A total lidocaine dose of 3.2 to 8.5 (mean +/- SEM = 5.7 +/- 0.5) mg/kg was administered to nose, larynx and bronchial tree over 9 to 45 (mean +/- SEM = 20 +/- 2.7) minutes. No complication occurred during the procedure. Peak SLC were 1-3.5 (mean +/- SEM = 2.5 +/- 0.2) micrograms/ml. The Vd beta was 1.79 +/- 0.19 L/kg, the t1/2 beta was 109 +/- 12 minutes, and the total body clearance 12.2 +/- 1.1 ml/min/kg. Peak SLC correlated well with the dose expressed as mg/kg (r = 0.59, p less than 0.025), and even better when related to body surface area (r = 0.63, p less than 0.01). Lidocaine doses up to 8.5 mg/kg proved safe and resulted in therapeutic SLC in our patients. Lidocaine dose up to 7 mg/kg appears to be safe provided that it does not exceed an upper limit of 175 mg/m2 and is gradually administered over a minimum of 15 minutes. Doses of 7-8.5 mg/kg appear to be safe when administered over longer periods. [\hyperlink{Luliconazole}{PMID: 2245677}, Y Amitai et al., 1990]

\hypertarget{pmid_22633503}{F}luconazole is a triazole antifungal agent that is widely used in the nursery. It is available in both intravenous and oral formulation, and is active against most of the fungal pathogens that require treatment when retrieved from culture samples in neonatal intensive care units. Although clinical use has been wide for over 15 years, there have been small safety and efficacy studies completed in young infants. Randomised clinical trials assessing effectiveness of this agent in prevention of systemic fungal infections in neonates have been published in the last decade, and one large additional randomised study has been recently completed. Nevertheless, a certain degree of uncertainty still exists regarding the kinetics and appropriate dosing of this agent in premature and term infants, as well as regarding safety. Areas of poignant debate include the feasibility of loading dose strategies, appropriate dosages in the early days of life in the different subgroups of preterm infants, and long-term safety of fluconazole administered in prophylaxis during the first weeks of life in extremely premature infants. This paper reviews the most recent evidence on fluconazole and its role in the NICU settings. [\hyperlink{Luliconazole}{PMID: 22633503}, E Castagnola et al., 2012]

\section*{Nystatin And Triamcinolone Acetonide}
\subsection*{Result}
\subsubsection*{Answer}

Unknown

\subsubsection*{{Explanation}}
\hypertarget{Nystatin And Triamcinolone Acetonide}
A review of the available abstracts reveals the following evidence regarding the safety of Nystatin and Triamcinolone Acetonide in children:

1. Nystatin:
   - Several studies have evaluated nystatin in infants and children, but none specifically address the combination of nystatin and triamcinolone acetonide.
   - Nystatin alone has been studied for prophylaxis and treatment in infants and children. For example, a study in infants with diaper dermatitis compared nystatin to clotrimazole and found both were safe and well-tolerated in infants (mean age \textasciitilde{}12 months) [\hyperlink{pmid_20553355}{PMID: 20553355}, P H Hoeger et al., 2010]. Another study in very low birthweight infants compared nystatin to fluconazole and found no statistically significant difference in systemic fungal infection rates, but a higher mortality rate in the nystatin group, raising questions about its safety in this specific population [\hyperlink{pmid_19504425}{PMID: 19504425}, Kimon Violaris et al., 2010]. However, these studies do not address the combination product.

2. Triamcinolone Acetonide:
   - Multiple studies have evaluated triamcinolone acetonide in children, primarily as an intranasal spray or for intraarticular/intramuscular use.
   - For intranasal triamcinolone acetonide:
     - In children aged 2–5 years, a randomized controlled trial found that triamcinolone acetonide aqueous nasal spray (110 mcg daily) for up to 6 months had a favorable safety profile, with no significant changes in serum cortisol or growth [\hyperlink{pmid_19441606}{PMID: 19441606}, Steven Weinstein et al., 2009].
     - In children aged 4–12 years, a 12-week double-blind study found both 82.5 mcg and 165 mcg daily doses were safe and well-tolerated [\hyperlink{pmid_1958002}{PMID: 1958002}, M J Welch et al., 1991].
     - In children aged 6–11 years, a 2-week study found 220 mcg daily was well tolerated [\hyperlink{pmid_8733987}{PMID: 8733987}, C H Banov et al.].
     - A 1- and 2-year follow-up study in children aged 6.1–14.3 years found no significant effect on statural growth [\hyperlink{pmid_18939734}{PMID: 18939734}, David P Skoner et al., 2008].
   - For intraarticular/intramuscular triamcinolone acetonide:
     - In children with juvenile idiopathic arthritis (age not specified, but pediatric), intraarticular triamcinolone acetonide was effective and no side effects were reported during the study period [\hyperlink{pmid_19346579}{PMID: 19346579}, Sumit Verma et al., 2009].
     - In children with nephrotic syndrome (median age 8.6 years, range 1.8–10.7), intramuscular triamcinolone acetonide was well tolerated, though growth velocity decreased during treatment and returned to normal after discontinuation [\hyperlink{pmid_15844000}{PMID: 15844000}, Tim Ulinski et al., 2005].
     - However, there are case reports of Cushing's syndrome following intralesional triamcinolone acetonide in children, suggesting potential for serious adverse effects with certain routes or doses [\hyperlink{pmid_7382013}{PMID: 7382013}, R R Augspurger et al., 1980].

3. Nystatin and Triamcinolone Acetonide Combination:
   - None of the abstracts reviewed specifically address the safety of the combination product (Nystatin and Triamcinolone Acetonide) in children. There are no targeted studies evaluating the safety of this combination in any pediatric age group.

Summary:
- There is evidence from targeted studies that triamcinolone acetonide (as a nasal spray or for certain other uses) is safe in children across various age ranges, with some caveats regarding route and duration.
- Nystatin alone has been studied in infants and children, with some evidence of safety, but also some concerns in specific populations.
- Critically, there are no abstracts providing targeted safety data for the combination of Nystatin and Triamcinolone Acetonide in children. Therefore, based on the available abstracts, the safety of this combination in children is unknown.

\subsection*{Abstracts}
\hypertarget{pmid_36610724}{T}he indications for nystatin as prophylaxis or treatment are limited. In the PASOAP (Pediatric Antifungal Stewardship Optimizing Antifungal Prescription) study, high use of nystatin in hospitalized children beyond the neonatal age was observed. In this report, we present the data on nystatin use in infants and children ≥ 3 months who participated in the PASOAP study. Nystatin was prescribed mainly for prophylaxis. Congenital heart disease, cystic fibrosis, and chronic renal disease were the most commonly reported conditions in children receiving prophylactic nystatin. There is sparse evidence supporting the use of nystatin prophylaxis beyond neonates; trials in specific pediatric patient groups are required. [\hyperlink{Nystatin And Triamcinolone Acetonide}{PMID: 36610724}, Harshani Jayawardena-Thabrew et al., 2022]

\hypertarget{pmid_19346579}{T}hirteen children with juvenile idiopathic arthritis (JIA) were treated with intraarticular steroid injection of triamcilone acetonide as a day care procedure. More than half (53.4\%) the children were free of pain, limp and NSAID's use, with improvement in functional score at 12 weeks. No side effects were reported during the period of the study. [\hyperlink{Nystatin And Triamcinolone Acetonide}{PMID: 19346579}, Sumit Verma et al., 2009]

\hypertarget{pmid_35179713}{S}ubtenon triamcinolone acetonide (Kenalog®; Bristol Myers Squibb) (STA) injections are commonly used in the treatment of adults in an outpatient setting. However, publications on detailing its outpatient use, safety, and efficacy in the pediatric population are scarce. We reviewed STA injections performed in children in the outpatient clinics at two tertiary centers from 2014 to 2020. All children were aged ≤ 18 years and had a diagnosis of non-infectious uveitis. STA injections were done using 0.5 cc (20 mg) triamcinolone injected superotemporally with only topical anesthesia. Data on the efficacy and safety of STA in treating inflammation and compiled data on visual acuity improvement and incidence of ocular complications were evaluated. Forty-eight eyes in 30 patients were included. The mean age of patients was 13.1 (range 7-18) years. There were no immediate complications observed in all injections performed. At the 3-month follow-up, inflammation had improved in 85.4\% of eyes, macular edema had resolved in 77.8\% of eyes, and there was significant vision improvement after STA. At 6 months after STA, the incidence of ocular hypertension was 12.5\% and no new cataracts had developed. STA injection with topical anesthesia was a well-tolerated, reasonable alternative for short-term treatment of uveitis among this pediatric population. [\hyperlink{Nystatin And Triamcinolone Acetonide}{PMID: 35179713}, Jennifer L Jung et al., 2022]

\hypertarget{pmid_15844000}{N}oncompliance is frequent in children and adolescents with nephrotic syndrome. Once suspected, noncompliance is difficult to confirm and often impossible to avoid. The standard oral glucocorticoid treatment for children has been shown to be efficient and safe. However, a small number of children/parents are noncompliant to the steroid treatment, resulting in multiple relapses. For these patients the use of steroids with prolonged half-life such as triamcinolone acetonide (TA) can be helpful. We studied seven children (six boys, one girl; median age at diagnosis 8.6 years, range 1.8-10.7) receiving conventional steroid treatment for a median of 30 months (8-74) before starting intramuscular (IM) TA treatment. The standard prednisone treatment was replaced by 1 monthly IM injection of TA (1 mg/kg per day oral prednisone replaced by 1 mg/kg per month IM TA). The treatment was tapered off by a reduction of 10-20\% of the initial dose per month over 6-8 months. After a mean observation period of 14 months (3-36) the results were evaluated in terms of number of relapses and treatment tolerance. Four children showed a clear decrease in number of relapses (1.8 to 0 per year); in the other three the number of relapses remained stable. Tolerance was excellent (no cataract, no arterial hypertension), and the cushingoid syndrome did not exceed the level experienced under conventional oral steroid therapy. However, growth velocity decreased during the TA treatment and returned to normal after discontinuation of TA. These preliminary results demonstrate that TA may be used in patients of suspected noncompliance in steroid-sensitive patients who respond with a complete remission during TA treatment over the observation period. Patients who do not benefit from the TA can be classified as very probably steroid-dependent. TA seems to be a useful therapeutic strategy in those patients for whom noncompliance is strongly suspected. [\hyperlink{Nystatin And Triamcinolone Acetonide}{PMID: 15844000}, Tim Ulinski et al., 2005]

\hypertarget{pmid_7382013}{W}e report 2 cases of Cushing's syndrome following intralesional triamcinolone acetonide injections of urethral strictures in children. The pharmacology of triamcinolone and its 2 parenteral forms, triamcinolone acetonide and triamcinolone diacetate, is discussed. For children we recommend the short-acting triamcinolone diacetate at 4-week intervals with dosage adjusted to age. In adults either type of triamcinolone may be used but triamcinolone acetonide should be given at 6-week intervals. [\hyperlink{Nystatin And Triamcinolone Acetonide}{PMID: 7382013}, R R Augspurger et al., 1980]

\hypertarget{pmid_29392100}{N}-acetylcysteine (NAC) is a well-known antidote for acetaminophen toxicity and is easily available over the counter. It has antioxidant and anti-inflammatory properties and an established tolerance and safety profile. Owing to its neuroprotective effects, its clinical use has recently expanded to include the treatment of different psychiatric and non-psychiatric disorders. Although a number of randomized controlled trials have documented the clinical evidence for NAC, there are no reviews that summarize the evidence. The present scoping review summarizes the study designs, the patient characteristics, the evidence and the limitations in randomized controlled trials designed to explore the efficacy of NAC for psychiatric conditions in the pediatric population. [\hyperlink{Nystatin And Triamcinolone Acetonide}{PMID: 29392100}, Sadiq Naveed et al., 2017]

\hypertarget{pmid_20386439}{A}lbumin has been regarded as the gold standard for maintaining adequate colloid osmotic pressure in children, but increased cost, the lack of clear-cut benefits for survival, and fear of transmission of unknown viruses have contributed to its replacement by hydroxyethyl starch and gelatin preparations. Each of the synthetic colloids has unique physicochemical characteristics that determine their likely efficacy and adverse effect profile. This review will examine the advantages and disadvantages of the use of different colloid solutions in children with a particular focus on their safety profile. Dextrans are rarely used because of their negative effects on coagulation and potential for anaphylactic reactions. Gelatin and albumin have little effect on hemostasis, but the disadvantages of gelatin include its high anaphylactoid potential and limited beneficial volume effect. Tetrastarches have significantly fewer adverse effects on coagulation and renal function than the older hydroxyethyl starches and are now approved for children. Dissolving tetrastarches in a plasma-adapted, balanced solution rather than in saline further improves safety with regard to coagulation and acid-base balance. Tetrastarches offer the best currently available compromise between cost-effectiveness and safety profile in children with preexisting normal renal function and coagulation. [\hyperlink{Nystatin And Triamcinolone Acetonide}{PMID: 20386439}, Sonja Saudan et al., 2010]

\hypertarget{pmid_23452680}{T}o examine the efficacy of N-acetylcysteine (NAC) for the treatment of pediatric trichotillomania (TTM) in a double-blind, placebo-controlled, add-on study. A total of 39 children and adolescents aged 8 to 17 years with pediatric trichotillomania were randomly assigned to receive NAC or matching placebo for 12 weeks. Our primary outcome was change in severity of hairpulling as measured by the Massachusetts General Hospital-Hairpulling Scale (MGH-HPS). Secondary measures assessed hairpulling severity, automatic versus focused pulling, clinician-rated improvement, and comorbid anxiety and depression. Outcomes were examined using linear mixed models to test the treatment×time interaction in an intention-to-treat population. No significant difference between N-acetylcysteine and placebo was found on any of the primary or secondary outcome measures. On several measures of hairpulling, subjects significantly improved with time regardless of treatment assignment. In the NAC group, 25\% of subjects were judged as treatment responders, compared to 21\% in the placebo group. We observed no benefit of NAC for the treatment of children with trichotillomania. Our findings stand in contrast to a previous, similarly designed trial in adults with TTM, which demonstrated a very large, statistically significant benefit of NAC. Based on the differing results of NAC in pediatric and adult TTM populations, the assumption that pharmacological interventions demonstrated to be effective in adults with TTM will be as effective in children, may be inaccurate. This trial highlights the importance of referring children with TTM to appropriate behavioral therapy before initiating pharmacological interventions, as behavioral therapy has demonstrated efficacy in both children and adults with trichotillomania. [\hyperlink{Nystatin And Triamcinolone Acetonide}{PMID: 23452680}, Michael H Bloch et al., 2013]

\hypertarget{pmid_19504425}{W}e compared the efficacy and safety of fluconazole and nystatin oral suspensions for the prevention of systemic fungal infection (SFI) in very low birthweight infants. A prospective, randomized clinical trial was conducted over a 15-month period, from May 1997 through September 1998, in 80 preterm infants with birthweights <1500 g. The infants were randomly assigned to receive oral fluconazole or nystatin, beginning within the first week of life. Prophylaxis was continued until full oral feedings were attained. Blood and urine cultures were obtained at enrollment and then weekly thereafter. Thirty-eight infants were randomly assigned to receive oral fluconazole (group I), and 42 infants were assigned to receive nystatin (group II). Birthweight, gestational age, and risk factors for fungal colonization and SFI at the time of randomization and during the hospital course were similar in both groups. SFI developed in two infants (5.3\%) in group I and six infants (14.3\%) in group II. The difference between these two rates was not statistically significant (relative risk, 0.37; 95\% confidence interval, 0.08 to 1.72). There were no deaths in group I and six deaths in group II (P = 0.03). Two infants died of neonatal sepsis, and four deaths were related to necrotizing enterocolitis and/or spontaneous intestinal perforation. No deaths were due to SFI. Enrollment was halted before completion and the study did not attain adequate power to detect a hypothesized drop in SFI rate from 15 to 5\%. Although the results cannot justify any conclusion about the relative efficacy of fluconazole versus nystatin in prevention of SFI, the significantly higher mortality rate in the nystatin group raises questions about the relative safety of this medication. [\hyperlink{Nystatin And Triamcinolone Acetonide}{PMID: 19504425}, Kimon Violaris et al., 2010]

\hypertarget{pmid_1958002}{T}riamcinolone acetonide aerosol (TAA), a topical corticosteroid, now available for intranasal use, has been shown to be highly effective in the treatment of both seasonal and perennial allergic rhinitis (PAR) in adults. To evaluate the efficacy and safety of TAA in children, 210 patients (ages 4 to 12 years) with PAR were randomly assigned to one of three treatment groups (placebo, TAA 82.5 micrograms/day, or TAA 165 micrograms/day). Medication was given tid over 12 weeks in a double-blind fashion. Response to medication was evaluated using symptom scoring, physician evaluation, and, in 44 patients, nasal airflow determinations by anterior rhinomanometry. The higher dose of TAA (165 micrograms/day) significantly improved rhinitis symptoms relative to placebo: the total nasal symptom score and most individual symptom scores (eg, nasal stuffiness, itch, sneezing) were significantly better, duration of rhinitis symptoms (hours per day) was significantly reduced, and nasal airflow in a subset of patients showed significant improvement. The lower dose of TAA (82.5 micrograms/day) was superior to placebo by the same parameters as the higher dose, but this improvement was not as consistently significant as the higher dose. There were no clinically significant adverse events; nasal irritation and epistaxis were rare with a similar incidence among treatment groups. In conclusion, TAA at 165 micrograms/day was effective in controlling the symptoms of PAR and in improving nasal airflow in pediatric patients; the lower dose (82.5 micrograms/day) was marginally effective. Both doses were safe and well-tolerated in the children studied. [\hyperlink{Nystatin And Triamcinolone Acetonide}{PMID: 1958002}, M J Welch et al., 1991]

\hypertarget{pmid_19441606}{I}ntranasal corticosteroids (INSs) are the most effective treatment for allergic rhinitis (AR). However, available INS safety and efficacy data in children younger than 6 years are limited. To report the first well-controlled study assessing the safety and efficacy of an INS in children aged 2 to 5 years with perennial AR. In a 4-week, multicenter, double-blind, parallel-group study, patients were randomized to receive triamcinolone acetonide aqueous nasal spray (TAA AQ), 110 microg once daily, or placebo. A subset of children continued into a 6-month, open-label phase. Efficacy end points included total nasal symptom scores. Safety measures included reports of adverse events, morning serum cortisol levels before and after cosyntropin infusion, and growth as measured using office stadiometry. A total of 474 patients were randomized to receive TAA AQ (n = 236) or placebo (n = 238); 436 entered the open-label extension phase. Adjusted mean (SE) changes from baseline during the double-blind period in instantaneous and reflective total nasal symptom scores were -2.28 (0.16) and -2.31 (0.15), respectively, in the TAA AQ group (P = .09) vs -1.92 (0.16) and -1.87 (0.15) in the placebo group (P = .03). Adverse event rates were comparable between treatment groups. There was no significant change from baseline in serum cortisol levels after cosyntropin infusion at study end. The distribution of children by stature-for-age percentile remained stable during the study. Use of TAA AQ, 110 microg once daily, for up to 6 months offers a favorable efficacy to safety ratio in children aged 2 to 5 years with perennial AR. [\hyperlink{Nystatin And Triamcinolone Acetonide}{PMID: 19441606}, Steven Weinstein et al., 2009]

\hypertarget{pmid_23886027}{T}his study examined the efficacy and safety of N-acetylcysteine (NAC) augmentation for treating irritability in children and adolescents with autism spectrum disorders (ASD). Forty children and adolescents met diagnostic criteria for ASD according to DSM-IV. They were randomly allocated into one of the two groups of NAC (1200 mg/day)+risperidone or placebo+risperidone. NAC and placebo were administered in the form of effervescent and in two divided doses for 8 weeks. Irritability subscale score of Aberrant Behavior Checklist (ABC) was considered as the main outcome measure. Adverse effects were also checked. The mean score of irritability in the NAC+risperidone and placebo+risperidone groups at baseline was 13.2(5.3) and 16.7(7.8), respectively. The scores after 8 weeks were 9.7(4.1) and 15.1(7.8), respectively. Repeated measures of ANOVA showed that there was a significant difference between the two groups after 8 weeks. The most common adverse effects in the NAC+risperidone group were constipation (16.1\%), increased appetite (16.1\%), fatigue (12.9\%), nervousness (12.9\%), and daytime drowsiness (12.9\%). There was no fatal adverse effect. Risperidone plus NAC more than risperidone plus placebo decreased irritability in children and adolescents with ASD. Meanwhile, it did not change the core symptoms of autism. Adverse effects were not common and NAC was generally tolerated well. This trial was registered at http://www.irct.ir. The registration number of this trial was IRCT201106103930N6. [\hyperlink{Nystatin And Triamcinolone Acetonide}{PMID: 23886027}, Ahmad Ghanizadeh et al., 2013]

\hypertarget{pmid_16679073}{T}he non-steroidal anti-inflammatory drugs (NSAIDs) and acetaminophen (paracetamol) are the most common analgesic drugs used in neonates and infants despite limited pharmacodynamic data. Both drugs act through inhibition of cyclooxygenase enzymes. Neonatal acetaminophen clearance is reduced in premature neonates (0.7 L h(-1) x 70 kg(-1)) and increases to 5 L h(-1) x 70 kg(-1) at term (40\% adult rates); adult rates are reached within the first year of life; NSAID clearance follows similar trends. Volume of distribution is increased in the neonatal period. Dosing of both drug groups is tempered by concerns about toxicity. Acetaminophen hepatotoxicity is less common in neonates than in older children and adults, possibly due to reduced oxidative enzyme activity (e.g. CYP 2E1). Data concerning NSAID adverse effects in the neonatal period are few. Renal function is reduced 20\% after NSAID use for patent ductus arteriosus closure in premature neonates and there is no increased frequency of intraventricular haemorrhage. No significant difference in the change in cerebral blood volume, change in cerebral blood flow, or tissue oxygenation index was found between administration of ibuprofen or placebo in neonates. Future studies should define concentration-response relationships for these drugs that are age and pathology specific. [\hyperlink{Nystatin And Triamcinolone Acetonide}{PMID: 16679073}, Evelyne Jacqz-Aigrain et al., 2006]

\hypertarget{pmid_1546817}{A} 4-week, double-blind, parallel group study compared the safety and efficacy of once-a-day intranasal administration of triamcinolone acetonide (Nasacort) versus placebo in 304 patients (155 adult and 149 adolescent) with seasonal allergic rhinitis. Patients were randomized to receive triamcinolone acetonide (110, 220, or 440 microgram) or placebo once daily each morning. Daily rhinitis symptoms scores, weekly patient and physician global assessments, and weekly nasal eosinophil smears were obtained. In each triamcinolone acetonide group, significant (P less than .05) improvement over placebo was noted in the nasal index (sum of ratings for stuffiness, discharge, and sneezing) by week 1, the first point of analysis, and maintained throughout the study. Triamcinolone acetonide groups also demonstrated significant (P less than .05) improvement over placebo in all individual rhinitis symptoms evaluated. The greatest improvement in symptoms was observed at the 440 microgram dose. A significant decrease in eosinophil counts paralleled clinical improvement in all triamcinolone acetonide groups. Physicians and patients rated triamcinolone acetonide significantly (P less than .05) more effective than placebo. Responses of adult and adolescent patients were comparable. Adverse experiences, clinical laboratory values, and results of physical examinations were unremarkable and comparable between the triamcinolone acetonide and placebo groups. We conclude that triamcinolone acetonide is safe, well tolerated, and superior to placebo as a once-a-day treatment for seasonal allergic rhinitis. [\hyperlink{Nystatin And Triamcinolone Acetonide}{PMID: 1546817}, S Findlay et al., 1992]

\hypertarget{pmid_32071590}{C}hildren and adolescents with autism spectrum disorder (ASD) often experience high levels of irritability, which adversely affects their functioning and behaviors. N-acetylcysteine (NAC), an antioxidant precursor to glutathione, has recently been studied for a variety of neuropsychiatric disorders. There is growing evidence to support its use to decrease irritability and self-injurious behaviors in youth with ASD. However, previous studies were limited to outpatient youth with mild symptoms of irritability, maintained on stable medication regimens, who do not meet criteria for higher levels of care. We describe the use of NAC among 4 youths (14-17 years) with ASD who had Aberrant Behavior Checklist-Irritability (ABC-I) scores of ≥ 20 and other psychotropic medication trials prior to treatment with NAC. In all of the cases, NAC appeared to be well tolerated. There was a reduction of symptoms of irritability and/or antipsychotic medication dosages in these cases; despite this, the authors cannot know whether use of NAC or other medication or behavioral strategies were responsible for such changes because this study was not a controlled trial. [\hyperlink{Nystatin And Triamcinolone Acetonide}{PMID: 32071590}, Matthew J Pesko et al., 2020]

\hypertarget{pmid_25754598}{T}his review evaluates the recent progress in clinical trials on oral triptans for acute migraine in children and adolescents. Randomized controlled trials (RCT) on the treatment of migraine in pediatric patients were rare and difficult to design. In particular, high placebo response in many of the trials made it difficult to prove efficacy of triptans. Using a "novel study design" for RCT, a study successfully proved the efficacy of an oral rizatriptan. This trial enrolled patients with unsatisfactory response to nonsteroidal anti-inflammatory or acetaminophen and with migraine lasting longer than 3 h. Rizatriptan was approved by Food and Drug Administration (FDA) (USA) for children and adolescents of 6-17 years. The triptan-NSAID combination drug for pediatric patients also showed efficacy. [\hyperlink{Nystatin And Triamcinolone Acetonide}{PMID: 25754598}, Fumihiko Sakai et al., 2015]

\hypertarget{pmid_20553355}{D}iaper dermatitis (DD) is the most common type of irritative dermatitis in infancy. It is frequently complicated by Candida superinfection. Comparison of efficacy and safety of two antifungal pastes (Imazol = 1\% clotrimazole; Multilind = 100,000 IU nystatin/g + 20\% zinc oxide) in infants with DD. A total of 96 infants were included in this multi-centre, controlled, randomized, evaluator-blinded phase IV trial and treated with pastes containing either clotrimazole (n = 45) or nystatin (n = 46) twice daily for 14 days. In all, 91 children (age 12.1 +/- 5.3 months; 48 females) with DD were evaluable. Total symptom score after 7 days (TSS7) was assessed as primary parameter. Secondary efficacy parameters were TSS at 14 days (TSS14), clinical and microbiological cure rates and global assessment (GA) of clinical response. TSS improved markedly with both pastes. Decreases in symptom score were 4.5 +/- 2.1 (day 7) and 6.1 +/- 1.9 (day 14) with clotrimazole compared with 4.2 +/- 2.3 and 5.4 +/- 2.4 with nystatin (P < 0.0001). With respect to TSS14, clotrimazole was superior to nystatin (P = 0.0434). Clinical cure rate was higher with clotrimazole [36.2\% (day 7) and 68.1\% (day 14)] compared with 28.6\% and 46.9\% (nystatin). GA was very good in 26 (55.3\%) clotrimazole-treated children (nystatin: 16 [32.7\%], P = 0.0257). Frequency of adverse events was comparable in both treatment groups. Clotrimazole was superior to nystatin with respect to reduction in symptom score and GA. Microbiological cure rate was 100\% for both agents. Both treatments were safe and well-tolerated. [\hyperlink{Nystatin And Triamcinolone Acetonide}{PMID: 20553355}, P H Hoeger et al., 2010]

\hypertarget{pmid_12506950}{O}ral thrush is a common condition in young infants. Nystatin treatment is associated with frequent recurrences and difficulty in administration. Fluconazole was compared with nystatin for the treatment of oral candidiasis in infants. Thirty-four infants were randomized to either nystatin oral suspension four times a day for 10 days or fluconazole suspension 3 mg/kg in a single daily dose for 7 days. Clinical cures for nystatin were 6 of 19 (32\%), and those for fluconazole were 15 of 15 (100\%), P < 0.0001. In this small pilot study fluconazole was shown to be superior to nystatin suspension for the treatment of oral thrush in otherwise healthy infants. [\hyperlink{Nystatin And Triamcinolone Acetonide}{PMID: 12506950}, R Alan Goins et al., 2002]

\hypertarget{pmid_28063133}{T}he antipyretic analgesics, paracetamol, and non-steroidal anti-inflammatory agents NSAIDs are one of the most widely used classes of medications in children. The aim of this review is to determine if there are any clinically relevant differences in safety between ibuprofen and paracetamol that may recommend one agent over the other in the management of fever and discomfort in children older than 3 months of age. [\hyperlink{Nystatin And Triamcinolone Acetonide}{PMID: 28063133}, Dipak J Kanabar et al., 2017]

\hypertarget{pmid_23078168}{P}aracetamol (acetaminophen) and ibuprofen are the most frequently purchased over-the-counter (OTC) medicines for children. Parents purchase these medicines for the treatment of fever and pain. In some countries other NSAIDs such as aspirin (acetylsalicylic acid) and dipyrone are available. We aimed to perform a narrative review of the efficacy and toxicity of OTC analgesic medicines for children in order to give guidance to health professionals and parents regarding the treatment of pain in a child. Neither aspirin nor dipyrone are recommended for OTC use because of the association with Reye's syndrome for the former and the risk of agranulocytosis for the latter. Both paracetamol and ibuprofen are effective for the treatment of mild pain in children. Adverse effects with both medicines are infrequent. Ibuprofen is an NSAID and therefore there is a greater risk of gastrointestinal adverse effects and hypersensitivity. Aspirin and dipyrone should be avoided. Paracetamol is the drug of first choice for mild pain in children because of its favourable safety profile. For the treatment of significant musculoskeletal pain, ibuprofen is the drug of first choice. [\hyperlink{Nystatin And Triamcinolone Acetonide}{PMID: 23078168}, Zeina Bárzaga Arencibia et al., 2012]

\hypertarget{pmid_6134564}{A}ll children aged under 15 years admitted to hospital in Newcastle upon Tyne between 1974 and 1981 with a diagnosis of poisoning were studied. After the introduction in 1976 of child resistant containers for salicylates and paracetamol, salicylate poisonings fell dramatically. The other most important medicines to cause poisoning in young children were tricyclic antidepressants, benzodiazapines, Lomotil (diphenoxylate and atropine), and iron preparations; these should also be packaged in child resistant containers by regulation. Few children had symptoms after poisoning with household products, but bleach, turpentine, and paraffin might also be packaged in child resistant containers. The numbers of adolescent girls admitted after deliberate self poisoning and of teenage boys admitted after ingestion of alcohol increased over the study period. [\hyperlink{Nystatin And Triamcinolone Acetonide}{PMID: 6134564}, G R Lawson et al., 1983]

\hypertarget{pmid_19505384}{T}onsillectomy and adenoidectomy remain the first choice treatment of chronic or recurrent acute infections of the upper respiratory tract in children. The aim of this study is to investigate the efficacy of the combination of thiamphenicol glycinate acetylcysteinate plus beclomethasone, administered as aerosol, in children awaiting tonsillectomy and/or adenoidectomy. The study comprised 204 children, aged 1 to 12 years, with chronic adenotonsillitis who had been listed for surgery due to obstructive symptoms and recurrent acute infections. Patients were randomized to treatment with thiamphenicol glycinate acetylcysteinate, dosage 250 mg/day in 2 administrations plus beclomethasone with a dosage of 400 microg/day in 2 administrations, or no treatment, control group, unless required. The drugs were administered by aerosol for 10 days/month over a period of 6 months. Clinical visits were at 4, 7 and 12 months after the start of treatment. The primary efficacy outcome was the reduction in the number of patients requiring surgery. Secondary efficacy measures were the reduction of nasal obstruction, the decrease in the number of infectious episodes and the tolerability of the treatment. Aerosol treatment with thiamphenicol glycinate acetylcysteinate plus beclomethasone resulted in a significantly lower proportion of patients requiring surgery (29 of 101; 29 percent) compared to patients in the control group (100 of 103; 97 percent) (p < 0.0001). Treatment was also associated with a reduction of nasal obstruction and a decrease in the number of infectious episodes. No treatment-related adverse events were reported and the aerosol therapy proved easy to administer to children. The aerosol therapy with the combination of thiamphenicol glycinate acetylcysteinate plus beclomethasone was able to prevent or postpone surgery in a substantial percentage of patients, without adverse events. These preliminary results suggest that this novel approach could play a role in the antibiotic prophylaxis of chronic infectious diseases of the upper airways. [\hyperlink{Nystatin And Triamcinolone Acetonide}{PMID: 19505384}, A Macchi et al., ]

\hypertarget{pmid_8733987}{T}riamcinolone acetonide (TAA) aerosol nasal inhaler has been shown to effectively relieve the symptoms of seasonal allergic rhinitis in adults and adolescents. We conducted a study to evaluate the efficacy and safety of once-daily administration of TAA aerosol nasal inhaler in pediatric patients aged 6 to 11 years with grass seasonal allergic rhinitis. This multicenter, randomized, double-blind, placebo-controlled, parallel-group study enrolled 116 children who were treated with either TAA aerosol nasal inhaler (220 micrograms/d) or placebo once daily for 2 weeks. Patients evaluated the severity of rhinitis symptoms (nasal stuffiness, discharge, sneezing, and itching) daily according to a four-point scale (0 = absent, 1 = mild, 2 = moderate, and 3 = severe). Patients' and physicians' global evaluations of overall treatment efficacy were assessed at the end of the 2-week treatment period. Patients treated with TAA aerosol nasal inhaler had significantly greater reductions in all nasal symptom scores overall and in virtually all symptoms at the end of week 1 and week 2 compared with those in the placebo group. Both patients' and physicians' global evaluations of efficacy favored TAA aerosol nasal inhaler over placebo. This study demonstrated that once-daily administration of 220 micrograms of TAA aerosol nasal inhaler was well tolerated and effectively reduced the symptoms of seasonal allergic rhinitis in pediatric patients. [\hyperlink{Nystatin And Triamcinolone Acetonide}{PMID: 8733987}, C H Banov et al., ]

\hypertarget{pmid_18939734}{G}uidelines recommend treatment with intranasal corticosteroids for patients with allergic rhinitis (AR), but concerns remain about possible adverse effects. To present the 1- and 2-year growth results for children with AR treated with triamcinolone acetonide aqueous nasal spray. Thirty-nine children (aged 6.1-14.3 years at study entry) were treated with triamcinolone acetonide aqueous for 1 year, and a subset of 30 children completed a second year of treatment. The dose was physician titered to achieve control over AR symptoms. For each child, statural heights at baseline and at the 1- and 2-year (where available) visits, together with growth rates, were measured and were compared with predicted values. There were no significant differences between measured and predicted heights at the 1- and 2-year visits. The mean (SD) measured--predicted difference was 0.3 (2.2) cm (95\% confidence interval, -0.4 to 1.0 cm) at the 1-year visit and 0.5 (3.0) cm (95\% confidence interval, -0.6 to 1.6 cm) at the 2-year visit. Mean differences in measured and predicted growth rates were nonsignificant at the 1- and 2-year visits. Triamcinolone acetonide aqueous titered to control AR symptoms and given for 1 or 2 years had no significant effect on statural growth in children with AR. [\hyperlink{Nystatin And Triamcinolone Acetonide}{PMID: 18939734}, David P Skoner et al., 2008]

\hypertarget{pmid_27027204}{C}urrent pharmacological treatments for Tourette Syndrome (TS), such as antipsychotic agents and α-2 agonists, are moderately effective in the treatment of tics, but have substantial side effects that limit their use. N-acetylcysteine (NAC) modulates glutamatergic systems, and has been used safely as an antioxidant agent with minimal side effects for decades. NAC has been increasingly studied for the treatment of other obsessive-compulsive spectrum disorders. We aim to examine the efficacy of NAC for the treatment of pediatric TS in a double-blind, placebo-controlled, add-on study. Thirty-one children and adolescents 8-17 years of age with TS were randomly assigned to receive NAC or matching placebo for 12 weeks. Our primary outcome was change in severity of tics as measured by the Yale Global Tic Severity Scale (YGTSS), Total tic score. Secondary measures assessed comorbid obsessive-compulsive disorder (OCD), depression, anxiety, and attention-deficit/hyperactivity disorder (ADHD). Linear mixed models in SAS were used to examine differences between NAC and placebo. Of 31 randomized subjects, 14 were assigned to placebo (two females; 11.5 + 2.8 years) and 17 to active NAC (five females; 12.4 + 1.4 years) treatment. No significant difference between NAC and placebo was found in reducing tic severity or any secondary outcomes. We found no evidence for efficacy of NAC in treating tic symptoms. Our findings stand in contrast to studies suggesting benefits of NAC in the treatment of other obsessive-compulsive spectrum disorders in adults, including OCD and trichotillomania, but are similar to a recent placebo-controlled trial of pediatric trichotillomania that found no benefit of NAC. [\hyperlink{Nystatin And Triamcinolone Acetonide}{PMID: 27027204}, Michael H Bloch et al., 2016]

\section*{Trihexyphenidyl Hydrochloride}
\subsection*{Result}
\subsubsection*{Answer}

Yes (Ages 1–18 years)

\subsubsection*{{Explanation}}
\hypertarget{Trihexyphenidyl Hydrochloride}
A review of the available abstracts reveals several studies specifically investigating the use of Trihexyphenidyl Hydrochloride in children, primarily for the treatment of dystonia associated with cerebral palsy and related movement disorders. The relevant studies are summarized below by age range:

1. **Ages 1–18 years**:  
   - A retrospective chart review of 101 children (mean age 7 years 10 months, range 1–18 years) with cerebral palsy treated with trihexyphenidyl for dystonia or sialorrhea found that 91\% tolerated the medication well, with a mean treatment duration of 3 years and 7 months. Side effects occurred in 69\% of subjects, mostly in those aged ≥7 years, but were generally manageable. 64\% continued treatment at study end, and 97 patients reported benefits. The study concludes that most children tolerated trihexyphenidyl well with gradual dose increases, and almost all demonstrated improvements in symptoms [\hyperlink{pmid_21310336}{PMID: 21310336}, Jorge Carranza-del Rio et al., 2011].

2. **Ages 4–15 years**:  
   - A prospective, open-label, multicenter pilot trial in 23 children (aged 4–15 years) with cerebral palsy and secondary dystonia found that trihexyphenidyl was generally well tolerated, though some children experienced nonserious adverse events (chorea, rash, hyperactivity). Three children withdrew due to these events, and three required dose reductions. The study concludes that trihexyphenidyl may be safe and effective for some children, but children with hyperkinetic dystonia may worsen. The authors call for larger, randomized trials to confirm these findings [\hyperlink{pmid_17690057}{PMID: 17690057}, Terence D Sanger et al., 2007].

3. **Ages (mean 8.2 ± 5.8 years)**:  
   - A retrospective analysis of 31 children with dystonia treated with high-dose trihexyphenidyl (>0.5 mg/kg/day) found that most caregivers reported improvement in motor function. Side effects were mostly transient, with one case of persistent hyperopia. The study concludes that trihexyphenidyl is effective, particularly in children without spasticity and with higher cognitive abilities [\hyperlink{pmid_21498790}{PMID: 21498790}, Hilla Ben-Pazi et al., 2011].

4. **Case report, age 8 years**:  
   - An 8-year-old female with dystonia after bilateral putamenal hemorrhages was treated with trihexyphenidyl and experienced improvements in fine motor control, language, and oral motor skills, with no adverse side effects reported [\hyperlink{pmid_10207932}{PMID: 10207932}, F S Pidcock et al., 1999].

5. **Ages not specified, but "children"**:  
   - A retrospective survey of 22 children with extrapyramidal cerebral palsy treated with trihexyphenidyl found improvements in upper extremity function, expressive language, and drooling, especially in younger children. The study suggests that younger children are more likely to respond, but calls for prospective masked studies to confirm findings [\hyperlink{pmid_11483397}{PMID: 11483397}, A H Hoon et al., 2001].

**Summary**:  
Multiple targeted studies, including retrospective reviews, prospective trials, and case reports, have investigated the safety and tolerability of trihexyphenidyl in children, primarily ages 1–18 years, with cerebral palsy and dystonia. These studies generally affirm that trihexyphenidyl is tolerated by most children, with side effects that are usually manageable or transient. Some subgroups (e.g., those with hyperkinetic dystonia) may experience worsening symptoms, and rare persistent side effects have been reported. The evidence supports that trihexyphenidyl can be considered safe for use in children within the studied age ranges, when monitored appropriately.

\subsection*{Abstracts}
\hypertarget{pmid_21310336}{A} paucity of information exists regarding medications to treat dystonia in children with cerebral palsy. This study sought to review the benefits and tolerability of trihexyphenidyl in children with cerebral palsy, treated for dystonia or sialorrhea or both in a pediatric tertiary care hospital, through a retrospective chart review. In total, 101 patients (61 boys and 40 girls) were evaluated. The mean age at drug initiation was 7 years and 10 months (range, 1-18 years). The mean initial dose was 0.095 mg/kg/day. The dose was increased by 10-20\% no sooner than every 2 weeks. The mean maximum dose reached was 0.55 mg/kg/day. Ninety-three patients (91\%) tolerated the medication well, with a mean duration of treatment of 3 years and 7 months. Side effects occurred in 69\% of subjects, the majority in patients aged ≥7 years, and soon after treatment initiation. Sixty-four percent continued the treatment at study end. Ninety-seven patients reported benefits, including reduction of dystonia in upper (59.4\%) and lower (37.6\%) extremities, sialorrhea (60.4\%), and speech issues (24.7\%). The majority of patients tolerated trihexyphenidyl well on a schedule of gradual dose increases, and almost all demonstrated improvements in dystonia or sialorrhea or both. [\hyperlink{Trihexyphenidyl Hydrochloride}{PMID: 21310336}, Jorge Carranza-del Rio et al., 2011]

\hypertarget{pmid_17690057}{A}lthough trihexyphenidyl is used clinically to treat both primary and secondary dystonia in children, limited evidence exists to support its effectiveness, particularly in dystonia secondary to disorders such as cerebral palsy. A prospective, open-label, multicenter pilot trial of high-dose trihexyphenidyl was conducted in 23 children aged 4 to 15 years with cerebral palsy judged to have secondary dystonia impairing function in the dominant upper extremity. All children were given trihexyphenidyl at increasing doses over a 9-week period up to a maximum of 0.75 mg/kg/d. Trihexyphenidyl was subsequently tapered off over the next 5 weeks. Objective motor assessments were performed at baseline, 9 weeks, and 15 weeks. The primary outcome measure was the Melbourne Assessment of Unilateral Upper Limb Function, tested in the dominant arm. Tolerability and safety were monitored closely throughout the trial. Of the 31 children who agreed to participate in the study, 5 failed to meet entry criteria and 3 withdrew due to nonserious adverse events (chorea, drug rash, and hyperactivity). Three children required a dosage reduction because of nonserious adverse events but continued to participate. The 23 children who completed the study showed a significant improvement in arm function at 15 weeks (P = .045) but not at 9 weeks (P = .985). Post hoc analysis showed that a subgroup (n = 10) with hyperkinetic dystonia (excess involuntary movements) worsened at 9 weeks (P = .04) but subsequently returned to baseline following taper of the medicine. The authors conclude that scientific evidence for the clinical use of trihexyphenidyl in cerebral palsy remains equivocal. Trihexyphenidyl may be a safe and effective for treatment for arm dystonia in some children with cerebral palsy if given sufficient time to respond to the medication. Post hoc analyses based on the type of movement disorder suggested that children with hyperkinetic forms of dystonia may worsen. A larger, randomized prospective trial stratified by the presence or absence of hyperkinetic movements is needed to confirm these results. [\hyperlink{Trihexyphenidyl Hydrochloride}{PMID: 17690057}, Terence D Sanger et al., 2007]

\hypertarget{pmid_18219837}{A}ntihistamines are an established first-line treatment for allergic rhinitis and are widely prescribed in infants for allergic symptoms. To establish the safety and tolerability of fexofenadine hydrochloride in children aged 6 months to 2 years in 2 studies (T/3001 and T/3002). Both studies had a multicenter, randomized, placebo-controlled design. Mean treatment duration was 8 days. Subjects were randomized (T/3001, n = 174; and T/3002, n = 219) to twice-daily fexofenadine hydrochloride, 15 or 30 mg, or placebo mixed with a standard vehicle. In the combined population, the incidence of treatment-emergent adverse events (TEAEs) was comparable between groups (placebo, 48.2\% [96/199]; fexofenadine hydrochloride, 15 mg, 40.0\% [34/85]; and fexofenadine hydrochloride, 30 mg, 35.2\% [38/108]). Vomiting was the most common TEAE (placebo, 13.6\%; fexofenadine hydrochloride, 15 mg, 14.1\%; and fexofenadine hydrochloride, 30 mg, 5.6\%). Most TEAEs were unrelated to study medication, as evaluated by investigators; those possibly related to study medication were mild or moderate in intensity. No clinical differences were seen between fexofenadine and placebo for vital signs, electrocardiographic results, or physical examination results. Fexofenadine hydrochloride, 15 or 30 mg, given for a mean duration of 8 days is well tolerated, with a good safety profile, in children aged 6 months to 2 years. [\hyperlink{Trihexyphenidyl Hydrochloride}{PMID: 18219837}, Frank C Hampel et al., 2007]

\hypertarget{pmid_18702885}{A}llergic rhinitis (AR) is a common chronic condition in children and may impact a child's quality of life. Increasing treatment compliance may improve quality of life. An oral suspension of fexofenadine hydrochloride (HCl) has been developed to ease administration to children and may, therefore, improve treatment compliance. The purpose of this study was to assess the pharmacokinetic behavior, safety, and tolerability of a single dose of fexofenadine HCl oral suspension administered to children aged 2-5 years with allergic rhinitis. Children (aged 2-5 years) with AR were recruited in a multicenter, open-label, single-dose study. Fexofenadine HCl (30 mg) was administered as a 6-mg/mL suspension (5 mL). Plasma samples were collected up to 24 hours postdose. Adverse events (AEs); electrocardiograms (ECGs); vital signs; and clinical laboratory tests for hematology, blood chemistry, and urinalysis were analyzed to evaluate safety and tolerability. Fifty subjects completed the study. Mean maximum plasma concentration of fexofenadine was 224 ng/mL, and mean area under the plasma concentration curve was 898 ng . hour/mL. Treatment-emergent AEs were mild in intensity and reported in a total of seven subjects. No trends or clinically meaningful changes in mean ECG, vital sign, or clinical laboratory test data occurred during the study. In children aged 2-5 years, the exposure after a 30-mg dose of fexofenadine HCl suspension was similar to the exposures previously seen after a 30- and 60-mg dose of fexofenadine HCl in children aged 6-11 years and in adults, respectively. The suspension was also well tolerated. [\hyperlink{Trihexyphenidyl Hydrochloride}{PMID: 18702885}, Nathan Segall et al., ]

\hypertarget{pmid_21498790}{T}here are conflicting reports regarding the efficacy of trihexyphenidyl, an anticholinergic drug, for treatment of dystonia in cerebral palsy. The author hypothesized that trihexyphenidyl may be more effective in specific subgroups and performed a retrospective analysis of 31 children (8.2 ± 5.8 years) with dystonia following treatment with high-dose trihexyphenidyl (>0.5 mg/kg/day). Main outcome measure was extent of motor improvement calculated according to the body areas affected. Most (21/31) caregivers reported improvement in 1 or more areas, mainly arm, hand, and oromotor function. Improvement was greater in children without spasticity (P = .02) and in those with higher cognitive function (P = .02). While a third of caregivers (10/31) reported tone reduction, and half (15/31) noted overall functional improvement. Side effects were transient, with the exception of hyperopia (n = 1), and occurred less frequently in children with a history of prematurity (P = .02). In summary, trihexyphenidyl is effective particularly in absence of spasticity and in children with higher cognitive abilities. [\hyperlink{Trihexyphenidyl Hydrochloride}{PMID: 21498790}, Hilla Ben-Pazi et al., 2011]

\hypertarget{pmid_17941284}{T}he safety of fexofenadine has been examined extensively in adults and school-age children. However, the safety of fexofenadine in children younger than 6 years has not been reported to date. To compare the safety and tolerability of twice-daily fexofenadine hydrochloride, 30 mg, and placebo in preschool children aged 2 to 5 years with allergic rhinitis. This was a multicenter, double-blind, randomized, placebo-controlled, parallel-group study, conducted between February 29, 2000, and June 14, 2001. Participants were randomized to either fexofenadine hydrochloride, 30 mg, or placebo twice daily for a 2-week period. To facilitate dosing, capsule content was mixed with applesauce (approximately 10 mL). Safety assessments depended on date of entry into the study because of an amendment to the protocol. Before the amendment, assessments included physical examination, vital signs reporting (oral temperature, heart rate, and respiratory rate), and adverse event (AE) reporting. After the amendment, safety assessments included laboratory testing (blood chemistry and hematology profiles), physical examination, 12-lead electrocardiography, and vital signs (oral temperature, blood pressure, heart rate, and respiratory rate) and AE reporting. Treatment-emergent AEs were observed in 116 of 231 participants receiving placebo and 111 of 222 receiving fexofenadine. These AEs were possibly related to study medication in 19 (8.2\%) and 21 (9.5\%) of the participants receiving placebo and fexofenadine, respectively, and most frequently involved the digestive system. No clinically relevant differences in laboratory measures, vital signs, and physical examinations were observed. The findings show that fexofenadine hydrochloride, 30 mg, is well tolerated and has a good safety profile in children aged 2 to 5 years with allergic rhinitis. [\hyperlink{Trihexyphenidyl Hydrochloride}{PMID: 17941284}, Henry Milgrom et al., 2007]

\hypertarget{pmid_21540483}{T}richlorophenols (TCPs) are organochlorine compounds which are ubiquitous in the environment and well known for their carcinogenic effects. However, little is known about their neurotoxicity in humans. Our goal was to examine the association between body burden of TCPs (ie, 2,4,5-TCP and 2,4,6-TCP) and attention deficit hyperactivity disorder (ADHD). We calculated ORs and 95\% CIs from logistic regression analyses using data from the 1999-2004 National Health and Nutrition Examination Survey (NHANES) to evaluate the association between urinary TCPs and parent-reported ADHD among 2546 children aged 6-15 years. Children with low levels (<3.58 μg/g) and high levels (≥3.58 μg/g) of urinary 2,4,6-TCP had a higher risk of parent-reported ADHD compared to children with levels below the limit of detection (OR 1.54, 95\% CI 0.97 to 2.43 and OR 1.77, 95\% CI 1.18 to 2.66, respectively; p for trend=0.006) after adjusting for covariates. No association was found between urinary 2,4,5-TCP and parent-reported ADHD. Exposure to TCP may increase the risk of behavioural impairment in children. The potential neurotoxicity of these chemicals should be considered in public health efforts to reduce environmental exposures/contamination, especially in countries where organochlorine pesticides are still commonly used. [\hyperlink{Trihexyphenidyl Hydrochloride}{PMID: 21540483}, Xiaohui Xu et al., 2011]

\hypertarget{pmid_2672786}{T}his study assessed the safety and efficacy of methylphenidate in children with seizures and attention-deficit disorder. Ten children, aged 6 years 10 months to 10 years 10 months, without seizures while receiving a single antiepileptic drug, were evaluated in a double-blind medication-placebo crossover study with methylphenidate hydrochloride was administered at 0.3 mg/kg per dose and given at 8 AM and 12 PM on school days only. The use of methylphenidate was associated with statistically significant improvements on the Conners' Teacher Rating Scale and on the Finger Tapping Task and with trends toward improvement on the Matching Familiar Figures Test and Discriminant Reaction Time tests. No child had seizures during the study period nor subsequently for those who continued receiving psychostimulants. There were no significant changes of epileptiform features or back-ground activity on electroencephalograms and no alterations in antiepileptic drug levels. Methylphenidate may be a safe and effective treatment for certain children with seizures and concurrent attention-deficit disorder. [\hyperlink{Trihexyphenidyl Hydrochloride}{PMID: 2672786}, H Feldman et al., 1989]

\hypertarget{pmid_23129068}{H}ydroxyurea (HU) is highly effective treatment for sickle cell disease (SCD). While pediatric use of HU is accepted clinical practice, barriers to use may impede its potential benefit. A survey of parents of children ages 5-17 years with SCD was performed across five institutions to assess factors associated with HU use. Of the 173 parent responses, 65 (38\%) had children currently taking HU. Among parents of children not taking HU, the most commonly cited reasons were that their hematology provider had not offered it, their child was not sufficiently symptomatic and concerns about potential side effects. Even parents of HU users reported widespread concern about effectiveness, long-term safety, and off-label use. In bivariate analyses, children's ages, parental demographics such as education level, or travel time to their hematology provider were not correlated with HU use. Bivariate analysis and multivariate logistic regression revealed three significant factors associated with current HU use: better parental knowledge about its major therapeutic effects (P < 0.001), sickle genotype (P = 0.005), and institution of clinical care (P = 0.04). Pervasive concerns about HU safety exist, even among parents of current users. Varying knowledge among parents appears to be independent of their demographics, and is associated with HU use. Inter-institutional variability in parental knowledge and drug uptake highlights potentially potent site-specific influences on likelihood of HU use. Overall, these survey data underscore the need for strategies to bolster parental understanding about benefits of HU and address concerns about its safety. [\hyperlink{Trihexyphenidyl Hydrochloride}{PMID: 23129068}, Suzette O Oyeku et al., 2013]

\hypertarget{pmid_31292919}{T}riclofos sodium (TFS) has been used for many years in children as a sedative for painless medical procedures. It is physiologically and pharmacologically similar to chloral hydrate, which has been censured for use in children with neurocognitive disorders. The aim of this study was to investigate the safety and efficacy of TFS sedation in a pediatric population with a high rate of neurocognitive disability. The database of the neurodiagnostic institute of a tertiary academic pediatric medical center was retrospectively reviewed for all children who underwent sedation with TFS in 2014. Data were collected on demographics, comorbidities, neurologic symptoms, sedation-related variables, and outcome. The study population consisted of 869 children (58.2\% male) of median age 25 months (range 5-200 months); 364 (41.2\%) had neurocognitive diagnoses, mainly seizures/epilepsy, hypotonia, or developmental delay. TFS was used for routine electroencephalography in 486 (53.8\%) patients and audiometry in 401 (46.2\%). Mean (± SD) dose of TFS was 50.2 ± 4.9 mg/kg. Median time to sedation was 45 min (range 5-245), and median duration of sedation was 35 min (range 5-190). Adequate sedation depth was achieved in 769 cases (88.5\%). Rates of sedation-related adverse events were low: apnea, 0; desaturation ≤ 90\%, 0.2\% (two patients); and emesis, 0.35\% (three patients). None of the children had hemodynamic instability or signs of poor perfusion. There was no association between desaturations and the presence of hypotonia or developmental delay. TFS, when administered in a controlled and monitored environment, may be safe for use in children, including those with underlying neurocognitive disorders. [\hyperlink{Trihexyphenidyl Hydrochloride}{PMID: 31292919}, Eytan Kaplan et al., 2019]

\hypertarget{pmid_10477679}{P}revious studies have determined the short-term toxicity profile, laboratory changes, and clinical efficacy associated with hydroxyurea (HU) therapy in adults with sickle cell anemia. The safety and efficacy of this agent in pediatric patients with sickle cell anemia has not been determined. Children with sickle cell anemia, age 5 to 15 years, were eligible for this multicenter Phase I/II trial. HU was started at 15 mg/kg/d and escalated to 30 mg/kg/d unless the patient experienced laboratory toxicity. Patients were monitored by 2-week visits to assess compliance, toxicity, clinical adverse events, growth parameters, and laboratory efficacy associated with HU treatment. Eighty-four children were enrolled between December 1994 and March 1996. Sixty-eight children reached maximum tolerated dose (MTD) and 52 were treated at MTD for 1 year. Significant hematologic changes included increases in hemoglobin concentration, mean corpuscular volume, mean corpuscular hemoglobin, and fetal hemoglobin parameters, and decreases in white blood cell, neutrophil, platelet, and reticulocyte counts. Laboratory toxicities typically were mild, transient, and were reversible upon temporary discontinuation of HU. No life-threatening clinical adverse events occurred and no child experienced growth failure. This Phase I/II trial shows that HU therapy is safe for children with sickle cell anemia when treatment was directed by a pediatric hematologist. HU in children induces similar laboratory changes as in adults. Phase III trials to determine if HU can prevent chronic organ damage in children with sickle cell anemia are warranted. [\hyperlink{Trihexyphenidyl Hydrochloride}{PMID: 10477679}, T R Kinney et al., 1999]

\hypertarget{pmid_17063023}{E}vidence on the caries-preventive effect of chlorhexidine (CHX) among high-risk children is inconclusive, possibly because obscured by fluoride exposure. We investigated the effect of CHX among initially 3-year-old subjects whose baseline d(3)ft was = 0 and whose only regular fluoride exposure came from toothpaste. The subjects were assigned to three groups: high-risk test (HRT, n = 70), high-risk control (HRC, n = 71), and low-risk control (LRC, n = 70). Risk classification was based on salivary mutans streptococcal levels (MS, </>or=1.0 x 10(5) cfu/ml). Basic measures (oral hygiene, dietary counselling every 4 months) were given to all groups. HRT also underwent CHX gel applications for 3 consecutive days at 3-month intervals for 15 months. Eighteen months after baseline d(3)ft increments and proportions of children with d(3)ft increment >or=1 (\%d(3)ft increment >or=1) among all groups were assessed. Anti-MS effect on high-risk children and caries-preventive effect on all children were statistically analysed by residual change analysis (MS), non-parametric tests and logistic regression analysis (caries). No differences were found between the groups in basic programme compliance. CHX significantly reduced MS levels. \%d(3)ft increment >or=1 and mean d(3)ft increments were 34.3\%, 0.56 (HRT), 32.4\%, 0.54 (HRC) and 11.4\%, 0.11 (LRC), with HRT/HRC values statistically significantly higher than LRC values and no significant difference between HRT and HRC. HRT children were not less likely to show new lesions than HRC children (OR = 1.09; 95\% confidence interval 0.54-2.19), while high-risk children were 4 times more likely to show new lesions than low-risk children (OR = 3.71; 95\% confidence interval 1.53-9.03). CHX gel applications showed moderate anti-MS effect but negligible caries-preventive effect. [\hyperlink{Trihexyphenidyl Hydrochloride}{PMID: 17063023}, S Petti et al., 2006]

\hypertarget{pmid_7771914}{T}he findings from case reports and patient questionnaire surveys have been interpreted as indicating that administration of stimulants is ill-advised for the treatment of attention-deficit hyperactivity disorder in children with tic disorder. Thirty-four prepubertal children with attention-deficit hyperactivity disorder and tic disorder received placebo and three dosages of methylphenidate hydrochloride (0.1, 0.3, and 0.5 mg/kg) twice daily for 2 weeks each, under double-blind conditions. Treatment effects were assessed using direct observations of child behavior in a simulated (clinic-based) classroom and using rating scales completed by the parents, teachers, and physician. Methylphenidate effectively suppressed hyperactive, disruptive, and aggressive behavior. There was no evidence that methylphenidate altered the severity of tic disorder, but it may have a weak effect on the frequency of motor (increase) and vocal (decrease) tics. Methylphenidate appears to be a safe and effective treatment for attention-deficit hyperactivity disorder in the majority of children with comorbid tic disorder. [\hyperlink{Trihexyphenidyl Hydrochloride}{PMID: 7771914}, K D Gadow et al., 1995]

\hypertarget{pmid_19747907}{C}hloral hydrate is used worldwide as a first-line agent for procedural sedation in paediatric patients undergoing painless diagnostic investigations. Chloral hydrate overdoses in children and adults have been reported to cause various toxicities, including central nervous system, respiratory and cardiac depression with sometimes fatal outcome. A 3-month-old girl was admitted after an unintentional administration of a 10-fold dose of chloral hydrate (667 mg/kg). She showed respiratory insufficiency in need of intubation and ventilation. Gastric endoscopy revealed esophagitis and gastric ulcerations. To assess the need for hemodialysis, serum trichloroethanol (TCE) was determined using a mass spectrometric quantification after a methyl tertiary butyl ether extraction using an external standard method. The serum TCE level 6 h after administration was 89 mg/L and declined to 20 mg/L within 24 h. The child could be extubated the next day; her further course was uneventful. The repeated determination of serum TCE levels prevented a technically difficult and risky hemodialysis in this very young patient. [\hyperlink{Trihexyphenidyl Hydrochloride}{PMID: 19747907}, Sultan Dogan-Duyar et al., 2010]

\hypertarget{pmid_10207932}{T}rihexyphenidyl has been found to be an effective treatment for dystonic movement disorders, improving gross motor function in patients with axial and torsional dystonia, tremors, and myoclonus. In this report, improvements in fine motor control, language, and oral motor skills are described with trihexyphenidyl in an 8-year-old female who developed dystonia after spontaneous bilateral putamenal hemorrhages. No adverse side effects occurred. The mechanism of action of trihexyphenidyl is believed to be in the basal ganglia where it inhibits muscarinic cholinergic receptors and increases the turnover of dopamine. [\hyperlink{Trihexyphenidyl Hydrochloride}{PMID: 10207932}, F S Pidcock et al., 1999]

\hypertarget{pmid_11483397}{T}rihexyphenidyl (Artane) is a centrally active muscarinic antagonist commonly used to treat patients with generalized dystonia. In a retrospective survey of 22 consecutive children with extrapyramidal cerebral palsy, we evaluated trihexyphenidyl on upper extremity and lower extremity function, expressive language, and drooling. Functional changes were assessed using a parental questionnaire (rating scale 1-5: from 1 = little or no change to 5 = tremendous change, with scores in either a positive or negative direction). Improvements of +4 or +5 were reported in eight children for upper extremity function, in eight children for verbal expressive language, in five for drooling, and in none for lower extremity function. Using bivariate linear regression modeling to investigate variables associated with treatment effects, there was a significant inverse relationship between age at initiation of medication and therapeutic response. Furthermore, beneficial responses were specific to upper-extremity function and expressive language. These results suggest that younger children are more likely to respond to trihexyphenidyl and that primary functional benefits include improved fine motor abilities and expressive language. A prospective masked study with a standardized clinical instrument is needed to confirm these findings. [\hyperlink{Trihexyphenidyl Hydrochloride}{PMID: 11483397}, A H Hoon et al., 2001]

\hypertarget{pmid_4938432}{O}ne hundred and three children with proved typhoid fever were treated with trimethoprim-sulphamethoxazole, and the results compared with those of a further 40 children treated with chloramphenicol. The bacteriological response to trimethoprim-sulphamethoxazole was unsatisfactory. From this study it seems that at present chloramphenicol is still the treatment of choice for typhoid fever. In view of the haematological changes occurring during therapy with trimethoprim-sulphamethoxazole caution is necessary and monitoring of the blood picture advisable, even at the recommended dose. [\hyperlink{Trihexyphenidyl Hydrochloride}{PMID: 4938432}, J N Scragg et al., 1971]

\hypertarget{pmid_31157521}{H}ydroxyurea (HU) increases fetal hemoglobin (HgbF) and ameliorates sickle cell disease (SCD) symptoms. Studies have demonstrated the safety and efficacy of HU in infants and children. Initiation of HU in infancy for children with SCD needs to be implemented in community practice. Starting in 2011, the Pediatric Sickle Cell Program of Northern Virginia initiated HU in infants with SCD. A prospective longitudinal database tracked the clinical course and outcomes. Twenty-four children with HgbSS who started HU by age 1 were continuously followed for a total of 95 person-years. Age at the time of analysis ranged from 2 to 7 years. Average hemoglobin at 6-month intervals ranged from 9.5 + 1.9 to 10.7 + 0.8 g/dL, and average HgbF ranged from 27.8 + 5.0\% to 34.1 + 6.6\%. Twenty-seven hospitalizations occurred (0.28/person-year), all before age 3, including 19 (70\%) for fever or infection, five (19\%) for splenic sequestration, and one (4\%) for pain in an infant prior to starting HU. The treat-and-release emergency department visits totaled 68 (0.72/person-year), including 62 visits (91\%) for fever, infection, or viral illness, and two visits (3\%) for pain/dactylitis in infants before HU initiation. Splenic sequestration accounted for all five transfusions. No pain episodes requiring medical attention were documented after HU initiation. No complicated acute chest syndrome, no abnormal or conditional transcranial Doppler ultrasound, and no overt strokes occurred. Implementation of HU in infancy for patients with SCD in community practice is feasible and is highly effective in preventing disease complications. [\hyperlink{Trihexyphenidyl Hydrochloride}{PMID: 31157521}, Ronay Thomas et al., 2019]

\hypertarget{pmid_15247700}{M}any children with urological disease require long-term treatment with antibiotics. In many cases the choice of medical instead of surgical management hinges on the implied safety of certain drugs. Recently some groups have advocated subureteral injection procedures to avoid long-term antibiotics for low grade reflux. We present a concise and relevant review on the use and adverse reactions of nitrofurantoin, trimethoprim and sulfamethoxazole in children. We reviewed the literature regarding the safety and toxicity of these drugs. Information regarding absorption, excretion and dosing was also gathered to explain better the mechanisms of toxicity. Adverse reactions in children reported in the literature related to nitrofurantoin are gastrointestinal disturbance (4.4/100 person-years at risk), cutaneous reactions (2\% to 3\%), pulmonary toxicity (9 patients), hepatoxicity (12 patients and 3 deaths), hematological toxicity (12 patients), neurotoxicity and an increased rate of sister chromatid exchanges. Adverse reactions in children related to trimethoprim/sulfamethoxazole are almost exclusively due to the sulfamethoxazole component, including cutaneous reactions (1.4 to 7.4 events per 100 person-years at risk), hematological toxicity (0\% to 72\% of patients) and hepatotoxicity (5 patients). The majority of adverse reactions were found in children on full dose therapy and not prophylaxis. The use of nitrofurantoin, trimethoprim and sulfamethoxazole is safe in children for long-term prophylactic therapy. The antibiotic safety issue should not be misconstrued as an argument for surgical therapy, whether minimally invasive or not. Adverse reactions exist to these medicines but they are less common than seen in adults, presumably because of the lower dose used for therapy, and the lack of significant comorbidities and drug interactions in children. Serious side effects are extremely rare and most are reversible by discontinuing therapy. The extremely low potential for significant adverse reactions should be discussed with parents. [\hyperlink{Trihexyphenidyl Hydrochloride}{PMID: 15247700}, Edward Karpman et al., 2004]

\hypertarget{pmid_33706380}{H}ydroxyurea (HU) is used in children with sickle cell disease (SCD) to increase fetal hemoglobin (HF), contributing to a decrease in physical symptoms and potential protection against cerebral microvasculopathy. There has been minimal investigation into the association between HU use and cognition in this population. This study examined the relationship between HU status and cognition in children with SCD. Thirty-seven children with SCD HbSS or HbS/β0 thalassaemia (sickle cell anemia; SCA) ages 4:0-11 years with no history of overt stroke or chronic transfusion completed a neuropsychological test battery. Other medical, laboratory, and demographic data were obtained. Neuropsychological function across 3 domains (verbal, nonverbal, and attention/executive) was compared for children on HU (n = 9) to those not taking HU (n = 28). Children on HU performed significantly better than children not taking HU on standardized measures of attention/executive functioning and nonverbal skills. Performance on verbal measures was similar between groups. These results suggest that treatment with HU may not only reduce physical symptoms, but may also provide potential benefit to cognition in children with SCA, particularly in regard to attention/executive functioning and nonverbal skills. Replication with larger samples and longitudinal studies are warranted. [\hyperlink{Trihexyphenidyl Hydrochloride}{PMID: 33706380}, Reem A Tarazi et al., 2021]

\hypertarget{pmid_36226857}{S}ickle cell disease (SCD) is a disease of abnormal hemoglobin associated with severe clinical phenotype and recurrent complications. Hydroxyurea (HU) is one of the US-FDA approved and commonly used drug for the treatment of adult SCD patients with clinical -severity. However, its use in the pediatric groups remains atypical. Despite a high prevalence of the disease in the state Chhattisgarh, there is a lack of evidence supporting its use in pediatric patients. This study aimed to evaluate the pharmacological and clinical efficacy and safety of HU in a large pediatric cohort with SCD from Central India. The study cohort consisted of 164 SCD (138 Hb SS and 26 Hb S beta-thalassemia) children (≤14 years of age) on HU therapy, who were monitored for toxicity, hematological and clinical efficacy at baseline (Pre-HU) and after 24 months (Post-HU). The results highlight the beneficial effects of HU at a mean dose of 18.7 ± 7.0 mg/kg/day. A significant improvement was observed, not only in physical and clinical parameters but also in hematological parameters which include fetal hemoglobin (Hb F), total hemoglobin, hematocrit, mean corpuscular volume (MCV) and mean corpuscular hemoglobin (MCH) levels, when evaluated against the baseline. We did not observe any significant adverse effects during the treatment period. Similar results were obtained on independent analysis of Hb SS and Hb Sβ patients. These findings strengthen the beneficial effect of hydroxyurea in pediatric population also without any serious adverse effects and builds up ground for expanding its use under regular monitoring. [\hyperlink{Trihexyphenidyl Hydrochloride}{PMID: 36226857}, Harsha Lad et al., 2023]

\hypertarget{pmid_19047254}{H}ydroxyurea is the only approved medication for the treatment of sickle cell disease in adults; there are no approved drugs for children. Our goal was to synthesize the published literature on the efficacy, effectiveness, and toxicity of hydroxyurea in children with sickle cell disease. Medline, Embase, TOXLine, and the Cumulative Index to Nursing and Allied Health Literature through June 2007 were used as data sources. We selected randomized trials, observational studies, and case reports (English language only) that evaluated the efficacy and toxicity of hydroxyurea in children with sickle cell disease. Two reviewers abstracted data sequentially on study design, patient characteristics, and outcomes and assessed study quality independently. We included 26 articles describing 1 randomized, controlled trial, 22 observational studies (11 with overlapping participants), and 3 case reports. Almost all study participants had sickle cell anemia. Fetal hemoglobin levels increased from 5\%-10\% to 15\%-20\% on hydroxyurea. Hemoglobin concentration increased modestly (approximately 1 g/L) but significantly across studies. The rate of hospitalization decreased in the single randomized, controlled trial and 5 observational studies by 56\% to 87\%, whereas the frequency of pain crisis decreased in 3 of 4 pediatric studies. New and recurrent neurologic events were decreased in 3 observational studies of hydroxyurea compared with historical controls. Common adverse events were reversible mild-to-moderate neutropenia, mild thrombocytopenia, severe anemia, rash or nail changes (10\%), and headache (5\%). Severe adverse events were rare and not clearly attributable to hydroxyurea. Hydroxyurea reduces hospitalization and increases total and fetal hemoglobin levels in children with severe sickle cell anemia. There was inadequate evidence to assess the efficacy of hydroxyurea in other groups. The small number of children in long-term studies limits conclusions about late toxicities. [\hyperlink{Trihexyphenidyl Hydrochloride}{PMID: 19047254}, John J Strouse et al., 2008]

\hypertarget{pmid_7767425}{T}o describe the epidemiologic findings associated with the use of methylphenidate hydrochloride among children aged 0 to 19 years in Michigan. A population-based data set of all prescriptions filed with the Michigan Triplicate Prescription Program during February and March 1992 was analyzed, maintaining complete anonymity. State of Michigan. All patients receiving a prescription for methylphenidate who are residents of Michigan, and all physicians prescribing methylphenidate. None. Descriptive data. Eleven of 1000 Michigan residents between the ages of 0 and 19 years received a prescription for methylphenidate during the study period. Eighty-four percent were boys. Boys aged 10 or 11 years received more prescriptions for methylphenidate than any other age group--43 per 1000. The number of children receiving prescriptions for methylphenidate ranged from 2.5 to 28 per 1000. The range for boys aged 10 or 11 years was from 9.6 to 117 per 1000. Primary care physicians wrote 84\% of all prescriptions; pediatricians wrote 59\% of the prescriptions for patients younger than 20 years old. Half of the prescriptions written by pediatricians were written by 5\% of the pediatricians in the state. Michigan has been among the states with the highest per capita consumption of methylphenidate for the past 10 years. The major use of methylphenidate is for treatment of attention deficit hyperactivity disorder. The number of boys in Michigan aged 10 or 11 years who were treated with methylphenidate was similar to the national prevalence of the disorder, 3\% to 5\%. A tenfold variation was noted in the percentage of children medicated when the data were analyzed by county. Relatively few pediatricians account for the largest proportion of prescriptions. Future studies are needed to link the use of methylphenidate with diagnostic and treatment considerations in attention deficit hyperactivity disorder. [\hyperlink{Trihexyphenidyl Hydrochloride}{PMID: 7767425}, M D Rappley et al., 1995]

\hypertarget{pmid_1056551}{T}he concentration of hexachlorophene was determined in serial blood samples taken from seven premature infants washed with pHisoHex. Results indicated that after a single wash with 5ml of pHisoHex, blood concentrations reached a maximum of 0.75-1.20mug/ml two to four days after application. The results obtained in this study confirm that the dermal absorption of hexachlorophene is greater in premature than in full-term infants. Use of hexachlorophene for infants. Use of hexachlorophene for infants of birth weight less than 2kg has now been discontinued at Queen Mary Hospital and the amount of pHisoHex used in the initial toileting of infants with birth weight more than 2kg has been limited to a single application of 5ml. [\hyperlink{Trihexyphenidyl Hydrochloride}{PMID: 1056551}, S J Greaves et al., 1975]

\hypertarget{pmid_23639119}{T}he Test of Everyday Attention for Children (TEA-Ch) is a reliable neuropsychological assessment of attention control in children. Methylphenidate (MPH) is an effective treatment to improve attentional difficulties in children with attention deficit/hyperactivity disorder (ADHD). Previous studies investigating the effects of MPH on attention performance of children with ADHD have produced mixed results and prior MPH usage may have confounded these results. No previous study has tested the effects of MPH on the entire TEA-Ch battery. This study investigated the effects of MPH on attention performance using the entire TEA-Ch in 51 medication-naïve children with ADHD compared with 35 nonmedicated typically developing children. All children were tested at baseline and after 6 weeks: The children with ADHD were medication-naïve at baseline, received MPH for 6 weeks and were tested whilst on medication at the second testing session. A beneficial effect of MPH administration was found on at least one subtest of each of the three forms of attention (selective, sustained, and attentional control) assessed by the TEA-Ch, independent of practice effects. MPH aided performance on the TEA-Ch tasks that were inherently nonarousing and that might require top-down control of attention. It is recommended that the TEA-Ch measures--Sky Search Count (selective attention),Score! (sustained attention), Creature Counting Time Taken for older children (attentional control), and Same Worlds (attentional control) be prioritized for use in future pharmacological studies using MPH. [\hyperlink{Trihexyphenidyl Hydrochloride}{PMID: 23639119}, Kate Paton et al., 2014]

\section*{Clevidipine}
\subsection*{Result}
\subsubsection*{Answer}

Unknown

\subsubsection*{{Explanation}}
\hypertarget{Clevidipine}
To determine if clevidipine is safe for use in children, I reviewed all available abstracts for targeted pediatric safety studies. The following summarizes the evidence by age range:

1. **Neonates, Infants, and Young Children (including <1 year):**
   - A retrospective cohort study reviewed clevidipine use in 38 pediatric patients on mechanical circulatory support, with a median age of 2.7 years and including neonates. No hypotensive or code events were reported, and the study concluded that clevidipine was effective for hypertension management and not associated with adverse events in this population. This is the largest pediatric cohort and included neonates, suggesting preliminary evidence of safety in this age group [\hyperlink{pmid_32796396}{PMID: 32796396}, May Wu et al., 2020].
   - Another retrospective study included 14 patients aged 11 months to 15 years undergoing congenital heart disease surgery. Clevidipine was effective for perioperative blood pressure control, with only minor adverse events (increased heart rate in two patients). However, it was not effective during cooling and cardiopulmonary bypass. No serious safety concerns were reported [\hyperlink{pmid_22477825}{PMID: 22477825}, Joseph D Tobias et al., 2011].

2. **Children and Adolescents (ages 9–18 years):**
   - A retrospective review of 10 pediatric patients aged 9–18 years found clevidipine effective for perioperative blood pressure control. No excessive hypotension was noted; two patients required metoprolol for reflex tachycardia, and one had elevated triglycerides. The study concluded that clevidipine was effective and did not report serious safety issues [\hyperlink{pmid_20837631}{PMID: 20837631}, Eric Towe et al.].
   - A case report described apparent resistance to clevidipine in a 13-year-old girl, but no safety concerns were raised; rather, the focus was on efficacy in the context of drug interactions [\hyperlink{pmid_27186138}{PMID: 27186138}, M Alysse Jacklen et al., 2014].

3. **General Pediatric Population:**
   - A review article notes that large clinical trials regarding the safety and efficacy of clevidipine in children are lacking, and only preliminary data are available [\hyperlink{pmid_24275788}{PMID: 24275788}, Joseph D Tobias et al., 2013].
   - Another review states that there is a lack of large-scale clinical trial data on the efficacy and safety of clevidipine in children during the perioperative period [\hyperlink{pmid_34472038}{PMID: 34472038}, Bo Xu et al., 2022].

**Summary:**  
There are several small, retrospective studies and case reports describing the use of clevidipine in children, including neonates, infants, and adolescents. These studies generally report effective blood pressure control and do not identify serious safety concerns, but they are limited by small sample sizes, retrospective design, and lack of control groups. There are no large, prospective, randomized controlled trials specifically designed to assess the safety of clevidipine in children. Therefore, while preliminary data suggest clevidipine may be safe in pediatric populations (including neonates and infants), definitive safety has not been established due to the absence of large, targeted safety studies.

\subsection*{Abstracts}
\hypertarget{pmid_20837631}{C}levidipine is a third-generation calcium channel antagonist of the dihydropyridine group. Like nicardipine, its primary physiologic effect is vasodilation, primarily of the arterial system with limited effects on capacitance vessels. Unlike other direct-acting vasodilators, it has an ultrashort half-life due to its metabolism by nonspecific blood and tissue esterases. To date, the majority of clinical experience with clevidipine has been in the adult cardiac surgery population, with no reports regarding its use in the pediatric population. We retrospectively reviewed our preliminary experience with the use of this novel agent in a cohort of 10 pediatric-aged patients ranging in age from 9 to 18 years. The indications for the use of clevidipine included control of perioperative hypertension in 4 patients, to provide controlled hypotension during orthopedic surgical procedures in 5 patients, and to improve distal perfusion during a toe-to-finger implant in 1 patient. One patient who presented to the operating room with hypertension received clevidipine preoperatively, intraoperatively, and postoperatively; 7 other patients received clevidipine only intraoperatively while the other 2 patients received clevidipine intraoperatively and postoperatively. The clevidipine infusion was started at 0.5 to 1 μg/kg per minute and titrated up to 3.5 μg/kg per minute as needed. No excessive hypotension was noted; however, intermittent doses of metoprolol were required to control reflex tachycardia in 2 of the 10 patients and an elevated triglyceride level was noted in 1 patient. Our preliminary experience demonstrates the efficacy of clevidipine for blood pressure control during the perioperative period. [\hyperlink{Clevidipine}{PMID: 20837631}, Eric Towe et al., ]

\hypertarget{pmid_20172984}{T}he pharmacology, pharmacokinetics, efficacy, safety, dosage and administration, and place in therapy of clevidipine are reviewed. Clevidipine is a new lipophilic, short-acting, third-generation dihydropyridine calcium channel blocker (CCB) approved for use in the management of acute hypertension when oral agents are not feasible. It exerts its hemodynamic effects through selective arterial vasodilation without effects on the venous circulation. Clevidipine has a half-life of approximately two minutes after i.v. administration, resulting in very rapid onset and offset of antihypertensive action. Unlike many current i.v. antihypertensive agents that are metabolized by the kidneys or liver, clevidipine is metabolized in the blood and tissues and does not accumulate in the body. Clevidipine does not appear to inhibit or induce cytochrome P-450 isoenzymes. Several Phase III clinical trials have reported the clinical efficacy and safety of clevidipine in patients with severe hypertension and in cardiac surgical patients with perioperative hypertension. The most frequent adverse events reported in clinical trials of clevidipine were headache, nausea, and vomiting. Risk of rebound hypertension, especially in patients not transitioned from clevidipine to oral antihypertensive therapy after prolonged infusions, should be monitored for at least eight hours after the drug is discontinued. Clevidipine, a novel third-generation dihydropyridine CCB, has demonstrated efficacy and safety in patients with acute hypertension and preoperative, perioperative, and postoperative hypertension. While its short duration of action and short half-life are appropriate for use in acute settings, more information on its safety is needed to assess its appropriate use in therapy. [\hyperlink{Clevidipine}{PMID: 20172984}, Uche Anadu Ndefo et al., 2010]

\hypertarget{pmid_27186138}{V}arious factors may be responsible for blood pressure alterations during perioperative care. When these physiologic alterations require treatment, several therapeutic options are available. Clevidipine is an ultrashort-acting, intravenous L-type calcium channel antagonist of the dihydropyridine class. Anecdotal experience has demonstrated its efficacy in various clinical scenarios in the pediatric population. We report apparent resistance to the vasodilatory effects of clevidipine in a 13-year-old girl who presented for anesthetic care during posterior spinal fusion for neuromuscular scoliosis whose chronic medication regimen included aripiprazole and methylphenidate for the treatment of depression and attention-deficit/hyperactivity disorder. We discuss the potential interaction of aripiprazole and methylphenidate with the calcium channel antagonists and cellular mechanisms responsible for the resistance to the vasodilatory effects of clevidipine.  [\hyperlink{Clevidipine}{PMID: 27186138}, M Alysse Jacklen et al., 2014] Various pharmacologic agents have been used for perioperative BP control in pediatric patients, including sodium nitroprusside, nitroglycerin, β-adrenergic antagonists, fenoldopam, and calcium channel antagonists. Of the calcium antagonists, the majority of the clinical experience remains with the dihydropyridine nicardipine. Clevidipine is a short-acting, intravenous calcium channel antagonist of the dihydropyridine class. It undergoes rapid metabolism by non-specific blood and tissue esterases with a half-life of less than 1 minute. As a dihydropyridine, its cellular and end-organ effects parallel those of nicardipine. The clevidipine trials in the adult population have demonstrated efficacy in rapidly controlling BP in various clinical scenarios with a favorable adverse effect profile similar to nicardipine. Data from large clinical trials regarding the safety and efficacy of clevidipine in children is lacking. This manuscript aims to review the commonly used pharmacologic agents for perioperative BP control in children, discuss the role of calcium channel antagonists such as nicardipine, and outline the preliminary data regarding clevidipine in the pediatric population.  [\hyperlink{Clevidipine}{PMID: 27186138}, Joseph D Tobias et al., 2013] Limited data exist regarding the management of hypertension in pediatric patients on mechanical circulatory support. Hypertension is a known risk factor for stroke and low cardiac output in patients requiring mechanical circulatory support and a narrow therapeutic window of blood pressure is often targeted. Traditional short-acting infusions to treat hypertension, such as sodium nitroprusside, may lead to accumulation of toxic metabolites in patients with renal dysfunction. Our primary objective was to describe use of clevidipine, a continuous short-acting calcium channel blocking medication, for blood pressure control in pediatric patients on mechanical circulatory support. Single-center retrospective cohort study. A 26-bed quaternary cardiovascular ICU in a university-based pediatric hospital in California. Mechanical circulatory support patients admitted to cardiovascular ICU who received clevidipine infusions between October 1, 2016, and March 31, 2019. Clevidipine infusion. Data from a cohort of 38 patients who received a total of 45 clevidipine infusions were reviewed. The cohort had a median age of 2.7 years and included neonates. No patient had record of hypotensive events, code events, or received low-dose epinephrine or code-dosed epinephrine related to a clevidipine infusion. Median duration of clevidipine infusion was 4.1 days (1.5-9.2 d). Eleven patients transitioned from clevidipine to enteral antihypertensive agents, and 26 clevidipine infusions were administered as a single agent without sodium nitroprusside. Seven patients were switched from sodium nitroprusside to clevidipine to avoid cyanide toxicity, a majority of whom had elevated serum creatinine. In this pediatric cardiac cohort, clevidipine infusions were effective at hypertension management and were not associated with hypotensive or code events. This report details the largest cohort and longest duration of clevidipine administration within a pediatric population and did not demonstrate hypotensive events, even among neonatal populations. Clevidipine may be a reasonable cost-effective alternative antihypertensive medication compared to traditional short-acting agents. [\hyperlink{Clevidipine}{PMID: 27186138}, May Wu et al., 2020]

\hypertarget{pmid_20412001}{D}rugs used to acutely lower blood pressure have specific indications and precautions for use. Clevidipine is a third-generation parenteral dihydropyridine calcium channel blocker that received United States Food and Drug Administration approval in August 2008 for blood pressure reduction when oral therapy is not feasible or desirable. Formulated as an injectable oil-in-water emulsion, the drug is a short-acting arterial-selective vasodilator. Clinical efficacy and safety trials of clevidipine have primarily focused on blood pressure management during cardiac surgery and in patients with acute severe hypertension (in intensive care units and emergency departments). In phase III trials, clevidipine demonstrated efficacy in blood pressure lowering, with a relatively low occurrence of adverse events. Reflex tachycardia, atrial fibrillation, and acute renal failure were observed in these studies and merit additional analysis. The lack of specific clinical outcomes documenting improved morbidity and mortality rates as compared with other agents, the small numbers of treated patients, and concerns regarding the lipid formulation necessitate further investigation to help define the therapeutic role of clevidipine. [\hyperlink{Clevidipine}{PMID: 20412001}, Abbie L Erickson et al., 2010]

\hypertarget{pmid_10220482}{C}levidipine is a new vascular selective calcium channel antagonist of the dihydropyridine type, structurally related to felodipine. Clinical trials have shown that the drug can be used to effectively control the blood pressure in connection with cardiac surgical procedures. The compound is tailored to be a short-acting drug and, due to incorporation of an ester linkage into the drug molecule, clevidipine is rapidly metabolized by ester hydrolysis. The pharmacokinetics of clevidipine and its primary metabolite, H 152/81, were studied in rats, rabbits, and dogs. In addition, the influence of the pharmacokinetics on the effect on mean arterial blood pressure was evaluated in anesthetized dogs. Compartmental nonlinear mixed effect regression analysis was used to calculate the population mean and individual pharmacokinetics of clevidipine, whereas nonlinear regression analysis of individual data was used to determine the pharmacokinetics of the primary metabolite. A linked Emax model was fitted to the individual pharmacodynamic/pharmacokinetic data in dogs. According to the results, clevidipine is a high-clearance drug with a relatively small volume of distribution, resulting in an extremely short half-life in all species studied. The median initial half-life of the individual value (Bayesian estimates) is 12, 20, and 22 s in the rabbit, rat, and dog, respectively. The primary metabolite is a high-clearance compound in the dog, whereas it is a low-clearance compound in the rat. A significant gender difference in the clearance of the metabolite was observed in the rat. The mean maximum reduction in arterial blood pressure is 38 +/- 12\% (Emax) and is achieved at 85 +/- 46 nM (EC50). The half-life for reaching equilibrium between the central and the effect compartment (T1/2ke0) is 47 +/- 49 s. [\hyperlink{Clevidipine}{PMID: 10220482}, H Ericsson et al., 1999]

\hypertarget{pmid_19419254}{C}levidipine is an arterial, selective, dihydropyridine calcium channel blocker with an ultrashort half-life. In this prospective, randomized, open-label, parallel-comparison trial series, the safety and efficacy of intravenous clevidipine with nitroglycerin, sodium nitroprusside and nicardipine in hypertensive patients during cardiac surgery were compared. No differences in the incidences of myocardial infarction, stroke or renal dysfunction were observed between treatment groups. Mortality was similar between the clevidipine-nitrogylcerine- and clevidipine-nicardipine-treated groups, whereas mortality appeared to be greater in the sodium nitroprusside group compared to clevidipine (p = 0.04 in a univariant analysis). Clevidipine was significantly more effective in blood pressure control compared with nitroglycerin (p = 0.0006) or sodium nitroprusside (p = 0.003) and was associated with fewer blood pressure excursions compared with nicardipine as a predetermined blood pressure range was narrowed. [\hyperlink{Clevidipine}{PMID: 19419254}, Solomon Aronson et al., 2009]

\hypertarget{pmid_18778189}{C}levidipine is an investigational agent undergoing late-stage clinical development to evaluate its potential as a novel short-acting intravenous agent for treating acute hypertension, either in hypertensive emergencies encountered in the emergency department and intensive care units, or in the perioperative period. Clevidipine has been evaluated in four Phase I studies, nine Phase II studies and six Phase III clinical studies. The patient populations studied include healthy volunteers, patients with essential hypertension, patients undergoing cardiac surgery, and patients presenting to the emergency department with hypertensive emergencies. Studies providing comparative data of clevidipine versus nitroglycerin, nicardipine or sodium nitroprusside are also available. This article reviews the results of clinical studies evaluating the pharmacological properties, clinical effects and safety profiles of clevidipine in various patient populations. Clevidipine is an effective agent for reducing acute elevation in blood pressure in various settings, including hypertensive emergencies and perioperative hypertension with a good safety profile. [\hyperlink{Clevidipine}{PMID: 18778189}, Ika Noviawaty et al., 2008]

\hypertarget{pmid_17898366}{C}levidipine is an ultrashort-acting, third-generation IV dihydropyridine calcium channel blocker that exerts rapid and titratable arterial blood pressure reduction, with fast termination of effect due to metabolism by blood and tissue esterases. As an arterial-selective vasodilator, clevidipine reduces peripheral vascular resistance directly, without dilating the venous capacitance bed. In this randomized, double-blind, placebo-controlled multicenter trial we evaluated the efficacy and tolerability of clevidipine in treating preoperative hypertension. One-hundred-fifty-two patients scheduled for cardiac surgery with current or recent hypertension were randomized to receive clevidipine or placebo preoperatively. One-hundred-five patients met postrandomization entrance criteria (systolic blood pressure [SBP] > or =160 mm Hg after inserting an arterial catheter) for reduction by > or =15\% from baseline in SBP. The patients thus received infusions of clevidipine (0.4-8.0 microg x kg(-1) x min(-1)) or 20\% lipid emulsion (placebo) for at least 30 min. Treatment failure was defined as failure to reduce SBP by > or =15\% from baseline or discontinuance of drug for any reason. Patients treated with clevidipine demonstrated a 92.5\% rate of treatment success and a significantly lower rate of treatment failure (7.5\%, 4 of 53) than patients receiving placebo (82.7\%, 43 of 52; P < 0.0001). Clevidipine achieved target blood pressures (SBP reduced by > or =15\%) at a median of 6.0 min (95\% confidence interval 6-8 min). A modest increase in heart rate from baseline occurred during clevidipine administration. Adverse events for each treatment group were similar. Clevidipine was effective in rapidly decreasing blood pressure preoperatively to targeted blood pressure levels and was well tolerated in patients scheduled for cardiac surgery. [\hyperlink{Clevidipine}{PMID: 17898366}, Jerrold H Levy et al., 2007]

\hypertarget{pmid_22477825}{T}o determine the efficacy and adverse effect profile of clevidipine when used for perioperative blood pressure (BP) control during surgery for congenital heart disease (CHD). We retrospectively reviewed our experience with the perioperative use of clevidipine in pediatric-aged patients undergoing surgery for CHD. The study cohort included 14 patients ranging from 11 months to 15 years (7.4 ± 4.6 years) and weighing from 5 to 41 kg (21.8 ± 11.1 kg). Clevidipine was administered as a continuous infusion for control of either postoperative BP or intraoperative mean arterial pressure (MAP) during cooling and cardiopulmonary bypass (CPB). It was administered as a bolus for BP control during emergence from anesthesia following cardiac surgery. The continuous infusion was started at 1 mcg/kg/min and increased in increments of 0.5 to 1 mcg/kg/min as needed. For postoperative BP control, dosing requirements varied from 1 to 7 mcg/kg/min (mean = 2.0 ± 1.2 mcg/kg/min). The target BP was achieved within 5 minutes in all patients. Two patients were treated with intravenous or oral propranolol for an increase in heart rate (HR) while receiving clevidipine. Despite doses up to 10 mcg/kg/min, effective control of MAP could not be achieved during CPB and cooling (core body temperature 28°C to 32°C). Bolus doses of clevidipine (10 to 15 mcg/kg) controlled BP during emergence from anesthesia with a decrease of the MAP from 97 ± 6 mm Hg to 71 ± 5 mm Hg (p<0.01). Clevidipine is effective for perioperative BP control in infants and children with CHD; however, it does not appear effective in controlling MAP during cooling and CPB. [\hyperlink{Clevidipine}{PMID: 22477825}, Joseph D Tobias et al., 2011]

\hypertarget{pmid_18635468}{A}cute postoperative hypertension is a well-known complication of cardiac surgery and is associated with postoperative morbidity. Clevidipine, an ultrashort-acting, third-generation dihydropyridine calcium channel blocker, exerts vascular-selective, arterial-specific vasodilation to decrease arterial blood pressure without negatively impacting cardiac function. In this double-blind, placebo-controlled trial, we examined the efficacy and safety of clevidipine in treating postoperative hypertension in cardiac surgery patients. Two hundred six patients undergoing cardiac surgery were randomized preoperatively. Of these, 110 met postrandomization inclusion criteria for the study [systolic blood pressure (SBP) >or=140 mm Hg within 4 h of admission to a postoperative setting, and clinically assessed as needing SBP reduction by >or=15\% from baseline]. Patients received an infusion of either clevidipine (0.4-8.0 microg kg(-1) min(-1)) or 20\% lipid emulsion (placebo) for 30 min to a maximum of 1 h unless treatment failure occurred sooner. The primary end point was the incidence of treatment failure, defined as the inability to decrease SBP by >or=15\% from baseline, or the discontinuation of study treatment for any reason within the 30-min period after study drug initiation. Clevidipine-treated patients had a significantly lower incidence of treatment failure than placebo patients [8.2\% (5 of 61) vs 79.6\% (39 of 49), P < 0.0001]. Treatment success was achieved in 91.8\% of clevidipine-treated patients. Median time to target SBP with clevidipine was 5.3 min (95\% confidence interval, 4-7 min). No clinically significant increase in heart rate from baseline was observed. Adverse event rates were similar for both treatment groups. Clevidipine is effective and safe in the rapid treatment of acute postoperative hypertension after cardiac surgery. [\hyperlink{Clevidipine}{PMID: 18635468}, Neil Singla et al., 2008]

\hypertarget{pmid_22457015}{C}levidipine is a rapidly-acting intravenous dihydropyridine antihypertensive acting via calcium channel blockade. This was a randomized, single-blind, parallel-design study of a 72-h continuous clevidipine infusion. Doses of 2, 4, 8, or 16.0 mg/h or placebo were evaluated in 61 subjects with mild to moderate essential hypertension. IV clevidipine or placebo was initiated at 2.0 mg/h and force-titrated in doubling increments every 3 min to target dose, then maintained for 72 h. Blood pressure and heart rate were measured during infusion, and for 4, 6 and 8 h after termination of infusion, although oral therapy could be restarted at 4 h. Clevidipine blood levels were obtained during infusion and for 1 hour after termination. Rapid onset of drug effect occurred at all clevidipine dose levels, with consistent pharmacokinetics and rapid offset after 72-h infusion. No evidence of tolerance to the clevidipine drug effect was observed at any dose level over the 72-h infusion. No evidence of rebound hypertension was found for either 4 or 6 h after termination of the clevidipine infusion. At 8 h following cessation of clevidipine, blood pressure was not significantly higher than at baseline. Placebo-treated subjects had blood pressures lower than baseline at 8 h following infusion termination; hence, placebo-adjusted blood pressures tended to be slightly higher than baseline. This study supports the use of up to 72 h of IV clevidipine therapy for the management of blood pressure, with consistent pharmacokinetic/pharmacodynamic characteristics and context insensitive half-life across the dose ranges evaluated. [\hyperlink{Clevidipine}{PMID: 22457015}, William B Smith et al., 2012]

\hypertarget{pmid_34472038}{A}cute hypertension, which may damage blood vessels, causes irreversible organ damage to the vasculature, central nervous system, kidney, and heart. Clevidipine, the first third-generation calcium channel antagonist approved by the Food and Drug Administration (FDA) in the past 20 years, is an ultra-short-acting calcium channel blocker that inhibits L-type calcium channels with high clearance and low distribution, can be rapidly metabolized into the corresponding inactive acid, and is rapidly hydrolyzed into inactive metabolites by esterase in arterial blood. Clevidipine is the same as nicardipine in that the main physiological effect is vasodilation and the main target is the arterial system, which has a limited effect on capacitor vessels. Unlike nitroglycerin, clevidipine has a limited effect on preload. In contrast to other direct-acting vasodilators, clevidipine has an ultra-short half-life due to metabolism by nonspecific blood and tissue esterases. Clevidipine trials conducted in adult populations have proven that it can rapidly control blood pressure in cardiac surgery situations and that adverse reactions to clevidipine are similar to those with other antihypertensive agents. In recent years, clinical trials have shown that clevidipine has excellent blood pressure-lowering capability in patients with acute neurological injury (hemorrhage, stroke, and subarachnoid and acute intracerebral hemorrhage), those  undergoing coronary artery bypass graft or spinal surgery, and in those with cerebral aneurysm/pheochromocytoma, acute heart failure, acute aortic syndromes, or renal insufficiency with severe hypertension, and it is equivalent to commonly used blood pressure-lowering medicines such as nicardipine or nitroglycerin. However, there is a lack of large-scale clinical trial data on the efficacy and safety of clevidipine in children during the perioperative period. [\hyperlink{Clevidipine}{PMID: 34472038}, Bo Xu et al., 2022]

\hypertarget{pmid_20088748}{C}levidipine butyrate is an ultrashort-acting intravenous dihydropyridine calcium-channel blocker that has been approved by the FDA for the reduction of blood pressure when oral therapy is not feasible. Hypertension is a global disease that affects more than 1 billion people worldwide and 75 million people in the USA. There are multiple agents available for the management of hypertension. The acute setting is where the challenge arises for developing new agents that not only decrease, but more importantly, optimally control blood pressure. Many drugs lower blood pressure; however, only a few have the capacity to precisely control hypertension in the acute phase. Clevidipine has unique pharmacodynamic and pharmacokinetic properties that enable the fast, safe and adequate reduction of blood pressure in hypertensive emergencies, with unique precision necessary to maintain the target blood pressure range. Its use in different clinical settings has been evaluated in several Phase I, II and III clinical studies. It is easily administered and titrated with minimal side effects, achieves fast control with low doses, is highly successful as monotherapy and allows excellent transition to oral medication. Thus, clevidipine is a promising new agent for the management of acute hypertension in a variety of clinical settings. [\hyperlink{Clevidipine}{PMID: 20088748}, Sergio D Bergese et al., 2010]

\hypertarget{pmid_28060017}{N}evirapine is the only nonnucleoside reverse transcriptase inhibitor currently available as a paediatric fixed-dose-combination tablet and is widely used in African children. Nonetheless, the number of investigations into pharmacokinetic determinants of virological suppression in African children is limited, and the predictive power of the current therapeutic range was never evaluated in this population, thereby limiting treatment optimization. We analysed data from 322 African children (aged 0.3-13 years) treated with nevirapine, lamivudine, and either abacavir, stavudine, or zidovudine, and followed up to 144 weeks. Nevirapine trough concentration (Cmin) and other factors were tested for associations with viral load more than 100 copies/ml and transaminase increases more than grade 1 using proportional hazard and logistic models in 219 initially antiretroviral treatment (ART)-naive children. Pre-ART viral load, adherence, and nevirapine Cmin were associated with viral load nonsuppression [hazard ratio = 2.08 (95\% confidence interval (CI): 1.50-2.90, P < 0.001) for 10-fold higher pre-ART viral load, hazard ratio = 0.78 (95\% CI: 0.68-0.90, P < 0.001) for 10\% improvement in adherence, and hazard ratio = 0.94 (95\% CI: 0.90-0.99, P = 0.014) for a 1 mg/l increase in nevirapine Cmin]. There were additional effects of pre-ART CD4 cell percentage and clinical site. The risk of virological nonsuppression decreased with increasing nevirapine Cmin, and there was no clear Cmin threshold predictive of virological nonsuppression. Transient transaminase elevations more than grade 1 were associated with high Cmin (>12.4 mg/l), hazard ratio = 5.18 (95\% CI 1.95-13.80, P < 0.001). Treatment initiation at lower pre-ART viral load and higher pre-ART CD4 cell percentage, increased adherence, and maintaining average Cmin higher than current target could improve virological suppression of African children treated with nevirapine without increasing toxicity. [\hyperlink{Clevidipine}{PMID: 28060017}, Andrzej Bienczak et al., 2017]

\hypertarget{pmid_10445679}{T}he pharmacokinetics of clevidipine, a potent short-acting vascular-selective calcium antagonist, was investigated during steady state and the postinfusion period in patients with mild to moderate hypertension. Furthermore, the dose-effect and blood concentration-effect relations and the tolerability of the drug were studied. Twenty patients were randomized to clevidipine intravenously at target dose rates of 0.18, 0.91, 2.74, and 5.48 microg/kg/min, respectively, or placebo. Each patient received in random order three infusion rates of clevidipine or placebo during three separate study days. Dose-dependent reduction in blood pressure and a modest increase in heart rate were noted. The extremely high clearance value and the small volume of distribution resulted in short half-lives of clevidipine, 2.2 and 16.8 min, respectively. The blood concentration and dose rate producing half the maximal effect (i.e. EC50 and ED50) were approximately 25 nM and 1.5 microg/kg/min, respectively. There was a linear relation between blood concentration and dose rate in the range studied. Clevidipine was safe and generally well tolerated; one patient was excluded because of adverse events at 2.74 microg/kg/min. In conclusion, clevidipine is a high-clearance calcium antagonist that may become a valuable contribution to the drugs used in conditions in which precise and rapid control of blood pressure is needed. [\hyperlink{Clevidipine}{PMID: 10445679}, J H Schwieler et al., 1999]

\hypertarget{pmid_21091269}{A}cute and severe hypertension is common, especially in patients with renal dysfunction (RD). Clevidipine is a rapidly acting (t½∼1 min) intravenous (IV) dihydropyridine calcium-channel blocker metabolized by blood and tissue esterases and may be useful in patients with RD. The purpose of this analysis was to assess the safety and efficacy of clevidipine in patients with RD. VELOCITY, a multicenter open-label study of severe hypertension, enrolled 126 patients with persistent systolic blood pressure (SBP) >180 mmHg. Investigators pre-specified a SBP initial target range (ITR) for each patient to be achieved within 30 min. Blood pressure monitoring was by cuff. Clevidipine was infused via peripheral IV at 2 mg/h for at least 3 min, then doubled every 3 min as needed to a maximum of 32 mg/h (non-weight-based treat-to-target protocol). Per protocol, clevidipine was continued for at least 18 h (96 h maximum). RD was diagnosed and reported as an end-organ injury by the investigator and was defined as requiring dialysis or an initial creatinine >2.0 mg/dl. Primary endpoints were the percentage of patients within the ITR by 30 min and the percentage below the ITR after 3 min of clevidipine infusion. Of the 24 patients with moderate to severe RD, most (13/24) were dialysis dependent. Forty-six percent were male, with mean age 51 ± 14 years; 63\% were black and 96\% had a hypertension history. Median time to achieve the ITR was 8.5 min. Almost 90\% of patients reached the ITR in 30 min without evidence of overshoot and were maintained on clevidipine through 18 h. Most patients (88\%) transitioned to oral antihypertensive therapy within 6 h of clevidipine termination. This report is the first demonstrating that clevidipine is safe and effective in RD complicated by severe hypertension. Prolonged infusion maintained blood pressure within a target range and allowed successful transition to oral therapy. [\hyperlink{Clevidipine}{PMID: 21091269}, W Frank Peacock et al., 2011]

\hypertarget{pmid_25145624}{O}lanzapine is frequently prescribed in young children for psychiatric conditions. It may be an option for chemotherapy-induced nausea and vomiting (CINV) control in children. The objective of this review was to describe the safety of olanzapine in children less than 13 years of age to determine if safety concerns would be a barrier to its use for CINV prevention. Electronic searches were performed in MEDLINE, EMBASE, Cochrane Central Register of Controlled Trials, Web of Science and Scopus. All studies in English reporting adverse effects associated with olanzapine use in children younger than 13 years or with a mean/median age less than 13 years were included. Adverse outcomes were synthesized for prospective studies. A total of 47 studies (17 prospective) involving 387 children aged 0.6-18 years were included; nine described olanzapine poisonings. Weight gain or sedation were reported in 78 \% [95 \% confidence interval (CI) 63-95] and 48 \% (95 \% CI 35-67), respectively. Extrapyramidal symptoms or electrocardiogram abnormalities were reported in 9 \% (95 \% CI 4-21) and 14 \% (95 \% CI 7-26), respectively. Elevation in liver function tests or blood glucose abnormalities were reported in 7 \% (95 \% CI 2-20) and 4 \% (95 \% CI 1-17), respectively. No deaths were attributed to olanzapine. No studies were identified with a primary focus on evaluating safety, and the adverse effects reported in the included studies were heterogeneous. Most adverse events associated with olanzapine use in children less than 13 years of age are of minor clinical significance. These findings support the exploration of olanzapine for the prevention of CINV in children in future trials. [\hyperlink{Clevidipine}{PMID: 25145624}, Jacqueline Flank et al., 2014]

\hypertarget{pmid_20029305}{I}n this article, you'll learn about six new drugs, including: * clevidipine, an I.V. antihypertensive for use when oral medication isn't feasible * romiplostim and eltrombopag olamine, new agents for chronic immune thrombocytopenic purpura * C1 inhibitor (human) for hereditary angioedema. Unless otherwise specified, the information in the following summaries applies to adults, not children. Consult the package insert for information about each drug's safety during pregnancy and breast-feeding. Also consult a pharmacist, the package insert, or a comprehensive drug reference for more details on precautions, drug interactions, and adverse reactions for all these drugs. [\hyperlink{Clevidipine}{PMID: 20029305}, Daniel A Hussar et al., ]

\hypertarget{pmid_24912730}{S}tudies on the efficacy and tolerability of rufinamide in infants and young children are scarce. Here we report on an open, retrospective, and pragmatic study about safety and efficacy of rufinamide in children aged less than four years, in terms of seizures types and epilepsy syndromes. Forty children (mean age 39.5 months; range 22-48) were enrolled in the study. The mean follow-up period was 12.2 months (range 5-21). Rufinamide was initiated at a mean age of 26.7 months (range 12-42). Final rufinamide mean dosage was 31.5 mg/kg/day if associated with valproic acid and 44.2 mg/kg/day if not. The highest seizure reduction rate was observed in the epileptic spasms (46\%) and drop attacks (42\%) groups. Seizure reduction was also observed in tonic seizures (35\%) and in the focal seizure (30\%) groups. In terms of epilepsy syndrome, rufinamide was effective in Lennox-Gastaut syndrome. Results were very poor for those affected by Dravet's syndrome. Globally, responder rate was 27.5\%, including two (5\%) patients seizure-free. Adverse reactions occurred in 37.5\% of children and were mainly represented by vomiting, drowsiness, irritability, and anorexia. Discontinuation rate due to treatment-emergent adverse events was 15\%. The present study concludes that rufinamide may be a safe and effective drug for a broad range of seizures and epilepsy syndromes in infants and young children and represents a valid therapeutic option in this population. [\hyperlink{Clevidipine}{PMID: 24912730}, Salvatore Grosso et al., 2014]

\hypertarget{pmid_11045391}{A}mlodipine has potential advantages in children since it can be dissolved into a liquid preparation and has a long elimination half-life, allowing for once-daily administration. The objective of this study was to compare the efficacy and compliance of amlodipine with that of standard long-acting calcium channel blockers (felodipine or nifedipine) in hypertensive children. A randomized, prospective, crossover study of 11 hypertensive children (9-17 years of age, 10 renal transplant patients) was performed with electronic monitoring of compliance. Each treatment arm was 30 days. No significant differences were observed in mean systolic (SBP) and diastolic blood pressures (DBP) between amlodipine and the other calcium channel blockers. Using 24-h blood pressure monitoring there were no significant differences over each drug treatment period in both mean day-time and night-time SBP and DBP. Patient compliance was similar in both the amlodipine and the nifedipine/felodipine treatment periods. These data suggest that amlodipine is as effective in pediatric nephrology patients as nifedipine and felodipine. Amlodipine may be optimally suited for treatment of young children because at present it is the only calcium channel blocker which can be administered once daily as a liquid preparation. [\hyperlink{Clevidipine}{PMID: 11045391}, J W Rogan et al., 2000]

\hypertarget{pmid_17561929}{T}here are more than 40 H(1)-antihistamines available worldwide. Most of these medications have never been optimally studied in prospective, randomized, double-masked, placebo-controlled trials in children. The aim was to perform a long-term study of levocetirizine safety in young atopic children. In the randomized, double-masked Early Prevention of Asthma in Atopic Children Study, 510 atopic children who were age 12-24 months at entry received either levocetirizine 0.125 mg/kg or placebo twice daily for 18 months. Safety was assessed by: reporting of adverse events, numbers of children discontinuing the study because of adverse events, height and body mass measurements, assessment of developmental milestones, and hematology and biochemistry tests. The population evaluated for safety consisted of 255 children given levocetirizine and 255 children given placebo. The treatment groups were similar demographically, and with regard to number of children with: one or more adverse events (levocetirizine, 96.9\%; placebo, 95.7\%); serious adverse events (levocetirizine, 12.2\%; placebo, 14.5\%); medication-attributed adverse events (levocetirizine, 5.1\%; placebo, 6.3\%); and adverse events that led to permanent discontinuation of study medication (levocetirizine, 2.0\%; placebo, 1.2\%). The most frequent adverse events related to: upper respiratory tract infections, transient gastroenteritis symptoms, or exacerbations of allergic diseases. There were no significant differences between the treatment groups in height, mass, attainment of developmental milestones, and hematology and biochemistry tests. The long-term safety of levocetirizine has been confirmed in young atopic children. [\hyperlink{Clevidipine}{PMID: 17561929}, F Estelle R Simons et al., 2007]

\hypertarget{pmid_23167664}{C}lozapine, a second generation antipsychotic which is relatively safe in overdose, has been used as an effective treatment alternative to traditional antipsychotics. The therapeutic use in children remains controversial. However, in accordance with the increasing prescription in adults, the accidental ingestion in childhood becomes more frequent. We report the youngest case of accidental clozapine ingestion. A 13-month-old girl presented with acute respiratory insufficiency and coma of unknown origin. The medical history, laboratory and radiological assessment did not link to aetiology until an almost spontaneous arousal after 22 h pointed towards intoxication. The initial standard drug screening using immunoassay had been negative. Hence, liquid chromatography mass spectrometry/mass spectrometry (LC-MS/MS) was performed, and clozapine was detected with a serum concentration of 736 ng/mL. This case illustrates the diagnostic and forensic pitfalls in a coma of unknown origin due to the limits of toxicological screening immunoassays. LC-MS/MS analysis by an established method showed clozapine metabolites (norclozapine and clozapine-N-oxide) are detectable for longer period, especially in urine, when compared with clozapine. The clinical course is presented in unique correlation with plasma and urine concentrations of clozapine and its metabolites. The elimination pattern of clozapine in toddlers is similar to adults, and the toxic dose was found to be lower when compared with school-age children and adults. [\hyperlink{Clevidipine}{PMID: 23167664}, N Toepfner et al., 2013]

\hypertarget{pmid_18534716}{W}e assess the safety and efficacy of intravenous clevidipine for treating patients with acute severe increase in blood pressure by using prespecified, non-weight-based titration dosing, with continuous maintenance infusion for 18 hours or longer. Prospective, open-label, single-arm evaluation of patients aged 18 years or older and presenting in the emergency department or ICU with severe hypertension (systolic blood pressure >180 mm Hg and/or diastolic blood pressure >115 mm Hg) and treated with clevidipine to achieve a predetermined, patient-specific systolic blood pressure target range. Clevidipine was initiated at 2 mg per hour and titrated as needed in doubling increments every 3 minutes to a maximum of 32 mg per hour, during 30 minutes, and then continued for a total duration of 18 to 96 hours. Study patients commonly presented with both acute hypertension and end-organ injury; 81\% (102/126) had demonstrable end-organ injury at baseline. Within 30 minutes of starting clevidipine, 88.9\% (104/117) of patients achieved target range. Median time to target range was 10.9 minutes. No concomitant intravenous antihypertensives were needed in 92.3\% (108/117) of patients receiving 18 hours or more of clevidipine infusion. Clevidipine was well tolerated with successful transition to oral antihypertensive therapy after infusion to a defined blood pressure target in 91.3\% (115/126) of patients. Clevidipine, dosed in a non-weight-based manner, was safe and effective in a cohort of patients with severe hypertension at a starting dose of 2 mg per hour, followed by simple titration during 18 hours or more of continuous infusion. Patients were effectively managed via simple blood pressure cuff monitoring throughout. [\hyperlink{Clevidipine}{PMID: 18534716}, Charles V Pollack et al., 2009]

\section*{Difluprednate}
\subsection*{Result}
\subsubsection*{Answer}

Yes (Ages 0-3 years, short-term use after cataract surgery)
Unknown (Ages >3 years or for other indications/long-term use)

\subsubsection*{{Explanation}}
\hypertarget{Difluprednate}
Based on the abstracts available, there is evidence from targeted studies evaluating the safety of difluprednate in children, specifically in the following age ranges:

1. **Ages 0-3 years**:  
A phase 3B, multicenter, randomized, double-masked, active-controlled study specifically evaluated the safety and efficacy of difluprednate 0.05\% ophthalmic emulsion in children aged 0-3 years who underwent uncomplicated cataract surgery. The study found that difluprednate had a safety and efficacy profile similar to prednisolone acetate 1\%. Adverse drug reactions included corneal edema (1 case) and increased intraocular pressure (IOP) or ocular hypertension (2 cases), which were also seen in the control group. Mean IOP values were slightly higher during treatment with difluprednate but normalized after cessation. The study concluded that difluprednate showed a similar safety profile to prednisolone acetate in this age group [\hyperlink{pmid_27367745}{PMID: 27367745}, M E Wilson et al., 2016].

2. **Pediatric patients (broader, but not precisely defined age range)**:  
A retrospective, observational case series evaluated 14 pediatric patients (26 eyes) with noninfectious uveitis treated with topical difluprednate. While difluprednate was effective in controlling inflammation, a high rate of steroid-induced IOP elevation (50\% of eyes) and cataract formation (39\% of eyes) was observed, with some requiring surgery. The authors recommend close monitoring due to these risks [\hyperlink{pmid_22265149}{PMID: 22265149}, Mark A Slabaugh et al., 2012]. However, this study does not provide a clear age breakdown and includes a small sample size.

3. **Case report in a child with pars planitis**:  
A single case report described a child with pars planitis treated with difluprednate for one year, who developed delayed, sudden-onset ocular hypertension and posterior subcapsular cataract, necessitating alternative treatment. The report suggests difluprednate may be useful short-term but highlights the risk of serious complications [\hyperlink{pmid_21034305}{PMID: 21034305}, Paul A Kurz et al., 2011].

4. **Other studies**:  
Other abstracts either focus on adults, do not specify pediatric age ranges, or are general reviews without targeted pediatric safety data.

**Summary**:  
- For children aged 0-3 years, a well-designed, targeted study supports that difluprednate is as safe as prednisolone acetate for short-term use after cataract surgery, with similar adverse event rates and no unexpected safety concerns.
- For pediatric patients with uveitis (age range not precisely defined), there is evidence of effectiveness but also a high risk of steroid-induced IOP elevation and cataract formation, indicating the need for close monitoring.
- For children outside the 0-3 year range, or for long-term use, the safety profile is less clear, and there are reports of significant adverse effects.

Therefore, difluprednate can be considered safe for short-term use in children aged 0-3 years after cataract surgery, based on targeted evidence. For other pediatric age groups or indications, safety is less certain due to limited or less rigorous data and reports of significant adverse effects.

\subsection*{Abstracts}
\hypertarget{pmid_27367745}{P}urposeTo evaluate safety and efficacy of difluprednate 0.05\% ophthalmic emulsion for treatment of postoperative inflammation after cataract surgery in pediatric patients.MethodsThis was a phase 3B, multicentre, randomized, double-masked, active-controlled study of patients aged 0-3 years who underwent uncomplicated cataract surgery in one eye, with/without intraocular lens implantation. Patients were randomized to receive difluprednate 0.05\% four times daily or prednisolone acetate 1\% for 14 days post surgery, followed by tapering for 14 days. Safety included evaluation of adverse events. Primary efficacy was the proportion of patients with an anterior cell grade of 0 (no cells) at day 14; secondary efficacy was a global inflammation score.ResultsForty patients were randomized to each treatment group. Adverse drug reactions included corneal oedema (difluprednate 0.5\%, n=1; prednisolone acetate 1\%, n=0) and increased intraocular pressure or ocular hypertension (n=2/group). Mean intraocular pressure values during treatment were 2-3 mm Hg higher with difluprednate 0.05\% compared with prednisolone acetate 1\%; mean values were similar between groups by the first week after treatment cessation. At 2 weeks post surgery, the incidence of complete clearing of anterior chamber cells was similar between groups (difluprednate 0.05\%, n=30 (78.9\%); prednisolone acetate 1\%, n=31 (77.5\%). Compared with prednisolone acetate 1\%, approximately twice as many difluprednate 0.05\%-treated patients had a global inflammation assessment score indicating no inflammation on day 1 (n=12 (30.8\%) vs n=7 (17.5\%) and day 8 (n=18 (48.7\%) vs n=10 (25.0\%).ConclusionsDifluprednate 0.05\% four times daily showed safety and efficacy profiles similar to prednisolone acetate 1\% four times daily in children 0-3 years undergoing cataract surgery.  [\hyperlink{Difluprednate}{PMID: 27367745}, M E Wilson et al., 2016] The aim of this study was to evaluate the efficacy and safety of difluprednate ophthalmic solution 0.05\% (Durezol; Alcon Laboratories, Fort Worth, TX) compared with prednisolone acetate ophthalmic suspension 1\% (Pred Forte; Allergan, Inc., Irvine, CA) for endogenous anterior uveitis. In this phase 3, multicenter, randomized, noninferiority trial, 90 patients with endogenous anterior uveitis [>10 anterior chamber (AC) cells and an AC flare score of ≥2 in at least 1 eye] received either difluprednate 4x /day (QID) (n=50) or prednisolone 8x/day (n=40) for 14 days, followed by a 2-week tapering regimen. The main outcome measure was change from baseline in AC cell grade on day 14. At day 14, mean AC cell grade improvement for difluprednate-treated patients was similar to prednisolone-treated patients (2.1 vs. 1.9, respectively), proving noninferiority. At day 14, 68.8\% of difluprednate patients had AC cell clearing (grade 0:≥ 1cell) compared with 61.5\% of prednisolone patients. In the prednisolone-treated group, 12.5\% of patients were withdrawn because of investigator-determined lack of efficacy; no difluprednate-treated patients were withdrawn for this reason (P=0.01). Clinically significant intraocular pressure elevation occurred in 3 difluprednate-treated patients (6.0\%) and 2 prednisolone-treated patients (5.0\%). Difluprednate administered QID is at least as effective as prednisolone administered 8x/day in resolving the inflammation and pain associated with endogenous anterior uveitis. Difluprednate provides effective treatment for anterior uveitis and requires less frequent dosing than prednisolone acetate. Trial NCT00501579 was registered at the National Institutes of Health Registry in July 2007 ( http://clinicaltrials.gov/ct2/show/NCT00501579?term=sirion\&rank=4 ). [\hyperlink{Difluprednate}{PMID: 27367745}, C Stephen Foster et al., 2010]

\hypertarget{pmid_22265149}{T}o evaluate the clinical effect of topical difluprednate in pediatric patients for treatment of noninfectious uveitis. Retrospective, observational case series. Twenty-six eyes of 14 pediatric patients with noninfectious uveitis who were treated with topical difluprednate were evaluated. Anterior and posterior cell grade, visual acuity, intraocular pressure (IOP), and cystoid macular edema (CME) were recorded at each visit. Main outcome measures were changes in anterior segment cell, CME, visual acuity, and IOP and development of a visually significant cataract. A significant (≥ 2-grade decrease or decrease to 0 in anterior segment cell) reduction in anterior segment inflammation was observed during treatment with topical difluprednate in 88\% of eyes (22/25) when used as an adjuvant to systemic immunomodulatory therapy. In addition, improvement in CME associated with uveitis was seen in response to topical therapy with difluprednate in 78\% of eyes with CME (7/9). A significant IOP response (IOP increase of ≥ 10 mm Hg from baseline and IOP ≥ 24 mm Hg) was seen in 50\% of eyes (13/26) and in 50\% of patients (7/14); 3 eyes of 2 patients required glaucoma surgery. Cataract formation or progression was observed in 39\% of eyes (10/26) and in 43\% of patients (6/14); 5 eyes of 3 patients required cataract surgery. Difluprednate is an effective agent for both control of anterior segment inflammation and reduction of CME in pediatric patients with uveitis when used as an adjuvant to systemic immunomodulatory therapy. A high rate of steroid-induced IOP elevation and cataract formation is seen in this population. Close monitoring of pediatric patients receiving difluprednate is recommended. [\hyperlink{Difluprednate}{PMID: 22265149}, Mark A Slabaugh et al., 2012]

\hypertarget{pmid_22364032}{A}cute respiratory infections are the second leading cause of morbidity in children under 18 years. Several drugs have been used with variable efficacy and safety, trying to reduce the associated symptoms and improve quality of life. To evaluate the efficacy and safety of buphenine, aminophenazone and diphenylpyraline hydrochloride when compared with placebo for the control of symptoms associated with common cold in children 6-24 months of age. Randomized clinical trial, double blind, placebo controlled, in 100 children < 24 months of any gender, with symptoms associated to common cold. They received the drug under study vs. placebo for seven days. Both groups received acetaminophen. The change on common cold related symptoms were evaluated. Statistic analysis was made with STATA 11.0 for Mac. Fifty-three children were randomized to study drug and forty-seven to placebo. Age of children in each group was similar (12.2 +/- 5.8 months vs. 12.7 +/- 5.8 months, p NS). There were significant differences between groups in relation to rhinorrea and sneezing resolution, with better results in Flumil group and no adverse events observed. The results in this study indicates that Flumil is a safe and effective drug for control of symptoms present in the common cold in children aged 6-24 months. [\hyperlink{Difluprednate}{PMID: 22364032}, Ericka Montijo-Barrios et al., ]

\hypertarget{pmid_19101421}{T}o assess the efficacy and safety of difluprednate ophthalmic emulsion 0.05\% (Durezol) 2 or 4 times a day compared with those of a placebo in the treatment of inflammation and pain associated with ocular surgery. Twenty-six clinics in the United States. One day after unilateral ocular surgery, patients who had an anterior chamber cell grade of 2 or higher (>10 cells) were treated with 1 drop of difluprednate 2 times or 4 times a day or with a placebo (vehicle) 2 times or 4 times a day in the study eye for 14 days. This was followed by a 14-day tapering period and a 7-day safety evaluation. Outcome measures included cleared anterior chamber inflammation (grade 0, <or=1 cell), absence of pain, and analysis of ocular adverse events. Of the 438 patients, 111 received difluprednate 2 times a day, 107 received difluprednate 4 times a day, and 220 received a placebo 2 or 4 times a day. Both difluprednate dosage regimens reduced postoperative ocular inflammation and pain safely and effectively compared with the placebo. A greater proportion of difluprednate-treated patients had a reduction in inflammation and pain at 8 days and 15 days. Three percent of patients in both difluprednate groups had a clinically significant IOP rise (>or=10 mm Hg and >or=21 mm Hg from baseline, respectively) versus 1\% in the placebo group. Difluprednate given 2 or 4 times a day cleared postoperative inflammation and reduced pain rapidly and effectively. There were no serious ocular adverse events. Fewer adverse events were reported in the difluprednate-treated groups than in the placebo group. [\hyperlink{Difluprednate}{PMID: 19101421}, Michael S Korenfeld et al., 2009]

\hypertarget{pmid_2275328}{T}he efficacy of flunitrazepam (0.04 mg.kg-1) as a premedicant was evaluated in 40 young children of less than 5 years of age in a double-blind, placebo-controlled study. Flunitrazepam was given by the rectal route 15 min prior to an inhalational mask induction with halothane. Sedation score, mask acceptance and induction score were significantly better in premedicated children than in the placebo group. There were no hypoxic episodes, prolonged sedation or other complications in either group. This suggests that flunitrazepam administered rectally in a low dose is an acceptable premedication in young children. [\hyperlink{Difluprednate}{PMID: 2275328}, C Estève et al., 1990]

\hypertarget{pmid_34547279}{T}o describe the effectiveness and side effect profile of difluprednate therapy in a series of patients with anterior scleritis. Retrospective, interventional case series. Data collected from all patients with anterior scleritis who used difluprednate as a single treatment agent from January 1, 2018, to January 1, 2020, including demographics, scleritis type, presence of nodules or necrosis, changes in scleritis activity, intraocular pressure (IOP), number of difluprednate drops used, best-corrected visual acuity (BCVA), and lens status. The primary outcome was clinical resolution of scleritis. Secondary outcomes included BCVA loss ≥2 lines, change in lens status or cataract surgery, and IOP ≥24 mm Hg. Twenty-five patients (35 eyes) were analyzed. The median age was 60 years (range 13-78); 60\% were female; 64\% were White. Forty percent had bilateral disease, and 44\% of patients had an associated systemic disease. The majority of eyes (66\%) had diffuse anterior scleritis. Eighty-three percent of eyes achieved resolution of scleritis, with a median time of resolution of 6 weeks. Eyes treated with an initial dose of ≥4 times daily were more likely to achieve disease resolution (hazard ratio [HR] = 3.43, 95\% confidence interval [CI] 1.19, 9.88, P = .02). Nine eyes had IOP elevation. Four eyes lost ≥2 lines of BCVA, and 1 due to cataract progression. One eye underwent cataract surgery. Difluprednate alone may effectively treat non-infectious anterior scleritis with a tolerable side effect profile. [\hyperlink{Difluprednate}{PMID: 34547279}, Paulina Liberman et al., 2022]

\hypertarget{pmid_21034305}{T}o report the effects of twice-daily difluprednate in a child with pars planitis (PP). Case report. PP was controlled with topical difluprednate for 1 year. Then an atypical pattern of steroid response--delayed, relatively sudden onset of recalcitrant ocular hypertension (OHT)--and posterior subcapsular cataract (PSC) formation necessitated alternative treatment. Although not a standard treatment, in select cases of PP topical difluprednate therapy could be a useful short-term treatment option while alternative treatments are considered or immunosuppressive agents build to therapeutic levels. Ophthalmologists must be aware of the potential for delayed onset of serious complications when using difluprednate. [\hyperlink{Difluprednate}{PMID: 21034305}, Paul A Kurz et al., 2011]

\hypertarget{pmid_8675429}{D}ivalproex sodium is an effective drug for the treatment of migraine. Most adverse drug events are transient and not of great clinical concern. Although rare, well-documented examples of liver toxicity have been reported in children under 2 years of age on polypharmacy. Additional cases occur in children under 10 who are receiving polypharmacy, particularly those who have intractable seizures and degenerative central nervous system disease. Clinicians who treat migraineurs with divalproex sodium do not need to be overly preoccupied with monitoring of drug levels and liver function tests. The most valuable test is clinical observation of the patient. [\hyperlink{Difluprednate}{PMID: 8675429}, S D Silberstein et al., 1996]

\hypertarget{pmid_21111374}{D}ifluprednate ophthalmic emulsion 0.05\% (Durezol™, Alcon, Fort Worth, Texas) is a topical difluorinated derivative of prednisolone with potent anti-inflammatory activity. Difluprednate 0.05\% has a reported associated increase in intraocular pressure (IOP) in 3\% of patients. Although the occurrence may be low, the possible elevation in IOP may be substantially higher than commonly encountered with other topical steroids. A 49-year-old black man presented with a traumatic anterior uveitis recalcitrant to traditional prednisolone acetate 1\% treatment. The patient was switched to difluprednate 0.05\% in an attempt to better control the ocular inflammation. Although the patient did not exhibit an IOP response after 4 weeks of treatment with prednisolone acetate 1\%, he did experience a pressure response within 2 weeks of initiating difluprednate treatment, resulting in an IOP increase from 9 mmHg to 48 mmHg with subsequent microcystic edema. A 44-year-old black woman presented with recurrent scleritis resistant to topical prednisolone acetate, loteprednol etabonate, and oral nonsteroidal anti-inflammatory therapy. Topical loteprednol 0.5\% was replaced by difluprednate 0.05\%. All evidence of ocular inflammation was eradicated with the changed therapy. IOP rose in the difluprednate-treated eye from 18 mmHg to 34 mmHg over the course of 18 days. In both cases, the IOP elevation was managed rapidly with the discontinuation of difluprednate and temporary use of IOP-reducing agents with no lasting adverse effects. Difluprednate 0.05\% is a new topical therapeutic option indicated for the treatment of inflammation and pain management associated with ocular surgery with an off-label potential for treatment of other anterior segment inflammatory conditions. However, clinicians need to be aware of the potential risk for significant and potentially rapid onset of IOP increase with this medication and manage patients accordingly. [\hyperlink{Difluprednate}{PMID: 21111374}, Kelly Meehan et al., 2010]

\hypertarget{pmid_21490354}{D}imenhydrinate is an over-the-counter drug that is commonly used for the treatment of nausea and vomiting. Many of my adult patients use it, but is it safe and useful in the pediatric population? Dimenhydrinate appears to be safe for use in the pediatric population. While little literature has been published about adverse effects of this medication, family physicians need to identify the cause of the vomiting before considering if the drug will be effective and need to ensure that patients safely use the medication and avoid potential interaction of the drug with other products. [\hyperlink{Difluprednate}{PMID: 21490354}, Paul Enarson et al., 2011]

\hypertarget{pmid_7560629}{T}his study compared the safety and efficacy of digoxin and flecainide in the prophylaxis of supraventricular tachycardia in infants. Recurrence of supraventricular tachycardia in infants is common. Digoxin is the conventional drug of first choice for prophylaxis, but its efficacy has not been tested in a controlled clinical trial, and there is no consensus on the drug of choice when digoxin is ineffective. We reviewed retrospectively the records of all infants with supraventricular tachycardia due to atrioventricular (AV) reentry admitted to our hospital between January 1986 and December 1993. Thirty-nine infants presented with sustained AV reentrant tachycardia at age 1 to 330 days (median 12). Intravenous flecainide was required to maintain immediate control in six patients who were then treated with oral flecainide. The other 33 patients were treated with oral digoxin. There was no recurrence of tachycardia in 14 (42\%) of the 33 patients (95\% confidence interval [CI] 25\% to 61\%). In the other 19 patients (58\%) (95\% CI 39\% to 75\%), digoxin was replaced by oral flecainide because of multiple recurrence of tachycardia. Full control was achieved in all 19 of these patients (100\%) (95\% CI 82\% to 100\%) and in 5 of the 6 patients treated with both intravenous and oral flecainide. Thus, overall, flecainide was effective in 24 (96\%) of 25 patients (95\% CI 80\% to 100\%). Comparison with previous natural history studies suggests that digoxin is ineffective in the prophylaxis of supraventricular tachycardia. Oral flecainide was effective in a small number of infants, with no adverse effects (95\% CI 0\% to 12\%), and may now be preferred as the primary prophylactic agent. [\hyperlink{Difluprednate}{PMID: 7560629}, J J O'Sullivan et al., 1995]

\hypertarget{pmid_19752076}{V}omiting is a common symptom in children with infectious gastroenteritis. It contributes to fluid loss and is a limiting factor for oral rehydration therapy. Dimenhydrinate has traditionally been used for children with gastroenteritis in countries such as Canada and Germany. We investigated the efficacy and safety of dimenhydrinate in children with acute gastroenteritis. We performed a prospective, randomized, placebo-controlled, multicenter trial. We randomly assigned 243 children with presumed gastroenteritis and vomiting to rectal dimenhydrinate or placebo. Children with no or mild dehydration were included. All children received oral rehydration therapy. Primary outcome was defined as weight gain within 18 to 24 hours after randomization. Secondary outcomes were number of vomiting episodes, fluid intake, parents' assessment of well-being, number of diarrheal episodes, and admission rate to hospital. We recorded potential adverse effects. Change of weight did not differ between children who received dimenhydrinate or placebo. The mean number of vomiting episodes between randomization and follow-up visit was 0.64 in the dimenhydrinate group and 1.36 in the placebo group. In total, 69.6\% of the children in the dimenhydrinate group versus 47.4\% in the placebo group were free of vomiting between randomization and the follow-up visit. Hospital admission rate, fluid intake, general well-being of the children, and potential adverse effects, including the number of diarrhea episodes, were similar in both groups. Dimenhydrinate reduces the frequency of vomiting in children with mild dehydration; however, the overall benefit is low, because it does not improve oral rehydration and clinical outcome. [\hyperlink{Difluprednate}{PMID: 19752076}, Ulrike Uhlig et al., 2009]

\hypertarget{pmid_23236934}{T}he use of midazolam for children was approved in March, 2010. Since the efficacy and safety data of midazolam used in children, excluding low-birth-weight infants and newborns, for "sedation under artificial respiration in intensive care units" were quite limited, a post-marketing survey was carried out to confirm the validity of the established dosage and administration. A consecutive enrollment method was adopted. Based on the data of 153 patients collected from 8 institutes, efficacy and safety profiles were analyzed. Among the 149 patients included in the safety analysis set, 6 adverse reactions were reported in 6 patients. The incidence of adverse events was 4.0\% (6/149). Reported adverse reactions included depressed level of consciousness: 1 event, delirium: 1 event, psychomotor hyperactivity: 1 event, hypotension: 2 events, and blood pressure increase: 1 event. Serious adverse drug reaction (ADR) reported in this survey was depressed level of consciousness. This ADR resolved on the following day after the treatment with flumazenil. Paradoxical reactions were reported in 1 patient, and tolerance was reported in 2 patients. The efficacy rate was 96.5\% (138/143). No additional safety issues (status of adverse reactions, status of adverse events, status of serious adverse events, etc.) and efficacy issue were manifest in the patients treated with the dosage and administration method established at the approval of the drug. [\hyperlink{Difluprednate}{PMID: 23236934}, Keizo Sogabe et al., 2012]

\hypertarget{pmid_9890791}{I}n 47 children followed for 1 year after the first "simple" febrile convulsion, dipropylacetate (Depakine, 20 mg/kg) was as effective in preventing new febrile convulsions (a single recurrence in 4\% of 47 children) as was phenobarbital (5 mg/kg) or primidone (25 mg/kg) (a single recurrence in 4\% of 47 children), and there were no side effects. Of 47 untreated children followed for 1 year, 55\% had 1 to 4 new febrile convulsions. All medications were given in divided doses morning and evening. [\hyperlink{Difluprednate}{PMID: 9890791}, G B Cavazzuti et al., 1975]

\hypertarget{pmid_24579280}{T}he aim of this video-based study was to examine the taste acceptance of children between the ages of 2 and 5 years regarding highly concentrated fluoride preparations in kindergarten-based preventive programs. The fluoride preparation Duraphat was applied to 16 children, Elmex fluid to 15 children, and Fluoridin N5 to 14 children. The procedure was conducted according to a standardized protocol and videotaped Three raters evaluated the children's nonverbal behavior as a measure of taste acceptance on the Frankl Behavior Rating Scale. The interrater reliability (intraclass correlation coefficient; ICC) was .86. In an interview, children indicated the taste of the fluoride preparations on a three-point "smiley" rating scale. The interviewer used a hand puppet during the survey to establish confidence between the children and examiners. Children's nonverbal behavior was significantly more positive after Fluoridin N5 and Duraphat were applied compared to the application of Elmex fluid. The same trend was found during the smiley assessment. The response of children who displayed cooperative positive behavior before the application of fluoride preparations was significantly more positive than those who displayed uncooperative negative behavior. To achieve a high acceptance of the application of fluoride preparations among preschool children, flavorful preparations should be used. [\hyperlink{Difluprednate}{PMID: 24579280}, Anne-Kathrin Kolb et al., 2013]

\hypertarget{pmid_8249087}{E}flornithine (difluoromethylornithine, DFMO) has recently been approved for the treatment of Trypanosoma brucei gambiense trypanosomiasis. Treatment failures have been infrequent but have occurred among patients treated with oral DFMO only, and among children. To investigate the higher frequency of failures observed in young patients, DFMO trough concentrations in serum and cerebrospinal fluid (CSF) were measured at the end of treatment in 13 children and 50 adults who had received 200 mg/kg intravenously every 12 h for 14 d. Mean DFMO concentration in CSF was significantly lower among children aged less than 12 years when compared to older patients (25.1 vs 68.9 nmol/mL, P < 0.001). Mean serum concentration was also lower in children (49.2 vs 87.5 nmol/mL, P = 0.03). Among patients who received DFMO as initial therapy for sleeping sickness, the mean CSF/serum ratio was lower in children (0.41 vs 0.91, P < 0.005). The 3 patients who failed DFMO treatment had CSF trough concentrations around or below 50 nmol/mL. Convulsions and anaemia were associated with higher drug levels and previous therapy with melarsoprol. The lower CSF drug concentrations observed in children could result from higher renal clearance and different CSF pharmacokinetics of DFMO in that age group. To avoid treatment failures, a 6-hourly regimen as well as higher DFMO dosage based on body surface area rather than on weight are recommended for children. [\hyperlink{Difluprednate}{PMID: 8249087}, F Milord et al., ]

\hypertarget{pmid_21276131}{D}iclofenac is an effective, opiate-sparing analgesic for acute pain in children, which is commonly used in pediatric surgical units. Recently, a Cochrane review concluded the major knowledge gap in diclofenac use is dosing information. A pharmacokinetic meta-analysis has been undertaken with the aim of recommending a dose for children aged 1-12 years. Studies containing diclofenac pharmacokinetic data were identified during a Cochrane systematic review, and authors were asked to provide raw data. A pooled population analysis was undertaken in NONMEM to define the pharmacokinetics of intravenous, oral, and rectal diclofenac in children. Simulations were performed to recommend a dose yielding an equivalent area under diclofenac concentration-time curve (AUC) to a 50-mg dispersible tablet in adults. Data from 111 children aged 1-14 years consisting of 375 samples following intravenous, oral suspension, and suppositories were used. Adult dispersible tablet and suspension data were added to provide a reference AUC and support the absorption modeling, respectively. A three-compartment model described disposition, a dual-absorption compartment model was used for suspension and dispersible tablet data, and single-absorption compartment model for suppositories. The estimate of clearance was 16.5 l·h(-1) ·70 kg(-1) and bioavailabilities were 0.36, 0.63, and 0.35 for suspension, suppository, and dispersible tablets, respectively. Single doses of 0.3 mg·kg(-1) for intravenous, 0.5 mg·kg(-1) for suppositories, and 1 mg·kg(-1) for oral diclofenac in children aged 1-12 years are recommended as they yield a similar AUC to 50 mg in adults. [\hyperlink{Difluprednate}{PMID: 21276131}, Joseph F Standing et al., 2011]

\hypertarget{pmid_16199414}{R}emifentanil is recommended for use in procedures with painful intraoperative stimuli but minimal postoperative pain. However, bradycardia and hypotension are known side-effects. We evaluated haemodynamic effects of i.v. glycopyrrolate during remifentanil-sevoflurane anaesthesia for cardiac catheterization of children with congenital heart disease. Forty-five children undergoing general anaesthesia with remifentanil and sevoflurane were randomly allocated to receive either saline, glycopyrrolate 6 microg kg(-1) or glycopyrrolate 12 microg kg(-1). After induction of anaesthesia with sevoflurane, i.v. placebo or glycopyrrolate was administered. An infusion of remifentanil at the rate of 0.15 microg kg(-1) min(-1) was started, sevoflurane continued at 0.6 MAC and cisatracurium 0.2 mg kg(-1) was given. Heart rate (HR) and non-invasive arterial pressures were monitored and noted every minute for the first 10 min and then every 2.5 min for subsequent maximum of 45 min. Baseline HR [mean (SD)] of 117 (20) beats min(-1) decreased significantly from 12.5 min onwards after starting the remifentanil infusion in the control group [106 (18) at 12.5 min and 99 (16) beats min(-1) at 45 min]. In the groups receiving glycopyrrolate, no significant decrease in HR was noticed. Glycopyrrolate at 12 microg kg(-1) induced tachycardia between 5 and 9 min after administration. Systolic and diastolic arterial pressures decreased gradually, but there were no significant differences in the pressures between groups. I.V. glycopyrrolate 6 microg kg(-1) prevents bradycardia during general anaesthesia with remifentanil and sevoflurane for cardiac catheterization in children with congenital heart disease. Administering 12 microg kg(-1) of glycopyrrolate temporarily induces tachycardia and offers no additional advantage. [\hyperlink{Difluprednate}{PMID: 16199414}, K Reyntjens et al., 2005]

\hypertarget{pmid_34826122}{A} topical formulation of diclofenac (FLECTOR diclofenac epolamine topical system (FDETS)) is approved in adults for the treatment of acute pain due to minor strains, sprains, and contusions; however, its safety and efficacy have not been investigated in a pediatric population. This study assessed the safety and efficacy of the FLECTOR (diclofenac epolamine) topical system in children. This was an open-label, single-arm, phase IV study at ten USA-based family medicine or pediatric practices in children aged 6-16 years with a clinically significant minor soft tissue injury sustained within the preceding 96 h and at least moderate spontaneous pain on the Wong-Baker FACES 104 patients were enrolled; 52 were 6-11 years old, and 52 were 12-16 years old (mean age 11.6 years). The maximum tolerability score experienced by any patient was 1 (faint redness). Fourteen adverse events (none serious) in nine patients (8.7\%) were considered possibly treatment-related. Reduction in pain during the study was somewhat greater for patients aged 6-11 versus 12-16 years (p < 0.011). The diclofenac plasma concentration tended to be higher in the younger age group compared with older patients: 1.83 versus 1.46 ng/mL at the first assessment and 2.49 versus 1.11 ng/mL at the last assessment (p = 0.002). The FLECTOR topical system safely and effectively provided pain relief for minor soft tissue injuries in the pediatric population, with minimal systemic nonsteroidal anti-inflammatory drug exposure and low potential risk of local or systemic adverse events. ClinicalTrials.gov identifier NCT02132247. [\hyperlink{Difluprednate}{PMID: 34826122}, Christopher A Jones et al., 2022]

\hypertarget{pmid_9132194}{T}o evaluate the safety and efficacy of intranasal diamorphine as an analgesic for use in children in accident and emergency (A\&E). A prospective, randomised clinical trial with consecutive recruitment of patients aged between 3 and 16 years with clinically suspected limb fractures. One group received 0.1 mg/kg intranasal diamorphine, and the other group received 0.2 mg/kg intramuscular morphine. At 0, 5, 10, 20, and 30 minutes pain scores, Glasgow coma score, and peripheral oxygen saturations were recorded; parental acceptability was assessed at 30 minutes. 58 children were recruited, with complete data collection in 51 (88\%); the median summed decrease in pain score was better for intranasal diamorphine than intramuscular morphine (9 v 8), though this was not significant (P = 0.4, Mann-Whitney U test). The episode was recorded as "acceptable" in all parents whose child received intranasal diamorphine, compared with only 55\% of parents in the intramuscular morphine group (P < 0.0001, Fisher's exact test). There was no incidence of decreased peripheral oxygen saturation or depression in the level of consciousness in any patient. Intranasal diamorphine is an effective, safe, and acceptable method of analgesia for children requiring opiates in the A \& E department. [\hyperlink{Difluprednate}{PMID: 9132194}, J A Wilson et al., 1997]

\hypertarget{pmid_10971664}{T}o determine the beneficial use of divalproex sodium as a prophylactic treatment for migraine in children. Previous studies for treatment of migraine in adults have shown a greater than 50\% reduction in migraine attack frequencies. Few data exist, however, regarding the efficacy and safety of divalproex sodium use in children with migraine. We studied the incidence of headache relief in our patients with migraine aged 16 years and younger treated with divalproex sodium prophylactically at our institution from July 1996 to December 1998 to determine medication dosage used, concomitant headache medications, and possible adverse effects. A total of 42 patients, ranging in age from 7 to 16 years (mean age, 11.3 years), were treated with divalproex sodium for headache. All had a history of migraine with or without aura. Baseline headache frequency during a minimum 6-month period was one to four headaches per month. Divalproex sodium dosage ranged from 15 mg/kg/day to 45 mg/kg/day. Of the 42 patients, 34 (80.9\%) successfully discontinued their abortive medications. After 4 months' treatment, 50\% headache reduction was seen in 78.5\% of patients, 75\% reduction in 14.2\% of patients, and 9. 5\% of patients became headache-free. These results indicate divalproex sodium to be an effective and well-tolerated treatment for the prophylaxis of migraine in children. [\hyperlink{Difluprednate}{PMID: 10971664}, J M Caruso et al., 2000]

\hypertarget{pmid_26646324}{B}isphosphonates are used in the treatment of vitamin D intoxication (VDI) after failure of conventional therapy including prednisolone. Safety concerns restrict the use of bisphosphonates from being used as first-line therapy for VDI in children. The aim of this study was to evaluate the efficacy and safety of pamidronate in comparison with prednisolone in children with VDI. We reviewed the hospital records of children consecutively diagnosed with VDI at two medical centers in a 15 year period. The subjects consisted of 21 children (age, 0.3-4.2 years) who were treated with prednisolone and/or bisphosphonates. Pamidronate (n = 18) or alendronate (n = 3) was used in six patients after unsuccessful prednisolone treatment, and in 15 patients from baseline. Initial serum calcium and 25-hydroxyvitamin D were 16.1 ± 1.9 mg/dL and 493 ± 219 ng/mL, respectively. The median time to reach normocalcemia in the pamidronate, alendronate and prednisolone groups was 3 days (range, 2-12 days), 4 days (range, 3-6 days) and 17 days (range, 12-26 days), respectively (P = 0.013). The pamidronate group had a fivefold shorter hospital stay than the prednisolone group. Three patients initially treated with prednisolone developed nephrocalcinosis but this did not occur in any patient treated with bisphosphonates from baseline. Apart from transient fever and moderate hypophosphatemia, no side-effect of bisphosphonate treatment was observed. Pamidronate is efficient and safe for the treatment of VDI in children. Pamidronate use significantly shortens the duration of treatment, and thereby may prevent the development of nephrocalcinosis. Instead of prednisolone, pamidronate should be used together with hydration and furosemide as the first-line therapy for VDI. [\hyperlink{Difluprednate}{PMID: 26646324}, Cengiz Kara et al., 2016]

\hypertarget{pmid_21750615}{T}o review the most recent published data regarding the novel potent steroid, difluprednate ophthalmic emulsion, 0.05\%. A comprehensive search of recent published literature including difluprednate was performed. Clinical studies relevant to the characteristics and clinical efficacy of difluprednate in controlling postoperative inflammation were included, and a synopsis of each study was developed. Several recent publications were identified in which difluprednate was shown to be efficacious in the treatment of postoperative inflammation in different clinical settings, including a novel perioperative regimen. Additional literature retrieved from this search included data on the relative potency of difluprednate, potential utility in the posterior segment, as well as the advantages of the emulsion formulation. Difluprednate has been studied extensively and shown in recent literature to be a safe and effective topical anti-inflammatory drug. The proven strength and unique formulation of difluprednate, along with its potent efficacy in treating and preventing inflammation, provides clinicians with a beneficial treatment option. [\hyperlink{Difluprednate}{PMID: 21750615}, Eric D Donnenfeld et al., 2011]

\hypertarget{pmid_20856594}{T}o evaluate the efficacy and safety of twice-daily difluprednate ophthalmic emulsion 0.05\% (Durezol(®)) versus placebo administered before surgery for managing inflammation and pain following cataract extraction. Eligible subjects (N = 121) were randomized 2:1 to topical treatment with 1 drop difluprednate or placebo administered twice daily for 16 days, followed by a 14-day tapering period. Dosing was initiated 24 hours before unilateral ocular surgery. Clinical signs of inflammation (anterior chamber [AC] cell and flare grade, bulbar conjunctival injection, ciliary injection, corneal edema, and chemosis), ocular pain/discomfort, intraocular pressure (IOP), and adverse events were assessed. Clearing of inflammation on day 14 (primary endpoint), defined as an AC cell grade of 0 (≤5 cells) and a flare grade of 0 (complete absence), was achieved in a significantly greater percentage of subjects treated with difluprednate, compared with placebo (74.7\% vs 42.5\%; P = 0.0006). A significantly greater percentage of difluprednate-treated subjects were free of ocular pain/discomfort on day 14 than placebo-treated subjects (64.6\% vs 30.0\%; P = 0.0004). Three subjects (3.7\%) in the difluprednate group had a clinically significant IOP rise (defined as ≥21 mmHg and a change from baseline ≥10 mmHg at same visit). Difluprednate, administered 2 times daily starting 24 hours before cataract surgery, was highly effective for managing ocular inflammation and relieving pain and discomfort postoperatively. Difluprednate was well tolerated and provides a convenient twice-daily option for managing postoperative ocular inflammation. [\hyperlink{Difluprednate}{PMID: 20856594}, Stephen Smith et al., 2010]

\section*{Buprenorphine Hydrochloride}
\subsection*{Result}
\subsubsection*{Answer}

Infants and Toddlers (<3 years): No
Children (3–5 years): Unknown
Children (5–12 years): Yes (for caudal epidural use in perioperative setting only)
Adolescents (13–19 years): Unknown

\subsubsection*{{Explanation}}
\hypertarget{Buprenorphine Hydrochloride}
To determine if Buprenorphine Hydrochloride is safe for use in children, I reviewed all available abstracts for targeted studies on safety in pediatric populations. Below is a summary of the evidence, organized by age range:

Infants and Toddlers (<3 years):
- Multiple retrospective case series and poison control data show that unintentional exposure to buprenorphine in children under 3 years can cause significant opioid toxicity, including drowsiness, miosis, and respiratory depression, sometimes requiring intensive care and naloxone administration. There are reports of both non-fatal and fatal outcomes in this age group. These studies do not represent intentional, therapeutic use, but rather accidental ingestion, and consistently highlight the risk of severe toxicity and even death in young children [\hyperlink{pmid_20921918}{PMID: 20921918}, Pedapati et al., 2011; \hyperlink{pmid_31675453}{PMID: 31675453}, von Fabeck et al., 2020; \hyperlink{pmid_23129079}{PMID: 23129079}, Kim et al., 2012; \hyperlink{pmid_30740990}{PMID: 30740990}, Lo Re et al., 2019; \hyperlink{pmid_29941678}{PMID: 29941678}, Post et al., 2018; \hyperlink{pmid_28292505}{PMID: 28292505}, Schwartz et al., 2017]. No studies affirm safety for therapeutic use in this age group.

Children (3–6 years):
- Poison center data and retrospective reviews show that accidental ingestion in this age group can also result in significant toxicity, though the risk of fatality appears lower than in infants and toddlers. There are no targeted clinical trials or prospective studies evaluating the safety of therapeutic buprenorphine use in this age group [\hyperlink{pmid_29941678}{PMID: 29941678}, Post et al., 2018; \hyperlink{pmid_23489089}{PMID: 23489089}, Soyka et al., 2013]. Safety for therapeutic use remains unproven.

Children (5–12 years):
- One randomized, double-blind study evaluated caudal (epidural) buprenorphine for postoperative analgesia in children aged 5–12 years. The study found superior analgesia compared to bupivacaine alone, with no major adverse effects and only increased nausea and vomiting in the buprenorphine group. The authors concluded that caudal buprenorphine is a safe and reliable means of providing postoperative pain relief in this age group [\hyperlink{pmid_7489418}{PMID: 7489418}, Kamal et al., 1995].
- A pharmacokinetic study in children aged 4–7 years receiving intravenous buprenorphine as premedication for minor surgery reported no safety concerns, but the study was not designed to assess safety outcomes [\hyperlink{pmid_2775626}{PMID: 2775626}, Olkkola et al., 1989].

Adolescents (13–19 years):
- Poison control data indicate that intentional and unintentional exposures occur, with some risk of serious outcomes, especially in cases of overdose or polysubstance use. No targeted clinical trials or safety studies for therapeutic use in adolescents were identified [\hyperlink{pmid_29941678}{PMID: 29941678}, Post et al., 2018].

General Pediatric Population:
- Multiple reviews and meta-analyses highlight a lack of robust, targeted safety studies for buprenorphine in children, especially for chronic or long-term use, and call for more research [\hyperlink{pmid_29588613}{PMID: 29588613}, Vicencio-Rosas et al., 2018; \hyperlink{pmid_21091589}{PMID: 21091589}, Michel et al., 2011; \hyperlink{pmid_31803545}{PMID: 31803545}, Haupt et al., 2019; \hyperlink{pmid_34567202}{PMID: 34567202}, Attinà et al., 2021]. Some small studies and case series suggest possible safety in certain settings, but the evidence is insufficient for broad recommendations.

Summary:
- For children under 5 years, especially infants and toddlers, there is clear evidence of risk and no evidence affirming safety for therapeutic use.
- For children aged 5–12 years, there is limited evidence (one small study) supporting safety for caudal epidural use in the perioperative setting.
- For adolescents, safety for therapeutic use is unknown due to lack of targeted studies.
- For all pediatric age groups, there is a lack of large, high-quality, targeted safety studies for therapeutic use of buprenorphine hydrochloride.

\subsection*{Abstracts}
\hypertarget{pmid_20921918}{S}ublingual buprenorphine is an alternative to methadone for office-based treatment of opioid dependence. Recent reports have examined a growing number of unintentional buprenorphine exposures in children resulting in significant toxicity, even after a single lick or taste of a sublingual tablet. Here, we report a series of unintentional buprenorphine exposures in toddlers over a 2.5-yr period that led to admission to the pediatric intensive care unit. The goals of this study were to determine: 1) the prevalence of symptomatic buprenorphine exposure in children <3 yrs of age; 2) the severity of toxicity associated with such exposures; and 3) effective clinical interventions. A retrospective case review was performed on records from the pediatric intensive care unit at an academic medical center located in the northeastern United States. Unintentional buprenorphine/naloxone exposure (n = 9) accounted for the largest single fraction of toxic ingestions among patients younger than 3 yrs within the study period (9/33, 27\%). All exposures occurred at the child's place of residence (n = 9, 100\%). Clinical signs of opioid toxicity were evident in all nine cases, with the most common symptom being drowsiness or lethargy (n = 9, 100\%), followed by miosis (n = 6, 67\%) and respiratory depression (n = 5, 56\%). Six patients were effectively treated with naloxone (n = 6, 67\%). The increased use and similarity to candy of the current formulation of buprenorphine pose a special risk to children, especially toddlers. Buprenorphine exposure in children <3 yrs old can cause significant opioid toxidrome. Naloxone is an effective agent for reversal of symptoms; however, given buprenorphine's high affinity and long action, higher doses or continuous infusion may be required. Adults on buprenorphine should be educated on the risks posed to young children in their household and the appropriate storage of medication. [\hyperlink{Buprenorphine Hydrochloride}{PMID: 20921918}, Ernest V Pedapati et al., 2011]

\hypertarget{pmid_31675453}{B}uprenorphine is a µ-partial agonist and k-antagonist acting on central opioid receptors. Patented for analgesia in 1968, buprenorphine has been used as opioid substitutive therapy since the 1990s, as well as methadone. The aim was to document pediatric poisoning, to discover the severity, and to evaluate the treatment with naloxone. All pediatric poisonings reported to the poison control center Marseille (France)-from January 1, 2009 to December 31, 2018-were included. Analysis put value on gender, age, estimated quantity, symptoms and their delay, place of treatment, medical treatment, utilization of antidotes, severity of intoxications, and patients' outcome. Fifty-four infant poisonings with buprenorphine were recorded, doses varied between 1 and 36 mg, and children showed mainly neurological (somnolence, miosis…) and gastroenteric (vomiting) effects. Pulmonary effects were described for four children. According to the poisoning severity score, 8 intoxications were classified as 'no symptoms or signs', 37 as minor poisonings, 3 as moderate, none as severe or fatal and 6 were unknown. Medical care was required for 46 children, and four of them were treated with naloxone. Buprenorphine poisoning can cause neurological, gastroenteric, and respiratory symptoms. Even licking a tablet leads to intoxication because of maximal tablet's absorption while placing it under the tongue. Hospital admission is necessary even at small doses. Naloxone was efficient in the four described cases. Parents have to be aware of the poisoning risk with buprenorphine. Recently, commercialized instantly dissolving formulations could cause more severe intoxications. [\hyperlink{Buprenorphine Hydrochloride}{PMID: 31675453}, Katharina von Fabeck et al., 2020]

\hypertarget{pmid_25623082}{T}he safety of a proprietary formulation of buprenorphine hydrochloride administered subcutaneously (SC) to young cats was investigated in a blinded, randomized study. Four cohorts of eight cats aged approximately 4 months were administered saline, 0.24, 0.72 or 1.20 mg/kg/day buprenorphine SC for nine consecutive days, representing 0×, 1×, 3× and 5× of the intended dose. Cats were monitored daily for evidence of clinical reactions, food and water intake and adverse events (AEs). Physical examinations, clinical pathology, vital signs and electrocardiograms (ECGs) were evaluated at protocol-specified time points. Complete necropsy and histopathologic examinations were performed following humane euthanasia. Four buprenorphine-treated cats experienced AEs during the study, two unrelated and two related to study drug administration. The two cats with AEs considered related to drug administration had clinical signs of hyperactivity, difficulty in handling, disorientation, agitation and dilated pupils in one 0.24 mg/kg/day cat and one 0.72 mg/kg/day cat. All of these clinical signs were observed simultaneously. There were no drug-related effects on survival, injection response, injection site inspections, body weight, food or water consumption, bleeding time, urinalysis, respiration rate, heart rate, ECGs, blood pressures, body temperatures, macroscopic examinations or organ weights. Once daily buprenorphine s.c. injections at doses of 0.24, 0.72 and 1.20 mg/kg/day for 9 consecutive days were well tolerated in young domestic cats. [\hyperlink{Buprenorphine Hydrochloride}{PMID: 25623082}, M K Sramek et al., 2015]

\hypertarget{pmid_23129079}{B}uprenorphine is a partial μ-opioid receptor agonist that is approved for the treatment of opioid dependency. It is generally believed to be safer than methadone because of its ceiling effect on respiratory depression. As more adults in US households use buprenorphine, an increasing number of children are being exposed. We report a fatal exposure to buprenorphine in a small child that occurred after ingestion of a caretaker's buprenorphine/naloxone. Postmortem toxicology analysis showed free serum concentrations of 52 ng/mL and 39 ng/mL for buprenorphine and norbuprenorphine, respectively. No other drugs were detected. Autopsy did not find signs of injury or trauma. The theoretical safety provided by the ceiling effect in respiratory depression from buprenorphine may not apply to children, and buprenorphine may cause dose-dependent respiratory depression. [\hyperlink{Buprenorphine Hydrochloride}{PMID: 23129079}, Hong K Kim et al., 2012]

\hypertarget{pmid_18381506}{T}here are few reports in children of overdoses of buprenorphine, a partial opioid agonist used in the treatment of opioid dependence and pain. The purpose of this study was to analyze buprenorphine overdoses in young children reported by US poison centers to the Researched Abuse, Diversion, and Addiction-Related Surveillance System. A retrospective review of buprenorphine overdoses in children < 6 years of age reported to the Researched Abuse, Diversion, and Addiction-Related Surveillance System from November 2002 through December 2005 was performed. Patients lost to follow-up and those ingesting multiple substances were excluded. Eighty-six cases met inclusion criteria. In the 54 children who developed toxicity, the clinical effects included drowsiness or lethargy (55\%), vomiting (21\%), miosis (21\%), respiratory depression (7\%), agitation or irritability (5\%), pallor (3\%), and coma (2\%). There were no fatalities. The mean time to onset of effects was 64.2 minutes, with a range of 20 minutes to 3 hours. Duration of clinical effects was under 2 hours in 11\%, 2 to 8 hours in 59\%, 8 to 24 hours in 26\%, and > 24 hours in 4\%. Children who ingested > or = 2 mg of buprenorphine were more likely to experience clinical effects, and all of the children who ingested > 4 mg experienced some effect. No child ingesting < 4 mg experienced a severe effect. Of the 22 children administered naloxone, 67\% had at least a partial response. Buprenorphine overdoses are generally well tolerated in children, with significant central nervous system and respiratory depression occurring in only 7\%. Any child ingesting > 2 mg and children < 2 years of age ingesting more than a lick or taste should be referred to the emergency department for a minimum of 6 hours of observation. Naloxone can be used to reverse respiratory depression. [\hyperlink{Buprenorphine Hydrochloride}{PMID: 18381506}, Bryan D Hayes et al., 2008]

\hypertarget{pmid_30740990}{T}he escalation of the opioid crisis has led to an increase in the treatment of opioid use disorder. In particular, recent legislation has allowed for office-based treatment with buprenorphine, a partial µ-opioid agonist that is believed to be safer than methadone due to a ceiling effect on respiratory depression in adults. An increasing number of children are being exposed to buprenorphine as more adults in US households receive take-home prescriptions. The ceiling effect seen in adults does not seem to apply to young children, and intoxication with severe symptoms including fatalities can occur. This article outlines the pharmacology of buprenorphine and reviews the current literature on overdose in children. We conclude with practical recommendations for limiting potential exposure and damage to children from accidental buprenorphine overdose. [\hyperlink{Buprenorphine Hydrochloride}{PMID: 30740990}, Meike Lo Re et al., 2019]

\hypertarget{pmid_23489089}{O}pioid maintenance therapy with methadone or buprenorphine is an established and first-line treatment for opioid dependence. Risk of diversion and toxicity of opioid prescription drugs, including buprenorphine, causes significant concerns. This is particularly the case in the United States, where the number of related emergency visits is increasing, especially in children. A systematic literature research (Medline, Pubmed) was performed to assess the risk associated with buprenorphine. The search, which was not limited to particular publication years, was performed with the key words buprenorphine AND toxicity (114 counts ) AND children (4 counts) and buprenorphine AND mortality AND children (5 counts). In addition, the author obtained information from relevant websites (NIDA, SAMSHA) and pharmacovigilance data from the manufacturer of buprenorphine. Clinical and toxicological data suggest a low risk for fatal intoxications associated with bupreorphine in adults. Data from emergency units indicate a dramatic, 20-fold increase in buprenorphine exposure in children over the past decade, mostly in those under 6. The US 'Researched Abuse, Diversion and Addiction-Related Surveillance' (RADARS) system indicates a lower risk of severe opioid intoxications with buprenorphine than with other opioids, with no fatal outcomes recorded. Correspondingly, data from spontaneous reports to the surveillance programme of the manufacturer of buprenorphine (13,600 buprenorphine exposures, 4879 of these in children under six) show a serious medical outcome in 34\% of children under the age of six but only one fatal outcome. Although exposure to buprenorphine and other opioids remains a significant concern in children, the drug seems rather to be safe with respect to severe outcomes, in particular death.  [\hyperlink{Buprenorphine Hydrochloride}{PMID: 23489089}, Michael Soyka et al., 2013] To develop key messages for methadone and buprenorphine safety education material based on an analysis of calls to the NYC Poison Control Center (NYC PCC) and designed for distribution to caregivers of young children. Retrospective review of all calls for children 5 years of age and younger involving methadone or buprenorphine from January 1, 2000, to June 15, 2014. A data abstraction form was completed for each case to capture patient demographics, exposure and caller sites, caller relation to patient, qualitative information regarding the exposure scenario, the product information, if naloxone was given, and the medical outcome of the case. A total of 123 cases were identified. The ages of the children ranged from 4 days to 5 years; 55\% were boys. All exposures occurred in a home environment. The majority of the calls were made to the NYC PCC by the doctor (74\%) or nurse (2\%) at a health care facility. Approximately one-fourth of the calls came from the home and were made by the parent (22\%) or grandparent (2\%). More than one-half of the exposures involved methadone (64\%). Naloxone was administered in 28\% of cases. Approximately one-fourth of the children did not experience any effect after the reported exposure, one-half (51\%) experienced some effect (minor, moderate, or major), and there was 1 death (1\%). More than one-half of the children were admitted to the hospital, with 40\% admitted to critical care and 13\% to noncritical care. Approximately 23\% were treated and released from the hospital, and 20\% were lost to follow-up or never arrived to the hospital. The remaining 4\% were managed on site without a visit to the hospital. Exposures to methadone and buprenorphine are dangerous with some leading to serious health effects. Safe storage and disposal instructions are needed for homes where children may be present. [\hyperlink{Buprenorphine Hydrochloride}{PMID: 23489089}, Lauren Schwartz et al., ]

\hypertarget{pmid_30106872}{I}ngestion of buprenorphine by young children is on the rise and can lead to life-threatening consequences and death. Exposure most often occurs when a child acquires the medication intended for adult use. However, buprenorphine is also prescribed by veterinarians and may be sent home, typically in non-child-resistant packaging, to be administered to the family pet. A previously healthy 2-year-old girl weighing 11.36 kg was found with a 1-mL syringe containing 0.6 mg/mL of buprenorphine in her mouth. The syringe had been in a plastic bag provided to the family by their veterinarian for the family dog. She was hospitalized for 24 hours but remained asymptomatic and was discharged healthy. This type of exposure to buprenorphine has not previously been described in the literature. Having this unsecured medication in the home increases the potential risk of exposure for young children and associated health consequences. Pediatricians should be aware of the potential dangers that veterinary pharmaceuticals can pose and educate parents about proper storage of medications. In addition, veterinarians should take extra precautions when dispensing these medications to pet owners with children. [\hyperlink{Buprenorphine Hydrochloride}{PMID: 30106872}, Kristin J Roberts et al., 2020]

\hypertarget{pmid_29941678}{T}o investigate buprenorphine exposures among children and adolescents ≤19 years old in the United States. Data were analyzed from calls to US poison control centers for 2007-2016 from the National Poison Data System. From 2007 to 2016, there were 11 275 children and adolescents ≤19 years old exposed to buprenorphine reported to US poison control centers. Most exposures were among children <6 years old (86.1\%), unintentional (89.2\%), and to a single substance (97.3\%). For single-substance exposures, children <6 years old had greater odds of hospital admission and of serious medical outcome than adolescents 13 to 19 years old. Adolescents accounted for 11.1\% of exposures; 77.1\% were intentional (including 12.0\% suspected suicide), and 27.7\% involved multiple substances. Among adolescents, the odds of hospital admission and a serious medical outcome were higher for multiple-substance exposures than single-substance exposures. Buprenorphine is important for the treatment of opioid use disorder, but pediatric exposure can result in serious adverse outcomes. Manufacturers should use unit-dose packaging for all buprenorphine products to help prevent unintentional exposure among young children. Health providers should inform caregivers of young children about the dangers of buprenorphine exposure and provide instructions on proper medication storage and disposal. Adolescents should receive information regarding the risks of substance abuse and misuse. Suspected suicide accounted for 12\% of adolescent exposures, highlighting the need for access to mental health services for this age group. [\hyperlink{Buprenorphine Hydrochloride}{PMID: 29941678}, Sara Post et al., 2018]

\hypertarget{pmid_29588613}{T}he usual management of moderate to severe pain is based on the use of opioids. Buprenorphine (BPN) is an opioid with an analgesic potency 50 times greater than that of morphine. It is widely used in various pain models and has demonstrated efficacy and safety in adult patients; however, there are insufficient clinical trials in pediatric populations. The aim of this study was to perform an updated meta-analysis on the implementation of BPN in the treatment of pain in the pediatric population. A bibliographic search was carried out in different biomedical databases to identify scientific papers and clinical trials with evidence of BPN use in children and adolescents. A total of 89 articles were found, of which 66 were selected. Analysis of these items revealed additional sources, and the final review included a total of 112 publications. Few studies were found regarding the efficacy and safety of BPN use in children. In recent years, the use of this drug in the pediatric population has become widespread, so it is imperative to perform clinical trials and pharmacological and pharmacovigilance studies, which will allow researchers to develop dosage schemes based on the evidence and minimize the risk of adverse effects. [\hyperlink{Buprenorphine Hydrochloride}{PMID: 29588613}, Erendira Vicencio-Rosas et al., 2018]

\hypertarget{pmid_16391920}{T}here is a paucity of relevant pediatric data on buprenorphine, especially with respect to the long-term application in children suffering chronic pain or to pediatric pharmacokinetic as well as pharmacodynamic data after repeated sublingual or long-term transdermal administration. Compared to adults, after single-dose buprenorphine, children seem to exhibit a larger clearance related to body weight and a longer duration of action. If combined with other opioids or sedatives or if the metabolite norbuprenorphine cumulates, it is difficult to estimate the risk of respiratory depression. Clear-cut evidence is missing that in children there is a ceiling of buprenorphine-induced respiratory depression. Due to its various application routes, long duration of action, and metabolism largely independent of renal function buprenorphine is of special clinical interest in pediatrics, especially for postoperative pain and cancer pain control. There is no reason to expect effects fundamentally different from those in adults. [\hyperlink{Buprenorphine Hydrochloride}{PMID: 16391920}, E Michel et al., 2006]

\hypertarget{pmid_21091589}{T}he transdermal therapeutic system (TTS) with buprenorphine is currently being used 'off-label' to treat chronic pediatric pain. We compiled available pharmacokinetic (PK), pharmacodynamic (PD), and clinical pediatric data on buprenorphine to rationalize treatment regimens. We conducted a systematic biomedical literature review focusing on pediatric buprenorphine data. There are few relevant pediatric buprenorphine data, particularly in children suffering chronic pain. There are no pediatric PK and PD data for children with chronic pain given sublingual or TTS formulations. Children given single dose buprenorphine have increased drug clearance referenced to bodyweight with a possible paradoxical longer duration of action. Buprenorphine metabolism is independent of renal function, which is advantageous in renal insufficiency. The risk of respiratory depression after buprenorphine is difficult to quantify. A concentration-response relationship for respiratory effects has not been described and it is unknown whether children have a ceiling effect similar to that described in healthy adult volunteers. Buprenorphine is of interest in pediatric postoperative pain and cancer pain control because of its multiple administration routes, long duration of action, and metabolism largely independent of renal function. There is little reason to expect buprenorphine effects in children out of infancy are fundamentally different to those in adults. From the limited pediatric data available, it appears that buprenorphine has no higher adverse potential than the more commonly used opioids. There is an urgent need for focused PK, PD, and safety studies in children before use in children becomes more widespread. [\hyperlink{Buprenorphine Hydrochloride}{PMID: 21091589}, Erik Michel et al., 2011]

\hypertarget{pmid_34567202}{P}ain is one of the main symptoms reported by sick children, particularly by those suffering from cancer. Opioids are very useful in controlling this symptom but they are burdened with significant side effects that limit their use in children. Buprenorphine is a strong opioid that, due to its particular pharmacological characteristics, ensures excellent pain relief with fewer side effects than other opioids. The transdermal formulation allows for good pain control associated with optimal compliance by patients and few limitations on daily life. Unfortunately, transdermal buprenorphine use remains off-label for the control of chronic pain in children; therefore, it is desirable that new studies can validate its use in the paediatric population. This review aims to analyse the clinical advantages of transdermal buprenorphine in the paediatric population and the possible side effects registered in daily clinical practice. [\hyperlink{Buprenorphine Hydrochloride}{PMID: 34567202}, Giorgio Attinà et al., 2021]

\hypertarget{pmid_483080}{B}uprenorphine hydrochloride, a new, potent, long-acting synthetic opiate analgesic, with partial agonist-antagonist activity, was administered intravenously to two groups of patients in an intensive care unit. Arterial blood was drawn for blood gas analysis before (control) and at regular intervals after drug administration, to determine the effects of intravenous buprenorphine on respiration in critically ill patients, each acting as his or her own control. Intravenous buprenorphine 0,4 mg (group I -- 10 patients) caused a significant reduction in mean respiration rate and an increase in mean PaCO2, but did not alter heart rate, PaO2 or base excess values. Intravenous buprenorphine 0,2 mg (group II -- 10 patients) was associated with a less significant reduction in the rate of breathing and elevation of PaCO2. Both 0,4 mg and 0,2 mg buprenorphine produced effective relief of pain, sedation, and reduction in restlessness, and allayed anxiety. Our results suggest that intravenous buprenorphine 0,2 mg can be safely recommended for the prolonged relief of postoperative pain in adults. [\hyperlink{Buprenorphine Hydrochloride}{PMID: 483080}, J W Downing et al., 1979]

\hypertarget{pmid_33174237}{B}uprenorphine has been used in pain and opioid addiction management for nearly 25 years. Compared to methadone, buprenorphine is thought to exhibit less side effects and respiratory depression in case of accidental or suicidal overdose. The aim was to describe the characteristics of exposures reported to a French Poison Control Center (PCC). We conducted a retrospective study including all buprenorphine exposures for which advice of our PCC was required between 2009 and 2018. After data extraction from the electronic medical files and anonymous transfer to an Access base, a statistical descriptive analysis was performed focusing on adolescents over 10 years old and adults. One hundred and ninety-nine cases were analyzed. The major circumstances of exposure were suicide attempts and overdoses in patients with previously identified substance abuse. Buprenorphine exposures have been reduced by 50\% between 2009 and 2018. Coingestions, often with benzodiazepines or antidepressants, were almost systematic and 79\% of all the series exhibited at least one symptom. Among the symptomatic cases, neurological effects were the most frequent (83\%) and respiratory symptoms occurred in 13\%. No deaths were registered. Severity did not exceed PSS1 in 80\% of all the cases. Treatment was mainly symptomatic even though naloxone was required in at least 5\% of the symptomatic cases. Within 24 h after exposure, 120 patients were discharged from the emergency department. Despite loss to follow-up, our results suggest that buprenorphine is relatively safe. [\hyperlink{Buprenorphine Hydrochloride}{PMID: 33174237}, Audrey Boulamery et al., 2021]

\hypertarget{pmid_34271853}{E}ffective postoperative analgesia is needed to prevent the negative effects of postoperative pain on patient outcomes. To compare the effectiveness of hydromorphone hydrochloride and sufentanil, combined with flurbiprofen axetil, for postoperative analgesia in pediatric patients. This prospective randomized controlled trial included 222 pediatric patients scheduled for repair of a structural congenital malformation under general anesthesia. Patients were randomized into 3 groups: hydromorphone hydrochloride 0.1 mg/kg (H1), hydromorphone hydrochloride 0.2 mg/kg; (H2) or sufentanil 1.5 µg/kg (S). Analgesics were diluted in 0.9\% saline to 100 ml and infused continuously at a basic flow rate of 2 mL per h. The primary outcome measure was the Face, Legs, Activity, Cry, and Consolability (FLACC) pain score. Secondary outcomes included heart rate (HR), respiration rate (RR), SpO The FLACC score was significantly lower in H1 and H2 groups compared to S. The Ramsay sedation score was significantly higher in H1 and H2 groups compared to S. Recovery time was shorter in H1 group compared to patients H2 group or S group. There were no significant differences in the PAED scale, HR, RR, SpO2, adverse reactions, satisfaction of parents with analgesia, or length and cost of hospital stay. Hydromorphone hydrochloride is a more effective analgesic than sufentanil for postoperative pain in pediatric patients following surgical repair of a structural congenital malformation, however, hydromorphone hydrochloride and sufentanil had similar safety profiles in this patient population. Chinese Clinical Trial Register ChiCTR-INR-17013935). Clinical trial registry URL: Date of registration: December 14, 2017. [\hyperlink{Buprenorphine Hydrochloride}{PMID: 34271853}, Yongying Pan et al., 2021]

\hypertarget{pmid_17941284}{T}he safety of fexofenadine has been examined extensively in adults and school-age children. However, the safety of fexofenadine in children younger than 6 years has not been reported to date. To compare the safety and tolerability of twice-daily fexofenadine hydrochloride, 30 mg, and placebo in preschool children aged 2 to 5 years with allergic rhinitis. This was a multicenter, double-blind, randomized, placebo-controlled, parallel-group study, conducted between February 29, 2000, and June 14, 2001. Participants were randomized to either fexofenadine hydrochloride, 30 mg, or placebo twice daily for a 2-week period. To facilitate dosing, capsule content was mixed with applesauce (approximately 10 mL). Safety assessments depended on date of entry into the study because of an amendment to the protocol. Before the amendment, assessments included physical examination, vital signs reporting (oral temperature, heart rate, and respiratory rate), and adverse event (AE) reporting. After the amendment, safety assessments included laboratory testing (blood chemistry and hematology profiles), physical examination, 12-lead electrocardiography, and vital signs (oral temperature, blood pressure, heart rate, and respiratory rate) and AE reporting. Treatment-emergent AEs were observed in 116 of 231 participants receiving placebo and 111 of 222 receiving fexofenadine. These AEs were possibly related to study medication in 19 (8.2\%) and 21 (9.5\%) of the participants receiving placebo and fexofenadine, respectively, and most frequently involved the digestive system. No clinically relevant differences in laboratory measures, vital signs, and physical examinations were observed. The findings show that fexofenadine hydrochloride, 30 mg, is well tolerated and has a good safety profile in children aged 2 to 5 years with allergic rhinitis. [\hyperlink{Buprenorphine Hydrochloride}{PMID: 17941284}, Henry Milgrom et al., 2007]

\hypertarget{pmid_33514404}{D}espite its recognized efficacy and tolerability profile, during the last decade a rise of adverse events following ibuprofen administration in children has been reported, including a possible role in worsening the clinical course of infections. Our aim was to critically evaluate the safety of ibuprofen during the course of pediatric infectious disease in order to promote its appropriate use in children. Ibuprofen is associated with severe necrotizing soft tissue infections (NSTI) during chickenpox course. Pre-hospital use of ibuprofen seems to increase the risk of complicated pneumonia in children. Conflicting data have been published in septic children, while ibuprofen in the setting of Cystic Fibrosis (CF) exacerbations is safe and efficacious. No data is yet available for ibuprofen use during COVID-19 course. Ibuprofen should not be recommended for chickenpox management. Due to possible higher risks of complicated pneumonia, we suggest caution on its use in children with respiratory symptoms. While it remains unclear whether ibuprofen may have harmful effects during systemic bacterial infection, its administration is recommended in CF course. Despite the lack of data, it is seems cautious to prefer the use of paracetamol during COVID-19 acute respiratory distress syndrome in children. [\hyperlink{Buprenorphine Hydrochloride}{PMID: 33514404}, Lucia Quaglietta et al., 2021]

\hypertarget{pmid_2775626}{B}uprenorphine (3 micrograms kg-1) was given intravenously as premedication to small children (age 4-7 years) undergoing minor surgery. Because of the rapid decline of the plasma buprenorphine concentrations, the terminal elimination half-life could not be estimated reliably. Given this constraint, values of clearance appeared to be higher than those in adults but values of Vss were similar. [\hyperlink{Buprenorphine Hydrochloride}{PMID: 2775626}, K T Olkkola et al., 1989]  A preliminary evaluation to review the scope and quality of evidence surrounding transdermal buprenorphine use in the pediatric setting for non-surgical pain was conducted. Our review revealed limited data available on the use of transdermal buprenorphine in pediatric patients. Most studies surrounding this subject involve accidental ingestion of buprenorphine and its use in the treatment of neonatal abstinence syndrome. While indicated for use only in adult populations, small studies have shown encouraging results in reducing pain in children with few, if any, adverse effects. This is reassuring from a clinical perspective, as we hope to highlight the available evidence and invite researchers to expand future studies. Through this review, we have identified significant gaps in the literature surrounding the safety and use of buprenorphine in the pediatric population. To our knowledge, there are no major studies investigating this subject, and it is our hope that future studies will explore the use of transdermal buprenorphine as an alternative pain management technique in pediatrics. The intent of our scoping review is to highlight the lack of research in this area; therefore, future studies may be conducted to support its use in North America. [\hyperlink{Buprenorphine Hydrochloride}{PMID: 2775626}, Thomas S Haupt et al., 2019]

\hypertarget{pmid_37954384}{C}hildhood pneumonia, often caused by acute upper respiratory tract infections or bronchitis, is one of the leading causes of mortality in children. Nebulized inhalation, as a low-risk treatment method, has garnered significant attention. However, its effectiveness and safety remain controversial. In this study, a systematic review of relevant literature on the use of budesonide (BUD) and ambroxol hydrochloride (AMB) inhalation in the treatment of childhood pneumonia was conducted, and a total of 10 articles were included. The meta-analysis revealed an odds ratio (OR) of 1.61 and an I [\hyperlink{Buprenorphine Hydrochloride}{PMID: 37954384}, Huanan Shen et al., 2023] Allergic rhinitis (AR) is a common chronic condition in children and may impact a child's quality of life. Increasing treatment compliance may improve quality of life. An oral suspension of fexofenadine hydrochloride (HCl) has been developed to ease administration to children and may, therefore, improve treatment compliance. The purpose of this study was to assess the pharmacokinetic behavior, safety, and tolerability of a single dose of fexofenadine HCl oral suspension administered to children aged 2-5 years with allergic rhinitis. Children (aged 2-5 years) with AR were recruited in a multicenter, open-label, single-dose study. Fexofenadine HCl (30 mg) was administered as a 6-mg/mL suspension (5 mL). Plasma samples were collected up to 24 hours postdose. Adverse events (AEs); electrocardiograms (ECGs); vital signs; and clinical laboratory tests for hematology, blood chemistry, and urinalysis were analyzed to evaluate safety and tolerability. Fifty subjects completed the study. Mean maximum plasma concentration of fexofenadine was 224 ng/mL, and mean area under the plasma concentration curve was 898 ng . hour/mL. Treatment-emergent AEs were mild in intensity and reported in a total of seven subjects. No trends or clinically meaningful changes in mean ECG, vital sign, or clinical laboratory test data occurred during the study. In children aged 2-5 years, the exposure after a 30-mg dose of fexofenadine HCl suspension was similar to the exposures previously seen after a 30- and 60-mg dose of fexofenadine HCl in children aged 6-11 years and in adults, respectively. The suspension was also well tolerated. [\hyperlink{Buprenorphine Hydrochloride}{PMID: 37954384}, Nathan Segall et al., ]

\hypertarget{pmid_7489418}{C}audal buprenorphine was investigated as a postoperative analgesic in a randomized double blind study in thirty children aged 5-12 years undergoing lower abdominal and lower limb surgery. Comparison was made between two groups of patients, one group receiving plain bupivacaine and the other a combination of plain bupivacaine with buprenorphine. Postoperative analgesia was assessed using a linear analogue scale, and by the response to direct questioning of children using an illustration of sequence of faces. Any untoward side effects and the need for additional analgesics were recorded. The degree and duration of analgesia was far superior in the buprenorphine group and there was a highly significant difference in the requirement of postoperative analgesia between the two groups. There were no major adverse side effects and no motor weakness in either groups, however the incidence of nausea and vomiting was higher in the buprenorphine group. It is concluded that a combination of bupivacaine with buprenorphine administered through the caudal epidural space is a safe and reliable means of providing postoperative pain relief in children for up to 24 h. [\hyperlink{Buprenorphine Hydrochloride}{PMID: 7489418}, R S Kamal et al., 1995]

\hypertarget{pmid_31992493}{B}uprenorphine prescriptions have increased dramatically within the United States, whereas methadone continues to be used widely. We investigated the trends and characteristics of buprenorphine and methadone exposures in the pediatric population. We identified pediatric exposures to buprenorphine and methadone using the National Poison Data System from 2013 to 2016. We descriptively assessed characteristics of the exposures. Trends in exposures were evaluated using generalized linear mixed models. Pediatric buprenorphine exposures increased from 2013 (1097) to 2016 (1226) while methadone calls decreased (486 to 396). After adjusting for the random effects of the geographical region, the mean number of pediatric buprenorphine exposures (per 100,000 pediatric population) increased from 1.3 to 1.5 (P = .05). Conversely, the mean number of methadone exposures decreased from 0.6 to 0.4 (P = .03). Children aged ≤3 years constituted the highest percentage of both exposures. Unintentional exposures accounted for most of the buprenorphine (86.9\%) and methadone (62.4\%) exposures. Major clinical effects were demonstrated in 2.3\% of buprenorphine exposures and were more frequent with methadone (13\%). West Virginia and Maryland demonstrated the highest incidence of buprenorphine and methadone exposures, respectively. Pediatric buprenorphine exposures increased but demonstrated less severe effects compared to methadone exposures, which decreased during the study period. [\hyperlink{Buprenorphine Hydrochloride}{PMID: 31992493}, Saumitra V Rege et al., 2020]

\section*{Clindamycin Palmitate Hydrochloride}
\subsection*{Result}
\subsubsection*{Answer}

Ages 1–14 years: Yes  
Ages 1–2 years: Unknown  
Infants and neonates (<1 year): Unknown  

\subsubsection*{{Explanation}}
\hypertarget{Clindamycin Palmitate Hydrochloride}
A review of the available abstracts reveals several studies specifically addressing the use of clindamycin palmitate hydrochloride in children. Below is a summary of the relevant evidence, organized by age range where possible:

1. Children with Upper Respiratory Illnesses (No specific age range, but "children" is used):
- A randomized study compared clindamycin palmitate to potassium phenoxymethyl penicillin in 103 children with upper respiratory illnesses and pharyngeal group A streptococci. Clindamycin palmitate was administered orally for 10 days. The study reported that clindamycin palmitate was as effective as penicillin in eradicating group A streptococci. Possible drug-related rashes were observed in 8 of 52 clindamycin palmitate-treated patients, but no severe adverse events were reported. The authors concluded that clindamycin palmitate should not be preferred to penicillin in non-allergic patients due to rash tendency and higher cost, but did not raise safety concerns that would preclude its use in children [\hyperlink{pmid_4208902}{PMID: 4208902}, M Stillerman et al., 1973].

2. Children with Osteomyelitis (No specific age range, but "children" is used):
- In a study of 29 children with osteomyelitis, clindamycin phosphate was given intravenously for about three weeks, followed by oral clindamycin palmitate at home for an additional six weeks. The results were described as good to excellent, with prompt clinical and bacteriologic response and no diarrhea or enterocolitis observed, even with high doses for up to nine weeks [\hyperlink{pmid_910760}{PMID: 910760}, W Rodriguez et al., 1977].

3. Pediatric Malaria (Ages 1 to 14 years, with subgroup analysis for 1–2 years):
- A study of 51 pediatric outpatients aged 1 to 14 years treated with fosmidomycin plus clindamycin (10 mg/kg every 12 hours for 3 days) found the combination generally well tolerated, though there were relatively high rates of neutropenia (16\%) and significant hemoglobin drops (14\%). Efficacy was lower in children aged 1–2 years (day 28 cure rate 62\%). The study highlights the need for further systematic studies, especially in children under 3 years, due to concerns about efficacy and some safety signals [\hyperlink{pmid_16870763}{PMID: 16870763}, Steffen Borrmann et al., 2006].

4. Infants and Neonates (Clindamycin phosphate, not palmitate, but relevant for safety):
- Several studies address intravenous clindamycin (not specifically the palmitate hydrochloride form) in infants and neonates, including those less than 1 year old. These studies report on pharmacokinetics and dosing, with no significant adverse events directly attributed to clindamycin [\hyperlink{pmid_6470871}{PMID: 6470871}, M J Bell et al., 1984; \hyperlink{pmid_31725114}{PMID: 31725114}, Rachel G Greenberg et al., 2020; \hyperlink{pmid_26926644}{PMID: 26926644}, Daniel Gonzalez et al., 2016]. However, these do not specifically address the palmitate hydrochloride form.

Summary:
- For children (age not always specified, but includes at least ages 1–14), clindamycin palmitate hydrochloride has been studied for safety and efficacy in targeted clinical trials. The studies affirm its general safety in children, with rash as a notable but non-severe adverse event. In the context of malaria treatment, some hematologic adverse events (neutropenia, hemoglobin drop) were observed, and efficacy was lower in children under 3 years, suggesting caution and the need for further study in this youngest subgroup.
- For infants and neonates, there is no targeted safety study of clindamycin palmitate hydrochloride; available studies focus on other forms of clindamycin.

Conclusion:
- For children (approximately ages 1–14), targeted studies affirm that clindamycin palmitate hydrochloride is generally safe, with some adverse events (rash, neutropenia, hemoglobin drop) that warrant monitoring.
- For children under 3 years, especially ages 1–2, safety is less certain due to lower efficacy and higher rates of some adverse events in the malaria study.
- For infants and neonates (<1 year), safety of clindamycin palmitate hydrochloride is unknown based on available abstracts.

\subsection*{Abstracts}
\hypertarget{pmid_4208902}{C}lindamycin palmitate and potassium phenoxymethyl penicillin were evaluated in 103 children with upper respiratory illnesses and pharyngeal group A streptococci, from November 1970 to July 1971. The children were assigned randomly by weight to one of the antibiotic regimens given orally for 10 days. Clindamycin palmitate and potassium phenoxymethyl penicillin dosages were 75 and 125 mg, respectively, in 5 ml tid for children weighing less than 25 kg, and 150 and 250 mg, respectively, in 10 ml bid for children weighing 25 kg or more. Recurrences of the original streptococcal group A, M, and T types within 3 weeks after the end of treatment were classified as failures. The failure rates were: clindamycin palmitate, 10\% (5 of 52), and potassium phenoxymethyl penicillin, 18\% (9 of 51). Possible drug-related rashes were observed in 8 of 52 clindamycin palmitate-treated patients. The geometric mean minimal inhibitory concentrations of clindamycin and penicillin against 103 isolates of group A streptococci were 0.033 and 0.007 mug/ml, respectively. The serum concentrations about 70 min after ingesting 150 mg of clindamycin palmitate averaged 3.8 mug/ml and after 250 mg of potassium phenoxymethyl penicillin averaged 0.9 mug/ml. Clindamycin palmitate was as effective as potassium phenoxymethyl penicillin in eradicating group A streptococci from the pharynx in tid and bid regimens. Nevertheless, because of its rash-producing tendency in some patients and higher cost, clindamycin palmitate should not be preferred to penicillin for treatment of streptococcal sore throat in the non-penicillin-allergic patient. [\hyperlink{Clindamycin Palmitate Hydrochloride}{PMID: 4208902}, M Stillerman et al., 1973]

\hypertarget{pmid_910760}{C}lindamycin phosphate was used in the treatment of 29 children with osteomyelitis of whom 25 had an acute and four a chronic type of infection. The usual dose was 50 mg/kg/day intravenously for approximately three weeks followed by oral clindamycin palmitate at home in a dose of 30 mg/kg/day for an additional six weeks. Staphylococcus aureus was isolated in 22 of 29 cases: 96\% of strains were penicillin resistant. The clinical and bacteriologic results in the present series were good to excellent. There was prompt clinical and bacteriologic response shortly after initiation of clindamycin therapy. Good bone penetration of the drug was observed. Long-term evaluation revealed satisfactory clinical and roentgenographic progress in all patients. No diarrhea or manifestations of enterocolitis appeared in any patient in spite of high doses of the drug for intervals up to nine weeks. [\hyperlink{Clindamycin Palmitate Hydrochloride}{PMID: 910760}, W Rodriguez et al., 1977]

\hypertarget{pmid_31725114}{D}espite the absence of adequate safety or efficacy data, clindamycin is widely prescribed in the neonatal intensive care unit. We evaluated the association between clindamycin exposure and adverse events, as well as antibiotic effectiveness in infants. This was a retrospective cohort study of infants receiving clindamycin before postnatal day 121 who were discharged from a Pediatrix Medical Group neonatal intensive care unit (1997-2015). Using a previously developed pharmacokinetic model, we performed simulations to predict clindamycin exposure based on available dosing data. We used multivariable logistic regression to evaluate the association between clindamycin exposure and safety outcomes during and after clindamycin therapy. We reported the proportion of infants with methicillin-resistant Staphylococcus aureus (MRSA) bacteremia and clearance of MRSA bacteremia. A total of 4089 infants received clindamycin at a median (25th-75th percentile) dose of 15 mg/kg/d (12-16). Clearance increased with older gestational age. Infants with the highest total clindamycin exposure had marginally increased odds of necrotizing enterocolitis within 7 days (adjusted odds ratio = 1.95 [1.04-3.63]), but exposure was not associated with death, sepsis, seizures, intestinal perforation or intestinal strictures. Of 25 infants who had MRSA bacteremia, 19 (76\%) cleared the infection by the end of the clindamycin course. Higher clindamycin exposure was not associated with increased odds of death or nonlaboratory adverse events. The use of pharmacokinetic models combined with available electronic health record data offers a valuable, cost-effective approach to analyzing the safety and effectiveness of drugs in infants when large-scale trials are not feasible. [\hyperlink{Clindamycin Palmitate Hydrochloride}{PMID: 31725114}, Rachel G Greenberg et al., 2020]

\hypertarget{pmid_26926644}{C}lindamycin may be active against methicillin-resistant Staphylococcus aureus, a common pathogen causing sepsis in infants, but optimal dosing in this population is unknown. We performed a multicenter, prospective pharmacokinetic (PK) and safety study of clindamycin in infants. We analyzed the data using a population PK analysis approach and included samples from two additional pediatric trials. Intravenous data were collected from 62 infants (135 plasma PK samples) with postnatal ages of <121 days (median [range] gestational age of 28 weeks [23 to 42] and postnatal age of 17 days [1 to 115]). In addition to body weight, postmenstrual age (PMA) and plasma protein concentrations (albumin and alpha-1 acid glycoprotein) were found to be significantly associated with clearance and volume of distribution, respectively. Clearance reached 50\% of the adult value at PMA of 39.5 weeks. Simulated PMA-based intravenous dosing regimens administered every 8 h (≤32 weeks PMA, 5 mg/kg; 32 to 40 weeks PMA, 7 mg/kg; >40 to 60 weeks PMA, 9 mg/kg) resulted in an unbound, steady-state concentration at half the dosing interval greater than a MIC for S. aureus of 0.12 μg/ml in >90\% of infants. There were no adverse events related to clindamycin use. (This study has been registered at ClinicalTrials.gov under registration no. NCT01728363.). [\hyperlink{Clindamycin Palmitate Hydrochloride}{PMID: 26926644}, Daniel Gonzalez et al., 2016]

\hypertarget{pmid_24447296}{C}hloral hydrate is the most commonly used sedative for paediatric diagnostic procedures in China with a success rate of around 80\%. Intranasal dexmedetomidine is used for rescue sedation in our centre. This prospective investigation evaluated 213 children aged one month to 10 years who were not adequately sedated following administration of chloral hydrate. Children were randomly assigned to receive rescue intranasal dexmedetomidine at 1 μg.kg(-1) (group 1), 1.5 μg.kg(-1) (group 2) or 2 μg.kg(-1) (group 3). The sedation level was assessed every 10 min using a modified observer's assessment of alertness/sedation scale. Successful rescue sedation in groups 1, 2 and 3 were 56 (83.6\%), 66 (89.2\%) and 51 (96.2\%), respectively. Increasing the rescue dose was associated with an increased success rate with an odds ratio of 4.12 (95\% CI 1.13-14.98), p = 0.032. We conclude that intranasal dexmedetomidine is effective for sedation in children who do not respond to chloral hydrate.  [\hyperlink{Clindamycin Palmitate Hydrochloride}{PMID: 24447296}, B L Li et al., 2014] Chloral hydrate is commonly used to sedate children for painless procedures. Children may recover more quickly after sedation with dexmedetomidine, which has a shorter half-life. We randomly allocated 196 children to chloral hydrate syrup 50 mg.kg [\hyperlink{Clindamycin Palmitate Hydrochloride}{PMID: 24447296}, V M Yuen et al., 2017] Sedation is often required for children undergoing diagnostic procedures. Chloral hydrate has been one of the sedative drugs most used in children over the last 3 decades, with supporting evidence for its efficacy and safety. Recently, chloral hydrate was banned in Italy and France, in consideration of evidence of its carcinogenicity and genotoxicity. Dexmedetomidine is a sedative with unique properties that has been increasingly used for procedural sedation in children. Several studies demonstrated its efficacy and safety for sedation in non-painful diagnostic procedures. Dexmedetomidine's impact on respiratory drive and airway patency and tone is much less when compared to the majority of other sedative agents. Administration via the intranasal route allows satisfactory procedural success rates. Studies that specifically compared intranasal dexmedetomidine and chloral hydrate for children undergoing non-painful procedures showed that dexmedetomidine was as effective as and safer than chloral hydrate. For these reasons, we suggest that intranasal dexmedetomidine could be a suitable alternative to chloral hydrate. [\hyperlink{Clindamycin Palmitate Hydrochloride}{PMID: 24447296}, Giorgio Cozzi et al., 2017]

\hypertarget{pmid_24949994}{C}lindamycin is commonly prescribed to treat children with skin and skin-structure infections (including those caused by community-acquired methicillin-resistant Staphylococcus aureus (CA-MRSA)), yet little is known about its pharmacokinetics (PK) across pediatric age groups. A population PK analysis was performed in NONMEM using samples collected in an opportunistic study from children receiving i.v. clindamycin per standard of care. The final model was used to optimize pediatric dosing to match adult exposure proven effective against CA-MRSA. A total of 194 plasma PK samples collected from 125 children were included in the analysis. A one-compartment model described the data well. The final model included body weight and a sigmoidal maturation relationship between postmenstrual age (PMA) and clearance (CL): CL (l/h) = 13.7 × (weight/70)(0.75) × (PMA(3.1)/(43.6(3.1) + PMA(3.1))); V (l) = 61.8 × (weight/70). Maturation reached 50\% of adult CL values at \textasciitilde{}44 weeks PMA. Our findings support age-based dosing.  [\hyperlink{Clindamycin Palmitate Hydrochloride}{PMID: 24949994}, D Gonzalez et al., 2014] To examine whether three cycles of a low-intensity chemotherapy consisting of cyclophosphamide [500 mg/m(2) - day 1], vinblastine [6 mg/m(2) - days 1 and 8] and prednisolone [40 mg/m(2) - days 1-7] (CVP) is safe and therapeutically effective in children and adolescents with early stage nodular lymphocyte predominant Hodgkin lymphoma [nLPHL]. Fifty-five children and adolescents with early stage nLPHL [median age 13 years, range 4-17 years] diagnosed between June 2005 and October 2010 in the UK and France are the subjects of this report. Staging investigations included conventional cross sectional as well as 18 fluro-deoxyglucose [FDG] PET imaging. Histology was confirmed as nLPHL by an expert pathology panel. Of the 45 patients, who received CVP as first line treatment, 36 [80\%, 95\% Confidence Interval [CI]: (68; 92)] either achieved a complete remission [CR] or CR unconfirmed [CRu], the remaining nine patients achieved a partial response. All nine subsequently achieved CR with salvage chemotherapy [n=7] or radiotherapy [n=2]. Ten patients received CVP at relapse after primary treatment that consisted of surgery alone and all achieved CR. To date, only three patients have relapsed after CVP chemotherapy and all had received CVP as first line treatment at initial diagnosis. The 40-month freedom from treatment failure and overall survival for the entire cohort were 75.4\% (SE ± 6\%) and 100\%, respectively. No significant early toxicity was observed. Our results show that CVP is an effective chemotherapy regimen in children and adolescents with early stage nLPHL that is well tolerated with minimal acute toxicity. [\hyperlink{Clindamycin Palmitate Hydrochloride}{PMID: 24949994}, Ananth Shankar et al., 2012]

\hypertarget{pmid_32373914}{T}o evaluate the practice and attitude of pediatrics nephrologists about cinacalcet use in children. An electronic structured questionnaire was answered by pediatric nephrologists practicing in the Kingdom of Saudi Arabia (KSA) and Gulf Council countries (GCC). A total of 42  pediatric nephrologists responded, of them, 42\% used cinacalcet for young children ≤5 years of age and 79\% used for children. There were wide variations in the method of administration (examples: crushed, divided, whole tablets), monitoring, doses and response definition, and follow-up. No serious complications after starting cinacalcet was observed in 50\%, while 40\% reported various complications, mainly hypocalcemia (70\%). Cinacalcet was stopped without achieving the target parathyroid hormone in more than half (55\%) of children because of intractable adverse effects (40\%), poor response (30\%), non-adherence (25\%), or high cost (5\%). Cinacalcet is used by the majority of pediatric nephrologists in KSA and GCC. A standard clinical guideline is needed to be followed by all users. [\hyperlink{Clindamycin Palmitate Hydrochloride}{PMID: 32373914}, Rafif A Al-Ahmad et al., 2020]

\hypertarget{pmid_10724028}{C}hildren infected with Chlamydia pneumoniae sometimes experience lower respiratory tract infections such as pneumonia and bronchitis. Although numerous anti-microbial compounds have been reported to be active against the organism, most of them have not been in a clinical trial in infants and children with C. pneumoniae infection. Clarithromycin has been shown to express anti-chlamydial effects in vitro. In this study, we evaluated the clinical anti-C. pneumoniae properties of clarithromycin in children with mainly lower respiratory tract infection. We administered clarithromycin orally to 21 infants and children at a dose of 10-15 mg/kg/day divided into two or three doses for 4-21 days. Clinical symptoms, roentgenographic and laboratory abnormal findings improved. The overall clinical efficacy rate was 85.7\% (18 of 21 cases). Administration of clarithromycin was considered to be a suitable treatment for improving lower respiratory infections in infants and children caused by C. pneumoniae. [\hyperlink{Clindamycin Palmitate Hydrochloride}{PMID: 10724028}, K Numazaki et al., 2000]

\hypertarget{pmid_27463797}{C}lindamycin hydrochloride (CLH) is a clinically important oral antibiotic with wide spectrum of antimicrobial activity that includes gram-positive aerobes (staphylococci, streptococci etc.), most anaerobic bacteria, Chlamydia and certain protozoa. The current study was focused to develop a stabilised clindamycin encapsulated poly lactic acid (PLA)/poly (D,L-lactide-co-glycolide) (PLGA) nano-formulation with better drug bioavailability at molecular level. Various nanoparticle (NPs) formulations of PLA and PLGA loaded with CLH were prepared by solvent evaporation method varying drug: polymer concentration (1:20, 1:10 and 1:5) and characterised (size, encapsulation efficiency, drug loading, scanning electron microscope, differential scanning calorimetry [DSC] and Fourier transform infrared [FTIR] studies). The ratio 1:10 was found to be optimal for a monodispersed and stable nano formulation for both the polymers. NP formulations demonstrated a significant controlled release profile extended up to 144 h (both CLH-PLA and CLH-PLGA). The thermal behaviour (DSC) studies confirmed the molecular dispersion of the drug within the system. The FTIR studies revealed the intactness as well as unaltered structure of drug. The CLH-PLA NPs showed enhanced antimicrobial activity against two pathogenic bacteria Streptococcus faecalis and Bacillus cereus. The results notably suggest that encapsulation of CLH into PLA/PLGA significantly increases the bioavailability of the drug and due to this enhanced drug activity; it can be widely applied for number of therapies.  [\hyperlink{Clindamycin Palmitate Hydrochloride}{PMID: 27463797}, Pradipta Ranjan Rauta et al., 2016] Chloral hydrate has been used extensively to sedate children, but at Brooke Army Medical Center, other drug combinations were becoming increasingly popular due to a perception that chloral hydrate had a high rate of failure, especially with younger or neurologically impaired children. Therefore, 50 children were given the drug before a diagnostic study, and patient data and a sedation score were recorded on a worksheet. Of 50 children, 43 (86\%) were "successfully sedated" on the first attempt with no side effects. Children with neurologic disorders had a much greater (27\% vs 4\%) failure rate than "normal" children. The sedation rate did not significantly differ by age, sex, or initial drug dosage. The study suggest that chloral hydrate is a safe and effective oral sedative but that children with neurologic disorders may need alternative drugs for sedation. [\hyperlink{Clindamycin Palmitate Hydrochloride}{PMID: 27463797}, P D Rumm et al., 1990]

\hypertarget{pmid_7247166}{C}lindamycin hydrochloride hydrate (Cleocin), a semisynthetic antibiotic shown experimentally to be effective in ocular toxoplasmosis in the rabbit, was used in the treatment of four patients with active retinochoroiditis secondary to toxoplasmosis. The drug was administered subjunctivally on alternate days for 30 days. Both subjective and objective evidence indicated beneficial results in these patients during the first 30 days. One of the four did not respond during the first 30 days but did respond during an extended period. One of those who responded initially had exacerbations when the drug was stopped and required treatment. [\hyperlink{Clindamycin Palmitate Hydrochloride}{PMID: 7247166}, J G Ferguson et al., 1981]

\hypertarget{pmid_35049570}{C}lindamycin hydrochloride is a widely used antibiotic for topical use, but its main disadvantage is poor skin penetration. Therefore, new approaches in the development of clindamycin topical formulations are of great importance. We aimed to investigate the effects of the type of gelling agent (carbomer and sodium carmellose), and the type and concentration of bile acids as penetration enhancers (0.1\% and 0.5\% of cholic and deoxycholic acid), on clindamycin release rate and permeation in a cellulose membrane in vitro model. Eight clindamycin hydrogel formulations were prepared using a 2 [\hyperlink{Clindamycin Palmitate Hydrochloride}{PMID: 35049570}, Nebojša Pavlović et al., 2022] Chloral hydrate (CH) is an oral sedative widely used to sedate infants and young children during auditory brainstem response (ABR) testing. The aim of this study was to record effectiveness, complications and safety of CH as a sedative for ABR. From January of 2003 until December of 2007, 1903 children were tested for ABR, 568 of them being under the age of 6 months. CH (8\%) was used for sedation at a dose of 40 mg/kg with a repeat dose, if necessary, for an adequate sedation, in 20-30 min. We recorded the effectiveness of CH as a sedative for ABR examination, as well as all complications related to the use of CH such as vomiting, rash, hyperactivity, respiratory distress and apnea. The statistical method used was the absolute and percentage frequency distribution of the occurrences. Sedation with CH was necessary to perform testing in 1591 (83.6\%) of the examined children. However, in the population of the examined infants, only 341 (60\%) were sedated with CH, because the remaining 227 (40\%) fell asleep by themselves. Complications included hyperactivity in 152 children (8\%), minor respiratory distress in 10 children (0.4\%), vomiting in 217 children (11.4\%), apnea in 4 children (0.2\%) and rash in 10 children (0.4\%). The complications of hyperactivity, vomiting and rash resolved without any medical treatment. The apnea cases were managed effectively by supplying ventilation to the children via a mask in the presence of an anesthesiologist. The use of CH at a dose of 40 mg/kg up to 80 mg/kg is safe and effective when administered in a setting with adequate equipment and the presence of well trained personnel. [\hyperlink{Clindamycin Palmitate Hydrochloride}{PMID: 35049570}, Eirini Avlonitou et al., 2011]

\hypertarget{pmid_16870763}{F}osmidomycin plus clindamycin was shown to be efficacious in the treatment of uncomplicated Plasmodium falciparum malaria in a small cohort of pediatric patients aged 7 to 14 years, but more data, including data on younger children with less antiparasitic immunity, are needed to determine the potential value of this new antimalarial combination. We conducted a single-arm study to improve the precision of efficacy estimates for an oral 3-day fixed-ratio combination of fosmidomycin and clindamycin at 30 and 10 mg/kg of body weight, respectively, every 12 hours for the treatment of uncomplicated P. falciparum malaria in 51 pediatric outpatients aged 1 to 14 years. Fosmidomycin plus clindamycin was generally well tolerated, but relatively high rates of treatment-associated neutropenia (8/51 [16\%]) and falls of hemoglobin concentrations of > or =2 g/dl (7/51 [14\%]) are of concern. Asexual parasites and fever were cleared within median periods of 42 h and 38 h, respectively. All patients who could be evaluated were parasitologically and clinically cured by day 14 (49/49; 95\% confidence interval [CI], 93 to 100\%). The per-protocol, PCR-adjusted day 28 cure rate was 89\% (42/47; 95\% CI, 77 to 96\%). Efficacy appeared to be significantly reduced in children aged 1 to 2 years, with a day 28 cure rate of only 62\% for this small subgroup (5/8). The inadequate efficacy in children of <3 years highlights the need for continued systematic studies of the current dosing regimen, which should include randomized trial designs. [\hyperlink{Clindamycin Palmitate Hydrochloride}{PMID: 16870763}, Steffen Borrmann et al., 2006]

\hypertarget{pmid_25044425}{C}lindamycin hydrochloride belongs to the antibiotic family of lincomycin. It has the same antibacterial spectrum as lincomycin, but the antibacterial activity is four to eight times stronger than that of lincomycin. There have been some adverse reactions in clinical use of clindamycin hydrochloride and its finished drug products. The impurities in drugs are directly related to their safety. In this study, two unknown impurities were isolated from the raw material of clindamycin hydrochloride through various chromatographic methods. Their structures were identified as clindamycin isomer (impurity 1) and dehydroclindamycin (impurity 2) by mass spectrometry and NMR spectroscopy. Both of them were found for the first time. The two impurities exhibit a similar but lower antibacterial activity compared with clindamycin hydrochloride.  [\hyperlink{Clindamycin Palmitate Hydrochloride}{PMID: 25044425}, Qiushi Sun et al., 2014] Although chloral hydrate (CH) has been used as a sedative for decades, it is not widely accepted as a valid choice for ophthalmic examinations in uncooperative children. This study aimed to systematically review the literature on the drug's safety and efficacy. We searched PubMed, EMBASE, ISI Web of Science, Scopus, CENTRAL, Google Scholar and Trip database to 1 October 2015, using the keywords 'chloral hydrate', 'paediatric' and 'procedural sedation OR diagnostic sedation'. A meta-analysis of randomised controlled trials (RCTs) was performed. A total of 6961 articles were screened and 104 were included in the review. Thirteen of these concerned paediatric ophthalmic examination, while 13 others were RCTs and were meta-analysed. CH was reported to have been administered in a total of 24 265 sedation episodes in children aged from <1 month to 18 years. The meta-analysis showed CH had a higher OR (2.95, 95\% CI 1.09 to 7.99) for successful sedation compared to other sedatives, but significant limitations apply. The commonest reported adverse events (AE) were not serious (eg, paradoxical reaction or transient vomiting) and required no intervention. Severe AE, including two deaths, were related to comorbidity, overdose or aspiration. Despite the paucity of high quality evidence, the existing literature suggests that the use of CH for procedural sedation in children appears to be an effective alternative to general anaesthesia, and it can be safe when administered in the hospital setting with appropriate monitoring and vigilance for intervention. [\hyperlink{Clindamycin Palmitate Hydrochloride}{PMID: 25044425}, Asimina Mataftsi et al., 2017]

\hypertarget{pmid_18805603}{W}e sought to evaluate efficacy, safety, and tolerability of a combination of clindamycin phosphate 1.2\% and benzoyl peroxide 2.5\% (clindamycin-BPO 2.5\%) aqueous gel in moderate to severe acne vulgaris. A total of 2813 patients, aged 12 years or older, were randomized to receive clindamycin-BPO 2.5\%, individual active ingredients, or vehicle in two identical, double-blind, controlled 12-week, 4-arm studies evaluating safety and efficacy (inflammatory and noninflammatory lesion counts) using Evaluator Global Severity Score and subject self-assessment. Clindamycin-BPO 2.5\% demonstrated statistical superiority to individual active ingredients and vehicle in reducing both inflammatory and noninflammatory lesions and acne severity. Visibly greater improvement was observed by patients with clindamycin-BPO 2.5\% as early as week 2. No substantive differences were seen in cutaneous tolerability among treatment groups and less than 1\% of patients discontinued treatment because of adverse events. Data from controlled studies may differ from clinical practice. Clindamycin-BPO 2.5\% provides statistically significant greater efficacy than individual active ingredients and vehicle with a highly favorable safety and tolerability profile. [\hyperlink{Clindamycin Palmitate Hydrochloride}{PMID: 18805603}, Diane Thiboutot et al., 2008]

\hypertarget{pmid_2026812}{C}hloral hydrate is commonly used to sedate children before CT. However, no prospective study has been published of the safety and efficacy of chloral hydrate at high dose levels for children undergoing CT. We define high dose levels of oral chloral hydrate to be 80-100 mg/kg, with a maximum total dose of 2 g. High dose chloral hydrate sedation was administered orally to 295 children for 326 CT examinations. Adverse reactions occurred in 7\% of the children, with vomiting being the most common (4.3\% of children). Hyperactivity and respiratory symptoms each occurred in less than 2\% of children. Prolonged sedation ( greater than 2 h) was not encountered in our series. Sedation was successful in producing motion free CT examinations, so that in 303 (93\%) of the cases, no repeat CT scans were needed. We conclude that high dose oral chloral hydrate provides safe and effective sedation for children undergoing CT. [\hyperlink{Clindamycin Palmitate Hydrochloride}{PMID: 2026812}, S B Greenberg et al., ]

\hypertarget{pmid_16553852}{T}he effects of subinhibitory concentrations of clindamycin on the morphological, biochemical and genetic characteristics of species of the Bacteroides fragilis group isolated from children with diarrhea were determined. The minimal inhibitory and subinhibitory concentrations for clindamycin were determined. Minimal inhibitory concentration values ranging from 0.25 to 512 microg mL(-1) were observed. Cultures grown with clindamycin were used to determine the macroscopic morphological characteristics, cellular viability, ultrastructural characteristics and DNA integrity. Clindamycin did not alter colonial morphology, but after 6 h elongated cells were observed. Also, extracellular vesicles and electron-lucent areas inside the cytoplasm were observed. Bacteria treated with clindamycin also showed fragmentation of DNA as determined by electrophoresis. The alterations produced by clindamycin might be indicative of a possible modification of the structures involved in bacterial pathogenesis. [\hyperlink{Clindamycin Palmitate Hydrochloride}{PMID: 16553852}, Elessandra Maria Silvestro et al., 2006]

\hypertarget{pmid_30141180}{C}alcimimetics, shown to control biochemical parameters of secondary hyperparathyroidism (SHPT), have well-established safety and pharmacokinetic profiles in adult end-stage renal disease subjects treated with dialysis; however, such studies are limited in pediatric subjects. In this study, the safety, tolerability, pharmacokinetics (PK), and pharmacodynamics (PD) of cinacalcet were evaluated in children with chronic kidney disease (CKD) and SHPT receiving dialysis. Twelve subjects received a single dose of cinacalcet (0.25 mg/kg) orally or by nasogastric or gastric tube. Subjects were randomized to one of two parathyroid hormone (PTH) and serum calcium sampling sequences: [(1) 2, 8, 48 h; or (2) 2, 12, 48 h] and assessed for 72 h after dosing. Median plasma cinacalcet t In conclusion, a single 0.25 mg/kg dose of cinacalcet was evaluated to be a safe starting dose in these children aged < 6 years. [\hyperlink{Clindamycin Palmitate Hydrochloride}{PMID: 30141180}, Winnie Y Sohn et al., 2019]

\hypertarget{pmid_6470871}{T}he pharmacokinetics of intravenously administered clindamycin phosphate was studied in 40 children less than 1 year of age. Mean peak serum concentrations were 10.92 micrograms/ml in premature infants less than 4 weeks of age, 10.45 micrograms/ml in term infants greater than 4 weeks, and 12.69 micrograms/ml in term infants less than 4 weeks of age. Mean trough concentrations were 5.52, 2.8, and 3.03 micrograms/ml, respectively, in the same groups. Serum half-life was significantly longer (8.68 vs 3.60 hours) in premature compared with term infants less than 4 weeks of age. Both premature and term infants less than 4 weeks had significantly decreased clearance when compared with infants greater than 4 weeks (0.294 and 0.678, respectively, vs 1.58 L/hr). Clearance was significantly greater (1.919 vs 0.310 L/hr) and serum half-life less (1.75 vs 7.57 hours) in infants with body weight greater than 3.5 kg. On the basis of these data it is recommended that in infants greater than 4 weeks or greater than 3.5 kg, intravenous clindamycin dosage be 20 mg/kg/day in four divided doses. In premature neonates less than 4 weeks, the dose should be reduced to 15 mg/kg/day in three divided doses. Term infants greater than 1 week of age may also receive 20 mg/kg/day in four doses. [\hyperlink{Clindamycin Palmitate Hydrochloride}{PMID: 6470871}, M J Bell et al., 1984]

\hypertarget{pmid_15951862}{D}iagnostic and therapeutic procedures in children are made easier using sedation. However, there is no consensus about which drug should be used to achieve this. Furthermore, none of the drugs used for sedation are risk free. The aim of this work is to study sedation indications, effectiveness, and safety at our center. A prospective observational study conducted at the Pediatric Day Care Unit, King Fahad National Guard Hospital, Riyadh, Saudi Arabia. The study covered 17.5 weeks in 2 periods: May 9th 1999 to June 13th 1999 and October 31st 2001 to February 11th 2002. Children <12 years were included. Collected data included demographics, indication, drug dosing and outcome. Data were reported as mean +/- SD. We included 148 patients, age 38 +/- 30 months. Adequate sedation was achieved in 79\% after initial chloral hydrate (CH) dose of 56.9 +/- 9.3 mg/kg, in 95\% after adding 18.5 +/- 6.4 mg/kg CH and in 96\% after adding second drug. Compared to nonrespondents, first CH dose respondents were younger and lower in weight. The CH side effects were few and mild. Chloral hydrate is a safe and effective agent for sedation in children with an age and weight dependent response. [\hyperlink{Clindamycin Palmitate Hydrochloride}{PMID: 15951862}, Omar M Hijazi et al., 2005]

\section*{Vigabatrin}
\subsection*{Result}
\subsubsection*{Answer}

Yes (Infants and young children up to 2 years, for short-term use)
Yes (Children up to 13 years, with caution and monitoring for visual toxicity)

\subsubsection*{{Explanation}}
\hypertarget{Vigabatrin}
Based on the available abstracts, there are multiple targeted studies evaluating the safety of Vigabatrin in children, particularly in infants and young children with infantile spasms and refractory epilepsy. The evidence can be summarized by age range as follows:

Infants and Young Children (up to 2 years):
- Several randomized controlled trials and open-label studies specifically evaluated Vigabatrin in infants with infantile spasms, including those with tuberous sclerosis complex. These studies consistently report that Vigabatrin is effective and generally well-tolerated in this age group, with few serious adverse events leading to discontinuation. For example, a randomized, single-masked, multicenter study with a 3-year follow-up in children under 2 years found Vigabatrin to be well tolerated and safe, with only nine patients discontinuing due to adverse events out of 142 [\hyperlink{pmid_11673582}{PMID: 11673582}, R D Elterman et al., 2001]. Another randomized, placebo-controlled trial in 40 children with newly diagnosed infantile spasms found no withdrawals due to adverse events [\hyperlink{pmid_10565592}{PMID: 10565592}, R E Appleton et al., 1999]. Additional studies confirm low rates of side effects and support its use as first-line therapy in this age group [\hyperlink{pmid_9095401}{PMID: 9095401}, C Chiron et al., 1997; \hyperlink{pmid_9377291}{PMID: 9377291}, M Rufo et al., 1997; \hyperlink{pmid_9733404}{PMID: 9733404}, H Siemes et al., 1998; \hyperlink{pmid_24910743}{PMID: 24910743}, Mohammad-Mahdi Taghdiri et al., 2013; \hyperlink{pmid_23748200}{PMID: 23748200}, M K Bakhshandeh Bali et al., 2014].

Children (up to 13 years):
- Studies in broader pediatric populations (up to 13 years) with refractory epilepsy or Lennox-Gastaut syndrome also report that Vigabatrin is generally well-tolerated, with mild and reversible side effects in most cases [\hyperlink{pmid_7925171}{PMID: 7925171}, M Feucht et al.; \hyperlink{pmid_8552215}{PMID: 8552215}, P Uldall et al., 1995; \hyperlink{pmid_24910743}{PMID: 24910743}, Mohammad-Mahdi Taghdiri et al., 2013]. However, there is consistent concern about the risk of visual field constriction and retinal toxicity, which appears to increase with longer duration of therapy and higher cumulative doses. Some studies report visual field constriction in over half of children tested after long-term use [\hyperlink{pmid_11944216}{PMID: 11944216}, Varda Gross-Tsur et al., 2002; \hyperlink{pmid_10894216}{PMID: 10894216}, V Gross-Tsur et al., 2000], while others report a lower incidence, especially with shorter treatment durations [\hyperlink{pmid_23748200}{PMID: 23748200}, M K Bakhshandeh Bali et al., 2014; \hyperlink{pmid_19215279}{PMID: 19215279}, Eija Gaily et al., 2009]. Most visual side effects are asymptomatic and detected only with specialized testing.

Long-term Safety:
- Long-term studies (up to 5 years) indicate that Vigabatrin remains effective and is generally well-tolerated, but the risk of visual field defects remains a concern, especially with prolonged use [\hyperlink{pmid_8552215}{PMID: 8552215}, P Uldall et al., 1995; \hyperlink{pmid_9733404}{PMID: 9733404}, H Siemes et al., 1998]. Some studies suggest that the risk of visual field loss may be lower in children treated in infancy compared to those treated at older ages [\hyperlink{pmid_19215279}{PMID: 19215279}, Eija Gaily et al., 2009].

Summary:
- For infants and young children (up to 2 years), multiple targeted studies affirm that Vigabatrin is safe for short-term use, with careful monitoring for visual side effects.
- For children up to 13 years, Vigabatrin is generally safe for use, but there is a significant risk of visual field constriction with long-term therapy, necessitating regular ophthalmologic monitoring.
- The safety of Vigabatrin in children is supported by targeted studies, but the risk of visual toxicity, particularly with prolonged use, must be weighed against the benefits, and monitoring is essential.

\subsection*{Abstracts}
\hypertarget{pmid_15941649}{V}igabatrin (VGB) is an important treatment option for infantile spasms. Vigabatrin-induced visual field defects are at present the most important safety issue in the use of the drug. The knowledge concerning VGB-associated visual dysfunction in pediatric patients, particularly in those who have been exposed to VGB in utero is limited. We explored ophthalmic and neurologic findings in two children who have been exposed prenatally to VGB. [\hyperlink{Vigabatrin}{PMID: 15941649}, Iiris Sorri et al., 2005]

\hypertarget{pmid_7925171}{T}he antiepileptic effect of vigabatrin (gamma-vinyl GABA, VGB) in children has been demonstrated in controlled and open studies. According to the literature, results were good to excellent in partial seizures (with and without becoming secondarily generalized) and promising in infantile spasms (IS). In patients with myoclonic epilepsies of early childhood and especially those with Lennox-Gastaut syndrome (LGS), the effect of VGB has been investigated only to a limited extent and the pattern of response was variable. The present open, add-on, dose-ranging study was initiated to assess the long-term effect and safety of VGB in a cohort of 20 children with LGS who were not responding sufficiently to first-line drug monotherapy with valproate (VPA) instead of adding classical second-line antiepileptic drugs [AEDs: benzodiazepines (BZD), phenobarbital (PB), primidone (PRM)], which usually are associated with rapid diminution of their antiepileptic properties and a high frequency of side effects. Eighty-five percent of children experienced a 50-100\% reduction in seizure frequency, even after dose reduction of VPA. No serious side effects occurred except in 1 patient who experienced dyskinesia. Mood changes, sedation, ataxia, and hypersalivation, well-known complications of other AEDs, were not observed. [\hyperlink{Vigabatrin}{PMID: 7925171}, M Feucht et al., ]

\hypertarget{pmid_8552215}{I}n an retrospective uncontrolled long-term study in 30 children with intractable epilepsy, it was found that treatment with vigabatrin resulted in a seizure reduction of more than 50\% at 1-year follow-up in 40\% of the children. The responders were all children with partial seizures. Side effects were mild and did not lead to discontinuation of the drug. Increased numbers of seizures were seen in three cases. A moderate weight increase was seen in 27\% of the children. At 5-year follow-up 7 children (23\%) still maintained a seizure reduction of more than 50\%. Trials of monotherapy in three seizure-free patients were unsuccessful. No further side effects were observed. A study of evoked potentials in 12 children showed no alteration in latency and amplitudes of VEP following treatment with vigabatrin. Our results show that in children vigabatrin seems to have a stable effect even though a few children may experience a breakthrough of seizures. The presented results together with previous reports on MRI-scans seem to indicate that even in children with a still maturing CNS vigabatrin is a safe drug. [\hyperlink{Vigabatrin}{PMID: 8552215}, P Uldall et al., 1995]

\hypertarget{pmid_9097366}{V}igabatrin is a structural analogue of gamma amino butyric acid (GABA), which binds irreversibly to GABA-transaminase causing increased brain levels of GABA. It is an important advance in the medical management of children with epilepsy. It appears to be particularly effective in the treatment of infantile spasms, especially when caused by tuberous sclerosis. It is also effective in the treatment of partial seizures and some generalized seizures including those of the Lennox-Gastaut syndrome. However, myoclonic seizures may be made worse by vigabatrin. It is not yet approved for use in the United States but it is approved throughout most of the rest of the world including Canada and Mexico. Release in the United States is expected in the near future. [\hyperlink{Vigabatrin}{PMID: 9097366}, W D Shields et al., 1997]

\hypertarget{pmid_38092645}{V}igabatrin is an anti-epileptic drug that inhibits the enzyme γ-aminobutyric acid (GABA)-transaminase. The anticonvulsant effect of vigabatrin involves increasing GABA levels and attenuating glutamate-glutamine cycling. Vigabatrin indications include infantile spasms and refractory focal seizures. Despite having a significant role in paediatric epileptology, vigabatrin has adverse effects, such as retinal toxicity, in up to 30\% of patients after 1 year of use and brain abnormalities on magnetic resonance imaging (MRI). The percentage of patients with brain abnormalities on MRI varies between 22-32\% of children using vigabatrin to treat infantile spasms. Risk factors for presenting these imaging abnormalities are cryptogenic infantile spasms, age <12 months old, high dosage, and possible concomitant hormonal therapy. Clinically, these abnormalities are usually asymptomatic. Histopathological analysis reveals white matter vacuolation and intramyelinic oedema. The typical findings of vigabatrin-associated brain abnormalities on MRI are bilateral and have a symmetrical hyperintense signal on T2-weighted imaging, with diffusion restriction, that often compromise the globi pallidi, thalami, subthalamic nuclei, cerebral peduncles, midbrain, dorsal brainstem, including the medial longitudinal fasciculi, and dentate nuclei of the cerebellum. In this article, the authors intend to review the clinical manifestations, histopathological features, imaging aspects, and differential diagnosis of vigabatrin-associated brain abnormalities on MRI. [\hyperlink{Vigabatrin}{PMID: 38092645}, D G Corrêa et al., 2023]

\hypertarget{pmid_11701271}{V}igabatrin (VGB) was found to be an effective anti-epileptic drug to reduce infantile spasms in about 50\% of patients and it has been found most effective in infantile spasms due to tuberous sclerosis (TSC) in which up to 95\% of infants had complete cessation of their spasms. VGB was synthesized to enhance inhibitory gamma-aminobutyric acidergic (GABAergic) transmission by elevating GABA levels via irreversible inhibition of GABA transaminase. The mechanism underlying the particular efficacy of VGB in TSC is still unknown. However, its efficacy suggests that epileptogenesis in TSC may be related to an impairment of GABAergic transmission. VGB should be considered as the first line monotheraphy for the treatment of infantile spasms in infants with confirmed diagnosis of TSC. The efficacy of VGB treatment can be assessed in less than 10 days, but usually a few days treatment with a dose of about 100 mg/kg/day stops infantile spasms. The cessation of the spasms is associated with a marked improvement of behaviour and mental development. Unfortunately, it has become clear that the use of VGB is associated with a late appearance of visual-field defects in up to 50\% of patients. Currently the minimum duration and doses of VGB treatment that can produce side effects are unknown. The feasibility of using short treatment periods (2-3 months) should be investigated. [\hyperlink{Vigabatrin}{PMID: 11701271}, P Curatolo et al., 2001]

\hypertarget{pmid_9095401}{V}igabatrin has been shown to be efficient in infants with infantile spasms and tuberous sclerosis, in open studies. In order to compare vigabatrin to oral steroids, a prospective randomized multicenter study was implemented using both drugs as monotherapy in newly diagnosed patients with infantile spasms and tuberous sclerosis. Eleven infants received vigabatrin (150 mg/kg per day) and 11 hydrocortisone (15 mg/kg per day) for 1 month. Spasm free patients continued vigabatrin or progressively stopped hydrocortisone in 1 month, non-responders were crossed to the other drug for a new 2 month-period. All vigabatrin patients (11/11) were spasm-free versus 5/11 hydrocortisone infants (P < 0.01). Seven patients were crossed to vigabatrin (six for inefficacy, one for adverse events) and became also totally controlled. Mean time to disappearance of infantile spasms was 3.5 days on vigabatrin versus 13 days on hydrocortisone (P < 0.01). Five patients exhibited side effects on vigabatrin but nine on hydrocortisone (P = 0.006). Vigabatrin should therefore be considered as the first choice treatment for infantile spasms due to tuberous sclerosis. [\hyperlink{Vigabatrin}{PMID: 9095401}, C Chiron et al., 1997]

\hypertarget{pmid_19215279}{T}he use of vigabatrin (VGB) as an antiepileptic drug (AED) has been limited by evidence showing that it causes vigabatrin-attributed visual field loss (VAVFL) in at least 20-40\% of patients exposed at school age or later. VGB is an effective drug for infantile spasms, but there are no reports on later visual field testing after such treatment. Our aim was to investigate the risk of VAVFL in school-age children who had received VGB in infancy. Visual fields of 16 children treated with VGB for infantile spasms were examined by Goldmann kinetic perimetry at age 6-12 years. Normal fields were defined as the temporal meridian extending to more than 70 degrees , and mild VAVFL between 50 and 70 degrees . Abnormal findings were always confirmed by repeating the test. Exposure data were collected from hospital charts. Vigabatrin was started at a mean age of 7.6 (range, 3.2-20.3) months. The mean duration of therapy was 21.0 (9.3-29.8) months and cumulative dose 655 g (209-1,109 g). Eight children were never treated with other AEDs, five received only adrenocorticotropic hormone (ACTH) in addition to VGB, and three children had been treated with other AEDs. Fifteen children had normal visual fields. Mild VAVFL was observed in one child (6\%) who had been treated with VGB for 19 months and who received a cumulative dose of 572 g. The risk of VAVFL may be lower in children who are treated with VGB in infancy compared to patients who receive VGB at a later age. [\hyperlink{Vigabatrin}{PMID: 19215279}, Eija Gaily et al., 2009]

\hypertarget{pmid_11673582}{I}nfantile spasms are a rare but devastating pediatric epilepsy that, outside the United States, is often treated with vigabatrin. The authors evaluated the efficacy and safety of vigabatrin in children with recent-onset infantile spasms. This 2-week, randomized, single-masked, multicenter study with a 3- year, open-label, dose-ranging follow-up study included patients who were younger than 2 years of age, had a diagnosed duration of infantile spasms of no more than 3 months, and had not previously been treated with adrenocorticotropic hormone, prednisone, or valproic acid. Patients were randomly assigned to receive low-dose (18-36 mg/kg/day) or high-dose (100-148 mg/kg/day) vigabatrin. Treatment responders were those who were free of infantile spasm for 7 consecutive days beginning within the first 14 days of vigabatrin therapy. Time to response to therapy was evaluated during the first 3 months, and safety was evaluated for the entire study period. Overall, 32 of 142 patients who were able to be evaluated for efficacy were treatment responders (8/75 receiving low-dose vigabatrin vs 24/67 receiving high doses, p < 0.001). Response increased dramatically after approximately 2 weeks of vigabatrin therapy and continued to increase over the 3-month follow-up period. Time to response was shorter in those receiving high-dose versus low-dose vigabatrin (p = 0.04) and in those with tuberous sclerosis versus other etiologies (p < 0.001). Vigabatrin was well tolerated and safe; only nine patients discontinued therapy because of adverse events. These results confirm previous reports of the efficacy and safety of vigabatrin in patients with infantile spasms, particularly among those with spasms secondary to tuberous sclerosis. [\hyperlink{Vigabatrin}{PMID: 11673582}, R D Elterman et al., 2001]

\hypertarget{pmid_22061182}{V}igabatrin is an effective antiepileptic drug (AED) for the treatment of refractory complex partial seizures (rCPS) and infantile spasms (IS). In clinical trials, vigabatrin was generally well-tolerated with an adverse event profile similar to that of other AEDs. The most common treatment-related adverse events were central nervous system effects, including drowsiness, dizziness, headache, and fatigue, with adjunctive vigabatrin in adults with rCPS, and sedation, somnolence, and irritability with vigabatrin monotherapy in infants with IS. Vigabatrin had little effect on cognitive function, mood, or behavior in a battery of neuropsychologic tests for rCPS. In placebo-controlled clinical trials, the incidence of depression and psychosis, but not other psychiatric adverse events, was greater with vigabatrin than placebo. Intramyelinic edema (IME) was initially identified in rats and dogs and led to a temporary suspension of clinical trials in the United States. IME was subsequently correlated with delays in evoked potential (EP) and increased T(2) -weighted signals on magnetic resonance imaging (MRI). Clinical trials of vigabatrin were allowed to resume after IME was not detected by neuropathologic assessments of autopsy and neurosurgical specimens or by serial EP or MRI assessments in older children and adults receiving vigabatrin. Subsequently, MRI abnormalities characterized by increased T(2) intensity and restricted diffusion were identified in infants treated with vigabatrin for IS. These abnormalities generally resolved with discontinuation of vigabatrin and, in some cases, during continued therapy. The benefit of improved seizure control must be balanced against the potential risks associated with vigabatrin, including abnormal MRI changes and other vigabatrin-related safety issues. [\hyperlink{Vigabatrin}{PMID: 22061182}, S D Walker et al., 2011]

\hypertarget{pmid_11944216}{V}igabatrin, one of the newer anti-epileptic drugs (AED), whose effect is mediated via elevated levels of brain GABA, has proved to be effective in drug resistant partial seizures and infantile spasms. Recently, visual field constriction was found in up to 30\% of adults, whereas information for the pediatric age group is sparse. We examined 24 visually-asymptomatic children, ages 3.5-18 years, treated for 3.01.6 years at doses of 25-90 mg/kg. These children underwent an ophthalmologic examination, visual evoked potentials, electroretinogram and when possible, perimetry. Over half of the children had at least one abnormal test and 11/17, who were able to undergo perimetric studies, had symmetrical, nasal visual field constriction. In view of the gravity and prevalence of visual field constriction, use of the AED vigabatrin should be weighed against its clinical benefits. [\hyperlink{Vigabatrin}{PMID: 11944216}, Varda Gross-Tsur et al., 2002]

\hypertarget{pmid_8951215}{V}igabatrin (VGB) is a recently-released antiepileptic drug which works by a clearly-defined mechanism of action: inhibition of GABA transaminase leading to an elevation of brain GABA concentration. It has been proven effective, mainly as an add-on agent, in complex partial and secondarily generalized seizures in both adults and children as well as in infantile spasms in both short and long-term controlled studies. World-wide experience now includes over 150,000 patients exposed to the drug. VGB has a favorable pharmacokinetic profile since it has little protein-binding, is mainly excreted unchanged by the kidney and has a long effective half-life allowing once or twice daily dosing. It is generally well-tolerated with very few cognitive effects but may cause significant behavioral side effects such as agitation, irritability, depression or psychosis in approximately 2-4\% of cases. Mild weight gain and possible exacerbation of absence and myoclonic seizures are other reported adverse effects. The role of VGB in other childhood epileptic syndromes apart from West syndrome is still being defined. [\hyperlink{Vigabatrin}{PMID: 8951215}, A Guberman et al., 1996]

\hypertarget{pmid_30192381}{V}igabatrin is an antiepileptic drug indicated as monotherapy in infantile spasms. However, the pharmacokinetic profile of this compound in infants and young children is still poorly understood, as is the minimal effective dose, critical information given the risk of exposure-related retinal toxicity with vigabatrin. A reasonable approach to determining this minimal dose would be to identify the lowest dose providing a low risk of exposure overlap with the 36-mg/kg dose, which is the highest dose associated with an increased risk for treatment failure, based on randomized dose-ranging data. A population pharmacokinetic model was consequently developed from 28 children (aged 0.4-5.7 years) for the active S(+)-enantiomer, using Monolix software. In parallel, a population model was developed from published adult data and scaled to children using theoretical allometry and maturation of the renal function. A one-compartment model with zero-order absorption and first-order elimination described the pediatric data. Mean population estimates (percentage interindividual variability) for the apparent clearance, apparent distribution volume, and absorption duration were 2.36 L/h (24.5\%), 17 L (38\%), and 0.682 hours, respectively. Apparent clearance and apparent distribution volume were related to body weight by empirical allometric equations. Monte Carlo simulations evidenced that a daily dose of 80 mg/kg should minimize exposure overlap with the 36-mg/kg dose. Similar results were obtained for the adult model scaled to children. Consequently, a minimal effective dose of 80 mg/kg/day could be considered for patients with infantile spasms. [\hyperlink{Vigabatrin}{PMID: 30192381}, Marwa Ounissi et al., 2019]

\hypertarget{pmid_10691111}{V}igabatrin (Sabril, Hoechst Marion Roussel) is an antiepilepsy drug (AED) presently marketed in 64 countries for the treatment of partial and secondarily generalized seizures. Vigabatrin (VGB) is marketed in a subset of these countries for the treatment of infantile spasms. Clinical experience in humans has shown that VGB provides effective seizure control with a wide margin of safety. However, animal toxicity studies raised concern when prolonged administration of VGB was shown to induce intramyelinic edema (IME) in some laboratory animal species. Animal and human data were reviewed with respect to the potential for VGB-induced IME. Surveillance of patients receiving VGB in clinical trials or by prescription has been conducted for >15 years to identify patients developing clinical abnormalities that might be IME related. The histologic lesions of VGB-induced IME in animals are reliably reproduced and correlate with changes in multimodality evoked potentials (EPs) and magnetic resonance imaging (MRI). Numerous studies of the effects of VGB on EP and MRI in epilepsy patients have demonstrated no clear-cut IME-related changes in these modalities. Additionally, autopsy and surgical brain samples from VGB-treated patients have been scrutinized for potential IME histopathology. In an estimated 350,000 patient-years of VGB exposure (approximately 175,000 patients exposed for 2 years at an average dose of 2 g/day), no definite case of VGB-induced IME has been identified. Comprehensive review of a variety of sources of data failed to identify any definite case of IME in humans treated with VGB. [\hyperlink{Vigabatrin}{PMID: 10691111}, J A Cohen et al., 2000]

\hypertarget{pmid_24910743}{A}pproximately one-third of all children with epilepsy do not achieve complete seizure improvement. This study evaluated the efficacy of Vigabatrin in children with intractable epilepsy. From November 2011 to October 2012, 73 children with refractory epilepsy (failure of seizure control with the use of two or more anticonvulsant drugs) who were referred to the Children's Medical Center and Mofid Children's Hospital were included in the study. The patients were treated with Vigabatrin in addition to their previous medication, and followed-up after three to four weeks to determine the daily frequency, severity, and duration of seizures in addition to any reported side effects. Of the 67 children, 41 (61.2\%) were males and 26 (38.8\%) females, their age ranging from three months to 13 years with an average of 3.1 [standard deviation (SD), 2.6] years. The mean daily frequency of seizures at baseline was 6.61 (SD, 5.9) seizures per day. Vigabatrin reduced the seizure frequency ≤2.9 (SD, 5.2) (56\% decline) and 3.0 (SD, 5.3) (54.5\% decline) per day after three and six months of treatment, respectively. A significant difference was observed between seizure frequencies at three (P<0.001) and six months (P<0.001) after Vigabatrin initiation compared with the baseline. Somnolence [3 (4.5\%)], horse laugh [1 (1.5\%)], urinary stones [1 (1.5\%)], increased appetite [1 (1.5\%)], and abnormal electroretinographic pattern [3 (4.5\%)] were the most common side effects in our patients. This study confirms the short-term efficacy and safety of Vigabatrin in children with refractory epilepsies. [\hyperlink{Vigabatrin}{PMID: 24910743}, Mohammad-Mahdi Taghdiri et al., 2013]

\hypertarget{pmid_23748200}{V}igabatrin is an antiepileptic drug that results in higher gamma-aminobutyrate levels in the brain and retina. Vigabatrin-induced visual field defects are usually asymptomatic and only detectable by perimetry. Further, perimetry requires good cooperation, and children aged under 10 years cannot do it. Electroretinogram response amplitude to full-field 30-Hz flicker shine has been offered to be more specific in predicting visual field defects. This study is scheduled to investigate the vigabatrin-associated visual complications in 67 epileptic children taking vigabatrin using full-field electroretinogram. Electroretinographic surveys showed normal range parameters despite 3 months of vigabatrin treatment, and just 3 (4.47\%) children had been visually impaired at the end of 6-month treatment. Among these 3 cases, 1 patient had persistent electroretinogram abnormality despite vigabatrin discontinuation. Our study suggests that vigabatrin is secure for short-term pediatric antiepileptic treatment, with few cases of visual impairments and that are often reversible.  [\hyperlink{Vigabatrin}{PMID: 23748200}, M K Bakhshandeh Bali et al., 2014] Vigabatrin has been studied in adult drug-resistant epilepsy since 1982 in single-blind and double-blind studies followed by long-term, open evaluations. These studies have provided evidence that vigabatrin is a potent and well-tolerated antiepileptic drug and support its potential value in pediatric epilepsy. The lack of any evidence of human neurotoxicity in these patients is also reassuring regarding its use in children. [\hyperlink{Vigabatrin}{PMID: 23748200}, M Dam et al., 1991]

\hypertarget{pmid_10183381}{I}n children with infantile spasms, vigabatrin monotherapy has been assessed in three published comparative trials.  The small numbers of patients make it impossible to draw precise conclusions on effectiveness.  However, a few days' treatment with a dose of about 100 mg/kg/day clears infantile spasms in a larger proportion of cases than a placebo or steroids.  Vigabatrin seems to be more effective in Bourneville disease.  The effect is sometimes transient: despite continued treatment, spasms or other types of epilepsy occur in approximately 50\% of patients who are initially improved.  In a trial versus ACTH, the lesser initial efficacy of vigabatrin was partly offset by a lower incidence of relapse and other types of seizures.  Vigabatrin is effective in some children who are resistant to ACTH or steroids.  As with steroids and ACTH, there is no proof that vigabatrin improves the long-term psychomotor development of these children.  In comparative trials the incidence of adverse events was statistically lower on vigabatrin than on steroids.  Most of the events were relatively mild neuropsychological effects, but a question mark still hangs over the possible neurotoxicity or oculotoxicity of vigabatrin during long-term administration. [\hyperlink{Vigabatrin}{PMID: 10183381}, Vigabatrin: new indication. An advance in infantile spasms., 1998]

\hypertarget{pmid_10565592}{V}igabatrin (VGB) has been shown to be an effective drug in the treatment of infantile spasms (West syndrome) in predominantly retrospective and open but also in prospective studies. This prospective, randomised, and placebo-controlled trial of VGB in infantile spasms was considered to be justified and feasible to confirm or refute these previous findings. Forty children with newly diagnosed infantile spasms received either VGB or placebo for 5 days in a double blind, placebo-controlled, parallel-group study, after which all the infants continuing in the study were treated openly with VGB for a minimum of 24 weeks. Compared with baseline, at the end of the double-blind phase, the patients treated with VGB had a 78\% (95\% confidence interval, 55-89\%) reduction in spasms compared with 26\% (-56-65\%) in the group treated with placebo (p = 0.020). Seven VGB-treated patients and two placebo-treated patients were spasm free on the final day of the double-blind period (p = 0.063). At the end of the study, 15 children (38\% of the original 40 patients or 42\% of the 36 patients who entered the open phase) were spasm free with VGB monotherapy. No patient withdrew from the study because of an adverse event. This unique randomized, placebo-controlled study is the first to demonstrate the efficacy of a specific drug in the treatment of West syndrome and supports the results of previously published open and prospective trials. It further confirms that VGB could be considered as the drug of first choice in treating infantile spasms. [\hyperlink{Vigabatrin}{PMID: 10565592}, R E Appleton et al., 1999]

\hypertarget{pmid_9733404}{A} multicentre, long-term, open-label, add-on study of vigabatrin was undertaken in 23 pretreated children with infantile spasms. After 3 months of vigabatrin therapy 11 of the 23 patients had become seizure-free. At this time two-thirds of these 11 children still received other antiepileptic drugs (AEDs) in addition to vigabatrin (mostly valproic acid and/or dexamethasone). After a mean follow-up time of 5 1/4 years (range: 4 1/4-6 1/2) 72\% of 18 evaluable patients (two children died, three were lost to follow-up) revealed seizure freedom for at least 1 year. The mean duration of vigabatrin therapy had been 2 1/2 years (range: 2 weeks to 4 3/4 years). Two-thirds of the 18 children continued to take AEDs, three of them undergoing vigabatrin monotherapy. Relapses of infantile spasms had occurred in 14\% of the children. The rate of vigabatrin side effects (10\%) was low. At follow-up, the EEG of 13 and the 18 patients demonstrated focal or multifocal epileptic discharges. Fifty-five percent had developed another epilepsy (focal epilepsy, secondary generalized epilepsy or myoclonic-astatic epilepsy). With respect to mental functions, three children were normal or slightly retarded, four showed moderate retardation and 11 revealed severe or very severe retardation. This long-term result is comparable to that in ACTH studies with unselected patients. The conclusions are: (1) vigabatrin is an effective drug for the short-term and long-term treatment of refractory infantile spasms; (2) the relapse rate is low; (3) vigabatrin is well tolerated; (4) with respect to secondary epilepsies and mental functions the long-term outcome in these pretreated children is similar to that in earlier studies with ACTH or corticosteroids. [\hyperlink{Vigabatrin}{PMID: 9733404}, H Siemes et al., 1998]

\hypertarget{pmid_10894216}{V}igabatrin is an anti-epileptic drug particularly useful for drug-resistant partial seizures and infantile spasms. Recently, vigabatrin-induced visual field constriction (VFC) and abnormal ocular electrophysiological studies were reported. In this study, we assessed visual fields, visual evoked potentials (VEPs), and electroretinography (ERG) in children treated with vigabatrin. Twenty-four visually asymptomatic children underwent a clinical ophthalmological examination, perimetry when appropriate, and VEP and ERG. Thirteen patients had at least one abnormal study. VFC was seen in 11 of 17 patients who had perimetry; 5 of 15 patients who underwent VEP testing and 4 of 11 who underwent ERG testing had abnormal examinations. For the most part, abnormal VEPs and ERGs were found in children who also had VFC. There was a consistent trend for longer treatment periods to correlate with VFC, abnormal ERGs, and VEPs. In summary, over half of the children treated with vigabatrin demonstrated VFC or abnormal ocular electrophysiological studies. Perimetry seemed to be the most sensitive modality for identifying vigabatrin toxicity. Abnormal ERGs and VEPs were primarily seen in children with VFC and may be useful in monitoring children who are not appropriate candidates for perimetry. Although the incidence of vigabatrin-induced VFC is worrisome, in the context of intractable seizures or infantile spasms, therapeutic benefits must be weighed against risks. [\hyperlink{Vigabatrin}{PMID: 10894216}, V Gross-Tsur et al., 2000]

\hypertarget{pmid_9377291}{V}igabatrin (VGB) was specifically synthesized to enhance inhibitory GABAergic transmission by elevating GABA levels via irreversible inhibition of GABA transaminase. This study was conducted to determine the efficacy of VGB introduced as monotherapy in 26 children fulfilled the criteria for diagnosis of West syndrome. Duration of follow-up was 24 months. Following the introduction of VGB, 13 patients became seizure free, a special efficacy in one case of tuberous sclerosis. Relapses of infantile spasms occurred in 8 infants, generalized seizures developed in 4 infants, and partial seizures in 3, of whom 2 were eventually rendered seizure-free by increasing the dose. EEG features and age affecting the response rate. VGB has been well tolerated in children with agitation being the most commonly reported side effects, and one case required discontinuation of therapy. VGB seems to be an effective an safe antiepileptic drug as primary monotherapy for West syndrome. [\hyperlink{Vigabatrin}{PMID: 9377291}, M Rufo et al., 1997]

\hypertarget{pmid_10073425}{T}he purpose of this report is to review the efficacy and safety of vigabatrin in the treatment of infantile spasms in infants suffering from tuberous sclerosis complex. We reviewed all studies published in the English-language literature investigating the use of vigabatrin in the treatment of infantile spasms. Ten studies gave results for the efficacy of vigabatrin in infantile spasms for infants both with and without underlying diagnoses of tuberous sclerosis. Of the 313 patients without tuberous sclerosis complex, 170 (54\%) had complete cessation of their infantile spasms; of the 77 patients with tuberous sclerosis complex, 73 (95\%) had complete cessation of their seizures. We conclude that vigabatrin should be considered as first-line monotherapy for the treatment of infantile spasms in infants with either a confirmed diagnosis of tuberous sclerosis or those at high risk, ie, those with a first-degree relative with tuberous sclerosis complex. Paradoxically, in those without tuberous sclerosis complex, vigabatrin might be less efficacious than suggested by studies including patients with tuberous sclerosis complex. [\hyperlink{Vigabatrin}{PMID: 10073425}, E Hancock et al., 1999]

\hypertarget{pmid_17625936}{V}igabatrin (VGB), a selective irreversible inhibitor of gamma-aminobutyric acid transaminase, has proved to be effective against cryptogenic and symptomatic infantile spasms (IS). Unfortunately, reports of serious visual field defects have led to a drastic reduction in the use of the drug. This review is based on a systematic search in the literature for evidence regarding efficacy and safety of VGB in IS. Based on a specific mechanism of action, there is a solid evidence of clinical efficacy of VGB in children with Tuberous Sclerosis. Similarly, VGB could represent a potential effective therapy also for spasms due to focal cortical dysplasia. In infants with spasms due to other causes, the risk of ophthalmologic toxicity should be carefully weighted against the benefit of controlling spasms. [\hyperlink{Vigabatrin}{PMID: 17625936}, Pasquale Parisi et al., 2007]

\hypertarget{pmid_34541635}{V}igabatrin (VGB), a second-generation antiepileptic drug, is effective for the treatment of infantile spasms and focal seizures, primarily in tuberous sclerosis complex (TSC) patients. However, reports of adverse events of VGB, including VGB-associated visual field loss and brain abnormalities in neuroimaging, have raised concerns about the broader use of VGB and thus significantly limited its application. The goal of this review was to summarise the recent therapeutic guidelines, the use of VGB in focal seizures and new VGB applications as a disease-modifying treatment in TSC patients. Moreover, we discuss the current opinions on potential VGB-associated toxicity and the safety of VGB. [\hyperlink{Vigabatrin}{PMID: 34541635}, Weronika Golec et al., 2021]

\section*{Pantoprazole Sodium}
\subsection*{Result}
\subsubsection*{Answer}

Neonates and Preterm Infants: Yes  
Infants (1–11 months): Yes  
Children (2–16 years): Yes  
Children (5–11 years): Yes  
Children (6–13 years): Yes  
Adolescents (12–16 years): Yes  

\subsubsection*{{Explanation}}
\hypertarget{Pantoprazole Sodium}
Based on the abstracts available, several targeted studies have evaluated the safety of Pantoprazole Sodium in children across different age ranges:

1. **Neonates and Preterm Infants**: A multicenter, randomized, open-label trial assessed the pharmacokinetics and safety of pantoprazole granules in neonates and preterm infants with GERD. The study found that pantoprazole was generally well tolerated, with no safety-related discontinuations reported [\hyperlink{pmid_20306184}{PMID: 20306184}, Robert M Ward et al., 2010].

2. **Infants (1–11 months)**: A randomized, double-blind, placebo-controlled study evaluated pantoprazole in infants with GERD. The study reported that pantoprazole significantly improved GERD symptom scores and was well tolerated. No between-group differences in adverse event frequency were noted, and serious adverse events were considered unrelated to treatment [\hyperlink{pmid_20400912}{PMID: 20400912}, Harland Winter et al., 2010].

3. **Children (2–16 years)**: A pharmacokinetic and safety study of single oral and intravenous doses of pantoprazole in children aged 2 to 16 years found that the doses used were safe and well tolerated, with all adverse events being mild and unrelated to the study drug [\hyperlink{pmid_18664620}{PMID: 18664620}, Gregory L Kearns et al., 2008].

4. **Children (6–13 years)**: A study of 15 children aged 6 to 13 years with reflux esophagitis treated with pantoprazole 20 mg daily for 28 days found that pantoprazole was safe and well tolerated, with only one patient experiencing a transient elevation of liver enzymes [\hyperlink{pmid_12548064}{PMID: 12548064}, Armando Madrazo-de la Garza et al., 2003].

5. **Children (5–11 years)**: A study of 53 children aged 5–11 years with endoscopically proven GERD treated with pantoprazole (10, 20, and 40 mg) for 8 weeks found that adverse events were similar among treatment groups, indicating safety [\hyperlink{pmid_16641576}{PMID: 16641576}, Vasundhara Tolia et al., 2006].

6. **Children (6–11 and 12–16 years)**: Two studies evaluated the pharmacokinetics and safety of pantoprazole delayed-release tablets in children aged 6–11 and 12–16 years with GERD. No serious drug-associated adverse events were reported, and pantoprazole was considered safe in these age groups [\hyperlink{pmid_20852004}{PMID: 20852004}, Robert M Ward et al., 2011; \hyperlink{pmid_16968960}{PMID: 16968960}, V Marc Tsou et al., 2006].

7. **Adolescents (12–16 years)**: A study using an age-appropriate questionnaire found that pantoprazole was safe, well tolerated, and effective in reducing GERD symptoms in adolescents [\hyperlink{pmid_16968960}{PMID: 16968960}, V Marc Tsou et al., 2006].

In summary, targeted studies have been conducted in neonates, infants, children, and adolescents, and these studies affirm the safety of Pantoprazole Sodium in these pediatric age groups.

\subsection*{Abstracts}
\hypertarget{pmid_18664620}{T}he primary objective was to determine the pharmacokinetics of single oral and intravenous doses of pantoprazole in children 2 to 16 years of age. The secondary objective was to assess the safety and tolerability of these doses. Male and female hospitalized and nonhospitalized patients from ages 5 to 16 years received single oral doses (20 mg or 40 mg), and those from ages 2 to 16 years received single intravenous doses (0.8 mg/kg or 1.6 mg/kg) of pantoprazole. The plasma concentration-time data for each patient were analyzed using noncompartmental methods. Routine safety and tolerability assessments were also obtained. The mean values for peak plasma concentration and total area under the plasma concentration-time curve increased with increasing dose. Pharmacokinetic values were similar in patients from ages 2 to 16 years and to those previously obtained in adults. Statistically significant differences were observed for dose-normalized pantoprazole area under the plasma concentration-time curve when compared between CYP2C19 extensive metabolizers with 1 versus 2 functional alleles. All adverse events were mild in severity and considered to be unrelated to study drug. The pharmacokinetic profile of oral and intravenous pantoprazole was similar in children ages 2 to 16 years. The doses used here were safe and well tolerated in this population. [\hyperlink{Pantoprazole Sodium}{PMID: 18664620}, Gregory L Kearns et al., 2008]

\hypertarget{pmid_24138461}{T}he aim of this study was to determine the safety and the efficacy of paediatrician-administered propofol in children undergoing different painful procedures. We conducted a retrospective study over a 12-year period in three Italian hospitals. A specific training protocol was developed in each institution to train paediatricians administering propofol for painful procedures. In this study, 36,516 procedural sedations were performed. Deep sedation was achieved in all patients. None of the children experienced severe side effects or prolonged hospitalisation. There were six calls to the emergency team (0.02\%): three for prolonged laryngospasm, one for bleeding, one for intestinal perforation and one during lumbar puncture. Nineteen patients (0.05\%) developed hypotension requiring saline solution administration, 128 children (0.4\%) needed O2 ventilation by face mask, mainly during upper endoscopy, 78 (0.2\%) patients experienced laryngospasm, and 15 (0.04\%) had bronchospasm. There were no differences in the incidence of major complications among the three hospitals, while minor complications were higher in children undergoing gastroscopy. This multicentre study demonstrates the safety and the efficacy of paediatrician-administered propofol for procedural sedation in children and highlights the importance of appropriate training for paediatricians to increase the safety of this procedure in children. [\hyperlink{Pantoprazole Sodium}{PMID: 24138461}, Antonio Chiaretti et al., 2014]

\hypertarget{pmid_20400912}{T}he objective of this study was to assess the efficacy of pantoprazole in infants with gastroesophageal reflux disease (GERD). Infants ages 1 through 11 months with GERD symptoms after 2 weeks of conservative treatment received open-label (OL) pantoprazole 1.2 mg x kg(-1) x day(-1) for 4 weeks followed by a 4-week randomized, double-blind (DB), placebo-controlled, withdrawal phase. The primary endpoint was withdrawal due to lack of efficacy in the DB phase. Mean weekly GERD symptom scores (WGSSs) were calculated from daily assessments of 5 GERD symptoms. Safety was assessed. One hundred twenty-eight patients entered OL treatment, and 106 made up the DB modified intent-to-treat population. Mean age was 5.1 months (82\% full-term, 64\% male). One third of patients had a GERD diagnostic test before OL study entry. WGSSs at week 4 were similar between groups. WGSSs decreased significantly from baseline during OL therapy (P < 0.001), when all patients received pantoprazole. The decrease in WGSSs was maintained during the DB phase in both treatment groups. There was no difference in withdrawal rates due to lack of efficacy (pantoprazole 6/52; placebo 6/54) or time to withdrawal during the DB phase. The greatest between-group difference in WGSS was slightly worse with placebo at week 5 (P = 0.09), mainly due to episodes of arching back (P = 0.028). No between-group differences in adverse event frequency were noted. Serious adverse events in 8 patients were considered unrelated to treatment. Pantoprazole significantly improved GERD symptom scores and was well tolerated. However, during the DB treatment phase, there were no significant differences noted between pantoprazole and placebo in withdrawal rates due to lack of efficacy. [\hyperlink{Pantoprazole Sodium}{PMID: 20400912}, Harland Winter et al., 2010]

\hypertarget{pmid_12973370}{P}antoprazole sodium is a substituted benzimidazole derivative which controls acid secretion by inhibition of gastric H(+)/K(+)-ATPase. The prodrug pantoprazole accumulates in the acidic space of the parietal cell where it is converted to the pharmacologically active principle, a thiophilic cyclic sulfenamide. The pH-dependent activation profile, i.e., activation at pH 1 versus activation at pH 4-6, is more favorable for pantoprazole than for the other proton pump inhibitors (PPIs) currently available. In vitro, pantoprazole interferes less potently than omeprazole with biological targets not related to gastric acid secretion. The gastric target sites for the pantoprazole sulfenamide are the cysteines 813 and 822 of the catalytic subunit of the H(+)/K(+)-ATPase. In contrast to omeprazole, the two binding sites are located right at the proton channel. In rats, dogs and humans, pantoprazole produces marked and prolonged inhibition of both basal and stimulated acid secretion. Overall, its antisecretory potency is equal to that of omeprazole. Antiulcer activity has been demonstrated for pantoprazole in two rat models. As seen with H(2)-receptor antagonists and other PPIs, pantoprazole causes an increase in serum gastrin concentration which reflects the degree of gastric acid inhibition. Pantoprazole is mainly metabolized by CYP3A4 and 2C19, but displays a lower affinity for these phase I cytochrome P450 enzymes than omeprazole. In contrast to the latter, pantoprazole is further conjugated with sulfate by the hepatic phase II metabolism. These two differences may explain why pantoprazole does not interfere with the metabolism of any other drug thus far tested in humans. [\hyperlink{Pantoprazole Sodium}{PMID: 12973370}, W Beil et al., 1999]

\hypertarget{pmid_36986577}{P}antoprazole is a model substance that requires dosage form adjustments to meet the needs of all patients. Pediatric pantoprazole formulations in Serbia are mostly compounded as capsules (divided powders), while in Western Europe liquid formulations are more common. The aim of this work was to examine and compare the characteristics of compounded liquid and solid dosage forms of pantoprazole. Three syrup bases were used: a sugar-free vehicle for oral solution (according to USP43-NF38), a vehicle with glucose and hydroxypropyl cellulose (according to the DAC/NRF2018) and a commercially available SyrSpend Alka base. Lactose monohydrate, microcrystalline cellulose and a commercially available capsule filler (excipient II, composition: pregelatinized corn starch, magnesium stearate, micronized silicon dioxide, micronized talc) were used as diluents in the capsule formulations. Pantoprazole concentration was determined by the usage of the HPLC method. Pharmaceutical technological procedures and microbiological stability measurements were performed according to the recommendations of the EP10. Although dose appropriate compounding with pantoprazole is suitable using both liquid vehicles as well as solid formulations, chemical stability is enhanced in solid formulation. Nevertheless, according to our results, if liquid formulation is a pH adjusted syrup, it could be safely kept in a refrigerator for up to 4 weeks. Additionally, liquid formulations could be readily applied, while solid formulation should be mixed with appropriate vehicles with higher pH values. [\hyperlink{Pantoprazole Sodium}{PMID: 36986577}, Nemanja Todorović et al., 2023]

\hypertarget{pmid_15960715}{T}his prospective, clinical trial evaluated the effects of short-term propofol administration on triglyceride levels and serum pancreatic enzymes in children undergoing sedation for magnetic resonance imaging. Laboratory parameters of 40 children, mean age (SD; range) 67 (66; 4-178) months undergoing short-term sedation were assessed before and 4 h after having received propofol. Mean (SD) propofol loading dose was 2.2 (1.1) mg.kg(-1) followed by continuous propofol infusion of 6.9 (0.9) mg.kg(-1).h(-1). Serum lipase levels (p = 0.035) and serum triglyceride levels (p = 0.003) were raised significantly after propofol administration but remained within normal limits. No significant changes in serum pancreatic-amylase levels were seen (p = 0.127). In two (5\%) children, pancreatic enzymes and in four (10\%) children triglyceride levels were raised above normal limits; however, no child showed clinical symptoms of pancreatitis. We conclude that even short-term propofol administration with standard doses of propofol may have a significant effect on serum triglyceride and pancreatic enzyme levels in children. [\hyperlink{Pantoprazole Sodium}{PMID: 15960715}, S Gottschling et al., 2005]

\hypertarget{pmid_31297294}{P}antoprazole sodium, a substituted benzimidazole derivative, is an irreversible proton pump inhibitor which is primarily used for the treatment of duodenal ulcers, gastric ulcers, and gastroesophageal reflux disease (GERD). The monographs of European Pharmacopoeia (Ph. Eur.) and United States Pharmacopoeia (USP) specify six impurities,  [\hyperlink{Pantoprazole Sodium}{PMID: 31297294}, Arun Kumar Awasthi et al., 2019] The objective of this study was to develop pediatric physiologically based pharmacokinetic (PBPK) models for pantoprazole and esomeprazole. Pediatric PBPK models were developed by Simcyp version 15 by incorporating cytochrome P450 (CYP)2C19 maturation and auto-inhibition. The predicted-to-observed pantoprazole clearance (CL) ratio ranged from 0.96-1.35 in children 1-17 years of age and 0.43-0.70 in term infants. The predicted-to-observed esomeprazole CL ratio ranged from 1.08-1.50 for children 6-17 years of age, and 0.15-0.33 for infants. The prediction was markedly improved by assuming no auto-inhibition of esomeprazole in infants in the PBPK model. Our results suggested that the CYP2C19 auto-inhibition model was appropriate for esomeprazole in adults and older children but could not be directly extended to infants. A better understanding of the complex interplay of enzyme maturation, inhibition, and compensatory mechanisms for CYP2C19 is necessary for PBPK modeling in infants. [\hyperlink{Pantoprazole Sodium}{PMID: 31297294}, Peng Duan et al., 2019]

\hypertarget{pmid_26858095}{S}edation is increasingly used to facilitate procedures on children in emergency departments (EDs). This overview of systematic reviews (SRs) examines the safety and efficacy of sedative agents commonly used for procedural sedation in children in the ED or similar settings. We followed standard SR methods: comprehensive search; dual study selection, quality assessment, data extraction. We included SRs of children (1 month to 18 years) where the indication for sedation was procedure-related and performed in the ED. Fourteen SRs were included (210 primary studies). The most data were available for propofol (six reviews/50,472 sedations) followed by ketamine (7/8,238), nitrous oxide (5/8,220), and midazolam (4/4,978). Inconsistent conclusions for propofol were reported across six reviews. Half concluded that propofol was sufficiently safe; three reviews noted a higher occurrence of adverse events, particularly respiratory depression (upper estimate 1.1\%; 5.4\% for hypotension requiring intervention). Efficacy of propofol was considered in four reviews and found adequate in three. Five reviews found ketamine to be efficacious and seven reviews showed it to be safe. All five reviews of nitrous oxide concluded it is safe (0.1\% incidence of respiratory events); most found it effective in cooperative children. Four reviews of midazolam made varying recommendations. To be effective, midazolam should be combined with another agent that increases the risk of adverse events (upper estimate 9.1\% for desaturation, 0.1\% for hypotension requiring intervention). This comprehensive examination of an extensive body of literature shows consistent safety and efficacy for nitrous oxide and ketamine, with very rare significant adverse events for propofol. There was considerable heterogeneity in outcomes and reporting across studies and previous reviews. Standardized outcome sets and reporting should be encouraged to facilitate evidence-based recommendations for care. [\hyperlink{Pantoprazole Sodium}{PMID: 26858095}, Lisa Hartling et al., 2016]

\hypertarget{pmid_31156811}{S}otalol hydrochloride (SOT) is an antiarrhythmic β-blocker which is highly effective for the treatment of supraventricular tachycardia in children. However, a licensed paediatric dosage form with sotalol is not currently available in Europe. The aim of this work was to formulate paediatric oral solutions with SOT 5 mg/mL for extemporaneous preparation in a hospital pharmacy with the lowest possible amount of excipients and to determine their stability. Three aqueous solutions were formulated. One preparation without any additives for neonates and two preparations for children from 1 month of age were compounded using citric acid to stabilise the pH value, potassium sorbate 0.1\% w/v as a preservative, and simple syrup or sodium saccharin as a sweetener. The samples were stored at room temperature and in a refrigerator, respectively, and the content of SOT and potassium sorbate was determined simultaneously using a validated high performance liquid chromatography method at different time points over 180 days. At least 95\% of the initial sotalol concentration remained throughout the 180-day study period in all three preparations at both temperatures. The content of potassium sorbate decreased by 17\% with sodium saccharin stored at room temperature. The three proposed oral aqueous solutions of SOT for neonates and infants were stable for 180 days. Storage in a refrigerator is preferred, particularly with sodium saccharin. The additive-free solution of SOT can be autoclaved to ensure microbiological stability and used particularly for neonates and in emergency situations. [\hyperlink{Pantoprazole Sodium}{PMID: 31156811}, Sylva Klovrzová et al., 2016]

\hypertarget{pmid_31110954}{V}arious publications on the use of sedation and anesthesia for diagnostic procedures in children have demonstrated that no ideal agent is available. Although propofol has been widely used for sedation during esophagogastroduodenoscopy in children, adverse events including hypoxia and hypotension, are concerns in propofol-based sedation. Propofol is used in combination with other sedatives in order to reduce potential complications. We aimed to analyze whether the administration of midazolam would improve the safety and efficacy of propofol-based sedation in diagnostic esophagogastroduodenoscopies in children. We retrospectively reviewed the hospital records of children who underwent diagnostic esophagogastroduodenoscopies during a 30-month period. Demographic characteristics, vital signs, medication dosages, induction times, sedation times, recovery times, and any complications observed, were examined. Baseline characteristics did not differ between the midazolam-propofol and propofol alone groups. No differences were observed between the two groups in terms of induction times, sedation times, recovery times, or the proportion of satisfactory endoscopist responses. No major procedural complications, such as cardiac arrest, apnea, or laryngospasm, occurred in any case. However, minor complications developed in 22 patients (10.7\%), 17 (16.2\%) in the midazolam-propofol group and five (5.0\%) in the propofol alone group ( The sedation protocol with propofol was safe and efficient. The administration of midazolam provided no additional benefit in propofol-based sedation. [\hyperlink{Pantoprazole Sodium}{PMID: 31110954}, Ulas Emre Akbulut et al., 2019]

\hypertarget{pmid_20852004}{C}hildren with gastroesophageal reflux disease (GERD) may benefit from gastric acid suppression with proton pump inhibitors such as pantoprazole. Effective treatment with pantoprazole requires correct dosing and understanding of the drug's kinetic profile in children. The aim of these studies was to characterize the pharmacokinetic (PK) profile of single and multiple doses of pantoprazole delayed-release tablets in pediatric patients with GERD aged 6 to 11 years (study 1) and 12 to 16 years (study 2). Patients were randomly assigned to receive pantoprazole 20 or 40 mg once daily. Plasma pantoprazole concentrations were obtained at intervals through 12 hours after the single dose and at 2 and 4 hours after multiple doses for PK evaluation. PK parameters were derived by standard noncompartmental methods and examined as a function of both drug dose and patient age. Safety was also monitored. Pantoprazole PK was dose independent (when dose normalized) and similar to PK reported from adult studies. There was no evidence of accumulation with multiple dosing or reports of serious drug-associated adverse events. In children aged 6 to 16 years with GERD, currently available pantoprazole delayed-release tablets can be used to provide systemic exposure similar to that in adults. [\hyperlink{Pantoprazole Sodium}{PMID: 20852004}, Robert M Ward et al., 2011]

\hypertarget{pmid_21767419}{P}ropofol is the sedative of choice in our hospital for all procedural sedations in children older than 3 months. Data were collected from all patients who underwent PSA with propofol in the period from November 2007 to December 2009. The procedure was performed by a paediatrician experienced in airway management, sedation and paediatric IC, and a specialized nurse. Patient characteristics, American Society of Anesthesiologists (ASA) classification, vital parameters and propofol dosage were registered on specially designed forms. Patient data were analyzed and compared with data from a non-matched historical cohort of patients who in the past had undergone PSA with chloral hydrate. 204 procedural sedations with intravenous propofol were performed in 196 patients. The mean cumulative induction dose was 3.39 mg/kg (SD: 1.34) and the mean maintenance dose was 4.05 mg/kg/h (SD: 2.23). The success rate was 99.5\%, compared to 88.6\% in the cohort that had received PSA with chloral hydrate. 1 procedure was aborted because of desaturation due to an obstructed airway, for which a jaw thrust was performed. No complications were observed in 199 procedures (97.5\%). In 4 procedures a mild and transient desaturation (85-89\%) occurred. The results suggest that propofol can be used safely and is effective for procedural sedation in selected children, provided that PSA is performed by experienced and trained staff. [\hyperlink{Pantoprazole Sodium}{PMID: 21767419}, Christine J P Bruijnen et al., 2011]

\hypertarget{pmid_20306184}{T}he pharmacokinetic profile of pantoprazole granules was assessed in neonates and preterm infants with gastroesophageal reflux disease (GERD) in a multicenter, randomized, open-label trial. Patients were randomly assigned to either the pantoprazole 1.25 mg (approx. 0.6 mg/kg) or 2.5 mg (approx. 1.2-mg/kg) group and treated for > or =5 consecutive days. Blood was sampled either at 0, 2, 8, and 18 h postdose or at 0, 1, 4, and 12 h postdose on day 1 and at 3 and 6 h postdose after > or =5 consecutive doses. Cytochrome P450 2C19 (CYP2C19) and CYP3A4 genotypes were determined. Safety was monitored. Population pharmacokinetics (popPK) analyses were conducted using nonlinear mixed-effects modeling. The popPK modeling of the pantoprazole 1.25 mg and 2.5 mg groups obtained mean (+/-standard deviation) estimates for the area under the plasma concentration versus time curve (AUC) of 3.54 (+/-2.82) and 7.27 (+/-5.30) microg h/mL, respectively, and mean estimates for half-life of 3.1 (+/-1.5) and 2.7 (+/-1.1) h, respectively. Pantoprazole did not accumulate following multiple-dose administration. The two patients with the CYP2C19 poor metabolizer genotype had a substantially higher AUC than extensive metabolizers. No safety-related discontinuations occurred. In preterm infants and neonates, pantoprazole granules were generally well tolerated, mean exposures with pantoprazole 2.5 mg were slightly higher than that in adults who received 40 mg. While the half-life was longer, accumulation did not occur. [\hyperlink{Pantoprazole Sodium}{PMID: 20306184}, Robert M Ward et al., 2010]

\hypertarget{pmid_12548064}{T}o investigate the efficacy and safety of oral pantoprazole, 20 mg (0.5 to 1.0 mg/kg/day) once daily for 28 days, in pediatric patients with reflux esophagitis. Patients in this study (n = 15; 6 to 13 years old, 9 boys) had reflux esophagitis grade Ic or II (Vandenplas classification). The efficacy of pantoprazole to reduce esophageal acid exposure time (pH < 4), reduce the number and duration of reflux episodes, and to increase the percentage of time with gastric pH > 3 was assessed by continuous 24-hour pH monitoring. The intensity of 5 common symptoms of esophagitis was scored before and after treatment on a 4-point scale. Esophagitis was assessed at baseline and after treatment by visual inspection and by the histology of biopsies from the distal third of the esophagus. Before treatment, the median percentage of time with intra-esophageal pH <4 was 9.3\%. After 28 days of therapy with pantoprazole, this value decreased to 2.7\% (P = 0.0006). The median percentage of time with intragastric pH > 3 increased from 21\% at baseline to 39\% on day 28 of therapy (P = 0.005). After 28 days of treatment, all patients experienced at least partial relief from reflux symptoms. Endoscopically confirmed healing of esophagitis was seen in 47\% of children (Savary-Miller classification). Histologic evidence of healing was not observed. Median serum gastrin levels were slightly elevated over baseline levels (from 74 pg/ml to 93 pg/ml). In one patient there was a transient elevation of serum GOT and GPT during treatment. Oral pantoprazole 20 mg daily provided gastric acid control in 15 pediatric patients with reflux esophagitis with partial clinical improvement of symptoms after 28 days of treatment. Pantoprazole was safe and well tolerated. [\hyperlink{Pantoprazole Sodium}{PMID: 12548064}, Armando Madrazo-de la Garza et al., 2003]

\hypertarget{pmid_11240876}{T}o document the safety and efficacy of an anaesthetic technique in paediatric patients undergoing transoesophageal echocardiography (TOE). Prospective descriptive study performed in a children's hospital with all patients undergoing TOE. Topical analgesia of the pharynx was achieved with lidocaine. Anaesthesia was induced with midazolam (25 microg.kg-1), fentanyl (1 microg.kg-1), and propofol (0.5-1 mg.kg-1), followed by a continuous infusion of propofol (5-10 mg.kg-1.h-1). Thirty patients are reported. The mean age was 11.4 +/- 5.1 years (range 1-22) and weight 40.5 +/- 22.1 kg (range 10-110). All the patients tolerated the procedure well. Two patients experienced brief oxygen desaturations during induction, 10 patients coughed during the procedure, and six patients had significant muscle activity requiring supplemental doses of propofol. None of the patients experienced nausea or vomiting. We conclude that our anaesthetic technique in spontaneously breathing paediatric patients during TOE is effective and appears to be safe in children with heart disease. [\hyperlink{Pantoprazole Sodium}{PMID: 11240876}, C M Heard et al., 2001]

\hypertarget{pmid_32682946}{P}osaconazole is approved for use in adults as an intravenous (IV) solution and two different oral formulations (a suspension and an improved bioavailability tablet). Data on the pharmacokinetics (PK), dosing and safety of posaconazole in children are limited. A novel powder for oral suspension (PFS) offers the bioavailability of the tablet formulated for weight-based dosing in children. A non-randomised, open-label, sequential dose-escalation, phase 1b trial evaluated the PK and safety of posaconazole IV and PFS in children aged 2 to 17 years with documented or expected neutropenia (ClinicalTrials.gov, NCT02452034; MSD protocol number, MK-5592-P097). Participants received posaconazole IV 3.5, 4.5 or 6.0 mg/kg/d for ≥10 days, with an option to switch to posaconazole PFS at the identical dose for ≤18 days. The target exposure was a mean within-dose cohort average steady-state plasma concentration (C [\hyperlink{Pantoprazole Sodium}{PMID: 32682946}, Andreas H Groll et al., 2020] To evaluate symptom improvement in 53 children (aged 5-11 years) with endoscopically proven gastroesophageal reflux disease (GERD) treated with pantoprazole (10, 20 and 40 mg) using the GERD Assessment of Symptoms in Pediatrics Questionnaire. The GERD Assessment of Symptoms in Pediatrics Questionnaire was used to measure the frequency and severity over the previous 7 days of abdominal/belly pain, chest pain/heartburn, difficulty swallowing, nausea, vomiting/regurgitation, burping/belching, choking when eating and pain after eating. Individual symptom scores were based on the product of the frequency and usual severity of each symptom. The sum of the individual symptom score values made up the composite symptom score (CSS). The primary end point was the change in the mean CSS from baseline to week 8. Mean frequency and severity of each symptom significantly decreased (from P < 0.006 to P < 0.001) over time. Similar significant decreases in CSS at week 8 versus baseline (P < 0.001) were seen in all groups. Significant decreases from baseline in CSS were noted from weeks 1 to 8 in the 20-mg (P < 0.003) and 40-mg (P < 0.001) groups. The 20- and 40-mg doses were significantly (P < 0.05) more effective than the 10-mg dose in improving GERD symptoms at week 1. Adverse events were similar among the treatment groups. Pantoprazole (20 and 40 mg) is effective in reducing endoscopically proven GERD symptoms in children. Both 20 and 40 mg pantoprazole significantly reduced symptoms as early as 1 week. [\hyperlink{Pantoprazole Sodium}{PMID: 32682946}, Vasundhara Tolia et al., 2006]

\hypertarget{pmid_31001938}{D}espite well-known advantages, propofol remains off-label in many countries for general anesthesia in children under 3 years of age due to insufficient evidence regarding its use in this population. This study aimed to evaluate the efficacy and safety of propofol compared with other general anesthetics in children under 3 years of age undergoing surgery through a systematic review and meta-analysis of existing randomized clinical trials. A comprehensive literature search was conducted of MEDLINE, Embase, and the Cochrane Central Register of Controlled Trials to find all randomized clinical trials comparing propofol with another general anesthetic that included children under 3 years of age. The relative risk or arcsine-transformed risk difference for dichotomous outcomes and the weighted or standardized mean difference for continuous outcomes were estimated using a random-effects model. A total of 249 young children from 6 publications were included. The children who received propofol had statistically significantly lower systolic and diastolic blood pressures, but hypotension was not observed in the propofol groups. The heart rate, stroke volume index, and cardiac index were not significantly different between the propofol and control groups. The propofol groups showed slightly shorter recovery times and a lower incidence of emergence agitation than the control groups, while no difference was observed for the incidence of hypotension, desaturation, and apnea. This systematic review and meta-analysis indicates that propofol use for general anesthesia in young healthy children undergoing surgery does not increase complications and that propofol could be at least comparable to other anesthetic agents. [\hyperlink{Pantoprazole Sodium}{PMID: 31001938}, Hyunsook Hong et al., 2019]

\hypertarget{pmid_16968960}{A}n age-appropriate questionnaire (GASP-Q) was used to assess the frequency and severity of the gastroesophageal reflux disease (GERD) symptoms: abdominal/belly pain, chest pain/heartburn, pain after eating, nausea, burping/belching, vomiting/regurgitation, choking when eating, and difficulty swallowing, in adolescents age 12 to 16 years. The primary objective was to compare the mean composite symptom score (CSS) at week 8 with baseline after treatment with 20 or 40 mg of pantoprazole. Statistically significant (p < 0.001) improvement in CSS occurred in both groups. Safety was comparable between the 2 groups. Pantoprazole was safe, well tolerated, and effective in reducing symptoms of GERD in adolescents. [\hyperlink{Pantoprazole Sodium}{PMID: 16968960}, V Marc Tsou et al., 2006]

\hypertarget{pmid_20484619}{T}he population pharmacokinetics of pantoprazole was characterized in pediatric patients from birth to 16 years using NONMEM and evaluated via bootstrap and predictive check. Data were described using a 2-compartment model with a typical parameterized in terms of clearance (CL) (95\% CI) of 1.93 L per hour (1.53, 2.61), given the reference covariates (female, full term, extensive/unknown CYP2C19 metabolizer status, non-African American, 10 kg weight, intravenous or tablet administration). Pantoprazole pharmacokinetic parameters appear to be similar in pediatric patients compared to adults when allometrically scaled. The effect of age on allometrically scaled CL was best described by a sigmoid Emax model with the age effect reaching an asymptote approximately equal to the adult CL by 1 year. CYP2C19 poor metabolizers exhibited reduced CL with the point estimate and 95\% CI more than 70\% lower than the typical value. Simulations from the final model indicated that the 1.2-mg/kg dose provides the best comparison to adults. [\hyperlink{Pantoprazole Sodium}{PMID: 20484619}, W Knebel et al., 2011]

\hypertarget{pmid_26719728}{S}ildenafil is a phosphodiesterase type-5 inhibitor approved for treatment of pulmonary arterial hypertension (PAH) in adults. Data from pediatric trials demonstrate a similar acute safety profile to the adult population but have raised concerns regarding the safety of long-term use in children. Interpretation of these trials remains controversial with major regulatory agencies differing in their recommendations - the US Food and Drug Administration recommends against the use of sildenafil for treatment of PAH in children, while the European Medicines Agency supports its use at "low doses". Here, we review the available pediatric data regarding dosing, acute, and long-term safety and efficacy of sildenafil for the treatment of PAH in children.  [\hyperlink{Pantoprazole Sodium}{PMID: 26719728}, Andrew L Dodgen et al., 2015] Use of propofol in pediatric age group has been marred by reports of its adverse effects like hypertriglyceridemia and acute pancreatitis, although a causal relation has not yet been established. This prospective, clinical trial was carried out to evaluate the effects of short-term propofol administration on serum lipid profile and serum pancreatic enzymes in children of ASA physical status I and II aged between 1 month and 36 months. Anesthesia was induced with Propofol (1\%) in the dose of 3 mg·kg(-1) intravenously and was maintained by propofol infusion (0.5\%) at the rate of 12 mg·kg(-1·) h(-1) for the first 20 min and at 8 mg·kg(-1·) h(-1) thereafter. The mean dose of propofol administered was 12.02 ± 2.75 mg·kg(-1) (fat load of 120.2 ± 27.5 mg·kg(-1) ). Lipid profile, serum amylase, and lipase were measured before induction of anesthesia, at 90 min, 4 h, and finally 24 h after induction. Serum lipase levels (P < 0.05), serum triglyceride levels (P < 0.05), and serum very low-density lipoproteins VLDL levels (P < 0.05) were raised significantly after propofol administration from baseline although remained within normal limits. Serum cholesterol levels and serum low-density lipoproteins LDL levels showed a statistically significant fall over 24 h. No significant changes in serum pancreatic amylase levels were seen (P > 0.05). None of the patients developed any clinical features of pancreatitis in the postoperative period. We conclude that despite a small, transient increase in serum triglycerides and pancreatic enzymes, short-term propofol administration in recommended dosages in children of ASA status I and II aged between 1 month and 36 months does not produce any clinically significant effect on serum lipids and pancreatic enzymes. [\hyperlink{Pantoprazole Sodium}{PMID: 26719728}, Munish Chauhan et al., 2013]

\hypertarget{pmid_11737739}{T}here is limited experience on sotalol use in the management of childhood arrhythmias. This study reviews the results of our experience with oral sotalol for treatment and prevention of tachyarrhythmias in children. The records of 62 patients (27 female, 35 male, mean age: 8.5+/-5.3 years) treated with sotalol for supraventricular or ventricular arrhythmias from 1994 to 1999 at our institution were reviewed. Demographic, clinical, echocardiographic, electrocardiographic (ECG), ambulatory ECG and electrophysiologic variables were collected. Forty-two (63.6\%) patients had re-entrant supraventricular tachycardia, eight patients (12.9\%) had atrial tachycardia, one patient (1.6\%) had junctional ectopic tachycardia, four patients (6.5\%) had ventricular tachycardia, and seven patients (11.3\%) had complex ventricular arrhythmias, as evidenced by surface or ambulatory ECG records; or revealed during the electrophysiological study. The mean sotalol dose was 3.9+/-1.2 mg/kg per day. In 15.5+/-13.9 months of sotalol use 50\% (n=31) had complete relief of symptoms and/or arrhythmia and 29\% (n=18) had partial relief. Sotalol was ineffective in 20\% (n=13). Sotalol was more effective in re-entrant type supraventricular tachycardias (P=0.012). Sotalol was the first choice in 35.5\% of patients. The sotalol therapy was initiated in inpatient settings in 40.3\% (25 patients). Complications due to sotalol were seen in six patients (five patients developed bradycardia/pauses, and one patient had torsades de pointes) for which the sotalol dose was modified. In patients with sick sinus syndrome, a pacemaker was implanted and in another patient sotalol was stopped. Sotalol, being an effective and safe drug particularly in children, is a good therapeutic alternative for the preventive treatment of childhood tachyarrhythmias. [\hyperlink{Pantoprazole Sodium}{PMID: 11737739}, A Celiker et al., 2001]

\hypertarget{pmid_25611962}{T}adalafil is a selective Phosphodiesterase-5 inhibitor that has been reported to have vasodilatory and antiproliferative effects on the pulmonary artery. In this study we evaluated the safety and efficacy of oral tadalafil in children with pulmonary arterial hypertension (PAH). This open label study, prospective and interventional was carried out in 25 known patients aged 2 month-5 years in 3 medical centers in Iran, between March 2013-Jun 2014. Tadalafil suspension was administrated at 1 mg/kg daily for all patients. Hemodynamic and safety parameters were assessed at baseline and then monthly for a total of 4 visits. 19 patients received tadalafil as initial therapy, in all visits significant improvements in mean pulmonary arterial pressure were observed (p<0.01). Of the 25 patients, 6 (24\%) had been on sildenafil for longer than 6 months. After transition from sildenafil to tadalafil clinical improvement was noted (p<0.05). Administration of tadalafil suspension was generally safe and well tolerated. Nausea was the most frequently reported adverse events which occurred in 3 patients during treatment. Oral tadalafil was administered easily and tolerated well and improved mean pulmonary artery pressure (MPAP) in children with PAH, which suggests that oral tadalafil may be more effective and safer than sildenafil in the treatment of PAH. [\hyperlink{Pantoprazole Sodium}{PMID: 25611962}, A Shiva et al., 2016]

\section*{Colesevelam Hydrochloride}
\subsection*{Result}
\subsubsection*{Answer}

10-17 years (with HeFH): Yes
Under 10 years: Unknown
Other pediatric indications: Unknown

\subsubsection*{{Explanation}}
\hypertarget{Colesevelam Hydrochloride}
To determine if Colesevelam Hydrochloride is safe for use in children, I reviewed all available abstracts for evidence of targeted safety studies in pediatric populations. The following abstracts are relevant:

1. \hyperlink{pmid_24600261}{PMID: 24600261}, Travis Sonnett et al., 2010: This review discusses a 32-week clinical trial of colesevelam in children and adolescents aged 10-17 years with heterozygous familial hypercholesterolemia (HeFH). The trial found significant benefits in lipid lowering and reported that adverse drug reactions were mild in severity. The review concludes that colesevelam has a mild side effect profile in this population, though it notes that further research would be beneficial for long-term effects.

2. \hyperlink{pmid_19879596}{PMID: 19879596}, Evan A Stein et al., 2010: This is a randomized, double-blind study of 194 children aged 10-17 years with HeFH. The study evaluated both efficacy and safety. Adverse events were evaluated, and the results indicate that colesevelam significantly lowered LDL cholesterol levels in this age group. The abstract does not report any significant safety concerns.

3. \hyperlink{pmid_20218749}{PMID: 20218749}, Caroline M Perry et al., 2010: This review summarizes a 32-week trial in pediatric patients aged 10-17 years with HeFH. Colesevelam was generally well tolerated for up to 26 weeks, with no major safety issues reported.

4. \hyperlink{pmid_23916045}{PMID: 23916045}, Michael Davidson et al., 2013: This review includes a summary of a multicenter study in pediatric patients with HeFH, showing that colesevelam was generally well tolerated, with gastrointestinal disorders being the most common adverse events, rarely leading to withdrawal.

Based on these abstracts, there is direct evidence from targeted clinical trials that colesevelam hydrochloride is safe for use in children aged 10-17 years with heterozygous familial hypercholesterolemia, with adverse events being generally mild and tolerability comparable to placebo. There is no evidence in the abstracts for safety in children younger than 10 years, or in pediatric populations without HeFH.

For children under 10 years, or for indications other than HeFH, the safety of colesevelam hydrochloride is unknown based on the available abstracts.

\subsection*{Abstracts}
\hypertarget{pmid_24600261}{C}olesevelam hydrochloride is a synthetic, nonsystemically absorbed polymer that functions as a bile acid sequestrant for the treatment of hypercholesterolemia. Recently, colesevelam was investigated for the treatment of heterozygous familial hypercholesterolemia (HeFH) in the pediatric/adolescent population aged 10-17 years. The purpose of this article is to review the disease state of HeFH in children and adolescents, review the pharmacologic mechanism of action, kinetics, and safety profile of colesevelam, analyze the results of a recent clinical trial of colesevelam in the pediatric/adolescent HeFH population, and discuss the role of colesevelam as a viable treatment option for HeFH. A literature search using Medline (1966-03 May 2010), PubMed (1950-03 May 2010), Science Direct (1994-03 May 2010), and International Pharmaceutical Abstracts (2004-2010) was performed using the search term colesevelam. English language, original research, and review articles were examined, and citations from these articles were also assessed. The manufacturer's prescribing information and the Food and Drug Administration review of the new drug application for the powder formulation were also examined. A 32-week trial was performed investigating the efficacy of colesevelam as monotherapy or combination therapy with a stable statin regimen. Upon completion of the trial, significant benefits were found in regard to the treatment of HeFH and the lowering of low-density lipoprotein cholesterol, total cholesterol, and other secondary measures. Safety and tolerability were also examined throughout the duration of the clinical trial, with adverse drug reactions considered mild in severity. Colesevelam has been shown to reduce low-density lipoprotein cholesterol levels significantly in pediatric/adolescent patients with HeFH, while maintaining a mild side effect profile. Although further research would be beneficial for long-term effects in this population, colesevelam should be considered when developing a treatment regimen for HeFH in the pediatric/adolescent population. [\hyperlink{Colesevelam Hydrochloride}{PMID: 24600261}, Travis Sonnett et al., 2010]

\hypertarget{pmid_19879596}{E}valuate the efficacy and safety of colesevelam hydrochloride in children with heterozygous familial hypercholesterolemia (heFH). This was a randomized, double-blind, 41-site study in 194 children aged 10 to 17 years (inclusive) with heFH (statin-naïve or on a stable statin regimen). After a 4-week stabilization period (period I), subjects were randomized 1:1:1 to placebo, colesevelam 1.875 g/d, or colesevelam 3.75 g/d for 8 weeks (period II). All then received open-label colesevelam 3.75 g/d for 18 weeks (period III), with follow-up 2 weeks later. The primary endpoint was percent change in low-density lipoprotein (LDL)-cholesterol from baseline to week 8. Secondary endpoints included percent change in other lipoprotein variables, including non-high-density lipoprotein (non-HDL)-cholesterol. Adverse events were also evaluated. At week 8, a significant difference from baseline in LDL-cholesterol was reported with colesevelam 1.875 g/d (-6.3\%; P = .031) and colesevelam 3.75 g/d (-12.5\%; P < .001) compared with placebo. Significant treatment effects were also reported for total cholesterol (-7.4\%), non-HDL-cholesterol (-10.9\%), HDL-cholesterol (+6.1\%), apolipoprotein A-I (+6.9\%), and apolipoprotein B (-8.3\%) and a nonsignificant effect for triglycerides (+5.1\%) with colesevelam 3.75 g/d compared with placebo at week 8. These treatment effects were maintained during period III. Colesevelam significantly lowered LDL-cholesterol levels in children with heFH. [\hyperlink{Colesevelam Hydrochloride}{PMID: 19879596}, Evan A Stein et al., 2010]

\hypertarget{pmid_20218749}{C}olesevelam hydrochloride (colesevelam), a non-absorbed, synthetic, lipid-lowering polymer, is a bile acid sequestrant. Colesevelam binds with high affinity to bile acids within the gastrointestinal tract, thereby inhibiting the reabsorption of bile acids, resulting in decreases in serum low-density lipoprotein cholesterol (LDL-C) levels. Colesevelam is available as tablets and as powder for oral suspension. At dosages of 3.75 g once daily or 1.875 g twice daily, colesevelam is approved in the US for the treatment of pediatric patients aged 10-17 years with heterozygous familial hypercholesterolemia. Colesevelam may be administered as monotherapy or in combination with an HMG-CoA reductase inhibitor (statin). A 32-week trial was conducted and consisted of a stablilization period ( approximately 4 weeks), a randomized period (8 weeks), an open-label period (18 weeks), and a 2-week follow-up period. In the 8-week, randomized, double-blind, placebo-controlled period of the trial, colesevelam (tablets), as monotherapy or with a statin, was an effective treatment for pediatric patients with heterozygous familial hypercholesterolemia. At week 8, recipients of colesevelam 3.75 g/day had significant percentage reductions from baseline in mean LDL-C levels (primary endpoint) compared with placebo recipients. Significant beneficial treatment effects for colesevelam 3.75 g/day versus placebo were also reported for several other lipid/lipoprotein parameters at week 8 of the study. The reported treatment effects on lipid/lipoprotein parameters were maintained over a subsequent 18-week, open-label, noncomparative period, when all patients received colesevelam 3.75 g/day. Colesevelam 3.75 g/day was generally well tolerated for up to 26 weeks by pediatric patients with heterozygous familial hypercholesterolemia. [\hyperlink{Colesevelam Hydrochloride}{PMID: 20218749}, Caroline M Perry et al., 2010]

\hypertarget{pmid_19862667}{T}he bile acid sequestrant, colesevelam hydrochloride, is approved for glycemic control in adults with type 2 diabetes. In three double-masked, placebo-controlled studies, colesevelam hydrochloride 3.75 g/day demonstrated its glycemic-lowering properties when added to existing metformin-, insulin-, or sulfonylurea-based therapy in adults with inadequately controlled type 2 diabetes. This was a 52-week open-label extension study conducted at 63 sites in the United States and one site in Mexico to further evaluate the safety and tolerability of colesevelam hydrochloride in subjects with type 2 diabetes. All subjects who completed the three double-masked, placebo-controlled studies were eligible to enroll in this open-label extension. In total, 509 subjects enrolled and received open-label colesevelam hydrochloride 3.75 g/day for 52 weeks. Safety and tolerability of colesevelam hydrochloride was evaluated by the incidence and severity of adverse events. In total, 360 subjects (70.7\%) completed the extension. Of the safety population, 361 subjects (70.9\%) experienced an adverse event, most (88.1\%) being mild or moderate in severity. Fifty-six adverse events (11.0\%) were drug-related; the most frequent drug-related adverse events were constipation and dyspepsia. Thirty-five subjects (6.9\%) discontinued due to an adverse event. Fifty-four subjects (10.6\%) experienced a serious adverse event; only one was considered drug-related (diverticulitis). Seventeen subjects (3.3\%) experienced hypoglycemia; most episodes were mild or moderate in severity. Glycemic improvements with colesevelam hydrochloride were seen without change in weight over 52 weeks (0.2 kg mean reduction from baseline). Colesevelam hydrochloride was safe and well-tolerated as long-term therapy for patients with type 2 diabetes. [\hyperlink{Colesevelam Hydrochloride}{PMID: 19862667}, A B Goldfine et al., 2010]

\hypertarget{pmid_11583720}{C}olesevelam hydrochloride is a novel, potent, non-absorbed lipid-lowering agent previously shown to reduce low density lipoprotein (LDL) cholesterol. To examine the efficacy and safety of coadministration of colesevelam and atorvastatin, administration of these agents alone or in combination was examined in a double-blind study of 94 hypercholesterolemic men and women (baseline LDL cholesterol > or =160 mg/dl). After 4 weeks on the American Heart Association Step I diet, patients were randomized among five groups: placebo; colesevelam 3.8 g/day; atorvastatin 10 mg/day; coadminstered colesevelam 3.8 g/day plus atorvastatin 10 mg/day; or atorvastatin 80 mg/day. Fasting lipids were measured at screening, baseline and 2 and 4 weeks of treatment. LDL cholesterol decreased by 12-53\% in all active treatment groups (P<0.01). LDL cholesterol reductions with combination therapy (48\%) were statistically superior to colesevelam (12\%) or low-dose atorvastatin (38\%) alone (P<0.01), but similar to those achieved with atorvastatin 80 mg/day (53\%). Total cholesterol decreased 6-39\% in all active treatment groups (P<0.05). High density lipoprotein cholesterol increased significantly for all groups including placebo (P<0.05). Triglycerides decreased in patients taking atorvastatin alone (P<0.05), but were unaffected by colesevelam alone or in combination. The frequency of side effects did not differ among groups. At recommended starting doses of each agent, coadministration of colesevelam and atorvastatin was well tolerated, efficacious and produced additive LDL cholesterol reductions comparable to those observed with the maximum atorvastatin dose. [\hyperlink{Colesevelam Hydrochloride}{PMID: 11583720}, D Hunninghake et al., 2001]

\hypertarget{pmid_12040732}{T}he pharmacology, pharmacodynamics, clinical efficacy, drug interactions, adverse effects, and dosage and administration of colesevelam hydrochloride are reviewed. Colesevelam hydrochloride is a nonabsorbed lipid-lowering agent approved for use alone or in combination with hydroxymethylglutaryl-coenzyme A (HMG-CoA) reductase inhibitors for the reduction of low-density-lipoprotein (LDL) cholesterol in patients with primary hypercholesterolemia. Colesevelam forms nonabsorbable complexes with bile acids in the gastrointestinal (GI) tract, resulting in changes in plasma lipid levels, including total, LDL, and high-density-lipoprotein cholesterol and triglycerides. Colesevelam has been reported to be four to six times as potent as traditional bile acid sequestrants (BASs), perhaps because of its greater binding affinity for glycocholic acid. Unlike cholestyramine and colestipol, colesevelam appears to reduce LDL cholesterol in a dose-dependent manner. In clinical trials, colesevelam demonstrated efficacy either alone or in combination with HMG-CoA reductase inhibitors in the treatment of primary hypercholesterolemia. Combination therapy appeared to be more effective than monotherapy. Although infection, headache, and GI adverse effects have been reported for colesevelam, the rates do not differ significantly from those occurring with placebo. The constipation that typically hinders compliance with traditional BASs is minimal. In one study, the rate of compliance with colesevelam was 93\%. There is little evidence of clinically significant interactions involving colesevelam. The maintenance dosage is three 625-mg tablets twice daily or six tablets once daily, taken with meals. Colesevelam provides an effective alternative to cholestyramine and colestipol while offering the potential for fewer adverse effects and better compliance. Studies are needed to directly compare colesevelam with traditional BASs. [\hyperlink{Colesevelam Hydrochloride}{PMID: 12040732}, Karen L Steinmetz et al., 2002]

\hypertarget{pmid_23805883}{C}olesevelam hydrochloride is used as an adjunct to diet and exercise to reduce elevated low-density lipoprotein (LDL) cholesterol in patients with primary hyperlipidemia as well as to improve glycemic control in patients with type 2 diabetes. This is likely to result in submission of abbreviated new drug applications (ANDA). This study was conducted to compare the efficacy of two tablet products of colesevelam hydrochloride based on the in vitro binding of bile acid sodium salts of glycocholic acid (GC), glycochenodeoxycholic acid (GCDA) and taurodeoxycholic acid (TDCA). Kinetic binding study was carried out with constant initial bile salt concentrations as a function of time. Equilibrium binding studies were conducted under conditions of constant incubation time and varying initial concentrations of bile acid sodium salts. The unbound concentration of bile salts was determined in the samples of these studies. Langmuir equation was utilized to calculate the binding constants k1 and k2. The amount of the three bile salts bound to both the products reached equilibrium at 3 h. The similarity factor (f2) was 99.5 based on the binding profile of total bile salts to the test and reference colesevelam tablets as a function of time. The 90\% confidence interval for the test to reference ratio of k2 values were 96.06-112.07 which is within the acceptance criteria of 80-120\%. It is concluded from the results that the test and reference tablets of colesevelam hydrochloride showed a similar in vitro binding profile and capacity to bile salts. [\hyperlink{Colesevelam Hydrochloride}{PMID: 23805883}, Yellela S R Krishnaiah et al., 2014]

\hypertarget{pmid_28741653}{C}hloral hydrate is commonly used to sedate children for painless procedures. Children may recover more quickly after sedation with dexmedetomidine, which has a shorter half-life. We randomly allocated 196 children to chloral hydrate syrup 50 mg.kg [\hyperlink{Colesevelam Hydrochloride}{PMID: 28741653}, V M Yuen et al., 2017] Colesevelam hydrochloride (HCl) was approved in January 2008 as an adjunct therapy for improving glycemic control in patients with type 2 diabetes mellitus (T2DM). Colesevelam HCl is a bile acid sequestrant that has been shown to significantly improve both glycemic control and the lipid profile in patients with T2DM when added to metformin-, sulfonylurea-, or insulin-based therapy. In addition, colesevelam HCl may be useful for reducing glucose and low-density lipoprotein cholesterol levels in patients with prediabetes (defined as fasting plasma glucose levels of 100-125 mg/dL or 2-hour poststimulation glucose levels of 140-199 mg/dL), who have an increased cardiovascular risk. As colesevelam HCl is a unique agent-with both significant glycemic and lipid benefits-it has the potential to play an important role in the management of T2DM. This article reviews the place of colesevelam HCl in therapy (both for T2DM and prediabetes), the benefits of early, intensive treatment of T2DM, and the importance of safe glycemic control later in the disease process. [\hyperlink{Colesevelam Hydrochloride}{PMID: 28741653}, Yehuda Handelsman et al., 2009]

\hypertarget{pmid_18458145}{H}yperglycemia is a risk factor for microvascular complications and may increase the risk of cardiovascular disease in patients with type 2 diabetes. This study tested the LDL cholesterol-lowering agent colesevelam HCl (colesevelam) as a potential novel treatment for improving glycemic control in patients with type 2 diabetes on sulfonylurea-based therapy. A 26-week, randomized, double-blind, placebo-controlled, parallel-group, multicenter study was carried out between August 2004 and August 2006 to evaluate the efficacy and safety of colesevelam for reducing A1C in adults with type 2 diabetes whose glycemic control was inadequate (A1C 7.5-9.5\%) with existing sulfonylurea monotherapy or sulfonylurea in combination with additional oral antidiabetes agents. In total, 461 patients were randomized (230 given colesevelam 3.75 g/day and 231 given placebo). The primary efficacy measurement was mean placebo-corrected change in A1C from baseline to week 26 in the intent-to-treat population (last observation carried forward). The least squares (LS) mean change in A1C from baseline to week 26 was -0.32\% in the colesevelam group and +0.23\% in the placebo group, resulting in a treatment difference of -0.54\% (P < 0.001). The LS mean percent change in LDL cholesterol from baseline to week 26 was -16.1\% in the colesevelam group and +0.6\% in the placebo group, resulting in a treatment difference of -16.7\% (P < 0.001). Furthermore, significant reductions in fasting plasma glucose, fructosamine, total cholesterol, non-HDL cholesterol, and apolipoprotein B were demonstrated in the colesevelam relative to placebo group at week 26. Colesevelam improved glycemic control and reduced LDL cholesterol levels in patients with type 2 diabetes receiving sulfonylurea-based therapy. [\hyperlink{Colesevelam Hydrochloride}{PMID: 18458145}, Vivian A Fonseca et al., 2008]

\hypertarget{pmid_11895050}{T}o assess whether colesevelam hydrochloride is absorbed in healthy volunteers. A single-center, open-label, radiolabeled study was performed with 16 healthy volunteers. Subjects were administered non-radiolabeled colesevelam hydrochloride 1.9 g twice daily for 4 weeks, followed by a single dose of [14C]-colesevelam 2.4 g (480 pCi). These subjects continued to receive non-radioactive colesevelam 1.9 g twice daily for 4 days after administration of the radiolabeled dose. Blood, urine, and feces were collected immediately prior to administration of [14C]-colesevelam and at specified intervals after administration. The whole-blood equivalent concentration of colesevelam was calculated using data collected throughout the 96 hours following radiolabeled drug administration. The proportion of [14C]-colesevelam excreted through urine or feces was calculated based on the amount of radioactivity recovered up to 216 hours after the radiolabeled dose. The mean cumulative total recovery of [14C]-colesevelam in urine and feces was 0.05\% and 74\%, respectively. Excluding 2 subjects for whom cumulative recovery was <25\%, the mean cumulative fecal recovery was 82\%. The mean maximum whole-blood equivalent concentration of colesevelam was 0.165+/-0.10 microg equiv/g 72 hours after administration of [14C]-colesevelam, which was estimated to be 0.04\% of the administered dose. All blood samples contained <4 times the number of background counts (dpm). The cumulative recovery data in urine and feces are consistent with the conclusion that colesevelam is not absorbed and is excreted entirely through the gastrointestinal system. [\hyperlink{Colesevelam Hydrochloride}{PMID: 11895050}, Dennis P Heller et al., 2002]

\hypertarget{pmid_11605698}{T}o evaluate the efficacy, tolerability, and safety of colesevelam hydrochloride, a new nonsystemic lipid-lowering agent. In this double-blind, placebo-controlled trial performed in 1998, 494 patients with primary hypercholesterolemia (low-density lipoprotein [LDL] cholesterol level > or = 130 mg/dL and < or = 220 mg/dL) were randomized to receive placebo or colesevelam (2.3 g/d, 3.0 g/d, 3.8 g/d, or 4.5 g/d) for 24 weeks. Fasting serum lipid profiles were measured to assess efficacy. Adverse events were monitored, and discontinuation rates and compliance rates were analyzed. The primary outcome measure was the mean absolute change of LDL cholesterol from baseline to the end of the 24-week treatment period. Colesevelam lowered mean LDL cholesterol levels 9\% to 18\% in a dose-dependent manner (P<.001), with a median LDL cholesterol reduction of 20\% at 4.5 g/d. The reduction in LDL cholesterol levels was maximal after 2 weeks and sustained throughout the study. Mean total cholesterol levels decreased 4\% to 10\% (P<.001), while median high-density lipoprotein cholesterol levels increased 3\% to 4\% (P<.001). Median triglyceride levels increased by 5\% to 10\% in placebo and colesevelam treatment groups relative to baseline (P<.05), but none of these differences were significantly different from placebo. Mean apolipoprotein B levels decreased 6\% to 12\% in an apparent dose-dependent manner (P<.001). No significant differences occurred in adverse events or discontinuation rates between groups, and compliance rates were between 88\% and 92\% for all groups. Colesevelam was efficacious, decreasing mean LDL cholesterol levels by up to 18\%, and well tolerated without serious adverse events. [\hyperlink{Colesevelam Hydrochloride}{PMID: 11605698}, W Insull et al., 2001]

\hypertarget{pmid_23916045}{F}amilial hypercholesterolemia (FH) is a common autosomal co-dominant genetic disorder that results in severely increased levels of LDL-C. Patients with FH are at an increased risk for premature coronary artery disease. Expert panels therefore recommend initiation of lipid-lowering therapy in childhood to reduce the very high lifetime risk of coronary artery disease. The bile acid sequestrant colesevelam is indicated to reduce elevated LDL-C levels in adults with primary hyperlipidemia and in boys and postmenarchal girls (aged 10-17 years) with heterozygous FH. The purpose of this article was to review currently available data on the use of colesevelam in the treatment of heterozygous FH. PubMed and Google Scholar were searched to identify clinical trials evaluating colesevelam in patients with heterozygous FH. The search returned 2 results (both multicenter, multinational studies): 1 study conducted in adults and the other in pediatric patients. In the study in adults with refractory FH, the addition of colesevelam to a maximally tolerated regimen of a statin plus ezetimibe provided a significantly greater reduction from baseline in LDL-C levels compared with placebo. Significantly greater reductions from baseline in LDL-C were also seen in pediatric patients with heterozygous FH receiving colesevelam (alone or in combination with statins) compared with placebo. Colesevelam was generally well tolerated in studies in patients with FH; consistent with other colesevelam studies, gastrointestinal disorders were the most common drug-related adverse events, but these events rarely led to study withdrawal. Currently available data demonstrate that colesevelam, alone or in combination therapy, is efficacious and well tolerated in the treatment of heterozygous FH in adults and pediatric patients, supporting its use as a treatment option in both of these patient populations. [\hyperlink{Colesevelam Hydrochloride}{PMID: 23916045}, Michael Davidson et al., 2013]

\hypertarget{pmid_20527137}{O}nly a few corticosteroids for topical use have proven safe and effective in pediatric populations down to 3 months of age. The authors report the results of a study designed to assess the efficacy and safety of hydrocortisone butyrate (HCB) 0.1\% in lipocream (LCr) vehicle in infants and children. A total of 264 boys and girls 3 months to less than 18 years old, with stable, mild to moderate atopic dermatitis affecting at least 10\% body surface area applied HCB 0.1\% in LCr or LCr alone twice daily for up to 1 month without occlusion. Primary end-points included: percent of patients who achieved treatment success based on physician global assessments. Secondary endpoint included: difference in pruritus and Eczema Area and Severity Index (EASI) at day 29. Treatment was significant (P < 0.001) for HCB 0.1\% LCr over vehicle. No serious nor significant adverse events were reported. Results are representative of a short duration treatment for a chronic disease. HCB 0.1\% in LCr is more effective than its vehicle in pediatric populations down to 3 months of age without significant adverse events when used twice a day for up to 1 month. [\hyperlink{Colesevelam Hydrochloride}{PMID: 20527137}, William Abramovits et al., ]

\hypertarget{pmid_25246305}{T}he aim of this study was to compare the efficacy and safety of different oral chloral hydrate and dexmedetomidine doses used for sedation during electroencephalography (EEG) in children. One hundred sixty children aged 1 to 9 years with American Society of Anesthesiologists physical status I-II who were uncooperative during EEG recording or who were referred to our electrodiagnostic unit for sleep EEG were included to the study. The patients were randomly assigned into 4 groups. In groups D1 and D2, patients received oral dexmedetomidine doses of 2 and 3 µg/kg, respectively. In group C1 and C2, patients received oral chloral hydrate doses of 50 and 100 mg/kg, respectively. The induction time was significantly shorter in group C2 compared with other groups (P = .000). The rate of adverse effects was significantly higher in group C2 compared with the dexmedetomidine groups (D1 and D2; P = .004). In conclusion, dexmedetomidine can be used safely for sedation during EEG in children.  [\hyperlink{Colesevelam Hydrochloride}{PMID: 25246305}, Hakan Gumus et al., 2015] The complications of type 2 diabetes mellitus (DM) can begin early in the progression from impaired glucose tolerance to type 2 DM. Metformin is recommended as initial drug therapy for managing hyperglycemia in type 2 DM. The bile acid sequestrant colesevelam hydrochloride (HCl) is approved in the United States for glycemic control in adults with type 2 DM. Colesevelam HCl improves glycemic control and reduces low-density lipoprotein-cholesterol in patients inadequately controlled on metformin-, sulfonylurea-, or insulin-based therapy. This trial is designed to evaluate whether initial therapy with metformin + colesevelam HCl provides greater glucose control and additional lipid and lipoprotein benefits, as compared to metformin alone in drug-naïve patients with type 2 DM, and whether treatment with colesevelam HCl has a beneficial effect on lipid and glucose levels in drug-naïve patients with impaired glucose tolerance and/or impaired fasting glucose (prediabetes). In this multicenter, randomized, double-blind, placebo-controlled, parallel-group trial, drug-naïve patients with type 2 DM will be randomized 1 : 1 to metformin + colesevelam HCl or metformin + matching placebo, while those with prediabetes will be randomized 1 : 1 to colesevelam HCl or placebo, for 16 weeks of treatment. The primary efficacy endpoint will be change in glycosylated hemoglobin (HbA(1c)) in patients with type 2 DM and change in low-density lipoprotein-cholesterol levels in patients with prediabetes. A potential limitation is that there is no direct comparator for the dual glucose- and lipid-lowering effect of colesevelam HCl in the prediabetes cohort. However, results of this trial will help to define the extent to which colesevelam HCl can help improve cardiometabolic risk factors for complications of type 2 DM in the first-line environment, and will also indicate the extent to which early intervention with colesevelam HCl can help to correct metabolic abnormalities associated with prediabetes. [\hyperlink{Colesevelam Hydrochloride}{PMID: 25246305}, Michael R Jones et al., 2009]

\hypertarget{pmid_33655976}{C}hildren evaluated in the emergency department for head trauma often undergo computed tomography (CT), with some uncooperative children requiring pharmacological sedation. Chloral hydrate (CH) is a sedative that has been widely used, but its rectal use for child sedation after head trauma has rarely been studied. The objective of this study was to document the safety and efficacy of rectal CH sedation for cranial CT in young children.We retrospectively studied all the children with head trauma who received rectal CH sedation for CT in the emergency department from 2016 to 2019. CH was administered rectally at a dose of 50 mg/kg body weight. When sedation was achieved, CT scanning was performed, and the children were monitored until recovery. The sedative safety and efficacy were analyzed.A total of 135 children were enrolled in the study group, and the mean age was 16.05 months. The mean onset time was 16.41 minutes. Successful sedation occurred in 97.0\% of children. The mean recovery time was 71.59 minutes. All of the vital signs were within normal limits after sedation, except 1 (0.7\%) with transient hypoxia. There was no drug-related vomiting reaction in the study group. Adverse effects occurred in 11 patients (8.1\%), but all recovered completely. Compared with oral CH sedation, rectal CH sedation was associated with quicker onset (P < .01), higher success rate (P < .01), and lower adverse event rate (P < .01).Rectal CH sedation can be a safe and effective method for CT imaging of young children with head trauma in the emergency department. [\hyperlink{Colesevelam Hydrochloride}{PMID: 33655976}, Quanmin Nie et al., 2021]

\hypertarget{pmid_24627951}{T}o determine the safety and efficacy of high-dose oral chloral hydrate for pediatric ophthalmic procedures. This study is a retrospective review of a quality audit of pediatric sedation for ophthalmic evaluation and imaging performed at King Khaled Eye Specialist Hospital between January 1 and December 31, 2011, in children aged 1 month to 6 years. Three hundred fifty-eight of 380 (94.2\%) sedation procedures were successful after a single dose of chloral hydrate, with 356 of 380 (93.7\%) children sedated within 45 minutes of the first dose. The total success rate of the sedation procedure increased to 97.9\% (372 of 380) when a second dose was administered. Children adequately sedated after a single dose of chloral hydrate were on average younger and weighed less than children who required additional doses. No major adverse events were documented. The use of chloral hydrate sedation for ophthalmic evaluation and imaging was safe and effective in this patient population with a high rate of procedure completion. [\hyperlink{Colesevelam Hydrochloride}{PMID: 24627951}, Michelle E Wilson et al., ]

\hypertarget{pmid_28275979}{S}edation is often required for children undergoing diagnostic procedures. Chloral hydrate has been one of the sedative drugs most used in children over the last 3 decades, with supporting evidence for its efficacy and safety. Recently, chloral hydrate was banned in Italy and France, in consideration of evidence of its carcinogenicity and genotoxicity. Dexmedetomidine is a sedative with unique properties that has been increasingly used for procedural sedation in children. Several studies demonstrated its efficacy and safety for sedation in non-painful diagnostic procedures. Dexmedetomidine's impact on respiratory drive and airway patency and tone is much less when compared to the majority of other sedative agents. Administration via the intranasal route allows satisfactory procedural success rates. Studies that specifically compared intranasal dexmedetomidine and chloral hydrate for children undergoing non-painful procedures showed that dexmedetomidine was as effective as and safer than chloral hydrate. For these reasons, we suggest that intranasal dexmedetomidine could be a suitable alternative to chloral hydrate. [\hyperlink{Colesevelam Hydrochloride}{PMID: 28275979}, Giorgio Cozzi et al., 2017]

\hypertarget{pmid_24953836}{B}ile acid malabsorption (BAM)-associated diarrhea is an important clinical issue in patients with Crohn's disease (CD). We analyzed the efficacy and safety of the bile acid sequestrant colesevelam for treatment of BAM-associated diarrhea in CD patients in a randomized, double-blind, placebo-controlled study. The primary endpoint was the proportion of patients with >30\% reduction of liquid stools/day from baseline to termination visit at week 4. Secondary endpoints were reduction of the number of liquid stools/day, improvement of stool consistency and quality of life. 26 patients were analyzed in the intention-to-treat (ITT) analysis. The primary endpoint was reached by 10 patients (69.7\%) in the colesevelam group compared to 3 patients (27.3\%) in the placebo group (risk difference RD=.394, 95\%CI:[-0.012; 0.706]; P=.0566). In the per-protocol analysis (n=22), the risk difference was statistically significant (RD=.470, 95\%CI:[0.018; 0.788], P(H0: RD=0)=0.0364; 95\% CI:[1.3;54.7]). Regarding secondary endpoints, in the ITT population colesevelam-treated patients had a significant reduction of liquid stools/day at week 4 (median 5.0 to 2.0; P=0.01), while patients treated with placebo had no significant reduction (median 4.0 to 3.0; P=0.42). Significantly more patients in the colesevelam group had improvement of stool consistency of at least one level in the Bristol stool chart, as compared to the placebo group (P=0.003). We found significant differences in favor for colesevelam treatment compared to placebo treatment for CD patients with BAM regarding the reduction of the number of liquid stools/day and stool consistency. ClinicalTrials.gov number: NCT01203254. [\hyperlink{Colesevelam Hydrochloride}{PMID: 24953836}, Florian Beigel et al., 2014]

\hypertarget{pmid_2402648}{C}hloral hydrate has been used extensively to sedate children, but at Brooke Army Medical Center, other drug combinations were becoming increasingly popular due to a perception that chloral hydrate had a high rate of failure, especially with younger or neurologically impaired children. Therefore, 50 children were given the drug before a diagnostic study, and patient data and a sedation score were recorded on a worksheet. Of 50 children, 43 (86\%) were "successfully sedated" on the first attempt with no side effects. Children with neurologic disorders had a much greater (27\% vs 4\%) failure rate than "normal" children. The sedation rate did not significantly differ by age, sex, or initial drug dosage. The study suggest that chloral hydrate is a safe and effective oral sedative but that children with neurologic disorders may need alternative drugs for sedation. [\hyperlink{Colesevelam Hydrochloride}{PMID: 2402648}, P D Rumm et al., 1990]

\hypertarget{pmid_23024102}{W}e conducted this single blind randomized clinical trial to compare the efficacy and safety of oral chloral hydrate and intranasal midazolam for induction of sedation for computerized tomography scan of brain in children. Participants aged 1-10 years (n=60) were randomized to receive 100 mg/kg chloral hydrate orally with intra nasal normal saline OR intranasal midazolam 0.2 mg/kg with oral normal saline. Adequate sedation (Ramsay sedation score of four) was obtained and CT scan completed successfully in 76.7\% of chloral hydrate group and in 40\% of midazolam group (P=0.004). No significant difference was seen for side effects frequency between the two drugs (10\% in chloral hydrate, 3.3\% in midazolam group; P=0.34). We conclude that oral chloral hydrate can be considered as a safe and effective drug for sedation in children undergoing CT scan of brain. [\hyperlink{Colesevelam Hydrochloride}{PMID: 23024102}, Razieh Fallah et al., 2013]

\hypertarget{pmid_23776789}{H}yperglycemia and hyperlipidemia are both risk factors for the development of various complications in patients with type 2 diabetes mellitus. Colesevelam hydrochloride is a novel agent that can improve both hypercholesterolemia and hyperglycemia in such patients. It is an orally administered bile acid sequestrant with high capacity for binding bile acids. This drug can offer potential new diabetes treatment along with other drugs. [\hyperlink{Colesevelam Hydrochloride}{PMID: 23776789}, Kavita Sekhri et al., 2011]

\hypertarget{pmid_31534313}{C}hloral hydrate (CH), as a sedation agent, is widely used in children for diagnostic or therapeutic procedures. However, it has not come into the market and is currently only used as hospital preparation in China. This review aims to systematically evaluate the efficacy of CH in children of all age groups for sedation before medical procedures. Seven electronic databases and three clinical trial registry platforms were searched and the deadline was September 2018. Randomized controlled trials (RCTs) evaluating the efficacy of CH for sedation in children were included by two reviewers. The extracted information included success rate of sedation, sedation latency and sedation duration. The Cochrane risk of bias tool was applied to assess the risk of bias. The outcomes were analyzed by Review Manager 5.3 software and expressed as relative risks (RR) or Mean Difference (MD) with 95\% confidence interval (CI). Heterogeneity was assessed with I-squared (I A total of 24 RCTs involving 3564 children of CH for sedation were included in the meta-analysis. Compared to placebo group, CH group had a significant increase in success rate of sedation when used for painless and painful procedure (RR=4.15, 95\% CI [1.21, 14.24], P=0.02; RR=1.28, 95\% CI [1.17, 1.40], P<0.01), which included 22 and 455 children for this analysis, respectively. Compared to midazolam group, CH group had a significant increase in success rate of sedation (RR=1.63, 95\% CI [1.48, 1.79], I From the extrapolation of the existing literature, CH oral solution is an appropriate effective alternative for sedation in pediatrics. [\hyperlink{Colesevelam Hydrochloride}{PMID: 31534313}, Zhe Chen et al., 2019]

\hypertarget{pmid_15951862}{D}iagnostic and therapeutic procedures in children are made easier using sedation. However, there is no consensus about which drug should be used to achieve this. Furthermore, none of the drugs used for sedation are risk free. The aim of this work is to study sedation indications, effectiveness, and safety at our center. A prospective observational study conducted at the Pediatric Day Care Unit, King Fahad National Guard Hospital, Riyadh, Saudi Arabia. The study covered 17.5 weeks in 2 periods: May 9th 1999 to June 13th 1999 and October 31st 2001 to February 11th 2002. Children <12 years were included. Collected data included demographics, indication, drug dosing and outcome. Data were reported as mean +/- SD. We included 148 patients, age 38 +/- 30 months. Adequate sedation was achieved in 79\% after initial chloral hydrate (CH) dose of 56.9 +/- 9.3 mg/kg, in 95\% after adding 18.5 +/- 6.4 mg/kg CH and in 96\% after adding second drug. Compared to nonrespondents, first CH dose respondents were younger and lower in weight. The CH side effects were few and mild. Chloral hydrate is a safe and effective agent for sedation in children with an age and weight dependent response. [\hyperlink{Colesevelam Hydrochloride}{PMID: 15951862}, Omar M Hijazi et al., 2005]

\section*{Famotidine}
\subsection*{Result}
\subsubsection*{Answer}

Neonates (0–1 month): Yes  
Infants (1–12 months): Yes  
Children (1–18 years): Yes  

\subsubsection*{{Explanation}}
\hypertarget{Famotidine}
Based on the abstracts provided, there are several targeted studies evaluating the safety of famotidine in children across different age ranges:

1. **Neonates (newborns):**
   - A study of 54 neonates with stress ulcers treated with intravenous famotidine (0.5 mg/kg every 12 hours) reported that famotidine was effective and did not lead to abnormal respiration, heart rate, or blood pressure. No loss of appetite, nausea, vomiting, diarrhea, constipation, or rashes were seen. No augmented side effects from other drugs were noted. The authors concluded famotidine is effective and safe for the treatment of stress ulcers in neonates [\hyperlink{pmid_18947477}{PMID: 18947477}, Yan-Yong Wu et al., 2008].

2. **Infants (1.3–10.5 months):**
   - In a randomized, placebo-controlled trial of 35 infants (1.3–10.5 months) with gastro-oesophageal reflux disease, famotidine (0.5 or 1.0 mg/kg) was compared to placebo. No serious adverse events were reported. Non-serious, possibly drug-related adverse experiences included agitation/irritability, somnolence, anorexia, headache, vomiting, hiccups, and candidiasis. The study notes that histamine-2 receptor antagonists may cause agitation and headache in infants, but no serious safety concerns were identified. The authors call for further large placebo-controlled studies [\hyperlink{pmid_12752346}{PMID: 12752346}, S R Orenstein et al., 2003].

3. **Children (over 1 year):**
   - A review of eight studies involving over 150 children (age not specified, but "children over the age of 1 year" are specifically mentioned) found that the pharmacokinetics and pharmacodynamics of intravenous famotidine are similar to adults, and that famotidine is safe and effective for peptic ulcer disease and esophagitis in pediatric patients. The review notes that more studies are needed in children under 1 year and those with renal impairment [\hyperlink{pmid_8853932}{PMID: 8853932}, L P James et al., 1996].
   - A study of 14 boys (age not specified, but described as "children") with gastric or duodenal ulcers treated with famotidine (0.5 mg/kg twice daily, max 40 mg/day) for 8 weeks found no side effects during treatment, and all ulcers healed [\hyperlink{pmid_7846741}{PMID: 7846741}, A Nagita et al., 1994].
   - A study of 17 severely handicapped, bedridden children (age not specified) with reflux esophagitis treated with famotidine (1–2 mg/kg/day) found improvement in symptoms in all cases and no side effects; mild transient liver enzyme elevations were not attributed to the drug [\hyperlink{pmid_2889529}{PMID: 2889529}, S Miyake et al., 1987].
   - A randomized trial in 58 children (aged 2–17 years) receiving famotidine as preoperative medication found no significant adverse effects and a reduction in risk for aspiration pneumonitis [\hyperlink{pmid_1887750}{PMID: 1887750}, J S Jahr et al., 1991].
   - A study of a midazolam-famotidine suppository in children aged 5 months to 7 years for premedication found it effective, with no mention of adverse effects [\hyperlink{pmid_7739102}{PMID: 7739102}, A Iwamoto et al., 1995].

4. **Children with functional GI disorders (4–18 years):**
   - Systematic reviews and RCTs in children aged 4–18 years with abdominal pain-related functional GI disorders or functional dyspepsia found that famotidine showed benefit in global symptom improvement, and no serious adverse effects were reported. However, the overall quality of evidence was very low, and the authors call for more high-quality trials [\hyperlink{pmid_25449223}{PMID: 25449223}, Judith J Korterink et al., 2015; \hyperlink{pmid_30360666}{PMID: 30360666}, Pamela D Browne et al., 2018].

5. **Children with autism spectrum disorders (3.8–8.1 years):**
   - A small pilot study of 9 boys with autism spectrum disorders (3.8–8.1 years) treated with famotidine (2 mg/kg/day) found no mention of adverse effects, but the study was not designed primarily for safety and was very small [\hyperlink{pmid_11459079}{PMID: 11459079}, L A Linday et al., 2001].

**Summary by age range:**
- **Neonates:** One targeted study supports safety for short-term use in neonates with stress ulcers.
- **Infants (1–12 months):** One RCT supports safety, with only non-serious adverse events reported.
- **Children (>1 year):** Multiple studies, including reviews and RCTs, support safety for use in children for various indications (ulcers, esophagitis, premedication, functional GI disorders), with no serious adverse effects reported.
- **Adolescents (up to 18 years):** Systematic reviews include children up to 18 years, with no serious adverse effects reported, but note the low quality of evidence.

**Limitations:** Some studies have small sample sizes, and for certain indications (e.g., functional GI disorders), the quality of evidence is low. However, across all age groups studied, no targeted study reports famotidine as unsafe in children.

\subsection*{Abstracts}
\hypertarget{pmid_8853932}{F}amotidine, an H2 receptor antagonist, has several potential advantages over cimetidine and ranitidine. These advantages include its potency, relatively longer elimination half-life, and lack of interaction with the cytochrome P450 isoforms. Eight studies addressing the use of famotidine in paediatric patients have been published. Data from these studies demonstrate that the pharmacokinetics and pharmacodynamics of intravenous famotidine appear to be similar in both children over the age of 1 year and adults. These data support a starting paediatric dosage for intravenous famotidine of 0.5 mg/kg every 8 to 12 hours. In addition, the safety and efficacy of famotidine in the treatment of peptic ulcer disease and esophagitis in paediatric patients is supported by these studies involving over 150 children. Future studies with famotidine in paediatrics should address its disposition in children under the age of 1 year and in children with compromised renal function, as well as the bioavailability of the oral formulation. [\hyperlink{Famotidine}{PMID: 8853932}, L P James et al., 1996]

\hypertarget{pmid_16028153}{B}ecause of concerns about arthrotoxicity, fluoroquinolones are restricted for use in children. This study describes the safety and efficacy of gatifloxacin when used for treatment of children with recurrent acute otitis media (ROM) or acute otitis media (AOM) treatment failure (AOMTF). We performed an analysis of 867 children included in 4 clinical trials who had ROM and/or AOMTF and were treated with gatifloxacin (10 mg/kg once daily for 10 days). Gatifloxacin had adverse event rates that were similar overall to those of a comparator antibiotic (amoxicillin-clavulanate), except for increased diarrhea in children <2 years old receiving amoxicillin-clavulanate. There was no evidence of arthrotoxicity, hepatotoxicity, alteration of glucose homeostasis, or central nervous system toxicity acutely or during 1 year follow-up in any child. Regarding efficacy, in 2 noncomparative trials, the gatifloxacin cure rate of AOM was 89\% (95\% confidence interval [CI], 83\%-95\%) at the test of cure (TOC) visit, 3-10 days after completion of therapy. In 2 comparative trials of gatifloxacin versus amoxicillin-clavulanate, the efficacy of gatifloxacin was 88\% (95\% CI, 82\%-94\%). Gatifloxacin led to better clinical outcomes than amoxicillin-clavulanate for AOMTF (91\% vs. 81\%; P=.029), for AOMTF and age <2 years old (89\% vs. 69\%; P=.009), and for severe AOM in children <2 years old (90\% vs. 75\%; P=.012). Among children with AOMTF previously treated with amoxicillin-clavulanate or ceftriaxone injections, gatifloxacin cure rates were high (88\% and 75\%, respectively). Gatifloxacin appears to be safe for children, with no evidence of producing arthrotoxicity in 867 children exposed to the antibiotic when used as treatment for ROM and AOMTF. [\hyperlink{Famotidine}{PMID: 16028153}, Michael E Pichichero et al., 2005]

\hypertarget{pmid_25449223}{T}o systematically review literature assessing efficacy and safety of pharmacologic treatments in children with abdominal pain-related functional gastrointestinal disorders (AP-FGIDs). MEDLINE and Cochrane Database were searched for systematic reviews and randomized controlled trials investigating efficacy and safety of pharmacologic agents in children aged 4-18 years with AP-FGIDs. Quality of evidence was assessed using Grades of Recommendation, Assessment, Development and Evaluation approach. We included 6 studies with 275 children (aged 4.5-18 years) evaluating antispasmodic, antidepressant, antireflux, antihistaminic, and laxative agents. Overall quality of evidence was very low. Compared with placebo, some evidence was found for peppermint oil in improving symptoms (OR 3.3 (95\% CI 0.9-12.0) and for cyproheptadine in reducing pain frequency (relative risk [RR] 2.43, 95\% CI 1.17-5.04) and pain intensity (RR 3.03, 95\% CI 1.29-7.11). Compared with placebo, amitriptyline showed 15\% improvement in overall quality of life score (P = .007) and famotidine only provides benefit in global symptom improvement (OR 11.0; 95\% CI 1.6-75.5; P = .02). Polyethylene glycol with tegaserod significantly decreased pain intensity compared with polyethylene glycol only (RR 3.60, 95\% CI 1.54-8.40). No serious adverse effects were reported. No studies were found concerning antidiarrheal agents, antibiotics, pain medication, anti-emetics, or antimigraine agents. Because of the lack of high-quality, placebo-controlled trials of pharmacologic treatment for pediatric AP-FGIDs, there is no evidence to support routine use of any pharmacologic therapy. Peppermint oil, cyproheptadine, and famotidine might be potential interventions, but well-designed randomized controlled trials are needed. [\hyperlink{Famotidine}{PMID: 25449223}, Judith J Korterink et al., 2015]

\hypertarget{pmid_8055765}{F}amotidine has been used for the treatment of peptic ulcers and Zollinger Ellison syndrome and is also useful in reflux and erosive oesophagitis. To evaluate the effects of Famotidine 20 mg given twice daily in the symptomatic relief of gastro-oesophageal reflux disease with normal oesophagus or mild endoscopic oesophagitis, patients were followed over a period of six weeks. 70\% of the patients had complete day-time heartburn relief during the study and 75\% had complete night-time heartburn relief during the study. Famotidine was found to be safe and there were no serious clinical or laboratory adverse experiences. [\hyperlink{Famotidine}{PMID: 8055765}, F A Okoth et al., 1994]

\hypertarget{pmid_7846741}{W}e treated 14 boys, six with gastric ulcers and eight with duodenal ulcers, to determine famotidine pharmacokinetics and its inhibition of gastric acid secretion (pharmacodynamics). Famotidine (1 mg/kg/day) was administered either intravenously or orally at a dose of 0.5 mg/kg twice a day (maximum: 40 mg/day). Blood samples were collected from all subjects and the intragastric pH monitored in eight. Pharmacokinetic parameters were calculated assuming a one-compartment model. Volume of distribution, elimination half-life, and area under the serum concentration-time curve were 1.52 +/- 0.37 l/kg, 2.29 +/- 0.38 h, and 1.14 +/- 0.32 ng.h/ml, respectively. The mean oral bioavailability of famotidine was 50.6\%. Both intravenously and orally administered famotidine neutralized gastric acidity during sleep but failed to continuously maintain the intragastric pH > 5.0. All subjects' ulcers healed within 8 weeks. There were no side effects noted during famotidine treatment. Twice daily administration of 0.5 mg/kg famotidine for 8 weeks appears to be a tolerated and effective treatment of children with gastroduodenal ulcers. [\hyperlink{Famotidine}{PMID: 7846741}, A Nagita et al., 1994]

\hypertarget{pmid_1887750}{A}spiration pneumonitis is a severe complication of anesthesia. The objectives of this study were to determine if preoperative famotidine, a new histamine2-receptor antagonist, given by mouth either the evening before or the morning of elective surgery, reduced gastric residual volume and increased gastric pH in pediatric patients. Either famotidine or placebo (or both) were orally administered to 58 children (aged 2-17 years). The patients were randomly assigned to four groups: Famotidine-Famotidine, Placebo-Placebo, Placebo-Famotidine, and Famotidine-Placebo; subjects in the Famotidine-Famotidine group received two doses of famotidine (0.5 mg.kg-1 per dose), those in the Placebo-Placebo group, two doses of placebo, those in the Placebo-Famotidine and Famotidine-Placebo group, one dose of each by mouth. The Famotidine-Famotidine group received one dose of famotidine at 22:00 the evening before surgery and a second dose 60-90 min before the scheduled time of surgery. The Placebo-Placebo group received two doses of placebo at the same times as the Famotidine-Famotidine group. The Placebo-Famotidine group received a dose of placebo the night before surgery and a dose of famotidine the morning of surgery; the Famotidine-Placebo group received famotidine the night before surgery and placebo the morning of surgery. The administration of famotidine on the morning of surgery significantly increased gastric pH (4.8 vs. 1.3) in comparison with placebo, as did two doses of famotidine (6.6). Famotidine failed to reduce gastric residual volume significantly in any group. The administration of famotidine significantly reduced the number of pediatric patients considered at higher risk for aspiration pneumonitis, despite not decreasing gastric residual volume. [\hyperlink{Famotidine}{PMID: 1887750}, J S Jahr et al., 1991]

\hypertarget{pmid_18947477}{T}o investigate the efficacy and safety of famotidine treatment for stress ulcers in neonates. Fifty-four neonates with stress ulcers from 2001 to 2006 were enrolled. Seven cases were confirmed with stress ulcers by gastroscopy. Famotidine was administered intravenously at a dosage of 0.5 mg/kg every other 12 hrs. After cessation of hematemesis and vomiting, famotidine was administered once a day for two days. Primary diseases and complications were concurrently treated. Clinical symptoms and gastric pH were assessed before and after famotidine treatment. Possible adverse effects of famotidine treatmentdouble ended arrowrelated were observed. After 24 hrs of famotidine treatment, hematemesis and vomiting ceased in 52 patients (96.3\%). Clinical symptoms disappeared in all of the 54 patients 48 hrs after famotidine treatment. Gastric pH value increased 6, 12, 24, 36 and 48 hrs after famotidine treatment from 2.07+/-0.22 (before treatment) to 5.01-5.15 (P<0.01). All of the 54 patients were successfully treated. Famotidine treatment did not lead to abnormal respiration, heart rate and blood pressure. Loss of appetite, nausea, vomiting, diarrhea, constipation and rashes were not seen after famotidine treatment. There were significant differences in white cell count, platelet count and hepatic enzyme levels before and after famotidine treatment. An augmented side effect of the other drugs concurrently used due to famotidine treatment was not noted. Famotidide is effective and safe for the treatment of stress ulcers in neonates. [\hyperlink{Famotidine}{PMID: 18947477}, Yan-Yong Wu et al., 2008]

\hypertarget{pmid_30360666}{C}hronic idiopathic nausea (CIN) and functional dyspepsia (FD) cause considerable strain on many children's lives and their families. Areas covered: This study aims to systematically assess the evidence on efficacy and safety of pharmacological treatments for CIN or FD in children. CENTRAL, EMBASE, and Medline were searched for Randomized Controlled Trials (RCTs) investigating pharmacological treatments of CIN and FD in children (4-18 years). Cochrane risk of bias tool was used to assess methodological quality of the included articles. Expert commentary: Three RCTs (256 children with FD, 2-16 years) were included. No studies were found for CIN. All studies showed considerable risk of bias, therefore results should be interpreted with caution. Compared with baseline, successful relief of dyspeptic symptoms was found for omeprazole (53.8\%), famotidine (44.4\%), ranitidine (43.2\%) and cimetidine (21.6\%) (p = 0.024). Compared with placebo, famotidine showed benefit in global symptom improvement (OR 11.0; 95\% CI 1.6-75.5; p = 0.02). Compared with baseline, mosapride versus pantoprazole reduced global symptoms (p = 0.011; p = 0.009). One study reported no occurrence of adverse events. This systematic review found no evidence to support the use of pharmacological drugs to treat CIN or FD in children. More high-quality clinical trials are needed. AP-FGID: Abdominal Pain Related Functional Gastrointestinal Disorders; BART: Biofeedback-Assisted Relaxation Training; CIN: Chronic Idiopathic Nausea; COS: Core Outcomes Sets; EPS: Epigastric Pain Syndrome; ESPGHAN: European Society for Pediatric Gastroenterology Hepatology and Nutrition; FAP: Functional Abdominal Pain; FD: Functional Dyspepsia; GERD: Gastroesophageal Reflux Disease; GES: Gastric Electrical Stimulation; H [\hyperlink{Famotidine}{PMID: 30360666}, Pamela D Browne et al., 2018] Using single subject research design, we performed pilot research to evaluate the safety and efficacy of famotidine for the treatment of children with autistic spectrum disorders. We studied 9 Caucasian boys, 3.8-8.1 years old, with a DSM-IV diagnosis of a pervasive developmental disorder, living with their families, receiving no chronic medications, and without significant gastrointestinal symptoms. The dose of oral famotidine was 2 mg/kg/day (given in two divided doses); the maximum total daily dose was 100 mg. Using single-subject research analysis and medication given in a randomized, double-blind, placebo-controlled, cross-over design, 4 of 9 children randomized (44\%) had evidence of behavioral improvement. Primary efficacy was based on data kept by primary caregivers, including a daily diary; daily visual analogue scales of affection, reciting, or aspects of social interaction; Aberrant Behavior Checklists (ABC, Aman); and Clinical Global Improvement scales. Children with marked stereotypy (meaningless, repetitive behaviors) did not respond. Our subjects did not have prominent gastrointestinal symptoms and endoscopy was not part of our protocol; thus, we cannot exclude the possibility that our subjects improved due to the effective treatment of asymptomatic esophagitis. The use of famotidine for the treatment of children with autistic spectrum disorders warrants further investigation. [\hyperlink{Famotidine}{PMID: 30360666}, L A Linday et al., 2001]

\hypertarget{pmid_31050796}{F}requently, infants and children require sedation to facilitate noninvasive procedures and imaging studies. Propofol and dexmedetomidine are used to achieve deep procedural sedation in children. The objective of this study was to compare the clinical safety and efficacy of propofol versus dexmedetomidine in pediatric patients undergoing sedation in a pediatric sedation unit. A retrospective analysis of patients sedated with either propofol or dexmedetomidine in a pediatric sedation unit by pediatric emergency physicians was performed. Both medications were dosed per protocol with propofol 2 mg/kg induction and 150 μg · kg A total of 2432 children were included- 1503 who received propofol and 929 who received dexmedetomidine. Propofol and dexmedetomidine resulted in successful completion of the study in 98.8\% and 99.7\%, respectively ( Propofol use led to significantly shorter recovery times, with an increased need for airway management, but rates of bag-mask ventilation (2.3\%), airway obstruction (1.1\%), and desaturation (1.6\%) were low. No patients required intubation. Propofol is a reasonable alternative to dexmedetomidine, with a clinically acceptable safety profile. [\hyperlink{Famotidine}{PMID: 31050796}, Nicole M Schacherer et al., 2019]

\hypertarget{pmid_8649608}{P}henytoin is widely used for the prevention and treatment of acute seizures in children. Although it has the advantage of being available in parenteral form, it cannot be given through the i.m. route. Furthermore, problems with venous accessibility and maintenance may complicate i.v. administration of phenytoin in newborns and very sick infants. Fosphenytoin, a new phenytoin prodrug, can be safely administered through the i.m. route, and, because of the physical characteristics of its formulation, it offers advantages over phenytoin for i.v. administration. Clinical studies with i.v. and i.m. fosphenytoin demonstrate that the efficacy, safety, and pharmacokinetics of this drug are similar in 5- to 18-year-old children and in young adults. The safety and pharmacokinetic profile of i.v. and i.m. fosphenytoin in younger children and infants is currently being investigated. [\hyperlink{Famotidine}{PMID: 8649608}, J M Pellock et al., 1996]

\hypertarget{pmid_22477789}{D}exmedetomidine was approved by the Food and Drug Administration in 1999 for the sedation of adults receiving mechanical ventilation in an intensive care setting. It provides sedation with minimal effects on respiratory function and may be used prior to, during, and following extubation. Based on its efficacy in adults, dexmedetomidine is now being explored as an alternative or adjunct to benzodiazepines and opioids in the pediatric intensive care setting. This review describes the studies evaluating the safety and efficacy dexmedetomidine in infants and children and provides recommendations on dosing and monitoring. The MEDLINE (1950-November 2009) database was searched for pertinent abstracts, using the key term dexmedetomidine. Additional references were obtained from the bibliographies of the articles reviewed and the manufacturer. All available English-language case reports, clinical trials, retrospective studies, and review articles were evaluated. Over two dozen case series and clinical studies have documented the utility of dexmedetomidine as a sedative in children requiring mechanical ventilation or procedural sedation. In several papers, dexmedetomidine use resulted in a reduction in the dose or discontinuation of other sedative agents. It may be of particular benefit in children with neurologic impairment or in those who do not tolerate benzodiazepines. The most frequent adverse effects reported with dexmedetomidine have been hypotension and bradycardia, in 10\% to 20\% of patients. These effects typically resolve with dose reduction. Dexmedetomidine offers an additional choice for the sedation of children receiving mechanical ventilation in the intensive care setting or requiring procedural sedation. While dexmedetomidine is well tolerated when used at recommended doses, it has the potential to cause hypotension and bradycardia and requires close monitoring. In addition to clinical trials currently underway, larger controlled studies are needed to further define the role of dexmedetomidine in pediatric intensive care. [\hyperlink{Famotidine}{PMID: 22477789}, Marcia L Buck et al., 2010]

\hypertarget{pmid_2866138}{E}xtensive preclinical safety studies with famotidine were performed or sponsored by Yamanouchi Phamaceutical Co, Ltd, Tokyo, Japan, and Merck, Sharp \& Dohme Research Laboratories, West Point, Pennsylvania, USA. These studies were performed in dogs, rats, mice and rabbits, receiving oral and intravenous administration of the compound. Minimal toxicologic effects (after acute, subacute, or chronic administration) have been observed even at extremely high dosage levels (4,000 mg/kg/day) and for extended periods of administration (2,000 mg/kg/day for 105 weeks). No evidence of teratogenic, mutagenic, or carcinogenic effects or alterations of reproductive function have been seen. Based on these data, there are no contraindications for administration of this compound to humans. [\hyperlink{Famotidine}{PMID: 2866138}, J D Burek et al., 1985]

\hypertarget{pmid_2877577}{F}amotidine is a potent histamine (H2)-receptor antagonist that binds to the H2 receptor in a competitive reversible manner as shown by in vivo, in vitro, and clinical studies. Famotidine has shown no evidence of carcinogenicity, mutagenicity, or teratogenicity in extensive and adequately designed safety assessment studies. The drug produces neither prolonged anacidity nor doses its use result in significant elevations of serum gastrin levels beyond those seen with other available H2-receptor antagonists when used as recommended for the treatment of ulcer disease. Taken together, these data demonstrate no undue or disproportionate risk to the use of famotidine. [\hyperlink{Famotidine}{PMID: 2877577}, R G Berlin et al., 1986]

\hypertarget{pmid_8851452}{F}amotidine is a specific, long-acting histamine2-receptor antagonist. It is indicated for the treatment of duodenal ulcer, gastric ulcer, gastroesophageal reflux disease, and Zollinger-Ellison syndrome. Since its introduction for the treatment of acid-related disorders in 1985, an estimated 18.8 million patients worldwide have been treated with famotidine. We present a comprehensive safety profile of oral famotidine, incorporating data from investigational trials, postmarketing studies, and reports of marketed use. The excellent tolerability profile of famotidine observed during investigational trials has remained substantially unchanged during postmarketing experience. Famotidine does not notably bind to cytochrome P-450 or gastric alcohol dehydrogenase and therefore has not been associated with clinically significant drug interactions. It is generally well tolerated in patients with cardiovascular, renal, or hepatic dysfunction or with Zollinger-Ellison syndrome who have tolerated doses up to 800 mg daily. [\hyperlink{Famotidine}{PMID: 8851452}, C W Howden et al., ]

\hypertarget{pmid_19681413}{T}his randomised controlled study evaluated the effects of fentanyl and dexmedetomidine on emergence characteristics of children having adenoidectomy and anaesthetised with sevoflurane. Ninety children, two to seven years of age and ASA physical status I, were studied. Children were randomly assigned to one of three groups of 30 children, with the study intervention injection given intravenously after intubation. Children in Group F received fentanyl 2.5 microg x kg(-1), children in Group D received dexmedetomidine 0.5 microG x kg(-1) and children in Group C received saline solution. Anaesthesia was induced with 50\% N2O and 8\% sevoflurane in O2 by mask and atracurium 0.6 mg x kg(-1) was administered for tracheal intubation. All children received paracetamol 40 mg/kg rectally one hour preoperatively and dexamethasone 0.5 mg x kg(-1) intravenously. The time to extubation was shorter in Group D than Group F. The eye-opening time was longer in Group F (16.1 +/- 5.3 minutes) than in Groups C (12.0 +/- 4.2 minutes) and D (12.7 +/- 3.2 minutes). The proportion of pain-free children in early recovery was significantly higher in Groups D (47\%) and F (43\%) than Group C (13\%) (P < 0.05). The proportion of children with agitation scores > 3 was lower in Groups D 17\% (5/30) and F 13\% (4/30) than in Group C 47\% (14/30) (P < 0.05). Fentanyl 2.5 microg x kg(-1) and dexmedetomidine 0.5 microg x kg(-1) had similar haemodynamic effects and emergence characteristics. Fentanyl has been safely used in children for many years. Further studies of dexmedetomidine safety and its interaction with other anaesthetic agents are required before recommending its routine use during general anaesthesia in children. [\hyperlink{Famotidine}{PMID: 19681413}, F Erdil et al., 2009]

\hypertarget{pmid_34853785}{A}sthma is the most common chronic disease in children, many of whom are managed solely with a short-acting β The aim of this study is to determine the efficacy and safety of as-needed budesonide-formoterol therapy compared with as-needed salbutamol in children aged 5 to 15 years with mild asthma, who only use a SABA. A 52-week, open-label, parallel group, phase III RCT will recruit 380 children aged 5 to 15 years with mild asthma. Participants will be randomised 1:1 to either budesonide-formoterol (Symbicort Rapihaler This is the first RCT to assess the safety and efficacy of as-needed budesonide-formoterol in children with mild asthma. The results will provide a much-needed evidence base for the treatment of mild asthma in children. [\hyperlink{Famotidine}{PMID: 34853785}, Lee Hatter et al., 2021]

\hypertarget{pmid_12752346}{G}astro-oesophageal reflux afflicts up to 7\% of all infants. Histamine-2 receptor antagonists are the most commonly prescribed medications for this disorder, but few controlled studies support this practice. To evaluate the safety and efficacy of famotidine for infant gastro-oesophageal reflux disease. Thirty-five infants, 1.3-10.5 months of age, entered an 8-week, multi-centre, randomized, placebo-controlled, two-phase trial: first 4 weeks, observer-blind comparison of famotidine 0.5 mg/kg and famotidine 1.0 mg/kg; second 4 weeks, double-blind withdrawal comparison (safety and efficacy) of each dose with placebo. No serious adverse events were reported. Eleven patients had 16 non-serious, possibly drug-related adverse experiences: 6 patients with agitation or irritability (manifested as head-rubbing in two), 3 patients with somnolence, 2 patients with anorexia, 2 with headache, 1 patient with vomiting, 1 patient with hiccups, and 1 patient with candidiasis. Of the 35 infants, 27 completed Part I. There were significant score improvements for famotidine 0.5 mg/kg in regurgitation frequency (P = 0.04), and for famotidine 1.0 mg/kg in crying time (P = 0.027) and regurgitation frequency (P = 0.004) and volume (P = 0.01). Eight infants completed Part II on double-blind treatment, which was insufficient for meaningful comparisons. Histamine-2 receptor antagonists may cause agitation and headache in infants. A possibly efficacious famotidine dose for infants is 0.5 mg/kg (frequency adjusted for age). As 1.0 mg/kg may be more efficacious in some, the dosage may require individualization based on response. Further sizeable placebo-controlled evaluations of histamine-2 receptor antagonists in infants with gastro-oesophageal reflux disease are warranted. [\hyperlink{Famotidine}{PMID: 12752346}, S R Orenstein et al., 2003]

\hypertarget{pmid_7739102}{W}e evaluated the efficacy of midazolam-famotidine suppository (M-F suppository) for premedication in children. After obtaining informed parental consent, we studied children aged 5m-7y, ASA I status, scheduled for minor elective surgery. The suppository group (n = 26) was given suppository of both midazolam 0.5 mg.kg-1 and famotidine 2 mg.kg-1, and the intramuscular injection group (n = 19) was given hydroxyzine 1 mg.kg-1. In the suppository group, gastric pH (3.90 +/- 0.34) was significantly higher, and gastric volume (1.88 +/- 0.54 ml) was significantly less than in the intramuscular injection group (1.82 +/- 0.15, 8.42 +/- 1.76 ml). The M-F suppository may offer similar sedative effect as an intramuscular injection of hydroxyzine. We concluded that the M-F suppository is an effective premedicant for pediatric patients. [\hyperlink{Famotidine}{PMID: 7739102}, A Iwamoto et al., 1995]

\hypertarget{pmid_32022483}{A}sthma affects over 6 million children in the United States alone. This study investigated the efficacy and long-term safety of mometasone furoate-formoterol (MF/F) and MF monotherapy in children with asthma. This phase 3, multicenter, randomized controlled trial evaluated metered-dose inhaler twice daily (BID) dosing with MF/F 100/10 µg or MF 100 µg in children, aged 5 to 11 years, with a history of asthma for greater than or equal to 6 months and confirmed bronchodilator reversibility, who were adequately controlled on inhaled corticosteroid/long-acting beta-agonist combination therapy for greater than or equal to 4 weeks. After a 2-week run-in on MF 100 µg BID, eligible patients received 24 weeks of double-blind treatment and were followed for safety up to 26 weeks. The primary efficacy endpoint was the change from baseline in AM postdose 60-minute AUC \%predicted FEV1\% across 12 weeks of treatment. A total of 181 participants received at least one dose of MF/F (n = 91) or MF (n = 90). MF/F was superior to MF across the 12-week evaluation period, with a treatment advantage of 5.21 percentage points (P < .001). Superior onset of action with MF/F over MF was achieved as early as 5 minutes postdose on day 1. Overall, approximately 50\% of participants experienced one or more treatment-emergent adverse events, with fewer occurring in the MF/F group. In children 5 to 11 years of age with persistent asthma, the addition of F to MF was well tolerated and provided significant, rapid, and sustained improvement in lung function compared with MF alone. [\hyperlink{Famotidine}{PMID: 32022483}, Cindy L J Weinstein et al., 2020]

\hypertarget{pmid_15690910}{T}he efficacy of the fluoroquinolone levofloxacin in the treatment of 35 children with bronchopulmonary disease exacerbation was practically the same as that of amoxycillin/clavulanate, cefotaxime or ceftriaxone. The clinical and bacteriological results were favourable. The eradication of the pathogens responsible for the bronchopulmonary inflammations in 86\% of the patients was stated. There is no doubt that fluoroquinolones should not be widely used in pediatrics. They should be considered as reserve drugs for the treatment of severe cases when the routine agents fail. Their use is justified when the situation is risky and the data on the pathogen susceptibility to the drugs are available. Still, levofloxacin is the most safe fluoroquinolone with low hepatotoxicity and lower effect on the central nervous system. The episodes of its negative cardiovascular action are less frequent. Moreover, the most frequent side effects of fluoroquinolones such as nausea, diarrhea or vomiting are less frequent with the use of levofloxacin. No signs of arthropathy in the patients treated with levofloxacin were observed in our trial. [\hyperlink{Famotidine}{PMID: 15690910}, I K Volkov et al., 2004]

\hypertarget{pmid_7961355}{T}he objective of this open study was to determine the efficacy and safety of fluoxetine for the treatment of children and adolescents with anxiety disorders. Twenty-one patients with overanxious disorders, social phobia, or separation anxiety disorder, who were unresponsive to previous psychopharmacological and psychotherapeutic interventions, were treated openly with fluoxetine for up to 10 months. Patients with lifetime histories of obsessive-compulsive disorder (OCD) or panic disorder, or with current major depression, were excluded. Beneficial and adverse effects of fluoxetine were ascertained using the improvement and severity subscales of the Clinical Global Impression Scale (CGIS) in two ways: (1) independent chart reviews by two child psychiatrists and (2) prospective assessments by the treating nurses and the patients' mothers. Eighty-one percent (n = 17) of patients showed moderate to marked improvement in anxiety symptoms. The severity of anxiety as measured by the CGIS was also significantly reduced from marked to mild (effect size: 2.3). There were no significant side effects. These results suggest that fluoxetine may be an effective and safe treatment for nondepressed children and adolescents with anxiety disorders other than OCD and panic disorder. Future investigations using double-blind, placebo-controlled methodologies are warranted. [\hyperlink{Famotidine}{PMID: 7961355}, B Birmaher et al., 1994]

\hypertarget{pmid_14567252}{T}he clinical efficacy and safety of clarithromycin (CAM) and cefdinir (CFDN) were evaluated in 65 pediatric outpatients with group A beta-hemolytic streptococcal tonsillopharyngitis. Treatment was "effective" or better in 26 (78.8\%) children receiving CAM and in 27 (87.1\%) receiving CFDN based on antigen clearance and the "Criteria for Evaluation in Clinical Trials of Antibacterial Agents in Children" proposed by Japan Society of Chemotherapy (p = NS). The causative organisms were eradicated in 94.7\% and 93.8\% of subjects in the CAM and CFDN groups, respectively (p = NS). Adverse drug reactions were limited to moderate diarrhea in one patient in each group, and subsided during treatment. Causative organisms exhibited good susceptibility to CAM and CFDN. These results suggest excellent efficacy, safety and usefulness of CAM and CFDN in the treatment of group A beta-hemolytic streptococcal tonsillopharyngitsis in children. [\hyperlink{Famotidine}{PMID: 14567252}, Tadafumi Nishimura et al., 2003]

\hypertarget{pmid_2889529}{V}omiting, hematemesis, and esophagitis resulting from gastroesophageal reflux or hiatal hernia are frequently observed in severely handicapped children. This study was conducted to determine whether the use of a new H2-antagonist, famotidine, could prevent recurrence of reflux esophagitis among such children. Seventeen severely handicapped, bedridden children admitted to a children's medical center between April 1985 and September 1986 were studied. All had vomiting or hematemesis as a main symptom, and the cause of esophagitis was suggested to be gastroesophageal reflux in 13 cases and hiatal hernia in four. Six had been previously treated with cimetidine or other drugs or a combination thereof without relief. Famotidine was administered at about 1 to 2 mg/kg/day, two times daily to patients weighing more than 10 kg and three times daily to those weighing less than 10 kg. In 13 cases, famotidine was administered intravenously for between seven and ten days and then given orally, while the rest were given the drug orally from the outset. The following results were obtained: (1) improvement was seen within seven days after start of famotidine treatment, and reduction of vomiting or hematemesis or both was reached within two weeks in 70\% of cases and within three weeks in 94\%; (2) famotidine was markedly effective in 29\% and moderately effective in 41\%; in no case was the drug ineffective; (3) no side effects were observed; five patients had transient, mild elevation of SGOT . SGPT, but this was not attributable to the drug. [\hyperlink{Famotidine}{PMID: 2889529}, S Miyake et al., 1987]

\hypertarget{pmid_31977308}{W}e sought to assess and compare safety and efficacy of fesoterodine and oxybutynin extended-release in the treatment of pediatric overactive bladder (OAB). We conducted a non-inferiority, randomized, double-blind, crossover trial comparing fesoterodine 4-8 mg and oxybutynin 10-20 mg once daily (QD) in children with OAB aged 5-14 years (2015-2018). Every child received the first medication for eight weeks, followed by crossover to the second antimuscarinic after a three-days washout. Dose up-titration was possible at mid-course. Patients could enter a fesoterodine 12-month extension. Endpoints were assessed through changes on voiding diaries, Patient's Perception of Bladder Condition score (PPBC), adverse events, vital signs, electrocardiogram, post-void residual, urinalysis, and blood tests. The Wilcoxon rank sum and Wilcoxon signed rank tests were used for statistical analysis. A total of 62 patients were randomized (two early dropouts). Expected class effects (dry mouth/constipation) were present but no significant difference was observed. There was a 10.1 beats/minute increase in heart rate with fesoterodine (p<0.01) (oxybutynin-1.9 beats/min; p=non-significant [ns]). No life-threatening or serious adverse events occurred. Efficacy was similar for both drugs. Bladder capacity improved over the 16 months of the study; baseline capacity of 125 mL (44.5\% expected bladder capacity for age [\%EBC]) to 171 mL (53.9 \%EBC) at the end of the extension phase. No clinical or statistical difference was shown between efficacy measures for fesoterodine or oxybutynin. The use of fesoterodine or oxybutynin appear safe and effective for the treatment of OAB in children. Based on our study, long-term treatment to achieve the ultimate goal of urinary continence is needed in this population. [\hyperlink{Famotidine}{PMID: 31977308}, Sophie Ramsay et al., 2020]

\section*{Voriconazole}
\subsection*{Result}
\subsubsection*{Answer}

Neonates (<3 months): Unknown  
Infants and Toddlers (3 months to <24 months): Unknown  
Children (2 to <12 years): Yes  
Adolescents (12 to <18 years): Yes  

\subsubsection*{{Explanation}}
\hypertarget{Voriconazole}
Based on the available abstracts, the safety of voriconazole in children has been specifically studied in several age groups. Below is a summary of the evidence for each age range:

Neonates and Infants (<3 months):
- There are very limited data. One case series reports safe use in critically ill newborns with cardiac disease, with no significant drug interactions or side effects observed [\hyperlink{pmid_18367773}{PMID: 18367773}, Vikas Kohli et al., 2008]. Another report describes two preterm infants treated successfully with voriconazole without side effects [\hyperlink{pmid_26146582}{PMID: 26146582}, Emel Altuncu et al., 2015]. However, these are small case series and not large, controlled safety studies.

Infants and Toddlers (3 months to <24 months):
- A retrospective analysis of 17 immunocompromised children aged 3 to <24 months (50 treatment episodes) found no signal for increased toxicity, with hepatic function parameters unchanged during treatment and only 6\% discontinuation due to adverse events. The study concludes no increased toxicity in this age group, but notes that most exposures were subtherapeutic and calls for more systematic study [\hyperlink{pmid_37553971}{PMID: 37553971}, Silke E Gastine et al., 2023].

Young Children (2 to <12 years):
- Multiple studies have specifically evaluated safety in this age group:
    - A pharmacokinetic and safety study in 40 immunocompromised children aged 2 to <12 years found that safety was consistent with the known profile of voriconazole, with the most common adverse events being transient elevated liver function tests [\hyperlink{pmid_21968355}{PMID: 21968355}, Timothy A Driscoll et al., 2011].
    - Another study in 9 pediatric patients (2 to 11 years) found voriconazole therapy was safe and well tolerated, with the recommended IV dose of 7 mg/kg [\hyperlink{pmid_20547816}{PMID: 20547816}, Claudia Michael et al., 2010].
    - A multicenter study in 2 to 11-year-olds found visual disturbances in 12.8\% of patients, but no withdrawals due to adverse events [\hyperlink{pmid_15155217}{PMID: 15155217}, Thomas J Walsh et al., 2004].
    - A prospective pilot study in 56 children <18 years (with most in the 2-12 range) undergoing stem cell transplant found 17.8\% developed adverse effects (mainly elevated hepatic enzymes), leading to withdrawal in 7 patients, but overall suggested voriconazole can be safely used as prophylaxis [\hyperlink{pmid_21572466}{PMID: 21572466}, J R Molina et al., 2012].
    - Additional studies confirm the need for higher dosing in children and the importance of therapeutic drug monitoring, but do not report unexpected or severe safety concerns [\hyperlink{pmid_19075073}{PMID: 19075073}, Mats O Karlsson et al., 2009; \hyperlink{pmid_20660687}{PMID: 20660687}, Thomas J Walsh et al., 2010; \hyperlink{pmid_19951112}{PMID: 19951112}, Michael Neely et al., 2010; \hyperlink{pmid_26558811}{PMID: 26558811}, Hyun Mi Kang et al., 2015; \hyperlink{pmid_33376032}{PMID: 33376032}, Yuki Hanai et al., 2021].

Adolescents (12 to <18 years):
- Several studies include adolescents up to 18 years, with safety findings similar to those in younger children. Adverse events such as hepatic enzyme elevation, visual disturbances, and rare neurologic or cardiac side effects are reported, but the overall safety profile is consistent with that in adults and younger children [\hyperlink{pmid_21572466}{PMID: 21572466}, J R Molina et al., 2012; \hyperlink{pmid_26558811}{PMID: 26558811}, Hyun Mi Kang et al., 2015; \hyperlink{pmid_27313918}{PMID: 27313918}, Sevliya Öcal Demir et al., 2016; \hyperlink{pmid_23484777}{PMID: 23484777}, Dilek Uludağ et al., 2013].

Summary:
- For children aged 2 to <18 years, multiple targeted studies affirm that voriconazole is generally safe, with a safety profile similar to adults, though adverse events (especially hepatic and visual) can occur and monitoring is recommended.
- For infants and toddlers (3 months to <24 months), limited retrospective data suggest no increased toxicity, but systematic safety studies are lacking.
- For neonates and infants (<3 months), only small case series exist, so safety is not established.


\subsection*{Abstracts}
\hypertarget{pmid_21968355}{V}oriconazole pharmacokinetics are not well characterized in children despite prior studies. To assess the appropriate pediatric dosing, a study was conducted in 40 immunocompromised children aged 2 to <12 years to evaluate the pharmacokinetics and safety of voriconazole following intravenous (IV)-to-oral (PO) switch regimens based on a previous population pharmacokinetic modeling: 7 mg/kg IV every 12 h (q12h) and 200 mg PO q12h. Area under the curve over the 12-h dosing interval (AUC(0-12)) was calculated using the noncompartmental method and compared to that for adults receiving approved dosing regimens (6 → 4 mg/kg IV q12h, 200 mg PO q12h). On average, the AUC(0-12) in children receiving 7 mg/kg IV q12h on day 1 and at IV steady state were 7.85 and 21.4 μg · h/ml, respectively, and approximately 44\% and 40\% lower, respectively, than those for adults at 6 → 4 mg/kg IV q12h. Large intersubject variability was observed. At steady state during oral treatment (200 mg q12h), children had higher average exposure than adults, with much larger intersubject variability. The exposure achieved with oral dosing in children tended to decrease as weight and age increased. The most common treatment-related adverse events were transient elevated liver function tests. No clear threshold of voriconazole exposure was identified that would predict the occurrence of treatment-related hepatic events. Overall, voriconazole IV doses higher than 7 mg/kg are needed in children to closely match adult exposures, and a weight-based oral dose may be more appropriate for children than a fixed dose. Safety of voriconazole in children was consistent with the known safety profile of voriconazole. [\hyperlink{Voriconazole}{PMID: 21968355}, Timothy A Driscoll et al., 2011]

\hypertarget{pmid_18367773}{V}oriconazole is a newer systemic antifungal agent effective against Candida and Aspergillus. There are few reports of its safe use in newborns. We report the first case series of safe Voriconazole use in critically ill newborns with cardiac disease along with several other cardiac drugs without any significant drug interaction or side-effect. [\hyperlink{Voriconazole}{PMID: 18367773}, Vikas Kohli et al., 2008]

\hypertarget{pmid_20547816}{T}he aim of this study was to investigate the pharmacokinetics and safety of voriconazole after intravenous (i.v.) administration in immunocompromised children (2 to 11 years old) and adults (20 to 60 years old) who required treatment for the prevention or therapy of systemic fungal infections. Nine pediatric patients were treated with a dose of 7 mg/kg i.v. every 12 h for a period of 10 days. Three children and 12 adults received two loading doses of 6 mg/kg i.v. every 12 h, followed by a maintenance dose of 5 mg/kg (children) or 4 mg/kg (adults) twice a day during the entire study period. Trough voriconazole levels in blood over 10 days of therapy and regular voriconazole levels in blood for up to 12 h postdose on day 3 were examined. Wide intra- and interindividual variations in plasma voriconazole levels were noted in each dose group and were most pronounced in the children receiving the 7-mg/kg dose. Five (56\%) of them frequently had trough voriconazole levels in plasma below 1 microg/ml or above 6 microg/ml. The recommended dose of 7 mg/kg i.v. in children provides exposure (area under the concentration-time curve) comparable to that observed in adults receiving 4 mg/kg i.v. The children had significantly higher C(max) values; other pharmacokinetic parameters were not significantly different from those of adults. Voriconazole exhibits nonlinear pharmacokinetics in the majority of children. Voriconazole therapy was safe and well tolerated in pediatric and adult patients. The European Medicines Agency-approved i.v. dose of 7 mg/kg can be recommended for children aged 2 to <12 years. [\hyperlink{Voriconazole}{PMID: 20547816}, Claudia Michael et al., 2010]

\hypertarget{pmid_22301479}{V}oriconazole is the treatment of choice for invasive aspergillosis and its use is increasing in pediatrics. Minimal pharmacokinetic data exist in young children. We report voriconazole concentrations for 10 children <3 years of age and pharmacokinetic parameters for 1 infant who had therapeutic drug monitoring performed. Trough concentrations were unpredictable based on dose, highlighting the need to follow values during therapy. [\hyperlink{Voriconazole}{PMID: 22301479}, Elizabeth H Doby et al., 2012]

\hypertarget{pmid_37553971}{V}oriconazole (VCZ) is an important first-line option for management of invasive fungal diseases and approved in paediatric patients ≥24 months at distinct dosing schedules that consider different developmental stages. Information on dosing and exposures in children <24 months of age is scarce. Here we report our experience in children <24 months who received VCZ due to the lack of alternative treatment options. This retrospective analysis includes 50 distinct treatment episodes in 17 immunocompromised children aged between 3 and <24 months, who received VCZ between 2004 and 2022 as prophylaxis (14 patients; 47 episodes) or as empirical treatment (3 patients; 3 episodes) by mouth (46 episodes) or intravenously (4 episodes) based on contraindications, intolerance or lack of alternative options. Trough concentrations were measured as clinically indicated, and tolerability was assessed based on hepatic function parameters and discontinuations due to adverse events (AEs). VCZ was administered for a median duration of 10 days (range: 1-138). Intravenous doses ranged from 4.9 to 7.0 mg/kg (median: 6.5) twice daily, and oral doses from 3.8 to 29 mg/kg (median: 9.5) twice daily, respectively. The median trough concentration was 0.63 mg/L (range: 0.01-16.2; 38 samples). Only 34.2\% of samples were in the recommended target range of 1-6 mg/L; 57.9\% had lower and 7.9\% higher trough concentrations. Hepatic function parameters analysed at baseline, during treatment and at end of treatment did not show significant changes during VCZ treatment. There was no correlation between dose and exposure or hepatic function parameters. In three episodes, VCZ was discontinued due to an AE (6\%; three patients). In conclusion, this retrospective analysis reveals no signal for increased toxicity in paediatric patients <24 months of age. Empirical dosing resulted in mostly subtherapeutic exposures which emphasises the need for more systematic study of the pharmacokinetics of VCZ in this age group. [\hyperlink{Voriconazole}{PMID: 37553971}, Silke E Gastine et al., 2023]

\hypertarget{pmid_19075073}{V}oriconazole is a potent triazole with broad-spectrum antifungal activity against clinically significant and emerging pathogens. The present population pharmacokinetic analysis evaluated voriconazole plasma concentration-time data from three studies of pediatric patients of 2 to <12 years of age, incorporating a range of single or multiple intravenous (i.v.) and/or oral (p.o.) doses. An appropriate pharmacokinetic model for this patient population was created using the nonlinear mixed-effect modeling approach. The final model described voriconazole elimination by a Michaelis-Menten process and distribution by a two-compartment model. It also incorporated a statistically significant (P < 0.001) influence of the CYP2C19 genotype and of the alanine aminotransferase level on clearance. The model was used in a number of deterministic simulations (based on various fixed, mg/kg of body weight, and individually adjusted doses) aimed at finding suitable i.v. and p.o. voriconazole dosing regimens for pediatric patients. As a result, 7 mg/kg twice a day (BID) i.v. or 200 mg BID p.o., irrespective of body weight, was recommended for this patient population. At these doses, the pediatric area-under-the-curve (AUC) distribution exhibited the least overall difference from the adult AUC distribution (at dose levels used in clinical practice). Loading doses or individual dosage adjustments according to baseline covariates are not considered necessary in administering voriconazole to children. [\hyperlink{Voriconazole}{PMID: 19075073}, Mats O Karlsson et al., 2009]

\hypertarget{pmid_21572466}{I}nvasive fungal disease (IFD) causes significant morbidity and mortality among children undergoing allo-SCT. In this prospective pilot study, we analyze voriconazole as primary antifungal prophylaxis. From October 2004 to July 2010, 56 children <18 years of age were enrolled in this study. Patients received voriconazole doses of 5 mg/kg per 12 h (n=23) or 7 mg/kg per 12 h (n=33), with a limiting dose of 200 mg/12 h, from day -1 to day +75 or later in patients with active acute GVHD. Patients were followed up for IFD for 6 months. In this series, 37 (66.1\%) patients successfully completed treatment (85.7\% during neutropenic period) without empirical or preemptive antifungal therapy, adverse effects or IFD. Nine (16.1\%) children needed preemptive (n=2) or empirical (n=7) antifungal therapy, and one (1.8\%) of them developed a fatal probable IFD during the study period. A total of 10 (17.8\%) children developed adverse effects related to voriconazole prophylaxis, leading to definitive withdrawal on median day 26.5 (in 7 patients after granulocytic recovery). The most frequent adverse effect was persistent elevation of hepatic enzymes in seven (12.5\%) children. There were no differences between doses of 5 and 7 mg/kg per 12 h. Our results suggest that voriconazole can be safely used as primary antifungal prophylaxis in children undergoing allo-SCT. [\hyperlink{Voriconazole}{PMID: 21572466}, J R Molina et al., 2012]

\hypertarget{pmid_13680324}{V}oriconazole is a new triazole active orally and parenterally that recently proved effective in the treatment of invasive aspergillosis and in empirical antifungal therapy for persistently febrile neutropenic patients. Limited data are available for pediatric patients. We report our experience with voriconazole in seven children with invasive aspergillosis, i.e., four girls and three boys with a median age of 5 (range 2-13) years affected by acute lymphoblastic leukemia (3), acute myeloid leukemia (2), refractory anemia with excess of blasts (1), and severe aplastic anemia (1). First-line therapy in all patients was liposomal amphotericin B (AmBisome) administered at a dosage of 3-5 mg/kg day. Voriconazole was administered for a median 8 (range 2-15) weeks. Response was complete and partial in two patients, respectively, stable in one, and there was no response (failure) in two. The voriconazole treatment was well tolerated. Four patients died-two of progressive aspergillosis. Three patients are alive and well 6, 5, and 4 months after the diagnosis of aspergillosis. Voriconazole appears to be an effective salvage treatment for invasive aspergillosis in pediatric patients, with good responses in patients who recover from neutropenia or are not relapsing. [\hyperlink{Voriconazole}{PMID: 13680324}, Simone Cesaro et al., 2003]

\hypertarget{pmid_15155217}{W}e conducted a multicenter study of the safety, tolerability, and plasma pharmacokinetics of the parenteral formulation of voriconazole in immunocompromised pediatric patients (2 to 11 years old). Single doses of 3 or 4 mg/kg of body weight were administered to six and five children, respectively. In the multiple-dose study, 28 patients received loading doses of 6 mg/kg every 12 h on day 1, followed by 3 mg/kg every 12 h on day 2 to day 4 and 4 mg/kg every 12 h on day 4 to day 8. Standard population pharmacokinetic approaches and generalized additive modeling were used to construct the structural pharmacokinetic and covariate models used in this analysis. In contrast to that in adult healthy volunteers, elimination of voriconazole was linear in children following doses of 3 and 4 mg/kg every 12 h. Body weight was more influential than age in accounting for the observed variability in voriconazole pharmacokinetics. Elimination capacity correlated with the CYP2C19 genotype. Exposures were similar at 4 mg/kg every 12 h in children (median area under the concentration-time curve (AUC), 14,227 ng. h/ml) and 3 mg/kg in adults (median AUC, 13,855 ng. h/ml). Visual disturbances occurred in 5 (12.8\%) of the 39 patients and were the only drug-related adverse events that occurred more than once. No withdrawals from the study were related to voriconazole. We conclude that pediatric patients have a higher capacity for elimination of voriconazole per kilogram of body weight than do adult healthy volunteers and that dosages of 4 mg/kg may be required in children to achieve exposures consistent with those in adults following dosages of 3 mg/kg. [\hyperlink{Voriconazole}{PMID: 15155217}, Thomas J Walsh et al., 2004]

\hypertarget{pmid_33376032}{T}his systematic review and meta-analysis was designed to determine the optimal trough concentration of voriconazole for children with invasive fungal infections (IFIs). We searched electronic databases (PubMed, Cochrane Central Register of Controlled Trials, ClinicalTrials.gov and Japana Centra Revuo Medicina) for clinical studies describing the voriconazole trough concentration. We used stepwise cut-off values of 1.0-2.0 mg/L for efficacy and 3.0-6.0 mg/L for safety. The efficacy outcomes were treatment success and all-cause mortality, and the safety outcomes were hepatotoxicity, neurotoxicity and all-cause adverse events. Nine studies involving 211 patients were included in the analysis. The probability of treatment success against IFIs was significantly increased at cut-off values of ≥1.0 mg/L (odds ratio [OR] = 2.65, 95\% confidence interval [CI] = 1.20-5.87). Our analysis did not find any relationship between the trough concentration and survival. Concerning safety, the occurrence of any outcomes did not significantly differ according to the voriconazole trough concentrations at any cut-off value. However, in a subgroup analysis of Asian study locations, a significantly higher risk of hepatotoxicity was demonstrated at voriconazole trough cut-off values ≥ 3.0 mg/L (OR = 8.40, 95\% CI = 1.36-51.92). Although a significant correlation between the voriconazole concentration and hepatotoxicity was evident in regression curve analysis, (y = 0.1198e Our findings suggest that the optimal trough concentration for increasing clinical success and minimizing hepatotoxicity during voriconazole therapy in children with IFIs, particularly for Asian populations, is 1.0-3.0 mg/L. [\hyperlink{Voriconazole}{PMID: 33376032}, Yuki Hanai et al., 2021]

\hypertarget{pmid_26558811}{V}oriconazole is an antifungal drug used to treat fungal infections. This was a retrospective study of 61 children with hemato-oncologic diseases or solid organ transplantation who were administered voriconazole for invasive fungal infections. Of the 61 patients, 31 (50.8\%) were in the therapeutic drug monitoring (TDM) group, and 30 (49.2\%) were in the non-TDM group. At 12 weeks, treatment failure rate in the non-TDM group was higher than the TDM group (78.6\% versus 40.0\%, p = 0.038). Drug discontinuation due to adverse events was less frequent in the TDM group than the non-TDM group (26.0\% versus 92.3\%, p = 0.001). Children required higher dosages to maintain drug levels within the targeted therapeutic range: an average of 8.3 mg/kg/dose in patients <12 years old and 6.9 mg/kg/dose for those ≥12 years old. Treatment failure rates were higher in patients whose voriconazole levels remained below 1.0 mg/L for more than 50\% of their treatment duration than those above 1.0 mg/L (71.4\% vs. 9.1\% after 12 weeks, p = 0.013). Serial monitoring of voriconazole levels in children is important for improving treatment response and preventing unnecessary drug discontinuation. Higher dosages are needed in children to reach therapeutic range.  [\hyperlink{Voriconazole}{PMID: 26558811}, Hyun Mi Kang et al., 2015] Although voriconazole, a triazole antifungal, is a safe drug, treatment with this agent is associated with certain adverse events such as hepatic, neurologic, and visual disturbances. The current report presents two cases, one a 9-year-old boy and the other a 17-year-old girl, who experienced neurologic side effects associated with voriconazole therapy. Our aim is to remind readers of the side effects of voriconazole therapy in order to prevent unnecessary investigations especially for psychological and ophthalmologic problems. The first case was a 9-year-old boy with cystic fibrosis and invasive aspergillosis that developed photophobia, altered color sensation, and fearful visual hallucination. The second case was a 17-year-old girl with cystic fibrosis and allergic bronchopulmonary aspergillosis, and she experienced photophobia, fatigue, impaired concentration, and insomnia, when the dose of voriconazole therapy was increased from 12 mg/kg/day to 16 mg/kg/day. The complaints of the two patients disappeared after discontinuation of voriconazole therapy. Our experience in these patients reminded us of the importance of being aware of the neurologic adverse events associated with voriconazole therapy in establishing early diagnosis and initiating prompt treatment. In addition, although serum voriconazole concentration was not measured in the present cases, therapeutic drug monitoring for voriconazole seems to be critically important in preventing neurologic side effects in pediatric patients.  [\hyperlink{Voriconazole}{PMID: 26558811}, Sevliya Öcal Demir et al., 2016] Voriconazole is used in antifungal prophylaxis. We performed a retrospective review of immunocompromised children receiving prophylaxis with voriconazole during major hospital renovation, who developed phototoxic skin reactions. The overall incidence of phototoxic skin reactions was 33\%. A voriconazole dose of ≥6 mg/kg of body weight per dose twice daily was associated with a significantly greater risk to develop phototoxic skin reactions compared with lower doses. [\hyperlink{Voriconazole}{PMID: 26558811}, Sara Bernhard et al., 2012]

\hypertarget{pmid_23484777}{V}oriconazole is a triazole antifungal drug that is used to treat invasive fungal infections, especially aspergillus. Here, we report two children who had severe bradycardia associated with voriconazole at a dose of 12 mg/kg per day. Bradycardia resolved in 24 hours in both after decreasing the dose to 10 mg/kg per day. Heart rates were in normal limits on follow-up. Bradycardia may be a side effect of voriconazole treatment in children under immunosuppressive treatment. Heart rate should be monitored in patients receiving voriconazole and other triazole treatments.  [\hyperlink{Voriconazole}{PMID: 23484777}, Dilek Uludağ et al., 2013] Voriconazole pharmacokinetic and pharmacodynamic data are lacking in children. Records at the Childrens Hospital Los Angeles were reviewed for children with > or =1 serum voriconazole concentration measured from 1 May 2006 through 1 June 2007. Information on demographic characteristics, dosing histories, serum concentrations, toxicity and survival, and outcomes was obtained. A total of 207 voriconazole measurements were obtained from 46 patients (age, 0.8-20.5 years). A 2-compartment Michaelis-Menten pharmacokinetic model fit the data best but explained only 80\% of the observed variability. The crude mortality rate was 28\%, and each trough serum voriconazole concentration <1000 ng/mL was associated with a 2.6-fold increased odds of death (95\% confidence interval, 1.6-4.8; P=.002). Serum voriconazole concentrations were not associated with hepatotoxicity. Simulations predicted an intravenous dose of 7 mg/kg or an oral dose of 200 mg twice daily would achieve a trough >1000 ng/mL in most patients, but with a wide range of possible concentrations. We found a pharmacodynamic association between a voriconazole trough >1000 ng/mL and survival and marked pharmacokinetic variability, particularly after enteral dosing, justifying the measurement of serum concentrations. [\hyperlink{Voriconazole}{PMID: 23484777}, Michael Neely et al., 2010]

\hypertarget{pmid_16243008}{T}here is increasing evidence for the efficacy of the antifungal voriconazole, particularly in immunosuppression. We describe our experience of using voriconazole in children with CF. We performed a retrospective case note review of children with CF treated with voriconazole in a single centre over an 18 month period. A total of 21 children aged 5 to 16 years (median 11.3) received voriconazole for between 1 and 50 (22) weeks. Voriconazole was used as monotherapy in 2 children with recurrent allergic bronchopulmonary aspergillosis (ABPA); significant and sustained improvements in clinical and serological parameters for up to 13 months were observed, without recourse to oral steroids. Voriconazole was used in combination with an immunomodulatory agent in a further 11 children with ABPA, with significant improvement in pulmonary function and serology. 8 children without ABPA but who had recurrent Aspergillus fumigatus isolates and increased symptoms also received voriconazole; this group did not improve with treatment. Adverse effects occurred in 7 children (33\%: photosensitivity reaction 3, nausea 2, rise in hepatic enzymes 1, hair loss 1). Voriconazole may be a useful adjunctive therapy for ABPA in CF. Voriconazole monotherapy appears to be an alternative treatment strategy when oral corticosteroids may not be suitable. [\hyperlink{Voriconazole}{PMID: 16243008}, Tom Hilliard et al., 2005]

\hypertarget{pmid_34943754}{V}oriconazole is a triazole antifungal agent commonly used for the treatment and prevention of invasive aspergillosis (IA). However, the study of voriconazole's use in children is limited. The present study was performed to explore maintenance dose to optimize voriconazole dosage in children and the factors affecting voriconazole trough concentration. This is a non-interventional retrospective clinical study conducted from 1 January 2016 to 31 December 2020. The study finally included 94 children with 145 voriconazole trough concentrations. The probability of achieving a targeted concentration of 1.0-5.5 µg/mL with empiric dosing increased from 43 (45.3\%) to 78 (53.8\%) after the TDM-guided adjustment. To achieve targeted concentration, the overall target maintenance dose for the age group of less than 2, 2 to 6, 6 to 12, and 12 to 18 years old was approximately 5.71, 6.67, 5.08 and 3.31 mg·kg-1/12\&nbsp;h, respectively ( [\hyperlink{Voriconazole}{PMID: 34943754}, Yi-Chang Zhao et al., 2021] The pharmacokinetics of voriconazole in children receiving 4 mg/kg intravenously (i.v.) demonstrate substantially lower plasma exposures (as defined by area under the concentration-time curve [AUC]) than those in adults receiving the same therapeutic dosage. These differences in pharmacokinetics between children and adults limit accurate prediction of pediatric voriconazole exposure based on adult dosages. We therefore studied the pharmacokinetics and tolerability of higher dosages of an i.v.-to-oral regimen of voriconazole in immunocompromised children aged 2 to <12 years in two dosage cohorts for the prevention of invasive fungal infections. The first cohort received 4 mg/kg i.v. every 12 h (q12h), then 6 mg/kg i.v. q12h, and then 4 mg/kg orally (p.o.) q12h; the second received 6 mg/kg i.v. q12h, then 8 mg/kg i.v. q12h, and then 6 mg/kg p.o. q12h. The mean values for the AUC over the dosing interval (AUCτ) for 4 mg/kg and 6 mg/kg i.v. in cohort 1 were 11,827 and 22,914 ng.h/ml, respectively, whereas the mean AUCτ values for 6 mg/kg and 8 mg/kg i.v. in cohort 2 were 17,249 and 29,776 ng.h/ml, respectively. High interpatient variability was observed. The bioavailability of the oral formulation in children was approximately 65\%. The safety profiles were similar in the two cohorts and age groups. The most common treatment-related adverse event was increased gamma glutamyl transpeptidase levels. There was no correlation between adverse events and voriconazole exposure. In summary, voriconazole was tolerated to a similar degree regardless of dosage and age; the mean plasma AUCτ for 8 mg/kg i.v. in children approached that for 4 mg/kg i.v. in adults, thus representing a rationally selected dosage for the pediatric population. [\hyperlink{Voriconazole}{PMID: 34943754}, Thomas J Walsh et al., 2010]

\hypertarget{pmid_30150475}{V}oriconazole is a broad-spectrum triazole antifungal and the first-line treatment for invasive aspergillosis (IA). The aim of this research was to study the dose adjustments of voriconazole as well as the affecting factors influencing voriconazole trough concentrations in Asian children to optimize its daily administration. Clinical data were analyzed of inpatients 2 to 14 years old who were subjected to voriconazole trough concentration monitoring from 1 June 2015 to 1 December 2017. A total of 138 voriconazole trough concentrations from 42 pediatric patients were included. Voriconazole trough concentrations at steady state ranged from 0.02 to 9.35 mg/liter, with high inter- and intraindividual variability. Only 50.0\% of children achieved the target range (1.0 to 5.5 mg/liter) at initial dosing, while 35.7\% of children were subtherapeutic, and 14.3\% of children were supratherapeutic at initial dosing. There was no correlation between initial trough concentrations and initial dosing. A total of 28.6\% of children (12/42) received an adjusted dose according to trough concentrations. Children <6, 6 to 12, and >12 years old required a median oral maintenance dose to achieve the target range of 11.1, 7.2, and 5.3 mg/kg twice daily, respectively ( [\hyperlink{Voriconazole}{PMID: 30150475}, Lin Hu et al., 2018] Voriconazole is used for treating or preventing invasive aspergillosis and other invasive fungal infections. To minimize adverse reactions and to maximize treatment effects, therapeutic drug monitoring should be performed. However, it is challenging to optimize daily voriconazole dosing because limited data have been published so far on pediatric patients. We retrospectively analyzed voriconazole concentrations in patients aged 0-18 years. In addition, a literature review was conducted. In our study cohort, younger age and oral administration were significantly associated with lower plasma voriconazole concentrations (P < 0.01). An unfavorable outcome was associated with low concentrations of voriconazole (P = 0.01). Reports of voriconazole administration in pediatric patients show that higher doses are required in younger children and in patients receiving oral administration. Hence, the current data suggest that we should escalate both initial and maintenance doses of voriconazole in pediatric patients, particularly in patients of younger age receiving an oral administration of voriconazole. [\hyperlink{Voriconazole}{PMID: 30150475}, Karin Kato et al., 2016]

\hypertarget{pmid_19841059}{V}oriconazole is a broad spectrum antifungal agent for treating life-threatening fungal infections. Its clearance is approximately 3-fold higher in children compared with adults. Voriconazole is cleared predominantly via hepatic metabolism in adults, mainly by CYP3A4, CYP2C19, and flavin-containing monooxygenase 3 (FMO3). In vitro metabolism of voriconazole by liver microsomes prepared from pediatric and adult tissues (n = 6/group) mirrored the in vivo clearance differences in children versus adults, and it showed that the oxidative metabolism was significantly faster in children compared with adults as indicated by the in vitro half-life (T(1/2)) of 33.8 + or - 15.3 versus 72.6 + or - 23.7 min, respectively. The K(m) for voriconazole metabolism to N-oxide, the major metabolite formed in humans, by liver microsomes from children and adults was similar (11 + or - 5.2 versus 9.3 + or - 3.6 microM, respectively). In contrast, apparent V(max) was approximately 3-fold higher in children compared with adults (120.5 + or - 99.9 versus 40 + or - 13.9 pmol/min/mg). The calculated in vivo clearance from in vitro data was found to be approximately 80\% of the observed plasma clearance values in both populations. Metabolism studies in which CYP3A4, CYP2C19, or FMO was selectively inhibited provided evidence that contribution of CYP2C19 and FMO toward voriconazole N-oxidation was much greater in children than in adults, whereas CYP3A4 played a larger role in adults. Although expression of CYP2C19 and FMO3 is not significantly different in children versus adults, these enzymes seem to contribute to higher metabolic clearance of voriconazole in children versus adults. [\hyperlink{Voriconazole}{PMID: 19841059}, Souzan B Yanni et al., 2010]

\hypertarget{pmid_18252755}{V}ery little information is available regarding the use of voriconazole drug monitoring in children with invasive fungal infections. The purpose of this study was to report the cases of five paediatric patients treated with voriconazole, in which plasma levels were used to monitor therapy. Five children treated with voriconazole were included in this case series. Voriconazole plasma levels were determined using either a bioassay or liquid chromatography-tandem mass spectrometry. The patients' ages ranged from 2 to 10 years old (mean 6.2 years). Three patients had acute leukaemia and two had suffered severe burn injuries. Doses administered varied from 3.4 mg/kg every 12 h to 8.1 mg/kg every 8 h. Plasma voriconazole concentrations were unpredictable for these paediatric patients. Subtherapeutic levels were frequently observed, despite progressive increments in dosage. For others, voriconazole levels markedly increased after a small increment in dosage. Phenobarbitone caused important drug interactions with voriconazole for one [corrected] of the patients. The dose administered did not correlate with exposure as measured by plasma levels of voriconazole. While the optimal dosage for voriconazole in children is still unknown, drug monitoring seems warranted to ensure adequate exposure, and after dose increments to prevent excessive exposure. Drug interactions significantly altered exposure. [\hyperlink{Voriconazole}{PMID: 18252755}, A C Pasqualotto et al., 2008]

\hypertarget{pmid_29893015}{V}oriconazole is a broad-spectrum antifungal agent and is mainly metabolized by the liver, yet there have been no reports about voriconazole treatment in patients with Child-Pugh class C cirrhosis. The objective of this study was to investigate the pharmacokinetic profile and safety of voriconazole treatment in this cohort of patients. A retrospective, multicenter study was performed in patients with Child-Pugh class C cirrhosis who received a voriconazole maintenance dose of 100 mg twice daily (group A) or 200 mg daily (group B) orally or intravenously. All voriconazole C A total of 51 voriconazole C The voriconazole maintenance dose of 100 mg twice daily or 200 mg daily orally or intravenously may be inappropriate in patients with Child-Pugh class C cirrhosis because of the higher voriconazole C [\hyperlink{Voriconazole}{PMID: 29893015}, T Wang et al., 2018] Data on safety and efficacy of voriconazole for invasive aspergillosis (IA) and invasive candidiasis/esophageal candidiasis (IC/EC) in pediatric patients are limited. Patients aged 2-<18 years with IA and IC/EC were enrolled in 2 prospective open-label, non-comparative studies of voriconazole. Patients followed dosing regimens based on age, weight and indication, with adjustments permitted. Treatment duration was 6-12 weeks for IA patients, ≥14 days after last positive Candida culture for IC patients and ≥7 days after signs/symptoms resolution for EC patients. Primary analysis for both the studies was safety and tolerability of voriconazole. Secondary end points included global response success at week 6 and end of treatment (EOT), all-causality mortality and time to death. Voriconazole exposure-response relationship was explored. Of 53 voriconazole-treated pediatric patients (31 IA; 22 IC/EC), 14 had proven/probable IA, 7 had confirmed IC and 10 had confirmed EC. Treatment-related hepatic and visual adverse events, respectively, were reported in 22.6\% and 16.1\% of IA patients, and 22.7\% and 27.3\% of IC/EC patients. All-causality mortality in IA patients was 14.3\% at week 6; no deaths were attributed to voriconazole. No deaths were reported for IC/EC patients. Global response success rate was 64.3\% (week 6 and EOT) in IA patients and 76.5\% (EOT) in IC/EC patients. There was no association between voriconazole exposure and efficacy; however, a slight positive association between voriconazole exposure and hepatic adverse events was established. Safety and efficacy outcomes in pediatric patients with IA and IC/EC were consistent with previous findings in adult patients. [\hyperlink{Voriconazole}{PMID: 29893015}, Judith M Martin et al., 2017]

\hypertarget{pmid_26146582}{H}erein, we report two preterms with invasive candidiasis refractory to liposomal amphotericin B (AMB) treatment in spite of low MIC levels (MIC: 0.5 mcg/mL). Both of the patients' blood cultures were persistently positive for C. parapsilosis despite high therapeutic doses (AMB: 7 mg/kg per day). After starting voriconazole blood cultures became negative and both of the patients were treated successfully without any side effects. In conclusion, although it is not a standard treatment in neonatal patients, our limited experience with these patients suggests that voriconazole appears to be a safe antifungal agent to be used in critically ill preterm infants with persistent fungemia despite AMB treatment.  [\hyperlink{Voriconazole}{PMID: 26146582}, Emel Altuncu et al., 2015]

\section*{Rivaroxaban}
\subsection*{Result}
\subsubsection*{Answer}

Ages 0.5–18 years (for VTE treatment and prevention): Yes  
Ages 2–8 years (post-Fontan procedure): Yes  
Ages <0.5 years: Unknown  
Other pediatric indications: Unknown  

\subsubsection*{{Explanation}}
\hypertarget{Rivaroxaban}
Based on the available abstracts, there is substantial evidence from targeted studies evaluating the safety of rivaroxaban in children for specific indications, primarily venous thromboembolism (VTE) treatment and thromboprophylaxis, across a range of pediatric age groups.

1. **Ages 0 to 18 years (including infants, children, and adolescents):**
   - The EINSTEIN-Jr program included children aged 0 to 18 years and evaluated the safety and efficacy of bodyweight-adjusted rivaroxaban regimens for the treatment of VTE. Multiple abstracts report that rivaroxaban was well tolerated, with a low risk of recurrent VTE and clinically relevant bleeding, and that its safety profile was similar to standard anticoagulation [\hyperlink{pmid_36495716}{PMID: 36495716}, Eman Hassan et al., 2023; \hyperlink{pmid_32246743}{PMID: 32246743}, Guy Young et al., 2020; \hyperlink{pmid_33351120}{PMID: 33351120}, Philip Connor et al., 2020; \hyperlink{pmid_34292671}{PMID: 34292671}, Stefan Willmann et al., 2021; \hyperlink{pmid_32935597}{PMID: 32935597}, Omri Cohen et al., 2020]. Real-world and clinical trial data confirm safety in children as young as 0.5 years (6 months).
   - A retrospective study in children <16 years found no bleeding events and a low recurrence of thrombosis, supporting safety in this age group [\hyperlink{pmid_36495716}{PMID: 36495716}, Eman Hassan et al., 2023].
   - A single-center case series in children aged 4 months to 15 years with congenital heart disease also found rivaroxaban to be safe, with only two clinically relevant bleeding episodes in 27 patients [\hyperlink{pmid_36681724}{PMID: 36681724}, Silvestre Duran et al., 2023].

2. **Ages 2 to 8 years (post-Fontan procedure):**
   - The UNIVERSE study specifically evaluated rivaroxaban in children aged 2 to 8 years after the Fontan procedure. The study found that rivaroxaban had a similar safety profile to acetylsalicylic acid (ASA), with low rates of major and clinically relevant nonmajor bleeding [\hyperlink{pmid_34558312}{PMID: 34558312}, Brian W McCrindle et al., 2021; \hyperlink{pmid_34524700}{PMID: 34524700}, Peijuan Zhu et al., 2022]. The study concluded that rivaroxaban is appropriate for thromboprophylaxis in this population.

3. **Infants and very young children (<2 years):**
   - The EINSTEIN-Jr program and related pharmacokinetic studies included children as young as 6 months, and some real-world data included infants. These studies found predictable pharmacokinetics and a safety profile consistent with older children and adults [\hyperlink{pmid_30534007}{PMID: 30534007}, Dagmar Kubitza et al., 2018; \hyperlink{pmid_34292671}{PMID: 34292671}, Stefan Willmann et al., 2021].

4. **General pediatric population (summary):**
   - Multiple reviews and summaries of the literature affirm that rivaroxaban has a favorable safety profile for VTE treatment and prevention in children, with no increased risk of bleeding compared to standard of care [\hyperlink{pmid_36610740}{PMID: 36610740}, Kimberly Mills et al., 2023; \hyperlink{pmid_34171956}{PMID: 34171956}, Eliška Boženková et al., 2021].

5. **Limitations:**
   - While the safety of rivaroxaban is affirmed for VTE treatment and thromboprophylaxis in children, the abstracts note that further studies are needed for other indications and in special pediatric populations (e.g., those with chronic illnesses) [\hyperlink{pmid_37390311}{PMID: 37390311}, Rukhmi Bhat et al., 2023].
   - Case reports of accidental ingestion in children did not report serious adverse effects, but these are not designed to assess therapeutic safety [\hyperlink{pmid_30083644}{PMID: 30083644}, Brendan M Carr et al., 2018; \hyperlink{pmid_29337837}{PMID: 29337837}, Julieta Weirthein et al., 2019].

**Conclusion:** There is strong evidence from targeted pediatric studies that rivaroxaban is safe for use in children aged 0.5 to 18 years for the treatment and prevention of VTE, including in specific subgroups such as post-Fontan procedure patients aged 2 to 8 years. Safety for other indications or in children younger than 6 months remains unknown.

\subsection*{Abstracts}
\hypertarget{pmid_36495716}{P}aediatric clinical practice for treatment of venous thromboembolism (VTE) is based on extrapolation from adult trials with minimal data on anticoagulation efficacy and safety in children. Based on EINSTEIN-Jr clinical trial data, rivaroxaban was approved to treat VTE and prevent its recurrence in children of all ages. To report the safety and efficacy of rivaroxaban use in paediatric VTE and to present real-world data, specifically about very young children. We conducted a retrospective observational study at Birmingham Children's Hospital. Data were collected from patients <16 years old who received rivaroxaban after its licensure in the period between March 2021 and June 2022. Rivaroxaban was used for treatment of acute VTE in 64 patients. Thrombosis was CVC-related in 26 patients, unprovoked in 3, while the rest had one or more risk factors for VTE. Safety and efficacy of rivaroxaban were assessed in 52 patients after excluding patients who were on current rivaroxaban treatment and those who were lost to follow up or stopped rivaroxaban due to intolerance. No bleeding events were reported, and recurrence of thrombosis occurred in only 3.6 \%. About 35 \% had normalised re-imaging, 40.3 \% improved, 9.6 \% were unchanged and 11.5 \% stopped rivaroxaban without re-imaging. Rivaroxaban was used for secondary VTE prophylaxis in 6 patients in our cohort with no recurrence of thrombosis or bleeding reports. Our real-world experience confirmed that rivaroxaban was well tolerated, effective and safe. Further real-world data and observational studies are essential to investigate the use of rivaroxaban among different risk groups. [\hyperlink{Rivaroxaban}{PMID: 36495716}, Eman Hassan et al., 2023]

\hypertarget{pmid_36681724}{R}ivaroxaban is a direct oral anticoagulant approved for therapeutic and prophylactic anticoagulation in both adults and children. Studies on rivaroxaban use in pediatric patients with congenital heart disease (CHD) are limited. Currently, warfarin (oral) and enoxaparin (injection) are the primary options for pediatric outpatient anticoagulation. Rivaroxaban may be a less burdensome alternative, but its use has not been well described in the pediatric CHD population. We describe our single-center experience. From May 2020-July 2022, we identified all pediatric CHD patients started on rivaroxaban. Dosing was based on recommendations reported in the EINSTEIN-Jr (Male et al. in Lancet Haematol 7:e18-e27, 2020) and UNIVERSE study (McCrindle et al. in J Am Heart Assoc 10:e021765, 2021) protocols. Qualitative outcomes on safety and efficacy are reported. There were 27 patients studied with an age range of 4 months-15 years at time of medication initiation. Single ventricle heart disease was present in 70\% (19/27) of patients. Of the 27 patients initiated on rivaroxaban, 15 (56\%) were started for VTE prophylaxis and 12 (44\%) were started for VTE treatment. No patients started on rivaroxaban for prophylaxis developed a VTE. There was resolution or lack of propagation in 10 of the 12 patients started for treatment. There were two clinically relevant bleeding episodes. Our single-center case series describes the experience with rivaroxaban for VTE prophylaxis and treatment in pediatric congenital heart disease. Larger studies are required to further investigate safety, efficacy, and use indications. [\hyperlink{Rivaroxaban}{PMID: 36681724}, Silvestre Duran et al., 2023]

\hypertarget{pmid_32935597}{A}nticoagulant therapy is in use for both prevention and treatment of venous and arterial thromboembolic disorders. Delivering safe and effective anticoagulation in the pediatric population is challenging, since the available standard therapy with parenteral UFH and LMWH is troublesome for most pediatric patients, and VKAs require frequent INR monitoring due to the unpredictable pharmacokinetics and numerous food and drug interactions. Rivaroxaban, a direct FXa inhibitor, offers the convenience of oral administration and predictable pharmacokinetics across a wide range of patients. Its safety and efficacy have been previously established in various adult indications. This review outlines pharmacologic and clinical aspects regarding rivaroxaban treatment in adults and children, and provides a broad appraisal of the The EINSTEIN-Jr program which evaluated the safety and efficacy of body-weight adjusted pediatric rivaroxaban regimens for the treatment of VTE in children. A review of the literature using the keywords rivaroxaban and pediatric venous thromboembolism was conducted within the National Center for Biotechnology (NCBI) and EMBASE databases. Rivaroxaban represents an appealing therapeutic alternative for VTE in children. Further research should explore additional indications for rivaroxaban in the pediatric population beyond that of VTE. [\hyperlink{Rivaroxaban}{PMID: 32935597}, Omri Cohen et al., 2020]

\hypertarget{pmid_30534007}{T}he EINSTEIN-Jr program will evaluate rivaroxaban for the treatment of venous thromboembolism (VTE) in children, targeting exposures similar to the 20 mg once-daily dose for adults. This was a multinational, single-dose, open-label, phase I study to describe the pharmacodynamics (PD), pharmacokinetics (PK) and safety of a single bodyweight-adjusted rivaroxaban dose in children aged 0.5-18 years. Children who had completed treatment for a venous thromboembolic event were enrolled into four age groups (0.5-2 years, 2-6 years, 6-12 years and 12-18 years) receiving rivaroxaban doses equivalent to 10 mg or 20 mg (either as a tablet or oral suspension). Blood samples for PK and PD analyses were collected within specified time windows. Fifty-nine children were evaluated. In all age groups, PD parameters (prothrombin time, activated partial thromboplastin time and anti-Factor Xa activity) showed a linear relationship versus rivaroxaban plasma concentrations and were in line with previously acquired adult data, as well as in vitro spiking experiments Bodyweight-adjusted, single-dose rivaroxaban had predictable PK/PD profiles in children across all age groups from 0.5 to 18 years. The PD assessments based on prothrombin time and activated partial thromboplastin time demonstrated that the anticoagulant effect of rivaroxaban was not affected by developmental hemostasis in children. ClinicalTrials.gov number, NCT01145859. [\hyperlink{Rivaroxaban}{PMID: 30534007}, Dagmar Kubitza et al., 2018]

\hypertarget{pmid_32246743}{R}ecently, the randomized EINSTEIN-Jr study showed similar efficacy and safety for rivaroxaban and standard anticoagulation for treatment of pediatric venous thromboembolism (VTE). The rivaroxaban dosing strategy was established based on phase 1 and 2 data in children and through pharmacokinetic (PK) modeling. Rivaroxaban treatment with tablets or the newly developed granules-for-oral suspension formulation was bodyweight-adjusted and administered once-daily, twice-daily, or thrice-daily for children with bodyweights of ≥30, ≥12 to <30, and <12 kg, respectively. Previously, these regimens were confirmed for children weighing ≥20 kg but only predicted in those <20 kg. Based on sparse blood sampling, the daily area under the plasma concentration-time curve [AUC Of the 335 children (aged 0-17 years) allocated to rivaroxaban, 316 (94.3\%) were evaluable for PK analyses. Rivaroxaban exposures were within the adult exposure range. No clustering was observed for any of the PK parameters with efficacy, bleeding, or adverse event outcomes. Results were similar for the tablet and suspension formulation. Acceptability and palatability of the suspension were favorable. Based on this analysis and the recently documented similar efficacy and safety of rivaroxaban compared with standard anticoagulation, we conclude that bodyweight-adjusted pediatric rivaroxaban regimens with either tablets or suspension are validated and provide for appropriate treatment of children with VTE. [\hyperlink{Rivaroxaban}{PMID: 32246743}, Guy Young et al., 2020]

\hypertarget{pmid_23642380}{R}ivaroxaban, a factor Xa inhibitor, is a new oral anticoagulant that has been developed as an alternative to vitamin K antagonists. However, its safety remains unclear. Reported randomized controlled trials comparing the safety of rivaroxaban with that of vitamin K antagonists (warfarin, acenocoumarol, phenprocoumon, and fluindione) were systematically searched. Inclusion was restricted to studies of ≥30 days' treatment duration. Safety end points examined included major and clinically relevant nonmajor bleeding, as well as mortality. Data were pooled across randomized controlled trials using random-effects meta-analysis models. Five randomized controlled trials including 23,063 patients that met the inclusion criteria were identified. Patients received treatment for nonvalvular atrial fibrillation (n = 14,264), deep vein thrombosis (n = 3,967), or acute symptomatic pulmonary embolism (n = 4,832). Overall, rivaroxaban was not associated with the risk of a composite end point of major or clinically relevant nonmajor bleeding (relative risk 0.99, 95\% confidence interval 0.93 to 1.06). However, rivaroxaban was associated with a significant decrease in fatal bleeding (relative risk 0.48, 95\% confidence interval 0.31 to 0.74). In 2 studies reporting intracranial bleeding events, rivaroxaban was associated with decreased risk compared with vitamin K antagonists. It was not associated with decreased risk for all-cause mortality (relative risk 0.89, 95\% confidence interval 0.73 to 1.09). In conclusion, with a decrease in fatal bleeding and no suggestion of an increase in all-cause mortality, rivaroxaban has a favorable safety profile with respect to bleeding. [\hyperlink{Rivaroxaban}{PMID: 23642380}, Guila Wasserlauf et al., 2013]

\hypertarget{pmid_34558312}{B}ackground Patients with single-ventricle physiology who undergo the Fontan procedure are at risk for thrombotic events associated with significant morbidity and mortality. The UNIVERSE Study evaluated the efficacy and safety of a novel liquid rivaroxaban formulation, using a body weight-adjusted dosing regimen, versus acetylsalicylic acid (ASA) in children post-Fontan. Methods and Results The UNIVERSE Study was a randomized, multicenter, 2-part, open-label study of rivaroxaban, in children who had undergone a Fontan procedure, to evaluate its dosing regimen, safety, and efficacy. Part A was the single-arm part of the study that determined the pharmacokinetics/pharmacodynamics and safety of rivaroxaban in 12 participants before proceeding to part B, whereby 100 participants were randomized 2:1 to open-label rivaroxaban versus ASA. The study period was 12 months. A total of 112 participants were enrolled across 35 sites in 10 countries. In part B, for safety outcomes, major bleeding occurred in one participant on rivaroxaban (epistaxis that required transfusion). Clinically relevant nonmajor bleeding occurred in 6\% of participants on rivaroxaban versus 9\% on ASA. Trivial bleeding occurred in 33\% of participants on rivaroxaban versus 35\% on ASA. For efficacy outcomes, 1 participant on rivaroxaban in part B had a pulmonary embolism (2\% overall event rate); and for ASA, 1 participant had ischemic stroke and 2 had venous thrombosis (9\% overall event rate). Conclusions In this study, participants who received rivaroxaban for thromboprophylaxis had a similar safety profile and fewer thrombotic events, albeit not statistically significant, compared with those in the ASA group. Registration URL: https://www.clinicaltrials.gov. Identifier: NCT02846532. [\hyperlink{Rivaroxaban}{PMID: 34558312}, Brian W McCrindle et al., 2021]

\hypertarget{pmid_33351120}{A}nticoagulant treatment of pediatric cerebral venous thrombosis has not been evaluated in randomized trials. We evaluated the safety and efficacy of rivaroxaban and standard anticoagulants in the predefined subgroup of children with cerebral venous thrombosis (CVT) who participated in the EINSTEIN-Jr trial. Children with CVT were randomized (2:1), after initial heparinization, to treatment with rivaroxaban or standard anticoagulants (continued on heparin or switched to vitamin K antagonist). The main treatment period was 3 months. The primary efficacy outcome, symptomatic recurrent venous thromboembolism (VTE), and principal safety outcome, major or clinically relevant nonmajor bleeding,were centrally evaluated by blinded investigators. Sinus recanalization on repeat brain imaging was a secondary outcome. Statistical analyses were exploratory. In total, 114 children with confirmed CVT were randomized. All children completed the follow-up. None of the 73 rivaroxaban recipients and 1 (2.4\%; CVT) of the 41 standard anticoagulant recipients had symptomatic, recurrent VTE after 3 months (absolute difference, 2.4\%; 95\% confidence interval [CI], -2.6\% to 13.5\%). Clinically relevant bleeding occurred in 5 (6.8\%; all nonmajor and noncerebral) rivaroxaban recipients and in 1 (2.5\%; major [subdural] bleeding) standard anticoagulant recipient (absolute difference, 4.4\%; 95\% CI, -6.7\% to 13.4\%). Complete or partial sinus recanalization occurred in 18 (25\%) and 39 (53\%) rivaroxaban recipients and in 6 (15\%) and 24 (59\%) standard anticoagulant recipients, respectively. In summary, in this substudy of a randomized trial with a limited sample size, children with CVT treated with rivaroxaban or standard anticoagulation had a low risk of recurrent VTE and clinically relevant bleeding. This trial was registered at clinicaltrials.gov as \#NCT02234843. [\hyperlink{Rivaroxaban}{PMID: 33351120}, Philip Connor et al., 2020]

\hypertarget{pmid_28712814}{R}ivaroxaban is a United States Food and Drug Administration-approved oral anticoagulant for venous thromboembolic disease; however, there is no information regarding the safety and its efficacy to support its use in patients after open or endovascular arterial interventions. We report the safety and efficacy of rivaroxaban vs warfarin in patients undergoing peripheral arterial interventions. This single-institution retrospective study analyzed all sequential patients from December 2012 to August 2014 (21 months) who were prescribed rivaroxaban or warfarin after a peripheral arterial procedure. Our study population was then compared using American College of Chest Physicians guidelines with patients then stratified as low, medium, or high risk for bleeding complications. Statistical analyses were performed using the Student t-test and χ There were 44 patients in the rivaroxaban group and 50 patients in the warfarin group. Differences between demographics and risk factors for bleeding between groups or reintervention rate were not statistically significant (P = .297). However, subgroup evaluation of the safety profile suggests that patients who were aged ≤65 years and on warfarin had an overall higher incidence of major bleeding (P = .020). Patients who were aged >65 years, undergoing open operation, had a significant risk for reintervention (P = .047) when they received rivaroxaban. Real-world experience using rivaroxaban and warfarin in patients after peripheral arterial procedures suggests a comparable safety and efficacy profile. Subgroup analysis of those requiring an open operation demonstrated a decreased bleeding risk when rivaroxaban was used (in those aged <65 years) but an increased risk for secondary interventions. Further studies with a larger cohort are required to validate our results. [\hyperlink{Rivaroxaban}{PMID: 28712814}, Anjan Talukdar et al., 2017]

\hypertarget{pmid_22884545}{C}urrent anticoagulation therapy in children is less than ideal, requiring regular venous monitoring and dosing adjustments. Limitations associated with conventional anticoagulants have prompted the development of novel drugs that specifically target key proteins in the coagulation system. Rivaroxaban is the first oral, direct Factor Xa inhibitor available for the prevention of venous thromboembolism in adults. Its predictable pharmacokinetic profile, high oral bioavailability and once-daily dosing make rivaroxaban an optimal anticoagulant that warrants investigation in children. The aim of this study was to investigate the age-related anticoagulant effect of rivaroxaban in vitro. Age-specific plasma pools were created (i.e. 28 days-23 months, 2-6, 7-11, 12-16 years and adults) and spiked with increasing concentrations of rivaroxaban (0-500 ng/ml). Commercially available PT, APTT and anti-Factor Xa assays, as well as sub-sampling thrombin generation assays, were used to measure rivaroxaban effect. The results of this study indicate that there are no significant differences in rivaroxaban effect across the age groups in vitro. In vivo studies are required to confirm the consistency of dose-response across the paediatric age groups. [\hyperlink{Rivaroxaban}{PMID: 22884545}, C Attard et al., 2012]

\hypertarget{pmid_30083644}{R}ivaroxaban, an oral anticoagulant, directly inhibits factor Xa (FXa). A 35-month-old boy was brought to the emergency department 15 minutes after ingesting 200 mg of rivaroxaban (16 mg/kg). Activated charcoal (AC) was administered; the patient was observed with monitoring of plasma anti-FXa levels and discharged the following day after an uneventful hospital observation. We identified two case series and seven case reports of potentially toxic rivaroxaban ingestion in the literature. No serious adverse effects were reported. The present case is the first reported use of anti-FXa monitoring after rivaroxaban ingestion. The magnitude of the effect of AC administration in this patient is unclear. [\hyperlink{Rivaroxaban}{PMID: 30083644}, Brendan M Carr et al., 2018]

\hypertarget{pmid_34292671}{R}ivaroxaban has been investigated in the EINSTEIN-Jr program for the treatment of acute venous thromboembolism (VTE) in children aged 0 to 18 years and in the UNIVERSE program for thromboprophylaxis in children aged 2 to 8 years with congenital heart disease after Fontan-procedure. Physiologically-based pharmacokinetic (PBPK) and population pharmacokinetic (PopPK) modeling were used throughout the pediatric development of rivaroxaban according to the learn-and-confirm paradigm. The development strategy was to match pediatric drug exposures to adult exposure proven to be safe and efficacious. In this analysis, a refined pediatric PopPK model for rivaroxaban based on integrated EINSTEIN-Jr data and interim PK data from part A of the UNIVERSE phase III study was developed and the influence of potential covariates and intrinsic factors on rivaroxaban exposure was assessed. The model adequately described the observed pediatric PK data. PK parameters and exposure metrics estimated by the PopPK model were compared to the predictions from a previously published pediatric PBPK model for rivaroxaban. Ninety-one percent of the individual post hoc clearance estimates were found within the 5th to 95th percentile of the PBPK model predictions. In patients below 2 years of age, however, clearance was underpredicted by the PBPK model. The iterative and integrative use of PBPK and PopPK modeling and simulation played a major role in the establishment of the bodyweight-adjusted rivaroxaban dosing regimen that was ultimately confirmed to be a safe and efficacious dosing regimen for children aged 0 to 18 years with acute VTE in the EINSTEIN-Jr phase III study. [\hyperlink{Rivaroxaban}{PMID: 34292671}, Stefan Willmann et al., 2021]

\hypertarget{pmid_20062915}{R}ivaroxaban is a novel, oral, direct factor Xa (FXa) inhibitor for the prevention and treatment of thromboembolic disorders. The aim of this study was to investigate the safety, pharmacokinetics (PK) and pharmacodynamics (PD) of rivaroxaban in healthy, elderly Chinese subjects. In this single-centre, single-blind, placebo-controlled, parallel-group, dose-escalation study, 79 subjects, aged 59-74years (mean 62.8), were randomised to receive once-daily oral doses of rivaroxaban 5, 10, 20, 30 or 40mg. Rivaroxaban was well tolerated: there was a low incidence of treatment-emergent adverse events and all events were of mild intensity. Rivaroxaban was absorbed rapidly, reaching maximum plasma concentrations within 2-4hours. The PK of rivaroxaban were dose dependent over the dose range tested. Maximal inhibition of FXa occurred 2-3hours after dosing and returned to baseline after 24-48hours, reflecting rivaroxaban plasma concentrations. Inhibition of FXa was associated with dose-dependent effects on global clotting tests. There were no clinically relevant differences in rivaroxaban plasma concentrations between male and female subjects. In conclusion, rivaroxaban was well tolerated and was found to have predictable PK and PD in healthy, elderly Chinese subjects. [\hyperlink{Rivaroxaban}{PMID: 20062915}, Ji Jiang et al., 2010]

\hypertarget{pmid_36610740}{T}he aim of this article is to provide an overview of the current literature for direct-acting oral anticoagulant (DOAC) use in pediatric patients and summarize ongoing trials. In treatment of venous thromboembolism (VTE) in pediatric patients, evidence supports use of both dabigatran and rivaroxaban. Dabigatran has been shown to be noninferior to standard of care (SOC) in terms of efficacy, with similar bleeding rates. Similarly, treatment with rivaroxaban in children with acute VTE resulted in a low recurrence risk and reduced thrombotic burden, without increased risk of bleeding, compared to SOC. Treatment of pediatric cerebral venous thrombosis as well as central venous catheter-related VTE with rivaroxaban appeared to be both safe and efficacious and similar to that with SOC. Dabigatran also has a favorable safety profile for prevention of VTE, and rivaroxaban has a favorable safety profile for VTE prevention in children with congenital heart disease. Many studies with several different DOACs are ongoing to evaluate both safety and efficacy in unique patient populations, as well as VTE prevention. The literature regarding pediatric VTE treatment and prophylaxis is growing, but the need for evidence-based pediatric guidelines remains. Additional long-term, postauthorization studies are warranted to further elucidate safety and efficacy in clinical scenarios excluded in clinical trials. Additional data on safety, efficacy, and dosing strategies for reversal agents are also necessary, especially as the use of DOACs becomes more common in the pediatric population. [\hyperlink{Rivaroxaban}{PMID: 36610740}, Kimberly Mills et al., 2023]

\hypertarget{pmid_24154682}{T}he direct factor Xa inhibitor rivaroxaban is approved for venous thromboembolism (VTE) treatment in adults. However, in all phase-III trials children or adolescents have not been included. For under-aged VTE patients, current standard treatment consists of low molecular weight heparin or Vitamin K antagonists. Rivaroxaban could be an attractive alternative, however, no data on the pharmacokinetics (PK) of rivaroxaban in adolescents are currently available. PATIENT, METHODS: We report PK data for rivaroxaban derived from a girl (age:15 years), who presented three month after acute deep vein thrombosis, already receiving rivaroxaban therapy. In the steady state of rivaroxaban therapy (20 mg once daily), plasma levels at baseline, 3 and 6 hours after intake of rivaroxaban were measured to evaluate the pharmacokinetics and changes of global coagulation tests. At baseline, a very low trough level of only 9.9 ng/ml rivaroxaban was found. At 3 hours, a peak concentration of 137.76 ng/ml rivaroxaban was observed with a rapid decrease within 6 hours after drug intake, when plasma levels of 34.45 ng/ml were measured. The patients INR and aPTT values reacted correspondingly. Our data indicate that adolescents may exhibit lower peak and trough levels after rivaroxaban intake compared to adult patients, but seem to have similar PK curves during the elimination phase. While our case is the first published case of a successful VTE treatment in an under-aged patient, we strongly discourage the routine use of rivaroxaban in non-adult patients, until data from phase II and III trials are available. [\hyperlink{Rivaroxaban}{PMID: 24154682}, J Beyer-Westendorf et al., 2014]

\hypertarget{pmid_34524700}{T}hrombosis remains an important complication for children with single-ventricle physiology following the Fontan procedure, and effective thromboprophylaxis is an important unmet medical need. To obviate conventional dose-finding studies and expedite clinical development, a rivaroxaban dose regimen for this indication was determined using a model-informed drug development approach. A physiologically based pharmacokinetic rivaroxaban model was used to predict a pediatric dosing regimen that would produce drug exposures similar to that of 10 mg once daily in adults. This regimen was used in an open-label, multicenter phase III study, which investigated the use of rivaroxaban for thromboprophylaxis in post-Fontan patients 2 to 8 years of age. The pharmacokinetics (PK) of rivaroxaban was assessed in part A (n = 12) and in part B (n = 64) of the UNIVERSE study. The safety and efficacy in the rivaroxaban group were compared to those in the acetylsalicylic acid group for 12 months. Pharmacodynamic end points were assessed in both parts of the study. Rivaroxaban exposures achieved in parts A and B were similar to the adult reference exposures. Prothrombin time also showed similarity to the adult reference. Exposure-response analysis did not identify a quantitative relationship between rivaroxaban exposures and efficacy/safety outcomes within the observed exposure ranges. A body weight-based dose regimen selected by physiologically based pharmacokinetic modeling was shown in the UNIVERSE study to be appropriate for thromboprophylaxis in the post-Fontan pediatric population. Model-based dose selection can support pediatric drug development and bridge adult dose data to pediatrics, thereby obviating the need for dose-finding studies in pediatric programs. [\hyperlink{Rivaroxaban}{PMID: 34524700}, Peijuan Zhu et al., 2022]

\hypertarget{pmid_29337837}{N}ovel oral anticoagulants offer equivalent or improved therapeutic profiles compared with warfarin, with less risk of bleeding, no interactions with food, and no need for routine laboratory monitoring. Caution must be exercised in using these drugs in certain patient populations, for example, renal insufficiency, those receiving additional antithrombotic therapy, those with questionable compliance, children, and those with a high risk of gastrointestinal bleeding. One of the novel oral anticoagulants, rivaroxaban, is a direct Factor Xa inhibitor, used to reduce risk of stroke and systemic embolism in patients with nonvalvular atrial fibrillation, deep vein thrombosis, and pulmonary embolism. We report a child who presented abnormal coagulation tests after unintended ingestion of 4 tablets of rivaroxaban. The patient was treated with fresh frozen plasma as well as admitted to intensive care and improved several hours later. We discuss his presentation and review of the literature on this topic. [\hyperlink{Rivaroxaban}{PMID: 29337837}, Julieta Weirthein et al., 2019]

\hypertarget{pmid_21848931}{R}ivaroxaban, an oral, direct factor Xa inhibitor, is a small molecule drug capable of inhibiting not only free factor Xa with high selectivity but also prothrombinase-bound and clot-associated factor Xa in a concentration-dependent manner. Clinical studies have demonstrated predictable anticoagulation and confirmed dose-proportional effects for rivaroxaban in humans with a rapid onset (within 2-4 h) and a half-life of 7-11 h and 11-13 h for young and elderly subjects, respectively. For a 10 mg dose, the oral bioavailability of rivaroxaban is high (80-100\%) and is not affected by food intake. These favourable pharmacological properties underpin the use of rivaroxaban in fixed dosing regimens, with no need for dose adjustment or routine coagulation monitoring. Rivaroxaban has a dual mode of excretion with the renal route accounting for one-third of the overall elimination of unchanged active drug. Rivaroxaban is a substrate of CYP3A4 and P-glycoprotein and therefore not recommended for concomitant use with strong inhibitors of both pathways, e.g. most azole antimycotics and protease inhibitors. Rivaroxaban is currently approved for the prevention of venous thromboembolism (VTE) in adult patients undergoing elective hip or knee replacement surgery. Studies using 10 mg rivaroxaban once daily in this indication demonstrated its suitability for a wide range of patients regardless of age, gender or body weight. Further studies in the treatment of VTE, prevention of cardiovascular events in patients with acute coronary syndrome, prevention of stroke in those with atrial fibrillation and prevention of VTE in hospitalized medically ill patients have been reported or are ongoing. [\hyperlink{Rivaroxaban}{PMID: 21848931}, Reinhold Kreutz et al., 2012]

\hypertarget{pmid_33148732}{T}o evaluate the short-term (12 weeks) safety and utilisation of rivaroxaban prescribed to new-user adult patients for the treatment of deep vein thrombosis and pulmonary embolism and for the prevention of recurrent deep vein thrombosis and pulmonary embolism in a secondary care setting in England and Wales. An observational cohort study using the technique of Specialist Cohort Event Monitoring. The Rivaroxaban Observational Safety Evaluation study was conducted across 87 participating National Health Service secondary care trusts in England and Wales. 1532 patients treated with rivaroxaban for the prevention and treatment of deep vein thrombosis/pulmonary embolism from September 2013 to January 2016. Non-interventional postauthorisation safety study of rivaroxaban. PRIMARY AND SECONDARY OUTCOME MEASURES: (1) Risk of major bleeding in gastrointestinal, intracranial, and urogenital sites and (2) risk of all major and clinically relevant non-major bleeds. Of a total of 4846 patients enrolled in the study from September 2013 to January 2016, 1532 were treated with rivaroxaban for the prevention and treatment of deep vein thrombosis/pulmonary embolism. The median age of the deep vein thrombosis/pulmonary embolism cohort was 63 years, and 54.6\% were men. The risk of major bleeding within the gastrointestinal, urogenital and intracranial primary sites was 0.7\% (n=11), 0.3\% (n=5) and 0.1\% (n=1), respectively. The risk of major bleeding in all sites was 1.5\% (n=23) at a rate of 8.3 events per 100 patient-years. In terms of the primary outcome risk of major bleeding in gastrointestinal, intracranial and urogenital sites, the risk estimates in the population using rivaroxaban for deep vein thrombosis/pulmonary embolism were low (<1\%) and consistent with the risk estimated from clinical trial data and in routine clinical practice. ClinicalTrials.gov Registry (NCT01871194); ENCePP Registry (EUPAS3979). [\hyperlink{Rivaroxaban}{PMID: 33148732}, Alison Evans et al., 2020]

\hypertarget{pmid_25074401}{R}ivaroxaban is a novel anticoagulant approved for use in patients with atrial fibrillation for stroke prevention. It is a factor Xa inhibitor, and its activity cannot be monitored with use of the international normalized ratio. A 5.6\% chance of major bleeding is associated with rivaroxaban use, including intracranial and gastrointestinal bleeds. We report the first case, to our knowledge, of isolated hemopericardium related to rivaroxaban use, which could potentially lead to death from cardiac tamponade. A 76-year-old man who was receiving rivaroxaban for atrial fibrillation presented to the emergency department with pleuritic chest pain and was found to have a hemopericardium. No signs of tamponade were evident, and his bleed remained stable after discontinuing rivaroxaban. The patient had also been taking saw palmetto, which may have contributed to the bleed by increasing rivaroxaban activity. A work-up for other causes of hemopericardium, including pacemaker lead misplacement and autoimmune disease-related pericarditis, was negative. Use of the Naranjo adverse drug reaction probability scale indicated a probable relationship (score of 5) between the patient's development of hemopericardium and rivaroxaban use. This case highlights the potential for bleeding complications associated with novel anticoagulants. Herbal products and various drugs may increase rivaroxaban levels by inhibiting P-glycoprotein and cytochrome P450 3A4 activity. Clinicians should be aware of these potential interactions with rivaroxaban and perform a review of not only the patient's drug therapy but also any herbal and food products that could alter the levels of anticoagulants. The lack of an antidote and the inability to dialyze rivaroxaban is a significant concern in situations of life-threatening bleeds. A laboratory test for monitoring rivaroxaban levels may be required for its safe use.  [\hyperlink{Rivaroxaban}{PMID: 25074401}, Poojita Shivamurthy et al., 2014] The contribution of coagulation activation to the pathogenesis of sickle cell disease (SCD) remains incompletely defined. We evaluated the efficacy and safety of rivaroxaban, an oral direct factor Xa inhibitor, in subjects with sickle cell anemia. In this pilot, single-center, randomized, double-blind, placebo-controlled, crossover study, eligible subjects with sickle cell anemia received rivaroxaban or placebo. The effect of rivaroxaban on coagulation activation, endothelial activation, inflammation, and microvascular blood flow was evaluated. Fourteen patients (HbSS - 14; females - 9) with mean age of 38 ± 10.6 years were randomized to receive rivaroxaban 20 mg daily or placebo for 4 weeks and, following a 2-week washout phase, were "crossed-over" to the treatment arm opposite to which they were initially assigned. Mean adherence to treatment with rivaroxaban, assessed by pill counts, was 85.6\% in the first treatment period and 93.6\% in the second period. Treatment with rivaroxaban resulted in a decrease from baseline of thrombin-antithrombin complex versus placebo (-34.4 ug/L [95\% CI: -69.4, 0.53] vs. 0.35 ug/L [95\% CI: -3.8, 4.5], p = .08), but the difference was not statistically significant. No significant differences were observed in changes from baseline of D-dimer, inflammatory, and endothelial activation markers or measures of microvascular blood flow. Rivaroxaban was well tolerated. Rivaroxaban was safe but did not significantly decrease coagulation activation, endothelial activation, or inflammation. Rivaroxaban did not improve microvascular blood flow. Adequately powered studies are required to further evaluate the efficacy of rivaroxaban in SCD. Clinicaltrials.gov Identifier: NCT02072668. [\hyperlink{Rivaroxaban}{PMID: 25074401}, Kenneth I Ataga et al., 2021]

\hypertarget{pmid_31132584}{T}he Fontan procedure is the final step of the 3-stage palliative procedure commonly performed in children with single ventricle physiology. Thrombosis remains an important complication in children after this procedure. To date, guideline recommendations for the type and duration of thromboprophylaxis after Fontan surgery are mainly based on extrapolation of knowledge gained from adults at risk for thrombosis in other clinical settings. Warfarin is being used off-label, and because of its multiple interactions with other drugs and food, a new alternative is highly desirable. Rivaroxaban, a direct Factor Xa inhibitor with a predictable pharmacokinetic profile, is a candidate to address this medical need. The UNIVERSE study is a prospective, open-label, active-controlled, multicenter study in children 2 to 8 years of age who have single ventricle physiology and had the Fontan procedure within the 4 months preceding enrollment. This study consists of 2 parts. In Part A, rivaroxaban pharmacokinetics, pharmacodynamics, safety, and tolerability are assessed to validate the pediatric dosing selected. In Part B, safety and efficacy of rivaroxaban versus acetylsalicylic acid are evaluated for thromboprophylaxis in children post-Fontan procedure. Children in each part will receive study drug for 12 months. Part A has been completed with 12 children enrolled. Enrollment into Part B is currently ongoing. The UNIVERSE study aims to provide dosing, pharmacokinetics/pharmacodynamics, safety, and efficacy information on the use of rivaroxaban, an oral anticoagulant, versus acetylsalicylic acid, an antiplatelet agent, in children with single ventricle physiology after the Fontan procedure. [\hyperlink{Rivaroxaban}{PMID: 31132584}, Liza Miriam Pina et al., 2019]

\hypertarget{pmid_37390311}{T}he direct oral anticoagulants (DOACs), rivaroxaban and dabigatran are newly licensed for the treatment and prevention of venous thromboembolism (VTE) in children and mark a renaissance in pediatric anticoagulation management. They provide a convenient option over standard-of-care anticoagulants (heparins, fondaparinux and vitamin K antagonists) due to their oral route of administration, child-friendly formulations, and significant reduction in monitoring. However, limitations related to therapeutic monitoring when needed and the lack of approved reversal agents for DOACs in children raise some safety concerns. There is accumulating experience of safety and efficacy of DOACs in the adults for a broad scope of indications, however the cumulative experience of using DOACs in pediatrics, specifically for those with coexisting chronic illnesses is sparse. Consequently, clinicians must often rely on their experience in treating VTE and extrapolation from adult data while using DOACs in these children. In this edition of "How I treat" the authors' share their experience of managing 4 scenarios that hematologists are likely to encounter in their day-to-day practice. Topics addressed include (1) appropriateness of indication; (2) use in special populations of children; (3) considerations for laboratory monitoring; (4) transition between anticoagulants; (5) major drug interactions; (6) perioperative management; and (7) anticoagulation reversal. [\hyperlink{Rivaroxaban}{PMID: 37390311}, Rukhmi Bhat et al., 2023]

\hypertarget{pmid_34171956}{V}enous thromboembolism in children is rare, but the incidence has increased sharply during the last years. The standard of care for treating this disease consists of warfarin, unfractionated heparin, low-molecular-weight heparins and fondaparinux. Lately, the usage of rivaroxaban (direct oral anticoagulant) was officially approved. According to a recent study, treatment with rivaroxaban resulted in a similarly low recurrence risk and reduced thrombotic burden without increased risk of bleeding. The usage of direct oral anticoagulants could overcome the limitation of currently used care (mainly the necessity of regular laboratory monitoring and parenteral application) while providing similar efficacy and safety to treat venous thromboembolism in children. [\hyperlink{Rivaroxaban}{PMID: 34171956}, Eliška Boženková et al., 2021]

\hypertarget{pmid_17595891}{R}ivaroxaban (BAY 59-7939) is an oral, direct Factor Xa (FXa) inhibitor being developed for the prevention and treatment of thromboembolic disorders. This analysis aimed to define population models for the pharmacokinetics (PK) and pharmacodynamics (PD) ofrivaroxaban in healthy males. Non-linear, mixed-effect modeling was used to analyze rivaroxaban plasma concentration and PD data (FXa activity and clotting tests) from subjects in a phase I, multiple-ascending-dose study. Subjects received 5 mg rivaroxaban once, twice or three times daily, or 10, 20 or 30 mg rivaroxaban twice daily. The population PK of rivaroxaban were well described by an oral, two-compartment model with first-order absorption and elimination from the central compartment. Population mean estimates for apparent oral clearance and volume of distribution for the central compartment were 9.2 1/h and 55 1, respectively, with moderate inter-individual variability (17.4\% and 30.7\%, respectively). Total volume of distribution for rivaroxaban at steady state was approximately 70 1. Residual (unexplained) variability was 25\%. FXa activity correlated with rivaroxaban plasma concentrations following an inhibitory Emax model; prothrombin time (PT) and rivaroxaban plasma concentrations correlated with a linear model, with a slope of 4.6 s/(100 microg/1). Inter-individual variability was low for the correlation with PT. The models derived were used to define sampling windows for population PK/PD modeling in Phase II studies. This analysis confirms that rivaroxaban has predictable, dose-proportional PK and PD. The linear correlation between rivaroxaban plasma concentrations and PT suggests that this test might be useful to assess rivaroxaban exposure in patients, if required. [\hyperlink{Rivaroxaban}{PMID: 17595891}, W Mueck et al., 2007]

\section*{Baloxavir Marboxil}
\subsection*{Result}
\subsubsection*{Answer}

Ages <1 year: Unknown
Ages 1 to <12 years: Yes
Ages 12-17 years: Yes

\subsubsection*{{Explanation}}
\hypertarget{Baloxavir Marboxil}
Based on the available abstracts, there are multiple targeted studies evaluating the safety of Baloxavir Marboxil in children across various age ranges. Here is a summary of the evidence by age group:

Ages 1 to <12 years:
- The MiniSTONE-2 trial was a double-blind, randomized, active-controlled study specifically enrolling children aged 1 to <12 years with acute influenza. The primary endpoint was safety. The incidence of adverse events (AEs) was similar between baloxavir and oseltamivir groups, with no deaths, serious AEs, or hospitalizations reported. The most common AEs were mild gastrointestinal symptoms. The study concluded that oral baloxavir is well tolerated and effective in otherwise healthy children with acute influenza [\hyperlink{pmid_32516282}{PMID: 32516282}, Jeffrey Baker et al., 2020].
- A post hoc analysis of MiniSTONE-2 focused on children 5-11 years old, confirming a low and balanced incidence of AEs between baloxavir and oseltamivir, with no serious AEs or deaths. The safety and efficacy results were similar to the overall 1-<12 years population [\hyperlink{pmid_37595103}{PMID: 37595103}, Jeffery B Baker et al., 2023].
- Another post hoc pooled analysis of two open-label studies in Japanese children aged <6 years (n=56) and ≥6 to <12 years (n=81) found that adverse events were reported in 39.0\% (<6 years) and 39.5\% (≥6 to <12 years), with no serious safety concerns. The study concluded that single-dose baloxavir was safe and effective across all age groups, though higher rates of fever recurrence were observed in children <6 years with influenza B [\hyperlink{pmid_36681802}{PMID: 36681802}, Nobuo Hirotsu et al., 2023].
- A multicenter, open-label study in Japan assessed a higher dose (2 mg/kg) of baloxavir granules in children <20 kg (all aged ≤6 years). Adverse events were reported in 53.3\% of patients, most commonly nasopharyngitis, diarrhea, and upper respiratory tract infection, but no serious AEs were reported. The study concluded baloxavir was well tolerated [\hyperlink{pmid_34045119}{PMID: 34045119}, Takuhiro Sonoyama et al., 2021].
- Another open-label study of baloxavir granules in 33 children (all <20 kg, including <10 kg) reported AEs in 54.5\% of children, but no deaths, serious AEs, or discontinuations. The regimen was considered well tolerated [\hyperlink{pmid_32433222}{PMID: 32433222}, Takato Yokoyama et al., 2020].
- A large postmarketing surveillance study in Japan included patients of any age, with a subgroup analysis for children <12 years. Adverse drug reactions (ADRs) were more common in children <12 years (14.1\%) than adults (10.0\%), but almost all ADRs were non-serious and resolved. The study concluded baloxavir is well tolerated regardless of age [\hyperlink{pmid_32409018}{PMID: 32409018}, Masami Nakazawa et al., 2020].
- A healthcare database study in Japan compared baloxavir to neuraminidase inhibitors in children aged 7-15 years (median age 10). Baloxavir did not increase subsequent medical resource utilization, and no safety concerns were raised [\hyperlink{pmid_33608939}{PMID: 33608939}, Masato Takeuchi et al., 2021].
- Additional studies and reviews confirm that baloxavir has been studied in children and is considered safe and well tolerated in this population [\hyperlink{pmid_33268716}{PMID: 33268716}, Yu Norikoshi et al., 2020; \hyperlink{pmid_31250840}{PMID: 31250840}, S C Locke et al., 2019].

Ages 12-17 years (Adolescents):
- The CAPSTONE-1 trial included a subgroup analysis of adolescents aged 12-17 years. Baloxavir was associated with fewer adverse events than placebo (17.1\% vs 34.1\%), and no safety concerns were identified. The study supports baloxavir as a treatment option in adolescents [\hyperlink{pmid_33340316}{PMID: 33340316}, Simon Portsmouth et al., 2021].

Ages <1 year:
- None of the abstracts provided data on the safety of baloxavir in children under 1 year of age.

Summary:
- For children aged 1 to <12 years, multiple targeted studies affirm the safety of baloxavir marboxil, with no serious safety concerns identified.
- For adolescents aged 12-17 years, targeted studies also affirm safety.
- For children under 1 year, there is no evidence available in the abstracts reviewed, so safety is unknown.

\subsection*{Abstracts}
\hypertarget{pmid_32516282}{B}aloxavir marboxil (baloxavir) is a novel, cap-dependent endonuclease inhibitor that has previously demonstrated efficacy in the treatment of influenza in adults and adolescents. We assessed the safety and efficacy of baloxavir in otherwise healthy children with acute influenza. MiniSTONE-2 (Clinicaltrials.gov: NCT03629184) was a double-blind, randomized, active controlled trial enrolling children 1-<12 years old with a clinical diagnosis of influenza. Children were randomized 2:1 to receive either a single dose of oral baloxavir or oral oseltamivir twice daily for 5 days. The primary endpoint was incidence, severity and timing of adverse events (AEs); efficacy was a secondary endpoint. In total, 173 children were randomized and dosed, 115 to the baloxavir group and 58 to the oseltamivir group. Characteristics of participants were similar between treatment groups. Overall, 122 AEs were reported in 84 (48.6\%) children. Incidence of AEs was similar between baloxavir and oseltamivir groups (46.1\% vs. 53.4\%, respectively). The most common AEs were gastrointestinal (vomiting/diarrhea) in both groups [baloxavir: 12 children (10.4\%); oseltamivir: 10 children (17.2\%)]. No deaths, serious AEs or hospitalizations were reported. Median time (95\% confidence interval) to alleviation of signs and symptoms of influenza was similar between groups: 138.1 (116.6-163.2) hours with baloxavir versus 150.0 (115.0-165.7) hours with oseltamivir. Oral baloxavir is well tolerated and effective at alleviating symptoms in otherwise healthy children with acute influenza. Baloxavir provides a new therapeutic option with a simple oral dosing regimen. [\hyperlink{Baloxavir Marboxil}{PMID: 32516282}, Jeffrey Baker et al., 2020]

\hypertarget{pmid_34045119}{B}aloxavir marboxil is an oral anti-influenza drug with demonstrated safety and efficacy in pediatric patients when a 2\% granules formulation is administered at 1 mg/kg. This study assessed safety, effectiveness, and pharmacokinetics of a higher dose (2 mg/kg) of baloxavir marboxil 2\% granules in pediatric patients weighing <20 kg. This multicenter, open-label, noncontrolled study was conducted at 15 sites in Japan (January 2019-March 2020; JapicCTI-194577). Patients aged <12 years with confirmed influenza received a single oral dose of baloxavir marboxil at 2 mg/kg if body weight was <10 kg or 20 mg if ≥ 10 to <20 kg. Safety, pharmacokinetics, effectiveness (time to illness alleviation [TTIA] of influenza; time to resolution of fever; virus titer), and polymerase acidic protein (PA) substituted viruses were assessed over 22 days. 45 patients, all aged ≤6 years, were enrolled. Adverse events were reported in 24 (53.3\%) patients, most commonly nasopharyngitis, diarrhea, and upper respiratory tract infection. Median (95\% confidence interval [CI]) TTIA was 37.8 (27.5-46.7) hours; median (95\% CI) time to resolution of fever was 22.0 (20.2-28.6) hours. A >4 log decrease in mean viral titer occurred at day 2 and a subsequent temporary 1-2 log increase in patients with influenza A(H3N2) and B. Treatment-emergent PA/I38X-substituted virus was detected in 16/39 (41.0\%) patients, but no prolonged TTIA or time to resolution of fever was associated with its presence. Baloxavir granules administered at 2 mg/kg in children <20 kg were well tolerated, with symptom alleviation similar to 1 mg/kg. [\hyperlink{Baloxavir Marboxil}{PMID: 34045119}, Takuhiro Sonoyama et al., 2021]

\hypertarget{pmid_33608939}{B}aloxavir marboxil is a novel antiviral agent for influenza, introduced into clinical practice in 2018. A concern remains about the variant virus with reduced susceptibility after baloxavir exposure and its clinical consequences such as healthcare-seeking behavior. Using a healthcare database in Japan, we compared the medical resource use following baloxavir and neuraminidase inhibitors (NAIs) treatment among children aged 7-15 years. The study period was from December 2018 to March 2019. The primary endpoint was the composite of hospitalization, laboratory and radiological tests, and antibiotic use over 1-9 days of antiviral treatment. As exploratory analyses, secondary outcomes being each single component of the primary composite were assessed and subgroup analyses comparing baloxavir with each NAI were done. Data from 115 867 prescriptions in 115 238 children were analyzed (median age: 10 years; severe influenza risk in 26\%; baloxavir accounting for 43\%). Overall, baloxavir use did not increase subsequent medical resource utilization in the composite endpoint (adjusted odds ratio [aOR]: 1.04; 95\% confidence interval [CI]: 0.99-1.09; P = 0.14), as were likelihoods of other secondary outcomes. In the subgroup analysis, baloxavir use was associated with higher medical resource use than oseltamivir (aOR: 1.21; 95\% CI: 1.13-1.31; P < 0.001) and lower resource use than zanamivir (aOR: 0.93; 95\% CI 0.86-1.00; P = 0.040). Based on a single-year experience in Japan, prescribing baloxavir rather than NAIs did not increase medical resource utilization within 9 days of treatment, except in one exploratory comparison with oseltamivir. [\hyperlink{Baloxavir Marboxil}{PMID: 33608939}, Masato Takeuchi et al., 2021] miniSTONE-2 (NCT03629184) was a global, phase 3, randomized, controlled study that investigated the safety and efficacy of single-dose baloxavir marboxil in otherwise healthy children 1-<12 years of age and showed a positive risk-benefit profile. This post hoc analysis evaluated the safety and efficacy of baloxavir versus oseltamivir in children 5-11 years old with influenza. Children received single-dose baloxavir or twice-daily oseltamivir for 5 days. Safety was the primary objective. Efficacy and virological outcomes included time to alleviation of symptoms, duration of fever and time to cessation of viral shedding by titer. Data were summarized descriptively. Ninety-four children 5-11 years old were included (61 baloxavir and 33 oseltamivir). Baseline characteristics were similar between the groups. The incidence of adverse events was balanced and low in both treatment groups, with the most common being vomiting (baloxavir 5\% vs. oseltamivir 18\%), diarrhea (5\% vs. 0\%) and otitis media (0\% vs. 5\%). No serious adverse events or deaths occurred. Median (95\% CI) time to alleviation of symptoms with baloxavir was 138.4 hours (116.7-163.4) versus 126.1 hours (95.9-165.7) for oseltamivir; duration of fever was comparable between groups [41.2 hours (23.5-51.4) vs. 51.3 hours (30.7-56.8), respectively]. Median time to cessation of viral shedding was shorter in the baloxavir group versus oseltamivir (1 vs. ≈3 days). Safety, efficacy and virological results in children 5-11 years were similar to those from the overall study population 1-<12 years of age. Single-dose baloxavir provides an additional treatment option for pediatric patients 5-11 years old with influenza. [\hyperlink{Baloxavir Marboxil}{PMID: 33608939}, Jeffery B Baker et al., 2023]

\hypertarget{pmid_36681802}{A}nti-influenza treatment is important for children and is recommended in many countries. This study assessed safety, clinical, and virologic outcomes of baloxavir marboxil (baloxavir) treatment in children based on age and influenza virus type/subtype. This was a post hoc pooled analysis of two open-label non-controlled studies of a single weight-based oral dose of baloxavir (day 1) in influenza virus-infected Japanese patients aged < 6 years (n = 56) and ≥ 6 to < 12 years (n = 81). Safety, time to illness alleviation (TTIA), time to resolution of fever (TTRF), recurrence of influenza illness symptoms and fever (after day 4), virus titer, and outcomes by polymerase acidic protein variants at position I38 (PA/I38X) were evaluated. Adverse events were reported in 39.0 and 39.5\% of patients < 6 years and ≥ 6 to < 12 years, respectively. Median (95\% confidence interval) TTIA was 43.2 (36.3-68.4) and 45.4 (38.9-61.0) hours, and TTRF was 32.2 (26.8-37.8) and 20.7 (19.2-23.8) hours, for patients < 6 years and ≥ 6 to < 12 years, respectively. Symptom and fever recurrence was more common in patients < 6 years with influenza B (54.5 and 50.0\%, respectively) compared with older patients (0 and 25.0\%, respectively). Virus titers declined (day 2) for both age groups. Transient virus titer increase and PA/I38X-variants were more common for patients < 6 years. The safety and effectiveness of single-dose baloxavir were observed in children across all age groups and influenza virus types. Higher rates of fever recurrence and transient virus titer increase were observed in children < 6 years. Japan Pharmaceutical Information Center Clinical Trials Information JapicCTI-163,417 (registered 02 November 2016) and JapicCTI-173,811 (registered 15 December 2017). [\hyperlink{Baloxavir Marboxil}{PMID: 36681802}, Nobuo Hirotsu et al., 2023]

\hypertarget{pmid_32409018}{B}aloxavir marboxil is an oral anti-influenza drug that inhibits the cap-dependent endonuclease of the virus polymerase acidic protein. In clinical trials, baloxavir reduced the time to alleviation of influenza symptoms and time to resolution of fever in adults, adolescents, and children. The purpose of this study is to collect data on the safety and effectiveness of baloxavir when used in clinical practice. This postmarketing surveillance (clinicaltrials.jp; JapicCTI-183882), conducted at 688 Japanese hospitals or clinics (March 2018 to March 2019), enrolled patients of any age with influenza A or B infection who received a single, weight-based dose of baloxavir. Adverse drug reactions (ADRs) were seen in 11.2\% of 3094 patients during the 7-day observation period; the most common ADR was diarrhea (6.1\%). ADRs were more common in children aged <12 years (14.1\%) than in adults (10.0\%). Almost all ADRs were non-serious (98.9\%) and were recovered or recovering (96.7\%). Median time to alleviation of symptoms (N = 2884) was 2.5 days (overall, influenza A, and influenza B groups). Median time to resolution of fever (N = 2946) was 1.5 days (overall, influenza A, and influenza B groups). Biphasic fever (increased temperature after previous fever resolution) was seen in 6.7\% of patients overall and 28.6\% of patients <6 years infected with influenza B, similar to rates published elsewhere with other influenza drugs and in untreated influenza. This postmarketing surveillance of >3000 patients suggests that baloxavir is well tolerated and effective regardless of patient age or influenza virus type. [\hyperlink{Baloxavir Marboxil}{PMID: 32409018}, Masami Nakazawa et al., 2020]

\hypertarget{pmid_31250840}{B}aloxavir marboxil is a newly approved antiviral agent with activity against influenza via a novel mechanism of action of inhibition of cap-dependent endonuclease (CEN). The novel agent was approved in October of 2018 in the United States for the treatment of acute uncomplicated influenza A and B in patients aged 12 years or older. Baloxavir is given as a single weight-based dose of 40 mg orally once for patients weighing less than 80 kg and 80 mg orally once for those weighing 80 kg or more within 48 hours of symptom onset. In comparison with current therapy, baloxavir is as effective in decreasing time to symptom alleviation as the drug of choice, oseltamivir, and significantly reduces viral load 1 day after treatment compared with placebo and oseltamivir. In safety analyses baloxavir was well tolerated with only mild adverse events reported (nausea, headache, diarrhea, bronchitis, nasopharyngitis), thus providing a safe and reliable alternative option to current therapy for acute uncomplicated influenza. Further studies are being conducted to evaluate the use of baloxavir in additional patient populations including pediatric patients less than 12 years of age and patients who are at high risk of complications related to influenza. [\hyperlink{Baloxavir Marboxil}{PMID: 31250840}, S C Locke et al., 2019]

\hypertarget{pmid_30998942}{B}aloxavir marboxil is a prodrug of baloxavir acid, an inhibitor of cap-dependent endonuclease, and suppresses the replication of influenza virus. The aim of this study was to investigate its pharmacokinetic characteristics in Japanese pediatrics. Population pharmacokinetic analysis was conducted for baloxavir acid with 328 plasma concentration data points in a clinical study of 107 Japanese pediatric influenza patients. The plasma baloxavir acid concentration profiles were well captured by a 2-compartment model including first-order absorption and lag time. Body weight was considered to be the most crucial covariate, which affects clearance and volume of distribution. The body weight-based dose regimen (10 mg for 10 kg to less than 20 kg pediatrics, 20 mg for 20 kg to less than 40 kg pediatrics, and 40 mg for at least 40 kg pediatrics) for Japanese pediatrics can provide comparable exposure to baloxavir acid to that for adults. In conclusion, the population pharmacokinetic model would be useful to comprehend the characteristics of baloxavir acid pharmacokinetics in pediatric patients. [\hyperlink{Baloxavir Marboxil}{PMID: 30998942}, Hiroki Koshimichi et al., 2019]

\hypertarget{pmid_35176206}{B}aloxavir marboxil is an endonuclease inhibitor indicated for the treatment of influenza in patients ≥12 years. No data exist for Chinese patients in global studies. This randomized, open-label, phase I study evaluated the pharmacokinetics (PK) and safety of baloxavir marboxil in healthy Chinese volunteers and was used to anticipate efficacy in Chinese patients. Patients received a single oral dose of baloxavir marboxil (40 or 80 mg [1:1]). Serial blood samples were collected predose and at various timepoints up to 14 days postdose. Baloxavir marboxil and acid plasma concentrations were determined by liquid chromatography tandem mass spectrometry. PK parameters of baloxavir acid were estimated by noncompartmental analysis. Adverse events (AEs) were recorded. Time to alleviation of symptoms (TTAS) was simulated for otherwise healthy (OwH) and high-risk (HR) Chinese and Asian patients. Thirty-two male patients received baloxavir marboxil. Baloxavir acid plasma concentration peaked 4 h postdose. Mean maximum concentration (C [\hyperlink{Baloxavir Marboxil}{PMID: 35176206}, Yanmei Liu et al., 2022] The novel anti-influenza virus agent baloxavir marboxil is a selective inhibitor of an influenza cap-dependent endonuclease. Although a single oral dose in tablet form of baloxavir marboxil is expected to improve drug compliance and rapidly reduce viral titers for pediatric patients with influenza, there is a concern that baloxavir marboxil-resistant influenza A variants could be generated. In this study, we investigated the frequency of prescription and pharmacy revisits for baloxavir marboxil at an outpatient clinic compared with that of neuraminidase inhibitors in pediatric patients with influenza. A total of 475 pediatric patients who were infected with the influenza virus visited the pharmacy between December 2019 and March 2020. Baloxavir marboxil (n = 149), oseltamivir (n = 161) and laninamivir (n = 162) were mainly prescribed and only a few patients were treated with peramivir (n = 2) or zanamivir (n = 1). Baloxavir marboxil-, oseltamivir- and laninamivir-treated pediatric patients were enrolled, and a log-rank test showed that the revisits of pediatric patients who were taking baloxavir marboxil was lower than those for oseltamivir (p < 0.001). Moreover, Cox proportional hazards models also revealed that baloxavir marboxil decreased the risk of revisits in comparison to oseltamivir (hazard ratio 0.28, 95\% confidence interval 0.11-0.70, p = 0.006), while no difference was found between laninamivir and baloxavir marboxil. Although there is a need to acquire appropriate and relevant information concerning resistant viruses, our results suggest that baloxavir marboxil may be a useful drug for treating pediatric patients with influenza infections. [\hyperlink{Baloxavir Marboxil}{PMID: 35176206}, Yu Norikoshi et al., 2020]

\hypertarget{pmid_33340316}{B}aloxavir marboxil has demonstrated safety and efficacy in treating adult and adolescent outpatients with acute influenza (CAPSTONE-1 trial). Here, we report a subgroup analysis of outcomes in adolescents from the trial. CAPSTONE-1 was a randomized, double-blind, placebo-controlled study. Eligible adolescent outpatients (aged 12-17 years of age) were randomized in a ratio of 2:1 to a single dose of baloxavir 40/80 mg if less than/greater than or equal to 80 kg or placebo. The main outcomes were the time to alleviation of symptoms (TTAS), duration of infectious virus detection, and incidence of adverse events (AEs). Among 117 adolescent patients, 90 (77\%) comprised the intent-to-treat infected population (63 baloxavir and 27 placebo; 88.9\% A(H3N2)). The median TTAS was 38.6 hours shorter (95\% confidence interval: -2.6, 68.4) in the baloxavir group compared with placebo (median TTAS, 54.1 hours vs 92.7 hours, P = .0055). The median time to sustained cessation of infectious virus detection was 72.0 hours for baloxavir compared with 120.0 hours for placebo recipients (P < .0001). Treatment-emergent PA/I38X-substituted viruses were detected in 5 of the 51 (9.8\%) baloxavir recipients. In the safety population (76 baloxavir and 41 placebo), AEs were less common in baloxavir than placebo recipients (17.1\% vs 34.1\%; P = .0421). In the baloxavir group, no AEs except for diarrhea were reported in 2 or more patients. Baloxavir demonstrated clinical and virologic efficacy in the otherwise healthy adolescents with acute influenza compared with placebo. There were no safety concerns identified. These results were similar to the adult population in CAPSTONE-1 and support baloxavir as a treatment option in adolescents. [\hyperlink{Baloxavir Marboxil}{PMID: 33340316}, Simon Portsmouth et al., 2021]

\hypertarget{pmid_30288682}{B}aloxavir marboxil, a prodrug that is metabolized to baloxavir acid, suppresses viral replication by inhibiting cap-dependent endonuclease. This first-in-human phase I study evaluated the safety, tolerability, and pharmacokinetics of baloxavir marboxil/baloxavir acid in healthy Japanese volunteers (Study 1), while food effects were evaluated in a separate phase I, crossover study in healthy Japanese volunteers (Study 2). Study 1 participants were randomized to single-dose oral baloxavir marboxil (6, 20, 40, 60, or 80 mg; n = 6 per dose) or placebo (n = 10), while Study 2 participants (n = 15) received single-dose oral baloxavir marboxil 20 mg in fasted, fed, and before-meal states. Baloxavir marboxil was well tolerated; there were few treatment-emergent adverse events and no serious adverse events/deaths. The mean plasma baloxavir acid concentration 24 h after single-dose (C Single-dose oral baloxavir marboxil was well tolerated, had a favorable safety profile, and had favorable pharmacokinetic characteristics, including a long half-life, supporting single oral dosing. The baloxavir acid area under the plasma concentration-time curve decreased with food intake by approximately 40\%. [\hyperlink{Baloxavir Marboxil}{PMID: 30288682}, Hiroki Koshimichi et al., 2018]

\hypertarget{pmid_30676002}{B}aloxavir marboxil (5-hydroxy-4-pyridone-3-carboxyl acid) is a new antiviral drug with special efficacy on influenza viruses that acts by inhibiting the cap-dependent endonuclease required for its replication. It is the first representative of the so-called inhibitors of influenza-like PB2. It has shown efficacy against influenza viruses A and B and most strains of animal origin (avian flu). Clinical trials conducted in healthy patients between 12 and 64 years without pathologies and not hospitalized (mild flu) have shown a reduction in the duration of symptoms similar to that obtained by oseltamivir. However, baloxavir is a much more potent inhibitor of viral replication than this drug. It has been shown as a safe and well tolerated drug. A single dose of 40-80 mg is administered the first 48 hours after onset of symptoms. In these trials, strains with moderate sensitivity (PA / I38T mutants) were detected in 2.2\% of influenza A (H1N1) pdm09 and in 9.7\% of influenza A (H3N2). Although these data could be a good drug to treat mild or moderate influenza, requiring trials in severe influenza and patients with chronic diseases to establish their true clinical utility. [\hyperlink{Baloxavir Marboxil}{PMID: 30676002}, J Reina et al., 2019]

\hypertarget{pmid_30476172}{B}aloxavir marboxil (formerly S-033188) is a first-in-class, orally available, cap-dependent endonuclease inhibitor licensed in Japan and the USA for the treatment of influenza virus infection. We evaluated the efficacy of delayed oral treatment with baloxavir marboxil in combination with a neuraminidase inhibitor in a mouse model of lethal influenza virus infection. The inhibitory potency of baloxavir acid (the active form of baloxavir marboxil) in combination with neuraminidase inhibitors was tested in vitro. The therapeutic effects of baloxavir marboxil and oseltamivir phosphate, or combinations thereof, were evaluated in mice lethally infected with influenza virus A/PR/8/34; treatments started 96 h post-infection. Combinations of baloxavir acid and neuraminidase inhibitor exhibited synergistic potency against viral replication by means of inhibition of cytopathic effects in vitro. In mice, baloxavir marboxil monotherapy (15 or 50 mg/kg twice daily) significantly and dose-dependently reduced virus titre 24 h after administration and completely prevented mortality, whereas oseltamivir phosphate treatments were not as effective. In this model, a suboptimal dose of baloxavir marboxil (0.5 mg/kg twice daily) in combination with oseltamivir phosphate provided additional efficacy compared with monotherapy in terms of virus-induced mortality, elevation of cytokine/chemokine levels and pathological changes in the lung. Baloxavir marboxil monotherapy with 96 h-delayed oral dosing achieved drastic reductions in virus titre, inflammatory response and mortality in a mouse model. Combination treatment with baloxavir acid and oseltamivir acid in vitro and baloxavir marboxil and oseltamivir phosphate in mice produced synergistic responses against influenza virus infections, suggesting that treating humans with the combination may be beneficial. [\hyperlink{Baloxavir Marboxil}{PMID: 30476172}, Keita Fukao et al., 2019]

\hypertarget{pmid_33061543}{B}aloxavir marboxil, a recently developed antiviral drug, has been used to treat influenza in some countries including Japan. The aim of this study was to determine the clinical efficacy of the drug, which currently remains unclear. Overall, 43 adult patients with seasonal influenza who visited the outpatient clinic of Teikyo University Hospital in Tokyo during the winter of 2018-2019 were enrolled. Of them, 14, 13, and 16 were prescribed baloxavir marboxil (40 or 80 mg once), oseltamivir (75 mg twice daily for 5 days), and laninamivir (40 mg once), respectively. A questionnaire was used to collect data about symptoms, and the Medical Outcome Study 8-Items Short Form Health Survey was employed to examine health-related quality-of-life (HRQOL) before and 7 days after admission. The main study endpoints included time to defervescence and the extent of improvement in HRQOL after treatment initiation. The data were analyzed with Welch's  No significant differences in clinical background characteristics were observed among the patients. The mean time to defervescence in the baloxavir group (median [interquartile range]; 1.0 [1.0-2.0] days) was significantly shorter than that in the laninamivir group (2.0 [1.5-3.5] days; p=0.0322). No significant differences in mean time to defervescence, change in HRQOL, and time for resolution of other symptoms were observed between the groups. The results suggest that baloxavir marboxil has a better antipyretic effect than oseltamivir and laninamivir. Moreover, baloxavir marboxil might be clinically more valuable than the other two drugs owing to higher medication adherence among patients. [\hyperlink{Baloxavir Marboxil}{PMID: 33061543}, Yusuke Yoshino et al., 2020]

\hypertarget{pmid_35056085}{B}aloxavir marboxil is a new drug developed in Japan by Shionogi to treat seasonal flu infection. This cap-dependent endonuclease inhibitor is a prodrug that releases the biologically active baloxavir acid. This new medicine has been marketed in Japan, the USA and Europe. It is well tolerated (more than 1\% of the patients experienced diarrhea, bronchitis, nausea, nasopharyngitis, and headache), and both influenza A and B viruses are sensitive, although the B strain is more resistant due to variations in the amino acid residues in the binding site. The drug is now in post-marketing pharmacovigilance phase, and its interest will be especially re-evaluated in the future during the annual flu outbreaks. It has been also introduced in a recent clinical trial against COVID-19 with favipiravir. [\hyperlink{Baloxavir Marboxil}{PMID: 35056085}, François Dufrasne et al., 2021]

\hypertarget{pmid_33858277}{B}aloxavir marboxil (baloxavir) is a single-dose antiviral which was previously found to be a cost-effective alternative to laninamivir in otherwise healthy adults in Japan. This study aimed at investigating the cost-effectiveness of baloxavir versus laninamivir in patients with influenza at high risk for complications. A decision tree was utilized to estimate costs and health gains associated with the use of antivirals. A lifetime horizon was applied to capture the long-term impact of influenza complications, and other events with associated costs and health outcomes were accounted for one influenza season. The study population was stratified into three categories: adolescents and non-elderly adults with high-risk conditions (HRC), elderly without other HRC, and elderly with other HRC. The cost-effectiveness was assessed from a public healthcare payer's perspective. The duration of influenza symptoms, probabilities of complications and probabilities of adverse events were obtained from a clinical trial and network meta-analysis. The costs of influenza and adverse events management were derived from the JammNet claims database. Utility values were informed by the clinical trial data and literature. Sensitivity analyses were also performed. The baloxavir strategy was associated with higher costs (+¥144) and higher quality-adjusted life-years (QALYs) in adults with HRC, elderly without HRC and elderly with HRC (+0.00078, +0.00183 and +0.00350 respectively). The overall incremental cost/QALY for baloxavir versus laninamivir was ¥68,855, which was below the willingness-to-pay threshold of ¥5 million/QALY gained. Key drivers of the model results were the probability of pneumonia and bronchitis. The probability of baloxavir being cost-effective was 72\%. This study suggests that influenza treatment with baloxavir is cost-effective compared with laninamivir in the adult high-risk population in Japan. [\hyperlink{Baloxavir Marboxil}{PMID: 33858277}, Mariia Dronova et al., 2021]

\hypertarget{pmid_34664769}{B}aloxavir marboxil, a novel influenza therapeutic agent, is a prodrug rapidly metabolized into its active form, baloxavir acid, which inhibits cap-dependent endonuclease. This study evaluated the pharmacokinetics (PKs) and safety of baloxavir acid in healthy Korean subjects and compared them with published data in Japanese subjects. This open-label and single-ascending dose study was conducted in 30 Korean male subjects, with a single oral dose of baloxavir marboxil (20, 40, or 80 mg) administered to eight subjects each; additionally, 80 mg was administered to six subjects (body weight >80 kg). Noncompartmental and population PK analyses were performed, and results were compared with those of Japanese subjects. Appropriateness of the body weight-based dosing regimen was evaluated by simulation. PK profiles of baloxavir acid revealed multicompartment behavior with a long half-life (80.8-98.3 h), demonstrating a dose-proportional increase. Baloxavir acid reached peak plasma concentration from 3.5 to 4.0 h postdosing. Body weight was identified as a significant covariate of apparent oral clearance and apparent volume of distribution, which was similar to that observed in Japanese subjects. Body weight-adjusted analysis revealed that exposure to baloxavir acid did not significantly differ between Korean and Japanese subjects. Simulated exposures to baloxavir acid demonstrated that the body weight-based dosing regimen for baloxavir marboxil was appropriate. Based on a PK study, clinical data including dosing regimen developed in Japan were adequately extrapolated to Korea, supporting the approval of baloxavir marboxil in Korean as a new treatment option for influenza. [\hyperlink{Baloxavir Marboxil}{PMID: 34664769}, Yun Kim et al., 2022]

\hypertarget{pmid_32433222}{A} granule formulation of baloxavir marboxil, a selective inhibitor of influenza cap-dependent endonuclease, was newly developed for children with difficulty swallowing tablets. A multicenter open-label study was conducted during the 2017-2018 influenza season to assess the safety, pharmacokinetics and clinical/virologic outcomes of single, oral, weight-based doses of baloxavir granules in Japanese children infected with influenza virus. The primary clinical endpoint was the time to illness alleviation of influenza. All 33 enrolled children completed the study and received baloxavir (1 mg/kg for 12 children weighing <10 kg, 10 mg for 21 children weighing 10 to <20 kg). Detected viruses were influenza B (36.4\%), A(H1N1)pdm09 (33.3\%) and A(H3N2) (27.3\%). Adverse events (AEs) were reported in 54.5\% of children. No deaths, serious AEs or AEs leading to discontinuation were reported. The mean (SD) plasma concentrations of baloxavir acid at 24 hours post-dose were 72.8 (24.0) and 51.3 (19.3) ng/mL in the 1-mg/kg and 10-mg dose groups, respectively. The median time to illness alleviation (95\% confidence interval) was 45.3 (28.5-64.1) hours. A >4-log decrease in infectious viral titer occurred on day 2 and a temporary 2-log increase on day 4. Polymerase acidic protein/I38T/M-substituted viruses were detected in 5 children infected with influenza A, but none with influenza B. Baloxavir granules and the weight-based dose regimen were considered to be well tolerated in children, with rapid influenza virus reduction and associated symptom alleviation. Evidence of baloxavir activity against influenza B was observed, but further data are required for confirmation. [\hyperlink{Baloxavir Marboxil}{PMID: 32433222}, Takato Yokoyama et al., 2020]

\hypertarget{pmid_32020174}{B}aloxavir marboxil (formerly S-033188) is a prodrug of baloxavir acid (S-033447) and inhibits cap-dependent endonuclease, an essential protein involved in the initiation of viral transcription by cleaving capped mRNA bound to PB2. Its adverse event profile is comparable to oseltamivir but is still vulnerable to resistance. The single-dose baloxavir marboxil is an appealing antiviral regimen for the treatment of influenza among outpatients when compared with longer, twice-daily regimens of oral and inhaled neuraminidase inhibitors. This review focuses on the mode of action, antiviral activity, pharmacokinetics, clinical indications, and safety profiles of this drug. Considerations for formulary addition and its place in therapy are also discussed. [\hyperlink{Baloxavir Marboxil}{PMID: 32020174}, George M Abraham et al., 2020]

\hypertarget{pmid_37052118}{B}aloxavir marboxil (BXM) is a polymerase acidic endonuclease inhibitor used as an antiviral drug. A simple, reliable, and robust liquid chromatographic method was developed and validated per International Council for Harmonisation of Technical Requirements for Pharmaceuticals for Human Use (ICH) Q2(R1) for estimating the assay and impurities of BXM in drug substance and pharmaceutical formulations. The chromatographic separation was carried out on C [\hyperlink{Baloxavir Marboxil}{PMID: 37052118}, Bhujanga Rao Nagulancha et al., 2023] Baloxavir marboxil (baloxavir) is the first cap-dependent endonuclease inhibitor being studied for the treatment of influenza in single oral dosing regimen. This network meta-analysis (NMA) evaluated the efficacy and safety of baloxavir compared to other antivirals for influenza in otherwise healthy patients. A systematic literature review was performed on 14 November 2016 in Medline, Embase, CENTRAL, and ICHUSHI to identify randomized controlled trials assessing antivirals for influenza. A NMA including 22 trials was performed to compare the efficacy and safety of baloxavir with other antivirals. The time to alleviation of all symptoms was significantly shorter for baloxavir compared to zanamivir (difference in median time 19.96 h; 95\% CrI [3.23, 39.07]). The time to cessation of viral shedding was significantly shorter for baloxavir than zanamivir and oseltamivir (47.00 h; 95\% CrI [28.18, 73.86] and 56.03 h [33.74, 87.86], respectively). The mean decline in virus titer from baseline to 24 h was significantly greater for baloxavir than for the other drugs. Other differences in efficacy outcomes were not significant. No significant differences were found between baloxavir and the other antivirals for safety, except total drug-related adverse events where baloxavir demonstrated a decrease compared to oseltamivir and laninamivir. The NMA suggests that baloxavir demonstrated better or similar efficacy results compared to other antivirals with a comparable safety profile. Baloxavir led to a significant decrease in viral titer versus zanamivir, oseltamivir and peramivir and decreased viral shedding versus zanamivir and oseltamivir. [\hyperlink{Baloxavir Marboxil}{PMID: 37052118}, V Taieb et al., 2020]

\hypertarget{pmid_30203386}{B}aloxavir marboxil is a prodrug that is metabolized to baloxavir acid, which suppresses viral replication by inhibiting cap-dependent endonuclease with a single oral administration. As the mode of action of baloxavir marboxil is different from that of neuraminidase inhibitors, such as oseltamivir, combination treatment with these drugs can be a treatment option, particularly for severe influenza infection. The aim of this study was to assess the drug-drug interaction between baloxavir marboxil and oseltamivir. Eighteen healthy adult subjects received three treatments in a crossover fashion: single administration of baloxavir marboxil 40 mg alone, repeated twice-daily administration of oseltamivir at 75 mg for 5 days, or single administration of baloxavir marboxil at 40 mg in combination with repeated twice-daily administration of oseltamivir at 75 mg for 5 days. The ratios (90\% confidence intervals) of maximum plasma concentration and area under the plasma concentration-time curve of baloxavir acid after co-administration compared to baloxavir marboxil alone were 1.03 (0.92-1.15) and 1.01 (0.96-1.06), respectively. The ratios (90\% confidence intervals) of maximum plasma concentration and area under the plasma concentration-time curve of oseltamivir carboxylate, the active form of oseltamivir, after co-administration compared to oseltamivir alone were 0.96 (0.93-1.00) and 0.99 (0.96-1.01), respectively, at steady state on day 5. Treatment-emergent adverse events reported were mild and not considered to be related to the study drug. The lack of a clinically meaningful drug-drug interaction between baloxavir marboxil and oseltamivir has been established. [\hyperlink{Baloxavir Marboxil}{PMID: 30203386}, Nao Kawaguchi et al., 2018]

\hypertarget{pmid_32045100}{B}aloxavir marboxil is a novel endonuclease inhibitor licensed for treatment of otherwise healthy or high-risk individuals infected with influenza. Viruses with reduced baloxavir susceptibility due to amino acid substitutions at residue 38 of the PA have been detected in some individuals following treatment. Here, we describe a genotypic pyrosequencing method that can be used to rapidly screen circulating influenza A and B viruses for substitutions in the PA/I38 codon and to quantify mixed viral populations. This method is suitable for surveillance of baloxavir susceptibility and to analyse samples from hospitalised patients undergoing baloxavir treatment to aid in clinical decision making. [\hyperlink{Baloxavir Marboxil}{PMID: 32045100}, Paulina Koszalka et al., 2020]

\hypertarget{pmid_30684563}{B}aloxavir marboxil (baloxavir) is an antiviral drug that inhibits the viral "cap-snatching" step in virus RNA transcription initiation. In Phase 2 study, baloxavir significantly shortend the time to alleviation of symptoms (TTAS) and showed significantly greater reduction in influenza virus titer compared with placebo. This provides additional outcomes including efficacy against virus types/subtypes and pharmacokinetic/pharmacodynamic (PK/PD) analysis. Subgroup analyses by virus types/subtype were conducted for the primary and key secondary endpoints. Blood samples were collected totally at 2 to 5 points including Day 2 after baloxavir dosing. PK/PD analyses were conducted for TTAS and change in virus titer using the liner model and the E The median TTAS in each baloxavir dose group was significantly shorter than in the placebo group for patients with A/H1N1pdm virus, and was numerically shorter than the placebo group for patients with A/H3N2 and type B virus. Baloxavir significantly reduced virus titer within 1 day after treatment: for A/H1N1pdm, A/H3N2, and B virus, all 3 doses of baloxavir marboxil reduced virus titer on Day 2 to a greater extent than placebo. No clear PK/PD relationships were found for the TTAS, but the larger reduction in virus titer was observed in increasing C These results support that baloxavir marboxil will be effective against a range of virus types/subtypes. [\hyperlink{Baloxavir Marboxil}{PMID: 30684563}, Akira Watanabe et al., 2019]

\section*{Adapalene And Benzoyl Peroxide And Clindamycin Phosphate}
\subsection*{Result}
\subsubsection*{Answer}

Unknown

\subsubsection*{{Explanation}}
\hypertarget{Adapalene And Benzoyl Peroxide And Clindamycin Phosphate}
Based on the abstracts provided, there is no study that specifically evaluates the safety of the combination of Adapalene AND Benzoyl Peroxide AND Clindamycin Phosphate in children. 

- There are studies on the safety and efficacy of clindamycin phosphate and benzoyl peroxide combinations in adolescents (aged 12 years and older) and in adolescents with skin of color, showing favorable safety profiles in these age groups [\hyperlink{pmid_18805603}{PMID: 18805603}, Diane Thiboutot et al., 2008; \hyperlink{pmid_22777222}{PMID: 22777222}, Lawrence F Eichenfield et al., 2012]. However, these studies do not include adapalene in the combination.
- There is a study on adapalene 0.1\% and benzoyl peroxide 2.5\% gel in children aged 9-11 years, which found the combination to be well tolerated and effective in this age group [\hyperlink{pmid_23839175}{PMID: 23839175}, Lawrence F Eichenfield et al., 2013]. However, this study does not include clindamycin phosphate in the combination.
- There are studies on clindamycin phosphate and benzoyl peroxide at different concentrations in patients aged 12 years and older, but again, these do not include adapalene [\hyperlink{pmid_25226009}{PMID: 25226009}, David M Pariser et al., 2014; \hyperlink{pmid_26345297}{PMID: 26345297}, Guy Webster et al., 2015].
- There is no abstract that evaluates the safety of the triple combination (Adapalene AND Benzoyl Peroxide AND Clindamycin Phosphate) in children of any age group.

Therefore, based on the abstracts available, the safety of Adapalene AND Benzoyl Peroxide AND Clindamycin Phosphate in children is unknown, as no targeted study has been done in this population.

\subsection*{Abstracts}
\hypertarget{pmid_18805603}{W}e sought to evaluate efficacy, safety, and tolerability of a combination of clindamycin phosphate 1.2\% and benzoyl peroxide 2.5\% (clindamycin-BPO 2.5\%) aqueous gel in moderate to severe acne vulgaris. A total of 2813 patients, aged 12 years or older, were randomized to receive clindamycin-BPO 2.5\%, individual active ingredients, or vehicle in two identical, double-blind, controlled 12-week, 4-arm studies evaluating safety and efficacy (inflammatory and noninflammatory lesion counts) using Evaluator Global Severity Score and subject self-assessment. Clindamycin-BPO 2.5\% demonstrated statistical superiority to individual active ingredients and vehicle in reducing both inflammatory and noninflammatory lesions and acne severity. Visibly greater improvement was observed by patients with clindamycin-BPO 2.5\% as early as week 2. No substantive differences were seen in cutaneous tolerability among treatment groups and less than 1\% of patients discontinued treatment because of adverse events. Data from controlled studies may differ from clinical practice. Clindamycin-BPO 2.5\% provides statistically significant greater efficacy than individual active ingredients and vehicle with a highly favorable safety and tolerability profile. [\hyperlink{Adapalene And Benzoyl Peroxide And Clindamycin Phosphate}{PMID: 18805603}, Diane Thiboutot et al., 2008]

\hypertarget{pmid_23839175}{E}valuate the efficacy and safety of adapalene 0.1\%-benzoyl peroxide 2.5\% gel (adapalene-BPO) in patients 9-11 years old with acne vulgaris.<BR> Enrolled subjects were male or female, with a score of 3 (moderate) on the Investigator's Global Assessment (IGA) scale. Subjects were randomized to receive adapalene-BPO or vehicle once daily for up to 12 weeks. Efficacy was evaluated by success rate (percentage of subjects rated "clear" or "almost clear") at each visit, median percentage changes from baseline in total, inflammatory and non-inflammatory lesion counts at each visit, the Children's Dermatology Life Quality Index (C DLQI) at baseline and week 12, and the Parent Assessment of Acne at week 12. Safety was assessed through evaluations of adverse events (AEs) and local tolerability [erythema, scaling, dryness, and stinging/burning on scales ranging from 0 (none) to 3 (severe)].<BR> A total of 142 subjects were randomized to adapalene-BPO and 143 to vehicle. At study endpoint (week 12), adapalene-BPO was significantly superior to vehicle regarding treatment success (49.3\% vs 15.9\%, respectively), and regarding percentage reduction in total lesion counts (68.6\% vs 19.3\%), inflammatory (63.2\% vs 14.3\%), and non-inflammatory lesion counts (70.7\% vs 14.6\%) (all P<.001). More subjects using adapalene-BPO reported that their acne had no effect on their quality of life, and parents noted that their child's acne significantly improved. Adapalene-BPO was well tolerated, with mean tolerability scores less than 1 (mild).<BR> In preadolescents with acne, adapalene-BPO leads to significantly superior treatment success and lesion count reduction compared to vehicle. [\hyperlink{Adapalene And Benzoyl Peroxide And Clindamycin Phosphate}{PMID: 23839175}, Lawrence F Eichenfield et al., 2013]

\hypertarget{pmid_8374650}{A}dvances in pediatric anesthesia can contribute to improved care of children in other environments. As an example, drugs and dosages established in preoperative sedation of children provide a base for their application in sedation and pain relief of children undergoing painful procedures in the emergency unit, oncology treatment area, and radiology suite. Midazolam, ketamine, fentanyl, propofol, chloral hydrate, and pentobarbital are reviewed from the past year's pediatric literature. Adverse sequelae of sedation including hypoxemia and hypoventilation or apnea confirm the need for an individual whose responsibility is observation and support of the sedated child rather than performing the procedure, a principle that is the cornerstone of revised guidelines for the use of sedation in children. Monitoring techniques may similarly be developed in the operating suite then applied in emergency areas or critical care units. We examine a qualitative device for detecting carbon dioxide in the exhaled gases of an intubated child. [\hyperlink{Adapalene And Benzoyl Peroxide And Clindamycin Phosphate}{PMID: 8374650}, A M Broennle et al., 1993]

\hypertarget{pmid_9132194}{T}o evaluate the safety and efficacy of intranasal diamorphine as an analgesic for use in children in accident and emergency (A\&E). A prospective, randomised clinical trial with consecutive recruitment of patients aged between 3 and 16 years with clinically suspected limb fractures. One group received 0.1 mg/kg intranasal diamorphine, and the other group received 0.2 mg/kg intramuscular morphine. At 0, 5, 10, 20, and 30 minutes pain scores, Glasgow coma score, and peripheral oxygen saturations were recorded; parental acceptability was assessed at 30 minutes. 58 children were recruited, with complete data collection in 51 (88\%); the median summed decrease in pain score was better for intranasal diamorphine than intramuscular morphine (9 v 8), though this was not significant (P = 0.4, Mann-Whitney U test). The episode was recorded as "acceptable" in all parents whose child received intranasal diamorphine, compared with only 55\% of parents in the intramuscular morphine group (P < 0.0001, Fisher's exact test). There was no incidence of decreased peripheral oxygen saturation or depression in the level of consciousness in any patient. Intranasal diamorphine is an effective, safe, and acceptable method of analgesia for children requiring opiates in the A \& E department. [\hyperlink{Adapalene And Benzoyl Peroxide And Clindamycin Phosphate}{PMID: 9132194}, J A Wilson et al., 1997]

\hypertarget{pmid_21313776}{I}n this paper we describe the assessment and medical treatment of pain in children according to the concept of the Centre of Pediatrics and Adolescent Medicine at the university of Freiburg, Germany. Opiate therapy in children as well as novel data about the association of paracetamol (acetaminophen) and wheezing/asthma bronchiale in children are discussed. Special aspects of analgesia for painful procedures and a nitrous oxide/oxygen mixture which has been recently introduced in Germany are described. The second part of the paper presents results of our prospective study about continuous infusion of fentanyl and midazolam in a fixed combination in 19 critically ill patients with a median age of 46 months, 40\% of these patients had an ARDS. The mortality rate was 21\%. A median dose of fentanyl of 3.9 microg/kg/h (midazolam 0.26 mg/kg/h) was infused. The fentanyl serum level (median 4.2 ng/ml, range 1.7-17.8 ng/ml) correlated significantly with the administered dose while the midazolam serum levels (median 911 ng/ml, range 234-4 651 ng/ml) correlated neither with the administered dose nor with any of the analysed parameters. A standard protocol for the assessment and treatment of pain should be established in every pediatric hospital. The data about the association of asthma bronchiale and paracetamol cannot be interpreted conclusively, but show that even for well known substances clinical trials may lead to new awareness. The study data about continuous infusion of fentanyl and midazolam show a good correlation of the fentanyl application to serum levels, while midazolam appears to be not the optimal substance for continuous sedation in this setting. [\hyperlink{Adapalene And Benzoyl Peroxide And Clindamycin Phosphate}{PMID: 21313776}, Cornelia Möllmann et al., 2011]

\hypertarget{pmid_7590052}{T}o assess the safety and efficacy of intravenous sedation in pediatric upper endoscopy, all elective outpatient procedures performed during a 2-year period (January 1, 1991 through December 31, 1992) were retrospectively reviewed. Of 614 children, 553 received intravenous meperidine and midazolam; 61 received fentanyl and midazolam. The mean dose of meperidine was 1.5 +/- 0.7 mg/kg and of fentanyl 0.0031 +/- 0.0014 mg/kg. Less midazolam was needed for children receiving fentanyl than for those receiving meperidine (0.05 +/- 0.03 mg/kg versus 0.08 +/- 0.05 mg/kg, p < 002). Recovery time (minutes) was shorter for those receiving fentanyl (74.7 +/- 22.8 versus 95.1 +/- 23.0, p < .003). Side effects occurred in 117 patients (19.1\%), of which the majority were mild (83\%); all were transient with no residual sequelae. Inability to complete the procedure occurred in fewer than 1\%. We conclude that both combinations of medication are safe and effective for children of all ages. The use of fentanyl/midazolam results in a shorter recovery time and a lower dose of midazolam. [\hyperlink{Adapalene And Benzoyl Peroxide And Clindamycin Phosphate}{PMID: 7590052}, E Chuang et al., 1995]

\hypertarget{pmid_23078168}{P}aracetamol (acetaminophen) and ibuprofen are the most frequently purchased over-the-counter (OTC) medicines for children. Parents purchase these medicines for the treatment of fever and pain. In some countries other NSAIDs such as aspirin (acetylsalicylic acid) and dipyrone are available. We aimed to perform a narrative review of the efficacy and toxicity of OTC analgesic medicines for children in order to give guidance to health professionals and parents regarding the treatment of pain in a child. Neither aspirin nor dipyrone are recommended for OTC use because of the association with Reye's syndrome for the former and the risk of agranulocytosis for the latter. Both paracetamol and ibuprofen are effective for the treatment of mild pain in children. Adverse effects with both medicines are infrequent. Ibuprofen is an NSAID and therefore there is a greater risk of gastrointestinal adverse effects and hypersensitivity. Aspirin and dipyrone should be avoided. Paracetamol is the drug of first choice for mild pain in children because of its favourable safety profile. For the treatment of significant musculoskeletal pain, ibuprofen is the drug of first choice. [\hyperlink{Adapalene And Benzoyl Peroxide And Clindamycin Phosphate}{PMID: 23078168}, Zeina Bárzaga Arencibia et al., 2012]

\hypertarget{pmid_10230186}{T}o review extant data on the efficacy and safety of anxiolytic medications (benzodiazepines, buspirone, and other serotonin 1A agonists), adrenergic agents (beta-blockers and alpha 2-adrenergic agonists clonidine and guanfacine), and the opiate antagonist naltrexone that have been used to treat various psychopathologies in children and adolescents. To identify critical gaps in our current knowledge about these agents and needs for further research. All available controlled trials of these medications in children and adolescents published in English through 1997 were reviewed. In addition, selected uncontrolled studies are included. The major finding, that there are virtually no controlled data that support the efficacy of most of these drugs for the treatment of psychiatric disorders in children and adolescents, is both surprising and unfortunate. For some drugs, e.g., buspirone and guanfacine, this is because no controlled studies have been carried out in children and/or adolescents. For other drugs, e.g., clonidine and naltrexone, most of the placebo-controlled studies have failed to demonstrate efficacy. The strongest recommendations for controlled studies of safety and efficacy in children and adolescents can be given for the following drugs: benzodiazepines for acute anxiety; buspirone (and newer serotonin 1A agonists as they become available) for anxiety and depression; beta-blockers for aggressive dyscontrol; guanfacine for attention-deficit/hyperactivity disorder; and naltrexone for hyperactivity, inattention, and aggression in autistic disorder. [\hyperlink{Adapalene And Benzoyl Peroxide And Clindamycin Phosphate}{PMID: 10230186}, M A Riddle et al., 1999]

\hypertarget{pmid_25746065}{S}ildenafil (Revatio®) and tadalafil (Adcirca®) are specific inhibitors of the phosphodiesterase-5 enzyme and produce pulmonary vasodilation by inhibiting the breakdown of cyclic guanosine monophosphate (cGMP) in the walls of pulmonary arterioles. We focus on the efficacy and safety of sildenafil and tadalafil in the treatment of pulmonary hypertension (PH) in children through a PubMed literature search. Although used since 1999 in the treatment of PH in children, it is only in the past few years that robust evidence for the use of sildenafil has emerged principally in the pivotal STARTS-1 study. The open-label extension of this study, STARTS-2, has revealed safety concerns substantiated by FDA post marketing surveillance leading to recommendations to use lower doses. More recently, tadalafil has been introduced allowing once daily dosing with apparently similar efficacy to sildenafil in children. Recently there have been suggestions that sildenafil and tadalafil may have a place in treating muscular dystrophy. [\hyperlink{Adapalene And Benzoyl Peroxide And Clindamycin Phosphate}{PMID: 25746065}, Alan G Magee et al., 2015]

\hypertarget{pmid_24406329}{T}o establish the safety of an intranasal diamorphine (IND) spray in children. An open-label, single-dose pharmacovigilance trial. Emergency departments in eight UK hospitals. Children aged 2-16 years with a fracture or other trauma. Adverse events (AE) specifically related to nasal irritation, respiratory and central nervous system depression. 226 patients received 0.1 mg/kg IND. No serious or severe AEs occurred. The incidence of treatment-emergent AEs (TEAEs) was 26.5\% (95\% CI 20.9\% to 32.8\%), 93\% being mild. 89\% were related to treatment, all being known effects of the drug or route of administration except for three events in two patients. 20.4\% (95\% CI 15.3\% to 26.2\%) patients reported nasal irritation, all mild except one moderate and one 'unknown' severity. No respiratory depression was reported. Three AEs related to reduced Glasgow Coma Score (GCS) occurred, all mild. There were no safety concerns raised during the conduct of the study. In addition to expected side effects, IND can cause mild nasal irritation in a proportion of patients. 2009-014982-16. [\hyperlink{Adapalene And Benzoyl Peroxide And Clindamycin Phosphate}{PMID: 24406329}, Jason Kendall et al., 2015]

\hypertarget{pmid_26345297}{T}o investigate the cutaneous safety and tolerability of clindamycin phosphate 1.2\%/benzoyl peroxide 3.75\% gel in moderate-to-severe acne patients. A safety assessment of 498 patients with moderate-to-severe acne receiving clindamycin phosphate 1.2\%/benzoyl peroxide 3.75\% gel or vehicle for 12 weeks. The vast majority (80-95\%) of patients reported no cutaneous safety or tolerability problems throughout the study. Mean scores for both active and vehicle were all <1 (where l=mild) and reduced over the duration of the study. When scaling, erythema, itching, burning, or stinging was reported it was generally mild. Moderate or severe reactions to clindamycin phosphate 1.2\%/benzoyl peroxide 3.75\% gel were rare and generally seen early in treatment. There were eight reports (3.3\%) of moderate erythema, four reports (1.7\%) of moderate scaling, three reports (1.2\%) of moderate itching, and one report of moderate burning (0.4\%) at Week 4. There was one report (0.4\%) of severe erythema and one report (0.4\%) of severe burning (both at Week 4), with one report (0.4\%) of severe stinging at Week 12. There were no substantive differences seen in cutaneous tolerability among treatment groups and younger patients tended to have milder reactions. It is not possible to determine the contributions of the individual active ingredients. Clindamycin phosphate 1.2\%/benzoyl peroxide 3.75\% gel has a favorable safety and tolerability profile with very low incidence of moderate or severe reactions. [\hyperlink{Adapalene And Benzoyl Peroxide And Clindamycin Phosphate}{PMID: 26345297}, Guy Webster et al., 2015]

\hypertarget{pmid_19740527}{E}noxaparin, a low molecular weight heparin (LMWH), is frequently used for the prevention and treatment of thromboembolic complications in infants and children (Sutor et al., 2004 [1]). Injection pain and the fear and anxiety associated with needle phobia in the pediatric population are well documented. Best practice pediatric pain management standards of care recommend mitigating the child's pain experience whenever possible. The use of topical anesthetics such as liposomal-lidocaine 4\% results in a rapid onset of anesthesia, minimal blanching, without vasoconstriction (Koh et al., 2004 [2]) or risk of methemoglobinemia. Topical lidocaine has been used to reduce the injection pain of enoxaparin, but there is no data available examining whether it will interfere with the absorption of LMWH. To determine if the topical lidocaine, Maxilene, interferes with enoxaparin absorption as measured by peak anti-Xa levels. Infants and children clinically prescribed enoxaparin were eligible for this study. Children in group 1 were pre-treated with Maxilene prior to enoxaparin injection on day 1 with no Maxilene pre-treatment on day 2. For group 2, the order was reversed. Peak anti-Xa levels were measured following each enoxaparin dose and were compared between the groups. 26 children of ages 14d-16 y (median 6.7 months) were enrolled. Anti-Xa levels following topical lidocaine administration were 0.070 U/mL (95\%CI 0.025; 0.114) lower than without prior topical lidocaine administration. Anti-Xa levels on the second day were on average 0.013 U/mL (95\%CI -0.066; 0.040) higher compared to day one regardless of the order of topical lidocaine administration. There were no reported incidences of local reactions such as redness, hives or blanching. Topical lidocaine (Maxilene) administration before enoxaparin injection results in a small, clinically non-significant, reduction in anti-Xa levels. [\hyperlink{Adapalene And Benzoyl Peroxide And Clindamycin Phosphate}{PMID: 19740527}, S M Duncan et al., 2010]

\hypertarget{pmid_17639417}{P}ain is the most common discomfort experienced by children undergoing major operations. It is most often not adequately treated because of inexperience and unfounded fears related to the use of opioid drugs. In adults, patient-controlled analgesia (PCA) is widely administered, while in children, its use with opioid drugs is still under evaluation for safety and efficacy. The objective of the study is to evaluate the safety and efficacy of an opioid drug (fentanil) administered by PCA associated with a sedative-adjuvant drug (midazolam) administered by continuous infusion in children having undergone major neurosurgical procedures. Sixteen children with moderate to severe postoperative pain were treated with fentanil by PCA (booster doses of 1 microg/kg) plus continuous infusion of midazolam (2 microg/kg per min) by an intravenous route. To evaluate safety and efficacy of this analgesic protocol, different subjective and objective parameters were monitored at 4-h intervals. In addition, patients' satisfaction was assessed by a questionnaire at the end of the treatment. All children experienced a good degree of analgesia and did not require any other analgesic drug during the treatment. Both subjective and objective parameters improved after starting pain-relieving treatment, and no major side effects occurred. The analysis of the answers of the questionnaire administered to the children showed a high grade of satisfaction. PCA with fentanil plus continuous infusion of midazolam is a safe and efficacious method for analgesia in children with moderate to severe postoperative neurosurgical pain. The association of midazolam to fentanil also contributes to control anxiety and stress in this subset of patients and does not show any important side effects. [\hyperlink{Adapalene And Benzoyl Peroxide And Clindamycin Phosphate}{PMID: 17639417}, Antonio Chiaretti et al., 2008]

\hypertarget{pmid_9631921}{C}hildren and adolescents with severe asthma frequently experience anxiety or depression with anxiety, which can undermine their response to treatment. In addition, these patients often receive theophylline and a variety of adrenergic stimulants, which can exacerbate or worsen anxiety. Such children occasionally are candidates for treatment with anxiolytic therapy. There is a paucity of drug disposition data in adolescents for benzodiazepines, the most frequently used antianxiety drugs. The authors monitored the steady state alprazolam plasma concentration in six children with severe asthma who were administered standard doses of alprazolam. In one patient administered concurrent therapy with troleandomycin, a recognized cytochrome 3A4 inhibitor, alprazolam plasma concentration was markedly elevated. Overall, the disposition data of alprazolam was consistent with data previously reported in adults. Alprazolam appeared to be safe and effective for use in adolescents with asthma. [\hyperlink{Adapalene And Benzoyl Peroxide And Clindamycin Phosphate}{PMID: 9631921}, C L DeVane et al., 1998]

\hypertarget{pmid_6359862}{F}ever and pain are the most common issues in pediatric patient management. Acetaminophen, aspirin, and dipyrone are the most commonly used drugs and are equivalent in their efficacy. Dipyrone, used in many parts of the world, but not in the United States, is an effective agent; however, it has been implicated in producing agranulocytosis and anaphylactic shock. The salicylates have anti-inflammatory effects making them appropriate for the treatment of patients with juvenile rheumatoid arthritis, but they are gastric irritants, may impair clotting, and, because of saturable kinetics, may lead to accumulation and toxicity. Acetaminophen is an effective antipyretic and analgesic with few side effects that is toxic only in massive overdose. [\hyperlink{Adapalene And Benzoyl Peroxide And Clindamycin Phosphate}{PMID: 6359862}, E Gladtke et al., 1983]

\hypertarget{pmid_22777222}{A}cne is common in adolescents and especially difficult to manage in people with color. A fixed combination of clindamycin phosphate and benzoyl peroxide (BPO) (clindamycin phosphate 1.2\%/BPO 2.5\% gel) was evaluated to determine its utility in treating moderate to severe acne in adolescents with skin of color. Three hundred thirty-seven adolescent acne subjects (aged 12 to <18 years) with skin of color were evaluated from 2 multicenter, double-blind studies. Subjects were randomized to receive clindamycin phosphate 1.2\%/BPO 2.5\% gel or vehicle, once daily for 12 weeks. Efficacy and tolerability were evaluated. Data were compared with an adolescent (A) and skin of color (B) cohort from the same pivotal study enrolling 2,813 subjects. Superior mean percent reductions in inflammatory, noninflammatory, and total lesion counts were observed in subjects receiving clindamycin phosphate 1.2\%/BPO 2.5\% gel compared to vehicle. At week 12, clindamycin phosphate 1.2\%/BPO 2.5\% gel showed similar lesion reduction compared to groups A and B (P<0.001). Treatment success with clindamycin phosphate 1.2\%/BPO 2.5\% gel, assessed by investigator and subject, was superior to vehicle and comparable to that seen in groups A and B (P<0.001). Clindamycin phosphate 1.2\%/ BPO 2.5\% gel was associated with a low incidence of treatment-related AEs and a favorable cutaneous tolerability profile. Clindamycin phosphate 1.2\%/BPO 2.5\% gel has been shown to be effective, safe, and well tolerated in moderate to severe acne in adolescents with skin of color. [\hyperlink{Adapalene And Benzoyl Peroxide And Clindamycin Phosphate}{PMID: 22777222}, Lawrence F Eichenfield et al., 2012]

\hypertarget{pmid_25226009}{T}o evaluate efficacy, safety, and tolerability of a fixed combination clindamycin phosphate 1.2\% and benzoyl peroxide 3.75\% (clindamycin-BP 3.75\%) aqueous gel in moderate to severe acne vulgaris. A total of 498 patients, 12-40 years of age, were randomized to receive clindamycin-BP 3.75\% or vehicle in a double-blind, controlled 12-week, 2-arm study evaluating safety and efficacy using inflammatory and noninflammatory lesion counts, Evaluator Global Severity Scores (EGSS) and subject self-assessment (SSA). In addition, patients completed a patient satisfaction survey (PSS), acne-specific QoL questionnaire, and assessed their facial skin for shininess/oiliness. Clindamycin-BP 3.75\% demonstrated statistical superiority to vehicle in reducing both inflammatory and noninflammatory lesions and acne severity. Clindamycin-BP 3.75\% showed greater efficacy relative to vehicle in assessments of skin oiliness, SSA and PSS. No substantive differences were seen in cutaneous tolerability among treatment groups and no patients discontinued treatment with Clindamycin-BP 3.75\% because of adverse events. Data from controlled studies may differ from clinical practice. It is not possible to determine the contributions from the individual active ingredients. Clindamycin-BP 3.75\% provides statistically significant greater efficacy than vehicle with a favorable safety and tolerability profile. [\hyperlink{Adapalene And Benzoyl Peroxide And Clindamycin Phosphate}{PMID: 25226009}, David M Pariser et al., 2014]

\hypertarget{pmid_22364032}{A}cute respiratory infections are the second leading cause of morbidity in children under 18 years. Several drugs have been used with variable efficacy and safety, trying to reduce the associated symptoms and improve quality of life. To evaluate the efficacy and safety of buphenine, aminophenazone and diphenylpyraline hydrochloride when compared with placebo for the control of symptoms associated with common cold in children 6-24 months of age. Randomized clinical trial, double blind, placebo controlled, in 100 children < 24 months of any gender, with symptoms associated to common cold. They received the drug under study vs. placebo for seven days. Both groups received acetaminophen. The change on common cold related symptoms were evaluated. Statistic analysis was made with STATA 11.0 for Mac. Fifty-three children were randomized to study drug and forty-seven to placebo. Age of children in each group was similar (12.2 +/- 5.8 months vs. 12.7 +/- 5.8 months, p NS). There were significant differences between groups in relation to rhinorrea and sneezing resolution, with better results in Flumil group and no adverse events observed. The results in this study indicates that Flumil is a safe and effective drug for control of symptoms present in the common cold in children aged 6-24 months. [\hyperlink{Adapalene And Benzoyl Peroxide And Clindamycin Phosphate}{PMID: 22364032}, Ericka Montijo-Barrios et al., ]

\hypertarget{pmid_9890791}{I}n 47 children followed for 1 year after the first "simple" febrile convulsion, dipropylacetate (Depakine, 20 mg/kg) was as effective in preventing new febrile convulsions (a single recurrence in 4\% of 47 children) as was phenobarbital (5 mg/kg) or primidone (25 mg/kg) (a single recurrence in 4\% of 47 children), and there were no side effects. Of 47 untreated children followed for 1 year, 55\% had 1 to 4 new febrile convulsions. All medications were given in divided doses morning and evening. [\hyperlink{Adapalene And Benzoyl Peroxide And Clindamycin Phosphate}{PMID: 9890791}, G B Cavazzuti et al., 1975]

\hypertarget{pmid_27714809}{I}nfusion of precipitate or destabilized emulsion can be harmful. The purpose of this study was to obtain Y-site compatibility data on intravenous drugs and total parenteral nutrition (TPN) relevant for children. Two three-in-one TPN admixtures (Olimel N5E and Numeta G16E) used for children of different age groups were tested with ten drugs (ampicillin, ceftazidime, clindamycin, dexamethasone, fluconazole, fosphenytoin, furosemide, metronidazole, ondansetron and paracetamol). Drug : TPN ratios were estimated from a wide range of age and weight classes, and the most extreme mixing ratios (drug > TPN, TPN > drug) in addition to 1 + 1 were chosen. Assessment of potential precipitation was performed by subvisual particle counting, visual examinations and measurements of turbidity and pH. Emulsion stability was investigated by estimation of percentage of droplets above 5 μm (PFAT5), mean droplet diameter and pH measurements. Complimentary theoretical evaluations were performed. Ampicillin, fosphenytoin and furosemide precipitated when mixed with TPN. The results for ceftazidime, clindamycin, dexamethasone, fluconazole, metronidazole, ondansetron and paracetamol suggest that they were compatible with either TPN in the tested concentrations. None of the drugs were found to destabilize the emulsions. Three drugs showed clear signs of precipitation when mixed with TPN and these products should not be co-administered in the same infusion line. [\hyperlink{Adapalene And Benzoyl Peroxide And Clindamycin Phosphate}{PMID: 27714809}, Vigdis Staven et al., 2017]

\hypertarget{pmid_23236934}{T}he use of midazolam for children was approved in March, 2010. Since the efficacy and safety data of midazolam used in children, excluding low-birth-weight infants and newborns, for "sedation under artificial respiration in intensive care units" were quite limited, a post-marketing survey was carried out to confirm the validity of the established dosage and administration. A consecutive enrollment method was adopted. Based on the data of 153 patients collected from 8 institutes, efficacy and safety profiles were analyzed. Among the 149 patients included in the safety analysis set, 6 adverse reactions were reported in 6 patients. The incidence of adverse events was 4.0\% (6/149). Reported adverse reactions included depressed level of consciousness: 1 event, delirium: 1 event, psychomotor hyperactivity: 1 event, hypotension: 2 events, and blood pressure increase: 1 event. Serious adverse drug reaction (ADR) reported in this survey was depressed level of consciousness. This ADR resolved on the following day after the treatment with flumazenil. Paradoxical reactions were reported in 1 patient, and tolerance was reported in 2 patients. The efficacy rate was 96.5\% (138/143). No additional safety issues (status of adverse reactions, status of adverse events, status of serious adverse events, etc.) and efficacy issue were manifest in the patients treated with the dosage and administration method established at the approval of the drug. [\hyperlink{Adapalene And Benzoyl Peroxide And Clindamycin Phosphate}{PMID: 23236934}, Keizo Sogabe et al., 2012]

\hypertarget{pmid_36421228}{C}oncerns about the safety of anesthetic agents in children arose after animal studies revealed disruptions in neurodevelopment after exposure to commonly used anesthetic drugs. These animal studies revealed that volatile inhalational agents, propofol, ketamine, and thiopental may have detrimental effects on neurodevelopment and cognitive function, but dexmedetomidine and xenon have been shown to have neuroprotective properties. The neurocognitive effects of benzodiazepines have not been extensively studied, so their effects on neurodevelopment are undetermined. However, experimental animal models may not truly represent the pathophysiological processes in children. Multiple landmark studies, including the MASK, PANDA, and GAS studies have provided reassurance that brief exposure to anesthesia is not associated with adverse neurocognitive outcomes in infants and children, regardless of the type of anesthetic agent used. [\hyperlink{Adapalene And Benzoyl Peroxide And Clindamycin Phosphate}{PMID: 36421228}, Dabin Ji et al., 2022]

\hypertarget{pmid_30283188}{S}urgery is a very stressful experience for patients. Children are the most susceptible to fear, anxiety, and stress due to their limited cognitive capabilities and dependency. In children, pharmacologic agents are frequently used as premedication to relieve the fear of surgery, to make child-parental separation easy, and to carry out a smooth induction of anesthesia. We conducted this study to compare the efficacy of intranasal fentanyl, midazolam, and dexmedetomidine as premedication in pediatric patients. The present study was conducted prospectively on 75 patients in the age group of 2-6 years of either sex of the American Society of Anesthesiologists physical Class I or II admitted in Guru Nanak Dev Hospital, attached to Government Medical College Amritsar, scheduled to undergo surgery under general anesthesia. The patients were divided into three groups of 25 each. Group F received intranasal fentanyl 1.5 μg/kg body weight, Group M received intranasal midazolam 0.3 mg/kg body weight, and Group D received intranasal dexmedetomidine 1 μg/kg body weight as nasal drops 50 min before surgery. Children who received intranasal fentanyl and intranasal midazolam had early onset of anxiolysis and sedation as compared to dexmedetomidine. In child-parent separation, quality of induction was better with fentanyl and dexmedetomidine as compared to midazolam. Intravenous cannulation score was best achieved with fentanyl as premedicant. Postoperative sedation was better with dexmedetomidine as compared to fentanyl and midazolam. Onset of action of fentanyl and midazolam is early as compared to that of dexmedetomidine. However, fentanyl provided better conditions for induction and emergence than midazolam. With dexmedetomidine onset of action was delayed and duration of action was prolonged which helped child to remain calm and sedated even after the surgery. [\hyperlink{Adapalene And Benzoyl Peroxide And Clindamycin Phosphate}{PMID: 30283188}, Veena Chatrath et al., ]

\hypertarget{pmid_26858095}{S}edation is increasingly used to facilitate procedures on children in emergency departments (EDs). This overview of systematic reviews (SRs) examines the safety and efficacy of sedative agents commonly used for procedural sedation in children in the ED or similar settings. We followed standard SR methods: comprehensive search; dual study selection, quality assessment, data extraction. We included SRs of children (1 month to 18 years) where the indication for sedation was procedure-related and performed in the ED. Fourteen SRs were included (210 primary studies). The most data were available for propofol (six reviews/50,472 sedations) followed by ketamine (7/8,238), nitrous oxide (5/8,220), and midazolam (4/4,978). Inconsistent conclusions for propofol were reported across six reviews. Half concluded that propofol was sufficiently safe; three reviews noted a higher occurrence of adverse events, particularly respiratory depression (upper estimate 1.1\%; 5.4\% for hypotension requiring intervention). Efficacy of propofol was considered in four reviews and found adequate in three. Five reviews found ketamine to be efficacious and seven reviews showed it to be safe. All five reviews of nitrous oxide concluded it is safe (0.1\% incidence of respiratory events); most found it effective in cooperative children. Four reviews of midazolam made varying recommendations. To be effective, midazolam should be combined with another agent that increases the risk of adverse events (upper estimate 9.1\% for desaturation, 0.1\% for hypotension requiring intervention). This comprehensive examination of an extensive body of literature shows consistent safety and efficacy for nitrous oxide and ketamine, with very rare significant adverse events for propofol. There was considerable heterogeneity in outcomes and reporting across studies and previous reviews. Standardized outcome sets and reporting should be encouraged to facilitate evidence-based recommendations for care. [\hyperlink{Adapalene And Benzoyl Peroxide And Clindamycin Phosphate}{PMID: 26858095}, Lisa Hartling et al., 2016]

\hypertarget{pmid_26926644}{C}lindamycin may be active against methicillin-resistant Staphylococcus aureus, a common pathogen causing sepsis in infants, but optimal dosing in this population is unknown. We performed a multicenter, prospective pharmacokinetic (PK) and safety study of clindamycin in infants. We analyzed the data using a population PK analysis approach and included samples from two additional pediatric trials. Intravenous data were collected from 62 infants (135 plasma PK samples) with postnatal ages of <121 days (median [range] gestational age of 28 weeks [23 to 42] and postnatal age of 17 days [1 to 115]). In addition to body weight, postmenstrual age (PMA) and plasma protein concentrations (albumin and alpha-1 acid glycoprotein) were found to be significantly associated with clearance and volume of distribution, respectively. Clearance reached 50\% of the adult value at PMA of 39.5 weeks. Simulated PMA-based intravenous dosing regimens administered every 8 h (≤32 weeks PMA, 5 mg/kg; 32 to 40 weeks PMA, 7 mg/kg; >40 to 60 weeks PMA, 9 mg/kg) resulted in an unbound, steady-state concentration at half the dosing interval greater than a MIC for S. aureus of 0.12 μg/ml in >90\% of infants. There were no adverse events related to clindamycin use. (This study has been registered at ClinicalTrials.gov under registration no. NCT01728363.). [\hyperlink{Adapalene And Benzoyl Peroxide And Clindamycin Phosphate}{PMID: 26926644}, Daniel Gonzalez et al., 2016]

\section*{Ciclopirox}
\subsection*{Result}
\subsubsection*{Answer}

3 months to 10 years: Yes
6 to 29 months: Yes
Younger than 3 months: Unknown
Older than 10 years: Unknown

\subsubsection*{{Explanation}}
\hypertarget{Ciclopirox}
Based on the abstracts available, there are two targeted studies that specifically address the safety of ciclopirox in children:

1. A multicenter, non-randomized, open-label, phase III study evaluated ciclopirox olamine cream 1\% for dermatomycosis in pediatric patients aged 3 months to 9 years. Twenty-one patients were included, with a median age of 2.7 years (range 3 months-9 years). No adverse events were reported in 62\% of patients; among the mild and moderate adverse events, only one case of irritative dermatitis was considered possibly related to the treatment. Safety evaluation was rated as excellent in 95\% of patients and good in 5\%. The study concludes that ciclopirox olamine cream 1\% is a safe and feasible treatment for superficial cutaneous mycotic infections, especially Candida spp. infection, in children aged between 3 months and 10 years [\hyperlink{pmid_25728876}{PMID: 25728876}, Elisabeth Gómez-Moyano et al.].

2. An open-label study investigated ciclopirox topical suspension 0.77\% in the treatment of diaper dermatitis due to Candida albicans in 44 subjects aged 6 to 29 months. All adverse events were mild to moderate and considered not related to the study medication. The study concludes that ciclopirox was safe and effective in this age group [\hyperlink{pmid_15696982}{PMID: 15696982}, Elizabeth Gallup et al.].

Other abstracts either do not specify pediatric populations or do not provide targeted safety data for children.

Summary by age range:
- 3 months to 10 years: Evidence from a targeted pediatric study supports the safety of ciclopirox olamine cream 1\% for superficial cutaneous mycotic infections.
- 6 to 29 months: Evidence from a targeted pediatric study supports the safety of ciclopirox topical suspension 0.77\% for diaper dermatitis due to Candida albicans.

No targeted safety data are available in the abstracts for children older than 10 years or for neonates younger than 3 months.

\subsection*{Abstracts}
\hypertarget{pmid_15696982}{C}iclopirox is a broad-spectrum antifungal, antibacterial, and anti-inflammatory agent. This open-label study investigated the safety and efficacy of ciclopirox topical suspension 0.77\% in the treatment of diaper dermatitis due to Candida albicans (C. albicans). Forty-four male and female subjects aged 6 to 29 months were included in the study. Study medication was applied topically to the affected diaper area twice daily for 1 week. Subjects were clinically evaluated at baseline and days 3, 7, and 14 (7 days post-treatment). Safety and efficacy variables included adverse events, mycological culture studies, KOH tests, Severity Scores, and Global Evaluation of Clinical Response. All adverse events were mild to moderate and considered not related to the study medication. Treatment provided statistically significant improvement (P < .05) for both the rate of mycological cure and reduction of Severity Score at each time point compared with baseline. Ciclopirox was safe and effective in the treatment of diaper dermatitis due to C. albicans. [\hyperlink{Ciclopirox}{PMID: 15696982}, Elizabeth Gallup et al., ]

\hypertarget{pmid_18698269}{T}o evaluate the ototoxicity of ciclopirox-containing solution as an otologic preparation for the treatment of otomycosis. Ciclopirox is a synthetic antimycotic agent available in a variety of formulations to treat superficial fungal infections. Ciclopirox has demonstrated both fungicidal and fungistatic activity in vitro against a broad spectrum of pathogenic fungi. It also possesses a broad-spectrum antibacterial properties, anti-inflammatory, and antiedema effect. The ototoxic effect of ciclopirox-containing solutions has not been known, so the current study was designed to observe the ototoxic effect of this solution experimentally. Experiments were performed in 22 young male albino guinea pigs (weight, 450-550 g). The 10 animals in the experimental group received ciclopirox solution, and the control group was divided into two groups of six animals each. The first group received saline solution (negative control) and the second received gentamicin (40 mg/mL; ototoxic control). Under general anesthesia, pretreatment auditory brainstem responses (ABRs) from the right ears were obtained from the animals in all groups. The right tympanic membranes were totally perforated, and a small piece of Gelfoam was applied to the middle ear directly to the round window membrane. Ear solutions were applied through transcanal approach to the middle ear twice a day in 2 weeks. Twenty-two animals of perforated tympanic membrane were observed during a 2-week period. Posttreatment ABRs were obtained in all groups in a week after the last administration. Baseline ABR results were normal in right ears of all animals tested. Animals undergoing placement of Gelfoam with either ciclopirox solution or saline in the middle ear showed no changes in the ABR threshold. The gentamicin group showed a significant change in the ABR threshold. In the guinea pig, when applied topically to the middle ear, ciclopirox does not cause a reduction in the ABR threshold. Because its safety has not yet been confirmed in patients, caution should be observed when prescribing this agent. [\hyperlink{Ciclopirox}{PMID: 18698269}, Serdar Baylancicek et al., 2008]

\hypertarget{pmid_18305467}{W}e conducted a prospective, open-label multicenter trial to evaluate the efficacy and safety of treating children with frequently relapsing nephrotic syndrome with cyclosporine. Patients were randomly divided into two groups with both initially receiving cyclosporine for 6 months to maintain a whole-blood trough level between 80 and 100 ng/ml. Over the next 18 months, the dose was adjusted to maintain a slightly lower (60-80 ng/ml) trough level in Group A, while Group B received a fixed dose of 2.5 mg/kg/day. The primary end point was the rate of sustained remission with analysis based on the intention-to-treat principle. After 2 years, the rate of sustained remission was significantly higher while the hazard ratio for relapse was significantly lower in Group A as compared with Group B. Mild arteriolar hyalinosis of the kidney was more frequently seen in Group A than in Group B, but no patient was diagnosed with striped interstitial fibrosis or tubular atrophy. We conclude that cyclosporine given to maintain targeted trough levels is an effective and relatively safe treatment for children with frequently relapsing nephrotic syndrome. [\hyperlink{Ciclopirox}{PMID: 18305467}, K Ishikura et al., 2008]

\hypertarget{pmid_21144334}{C}yclosporine has been found to be effective and safe in many inflammatory skin disorders such as psoriasis and atopic dermatitis (AD), in adults and in children. Its use in paediatrics is still under scope. We present three patients who started cyclosporine but stopped due to complications. It is our aim to warn about potential side effects of cyclosporine and recommend cautious utilization. Two children, aged 4 and 13 years, with AD and one child, aged 2 years, with erythrodermic psoriasis, were treated with oral cyclosporine. developed secondary impetigo on the 6th day of treatment. Started topical corticosteroids and topical calcineurin inhibitors afterwards, with no relapses. developed herpetic infection, hepatic and renal impairment (eventual drug interaction) on the 4th day of treatment. THIRD CASE: Psoriasis and impetigo, treated with flucloxacillin, gentamicin. Generalized angioedema and urticariform lesions after 6 days of cyclosporine. Beta lactam hypersensitivity reaction under study. Eventual cyclosporine toxicity to consider. The data on cyclosporine use in children is still scarce. Use should be limited to cases with precise indication, after considering risks and benefits. [\hyperlink{Ciclopirox}{PMID: 21144334}, João Antunes et al., ]

\hypertarget{pmid_15985032}{C}iclopirox is an antifungal agent and is effective against both Gram-positive and Gram-negative bacteria. These properties may give ciclopirox an advantage over other antifungal agents in the treatment of interdigital tinea pedis with secondary bacterial infection (dermatophytosis complex). To evaluate the efficacy of ciclopirox 0.77\% gel in the treatment of tinea pedis interdigitalis with secondary bacterial infection in a prospective, randomized, double-blind, placebo-controlled clinical study. One hundred subjects were enrolled in this 8-week study (twice-daily ciclopirox, 40 subjects; once-daily ciclopirox, 40 subjects; twice-daily vehicle, 20 subjects). Mycologic sampling, bacterial swabs, and evaluations for symptoms and signs of tinea pedis were performed on a target webspace at baseline and at weeks 2, 4, and 8. Global evaluations were made by both investigator and subject at each visit. Ciclopirox gel applied once or twice daily significantly reduced the signs and symptoms at week 8, compared with vehicle (P<0.0036). The mycologic cure and complete cure rates were much higher for the ciclopirox regimens than for the vehicle regimen. Early reduction of bacterial counts was noted with the ciclopirox regimens. There was no significant difference in the adverse event rate between the ciclopirox groups and the placebo group. Ciclopirox 0.77\% gel, applied once or twice daily, is effective and safe in the treatment of tinea pedis interdigitalis with concomitant bacterial infection (dermatophytosis complex). [\hyperlink{Ciclopirox}{PMID: 15985032}, Aditya K Gupta et al., 2005]

\hypertarget{pmid_19617660}{C}yclosporine A is used in the treatment of idiopathic nephrotic syndrome. We conducted this study to evaluate the effect of cyclosporine and its combination with ketoconazole in Egyptian nephrotic children with steroid-resistant and steroid-dependant minimal change. Forty-eight children with minimal change lesions who received cyclosporine with or without ketoconazole were studied. Their mean age was 5.17 +/- 1.59 years, and they were 31 boys and 17 girls. The mean duration of the disease was 6.22 +/- 3.16 years. Thirty-one of the children were steroid dependent and 17 were steroid resistant. Cyclosporine treatment was commenced after remission was attained and adjusted to a target trough level of 100 ng/mL. The mean cyclosporine therapy at a dose of 2.07 +/- 0.91 mg/kg was administered for a mean of 25.75 +/- 1.95 months. Thirty-three patients received adjunctive ketoconazole therapy. Thirty-eight patients (79.2\%) responded well to cyclosporine. Steroid therapy could be discontinued in 43 patients (89.6\%), but 9 experienced relapse. Ten patients (20.8\%) were resistant to cyclosporine therapy. Fifteen patients received cyclosporine alone, while 33 received concomitant cyclosporine and ketoconazole. The response to cyclosporine was significantly better in those on ketoconazole. The economic effect of ketoconazole therapy was a reduction in the costs of cyclosporine treatment by 47.4\% at 1 year of treatment. Cyclosporine treatment in children with minimal change nephrotic syndrome is effective in preventing relapse and decreasing steroid toxicity. Its combination with low-dose ketoconazole is safe, reduces treatment costs, and improves the response to cyclosporine. [\hyperlink{Ciclopirox}{PMID: 19617660}, Alaa Sabry et al., 2009]

\hypertarget{pmid_25728876}{T}here is scarce information on the use of ciclopirox olamine in children. The aim of this study was to evaluate the efficacy and safety of ciclopirox olamine cream 1\% for the treatment of dermatomycosis in pediatric patients. A multicenter, non-randomized, open-label, phase iii study was conducted on patients aged 3 months to 9 years diagnosed with dermatomycosis confirmed by direct microscopy and culture, and treated with ciclopirox olamine cream 1\% for 28 days. Clinical and microbiological evaluations were performed before starting the treatment therapy, at 7, 14 and 28 days after starting the treatment, and 28 days after its completion. Twenty-one patients with a median age of 2.7 years (range 3 months-9 years) were included. The most frequent mycosis location was the inguinal region (72\%). The most frequently isolated etiological agent was Candida spp. (71\%). No adverse events were reported in 62\% of the patients. Among the mild and moderate reported adverse events, only one, irritative dermatitis, was considered as possibly related to the treatment. Safety evaluation was excellent in 95\% of the patients, and good in 5\%. After the first week of treatment, 12 patients out of 13 (92\%) showed a clinical improvement, and 5 out of 7 (71\%) had both clinical and mycological improvements. At the end of the treatment, clinical cure was observed in 7 out of 9 patients (78\%). No relapses occurred. Ciclopirox olamine cream 1\% is a safe and feasible treatment for superficial cutaneous mycotic infections, especially Candida spp. infection, in children aged between 3 months and 10 years. [\hyperlink{Ciclopirox}{PMID: 25728876}, Elisabeth Gómez-Moyano et al., ] the aim of this study was to report single centre experience with cyclosporine used in treatment of children with inflammatory bowel disease with regard to safety and efficacy. retrospective analysis included 23 patients, 21 with ulcerative colitis and 2 with Crohn's disease, aged 2.75 to 18.5 years. They were treated with cyclosporine during the last 5 years. Before cyclosporine therapy was started they received steroids and azathioprine. Cyclosporine treatment was given in severe steroid-resistant exacerbation of the disease (n = 10) or steroid-dependence (n = 13). Cyclosporine dose was set to obtain therapeutic levels (serum concentration > 100 ng/ml and < 200 ng/ml). Cyclosporine treatment was continued up to 2 months in 6 cases, 2 to 6 months in 8 patients and more than 6 months in 9 patients. Complications were reported in 2 patients: hirsutism and gingival hypertrophy. Cyclosporine treatment was stopped in the second case. None of the two patients with Crohn's disease improved during the treatment. Short-term improvement was observed in 11 patients with ulcerative colitis. Long-term recovery (> 6 months) was obtained in 6 cases. In 10 patients with severe exacerbation of ulcerative colitis colectomy was performed, in 4 of them elective surgery was performed when the clinical status improved. cyclosporine appears to be a safe and relatively effective treatment of ulcerative colitis in children. Cyclosporine is less effective in maintaining remission and it did not allow to avoid colectomy in severe exacerbation. Still case controlled studies are needed to show the efficacy of this treatment. [\hyperlink{Ciclopirox}{PMID: 25728876}, Piotr Socha et al., ]

\hypertarget{pmid_20530497}{W}e previously established a treatment protocol for conventional cyclosporine (Sandimmune, Novartis, Basel, Switzerland) in children with frequently relapsing nephrotic syndrome; ∼50\% of patients remained relapse free for 2 years, without serious adverse events. Recently, microemulsified cyclosporine (Neoral, Novartis), which has a more stable absorption profile than conventional cyclosporine, has been developed. We tested the hypothesis that microemulsified cyclosporine is at least as effective as conventional cyclosporine. To evaluate the safety and efficacy of microemulsified cyclosporine, a prospective, multicentre trial was conducted according to the previously established protocol, using microemulsified cyclosporine instead of conventional cyclosporine. The duration of treatment was 24 months. During the first 6 months, patients received microemulsified cyclosporine in a dose that maintained the trough level between 80 and 100 ng/mL of cyclosporine. For the next 18 months, the dose was adjusted to maintain a level between 60 and 80 ng/mL. A total of 62 patients (median age, 5.4 years; 48 males, 14 females) were studied. The frequency of relapse decreased from 4.6 ± 1.4 to 0.7 ± 1.5 times per year (P < 0.0001). The probability of relapse-free survival at Month 24 was 58.1\% (95\% confidence interval, 45.8-70.3\%). The probability of progression (to frequently relapsing nephrotic syndrome)-free survival at Month 24 was 88.5\% (95\% confidence interval, 80.4-96.5\%). Cyclosporine nephrotoxicity was detected in only 8.6\% of patients who underwent renal biopsy after 2 years of treatment. Antihypertensive agents were administered to 12.9\% of the patients to control hypertension without severe sequelae. Microemulsified cyclosporine administered according to our treatment protocol is safe and effective in children with frequently relapsing nephrotic syndrome. [\hyperlink{Ciclopirox}{PMID: 20530497}, Kenji Ishikura et al., 2010]

\hypertarget{pmid_2642107}{A}lthough cyclosporine has improved allograft survival in renal transplant patients, problems with drug toxicity remain, raising the question whether cyclosporine should be stopped at some point post-transplant. However, the relative safety of converting from cyclosporine to another immunosuppressive agent, or simply stopping cyclosporine remains an issue of debate and has not been evaluated in children. We have developed a protocol to convert children, who are 6 months post-transplant and have stable kidney function, from cyclosporine and prednisone to azathioprine and prednisone. Eleven children have undergone conversion because of suspected/potential nephrotoxicity or because of other difficulties with cyclosporine (expense, hirsutism). These children were compared with a control group of 12 children who met all criteria for conversion at 6 months but remained on cyclosporine. Allograft survival was similar in both groups but the children converted from cyclosporine experienced an improvement in renal function as measured by calculated creatinine clearance. There were no episodes of rejection for a period of 4 months post-conversion and all rejection episodes that developed subsequently occurred during or after the change from daily to alternate-day prednisone. We believe that conversion from cyclosporine to azathioprine can be accomplished safely in children with stable allograft function but long-term risks and benefits need further evaluation. [\hyperlink{Ciclopirox}{PMID: 2642107}, B A Kaiser et al., 1989]

\hypertarget{pmid_15334276}{C}iclopirox (Loprox) is a broad-spectrum antifungal medication that also has antibacterial and anti-inflammatory properties. Its main mode of action is thought to be its high affinity for trivalent cations, which inhibit essential co-factors in enzymes. Clinical trials have shown that ciclopirox gel is a successful treatment for seborrheic dermatitis of the scalp as well as for tinea pedis. Adverse effects are generally mild and include a skin-burning sensation, contact dermatitis, and pruritus. Ciclopirox is indicated in the US for the treatment of tinea pedis, tinea corporis, pityriasis versicolor, seborrheic dermatitis, and cutaneous candidiasis. [\hyperlink{Ciclopirox}{PMID: 15334276}, A K Gupta et al., ]

\hypertarget{pmid_8151150}{T}he efficacy and safety of aciclovir granules (containing 40\% w/w aciclovir) were evaluated in the treatment of chickenpox in otherwise healthy children. Patients presenting with chickenpox received aciclovir granules at a dose of 20 mg/kg four times daily for five to seven days. Overall 51 children received treatment with aciclovir. A further 53 patients receiving conventional symptomatic therapy acted as a control. In the aciclovir group the overall efficacy rate was 92.2\%. There were reductions in the numbers of lesions, fever, itching and the duration of symptoms. No adverse experiences were reported. Overall this formulation of aciclovir appears to be a safe and effective treatment for chickenpox in this patient population. However the need for anti-viral therapy in otherwise healthy children is still the subject of debate and it might be appropriate to identify sub-groups for whom such therapy is justified. [\hyperlink{Ciclopirox}{PMID: 8151150}, H Kamiya et al., 1994]

\hypertarget{pmid_15790671}{C}iclopirox is a topical antifungal agent of the hydroxypyridone class whose mode of action is poorly understood. In order to elucidate the mechanism of action of ciclopirox, we analysed the growth, cellular integrity, biochemical properties, viability and transcriptional profile of the polymorphic yeast Candida albicans following exposure to this antifungal agent. Multiple biochemical assays served to identify factors that were critical for antifungal activity and to identify proteins whose activities changed in drug-exposed cells. Genome-wide transcriptional profiling was used to identify genes that were up-regulated in response to the cellular effects of the drug. Ciclopirox inhibited growth of C. albicans yeast and hyphal cells in a dose-dependent manner. This effect was reduced (i) by the addition of iron ions or the metabolic inhibitor 2-deoxy-D-glucose to growth media, (ii) in media that lacked glucose, and (iii) for cells that were pre-incubated with hydrogen peroxide or menadione [which caused induction of proteins involved in detoxification of reactive oxygen species (ROS)]. In contrast, cells pre-cultured under poor oxygen conditions (which had decreased activity of proteins involved in ROS detoxification) were more susceptible to ciclopirox. Treatment with ciclopirox did not directly cause cell membrane damage and did not change intracellular levels of ATP. Finally, the transcriptional profiling pattern of drug-treated cells strongly resembled iron-limited conditions. These data indicate that metabolic activity, oxygen accessibility and iron levels are critical parameters in the mode of action of ciclopirox olamine. [\hyperlink{Ciclopirox}{PMID: 15790671}, Hans-Christian Sigle et al., 2005]

\hypertarget{pmid_15271196}{S}eborrheic dermatitis is a common inflammatory skin disorder. Yeasts of the genus Malassezia have been implicated in the etiology of seborrheic dermatitis, although this connection remains controversial. Ciclopirox is a synthetic, hydroxypyridone-derived, broad-spectrum antifungal agent with anti-inflammatory properties. A total of 499 US patients with seborrheic dermatitis of the scalp were randomized to apply either ciclopirox shampoo 1\% or vehicle twice weekly for 4 weeks. The main efficacy parameters were based on 6-point ordinal scales describing the disease's signs and symptoms (scaling, erythema and itching) and a 6-point scale providing a global evaluation of the status of seborrheic dermatitis. Ciclopirox was significantly better than vehicle in effectively treating seborrheic dermatitis. 'Effective treatment' (score of 0 or 1 for disease status, scaling and erythema) was achieved in 26.0\% of ciclopirox-treated patients compared with 12.9\% of vehicle-treated patients (P = 0.0001; OR: 2.383, 95\% CI: 1.494-3.799). The majority of subjects experienced adverse events that were mild in intensity, with skin and appendage reactions the most commonly reported, at similar frequency in both groups. Ciclopirox shampoo 1\% is effective and safe in the treatment of seborrheic dermatitis of the scalp. [\hyperlink{Ciclopirox}{PMID: 15271196}, Mark Lebwohl et al., 2004]

\hypertarget{pmid_14567368}{B}ACKGROUND Tinea pedis (athlete's foot) is the most common fungal infection in the general population. Ciclopirox, a broad-spectrum hydroxypyridone antifungal, has proven efficacy against the organisms commonly implicated in tinea pedis; Trichophyton rubrum, T.mentagrophytes and Epidermophyton floccosum. Two multicenter, double-blind, clinical studies compared the efficacy and safety of ciclopirox gel with that of its vehicle base in subjects with moderate interdigital tinea pedis with or without plantar involvement. Three hundred and seventy-four subjects were enrolled and randomized to one of two treatment groups: ciclopirox gel 0.77\%, or ciclopirox gel vehicle, applied twice daily for 28 days, with a final visit up to day 50. The primary efficacy variable was Treatment Success defined as combined mycological cure and clinical improvement >/= 75\%. Secondary measures of effectiveness were Global Clinical Response, Sign and Symptom Severity Scores, Mycological Evaluation (KOH examination and final culture result), Mycological Cure (negative KOH and negative final culture results) and Treatment Cure (combined clinical and mycological cure). At endpoint (final post-baseline visit), 60\% of the ciclopirox subjects achieved treatment success compared to 6\% of the vehicle subjects. At the same time point, 66\% of ciclopirox subjects compared with 19\% of vehicle subjects were either cleared or had excellent improvement. Pooled data showed that 85\% of ciclopirox subjects were mycologically cured, compared to only 16\% of vehicle subjects at day 43, 2 weeks post-treatment. Ciclopirox gel 0.77\% applied twice daily for 4 weeks is an effective treatment of moderate interdigital tinea pedis due to T. rubrum, T. mentagrophytes and E. floccosum and is associated with a low incidence of minor adverse effects. [\hyperlink{Ciclopirox}{PMID: 14567368}, Raza Aly et al., 2003]

\hypertarget{pmid_8518000}{F}ive children with multiple relapsing steroid-dependent nephrotic syndrome were treated with continuous cyclosporin for periods ranging from 18 to 48 months. Renal biopsy showed mild mesangial proliferation in three of the children and minimal change in two. All children previously had been treated with cyclophosphamide. Cyclosporin was started during remission at 5 mg/kg per day. If a relapse occurred the dose was increased until a trough blood level of 100-250 ng/mL (HPLC) was achieved. In the initial 12 months of treatment, the mean number of relapses decreased from 6.4 +/- 0.54 (s.d.) per annum to 1.6 +/- 1.3 per annum (P < 0.01). Cyclosporin was effective in maintaining long-term remission in four of the five patients. Side effects included hypertrichosis (5) and gum hyperplasia (1). The mean creatinine clearance decreased from 126 +/- 16 to 97 +/- 22 mL/min per 1.73 m2 (P = NS). A renal biopsy in all five patients after 12 months therapy showed no nephrotoxicity. A further biopsy in one patient after 4 years therapy showed interstitial fibrosis. Cyclosporin should be considered in children with steroid-dependent nephrotic syndrome who show signs of steroid toxicity and have only a short remission period after cyclophosphamide. Serial renal biopsies are recommended if prolonged therapy is used. [\hyperlink{Ciclopirox}{PMID: 8518000}, K L Webb et al., 1993]

\hypertarget{pmid_12836096}{I}n a systematic review and meta-analysis of randomized controlled trials (RCT), we aimed to evaluate the benefits and harms of all interventions for children with steroid-resistant nephrotic syndrome (SRNS). Nine RCTs involving 225 children were included. Cyclosporin when compared with placebo or no treatment significantly increased the number of children who achieved complete remission [3 trials, 49 children, relative risk (RR) for persistent nephrotic syndrome 0.64, 95\% confidence intervals (CI), 0.47-0.88]. There was no significant difference in the number of children who achieved complete remission between oral cyclophosphamide with prednisone and prednisone alone [2 trials, 91 children, RR 1.01, 95\% CI 0.74-1.36], between intravenous cyclophosphamide and oral cyclophosphamide [1 study, 11 children, RR 0.09, 95\% CI 0.01-1.39], and between azathioprine with prednisone and prednisone alone [1 trial, 31 children, RR 1.01, 95\% CI 0.77-1.32]. No RCTs were identified comparing combination regimens comprising high-dose steroids, alkylating agents or cyclosporin with single agents, placebo, or no treatment. Further adequately powered and well-designed RCTs are needed to confirm the efficacy of cyclosporin and to evaluate regimens of high-dose steroids with alkylating agents or cyclosporin for SRNS. [\hyperlink{Ciclopirox}{PMID: 12836096}, Doaa Habashy et al., 2003]

\hypertarget{pmid_8881900}{C}yclosporin has been shown to be effective in the treatment of adult atopic dermatitis, but there are no clinical trials evaluating its use in childhood. Atopic dermatitis is more common in children and the severe form can be associated with considerable morbidity. We report on 18 children with severe refractory atopic dermatitis who have been treated with cyclosporin on an open basis. The drug was given at an initial daily dose of 5 or 6 mg/kg and in some patients the dose was reduced according to response. Sixteen patients showed a good or excellent response to treatment, one a moderate response and one patient failed to improve. The treatment was well tolerated and there were no significant changes in serum creatinine or blood pressure. Long remission after withdrawal of treatment was seen in some patients, although most relapsed within a few weeks. We suggest that cyclosporin is an effective and safe short-term treatment for severe atopic dermatitis in childhood. [\hyperlink{Ciclopirox}{PMID: 8881900}, I Zaki et al., 1996]

\hypertarget{pmid_11107710}{C}iclopirox 8\% nail lacquer has recently become the first topical antifungal agent to be approved by the US Food and Drug Administration for the treatment of onychomycosis. This article reviews the results of the two pivotal clinical trials of this drug that have been performed in the United States as well as those that have been carried out in other countries. The two US studies were both double-blind, vehicle-controlled, parallel-group, multicenter studies designed to determine the efficacy and safety of ciclopirox nail lacquer in the treatment of mild-to-moderate onychomycosis of the toenails caused by dermatophytes. The combined results show a 34\% mycologic cure rate, as compared with 10\% for the placebo. Data from the ten studies conducted worldwide show a meta-analytic mean (+/- SE) mycologic cure rate of 52.6\% +/- 4.2\%. As expected for a topical agent, ciclopirox nail lacquer was found to be extremely safe, with mild, transient irritation at the site of application reported as the most common adverse event. Ciclopirox nail lacquer may also have potential for use in combination or adjunctive therapy. Further studies will help to better position this agent for the treatment of this widespread podiatric condition. [\hyperlink{Ciclopirox}{PMID: 11107710}, A K Gupta et al., ]

\hypertarget{pmid_11051135}{C}iclopirox is a synthetic hydroxypyridone antifungal agent. In contrast to the azoles, glucuronidation is the main metabolic pathway of ciclopirox; therefore interactions with drugs metabolized via the cytochrome P450 system are unlikely Ciclopirox is also distinct from the common systemic agents, which interfere with sterol biosynthesis. In fact, ciclopirox chelates trivalent cations (such as Fe3+), inhibits metal-dependent enzymes that are responsible for degradation of toxic metabolites in the fungal cells, and targets diverse metabolic (eg, respiratory) and energy producing processes in microbial cells. Ciclopirox is a broad spectrum antimicrobial with activity against all the usual dermatophytes as well as yeast and nondermatophyte molds. It has demonstrated activity against gram positive and negative bacteria, including resistant strains of Staphlococcus aureus. Ciclopirox exhibits fungal inhibitory activity (minimum inhibitory concentration < 4 microg/mL for dermatophytes) as well as fungicidal activity; to date resistance to the drug has not been identified. Ciclopirox has been formulated in a nail lacquer delivery system. After evaporation of volatile solvents in the lacquer, the concentration of ciclopirox in the remaining lacquer film reaches approximately 35\%, providing a high concentration gradient for penetration into the nail. Radiolabel data demonstrate penetration into infected nails after only 1 application of the lacquer. Ciclopirox nail lacquer is a topical product that provides an active fungicidal agent in a delivery system capable of promoting nail penetration. With repeated applications, the antifungal agent is homogeneously distributed through all layers of the toenail achieving concentrations of ciclopirox in excess of inhibitory and fungicidal concentrations for most pathogens. Although ciclopirox readily penetrates nails, very low levels of ciclopirox are recoverable systemically, even after chronic use. Ciclopirox nail lacquer 8\% is a topical product that provides an active fungicidal agent in a delivery system capable of penetrating nails. [\hyperlink{Ciclopirox}{PMID: 11051135}, M Bohn et al., 2000]

\hypertarget{pmid_1868743}{W}e report on our experience with acyclovir (Zovirax) capsules for the symptomatic treatment of chickenpox in two children and an adult. [\hyperlink{Ciclopirox}{PMID: 1868743}, C Marino et al., 1991]

\hypertarget{pmid_8647967}{S}evere atopic dermatitis (AD) remains difficult to treat. Cyclosporine is effective in adults but has not previously been investigated in children with AD. The aims were to investigate the efficacy, safety, and tolerability of cyclosporine in severe refractory childhood AD. Subjects 2 to 16 years of age were treated for 6 weeks with cyclosporine, 5 mg/kg per day, in an open study. Disease activity was monitored every 2 weeks by means of sign scores, visual analogue scales for symptoms, and quality-of-life questionnaires. Adverse events were monitored. Efficacy and tolerability were assessed with five-point scales. Twenty-seven children were treated. Significant improvements were seen in all measures of disease activity. Twenty-two showed marked improvement or total clearing. Quality of life improved for both the children and their families. Tolerability was considered good or very good in 25 subjects. Cyclosporine may offer an effective, safe, and well-tolerated short-term treatment option for children with severe AD. [\hyperlink{Ciclopirox}{PMID: 8647967}, J Berth-Jones et al., 1996]

\hypertarget{pmid_12895183}{S}eborrheic dermatitis is a common inflammatory skin disorder that usually occurs in patients with pre-existing seborrhea. The etiology of seborrheic dermatitis is uncertain. Typically, sites dense with sebaceous glands support growth of the lipophilic yeast Malassezia furfur. Ciclopirox (Loprox) gel is a hydroxypyridone, broad-spectrum antifungal agent proven effective against the yeast M. furfur. A multicenter, randomized, double-blind, vehicle controlled study of 178 subjects evaluated the efficacy of ciclopirox gel in treating seborrheic dermatitis of the scalp. One hundred and seventy-eight subjects were randomized to apply either ciclopirox gel 0.77\% twice daily, or vehicle twice daily for 28 days. Subjects' signs and symptoms of severity (erythema, scaling, pruritus and burning) were rated on a scale of 0-3 (none to severe); for inclusion, a minimum score of 4, for the sum of the individual ratings was required. Efficacy evaluations were performed at baseline, days 4, 8, 15, 22, 29, and at end-point (final visit, up to day 33). The primary efficacy variable was clinical response assessed by a global improvement, based on a scale of 0-5 (100\% clearance to flare of treatment area). Changes in signs/symptoms severity scores within the target lesion were also evaluated. Global evaluation scores demonstrated that significantly more ciclopirox-treated subjects achieved over 75\% improvement compared with vehicle at days 22, 29, and endpoint (P < 0.01). Change-from-baseline mean score for total signs and symptoms was significantly greater in ciclopirox subjects compared with vehicle subjects at the same time points as above (P < 0.001), as well as day 15 (P < 0.01). Twenty-nine percent of subjects rated ciclopirox as having excellent cosmetic acceptability. There were only mild adverse events, with the most common being burning sensation in 13\% of ciclopirox subjects and 9\% of vehicle subjects. Ciclopirox gel is effective and safe in the treatment of seborrheic dermatitis of the scalp. [\hyperlink{Ciclopirox}{PMID: 12895183}, Raza Aly et al., 2003]

\hypertarget{pmid_25059452}{C}yclosporine is a systemic therapy used for control of severe atopic dermatitis (AD) in children. Although traditionally recommended at a dose of 5 mg/kg/day for 6 months, a longer duration of treatment may be necessary to bring a child with active and severe disease into remission. There are few data on the short- and long-term effectiveness of longer courses of therapy. This was a retrospective chart review of children treated with cyclosporine at a Canadian hospital-affiliated clinic between 2000 and 2013. Fifteen patients with adequate follow-up were identified. Twelve (80\%) were male and the mean age at initiation of cyclosporine was 11.2 ± 3.4 years. The mean duration of cyclosporine therapy was 10.9 ± 2.7 months (range 7-15 months) at a starting dose of 2.8 ± 0.6 mg/kg/day. Of 12 patients (80\%) who responded to cyclosporine, 5 patients (42\%) had relapsed at a follow-up of 22.7 ± 15.0 months. The duration of therapy was longer in patients who did not relapse (17.7 ± 10.7 months) than in those who did (10.2 ± 2.7 months) (p = 0.06). Adverse events led to discontinuation in three patients (20\%) and included infection-related complications in two patients and reversible renal toxicity in one. These results suggest that a longer duration of low-dose cyclosporine may help decrease the risk of relapse in patients with severe AD who are resistant to topical therapies. [\hyperlink{Ciclopirox}{PMID: 25059452}, Cathryn Sibbald et al., ]

\hypertarget{pmid_23700934}{T}wo hundred fifty patients, including 100 children with frequent and prolonged diseases at the age of 4 to 7 years, 76 children at the age of 7 to 18 years and 74 subjects at the age of 22 to 57 years were observed. The patients were treated with cycloferon in two courses with a 2-week interval according to the standard scheme. The tonsil surface microflora and its susceptibility to antibiotics were determined. Cycloferon lowered the Staphylococcus aureus titre and increased the culture susceptibility to benzylpenicillin, oxacillin, rifampicin, and erythromycin, reducing the variety of the fauces nonpathogenic microflora. The use of cycloferon induced no adverse (pathologic) reactions in 94.8\% of the cases. In 4.4\% of the children under school age the adverse reactions were transitory and did not require discontinuation of the drug use. Unforeseen reactions were recorded in 0.8\% of the children and the use of the drug in them was discontinued. The use of cycloferon in two courses with a 2-week interval according to the standard scheme is recommended for prophylaxis of acute respiratory diseases in the group of children with frequent and prolonged diseases during epidemiologically unfavourable periods and for complex therapy of rhinopharinx infections as an agent increasing efficacy of other antibacterials. [\hyperlink{Ciclopirox}{PMID: 23700934}, S A Lialikov et al., 2012]

\section*{Alitretinoin}
\subsection*{Result}
\subsubsection*{Answer}

Ages <5.8 years: Unknown  
Ages 5.8–15.8 years: Yes  
Ages >15.8 years: Unknown  

\subsubsection*{{Explanation}}
\hypertarget{Alitretinoin}
A review of the available abstracts reveals the following evidence regarding the safety of alitretinoin in children:

1. One targeted study specifically evaluated the safety and efficacy of alitretinoin in children and adolescents under 18 years of age. This retrospective chart review included 13 children (median age 11.5 years, range 5.8-15.8 years) treated with alitretinoin for chronic hand eczema (CHE) and other inflammatory skin diseases. The study found that alitretinoin was "highly effective and safe" for pediatric CHE, with good tolerability overall. The most common adverse event was headache (77\% of patients), which led to therapy interruption in one subject. The authors concluded that alitretinoin "should thus be considered in children with refractory disease under topical therapy," but also noted that larger studies are required to corroborate these findings [\hyperlink{pmid_31742759}{PMID: 31742759}, I Luchsinger et al., 2020].

2. Other abstracts on alitretinoin focus exclusively on adult populations or do not specify inclusion of children, and therefore do not provide evidence for or against safety in pediatric age groups [\hyperlink{pmid_25886084}{PMID: 25886084}, Marc Morris et al., 2016; \hyperlink{pmid_21443602}{PMID: 21443602}, A H Schmitt-Hoffmann et al., 2011; \hyperlink{pmid_15611422}{PMID: 15611422}, Thomas Ruzicka et al., 2004; \hyperlink{pmid_21070335}{PMID: 21070335}, T Dirschka et al., 2011].

Summary by age range:
- Ages 5.8–15.8 years: There is limited but direct evidence from a small retrospective study suggesting alitretinoin is safe for use in children within this age range, specifically for chronic hand eczema and certain other inflammatory skin diseases. However, the sample size is small, and the authors recommend larger studies for confirmation.
- Ages <5.8 years: No data are available in the abstracts regarding safety in children younger than 5.8 years.
- Ages 16–17 years: The study included adolescents up to 15.8 years; there is no direct evidence for those aged 16–17 years, but the upper end of the studied range approaches this group.

Conclusion: For children aged 5.8–15.8 years, there is some evidence supporting safety, but it is based on a small cohort and requires further confirmation. For children younger than 5.8 years or older than 15.8 years, safety is unknown due to lack of data.

\subsection*{Abstracts}
\hypertarget{pmid_31742759}{A}litretinoin is a systemic retinoid licensed for use in adult patients suffering from chronic hand eczema recalcitrant to potent topical steroids. Experience with its use in childhood is lacking. To report on the efficacy and safety of alitretinoin treatment in a cohort of children and adolescents with chronic hand eczema (CHE) and other inflammatory skin diseases. We performed a retrospective chart review of all consecutive patients under the age of 18 years treated with alitretinoin at our paediatric skin centre. Physician's Global Assessment (PGA) was used as the primary outcome measure. Thirteen children (9 girls and 4 boys) were enrolled in this study. The median age at start of treatment with alitretinoin was 11.5 years (range 5.8-15.8 years). Nine children were diagnosed with CHE, two with severe atopic dermatitis (AD), and two with inherited ichthyosis [netherton syndrome (NS), autosomal recessive congenital ichthyosis (ARCI)]. Moderate to excellent response (PGA decrease of ≥1 point) was observed in 7 (78\%) of the nine patients with CHE, one of the two patients with extensive AD and in the one patient with ARCI. In the remaining four subjects, no convincing effect was documented. Tolerability was overall very good. The most common adverse event was headache in 10 patients (77\%) during the initiation of treatment, leading to interruption of therapy in one subject. Alitretinoin seems to be highly effective and safe for the treatment of paediatric CHE and should thus be considered in children with refractory disease under topical therapy. Larger studies are required to corroborate these findings. [\hyperlink{Alitretinoin}{PMID: 31742759}, I Luchsinger et al., 2020]

\hypertarget{pmid_31321320}{A}zithromycin is widely used in children not only in the treatment of individual children with infectious diseases, but also as mass drug administration (MDA) within a community to eradicate or control specific tropical diseases. MDA has also been reported to have a beneficial effect on child mortality and morbidity. However, concerns have been raised about the safety of azithromycin, especially in young children. The aim of this review is to systematically identify the safety of azithromycin in children of all ages. MEDLINE, PubMed, Cochrane Central Register of Controlled Trials, Embase, CINAHL, International Pharmaceutical Abstracts and adverse drug reaction (ADR) monitoring systems will be systematically searched for randomised controlled trials (RCTs), cohort studies, case-control studies, cross-sectional studies, case series and case reports evaluating the safety of azithromycin in children. The Cochrane risk of bias tool, Newcastle-Ottawa and quality assessment tools, and The Joanna Briggs Institute Critical Appraisal tools will be used for quality assessment. Meta-analyses will be conducted to the incidence of ADRs from RCTs if appropriate. Subgroup analyses will be performed in different age and azithromycin dosage groups. Formal ethical approval is not required as no primary data are collected. This systematic review will be disseminated through a peer-reviewed publication. CRD42018112629. [\hyperlink{Alitretinoin}{PMID: 31321320}, Peipei Xu et al., 2019]

\hypertarget{pmid_25886084}{A}litretinoin is approved for the treatment of adults with severe chronic hand eczema (CHE) refractory to potent topical steroids. In the 6 years since launch, approximately 250 000 patients have been treated with alitretinoin. To compare the postmarketing safety surveillance experience of alitretinoin with data from clinical trials and key safety issues with other retinoids. An integrated safety analysis of the pivotal studies of alitretinoin and postmarketing adverse event (AE) reports received since approval for alitretinoin were analyzed. In the pivotal trials, headache, erythema, nausea, increased blood triglycerides and increased blood creatinine phosphokinase were the most frequently reported AEs. Headache, hyperlipidemia and nausea were also frequently reported postmarketing AEs, but depression was relatively more frequently reported than in the pivotal trials. Inflammatory bowel disease and benign intracranial hypertension were rare, and very few cases have been reported in postmarketing surveillance. There have been no reports of teratogenicity in humans consequent to fetal exposure. Safety data collected in pivotal trials and postmarketing surveillance suggest that alitretinoin is well tolerated by patients with CHE with a relatively low incidence of serious reactions. The adverse reaction profile is congruent with reported effects of other marketed oral retinoids. [\hyperlink{Alitretinoin}{PMID: 25886084}, Marc Morris et al., 2016]

\hypertarget{pmid_17561929}{T}here are more than 40 H(1)-antihistamines available worldwide. Most of these medications have never been optimally studied in prospective, randomized, double-masked, placebo-controlled trials in children. The aim was to perform a long-term study of levocetirizine safety in young atopic children. In the randomized, double-masked Early Prevention of Asthma in Atopic Children Study, 510 atopic children who were age 12-24 months at entry received either levocetirizine 0.125 mg/kg or placebo twice daily for 18 months. Safety was assessed by: reporting of adverse events, numbers of children discontinuing the study because of adverse events, height and body mass measurements, assessment of developmental milestones, and hematology and biochemistry tests. The population evaluated for safety consisted of 255 children given levocetirizine and 255 children given placebo. The treatment groups were similar demographically, and with regard to number of children with: one or more adverse events (levocetirizine, 96.9\%; placebo, 95.7\%); serious adverse events (levocetirizine, 12.2\%; placebo, 14.5\%); medication-attributed adverse events (levocetirizine, 5.1\%; placebo, 6.3\%); and adverse events that led to permanent discontinuation of study medication (levocetirizine, 2.0\%; placebo, 1.2\%). The most frequent adverse events related to: upper respiratory tract infections, transient gastroenteritis symptoms, or exacerbations of allergic diseases. There were no significant differences between the treatment groups in height, mass, attainment of developmental milestones, and hematology and biochemistry tests. The long-term safety of levocetirizine has been confirmed in young atopic children. [\hyperlink{Alitretinoin}{PMID: 17561929}, F Estelle R Simons et al., 2007]

\hypertarget{pmid_9091512}{A}lthough topically applied all-trans-retinoic acid (tretinoin) undergoes minimal absorption and adds negligibly to normal endogenous levels, its safety in humans is occasionally questioned because oral ingestion of retinoids at therapeutic levels is known to entail teratogenic risks. To assess the actual potential for developmental toxicity from treatment with topical tretinoin. Risk assessments were conducted on four known human developmental toxicants (valproic acid, methotrexate, thalidomide, and isotretinoin) and a potential developmental toxicant (acetylsalicylic acid). The margin of safety for each chemical was calculated from the ratio of animal no-observed adverse effect levels to human lowest-observed adverse effect levels or estimated exposure doses. The derived safety margin of more than 100 for topical tretinoin (with 2\% absorption) contrasted sharply with the near unity values for valproic acid, methotrexate, thalidomide, and isotretinoin and was larger than that for acetylsalicylic acid. These data support other epidemiologic and animal data that topical tretinoin is not a potential human developmental toxicant. [\hyperlink{Alitretinoin}{PMID: 9091512}, E M Johnson et al., 1997]

\hypertarget{pmid_29098909}{T}here is a few evidence-based information regarding the efficacy and safety of acitretin treatment in children with pustular psoriasis (PP). This study aimed to provide an additional evidence for this field. A retrospective study was undertaken for 15 children with PP who received acitretin in doses of 0.6-1.0 mg/kg/day for 4-6 weeks, the transition dose of 0.2-0.4 mg/kg/day for 4-6 weeks and maintenance dose of 0.2-0.3 mg/kg/day. Additionally, a literature review on this topic is conducted. Of 15 children with generalized PP (GPP, n = 10), palmoplantar psoriasis (PPP, n = 3), and acrodermatitis continua of Hallopeau (ACH, n = 2), 93.3\% (14/15) showed good response, only one case with ACH exhibited moderate response. During the 10-32 months of follow-up, acitrerin monotherapy for children cases with PP overall showed good efficacy and safety. In the literature review, a total of 107 childhood PP cases treated with acitretin in 21 studies were included in the analysis. The clinical effectiveness was obtained in 88.8\% (95/107) patients treated with acitretin as monotherapy or combination therapy, and most of cases (92.6\%, 100/107) treated by acitretin did not report side effects during the treatment and follow-up of acitretin. This study is just included a small sample sizes and no standardized studies were used in the literature. Acitretin therapy for children with PP (monotherapy or combination therapy), all showed a satisfactory therapeutic effect and safety, independent of the short or long-tern therapeutic procedures. [\hyperlink{Alitretinoin}{PMID: 29098909}, Pingjiao Chen et al., 2018]

\hypertarget{pmid_16028153}{B}ecause of concerns about arthrotoxicity, fluoroquinolones are restricted for use in children. This study describes the safety and efficacy of gatifloxacin when used for treatment of children with recurrent acute otitis media (ROM) or acute otitis media (AOM) treatment failure (AOMTF). We performed an analysis of 867 children included in 4 clinical trials who had ROM and/or AOMTF and were treated with gatifloxacin (10 mg/kg once daily for 10 days). Gatifloxacin had adverse event rates that were similar overall to those of a comparator antibiotic (amoxicillin-clavulanate), except for increased diarrhea in children <2 years old receiving amoxicillin-clavulanate. There was no evidence of arthrotoxicity, hepatotoxicity, alteration of glucose homeostasis, or central nervous system toxicity acutely or during 1 year follow-up in any child. Regarding efficacy, in 2 noncomparative trials, the gatifloxacin cure rate of AOM was 89\% (95\% confidence interval [CI], 83\%-95\%) at the test of cure (TOC) visit, 3-10 days after completion of therapy. In 2 comparative trials of gatifloxacin versus amoxicillin-clavulanate, the efficacy of gatifloxacin was 88\% (95\% CI, 82\%-94\%). Gatifloxacin led to better clinical outcomes than amoxicillin-clavulanate for AOMTF (91\% vs. 81\%; P=.029), for AOMTF and age <2 years old (89\% vs. 69\%; P=.009), and for severe AOM in children <2 years old (90\% vs. 75\%; P=.012). Among children with AOMTF previously treated with amoxicillin-clavulanate or ceftriaxone injections, gatifloxacin cure rates were high (88\% and 75\%, respectively). Gatifloxacin appears to be safe for children, with no evidence of producing arthrotoxicity in 867 children exposed to the antibiotic when used as treatment for ROM and AOMTF. [\hyperlink{Alitretinoin}{PMID: 16028153}, Michael E Pichichero et al., 2005]

\hypertarget{pmid_20819318}{A}llergic rhinitis (AR) and chronic idiopathic urticaria (CIU) are common causes of substantial illness and disability in preschool children. Antihistamines are commonly used to treat preschool children with these conditions, but their use is based mostly on extrapolated efficacy from adult populations; it is thus important to characterize the safety of antihistamines in the pediatric population. This study was designed to assess the safety of levocetirizine dihydrochloride oral liquid drops in infants and children with AR or CIU. Two multicenter, double-blind, randomized, parallel-group studies randomized infants aged 6-11 months (study 1, n = 69) and children aged 1-5 years (study 2, n = 173) to levocetirizine, 1.25 mg (q.d. or b.i.d., respectively), or placebo for 2 weeks, using a 2:1 ratio. Safety evaluations included treatment-emergent adverse events (TEAEs), vital signs, electrocardiographic (ECG) assessments, and laboratory tests. The overall incidence of TEAEs was similar between levocetirizine and placebo in both studies. Most TEAEs were mild or moderate in intensity. TEAEs prompted discontinuation of therapy in three patients receiving levocetirizine in study 1. No clinically relevant changes from baseline in vital signs or laboratory parameters were apparent in either study; changes from baseline in these evaluations were similar between groups. No significant changes were observed in ECG parameters, including corrected QT interval. Levocetirizine, 1.25 and 2.5 mg/day, was well tolerated in infants aged 6-11 months and in children aged 1-5 years, respectively, with AR or CIU. [\hyperlink{Alitretinoin}{PMID: 20819318}, Frank Hampel et al., ]

\hypertarget{pmid_7901684}{T}retinoin is effective in acute promyelocytic leukaemia in adults. Data about its efficacy and safety in children are limited. We have treated 9 children with tretinoin at 45 mg/m2 per day. Pseudotumour cerebri or hyperleucocytosis occurred in 5 patients. Retinoic acid syndrome was seen in 3 cases. 1 of 2 children who developed hyperleucocytosis, pseudotumour cerebri, and retinoic acid syndrome died despite steroids and mechanical ventilation. Complete remissions with tretinoin alone were achieved in 15 patients. All 8 surviving children received consolidation chemotherapy. Our experience with tretinoin therapy suggests that toxicity is frequent in children. [\hyperlink{Alitretinoin}{PMID: 7901684}, H H Mahmoud et al., 1993]

\hypertarget{pmid_10563619}{T}o compare the safety and efficacy of add-on lamotrigine and placebo in the treatment of children and adolescents with partial seizures. Add-on and monotherapy lamotrigine is safe and effective in adults with partial seizures, and reports of preliminary uncontrolled trials suggest similar benefits in children. We studied 201 children with diagnoses of partial seizures of any subtype currently receiving stable conventional regimens of antiepileptic therapy at 40 study sites in the United States and France. After a baseline observation period (to confirm that more than four seizures occurred in each of two consecutive 4-week periods), patients were randomized to add-on lamotrigine or placebo therapy. A 6-week dose-escalation period was followed by a 12-week maintenance period. Compared with placebo, lamotrigine significantly reduced the frequency of all partial seizures and the frequency of secondarily generalized partial seizures in these treatment-resistant patients. The most commonly reported adverse events in the lamotrigine-treated patients were vomiting, somnolence, and infection; the frequency of these and other adverse events was similar to that in the placebo-treated group, with the exception of ataxia, dizziness, tremor, and nausea, which were more frequent in the lamotrigine-treated group. The frequency of withdrawals for adverse events was similar between groups. Two patients were hospitalized for skin rash, which resolved after discontinuation of lamotrigine therapy. Lamotrigine was effective for the adjunctive treatment of partial seizures in children and demonstrated an acceptable safety profile. [\hyperlink{Alitretinoin}{PMID: 10563619}, M Duchowny et al., 1999]

\hypertarget{pmid_29599698}{A}litretinoin is a new oral retinoid authorized for use in grownups that have severe chronic hand eczema (CHE). A comprehensive search to solicit all studies of alitretinoin for the treatment of CHE. A comprehensive search to solicit all studies of alitretinoin for the treatment of CHE including randomized controlled trials (RCTs) or uncontrolled trials, re-treatment studies, open-label studies, or observational studies, along with case series of 10 or more participants. Physician global assessment (PGA), patient global assessment (PaGA) and modified total lesion symptom score (mTLSS) are the methods and outcomes criteria. Generated effect size and 95\% confidence intervals were calculated for the outcomes. Heterogeneity and publication bias were also tested for all selected trials. When a noteworthy Q statistic ( [\hyperlink{Alitretinoin}{PMID: 29599698}, Mohammed Saleh Al-Dhubaibi et al., ] Recent studies have found that alitretinoin can induce clinically significant responses in subjects with severe chronic hand eczema (CHE) unresponsive to topical corticosteroids. To assess the pharmacokinetics (PK), efficacy and safety of alitretinoin 10 or 30 mg once daily. This was a randomized, double-blind study, which enrolled 32 subjects aged 18-75 years with CHE unresponsive to potent topical corticosteroids. Subjects received alitretinoin 10 mg (n = 16) or 30 mg (n = 16) once daily for 12 or 24 weeks. Standard PK variables [area under the curve (AUC) of plasma concentration vs. time, maximum plasma concentration (C(max)), time to maximum plasma concentration (t(max)), elimination half-life (t(1/2)), total systemic clearance (CL/F) and volume of distribution (Vd/F)] were determined for alitretinoin and metabolites. Efficacy was assessed using the Physician's Global Assessment (PGA) scale. Chronic administration of alitretinoin for up to 24 weeks did not result in accumulation or time-dependent changes in the disposition of alitretinoin. Exposure was found to be proportional to dose. Systemic exposure (AUC) to alitretinoin was proportional to dose for 10 and 30 mg alitretinoin; 62.8\% of subjects achieved clear/almost clear hands in the 30 mg group and 12.5\% in the 10 mg group. Alitretinoin was well tolerated. Chronic administration of alitretinoin for 12-24 weeks did not lead to accumulation or time-dependent changes in drug exposure. Alitretinoin was effective and well tolerated in the treatment of subjects with moderate or severe CHE unresponsive to potent topical corticosteroids. [\hyperlink{Alitretinoin}{PMID: 29599698}, A H Schmitt-Hoffmann et al., 2011]

\hypertarget{pmid_15611422}{T}o assess the efficacy and safety of oral alitretinoin (9-cis-retinoic acid), 10 mg/d, 20 mg/d, and 40 mg/d, compared with placebo control, in the treatment of chronic hand dermatitis. Multicenter, randomized, double-blind, placebo-control, prospective trial. A total of 43 outpatient clinics in 10 European countries. Of 348 patients screened, 319 with moderate or severe refractory chronic hand dermatitis were randomized, in the ratio of 1:1:1:1, to 4 treatment groups and received allocated intervention. Of 75 patients who withdrew, 24 withdrew owing to adverse events. Placebo or 10 mg, 20 mg, or 40 mg of oral alitretinoin (9-cis-retinoic acid) taken once daily for 12 weeks. Safety was assessed for all patients during a follow-up period of 4 weeks, and responders were observed for a follow-up period of 3 months. Physician's global assessment of overall chronic hand dermatitis severity. Alitretinoin led to a significant and dose-dependent improvement in disease status, with responses in up to 53\% of patients, and up to a 70\% mean reduction in disease signs and symptoms. Treatment was generally well tolerated, with dose-dependent effects comprising headache, flushing, mucocutaneous events, hyperlipidemia, and decreased hemoglobin and decreased free thyroxin levels. Three months after discontinuation of treatment, the rate of relapse was 26\%, independent of dose. Alitretinoin given at well-tolerated doses induced substantial clearing of chronic hand dermatitis in patients refractory to conventional therapy. [\hyperlink{Alitretinoin}{PMID: 15611422}, Thomas Ruzicka et al., 2004]

\hypertarget{pmid_25135766}{R}ecently, an association between childhood growth stunting and aflatoxin (AF) exposure has been identified. In Ghana, homemade nutritional supplements often consist of AF-prone commodities. In this study, children were enrolled in a clinical intervention trial to determine the safety and efficacy of Uniform Particle Size NovaSil (UPSN), a refined calcium montmorillonite known to be safe in adults. Participants ingested 0.75 or 1.5 g UPSN or 1.5 g calcium carbonate placebo per day for 14 days. Hematological and serum biochemistry parameters in the UPSN groups were not significantly different from the placebo-controlled group. Importantly, there were no adverse events attributable to UPSN treatment. A significant reduction in urinary metabolite (AFM1) was observed in the high-dose group compared with placebo. Results indicate that UPSN is safe for children at doses up to 1.5 g/day for a period of 2 weeks and can reduce exposure to AFs, resulting in increased quality and efficacy of contaminated foods.  [\hyperlink{Alitretinoin}{PMID: 25135766}, Nicole J Mitchell et al., 2014] Infants and children under five years of age are the most vulnerable to malaria with over 1,700 deaths per day from malaria in this group. However, until recently, there were no WHO-endorsed paediatric anti-malarial formulations available. Artemisinin-based combination therapy is the current standard of care for patients with uncomplicated falciparum malaria in Africa. Artemether/lumefantrine (AL) meets WHO pre-qualification criteria for efficacy, safety and quality. Coartem, a fixed dose combination of artemether and lumefantrine, has consistently achieved cure rates of >95\% in clinical trials. However, AL tablets are inconvenient for caregivers to administer as they need to be crushed and mixed with water or food for infants and young children. Further, in common with other anti-malarials, they have a bitter taste, which may result in children spitting the medicine out and not receiving the full therapeutic dose. There was a clear unmet medical need for a formulation of AL specifically designed for children. Ahead of a call from WHO for child-friendly medicines, Novartis, working in partnership with Medicines for Malaria Venture (MMV), started the development of a new formulation of AL for infants and young children: Coartem Dispersible. The excellent efficacy, safety and tolerability already demonstrated by AL tablets were confirmed with dispersible AL in a large trial comparing the crushed tablets with dispersible tablets in 899 African children with falciparum malaria. In the evaluable population, 28-day PCR-corrected cure rates of >96\% were achieved. Further, its sweet taste means that it is palatable for children, and the dispersible formulation makes it easier for caregivers to administer than bitter crushed tablets. Easing administration may foster compliance, hence improving therapeutic outcomes in infants and young children and helping to preserve the efficacy of ACT. [\hyperlink{Alitretinoin}{PMID: 25135766}, Salim Abdulla et al., 2009]

\hypertarget{pmid_21443599}{A}litretinoin, like all retinoids, is teratogenic, and can only be given to women of childbearing potential if pregnancy is excluded and a strict contraceptive programme is followed. This study was designed to determine whether alitretinoin in the semen of men treated with alitretinoin poses a teratogenic risk to their female partners. In total, 24 healthy men aged 18-45 years received alitretinoin 20 mg (n = 12) or 40 mg (n = 12), once daily for 14 days. Subjects in the 40 mg dose group provided ejaculate at baseline, on day 1, before and approximately 4 h after dosing on day 2, and at follow-up on study day 21 (± 2). Alitretinoin and 4-oxo-alitretinoin were detected in 11 of the 12 semen samples. The highest level of alitretinoin in semen was 7.92 ng/mL. Assuming an ejaculate volume of 10 mL, the amount of drug transferred in semen would be about 80 ng, 1/375,000 of a single 30 mg capsule. Complete absorption of 80 ng of alitretinoin from semen, presuming a volume of distribution confined to 5 L of circulating blood in the partner, would lead to an increase in plasma alitretinoin concentration of 0.016 ng/mL, which appears to be negligible compared with measured endogenous plasma levels. Increases in plasma levels of related retinoids are also negligible. Alitretinoin in the semen of men receiving up to 40 mg of oral alitretinoin per day is unlikely to be associated with teratogenic risk in their female partners. Barrier contraception is therefore not required for men taking alitretinoin. [\hyperlink{Alitretinoin}{PMID: 21443599}, A H Schmitt-Hoffmann et al., 2011]

\hypertarget{pmid_21070335}{B}linded, controlled studies have found that oral alitretinoin is well tolerated and effective in the treatment of severe chronic hand eczema (CHE). To assess the safety and efficacy of oral alitretinoin in patients with severe CHE in an open-label study using flexible dosing and a new measurement of patient-relevant benefits. In total, 249 patients aged 18-75 years with severe CHE unresponsive to treatment with topical corticosteroids received alitretinoin 30 mg once daily for up to 24 weeks. Safety assessments included adverse events (AEs) and laboratory tests. Efficacy assessments included Physician's Global Assessment (PGA), the Modified Total Lesion Symptom Score, Patient's Global Assessment and extent of disease, as well as intensity of pain and pruritus as determined by visual analogue scale (VAS) and a categorical scale for pruritus. Alitretinoin was well tolerated when given for up to 24 weeks. Dose reduction occurred in 16.5\% of patients. Dose interruption was required for 15.7\% of patients, most commonly for headache. AEs and laboratory changes comprised effects typical of the retinoid class. A PGA response of 'clear' or 'almost clear' hands was reported for 46.6\% of patients, similar to the response rate seen in blinded trials. Results of VAS and categorical assessments of pruritus provided supporting evidence of efficacy, and treatment was assessed as providing meaningful benefits to patients. Oral alitretinoin 30 mg was well tolerated and effective, and provided distinct therapeutic benefits in severe CHE, as assessed by patients. [\hyperlink{Alitretinoin}{PMID: 21070335}, T Dirschka et al., 2011]

\hypertarget{pmid_24175945}{T}he purpose of the study was to compare the safety of artemether-lumefantrine (AL) with other artemisinin-based combinations in children. A search of EMBASE (from 1974 to April 2013), MEDLINE (from 1946 to April 2013) and the Cochrane library of registered controlled trials for randomized controlled trials (RCTs) which compared AL with other artemisinin-based combinations was done. Only studies involving children ≤ 17 years old in which safety of AL was an outcome measure were included. Four thousand, seven hundred and twenty six adverse events (AEs) were recorded in 6,000 patients receiving AL. Common AEs (≥ 1/100 and <1/10) included: coryza, vomiting, anaemia, diarrhoea, vomiting and abdominal pain; while cough was the only very commonly reported AE (≥ 1/10). AL-treated children have a higher risk of body weakness (64.9\%) than those on artesunate-mefloquine (58.2\%) (p = 0.004, RR: 1.12 95\% CI: 1.04-1.21). The risk of vomiting was significantly lower in patients on AL (8.8\%) than artesunate-amodiaquine (10.6\%) (p = 0.002, RR: 0.76, 95\% CI: 0.63-0.90). Similarly, children on AL had a lower risk of vomiting (1.2\%) than chlorproguanil-dapsone-artesunate (ACD) treated children (5.2\%) (p = 0.002, RR: 0.63, 95\% CI: 0.47-0.85). The risk of serious adverse events was significantly lower for AL (1.3\%) than ACD (5.2\%) (p = 0.002, RR: 0.45, 95\% CI: 0.27-0.74). Artemether-lumefantrine combination is as safe as ASAQ and DP for use in children. Common adverse events are cough and gastrointestinal symptoms. More studies comparing AL with artesunate-mefloquine and artesunate-azithromycin are needed to determine the comparative safety of these drugs. [\hyperlink{Alitretinoin}{PMID: 24175945}, Oluwaseun Egunsola et al., 2013]

\hypertarget{pmid_27443789}{A}citretin is licensed for and is most commonly used to treat psoriasis. Little information exists about its efficacy and safety in childhood and adolescent psoriasis. Retrospective analysis of a group of children and adolescents (<17 years of age) with moderate to severe plaque psoriasis treated with acitretin between 2010 and 2014 at Italian dermatology clinics. Patients were identified through databases or registries. The study population consisted of 18 patients with a median age of 9.5 years at the start of therapy. The median maintenance dosage per day was 0.41 mg/kg. Eight patients (44.4\%) achieved complete clearance or good improvement of their psoriasis, defined as improvement from baseline of 75\% or more on the Psoriasis Area and Severity Index at week 16. Three had three or more courses of treatment with short disease-free intervals. In three patients, acitretin treatment was ongoing at the time of data collection. The mean total duration of treatment in responders was 22.7 months. One patient discontinued treatment because of arthralgia. The remaining nine patients (50\%) discontinued treatment because it was ineffective. Mucocutaneous adverse effects occurred in all patients, but did not affect therapy maintenance. In this retrospective case series, acitretin was a moderately effective, well-tolerated treatment in children with moderate to severe plaque psoriasis. Given the small number of patients, statements about long-term safety are not possible. [\hyperlink{Alitretinoin}{PMID: 27443789}, Vito Di Lernia et al., 2016]

\hypertarget{pmid_29379562}{T}he aim of the present research was to compare the effectiveness and tolerability of melatonin and amitriptyline in pediatric migraine prevention. In a parallel single-blinded randomized clinical trial, 5-15 yr old children with diagnosis of migraine that preventive therapy was indicated in whom and were referred to Pediatric Neurology Clinic of Shahid Sadoughi Medical Sciences University, Yazd-Iran from 2013-2014, were randomly allocated to receive 1 mg/kg amitriptyline or 0.3 mg/kg melatonin for three consecutive months. Forty one girls (51.3\%) and 39 boys (48.7\%) with mean age of 10.44±2.26 yr were evaluated. Good response was seen in 82.5\% of amitriptyline and 62\%.5 of melatonin groups and amitriptyline was statistically significant more effective ( Amitriptyline and melatonin are effective and safe in pediatric migraine prophylaxis but amitriptyline can be considered as a more effective drug. [\hyperlink{Alitretinoin}{PMID: 29379562}, Razieh Fallah et al., 2018]

\hypertarget{pmid_31488494}{B}iannual mass azithromycin distribution to children aged 1-59 months has been shown to reduce all-cause mortality. Children under 28 days of age were not treated in studies evaluating mass azithromycin distribution for child mortality due to concerns related to infantile hypertrophic pyloric stenosis (IHPS). Here, we report the design of a randomised controlled trial to evaluate the efficacy and safety of administration of a single dose of oral azithromycin during the neonatal period. The  This study was approved by the Institutional Review Boards at the University of California, San Francisco in San Francisco, USA (Protocol \#18-25027) and the Comité National d'Ethique pour la Recherche in Ouagadougou, Burkina Faso (Protocol \#2018-10-123). The findings of this trial will be presented at local, regional and international meetings and published in open access peer-reviewed journals. NCT03682653; Pre-results. [\hyperlink{Alitretinoin}{PMID: 31488494}, Ali Sie et al., 2019]

\hypertarget{pmid_19948038}{A}rtemisinin combination therapy has become the standard of care for uncomplicated malaria in most of Africa. However, there is limited data on the safety and tolerability of these drugs, especially in young children and patients co-infected with HIV. A longitudinal, randomized controlled trial was conducted in a cohort of HIV-infected and uninfected children aged 4-22 months in Tororo, Uganda. Participants were randomized to treatment with artemether-lumefantrine (AL) or dihydroartemisinin-piperaquine (DP) upon diagnosis of their first episode of uncomplicated malaria and received the same regimen for all subsequent episodes. Participants were actively monitored for adverse events for 28 days and then passively for up to 63 days after treatment. This study was registered in ClinicalTrials.gov (registration \# NCT00527800). A total of 122 children were randomized to AL and 124 to DP, resulting in 412 and 425 treatments, respectively. Most adverse events were rare, with only cough, diarrhoea, vomiting, and anaemia occurring in more than 1\% of treatments. There were no differences in the risk of these events between treatment groups. Younger age was associated with an increased risk of diarrhoea in both the AL and DP treatment arms. Retreatment for malaria within 17-28 days was associated with an increased risk of vomiting in the DP treatment arm (HR = 6.47, 95\% CI 2.31-18.1, p < 0.001). There was no increase in the risk of diarrhoea or vomiting for children who were HIV-infected or on concomitant therapy with antiretrovirals or trimethoprim-sulphamethoxazole prophylaxis. Both AL and DP were safe and well tolerated for the treatment of uncomplicated malaria in young HIV-infected and uninfected children. ClinicalTrials.gov: NCT00527800; http://clinicaltrials.gov/ct2/show/NCT00527800. [\hyperlink{Alitretinoin}{PMID: 19948038}, Shereen Katrak et al., 2009]

\hypertarget{pmid_36302965}{E}arly supports to enhance social development in children with autism are widely promoted. While oxytocin has a crucial role in mammalian social development, its potential role as a medication to enhance social development in humans remains unclear. We investigated the efficacy, tolerability, and safety of intranasal oxytocin in young children with autism using a double-blind, randomized, placebo-controlled, clinical trial, following a placebo lead-in phase. A total of 87 children (aged between 3 and 12 years) with autism received 16 International Units (IU) of oxytocin (n = 45) or placebo (n = 42) nasal spray, morning and night (32 IU per day) for twelve weeks, following a 3-week placebo lead-in phase. Overall, there was no effect of oxytocin treatment over time on the caregiver-rated Social Responsiveness Scale (SRS-2) (p = 0.686). However, a significant interaction with age (p = 0.028) showed that for younger children, aged 3-5 years, there was some indication of a treatment effect. Younger children who received oxytocin showed improvement on caregiver-rated social responsiveness ( SRS-2). There was no other evidence of benefit in the sample as a whole, or in the younger age group, on the clinician-rated Clinical Global Improvement Scale (CGI-S), or any secondary measure. Importantly, placebo effects in the lead-in phase were evident and there was support for washout of the placebo response in the randomised phase. Oxytocin was well tolerated, with more adverse side effects reported in the placebo group. This study suggests the need for further clinical trials to test the benefits of oxytocin treatment in younger populations with autism.Trial registration www.anzctr.org.au (ACTRN12617000441314). [\hyperlink{Alitretinoin}{PMID: 36302965}, Adam J Guastella et al., 2023]

\hypertarget{pmid_21690106}{T}wo recent trials of angiotensin II receptor blockers (ARBs) were performed in children 0-5 years of age. Data from the published reports of these trials together with additional information from the sponsoring drug companies were obtained. Three deaths occurred in the 183 (1.6\%) hypertensive children participating in the two trials. At least two of these deaths occurred in children known to be susceptible to drugs acting on the renin-angiotensin system, that is, children with ongoing nephrotic syndrome and acute gastroenteritis. Clinicians who prescribe ARBs in preschool children need to be aware of the risk of drug toxicity especially in children susceptible to intravascular dehydration. Clinicians should consider discontinuing the drugs in the presence of acute diarrhoea. [\hyperlink{Alitretinoin}{PMID: 21690106}, Kjell Tullus et al., 2011]

\hypertarget{pmid_18608715}{I}nvestigation into the clinical efficacy, side effects and safety of oral acitretin on severe inherited disorders of keratinization in children. Acitretin was given as a treatment dose of 0.77-1.07 mg/kg x per day (mean 0.86+/-0.11) and maintenance dose of 0-0.94 mg/kg x per day (mean 0.33+/-0.26) to 28 children with severe inherited disorders of keratinization. Body height and weight were chosen as the monitoring indexes to evaluate the growth and development and other common side effects as the safety evaluation of the children for a follow-up of 2-36 months. After 2-4 months of treatment, the clinical cure rate was 82.1\% and the effective rate was 17.9\%. Most cases, such as bullous ichthyosiform erythroderma, lamellar ichthyosis, pityriasis rubra pilaris, and inflammatory linear verrucous epidermal nevus showed remarkable therapeutic response; non-bullous ichthyosiform erythroderma was also effective. Two cases with Darier's disease were previously shown to be resistant to acitretin therapy, but improved after 6 months of treatment. No previous investigation had been made on a negative effect on the growth and development of such children. Acitretin showed a satisfactory therapeutic effect on severe inherited disorders of keratinization in children. [\hyperlink{Alitretinoin}{PMID: 18608715}, Xi-Bao Zhang et al., 2008]

\section*{Travoprost}
\subsection*{Result}
\subsubsection*{Answer}

2 months to <18 years: Yes
<2 months: Unknown

\subsubsection*{{Explanation}}
\hypertarget{Travoprost}
A review of the available abstracts reveals the following evidence regarding the safety of Travoprost in children:

1. \hyperlink{pmid_19233013}{PMID: 19233013}, Tammy L Yanovitch et al., 2009: This retrospective study specifically examined the use of Travoprost in children. The study found that Travoprost had minimal adverse events in children and reduced intraocular pressure (IOP) in select cases of pediatric glaucoma. However, the abstract does not specify the exact age range of the children included.

2. \hyperlink{pmid_28887006}{PMID: 28887006}, El Roy Dixon et al., 2017: This is a double-masked, parallel-group, randomized controlled trial evaluating the efficacy and safety of Travoprost in pediatric patients aged 2 months to <18 years with glaucoma or ocular hypertension. The study included 157 patients (mean age 9.6 years), with 77 receiving Travoprost. The most common treatment-related adverse events were ocular hyperemia and eyelash growth. No serious adverse events or treatment-related systemic adverse events were reported. The study concluded that Travoprost was well-tolerated in this pediatric population.

Other abstracts mentioning Travoprost (e.g., \hyperlink{pmid_20819599}{PMID: 20819599}, 21060666, 20622697, 25759110, 22167541, 32311201) focus on adult populations or do not specify pediatric data.

Summary by age range:
- 2 months to <18 years: There is direct evidence from a randomized controlled trial affirming the safety of Travoprost in this age group, with no serious adverse events reported and only mild, expected side effects (ocular hyperemia, eyelash growth) [\hyperlink{pmid_28887006}{PMID: 28887006}, El Roy Dixon et al., 2017].
- <2 months: The available targeted pediatric study did not include infants younger than 2 months, so safety in this group is unknown.

\subsection*{Abstracts}
\hypertarget{pmid_19233013}{B}ecause of the limited ability to perform controlled, randomized studies in children, the safety and effectiveness of topical medications in pediatric glaucoma is sometimes difficult to determine. Although travoprost has been commercially available since 2001, there are no published reports on its use in children. This retrospective study found travoprost to have minimal adverse events in children and to reduce IOP in select cases of pediatric glaucoma. [\hyperlink{Travoprost}{PMID: 19233013}, Tammy L Yanovitch et al., 2009]

\hypertarget{pmid_28887006}{T}o evaluate efficacy and safety of travoprost in pediatric patients with ocular hypertension or glaucoma and demonstrate its noninferiority to timolol. Patients aged 2 months to <18 years with glaucoma or ocular hypertension were randomized to receive travoprost (0.004\%) or timolol eye drops (0.25\% for patients aged 2 months to <3 years and 0.5\% for patients ≥3 years old) for 3 months in this double-masked, parallel-group study. Intraocular pressure (IOP) was measured and patients were evaluated at 2 weeks, 6 weeks, and 3 months after treatment. Change in IOP from baseline to 3 months was the primary endpoint, and the test of noninferiority was based on a margin of +3.0 mm Hg using the 95\% 2-sided confidence interval of the mean change. Of 157 patients included (mean age, 9.6 years), 77 received travoprost and 75 timolol. All patients experienced a significant reduction in IOP in the study eye at 3 months: the mean IOP change from baseline was -5.4 mm Hg for travoprost; -5.3 mm Hg, for timolol. The mean difference between travoprost and timolol at month 3 was -0.1 mm Hg (95\% CI, -1.5 to 1.4 mm Hg). The most common treatment-related adverse events for the travoprost group were ocular hyperemia and eyelash growth. No serious adverse events were reported. This study found travoprost to be noninferior to timolol in lowering IOP in patients with pediatric glaucoma or ocular hypertension. Travoprost was well-tolerated, and no treatment-related systemic adverse events were reported. [\hyperlink{Travoprost}{PMID: 28887006}, El Roy Dixon et al., 2017]

\hypertarget{pmid_20819599}{T}ravoprost has been widely used for the treatment of patients with open-angle glaucoma (OAG) or ocular hypertension (OH). The aim of this study was to evaluate the intraocular pressure (IOP) lowering efficacy of travoprost 0.004\% monotherapy in patients previously treated with other topical hypotensive medications, and in previously untreated patients. This open-label, 12-week study in 1651 adult patients with ocular hypertension or open-angle glaucoma who were untreated or required a change in therapy (due to either inadequate efficacy or safety issues) as judged by the investigator was conducted at 6 sites in China. Previously treated patients were instructed to discontinue their prior medications at the first visit. All the patients were dosed with travoprost 0.004\% once-daily at 8 p.m. in both eyes for 12 weeks. Efficacy and safety evaluations were conducted at week 4 and 12. IOP measurements were performed at the same time of day at the follow-up visits. For patients transitioned to travoprost, mean IOP reductions from baseline in untreated and treated patients with different prior medications at week 12 were: latanoprost, (4.3 +/- 4.6) mmHg; beta-blocker, (6.3 +/- 4.0) mmHg; alpha-agonist, (7.5 +/- 4.3) mmHg; topical carbonic anhydrase inhibitors, (8.0 +/- 4.9) mmHg. All mean IOP changes from baseline were statistically significant (P < 0.001). No treatment-related serious adverse events were reported in this study. In patients treated with other hypotensive medications or untreated, the IOP reduction with travoprost was significant. The results of this study demonstrated the potential benefit of using travoprost as a replacement therapy in order to ensure adequate IOP control. Travoprost administered once daily was safe and well tolerated in patients with glaucoma or ocular hypertension. [\hyperlink{Travoprost}{PMID: 20819599}, Jian Ge et al., 2010]

\hypertarget{pmid_21060666}{T}ravoprost is a prostaglandin analog used in the management of glaucoma and ocular hypertension for reducing intraocular pressure (IOP). The IOP-lowering efficacy of travoprost has been shown to be similar to that of other prostaglandins, including latanoprost and bimatoprost. When compared with fixed combinations of timolol and either latanoprost or dorzolamide, travoprost alone can reduce mean IOP in a similar or superior manner. Concomitant therapy of travoprost and timolol can reach even greater IOP reductions than fixed combinations at some time points, but with no difference in the early morning, when IOP is usually higher. In addition, the long duration of action of travoprost can also provide better control of IOP fluctuation, probably due to its stronger prostaglandin F receptor mechanism. The side effects of travoprost do not represent a risk to the vision or health of the patient. The proven efficacy and safety combined with convenient once-daily dosing for travoprost increases patient compliance with treatment for glaucoma. [\hyperlink{Travoprost}{PMID: 21060666}, Emilio Rintaro Suzuki et al., 2010]

\hypertarget{pmid_20491260}{T}he efficacy of preparation "Tachocomb" application in 224 children, ageing from 1 day to 18 years old, was studied up. The spectrum of the preparation studying had included the investigation of its properties, which secure an additional isolation for the intestinal and vascular walls defects, and hemostatic effect in the hemorrhage area as well. The data obtained trust the safety and efficacy of "Tachocomb" platelets application in pediatric surgery and permit to recommend them for improvement of the surgical treatment results. [\hyperlink{Travoprost}{PMID: 20491260}, D Iu Krivchenia et al., 2010]

\hypertarget{pmid_30789136}{I}n this retrospective study, we assessed the safety of window period prophylaxis and proportion of tuberculin skin test (TST) conversions in children <5 years of age who were exposed to an adult with tuberculosis disease during 2007-2017. Children included in this study had unremarkable examination and chest radiograph findings and negative test results for TB infection. In total, 752 children (41\% cohabitating with the index patient) received prophylaxis during the window period, usually directly observed therapy with isoniazid. Hepatotoxicity and tuberculosis disease did not develop in any child. TST conversion occurred in 37 (4.9\%) children and was associated with the index patient being the child's parent (odds ratio 3.2, 95\% CI 1.2-8.2). TST conversion was not associated with sputum smear results, culture positivity, or cohabitation. Thresholds for initiation of window prophylaxis in exposed young children should be low given the safety of medication and difficulties with risk stratification. [\hyperlink{Travoprost}{PMID: 30789136}, Andrea T Cruz et al., 2019]

\hypertarget{pmid_15510514}{L}atanoprost is a prostaglandin F2alpha analog that reduces intraocular pressure by 20-40\% in adults with open-angle glaucoma. The efficacy and safety of this drug in children has not been widely reported. In our study we evaluated the effect of latanoprost in 14 children aged 12-18 years (mean 15 years): 10 patients with glaucoma juvenile (I group); 2 patients with secondary glaucoma because of uveitis recidivans and 2 patients with aniridia and albinismus (II group). In the I group the average IOP decrement was 9 mmHg or 36.5\% (range 29-44\%). In the II group the average IOP decrement was 6.5 mmHg or 23.5\% (range 11-33\%). In one child with aniridia after one year of treatment IOL rose again to 26 mmHg and antiglaucomatous surgery was necessary. Ocular side effects in children of latanoprost are mild. [\hyperlink{Travoprost}{PMID: 15510514}, Beata Urban et al., 2004]

\hypertarget{pmid_32943036}{D}rooling is common in children with neurological disorders, but its management is very challenging, Scopolamine transdermal patch (STP) appears to be useful in controlling drooling, although it is not approved for this indication and there are limited clinical studies about its effectiveness. This study aimed (1) to assess the impact of STP use on the severity of drooling and on the frequency of emergency department (ED) and hospital readmission (RA) visits related to drooling, and (2) to determine the level of family satisfaction with STP when used in children with neurological disorders. This is a retrospective cohort study of all pediatric patients aged 3-14 years, with non-progressive neurodevelopmental disability, who used STP for more than one year during the period between April 2015 and July 2018 (n = 44). Data on demographics, clinical status, comorbidities, STP dose and duration, other medications, ED and RA visits were collected. Follow-up phone-call interviews with parents/caregivers were performed using a parent-reported frequency and severity rating scale of sialorrhea. Absolute and relative risk reductions were calculated to assess the impact of STP on ED and RA visits. Significance was considered at p-value of ≤ 0.05. STP use showed significant reduction in severity of drooling (p < 0.001), wiping of the child's mouth (p < 0.001), bibs or clothing changes (p < 0.001), choking and aspiration of saliva (p = 0.001). The Relative Risk Reduction of the drooling-related ED and RA visits were 86\% and 67\% respectively. Nearly two-thirds (60\%) of caregivers were satisfied with using STP. This is the first study of its kind done in Saudi Arabia demonstrating favorable impact of STP use by children on the consequences associated with drooling and with the frequency of ER and RA visits due to drooling. Development of a medication use protocol is recommended to standardize STP treatment in order to optimize its effectiveness. This study serves as baseline information for future prospective interventional studies. [\hyperlink{Travoprost}{PMID: 32943036}, Majed Al Jeraisy et al., 2020]

\hypertarget{pmid_35049017}{D}espite its low incidence, pulmonary hypertension in children places a substantial burden on families and society because survival can be shorter than 10 months and treatment options are limited and ineffective. Drugs to treat pulmonary hypertension include endothelin antagonists, phosphodiesterase type 5 inhibitors and prostacyclin, which is the most widely used to treat pediatric pulmonary hypertension. The main aim of this study was to provide a comprehensive overview of the advantages and disadvantages of prostacyclin and its analogs for treating pulmonary hypertension in children. To retrieve a thorough collection of studies, we performed a search in PubMed using the following combination of keywords: (Prostacyclins) or (Epoprostenol) or (Iloprost) or (Treprostinil) or (Beraprost), (children) and (pulmonary arterial hypertension). The time limits used for the search were December 1983 to May 2021. The search retrieved a total of 238 articles. Titles and abstracts of articles were screened for relevance, and all relevant articles published in English were included. Epoprostenol can be effective against severe pulmonary hypertension. Iloprost can treat severe persistent pulmonary hypertension in newborns and inhaled iloprost can be used in pulmonary vasoreactivity testing. Treprostinil is a long-acting prostacyclin analog, and it shows the highest antiproliferative activity among prostacyclins. Beraprost may be effective in premature infants, but available evidence comes from only one patient, so more clinical testing is needed. [\hyperlink{Travoprost}{PMID: 35049017}, Y Wu et al., 2022]

\hypertarget{pmid_24576066}{T}he benzalkonium chloride (BAK) content of tafluprost ophthalmic solution (Tapros(®): tafluprost) has been reduced to balance corneal safety and preservative effectiveness (old formulation: 0.01\%; new formulation: 0.001\%). However, no reports have been published on its clinical effect. Therefore, we conducted a clinical research study to compare the safety of BAK-reduced tafluprost on the ocular surface with other prostaglandin ophthalmic solutions. This clinical study included 28 glaucoma patients (28 eyes) with a treatment history of latanoprost ophthalmic solution (Xalatan(®)) or travoprost ophthalmic solution (Travatan Z(®)), who presented with corneal epithelial disorders. The subjects were switched to BAK-reduced tafluprost, and its effect on the ocular surface was examined after 1 and 2 months of treatment [using fluorescein staining score, hyperemia, tear film breakup time, and intraocular pressure (IOP) lowering]. In all analyzed subjects (N=27), the fluorescein staining score was significantly improved after switching to BAK-reduced tafluprost (P<0.0001). Conversely, the IOP-lowering effect was not notably changed. The subjects switched from latanoprost (n=10) showed significant improvement in fluorescein staining score (P<0.05) as well as in IOP lowering (P<0.01). The subjects switched from travoprost (n=17) also showed significant improvement in fluorescein staining score (P<0.001), but without a significant change in IOP lowering. Tafluprost with reduced BAK has potential as a superior antiglaucoma drug, not only for its IOP-lowering effect, but also for its good corneal safety profile. [\hyperlink{Travoprost}{PMID: 24576066}, Takeshi Kumagami et al., 2014]

\hypertarget{pmid_29929405}{T}razodone is an atypical antidepressant with no established safety in children. Previous case reports showed no complications at doses 50-500 mg in children. Our study objective is to characterize the clinical effects, dose-related toxicity, and establish triage dose for acute trazodone ingestions in children ≤6 years of age. Cases with acute trazodone ingestions in children ≤6 years of age between 2000 and 2015 were retrospectively reviewed. Data were analyzed for dose (mg/kg), clinical effects, management site, treatment, and outcome. Cases with coingestions, unknown outcome, or unknown dose were excluded. A total of 84 patients (mean age 26.7 months, 35 females, 49 males) were included. Of those, 52 (61.9\%) had no clinical effects; 29 (34.5\%) had minor effects (vomiting, dizziness, headache); and three (3.6\%) had moderate effects (ataxia, slurred speech, priapism). No major effects or deaths were observed. Moderate effects were manifested at doses ≥6.9 mg/kg. Priapism occurred in a 2-year-old child at a dose of 6.9 mg/kg. Sixteen (19\%) patients were managed at home and 68 (81\%) patients were referred to a HCF. Among those referred to a HCF, three (4.4\%) patients had moderate effects with ingested dose ≥6.9 mg/kg. However, 27 (39.7\%) patients of those referred to a HCF had an ingested dose <6 mg/kg and none of them manifested symptoms beyond minor effects. All referred patients had uneventful recovery and no sequela. Children should be referred for further evaluation in acute unintentional trazodone ingestions with doses ≥6 mg/kg. [\hyperlink{Travoprost}{PMID: 29929405}, Tharwat El Zahran et al., 2019]

\hypertarget{pmid_20622697}{T}ravoprost is a prostaglandin analog used in the treatment of open-angle glaucoma. This drug is safe and efficacious and has a low incidence of systemic and local side effects. Common local side effects are conjunctival hyperemia, iris pigmentation, and hypertrichosis of the eyelashes. The authors present a case of a patient who developed marked hypertrichosis of the cheek vellus 3 months after starting treatment with travoprost. [\hyperlink{Travoprost}{PMID: 20622697}, Santiago Ortiz-Perez et al., ]

\hypertarget{pmid_17803434}{T}he aim of this study was to determine the safety, tolerability, and pharmaco-dynamics of a novel prostanoid fluoroprostaglandin (FP)-receptor agonist, tafluprost (AFP-168), in healthy males. This was a phase I study in healthy males 18-45 years of age (N = 49). Participants were randomized to receive 1 of 4 eye drops: tafluprost 0.0025\% or 0.005\%, latanoprost 0.005\%, or a placebo, administered once-daily for 7 days, with 1 drop per eye. Safety and tolerability assessments and intraocular pressure (IOP) measurements were performed at defined intervals. Tafluprost was generally well tolerated. No serious adverse events were reported and no participants withdrew owing to an adverse event. IOP decreased over time, compared with baseline, in all 4 treatment groups. Treatment with tafluprost 0.005\% resulted in a significantly greater reduction in IOP, compared with either latanoprost 0.005\% or a placebo, at various time points during treatment. Ocular hyperemia and photophobia were more common with tafluprost 0.0025\% or 0.005\%, compared with latanoprost 0.005\%. Tafluprost eye drops 0.0025\% and 0.005\% were generally well tolerated and safe. Tafluprost 0.005\% reduced IOP more than placebo or latanoprost 0.005\%. Therefore, tafluprost looks promising for further investigation. [\hyperlink{Travoprost}{PMID: 17803434}, Andrew Sutton et al., 2007]

\hypertarget{pmid_19952926}{T}he testing and treatment of children at risk for Mycobacterium tuberculosis infection represents an important public health priority in the United States. Until recently, diagnosis has relied upon the tuberculin skin test (TST). New interferon-gamma release assays (IGRAs) offer improvements over TST, but these tests have not been studied in children until recently. Evidence regarding IGRA performance in children is accumulating rapidly. Overall, the findings demonstrate performance of IGRAs equivalent or superior to that of the TST. However, IGRAs have biological limitations similar to TST and some technical problems of their own, and critical gaps in our knowledge remain. Current evidence supports usage of IGRAs in children aged 5 years or older. IGRAs are preferred over TST when specificity is paramount or wherein patients might fail to return for TST reading. Evidence for use in children aged less than 5 years is insufficient at this time: the sensitivity is poorly defined, and TST is preferred for testing these children. Future IGRA research should focus on children aged less than 5 years for informing expanded usage in this vulnerable population. [\hyperlink{Travoprost}{PMID: 19952926}, Deborah A Lewinsohn et al., 2010]

\hypertarget{pmid_25759110}{T}o evaluate the safety and tolerability of Polyquad-preserved Travoprost (PQ-Travoprost) in patients previously treated with benzalkonium chloride (BAK)-preserved Latanoprost. Cohort 6-month study on open-angle glaucoma or ocular hypertension patients. Complete ophthalmic examination, intraocular pressure (IOP) measurement and ocular surface status (tear film break-up time [TF-BUT], corneal staining and ocular surface disease index [OSDI]) were evaluated at baseline and 6 months later. A total of 44 patients were enrolled. Median (interquartile range [IQR]) baseline IOP was 18 (15.5 - 21) and 16 (14 - 17) mmHg (p < 0.0001) after 6 months. At baseline, 18 (40.9\%) patients presented an IOP of < 18 mmHg, 11 (25\%) < 16 mmHg, 2 (4.3\%) < 14 mmHg and 1 (2.3\%) < 12 mmHg; 6 months later the proportions were 36 (81.8\%) (p < 0.0001), 21 (47.7\%) (p = 0.00075), 8 (18.2\%) (p = 0.0143) and 6 (13.6\%) (p = 0.0253). Concerning safety, TF-BUT improved from 8 [IQR 6 - 10] to 10 [IQR 8 - 12] s (p < 0.0001). No eye developed corneal staining; punctate keratitis was absent in 13 (29.5\%) patients at baseline and in 31 (70.4\%) after 6 months (p < 0.001). OSDI changed from 16 (10 - 30) to 9 (2 - 20). No patient treated with PQ-Travoprost developed ocular surface disease after 6 months of monotherapy, whereas many patients reached a good IOP control with lower IOP values. Ocular surface status statistically improved when examined by TF-BUT and corneal staining. [\hyperlink{Travoprost}{PMID: 25759110}, Gemma Caterina Maria Rossi et al., 2015]

\hypertarget{pmid_17317374}{I}ntravenous epoprostenol was the first agent approved by the United States Food and Drug Administration for the management of pulmonary arterial hypertension (PAH). However, epoprostenol therapy carries the risks of a short half-life (<6 minutes) and side effects, including jaw pain, flushing, and headache. Recently, intravenous treprostinil has been studied, primarily in adults with PAH, and found to provide effective therapy. The effects of continuous intravenous treprostinil were retrospectively evaluated in 13 children with stable PAH who had been treated with epoprostenol for >1 year. Children were transitioned in the hospital over 24 hours using a rapid or slow strategy. The children were a mean age of 11 years (range 3 to 17) and were transitioned to treprostinil from August 2004 to August 2005. The baseline 6-minute walking distance was on average 516 +/- 115 m (n = 9) and did not change after transition. Patients were treated with treprostinil for 1.1 +/- 0.5 years. There were 2 deaths, and 2 patients transitioned to other therapy. Seven patients experienced > or =1 central-line infection. Despite a higher dose of treprostinil, the side effects were subjectively diminished. In conclusion, treprostinil provides an alternative therapy in children with PAH, with fewer side effects. However, evaluation regarding rates of infection requires further exploration. [\hyperlink{Travoprost}{PMID: 17317374}, D Dunbar Ivy et al., 2007]

\hypertarget{pmid_31892518}{T}he tuberculin skin test (TST) has been preferred for screening young children for latent tuberculosis infection (LTBI) because of concerns that interferon-γ release assays (IGRAs) may be less sensitive in this high-risk population. In this study, we compared the predictive value of IGRAs to the TST for progression to tuberculosis disease in children, including those <5 years old. Children <15 years old at risk for LTBI or progression to disease were tested with TST, QuantiFERON-TB Gold In-Tube test (QFT-GIT), and T-SPOT. Of 3593 children enrolled September 2012 to April 2016, 92\% were born outside the United States; 25\% were <5 years old. Four children developed tuberculosis over a median 4.3 years of follow-up. Sensitivities for progression to disease for TST and IGRAs were low (50\%-75\%), with wide confidence intervals (CIs). Specificities for TST, QFT-GIT, and T-SPOT were 73.4\% (95\% CI: 71.9-74.8), 90.1\% (95\% CI: 89.1-91.1), and 92.9\% (95\% CI: 92.0-93.7), respectively. Positive and negative predictive values for TST, QFT-GIT, and T-SPOT were 0.2 (95\% CI: 0.1-0.8), 0.9 (95\% CI: 0.3-2.5), and 0.8 (95\% CI: 0.2-2.9) and 99.9 (95\% CI: 99.7-100), 100 (95\% CI: 99.8-100), and 99.9 (95\% CI: 99.8-100), respectively. Of 533 children with TST-positive, IGRA-negative results not treated for LTBI, including 54 children <2 years old, none developed disease. Although both types of tests poorly predict disease progression, IGRAs are no less predictive than the TST and offer high specificity and negative predictive values. Results from this study support the use of IGRAs for children, especially those who are not born in the United States. [\hyperlink{Travoprost}{PMID: 31892518}, Amina Ahmed et al., 2020]

\hypertarget{pmid_29426959}{I}nhaled prostacyclin analogue iloprost is currently utilized in adult patients with pulmonary arterial hypertension (PAH), but little information is available on its use in the pediatric population. This study evaluated the safety and acute haemodynamic effects of inhaled iloprost in children with PAH associated with congenital heart disease (CHD). Children with PAH-CHD who underwent cardiac catheterization and iloprost administration in our catheter laboratory between June 2007 and October 2015 were included. Iloprost was administered by atomization inhalation and changes in hemodynamic parameters were recorded. In total, 100 children with PAH-CHD were enrolled. Median age was 13 [1.5-18.0] years and 34\% were male. A ventricular septal defect was present in 84\%, a patent duct in 12\%, a complete atrioventricular septal defect in 2\%, and an isolated atrial septal defect in 2\%. Pulmonary vascular resistance indexed (PVRI) was above 8 WU m [\hyperlink{Travoprost}{PMID: 29426959}, Qiangqiang Li et al., 2018] Long-term data are needed regarding effective and safe glaucoma treatment modalities. This study evaluated 4-year outcomes of second-generation trabecular micro-bypass stent implantation (iStent inject) combined with topical travoprost in open-angle glaucoma (OAG). Prospective, non-randomized, multi-surgeon study at a tertiary care ophthalmology centre. OAG subjects with preoperative intraocular pressure (IOP) 18 to 30 mmHg on two medications and 22 to 38 mmHg post-washout. Subjects (n = 53) underwent standalone iStent inject implantation and started travoprost on postoperative Day 1. Measures included IOP, medications, comprehensive ophthalmic examinations and testing, and adverse events (AEs). Annual medication washouts were performed. Mean medicated and unmedicated IOP; and proportions of eyes with IOP ≤18mmHg, ≤15 mmHg, or ≥20\% reduced while on travoprost vs screening IOP on two medications. At 48 months postoperative, 85\% of eyes reduced IOP ≥20\% on travoprost vs screening IOP on 2 medications; 92\% of eyes had IOP ≤18 mmHg on travoprost; and 83\% had IOP ≤15 mmHg on travoprost. At Month 49 (post-washout), 90\% of eyes reduced IOP ≥20\% vs preoperative washout IOP. Throughout follow-up, mean IOP on travoprost was 11.9 to 13.0 mmHg (34\%-40\% reduced vs 19.7 mmHg on 2 medications preoperatively; P < .0001 throughout), and post-washout IOP was 16.5 to 17.7 mmHg (28\%-34\% reduced vs 24.9 mmHg preoperatively; P < .0001 throughout). Favourable safety included minimal AEs; stable visual acuity, cup-to-disc ratio and visual fields; and no secondary surgeries. Combining iStent inject implantation with topical prostaglandin produced significant and safe 4-year IOP and medication reductions in OAG. [\hyperlink{Travoprost}{PMID: 29426959}, John Berdahl et al., 2020]

\hypertarget{pmid_22167541}{T}o demonstrate that the intraocular pressure (IOP)-lowering effect of travoprost 0.004\% preserved with polyquaternium-1 (travoprost benzalkonium chloride [BAK]-free) is non-inferior to that of travoprost 0.004\% preserved with benzalkonium chloride (travoprost BAK) in patients with ocular hypertension or open-angle glaucoma. A total of 371 patients randomly received travoprost BAK-free (n=185) or travoprost BAK (n=186) dosed once daily in the evening for 3 months. Patients were evaluated at 9 am, 11 AM, and 4 PM at baseline, weeks 2 and 6, and month 3. Intraocular pressure was also evaluated 36 and 60 hours after the month 3 visit. Travoprost BAK-free is non-inferior to travoprost BAK. The 95\% upper confidence limits for the difference in mean IOP at month 3 (primary efficacy) were 0.5 mmHg, 0.6 mmHg, and 0.5 mmHg, at 9 AM, 11 AM, and 4 PM, respectively. Mean IOP reductions from baseline ranged from 7.6 to 8.7 mmHg in the travoprost BAK-free group and from 7.7 to 9.2 mmHg in the travoprost BAK group. At 36 and 60 hours after the last dose, mean IOP remained 6.8 mmHg and 5.7 mmHg below baseline in the travoprost BAK-free group, vs 7.3 mmHg and 6.0 mmHg in the travoprost BAK group, respectively. The safety profile of travoprost BAK-free was similar to that of travoprost BAK. Travoprost BAK-free safely and effectively lowers IOP in eyes with open-angle glaucoma or ocular hypertension. This BAK-free formulation has comparable safety, efficacy, and duration of IOP-lowering effect to travoprost preserved with BAK. Travoprost BAK-free is an effective option for IOP reduction while avoiding BAK exposure. [\hyperlink{Travoprost}{PMID: 22167541}, Stefano Gandolfi et al., ]

\hypertarget{pmid_22050687}{T}o evaluate the changes in intraocular pressure (IOP) and pupil size in 12 Beagles with inherited glaucoma after instillations of 0.033, 0.0033, 0.001, 0.00033, and 0.0001\% travoprost (Travatan®-Alcon Laboratories, Inc., Ft Worth, TX, USA) in multiple single-dose studies. Intraocular pressure and pupil diameter (PD) measurements were obtained at 9 am, 12 pm, 3 pm, and 9 am the following day (24 h) in two groups of six glaucoma dogs. After 7 days, the vehicle or concentration was repeated in the contralateral eye of the same animals. Concentrations of 0.00033, 0.001, and 0.0033\% travoprost significantly lowered IOP and PD, but the 0.0001\% concentration provided limited IOP changes, although PD changes were still significant. This suggests travoprost is effective in the dog to lower IOP and reduce pupil size at concentrations starting between 0.0001 and 0.00033\%. The dose response for travoprost in the glaucomatous Beagle indicates this model is highly sensitive to this group of drugs, even at concentrations as low as 0.00033\% (1/12 the commercially available concentration). [\hyperlink{Travoprost}{PMID: 22050687}, Edward O Mackay et al., 2012]

\hypertarget{pmid_18496412}{T}o review the physiology and the published literature on the role of vasopressin in shock in children. We searched MEDLINE (1966-2007), EMBASE (1980-2007), and the Cochrane Library, using the terms vasopressin, terlipressin, and shock and synonyms or related terms for relevant studies in pediatrics. We searched the online ISRCTN-Current Controlled Trials registry for ongoing trials. We reviewed the reference lists of all identified studies and reviews as well as personal files to identify other published studies. Beneficial effects have been reported in vasodilatory shock and asystolic cardiac arrest in adults. Solid evidence for vasopressin use in children is scant. Observational studies have reported an improvement in blood pressure and rapid weaning off catecholamines during administration of low-dose vasopressin. Dosing in children is extrapolated from adult studies. Vasopressin offers promise in shock and cardiac arrest in children. However, in view of the limited experience with vasopressin, it should be used with caution. Results of a double-blind, randomized controlled trial in children with vasodilatory shock will be available soon. [\hyperlink{Travoprost}{PMID: 18496412}, Karen Choong et al., 2008]

\hypertarget{pmid_23000726}{F}ew studies have examined the efficacy or safety of a transdermal β(2) agonist as add-on medication to long-term leukotriene receptor antagonist (LTRA) therapy in pediatric asthma patients. In this randomized, open-label, multicenter clinical trial, children aged 4-12 years on long-term LTRA therapy were treated with tulobuterol patches (1-2mg daily) or oral sustained-release theophylline (usual dose, 4-5mg/kg daily) for 4 weeks. LTRAs were continued throughout the trial. Outcomes included volume of peak expiratory flow (\% PEF), fractional exhaled nitric oxide (FeNO), clinical symptoms and adverse events. Thirty-three and 31 patients were treated with tulobuterol patches and theophylline, respectively. \% PEF measured in the morning and before bedtime was significantly higher at all times in the treatment period compared with baseline in the tulobuterol patch group (p < 0.001), and was significantly higher in the tulobuterol patch group compared with the theophylline group. FeNO was similar and unchanged from baseline in both groups. There were no drug-related adverse events in either group. These results suggest that short-term use of a transdermal β(2) agonist is an effective therapy for pediatric asthma without inducing airway inflammation in children on long-term LTRA therapy. [\hyperlink{Travoprost}{PMID: 23000726}, Toshio Katsunuma et al., 2013]

\hypertarget{pmid_29650343}{C}ontinuous intravenous epoprostenol was the first treatment approved for pulmonary arterial hypertension (PAH) but administration through a central venous line carries risks of thrombosis and sepsis, particularly in children. We sought to evaluate the safety, efficacy and management of subcutaneous (SC) treprostinil in children with PAH. Fifty-six children (median age 65, range 1-200 months) were treated with SC treprostinil. Clinical status, echocardiography, NT-proBNP, and site pain and infection were evaluated. Right heart catheterization was performed in 54 patients before starting SC treprostinil infusion and was repeated at 6 months in 31 patients. Treatment was well tolerated in 79\% of patients. Site pain resistant to simple analgesics occurred in 12 patients (21\%), but could be managed in 9/12 children. At 6 months, 3 patients had died, 4 had received a Potts shunt and 1 underwent lung transplantation. Among the 48 treated patients, 40 (83\%) showed significant improvement in WHO functional class, 6 minute walk distance, NT-proBNP and pulmonary vascular resistance (p < 0.01 for all parameters). At last follow-up (median 37 months), ten patients had died, 2 underwent a lung transplantation and 8 underwent a Potts shunt. In 30 of the 36 remaining treated patients, improvement of clinical status was sustained. No children developed sepsis and 12 had minor site infections. Subcutaneous treprostinil infusion is an effective therapy without serious side effects in children with PAH. Site pain can be managed with simple analgesics in most children. [\hyperlink{Travoprost}{PMID: 29650343}, Marilyne Levy et al., 2018]

\hypertarget{pmid_9272316}{A} study was conducted to determine the safety and efficacy of topically applied ciprofloxacin ophthalmic solution in a pediatric population experiencing acute bacterial conjunctivitis. Topically applied ciprofloxacin (3 mg/mL) is known to be a safe and effective treatment for adults suffering from bacterial conjunctivitis; however, the safety and effectiveness of this broad-spectrum fluoroquinolone in pediatric patients is not well established. Ciprofloxacin was evaluated in a double-blind, randomized, controlled study of 257 patients ranging from 0 (i.e., less than 1 year) to 12 years of age from 33 geographically diverse medical centers. The children received either 0.3\% ciprofloxacin ophthalmic solution (Ciloxan, Alcon Laboratories, Fort Worth, TX) or tobramycin ophthalmic solution (Tobrex, Alcon Laboratories, Fort Worth, TX). Both test medications were administered topically every 2 hours on days 1 and 2 followed by every 4 hours on days 3 through 7. Eyes were cultured prior to enrollment and again on day 7. Treatment efficacy as determined by microbiological culture data and physicians' judgment of overall resolution was similar for the ciprofloxacin and tobramycin groups. Microbiological eradication was observed in 90.1\% of the ciprofloxacin group and 84.3\% of the tobramycin group (P = 0.29). Physicians judged 87.0\% of the ciprofloxacin patients and 89.9\% of the tobramycin patients clinically cured on day 7 (P > 0.5). There were no serious adverse medical events attributable to either treatment. This study showed that topically applied ciprofloxacin ophthalmic solution is safe and effective in a pediatric population experiencing acute bacterial conjunctivitis. [\hyperlink{Travoprost}{PMID: 9272316}, R D Gross et al., 1997]

\section*{Azathioprine Sodium}
\subsection*{Result}
\subsubsection*{Answer}

Ages 2–18: Yes  
Ages <2: Unknown  

\subsubsection*{{Explanation}}
\hypertarget{Azathioprine Sodium}
Based on the abstracts available, several targeted studies have evaluated the safety of azathioprine (including azathioprine sodium) in children for various conditions. Below is a summary of the relevant evidence, organized by age ranges and indications:

1. **Children with Inflammatory Bowel Disease (IBD), including Crohn's Disease and Autoimmune Hepatitis (AIH):**
   - A retrospective cohort study of 107 children (age not specified, but all pediatric) with IBD treated with azathioprine at 3 mg/kg found that only 2 children discontinued due to persistent adverse effects, and the drug was described as "safe and well-tolerated" as maintenance therapy [\hyperlink{pmid_12656694}{PMID: 12656694}, D Fuentes et al., 2003].
   - Another retrospective study of 123 Italian children (mean age at start of therapy 11.8 years) with IBD found that 39\% experienced side effects, but only 7\% had severe toxicity requiring discontinuation. The authors concluded azathioprine was efficacious in 70\% and induced severe toxicity in 7\% [\hyperlink{pmid_12030954}{PMID: 12030954}, A Barabino et al., 2002].
   - In pediatric autoimmune hepatitis, a retrospective analysis of 56 children (median age 11 years) found no specific safety concerns directly attributed to azathioprine, and the drug is described as the "mainstay" of maintenance therapy [\hyperlink{pmid_30234756}{PMID: 30234756}, Rishi Bolia et al., 2018].

2. **Children with Severe Atopic Dermatitis:**
   - A prospective cohort of 82 children (mean age 8.3 years) treated with azathioprine for atopic dermatitis found that 41\% had adverse effects on blood indices (22\% pronounced), but only 2 stopped therapy due to these effects. Clinical adverse effects occurred in 20\%, with 2 stopping therapy. The authors concluded oral azathioprine was associated with "few pronounced adverse effects" for the duration and dosage used [\hyperlink{pmid_25440430}{PMID: 25440430}, Nicholas R Fuggle et al., 2015].
   - A retrospective study of 7 children (mean age 10 years) with severe atopic dermatitis treated for up to 38 months found only mild, transient leukopenia in 2 patients, and concluded azathioprine is "safe" when adjusted to thiopurine methyltransferase activity [\hyperlink{pmid_20525484}{PMID: 20525484}, R M Martel et al., 2010].
   - Another retrospective review of 48 children (age not specified, but pediatric) with severe atopic eczema found no cases of neutropenia and described the short-term adverse effect profile as "entirely acceptable" [\hyperlink{pmid_12174104}{PMID: 12174104}, L-A Murphy et al., 2002].
   - A prospective study of 12 children with severe, recalcitrant atopic dermatitis found "few adverse effects" and clinical improvement in all but one patient [\hyperlink{pmid_22892285}{PMID: 22892285}, Maura Caufield et al., 2013].

3. **Other Pediatric Indications:**
   - A retrospective chart review of 95 children (age not specified, but pediatric) treated with azathioprine for dermatological conditions found that 3\% discontinued due to abnormal blood tests, but no significant adverse events were reported with less frequent monitoring [\hyperlink{pmid_27435804}{PMID: 27435804}, Joy Yee et al., 2018].
   - A case report of a 13-year-old boy who ingested a large overdose of azathioprine found only moderate, transient decreases in blood counts and no serious adverse events [\hyperlink{pmid_9629549}{PMID: 9629549}, C Krüger et al.].

4. **In Utero Exposure:**
   - One cohort study of children exposed to azathioprine in utero (mothers with SLE) found an association with increased need for special educational services, suggesting possible developmental concerns, but this is not the same as direct pediatric use and is confounded by maternal disease [\hyperlink{pmid_23139238}{PMID: 23139238}, Wendy Marder et al., 2013].
   - A review concluded that long-term safety data for children exposed in utero are lacking and that large-scale studies are needed [\hyperlink{pmid_32240856}{PMID: 32240856}, Cristina Belizna et al., 2020].

**Summary by Age Range:**
- Most studies included children from infancy up to 18 years, with mean or median ages typically between 8 and 12 years.
- No studies specifically addressed safety in neonates or infants under 1 year, but some included children as young as 2 years.
- Across studies in children (primarily ages 2–18), azathioprine was generally well-tolerated, with adverse effects occurring in a minority and rarely requiring discontinuation.

**Conclusion:**
There is evidence from targeted pediatric studies (primarily ages 2–18) that azathioprine sodium is generally safe for use in children with IBD, autoimmune hepatitis, and severe atopic dermatitis, with appropriate monitoring. Safety in neonates and infants under 2 years is not directly addressed in the available abstracts, so safety in this age group is unknown.

\subsection*{Abstracts}
\hypertarget{pmid_12656694}{A}zathioprine is widely used as maintenance therapy in children with moderate to severe inflammatory bowel disease (IBD). There is no data on safety at higher doses and its impact on growth and surgical morbidity in children. This retrospective cohort study included all children treated with azathioprine and diagnosed with IBD between 1996-2001. Outcome measures included indications for azathioprine use, adverse-effects and reasons for treatment discontinuation. Height and weight at diagnosis, treatment onset and current follow-up was recorded, and Z scores for height standardised for time. 107 children received azathioprine at 3 mg/kg. 61\% had Crohn's disease and 83\% started azathioprine within 2 years of diagnosis. Only 2/107 children had to stop azathioprine because of persistent adverse effects and 16/107 required surgery. There was a trend toward better growth in a group of children with Crohn's disease following treatment with high dose azathioprine therapy (P = 0.08). Azathioprine is a safe and well-tolerated maintenance therapy at 3 mg/kg for children with IBD. The prevalence of surgery and growth failure in a cohort of children with moderate to severe IBD appears less than previously reported. In children with Crohn's disease, growth velocity may be maximised by an emphasis on nutritional therapy and the use of high dose azathioprine. [\hyperlink{Azathioprine Sodium}{PMID: 12656694}, D Fuentes et al., 2003]

\hypertarget{pmid_25440430}{A}zathioprine is efficacious in the treatment of severe childhood atopic dermatitis; however, robust data on adverse effects in this population are lacking. We sought to assess adverse effects of azathioprine treatment in a pediatric atopic dermatitis cohort, and make recommendations for monitoring based on these data. Blood test results for all 82 children prescribed oral azathioprine for atopic dermatitis in our department between 2010 and 2012 were collated prospectively, and clinical notes were reviewed retrospectively. Mean age at commencing azathioprine was 8.3 years (SEM 0.4). Mean maximum doses were 2.4 mg/kg (SEM 0.1) and 1.5 mg/kg (SEM 0.1) for normal and reduced serum thiopurine-S-methyltransferase levels, respectively. Adverse effects on blood indices occurred in 34 of 82 patients (41\%), with pronounced effects in 18 of 82 (22\%) after a median time of 0.4 years. Two patients stopped therapy as a result of abnormal blood indices. Clinical adverse effects occurred in 16 of 82 (20\%), two resulting in cessation of therapy. Incidence of adverse effects was unaffected by age, sex, thiopurine-S-methyltransferase level, and drug dose on multivariate regression. Comparison with other studies is limited by varying definitions of adverse effects. Oral azathioprine was associated with few pronounced adverse effects for the duration of use and dosage in this cohort. Recommendations for monitoring are made. [\hyperlink{Azathioprine Sodium}{PMID: 25440430}, Nicholas R Fuggle et al., 2015]

\hypertarget{pmid_32240856}{A}zathioprine (AZA), an oral immunosuppressant, is safe during pregnancy. Some reports suggested different impairments in the offspring of mothers with autoimmune diseases (AI) exposed in utero to AZA. These observations are available from retrospective studies or case reports. However, data with respect to the long-term safety in the antenatally exposed child are still lacking. The aim of this study is to summarize the current knowledge in this field and to focus on the need for a prospective study on this population. We performed a PubMed search using several search terms. The actual data show that although the risk of congenital anomalies in offspring, as well as the infertility risk, are similar to those found in general population, there is a higher incidence of prematurity, of lower weight at birth and an intra-uterine delay of development. There is also an increased risk of materno- fetal infections, especially cytomegalovirus infection. Some authors raise the interrogations about neurocognitive impairment. Even though the adverse outcomes might well be a consequence of maternal illness and disease activity, interest has been raised about a contribution of this drug. However, the interferences between the external agent (in utero exposure to AZA), with the host (child genetic susceptibility, immune system anomalies, emotional status), environment (public health, social context, availability of health care), economic, social, and behavioral conditions, cultural patterns, are complex and represent confounding factors. In conclusion, it is necessary to perform studies on the medium and long-term outcome of children born by mothers with autoimmune diseases, treated with AZA, in order to show the safety of AZA exposure. Only large-scale population studies with long-term follow-up will allow to formally conclude in this field. TAKE HOME MESSAGES. [\hyperlink{Azathioprine Sodium}{PMID: 32240856}, Cristina Belizna et al., 2020]

\hypertarget{pmid_27435804}{S}ystemic oral immunomodulators azathioprine, methotrexate and cyclosporin are widely used in paediatric dermatology. Routine blood tests are performed to minimise drug-related adverse events. However, the frequency of monitoring tests may lead to significant fearful experiences for patients. We reviewed haematological abnormalities and clinical side-effects in a paediatric clinic population commencing immunomodulators for dermatological conditions, where haematological profiles are monitored less frequently than in current recommendations. A retrospective chart review of children started on azathioprine, methotrexate or cyclosporin for a dermatological condition between 2001-2015 from a primarily paediatric, private dermatology practice was performed. Blood tests were done at baseline, 1 month, 2 months and then 3-monthly for children on azathioprine. Children on methotrexate and cyclosporin had tests done at baseline, after 1 month and then 3-monthly. In total, 242 children were included in this study. Azathioprine, methotrexate and cyclosporin cohorts had 95, 97 and 50 patients treated for a mean duration of 656, 758 and 313 days, respectively. Isolated abnormal blood tests indicated the cessation of azathioprine in 3/95 (3\%), methotrexate in 5/97 (5\%) and cyclosporin in 2/50 (4\%) of patients. Abnormal blood test results were not associated with any reported clinical side-effects in the azathioprine (P = 0.726), methotrexate (P = 0.06) or cyclosporin groups (P = 0.250). In our experience, less frequent monitoring did not result in any significant adverse events over a 15-year period. We suggest that haematological monitoring during immunosuppressants use can be safely reduced from current recommendations. [\hyperlink{Azathioprine Sodium}{PMID: 27435804}, Joy Yee et al., 2018]

\hypertarget{pmid_16028153}{B}ecause of concerns about arthrotoxicity, fluoroquinolones are restricted for use in children. This study describes the safety and efficacy of gatifloxacin when used for treatment of children with recurrent acute otitis media (ROM) or acute otitis media (AOM) treatment failure (AOMTF). We performed an analysis of 867 children included in 4 clinical trials who had ROM and/or AOMTF and were treated with gatifloxacin (10 mg/kg once daily for 10 days). Gatifloxacin had adverse event rates that were similar overall to those of a comparator antibiotic (amoxicillin-clavulanate), except for increased diarrhea in children <2 years old receiving amoxicillin-clavulanate. There was no evidence of arthrotoxicity, hepatotoxicity, alteration of glucose homeostasis, or central nervous system toxicity acutely or during 1 year follow-up in any child. Regarding efficacy, in 2 noncomparative trials, the gatifloxacin cure rate of AOM was 89\% (95\% confidence interval [CI], 83\%-95\%) at the test of cure (TOC) visit, 3-10 days after completion of therapy. In 2 comparative trials of gatifloxacin versus amoxicillin-clavulanate, the efficacy of gatifloxacin was 88\% (95\% CI, 82\%-94\%). Gatifloxacin led to better clinical outcomes than amoxicillin-clavulanate for AOMTF (91\% vs. 81\%; P=.029), for AOMTF and age <2 years old (89\% vs. 69\%; P=.009), and for severe AOM in children <2 years old (90\% vs. 75\%; P=.012). Among children with AOMTF previously treated with amoxicillin-clavulanate or ceftriaxone injections, gatifloxacin cure rates were high (88\% and 75\%, respectively). Gatifloxacin appears to be safe for children, with no evidence of producing arthrotoxicity in 867 children exposed to the antibiotic when used as treatment for ROM and AOMTF. [\hyperlink{Azathioprine Sodium}{PMID: 16028153}, Michael E Pichichero et al., 2005]

\hypertarget{pmid_21555791}{A}n open-labelled, non-comparative study was conducted in 117 children aged 2-12 years to evaluate the efficacy and safety of azithromycin (20mg/ kg/day for 6 days) for the treatment of uncomplicated typhoid fever. Of the patients enrolled based on a clinical definition of typhoid fever, 109 (93.1\%) completed the study.Mean (SD) of duration of fever at presentation was 9.1(4.5) days. Clinical cure was seen in 102 (93.5\%) subjects, while 7 were withdrawn from the study because of clinical deterioration. Mean day of response was 3.45±1.97. BACTEC blood culture was positive for Salmonella typhi in 17/109 (15.5\%) and all achieved bacteriological cure. No serious adverse event was observed. Global well being assessed by the investigator and subjects was good in 95\% cases which was done at the end of the treatment. Azithromycin was found to be safe and efficacious for the management of uncomplicated typhoid fever. [\hyperlink{Azathioprine Sodium}{PMID: 21555791}, Anju Aggarwal et al., 2011]

\hypertarget{pmid_9629549}{W}e report on a 13 years 7 months old boy who ingested 650 mg azathioprine in a suicide attempt. His baseline medication was azathioprine and methotrexate for control of juvenile chronic polyarthritis. After the induction of vomiting by ipecacuanha sirup and administration of charcoal (1 g/kg), he was closely followed for haematological, hepatic, and renal side effects. During the following days, no serious adverse events were noted except that the thrombocyte (from 403,000 down to 199,000/microliter) and total leukocyte count decreased moderately (from 12,000 down to 7100/microliter). On the basis of this case report and the available literature, the potential acute toxicity of azathioprine and possible treatment modalities are discussed. [\hyperlink{Azathioprine Sodium}{PMID: 9629549}, C Krüger et al., ]

\hypertarget{pmid_31321320}{A}zithromycin is widely used in children not only in the treatment of individual children with infectious diseases, but also as mass drug administration (MDA) within a community to eradicate or control specific tropical diseases. MDA has also been reported to have a beneficial effect on child mortality and morbidity. However, concerns have been raised about the safety of azithromycin, especially in young children. The aim of this review is to systematically identify the safety of azithromycin in children of all ages. MEDLINE, PubMed, Cochrane Central Register of Controlled Trials, Embase, CINAHL, International Pharmaceutical Abstracts and adverse drug reaction (ADR) monitoring systems will be systematically searched for randomised controlled trials (RCTs), cohort studies, case-control studies, cross-sectional studies, case series and case reports evaluating the safety of azithromycin in children. The Cochrane risk of bias tool, Newcastle-Ottawa and quality assessment tools, and The Joanna Briggs Institute Critical Appraisal tools will be used for quality assessment. Meta-analyses will be conducted to the incidence of ADRs from RCTs if appropriate. Subgroup analyses will be performed in different age and azithromycin dosage groups. Formal ethical approval is not required as no primary data are collected. This systematic review will be disseminated through a peer-reviewed publication. CRD42018112629. [\hyperlink{Azathioprine Sodium}{PMID: 31321320}, Peipei Xu et al., 2019]

\hypertarget{pmid_14770073}{T}hree clinical trials have examined the efficacy and safety of single dose azithromycin (30 mg/kg) in children with uncomplicated acute otitis media (AOM). In the first trial, a small pilot study, the clinical and microbiologic efficacy of single dose azithromycin was comparable with that of 3-day azithromycin or single dose ceftriaxone. A second, non-comparative trial confirmed the clinical and microbiologic efficacy of the single dose regimen. The third study, a large double blind, double dummy trial, demonstrated comparable clinical success rates between single dose azithromycin and 10-day standard amoxicillin/clavulanate. The incidence of drug-related adverse events in patients treated with single dose azithromycin was low in all three trials and similar to rates that have been reported for other antimicrobial agents used for the treatment of patients with AOM. In the amoxicillin/clavulanate trial, compliance with single dose azithromycin was significantly better than with the amoxicillin/clavulanate regimen (P < 0.001). We conclude that a single dose of azithromycin (30 mg/kg) is safe and effective for the treatment of uncomplicated AOM in children. [\hyperlink{Azathioprine Sodium}{PMID: 14770073}, Adriano Arguedas et al., 2004]

\hypertarget{pmid_20525484}{I}n a small number of cases of childhood atopic dermatitis, topical therapy is ineffective, necessitating prolonged use of systemic immunosuppressants. Over the last few years, a better understanding of the metabolic pathways involved in azathioprine breakdown has enabled us to use this drug more safely. In this study, we evaluated the toxicity of azathioprine treatment adjusted to thiopurine methyltransferase activity in children with severe atopic dermatitis. We performed a retrospective study of the side effects of azathioprine therapy adjusted to thiopurine methyltransferase activity in children aged under 14 years with atopic dermatitis who were treated in the dermatology department of Hospital Universitario Insular de Gran Canaria in Gran Canaria, Spain. Side effects were evaluated by analysis of leukocyte count and transaminase levels at baseline, after 1 month of treatment, and every 3 months thereafter. During the last 4 years, 7 children (mean age, 10 years) with severe atopic dermatitis received azathioprine in our department. Mean duration of treatment was 12 months (range, 1 to 38 months). Only 2 patients presented mild transient leukopenia that did not require treatment to be suspended. Our experience shows that, when adjusted to thiopurine methyltransferase activity, azathioprine is a safe drug for the treatment of children with severe atopic dermatitis. However, clinical trials should be performed to compare the risk-benefit ratios of the different immunosuppressants used to treat these patients. [\hyperlink{Azathioprine Sodium}{PMID: 20525484}, R M Martel et al., 2010] 6-Mercaptopurine (6-MP) maintains remission in pediatric Crohn's disease (CD). Azathioprine, a prodrug of 6-MP, is used for maintenance of remission of CD in Europe. We evaluated to what extent azathioprine is used in newly diagnosed pediatric CD patients and whether maintenance of remission differed between patients using azathioprine or not. Charts of children (diagnosed 1998-2003, follow-up > or = 18 mo) were reviewed. Active disease was defined as Pediatric Crohn's Disease Activity Index (PCDAI) greater than 10 or systemic corticosteroid use. Remission was defined as PCDAI 10 or less without use of corticosteroids. Eighty-eight children (55M/33F, age 12 +/- 3 yr) were included. Seventy-two (82\%) patients received azathioprine during the follow-up period (38 +/- 17 mo). Patients diagnosed after 2000 received azathioprine significantly earlier during the course of disease compared with those diagnosed earlier (median, at 233 vs. 686 days; P < 0.05). At initial presentation, moderate-severe disease activity and prescription of corticosteroids were more prevalent in patients using azathioprine compared with nonazathioprine patients (75\% vs. 52\%; P < 0.05; and 89\% vs. 58\%; P < 0.005, respectively). Duration of corticosteroid use was longer in patients receiving azathioprine (232 vs. 168 days; P < 0.005). Median maintenance of first remission in patients who initially used corticosteroids, however, was longer in patients receiving azathioprine compared with nonazathioprine patients (PCDAI, 544 vs. 254 days, P = 0.08; corticosteroid free, 575 vs. 259 days, P < 0.05, respectively). We conclude that, since 2000, azathioprine is being introduced earlier in the treatment of newly diagnosed pediatric CD patients. The use of azathioprine is associated with prolonged maintenance of the first remission. [\hyperlink{Azathioprine Sodium}{PMID: 20525484}, Gerald J Jaspers et al., 2006]

\hypertarget{pmid_12174106}{T}here is a limited range of treatments for severe atopic dermatitis (AD). Azathioprine has often been used but there has been no randomized controlled trial of this drug to confirm its efficacy in AD. To establish or refute the efficacy of azathioprine in severe AD. To investigate the safety and tolerability of azathioprine in this patient population. We performed a double-blind, randomized, placebo-controlled, crossover trial of azathioprine in adult patients with severe AD. Each treatment period was of 3 months' duration. Treatments were azathioprine 2.5 mg kg(-1) day(-1) and matched placebo. Disease activity was monitored using the SASSAD sign score. In addition, severity of pruritus, sleep disturbance and disruption of work/daytime activity were monitored using visual analogue scales. Adverse events were recorded and haematological and biochemical monitoring was performed. Thirty-seven subjects were enrolled, mean age 38 years (range 17-73). Sixteen were withdrawn, 12 during azathioprine treatment and four during placebo treatment. The SASSAD score fell by 26\% during treatment with azathioprine vs. 3\% on placebo (P < 0.01). Pruritus, sleep disturbance and disruption of work/daytime activity all improved significantly on active treatment but not on placebo. The difference in mean improvement between azathioprine and placebo was significant for disruption of work/daytime activity (P < 0.02) but not for pruritus or sleep disturbance. Gastrointestinal disturbances were reported by 14 patients during azathioprine treatment and four were withdrawn as a result of severe nausea and vomiting. Leukopenia was observed in two patients and deranged liver enzymes in eight during treatment with azathioprine. Azathioprine is an effective and useful drug in severe AD although it is not always well-tolerated. Monitoring of the full blood count and liver enzymes is advisable during treatment. [\hyperlink{Azathioprine Sodium}{PMID: 12174106}, J Berth-Jones et al., 2002]

\hypertarget{pmid_8818855}{A}dverse events and laboratory abnormalities were monitored over 35 days after the commencement of treatment in 45 open studies conducted in Europe, South America, Africa and Asia. These studies were to assess the clinical efficacies of azithromycin and comparator antimicrobial agents in the treatment of paediatric acute bacterial infections. Children (6 months-16 years of age) had been treated with an oral suspension of azithromycin (10 mg/kg given once daily for 3 consecutive days) or with the approved oral dosing regimen of the comparator (amoxycillin, co-amoxiclav, cefixime, cefaclor, clarithromycin, erythromycin, penicillin V, cloxacillin, or roxithromycin). Adverse events were recorded in 232/2655 (8.7\%) children treated with azithromycin and in 180/1844 (9.8\%) who received comparator treatment. The majority of the treatment-related adverse events were classed as being of only mild or moderate severity and were gastrointestinal: 140 (5.3\%) in azithromycin- and 120 (6.5\%) in comparator-treated children. Co-amoxiclav was responsible for proportionately more of such events than any other agent. Treatment was discontinued prematurely due to an adverse event in 34 (1.3\%) azithromycin- and in 31 (1.7\%) comparator-treated children. Incidences of clinically significant laboratory abnormalities were low and occurred with comparable frequency in both treatment groups. The present analysis confirms that azithromycin can be safely used to treat bacterial infections in children of all ages. [\hyperlink{Azathioprine Sodium}{PMID: 8818855}, G Treadway et al., 1996]

\hypertarget{pmid_11136494}{T}he clinical effectiveness of amiodarone must be weighed against the likelihood of adverse effects. Adverse effects are less common in children than in adults, yet there have been no large studies assessing the efficacy and safety of amiodarone in the first 9 months of life. We sought to assess the safety and efficacy of amiodarone as primary therapy for supraventricular tachycardia in infancy. We evaluated the clinical course of 50 consecutive infants and neonates (1.0+/-1.5 months, 35 male) treated with amiodarone for supraventricular tachyarrhythmias between July 1994 and July 1999. At presentation, congenital heart disease, congestive heart failure, or ventricular dysfunction were present in 24\%, 36\%, and 44\% of the infants, respectively. Infants received a 7- to 10-day load of amiodarone at either 10 or 20 mg/kg/d. If this failed to control the arrhythmia, oral propranolol (2 mg/kg/d) was added. Patients were followed up for 16.0+/-13.0 months, and antiarrhythmic drugs were discontinued as tolerated. Rhythm control was achieved in all patients. Of the 34 patients who have reached 1 year of age, 23 (68\%) have remained free of arrhythmia, despite discontinuation of propranolol and amiodarone. Growth and development remained normal for age. Higher loading doses of amiodarone were associated with an increase in the corrected QT interval, but no proarrhythmia was seen. There were no side effects necessitating drug withdrawal. Amiodarone is an effective and safe therapy for tachycardia control in infancy. [\hyperlink{Azathioprine Sodium}{PMID: 11136494}, S P Etheridge et al., 2001]

\hypertarget{pmid_22892285}{A}zathioprine is prescribed as a corticosteroid-sparing agent for many inflammatory conditions, including refractory atopic dermatitis (AD). There are limited prospective data on its appropriate use and monitoring for children with AD. This study was designed to assess clinical response to azathioprine, determine the necessity for repeated measurement of thiopurine methyltransferase (TPMT) activity during treatment, and test the utility of measuring levels of the metabolites 6-thioguanine nucleotide and 6-methylmercaptopurine. Twelve children with severe, recalcitrant AD were treated with oral azathioprine and followed prospectively. Disease severity was determined by the SCORing AD index. Baseline TPMT activity was measured and this was repeated along with 6-thioguanine nucleotide and 6-methylmercaptopurine measurement at times of stable improvement, inadequate response, or change in response. Azathioprine therapy was associated with clinical improvement in all but 1 patient. There were few adverse effects. Three patients showed a significant change in TPMT activity during treatment: 2 had a mild decrease and 1 demonstrated enzyme inducibility with an increase from the intermediate to the normal activity range. These changes, but not 6-thioguanine nucleotide or 6-methylmercaptopurine levels, inversely correlated with the clinical response to therapy. Small sample size is a limitation. Azathioprine can be of benefit in the treatment of recalcitrant pediatric AD. Repeat assessment of TPMT activity may be helpful for evaluation of nonresponse or change in response and warrants further study. In contrast, measurement of thiopurine metabolites during treatment was not clinically useful. [\hyperlink{Azathioprine Sodium}{PMID: 22892285}, Maura Caufield et al., 2013]

\hypertarget{pmid_12174104}{A}topic eczema is a chronic inflammatory skin disease, which in most children can be adequately controlled using topical therapy. However, in a small number of children it is necessary to use systemic treatments to gain an acceptable level of disease control. To evaluate azathioprine as a treatment for severe atopic eczema in children, and the value of pretreatment thiopurine methyltransferase (TPMT) levels in the identification of patients at high risk of myelosuppression. Between January 1997 and May 2000, 91 children had erythrocyte TPMT assays with the intention of treating their atopic eczema with azathioprine. This study is based on retrospective examination of data taken from the hospital notes of these children, who had attended Great Ormond Street Hospital for Children and St John's Institute of Dermatology, London. The distribution of TPMT values corresponded closely to that previously described in adults. Forty-eight children were commenced on treatment with azathioprine. Twenty-eight had an excellent response to treatment, 13 had a good response and seven had a poor response. No patient developed neutropenia. Azathioprine may prove a very valuable treatment for severe atopic eczema in children. We consider its short-term adverse effect profile in children with normal TPMT activity to have been entirely acceptable with our treatment protocol. As result, we now feel confident to initiate therapy at dose levels of 2.5-3.5 mg kg(-1) in those with a normal TPMT level, and to reduce the frequency with which we undertake tests of bone marrow and liver function. [\hyperlink{Azathioprine Sodium}{PMID: 12174104}, L-A Murphy et al., 2002]

\hypertarget{pmid_8878240}{I}n this multicenter, open label trial the investigators evaluated the efficacy and safety of azithromycin suspension administered once daily for 5 days for the treatment of clinically and bacteriologically established acute otitis media. Two hundred eligible children with acute otitis media from 10 US centers were treated with 10 mg/kg of azithromycin oral suspension on Day 1, followed by 5 mg/kg once daily for the next 4 days. Tympanocentesis and subsequent culture of middle ear effusion were performed at baseline. Clinical efficacy was evaluated on Days 6, 11 and 30. Analysis of clinical efficacy in evaluable patients 11 days after the initiation of therapy showed that the rate of satisfactory responses (cured or improved) ranged from 79.6 to 82.4\% in patients infected with Streptococcus pneumoniae, Haemophilus influenzae, or Moraxella catarrhalis. Satisfactory clinical response at Day 30 was reported in 70\% of evaluable patients, and eradication of S. pneumoniae, H. influenzae and M. catarrhalis was presumed in 64 to 73\%. Relapses occurred in 14\% of the evaluable patients. Among the treated patients 8.5\% reported mild or moderate side effects. Azithromycin is an effective, safe and well-tolerated treatment for children with acute otitis media. [\hyperlink{Azathioprine Sodium}{PMID: 8878240}, J McCarty et al., 1996]

\hypertarget{pmid_33729325}{A}zathioprine is a common first-line therapy for neuromyelitis optica spectrum disorder (NMOSD). The aim of this study was to determine whether long-term treatment (>10 years) with azathioprine is safe in NMOSD. Methods: We conducted a retrospective medical record review of all patients at the School of Medicine of the University of São Paulo (São Paulo, Brazil) who fulfilled the 2015 international consensus diagnostic criteria for NMOSD and were treated with azathioprine for at least 10 years. Out of 375 patients assessed for eligibility, 19 were included in this analysis. These patients' median age was 44 years (range=28-61); they were mostly female (17/19) and AQP4-IgG seropositive (18/19). The median disease duration was 15 years (range=10-39) and most patients presented a relapsing clinical course (84.2\%). The median duration of treatment was 11.9 years (range=10.0-23.8). The median annualized relapse rates (ARR) pre- and post-treatment with azathioprine were 1 (range=0.1-2) and 0.1 (range=0-0.35); p=0.09. Three patients (15.7\%) had records of adverse events during the follow-up, which consisted of chronic B12 vitamin deficiency, pulmonary tuberculosis and breast cancer. Azathioprine may be considered a safe agent for long-term treatment (>10 years) of NMOSD, but continuous vigilance for infections and malignancies is required. [\hyperlink{Azathioprine Sodium}{PMID: 33729325}, Ana Beatriz Ayroza Galvão Ribeiro Gomes et al., 2021]

\hypertarget{pmid_8988412}{A}zithromycin (AZM), 10\% fine granules or 100 mg capsules, were given orally to 27 children with various pediatric infections. The results of the study are shown below. 1. Pharmacokinetic investigation. We studied plasma and urinary concentrations after 100 mg AZM capsules were given. One patient received 8.3 mg/kg of AZM once a day for 3 days, and AZM concentration in plasma was 0.033 microgram/ml 48 hours after the final dosing. Doses of 8.3 and 12.5 mg/kg body weight of AZM were respectively given to two patients once daily for 3 days. As a result, AZM concentrations in urine during a period between 96 and 120 hours post-dosing were 1.67 and 4.53 micrograms/ml, respectively, and urinary excretion rate in 120 hours after the first dosing was 10.54\% in the patient that was given 12.5 mg/kg. 2. Clinical investigation. Clinical efficacies were examined in 24 patients. Excellent results were obtained in 7 patients, good results in 14 patients, hence the clinical efficacy rate was 87.5\%. Bacteriologically, Haemophilus influenzae strains isolates from 2 patients were eradicated in 1 and decreased in the other. Safety was evaluated in 26 patients. An adverse reaction was observed in 1 patient (urticaria). Abnormal laboratory test results were observed in 2 patients, decreased WBC in 1 and elevation of eosinophils in the other. The above results suggest that AZM is a useful oral antibiotic for pediatric patients with infection with susceptible organisms. [\hyperlink{Azathioprine Sodium}{PMID: 8988412}, Y Kobayashi et al., 1996]

\hypertarget{pmid_8396100}{I}n this open study, a three-day regimen of azithromycin (single daily dose of 10 mg/kg) was compared with a ten-day regimen of amoxycillin paediatric suspension (30 mg/kg/day in three divided doses; children > 20 kg received 250 mg tid daily) in 154 children (aged 2-12 years) with a clinical diagnosis of acute otitis media (13 recurrent). Full clinical, bacteriological and laboratory safety assessments were performed during and after the study. Of the 77 azithromycin patients, 61 (79\%) were considered cured, 15 (19\%) improved and one (1\%) failed, compared with 45 (58\%) cured, 28 (36\%) improved and four (5\%) failed among the 77 amoxycillin patients. Excluding from analysis the 13 patients with recurrent otitis media, azithromycin was found to be significantly superior to amoxycillin (P = 0.003). The incidence of side-effects was low, with only two (3\%) and three (4\%) patients reporting adverse events with azithromycin and amoxycillin, respectively. These were gastrointestinal in nature and of mild or moderate severity, except for one case of severe diarrhoea in the amoxycillin group. No treatment-related abnormalities in the laboratory safety tests were observed, and no patients withdrew from therapy. A three-day regimen of azithromycin was therefore shown to be more effective than, and as well tolerated as, amoxycillin in the treatment of children with acute otitis media. [\hyperlink{Azathioprine Sodium}{PMID: 8396100}, E Mohs et al., 1993]

\hypertarget{pmid_24370666}{A}lthough paediatric patients frequently suffer from intoxications with atypical antipsychotics, the number of studies in young children, which have assessed the effects of acute exposure to this class of drugs, is very limited. The aim of this study was to achieve a better characterization of the acute toxicity profile in young children of the atypical antipsychotics clozapine, olanzapine, quetiapine, and risperidone. We performed a multicentre retrospective analysis of cases with atypical antipsychotics intoxication in children younger than 6 years, reported by physicians to German, Austrian, and Swiss Poisons Centres for the 9-year period between January 1, 2001 and December 31, 2009. One hundred and six cases (31 clozapine, 29 olanzapine, 12 quetiapine, and 34 risperidone) were available for analysis. Forty-seven of the children showed minor, 28 moderate, and 2 severe symptoms. Twenty-nine cases were asymptomatic. No fatalities were recorded. Symptoms predominantly involved the central nervous and cardiovascular systems. Minor reduction in vigilance (Glasgow Coma Scale score >9) (62 \%) was the most frequently reported symptom, followed by miosis (12 \%) and mild tachycardia (10 \%). Extrapyramidal motor symptoms were observed in one case (1 \%) after ingestion of risperidone. In most cases, surveillance and supportive care were sufficient to achieve a good outcome, and all children made full recovery. Paediatric antipsychotic exposure can result in significant poisoning; however, in most cases only minor or moderate symptoms occurred and were followed by complete recovery. Symptomatic patients should be monitored for central nervous system depression and an electrocardiogram should be obtained. [\hyperlink{Azathioprine Sodium}{PMID: 24370666}, Marianne Meli et al., 2014]

\hypertarget{pmid_34465327}{A}zithromycin has recently been shown to reduce all-cause childhood mortality in sub-Saharan Africa. One potential mechanism of this effect is via the anti-malarial effect of azithromycin, which may help treat or prevent malaria infection. This study evaluated short- and longer-term effects of azithromycin on malaria outcomes in children. Children aged 8 days to 59 months were randomized in a 1:1 fashion to a single oral dose of azithromycin (20 mg/kg) or matching placebo. Children were evaluated for malaria via thin and thick smear and rapid diagnostic test (for those with tympanic temperature ≥ 37.5 °C) at baseline and 14 days and 6 months after treatment. Malaria outcomes in children receiving azithromycin versus placebo were compared at each follow-up timepoint separately. Of 450 children enrolled, 230 were randomized to azithromycin and 220 to placebo. Children were a median of 26 months and 51\% were female, and 17\% were positive for malaria parasitaemia at baseline. There was no evidence of a difference in malaria parasitaemia at 14 days or 6 months after treatment. In the azithromycin arm, 20\% of children were positive for parasitaemia at 14 days compared to 17\% in the placebo arm (P = 0.43) and 7.6\% vs. 5.6\% in the azithromycin compared to placebo arms at 6 months (P = 0.47). Azithromycin did not affect malaria outcomes in this study, possibly due to the individually randomized nature of the trial. Trial registration This study is registered at clinicaltrials.gov (NCT03676751; registered 19 September 2018). [\hyperlink{Azathioprine Sodium}{PMID: 34465327}, Boubacar Coulibaly et al., 2021]

\hypertarget{pmid_23139238}{A}zathioprine (AZA) is recognized among immunosuppressive medications as relatively safe during pregnancy for women with systemic lupus erythematosus (SLE) requiring aggressive treatment. This pilot study aimed to determine whether SLE therapy during pregnancy was associated with developmental delays in offspring. This cohort study included SLE patients with at least one live birth postdiagnosis. Medical histories were obtained via interviews and chart review. Multiple logistic regression was used to examine associations between SLE therapy during pregnancy and maternal report of special educational (SE) requirements (as proxy for developmental delays) among offspring. Propensity scoring (incorporating corticosteroid use, lupus flare, and lupus nephritis) was used to account for disease severity. Of 60 eligible offspring from 38 mothers, 15 required SE services, the most common indication for which was speech delay. Seven (54\%) of the 13 children with in utero AZA exposure utilized SE services versus 8 (17\%) of 47 nonexposed children (P < 0.01). After adjustment for pregnancy duration, small for gestational age, propensity score, maternal education level, and antiphospholipid antibody syndrome, AZA was significantly associated with SE utilization occurring from age 2 years onward (odds ratio 6.6, 95\% confidence interval 1.0-43.3), and bordered on significance for utilization at any age or age <2 years. AZA exposure during SLE pregnancy was independently associated with increased SE utilization in offspring, after controlling for confounders. Further research is indicated to fully characterize developmental outcomes among offspring with in utero AZA exposure. Vigilance and early interventions for suspected developmental delays among exposed offspring may be warranted. [\hyperlink{Azathioprine Sodium}{PMID: 23139238}, Wendy Marder et al., 2013]

\hypertarget{pmid_30234756}{A}zathioprine (AZA) is the mainstay of maintenance therapy in pediatric autoimmune hepatitis (AIH). The use of thiopurines metabolites to individualize therapy and avoid toxicity has not, however, been clearly defined. Retrospective analysis of children ≤18 years diagnosed with AIH between January 2001 and 2016. Standard definitions were used for treatment response and disease flare. Thiopurine metabolite levels were correlated with the corresponding liver function test. A total of 56 children (32 girls) were diagnosed with AIH at a median age of 11 years (interquartile range [IQR] 9). No difference in 6-thioguanine-nucleotide (6-TG) levels (271[IQR 251] pmol/8 × 10 red blood cell vs 224 [IQR 147] pmol/8 × 10 red blood cell, P = 0.06) was observed in children in remission when compared with those who were not in remission. No correlation was observed between the 6-TG and alanine aminotransferase levels (r = -0.179, P = 0.109) or between 6-methyl-mercaptopurine (6-MMP) and alanine aminotransferase levels (r = 0.139, P = 0.213). The 6-MMP/6-TG ratio was significantly lower in patients who were in remission (2[7] vs 5 (10), P = 0.04). Using a quartile analysis, we found that having a ratio of <4 was significantly associated with being in remission with OR 2.50 (95\% confidence interval 1.02-6.10), P = 0.047. Use of allopurinol with low-dose AZA in 6 children with preferential 6-MMP production brought about remission in 5/6 (83.3\%). Thiopurine metabolite levels should be measured in patients with AIH who have experienced a loss of remission. A 6-MMP/6-TG ratio of <4 with the addition of allopurinol could be considered in these patients. [\hyperlink{Azathioprine Sodium}{PMID: 30234756}, Rishi Bolia et al., 2018]

\hypertarget{pmid_12030954}{T}o assess the efficacy and safety of azathioprine in a paediatric population with inflammatory bowel disease. One hundred and twenty-three Italian children treated with azathioprine were studied retrospectively. The treatment duration and causes of its discontinuation, side-effects and variation in corticosteroid dose were assessed. The mean age at inflammatory bowel disease diagnosis was 9.8 +/- 3.6 years, and at the start of azathioprine therapy 11.8 +/- 4.3 years. The mean duration of treatment was 19 +/- 16 months. Fifty patients (41\%) stopped treatment due to surgery (12\%), prolonged remission (11\%), non-response (7\%), severe side-effects (7\%) and poor compliance (3\%). Of the 73 patients (59\%) remaining on azathioprine, 11 had never been treated with corticosteroids, 27 were able to stop them and 35 were still on a very low daily dose (91\% < 0.3 mg/kg). The difference in the daily corticosteroid dose between the beginning of azathioprine treatment (1 +/- 0.6 mg/kg) and the conclusion of the study (0.18 +/- 0.16 mg/kg) was statistically significant. Side-effects were recorded in 48 of the 123 patients (39\%), but only eight required discontinuation of azathioprine. Azathioprine was efficacious in 70\% of patients, but ineffective in 20\% and induced severe toxicity in 7\%. Corticosteroids were stopped or markedly reduced in 62\% of patients, but they were never given in 9\%. [\hyperlink{Azathioprine Sodium}{PMID: 12030954}, A Barabino et al., 2002]

\section*{Bupivacaine Hydrochloride}
\subsection*{Result}
\subsubsection*{Answer}

Yes (Infants, 6 months and older)  
Yes (Toddlers and Young Children, 1–5 years)  
Yes (Children, 6–12 years)  
Yes (Adolescents, 12–18 years)  
Unknown (Neonates and infants under 6 months: limited direct safety data, but some case series suggest use without reported toxicity)

\subsubsection*{{Explanation}}
\hypertarget{Bupivacaine Hydrochloride}
A review of the available abstracts reveals multiple targeted studies evaluating the safety of Bupivacaine Hydrochloride in children across various age ranges and surgical contexts. Below is a summary of the evidence by age group:

Infants (under 1 year, including neonates and infants up to 8 months):
- A study of 12 cases used hyperbaric bupivacaine for spinal anesthesia in children younger than 8 months. The abstract does not report adverse events or toxicity, but also does not explicitly affirm safety; it is primarily a case series documenting use [\hyperlink{pmid_2265161}{PMID: 2265161}, S K Parkinson et al.].
- A pharmacokinetic study included children as young as 3 months, showing that bupivacaine concentrations after intercostal nerve block remained below toxic levels, but the abstract does not explicitly state safety outcomes [\hyperlink{pmid_3706800}{PMID: 3706800}, P Rothstein et al., 1986].
- Another study included infants weighing less than 12 kg (some likely under 1 year) and found that caudal anesthesia with bupivacaine at 2.5 mg/kg resulted in plasma concentrations below toxic values, but recommends caution with dosing in small infants [\hyperlink{pmid_3729087}{PMID: 3729087}, J Camboulives et al., 1986].
- A large prospective study of 1132 children aged 6 months to 14 years found spinal anesthesia with hyperbaric bupivacaine to be effective and safe, with a low incidence of complications and no neurological complications reported [\hyperlink{pmid_15200653}{PMID: 15200653}, Franco Puncuh et al., 2004].

Toddlers and Young Children (1–5 years):
- Multiple studies, including those on caudal, epidural, and local infiltration, report safe use of bupivacaine in children as young as 1 year, with plasma concentrations below toxic thresholds and no significant adverse events [\hyperlink{pmid_3706800}{PMID: 3706800}, P Rothstein et al., 1986; \hyperlink{pmid_3729087}{PMID: 3729087}, J Camboulives et al., 1986; \hyperlink{pmid_29520391}{PMID: 29520391}, Kyoung Lee et al., 2018].
- A study specifically divided children into age groups (under 2, 3–4, and 5+ years) and found no hemodynamic or cardiac adverse effects after local infiltration of bupivacaine and lidocaine, with no adverse events reported in any group [\hyperlink{pmid_29520391}{PMID: 29520391}, Kyoung Lee et al., 2018].

Children (6–12 years):
- Several randomized controlled trials and prospective studies in this age group (including caudal, epidural, local infiltration, and nerve blocks) consistently report that bupivacaine hydrochloride, when used at recommended doses, is not associated with toxic plasma levels or significant adverse events [\hyperlink{pmid_37871332}{PMID: 37871332}, Andrew T Gabrielson et al., 2024; \hyperlink{pmid_34534923}{PMID: 34534923}, Christopher F Tirotta et al., 2021; \hyperlink{pmid_29520391}{PMID: 29520391}, Kyoung Lee et al., 2018; \hyperlink{pmid_15200653}{PMID: 15200653}, Franco Puncuh et al., 2004].
- A large study of 1132 children aged 6 months to 14 years found spinal anesthesia with bupivacaine to be safe and effective [\hyperlink{pmid_15200653}{PMID: 15200653}, Franco Puncuh et al., 2004].
- A study of 55 children (grouped by age, including 5+ years) found no adverse events or significant changes in hemodynamic or ECG parameters [\hyperlink{pmid_29520391}{PMID: 29520391}, Kyoung Lee et al., 2018].
- A study of 104 children aged 6–18 years undergoing urologic surgery found bupivacaine hydrochloride (alone or with liposomal bupivacaine) to be well tolerated, with no significant safety concerns [\hyperlink{pmid_37871332}{PMID: 37871332}, Andrew T Gabrielson et al., 2024].

Adolescents (12–18 years):
- The above studies include adolescents up to 16 or 18 years and report no significant safety concerns when bupivacaine is used at recommended doses [\hyperlink{pmid_37871332}{PMID: 37871332}, Andrew T Gabrielson et al., 2024; \hyperlink{pmid_34534923}{PMID: 34534923}, Christopher F Tirotta et al., 2021; \hyperlink{pmid_15200653}{PMID: 15200653}, Franco Puncuh et al., 2004].

Summary:
Across all pediatric age groups, there are multiple targeted studies (including large prospective cohorts and randomized trials) that affirm the safety of Bupivacaine Hydrochloride for various regional anesthesia techniques (caudal, epidural, spinal, local infiltration, and nerve blocks) when used at recommended doses. These studies report plasma concentrations below toxic thresholds and a low incidence of adverse events, with no reports of serious toxicity or mortality attributable to bupivacaine in these pediatric populations. Some studies recommend caution with dosing in small infants, but overall, the evidence supports safety in children from infancy through adolescence.

\subsection*{Abstracts}
\hypertarget{pmid_3706800}{A} pharmacokinetic evaluation of bupivacaine was carried out after intercostal nerve blocks performed on 28 occasions in 27 children varying in age from 3 months to 16 yr. Bupivacaine HCl, 0.5\%, with epinephrine 1:200,000 was employed. Doses of 2 mg/kg, 3 mg/kg, and 4 mg/kg resulted in peak whole blood arterial bupivacaine (base) concentrations (mean +/- SD) of 0.77 +/- 0.25 microgram/ml, 1.37 +/- 0.23 microgram/ml, and 1.87 +/- 0.53 microgram/ml, respectively. Calculated pharmacokinetic parameters (mean +/- SD) were the following: apparent volume of distribution (VD beta), 2.8 +/- 0.8 L/kg; steady-state volume of distribution (VDss), 2.7 +/- 0.7 L/kg; elimination half-life (t1/2 beta), 147 +/- 80 min; and total body clearance (Cl), 16.0 +/- 7.4 ml X min-1 X kg-1, or 382 +/- 201 ml X min-1 X m-2. Compared with data reported for adult patients, our data indicate that the volume of distribution is greater and clearance is more rapid in children than in adults. The absorption of local anesthetic from the intercostal space appears to be more rapid in children than adults. In an additional group of 11 children, the relationship of the bupivacaine blood:plasma concentration ratio (lambda) to hematocrit was investigated. Hematocrit in this group ranged from 30 to 59, and lambda varied from 0.47 to 0.82. There was a significant relationship between lambda and hematocrit defined by the equation lambda = -0.0079 Hct + 1.028 (r = 0.72, P less than 0.05). Reporting bupivacaine concentration in terms of plasma concentration may introduce an artifact that is dependent on the hematocrit, and we therefore suggest that whole blood concentration values be reported by investigators in the future. [\hyperlink{Bupivacaine Hydrochloride}{PMID: 3706800}, P Rothstein et al., 1986]

\hypertarget{pmid_6859502}{P}lasma bupivacaine concentrations were measured in 45 children, whose ages ranged from 4 months to 12 years, following administration of caudal epidural analgesia. Using 3 mg/kg of bupivacaine 0.25\%, mean blood levels of 1.2-1.4 microgram/ml were reached, which are well within the limits of projected toxic levels. Simultaneous arterial and venous sampling showed a small but significant difference between these two sampling sites fo the first fifteen minutes. [\hyperlink{Bupivacaine Hydrochloride}{PMID: 6859502}, R L Eyres et al., 1983]

\hypertarget{pmid_37871332}{W}e sought to determine if the addition of liposomal bupivacaine to bupivacaine hydrochloride improves opioid-free rate and postoperative pain scores among children undergoing ambulatory urologic surgery. A prospective, phase 3, single-blinded, single-center randomized trial with superiority design was conducted in children 6 to 18 years undergoing ambulatory urologic procedures between October 2021 and April 2023. Patients were randomized 1:1 to receive dorsal penile nerve block (penile procedures) or incisional infiltration with spermatic cord block (inguinal/scrotal procedures) with weight-based liposomal bupivacaine plus bupivacaine hydrochloride or bupivacaine hydrochloride alone. The primary outcome was opioid-free rate at 48 hours. Secondary outcomes included parents' postoperative pain measure scores, numerical pain scale scores, and weight-based opioid utilization at 48 hours and 10 to 14 days. We randomized 104 participants, with > 98\% (102/104) with complete follow-up data at 48 hours and 10 to 14 days. At interim analysis, there was no significant difference in opioid-free rate at 48 hours between arms (60\% in the intervention vs 62\% in the control group; estimated difference in proportion -1.9\% [95\% CI, -20\%-16\%];  The addition of liposomal bupivacaine to bupivacaine hydrochloride did not significantly improve opioid-sparing effect or postoperative pain compared with bupivacaine hydrochloride alone among children ≥ 6 years undergoing ambulatory urologic surgery. [\hyperlink{Bupivacaine Hydrochloride}{PMID: 37871332}, Andrew T Gabrielson et al., 2024]

\hypertarget{pmid_29520391}{L}ocal anesthetic agents such as bupivacaine and lidocaine are commonly used after surgery for pain control. The aim of this prospective study was to evaluate the safety of a mixture of bupivacaine and lidocaine in children who underwent urologic inguinal and scrotal surgery. Fifty-five patients who underwent pediatric urologic outpatient surgeries, were prospectively enrolled in this study. The patients were divided into three groups according to age (group I: under 2 years, group II: between 3-4 years, and group III: 5 years and above). Patients were further sub-divided into unilateral and bilateral groups. All patients were injected with a mixture of 0.5\% bupivacaine and 2\% lidocaine (2:1 volume ratio) at the surgical site, just before the surgery ended. Hemodynamic and electrocardiographic parameters were measured before local anesthesia, 30 minutes after administration of local anesthesia, and 60 minutes after administration. The patients' mean age was 40.5±39.9 months. All patients had normal hemodynamic and electrocardiographic parameters before local anesthesia, after 30 minutes, and after 60 minutes. Also, results of all intervals were within normal values, when analyzed by age and laterality. No mixture related adverse events (nausea, vomiting, pruritus, sedation, respiratory depression) or those related to electrocardiographic parameters (arrhythmias and asystole) were reported in any patients. A mixture of bupivacaine and lidocaine can be safely used in children undergoing urologic inguinal and scrotal surgery. An appropriate dose has no clinically significant hemodynamic or cardiac changes and adverse effects. [\hyperlink{Bupivacaine Hydrochloride}{PMID: 29520391}, Kyoung Lee et al., 2018]

\hypertarget{pmid_10551577}{B}upivacaine provides reliable, long-lasting anesthesia and analgesia when given via the caudal route. Ropivacaine is a newer, long-acting local anesthetic that (at a concentration providing similar pain relief) has less motor nerve blockade and may have less cardiotoxicity than bupivacaine. In a double-blind trial, 81 healthy children, undergoing ambulatory surgical procedures, were randomly allocated to receive caudal analgesia with either bupivacaine or ropivacaine, 0.25\%, 1 mVkg. All blocks were placed by an attending anesthesiologist or an anesthesia fellow after induction of general anesthesia. Data were available for 75 children. There were no significant differences between the two groups in baseline characteristics or in anesthesia, surgery, recovery room, or day surgery unit durations. The quality and duration of postoperative pain relief did not differ. Motor and sensory effects were similar. Time to first micturition did not differ. Ropivacaine (0.25\%, 1 ml/kg) provided adequate postoperative analgesia with no difference from bupivacaine (0.25\%, 1 ml/kg) in quality and duration of pain relief, motor and sensory effects, or time to first micturition in our study children. [\hyperlink{Bupivacaine Hydrochloride}{PMID: 10551577}, S Khalil et al., 1999]

\hypertarget{pmid_564643}{B}upivacaine (Marcaine) hydrochloride, a long-acting local anesthetic drug, was used in concentrations of 0.25, 0.5, or 0.75 percent with and without a vasoconstrictor, in amounts ranging from 25 to over 600 mg, for caudal, epidural (peridural), or peripheral nerve block for 11,080 surgical, obstetrical, diagnostic, or therapeutic procedures. Onset of anesthesia occurred in 4 to 10 minutes and maximum anesthesia in 15 to 35 minutes. Concentrations of 0.25, 0.5, and 0.75 percent consistently produced complete sensory anesthesia of the integumentary and musculoskeletal systems. With 0.25 and 0.5 percent, motor blockade ranged from minimal to complete. In intra-abdominal surgery, only 0.75 percent consistently produced profound muscle relaxation. Fifteen systemic toxic reactions occurred, but no untoward sequelae resulted from them. One inadvertent subarachnoid injection of 110 mg resulted in a total spinal block with an uneventful recovery. [\hyperlink{Bupivacaine Hydrochloride}{PMID: 564643}, D C Moore et al., ]

\hypertarget{pmid_9567153}{E}pidural anaesthesia is extremely useful in providing postoperative analgesia for children after surgery of the lower body. Although results on early pharmacokinetics in children have previously been reported, no data are available on the long-term effects of epidural anaesthesia. The aim of this investigation was the assessment of plasma bupivacaine levels in children with continuous epidural anaesthesia in the postoperative period. A catheter with an outer diameter of 0.63 mm was inserted through a 19G Tuohy cannula into the epidural space. A maximum dose of 0.4 mg/kg/h bupivacaine was administered for continuous epidural infusion. Careful monitoring was performed to detect early signs of local anaesthetic intoxication. Two milliliters of blood were obtained in each patient per day and nepholometric serum measurement were performed to determine alpha 1-acid glycoprotein and albumin levels. Bupivacaine plasma concentrations were assessed according to the method described by Sattler et al. [25]. Ten children were included in the investigation. The measured albumin and alpha 1-acid glycoprotein concentrations were within the range described by other investigators. At the onset of pain therapy maximum levels of 0.5 microgram/ml were recorded after a loading dose of bupivacaine and levels of up to 2.2 micrograms/ml were achieved following continuous infusion. There were no neurologic complications or signs of local anesthetic intoxication. In conclusion our results show that a dose of up to 0.4 mg/kg/h bupivacaine during continuous epidural infusion is not associated with toxic complications. Careful monitoring of the children by experienced staff is mandatory. [\hyperlink{Bupivacaine Hydrochloride}{PMID: 9567153}, A Scherhag et al., 1998]

\hypertarget{pmid_2872745}{B}upivacaine was utilized for postoperative analgesia in patients undergoing orchiopexy and hernia repair. In a study of 75 pediatric patients, ranging in ages from twelve months to twelve years, who had undergone orchiopexy and hernia repair during a three-year period, 42 received bupivacaine hydrochloride as a local infiltration block anesthesia to relieve postoperative pain; 33 patients did not receive bupivacaine. Patients receiving bupivacaine had less postoperative pain and were more comfortable when leaving the hospital within a few hours after surgery. [\hyperlink{Bupivacaine Hydrochloride}{PMID: 2872745}, R Baghdassarian et al., 1986]

\hypertarget{pmid_34534923}{T}o evaluate the pharmacokinetics and safety of liposomal bupivacaine in pediatric patients undergoing spine or cardiac surgery. Multicenter, open-label, phase 3, randomized trial (PLAY; NCT03682302). Operating room. Two separate age groups were evaluated (age group 1: patients 12 to <17 years undergoing spine surgery; age group 2: patients 6 to <12 years undergoing spine or cardiac surgery). Randomized allocation of liposomal bupivacaine 4 mg/kg or bupivacaine hydrochloride (HCl) 2 mg/kg via local infiltration at the end of spine surgery (age group 1); liposomal bupivacaine 4 mg/kg via local infiltration at the end of spine or cardiac surgery (age group 2). The primary and secondary objectives were to evaluate the pharmacokinetics (eg, maximum plasma bupivacaine concentrations [C Baseline characteristics were comparable across groups. Mean C Plasma bupivacaine levels following local infiltration with liposomal bupivacaine remained below the toxic threshold in adults (\textasciitilde{}2000-4000 ng/mL) across age groups and procedures. AEs were mild to moderate, supporting the safety of liposomal bupivacaine in pediatric patients undergoing spine or cardiac surgery. Clinical trial number and registry URL: ClinicalTrials.gov identifier: NCT03682302. [\hyperlink{Bupivacaine Hydrochloride}{PMID: 34534923}, Christopher F Tirotta et al., 2021]

\hypertarget{pmid_264254}{A} study was developed in an attempt to investigate the possible usefulness of the local anesthetic agent, bupivacaine hydrochloride, for oral surgery. The results show that bupivacaine hydrochloride is an effective local anesthetic agent. It has a rapid onset time, a high frequency of surgical anesthesia, a long duration, and a low incidence of side effects. In comparison to lidocaine, bupivacaine has a greater potency, a lower toxicity at equipotent doses a longer duration, a possible pain-free period after return of normal sensation, and it does not require a vasoconstrictor for consistent profoundness. [\hyperlink{Bupivacaine Hydrochloride}{PMID: 264254}, J L Laskin et al., 1977]

\hypertarget{pmid_36997075}{T}his study evaluates the tolerability and efficacy of preoperative dorsal penile nerve block with Exparel plus bupivacaine hydrochloride in children>6 years old undergoing ambulatory urologic surgery. We demonstrate that the drug combination is well-tolerated, with appropriate analgesic efficacy in the recovery room as well as at 48-hour and 10-14 day follow-up periods. These preliminary data justify the need to perform a prospective, randomized trial comparing Exparel plus bupivacaine hydrochloride to other common local anesthetic regimens used in pediatric urologic surgery. [\hyperlink{Bupivacaine Hydrochloride}{PMID: 36997075}, Andrew T Gabrielson et al., 2023]

\hypertarget{pmid_25185333}{B}upivacaine hydrochloride is frequently used in veterinary dental procedures to reduce the amount of general anesthesia needed and to reduce post-procedural pain. The aim of this study was to develop a novel method to test local anesthetic duration in mammals. Six infant pigs were placed under deep/surgical anesthesia with 3 \% isoflurane and oxygen while 0.5 ml of 0.5\% bupivacaine hydrochloride was injected to block the two greater palatine and the nasopalatine nerves. They were then maintained under light anesthesia with 0.5-1.0\% isoflurane. Beginning 15-minutes after the injection, 7 sites in the oral cavity were stimulated using a pointed dental waxing instrument, including 3 sites on the hard palate. The response, or lack of response, to the stimulus was recorded on video and in written record The bupivacaine hydrochloride injections lasted 1 to 3-hours before the animals responded to the sensory stimulation with a reflexive movement This study provides evidence that bupivacaine used to anesthetize the hard palate has a relatively short and variable duration of action far below what is expected based on its pharmacokinetic properties. [\hyperlink{Bupivacaine Hydrochloride}{PMID: 25185333}, Shaina Devi Holman et al., 2014]

\hypertarget{pmid_3729087}{S}erum concentrations of lidocaine and plasma concentrations of bupivacaine were measured so as to assess the risk of systemic toxicity following their administration by the caudal route in children, and study their pharmacokinetic profiles according to age. The serum concentrations of lidocaine were measured by immuno-enzymology in 37 children (23 +/- 13 kg) during the first hour after administration of 7 mg . kg-1. The plasma concentrations of bupivacaine were measured by high performance liquid chromatography in 40 children (18.03 +/- 8.90 kg) during the first hour after administration of 2.5 mg . kg-1. The greatest concentrations observed between 15 and 30 min after the injection were of 2.40 +/- 0.86 micrograms . ml for lidocaine and 0.93 +/- 0.44 microgram . ml-1 for bupivacaine. Higher values were observed in infants weighing less than 12 kg where they reached 2.89 +/- 0.72 and 1.52 +/- 0.68 micrograms . ml-1 respectively. These results showed that caudal anaesthesia with lidocaine (7 ml . kg-1) and bupivacaine (2.5 ml . kg-1) was a safe technique for children, giving average plasma concentrations inferior to toxic values. However, it seemed prudent not to give more than the prescribed doses in the small infant. [\hyperlink{Bupivacaine Hydrochloride}{PMID: 3729087}, J Camboulives et al., 1986]

\hypertarget{pmid_15200653}{S}pinal anaesthesia has been used in children for over 100 years and in the last two decades its popularity for newborns and infants has increased, but there are still unanswered questions with the technique. We evaluated the characteristics of spinal block including ease of performance, efficacy, adverse effects and complications in 1132 children, aged 6 months to 14 years, undergoing surgery in the lower part of the body. Local ethical committee approved the protocol of this prospective study, and parents gave written informed consent and older children their assent. All patients were sedated with midazolam, thiopental or propofol intravenously during spontaneous ventilation. No inhalation anaesthetics were used. Spinal block was performed with 0.5\% hyperbaric bupivacaine at a dose of 0.2 mg x kg(-1). Efficacy, safety and ease of performance of the spinal block were shown to be satisfactory in most children. Only 27 of the 1132 children needed any supplementation. The incidence and severity of complications was low. Only nine of 942 children, < 10 years of age and eight of 190 children, 10 years or older, developed hypotension. The incidences of postdural puncture headache, in five of the 1132 children, and backache, in nine of the 1132, were low. No other neurological complications were reported. Spinal anaesthesia with hyperbaric bupivacaine is a feasible anaesthetic method in children for surgery in the lower part of the body. [\hyperlink{Bupivacaine Hydrochloride}{PMID: 15200653}, Franco Puncuh et al., 2004]

\hypertarget{pmid_12060325}{S}ince 1970, bupivacaine 0.25\% in a dose of 4 mg x kg-1 (1.6 ml x kg-1) has been used at the Hospital Infantil de México for caudal block in children undergoing surgical correction of congenital pyloric stenosis (CPS). Although this dose is considered unsafe, in our experience, it has been associated with a high success rate and a low incidence of adverse events. This experience has not been previously documented. A retrospective cohort of patients undergoing surgical correction of CPS was studied. Nineteen patients received general anaesthesia while 223 received caudal block. The latter were then grouped according to the sedation technique. The rate of successful caudal blocks and complications were considered the major outcomes of the study, whereas the postsurgical fasting period and hospital stay were considered secondary outcomes. The rate of success of caudal block was 96\%. Anaesthetic complications related to bupivacaine were present in 1.3\%. Mortality occurred in the postoperatory period in one septic patient who also was suffering from gastroschisis that required general anaesthesia. Postoperatory fasting period and hospital stay tended to be higher with general anaesthesia than caudal block. However, of the 19 patients receiving general anaesthesia, five suffered serious comorbidity and nine were failed caudal blocks. Caudal block with bupivacaine 0.25\% (4 mg x kg-1) was associated with a low rate of anaesthetic complications. Further prospective studies to clarify the risks and benefits are required. [\hyperlink{Bupivacaine Hydrochloride}{PMID: 12060325}, Diana Moyao-García et al., 2002]

\hypertarget{pmid_10673893}{R}opivacaine is assumed to be less toxic than bupivacaine but there are no reports concerning its long-term use in paediatric anaesthesia. We report the use of ropivacaine for long-term epidural anaesthesia in a 21-month-old girl. In two consecutive periods of 3 days each, 0.5\% bupivacaine and 0.5\% or 0.75\% ropivacaine were administered to facilitate painful vaginal brachytherapy. The mean dose of bupivacaine increased from 1.05 to 1.32 mg kg-1 h-1 and that of ropivacaine increased from 1.40 to 3.86 mg kg-1 h-1. No toxic side effects were observed. We conclude that both epidural ropivacaine and bupivacaine were effective and safe during long-term epidural anaesthesia in this particular case. However, the doses were potentially toxic and should therefore be used with extreme caution. [\hyperlink{Bupivacaine Hydrochloride}{PMID: 10673893}, B Gustorff et al., 1999]

\hypertarget{pmid_8880822}{W}e studied the haemodynamic and cardiovascular effects of epidural anaesthesia with plain bupivacaine 0.75 ml.kg-1 in 13 unpremedicated ASA 1 children using measurements of heart rate, blood pressure and M-mode echocardiography. Under general anaesthesia, M-mode echocardiographic evaluation of left ventricular function in each patient was performed at four points (after general anaesthesia, point A; 5 min, 10 min and 25 min after epidural anaesthesia, point B; point C; and point D, respectively). Results were compared between point A and B, A and C, A and D, B and C, B and D, C and D. HR decreased significantly at 10 min (point C) and 25 min (point D) and MBP decreased at 5 min (point B) and 10 min (point D) compared to point A. No other M-mode cardiographic indices were changed at any point. Epidural anaesthesia using 0.25\% bupivacaine 0.75 ml.kg-1 did not affect LV function in young children. [\hyperlink{Bupivacaine Hydrochloride}{PMID: 8880822}, M H Tsuji et al., 1996]

\hypertarget{pmid_24691852}{T}he objective of this study is to compare the topical administration of bupivacaine hydrochloride, saline and bupivacaine hydrochloride infiltration on post-tonsillectomy pain in children. Sixty children undergoing tonsillectomy were enrolled in the study. Patients were randomized into three groups using sealed envelopes. Group 1 (n = 20) received topical 0.5 \% bupivacaine hydrochloride, group 2 (n = 20) received topical 0.9 \% NaCl (saline), and group 3 (n = 20) received 0.5 \% bupivacaine hydrochloride infiltrated around each tonsil. Pain was evaluated using McGrath's face scale. Pain scores in topical bupivacaine hydrochloride group was significantly lesser than the topical saline group at 5th, 13th, 17th and 21st hours, until the 6th day (p < 0.017). Moreover, pain scores of topical bupivacaine hydrochloride group was superior to bupivacaine hydrochloride infiltration group at 5th, 13th, 17th hours and 2nd, 3rd, 4th and 5th day (p < 0.017). There were significantly lesser morbidities in topical bupivacaine hydrochloride than saline group in 1st and 4th day (p < 0.017). Topical administration of bupivacaine hydrochloride proved to provide more efficient pain control than bupivacaine hydrochloride infiltration. [\hyperlink{Bupivacaine Hydrochloride}{PMID: 24691852}, Mehmet Haksever et al., 2014]

\hypertarget{pmid_28431423}{W}e evaluated blood bupivacaine concentrations in children having a single-shot sciatic and continuous femoral blocks after anterior cruciate ligament repair. Dried blood spot samples were analyzed for bupivacaine levels at 0, 5, 15, 30, 60, and 120 minutes and 4, 24, and 48 hours. The highest 99\% upper confidence interval limit was 135 ng/mL at the 4-hour evaluation point. The 99\% upper confidence interval was below potentially toxic levels (1500 ng/mL) across all sampling times. The risk of local anesthetic toxicity in pediatric patients receiving single-shot sciatic and continuous femoral nerve blocks is very low. [\hyperlink{Bupivacaine Hydrochloride}{PMID: 28431423}, Santhanam Suresh et al., 2017]

\hypertarget{pmid_3837258}{B}upivacaine induced caudal block as unique anaesthetic procedure in 80 children who underwent surgical correction of hypospadias, resulted of great efficacy and easy employment. Furthermore, the method permitted to avoid intubation and the use of general anaesthetic drugs with their relative complications and provided a great postoperative analgesia. [\hyperlink{Bupivacaine Hydrochloride}{PMID: 3837258}, M Heinen et al., ]

\hypertarget{pmid_3396545}{B}lood concentrations and pharmacokinetic parameters of bupivacaine were measured after epidural injection in children aged from 1 to 7 years. The children were allocated to two groups. In Group 1 (five children), pharmacokinetic parameters were measured after a single bolus injection of bupivacaine 0.25\% with adrenaline 1:200,000. In Group 2 (eight children), pharmacokinetic parameters were measured after the initial injection and the second injection. The same local anaesthetic was used. The volume of the second injection was half of the initial volume. In children of Group 1, maximum mean concentration (CPmax) was 0.64 +/- 0.05 microgram ml-1, time to maximum concentration (Tmax) 19.2 +/- 3.9 min, vascular absorption (T1/2 abs) 4.3 +/- 1.5 min, terminal half-life (T1/2 beta) 227 +/- 37.7 min, volume of distribution (Vd) 3.4 +/- 0.51 kg-1, and total body clearance (Clt) 11.0 +/- 2.0 ml min-1 kg-1. When compared to an adult's pharmacokinetic parameters, both Vd and Clt were increased, so that T1/2 beta remained essentially unchanged. In children of Group 2, the first repeat injection was performed at 110 +/- 6.9 min. Mean CPmax increased significantly (20\% after the second injection), whereas the values of the pharmacokinetic parameters measured did not differ significantly from those measured in children in Group 1. The results obtained in the present study demonstrate that the pharmacokinetic parameters of bupivacaine in children do not differ markedly from those reported in adults and that in the recommended dosage, the mean maximum concentrations, even after the second injection, are less than the presumed toxic levels. [\hyperlink{Bupivacaine Hydrochloride}{PMID: 3396545}, I Murat et al., 1988]

\hypertarget{pmid_7740909}{T}he authors discuss their experience with chloroprocaine for epidural anesthesia in five pediatric patients. While bupivacaine remains the most commonly used local anesthetic in children, recent reports of toxicity document the risks of this agent. The major advantage of chloroprocaine is its rapid metabolism, which thereby minimizes the risks of toxicity, especially in patients with preexisting problems such as young age or underlying hepatic dysfunction, which may limit the metabolism of local anesthetics of the amide class. In three cases, the epidural infusion was combined with the general anesthetic. The cases included hepatic resection, repair of bladder exstrophy, and correction of duodenal atresia. In two other cases, epidural anesthesia was used instead of general anesthesia in a former preterm infant who was undergoing inguinal herniorrhaphy and for lower extremity orthopedic procedures in a child with myotonic dystrophy. In all cases, chloroprocaine was chosen because of preexisting or associated conditions that might increase the risk of bupivacaine toxicity, such as hepatic resection, repeated dosing in a neonate, or the need for higher concentrations of local anesthetic to achieve adequate surgical conditions. Adequate intraoperative conditions were achieved in all five patients. No complications related to chloroprocaine epidural anesthesia were noted. This initial experience suggests that chloroprocaine offers an acceptable alternative to bupivacaine for epidural anesthesia in the pediatric population. [\hyperlink{Bupivacaine Hydrochloride}{PMID: 7740909}, J D Tobias et al., 1995]

\hypertarget{pmid_7102098}{T}he results of more than hundred caudal anaesthesias in surgical paediatric urology in children aged one to twelve years are reported. Modifying the method of Schulte-Steinberg, we used Bupivacain - CO2 0.5\%, and attained sufficient analgesia in all cases for whatever surgery applied. There were no complications except in one case where an overdose had been applied. We feel that the postoperative period is easier to bear because of the longer lasting local analgesia. Earlier onset of oral food intake is possible. [\hyperlink{Bupivacaine Hydrochloride}{PMID: 7102098}, U Hofmann et al., 1982]

\hypertarget{pmid_1773485}{T}he local anaesthetic bupivacaine could be very useful for analgesia in pediatric neurosurgery. Since systemic toxic reactions to bupivacaine are correlated with high plasma levels it was important, as an adjunct to clinical evaluation, to measure plasma bupivacaine. This report describes a high-performance liquid chromatography (HPLC) method for the quantitation of plasma bupivacaine. Sample preparation involves extraction into ether followed by back-extraction into HCl. After evaporation, the acid extract is redissolved and separated by reversed-phase chromatography. The assay is linear to 5 mg bupivacaine/L and shows excellent recovery and precision. With samples from children undergoing brain surgery following scalp infiltration with either 0.125\% or 0.25\% bupivacaine, plasma levels peak within 10 min, then fall rapidly to a plateau by 30 min. This plateau is maintained for at least 120 min. In no case did we find supposed toxic levels of bupivacaine. [\hyperlink{Bupivacaine Hydrochloride}{PMID: 1773485}, P St Louis et al., 1991]

\hypertarget{pmid_2265161}{P}revious studies have established the efficacy of spinal anesthesia in infants with hyperbaric lidocaine, hyperbaric tetracaine and isobaric and hyperbaric bupivacaine. Use of the commercially available hyperbaric spinal anesthesia solution of 0.75\% bupivacaine in 8.25\% dextrose has not heretofore been documented in this patient population. We present a series of 12 cases in which this hyperbaric bupivacaine solution with epinephrine was used for spinal anesthesia in children younger than 8 months of age. [\hyperlink{Bupivacaine Hydrochloride}{PMID: 2265161}, S K Parkinson et al., ]

\section*{Busulfan}
\subsection*{Result}
\subsubsection*{Answer}

Infants (<2 years): Unknown  
Children (2–12 years): Yes  
Adolescents (12–18 years): Yes  

\subsubsection*{{Explanation}}
\hypertarget{Busulfan}
To determine if Busulfan is safe for use in children, I reviewed the available abstracts for targeted studies on safety in pediatric populations, focusing on specific age ranges and whether the studies affirm safety.

**Infants and Young Children (0–2 years):**
- Several pharmacokinetic studies have been conducted in infants and young children, showing that Busulfan disposition is age-dependent, with higher clearance and larger volume of distribution in this group. These studies emphasize the need for individualized dosing and therapeutic drug monitoring due to high interpatient variability, but do not directly affirm safety or report on adverse events in a systematic way [\hyperlink{pmid_8338934}{PMID: 8338934}, G Vassal et al., 1993; \hyperlink{pmid_24029650}{PMID: 24029650}, Radojka M Savic et al., 2013; \hyperlink{pmid_2328321}{PMID: 2328321}, L B Grochow et al., 1990]. One study in rats showed significant toxicity when Busulfan was administered during pregnancy, but this is not directly translatable to human infants [\hyperlink{pmid_31706849}{PMID: 31706849}, Henrique J C B Gouveia et al., 2020]. Another rat study showed systemic histopathological changes in infant rats after Busulfan administration [\hyperlink{pmid_24791064}{PMID: 24791064}, Toko Ohira et al., 2014]. These animal studies suggest potential for harm, but human data are lacking on definitive safety or unsafety in this age group.

**Children (2–12 years):**
- Multiple studies have evaluated Busulfan as part of conditioning regimens for hematopoietic stem cell transplantation (HSCT) in children, including targeted safety assessments:
    - A prospective pilot study in 30 children aged 2–21 years (median 8) using high-dose Busulfan/melphalan reported that the regimen was "well-tolerated, with an acceptable transplant-related mortality" (6.6\%), and concluded that the regimen is feasible and safe in children [\hyperlink{pmid_10642802}{PMID: 10642802}, M A Diaz et al., 1999].
    - A study of 82 children (age not specified, but all pediatric) found no significant correlation between Busulfan exposure and toxic events or graft failure, suggesting that with monitoring, Busulfan can be used safely [\hyperlink{pmid_28801891}{PMID: 28801891}, Maura Faraci et al., 2018].
    - A meta-analysis of 13 studies involving 548 pediatric patients (aged 0.3–18 years) found that Busulfan exposure above certain thresholds increased risk of veno-occlusive disease (VOD), but that therapeutic drug monitoring and dose adjustment can mitigate this risk. The authors call for further well-designed trials but do not report Busulfan as unsafe [\hyperlink{pmid_32312247}{PMID: 32312247}, Xinying Feng et al., 2020].
    - A study of 954 pediatric transplant procedures found a low incidence of seizures (1.3\%) with Busulfan, most of which were associated with pre-existing risk factors, and concluded that the risk is low with prophylaxis [\hyperlink{pmid_24201160}{PMID: 24201160}, Désirée Caselli et al., 2014].
    - Several pharmacokinetic studies in children (including those aged 0.45–17.2 years) affirm that with individualized dosing and monitoring, Busulfan can be administered safely, with most patients achieving target exposures and no unexpected toxicities [\hyperlink{pmid_19049661}{PMID: 19049661}, L Nguyen et al., 2008; \hyperlink{pmid_17001185}{PMID: 17001185}, Juliette Zwaveling et al., 2006; \hyperlink{pmid_32140913}{PMID: 32140913}, Abdullah Alsultan et al., 2020].
    - A study comparing IV and oral Busulfan in children (0.7–13.1 years) found IV administration had less variability and fewer adverse effects, supporting its use in pediatric patients [\hyperlink{pmid_22742881}{PMID: 22742881}, G J Veal et al., 2012].
    - Another study in 3 pediatric patients (age not specified) in Taiwan found IV Busulfan had better compliance and fewer adverse effects than oral Busulfan [\hyperlink{pmid_15181963}{PMID: 15181963}, Ming-Yang Lee et al., 2004].

**Adolescents (12–18 years):**
- The above studies often include adolescents up to 17 or 18 years old, and the findings are consistent: with appropriate dosing and monitoring, Busulfan is feasible and generally safe for use in this age group as part of HSCT conditioning regimens [\hyperlink{pmid_10642802}{PMID: 10642802}, M A Diaz et al., 1999; \hyperlink{pmid_19049661}{PMID: 19049661}, L Nguyen et al., 2008; \hyperlink{pmid_32312247}{PMID: 32312247}, Xinying Feng et al., 2020].

**Summary:**
- For infants and very young children (<2 years), while pharmacokinetic studies exist, there is insufficient direct evidence from targeted safety studies affirming Busulfan's safety, and animal studies suggest potential for harm.
- For children (2–12 years) and adolescents (12–18 years), multiple targeted studies and meta-analyses affirm that Busulfan can be used safely as part of HSCT regimens when individualized dosing and therapeutic drug monitoring are employed. Risks such as VOD and seizures exist but are manageable and do not preclude its use.

\subsection*{Abstracts}
\hypertarget{pmid_32140913}{B}ackground Busulfan is an antineoplastic drug that is used widely as part of a conditioning regimen in pediatric patients undergoing hematopoietic stem cell transplantation. It has a narrow therapeutic index and highly variable pharmacokinetics; therefore therapeutic drug monitoring is recommended to optimize busulfan dosing. Objective To study the population pharmacokinetics of busulfan in Saudi pediatric patients to optimize its dosing. Settings King Abdullah Specialist Children's Hospital in Riyadh, Saudi Arabia. Methods This pharmacokinetic observational study was conducted between January 2016 and December 2018. All pediatric patients receiving IV busulfan and undergoing routine therapeutic drug monitoring were included. Population pharmacokinetics modeling was conducted using Monolix2019R1. Pharmacokinetic data of busulfan in children. Results The study included 59 patients and 513 samples. The mean ± SD age was 6.10 ± 3.17 years, and the dose administered was 0.994 ± 0.15 mg/kg. The mean ± SD Cmax and area under the curve (AUC) were 900.60 ± 402.8 ng/mL and 1031.14 ± 300.75 µM min, respectively. Based on our simulations, the European Medicines Agency recommended dose were adequate for most patient's groups to achieve the conventional target of an AUC [\hyperlink{Busulfan}{PMID: 32140913}, Abdullah Alsultan et al., 2020] Busulfan is widely used as a component of the myeloablative therapy in bone marrow transplantation. Recent studies have shown that the drug disposition is altered in children and is associated with less therapeutic effectiveness, lower toxicities, and higher rates of engraftment failure. We have evaluated the bioavailability of the drug in two groups of patients: eight children between 1.5 and 6 years of age and eight older children and adults between 13 and 60 years. Oral bioavailability showed a large interindividual variation. In children, the bioavailability ranged from 0.22 to 1.20, and for adults, it was within the range 0.47 to 1.03. The elimination half-life after intravenous administration in children (2.46 +/- 0.27 hours; mean +/- SD) did not differ from that obtained for adults (2.61 +/- 0.62 hours). However, busulfan clearance normalized to body weight was significantly higher in children (3.62 +/- 0.78 mL.min-1.kg-1) than that in adults (2.49 +/- 0.52 mL.min-1.kg-1). Also, the distribution volume normalized for body weight was significantly higher in children (0.74 +/- 0.10 L.kg-1) compared with 0.56 +/- 0.10 L. kg-1 in adults. The difference in clearance between children and adults was not statistically significant when normalized to body surface area, which most probably shows that busulfan dosage should be calculated on the basis of surface area rather than body weight. However, to avoid drug-related toxicities, drug monitoring and an individual dose adjustment should be considered because of the variability in busulfan bioavailability. [\hyperlink{Busulfan}{PMID: 32140913}, M Hassan et al., 1994]

\hypertarget{pmid_2328321}{C}hildren receive busulfan orally as part of myeloablative therapy before bone marrow transplantation for malignant and nonmalignant conditions. Children have been reported to have a low incidence of severe toxicity and significant rates of failure to achieve full engraftment. We evaluated the disposition of busulfan in children between 2 months and 3.6 years of age with lysosomal storage diseases, leukemia, and immunodeficiency disorders receiving oral doses of 1 or 2 mg/kg using a gas chromatographic assay. Peak concentrations were lower than those previously reported for adults, ranging from 1.4 to 5.2 mumol/L. The harmonic mean of the elimination half-life was 92 minutes, which is only slightly faster than that for adults (140 minutes). However, the area under the curve ranged from 400 to 1,000 (715 +/- 240) mumol.min/L, substantially lower than in adults receiving 1 mg/kg (range, 710 to 5,100 mumol.min/L; mean +/- SD, 2,180 +/- 1,200). The apparent volume of distribution (assuming complete bioavailability) ranged from 0.28 to 3.53 L/kg (1.42 +/- 0.86), which is more than twice that reported for adults (0.60 +/- 0.42). Busulfan clearance rate normalized to surface area is twice as high in children (200 +/- 100 mL/min/m2) as it is in adults (95 +/- 54 mL/min/m2). Alterations in bioavailability (absorption or first pass elimination) or in actual volume of distribution may account for these differences in drug disposition. The observed differences suggest the need for separate phase I dose escalation studies in children with accompanying pharmacokinetic assessment. [\hyperlink{Busulfan}{PMID: 2328321}, L B Grochow et al., 1990]

\hypertarget{pmid_28801891}{T}he aim of this report is to describe the experience in the management of busulphan-based conditioning regimen administered before hematopoietic stem cell transplantation (HSCT) in children. We report the values of the first dose AUC (area under the concentration-time curve, normal target between 3600 and 4800 ng·h/mL) in children treated with oral and intravenous busulphan, and we analyze the impact of some clinical variables in this cohort of patients. 82 children treated with busulphan before HSCT were eligible for the study: 57 received oral busulphan with a mean AUC of 3586 ng·h/mL, while 25 received intravenous busulphan with a mean AUC of 4158 ng·h/mL. Dose adjustment was based on first dose AUC. The dose was increased in 36 children (43.9\%) and decreased in 26 patients (31.7\%). Age at HSCT (P = 0.015), cumulative dose of busulphan as mg/m We concluded that older age at HSCT, intravenous administration of busulphan, cumulative, and prescribed dose of busulphan are associated with higher AUC levels. The absence of significant correlations between toxic events, graft failure, and AUC suggests the efficacy of busulphan concentrations monitoring in our patients. [\hyperlink{Busulfan}{PMID: 28801891}, Maura Faraci et al., 2018]

\hypertarget{pmid_19049661}{B}usulfan (Bu) is commonly used in preparative conditioning regimen prior to bone marrow transplantation in infants (< 1 year old), children and adolescents (up to 17 years old). The clinical development of an intravenous form of busulfan (Busilvex) was based on pharmacokinetic (PK) modeling and simulation techniques. A retrospective population PK analysis was initially performed from a first study in 24 pediatric patients (0.45-16.7 years old) and a log-linear relationship between body weight and Busilvex clearance was demonstrated with no age-dependency. For an optimal area under the curve (AUC) targeting, a new Bu dosing regimen [i.e. 5 dose levels (0.80 to 1.20 mg/kg) adjusted to 5 discrete weight categories] was developed and assessed through population PK-based simulations. The benefit from this new dosing strategy was validated in a second trial including 55 children (0.30-17.2 years old). This prospective trial confirmed the previous simulations: an efficient therapeutic targeting whatever the patient's age or body weight. Over 80\% of the children were within the desired plasma exposure window, and the initial PK model was validated on the confirmatory dataset. [\hyperlink{Busulfan}{PMID: 19049661}, L Nguyen et al., 2008]

\hypertarget{pmid_31706849}{B}usulfan is a bifunctional alkylating agent used for myeloablative conditioning and in the treatment of chronic myeloid leukemia due to its ability to cause DNA damage. However, in rodent experiments, busulfan presented a potential teratogenic and cytotoxic effect. Studies have evaluated the effects of busulfan on fetuses after administration in pregnancy or directly on pups during the lactation period. There are no studies on the effects of busulfan administration during pregnancy on offspring development after birth. We investigated the effects of busulfan on somatic and reflex development and encephalic morphology in young rats after exposure in pregnancy. The pregnant rats were exposed to busulfan (10 mg/kg, intraperitoneal) during the early developmental stage (days 12-14 of the gestational period). After birth, we evaluated the somatic growth, maturation of physical features and reflex-ontogeny during the lactation period. We also assessed the effects of busulfan on encephalic weight and cortical morphometry at 28 days of postnatal life. As a result, busulfan-induced pathological changes included: microcephaly, evaluated by the reduction of cranial axes, delay in reflex maturation and physical features, as well as a decrease in the morphometric parameters of somatosensory and motor cortex. Thus, these results suggest that the administration of a DNA alkylating agent, such as busulfan, during the gestational period can cause damage to the central nervous system in the pups throughout their postnatal development. [\hyperlink{Busulfan}{PMID: 31706849}, Henrique J C B Gouveia et al., 2020]

\hypertarget{pmid_10642802}{W}e conducted a prospective pilot study to assess the feasibility and safety of high-dose busulfan/melphalan as conditioning therapy prior to autologous PBPC transplantation in pediatric patients with high-risk solid tumors. From January 1995 to January 1999, 30 patients aged 2-21 years (median 8) were entered into the study. There were 14 females and 16 males. Diagnoses included neuroblastoma in 10 patients; Ewing's sarcoma and peripheral neuroectodermal tumor (PNET) in 15 patients and rhabdomyosarcoma in five patients. Treatment consisted of busulfan 16 mg/kg, orally over 4 days (from days -5 to -2) in 6 hourly divided doses, and melphalan at a dose of 140 mg/m2 given by intravenous infusion over 5 min on day -1. G-CSF mobilized PBPC were used as autologous stem-cell rescue. One patient developed a single generalized convulsion during busulfan therapy. The most relevant non-hematologic toxicity was gastrointestinal, manifesting as grade 2-3 mucositis and diarrhea in 12 patients. Two patients died of procedure-related complications, one from veno-occlusive disease of liver and multiorgan failure and the other from adult respiratory distress syndrome. Probability of treatment-related mortality was 6.6 +/- 4.5\%. With a median follow-up of 18 months (range, 1-48), 19 patients are alive and disease-free, the actuarial EFS at 4 years being 55 +/- 12\% for the whole group. We conclude that high-dose busulfan/melphalan for autologous transplantation in children with solid tumors is feasible even in small patients. It is well-tolerated, with an acceptable transplant-related mortality and has proven antitumor activity. [\hyperlink{Busulfan}{PMID: 10642802}, M A Diaz et al., 1999]

\hypertarget{pmid_22742881}{B}usulfan is widely used in a neuroblastoma setting, with several studies reporting marked inter-patient variability in busulfan pharmacokinetics and pharmacodynamics. The current study reports on the pharmacokinetics of oral versus intravenous (IV) busulfan in high-risk neuroblastoma patients treated on the European HR-NBL-1/SIOPEN study. Busulfan was administered four times daily for 4 days to children aged 0.7-13.1 years, either orally (1.45-1.55 mg/kg) or by the IV route (0.8-1.2mg/kg according to body weight strata). Blood samples were obtained prior to administration, 2, 4, and 6h after the start of administration on dose 1. Busulfan analysis was carried out by gas chromatography-mass spectrometry and data analysed using a NONMEM population pharmacokinetic approach. Busulfan plasma concentrations obtained from 38 patients receiving IV busulfan and 25 patients receiving oral busulfan, were fitted simultaneously using a one-compartment pharmacokinetic model. Lower variability in drug exposure was observed following IV administration, with a mean busulfan area under the plasma concentration versus time curve (AUC) of 1146 ± 187 μM.min (range 838-1622), as compared to 953 ± 290 μM.min (range 434-1427) following oral busulfan. A total of 87\% of children treated with IV busulfan achieved AUC values within the target of 900-1500 μM.min versus 56\% of patients following oral busulfan. Busulfan AUC values were significantly higher in HR-NBL-1/SIOPEN trial patients who experienced hepatic toxicity or veno-occlusive disease (VOD) (1177 ± 189 μM.min versus 913 ± 256 μM.min; p=0.0086). Further stratification based on route of administration suggested that the incidence of hepatic toxicity was related to both high busulfan AUC and oral drug administration. The reduced pharmacokinetic variability and improved control of busulfan AUC observed following IV administration support its utility within the ongoing HR-NBL-1/SIOPEN trial. [\hyperlink{Busulfan}{PMID: 22742881}, G J Veal et al., 2012]

\hypertarget{pmid_11966670}{I}ntravenous formulations of busulfan have recently become available. Although busulfan is used frequently in children as part of a myeloablative regimen prior to bone marrow transplantation, pharmacokinetic data on intravenous busulfan in children are scarce. The aim was to investigate intravenous busulfan pharmacokinetics in children and to suggest a limited sampling strategy in order to determine busulfan systemic exposure with the minimum of inconvenience and risk for the patient. Plasma pharmacokinetics after the first administration was investigated in six children using nonlinear mixed effect modelling. Pharmacokinetics showed little variability and were described adequately with a one-compartment model (population estimates CL,av=0.29 l h(-1) kg(-1); V,av=0.84 l kg(-1); t(1/2)=1.7-2.8 h). Combined with limited sampling and a Bayesian fitting procedure, the model can adequately estimate the systemic exposure to intravenous busulfan, which in children appears to be at the lower end of the adult range. Busulfan systemic exposure in children during intravenous administration can be estimated adequately with limited sampling and a Bayesian fitting procedure from a one-compartment model. Intravenous busulfan pharmacokinetics in children should be the subject of more research. [\hyperlink{Busulfan}{PMID: 11966670}, Serge Cremers et al., 2002]

\hypertarget{pmid_24791064}{B}usulfan is an antineoplastic bifunctional alkylating agent. We previously reported the busulfan-induced systemic histopathological changes in fetal rats and the sequence of brain lesions in fetal and infant rats. In the present study, in order to clarify the nature and sequence of busulfan-induced systemic histopathological changes in infant rats, 6-day-old male infant rats were subcutaneously administered 20 mg/kg of busulfan and histopathologically examined at 1, 2, 4, 7 and 14 days after treatment (DAT). As a result, histopathological changes characterized by pyknosis of component cells were observed in the heart, lungs, stomach, intestines, liver, kidneys, testes, epididymides, hematopoietic and lymphoid tissues, dorsal skin and femur as well as in the brain and eyes (data not shown in this paper). Such pyknosis transiently appeared until 7 DAT with prominence at 2 and/or 4 DAT in each tissue, except for the thymus, in which pyknosis peaked at 1 DAT. Most of the pyknotic nuclei were immunohistochemically positive for cleaved caspase-3, indicating that pyknotic cells were apoptotic. Different from the reports of fetal and adult rats, apoptosis was also found in cardiomyocytes and osteoblasts in infant rats.  [\hyperlink{Busulfan}{PMID: 24791064}, Toko Ohira et al., 2014] Busulphan levels in plasma were measured in 27 patients during conditioning therapy (1 mg/kg x 4 for 4 days) before bone marrow transplantation. The mean minimal concentration found in children aged less than 5 years (237 ng ml-1) was lower than that observed in adults or older children (607 and 573 ng ml-1, respectively). The AUC for the last dose was significantly lower in young children (2.315 h ng ml-1) than in adults or older children (6,134 and 5,937 h ng ml-1, respectively). The elimination half-life for the last dose in young children was shorter (2.05 h) than that in either adults (2.59 h) or older children (2.79 h). When the AUC was normalized for body surface area, the difference between young children and the other groups was smaller but remained statistically significant. The total body clearance was significantly higher in young children (7.3 ml min-1 kg-1) as compared with both older children and adults (3.02 and 2.7 ml min-1 kg-1, respectively). The plasma levels of busulphan showed circadian rhythmicity, especially in young children. The concentration measured during the night in some patients was up to 3-fold that observed during daytime. We conclude that the busulphan dosage for children must be reconsidered and that further studies are urgently needed to develop an optimal therapy. [\hyperlink{Busulfan}{PMID: 24791064}, M Hassan et al., 1991]

\hypertarget{pmid_7803883}{T}o review the current published studies evaluating the pharmacokinetics, clinical efficacy, safety, and toxicity of busulfan in pediatric and adult patients. English-language literature published between 1953 and 1993 was analyzed; pertinent literature was reviewed. Emphasis was placed on pharmacologic studies and clinical trials involving busulfan therapy both in myeloproliferative disorders and in conditioning regimens for autologous or allogeneic bone marrow transplantation. Data from both pediatric and adult studies were evaluated; emphasis was placed on the relationship between plasma concentrations of busulfan and its efficacy and toxicity. Busulfan has been used widely at conventional dosages (1-12 mg/d) for the treatment of patients with chronic myelogenous leukemia (CML). Busulfan at high doses (usually 16 mg/kg) given with other cytotoxic drugs (especially cyclophosphamide) is a common preparative regimen in patients undergoing allogeneic or autologous bone marrow transplantation (BMT) for acute or chronic leukemia and other nonmalignant disorders (e.g., hemoglobinopathies, inborn error of immune system, congenital metabolic disorders). Pharmacokinetics of high-dose busulfan are age-dependent. Busulfan systemic exposure and, thus, tissue and tumor exposure are lower in children than with adults. Relationships between toxicity (principally neutropenia, hepatic veno-occlusive disease, incidence of seizures) and drug exposure were found for busulfan. Busulfan is a useful, sufficiently safe drug in the treatment of patients with CML. At higher dosages, busulfan is a fundamental part of myeloablative therapies for patients undergoing BMT. As the pharmacokinetics and metabolism of busulfan is further understood, there is great potential for improving treatment outcome. An assessment of maximal tolerated exposure determined by therapeutic drug monitoring may decrease the incidence and lethality of regimen-related toxicities. [\hyperlink{Busulfan}{PMID: 7803883}, I Buggia et al., 1994]

\hypertarget{pmid_24201160}{B}usulphan (BU) is associated with neurotoxicity and risk of seizures. Hence, seizure prophylaxis is routinely utilized during BU administration for stem cell transplantation (SCT). We collected data on the incidence of seizures among children undergoing SCT in Italy. Fourteen pediatric transplantation centers agreed to report unselected data on children receiving BU as part of the conditioning regimen for SCT between 2005 and 2012. Data on 954 pediatric transplantation procedures were collected; of them, 66\% of the patients received BU orally, and the remaining 34\%, i.v. All the patients received prophylaxis of seizures, according to local protocols, consisting of different schedules and drugs. A total of 13 patients (1.3\%) developed seizures; of them, 3 had a history of epilepsy (or other seizure-related pre-existing condition); 3 had documented brain lesions potentially causing seizures per se; 1 had febrile seizures, 1 severe hypo-osmolality. In the remaining 5 patients, seizures were considered not explained and, thus, potentially related to BU administration. The incidence of seizures in children receiving BU-containing regimen was very low (1.3\%); furthermore, most of them had at least 1-either pre-existing or concurrent-associated risk factor for seizures.  [\hyperlink{Busulfan}{PMID: 24201160}, Désirée Caselli et al., 2014] Busulfan disposition is age-dependent with a higher clearance and a larger volume of distribution in children than in adults. The optimal dosage of busulfan needed to achieve bone marrow (BM) displacement in young children with malignant or nonmalignant disease remains to be defined. Using a gas chromatography-mass spectrometry assay, we evaluated plasma pharmacokinetics of busulfan in 33 children (median age, 9 months; range, 2 months to 2.75 years) with immune deficiencies, lysosomal storage diseases, acute leukemias, and malignant lymphohistiocytosis after an oral dose ranging from 0.9 to 2.6 mg/kg. The busulfan clearance (assuming a bioavailability of 1) ranged from 2.1 to 13.4 mL/min/kg with a mean of 6.8 mL/min/kg, which is higher than that reported in older children (4.5 mL/min/kg) and adults (2.9 mL/min/kg). Six children with lysosomal storage disease (5 with Hurler's disease, 1 with San Filippo's disease) had a prolonged elimination half-life (4.9 v 2.4 hours), a larger volume of distribution (3.4 v 1.2 L/kg) and a faster clearance (8.7 v 6.3 mL/min/kg) than the other 27 children. This suggests that a higher dose of busulfan will be required to achieve BM displacement in children with lysosomal storage disease. Over the dose range of 0.9 to 2.6 mg/kg, busulfan pharmacokinetics were linear. However, only 46\% of the interpatient variation in systemic exposure could be ascribed to the dose. Given the wide interpatient variability in busulfan disposition, dose adjustment and drug monitoring will be needed to achieve the optimal dosage of busulfan in young children. The plasma busulfan levels required to achieve BM displacement need to be defined, especially in lysosomal storage diseases. [\hyperlink{Busulfan}{PMID: 24201160}, G Vassal et al., 1993]

\hypertarget{pmid_17001185}{W}e studied the pharmacokinetics and clinical outcome of a new once-daily intravenous area under the curve-targeted dosing scheme for busulfan based on body surface area. Eighteen children undergoing busulfan-based conditioning for allogeneic stem cell transplantation were enrolled. The age of the children ranged from 0.5 to 16 years. For all children, the starting dose was 80 mg/m. Unlimited dose adjustment was allowed to reach the target area under the curve (3800 micromol/l . min). This target area under the curve was determined on the basis of a previous study in our hospital. Pharmacokinetic studies were performed after the first dose. The median area under the curve on day 1 was 2616 (range 1781-5040) micromol/l . min at a dose of 80 mg/m. This resulted in a median dose increment to 114 (range 62-168) mg/m to reach the target area under the curve. In only one patient, the dose was decreased. Donor engraftment was established in 14 out of 18 patients (78\%). Two of the four patients were successfully retransplanted. Relapse occurred in two patients (one died, one received additional treatment). Fourteen patients survived with a median follow-up of 1.6 years (1.0-2.2 years). The disease-free survival was 66\% (12 of 18 patients). Despite the high systemic peak levels, there was no new unexpected or unusual toxicity. Moderate veno-occlusive disease was seen in one patient only. We conclude that intravenous busulfan in children administered once daily is safe, convenient and feasible, and can be dosed surface-based, independent of age. There was very limited (liver) toxicity, but the rejection rate was relative high, which can be probably overcome by a higher exposure to busulfan. Future investigations should be aimed at further optimizing the target area under the curve of intravenous busulfan for specific patient groups. [\hyperlink{Busulfan}{PMID: 17001185}, Juliette Zwaveling et al., 2006]

\hypertarget{pmid_20677921}{H}igh-dose Busulfan in combination chemotherapy has been used commonly for hematopoietic stem cell transplantation. It crosses the blood-brain barrier and could cause seizure. Benzodiazepines have been used as anticonvulsant prophylaxis. This is a prospective study using oral lorazepam together with busulfan-based conditioning regimen in 30 children undergoing hematopoietic stem cell transplantation. The dose of lorazepam used ranged from 0.017 to 0.039 mg/kg (median = 0.026 mg/kg) per dose. None of the patients developed seizure while receiving oral lorazepam or within 72 hours of the last dose of Busulfan. Oral lorazepam was tolerated by the patients, but all patients needed dose reduction due to some adverse effects. In the authors' experience, oral lorazepam is a useful anticonvulsant prophylaxis for children receiving high-dose busulfan. [\hyperlink{Busulfan}{PMID: 20677921}, Amir Ali Hamidieh et al., 2010]

\hypertarget{pmid_24029650}{L}ittle information is currently available regarding the pharmacokinetics (PK) of busulfan in infants and small children to help guide decisions for safe and efficacious drug therapy. The objective of this study was to develop an algorithm for individualized dosing of i.v. busulfan in infants and children weighing ≤12 kg, that would achieve targeted exposure with the first dose of busulfan. Population PK modeling was conducted using intensive time-concentration data collected through the routine therapeutic drug monitoring of busulfan in 149 patients from 8 centers. Busulfan PK was well described by a 1-compartment base model with linear elimination. The important clinical covariates affecting busulfan PK were actual body weight and age. Based on our model, the predicted clearance of busulfan increases approximately 1.7-fold between 6 weeks to 2 years of life. For infants age <5 months, the model-predicted doses (mg/kg) required to achieve a therapeutic concentration at steady state of 600-900 ng/mL (area under the curve range, 900-1350 μM·min) were much lower compared with standard busulfan doses of 1.1 mg/kg. These results could help guide clinicians and inform better dosing decisions for busulfan in young infants and small children undergoing hematopoietic cell transplantation.  [\hyperlink{Busulfan}{PMID: 24029650}, Radojka M Savic et al., 2013] Busulphan (1, 4-bis [methanesulfonyl-y] butane) is a bi-functional alkylating agent that, in combination with cyclophosphamide, has been commonly used in conditioning regimens before hematological stem cell transplantation (HSCT) for nearly 20 years. Busulfan has a very narrow therapeutic index, and acute toxicity may be related to absorption and disposition of the drug and metabolites. Precise delivery of the oral formulation is compromised by erratic gastrointestinal absorption, particularly in infants and small children. An intravenous busulfan formula was approved nearly 40 years after the approval of the oral formulation. Busulfan levels expressed as the area under the concentration-time curve (AUC) higher than 1500 microM* minute were reported to increase the risk of developing veno-occlusive disease (VOD), while low levels may result in engraftment failure or disease relapse. VOD occurs in 11-40\% of patients undergoing HSCT and is associated with death in 3.3\% of patients. Measurement of individual plasma busulfan levels during oral or intravenous dosing to obtain an AUC is likely to provide the necessary elements to monitor the drug disposition, ensuring efficacy and preventing toxicity of patients undergoing HSCT. It is also important to consider the busulfan drug-drug interactions and adverse drug reactions that can develop during the therapeutic process. Busulfan therapeutic drug monitoring and dose-adjustment should be performed in specialized laboratories staffed by well-trained personnel. [\hyperlink{Busulfan}{PMID: 24029650}, Norberto Krivoy et al., 2008]

\hypertarget{pmid_32312247}{B}usulfan (Bu) is a key component of several conditioning regimens used before hematopoietic stem cell transplantation (HSCT). However, the optimum systemic exposure (expressed as the area under the concentration-time curve [AUC]) of Bu for clinical outcome in children is controversial. Research on pertinent literature was carried out at PubMed, EMBASE, Web of science, the Cochrane Library and ClinicalTrials.gov. Observational studies were included, which compared clinical outcomes above and below the area under the concentration-time curve (AUC) cut-off value, which we set as 800, 900, 1000, 1125, 1350, and 1500 μM × min. The primary efficacy outcome was notable in the rate of graft failure. In the safety outcomes, incidents of veno-occlusive disease (VOD) were recorded, as well as other adverse events. Thirteen studies involving 548 pediatric patients (aged 0.3-18 years) were included. Pooled results showed that, compared with the mean Bu AUC (i.e., the average value of AUC measured multiple times for each patient) of > 900 μM × min, the mean AUC value of < 900 μM × min significantly increased the incidence of graft failure (RR = 3.666, 95\% CI: 1.419, 9.467). The incidence of VOD was significantly decreased with the mean AUC < 1350 μM × min (RR = 0.370, 95\% CI: 0.205-0.666) and < 1500 μM × min (RR = 0.409, 95\% CI: 0182-0.920). In children, Bu mean AUC above the cut-off value of 900 μM × min (after every 6-h dosing) was associated with decreased rates of graft failure, while the cut-off value of 1350 μM × min were associated with increased risk of VOD, particularly for the patients without VOD prophylaxis therapy. Further well-designed prospective and multi centric randomized controlled trials with larger sample size are necessary before putting our result into clinical practices. [\hyperlink{Busulfan}{PMID: 32312247}, Xinying Feng et al., 2020]

\hypertarget{pmid_15181963}{S}ome studies have proved that intravenous busulfan with cyclophosphamide (used as a component of conditioning regimens for hematopoietic stem cell transplantation) is safer and has fewer complication than oral busulfan in adults, whereas the same proof in pediatric patients is only limited, with no reported data so far from Asian countries. In this study, we aimed to evaluate the efficacy and complications of IV busulfan in pediatric patients. Three pediatric patients with acute myeloid leukemia were treated by intravenous busulfan combined with cyclophosphamide to compare retrospectively with the treatment with oral busulfan plus intravenous cyclophosphamide in another three pediatric cases having transplantation in the same institute. The results showed that the intravenous busulfan-based regimen had better compliance and less adverse effects including mucositis, hepatic toxicity, transplant-related hepatic veno-occlusive disease, and acute graft-versus-host disease than oral busulfan-based treatment. The conditioning regimen of intravenous busulfan combined with cyclophosphamide is an acceptable alternative for pediatric patients with hematological malignancies in Taiwan. The long-term benefit and adverse effects of intravenous busulfan should be further explored. [\hyperlink{Busulfan}{PMID: 15181963}, Ming-Yang Lee et al., 2004]

\hypertarget{pmid_24174393}{B}usulfan (Bu) is a DNA-alkylating agent used for myeloablative conditioning in stem cell transplantation in children and adults. While the use of intravenous rather than oral administration of Bu has reduced inter-individual variability in plasma levels, toxicity still occurs frequently after hematopoietic stem cell transplantation (HSCT). Toxicity (especially hepatotoxic effects) of intravenous (IV) Bu may be related to both Bu and/or N,N-dimethylacetamide (DMA), the solvent of Bu. In this study, we assessed the relation between the exposure of Bu and DMA with regards to the clinical outcome in children from two cohorts. In a two-centre study Bu and DMA AUC (area under the curve) were correlated in pediatric stem cell recipients to the risk of developing SOS and to the clinical outcome. In patients receiving Bu four times per day Bu levels >1,500 µmol/L minute correlate to an increased risk of developing a SOS. In the collective cohort, summarizing data of all 53 patients of this study, neither high area under the curve (AUC) of Bu nor high AUC of DMA appears to be an independent risk factor for the development of SOS in children. In this study neither Bu nor DMA was observed as an independent risk factor for the development of SOS. To identify subgroups (e.g., infants), in which Bu or DMA might be risk factors for the induction of SOS, larger cohorts have to be evaluated. [\hyperlink{Busulfan}{PMID: 24174393}, Kornelius Kerl et al., 2014]

\hypertarget{pmid_9334898}{B}uspirone is a nonbenzodiazepine anxiolytic that has been effective in uncontrolled trials for treating childhood anxiety disorders. A 4-year-old boy with a history of laryngomalacia (congenital structural abnormality with airway collapse and obstruction on inhalation), pharyngeal dysphagia (difficulty in swallowing), poor weight gain, delayed self-feeding skills, and anxiety symptoms is described. An open trial of buspirone, increased gradually to 12.5 mg daily in divided doses over a period of 22 weeks, was associated with decreased anxiety, improved self-feeding skills, and weight gain. Based on parental reports, buspirone appeared to decrease separation and social anxiety, as well as anxiety associated with eating. Drug discontinuation was associated with symptom relapse, whereas drug readministration lead to the same clinical benefits that had been observed previously. The medication was well tolerated, and its benefits have persisted for over 1 year. No new recommendations can be made regarding the use of buspirone in preschool children or in the treatment of anxious behaviors adversely affecting medical conditions in children and adolescents. [\hyperlink{Busulfan}{PMID: 9334898}, G L Hanna et al., 1997]

\hypertarget{pmid_22455797}{T}he wide variability in pharmacokinetics of busulfan in children is one factor influencing outcomes such as toxicity and event-free survival. A meta-analysis was conducted to describe the pharmacokinetics of busulfan in patients from 0.1 to 26 years of age, elucidate patient characteristics that explain the variability in exposure between patients and optimize dosing accordingly. Data were collected from 245 consecutive patients (from 3 to 100 kg) who underwent haematopoietic stem cell transplantation (HSCT) in four participating centres. The inter-patient, inter-occasion and residual variability in the pharmacokinetics of busulfan were estimated with a population analysis using the nonlinear mixed-effects modelling software NONMEM VI. Covariates were selected on the basis of their known or theoretical relationships with busulfan pharmacokinetics and were plotted independently against the individual pharmacokinetic parameters and the weighted residuals of the model without covariates to visualize relations. Potential covariates were formally tested in the model. In a two-compartment model, body weight was the most predictive covariate for clearance, volume of distribution and inter-compartmental clearance and explained 65\%, 75\% and 40\% of the observed variability, respectively. The relationship between body weight and clearance was characterized best using an allometric equation with a scaling exponent that changed with body weight from 1.2 in neonates to 0.55 in young adults. This implies that an increase in body weight in neonates results in a larger increase in busulfan clearance than an increase in body weight in older children or adults. Clearance on the first day was 12\% higher than that of subsequent days (p < 0.001). Inter-occasion variability on clearance was 15\% between the 4 days. Based on the final pharmacokinetic-model, an individualized dosing nomogram was developed. The model-based individual dosing nomogram is expected to result in predictive busulfan exposures in patients ranging between 3 and 65 kg and thereby to a safer and more effective conditioning regimen for HSCT in children. [\hyperlink{Busulfan}{PMID: 22455797}, Imke H Bartelink et al., 2012]

\hypertarget{pmid_10570028}{T}he relationship between age and busulfan apparent oral clearance (Cl/F) expressed relative to adjusted ideal body weight and body surface area (bsa) was evaluated in 135 children aged 0 to 16 years undergoing hematopoietic stem cell transplantation for various disorders. Busulfan plasma levels were measured by gas chromatography-mass spectrometry after the first daily dose of the 4-day dosing regimen. Cl/F expressed relative to adjusted ideal body weight (ml/min/kg) and bsa (ml/min/m(2)) was lower in 9- to 16-year-old (y.o.) compared with 0- to 4-y.o. children (49 and 30\%; p<.001). We hypothesized that the greater busulfan Cl/F observed in young children was in part due to enhanced (first-pass intestinal) metabolism. Busulfan conjugation rate was compared in incubations with human small intestinal biopsy specimens from healthy young (1- to 3-y.o.) and older (9- to 17-y.o.) children. Villin content in biopsy specimens was determined by Western blot and busulfan conjugation rate was expressed relative to villin content to control for differences in epithelial cell content in pinch biopsies. Intestinal biopsy specimens from young children had a 77\% higher busulfan conjugation rate (p =.037) compared with older children. We have previously shown that glutathione-S-transferase (GST) A1-1 is the major isoform involved in busulfan conjugation, and that this enzyme is expressed uniformly along the length of adult small intestine. Thus, the greater busulfan conjugation activity in intestinal biopsies of the young children was most likely due to enhanced GSTA1-1 expression. We conclude that age dependence in busulfan Cl/F appears to result at least in part from enhanced intestinal GSTA1-1 expression in young children. [\hyperlink{Busulfan}{PMID: 10570028}, J P Gibbs et al., 1999]

\hypertarget{pmid_8932835}{B}usulphan pharmacokinetics were investigated in 20 children, who underwent bone marrow transplantation for either leukemia or inherited disorders. Busulphan (1.90-6.02 mg/kg/day) was administered orally as a single dose or twice daily. Busulphan kinetics were found to be linear within the studied range. Children with inherited disorders eliminated busulphan significantly faster after the first and the last dose with half-lives (t1/2) of 1.93 and 1.71 h, respectively compared to children with leukemia (3.16 and 2.70 h, respectively). The area under plasma concentration curves (AUCs, corrected for mg/kg) as an expression for the systemic exposure of busulphan were significantly higher in children with leukemia, 22.4 and 19.04 mumol/l.h (5527 and 4690 ng.h.ml-1) after the first and the last dose, respectively, compared to 11.2 and 8.2 mumol/l.h (2768 and 2029 ng.h.ml-1) found in children with inherited disorders. The present results confirm those reported by others, ie busulphan pharmacokinetics can be influenced by the underlying disease and its status. Our population pharmacokinetic analysis showed a negative correlation between the weight corrected clearance and the age in both groups of children. However, clearance was about 42\% higher in children with inherited disorders compared to those with leukemia. To estimate AUC for the first dose, we evaluated a limited sampling model based on three concentrations (1, 3 and 6 h). A high correlation (r = 0.998, P < 0.0001, n = 40) between the estimated and the determined AUC was found. The present model is reliable and adequate for studying more patients, with a long-term follow-up combined with drug monitoring in correlation with drug efficacy and toxicity to define the optimal busulphan dosage required. [\hyperlink{Busulfan}{PMID: 8932835}, M Hassan et al., 1996]

\section*{Butorphanol Tartrate}
\subsection*{Result}
\subsubsection*{Answer}

Yes (Ages 6 weeks to 17 years)
Unknown (Neonates, under 6 weeks)

\subsubsection*{{Explanation}}
\hypertarget{Butorphanol Tartrate}
Based on the available abstracts, several targeted studies have evaluated the safety of Butorphanol Tartrate in children across different age ranges:

1. **Infants and Young Children (Ages 1 to 4 years):**
   - A study of 103 patients aged 1 to 4 years undergoing various surgeries assessed the use of butorphanol tartrate (moradol) as part of combined anesthesia. The study monitored hemodynamic parameters, stress hormones, and metabolic markers, and found no serious side effects. The authors concluded that butorphanol provided effective analgesia with minimal impact on adrenal and thyroid function, and no serious adverse events were observed [\hyperlink{pmid_9045578}{PMID: 9045578}, T S Agzamkhodzhaev et al.].

2. **Infants and Children (Ages 6 weeks to 7 years):**
   - A prospective, randomized, double-blind study in 60 postoperative pediatric patients aged 6 weeks to 7 years compared epidural and intravenous butorphanol for side effects and efficacy. The study found that butorphanol was effective for analgesia, with sedation being the most common side effect, but no significant increase in serious adverse events [\hyperlink{pmid_7639376}{PMID: 7639376}, A G Bailey et al., 1994].

3. **Children (Ages 6 months and older):**
   - A double-blind, placebo-controlled study of 60 children aged 6 months or older undergoing bilateral myringotomy and tube placement found that transnasal butorphanol at 25 mcg/kg provided effective postoperative pain relief with no significant safety concerns reported [\hyperlink{pmid_9710397}{PMID: 9710397}, R E Bennie et al., 1998].

4. **Children (Ages 1.5 to 13 years):**
   - A randomized, double-blind study of 156 children aged 1.5 to 13 years compared butorphanol and morphine for postoperative pain. Both drugs were effective, and butorphanol was associated with less vomiting. No serious adverse events were reported, and the study concluded that butorphanol had few advantages over morphine in this population [\hyperlink{pmid_7628027}{PMID: 7628027}, W M Splinter et al., 1995].

5. **Children (Ages 8 to 17 years):**
   - A report of eight patients aged 8 to 17 years receiving transnasal butorphanol for postoperative pain found adequate analgesia with only mild, transient side effects (nausea, dizziness, bitter taste, mild throat irritation), none of which precluded further use [\hyperlink{pmid_8521312}{PMID: 8521312}, J D Tobias et al., 1995].

6. **Children (Postoperative, age not specified but implied to be pediatric):**
   - A retrospective study compared epidural butorphanol/bupivacaine with fentanyl/bupivacaine in 191 children after urological procedures. Butorphanol provided similar analgesia to fentanyl with fewer side effects such as pruritus and no cases of clinically significant respiratory depression in the butorphanol group [\hyperlink{pmid_21269000}{PMID: 21269000}, Alexandra Szabova et al.].

7. **Adolescents:**
   - A pilot study in 27 adolescents with postoperative orthopedic pain found that butorphanol provided good or excellent pain relief in 89\% of patients, with only urinary retention as a notable side effect (possibly related to other medications or surgery). Safety and tolerance were rated as good or excellent in all patients [\hyperlink{pmid_3079008}{PMID: 3079008}, N L Steg et al., 1988].

In summary, multiple targeted studies in children from 6 weeks to 17 years have evaluated the safety of Butorphanol Tartrate for analgesia in various clinical settings. These studies consistently report effective pain relief with no serious or unexpected adverse events, and side effects (such as sedation, nausea, or mild throat irritation) were generally mild and manageable. No studies in these abstracts reported evidence of harm or unsafe use in children.

There is no evidence from these abstracts regarding the safety of Butorphanol Tartrate in neonates (under 6 weeks of age), so safety in this age group remains unknown.

\subsection*{Abstracts}
\hypertarget{pmid_9045578}{T}he purpose of this study was to assess the efficacy of moradol (butorphanol tartrate) as an analgetic component of combined total anesthesia in children. The adequacy of anesthesia was assessed by echography, electrocardiography, and electroencephalography, measurements of stress hormones, electrolyte balance, and metabolic parameters at various stages of anesthesia and surgery. A total of 103 patients aged 1 to 4 years were examined, subjected to abdominal, thoracal, urological, orthopaedic, and ENT surgery. The stability of the basic hemodynamic parameters and some stress hormones in the blood during surgery indicated effective protection of the organism from surgical trauma, on the one hand, and evidenced the minimal effect of anesthesia on the adrenal and thyroid function, on the other. A single injection of moradol during induction narcosis was sufficient for 2.5 to 3 hours of surgery. No serious side effects of the drug were observed. Prolonged (10 hours on average) anesthesia persisted after the operation, this decreasing the postoperative use of analgesics. [\hyperlink{Butorphanol Tartrate}{PMID: 9045578}, T S Agzamkhodzhaev et al., ]

\hypertarget{pmid_8458045}{B}utorphanol tartrate, a synthetically derived opioid agonist-antagonist analgesic, was tested in a large group of postpartum women (N = 76) to assess the safety and analgesic efficacy of a recently approved transnasal preparation of this drug in the relief of postepisiotomy pain. The safety and efficacy of intravenous and intramuscular administration of butorphanol tartrate has been established over 14 years of clinical use. The new nasal spray dosage form offers a similar degree of efficacy with a rapid onset of action. Compared with the injectables and other drugs in this class, transnasal butorphanol has a longer duration of action (4 to 5 hours). In this double-blind, parallel-group, dose-response study, 76 female patients ages 17 to 37 years with moderate to severe postepisiotomy pain were randomly assigned to receive a single dose of transnasal butorphanol (0.25, 0.5, 1, or 2 mg) or placebo. The patients were evaluated for 6 hours. The results of the study indicate that the 1-mg and 2-mg doses were associated with greater efficacy compared with placebo using several markers for efficacy, including the pain relief score and time to remedication. The drug was well tolerated, dizziness and drowsiness being the most frequently reported adverse effects. Adverse effects appeared to be dose related. [\hyperlink{Butorphanol Tartrate}{PMID: 8458045}, T H Joyce et al., ]

\hypertarget{pmid_9710397}{M}ore than 70\% of children require analgesics after bilateral myringotomy and tube placement (BMT). Because anesthesia for BMT is generally provided by face mask without placement of an intravenous catheter, an alternative route for analgesia administration is needed. Transnasal butorphanol is effective in relieving postoperative pain in adults and children. The effectiveness of transnasal butorphanol for postoperative pain management in children undergoing BMT was studied. This double-blinded, placebo-controlled study compared the postoperative analgesic effects of transnasal butorphanol administered after the induction of anesthesia. Sixty children classified as American Society of Anesthesiologists physical status 1 or 2 who were aged 6 months or older and scheduled for elective BMT were randomized to receive transnasal placebo or 5, 15, or 25 microg/kg butorphanol. Postoperative pain was assessed using the Children's Hospital of Eastern Ontario Pain Scale (CHEOPS) on arrival in the postanesthesia care unit and at 5, 10, 15, 30, 45, and 60 min. The CHEOP scores were significantly less in the 25 microg/kg transnasal butorphanol group compared with controls. Significantly fewer children received rescue analgesia in the 25 microg/kg transnasal butorphanol group compared with controls (n = 1 and 8, respectively; P = 0.02). Transnasal butorphanol given in a dose of 25 microg/kg after induction of anesthesia provided adequate postoperative pain relief in children undergoing BMT. [\hyperlink{Butorphanol Tartrate}{PMID: 9710397}, R E Bennie et al., 1998]

\hypertarget{pmid_8521312}{T}he authors present their experience with transnasal butorphanol to provide analgesia following orthopaedic and plastic surgical procedures in children. Transnasal butorphanol was administered to eight patients ranging in age from eight to 17 years and in weight from 34 to 64 kg. Following the surgical procedure, the patient and a parent were instructed on how to use the medication. They were instructed to administer one spray into one nostril every three h as needed for pain. The quality of analgesia was assessed twice a day using a visual analogue score of 0 to 10 (0 = no pain, 10 = worst pain imaginable). Intranasal butorphanol provided adequate analgesia in all eight patients with visual analogue scores of zero to two. Adverse effects from the medication included one report of nausea, one complaint of transient dizziness, and two reports of a bitter taste and some mild throat irritation. None of these was severe enough to preclude its subsequent use. Our preliminary experience suggests that transnasal butorphanol may offer an alternative route of delivery when intravenous administration is not feasible. Future studies are needed to compare its efficacy to intravenous and non-parenteral routes of administration. It may prove to be useful in other situations when intravenous access is lacking such as prior to invasive procedures in the outpatient clinic or emergency room. [\hyperlink{Butorphanol Tartrate}{PMID: 8521312}, J D Tobias et al., 1995]

\hypertarget{pmid_21269000}{T}he aim of this retrospective study is to compare safety and efficacy of postoperative epidural butorphanol/bupivacaine with the gold-standard epidural analgesic infusion fentanyl/bupivacaine in children. With the Institutional Review Board's approval, the authors searched their Pain Management Database and divided children who received epidural analgesia into two groups. Each butorphanol group subject was matched with two fentanyl group subjects. Demographic data, pain scores, epidural interventions, epidural side effects, use of rescue opioid analgesia and adjuvant analgesics, causes of epidural failure, time of first oral intake and ambulation, and length of stay were statistically compared. A total of 191 patients were identified between 2000 and 2007; 58 in epidural butorphanol/bupivacaine and 133 in fentanyl/bupivacaine groups. Demographic data were comparable between the groups. The number of children with good pain control on postoperative days 1 and 2 in butorphanol (84 and 82 percent) and fentanyl (93 and 91 percent) groups were statistically similar (p = 0.06 and 0.13, respectively). Incidences of epidural side effects, especially pruritus, were significantly higher in the fentanyl group. Significantly more children in the butorphanol group required epidural rate changes when compared with those in the fentanyl group. Incidence of failed epidurals was significantly higher in the fentanyl group when compared with that in the butorphanol group. Clinically significant respiratory depression occurred in two children in the fentanyl group and in none of the children in the butorphanol group (p > 0.99). Epidural butorphanol provided similar analgesia to epidural fentanyl after urological procedures in children, but butorphanol caused less pruritus than fentanyl. Epidural analgesia with butorphanol/bupivacaine is effective in children undergoing urological procedures. When compared with epidural fentanyl, epidural butorphanol causes significantly less itching. [\hyperlink{Butorphanol Tartrate}{PMID: 21269000}, Alexandra Szabova et al., ]

\hypertarget{pmid_24370240}{T}o evaluate antinociceptive effects and pharmacokinetics of butorphanol tartrate after IM administration to American kestrels (Falco sparverius). Fifteen 2- to 3-year-old American kestrels (6 males and 9 females). Butorphanol (1, 3, and 6 mg/kg) and saline (0.9\% NaCl) solution were administered IM to birds in a crossover experimental design. Agitation-sedation scores and foot withdrawal response to a thermal stimulus were determined 30 to 60 minutes before (baseline) and 0.5, 1.5, 3, and 6 hours after treatment. For the pharmacokinetic analysis, butorphanol (6 mg/kg, IM) was administered in the pectoral muscles of each of 12 birds. In male kestrels, butorphanol did not significantly increase thermal thresholds for foot withdrawal, compared with results for saline solution administration. However, at 1.5 hours after administration of 6 mg of butorphanol/kg, the thermal threshold was significantly decreased, compared with the baseline value. Foot withdrawal threshold for female kestrels after butorphanol administration did not differ significantly from that after saline solution administration. However, compared with the baseline value, withdrawal threshold was significantly increased for 1 mg/kg at 0.5 and 6 hours, 3 mg/kg at 6 hours, and 6 mg/kg at 3 hours. There were no significant differences in mean sedation-agitation scores, except for males at 1.5 hours after administration of 6 mg/kg. Butorphanol did not cause thermal antinociception suggestive of analgesia in American kestrels. Sex-dependent responses were identified. Further studies are needed to evaluate the analgesic effects of butorphanol in raptors. [\hyperlink{Butorphanol Tartrate}{PMID: 24370240}, David Sanchez-Migallon Guzman et al., 2014]

\hypertarget{pmid_2782703}{T}he effects of butorphanol tartrate on arterial pressure, jejunal blood flow, vascular resistance, oxygen extraction, and oxygen uptake were determined in 10 anesthetized ponies ventilated with a mixture of halothane and 100\% oxygen, using isolated autoperfused jejunal segments. Physiologic saline solution or butorphanol tartrate (0.2 mg/kg of body weight) was administered as a single bolus into the left jugular vein. By 2 minutes, butorphanol decreased arterial blood pressure and intestinal blood flow, and increased intestinal oxygen extraction. However, intestinal vascular resistance and oxygen uptake were unaffected. Results of this study indicate that butorphanol tartrate induces a hypotension that secondarily decreases intestinal blood flow, but intestinal vascular resistance and metabolism are not adversely affected. We conclude that butorphanol tartrate does not compromise intestinal viability in halothane-anesthetized ponies and, therefore, may be a good analgesic choice for the equid destined for abdominal surgery. [\hyperlink{Butorphanol Tartrate}{PMID: 2782703}, J A Stick et al., 1989]

\hypertarget{pmid_359109}{B}utorphanol tartrate 1 mg and 2 mg were compared in 80 normal mothers at term in a double-blind study with meperidine hydrochloride 40 mg and 80 mg for the relief of pain in labour. Butorphanol was found to be as effective as meperidine in relieving pain in labour. The foetal condition, as measured by ECG monitoring, Apgar scores, time to sustained respiration, umbilical venous H+ (pH) and PCO2, and a general nursery survey were comparable for meperidine and butorphanol. No psychomimetic phenomena were seen. Assays indicated that both butorphanol and meperidine crossed the placenta. The mean concentration of butorphanol in neonatal serum was 0.84 times maternal serum at 1.5 to 3.5 hours after intramuscular administration of a single or two successive doses of butorphanol 1 mg or 2 mg to the mother. The mean concentrations for meperidine in neonatal serum was 0.89 times maternal serum at 0.85 to 3.6 hours after intramuscular administration of meperidine 40 mg or 80 mg to the mother. Neither analgesic caused severe depression of the infant except for one meperidine-treated case. [\hyperlink{Butorphanol Tartrate}{PMID: 359109}, A L Maduska et al., 1978]

\hypertarget{pmid_2830756}{B}utorphanol tartrate is a highly effective opioid agonist-antagonist analgesic with qualitative as well as quantitative differences from the pure agonists. These differences are thought to be due to interaction with a distinct subset of opioid receptors. Although it relieves severe pain, the drug does not usually elevate mood, and it may occasionally cause dysphoria. Counterbalancing its disadvantages is a wealth of clinical experience with the drug showing an impressive record of safety. Butorphanol produces limited respiratory depression and smooth muscle spasm, and both effects are reversible with naloxone. The most prominent side effect is sedation, a property that is generally quite useful in the perioperative period. Butorphanol is a weak morphine antagonist, so it may interact with agonists like morphine or fentanyl. The chief advantages of this agent are its low toxicity and very low potential for abuse. [\hyperlink{Butorphanol Tartrate}{PMID: 2830756}, C E Rosow et al., 1988]

\hypertarget{pmid_15613167}{B}utorphanol tartrate is a synthetic mixed agonist-antagonist opioid analgesic. Its transnasal dosage form, which may be self-administered when the use of an opioid analgesic is appropriate, was previously shown to provide rapid relief of migraine pain. In this double-blind, parallel-group, outpatient study, we compared butorphanol nasal spray 1 mg followed in 1 hour by an optional second 1-mg dose with the orally administered analgesic, Fiorinal with Codeine (one capsule containing butalbital 50 mg, caffeine 40 mg, aspirin 325 mg, and codeine phosphate 30 mg). Patients (N=321) were assigned by randomization to one of two treatment groups (butorphanol or Fiorinal with Codeine) and instructed to self-administer medication when migraine pain reached an intensity of moderate or severe and to record study-related events in a diary for 24 hours posttreatment. Efficacy analyses were performed on data from 275 patients who took study medication and returned a patient diary; 136 in the butorphanol group and 139 in the Fiorinal with Codeine group. During the first 2 hours after treatment, butorphanol was more effective than Fiorinal with Codeine in treating migraine pain as measured by pain intensity difference scores, percentage of responders (pain decreased to mild or none), percentage of pain-free patients, and degree of pain relief, with a more rapid time to onset of 15 minutes. A similar percentage of patients in the two groups used rescue medication during the first 4 hours, after which more butorphanol-treated than Fiorinal with Codeine-treated patients used rescue medication. Butorphanol patients had more side effects, less improvement in digestive symptoms, and less improvement in functional ability than Fiorinal with Codeine patients. [\hyperlink{Butorphanol Tartrate}{PMID: 15613167}, J Goldstein et al., ]

\hypertarget{pmid_8330463}{T}he safety, tolerance, and pharmacokinetics of transnasal butorphanol were evaluated in a double-blind, multiple-dose phase I study. A total of 18 subjects received either placebo (n = 6) or a single transnasal dose of 2 mg butorphanol tartrate on the first day and 1, 2, and 4 mg doses of butorphanol tartrate every 6 hours on days 2 through 6, 7 through 11, and 12 through 16, respectively. Safety assessment was performed on days 7, 12, and 17. Serial blood samples were collected on days 1, 6, 11, and 16, and the plasma was analyzed for unchanged butorphanol by a validated and specific radioimmunoassay. Butorphanol was rapidly absorbed and peak levels in plasma were generally attained within 1 hour after the nasal administration. The values of maximum concentration, minimum concentration, and area under the concentration versus time curve from time zero to the dosing interval [AUC(0-tau)] increased as the administered dose increased in a dose-proportional manner. The values of AUC from time zero to infinity after a single dose of 2 mg butorphanol tartrate, 10.9 ng.hr/ml, were identical to the values of AUC(0-tau) after a multiple administration of 2 mg dose, 10.4 ng.hr/ml. Mean elimination half-life value was 5.45 hours. Steady state was reached in fewer than eight doses when given every 6 hours. Transnasal butorphanol was well tolerated by all subjects. After repeated administration of transnasal butorphanol, no significant changes were observed in the nasal examination, which included evaluation of color, wetness, and thickness of nostril membrane, air flow, airway patency, and general nasal conditions.(ABSTRACT TRUNCATED AT 250 WORDS) [\hyperlink{Butorphanol Tartrate}{PMID: 8330463}, W C Shyu et al., 1993] A pilot study was conducted to evaluate the use of butorphanol, administered intramuscularly in 0.7-mg to 3-mg doses, in 27 adolescents with postoperative orthopedic pain. Butorphanol provided good or excellent pain relief in 24 (89\%) patients. The duration of the relief was about three to four hours. The only adverse effect experienced in more than one patient was urinary retention, possibly associated with the use of fentanyl, which was administered for balanced anesthesia, and/or with the surgical procedure (spinal fusion). Tolerance and safety were rated as good or excellent in 100\% of the patients. [\hyperlink{Butorphanol Tartrate}{PMID: 8330463}, N L Steg et al., 1988]

\hypertarget{pmid_7639376}{W}e performed a prospective, randomized, double-blinded study in 60 postoperative pediatric patients aged 6 wk to 7 yr to compare the efficacy of butorphanol given epidurally or intravenously in preventing the side effects of epidural morphine. Three groups of patients received 60 micrograms/kg epidural morphine; 20 patients also received epidural butorphanol 30 micrograms/kg, and 20 patients also received 30 micrograms/kg intravenous butorphanol. All patients were evaluated for analgesia, sedation, vomiting, urinary retention, pruritus, and respiratory depression for 24 h postoperatively. Although the overall incidence of side effects was not different in the three groups, the epidural butorphanol group had a significant decrease in severity of pruritus. Sedation was seen more frequently in the groups receiving butorphanol, but was most pronounced in the epidural butorphanol group. We conclude that butorphanol has little or no effect on the side effects of epidural morphine. [\hyperlink{Butorphanol Tartrate}{PMID: 7639376}, A G Bailey et al., 1994]

\hypertarget{pmid_3344597}{T}he analgesic efficacy and safety of butorphanol tartrate are discussed in 2 groups of patients who underwent urological procedures. The first group of 83 patients is presented as a retrospective review of the postoperative use of butorphanol. The second group of patients was involved in a double-blind, randomized comparative trial of butorphanol (2 or 4 mg) and meperidine (80 mg) for the relief of moderate to severe pain due to renal colic. Eighty-three patients with documented upper urinary tract calculi were evaluated for efficacy; 120 patients were evaluated for safety. Butorphanol 4 mg (i.m.) was more effective than butorphanol 2 mg (i.m.) and equivalent to meperidine 80 mg (i.m.). There were no statistically significant differences among the three treatment groups in regard to side effects. Overall, in the urology patients studied, butorphanol was found to be an effective and well tolerated agent that possesses important safety advantages when compared with the narcotic analgesics. [\hyperlink{Butorphanol Tartrate}{PMID: 3344597}, H H Henry et al., 1988]

\hypertarget{pmid_15334604}{B}utorphanol (17-cyclobutylmethyl-3,14-dihydroxymorphinan) tartrate (Stadol) is a mixed agonist-antagonist opioid analgesic agent that is about five to seven times as potent as morphine in analgesic effects. The chronic use of butorphanol produces physical dependence in humans and animals. Phosphorylation plays a very important role in developing butorphanol dependence; however, global phosphorylation events induced by chronic butorphanol administration have not been reported. The aim of this study is to determine the alteration of tyrosine phosphorylation of brain frontal cortical proteins in butorphanol-dependent rats using a proteomic approach. Dependence was produced by continuous intracerebroventricular (i.c.v.) infusion of butorphanol (26 nmol/microl/hr) for 72 hr via osmotic minipump in rats. Similar patterns of protein expression were detected by two-dimensional electrophoresis (2-DE) in brain frontal cortex of butorphanol-dependent and saline-treated control rats. All 65 phosphotyrosyl (p-Tyr) protein spots detected in pH 3-10 phosphotyrosine 2-DE of control rat brains were detected in butorphanol-dependent rat brains. The densities of most p-Tyr protein spots were increased in butorphanol-dependent rat brains compared to saline-treated control samples. Eighteen additional p-Tyr protein spots were detected in pH 3-10 2-DE images of butorphanol-dependent rat brains. Immobilized pH strips with three different narrow pH ranges were examined to improve the resolution of p-Tyr proteins in 2-DE gels. Fifty-three p-Tyr protein spots were identified as known proteins involved in cell cytoskeleton, cell metabolism, and cell signaling. This proteomic approach can provide useful information for understanding the complex mechanism of butorphanol dependence in vivo. [\hyperlink{Butorphanol Tartrate}{PMID: 15334604}, Seong-Youl Kim et al., 2004]

\hypertarget{pmid_10791933}{T}o evaluate effects of butorphanol tartrate and buprenorphine hydrochloride on withdrawal threshold to a noxious stimulus in conscious African grey parrots. 29 African grey parrots (Psittacus erithacus erithacus and Psittacus erithacus timneh). Birds were fitted with an electrode on the medial metatarsal region of the right leg, placed into a test box, and allowed to acclimate. An electrical stimulus (range, 0.0 to 1.46 mA) was delivered to each bird's foot through an aluminum perch. A withdrawal response was recorded when the bird lifted its foot from the perch or vigorously flinched its wings. Baseline threshold to a noxious electrical stimulus was determined. Birds then were randomly assigned to receive an i.m. injection of saline (0.9\% NaCl) solution, butorphanol (1.0 mg/kg of body weight), or buprenorphine (0.1 mg/kg), and threshold values were determined again. Butorphanol significantly increased threshold value, but saline solution or buprenorphine did not significantly change threshold values. Butorphanol had an analgesic effect, significantly increasing the threshold to electrical stimuli in African grey parrots. Buprenorphine at the dosage used did not change the threshold to electrical stimulus. Butorphanol provided an analgesic response in half of the birds tested. Butorphanol would be expected to provide analgesia to African grey parrots in a clinical setting. [\hyperlink{Butorphanol Tartrate}{PMID: 10791933}, J R Paul-Murphy et al., 1999]

\hypertarget{pmid_7886096}{I}n the present series of studies we examined the effect of butorphanol tartrate on food-reinforced operant responding in satiated rats. In the first experiment, 8.0 mg/kg butorphanol was administered subcutaneously, once per day for 4 days, to satiated rats responding under an fixed ratio 10 (FR 10) reinforcement schedule. In the second experiment, butorphanol (0, 0.3, 1.0, 3.0, 10.0 mg/kg) was administered to satiated rats responding under an FR 80 (first pellet) FR 3 (subsequent pellets) reinforcement schedule for 4 consecutive days. Repeated butorphanol administration increased total amount of food consumed over sessions in both experiments. Under the FR 80 schedule component, butorphanol initially increased latency to acquire the first pellet, an effect attenuated by repeated administration. Whereas vehicle administration was associated with consumption of relatively large quantities of food within the first 10 min of receiving the first pellet, butorphanol was associated with continued feeding as the session progressed. These data suggest that butorphanol-induced food intake is associated with maintenance rather than initiation of feeding. [\hyperlink{Butorphanol Tartrate}{PMID: 7886096}, J M Rudski et al., 1994]

\hypertarget{pmid_359382}{B}utorphanol tartrate, a synthetic morphinan which has analgesic agonist and antagonist properties, was compared with meperidine for balanced anaesthesia. The two agents were found to be comparable in efficacy, maintenance of cardiovascular stability, and incidence of side-effects. Butorphanol has the advantage of being non-narcotic and having a lower propensity for addiction. Because of its antagonist properties, there appears to be a limit to its depressant effects on respiration. [\hyperlink{Butorphanol Tartrate}{PMID: 359382}, L C Stehling et al., 1978]

\hypertarget{pmid_7628027}{T}he purpose of this study was to compare the side effects and efficacy of equianalgesic doses of morphine (M) and butorphanol (B) in children undergoing similar surgical procedures associated with moderate postoperative pain. We studied 156 healthy children aged 1.5-13 yr who underwent elective inguinal herniorrhaphy or orchidopexy. After induction of anaesthesia subjects were given 150 micrograms.kg-1 M or 30 micrograms.kg-1 B following a randomized, stratified, blocked and double-blind design. A standardized anaesthetic was administered, which included 1.5\% halothane, vecuronium, droperidol and mechanical ventilation. The postsurgical four-hour follow-up included assessment of pain, vomiting and respiratory depression. Pain was assessed with mCHEOPS and analgesics were administered when indicated in the recovery room. Each opioid was administered to a group of 78 patients. Within each group, 25 subjects had an iv induction, 21 children had an orchidopexy and 57 had inguinal hernia repairs. The groups were similar with respect to age, weight, and length of surgery. The choice of opioid did not affect recovery times from anaesthesia. Analgesic requirements were similar among the groups. Ten minutes after arrival in the recovery room the B-subjects had a lower pain score than the M-patients. Postoperative vomiting was less among the B-subjects: 14\% vs 28\%, P = 0.03. Two M-patients required an unscheduled admission to hospital because of vomiting. It is concluded that butorphanol has few advantages over morphine in the population studied. [\hyperlink{Butorphanol Tartrate}{PMID: 7628027}, W M Splinter et al., 1995]

\hypertarget{pmid_34853785}{A}sthma is the most common chronic disease in children, many of whom are managed solely with a short-acting β The aim of this study is to determine the efficacy and safety of as-needed budesonide-formoterol therapy compared with as-needed salbutamol in children aged 5 to 15 years with mild asthma, who only use a SABA. A 52-week, open-label, parallel group, phase III RCT will recruit 380 children aged 5 to 15 years with mild asthma. Participants will be randomised 1:1 to either budesonide-formoterol (Symbicort Rapihaler This is the first RCT to assess the safety and efficacy of as-needed budesonide-formoterol in children with mild asthma. The results will provide a much-needed evidence base for the treatment of mild asthma in children. [\hyperlink{Butorphanol Tartrate}{PMID: 34853785}, Lee Hatter et al., 2021]

\hypertarget{pmid_9153436}{B}utorphanol (Stadol, Bristol-Meyers Squibb, Princeton, NJ) is a synthetically derived opiate. As a nasal spray, it was approved for release in 1991 and was subsequently promoted as a safe treatment for migraine. Since then, there have been numerous reports of problems with butorphanol similar to those of any narcotic, especially dependence-addiction and major psychological disturbances. These problems have been documented by the Food and Drug Administration, but the information can be obtained only through the Freedom of Information Act. The experience with butorphanol indicates the need for physicians to have additional sources of information about drugs than are presently available. [\hyperlink{Butorphanol Tartrate}{PMID: 9153436}, M A Fisher et al., 1997]

\hypertarget{pmid_24592110}{P}arenteral opioids can be administered with ease at a very low cost with high efficacy as labour analgesia. However, there are insufficient data available to accept the benefits of parenteral opioids over other proven methods of labour analgesia. Butorphanol, a new synthetic opioid, has emerged as a promising agent in terms of efficacy and a better safety profile. This study investigates the effect of butorphanol as a labour analgesia to gather further evidence of its safety and efficacy to pave the way for its widespread use in low resource settings. One hundred low risk term consenting pregnant women were recruited to take part in a prospective cohort study. Intramuscular injections of butorphanol tartrate 1 mg (Butrum 1/2mg, Aristo, Mumbai, India) were given in the active phase of labour and repeated two hourly. Pain relief was noted on a 10-point visual pain analogue scale (VPAS). Obstetric and neonatal outcome measures were mode of delivery, duration of labour, Apgar scores at 1 and 5 minutes and Neonatal Intensive Care Unit admissions. Collected data were analysed for statistically significant pain relief between pre- and post-administration VPAS scores and also for the incidence of adverse outcomes. Pain started to decrease significantly within 15 minutes of administration and reached the nadir (3.08 SD0.51) at the end of two hours. The pain remained below four on the VPAS until the end of six hours and was still significantly low after eight hours. The incidence of adverse outcomes was low in the present study. Butorphanol is an effective parenteral opioid analgesic which can be administered with reasonable safety for the mother and the neonate. The study has the drawback of lack of control and small sample size. [\hyperlink{Butorphanol Tartrate}{PMID: 24592110}, Ajay Halder et al., 2013]

\hypertarget{pmid_23126609}{P}ain management is an important component of foal nursing care, and no objective data currently exist regarding the analgesic efficacy of opioids in foals. To evaluate the somatic antinociceptive effects of 2 commonly used doses of intravenous (i.v.) butorphanol in healthy foals. Our hypothesis was that thermal nociceptive threshold would increase following i.v. butorphanol in a dose-dependent manner in both neonatal and older pony foals. Seven healthy neonatal pony foals (age 1-2 weeks), and 11 healthy older pony foals (age 4-8 weeks). Five foals were used during both age periods. Treatments, which included saline (0.5 ml), butorphanol (0.05 mg/kg bwt) and butorphanol (0.1 mg/kg bwt), were administered i.v. in a randomised crossover design with at least 2 days between treatments. Response variables included thermal nociceptive threshold, skin temperature and behaviour score. Data within each age period were analysed using a 2-way repeated measures ANOVA, followed by a Holm-Sidak multiple comparison procedure if warranted. There was a significant (P<0.05) increase in thermal threshold, relative to Time 0, following butorphanol (0.1 mg/kg bwt) administration in both age groups. No significant time or treatment effects were apparent for skin temperature. Significant time, but not treatment, effects were evident for behaviour score in both age groups. Butorphanol (0.1 mg/kg bwt, but not 0.05 mg/kg bwt) significantly increased thermal nociceptive threshold in neonatal and older foals without apparent adverse behavioural effects. Butorphanol shows analgesic potential in foals for management of somatic painful conditions. [\hyperlink{Butorphanol Tartrate}{PMID: 23126609}, K T McGowan et al., 2013]

\hypertarget{pmid_9470973}{B}utorphanol is an opioid used as analgesic in humans and other species. In horses, it can cause locomotor stimulation at low doses. This drug is not well chromatographed by GC and so, it is necessary to transform it into a more suitable compound, which can be done by derivatization. The derivatization of a drug is used to impart volatility, masking polar groups to improve the results in gas chromatographic analysis. We have evaluated N,O-bis(trimethylsilyl)- trifluoracetamide (BSTFA)+ 1\% trimethylchlorsilane (TMCS) and N-methyl-N-trimethylsilil-trifluoroacetamide (MSTFA) as derivatizing reagents for butorphanol at 30, 60 and 80 degrees C during 15, 30 and 60 min. The effects of dilution of these reagents with toluene and the evaporation before the derivatization were tested. Both reagents can be used for butorphanol derivatization and analysis and the dilution and evaporation steps did not alter the final results. The best derivatization conditions were 15 min at 80 degrees C, although 60 degrees C, although 60 degrees C during 60 min were also suitable. [\hyperlink{Butorphanol Tartrate}{PMID: 9470973}, M H Andraus et al., ]

\hypertarget{pmid_22918011}{B}utorphanol tartrate, a mixed synthetic agonistantagonist opioid analgesic has been used for management of postoperative pain in minor and major surgical procedures.(14,20) Tramadol hydrochloride is a centrally acting opioid which is effectively used in postoperative pain in various minor and major surgeries. Twenty subjects selected randomly received butorphanol tartrate 1 mg intramuscular and 20 subjects received tramadol hydrochloride 50 mg intramuscular after the removal of mandibular third molars. Time of injection, amount of anesthetic injected, duration of surgery, adverse effects were recorded.(21) The mean amount of LA administered in butorphanol group was 2.6450 ml and in tramadol group was 2.640 ml respectively, the mean duration for surgery was 56.75 and 53.5 minutes for butorphanol and tramadol groups respectively which was statistically not significant. Pain assessment was done with VAS which showed mean of 19.2 and 15.5 mm (p = 0.001) which was significant for butorphanol and tramadol respectively after 12 hours. The mean time for rescue medication requirement was 5.9 hours (for tramadol) and 8.4 hours (for butorphanol). Effective analgesic activity was seen by butorphanol 1 mg intramuscular then tramadol 50 mg. Butorphanol 1 mg was more effective than tramadol 50 mg in respect to postoperative analgesia. [\hyperlink{Butorphanol Tartrate}{PMID: 22918011}, Syed Sirajul Hassan et al., 2012]

\section*{Technetium Tc-99M Sestamibi Kit}
\subsection*{Result}
\subsubsection*{Answer}

Unknown

\subsubsection*{{Explanation}}
\hypertarget{Technetium Tc-99M Sestamibi Kit}
To determine if Technetium Tc-99M Sestamibi Kit is safe for use in children, I reviewed all available abstracts for evidence of targeted safety studies in pediatric populations.

Relevant abstracts:

1. \hyperlink{pmid_8523231}{PMID: 8523231} (Martinez et al., 1995): This abstract describes the use of Technetium 99m-sestamibi in three children with hyperparathyroidism during surgical exploration. The radiopharmaceutical was used in conjunction with a gamma-detecting probe to assist in identifying abnormal parathyroid tissue. However, the abstract does not mention any assessment of safety, adverse events, or a systematic evaluation of safety in these children.

2. \hyperlink{pmid_7628151}{PMID: 7628151} (Aktolun et al., 1995): This report describes the use of Tc-99m sestamibi in a 2-year-old girl with bilateral retinoblastoma to detect distant metastases. The findings focus on the diagnostic effectiveness of the scan, with no mention of safety outcomes or adverse effects.

3. \hyperlink{pmid_12411545}{PMID: 12411545} (Kirton et al., 2002): This pilot study involved 20 children with CNS malignancies who underwent Tc-99m MIBI (sestamibi) SPECT imaging. The study explores the clinical usefulness of the tracer in tumor detection and monitoring but does not report on safety, adverse events, or a targeted safety assessment.

4. \hyperlink{pmid_2148347}{PMID: 2148347} (O'Tuama et al., 1990): This case report describes a 5-year-old girl with a brain stem astrocytoma who underwent Tc-99m Hexamibi (sestamibi) SPECT imaging. The focus is on tumor imaging, with no mention of safety or adverse effects.

Summary:
- Several abstracts report the use of Technetium Tc-99M Sestamibi Kit in children, including those as young as 2 years old.
- None of the abstracts present a targeted safety study or systematic safety assessment in pediatric populations.
- No adverse events or safety concerns are reported, but the absence of safety data does not equate to evidence of safety.
- No age-specific safety conclusions can be drawn for any pediatric age range based on the available abstracts.

Conclusion:
Based on the abstracts reviewed, there is no definitive evidence from targeted safety studies affirming that Technetium Tc-99M Sestamibi Kit is safe or unsafe for use in children of any age range. Therefore, the safety of Technetium Tc-99M Sestamibi Kit in children is unknown.

\subsection*{Abstracts}
\hypertarget{pmid_1834815}{T}echnetium-99m-2-methoxy isobutyl isonitrile (99mTc-sestamibi) is a radiopharmaceutical that can be useful in the evaluation of patients with acute myocardial infarction. The current method for preparation requires a lengthy boiling water bath procedure and the recommended quality control procedure is cumbersome and time-consuming. Using a microwave oven, the heating time necessary to provide a labeling efficiency (averaging 97\% for sestamibi-labeled with maximum allowable 99mTc activity and volume has been reduced to 10 sec. A new mini-paper chromatography (MPC) system has been developed to analyze the radiochemical purity of 99mTc-sestamibi involving a 1:1 chloroform/tetrahydrofuran developing solvent. The recommended thin-layer chromatography (TLC) system involving the use of an AI2O3-coated plate requires an average time for drying and development of 34.8 +/- 1.6 min (n = 58) to complete, whereas the new MPC system has an average developing time of 2.3 +/- 0.1 min (n = 26). For radiochemical purity values ranging from 71-99\% (n = 31), the MPC and TLC methods correlated closely (r = 0.99) with a regression line of MPC\% = 1.05 TLC\%--5.75. The combined use of the microwave oven heating method and our quick quality control system will facilitate the rapid, emergency use of 99mTc-sestamibi and eliminate the need for advance preparation of multiple kits each day. [\hyperlink{Technetium Tc-99M Sestamibi Kit}{PMID: 1834815}, J C Hung et al., 1991]

\hypertarget{pmid_1835138}{T}echnetium 99m sestamibi is a promising new radiopharmaceutical that can assess myocardium at risk, infarct size, and treatment efficacy in acute myocardial infarction. The minimal redistribution of this radiopharmaceutical makes it ideal for the measurement of myocardium at risk, as demonstrated by several animal studies. The high-count density images are readily quantitated, and techniques have been developed and validated for this purpose. Early clinical studies have shown that myocardium at risk varies widely, even for a coronary occlusion in a similar location, a finding similar to that reported previously in several different animal infarction models. The clinical use of this radiopharmaceutical to measure final infarct size and treatment benefit, or myocardial salvage, has now been demonstrated using both planar and tomographic imaging techniques. Evidence of benefit is often evident by 18 to 48 hours after reperfusion therapy, although the full extent of improvement is not evident until later. The current 6-hour shelf life and 30-minute preparation time are logistical barriers to widespread clinical use. This radiopharmaceutical provides a new, powerful measurement tool for the assessment of treatment efficacy in acute myocardial infarction that is probably superior to other currently available methods. [\hyperlink{Technetium Tc-99M Sestamibi Kit}{PMID: 1835138}, R J Gibbons et al., 1991]

\hypertarget{pmid_2145747}{R}ecently, efforts have been directed at the development of technetium-99m (Tc-99m)-labeled isonitrile compounds for assessment of regional perfusion and viability after experimental myocardial infarction or ischemia. One of the most promising of these agents, Tc-99m sestamibi, has undergone rather extensive laboratory investigation. Like thallium-201 (Tl-201), the uptake of Tc-99m sestamibi in myocardial tissue is proportional to myocardial blood flow after intravenous injection. Similar to other diffusible indicators, Tc-99m sestamibi underestimates blood flow at high flow rates. In low flow regions, the myocardial uptake of this agent is higher relative to nonischemic uptake than is microsphere-determined blood flow. This is attributed to increased extraction at low flows. This first-pass myocardial extraction fraction for Tc-99m sestamibi is less than that for Tl-201. However, Tc-99m sestamibi has a higher parenchymal cell permeability and higher volume of distribution than T;-201. Tc-99m sestamibi shows minimal "delayed redistribution" after initial intravenous administration. Uptake of Tc-99m sestamibi is not altered by myocardial "stunning" or with ischemic dysfunction produced by sustained low coronary flow. The uptake of the isonitrile is still proportional to blood flow in these situations. In intact animal models, myocardial uptake of Tc-99m sestamibi during coronary occlusion delineates the in vivo area at risk. When Tc-99m sestamibi is administered after reperfusion following variable periods of preceding coronary occlusion, Tc-99m sestamibi uptake delineates the area of viable myocardium that is salvaged and not simply the degree of reflow. This suggests that serial Tc-99m sestamibi imaging might be useful in assessing the efficacy of coronary reperfusion after thrombolytic therapy. [\hyperlink{Technetium Tc-99M Sestamibi Kit}{PMID: 2145747}, G A Beller et al., 1990]

\hypertarget{pmid_8523231}{T}echnetium 99m-sestamibi, a radiopharmaceutical used for the diagnostic imaging of abnormal parathyroid tissue, and the Neoprobe 1000, a hand-held, gamma-detecting probe, were used concurrently, during surgical exploration, in three children with hyperparathyroidism. This novel combination assisted with the identification of an ectopic mediastinal parathyroid adenoma and with the localization of multiple hyperplastic parathyroid glands. 99mTc-sestamibi combined with the Neoprobe 1000 may prove to be a useful adjunctive technique for the intraoperative localization of abnormal parathyroid tissue in selected patients. [\hyperlink{Technetium Tc-99M Sestamibi Kit}{PMID: 8523231}, D A Martinez et al., 1995]

\hypertarget{pmid_2145744}{T}echnetium-99m (Tc-99m) sestamibi has been used to evaluate the efficacy of thrombolytic therapy. Improved image quality due to the higher photon energy of Tc-99m and the increased allowable doses of this radiopharmaceutical along with its lack of redistribution makes Tc-99m sestamibi an acceptable imaging agent for such studies. This imaging agent was used for serial quantitative planar and tomographic imaging to assess the initial risk area of infarction, its change over time and the relation to infarct-related artery patency in patients with a first acute myocardial infarction. Twenty-three of 30 patients were treated with recombinant tissue-type plasminogen activator (rt-PA) within 4 hours after onset of acute chest pain. Seven patients were treated in the conventional manner and did not receive thrombolytic therapy. The initial area at risk varied greatly both in patients treated with rt-PA and in those who received conventional therapy. Patients with successful thrombolysis and patient infarct arteries had a significantly greater reduction of Tc-99m sestamibi defect size than patients who had persistent coronary occlusion. Serial imaging with Tc-99m sestamibi could find important application in future clinical research evaluating the efficacy of new thrombolytic agents. Direct measurements of the amount of hypoperfused myocardium before and after thrombolysis could provide rapid and unequivocal results using fewer patients and avoiding the use of "mortality" as an end point. This approach has not yet been widely tested in the clinical arena. [\hyperlink{Technetium Tc-99M Sestamibi Kit}{PMID: 2145744}, F J Wackers et al., 1990]

\hypertarget{pmid_7628151}{T}c-99m sestamibi has been shown to accumulate in several primary malignant tumors, but data regarding its use in the detection of distant metastases are limited. Despite its physical limitations, Tl-201 now has a definite place in the routine evaluation of certain primary and metastatic tumors. This report describes the value of Tc-99m sestamibi and its superiority to Tl-201 in the visualization of distant metastases in a 2-year-old girl with bilateral retinoblastoma. Three sites of soft tissue and bone metastasis were demonstrated by Tc-99m sestamibi imaging. Of these metastases, the one in the parieto-occipital region showed evident Tl-201 uptake, while the other two in the right arm and chest wall showed only slightly increased uptake, which could hardly be spotted without the confirmation of a later Tc-99m sestamibi scan. The scintigraphic findings were confirmed with histopathologic examination. Tc-99m sestamibi scan is effective and superior to Tl-201 in the detection of distant soft tissue and bone metastases from retinoblastoma. [\hyperlink{Technetium Tc-99M Sestamibi Kit}{PMID: 7628151}, C Aktolun et al., 1995]

\hypertarget{pmid_12552335}{S}everal years ago technetium-99m tetrofosmin was reported to localise parathyroid adenomas. The aim of this study was to compare the sensitivity of this radiopharmaceutical with that of (99m)Tc-sestamibi using a double-phase parathyroid scintigraphy protocol. Scans of 12 patients were evaluated visually and lesion to thyroid ratios were calculated. Nine of the patients were subsequently operated on; a total of eight parathyroid adenomas or hyperplastic glands were histologically confirmed in seven of the patients, while in one patient a parathyroid carcinoma was histologically proven. All of these patients had positive (99m)Tc-sestamibi scintigrams, whereas only two (99m)Tc-tetrofosmin scintigrams were positive. With (99m)Tc-sestamibi there was a significant increase in the lesion to thyroid ratio from 10 min to 90 min and 150 min p.i. which was not seen on scintigraphy with (99m)Tc-tetrofosmin. This makes (99m)Tc-tetrofosmin less suitable for double-phase parathyroid scintigraphy. [\hyperlink{Technetium Tc-99M Sestamibi Kit}{PMID: 12552335}, Alida C Fröberg et al., 2003]

\hypertarget{pmid_9225801}{T}he purpose of this study was to assess the normal perfusion pattern of the pediatric brain with 99mTc-ethylcysteinate dimer (99mTc-ECD). Tomographic imaging was performed with a dedicated system with high sensitivity and resolution. Sixteen children, referred for brain imaging in the workup of seizure disorder, were included since they turned out negative after a 1-yr follow-up. A standardized brain presentation was obtained after reslicing and reorienting of the three-dimensional volumetric dataset. Quantitative analysis did not reveal significant left-right uptake differences per patient. Three age clusters were investigated that showed differences in regional uptake, mainly a relatively increased uptake in basal ganglia, visual and motor cortex. An uptake ratio or perfusion index was calculated after normalization. Normal limits were established for the children in the three groups. Technetium-99m-ECD is a safe agent for children and should be the radiopharmaceutical of choice for brain perfusion studies because of favorable radiation dosimetry and stability. The age dependence of perfusion necessitates a database comparison before concluding that the observed perfusion pattern is normal. [\hyperlink{Technetium Tc-99M Sestamibi Kit}{PMID: 9225801}, C Schiepers et al., 1997]

\hypertarget{pmid_8404951}{T}echnetium-99m exametazime (99mTc-d,l-HMPAO), prepared by the reconstitution of the Ceretec kit, is widely used in the investigation of regional cerebral blood flow. The radiochemical purity specification of not less than 80\% lipophilic complex requires that the kit is used within 30 min of reconstitution. In certain circumstances this imposes restrictions on its clinical availability. A number of approaches to extending the 30-min shelf life have been proposed and these are discussed. A method of stabilising the kit by the addition of 200 micrograms cobalt chloride hexahydrate (CoCl2 x 6H2O) in 2 ml of water is described. The addition of this solution can extend the shelf life of the reconstituted kit to at least 5 h post reconstitution. [\hyperlink{Technetium Tc-99M Sestamibi Kit}{PMID: 8404951}, P S Weisner et al., 1993]

\hypertarget{pmid_18818954}{A} randomized, open, coordinated multi-center trial compared the bacteriological and clinical efficacy and safety of orally administered ceftibuten and trimethoprim-sulfamethoxazole (TMP-SMX) in children with febrile urinary tract infection (UTI). Children aged 1 month to 12 years presenting with presumptive first-time febrile UTI were eligible for enrollment. A 2:1 assignment to treatment with ceftibuten 9 mg/kg once daily (n = 368) or TMP-SMX (3 mg + 15 mg)/kg twice daily (n = 179) for 10 days was performed. Escherichia coli was recovered in 96\% of the cases. Among the E. coli isolates, 14\% were resistant to TMP-SMX but none to ceftibuten. In the modified intention-to-treat population, the bacteriological elimination rates at follow-up did not differ significantly between patients treated with ceftibuten and those treated with TMP-SMX [91 vs. 95\%, with a 95\% confidence interval (CI) for difference of -9.7 to 1.0]. However, the clinical cure rate was significantly higher among those treated with ceftibuten (93 vs. 83\%, with a 95\% CI for difference of 2.4 to 17.0). Adverse events were similar for both regimens and consisted mainly of gastrointestinal disturbances. In conclusion, ceftibuten is a safe and effective drug for the empirical treatment of febrile UTI in young children. [\hyperlink{Technetium Tc-99M Sestamibi Kit}{PMID: 18818954}, Staffan Mårild et al., 2009]

\hypertarget{pmid_12411545}{S}PECT has the potential to add valuable information to the diagnosis and management of central nervous system (CNS) malignancy. Radioactive tracers including (99m)Tc-methoxyisobutylisonitrile (MIBI), or sestamibi, have been shown to be sensitive markers for brain tumors; however, their role in imaging children is poorly defined. We undertook a pilot study of 29 pairs of (99m)Tc-MIBI and MRI images from 20 children to explore the clinical usefulness of this tracer in CNS malignancy. Tumor types that took up (99m)Tc-MIBI included brain stem glioma, fibrillary astrocytoma, other low-grade astrocytomas, and glioblastoma multiforme. Most tumors positive for (99m)Tc-MIBI uptake were astrocytomas, including those in the brain stem, cerebellum, and cortex. This method of nuclear imaging not only was able to identify the presence of a tumor but also could identify changes in the same tumor over time. Some correlation between histologic grade and (99m)Tc-MIBI uptake was observed. Several tumors, including craniopharyngioma, medulloblastoma, and optic glioma, were evident on MRI but not on (99m)Tc-MIBI SPECT. The results suggest that this modality is a potentially useful tool in the diagnosis and management of CNS malignancies, particularly higher-grade astrocytomas, in children. [\hyperlink{Technetium Tc-99M Sestamibi Kit}{PMID: 12411545}, Adam Kirton et al., 2002]

\hypertarget{pmid_9591536}{A} retrospective chart review of 43 patients who underwent technetium 99m (Tc-99m) sestamibi scans from June 1995 to January 1997 was performed. Only those who underwent subsequent parathyroid exploration with excision were included in the study. Twenty subjects (13 women and seven men) were included in the study. Ages ranged from 21 to 84 years (mean, 58 years). All patients had laboratory values and clinical findings consistent with primary hyperparathyroidism. Two patients had preoperative magnetic resonance imaging (MRI) scans (one patient with recurrent disease), and one had a preoperative computed tomography (CT) scan. The remaining patients had the sestamibi scan as the only preoperative localization study. There were 18 pathologic diagnoses of parathyroid adenoma and two of parathyroid hyperplasia. Sestamibi failed to correctly identify the location of the parathyroid lesion in two cases. In 18 cases the preoperative sestamibi scan correctly localized the lesion, a predictive value of 90\%. We conclude that the Tc-99m sestamibi scan is an accurate preoperative tool that can be used as a single modality to localize parathyroid adenomas. [\hyperlink{Technetium Tc-99M Sestamibi Kit}{PMID: 9591536}, E F George et al., 1998]

\hypertarget{pmid_2145751}{U}nlike thallium-201, technetium-99m (Tc-99m) sestamibi does not redistribute in the myocardium after injection. Thus, 2 separate injections, 1 at rest and the other at stress (or after dipyridamole), are required to differentiate ischemia from scar. From a physical viewpoint, a 24-hour interval between the 2 injections is preferable for detection of coronary artery disease (CAD) with Tc-99m sestamibi imaging. However, same-day studies are more convenient in clinical practice. Results of studies using different Tc-99m sestamibi injection protocols are presented with emphasis on the advantages of a rest-stress injection sequence with a low dose at rest (7 mCi) followed 2 hours later by a higher dose at stress (25 mCi). A prospective study was conducted in a patient population with proven CAD using same-day studies to compare a rest-stress (7 and 25 mCi, respectively) to a stress-rest (7 and 25 mCi) Tc-99m sestamibi injection sequence. There was an agreement in 87.3\% of the analyzed segments between the 2 protocols. However, the largest discordance for type of defect applied to 7.4\% of the segments judged ischemic in the rest-stress protocol, which were called scars on stress-rest. This study showed that a rest-stress sequence is preferable when using a same-day protocol with a short time interval (less than 2 hours) between the 2 Tc-99m sestamibi injections because the rest image performed initially represents a "true" rest study, which is not necessarily the case with the stress-rest sequence. Preliminary studies were performed to evaluate dipyridamole with Tc-99m sestamibi imaging in normal subjects and in patients with CAD. These studies showed that treadmill and dipyridamole Tc-99m sestamibi imaging are comparable and the results are similar to those obtained with thallium-201. [\hyperlink{Technetium Tc-99M Sestamibi Kit}{PMID: 2145751}, R Taillefer et al., 1990]

\hypertarget{pmid_442412}{T}he serial scintiphotography following intravenous injection of Tc99m-ertechnetate was used for examination of 50 children aged from 11 months to 14 years. The method is founded on the Tc99m-pertechnetate property of selective accumulation in the gastric mucous membrane and in Meckel's diverticulum when the latter contains the ectopic gastric mucous membrane. The inflamed diverticulum can accumulate the radionuclide, as well. Meckel's diverticulum was suspected in 7 children during examination; in 6 of them Meckel's diverticulum was found peroperatively, and in one case there was enterocyst of the ileum. [\hyperlink{Technetium Tc-99M Sestamibi Kit}{PMID: 442412}, Iu A Tikhonov et al., 1979]

\hypertarget{pmid_14978536}{I}nvestigation of the diagnostic role of technetium-99m methoxyisobutylisonitrile ((99m)Tc sestamibi) scintimammography in non-palpable, suspicious breast lesions described as microcalcification, mass and increased density using mammography. 35 women with non-palpable breast lesions were enrolled in the study. Anterior, left and right lateral, ipsilateral posterior oblique images were obtained 15 min after the injection of 740 MBq of (99m)Tc sestamibi. All scintigraphic images were evaluated visually and focal increased (99m)Tc sestamibi uptake was accepted as malignant lesion. Breast lesions were classified as microcalcification (13 women), mammographic mass (16 women) and increased density (6 women). Excisional biopsy was performed in all of them irrespective of the scintigraphic results. The focally increased (99m)Tc sestamibi uptake was seen in 11 breast lesions with malignant lesions and in 4 breast lesions with benign lesions. The diffuse uptake of (99m)Tc sestamibi was seen in 18 breast lesions with benign lesions and 2 breast lesions with malignant lesions. There was no false positive result of (99m)Tc sestamibi in microcalcification group and there was no false negative result of the mammographic mass and increased density groups. Scintimammography might be a complementary method in decision making for the non-palpable, suspicious breast lesions that were evaluated as microcalcification, mass and increased density mammograpically. [\hyperlink{Technetium Tc-99M Sestamibi Kit}{PMID: 14978536}, R Bekiş et al., 2004]

\hypertarget{pmid_1533370}{A}bnormally high uptake of technetium-99m hexakis-2-methoxyisobutylisonitrile (99mTc-SESTAMIBI) in the right ventricle and in the septum was observed in a 47-year-old woman initially presenting with dysarthria and left hemiparesis. Endomyocardial biopsy demonstrated a high-grade malignant non-Hodgkin's lymphoma. Complete remission was achieved by combined cyclophosphamide, doxorubicin, vincristine and prednisone (CHOP) chemotherapy and radiotherapy of the heart and mediastinum. The post-remission single photon emission tomography (SPET) 99mTc-SESTAMIBI study showed a homogeneous distribution pattern, in agreement with echocardiography computed tomography and magnetic resonance imaging. Increased uptake of 99mTc-SESTAMIBI, a myocardial perfusion agent, has been observed in some benign and malignant tumours. It may prove to be useful in the diagnosis and follow-up of malignancies. [\hyperlink{Technetium Tc-99M Sestamibi Kit}{PMID: 1533370}, G Medolago et al., 1992]

\hypertarget{pmid_26465108}{T}o evaluate whether scintigraphy with technetium-99m-labeled ceftizoxime ((99m)Tc-CFT) can differentiate mediastinitis from aseptic inflammation associated with sternotomy. Twenty female Wistar rats were randomly distributed into four groups: S (control) -partial upper median sternotomy with no treatment; SW (control) - sternotomy and treatment of sternal wounds with bone wax; SB - sternotomy and infection with Staphylococcus aureus; SWB - sternotomy with bone wax treatment and bacterial infection. Scintigraphy with (99m)Tc-CFT was performed eight days after surgery and images were collected 210 and 360 min after infusion of the radiopharmaceutical. No animals exhibited clinical signs of wound infection at the end of the experiment, although histological data verified acute inflammatory response in those experimentally infected with bacteria. Scintigraphic images revealed that tropism of (99m)Tc-CFT to infected sternums was greater than to their non-infected counterparts. Mean counts of radioactivity in bacteria-infected sternal regions (SB and SWB) were significantly higher (p = 0.0007) than those of the respective controls (S and SW). Scintigraphy with technetium-99m-labeled ceftizoxime is a method that can potentially detect infection post sternotomy and differentiate from aseptic inflammation in animals experimentally inoculated with S. aureus. [\hyperlink{Technetium Tc-99M Sestamibi Kit}{PMID: 26465108}, Paulo Henrique Nogueira Costa et al., 2015]

\hypertarget{pmid_1835728}{T}he sensitivity and specificity of technetium-99m hexakis-2-methoxy-2-isobutyl-isonitrile (sestamibi) single-photon emission computed tomographic (SPECT) imaging for the diagnosis of coronary artery disease were studied in 45 patients admitted to the hospital for clinical suspicion of unstable angina. Only patients without prior myocardial infarction were included and all patients had technetium-99m sestamibi injection and a 12-lead electrocardiogram (ECG) during and less than or equal to 4 h after an episode of chest pain. Coronary angiography performed in all patients during hospitalization showed significant coronary artery disease (greater than or equal to 50\% luminal diameter reduction) in 26 of the 45 patients. The SPECT studies obtained after injection of technetium-99m sestamibi during an episode of spontaneous chest pain showed a sensitivity of 96\% for the detection of coronary artery disease; the 12-lead ECG obtained at the time of the injection had a sensitivity of 35\%. With the patient in the pain-free state, respective sensitivity values were 65\% and 38\%. Specificity for the radionuclide study was 79\% during pain and 84\% in the pain-free state; for the ECG, it was 74\% both during and between episodes of pain. The site of the perfusion defect corresponded to the most severe coronary artery lesion in 88\% of patients. The severity of the perfusion defect correlated with the extent of coronary artery disease: the defect score was 5.3 +/- 3.3 with one-vessel disease, 4.9 +/- 2.8 with two-vessel disease and 10.5 +/- 5.0 with three-vessel disease (p less than 0.01).(ABSTRACT TRUNCATED AT 250 WORDS) [\hyperlink{Technetium Tc-99M Sestamibi Kit}{PMID: 1835728}, L Bilodeau et al., 1991] Since the introduction of technetium-99m methoxy-isobutyl isonitrile (Tc-99m sestamibi) in Europe, there has been a growing interest in its use. Several European multicenter trials have been conducted to evaluate this new agent in relation to the traditional perfusion marker thallium-201, and other studies are in progress to understand the use of this perfusion marker for the diagnosis of coronary disease, for use in conjunction with pharmacologic vasodilation, for use in the assessment of ventricular function and wall motion and for the assessment of interventions. [\hyperlink{Technetium Tc-99M Sestamibi Kit}{PMID: 1835728}, H Sochor et al., 1990]

\hypertarget{pmid_2148347}{T}echnetium-99m-Hexamibi [Hexakis (methoxyisobutylisonitrile) technetium (i)] was developed as a myocardial perfusion agent with biologic properties similar to those of thallium-201 (201TI). As 201TI has recently been observed to be of value for the diagnosis of brain tumors when used in conjunction with single-photon emission computed tomography (SPECT) imaging technology, the possibility that the biologic similarity of the two radiopharmaceuticals extended to their affinity for tumors was tested. A 5-yr-old female patient with a brain stem astrocytoma showed marked focal uptake of 99mTc-Hexamibi at the site of tumor recurrence as defined by biopsy and prior 201TI/SPECT study. Tumor-to-normal cortex radioactivity ratios for the 99mTc-Hexamibi/SPECT study were 132:1 and the spatial resolution of the 99mTc-Hexamibi images was high. This observation suggests that 99mTc-Hexamibi merits further study as a potential agent for SPECT imaging of human brain tumors. [\hyperlink{Technetium Tc-99M Sestamibi Kit}{PMID: 2148347}, L A O'Tuama et al., 1990]

\hypertarget{pmid_9179242}{T}hallium-201 with a half-life of 73 hours, decays by electron capture and as a consequence emits numerous Auger electrons. When it localises in the cell nucleus it causes enhanced biological effects. Technetium-99m with a half-life of 6 hours, radiates gamma rays and the side effects are not as significant. Tl-201 and Tc-99m labelled with SESTAMIBI are used for the SPECT perfusion image of the heart. In this study the tissue of interest are the testes which, after irradiation, can develop stochastic effects: both somatic (cancer) and hereditary. The activities of the radiopharmaceuticals used in common practice (30 mCi of Tc-99m and 3 mCi of Tl-201) cause different probabilities for the induction of stochastic effects in the testes. The probabilities are about 30 times higher for Tl-201 than for Tc-99m. These results, in combination with the fact that the higher activity of Tc-99m yields better images within shorter time, must make the clinician carefully assess the choice of the radiopharmaceutical to be used for the studies of the heart, especially for patients of reproductive age. [\hyperlink{Technetium Tc-99M Sestamibi Kit}{PMID: 9179242}, A Manetou et al., ]

\hypertarget{pmid_26428089}{C}orrosive esophageal injury due to accidental ingestion is a serious clinical problem in children particularly in developing countries. The present study was conducted to evaluate the diagnostic utility of technetium-99m-pyrophosphate ((99m)Tc-PYP) scintigraphy in the early stage of esophageal burns by using different concentrations of sodium hydroxide (NaOH) in an experimental rat model. Twenty-eight male Sprague-Dawley rats, weighing 200-250 g, were used in the study. Esophageal burn model was created in 21 rats by gastrically infusion of various concentrations of NaOH. The rats were divided randomly into three groups: mild-burn group (n = 7) received 15\% NaOH, moderate-burn group (n = 7) received 30\% NaOH and severe-burn group (n = 7) received 45\% NaOH. Seven rats were identified as control group and received normal saline. Three hours after burn injury, 1-mCi (99m)Tc-PYP was administered through tail vein. Two hours after (99m)Tc-PYP administration, static imaging with gamma camera was performed. Then, histopathologic assessment of esophageal samples was achieved properly. All NaOH-applied groups (mild, moderate, and severe) showed a significant higher uptake ratio when compared to control group (P < 0.005). NaOH-applied groups displayed important histologic alterations such as mucosal disintegration, edema, inflammation, and stromal damage when compared to control group. Pearson correlation analysis revealed a significant correlation between the (99m)Tc-PYP uptake ratio and histologic score (P < 0.0005). The scintigraphic imaging may provide advantages in the early stage of esophageal burns in some patients whom endoscopic procedure is contraindicated because of its high risk of complications such as bleeding and perforation. [\hyperlink{Technetium Tc-99M Sestamibi Kit}{PMID: 26428089}, Öznur Dilek Çiftçi et al., 2016]

\hypertarget{pmid_2145748}{T}echnetium-99m (Tc-99m) sestamibi is a new myocardial perfusion imaging agent that offers significant advantages over thallium-201 (Tl-201) for myocardial perfusion imaging. The results of the current clinical trials using acquisition and processing parameters similar to those for Tl-201 and a separate (2-day) injection protocol suggest that Tc-99m sestamibi and Tl-201 single photon emission computed tomography (SPECT) provide similar information with respect to detection of myocardial perfusion defects, assessment of the pattern of defect reversibility, overall detection of coronary artery disease (CAD) and detection of disease in individual coronary arteries. Tc-99m sestamibi SPECT appears to be superior to Tc-99m sestamibi planar imaging because the former provides a higher defect contrast and is more accurate for detection of disease in individual coronary arteries. Research is currently under way addressing optimization of acquisition and processing of Tc-99m sestamibi studies and development of quantitative algorithms for detection and localization of CAD and sizing of transmural and nontransmural myocardial perfusion defects. It is expected that with the implementation of the final results of these new developments, further significant improvement in image quality will be attained, which in turn will further increase the confidence in image interpretation. Development of algorithms for analysis of end-diastolic myocardial images may allow better evaluation of small and nontransmural myocardial defects. Furthermore, gated studies may provide valuable information with respect to regional myocardial wall motion and wall thickening. With the implementation of algorithms for attenuation and scatter correction, the overall specificity of Tc-99m sestamibi SPECT should improve significantly because of a substantial decrease in the occurrence of attenuation-related image artifacts.(ABSTRACT TRUNCATED AT 250 WORDS) [\hyperlink{Technetium Tc-99M Sestamibi Kit}{PMID: 2145748}, J Maddahi et al., 1990] Sedation is often required for young children during transthoracic echocardiography. Dexmedetomidine and ketamine are two sedatives that are commonly used in children for procedural sedation, but they have some disadvantages when they are used alone. The aim of this retrospective study was to analyze the effects and safety of intranasal sedation with a combination of dexmedetomidine and ketamine during transthoracic echocardiography in young children and to analyze risk factors for sedation failure. After IRB approval, we retrospectively evaluated data on patients who underwent echocardiography between May 2016 and August 2017 utilizing a combination of dexmedetomidine 2 μg/kg and ketamine 1 mg/kg. We collected information including heart rate, pulse oxygen saturation, sedation onset time, exam time, recovery time, and adverse reactions. Stepwise logistic regression analyses were performed to analyze the risk factors for sedation failure. Sedation was successful in 2212 patients (96\%) and took effect in 15.7 (IQR: 10-23) min, while sedation failed in 92 patients. Cyanotic heart disease, history of sedation failure, history of congenital heart disease surgery, and fever were independent risk factors for sedation failure. Most of the patients in this study had an American Society of Anesthesiologists (ASA) grade of II to III, but no severe adverse reactions were observed. Intranasal sedation with a combination of dexmedetomidine and ketamine is effective and appears to have an acceptable safety profile for young children during transthoracic echocardiography. [\hyperlink{Technetium Tc-99M Sestamibi Kit}{PMID: 2145748}, Jianxia Liu et al., 2019]

\hypertarget{pmid_34839268}{P}rophylactic antiseizure medications (ASMs) for pediatric traumatic brain injury (TBI) are understudied. We evaluated clinical and radiographic features that inform prescription of ASMs for pediatric TBI. We hypothesized that despite a lack of evidence, levetiracetam is the preferred prophylactic ASM but that prophylaxis is inconsistently prescribed. This retrospective study assessed children admitted with TBI from January 1, 2017, to December 31, 2019. TBI severity was defined using Glasgow Coma Scale (GCS) scores. Two independent neuroradiologists reviewed initial head computed tomography and brain magnetic resonance imaging. Fisher exact tests and descriptive and regression analyses were conducted. Among 167 children with TBI, 44 (26\%) received ASM prophylaxis. All 44 (100\%) received levetiracetam. Prophylaxis was more commonly prescribed for younger children, those with neurosurgical intervention, and abnormal neuroimaging (particularly intraparenchymal hematoma) (odds ratio = 10.3, confidence interval 1.8 to 58.9), or GCS ≤12. Six children (13.6\%), all on ASM, developed early posttraumatic seizures (EPTSs). Of children with GCS ≤12, four of 17 (23.5\%) on levetiracetam prophylaxis developed EPTSs, higher than the reported rate for phenytoin. Although some studies suggest it may be inferior to phenytoin, levetiracetam was exclusively used for EPTS prophylaxis. Intraparenchymal hematoma >1 cm was the single neuroimaging feature associated with ASM prophylaxis regardless of the GCS score. Yet these trends are not equivalent to optimal evidence-based management. We still observed important variability in neuroimaging characteristics and TBI severity for children on prophylaxis. Thus, further study of ASM prophylaxis and prevention of pediatric EPTSs is warranted. [\hyperlink{Technetium Tc-99M Sestamibi Kit}{PMID: 34839268}, Taryn-Leigh Surtees et al., 2022]

\section*{Ceftriaxone Sodium}
\subsection*{Result}
\subsubsection*{Answer}

Neonates (≤28 days): Unknown
Infants (1 month to 2 years): Yes
Children (>2 months to 18 years): Yes

\subsubsection*{{Explanation}}
\hypertarget{Ceftriaxone Sodium}
Based on the available abstracts, the safety of Ceftriaxone Sodium in children has been directly studied in multiple age groups. Below is a summary of the evidence by age range:

Neonates (≤28 days):
- Several abstracts specifically address neonates. One systematic review found that the available literature on neonates is limited, with small sample sizes and inconsistent outcome ascertainment, making it impossible to estimate the magnitude of risk. There are reports of bilirubin displacement, self-resolving biliary sludge, and serious cardiopulmonary events (including deaths) when ceftriaxone is administered with calcium-containing solutions. The review concludes that concomitant administration of intravenous ceftriaxone and calcium-containing solutions should be avoided in neonates, and further controlled studies are needed to assess safety, especially regarding hyperbilirubinemia [\hyperlink{pmid_27718120}{PMID: 27718120}, Patrick C Donnelly et al., 2017]. Another review recommends that ceftriaxone should be avoided or minimized in neonates, especially those with hyperbilirubinemia or receiving calcium [\hyperlink{pmid_18759704}{PMID: 18759704}, Scott V Monte et al., 2008]. However, a pharmacokinetic and safety study in 39 neonates found no drug-associated changes in biochemical or hematological parameters and concluded ceftriaxone is safe and well tolerated for newborn sepsis [\hyperlink{pmid_4076254}{PMID: 4076254}, A Mulhall et al., 1985]. Overall, the evidence is mixed, with some studies affirming safety in neonates, but systematic reviews highlight unresolved safety concerns, especially with certain risk factors.

Infants (1 month to 2 years):
- A pharmacokinetic study in 66 infants (1 month to 2 years) found that age and weight significantly affect ceftriaxone pharmacokinetics, but did not report significant safety concerns [\hyperlink{pmid_32816735}{PMID: 32816735}, Ya-Kun Wang et al., 2020]. Other studies in this age group (and older children) report mild, transient side effects such as diarrhea, eosinophilia, and transient laboratory abnormalities, but generally affirm safety [\hyperlink{pmid_6330022}{PMID: 6330022}, T Chonmaitree et al., 1984; \hyperlink{pmid_6318653}{PMID: 6318653}, S C Aronoff et al., 1983; \hyperlink{pmid_3985602}{PMID: 3985602}, B L Congeni et al., 1985].

Children (>2 months to 18 years):
- Multiple studies and reviews have evaluated ceftriaxone in children for a variety of infections, including sepsis, respiratory tract infections, otitis media, and typhoid fever. These studies consistently report high efficacy and a low incidence of mild, transient side effects (e.g., diarrhea, rash, laboratory abnormalities). Some studies specifically note the occurrence of asymptomatic biliary pseudolithiasis and nephrolithiasis, especially with higher doses, longer duration, and in children over 12 months, but these complications are generally self-limited and do not require discontinuation if the patient is asymptomatic [\hyperlink{pmid_18246742}{PMID: 18246742}, Ahmet Soysal et al., ; \hyperlink{pmid_24910741}{PMID: 24910741}, Azita Fesharakinia et al., 2013]. Rare but serious hypersensitivity reactions (e.g., anaphylaxis) have been reported [\hyperlink{pmid_24592804}{PMID: 24592804}, D Shrestha et al., 2013; \hyperlink{pmid_24130394}{PMID: 24130394}, Russelian Arulraj et al., ], but these are uncommon.

Summary:
- For neonates, the safety of ceftriaxone sodium is uncertain due to conflicting evidence and unresolved concerns about specific risks (e.g., hyperbilirubinemia, calcium co-administration).
- For infants and older children, multiple targeted studies affirm the safety of ceftriaxone sodium, with mild and transient side effects being most common. Rare but serious adverse events can occur, as with most antibiotics.

\subsection*{Abstracts}
\hypertarget{pmid_28827252}{C}eftriaxone is widely used in children in the treatment of sepsis. However, concerns have been raised about the safety of ceftriaxone, especially in young children. The aim of this review is to systematically evaluate the safety of ceftriaxone in children of all age groups. MEDLINE, PubMed, Cochrane Central Register of Controlled Trials, EMBASE, CINAHL, International Pharmaceutical Abstracts and adverse drug reaction (ADR) monitoring systems will be systematically searched for randomised controlled trials (RCTs), cohort studies, case-control studies, cross-sectional studies, case series and case reports evaluating the safety of ceftriaxone in children. The Cochrane risk of bias tool, Newcastle-Ottawa and quality assessment tools developed by the National Institutes of Health will be used for quality assessment. Meta-analysis of the incidence of ADRs from RCTs and prospective studies will be done. Subgroup analyses will be performed for age and dosage regimen. Formal ethical approval is not required as no primary data are collected. This systematic review will be disseminated through a peer-reviewed publication and at conference meetings. CRD42017055428. [\hyperlink{Ceftriaxone Sodium}{PMID: 28827252}, Linan Zeng et al., 2017]

\hypertarget{pmid_6330022}{T}he clinical efficacy and safety of ceftriaxone, a long half-life cephalosporin were evaluated in 48 children with a variety of serious bacterial infections. Clinical cure was achieved in 92\% (44 of 48) of patients. Peak serum bactericidal titres for Haemophilus influenzae type b, Streptococcus pneumoniae, Str. pyogenes and Escherichia coli were greater than or equal to 1:1024. Mean peak and trough ceftriaxone levels were 173 and 42 mg/l, respectively. Mild and transient diarrhoea was observed in 10\% of patients. Laboratory side effects encountered were eosinophilia, thrombocytosis and neutropenia in another 8\%. Ceftriaxone is a useful antibiotic for common childhood infections. Its prolonged half-life allows twice daily administration which reduces problems related to intravenous therapy as well as the cost and personnel time. [\hyperlink{Ceftriaxone Sodium}{PMID: 6330022}, T Chonmaitree et al., 1984]

\hypertarget{pmid_3969363}{C}eftriaxone is a new parenteral cephalosporin with a prolonged half-life and an expanded Gram-negative spectrum. Before it can be used as a single agent for infections of unknown etiology, its efficacy in treating infections caused by Gram-positive organisms, particularly Staphylococcus aureus, must be proven. Ceftriaxone was administered to 12 children for treatment of infections due to S. aureus alone or in the presence of other organisms. Sites of infection included soft tissue, respiratory tract, bone and joint. Patients received ceftriaxone at 68 to 100 mg/kg/day in two doses for 3 to 20 days. Clinical and bacteriologic responses were satisfactory in all patients. One patient experienced abdominal pain during infusion and another developed a skin rash. Five patients had platelet counts of 500,000/mm3 or greater; four had an eosinophil count of 7\% or greater and one patient had transient neutropenia. These abnormalities resolved during or after therapy. Ceftriaxone was a safe and effective single antibiotic for the treatment of infections caused by S. aureus in children. [\hyperlink{Ceftriaxone Sodium}{PMID: 3969363}, S J Nelson et al., ]

\hypertarget{pmid_27718120}{C}eftriaxone is a third-generation cephalosporin with broad-spectrum activity against both Gram-positive and Gram-negative bacteria. Despite its effectiveness, its use for the treatment of infections in neonatal patients has been limited because of concern about its potential toxicity. Our aim was to review the literature for an association between ceftriaxone and cardiopulmonary events, hyperbilirubinemia, and pseudolithiasis among neonates. We searched PubMed and EMBASE and included studies that evaluated ceftriaxone safety in neonates. Study bias was evaluated in the following domains: exposure measurement, outcome measurement, attrition, generalizability, confounding, statistical analysis, and reporting. We included nine studies regarding ceftriaxone side effects, primarily spontaneous reports, published case reports, and small case series. Reports of bilirubin displacement attributed to ceftriaxone included increases in serum bilirubin necessitating antibiotic change in a subset of infants after administration of ceftriaxone. One study described self-resolving biliary sludge after ceftriaxone administration in six of 80 infants. Cardiopulmonary adverse events included a report of eight cardiopulmonary events related to concomitant ceftriaxone-calcium infusion, including seven infant deaths. Additional cardiopulmonary events reported included perinatal asphyxia, pulmonary hypertension, and thrombocytosis. However, the available literature had small sample sizes, poor external validity, and inconsistent outcome ascertainment methods, which made it impossible to estimate the magnitude of risk. Concomitant administration of intravenous ceftriaxone and calcium-containing solutions should be avoided in neonates. However, further controlled studies are needed to assess whether bilirubin displacement associated with the use of ceftriaxone is clinically relevant, particularly in healthy term and near-term neonates with mild hyperbilirubinemia. [\hyperlink{Ceftriaxone Sodium}{PMID: 27718120}, Patrick C Donnelly et al., 2017]

\hypertarget{pmid_3912733}{A}A. have tested a new drug (Ceftriaxone) on 40 children affected by upper and lower respiratory tract infectious diseases. As shown by results, this new drug has been remarkably effective and easy to use since it may be administered once in a day; moreover, the tested drug has not caused any kind of tissue or parenchymal involvement. [\hyperlink{Ceftriaxone Sodium}{PMID: 3912733}, F De Francesco et al., ]

\hypertarget{pmid_6314805}{C}eftriaxone is an investigational cephalosporin with a half-life of five to eight hours. In an uncontrolled study, we evaluated its efficacy and safety in 30 pediatric and 12 young adult patients with serious bacterial infections. This agent was administered to children at a dosage of 50 to 75 mg/kg/day intravenously in two divided doses. Those with CNS infections received 100 mg/kg/day. In adults, the dosage was 1 g either once or twice daily. The diseases we treated included pneumonia (17), sepsis (eight), ventriculoperitoneal shunt infections (three), osteomyelitis (three), brain abscess (two), peritonitis (two), and miscellaneous (seven). Clinical cures were achieved in all cases, although one child with cystic fibrosis and Pseudomonas pneumonia had persistent colonization in his sputum. No serious side effects were observed. Although not the agent of choice for many of these pathogens, ceftriaxone appears to represent an important alternative to therapy. [\hyperlink{Ceftriaxone Sodium}{PMID: 6314805}, R W Steele et al., 1983]

\hypertarget{pmid_18246742}{C}eftriaxone, a third-generation cephalosporin, is widely used for treating infection during childhood. It is mainly eliminated in the urine, but approximately 40\% of a given dose is unmetabolized and secreted into bile. The aim of this study was to investigate the frequency, clinical characteristics, and outcome of biliary sludge (BS) in addition to potential contributing risk factors in children who receive ceftriaxone. Biliary ultrasonography was performed at the time of randomization before ceftriaxone treatment was started, on the 5th and 10th days, and at the end of the treatment. If BS was detected, patients were followed-up weekly by sonographic examination until the BS or biliary lithiasis (BL) disappeared. A total of 114 children (56 girls, 58 boys; age range: 2-180 months, mean 47.5 +/- 46.3 mos) were enrolled in the study. Fourteen (12\%) subjects developed BS and 10 (9\%) developed BL on the 5th day of treatment. On the 10th day of treatment, 20 (18\%) subjects developed BS and 15 (13\%) developed BL. In total, 35 (31\%) of all subjects developed biliary precipitation (BP), of whom 20 (57\%) were diagnosed as BS and 15 (43\%) as BL. All subjects who developed BP were found to be asymptomatic during the course of therapy. Patient age over 12 months, daily total dose of ceftriaxone of more than 2 g, and duration of treatment longer than five days were found to be associated with BP. Ceftriaxone frequently causes transient BPs and its probability increases if the child is over 12 months of age, the dose is over 2 g/day, or the duration is over five days. Neither radiologic investigation nor the discontinuation of treatment with ceftriaxone is necessary as long as the patient is asymptomatic. [\hyperlink{Ceftriaxone Sodium}{PMID: 18246742}, Ahmet Soysal et al., ]

\hypertarget{pmid_18759704}{I}solated reports of neonatal and infant deaths associated with ceftriaxone-calcium precipitation in the lungs and kidneys have prompted a recommendation from the US FDA in June 2007 advising that in patients of all ages, calcium-containing solutions should not be administered simultaneously or within 48 h of the last ceftriaxone dose. To provide a comprehensive review of the literature surrounding the safety of ceftriaxone in the neonatal (< or = 28 days) and geriatric populations (> or = 65 years). Multi-database literature search for original research articles, review articles and case reports pertaining to safety of ceftriaxone in the neonatal and geriatric populations. Ceftriaxone should be avoided or significantly minimized in neonates (especially those treated concomitantly with intravenous calcium solutions and those with hyperbilirubinemia), and potentially restricted in the geriatric population treated concomitantly with intravenous calcium. [\hyperlink{Ceftriaxone Sodium}{PMID: 18759704}, Scott V Monte et al., 2008]

\hypertarget{pmid_24910741}{C}eftriaxone is a third-generation cephalosporin which is widely used for treatment of infection in children accompanied by complications like urinary tract lithiasis and gallbladder psudolithiasis or sludge. The aim of this study was to investigate the incidence and predisposing factors that contribute to these complications in children. This quasi-experimental and before- and after-study was conducted in 96 children who were hospitalized for treatment of different bacterial infections and received 50-100 mg/kg/day ceftriaxone divided into two equal doses intravenously under conditions of adequate hydration. Sonographic examinations of urinary tract and gallbladder were carried out before and after treatment for that purpose. Patients with positive sonographic findings after treatment were followed with serial sonographic examinations. Post-treatment sonography demonstrated nephrolithiasis in 6 (6.3\%) and gallbladder stone in one (1\%), all were asymptomatic. Comparison of the groups with and without nephrolithiasis demonstrated no significant differences with respect to age, body weight, diagnosis, season of hospitalization, dosage of drug and the duration of treatment. Nephrolithiasis had a significant relation with male gender (P=0.02). Our results showed that pediatric patients may develop small sized, asymptomatic renal stones during a 2-6 day course of normal or moderate dose of ceftriaxone therapy. Close monitoring of ceftriaxone treated patients especially on high dose long term therapy for nephrolithiasis and gallbladder psudolithiasis or sludge is recommended. [\hyperlink{Ceftriaxone Sodium}{PMID: 24910741}, Azita Fesharakinia et al., 2013]

\hypertarget{pmid_14759321}{T}o compare the efficacy and safety of a single ceftriaxone injection with 10-day oral amoxicillin in the treatment for children's acute otitis media. This study was a prospective, comparative, open randomized, multicenter trial. In the ceftriaxone group, a single dose sodium ceftriaxone (50 mg/kg, total dose < 1 g) was injected. In the amoxicillin group, the oral amoxicillin [40 mg/(kg.d), tid] was used for 10 days. Totally 236 cases aged from 0.5 to 12 years were enrolled and 212 cases completed the study. These patients were followed up twice and clinical signs and symptoms were recorded, otoscopy, peripheral blood WBC count, hearing test (pure tone test) and tympanography were performed. In the ceftriaxone group, 103/106 cases were cured or improved (97.17\%), while in the amoxicillin group 96/106 cases were cured or improved (90.57\%) (P < 0.05). Ceftriaxone was significantly better than amoxicillin in the treatment. Totally 4 cases had side effects such as papular skin rash, urticaria around mouth, skin pigmentation, two cases in the ceftriaxone group and other two cases in the amoxicillin group. There was no significant difference between the 2 groups in side effects. Ceftriaxone injection was significantly better than ten-day oral amoxicillin for treatment of acute otitis media in children. The single dose regimen with ceftriaxone seems to be a good choice for children, particularly for. [\hyperlink{Ceftriaxone Sodium}{PMID: 14759321}, Ya-mei Zhang et al., 2003]

\hypertarget{pmid_3985602}{C}eftriaxone administered as a single daily dose of 50 mg/kg was evaluated in the treatment of 35 children with a variety of nonmeningitic bacterial infections. In two of the patients, the drug was discontinued before the response to the drug could be evaluated. All of the remaining patients had a satisfactory response. In 22 of the patients, plasma was available for the determination of ceftriaxone levels 1 h after a dose and immediately before the next dose. All but one of these patients had trough ceftriaxone levels which exceeded the MIC of the infecting organism, although marginally so for Staphylococcus aureus. Ceftriaxone appears to be safe and effective in the treatment of a variety of bacterial pathogens in children when administered at a single daily dose of 50 mg/kg. This drug may be especially useful in those patients in whom outpatient antibiotic therapy is contemplated or in whom maintenance of intravenous access is difficult. [\hyperlink{Ceftriaxone Sodium}{PMID: 3985602}, B L Congeni et al., 1985]

\hypertarget{pmid_6318653}{T}hirty-four patients aged 1 month to 19 years were treated with ceftriaxone for suspected bacterial infections. Bacterial pathogens were isolated from 25 children. The overall bacterial cure rate was 88\%, with an overall clinical response rate of 96\%. No side effects requiring cessation of therapy were observed. Ceftriaxone proved to be safe and effective in the treatment of serious infections in children. [\hyperlink{Ceftriaxone Sodium}{PMID: 6318653}, S C Aronoff et al., 1983]

\hypertarget{pmid_6098699}{C}eftriaxone CTRX was evaluated about its antibacterial activity against clinical isolates at our department and tried clinically in 10 children of 6 months to 10 years and 6 months of age. The antibacterial activity was equal to cefotaxime or higher while the clinical results were almost satisfactory. Three out of 4 strains were eradicated (75\%). As to the adverse reaction, eosinophilia was observed only in 1 case. [\hyperlink{Ceftriaxone Sodium}{PMID: 6098699}, H Hoshina et al., 1984]

\hypertarget{pmid_6098721}{C}eftriaxone (Ro 13-9904, CTRX), developed by F. Hoffmann-La Roche Ltd. in Switzerland, was used for the pediatric infections and the following results were obtained. The mean blood level of CTRX in 2 children after a 60-minute intravenous drip infusion with 20 mg/kg was 58.6 micrograms/ml at 30 minutes, 75.0 micrograms/ml at 1 hour, 39.85 micrograms/ml at 2 hours, 27.74 micrograms/ml at 4 hours, 20.71 micrograms/ml at 6 hours, 11.72 micrograms/ml at 12 hours and 3.91 micrograms/ml at 24 hours while the half-life time was 5.9 hours in one child and 7.6 hours in the other. CTRX was used in 22 children with acute infections consisting of 3 with acute pharyngeal tonsillitis, 4 with acute bronchitis, 8 with bronchopneumonia, 6 with infections of skin soft tissue and 1 with salmonellosis. The results were excellent in 5 cases and good in 17, indicating an efficacy rate of 100\%. Out of 10 cases where the causative strains were detected, 4 cases were followed about the activities of the respective bacteria, i.e., H. influenzae, Streptococcus group A, S. aureus and Salmonella group B, all of which were eradicated after the end of administration. The daily dose of CTRX ranged from 30 to 50 mg/kg and generally a larger dose was used for serious infections. CTRX was administered twice daily in 20 out of 22 cases, by an intravenous injection in 4 and an intravenous drip infusion in 18, for 2 to 4 days in 16 and 5 to 8 1/2 days in 6. No clinical adverse reactions were observed while the laboratory test found a slight elevation of GOT in one and that of GOT and GPT in another. From the above results, CTRX was judged to be a highly useful drug for treatment of pediatric infections. [\hyperlink{Ceftriaxone Sodium}{PMID: 6098721}, M Minamitani et al., 1984]

\hypertarget{pmid_12830336}{C}eftriaxone is a widely used third-generation cephalosporin. It is generally very safe, but complications of biliary pseudolithiasis and, rarely, nephrolithiasis have been reported in children. These complications generally resolve spontaneously with cessation of the ceftriaxone therapy; however, they may symptomatically mimic more serious clinical problems, such as cholecystitis. We report a case of both ceftriaxone-induced biliary pseudolithiasis and nephrolithiasis. [\hyperlink{Ceftriaxone Sodium}{PMID: 12830336}, Jeffrey S Prince et al., 2003]

\hypertarget{pmid_1918222}{C}eftriaxone is generally recognized as safe and effective when used as a single drug in the therapy of septicemia and other serious infections involving bacteremia in both adults and children. An advantage of ceftriaxone over other third-generation cephalosporins is its long serum half-life, which allows the drug to be given every 12 hours in children or less frequently in adults. [\hyperlink{Ceftriaxone Sodium}{PMID: 1918222}, M T Foster et al., 1991]

\hypertarget{pmid_24592804}{C}eftriaxone is a widely used antibiotic in pediatric clinical practice. Usually ceftriaxone is well tolerated and serious adverse effect like anaphylaxis is rare. We report a near fatal anaphylaxis reaction in a child after the first dose of intravenous ceftriaxone who revived successfully. [\hyperlink{Ceftriaxone Sodium}{PMID: 24592804}, D Shrestha et al., 2013]

\hypertarget{pmid_6317277}{P}harmacokinetic variables were studied in children with central nervous system infections who received a single dose of ceftriaxone sodium. After initial lumbar puncture of children with documented or suspected bacterial meningitis, ventriculitis, or both, therapy was initiated with i.v. ampicillin and chloramphenicol. Children were randomly selected to receive a single i.v. dose of ceftriaxone. Concentrations of ceftriaxone were measured in plasma at intervals from 0 to 720 minutes after the beginning of the infusion and in cerebrospinal fluid (CSF) at one to five hours after the dose. Blood samples were obtained immediately after the second lumbar puncture for assessment of drug penetration into CSF. Elimination rate constant, elimination half-life, apparent volume of distribution, and plasma clearance were determined from samples obtained 30-720 minutes after the start of the infusion. In two children with ventriculoperitoneal shunts, serial determinations of ceftriaxone in CSF were obtained. All eight children who received 75 mg/kg and five of eight who received 50 mg/kg had positive CSF cultures. Volume of distribution was less after the 50 mg/kg dose than after the 75 mg/kg dose. In the children with shunts, adequate CSF drug concentrations were maintained throughout 12 hours of testing. These data support a 12-hour dosage interval, but clinical studies are needed to evaluate efficacy of the drug at both 12-hour and 24-hour dosage regimens. [\hyperlink{Ceftriaxone Sodium}{PMID: 6317277}, M D Reed et al., ]

\hypertarget{pmid_7334587}{C}linical evaluation was carried out on cefroxadine dry syrup (containing 100 mg of cefroxadine per 1 g) for child use, and the following results were obtained. 1. Serum levels: Peak serum levels at 1 hour after single administration of CXD 100 mg (9.1 mg/kg) to a 4-year old child (11kg) and 300 mg (12.8 mg/kg) to a 8-year old child (23.5 kg) were 20.32 microgram/ml and 18.75 microgram/ml, respectively. They declined to 0.78 microgram/ml and 0.88 microgram/ml respectively after 6 hours and to undetectable levels after 8 hours. Half-life was 1 hour and 1.2 hours, respectively. CXD has shown the same concentration pattern as CEX, except for the fact that serum levels were peaked after 30 minutes and not detectable after 6 hours. 2. Clinical responses: CXD was administered, for 7 days, to 33 children with scarlet fever in the dosage of greater than or equal 20 approximately less than 60 mg/kg/day (7 children in greater than or equal to 20 approximately 30 mg/kg/day, 21 in greater than or equal to 30 approximately less than 40 mg/kg/day and 5 in greater than or equal to 40 approximately less than 60 mg/kg/day). Clinical responses were excellent in 19 cases and good in 14 cases, with an efficacy rate of 100\%. All strains of group A Streptococcus isolated from the pharynx of 22 children were eradicated within 24 hours. In 1 case each of acute pharyngitis, acute tonsillitis, acute laryngotracheitis and staphylococcal scalded skin syndrome, the dosage of greater than or equal to 30 approximately less than 45 mg/kg/day produced a 100\% good clinical response and eliminated the causative pathogens. 3. Side effect: Only 2 cases of eosinophilia were observed in hematologic study as well as in hepatic and renal function tests before and after administration. [\hyperlink{Ceftriaxone Sodium}{PMID: 7334587}, M Minamitani et al., 1981]

\hypertarget{pmid_24130394}{C}eftriaxone is a commonly used antibiotic in children for various infections like respiratory tract infection, urinary tract infection and enteric fever. Hypersensitive reactions following ceftriaxone therapy are uncommon but are potentially life-threatening. The rash can resemble viral exanthems and may lead to a delay in the recognition and prompt treatment. Here we report a 7-year-old boy who presented with fever and rash with emphasis on recognizing ceftriaxone hypersensitivity and its management.  [\hyperlink{Ceftriaxone Sodium}{PMID: 24130394}, Russelian Arulraj et al., ] The clinical and pharmacokinetic studies of cefroxadine (CXD) dry syrup were conducted, and the following results were obtained. 1. A single dose of CXD either 10 mg/kg or 20 mg/kg was given to 2 patients each, and serum levels were peaked in the range of about 10 to 11 microgram/ml. 2. About 30 mg/kg of CXD per day was administered to 47 infants and children (37; upper and lower respiratory tract infection, 3; urinary tract infection, 7; Others) aged from 6 months to 8 years and 1 month weighing 8 to 29 kg, and a 97.8\% (44 out of 45) of clinical response was obtained except for 2 cases whose efficacies were uncertain. 3. As the drug-induced side effects, only transient loose stool was observed in 2 cases. This, however, allowed to continue the treatment. [\hyperlink{Ceftriaxone Sodium}{PMID: 24130394}, S Nakazawa et al., 1981]

\hypertarget{pmid_32816735}{C}eftriaxone is a third-generation cephalosporin used to treat infants with community-acquired pneumonia. Currently, there is a large variability in the amount of ceftriaxone used for this purpose in this particular age group, and an evidence-based optimal dose is still unavailable. Therefore, we investigated the population pharmacokinetics of ceftriaxone in infants and performed a developmental pharmacokinetic-pharmacodynamic analysis to determine the optimal dose of ceftriaxone for the treatment of infants with community-acquired pneumonia. A prospective, open-label pharmacokinetic study of ceftriaxone was conducted in infants (between 1 month and 2 years of age), adopting an opportunistic sampling strategy to collect blood samples and applying high-performance liquid chromatography to quantify ceftriaxone concentrations. Developmental population pharmacokinetic-pharmacodynamic analysis was conducted using nonlinear mixed effects modeling (NONMEM) software. Sixty-six infants were included, and 169 samples were available for pharmacokinetic analysis. A one-compartment model with first-order elimination matched the data best. Covariate analysis elucidated that age and weight significantly affected ceftriaxone pharmacokinetics. According to the results of a Monte Carlo simulation, with a pharmacokinetic-pharmacodynamic target of a free drug concentration above the MIC during 70\% of the dosing interval (70\%  [\hyperlink{Ceftriaxone Sodium}{PMID: 32816735}, Ya-Kun Wang et al., 2020] Ten children, diagnosed as having typhoid fever, were enrolled in this study between April and September, 1989. Ceftriaxone was administered intravenously, in two dosages adding to 50-100 mg/kg/day over as short a period as five days. The mean period of defervescence was 3.2 days. No adverse reactions to the drug occurred; all those fulfilling the prescribed course were cured. To date, no relapse has been reported nor has any patient become a chronic carrier. Shortterm use of Ceftriaxone had the advantages of rapid response, abscence of serious side effects, and low failure rate in treating children with typhoid fever. [\hyperlink{Ceftriaxone Sodium}{PMID: 32816735}, S L Yen et al., ]

\hypertarget{pmid_1810586}{C}hildren with sickle cell disease have a greatly increased potential for developing rapid and at times fatal sepsis from Streptococcus pneumoniae. Hospitalization and parenteral antibiotic treatment in all febrile children with sickle cell disease have thus become the standard of care at most sickle cell centers. As an alternative approach, we managed selected febrile children with sickle cell disease on an ambulatory basis with parenteral ceftriaxone to determine its safety and effectiveness in preventing sepsis and reducing the number of days of hospitalization. Twenty of 40 children who presented with significant fever met the study criteria and received ceftriaxone on an ambulatory basis. Three were subsequently hospitalized. Compared with a previous year, when all febrile children were admitted, ceftriaxone use reduced the days of hospitalization from 214 (6.3 +/- 1.6 days/patient) to 111 days (2.8 +/- 0.7 days/patient). The empiric use of ceftriaxone appears safe and effective, but it requires an expanded study over an extended period. [\hyperlink{Ceftriaxone Sodium}{PMID: 1810586}, S S Bakshi et al., 1991]

\hypertarget{pmid_4076254}{T}he pharmacokinetics and safety of ceftriaxone were examined in 39 neonates who required antibiotics for clinically suspected sepsis. The drug was administered as a once daily dose of 50 mg/kg by the intravenous (IV) or intramuscular (IM) route. Ceftriaxone was assayed in 49 series of blood samples, 3 samples of cerebrospinal fluid (CSF) and 15 samples of urine by a microbiological technique. Blood was collected before, during and after treatment for biochemical analysis. Routine haematological investigations were also monitored. There was no significant difference between the maximum plasma concentrations following IV (153 +/- 39 mg/l) or IM (141 +/- 19 mg/l) administration (first dose). The mean elimination half-life, total body clearance, and volume of distribution following the first dose were 15.4 +/- 5.4 h, 0.28 +/- 0.12 ml/min per kg and 325 +/- 59 ml/kg respectively. Clearance increased with increasing postnatal age and body temperature (P less than 0.0002) and decreasing plasma creatinine concentration (P less than 0.005). Increasing plasma protein concentration was associated with a decrease in volume of distribution (P less than 0.001). There were no drug-associated changes in any of the biochemical or haematological parameters examined. Ceftriaxone is a safe and well tolerated antibiotic for use in the treatment of newborn sepsis and possibly meningitis. A once daily administration of 50 mg/kg by the IV and IM routes provides satisfactory plasma concentrations throughout the dosage interval whilst avoiding accumulation. [\hyperlink{Ceftriaxone Sodium}{PMID: 4076254}, A Mulhall et al., 1985]

\section*{Cefuroxime Sodium}
\subsection*{Result}
\subsubsection*{Answer}

Yes (Neonates: <1 month)
Yes (Infants and Children: 1 month–12 years)
Yes (Adolescents: 12–17 years)

\subsubsection*{{Explanation}}
\hypertarget{Cefuroxime Sodium}
A review of the available abstracts reveals multiple targeted studies evaluating the safety of Cefuroxime Sodium (and its oral prodrug, cefuroxime axetil) in children across various age ranges. Below is a summary of the relevant evidence, organized by age group:

Neonates (less than 1 month):
- A study of 28 neonates with suspected or proven infection (all with complications at birth or early neonatal life) received intramuscular or intravenous cefuroxime (50 mg/kg twice daily for 5 days). No adverse clinical side effects or laboratory changes attributable to cefuroxime were observed. The study concludes that cefuroxime is a safe, well-tolerated, and rapidly absorbed drug for the treatment of neonates with suspected or proven infections [\hyperlink{pmid_7065695}{PMID: 7065695}, J de Louvois et al., 1982].

Infants and Children (1 month to 12 years):
- An open, multicenter trial involving 304 children aged 3 months to 12 years with acute upper respiratory infections and/or acute otitis media treated with cefuroxime axetil suspension reported a 93\% cure rate and only minor adverse reactions (4.9\%, mostly vomiting). The study concludes cefuroxime axetil suspension was safe and effective in this population [\hyperlink{pmid_8234058}{PMID: 8234058}, J Barliński et al.].
- A study of 84 children aged 3 months to 5 years with community-acquired pneumonia treated with intravenous cefuroxime followed by oral cefuroxime axetil suspension found 97.6\% were cured or improved, with no significant safety concerns reported. The study affirms the safety and efficacy of cefuroxime in this age group [\hyperlink{pmid_8088980}{PMID: 8088980}, I Shalit et al., 1994].
- A randomized trial in 40 children aged 3 months to 5 years with parapneumonic pleural effusion or empyema compared cefuroxime to other antibiotics and found adverse effects attributed to cefuroxime were mild and infrequent, supporting its safety [\hyperlink{pmid_11969360}{PMID: 11969360}, G C Palacios et al., 2002].
- A study of 36 children aged 3 months to 12 years receiving cefuroxime axetil suspension for respiratory or soft-tissue infections found that, while 3 of 35 were withdrawn due to adverse events (one drug-related hypersensitivity), the majority tolerated the drug well, and the study concluded the safety profile was favorable [\hyperlink{pmid_1763541}{PMID: 1763541}, D A Powell et al., 1991].
- A large retrospective analysis of 886 children aged 4 months to 17 years with acute ENT diseases treated with cefuroxime found the therapy to be safe and effective, with a beneficial effect in 98.9\% of patients [\hyperlink{pmid_15732828}{PMID: 15732828}, Magdalena Lapienis et al., 2004].
- A randomized study in children aged 6 months to 12 years with early Lyme disease compared cefuroxime axetil to amoxicillin and found both to be safe and efficacious, with only mild diarrhea as a notable side effect [\hyperlink{pmid_12042561}{PMID: 12042561}, Stephen C Eppes et al., 2002].
- Additional studies in children (age ranges not always specified, but including infants and older children) with various infections (respiratory, urinary, soft-tissue, and meningitis) consistently report good efficacy and no significant side effects attributable to cefuroxime [\hyperlink{pmid_434912}{PMID: 434912}, J A Kuzemko et al., 1979; \hyperlink{pmid_41955}{PMID: 41955}, K Ohnuma et al., 1979; \hyperlink{pmid_390177}{PMID: 390177}, Y Nakamura et al., 1979; \hyperlink{pmid_3877456}{PMID: 3877456}, W J Barson et al., 1985].

Adolescents (12–17 years):
- The large retrospective study mentioned above included children up to 17 years of age and found cefuroxime therapy to be safe and effective [\hyperlink{pmid_15732828}{PMID: 15732828}, Magdalena Lapienis et al., 2004].

Summary:
There are multiple targeted studies, including randomized controlled trials and large retrospective analyses, specifically evaluating the safety of cefuroxime sodium (and cefuroxime axetil) in neonates, infants, children, and adolescents. These studies consistently affirm the safety of cefuroxime sodium in these pediatric populations, with only mild and infrequent adverse effects reported.

\subsection*{Abstracts}
\hypertarget{pmid_513298}{H}aving resistance to beta-lactamase-producing strains and showing resistance to not only cephalosporin resistant strains of E. coli and Klebsiella but also to Citrobacter, Proteus and Enterobacter, Cefuroxime (CXM) was used in pediatric field for both fundamental and clinical studies. CXM was found to be a useful antibiotic in views of high clinical efficacy rate obtained and no side effect noted. As for the dose, the single dose of 25 mg/kg achieved sufficient blood levels. Also in view of good clinical effect, the dose of 25 mg/kg three or four times daily seems appropriate for treatment of children. [\hyperlink{Cefuroxime Sodium}{PMID: 513298}, M Hotta et al., 1979]

\hypertarget{pmid_3531565}{P}harmacokinetic and clinical studies of cefixime (CFIX) in children were done and the following results were obtained. Serum and urinary concentrations of CFIX were determined in 6 children aged 5 to 14 years given single doses of 1.5 or 6.0 mg/kg. Mean serum concentrations peaked at 4 hours after the administration of either 1.5 or 6.0 mg/kg, and respective peak values were 0.71 and 4.46 micrograms/ml. Biological half-lives for the low and the high doses were 5.28 and 4.45 hours, respectively. The 12-hours urinary recovery ranged from 7.0 to 13.8\% after administration of 1.5 mg/kg, and the 8-hours urinary recovery was 18.1\% after administration of 6.0 mg/kg. Therapeutic responses were recorded as excellent or good in 43 (97.7\%) of the children, comprising 13 with tonsillitis and 31 with scarlet fever. The microbiological effectiveness of CFIX on identified pathogens comprising 29 strains of S. pyogenes and 2 strains of S. aureus was satisfactory as evidence by a high eradication rate of 93.5\%. No clinical side effects were observed. Abnormal laboratory findings were elevation of GOT and/or GPT in 4 patients and eosinophilia in 1 patient. In conclusion, CFIX was found to be efficacious and safe for the treatment of bacterial infections in children. [\hyperlink{Cefuroxime Sodium}{PMID: 3531565}, T Nishimura et al., 1986]

\hypertarget{pmid_7933536}{W}e administrated cefodizime (40 mg/kg) to 13 patients with simple herniorrhaphy in the pediatric field and determined its concentrations in tissues and serums. The mean serum and tissue levels of cefodizime after administration were 43.1 +/- 13.3 micrograms/ml, and 23.1 +/- 6.4 micrograms/g, respectively, at 3 hours. Cefodizime concentrations of the tissue and serum were maintained at relatively high levels for many hours. The ratio of cefodizime concentrations in tissue to serum became high at 3 hours after administration, and this suggests that tissue concentrations decreased more slowly than serum levels, and cefodizime concentrations in tissue were maintained at fairly high levels over a long period. No side effects caused by cefodizime were observed. From pharmacokinetic and clinical observations, cefodizime appears to be a safe and effective injectable antibiotic for the treatment of infections in children. [\hyperlink{Cefuroxime Sodium}{PMID: 7933536}, K Matsuura et al., 1994]

\hypertarget{pmid_8234058}{T}he study aimed at assessing the clinical efficiency, safety, and tolerance of cefuroxime axetil suspension in the treatment of children with the acute upper respiratory infections and/or the acute otitis media. The trial was open, multicenter, involving 304 children aged between 3 months and 12 years. They were recruited from 18 general practice centers in Poland. Children were given cefuroxime axetil suspension in the dose of 10 mg/kg body weight (upper respiratory) or 15 mg/kg otitis media. max. 250 mg) bid. Children were examined prior to the treatment, 3-4 days following the start of therapy, 1-2 days after completion of the treatment, and followed-up for 14 days. Post-therapy examination has shown 93\% cure rate. During the follow-up period 0.77\% of patients relapsed. Only minor adverse reactions were reported by 4.9\% of patients. Most common complaint was vomiting. Cefuroxime axetil suspension was safe and effective therapy in the acute upper respiratory infections and the acute otitis media in childhood. [\hyperlink{Cefuroxime Sodium}{PMID: 8234058}, J Barliński et al., ]

\hypertarget{pmid_3866088}{A} clinical trial of ceftizoxime suppositories (CZX-S) was performed to evaluate the therapeutic effectiveness in children with bacterial infection. The subjects were 10 children comprising 4 with pneumonia, 3 with lacunar tonsillitis, 2 with pharyngitis, and 1 with UTI. They were given 1 suppository containing either 125 mg or 250 mg of CZX 2 to 4 times a day. The daily per kg body weight dose ranged from 17.1 to 60.0 mg. The result was "markedly effective" in 3, "effective" in 6, and "failure" was recorded in 1. Bacteriologically, successful eradication of causative organisms was confirmed in all the 4 children who underwent the test. No clinical side effects were observed. The only laboratory test abnormality recorded in a single patient was eosinophilia, which was not definitely ascribable to CZX-S. In conclusion, CZX-S have proved to be a clinically safe and effective antibiotic preparation in infantile infection, even in children whose treatment with conventional antibiotics is associated with difficulties. [\hyperlink{Cefuroxime Sodium}{PMID: 3866088}, T Hosoda et al., 1985]

\hypertarget{pmid_7065695}{T}he new broad spectrum cephalosporin, cefuroxime, was used to treat 28 neonates with suspected or proved infection. All of them had had complications at birth or in early neonatal life which were known to predispose to infection. The treatment regimen consisted of intramuscular or intravenous cefuroxime (50 mg/kg twice a day) for 5 days. Previously, such infants would have received gentamicin with penicillin or ampicillin. Pathogenic or potentially pathogenic bacteria were isolated from  7 (25\%) of them. All of these organisms were sensitive to cefuroxime. None of the babies had meningitis, but blood cultures from 2 gave positive results. There was significant clinical improvement in 27 of them after 5 days of treatment and each was well on discharge from hospital. Serum urea, total protein, albumin, and alanine transaminase levels were estimated before, during, and after cefuroxime treatment. There were no changes attributable to cefuroxime nor were any changes in haemoglobin, packed cell volume, or total differential white cell counts observed. There were no adverse clinical side effects. One hundred and ninety-four samples of serum were assayed for cefuroxime. The mean peak level after intramuscular injection (42.7 mg/l) was reached in 0.8 hours, and the mean trough level was 10.5 mg/l. The mean half-life of cefuroxime in infants aged less than 4 days was 5.8 hours. In 4 infants older than 8 days, it ranged from 1.6-3.8 hours. Half-life was not associated with birthweight. Cefuroxime is a safe, well-tolerated, and rapidly absorbed drug for the treatment of neonates with suspected or proved infections; it is a useful alternative to gentamicin, if the use of an aminoglycoside is not clearly indicated. [\hyperlink{Cefuroxime Sodium}{PMID: 7065695}, J de Louvois et al., 1982]

\hypertarget{pmid_8088980}{F}or children with acute respiratory infections in hospital, it is desirable to transfer from parenteral to oral therapy at the earliest opportunity. The introduction of a pediatric suspension of cefuroxime axetil provides a continuous course of one antibiotic with transition from injectable to oral therapy. This open study was designed to investigate the efficacy of cefuroxime in pediatric patients aged 3 months to 5 years with community-acquired pneumonia. Children had evidence of lobar pneumonia on chest X-ray, a white blood cell count of > 15,000/mm3 and a rectal temperature of > or = 38.5 degrees C on enrollment. Cefuroxime was given by i.v. injection at 75 mg/kg per day in three divided doses for 48-72 h followed by oral cefuroxime suspension at 30 mg/kg per day in two divided doses. Of 84 evaluable patients 82 (97.6\%) were cured or improved post-treatment, and of 74 evaluable children who returned for follow-up assessment 73 (98.6\%) remained well. Oral therapy with twice daily cefuroxime axetil suspension following 2-3 days of i.v. cefuroxime administration was confirmed as effective and safe treatment for lobar pneumonia in children under 5 years of age. [\hyperlink{Cefuroxime Sodium}{PMID: 8088980}, I Shalit et al., 1994]

\hypertarget{pmid_2693753}{C}linical usefulness of cefixime (CFIX), a new oral cephalosporin antibiotic, in pediatric field was investigated. The results obtained were summarized as follows. 1. The clinical efficacy of CFIX was investigated in a total of 138 children including 49 with upper respiratory tract infections (RTI), 22 with acute bronchitis, 18 with pneumonia, 19 with scarlet fever and 21 with urinary tract infections (UTI). 2. Clinical effectiveness was excellent in 58, good in 60, fair in 14 and poor in 3, with an overall efficacy rate of 87.4\%. The efficacy rate classified according to types of infection were 85.7\% in upper RTI, 89.5\% in acute bronchitis, 94.4\% in pneumonia, 78.9\% in scarlet fever, and 90.5\% in UTI. 3. Out of the suspected causative organisms, 43 strains of a total of 50 strains isolated were eradicated. The bacteriological eradication rate was 86.0\%. (Haemophilus influenzae 100\%, Haemophilus parainfluenzae 100\%, Streptococcus pyogenes 88.5\%, Escherichia coli 85.7\%). 4. One hundred forty four children were analyzed for side effect. Side effects were observed in 2 children (1.4\%) with diarrhea in 1 and anorexia in another. Abnormal laboratory test results were recorded in 4 children (3.3\%). The above results suggest that CFIX is a very useful new oral cephalosporin for the treatment of bacterial infections in children. [\hyperlink{Cefuroxime Sodium}{PMID: 2693753}, H Mikawa et al., 1989]

\hypertarget{pmid_434912}{C}efuroxime (25 mg/kg) given intravenously every four hours to 7 children with bacterial meningitis resulted in satisfactory therapeutic blood and CSF levels. All children made a full recovery and side effects were absent. [\hyperlink{Cefuroxime Sodium}{PMID: 434912}, J A Kuzemko et al., 1979]

\hypertarget{pmid_3761541}{W}e used cefixime (CFIX), a newly developed oral cephalosporin antibiotic, to treat 21 children with various infections. The results are summarized as follows. The serum half-lives of CFIX after an administration of 6 mg/kg to each of 2 children were 2.56 and 2.79 hours. The serum concentrations were high enough to ensure the therapeutic response. The clinical response was "excellent" in 16 children and "good" in 5, with a 100\% efficacy rate. No side effects were recorded. The only abnormal finding was slight eosinophilia in 1 child. [\hyperlink{Cefuroxime Sodium}{PMID: 3761541}, S Furukawa et al., 1986]

\hypertarget{pmid_3877456}{A}lthough it is used extensively in Europe, there is a limited amount of published data concerning pediatric clinical experience with cefuroxime in the United States. Thirty-six children, ranging from 3.5 to 57 months of age, received intravenous cefuroxime (75 mg/kg/day in three divided doses) for soft-tissue infections of the face or epiglottis. Infections treated included preseptal (19 patients) and buccal (13 patients) cellulitis and epiglottitis (four patients). Blood cultures were positive in 22 patients, yielding Haemophilus influenzae type b in 17 (four were beta-lactamase-positive), Streptococcus pneumoniae in four; and beta-lactamase-positive, nontypable H influenzae in one. An additional five patients with buccal cellulitis had negative blood cultures but H influenzae type b antigenuria. A satisfactory clinical response was noted in all patients, and repeated blood cultures performed in initially bacteremic patients were sterile. Cefuroxime therapy was well tolerated, and abnormal laboratory results were infrequent, except for absolute granulocytopenia (granulocytes, less than 1,500/cu mm), which occurred in six patients but could not be ascribed to a drug effect because of the uncontrolled design of our study. Treatment with cefuroxime appears to be a safe and effective therapy for pediatric soft-tissue infections due to H influenzae and S pneumoniae. [\hyperlink{Cefuroxime Sodium}{PMID: 3877456}, W J Barson et al., 1985]

\hypertarget{pmid_15732828}{T}he authors present results of retrospective clinical analysis of usefulness the cefuroxime therapy of acute ENT diseases in children. The study group consist of 886 patients, aged 4 m. to 17 year, hospitalized at the Department of Paediatric Otolaryngology between 1997-2002. The efficacy of therapy was estimated on the ground of 4 degree scale. Particular attention was paid on measuring an average time of intravenous and oral administration of drug and on side effects of treatment. The results of the study shown that cefuroxime therapy is safe and effective. Beneficial therapeutic effect was obtained in 98.9\% of patients. [\hyperlink{Cefuroxime Sodium}{PMID: 15732828}, Magdalena Lapienis et al., 2004]

\hypertarget{pmid_28827252}{C}eftriaxone is widely used in children in the treatment of sepsis. However, concerns have been raised about the safety of ceftriaxone, especially in young children. The aim of this review is to systematically evaluate the safety of ceftriaxone in children of all age groups. MEDLINE, PubMed, Cochrane Central Register of Controlled Trials, EMBASE, CINAHL, International Pharmaceutical Abstracts and adverse drug reaction (ADR) monitoring systems will be systematically searched for randomised controlled trials (RCTs), cohort studies, case-control studies, cross-sectional studies, case series and case reports evaluating the safety of ceftriaxone in children. The Cochrane risk of bias tool, Newcastle-Ottawa and quality assessment tools developed by the National Institutes of Health will be used for quality assessment. Meta-analysis of the incidence of ADRs from RCTs and prospective studies will be done. Subgroup analyses will be performed for age and dosage regimen. Formal ethical approval is not required as no primary data are collected. This systematic review will be disseminated through a peer-reviewed publication and at conference meetings. CRD42017055428. [\hyperlink{Cefuroxime Sodium}{PMID: 28827252}, Linan Zeng et al., 2017]

\hypertarget{pmid_3494006}{C}efuroxime axetil tablets were given to 12 children (aged 19 months to 13.5 years) for a total of 14 episodes of lower respiratory tract infection. Doses ranged from 15 to 32 mg/kg/day. Six infections were regarded as cured and seven improved. In four cases, Haemophilus influenzae was present at the end of treatment. Serum levels of cefuroxime showed great variability. Absorption and penetration of the drug into the lower respiratory mucosa may not be sufficient to kill organisms which are sensitive in vitro. Cefuroxime axetil tablets were acceptable to most children. [\hyperlink{Cefuroxime Sodium}{PMID: 3494006}, J W Carson et al., 1987]

\hypertarget{pmid_41955}{C}efuroxime, a new synthetic cephalosporin, was administered to 10 pediatric patients (6 with respiratory tract infection, 2 with urinary tract infection, 1 with sepsis of E. coli and 1 with enterocolitis). The clinical result was good and excellent in all the 10 cases. No side effect was observed in any of them. [\hyperlink{Cefuroxime Sodium}{PMID: 41955}, K Ohnuma et al., 1979]

\hypertarget{pmid_34834338}{C}efixime (CEF) is a cephalosporin included in the WHO Model List of Essential Medicines for Children. Liquid formulations are considered the best choice for pediatric use, due to their great ease of administration and dose-adaptability. Owing to its very low aqueous solubility and poor stability, CEF is only available as a powder for oral suspensions, which can lead to reduced compliance by children, due to its unpleasant texture and taste, and possible non-homogeneous dosage. The aim of this work was to develop an oral pediatric CEF solution endowed with good palatability, exploiting the solubilizing and taste-masking properties of cyclodextrins (CDs), joined to the use of amino acids as an auxiliary third component. Solubility studies indicated sulfobutylether-β-cyclodextrin (SBEβCD) and Histidine (His) as the most effective CD and amino acid, respectively, even though no synergistic effect on drug solubility improvement by their combined use was found. Molecular Dynamic and  [\hyperlink{Cefuroxime Sodium}{PMID: 34834338}, Marzia Cirri et al., 2021] Cefixime (CFIX), a new oral cephalosporin, was administered clinically at a daily dose of 3.4 mg/kg to 10.4 mg/kg to each of 12 children, aged from 2 months to 14 years old. An additional separate study was done to compare the serum and urinary levels of CFIX in 3 children when each was administered with 100 mg of the drug in capsule with the serum and urinary levels of the drug in the same children when each was given the same amount of drug in the form of 5\% granules. The results of these trials are summarized below. Peak serum levels of CFIX administered in capsules and 5\% granules averaged 1.4 micrograms/ml and 1.9 micrograms/ml, respectively. The half-life of the former was 5.13 hours, while that of the latter was 4.17 hours. The difference in the peak levels was statistically insignificant. The urinary excretion of CFIX in either form of the drug (capsules and granules) was about 14-18\% in 12 hours. In 9 cases of respiratory infections, therapeutic results were excellent in 3 cases, good in 6 cases, and the effective rate was 100\%. In 2 cases of urinary tract infection, results were excellent in 1 case and good in 1 case. The drug efficacy was poor in 1 case of purulent cervical lymphadenitis, probably caused by Staphylococcus aureus. No adverse reactions attributable to the drug were observed. CFIX may be expected to be a highly effective and safe agent in moderate respiratory and urinary tract infections of children. [\hyperlink{Cefuroxime Sodium}{PMID: 34834338}, N Nakayama et al., 1986]

\hypertarget{pmid_1763541}{T}he tolerability, safety, and efficacy of cefuroxime axetil suspension was studied in 36 children (aged 3 mo to 12 y) who had been hospitalized for respiratory tract or soft-tissue infections. After receiving parenteral antibiotics for a mean of 3.7 days, children were discharged home to receive cefuroxime axetil suspension at doses of 10, 15, or 20 mg/kg every 8 or 12 hours for a mean of 8.2 days. One child was lost to follow-up. Three of 35 evaluated patients were withdrawn from therapy because of adverse events, one of which was a drug-related hypersensitivity reaction. Of the 32 children who completed therapy, 9 developed mild reactions including oral thrush, diarrhea, or diaper dermatitis; none were withdrawn from therapy. Complete clinical cure occurred in 28 children (80 percent); 4 (11.4 percent) were clinically improved but still required an additional antibiotic within one week of completing therapy with cefuroxime axetil suspension. This favorable tolerability and safety of cefuroxime axetil suspension warrants further efficacy trials in pediatric patients. [\hyperlink{Cefuroxime Sodium}{PMID: 1763541}, D A Powell et al., 1991]

\hypertarget{pmid_390177}{C}efuroxime, a new cephalosporin C antibiotic, was administered to 15 children with respiratory tract infection, urinary tract infection, or subcutaneous tumour. The following results were obtained. 1) CXM 30 approximately 100 mg/kg/day were used in treatment of respiratory tract infection. Eight of the eleven patients treated responded to the therapy. 2) CXM 45 approximately 75 mg/kg/day were given to 3 patients with urinary tract infection. Excellent results were obtained in all these cases. 3) One patient with subcutaneous tumour responded to CXM therapy. 4) Clinical isolates from the foci involved, i.e., Staphylococcus aureus (4 strains), Group A Streptococcus hemolyticus (1 strain), Streptococcus pneumoniae (1 strain), Haemophilus influenzae (1 strain), and Escherichia coli (3 strains) were all eliminated by CXM therapy except 2 unassessable strains. 5) No noteworthy side effect was noted. [\hyperlink{Cefuroxime Sodium}{PMID: 390177}, Y Nakamura et al., 1979]

\hypertarget{pmid_37698043}{T}he World Health Organization recommends that infants be exclusively breastfed for the first 6 months. Antibiotics are among the most commonly prescribed drugs for pregnant and lactating women. The vast majority of drugs pass into breast milk, which may create a risk for the infant. In cases where drug exposure may pose a risk, breastfeeding should be discontinued. Therefore, the mother's drug use should be decided by considering the most accurate and recent data. Cefuroxime is a second-generation cephalosporin antibiotic with a broad spectrum of activity against Gram-negative and -positive microorganisms. In this study, we aimed to develop the LC-MS/MS method using salt-assisted liquid-liquid micro-extraction (SALLME) for the determination of cefuroxime in breast milk. The method was validated according to the European Medicines Agency (EMA) guidelines. Cefuroxime and the internal standard cefixime were extracted from plasma by a SALLME technique. The results obtained from the entire validation study are at an acceptable level according to the EMA criteria. The calibration curve of cefuroxime was between 25 and 1000 ng/ml, with correlation coefficients of >0.99. The lower limit of quantitation was 25 ng/ml for cefuroxime. Furthermore, the developed method was applied for the determination of cefuroxime in real patient breast milk. [\hyperlink{Cefuroxime Sodium}{PMID: 37698043}, Aykut Kul et al., 2023]

\hypertarget{pmid_1571464}{C}efotaxime has been used to treat serious bacterial infections in children since 1982. With the predominant use of cephalosporins in pediatrics, reports of adverse effects of certain compounds have increased. A retrospective review is presented of 2,243 cases of children receiving therapy with cefotaxime in order to evaluate the safety profile and efficacy of cefotaxime in the treatment of serious infections in hospitalized children. Overall, 57 (2.5\%) children experienced adverse reactions. These included local reactions in 6 (0.3\%), rash in 28 (1.2\%), diarrhea in 15 (0.97\%), vomiting in 10 (0.7\%), abdominal pain in 1 (0.1\%), headache in 3 (0.4\%), and drug fever in 1 (0.1\%). No cases of hemolytic anemia, bleeding, or hyperbilirubinemia were found. Efficacy of treatment for different disease categories ranged from 90.5\% to 100\%. The percentage of children in any treatment group with a particular laboratory abnormality following initiation of cefotaxime therapy ranged from 0\% to 2.6\%, and rates of superinfection with bacteria or Candida were 0.4\% to 1.7\%. Cefotaxime has the distinct advantage of high rates of efficacy and low rates of complications and superinfection among children hospitalized for serious infections. [\hyperlink{Cefuroxime Sodium}{PMID: 1571464}, R F Jacobs et al., 1992]

\hypertarget{pmid_10496153}{A}n open-labeled and randomized trial was conducted to compare the efficacy and safety of once daily cefpodoxime proxetil suspension (10mg/kg/day) and thrice daily cefaclor (45mg/kg/day) in the treatment of acute otitis media in children. A total of 57 children aged from 6 months to 9 years were enrolled; 23 were treated with cefpodoxime and 34 with cefaclor. Satisfactory clinical outcome, either cure or improvement, was achieved at the end of treatment in 90\% of patients in the cefaclor group and 95\% of patients in the cefpodoxime group (p > 0.05). Clinical recurrence was identified at the follow-up visits in one case of the cefaclor group (3\%), and none in the cefpodoxime group (p > 0.05). These drugs were well tolerated by 14/21 (67\%) in the cefpodoxime-treated group and 27/32 (84\%) in the cefaclor-treated group. The incidence of adverse events was slightly higher in the cefpodoxime group than in the cefaclor group, however the difference did not reach statistical significance (p > 0.05). The daily cost of once-daily cefpodoxime was lower than that of thrice-daily cefaclor. We conclude that cefpodoxime administered once daily is as effective and safe as cefaclor administered thrice daily in the treatment of acute otitis media in children. The less dosing frequency and lower daily price of cefpodoxime provide additional benefits. [\hyperlink{Cefuroxime Sodium}{PMID: 10496153}, H Y Tsai et al., 1998]

\hypertarget{pmid_3761545}{C}efixime (CFIX) was evaluated for pharmacokinetics, therapeutic effectiveness on infection, safety, and bacteriological effectiveness in pediatrics. The following is a summary of the results. Pharmacokinetics in 4 children, 2 each receiving a single dose of 1.5 mg or 6.0 mg per kg body weight, were examined. Peak serum CFIX concentrations after the dose of 1.5 mg/kg were 1.12 and 1.34 micrograms/ml, and the serum half-lives were 1.83 and 3.53 hours. For the children administered with 6.0 mg/kg of CFIX, the respective figures were 2.50 and 7.46 micrograms/ml, and 6.77 and 6.64 hours. The 12-hour urinary recoveries were 4.9 and 34.1\% and 9.4 and 25.4\% for the small and the large doses, respectively. Therapeutic effectiveness in 19 children with infections was "excellent" in 14 and "good" in 5, with an effectiveness rate of 100\%. Bacteriological effectiveness was evaluated in 10 children. Classified by causative organisms, 5 cases had H. influenzae, 2 each H. parainfluenzae and S. pyogenes, and 1 mixed infection by H. influenzae and S. pneumoniae. Only the H. influenzae in the child with mixed infection resisted the therapy, and all the other pathogens were successfully eradicated. No side effects were recorded. The only abnormal laboratory test finding attributed to CFIX was eosinophilia in 2 children. [\hyperlink{Cefuroxime Sodium}{PMID: 3761545}, M Miyazaki et al., 1986]

\hypertarget{pmid_12042561}{C}efuroxime axetil has been shown to have efficacy comparable to doxycycline in adults with early Lyme disease (LD). Because of toxicity, doxycycline is usually avoided in children. For children who are unable to tolerate amoxicillin, there is currently no proven alternative oral therapy for LD. This randomized, unblinded study compared 2 dosage regimens of cefuroxime axetil (20 mg/kg/d and 30 mg/kg/d) with amoxicillin (50 mg/kg/d), each given for 20 days. Children were enrolled if they were 6 months to 12 years of age, had erythema migrans, and met other eligibility requirements. Serologic testing occurred at entry and after 6 months. Follow-up evaluations for safety, tolerability, and efficacy occurred at 10 and 20 days, 6 months, and 1 year. Forty-three children were randomized (13 in the amoxicillin group, 15 in each cefuroxime axetil group); 39 completed 12 months of follow-up. At the completion of treatment, there was total resolution of erythema migrans in 67\% of the amoxicillin group, 92\% of the low-dose cefuroxime group, and 87\% of the high-dose cefuroxime group, and resolution of constitutional symptoms occurred in 100\%, 69\%, and 87\%, respectively. All patients had a good outcome, with no long-term problems associated with LD. One patient, who was well at the first 2 follow-up visits, was treated with doxycycline because of new constitutional symptoms. Mild diarrhea occurred in a small number of participants in each group (1 patient was diagnosed and treated for Clostridium difficile-associated diarrhea, which occurred after completing the full course of study medication). No hypersensitivity reactions occurred. The number of patients in this trial was not sufficient to demonstrate a statistically significant difference between the 3 groups; however, both amoxicillin and cefuroxime axetil seem to be safe, efficacious treatments for children with early LD. [\hyperlink{Cefuroxime Sodium}{PMID: 12042561}, Stephen C Eppes et al., 2002]

\hypertarget{pmid_11969360}{T}he aim of this study was to evaluate the efficacy of cefuroxime, compared with the combination of dicloxacillin/chloramphenicol, for the treatment of children with parapneumonic pleural effusion or empyema. Forty patients, aged 3 months to 5 years, with pleural effusion or empyema were randomized to receive cefuroxime (100 mg/kg/day) IV (n=20) or chloramphenicol (100 mg/kg/day) plus dicloxacillin (200 mg/kg/day) IV (n=20). Both groups were similar in age, days of illness, clinical and radiological findings, and etiology. Most patients (70\%) had an empyema at presentation. There was no difference in clinical outcomes, including days to defervescence, duration of respiratory distress, duration of chest tube drainage, and days to discharge from hospital. The complication rates were similar in both groups. Pleural thickening occurred in four patients, bronchopleural fistula in two, and loculated empyema in one patient of each treatment group. Adverse effects attributed to cefuroxime were mild and infrequent. These results suggest that cefuroxime is an effective and well-tolerated alternative for the treatment of children with pleural effusion and empyema. [\hyperlink{Cefuroxime Sodium}{PMID: 11969360}, G C Palacios et al., 2002]

\section*{Chloroprocaine Hydrochloride}
\subsection*{Result}
\subsubsection*{Answer}

Yes (infants ≤6 months, for regional anesthesia)
Unknown (children >6 months, for other uses or repeated/prolonged dosing)

\subsubsection*{{Explanation}}
\hypertarget{Chloroprocaine Hydrochloride}
A review of the available abstracts reveals several that specifically address the safety of Chloroprocaine Hydrochloride (2-chloroprocaine) in children:

1. **Infants (≤6 months):**
   - A retrospective cohort study compared continuous epidural infusions of 1.5\% 2-chloroprocaine to 0.1\% ropivacaine in full-term infants 6 months or younger following thoracotomy. The study found that 2-chloroprocaine was not inferior to ropivacaine in terms of analgesia, and no significant safety concerns were reported. The study supports the use of 2-chloroprocaine for continuous epidural infusion in this age group [\hyperlink{pmid_26306545}{PMID: 26306545}, Wallis T Muhly et al., 2015].
   - Another case report described a brief episode of local anesthetic systemic toxicity in an infant after 3\% 2-chloroprocaine via paravertebral catheter, with complete recovery and no long-term effects. The authors note that such events are infrequent and support continued use in infants, with caution regarding volume [\hyperlink{pmid_27089835}{PMID: 27089835}, Maria A Hernandez et al., 2016].
   - A review article summarizes the literature on 2-chloroprocaine for regional anesthesia in infants and children, noting its rapid metabolism and attractiveness for use in neonates and infants, especially when hepatic function is a concern. The review presents dosing regimens and discusses applications, but does not report new safety data [\hyperlink{pmid_28321983}{PMID: 28321983}, Giorgio Veneziano et al., 2017].

2. **Children (general pediatric population):**
   - An early case series describes the use of chloroprocaine for epidural anesthesia in five pediatric patients (including neonates and infants) with various surgical indications. No complications related to chloroprocaine were observed, and the authors suggest it is an acceptable alternative to bupivacaine in the pediatric population [\hyperlink{pmid_7740909}{PMID: 7740909}, J D Tobias et al., 1995].
   - A preclinical study in juvenile rats evaluated the safety of single maximum tolerated doses of intrathecal 2-chloroprocaine at different developmental stages (postnatal days 7, 14, 21). No evidence of developmental neurotoxicity was found, but the authors caution that results cannot be extrapolated to repeated dosing or prolonged infusion [\hyperlink{pmid_34801587}{PMID: 34801587}, Suellen M Walker et al., 2022].

**Summary by Age Range:**
- For infants (≤6 months), there is targeted clinical evidence supporting the safety of chloroprocaine hydrochloride for continuous epidural infusion, with no significant adverse events reported in the studies reviewed.
- For the general pediatric population (including neonates, infants, and children), small case series and reviews suggest safety when used for regional anesthesia, but the data are limited in size and scope.
- There is no evidence from the abstracts of targeted studies showing chloroprocaine hydrochloride is unsafe in children.
- There is no evidence from the abstracts of targeted studies in older children (beyond infancy) or for repeated/prolonged use beyond single or short-term dosing.

**Conclusion:** Based on the abstracts, chloroprocaine hydrochloride appears to be safe for use in infants (≤6 months) for regional anesthesia, with some supportive evidence in the broader pediatric population, but data are limited and primarily from small studies or case series. Safety for repeated or prolonged use is unknown.

\subsection*{Abstracts}
\hypertarget{pmid_7740909}{T}he authors discuss their experience with chloroprocaine for epidural anesthesia in five pediatric patients. While bupivacaine remains the most commonly used local anesthetic in children, recent reports of toxicity document the risks of this agent. The major advantage of chloroprocaine is its rapid metabolism, which thereby minimizes the risks of toxicity, especially in patients with preexisting problems such as young age or underlying hepatic dysfunction, which may limit the metabolism of local anesthetics of the amide class. In three cases, the epidural infusion was combined with the general anesthetic. The cases included hepatic resection, repair of bladder exstrophy, and correction of duodenal atresia. In two other cases, epidural anesthesia was used instead of general anesthesia in a former preterm infant who was undergoing inguinal herniorrhaphy and for lower extremity orthopedic procedures in a child with myotonic dystrophy. In all cases, chloroprocaine was chosen because of preexisting or associated conditions that might increase the risk of bupivacaine toxicity, such as hepatic resection, repeated dosing in a neonate, or the need for higher concentrations of local anesthetic to achieve adequate surgical conditions. Adequate intraoperative conditions were achieved in all five patients. No complications related to chloroprocaine epidural anesthesia were noted. This initial experience suggests that chloroprocaine offers an acceptable alternative to bupivacaine for epidural anesthesia in the pediatric population. [\hyperlink{Chloroprocaine Hydrochloride}{PMID: 7740909}, J D Tobias et al., 1995]

\hypertarget{pmid_10851644}{C}iprofloxacin clinical and bacteriological efficacies, as well as tolerability mainly with respect to chondrotoxicity were evaluated in the treatment of children with mucoviscidosis. The drug was shown to have high clinical and moderate bacteriological efficacies. As for its tolerability, adverse reactions chiefly associated with affection of the gastrointestinal tract, i.e. nausea, stomach pain, diarrhea, increased transaminase levels were recorded. The arthrotoxicity episode was single and transitory. The use of ciprofloxacin had no negative effect on the children growth. No chondrotoxic effect of ciprofloxacin in the treatment of children was observed which is explained in the paper. It is concluded that ciprofloxacin is in general an efficient and safe antibiotic useful for the treatment of children. [\hyperlink{Chloroprocaine Hydrochloride}{PMID: 10851644}, S S Postnikov et al., 2000]

\hypertarget{pmid_2402648}{C}hloral hydrate has been used extensively to sedate children, but at Brooke Army Medical Center, other drug combinations were becoming increasingly popular due to a perception that chloral hydrate had a high rate of failure, especially with younger or neurologically impaired children. Therefore, 50 children were given the drug before a diagnostic study, and patient data and a sedation score were recorded on a worksheet. Of 50 children, 43 (86\%) were "successfully sedated" on the first attempt with no side effects. Children with neurologic disorders had a much greater (27\% vs 4\%) failure rate than "normal" children. The sedation rate did not significantly differ by age, sex, or initial drug dosage. The study suggest that chloral hydrate is a safe and effective oral sedative but that children with neurologic disorders may need alternative drugs for sedation. [\hyperlink{Chloroprocaine Hydrochloride}{PMID: 2402648}, P D Rumm et al., 1990]

\hypertarget{pmid_21531030}{C}hloral hydrate (CH) is an oral sedative widely used to sedate infants and young children during auditory brainstem response (ABR) testing. The aim of this study was to record effectiveness, complications and safety of CH as a sedative for ABR. From January of 2003 until December of 2007, 1903 children were tested for ABR, 568 of them being under the age of 6 months. CH (8\%) was used for sedation at a dose of 40 mg/kg with a repeat dose, if necessary, for an adequate sedation, in 20-30 min. We recorded the effectiveness of CH as a sedative for ABR examination, as well as all complications related to the use of CH such as vomiting, rash, hyperactivity, respiratory distress and apnea. The statistical method used was the absolute and percentage frequency distribution of the occurrences. Sedation with CH was necessary to perform testing in 1591 (83.6\%) of the examined children. However, in the population of the examined infants, only 341 (60\%) were sedated with CH, because the remaining 227 (40\%) fell asleep by themselves. Complications included hyperactivity in 152 children (8\%), minor respiratory distress in 10 children (0.4\%), vomiting in 217 children (11.4\%), apnea in 4 children (0.2\%) and rash in 10 children (0.4\%). The complications of hyperactivity, vomiting and rash resolved without any medical treatment. The apnea cases were managed effectively by supplying ventilation to the children via a mask in the presence of an anesthesiologist. The use of CH at a dose of 40 mg/kg up to 80 mg/kg is safe and effective when administered in a setting with adequate equipment and the presence of well trained personnel. [\hyperlink{Chloroprocaine Hydrochloride}{PMID: 21531030}, Eirini Avlonitou et al., 2011]

\hypertarget{pmid_2026812}{C}hloral hydrate is commonly used to sedate children before CT. However, no prospective study has been published of the safety and efficacy of chloral hydrate at high dose levels for children undergoing CT. We define high dose levels of oral chloral hydrate to be 80-100 mg/kg, with a maximum total dose of 2 g. High dose chloral hydrate sedation was administered orally to 295 children for 326 CT examinations. Adverse reactions occurred in 7\% of the children, with vomiting being the most common (4.3\% of children). Hyperactivity and respiratory symptoms each occurred in less than 2\% of children. Prolonged sedation ( greater than 2 h) was not encountered in our series. Sedation was successful in producing motion free CT examinations, so that in 303 (93\%) of the cases, no repeat CT scans were needed. We conclude that high dose oral chloral hydrate provides safe and effective sedation for children undergoing CT. [\hyperlink{Chloroprocaine Hydrochloride}{PMID: 2026812}, S B Greenberg et al., ]

\hypertarget{pmid_28827252}{C}eftriaxone is widely used in children in the treatment of sepsis. However, concerns have been raised about the safety of ceftriaxone, especially in young children. The aim of this review is to systematically evaluate the safety of ceftriaxone in children of all age groups. MEDLINE, PubMed, Cochrane Central Register of Controlled Trials, EMBASE, CINAHL, International Pharmaceutical Abstracts and adverse drug reaction (ADR) monitoring systems will be systematically searched for randomised controlled trials (RCTs), cohort studies, case-control studies, cross-sectional studies, case series and case reports evaluating the safety of ceftriaxone in children. The Cochrane risk of bias tool, Newcastle-Ottawa and quality assessment tools developed by the National Institutes of Health will be used for quality assessment. Meta-analysis of the incidence of ADRs from RCTs and prospective studies will be done. Subgroup analyses will be performed for age and dosage regimen. Formal ethical approval is not required as no primary data are collected. This systematic review will be disseminated through a peer-reviewed publication and at conference meetings. CRD42017055428. [\hyperlink{Chloroprocaine Hydrochloride}{PMID: 28827252}, Linan Zeng et al., 2017]

\hypertarget{pmid_28741653}{C}hloral hydrate is commonly used to sedate children for painless procedures. Children may recover more quickly after sedation with dexmedetomidine, which has a shorter half-life. We randomly allocated 196 children to chloral hydrate syrup 50 mg.kg [\hyperlink{Chloroprocaine Hydrochloride}{PMID: 28741653}, V M Yuen et al., 2017] Chloral hydrate (CH), as a sedation agent, is widely used in children for diagnostic or therapeutic procedures. However, it has not come into the market and is currently only used as hospital preparation in China. This review aims to systematically evaluate the efficacy of CH in children of all age groups for sedation before medical procedures. Seven electronic databases and three clinical trial registry platforms were searched and the deadline was September 2018. Randomized controlled trials (RCTs) evaluating the efficacy of CH for sedation in children were included by two reviewers. The extracted information included success rate of sedation, sedation latency and sedation duration. The Cochrane risk of bias tool was applied to assess the risk of bias. The outcomes were analyzed by Review Manager 5.3 software and expressed as relative risks (RR) or Mean Difference (MD) with 95\% confidence interval (CI). Heterogeneity was assessed with I-squared (I A total of 24 RCTs involving 3564 children of CH for sedation were included in the meta-analysis. Compared to placebo group, CH group had a significant increase in success rate of sedation when used for painless and painful procedure (RR=4.15, 95\% CI [1.21, 14.24], P=0.02; RR=1.28, 95\% CI [1.17, 1.40], P<0.01), which included 22 and 455 children for this analysis, respectively. Compared to midazolam group, CH group had a significant increase in success rate of sedation (RR=1.63, 95\% CI [1.48, 1.79], I From the extrapolation of the existing literature, CH oral solution is an appropriate effective alternative for sedation in pediatrics. [\hyperlink{Chloroprocaine Hydrochloride}{PMID: 28741653}, Zhe Chen et al., 2019]

\hypertarget{pmid_24445981}{T}o compare efficacy and safety of chloral hydrate (CH), chloral hydrate and promethazine (CH + P) and chloral hydrate and hydroxyzine (CH + H) in electroencephalography (EEG) sedation. In a parallel single-blinded randomized clinical trial, ninety 1-7 y-old uncooperative kids who were referred to Pediatric Neurology Clinic of Shahid Sadoughi University, Yazd, Iran from April through August 2012, were randomly assigned to receive 40 mg/kg of chloral hydrate or 40 mg/kg of chloral hydrate and 1 mg/kg of promethazine or 40 mg/kg of chloral hydrate and 2 mg/kg of hydroxyzine. The primary endpoint was efficacy in sufficient sedation (obtaining four Ramsay sedation score) and successful completion of EEG. Secondary endpoint was clinical adverse events. Thirty nine girls (43.3 \%) and 51 boys (56.7 \%) with mean age of 3.34 ± 1.47 y were assessed. Sufficient sedation and completion of EEG were achieved in 70 \% (N = 21) of chloral hydrate group, in 83.3 \% (N = 25) of CH + H group and in 96.7 \% (N = 29) of CH + P group (p = 0.02). Mild clinical adverse events including vomiting [16.7 \% (N = 5) in CH, 6.7 \% (N = 2) in CH + P, 6.7 \% (N = 2) in CH + H], agitation in 3.3 \% of CH + P (N = 1) group and mild transient hypotension in 3.3 \% of CH + H (N = 1) group occurred. Safety of these three sedation regimens was not statistically significant different (p = 0.14). Combination of chloral hydrate-antihistamines can be used as the most effective and safe sedation regimen in drug induced sleep electroencephalography of kids. [\hyperlink{Chloroprocaine Hydrochloride}{PMID: 24445981}, Razieh Fallah et al., 2014]

\hypertarget{pmid_22246409}{C}hloral hydrate (CH) is safe and effective for sedation of suitable children. The purpose of this study was to assess whether adequate sedation is achieved with reduced CH doses. We retrospectively recorded outpatient CH sedations over 1 year. We defined standard doses of CH as 50 mg/kg (infants) and 75 mg/kg (children >1 year). A reduced dose was defined as at least 20\% lower than the standard dose. In total, 653 children received CH sedation (age, 1 month-3 years 10 months), 42\% were given a reduced initial dose. Augmentation dose was required in 10.9\% of all children, and in a higher proportion of children >1 year (15.7\%) compared to infants (5.7\%; P < 0.001). Sedation was successful in 96.7\%, and more frequently successful in infants (98.3\%) than children >1 year (95.3\%; P = 0.03). A reduced initial dose had no negative effect on outcome (P = 0.19) or time to sedation. No significant complications were seen. We advocate sedation with reduced CH doses (40 mg/kg for infants; 60 mg/kg for children >1 year of age) for outpatient imaging procedures when the child is judged to be quiet or sleepy on arrival. [\hyperlink{Chloroprocaine Hydrochloride}{PMID: 22246409}, Jennifer Bracken et al., 2012]

\hypertarget{pmid_16520840}{C}hloral hydrate is generally considered to be a safe hypnotic drug, and is commonly used for short-term sedation before diagnostic procedures. Its irritant actions to the mucous membranes are usually limited. We report a rare complication of chloral hydrate overdose in an infant. An 8-month-old male infant became unconscious and required ventilation support after an overdose of chloral hydrate was administered to provide sedation for an ophthalmologic examination. White plaques and sloughing of the oropharyngeal mucosa were observed on the next day. Esophagogastroscopy revealed severe corrosive lesions on the whole esophagus. The child recovered after supportive treatment and his oral intake remained well without dysphagia after 1 year. This report illustrates the potential corrosive effect of chloral hydrate. Strict attention should be paid to the dosing and administration protocol of chloral hydrate in infants. The condition of the oropharyngeal mucosa should be carefully monitored after chloral hydrate administration. [\hyperlink{Chloroprocaine Hydrochloride}{PMID: 16520840}, Yu-Cheng Lin et al., 2006]

\hypertarget{pmid_26306545}{C}ontinuous thoracic epidural analgesia is useful in the management of infants following thoracotomy. Concerns about drug accumulation and toxicity limit the amount of amide local anesthetics that can be delivered. Continuous epidural infusions of the ester local anesthetic chloroprocaine result in little drug accumulation allowing for higher infusion rates. We retrospectively compared patients managed with 1.5\% 2- chloroprocaine or 0.1\% ropivacaine epidural infusions to determine if the increased infusion rate resulted in similar or improved analgesia. This retrospective cohort comparison consisted of full term infants 6 months or younger who underwent thoracotomy for congenital lung lesion resection. Patients were included if they were managed with either a 1.5\% 2-chloroprocaine (Group C) (n = 26) or 0.1\% ropivacaine (Group R) (n = 28) infusion administered through a caudally placed thoracic epidural catheter. The primary outcome was morphine administration at 0-24 h. Patients were similar in age, weight, length of stay, epidural location and duration. There was weak evidence for a difference in morphine use in the first 24 h in Group C compared to Group R (P = 0.08) but no difference 24-48 h. Group C was more commonly managed with ketorolac at 0-24 h (P = 0.03) and 24-48 h (P =< 0.01). The use of 2-chloroprocaine for continuous epidural infusion in infants following thoracotomy was not inferior to ropivacaine and there was weak evidence for a reduction in opioid consumption in the first 24 h postoperatively. However, the 2-chloroprocaine group was more likely to receive ketorolac. [\hyperlink{Chloroprocaine Hydrochloride}{PMID: 26306545}, Wallis T Muhly et al., 2015]

\hypertarget{pmid_24447296}{C}hloral hydrate is the most commonly used sedative for paediatric diagnostic procedures in China with a success rate of around 80\%. Intranasal dexmedetomidine is used for rescue sedation in our centre. This prospective investigation evaluated 213 children aged one month to 10 years who were not adequately sedated following administration of chloral hydrate. Children were randomly assigned to receive rescue intranasal dexmedetomidine at 1 μg.kg(-1) (group 1), 1.5 μg.kg(-1) (group 2) or 2 μg.kg(-1) (group 3). The sedation level was assessed every 10 min using a modified observer's assessment of alertness/sedation scale. Successful rescue sedation in groups 1, 2 and 3 were 56 (83.6\%), 66 (89.2\%) and 51 (96.2\%), respectively. Increasing the rescue dose was associated with an increased success rate with an odds ratio of 4.12 (95\% CI 1.13-14.98), p = 0.032. We conclude that intranasal dexmedetomidine is effective for sedation in children who do not respond to chloral hydrate.  [\hyperlink{Chloroprocaine Hydrochloride}{PMID: 24447296}, B L Li et al., 2014] Continuous epidural infusions are an effective and safe method of providing anesthesia and postoperative analgesia in infants and children with multiple advantages over systemic medications, including earlier tracheal extubation, decreased perioperative stress response, earlier return of bowel function, and decreased exposure to volatile anesthetic agents with uncertain long-term neurocognitive effects. Despite these benefits, local anesthetic toxicity remains a concern in neonates and infants because of their decreased metabolic capacity for amide local anesthetics. Chloroprocaine, an ester local anesthetic agent, which is rapidly metabolized in plasma at all ages, is an attractive alternative for this special population, particularly in the presence of superimposed liver impairment or when higher infusion rates are needed for surgical incisions stretching many dermatomes. The current manuscript reviews the literature pertaining to the use of 2-chloroprocaine for regional anesthesia in infants and children. Dosing regimens are presented and the applications of 2-chloroprocaine in this population are discussed. [\hyperlink{Chloroprocaine Hydrochloride}{PMID: 24447296}, Giorgio Veneziano et al., 2017]

\hypertarget{pmid_28242616}{A}lthough chloral hydrate (CH) has been used as a sedative for decades, it is not widely accepted as a valid choice for ophthalmic examinations in uncooperative children. This study aimed to systematically review the literature on the drug's safety and efficacy. We searched PubMed, EMBASE, ISI Web of Science, Scopus, CENTRAL, Google Scholar and Trip database to 1 October 2015, using the keywords 'chloral hydrate', 'paediatric' and 'procedural sedation OR diagnostic sedation'. A meta-analysis of randomised controlled trials (RCTs) was performed. A total of 6961 articles were screened and 104 were included in the review. Thirteen of these concerned paediatric ophthalmic examination, while 13 others were RCTs and were meta-analysed. CH was reported to have been administered in a total of 24 265 sedation episodes in children aged from <1 month to 18 years. The meta-analysis showed CH had a higher OR (2.95, 95\% CI 1.09 to 7.99) for successful sedation compared to other sedatives, but significant limitations apply. The commonest reported adverse events (AE) were not serious (eg, paradoxical reaction or transient vomiting) and required no intervention. Severe AE, including two deaths, were related to comorbidity, overdose or aspiration. Despite the paucity of high quality evidence, the existing literature suggests that the use of CH for procedural sedation in children appears to be an effective alternative to general anaesthesia, and it can be safe when administered in the hospital setting with appropriate monitoring and vigilance for intervention. [\hyperlink{Chloroprocaine Hydrochloride}{PMID: 28242616}, Asimina Mataftsi et al., 2017]

\hypertarget{pmid_20112608}{C}hloral hydrate is generally considered a safe sedative-hypnotic drug, and is commonly used for sedation of infants and young children before diagnostic procedures. Even chloral hydrate administered within the recommended maximal dose limits can cause serious morbidity and mortality. Here the authors describe a four-month-old girl with a life-threatening central nervous system and respiratory depression after administration of a therapeutic dose of chloral hydrate. The patient gradually recovered with supportive treatment including ventilation therapy. [\hyperlink{Chloroprocaine Hydrochloride}{PMID: 20112608}, Emre Ceçen et al., ]

\hypertarget{pmid_31264154}{C}hlordecone was used intensively as an insecticide in the French West Indies. Because of its high persistence, the resulting contamination of food and water has led to chronic exposure of the general population as evidenced by its presence in the blood of people of Guadeloupe, in particular in pregnant women and newborns, and in maternal breast milk. Chlordecone is recognized as a reproductive and developmental toxicant, is neurotoxic and carcinogenic in rodents, and is considered as an endocrine-disrupting compound with well-established estrogenic and progestogenic properties both in vitro and in vivo. The question arises of its potential consequences on child neurodevelopment following prenatal and childhood exposure, in particular on behavioral sexual dimorphism in childhood. We followed 116 children from the TIMOUN mother-child cohort study in Guadeloupe, who were examined at age 7. These children were invited to participate in a 7-min structured play session in which they could choose between different toys considered as feminine, masculine, or neutral. The play session was video recorded, and the percentage of the time spent playing with feminine or masculine toys was calculated. We estimated associations between playtime and prenatal exposure to chlordecone (assessed by concentration in cord blood) or childhood exposure (determined from concentrations in child blood obtained at the 7-year follow-up), taking into account confounders and co-exposures to other environmental chemicals. We used a two-group regression model to take into account sex differences in play behavior. Our results do not indicate any modification in sex-typed toy preference among 7-year-old children in relation with either prenatal or childhood exposure to chlordecone. [\hyperlink{Chloroprocaine Hydrochloride}{PMID: 31264154}, Sylvaine Cordier et al., 2020]

\hypertarget{pmid_11847958}{I}nformation regarding the treatment of anthrax infection is scarce in adults and is even more limited in children. Children, however, may be at a greater risk for developing an infection and systemic disease if exposed to anthrax than adults. The Centers for Disease Control and Prevention (CDC) recommends the use of doxycycline or ciprofloxacin for prophylaxis and treatment in children. Doxycycline currently is not indicated for use in children < 8 years old, due to staining of teeth and inhibition of bone growth associated with tetracyclines. Doxycycline, however, may have less adverse effect on teeth than its precursors. Ciprofloxacin has a pediatric indication only when a child is potentially exposed to inhaled anthrax. Ciprofloxacin is contraindicated in pediatric patients because fluoroquinolones were shown to cause cartilage toxicity in immature animals. Although children of various ages have received ciprofloxacin, there are few reports of cartilage toxicity. Because anthrax is a potentially fatal infection, the benefits to using these antibiotics greatly outweigh the risks. Therefore, the use of these antibiotics in children can be recommended, despite the lack of adequate efficacy and safety studies in pediatric patients with anthrax. [\hyperlink{Chloroprocaine Hydrochloride}{PMID: 11847958}, Sandra Benavides et al., 2002]

\hypertarget{pmid_20527137}{O}nly a few corticosteroids for topical use have proven safe and effective in pediatric populations down to 3 months of age. The authors report the results of a study designed to assess the efficacy and safety of hydrocortisone butyrate (HCB) 0.1\% in lipocream (LCr) vehicle in infants and children. A total of 264 boys and girls 3 months to less than 18 years old, with stable, mild to moderate atopic dermatitis affecting at least 10\% body surface area applied HCB 0.1\% in LCr or LCr alone twice daily for up to 1 month without occlusion. Primary end-points included: percent of patients who achieved treatment success based on physician global assessments. Secondary endpoint included: difference in pruritus and Eczema Area and Severity Index (EASI) at day 29. Treatment was significant (P < 0.001) for HCB 0.1\% LCr over vehicle. No serious nor significant adverse events were reported. Results are representative of a short duration treatment for a chronic disease. HCB 0.1\% in LCr is more effective than its vehicle in pediatric populations down to 3 months of age without significant adverse events when used twice a day for up to 1 month. [\hyperlink{Chloroprocaine Hydrochloride}{PMID: 20527137}, William Abramovits et al., ]

\hypertarget{pmid_27089835}{R}egional anesthesia use in pediatric patients has a good safety profile. 2-Chloroprocaine is used frequently in infants due to rapid onset, lack of accumulation, and rapid plasma degradation. We present a case of local anesthetic systemic toxicity following the administration of 3\% 2-chloroprocaine through a paravertebral catheter in an infant. The episode lasted 40 s followed by complete recovery. The infrequent reporting of local anesthetic systemic toxicity and limited duration of symptoms supports the continued use of 2-chloroprocaine in infants. Volume should be restricted to the smallest amount providing analgesia. [\hyperlink{Chloroprocaine Hydrochloride}{PMID: 27089835}, Maria A Hernandez et al., 2016]

\hypertarget{pmid_33655976}{C}hildren evaluated in the emergency department for head trauma often undergo computed tomography (CT), with some uncooperative children requiring pharmacological sedation. Chloral hydrate (CH) is a sedative that has been widely used, but its rectal use for child sedation after head trauma has rarely been studied. The objective of this study was to document the safety and efficacy of rectal CH sedation for cranial CT in young children.We retrospectively studied all the children with head trauma who received rectal CH sedation for CT in the emergency department from 2016 to 2019. CH was administered rectally at a dose of 50 mg/kg body weight. When sedation was achieved, CT scanning was performed, and the children were monitored until recovery. The sedative safety and efficacy were analyzed.A total of 135 children were enrolled in the study group, and the mean age was 16.05 months. The mean onset time was 16.41 minutes. Successful sedation occurred in 97.0\% of children. The mean recovery time was 71.59 minutes. All of the vital signs were within normal limits after sedation, except 1 (0.7\%) with transient hypoxia. There was no drug-related vomiting reaction in the study group. Adverse effects occurred in 11 patients (8.1\%), but all recovered completely. Compared with oral CH sedation, rectal CH sedation was associated with quicker onset (P < .01), higher success rate (P < .01), and lower adverse event rate (P < .01).Rectal CH sedation can be a safe and effective method for CT imaging of young children with head trauma in the emergency department. [\hyperlink{Chloroprocaine Hydrochloride}{PMID: 33655976}, Quanmin Nie et al., 2021]

\hypertarget{pmid_15951862}{D}iagnostic and therapeutic procedures in children are made easier using sedation. However, there is no consensus about which drug should be used to achieve this. Furthermore, none of the drugs used for sedation are risk free. The aim of this work is to study sedation indications, effectiveness, and safety at our center. A prospective observational study conducted at the Pediatric Day Care Unit, King Fahad National Guard Hospital, Riyadh, Saudi Arabia. The study covered 17.5 weeks in 2 periods: May 9th 1999 to June 13th 1999 and October 31st 2001 to February 11th 2002. Children <12 years were included. Collected data included demographics, indication, drug dosing and outcome. Data were reported as mean +/- SD. We included 148 patients, age 38 +/- 30 months. Adequate sedation was achieved in 79\% after initial chloral hydrate (CH) dose of 56.9 +/- 9.3 mg/kg, in 95\% after adding 18.5 +/- 6.4 mg/kg CH and in 96\% after adding second drug. Compared to nonrespondents, first CH dose respondents were younger and lower in weight. The CH side effects were few and mild. Chloral hydrate is a safe and effective agent for sedation in children with an age and weight dependent response. [\hyperlink{Chloroprocaine Hydrochloride}{PMID: 15951862}, Omar M Hijazi et al., 2005]

\hypertarget{pmid_34801587}{S}pinally-administered local anesthetics provide effective perioperative anesthesia and/or analgesia for children of all ages. New preparations and drugs require preclinical safety testing in developmental models. We evaluated age-dependent efficacy and safety following 1 \% preservative-free 2-chloroprocaine (2-CP) in juvenile Sprague-Dawley rats. Percutaneous lumbar intrathecal 2-CP was administered at postnatal day (P)7, 14 or 21. Mechanical withdrawal threshold pre- and post-injection evaluated the degree and duration of sensory block, compared to intrathecal saline and naive controls. Tissue analyses one- or seven-days following injection included histopathology of spinal cord, cauda equina and brain sections, and quantification of neuronal apoptosis and glial reactivity in lumbar spinal cord. Following intrathecal 2-CP or saline at P7, outcomes assessed between P30 and P72 included: spinal reflex sensitivity (hindlimb thermal latency, mechanical threshold); social approach (novel rat versus object); locomotor activity and anxiety (open field with brightly-lit center); exploratory behavior (rearings, holepoking); sensorimotor gating (acoustic startle, prepulse inhibition); and learning (Morris Water Maze). Maximum tolerated doses of intrathecal 2-CP varied with age (1.0 μL/g at P7, 0.75 μL/g at P14, 0.5 μL/g at P21) and produced motor and sensory block for 10-15 min. Tissue analyses found no significant differences across intrathecal 2-CP, saline or naïve groups. Adult behavioral measures showed expected sex-dependent differences, that did not differ between 2-CP and saline groups. Single maximum tolerated in vivo doses of intrathecal 2-CP produced reversible spinal anesthesia in juvenile rodents without detectable evidence of developmental neurotoxicity. Current results cannot be extrapolated to repeated dosing or prolonged infusion. [\hyperlink{Chloroprocaine Hydrochloride}{PMID: 34801587}, Suellen M Walker et al., 2022]

\hypertarget{pmid_7633153}{W}e evaluated the safety of ciprofloxacin administered in a dose of 15-25 mg/kg for 9-16 days, in a case series of 58 children who were between 8 months and 13 years of age. No arthropathy was observed during therapy and follow-up. Blinded evaluation of 22 pairs of nuclear magnetic resonance scans obtained before and between day 10 and 15 of therapy did not reveal any cartilage damage. After the first dose of ciprofloxacin (10 mg/kg), serum fluoride levels increased at 12 h in 15 of 19 (79\%) patients; 24-h urinary fluoride excretion was higher on day 7 compared with basal values in 16 of 18 (88.9\%) patients. Height z scores of 53 patients at a mean of 22.5 months of follow-up were not significantly different from basal scores (p = 0.12). In conclusion, ciprofloxacin may be recommended for use in children for short duration when effective alternative antibacterials are unavailable. However, there is a need for further studies to evaluate the tissue accumulation of fluoride and its potential to cause toxic effects. [\hyperlink{Chloroprocaine Hydrochloride}{PMID: 7633153}, K M Pradhan et al., 1995]

\hypertarget{pmid_18278305}{C}hloral hydrate and hydroxyzine are a drug combination frequently used by practitioners to sedate pediatric dental patients, but their effectiveness has not been compared to a negative control group in humans. The aim of this crossover, double-blinded study was to evaluate the effect of these drugs compared to a placebo, administered to young children for dental treatment. Thirty-five dental sedation sessions were carried out on 12 uncooperative ASA I children aged less than 5 years old. In each session patients were randomly assigned to groups P (placebo), CH (chloral hydrate 75 mg/kg) and CHH (chloral hydrate 50 mg/kg plus hydroxyzine 2.0 mg/kg). Vital signs and behavioral variables were evaluated every 15 min. Comparisons were statistically analyzed using Friedman and Wilcoxon tests. P, CH and CHH had no differences concerning vital signs, except for breathing rate. All vital signs were in the normal range. CH and CHH promoted more sleep in the first 30 min of treatment. Overall behavior was better in CH and CHH than in P. CH, CHH and P were effective in 62.5\%, 61.5\% and 11.1\% of the cases, respectively. Chloral hydrate was safe and relatively effective, causing more satisfactory behavioral and physiological outcomes than a placebo. [\hyperlink{Chloroprocaine Hydrochloride}{PMID: 18278305}, Luciane Ribeiro de Rezende Sucasas da Costa et al., 2007]

\section*{Cisatracurium Besylate}
\subsection*{Result}
\subsubsection*{Answer}

Yes (1 month to 16 years)

\subsubsection*{{Explanation}}
\hypertarget{Cisatracurium Besylate}
A review of the available abstracts reveals multiple targeted studies evaluating the safety and efficacy of Cisatracurium Besylate in children across various age ranges. Below is a summary of the relevant evidence, organized by age group:

Infants (1 month to 1 year):
- Several studies specifically included infants. For example, Taivainen et al. (2000) studied 27 infants (aged 1–23 months) and 24 children (aged 2–12.5 years), administering 0.15 mg/kg cisatracurium. The study found that cisatracurium was effective and well tolerated in both infants and children, with negligible changes in blood pressure and heart rate, and no significant adverse effects reported [\hyperlink{pmid_11069329}{PMID: 11069329}, T Taivainen et al., 2000].
- Soltész et al. (2002) compared 15 infants and 15 children, finding that infants were more sensitive to cisatracurium, but no safety concerns were raised. The authors recommend neuromuscular monitoring due to interindividual differences [\hyperlink{pmid_12125308}{PMID: 12125308}, S Soltész et al., 2002].
- Meakin et al. (2007) included infants as young as 1 month and found that cisatracurium 0.15 mg/kg produced acceptable intubating conditions in the great majority of infants and children, with negligible hemodynamic changes [\hyperlink{pmid_17238881}{PMID: 17238881}, George H Meakin et al., 2007].

Children (1 year to 12 years):
- Multiple studies, including Agavelian et al. (children aged 2–12 years), Meretoja et al. (children 2–12 years), and Imbeault et al. (children 1–6 years), all report effective neuromuscular blockade, good intubating conditions, and no significant adverse effects or hemodynamic instability [\hyperlink{pmid_10584360}{PMID: 10584360}, E G Agavelian et al.; \hyperlink{pmid_8880817}{PMID: 8880817}, O A Meretoja et al., 1996; \hyperlink{pmid_16492821}{PMID: 16492821}, Karynn Imbeault et al., 2006].
- WangNing ShangGuan et al. (2008) studied children aged 15–60 months and found no effect on heart rate or blood pressure at any dose, with no adverse events reported [\hyperlink{pmid_18929279}{PMID: 18929279}, WangNing ShangGuan et al., 2008].
- Burmester et al. (2005) conducted a randomized, double-blind study in critically ill children aged 3 months to 16 years, finding that recovery of neuromuscular function after discontinuation of cisatracurium was significantly faster than with vecuronium, with no safety concerns raised [\hyperlink{pmid_15815895}{PMID: 15815895}, Margarita Burmester et al., 2005].

Infants and Children (0.3–9.6 years):
- de Ruiter et al. (2001) studied 32 infants (0.3–1.0 years) and 32 children (3.1–9.6 years), concluding that cisatracurium is equipotent in infants and children when dosed by body weight, with no safety issues reported [\hyperlink{pmid_11388529}{PMID: 11388529}, J de Ruiter et al., 2001].

Summary:
Across these studies, which include infants as young as 1 month and children up to 16 years, cisatracurium besylate was consistently found to be effective and well tolerated, with no significant adverse effects or hemodynamic instability reported. The studies were specifically designed to assess safety and efficacy in pediatric populations, and several recommend the use of neuromuscular monitoring due to interindividual variability, especially in infants.

There are no abstracts indicating that cisatracurium besylate is unsafe in children, nor are there any that report serious adverse events in the studied pediatric populations.

Therefore, based on the available abstracts, cisatracurium besylate is affirmed as safe for use in children from 1 month to 16 years, when used with appropriate monitoring.

\subsection*{Abstracts}
\hypertarget{pmid_10584360}{C}isatracurium besilate was used in 21 children aged 2-12 years (ASA classes I-II) for intravenous anesthesia anesthetic + nitrous oxide + oxygen (group 1, 11 pts) and halothane + nitrous oxide + oxygen (group 2, 11 pts). In group 1 the initial nimbex dose was 0.15 mg/kg, in group 2 0.12 mg/kg, which created good or excellent conditions for intubation within 2 min and induced a neuromuscular blocking (NMB) for 44.2 +/- 4.7 and 37.6 +/- 5.9 min, respectively. NMB was maintained by bolus injections of nimbex in doses of 0.03 mg/kg (group 1) and 0.02 mg/kg (group 2). The duration of myoplegia was 18.6 +/- 3.6 and 13.2 +/- 2.3 min, respectively. Clinically the relaxation was sufficient throughout the operation. Neuromuscular conduction recovered spontaneously in all cases, extubation was carried out 42.2 +/- 3.8 min after the last bolus injection of nimbex in group 1 and after 35.4 +/- 7.4 min in group 2. No appreciable fluctuations of hemodynamic parameters or other side effects (skin hyperemia or bronchial spasm) were observed during anesthesia. Studies of the benzyl isoquinoline myorelaxant cisatracurium besilate demonstrated that this effective and safe drug with an average duration of effect can be used in pediatric anesthesiology. [\hyperlink{Cisatracurium Besylate}{PMID: 10584360}, E G Agavelian et al., ]

\hypertarget{pmid_9129870}{C}isatracurium besilate (besylate) is a nondepolarising neuromuscular blocking agent with an intermediate duration of action. It is the R-cis, R'-cis isomer of atracurium besilate and is approximately 3-fold more potent than the mixture of isomers that constitute the parent drug. The ED95 for cisatracurium besilate (dose required to produce 95\% suppression of twitch response to nerve stimulation) in adults is 0.05 mg/kg during N2O/O2 opioid anaesthesia. As for atracurium besilate, the primary route of elimination of cisatracurium besilate is by spontaneous degradation. Cisatracurium besilate is not associated with dose-related histamine release (at bolus doses of < or = 8 x ED95) and, consistent with this, has demonstrated cardiovascular stability in both healthy patients (< or = 8 x ED95) and those with coronary artery disease (< or = 6 x ED95). In clinical trials, cisatracurium besilate has been used successfully to facilitate intubation (at 2 to 4 x ED95) and as a muscle relaxant during surgery and in intensive care. Compared with vecuronium, cisatracurium besilate was associated with a significantly faster recovery after continuous infusion in patients in intensive care. Relative to atracurium besilate, cisatracurium besilate has a lower propensity to cause histamine release is more potent but has a slightly longer onset time at equipotent doses. It also offers a more predictable recovery profile than vecuronium after prolonged use in patients in intensive care. Thus, comparative data provide some indication of the potential of cisatracurium besilate as an intermediate-duration neuromuscular blocking agent but further comparisons with other like agents are required to define precisely its relative merits. [\hyperlink{Cisatracurium Besylate}{PMID: 9129870}, H M Bryson et al., 1997]

\hypertarget{pmid_11388529}{T}o determine the effect of age on the dose-response relation and infusion requirement of cisatracurium besylate in pediatric patients, 32 infants (mean age, 0.7 yr; range, 0.3-1.0 yr) and 32 children (mean age, 4.9 yr; range, 3.1-9.6 yr) were studied during thiopentone-nitrous oxideoxygen-narcotic anesthesia. Potency was determined using a single-dose (20, 26, 33, or 40 microg/kg) technique. Neuromuscular block was assessed by monitoring the electromyographic response of the adductor pollicis to supramaximal train-of-four stimulation of the ulnar nerve at 2 Hz. Least-squares linear regression analysis of the log-probit transformation of dose and maximal response yielded median effective dose (ED50) and 95\% effective dose (ED95) values for infants (29+/-3 microg/kg and 43+/-9 microg/kg, respectively) that were similar to those for children (29+/-2 microg/kg and 47+/-7 microg/kg, respectively). The mean infusion rate necessary to maintain 90-99\% neuromuscular block during the first hour in infants (1.9+/-0.4 microg x kg(-1) x min(-1); range: 1.3-2.5 microg x kg(-1) x min(-1)) was similar to that in children (2.0+/-0.5 microg x kg(-1) x min(-1); range: 1.3-2.9 microg x kg(-1) x min(-1)). The authors conclude that cisatracurium is equipotent in infants and children when dose is referenced to body weight during balanced anesthesia. [\hyperlink{Cisatracurium Besylate}{PMID: 11388529}, J de Ruiter et al., 2001]

\hypertarget{pmid_9606456}{T}he stability of cisatracurium besylate was studied. Cisatracurium (as besylate) 2 mg/mL in 5- and 10-mL unopened vials and 10 mg/mL in 20-mL unopened vials, as well as 3 mL of solution from additional 2-mg/mL vials, repackaged in 3-mL sealed plastic syringes, was stored at 4 and 23 degrees C in the dark and in normal fluorescent room light. Admixtures of cisatracurium (as besylate) 0.1, 2, or 5 mg/mL in polyvinyl chloride (PVC) minibags of 5\% dextrose injection or 0.9\% sodium chloride injection were stored at 4 and 23 degrees C in normal fluorescent room light. Triplicate samples for each storage condition were taken initially and at 1, 3, 5, 7, 14, 21, and 30 days; samples from vials were also removed at 45 and 90 days. Solutions were stored in sterile vials at -70 degrees C and then thawed at room temperature before analysis of chemical stability by high-performance liquid chromatography. Physical stability was assessed as well. Cisatracurium besylate was physically stable in all samples throughout the study. Cisatracurium (as besylate) 2 mg/mL exhibited drug losses at 23 degrees C in vials at 45 days and in syringes at 30 days. Cisatracurium (as besylate) 0.1, 2, and 5 mg/mL in 5\% dextrose injection and in 0.9\% sodium chloride injection was stable for at least 30 days at 4 degrees C, but substantial drug losses occurred at 23 degrees C. Admixtures prepared with cisatracurium (as besylate) 0.1 mg/mL and with 5\% dextrose injection exhibited the greatest losses. Cisatracurium besylate was stable in most samples for at least 30 days at 4 and 23 degrees C; admixtures containing cisatracurium (as besylate) 0.1 or 2 mg/mL exhibited substantial drug loss at 23 degrees C. [\hyperlink{Cisatracurium Besylate}{PMID: 9606456}, Q A Xu et al., 1998]

\hypertarget{pmid_16492821}{W}e studied the pharmacokinetics and pharmacodynamics of cisatracurium in 9 children (mean weight, 17.1 kg) aged 1-6 yr (mean, 3.75 yr) during propofol-nitrous oxide anesthesia. Neuromuscular monitoring was performed. Venous samples were taken before injection of a 0.1 mg/kg dose of cisatracurium and then at 2, 5, 10, 30, 60, 90, and 120 min. Cisatracurium plasma concentrations were determined by high performance liquid chromatography. Onset time was 2.5 +/- 0.8 min, recovery to 25\% of baseline twitch height was 37.6 +/- 10.2 min, and the 25\%-75\% recovery index was 10.9 +/- 3.7 min. Distribution and elimination half-lives were 3.5 +/- 0.9 min and 22.9 +/- 4.5 min, respectively. Steady-state volume of distribution (0.207 +/- 0.031 L/kg) and total body clearance (6.8 +/- 0.7 mL/min/kg) were significantly larger than those published for adults. Pharmacodynamic results were comparable to those obtained in pediatric studies during halothane or opioid anesthesia with the exception of a longer recovery to 25\% baseline. Although the plasma-effect compartment equilibration rate constant was twofold faster (0.115 +/- 0.025 min(-1)) than that published for cisatracurium in adults, the effect compartment concentration corresponding to 50\% block was similar (129 +/- 27 ng/mL). [\hyperlink{Cisatracurium Besylate}{PMID: 16492821}, Karynn Imbeault et al., 2006]

\hypertarget{pmid_9262747}{T}he compatibility of cisatracurium besylate with 91 other drugs during simulated Y-site injection was studied. Five milliliters of cisatracurium 0.1, 2, and 5 mg/mL (as besylate) in 5\% dextrose injection was combined with 5 mL of each of 91 drugs in 5\% dextrose injection or 0.9\% sodium chloride injection. All combinations were prepared in duplicate and stored at approximately 23 degrees C. Samples were visually examined under normal laboratory fluorescent light and, if there was no obvious visual incompatibility, under high-intensity monodirectional light. Turbidity was measured as well. Particle sizing and counting was performed for selected combinations. All evaluations were performed at intervals up to four hours. Cisatracurium besylate at all three concentrations was compatible with most of the drugs tested. However, one drug (cefoperazone) was incompatible with cisatracurium besylate at all three concentrations, 14 (including many cephalosporins) were incompatible with cisatracurium besylate 2 and 5 mg/ mL, and 12 were incompatible with cisatracurium 5 mg/ mL. During simulated Y-site administration, cisatracurium 0.1, 2, and 5 mg/mL (as besylate) in 5\% dextrose injection was compatible with 64 of 91 drugs for four hours at approximately 23 degrees C. Twenty-seven drugs were incompatible with cisatracurium besylate at one or more concentrations. [\hyperlink{Cisatracurium Besylate}{PMID: 9262747}, L A Trissel et al., 1997]

\hypertarget{pmid_12125308}{T}o compare the onset, duration and maximum effect of 0.1 mg/kg cisatracurium during balanced anesthesia with sevoflurane and remifentanil between infants and children. We measured the time course of the neuromuscular blockade in 15 infants and 15 children by electromyography. Anesthesia was induced with propofol/remifentanil and maintained with sevoflurane (constant 2\% endtidal) and remifentanil according to the patients individual requirements. After injection of 0.1 mg/kg cisatracurium we measured the following parameters: onset time: time between the beginning of injection of cisatracurium and maximum T1 depression, clinical duration: time between injection of the drug and recovery of T1 to 25\%, recovery index: time between recovery of T1 from 25\% to 75\%. TOFR 0.9: time between injection of cisatracurium and recovery of the train-of-four ratio to 90\%. In addition, we determined the maximum neuromuscular blockade Tmax after 0.1 mg/kg cistracurium. Both groups differed significantly with regard to onset time and clinical duration. In the infants, the onset time was shorter (74 s vs. 198 s) and the clinical duration longer (55 min vs. 41 min) compared to the older children. The TOFR 0.9 was 73 min (range 56-86 min) in the group of the infants and 59 min (range 43-72 min) in the group of the older children (p < 0.001). Tmax was 100\% (range 97-100\%) in the infants and 98\% (range 92-100\%) in the children (p < 0.01). However, the recovery index was comparable in both groups (21 vs. 16 min). Infants are substantially more sensitive to cisatracurium than children, which can be demonstrated in a significantly shorter onset time, a prolonged clinical duration and a delayed neuromuscular recovery. As there exist large interindividual differences, we recommend the use of neuromuscular monitoring in the routine practice of pediatric anesthesia. [\hyperlink{Cisatracurium Besylate}{PMID: 12125308}, S Soltész et al., 2002]

\hypertarget{pmid_18929279}{T}o describe, in pediatric patients, the effects of three doses of cisatracurium during nitrous oxide-propofol anesthesia and to determine if larger doses result in faster onset time. College hospital. 75 ASA physical status I and II children, aged 15 to 60 months, undergoing surgery for hypospadias or undescendent testis. Patients were randomly assigned to one of three groups according to the dose of cisatracurium: Group A = 0.1 mg/kg (two x effective dose), Group B = 0.15 mg/kg (three x effective dose), and Group C = 0.2 mg/kg (4 x effective dose). Neuromuscular block was assessed with TOF-Guard (Biometer International, Odense, Denmark) accelerometry. Onset time (from cisatracurium injection to maximal depression of time to first twitch), duration of peak effect (time from cisatracurium injection to 5\% recovery of time to first twitch), duration of clinical action (time from cisatracurium injection to 25\% recovery of time to first twitch), and recovery index (recovery of time to first twitch from 25\% to 75\%) were recorded. Cisatracurium had no effect on heart rate or blood pressure at any dose. Compared with Group A, onset times in Groups B and C were shorter; and durations of peak effect and clinical action in Groups B and C were longer (P < 0.01) than those in Group A. There was no difference in recovery index among the three groups. There was no difference in onset times between Groups B and C. Compared with Group B, durations of peak effect and clinical action in Group C were longer (P < 0.05 or P < 0.01). Four times the effective dose of cisatracurium did not significantly shorten onset time beyond that produced with three times the effective dose in young children. [\hyperlink{Cisatracurium Besylate}{PMID: 18929279}, WangNing ShangGuan et al., 2008]

\hypertarget{pmid_17238881}{T}he aims of the present study were to determine the tracheal intubating conditions, onset time, duration of action, and hemodynamic responses following the administration of cisatracurium 0.15 mg x kg(-1) to infants and children. One hundred and eighty-one infants and children aged 1 month to 12 years were randomized to two groups to receive anesthesia with nitrous oxide-oxygen-halothane (group H) or nitrous oxide-oxygen-thiopental-fentanyl (group TF). Intubation conditions were assessed 120 s after cisatracurium administration using a 4-part scale. Neuromuscular transmission was monitored by recording the evoked compound electromyogram of the adductor pollicis. The proportion of patients with excellent or good intubating conditions was similar in both groups (88 of 90, 98\% in group H; 85 of 90, 94\% in group TF). However, there was a significantly greater proportion of excellent intubating conditions in group H (79 of 90, 88\%) compared with group TF (65 of 90, 72\%) (P = 0.01) and recovery time was significantly longer in group H compared with group TF (P < 0.001). There was also a higher proportion of excellent intubating conditions in infants compared with older subjects (P = 0.02) and a shorter onset time (P < 0.001) and longer recovery time (P < 0.001) in younger compared with older patients. Changes in heart rate and arterial pressure were negligible 1 min following the cisatracurium administration. Cisatracurium 0.15 mg x kg(-1) produces acceptable intubating conditions at 120 s in the great majority of infants and children. Anesthesia background and age have significant effects on intubating conditions and duration of action of cisatracurium. [\hyperlink{Cisatracurium Besylate}{PMID: 17238881}, George H Meakin et al., 2007]

\hypertarget{pmid_8880817}{C}isatracurium, 51W89, is one of the ten stereoisomers of Tracrium which, unlike atracurium, has been reported to have a lack of histamine mediated cardiovascular effects at doses as high as 8 x ED95 in adults. We compared the time-course of neuromuscular effects of 80 micrograms.kg-1 or 100 micrograms.kg-1 cisatracurium during N2O-O2-halothane or N2O-O2-opioid anaesthesia, respectively, in 32 children 2-12 years old. Neuromuscular function was monitored by evoked adductor pollicis EMG. Even-numbered patients (n = 16) were allowed to obtain full spontaneous recovery of neuromuscular function and odd-numbered patients (n = 16) received neostigmine 45 micrograms.kg-1 together with glycopyrrolate at the time of 25\% EMG recovery. Data are expressed as median with 10th to 90th percentile range. Cisatracurium had an onset time (time from administration to maximal effect) of 2.2 (1.7-3.8) or 2.3 (1.8-4.9) min, a clinical duration (time to 25\% EMG recovery) of 34 (22-40) or 27 (24-33) min, and a spontaneous 25-75\% recovery time (time from 25 to 75\% EMG recovery) of 11 (9-13) or 11 (7-12) min during halothane or balanced anaesthesia, respectively (NS). Train-of-four ratio recovered to 0.70 in 2.5 (1.8-3.0) or 3.2 (2.1-4.3) min following neostigmine during halothane or balanced anaesthesia, respectively (NS). Changes in blood pressure or heart rate following cisatracurium were negligible. We regard cisatracurium as a safe and promising intermediate duration muscle relaxant the effects of which can easily be reversed with neostigmine. [\hyperlink{Cisatracurium Besylate}{PMID: 8880817}, O A Meretoja et al., 1996]

\hypertarget{pmid_9989341}{C}isatracurium besilate, one of the 10 stereoisomers that comprise atracurium besilate, is a nondepolarising neuromuscular blocking agent with an intermediate duration of action. Following a 5- to 10-sec intravenous bolus dose of cisatracurium besilate to healthy young adult surgical patients, elderly patients and patients with renal or hepatic failure, the concentration versus time profile of cisatracurium besilate is best characterised by a 2-compartment model. The volume of distribution (Vd) of cisatracurium besilate is small because of its relatively large molecular weight and high polarity. Cisatracurium besilate undergoes Hofmann elimination, a process dependent on pH and temperature. Unlike atracurium besilate, cisatracurium besilate does not appear to be degraded directly by ester hydrolysis. Hofmann elimination, an organ independent elimination pathway, occurs in plasma and tissue, and is responsible for approximately 77\% of the overall elimination of cisatracurium besilate. The total body clearance (CL), steady-state Vd and elimination half-life of cisatracurium besilate in patients with normal organ function are approximately 0.28 L/h/kg (4.7 ml/min/kg), 0.145 L/kg and 25 minutes, respectively. The magnitude of interpatient variability in the CL of cisatracurium besilate is low (16\%), a finding consistent with the strict physiological control of the factors that effect the Hofmann elimination of cisatracurium besilate (i.e. temperature and pH). There is a unique relationship between plasma clearance and Vd because the primary elimination pathway for cisatracurium besilate is not dependent on organ function. There are minor differences in the pharmacokinetics of cisatracurium besilate in various patient populations. These differences are not associated with clinically significant differences in the recovery profile of cisatracurium besilate, but may be associated with differences in the time to onset of neuromuscular block. [\hyperlink{Cisatracurium Besylate}{PMID: 9989341}, D F Kisor et al., 1999]

\hypertarget{pmid_15815895}{T}o evaluate and compare the efficacy, infusion rate and recovery profile of vecuronium and cisatracurium continuous infusion in critically ill children requiring mechanical ventilation. Prospective, randomised, double-blind, single-centre study in critically ill children in a paediatric intensive care unit in a tertiary children's hospital. Thirty-seven children from 3 months to 16 years old (median 4.1 year) were randomised to receive either drug; those already receiving more than 6 h of neuromuscular blocking drugs were excluded. The Train-of-Four (TOF) Watch maintained neuromuscular blockade to at least one twitch in the TOF response. Recovery time was measured from cessation of infusion until spontaneous TOF ratio recovery of 70\%. The cisatracurium infusion rate in nineteen children averaged 3.9+/-1.3 microg kg(-1) min(-1) with a median duration of 63 h (IQR 23-88). The vecuronium infusion rate in 18 children averaged mean 2.6+/-1.3 microg kg(-1) min(-1) with a median duration of 40 h (IQR 27-72). Median time to recovery was significantly shorter with cisatracurium (52 min, 35-73) than with vecuronium (123 min, 80-480). Prolonged recovery of neuromuscular function (>24 h) occurred in one child (6\%) on vecuronium. Recovery of neuromuscular function after discontinuation of neuromuscular blocking drug infusion in children is significantly faster with cisatracurium than vecuronium. Neuromuscular monitoring was not sufficient to eliminate prolonged recovery in children on vecuronium infusions. [\hyperlink{Cisatracurium Besylate}{PMID: 15815895}, Margarita Burmester et al., 2005]

\hypertarget{pmid_11069329}{W}e studied the neuromuscular and cardiovascular effects of a single, rapidly administered intravenous dose of cisatracurium 0.15 mg.kg(-1) in 27 infants (aged 1-23 months) and 24 children (aged 2-12.5 years). After midazolam premedication, anaesthesia was induced and maintained with thiopental and alfentanil in addition to nitrous oxide in oxygen. Neuromuscular function was monitored by evoked adductor pollicis electromyography. At least 15 min after intubation, each patient received cisatracurium 0.15 mg.kg(-1) over 5 s. Complete neuromuscular blockade was produced by this dose in all but one infant. The mean (SD) onset time of maximal blockade was more rapid in infants [2.0 (0.8) min] than in children [3.0 (1.2) min], p = 0. 0011. The clinical duration of action of cisatracurium (recovery of evoked response to 25\% of control) was significantly longer in infants [43.3 (6.2) min] than in children [36.0 (5.4) min], p < 0.0001. Once neuromuscular function started to recover, the rate of recovery was similar in both age groups. Changes in blood pressure and heart rate after the administration of cisatracurium were negligible in both age groups. Cisatracurium, at a dose of 0.15 mg. kg(-1), was effective and well tolerated in infants and children. [\hyperlink{Cisatracurium Besylate}{PMID: 11069329}, T Taivainen et al., 2000]

\hypertarget{pmid_15839217}{T}he authors have studied the effects of the muscular relaxants rocuronium bromide and cysatracurium besylate on neuromuscular conduction, respiration mechanics, and hemodynamics in 120 children during endosurgical operations. The paper comparatively assesses the muscular relaxants and the procedures of their use, which made it possible to ensure controlled and deep relaxation. [\hyperlink{Cisatracurium Besylate}{PMID: 15839217}, V V Makushkin et al., ]

\hypertarget{pmid_28827252}{C}eftriaxone is widely used in children in the treatment of sepsis. However, concerns have been raised about the safety of ceftriaxone, especially in young children. The aim of this review is to systematically evaluate the safety of ceftriaxone in children of all age groups. MEDLINE, PubMed, Cochrane Central Register of Controlled Trials, EMBASE, CINAHL, International Pharmaceutical Abstracts and adverse drug reaction (ADR) monitoring systems will be systematically searched for randomised controlled trials (RCTs), cohort studies, case-control studies, cross-sectional studies, case series and case reports evaluating the safety of ceftriaxone in children. The Cochrane risk of bias tool, Newcastle-Ottawa and quality assessment tools developed by the National Institutes of Health will be used for quality assessment. Meta-analysis of the incidence of ADRs from RCTs and prospective studies will be done. Subgroup analyses will be performed for age and dosage regimen. Formal ethical approval is not required as no primary data are collected. This systematic review will be disseminated through a peer-reviewed publication and at conference meetings. CRD42017055428. [\hyperlink{Cisatracurium Besylate}{PMID: 28827252}, Linan Zeng et al., 2017]

\hypertarget{pmid_31784219}{C}isatracurium besylate has been determined by fast and highly sensitive spectrofluorimetric method based on measuring the fluorescence intensity of its methanolic solution at 312 nm after excitation at 230 nm (Method I). The linearity occurred over the concentration range of 10.0-130.0 ng/mL with detection limit of 1.07 ng/mL. The method was further extended for the determination of the studied drug in spiked human plasma with good percentage recoveries (97.43-103.50\%). Cisatracurium is co-administered with nalbuphine during surgery. The simultaneous determination of both drugs was based on synchronous spectrofluorimetric technique. First derivative synchronous spectrofluorimetric amplitude was measured in methanol at Δ λ = 60 nm and each drug could be estimated at the zero crossing point of the other. Hence, cisatracurium could be measured at 284.6 nm while nalbuphine at 276.3 nm (Method II). The method was linear over the ranges of 50.0-750.0 ng/mL and 0.5-7.0 μg/mL with the detection limits of 2.16 ng/mL and 0.04 μg/mL for cisatracurium and nalbuphine, respectively. The method was further extended for the simultaneous determination of both drugs in spiked human urine with mean percentage recoveries of 99.99 ± 2.06 and 99.53 ± 6.17 for cisatracurium and nalbuphine, respectively. Both methods were validated in agreement with Guidelines adopted by International Council of Harmonization (ICH). [\hyperlink{Cisatracurium Besylate}{PMID: 31784219}, Mona E El Sharkasy et al., 2020]

\hypertarget{pmid_12542606}{G}astroesophageal reflux is a common problem in infancy. Cisapride is a commonly used therapy for gastroesophageal reflux in children. In view of recent concern regarding adverse effects this study aims to evaluate the benefits and risks of cisapride for the treatment of gastroesophageal reflux in children. A meta-analysis of randomized controlled trials of cisapride using a random-effects model. Ten trials involving 415 children were identified. There was no evidence of a significant reduction in vomiting severity with cisapride as measured by a clinical score (five trials, standardized weighted mean difference -0.18; 95\% confidence interval (CI) -0.51 to 0.15). Twenty-four-hour esophageal pH monitoring data showed the mean reflux index was significantly lower in the children treated with cisapride compared with controls (five trials, weighted mean difference -6.24; 95\% CI -8.81 to -3.67). With cisapride treatment, there was no reduction in the mean number of reflux episodes lasting greater than 5 min (three trials, weighted mean difference -0.72; 95\% CI -1.92 to 0.47) or in the number of children with esophagitis at final follow up compared with baseline (two trials, relative risk 0.80; 95\% CI 0.40 to 1.61). There was no significant difference in reported side-effects or adverse events (six trials, relative risk 1.16; 95\% CI 0.95 to 1.41). No clinically important benefits of cisapride in children with gastroesophageal reflux have been demonstrated. Nor was there any evidence of adverse or harmful events. [\hyperlink{Cisatracurium Besylate}{PMID: 12542606}, Jacqueline R Dalby-Payne et al., 2003]

\hypertarget{pmid_22378696}{A}nti-seizure prophylaxis is routinely utilized during busulfan administration for HSCT. We evaluated the feasibility and efficacy of levetiracetam in children undergoing HSCT. A total of 28 children and young adults received levetiracetam during HSCT and the outcomes and costs were compared to a historical, but similar cohort of 25 patients who had received fosphenytoin. Levetiracetam was well tolerated and was efficacious in preventing seizures. Cost of drug, administration, and monitoring were also similar among the two groups. Due to non-induction of the hepatic cytochrome P450 enzymes, levetiracetam may lead to better dose assurance of busulfan in targeted dose regimens for HSCT. [\hyperlink{Cisatracurium Besylate}{PMID: 22378696}, Sandeep Soni et al., 2012]

\hypertarget{pmid_9831411}{T}his paper provides a comprehensive review of the current knowledge on cisapride in different clinical conditions in children: different manifestations of gastro-oesophageal reflux, such as (excessive) regurgitation, oesophagitis, chronic respiratory disease or uncontrolled asthma, cystic fibrosis, chronic dyspepsia, constipation and pseudo-obstruction, and as an aid to small bowel capsule-biopsy. It discusses, in depth, the safety profile of cisapride in paediatric patients. [\hyperlink{Cisatracurium Besylate}{PMID: 9831411}, Y Vandenplas et al., 1998]

\hypertarget{pmid_9297378}{C}isatracurium (51W89) is one of the ten stereoisomers of atracurium, accounting for about 15\% of the racemate. The ED95 of cisatracurium was determined to be about 50 micrograms/kg (cation, molecular weight 929), while the ED95 of atracurium (besylate salt, molecular weight 1245) was 250 micrograms/kg. Thus, on a molar basis in adult patients, cisatracurium is about 3.5 times as potent as the racemic atracurium mixture. We compared atracurium with cisatracurium in healthy adult patients and found an almost identical pharmacodynamic profile. In children, an ED95 of about 40 micrograms/kg was determined, while a 1-min-longer onset of cisatracurium was found in geriatric than in young adult patients. The presence of chronic renal failure did not prolong the duration of action of cisatracurium. The recovery of neuromuscular transmission from a cisatracurium infusion of up to 145 h was investigated in intensive care unit patients. Their time from the end of infusion to a train-of-four ratio > 0.7 (68 +/- 18 min) was on average only some 70\% longer than after an infusion of cisatracurium for 2 h in normal surgical patients. In another study, no signs of histamine release nor any clinically relevant cardiovascular effects of cisatracurium were found in doses up to eight times ED95. [\hyperlink{Cisatracurium Besylate}{PMID: 9297378}, H Mellinghoff et al., 1997]

\hypertarget{pmid_34462863}{C}ommunity-acquired pneumonia (CAP)/community-acquired bacterial pneumonia (CABP) and complicated skin and soft tissue infection (cSSTI)/acute bacterial skin and skin structure infection (ABSSSI) represent major causes of morbidity and mortality in children. β-Lactams are the cornerstone of antibiotic treatment for many serious bacterial infections in children; however, most of these agents have no activity against methicillin-resistant Staphylococcus aureus (MRSA). Ceftaroline fosamil, a β-lactam with broad-spectrum in vitro activity against Gram-positive pathogens (including MRSA and multidrug-resistant Streptococcus pneumoniae) and common Gram-negative organisms, is approved in the European Union and the United States for children with CAP/CABP or cSSTI/ABSSSI. Ceftaroline fosamil has completed a pediatric investigation plan including safety, efficacy, and pharmacokinetic evaluations in patients with ages ranging from birth to 17 years. It has demonstrated similar clinical and microbiological efficacy to best available existing treatments in phase III-IV trials in patients aged ≥ 2 months to < 18 years with CABP or ABSSSI, with a safety profile consistent with the cephalosporin class. It is also approved in the European Union for neonates with CAP or cSSTI, and in the US for neonates with ABSSSI. Ceftaroline fosamil dosing for children (including renal function adjustments) is supported by pharmacokinetic/pharmacodynamic modeling and simulations in appropriate age groups, and includes the option of 5- to 60-min intravenous infusions for standard doses, and a high dose for cSSTI patients with MRSA isolates, with a ceftaroline minimum inhibitory concentration of 2-4 mg/L. Considered together, these data suggest ceftaroline fosamil may be beneficial in the management of CAP/CABP and cSSTI/ABSSSI in children. [\hyperlink{Cisatracurium Besylate}{PMID: 34462863}, Susanna Esposito et al., 2021]

\hypertarget{pmid_11473856}{T}he purpose of this investigation was to compare the costs of intermediate-acting neuromuscular blocking drugs in children during routine ambulatory surgery. We studied 200 healthy, 2-10-yr-old children undergoing elective dental restorative surgery. During Part 1 of the study, children received an inhaled anesthetic with halothane and nitrous oxide, whereas in Part 2, the anesthetic was IV propofol with nitrous oxide. The study drugs were atracurium, cisatracurium, mivacurium, rocuronium, and vecuronium. Patients were initially administered 2x the effective dose for 95\% of the study drug. After recovery to 10\% of baseline neuromuscular function, the neuromuscular blockade was rigidly maintained with an infusion of the study drug at about 10\% of baseline function. Neuromuscular drug costs were approximated as drug usage x cost/unit. The initial drug costs were not substantially different for both Parts 1 and 2, but over time, mivacurium became the most expensive drug and cisatracurium the least expensive. In conclusion, based on current costs, cisatracurium is the least expensive intermediate-acting neuromuscular drug. For children undergoing minor ambulatory procedures of 1-2 h, and continuous intraoperative neuromuscular blockade is indicated, cisatracurium currently is the least expensive drug. [\hyperlink{Cisatracurium Besylate}{PMID: 11473856}, W M Splinter et al., 2001]

\hypertarget{pmid_3866088}{A} clinical trial of ceftizoxime suppositories (CZX-S) was performed to evaluate the therapeutic effectiveness in children with bacterial infection. The subjects were 10 children comprising 4 with pneumonia, 3 with lacunar tonsillitis, 2 with pharyngitis, and 1 with UTI. They were given 1 suppository containing either 125 mg or 250 mg of CZX 2 to 4 times a day. The daily per kg body weight dose ranged from 17.1 to 60.0 mg. The result was "markedly effective" in 3, "effective" in 6, and "failure" was recorded in 1. Bacteriologically, successful eradication of causative organisms was confirmed in all the 4 children who underwent the test. No clinical side effects were observed. The only laboratory test abnormality recorded in a single patient was eosinophilia, which was not definitely ascribable to CZX-S. In conclusion, CZX-S have proved to be a clinically safe and effective antibiotic preparation in infantile infection, even in children whose treatment with conventional antibiotics is associated with difficulties. [\hyperlink{Cisatracurium Besylate}{PMID: 3866088}, T Hosoda et al., 1985]

\hypertarget{pmid_9719722}{C}isatracurium is one of ten isomers that form the racemic mix of atracurium (51W89 or 1 R-cis, 1'R-cis atracurium). It is three times more potent than atracurium itself and hemodynamically stable thanks to its scarce release of histamine. Cisatracurium is hydrolyzed mainly by the pathway of Hofmann (77\%) and to a lesser degree it is metabolized by organ-dependent modes (mainly by the kidney (16\%)). Dose therefore hardly needs to be changed for elderly patients or those with liver, kidney or cardiovascular disease. The calculated ED95 is 0.05 mg.kg-1 (0.04 mg.kg-1 in children), although a dose two to four times greater is used under clinical conditions to shorten tracheal intubation time because of low onset of blockade, particularly in comparison with rocuronium. The period of deep blockade (lack of response to neurostimulation) is prolonged by the higher dose, but recovery is dose-independent and recovery indices are similar. Cisatracurium has proven useful in intensive care because of its hemodynamic stability, which is comparable to that of steroid derivatives but with faster recovery from blockade once administration is discontinued. Its metabolism predominantly through Hofmann's pathway, with less laudanosine formation than is produced by atracurium, is also appreciated. Cisatracurium is described as the nondepolarizing muscle relaxant of choice for medium-to-long-term surgery on hemodynamically unstable patients or those with kidney or liver disease, and for neuromuscular blockade in intensive care. [\hyperlink{Cisatracurium Besylate}{PMID: 9719722}, J R Ortiz et al., ]

\hypertarget{pmid_8788285}{T}o establish whether cisapride is beneficial in children with intractable constipation, an open trial was performed. Chronically constipated children who had failed at least 12 weeks of medical therapy received cisapride at a dose of 0.2 mg/kg/dose TID for 12 weeks. Children with pelvic floor dyssynergia were excluded. Patients were followed prospectively for at least 12 months. Thirty children were initially enrolled, and 27 (14 boys, 13 girls) completed the study. At the end of 12 weeks of cisapride treatment, there was a significant increase in the number of bowel movements per week (1.43 +/- 0.52 to 6.48 +/- 4.16; p < 0.05) and significant decreases in the number of accidents per day (2.86 +/- 2.71 to 0.52 +/- 1.23; p < 0.05) and doses of laxatives used per week (14.33 +/- 5.84 to 3.37 +/- 7.10; p < 0.05). Encopresis disappeared in 65.2\% of cases (p < 0.0001) and improved in 26\%. Sixty-nine percent of the patients stopped using laxatives (p < 0.001). After 12 weeks 18 patients (66.6\%) were asymptomatic, seven (25.9\%) showed some improvement in bowel movement frequency and number of accidents, and two (7.4\%) showed no improvement. The cisapride was well tolerated. After long-term follow-up (20 +/- 9.8 months), 37\% of patients had recovered (asymptomatic and off laxatives and cisapride) and 29.6\% were still asymptomatic but were using laxatives or cisapride. There were no differences in baseline characteristics between recovered and nonrecovered patients. We conclude that cisapride is effective in the treatment of some children with intractable constipation without pelvic floor dyssynergia. [\hyperlink{Cisatracurium Besylate}{PMID: 8788285}, S Nurko et al., 1996]

\section*{Clonidine Hydrochloride}
\subsection*{Result}
\subsubsection*{Answer}

Neonates/Infants (<1 year): Unknown  
Children (1-12 years): Yes  
Adolescents (13-18 years): Yes  

\subsubsection*{{Explanation}}
\hypertarget{Clonidine Hydrochloride}
A review of the available abstracts reveals multiple targeted studies evaluating the safety of clonidine hydrochloride in children across various age ranges and indications. Below is a summary of the evidence, organized by age group:

Neonates and Infants:
- One pilot study investigated clonidine for neonatal narcotic abstinence syndrome in infants passively addicted to narcotics. Six of seven infants responded to clonidine (3-4 mcg/kg/day) with no toxic side effects observed at the dosage used. The authors suggest clonidine may be safe for this indication, but note that its use remains investigational and comparative safety with other regimens is undetermined [\hyperlink{pmid_6597462}{PMID: 6597462}, E L Hoder et al., 1984].
- A large retrospective study of IV clonidine infusion for analgosedation in critically ill children (age 0-18 years, median age 12.9 months) found that while bradycardia and hypotension were common, these were not clinically significant and did not require intervention. Younger age was a risk factor for bradycardia, but overall, clonidine was hemodynamically well tolerated [\hyperlink{pmid_29912068}{PMID: 29912068}, Niina Kleiber et al., 2018].
- A review article notes that while clonidine appears safe in healthy children in the perioperative setting, there is an urgent need for high-quality trials in children younger than 12 months and those with hemodynamic instability [\hyperlink{pmid_31045639}{PMID: 31045639}, Arash Afshari et al., 2019].

Children (1-12 years):
- Multiple randomized controlled trials and observational studies in children aged 1-12 years (including studies of premedication, ADHD, tic disorders, and pain management) consistently report that clonidine is generally well tolerated. The most common side effects are mild-to-moderate somnolence and occasional bradycardia, which rarely require intervention [\hyperlink{pmid_21241954}{PMID: 21241954}, Rakesh Jain et al., 2011; \hyperlink{pmid_18182964}{PMID: 18182964}, W Burleson Daviss et al., 2008; \hyperlink{pmid_37203849}{PMID: 37203849}, K-D Zeng et al., 2023; \hyperlink{pmid_26083572}{PMID: 26083572}, Peter G Larsson et al., 2015; \hyperlink{pmid_11448249}{PMID: 11448249}, V Agarwal et al., 2001; \hyperlink{pmid_8836273}{PMID: 8836273}, K Nishina et al., 1996; \hyperlink{pmid_8561317}{PMID: 8561317}, K Mikawa et al., 1996].
- Studies of oral, intravenous, and topical clonidine in children (including those with ADHD, tic disorders, and for perioperative use) affirm its safety profile, with adverse events being mild and transient. Bradycardia is noted but is infrequent and not clinically significant in most cases.
- A case report of topical clonidine for herpetic neuralgia in a 9-year-old found no side effects [\hyperlink{pmid_12166288}{PMID: 12166288}, Reiko Hagihara et al., 2002].
- Studies of compounded oral and capsule formulations for pediatric use confirm the stability and suitability of these preparations for children [\hyperlink{pmid_30124097}{PMID: 30124097}, V Merino-Bohórquez et al., 2019; \hyperlink{pmid_36186250}{PMID: 36186250}, Maya Wasilewski et al., 2022; \hyperlink{pmid_22580108}{PMID: 22580108}, A L de Goede et al., 2012; \hyperlink{pmid_34465373}{PMID: 34465373}, Jumpei Saito et al., 2021].

Adolescents (13-18 years):
- Randomized controlled trials of clonidine (including extended-release formulations) for ADHD in children and adolescents (up to 17 years) show that clonidine is generally well tolerated, with mild adverse events such as somnolence and occasional bradycardia [\hyperlink{pmid_21241954}{PMID: 21241954}, Rakesh Jain et al., 2011; \hyperlink{pmid_21555501}{PMID: 21555501}, Scott H Kollins et al., 2011].
- A large poison center database review (children <19 years) found that most clonidine exposures resulted in no or minor effects, but moderate and major effects (including one fatality in a 23-month-old) can occur, especially in overdose situations [\hyperlink{pmid_11929375}{PMID: 11929375}, Wendy Klein-Schwartz et al., 2002]. This highlights the importance of dosing accuracy and monitoring.

Overdose and Poisoning:
- Several case reports and retrospective studies document significant toxicity (lethargy, bradycardia, hypotension, respiratory depression) in children following accidental or dosing errors with clonidine, sometimes requiring intensive care but with most children recovering without sequelae [\hyperlink{pmid_6523727}{PMID: 6523727}, J T Algren et al., 1984; \hyperlink{pmid_6849304}{PMID: 6849304}, M Artman et al., 1983; \hyperlink{pmid_19655285}{PMID: 19655285}, Mariya Farooqi et al., 2009; \hyperlink{pmid_11483818}{PMID: 11483818}, M J Romano et al., 2001; \hyperlink{pmid_33036823}{PMID: 33036823}, Michael S Toce et al., 2021]. These reports do not contradict the safety of therapeutic use but underscore the risk of overdose and the need for careful dosing and monitoring.

Summary:
- For neonates and infants, limited data suggest clonidine may be safe when used under close monitoring, but more high-quality studies are needed, especially in those under 12 months.
- For children (1-12 years) and adolescents (13-18 years), multiple targeted studies affirm the safety of clonidine hydrochloride for various indications, with mild and manageable side effects.
- Overdose can cause serious toxicity in all pediatric age groups, so careful dosing and monitoring are essential.

\subsection*{Abstracts}
\hypertarget{pmid_30124097}{P}ediatric patients present changing physiological features. Because of the lack of land suitable for commercial management, pediatric specialties very often need to prepare extemporaneous formulations to improve the dosage and administration of drugs for children. Oral liquid formulations are the most suitable for pediatric patients. Clonidine is widely used in the pediatric population for opioid withdrawal, hypertensive crisis, attention deficit disorders and hyperactivity syndrome, and as an analgesic in neuropathic cancer pain. The objective was to study the physicochemical and microbiological stability and determine the shelf life of an oral solution containing 20 µg/mL clonidine hydrochloride in different storage conditions (5 ± 3 °C, 25 ± 3 °C, and 40 ± 2 °C). Using raw material with excipients safe for all pediatric age groups, two oral liquid formulations of clonidine hydrochloride were designed (with and without preservatives). Solutions stored at 5 ± 3 °C (with and without preservatives) were physically and microbiologically stable for at least 90 days in closed containers and for 42 days after opening. Two oral solutions of clonidine hydrochloride 20 µg/mL were developed for pediatric use from raw materials that are readily available and easy to process, containing safe excipients that are stable over a long period of time. [\hyperlink{Clonidine Hydrochloride}{PMID: 30124097}, V Merino-Bohórquez et al., 2019]

\hypertarget{pmid_37203849}{T}he current research was designed to assess the efficacy of clonidine in the treatment of children with tic disorder co-morbid with attention deficit hyperactivity disorder. A total of 154 children with tic disorder co-morbid with attention deficit hyperactivity disorder admitted to our hospital from July 2019 to July 2022 were recruited and assigned to receive either methylphenidate hydrochloride plus haloperidol (observation group) or clonidine (experimental group), with 77 cases in each group. Outcome measures included clinical efficacy, Yale Global Tic Severity Scale (YGTSS) scores, Conners Parent Symptom Questionnaire (PSQ) scores, and adverse events. Clonidine was associated with markedly higher clinical efficacy vs. methylphenidate hydrochloride plus haloperidol (p<0.05). Clonidine offered more significant mitigation of the tic disorder vs. methylphenidate hydrochloride plus haloperidol, as evinced by the lower kinetic tic scores, vocal tic scores, and total scores (p<0.05). Children exhibited markedly milder tic symptoms after clonidine monotherapy vs. those with dual therapy of methylphenidate hydrochloride and haloperidol, suggested by the lower scores of character problems, learning problems, psychosomatic disorders, hyperactivity/impulsivity, anxiety index, and hyperactivity index (p<0.05). Clonidine features a higher safety profile than methylphenidate hydrochloride plus haloperidol by reducing the incidence of adverse events (p<0.05). Clonidine effectively alleviates tic symptoms, reduces attention deficit and hyperactivity/impulsivity in children with tic disorder co-morbid attention deficit hyperactivity disorder, and features a high safety profile. [\hyperlink{Clonidine Hydrochloride}{PMID: 37203849}, K-D Zeng et al., 2023]

\hypertarget{pmid_6523727}{C}lonidine hydrochloride (CH) is an antihypertensive drug with complex pharmacologic activity including central and peripheral alpha-adrenergic stimulation and CNS depression. We reviewed the records of 5 children admitted to our Pediatric Intensive Care Unit following accidental ingestion of CH. All patients presented with lethargy or stupor, beginning 20-60 minutes after ingestion. Respiratory depression or apnea occurred in 4, requiring endotracheal intubation in 2 and mechanical ventilation in 1. All 5 developed mild to moderate hypertension, and 3 developed asymptomatic bradycardia. The dose of CH ingested was estimated to be 0.2-0.4 mg in 4 out of 5 patients. Treatment consisted of efforts to prevent absorption of CH from the GI tract and supportive care. All signs of CH toxicity resolved within 6-14 hours. Four patients were transferred from ICU within 24 hours and discharged home the following day. One patient developed post-extubation stridor and atelectasis. Significant toxicity occurred even though the amount of CH ingested was relatively small in at least 4 or 5 patients. Transient hypertension occurred early in the hospital course of all patients and resolved without treatment. Hypotension and symptomatic bradycardia were not observed. Apnea was the most serious abnormality observed. All patients recovered without significant morbidity. [\hyperlink{Clonidine Hydrochloride}{PMID: 6523727}, J T Algren et al., 1984]

\hypertarget{pmid_21241954}{T}his study examined the efficacy and safety of clonidine hydrochloride extended-release tablets (CLON-XR) in children and adolescents with attention-deficit/hyperactivity disorder (ADHD). This 8-week, placebo-controlled, fixed-dose trial, including 3 weeks of dose escalation, of patients 6 to 17 years old with ADHD evaluated the efficacy and safety of CLON-XR 0.2 mg/day or CLON-XR 0.4 mg/day versus placebo in three separate treatment arms. Primary endpoint was mean change in ADHD Rating Scale-IV (ADHD-RS-IV) total score from baseline to week 5 versus placebo using a last observation carried forward method. Secondary endpoints were improvement in ADHD-RS-IV inattention and hyperactivity/impulsivity subscales, Conners Parent Rating Scale-Revised: Long Form, Clinical Global Impression of Severity, Clinical Global Impression of Improvement, and Parent Global Assessment from baseline to week 5. Patients (N = 236) were randomized to receive placebo (n = 78), CLON-XR 0.2 mg/day (n = 78), or CLON-XR 0.4 mg/day (n = 80). Improvement from baseline in ADHD-RS-IV total score was significantly greater in both CLON-XR groups versus placebo at week 5. A significant improvement in ADHD-RS-IV total score occurred between groups as soon as week 2 and was maintained throughout the treatment period. In addition, improvement in ADHD-RS-IV inattention and hyperactivity/impulsivity subscales, Conners Parent Rating Scale-Revised: Long Form, Clinical Global Impression of Improvement, Clinical Global Impression of Severity, and Parent Global Assessment, occurred in both treatment groups versus placebo. The most common treatment-emergent adverse event was mild-to-moderate somnolence. Changes on electrocardiogram were minor and reflected the known pharmacology of clonidine. Clonidine hydrochloride extended-release tablets were generally well tolerated by patients in the study and significantly improved ADHD symptoms in this pediatric population. [\hyperlink{Clonidine Hydrochloride}{PMID: 21241954}, Rakesh Jain et al., 2011]

\hypertarget{pmid_12166288}{W}e report a case of effective treatment with clonidine ointment for herpetic neuralgia in a child. Clonidine hydrochloride is an alpha 2-agonist. It is generally administered intravenously, intramuscularly, intrathecally and orally. However, there have been only a few reports on transdermal usage. In our department, we have investigated the analgesic effect of topical application of clonidine in adults, and we have obtained sufficient evidence on the effects of clonidine. Therefore, we decided to use clonidine to a child. A 9-year-old child who had undergone BMT and developed herpes zoster was experiencing severe pain, itch, and insomnia. Many drugs were ineffective in relieving the pain, itch, and insomnia. To remove the symptoms, we tried clonidine ointment. Immediate improvement was observed in all the symptoms. Therefore, clonidine ointment was thought to be effective and we decided to prescribe clonidine ointment and amitriptyline hydrochloride. The application of clonidine showed no side effects such as bradycardia and low blood pressure. We conclude that clonidine can be administered to children without causing side effects. [\hyperlink{Clonidine Hydrochloride}{PMID: 12166288}, Reiko Hagihara et al., 2002]

\hypertarget{pmid_11929375}{T}o analyze the trends, demographics, and toxic effects associated with pediatric clonidine hydrochloride exposures reported to poison centers. Retrospective. Clonidine-only exposures followed up to known outcome in children younger than 19 years reported to the American Association of Poison Control Center's database from January 1, 1993, through December 31, 1999. Frequency of exposures over time, acuity, reason, symptoms, management site, treatment, and outcome. There were 10 060 reported exposures with 57\% reported for children younger than 6 years, 34\% for children between 6 and 12 years old, and 9\% for adolescents between 13 and 18 years old. In 1999 there were 2.5 times as many exposures as in 1993. In 6- through 12-year-olds, clonidine was the child's medication in 35\% of the exposures, compared with 10\% in children younger than 6 years and 26\% in adolescents. The proportion of cases involving the child's medication increased over 7 years. While unintentional overdose was most common in children younger than 6 years, therapeutic errors and suicide attempts predominated in 6- through 12-year-olds and adolescents, respectively. In 6042 symptomatic children (60\%), the most common symptoms were lethargy (80\%), bradycardia (17\%), hypotension (15\%), and respiratory depression (5\%). Most exposures resulted in no effect (40\%) or minor effects (39\%). Moderate effects occurred in 1907 children (19\%), major effects in 230 children (2\%); there was 1 fatality in a 23-month-old. While most of the clonidine exposures resulted in minimal toxic effects, serious toxic effects and death can occur. The trend toward increasing the number of exposures in children, especially with evidence of toxic effects in children receiving clonidine therapeutically, is cause for concern. [\hyperlink{Clonidine Hydrochloride}{PMID: 11929375}, Wendy Klein-Schwartz et al., 2002]

\hypertarget{pmid_6597462}{C}lonidine hydrochloride, an alpha 2-adrenergic agonists, was used to treat seven infants who were passively addicted to narcotics because of maternal methadone maintenance. In six of seven infants, the major symptoms of narcotic withdrawal were ameliorated after a total daily oral dose of 3-4 micrograms/kg/day was achieved. One infant failed to respond. No toxic side effects of clonidine were observed at the dosage level used. The results of this pilot study suggest that clonidine may be a safe therapeutic agent for the treatment of neonatal narcotic abstinence syndrome (NNAS). Clonidine treatment of NNAS remains strictly investigational at this time. The relative efficacy and safety of clonidine versus other currently used drug regimens for NNAS also remains to be determined. [\hyperlink{Clonidine Hydrochloride}{PMID: 6597462}, E L Hoder et al., 1984]

\hypertarget{pmid_36186250}{C}lonidine hydrochloride is an antihypertensive, centrally acting α2 adrenergic agonist with various pediatric indications. For pediatric patients, 20-mcg clonidine hydrochloride capsules can be compounded from commercial tablets or from a pre-compounded titrated powder. These methods should be compared to ensure the best quality for the high-risk patients, and a beyond-use date should be established. Eight experimental batches were made from commercial tablets and 8 were made from microcrystalline cellulose (MCC)-based titrated powders. Quality controls were performed to determine the best compounding protocol. Stability study was conducted on capsules compounded with the best method. Of 8 batches manufactured from commercial tablets, 7 were compliant for both clonidine mean content and content uniformity, whereas 7 of 8 batches manufactured from titrated powders were not. A clonidine loss during compounding was evidenced by surface sampling analyses. Clonidine hydrochloride 20-mcg capsules' mean content remained higher than 90\% of initial content for 1 year when stored at 25°C with 60\% relative humidity and protected from light. Commercial tablets should be preferred to 1\% clonidine hydrochloride and MCC titrated powder made from the active pharmaceutical ingredient. Twenty-microgram clonidine hydrochloride capsules made from commercial tablets are stable for 1 year when stored under managed ambient storage condition. [\hyperlink{Clonidine Hydrochloride}{PMID: 36186250}, Maya Wasilewski et al., 2022]

\hypertarget{pmid_26083572}{C}lonidine has been advocated as a valid alternative for premedication in children but one of the few limitations is its association with reduced heart rate (HR), which thus raises the question of the safety of clonidine as premedication in children. The aim of this study was to investigate the incidence of bradycardia in children premedicated with oral or intravenous clonidine as compared to children not receiving pharmacologic premedication. An open, nonrandomized, observational study design was used. During the preoperative assessment visit the children were prescribed no premedication, intravenous or oral clonidine. On arrival to the operating room (OR) HR was recorded by connecting the patient to standard monitoring with pulseoximetry and/or Electrocardiogram. The primary outcome measure was the number of patients with a HR below 85\% of the lower limit of the normal range (1st centile), which was defined as bradycardia that might need clinical intervention. One thousand five hundred and seven patients were included in the analysis. 600 and 85 patients did not receive any premedication (Group 0), 305 patients received iv Clonidine (Group CIV), and 517 patients were given oral Clonidine (Group CPO). One patient in Group 0 (0.15\%; 95\% CI: 0-0.81\%), none in Group CIV (0\%; 95\% CI: 0.00-0.98\%), and 5 patients in Group CPO (0.97\%; 95\% CI: 0.31-2.24\%) were observed to have a HR of <85\% of the 1st centile. The incidence of bradycardia following oral or intravenous premedication with clonidine in a pediatric population scheduled for anesthesia is low. Thus, it does not appear rational to refrain from using clonidine as premedication in children only due to fear of bradycardia. [\hyperlink{Clonidine Hydrochloride}{PMID: 26083572}, Peter G Larsson et al., 2015]

\hypertarget{pmid_29912068}{C}lonidine is an antihypertensive drug used for analgosedation in the PICU. Lack of reliable data on its hemodynamic tolerance limits its use. This study explores the hemodynamic tolerance of IV clonidine infusion in a broad population of children with high severity of disease. Retrospective analysis of prospectively collected data. A tertiary and quaternary referral PICU. Critically ill children age 0-18 years old who received an IV clonidine infusion for analgosedation of at least 1 hour. None. The primary endpoints were the prevalences of bradycardia and hypotension. Secondary endpoints were changes in heart rate, blood pressure, Vasoactive-Inotropic Score, COMFORT Behavior score (a sedation scoring scale), and body temperature during the infusion. The association of bradycardia with other hemodynamic variables was explored, as well as potential risk factors for severe bradycardia. One-hundred eighty-six children (median age, 12.9 mo [interquartile range, 3.5-60.6 mo]) receiving a maximum median clonidine infusion of 0.7 µg/kg/hr (interquartile range, 0.3-1.5) were included. Severe bradycardia and systolic hypotension occurred in 72 patients (40.2\%) and 105 patients (58\%), respectively. Clonidine-associated bradycardia was hemodynamically well tolerated, as it was not related with hypotension and the need for vasoactive drugs decreased in parallel with a sedation score guided clonidine infusion rate increase. Younger age was the only identified risk factor for clonidine-associated bradycardia. Although administration of clonidine is often associated with bradycardia and hypotension, these complications do not seem clinically significant in a mixed PICU population with a high degree of disease severity. Clonidine may have a vasoactive-inotropic sparing effect. [\hyperlink{Clonidine Hydrochloride}{PMID: 29912068}, Niina Kleiber et al., 2018]

\hypertarget{pmid_8836273}{O}ral clonidine given as a premedicant in adults has been shown to reduce intraoperative inhalation anaesthetic requirements and provide perioperative haemodynamic stability. We conducted the current study to ascertain whether or not these beneficial effects of clonidine can be reproduced in children. In a prospective, randomized, double-blind, controlled clinical trial, 60 children (ASA I) aged 5-11 yr, received placebo (control), 2 micrograms kg-1 clonidine, or 4 micrograms kg-1 clonidine orally 105 min before induction of anaesthesia. Anaesthesia was induced with halothane, nitrous oxide in oxygen via mask and maintained with halothane and 60\% nitrous oxide in oxygen. The halothane concentration was titrated to the concentration required to maintain haemodynamic stability (defined as 20\% of blood pressure (BP) and heart rate (HR)) for maintenance of anaesthesia. The end-tidal concentration of halothane was monitored throughout anaesthesia. On completion of surgery, nitrous oxide and halothane were discontinued. Following confirmation of recovery from anaesthesia and muscle relaxation, the endotracheal tube was removed. Higher inspired concentrations of halothane (\%) were required in the control and 2 micrograms kg-1 clonidine-treated groups (mean SD: 1.1 +/- 0.2 and 1.0 +/- 0.2, respectively) than in the 4 micrograms kg-1 clonidine-treated group (0.6 +/- 0.1) for haemodynamic stability (P < 0.05). Clonidine, 4 micrograms kg-1, significantly reduced the intraoperative lability (coefficient of variation) of systolic and diastolic BP and HR compared with the other two regimens. Oral clonidine premedication at a dose of 4 micrograms kg-1 provided intraoperative haemodynamic stability and reduced anaesthetic requirements in children. However, we are unable to extrapolate these observations to younger children and infants. [\hyperlink{Clonidine Hydrochloride}{PMID: 8836273}, K Nishina et al., 1996]

\hypertarget{pmid_11448249}{A} 12-week, double-blind, randomized, placebo-controlled trial of oral clonidine in three fixed doses (4, 6, and 8 mcg/kg/day) using a crossover design was conducted with 10 children who had hyperkinetic disorder (mean age 7.6 years +/-.54). All had comorbid mental retardation. Both parents' ratings on the Parent Symptom Questionnaire and clinicians' ratings on the Hillside Behaviour Rating Scale showed a marked dose-related response to clonidine in hyperactivity, impulsivity, and inattention. Drowsiness was a common side effect of clonidine. It wore off by the 2nd to 4th week in most cases. Thus, clonidine is a safe and effective medication in young hyperkinetic children with comorbid mental retardation. [\hyperlink{Clonidine Hydrochloride}{PMID: 11448249}, V Agarwal et al., 2001]

\hypertarget{pmid_34465373}{C}lonidine hydrochloride is used to treat sedative agent withdrawals, malignant hypertension, and anesthesia complications. Clonidine is also prescribed off-label to pediatric patients at a dose of 1 μg/kg. The commercially available enteral form of clonidine, Catapres® tablets, is often compounded into a powder form by pharmacists to achieve dosage adjustments for administration to pediatric patients. However, the stability and quality of compounded clonidine powder have not been verified. The objectives of this study were to formulate a 0.2 mg/g oral clonidine hydrochloride powder and assess the stability and physical properties of this compounded product in storage. A 0.2 mg/g clonidine powder was prepared by adding lactose monohydrate to crushed and filtrated clonidine tablets. The powder was stored in polycarbonate amber bottles or coated paper packages laminated with cellophane and polyethylene. The stability of clonidine at 25 °C ± 2 °C and 60\% ± 5\% relative humidity was examined over a 120-d period in "bottle (closed)," "bottle (in use)," and "laminated paper" storage conditions. Drug dissolution and powder X-ray diffraction analysis were conducted to assess physicochemical stabilities. Validated liquid chromatography-diode array detection was used to detect and quantify clonidine and its degradation product, 2,6-dichloroaniline (2,6-DCA). Clonidine content was maintained between 90.0 and 110.0\% of the initial contents in all packaging and storage conditions. After 120 d of storage, 2,6-DCA was not detected, and no crystallographic and dissolution changes were observed. Compounded clonidine powder stability was maintained for 120 d at 25 °C ± 2 °C and 60\% ± 5\% relative humidity. This information may contribute to the management of clonidine compounded powder in community and hospital pharmacies in Japan. [\hyperlink{Clonidine Hydrochloride}{PMID: 34465373}, Jumpei Saito et al., 2021]

\hypertarget{pmid_22580108}{M}any drugs are unavailable in suitable paediatric dosage forms. We describe the development and validation of a stable paediatric oral formulation of clonidine hydrochloride 50 μg/ml, allowing individualised paediatric dosing and easy administration. Stability of the extemporaneously compounded formulation of clonidine hydrochloride was assessed using a validated HPLC method. Clonidine hydrochloride was stable in the buffered aqueous solution at room temperature for up to 9 months. The described formulation is chemically stable for at least 9 months when stored in brown 100 ml PET bottles at room temperature, enabling adequate oral treatment in paediatric patients. [\hyperlink{Clonidine Hydrochloride}{PMID: 22580108}, A L de Goede et al., 2012]

\hypertarget{pmid_12673018}{C}lonidine hydrochloride has been used for pre-anesthetic medication to provide a pre-operative sedation in pediatric surgery. The purpose of this study is to determine the plasma clonidine concentration, which gives satisfactory sedation in pediatric surgery. Sixteen pediatric patients (age: 1-11 years, weight: 9-33 kg) received either 2 or 4 microg/kg of clonidine lollipop before entering the operating room. Plasma clonidine concentrations were determined 120 min after administration of clonidine lollipop. Pre-operative sedation was evaluated by 5-point scoring systems at entering the operating room. The changes in systolic blood pressure (SBP), diastolic blood pressure (DBP) and heart rate (HR) were also assessed before and after administration of clonidine lollipop. The patients with satisfactory sedation had higher plasma clonidine concentration than that of the patients with unsatisfactory sedation (0.45+/-0.16 ng/ml vs. 0.26+/-0.16 ng/ml, p<0.05). The clonidine concentrations in the satisfactory group ranged from 0.28 to 0.81 ng/ml. There was no significant difference in hemodynamic parameters (SBP, DBP and HR) before and after administration of clonidine lollipop in both satisfactory and unsatisfactory sedation groups. Plasma clonidine concentration of 0.3-0.8 ng/ml would be sufficient to produce satisfactory sedation without changes in hemodynamic parameters in pediatric surgery. [\hyperlink{Clonidine Hydrochloride}{PMID: 12673018}, Kenji Sumiya et al., 2003]

\hypertarget{pmid_18182964}{T}o examine the safety and tolerability of clonidine used alone or with methylphenidate in children with attention-deficit/hyperactivity disorder (ADHD). In a 16-week multicenter, double-blind trial, 122 children with ADHD were randomly assigned to clonidine (n = 31), methylphenidate (n = 29), clonidine and methylphenidate (n = 32), or placebo (n = 30). Doses were flexibly titrated up to 0.6 mg/day for clonidine and 60 mg/day for methylphenidate (both with divided dosing). Groups were compared regarding adverse events and changes from baseline to week 16 in electrocardiograms and vital signs. There were more incidents of bradycardia in subjects treated with clonidine compared with those not treated with clonidine (17.5\% versus 3.4\%; p =.02), but no other significant group differences regarding electrocardiogram and other cardiovascular outcomes. There were no suggestions of interactions between clonidine and methylphenidate regarding cardiovascular outcomes. Moderate or severe adverse events were more common in subjects on clonidine (79.4\% versus 49.2\%; p =.0006) but not associated with higher rates of early study withdrawal. Drowsiness was common on clonidine, but generally resolved by 6 to 8 weeks. Clonidine, used alone or with methylphenidate, appears safe and well tolerated in childhood ADHD. Physicians prescribing clonidine should monitor for bradycardia and advise patients about the high likelihood of initial drowsiness. [\hyperlink{Clonidine Hydrochloride}{PMID: 18182964}, W Burleson Daviss et al., 2008]

\hypertarget{pmid_6849304}{C}lonidine hydrochloride poisoning in children has become more frequent with increasing availability of this drug. We report four cases of accidental clonidine poisoning that demonstrate the various signs and symptoms of clonidine poisoning. The most frequent and significant toxic effects are depression of consciousness, bradycardia, hypotension, and respiratory depression. Ventilatory support must be available if apnea occurs. Bradycardia can be treated with atropine sulfate, epinephrine chloride, dopamine hydrochloride, or tolazoline hydrochloride. Hypotension is treated with intravenous fluids and dopamine, reserving tolazoline for refractory cases. Hypothermia is common but is of minor clinical significance. Paradoxical hypertension should be treated with tolazoline. Clonidine may not be detected in body fluids by routine toxicology-screening procedures, so poisoning should be suspected on clinical grounds. [\hyperlink{Clonidine Hydrochloride}{PMID: 6849304}, M Artman et al., 1983]

\hypertarget{pmid_16301230}{C}lonidine is effective in treating sevoflurane-induced postanesthesia agitation in children. We conducted a study on 169 children to quantify the risk reduction of clonidine agitation in patients admitted to our day-surgery pediatric clinic. Children were randomly allocated to receive clonidine 2 mug/kg or placebo before general anesthesia with sevoflurane that was also supplemented with a regional or central block. An observer blinded to the anesthetic technique assessed recovery variables and the presence of agitation. Pain and discomfort scores were significantly decreased in the clonidine group; the incidence of agitation was reduced by 57\% (P = 0.029) and the incidence of severe agitation by 67\% (P = 0.064). Relative risks for developing agitation and severe agitation were 0.43 (95\% confidence interval, 0.24-0.78) and 0.32 (0.09-1.17), respectively. Clonidine produces a substantial reduction in the risk of postsevoflurane agitation in children. [\hyperlink{Clonidine Hydrochloride}{PMID: 16301230}, Simonetta Tesoro et al., 2005]

\hypertarget{pmid_19655285}{C}lonidine is frequently prescribed to children. Clonidine overdose in children has resulted in major clinical effects and deaths. A 3.5-year-old male with a history of a seizure disorder and night terrors presented following difficulty walking, excessive sleeping, agitation when awake, and possible seizure activity. Chronic medications were valproic acid (VPA) and clonidine. On presentation, he alternated between poor responsiveness and agitation, with initial vitals: blood pressure, BP 144/76 mmHg; heart rate, 65 bpm; respiratory rate, 18 bpm; temperature 99.5 degrees F; and pulse oximetry 96\% on room air. VPA level was 35 microg/mL. A toxicology consult the next day noted a dry mouth, 2-mm pupils, intermittent gasping, and central nervous system (CNS) depression, with a diagnostic impression of clonidine overdose. The caregiver had been giving 1 mL (0.1 mg) qd of a pharmacy-compounded clonidine suspension by a provided syringe. The pharmacy procedure record agreed with the physicians order. The amount dispensed was a 30-day supply but the bottle was empty on day 19, leading us to suspect a possible accelerated dosing error. The concentration in the bottle thus could not be confirmed. The child slowly returned to his baseline state over 48 hours. A serum clonidine level drawn approximately 18 hours after his last dose later returned at 300 ng/mL (reference range = 0.5-4.5 ng/mL). Compounding and liquid dosing errors are common in children and may result in massive overdoses. There was an accelerated dosing error, but whether a compounding or suspension error or even an acute overdose occurred as well is unknown. Particular care should be taken with medications that have low therapeutic indices, that are extemporaneously compounded, and are prepared as liquids, where medication errors are more likely. [\hyperlink{Clonidine Hydrochloride}{PMID: 19655285}, Mariya Farooqi et al., 2009]

\hypertarget{pmid_3053867}{W}e evaluated the tolerance and effectiveness of the oral clonidine test for GH in 75 children, 84\% with hyposomia and 16\% with other diseases. The test was well tolerated, since 97\% of the examined children had no side effects with the exception of occasional drowsiness, pallor and myosis of short duration. Two of the children at the end of the test, had more severe symptoms 30 min after (deep asthenia, pallor and a further small blood pressure drop) which however, resolved after 4-6 h. No correlation was observed between the clinical picture and the drops in blood pressure and/or plasma cortisol in the children examined. We confirm the effectiveness of the clonidine test in the release of GH since in our study we observed no negative false subnormal responses. [\hyperlink{Clonidine Hydrochloride}{PMID: 3053867}, R De Angelis et al., 1988]

\hypertarget{pmid_33036823}{P}ediatric clonidine ingestions frequently result in emergency department visits and admission for cardiac monitoring. Detailed information on the clinical course and specifically time of vital sign abnormalities of these patients is lacking. The objective of this study was to provide descriptive analysis of the rates and times to vital sign abnormalities, treatment, disposition, and outcomes in a single-center cohort of pediatric patients with report of clonidine poisoning. We performed a retrospective cohort study of patients younger than 21 years who presented to a large, urban, tertiary care center with a report of single substance clonidine exposure between January 2004 and November 2017. Patients were dichotomized into younger (≤9 years or younger) and older (10-21 years) groups based on the expected physiologic and psychologic differences between older and younger children. Eighty-eight patients met our inclusion criteria. Younger patients (≤9 years or younger; n = 47) were more likely to be exposed to someone else's medication (53\%) and older patients (10-21 years; n = 41) overwhelmingly (85\%) were exposed to their own medication. Thirty-nine (45\%) became bradycardic, 27 (32\%) became bradypneic, and 38 (44\%) became hypotensive. Eighty percent of patients had depressed mental status. Thirty-three (38\%) patients received at least one dose of naloxone (median 0.07 mg/kg; interquartile range 0.03-0.11 mg/kg). Of those who received naloxone, 50\% had a documented clinical response. In this study of patients at a pediatric tertiary referral center, pediatric patients with report of clonidine exposures were likely to exhibit altered mental status and frequently develop vital sign abnormalities. Naloxone exhibited some effectiveness; given its wide safety margin, high-dose naloxone should be used in critically poisoned non-opioid-dependent patients. Because adolescents are much more likely to ingest their own clonidine medication, counseling with parents and other caregivers regarding safe medication storage is paramount. [\hyperlink{Clonidine Hydrochloride}{PMID: 33036823}, Michael S Toce et al., 2021]

\hypertarget{pmid_31045639}{C}lonidine, an α2-receptor agonist is a widely used drug in pediatrics with a large scope of indications ranging from prevention of postoperative emergence agitation, analgesia, anxiolysis, sedation, weaning to shivering. In the era of 'opioid-free' medicine with much attention be directed toward increasing problems with opioid use, clonidine due to its global availability, low cost and safety profile has become an even more interesting option. Increasing evidence from randomised clinical trials support the use of clonidine in healthy children in the perioperative setting. Clonidine appears to significantly reduce postoperative emergence agitation, opioid consumption, shivering, nausea and vomiting. In addition, emerging evidence support the use of clonidine for sedation of critically ill children in ICUs. In this review, the current evidence for clonidine in pediatrics is described and analyzed including a meta-analysis for prevention of emergence agitation. Clonidine appears a safe and beneficial drug with moderate to high-quality evidence supporting its use in pediatric anesthesia. However, for some indications and populations such as children younger than 12 months old and those with hemodynamic instability, there is an urgent need for high-quality trials. [\hyperlink{Clonidine Hydrochloride}{PMID: 31045639}, Arash Afshari et al., 2019]

\hypertarget{pmid_21555501}{T}o assess the efficacy and safety of clonidine hydrochloride extended-release tablets (CLON-XR) combined with stimulants (ie, methylphenidate or amphetamine) for attention-deficit/hyperactivity disorder (ADHD). In this phase 3, double-blind, placebo-controlled trial, children and adolescents with hyperactive- or combined-subtype ADHD who had an inadequate response to their stable stimulant regimen were randomized to receive CLON-XR or placebo in combination with their baseline stimulant medication. Predefined efficacy measures evaluated change from baseline to week 5. Safety was assessed by spontaneously reported adverse events, vital signs, electrocardiogram recordings, and clinical laboratory values. Improvement from baseline for all efficacy measures was evaluated using analysis of covariance. Of 198 patients randomized, 102 received CLON-XR plus stimulant and 96 received placebo plus stimulant. At week 5, greater improvement from baseline in ADHD Rating Scale IV (ADHD-RS-IV) total score (95\% confidence interval: -7.83 to -1.13; P = .009), ADHD-RS-IV hyperactivity and inattention subscale scores (P = .014 and P = .017, respectively), Conners' Parent Rating Scale scores (P < .062), Clinical Global Impression of Severity (P = .021), Clinical Global Impression of Improvement (P = .006), and Parent Global Assessment (P = .001) was observed in the CLON-XR plus stimulant group versus the placebo plus stimulant group. Adverse events and changes in vital signs in the CLON-XR group were generally mild. The results of this study suggest that CLON-XR in combination with stimulants is useful in reducing ADHD in children and adolescents with partial response to stimulants. [\hyperlink{Clonidine Hydrochloride}{PMID: 21555501}, Scott H Kollins et al., 2011]

\hypertarget{pmid_11483818}{A} 5-year-old child who weighed 17.5 kg received 50 mg of clonidine. The amount ingested was confirmed by analysis of the suspension administered (clonidine HCl 9.78 mg/mL). To our knowledge, this represents the largest ingestion in a child and the largest ingestion on a milligram per kilogram basis in the medical literature. The child's initial presentation included hyperventilation, an unusual feature of clonidine toxicity. The child was discharged without sequela 42 hours after admission. A serum concentration of clonidine 17 hours postingestion was 64 ng/mL, the highest reported to date in a pediatric patient. The intoxication was traced to a pharmacy compounding error in which milligrams were substituted for micrograms. Increased prescribing of clonidine in young children coupled with the requirement to compound clonidine in a suspension and the narrow therapeutic index suggests that the frequency of severe ingestions in children will increase in the future. [\hyperlink{Clonidine Hydrochloride}{PMID: 11483818}, M J Romano et al., 2001]

\hypertarget{pmid_8561317}{C}lonidine is an effective preanesthetic medication in children, providing a preoperative sedative effect. The analgesic properties of the drug have been well documented in adults. The current study was designed to investigate the effect of oral clonidine given preoperatively on postoperative pain in children undergoing minor surgery. In a prospective, randomized, controlled clinical trial, 90 children aged 5-12 yr undergoing elective ophthalmic, urologic, and otologic surgery received placebo (control), clonidine 2 micrograms/kg, or clonidine 4 micrograms/kg. These drugs were administered 105 min before the estimated time of induction of anesthesia and followed by treatment with oral atropine 0.03 mg/kg 60 min before anesthesia. Anesthesia was induced and maintained with halothane and nitrous oxide in oxygen. Postoperative pain was assessed by a blinded observer using an objective pain scale (OPS). Clonidine 4 micrograms/kg provided lower OPS (highest) scores during 12 h after surgery and reduced requirement for postoperative supplementary analgesic (diclofenac suppository) compared with the other two regimens. These data suggest that oral clonidine premedication (4 micrograms/kg) is a possible approach to facilitating postoperative analgesia in children undergoing minor surgery. [\hyperlink{Clonidine Hydrochloride}{PMID: 8561317}, K Mikawa et al., 1996]

\section*{Colistimethate Sodium}
\subsection*{Result}
\subsubsection*{Answer}

Yes (Neonates, including preterm and extremely low birth weight)
Yes (Infants and children up to 18 years)

\subsubsection*{{Explanation}}
\hypertarget{Colistimethate Sodium}
Based on the abstracts available, there are multiple targeted studies evaluating the safety of Colistimethate Sodium (colistin) in children, including neonates, infants, and older pediatric patients. Below is a summary of the relevant evidence by age range:

Neonates (including preterm and extremely low birth weight):
- Several studies specifically evaluated intravenous Colistimethate Sodium in neonates with multidrug-resistant Gram-negative infections. These studies report favorable clinical outcomes, low rates of nephrotoxicity, and no neurotoxicity. For example, in a study of 18 critically ill neonates (10 term, 8 preterm), colistimethate was found to be safe and efficacious, with only two cases of increased serum creatinine, both associated with multiple organ dysfunction and/or co-administration of another nephrotoxic drug [\hyperlink{pmid_21245777}{PMID: 21245777}, Mamta Jajoo et al., 2011]. Another study of 21 neonates with multidrug-resistant Acinetobacter sepsis found no renal impairment and concluded colistin was effective and safe [\hyperlink{pmid_26868136}{PMID: 26868136}, Manar Al-Lawama et al., 2016].

Infants and Children (up to 18 years):
- Multiple prospective and retrospective studies, as well as case series, have evaluated intravenous Colistimethate Sodium in children with severe infections due to multidrug-resistant Gram-negative bacteria. These studies consistently report that Colistimethate Sodium is effective and generally safe, with low rates of nephrotoxicity and no significant neurotoxicity observed [\hyperlink{pmid_25691180}{PMID: 25691180}, Poddutoor Preetham Kumar et al., 2015; \hyperlink{pmid_20003141}{PMID: 20003141}, Solmaz Celebi et al., 2010; \hyperlink{pmid_27994915}{PMID: 27994915}, Ayşe Karaaslan et al., 2016; \hyperlink{pmid_20119725}{PMID: 20119725}, Elias Iosifidis et al., 2010; \hyperlink{pmid_19116601}{PMID: 19116601}, Matthew E Falagas et al., 2009]. One study included children from 8 days to 15 years and found Colistimethate Sodium to be safe and effective [\hyperlink{pmid_20003141}{PMID: 20003141}, Solmaz Celebi et al., 2010].

- Inhaled Colistimethate Sodium has also been studied in children with cystic fibrosis aged ≥6 years, showing good tolerability and safety [\hyperlink{pmid_23135343}{PMID: 23135343}, Antje Schuster et al., 2013]. A small case series of inhaled colistin in critically ill children without cystic fibrosis also reported no toxicity [\hyperlink{pmid_20658485}{PMID: 20658485}, Matthew E Falagas et al., 2010].

- Pharmacokinetic studies in children (including those as young as 3 months) highlight variability in drug exposure and suggest that current dosing may be suboptimal, but these studies do not report significant safety concerns [\hyperlink{pmid_33782000}{PMID: 33782000}, Charalampos Antachopoulos et al., 2021; \hyperlink{pmid_30722017}{PMID: 30722017}, Mong How Ooi et al., 2019; \hyperlink{pmid_25874300}{PMID: 25874300}, V I Zakharevich et al., 2015; \hyperlink{pmid_27276179}{PMID: 27276179}, Narongsak Nakwan et al., 2016]. One study in neonates noted that standard dosing may result in suboptimal plasma concentrations, but did not report safety issues [\hyperlink{pmid_27276179}{PMID: 27276179}, Narongsak Nakwan et al., 2016].

- Studies of loading dose strategies in children suggest possible benefits in efficacy, with no increase in nephrotoxicity or other adverse events, but further research is needed [\hyperlink{pmid_31956632}{PMID: 31956632}, Shiva Fatehi et al.; \hyperlink{pmid_32179149}{PMID: 32179149}, Noppadol Wacharachaisurapol et al., 2020].

Summary:
There are multiple targeted studies in neonates, infants, and children (up to 18 years) that affirm the safety of Colistimethate Sodium for the treatment of severe infections caused by multidrug-resistant Gram-negative bacteria. These studies report low rates of nephrotoxicity and no significant neurotoxicity, even in critically ill and preterm neonates. However, optimal dosing in children, especially neonates, may require further study to ensure efficacy, but current evidence supports safety.

\subsection*{Abstracts}
\hypertarget{pmid_25691180}{T}o observe the safety and efficacy of Colistimethate sodium in children infected with gram-negative bacteria, susceptible only to colistimethate sodium. This prospective observational study done over 2 years observed children who received colistin for >48 h, for renal failure as defined by p-RIFLE criteria. Out of 68 children, 52 (76.5\%) survived. There were three children with evidence of acute kidney injury and none had neurotoxicity. Serum creatinine significantly decreased at 48 h and at end of treatment, from that at beginning of therapy (P=0.007). Colistimethate sodium is effective against carbapenem-resistant Gram-negative bacteria, and is safe in children. [\hyperlink{Colistimethate Sodium}{PMID: 25691180}, Poddutoor Preetham Kumar et al., 2015]

\hypertarget{pmid_20003141}{T}he aim of the present study was to assess the efficacy and safety of colistimethate sodium therapy in multidrug-resistant nosocomial infections caused by Pseudomonas aeruginosa or Acinetobacter baumannii in neonates and children. Pediatric patients hospitalized at the Uludag University Hospital who had nosocomial infections caused by multidrug-resistant P. aeruginosa or A. baumannii, were enrolled in the study. Colistimethate sodium at a dosage of 50-75 x 10(3) U/kg per day was given i.v. divided into three doses. Fifteen patients received 17 courses of colistimethate sodium for the following infections: ventilator-associated pneumonia (n= 14), catheter-related sepsis (n= 1) and skin and soft-tissue infection (n= 2). The mean age of patients was 53.2 + 74.7 months (range, 8 days-15 years) and 60\% were male. Mortality was 26.6\%. Colistimethate sodium appears to be safe and effective for the treatment of severe infections caused by multidrug-resistant P. aeruginosa or A. baumannii in pediatric patients. [\hyperlink{Colistimethate Sodium}{PMID: 20003141}, Solmaz Celebi et al., 2010]

\hypertarget{pmid_20585114}{U}sing a liquid chromatography-tandem mass spectrometry method, the serum and cerebrospinal fluid (CSF) concentrations of colistin were determined in patients aged 1 months to 14 years receiving intravenous colistimethate sodium (60,000 to 225,000 IU/kg of body weight/day). Only in one of five courses studied (a 14-year-old receiving 225,000 IU/kg/day) did serum concentrations exceed the 2 microg/ml CLSI/EUCAST breakpoint defining susceptibility to colistin for Pseudomonas and Acinetobacter. CSF colistin concentrations were <0.2 microg/ml but increased in the presence of meningitis (approximately 0.5 microg/ml or 34 to 67\% of serum levels). [\hyperlink{Colistimethate Sodium}{PMID: 20585114}, Charalampos Antachopoulos et al., 2010]

\hypertarget{pmid_23135343}{T}o assess efficacy and safety of a new dry powder formulation of inhaled colistimethate sodium in patients with cystic fibrosis (CF) aged ≥6 years with chronic Pseudomonas aeruginosa lung infection. A prospective, centrally randomised, phase III, open-label study in patients with stable CF aged ≥6 years with chronic P aeruginosa lung infection. Patients were randomised to Colobreathe dry powder for inhalation (CDPI, one capsule containing colistimethate sodium 1 662 500 IU, twice daily) or three 28-day cycles with twice-daily 300 mg/5 ml tobramycin inhaler solution (TIS). Study duration was 24 weeks. 380 patients were randomised. After logarithmic transformation of data due to a non-normal distribution, adjusted mean difference between treatment groups (CDPI vs TIS) in change in forced expiratory volume in 1 s (FEV1\% predicted) at week 24 was -0.98\% (95\% CI -2.74\% to 0.86\%) in the intention-to-treat population (n=373) and -0.56\% (95\% CI -2.71\% to 1.70\%) in the per protocol population (n=261). The proportion of colistin-resistant isolates in both groups was ≤1.1\%. The number of adverse events was similar in both groups. Significantly more patients receiving CDPI rated their device as 'very easy or easy to use' (90.7\% vs 53.9\% respectively; p<0.001). CDPI demonstrated efficacy by virtue of non-inferiority to TIS in lung function after 24 weeks of treatment. There was no emergence of resistance of P aeruginosa to colistin. Overall, CDPI was well tolerated. TRIAL REG NO: EudraCT 2004-003675-36. [\hyperlink{Colistimethate Sodium}{PMID: 23135343}, Antje Schuster et al., 2013]

\hypertarget{pmid_21245777}{N}osocomial infection due to multidrug-resistant Gram-negative pathogens in intensive care units is a challenge for clinicians and microbiologists, and has led to resurgence of parenteral colistin use in the last decade. Safety and efficacy data regarding intravenous colistin (colistimethate) use in neonates is sparse. We present our experience of efficacy and safety of colistimethate in the treatment of sepsis in critically sick term and preterm neonates. The records of the neonates who received colistimethate in a neonatal intensive care unit of a tertiary care center from January 2009 to December 2009 were reviewed. Eighteen critically sick neonates (10 term and 8 preterm) received 21 courses of colistimethate (dose ranging from 50,000 to 75,000 IU/kg/d) for treatment of pneumonia, blood stream infections, meningitis, and empyema thoracis. The isolated pathogens in decreasing order of frequency were Acinetobacter baumannii, Klebsiella pneumoniae, Pseudomonos aeruginosa, and Enterobacter. Mean duration of colistimethate was 13.1 days/course (range: 5-21 days). At least one other antibiotic was coadministered in all courses. A favorable clinical outcome occurred in 16 of 21 (76\%) courses, 5 patients died due to severe sepsis with multiple organ dysfunction. Microbiologic clearance was documented in 17 courses. Increase in serum creatinine by > 0.5 mg/dL above baseline in 2 babies was associated with the presence of multiple organ dysfunction syndrome in both and coadministration of netilmicin in one. Colistimethate intravenous administration appears to be safe and efficacious for multidrug-resistant Gram-negative infections in neonates, including preterm and extremely low birth weight neonates. [\hyperlink{Colistimethate Sodium}{PMID: 21245777}, Mamta Jajoo et al., 2011]

\hypertarget{pmid_28741653}{C}hloral hydrate is commonly used to sedate children for painless procedures. Children may recover more quickly after sedation with dexmedetomidine, which has a shorter half-life. We randomly allocated 196 children to chloral hydrate syrup 50 mg.kg [\hyperlink{Colistimethate Sodium}{PMID: 28741653}, V M Yuen et al., 2017] Limited pharmacokinetic (PK) data suggest that currently recommended pediatric dosages of colistimethate sodium (CMS) by the Food and Drug Administration and European Medicines Agency may lead to suboptimal exposure, resulting in plasma colistin concentrations that are frequently <2 mg/liter. We conducted a population PK study in 17 critically ill patients 3 months to 13.75 years (median, 3.3 years) old who received CMS for infections caused by carbapenem-resistant Gram-negative bacteria. CMS was dosed at 200,000 IU/kg/day (6.6 mg colistin base activity [CBA]/kg/day; 6 patients), 300,000 IU/kg/day (9.9 mg CBA/kg/day; 10 patients), and 350,000 IU/kg/day (11.6 mg CBA/kg/day; 1 patient). Plasma colistin concentrations were determined using ultraperformance liquid chromatography combined with electrospray ionization-tandem mass spectrometry. Colistin PK was described by a one-compartment disposition model, including creatinine clearance, body weight, and the presence or absence of systemic inflammatory response syndrome (SIRS) as covariates ( [\hyperlink{Colistimethate Sodium}{PMID: 28741653}, Charalampos Antachopoulos et al., 2021] The increasing frequency of infections caused by multidrug-resistant (MDR) Gram-negative bacteria has led to the reappraisal of colistimethate use. We present a case series of critically ill pediatric patients without cystic fibrosis who received intravenous colistimethate treatment. All available relevant medical records were reviewed. Seven children without cystic fibrosis (mean age 7.7 years; 2 female), admitted to the intensive care unit of a tertiary-care pediatric hospital in Athens, Greece, were identified to have received intravenous colistimethate during October 2004 to May 2008. MDR Acinetobacter baumannii, Pseudomonas aeruginosa, and/or Klebsiella pneumoniae were isolated from blood and/or bronchial secretions specimens in 6 of 7 reported patients. All isolates were susceptible to colistin. All 7 patients received intravenous colistimethate in a dosage of 5 mg/kg daily (divided in 3 equal doses, administered every 8 hours). Five children received colistimethate for 10 days and the remaining 2 for 2 and 23 days, respectively. The infections caused by MDR Gram-negative bacteria were improved in 6 children with microbiologically documented infections. Five of the 7 children were discharged from the ICU. The remaining 2 children died (1 of them had received colistimethate for 2 days); their death was not attributed to MDR Gram-negative infection. No nephrotoxicity or other type of toxicity of colistimethate was noted in this case-series. Although the small number of included cases precludes any firm conclusions, our study suggests that colistimethate may have a role for the treatment of infections caused by MDR Gram-negative bacteria in critically ill pediatric patients. [\hyperlink{Colistimethate Sodium}{PMID: 28741653}, Matthew E Falagas et al., 2009]

\hypertarget{pmid_31956632}{P}harmacokinetic and clinical studies recommend applying loading dose of colistin for the treatment of severe infections in the critically ill adults. Pharmacokinetic studies of colistin in children also highlight the need for a loading dose. However, there are no clinical studies evaluating the effectiveness of colistin loading dose in children. In a randomized trial, children with ventilator-associated pneumonia or central line-associated bloodstream infection (CLABSI) for whom colistin was initiated, were enrolled. Patients were randomized into two groups; loading dose and conventional dose treatment arms. In the conventional treatment arm, colistimethate sodium was initiated with maintenance dose. In the loading dose group, colistimethate sodium was commenced with a loading dose of 150,000 international unit/kg, then on the maintenance dose. Both treatment arms also received meropenem as combination therapy. Primary outcomes were overall efficacy, clinical improvement and microbiological cure. Secondary outcomes were colistin-induced nephrotoxicity and development of resistance. Thirty children completed this study. There was a significantly higher overall efficacy in the group received loading dose (42.9 vs. 6.3\%,  This preliminary study suggests that colistin loading dose might have some benefits in critically ill children, specifically in children with CLABSI. Further trials are required to elucidate colistin best dosing strategy in critically ill children with severe infections. [\hyperlink{Colistimethate Sodium}{PMID: 31956632}, Shiva Fatehi et al., ]

\hypertarget{pmid_27994915}{T}he emergence of infections due to multidrug-resistant  In this study, we aimed to evaluate the clinical efficacy and safety of colistin use in critically ill pediatric patients. This study has a retrospective study design. Sixty-one critically ill children were identified through the department's patient files archive during the period from January 2011 to November 2014. Twenty-nine females and thirty-two males with a mean±standard deviation (SD) age of 61±9 months (range 0-216, median 12 months) received IV colistin due to MDR-GNB infections. Bacteremia (n=23, 37.7\%) was the leading diagnosis, followed by pneumonia (n=19, 31\%), clinical sepsis (n=7, 11.4\%), wound infection (n=6, 9.8\%), urinary tract infection (n=5, 8.1\%) and meningitis (n=1, 1.6\%). All of the isolates were resistant to carbapenems; however, all were susceptible to colistin. The isolated microorganisms in decreasing order of frequency were:  Colistin appears to be a safe and efficacious drug for treating MDR-GNB infections in children. [\hyperlink{Colistimethate Sodium}{PMID: 27994915}, Ayşe Karaaslan et al., 2016]

\hypertarget{pmid_30722017}{I}ntravenous colistin is widely used to treat infections in pediatric patients. Unfortunately, there is a paucity of pharmacological information to guide the selection of dosage regimens. The daily dose recommended by the US Food and Drug Administration (FDA) and European Medicines Agency (EMA) is the same body weight-based dose traditionally used in adults. The aim was to increase our understanding of the patient factors that influence the plasma concentration of colistin, and assess the likely appropriateness of the FDA and EMA dosage recommendations. There were 5 patients, with a median age of 1.75 (range 0.1-6.25) years, a median weight of 10.7 (2.9-21.5) kg, and a median creatinine clearance of 179 (44-384) mL/min/1.73m2, who received intravenous infusions of colistimethate each 8 hours. The median daily dose was 0.21 (0.20-0.21) million international units/kg, equivalent to 6.8 (6.5-6.9) mg of colistin base activity per kg/day. Plasma concentrations of colistimethate and formed colistin were subjected to population pharmacokinetic modeling to explore the patient factors influencing the concentration of colistin. The median, average, steady-state plasma concentration of colistin (Css,avg) was 0.88 mg/L; individual values ranged widely (0.41-3.50 mg/L), even though all patients received the same body weight-based daily dose. Although the daily doses were \textasciitilde{}33\% above the upper limit of the FDA- and EMA-recommended dose range, only 2 patients achieved Css,avg ≥2mg/L; the remaining 3 patients had Css,avg <1mg/L. The pharmacokinetic covariate analysis revealed that clearances of colistimethate and colistin were related to creatinine clearance. The FDA and EMA dosage recommendations may be suboptimal for many pediatric patients. Renal functioning is an important determinant of dosing in these patients. [\hyperlink{Colistimethate Sodium}{PMID: 30722017}, Mong How Ooi et al., 2019]

\hypertarget{pmid_3168097}{G}olytely solution is now commonly used in preoperative bowel preparation or in colonoscopy and barium enema in adults. Studies have demonstrated its effectiveness and good acceptance in regards to clinical as well as biological point of view. In children, it has been used more recently, but since 1984 several teams agree to find the method excellent. Our study aimed to confirm there was no electrolytic movement caused by golytely and that using it without reserve in children was possible, even in the very young ones. Children are generally very sensible to those movements and mainly as they have a general anaesthesia in the hours following the golytely administration for investigation or surgery. Up to now, 54 children from 4 months to 18 years aged have been studied. Besides the good quality of the preparation noted by the operator and the good clinical tolerance, no significant change of the sodium, potassium, chloride, creatinine and proteins has been noticed. Only urea has decreased very lightly but not out of norms. These results confirm that golytely is safe and effective in preparing the bowel in children. [\hyperlink{Colistimethate Sodium}{PMID: 3168097}, A Bichet-Sicard et al., 1988]

\hypertarget{pmid_37283451}{D}ata regarding the treatment of childhood granulomatous periorificial dermatitis (CGPD) using oral therapies are limited. This study included 31 Chinese children with CGPD treated with oral roxithromycin. After 12 weeks of treatment, 90.3\% of the patients recovered, and there were no severe adverse effects. Our results suggest that oral roxithromycin is an effective and safe treatment for CGPD. [\hyperlink{Colistimethate Sodium}{PMID: 37283451}, Senfen Wang et al., ]

\hypertarget{pmid_27276179}{I}n this study, we sought to evaluate the pharmacokinetics of colistin after intravenous administration of colistimethate sodium (CMS) in the critically ill neonates with Gram-negative bacterial infections. A single intravenous dose of CMS [approximately 150,000 IU/kg, equivalent to 5 mg/kg colistin base activity (CBA)] was administered to 7 critically ill neonates. Mean (±SD) maximum plasma colistin concentration and area under the time-concentration curve from 0 to infinity were 3.0 ± 0.7 µg/mL and 25.3 ± 10.4 µg·h/mL, respectively. Time to maximum concentration, half-life, apparent volume of distribution and clearance were 1.3 ± 0.9 hours, 9.0 ± 6.5 hours, 7.7 ± 9.3 L/kg and 0.6 ± 0.3 L/h/kg, respectively. After a dose regimen of 5 mg/kg CBA every 24 hours, the average concentration expected at steady state is 1.1 ± 0.4 µg/mL. In critically ill neonates, a single intravenous dose of 5 mg CBA/kg (approximately 150,000 IU/kg of CMS) resulted in suboptimal plasma concentrations of colistin. According to our pharmacokinetics data, the dosage of CMS currently used in critically ill neonates is insufficient. [\hyperlink{Colistimethate Sodium}{PMID: 27276179}, Narongsak Nakwan et al., 2016]

\hypertarget{pmid_26868136}{N}eonatal sepsis caused by multidrug-resistant gram-negative bacteria has been reported in different parts of the world. It is a major threat to neonatal care, carrying a high rate of morbidity and mortality. While Colistin is the treatment of choice, few studies have reported its use in neonatal patients. A retrospective descriptive study of all neonatal patients who had multidrug-resistant Acinetobacter sepsis and were treated with Colistin over a 2-year period. Patients' charts and hospital laboratory data were reviewed. During the study period, 21 newborns were treated with Colistin. All had sepsis evident by positive blood culture and clinical signs of sepsis. The median gestational age and birth weight were 33 weeks (26-39) and 1700 g (700-3600), respectively. Nine (43 \%) were very low birth weight infants. Eighteen (86 \%) were preterm infants. Nineteen (91 \%) newborns survived. No renal impairment is documented in any of our patients. Fourteen (67 \%) of our patients had elevated eosinophil counts following Colistin treatment, for those patients, the average eosinophilic counts ± standard deviation before and after Colistin therapy were 149.08 ± 190.38 to 1193 ± 523.29, respectively, with a p value of less than 0.0001. Our study showed that Colistin was both effective and safe for treating multidrug-resistant Acinetobacter neonatal sepsis. This is a retrospective study. No universal protocol was used for the patients. The factors that might affect the response or cause side effects are difficult to evaluate. [\hyperlink{Colistimethate Sodium}{PMID: 26868136}, Manar Al-Lawama et al., 2016]

\hypertarget{pmid_15951862}{D}iagnostic and therapeutic procedures in children are made easier using sedation. However, there is no consensus about which drug should be used to achieve this. Furthermore, none of the drugs used for sedation are risk free. The aim of this work is to study sedation indications, effectiveness, and safety at our center. A prospective observational study conducted at the Pediatric Day Care Unit, King Fahad National Guard Hospital, Riyadh, Saudi Arabia. The study covered 17.5 weeks in 2 periods: May 9th 1999 to June 13th 1999 and October 31st 2001 to February 11th 2002. Children <12 years were included. Collected data included demographics, indication, drug dosing and outcome. Data were reported as mean +/- SD. We included 148 patients, age 38 +/- 30 months. Adequate sedation was achieved in 79\% after initial chloral hydrate (CH) dose of 56.9 +/- 9.3 mg/kg, in 95\% after adding 18.5 +/- 6.4 mg/kg CH and in 96\% after adding second drug. Compared to nonrespondents, first CH dose respondents were younger and lower in weight. The CH side effects were few and mild. Chloral hydrate is a safe and effective agent for sedation in children with an age and weight dependent response. [\hyperlink{Colistimethate Sodium}{PMID: 15951862}, Omar M Hijazi et al., 2005]

\hypertarget{pmid_27083755}{T}o review the evidence for the efficacy and safety of colchicine in children with pericarditis. Systematic review. The following databases were searched for studies about colchicine in children with pericarditis (June 2015): Cochrane Central, Medline, EMBASE and LILACS. All observational and experimental studies on humans with any length of follow-up and no limitations on language or publication status were included. The outcomes studied were recurrences of pericarditis and adverse events. Two authors extracted data and assessed quality of included studies using the Cochrane risk of bias tool for non-randomised trials. Two case series and nine case reports reported the use of colchicine in a total of 86 children with pericarditis. Five articles including 74 paediatric patients were in favour of colchicine in preventing further pericarditis recurrences. Six studies including 12 patients showed that colchicine did not prevent recurrences of pericarditis. No randomised controlled trials (RCTs) were found. Although colchicine is an established treatment for pericarditis in adults, it is not routinely used in children. There is not enough evidence to support or discourage the use of colchicine in children with pericarditis. Further research in the form of large double-blind RCTs is needed to establish the efficacy of colchicine in children with pericarditis. [\hyperlink{Colistimethate Sodium}{PMID: 27083755}, Samer Alabed et al., 2016]

\hypertarget{pmid_20645317}{P}ediatric ulcerative colitis (UC) has a more severe phenotype, reflected by more extensive disease and a higher rate of acute severe exacerbations. The pooled steroid-failure rate among 291 children from five studies is 34\% (95\% confidence interval [CI]: 27\%-41\%). It is suggested that corticosteroids should be dosed between 1-1.5 mg/kg up to 40-60 mg daily. Food restriction has a limited role in severe UC and should be generally discouraged in children who do not have a surgical abdomen. Appraisal of radiologic findings in children must recognize the variation in colonic width with age and size. Data suggest that the Pediatric UC Activity Index (PUCAI), determined at day 3, should be used to screen for patients likely to fail corticosteroids (>45 points), and at day 5 to dictate the introduction of second-line therapy (>65-70 points). Cyclosporine is successful in children with severe colitis but its use should be restricted to 3-4 months while bridging to thiopurine treatment (pooled short-term success rate 81\% [95\% CI: 76\%-86\%]; n = 94 from eight studies). Infliximab may be as effective as cyclosporine (75\% pooled short-term response (95\% CI: 67\%-83\%); n = 126, six studies) with a pooled 1-year response of 64\% (95\% CI: 56\%-72\%). In toxic megacolon, in patients refractory to one salvage medical therapy, and in chronic severe disease, colectomy may be preferred. Decision-making regarding colectomy in children must consider the toxicity of medication consumed over many future years, the quality of life and self-image associated with either choice, as well as both functional outcomes and, in females, fertility following pouch procedures. [\hyperlink{Colistimethate Sodium}{PMID: 20645317}, Dan Turner et al., 2011]

\hypertarget{pmid_32179149}{U}se of colistin in children is rising in line with the increase of multidrug-resistant Gram-negative bacteria (MDR-GNB). In adults, a colistin loading dose is recommended to achieve therapeutic concentrations within 12-24 h. Here we aimed to describe the pharmacokinetic (PK) parameters of a loading dose versus a recommended initial dose of intravenous colistimethate sodium (CMS) in paediatric patients. A prospective, open-label, PK study was conducted in paediatric patients (age 2-18 years) with normal renal function. Patients (n = 20) were randomly assigned to receive either a CMS loading dose (LD group) of 4 mg of colistin base activity (CBA)/kg/dose or a standard initial dose (NLD group) of 2.5 mg (12-h interval) or 1.7 mg (8-h interval) of CBA/kg/dose. Serial blood samples were collected. Plasma concentrations of formed colistin were measured by LC-MS/MS. PK parameters were reported. Acute kidney injury (AKI) was monitored by serum creatinine and urine NGAL. The median (interquartile range) age and body weight were 8.5 (3.5-11.3) years and 21.5 (13.5-20.0) kg. The mean (standard deviation) of first-dose PK parameters of the LD group versus the NLD group were: C [\hyperlink{Colistimethate Sodium}{PMID: 32179149}, Noppadol Wacharachaisurapol et al., 2020] Emergence of multidrug-resistant Gram-negative nosocomial pathogens has led to resurgence of colistin use. Safety and efficacy data regarding colistin use in pediatric patients are sparse, while optimal dosage has not been defined. We present a case series of neonates and children without cystic fibrosis treated with various doses of colistin intravenously. The records of patients who received colistin in a tertiary-care hospital from January 2007 to March 2009 were reviewed. Thirteen patients (median age 5 years, range 22 days to 14 years) received 19 courses of colistin as treatment of pneumonia, central nervous system infection, bacteremia, or complicated soft tissue infection. The isolated pathogens were Acinetobacter baumannii, Enterobacter cloacae, Klebsiella pneumoniae, Pseudomonas aeruginosa, and Stenotrophomonas maltophilia. Daily dose of colistin (colistimethate) ranged between 40,000 and 225,000 IU/kg. Duration of administration ranged from 1 to 133 days. Other antimicrobials were co-administered in 18/19 courses. Increase of serum creatinine in one patient was associated with co-administration of colistin and gentamicin. Sixteen of 19 courses had a favorable outcome, and only two of the three deaths were infection-related. Colistin intravenous administration appears well tolerated even at higher than previously recommended doses and of prolonged duration. [\hyperlink{Colistimethate Sodium}{PMID: 32179149}, Elias Iosifidis et al., 2010]

\hypertarget{pmid_28275979}{S}edation is often required for children undergoing diagnostic procedures. Chloral hydrate has been one of the sedative drugs most used in children over the last 3 decades, with supporting evidence for its efficacy and safety. Recently, chloral hydrate was banned in Italy and France, in consideration of evidence of its carcinogenicity and genotoxicity. Dexmedetomidine is a sedative with unique properties that has been increasingly used for procedural sedation in children. Several studies demonstrated its efficacy and safety for sedation in non-painful diagnostic procedures. Dexmedetomidine's impact on respiratory drive and airway patency and tone is much less when compared to the majority of other sedative agents. Administration via the intranasal route allows satisfactory procedural success rates. Studies that specifically compared intranasal dexmedetomidine and chloral hydrate for children undergoing non-painful procedures showed that dexmedetomidine was as effective as and safer than chloral hydrate. For these reasons, we suggest that intranasal dexmedetomidine could be a suitable alternative to chloral hydrate. [\hyperlink{Colistimethate Sodium}{PMID: 28275979}, Giorgio Cozzi et al., 2017]

\hypertarget{pmid_21531030}{C}hloral hydrate (CH) is an oral sedative widely used to sedate infants and young children during auditory brainstem response (ABR) testing. The aim of this study was to record effectiveness, complications and safety of CH as a sedative for ABR. From January of 2003 until December of 2007, 1903 children were tested for ABR, 568 of them being under the age of 6 months. CH (8\%) was used for sedation at a dose of 40 mg/kg with a repeat dose, if necessary, for an adequate sedation, in 20-30 min. We recorded the effectiveness of CH as a sedative for ABR examination, as well as all complications related to the use of CH such as vomiting, rash, hyperactivity, respiratory distress and apnea. The statistical method used was the absolute and percentage frequency distribution of the occurrences. Sedation with CH was necessary to perform testing in 1591 (83.6\%) of the examined children. However, in the population of the examined infants, only 341 (60\%) were sedated with CH, because the remaining 227 (40\%) fell asleep by themselves. Complications included hyperactivity in 152 children (8\%), minor respiratory distress in 10 children (0.4\%), vomiting in 217 children (11.4\%), apnea in 4 children (0.2\%) and rash in 10 children (0.4\%). The complications of hyperactivity, vomiting and rash resolved without any medical treatment. The apnea cases were managed effectively by supplying ventilation to the children via a mask in the presence of an anesthesiologist. The use of CH at a dose of 40 mg/kg up to 80 mg/kg is safe and effective when administered in a setting with adequate equipment and the presence of well trained personnel. [\hyperlink{Colistimethate Sodium}{PMID: 21531030}, Eirini Avlonitou et al., 2011]

\hypertarget{pmid_25874300}{T}he prevalence of hospital strains of P. aeruginosa, A. baumannii and K. pneumoniae characterizing by multiple drug resistance to overwhelming majority of antibiotics anew evoked an increased interest to colistin. However, until now there is not enough information concerning pharmacokinetics of colistin to optimize dosage of this pharmaceutical. The study was carried out to analyze pharmacokinetics of both colistin and sodium colistimitate in children with chemically induced neutropenia. To quantitatively detect colistin in blood serum the technique of highly effective fluid chromatography with mass spectrometry was applied. The concentration of colistin was detected in 21 children with chemically induced neutropenia (13 patients with septicemia, 8 children of control group) after intravenous injection of sodium colistimitate. The significant variability of pharmacokinetic parameters of colistin was established both in patients with septicemia and in control group. The technique of highly effective fluid chromatography with mass spectrometry can be applied for therapeutic medicinal monitoring and optimization regimen of dosage. [\hyperlink{Colistimethate Sodium}{PMID: 25874300}, V I Zakharevich et al., 2015]

\hypertarget{pmid_20658485}{D}ata regarding the role of inhaled colistin in critically ill pediatric patients without cystic fibrosis are scarce. Three children (one female), admitted to the intensive care unit (ICU) of a tertiary-care pediatric hospital in Athens, Greece, during 2004-2009 received inhaled colistin as monotherapy for tracheobronchitis (two children), and as adjunctive therapy for necrotizing pneumonia (one child). Colistin susceptible Acinetobacter baumannii and Pseudomonas aeruginosa were isolated from the cases' bronchial secretions specimens. All three children received inhaled colistin at a dosage of 75 mg diluted in 3 ml of normal saline twice daily (1,875,000 IU of colistin daily), for a duration of 25, 32, and 15 days, respectively. All three children recovered from the infections. Also, a gradual reduction, and finally total elimination of the microbial load in bronchial secretions was observed during inhaled colistin treatment in the reported cases. All three cases were discharged from the ICU. No bronchoconstriction or any other type of toxicity of colistin was observed. In conclusion, inhaled colistin was effective and safe for the treatment of two children with tracheobronchitis, and one child with necrotizing pneumonia. Further studies are needed to clarify further the role of inhaled colistin in pediatric critically ill patients without cystic fibrosis. [\hyperlink{Colistimethate Sodium}{PMID: 20658485}, Matthew E Falagas et al., 2010]

\hypertarget{pmid_33655976}{C}hildren evaluated in the emergency department for head trauma often undergo computed tomography (CT), with some uncooperative children requiring pharmacological sedation. Chloral hydrate (CH) is a sedative that has been widely used, but its rectal use for child sedation after head trauma has rarely been studied. The objective of this study was to document the safety and efficacy of rectal CH sedation for cranial CT in young children.We retrospectively studied all the children with head trauma who received rectal CH sedation for CT in the emergency department from 2016 to 2019. CH was administered rectally at a dose of 50 mg/kg body weight. When sedation was achieved, CT scanning was performed, and the children were monitored until recovery. The sedative safety and efficacy were analyzed.A total of 135 children were enrolled in the study group, and the mean age was 16.05 months. The mean onset time was 16.41 minutes. Successful sedation occurred in 97.0\% of children. The mean recovery time was 71.59 minutes. All of the vital signs were within normal limits after sedation, except 1 (0.7\%) with transient hypoxia. There was no drug-related vomiting reaction in the study group. Adverse effects occurred in 11 patients (8.1\%), but all recovered completely. Compared with oral CH sedation, rectal CH sedation was associated with quicker onset (P < .01), higher success rate (P < .01), and lower adverse event rate (P < .01).Rectal CH sedation can be a safe and effective method for CT imaging of young children with head trauma in the emergency department. [\hyperlink{Colistimethate Sodium}{PMID: 33655976}, Quanmin Nie et al., 2021]

\section*{Cosyntropin}
\subsection*{Result}
\subsubsection*{Answer}

Neonates/Infants (up to 1 month): Unknown  
Children (general pediatric population, postdural puncture headache): Yes  
Other pediatric age groups/indications: Unknown  

\subsubsection*{{Explanation}}
\hypertarget{Cosyntropin}
To determine if Cosyntropin is safe for use in children, I reviewed all available abstracts for targeted pediatric safety studies.

1. **Neonates and Infants (Extremely Low Birth Weight Infants, \textasciitilde{}18-28 days old):**
   - One prospective, nested study tested different cosyntropin doses (0.1 and 1.0 mcg/kg) in 276 extremely low birth weight infants (500-999g, 18-28 days old) to assess adrenal function. The study focused on the diagnostic utility of cosyntropin and reported no mention of adverse effects or safety concerns. However, the primary outcome was not safety, and the abstract does not explicitly affirm safety, only that the test was performed and recommendations were made for dosing [\hyperlink{pmid_16159938}{PMID: 16159938}, Kristi L Watterberg et al., 2005].

2. **Children (Pediatric Patients, Age Range Not Specified, but likely broad):**
   - A retrospective chart review described the use of intravenous cosyntropin for refractory pediatric postdural puncture headache in a tertiary care pediatric hospital. Over 5 years, 26 patients received 37 doses (5-15 mcg/kg, median 10.4 mcg/kg). No adverse effects related to treatment were identified. The study concludes that cosyntropin is a potential alternative to epidural blood patch in pediatric patients who fail conservative management, but also notes that further research is warranted [\hyperlink{pmid_31789828}{PMID: 31789828}, Michelle K Zapapas et al., 2020]. This study is specifically about children and reports no adverse effects, supporting safety in this context.

3. **Children (Pediatric ICU, Age Range Not Specified):**
   - A retrospective cohort study examined cosyntropin stimulation test results in 76 pediatric ICU patients with catecholamine- and fluid-resistant shock. The study does not mention adverse effects or safety concerns, nor does it focus on safety as a primary outcome [\hyperlink{pmid_29151100}{PMID: 29151100}, Pallavi Iyer et al., 2018].

4. **Other Studies:**
   - Several other abstracts mention cosyntropin stimulation testing in children as a diagnostic tool (e.g., for adrenal suppression after corticosteroid use or as a safety measure in drug trials), but these do not focus on cosyntropin safety itself and do not report adverse events related to cosyntropin administration [\hyperlink{pmid_17300659}{PMID: 17300659}, Lawrence Eichenfield et al., 2007; \hyperlink{pmid_19441606}{PMID: 19441606}, Steven Weinstein et al., 2009].

**Summary by Age Range:**
- **Neonates/Infants (up to 1 month):** No targeted safety study; safety is unknown.
- **Children (general pediatric population, including those with postdural puncture headache):** One targeted study (n=26) found no adverse effects, suggesting cosyntropin is safe for this indication, but the sample size is small and the study is retrospective.
- **Other pediatric age groups:** No targeted safety studies; safety is unknown.

**Conclusion:** There is one targeted pediatric study (for postdural puncture headache) that affirms safety in children, but for other age groups and indications, safety is either not addressed or unknown.

\subsection*{Abstracts}
\hypertarget{pmid_16159938}{V}arious cosyntropin doses are used to test adrenal function in premature infants, without consensus on appropriate dose or adequate response. The objective of this study was to test the cortisol response of extremely low birth weight infants to different cosyntropin doses and evaluate whether these doses differentiate between groups of infants with clinical conditions previously associated with differential response to cosyntropin. The design was a prospective, nested study conducted within a randomized clinical trial of low-dose hydrocortisone from November 1, 2001, to April 30, 2003. The setting was nine newborn intensive care units. The patients included infants with 500-999 g birth weight. The drug used was cosyntropin, at 1.0 or 0.1 microg/kg, given between 18 and 28 d of birth. We measured the cortisol response to cosyntropin. Two hundred seventy-six infants were tested. Previous hydrocortisone treatment did not suppress basal or stimulated cortisol values. Cosyntropin, at 1.0 vs. 0.1 microg/kg, yielded higher cortisol values (P < 0.001) and fewer negative responses (2 vs. 21\%). The higher dose, but not the lower dose, showed different responses for girls vs. boys (P = 0.02), infants receiving enteral nutrition vs. not (P < 0.001), infants exposed to chorioamnionitis vs. not (P = 0.04), and those receiving mechanical ventilation vs. not (P = 0.02), as well as a positive correlation with fetal growth (P = 0.03). A response curve for the 1.0-microg/kg dose for infants receiving enteral nutrition (proxy for clinically well infants) showed a 10th percentile of 16.96 microg/dl. Infants with responses less than the 10th percentile had more bronchopulmonary dysplasia and longer length of stay. A cosyntropin dose of 0.1 microg/kg did not differentiate between groups of infants with clinical conditions that affect response. We recommend 1.0 microg/kg cosyntropin to test adrenal function in these infants. [\hyperlink{Cosyntropin}{PMID: 16159938}, Kristi L Watterberg et al., 2005]

\hypertarget{pmid_21958958}{T}o evaluate the analgesic effect and tolerability of paracetamol syrup compared to placebo and ketoprofen lysine salt in children with pharyngotonsillitis cared by family pediatricians. A double-blind, randomized, placebo-controlled trial of a 12 mg/kg single dose of paracetamol paralleled by open-label ketoprofren lysine salt sachet 40 mg. Six to 12 years old children with diagnosis of pharyngo-tonsillitis and a Children's Sore Throat Pain (CSTP) Thermometer score > 120 mm were enrolled. Primary endpoint was the Sum of Pain Intensity Differences (SPID) of the CSTP Intensity scale by the child. 97 children were equally randomized to paracetamol, placebo or ketoprofen. Paracetamol was significantly more effective than placebo in the SPID of children and parents (P < 0.05) but not in the SPID reported by investigators, 1 hour after drug administration. Global evaluation of efficacy showed a statistically significant advantage of paracetamol over placebo after 1 hour either for children, parents or investigators. Patients treated in open fashion with ketoprofen lysine salt, showed similar improvement in pain over time. All treatments were well-tolerated. A single oral dose of paracetamol or ketoprofen lysine salt are safe and effective analgesic treatments for children with sore throat in daily pediatric ambulatory care. [\hyperlink{Cosyntropin}{PMID: 21958958}, Nicolino Ruperto et al., 2011]

\hypertarget{pmid_31789828}{P}ostdural puncture headache is a challenging complication of diagnostic, therapeutic, and unintentional lumbar puncture. Literature evidence supports cosyntropin as a viable noninvasive therapy for adults who have failed conservative management, but pediatric data are limited. The purpose of this retrospective chart review was to describe the use of intravenous cosyntropin for refractory pediatric postdural puncture headache at a single free-standing tertiary care pediatric hospital. Patients who had received cosyntropin were identified. Charts were retrospectively reviewed for indication, dosing information, efficacy, and side effects. The response was defined as a 50\% reduction in pain score, with a secondary efficacy measure of time to discharge after the first dose. Over a 5-year period, 26 patients received 37 doses of cosyntropin. Dosing ranged from 5 to 15 mcg/kg (median, 10.4 mcg/kg). There was a significant reduction in pain scores after the first dose of cosyntropin (P=0.008). Eighty-one percent of patients (n=21) achieved either a 50\% reduction in pain or were discharged within 24 hours after the first dose. The median time to 50\% pain reduction in 13 patients who achieved it before or discharge was 5 hours (range, 1 to 30 h). The median time to discharge after the first dose was 20 hours (range, 2 to 72 h). Ten patients received >1 dose of cosyntropin. Three patients required an epidural blood patch. No adverse effects related to treatment were identified. This study suggests that while further research is warranted, cosyntropin is a potential alternative to epidural blood patch for pediatric patients with postdural puncture headache who fail conservative management. [\hyperlink{Cosyntropin}{PMID: 31789828}, Michelle K Zapapas et al., 2020]

\hypertarget{pmid_17427394}{C}osyntropin (adrenocorticotropic hormone [ACTH]) stimulation tests are used to evaluate adrenal function. Low-dose ACTH stimulation tests are the most accurate method for diagnosing relative adrenal insufficiency in critically ill humans but have not been evaluated in foals. Peak serum cortisol concentrations in healthy foals will not be significantly different after intravenous administration of 1, 10, 100, and 250 microg of cosyntropin. 14 healthy neonatal foals, 3-4 days of age. A randomized cross-over model was used in which cosyntropin (1, 10, 100, or 250 microg) was administered intravenously on days 3 and 4 of life. Blood samples were collected before and 30, 60, 90, 120, and 150 minutes after administration of cosyntropin for determination of serum cortisol concentration. Serum cortisol concentrations did not significantly increase after administration of 1 microg of cosyntropin. Cortisol concentration peaked 30 minutes after administration of 10 microg of cosyntropin and 90 minutes after 100 and 250 microg of cosyntropin. There was no relationship between cosyntropin dose and serum cortisol concentration at 30 minutes. Compared with the 10-microg dose, 100 and 250 microg of cosyntropin induced significantly greater cortisol concentrations at 90 minutes, at which point the 10-microg cosyntropin-dose cortisol values were indistinguishable from baseline. There was no significant difference in the area under the cortisol concentration curve between the 100- and 250-microg doses. No effect of day of testing or foal weight on peak cortisol concentration was detected. The results of this study suggest that 10- and 100-microg doses of cosyntropin would be appropriate for evaluating adrenal function in neonatal foals. [\hyperlink{Cosyntropin}{PMID: 17427394}, Kelsey A Hart et al., ]

\hypertarget{pmid_23515245}{T}o describe the rationale, design and first data from PATRO Children, a postmarketing surveillance of the long-term efficacy and safety of somatropin (Omnitrope(®)) for the treatment of children requiring growth hormone treatment. PATRO Children is a multicentre, open, longitudinal, noninterventional study being conducted in children's hospitals and specialised endocrinology clinics. The primary objective is to assess the long-term safety of Omnitrope(®) in routine clinical practice. Eligible patients are infants, children and adolescents (male or female) who are receiving treatment with Omnitrope(®) and who have provided informed consent. Patients who have been treated with another recombinant human growth hormone (rhGH) product before starting Omnitrope(®) are eligible for inclusion. All adverse events (AEs) are monitored and recorded, with particular emphasis on: long-term safety; the recording of malignancies; the occurrence and clinical impact of anti-hGH antibodies; the development of diabetes during Omnitrope(®) treatment in children short for gestational age (SGA); safety issues in patients with Prader-Willi syndrome (PWS). Efficacy assessments include auxological parameters, plus insulin-like growth factor-1 and insulin-like growth factor binding protein-3. As of September 2012, 1837 patients were enrolled in the study from 184 sites in 10 European countries. To date, efficacy data are reassuring and consistent with previous studies. In addition, there have been no confirmed cases of diabetes occurring under Omnitrope(®) treatment, no reports of malignancy and no safety issues in PWS patients. The efficacy and safety profile of Omnitrope(®) in the PATRO Children study so far are as expected. The ongoing study will extend the safety database for Omnitrope(®), and rhGH products more generally, in paediatric indications. Of particular interest, PATRO Children will add important information on the diabetogenic potential of rhGH in children born SGA, the risk of malignancies in children receiving rhGH, and AEs with a possible causal relationship to rhGH treatment in children with PWS. [\hyperlink{Cosyntropin}{PMID: 23515245}, Roland Pfäffle et al., 2013]

\hypertarget{pmid_37625985}{R}esearch suggests that postpartum post-dural puncture headache (PDPH) might be prevented or treated by administering intravenous cosyntropin. In this retrospective cohort study, we questioned whether prophylactic (1 mg) and therapeutic (7 µg/kg) intravenous cosyntropin following unintentional dural puncture (UDP) was effective in decreasing the incidence of PDPH and therapeutic epidural blood patch (EBP) after birth. Two tertiary-care American university hospitals collected data from November 1999 to May 2017. Two hundred and fifty-three postpartum patients who experienced an UDP were analyzed. In one institution 32 patients were exposed to and 32 patients were not given prophylactic cosyntropin; in the other institution, once PDPH developed, 36 patients were given and 153 patients were not given therapeutic cosyntropin. The primary outcome for the prophylactic cosyntropin analysis was the incidence of PDPH and for the therapeutic cosyntropin analysis in exposed vs. unexposed patients, the receipt of an EBP. The secondary outcome for the prophylactic cosyntropin groups was the receipt of an EBP. In the prophylactic cosyntropin analysis no significant difference was found in the risk of PDPH between those exposed to cosyntropin (19/32, 59\%) and unexposed patients (17/32, 53\%; odds ratio (OR) 1.37, 95\% CI 0.48 to 3.98, P = 0.56), or in the incidence of EBP between exposed (12/32, 38\%) and unexposed patients (6/32, 19\%; OR 2.6, 95\% CI 0.83 to 8.13, P = 0.095). In the therapeutic cosyntropin analysis, in patients exposed to cosyntropin the incidence of EBP was significantly higher (20/36, 56\% vs. 43/153, 28\%; OR 3.20, 95\% CI 1.52 to 6.74, P = 0.002). Our data show no benefits from the use of cosyntropin for preventing or treating postpartum PDPH. [\hyperlink{Cosyntropin}{PMID: 37625985}, C Pancaro et al., 2023]

\hypertarget{pmid_27845313}{T}o study the efficacy and safety of the complex metabolic neuroprotector cytoflavin in children with the consequences of perinatal hypoxic brain damages. Patients, aged 4-8 years, were stratified into three groups: 35 with infant cerebral palsy, 64 with the minimal brain dysfunction and 47 with sensorineural hearing loss. The control group consisted of 30 children. Monotherapy with cytoflavin was carried out in the dose of one tablet twice a day for 25 days. Neurologic status, neurophysiological examination andneuropsychophysiological testing were performed before and after treatment. The efficacy of cytoflavin in children of preschool and early school agewas demonstrated. A complex neuroprotective action, including vasoactive, nootropic and antiasthenic effects, was revealed. Side-effects of cytoflavin were not observed. [\hyperlink{Cosyntropin}{PMID: 27845313}, S Yu Lavrick et al., ]

\hypertarget{pmid_11882229}{T}onsillectomy is commonly performed in children, but unfortunately it is associated with intense postoperative pain. The use and optimal timing of nonsteroidal anti-inflammatory drugs (e.g. ketoprofen) during tonsillectomy is controversial. We evaluated the safety and efficacy of ketoprofen in 109 children, aged 3-16 years, during and after tonsillectomy in 1998-2000. Standardized anaesthesia was used. Forty-seven children received ketoprofen 0.5 mg.kg-1 at induction (preketoprofen group) and 42 children after surgery (postketoprofen group), followed by continuous ketoprofen infusion of 3 mg.kg-1 over 24 h in both groups; 20 children received normal saline (placebo group). Oxycodone was used for rescue analgesia. Pre- and postketoprofen groups did not differ in experienced pain or in opioid consumption in the first 24 h after surgery; demonstrating that ketoprofen did not have a pre-emptive effect. Patients in the placebo group received 30 more oxycodone doses than did patients in the ketoprofen groups, but the difference was not significant (P=0.074). Two patients (5) in the postketoprofen group had postoperative bleeding at 4 h and 26 h, respectively. Both patients required electrocautery to stop bleeding. Neither the incidence nor the severity of adverse events differed between study groups. This study demonstrates that ketoprofen did not have a preemptive effect and, at the dose used, did not perform statistically significantly better than placebo. [\hyperlink{Cosyntropin}{PMID: 11882229}, Hannu Kokki et al., 2002]

\hypertarget{pmid_32968391}{T}o assess the safety as well as efficacy of desmopressin monotherapy alone and in combination (desmopressin + oxybutynin) in treating nocturnal urinary incontinence among children with 7 to 13 years. This randomized controlled trial has been carried out in National Institute of Child Health from September 2018 to March 2019 with the utilization of convenient sampling technique. Data has been collected after taking ethical approval and informed consent of the Parents with complete confidentiality. The sample size was 84 and equal number of patients was divided in two groups. Group-I was given desmopressin at monotherapy at a dose of 0.2 mg and Group-II was given desmopressin and oxybutynin at the dose of 0.2 mg desmopressin and 5 mg oxybutynin patients were diagnosed on the basis of history. Routine lab investigation included Urine DR and ultrasound abdomen. In this study significant differences between two groups were found with respect to socio economic status, lack of education of parents (P Less than 0.05). The frequency, urgency and incontinence of this ailment was significantly controlled by combination therapy (desmopressin + oxybutynin) as compared to desmopressin as monotherapy (P Less than 0.05) as patient was followed after one, two and three monthly basis. Desmopressin combination with oxybutynin is more effective as compared to monotherapy treatment. The affectivity of the combination therapy was very high with least side effects and all the children recovered from the condition at third month of treatment. Furthermore, headache was observed to be common with monotherapy and loss of appetite was observed with combination therapy. [\hyperlink{Cosyntropin}{PMID: 32968391}, Asiya Kazi et al., ]

\hypertarget{pmid_17300659}{C}orticosteroids are currently the first line of treatment for patients with atopic dermatitis. In the pediatric population however, the potential impact of adrenal suppression is always an important safety concern. Twenty boys and girls, 5-12 years of age, with normal adrenal function and a history of atopic dermatitis were maximally treated three times daily with a lipid-rich, moisturizing formulation of hydrocortisone butyrate 0.1\% for up to 4 weeks. At the conclusion of the 4-week treatment period, cosyntropin injection stimulation testing showed no evidence of adrenal suppression. In addition, the therapy was noted to be highly efficacious, with a clinical success rate of 80\% (Physician Global Score of (0) clear or (1) almost clear). No local side effects associated with prolonged use of topical corticosteroids were reported. In summary, this study supports the contention that this lipid-rich, moisturizing formulation of hydrocortisone butyrate 0.1\% was a well-tolerated and beneficial treatment for atopic dermatitis, demonstrating no adrenal suppression in the pediatric population aged 5-12 years. The relevance of these findings for children below 5 years of age, because of difference in body mass/surface area ratios, remains to be determined. [\hyperlink{Cosyntropin}{PMID: 17300659}, Lawrence Eichenfield et al., ]

\hypertarget{pmid_19337398}{T}he purpose of this study was to investigate total baseline plasma cortisol and adrenocorticotropic hormone (ACTH) concentrations, and ACTH-stimulated cortisol concentrations in foals from birth to 12 wk of age. Plasma (baseline) cortisol and ACTH concentrations were measured in 13 healthy foals at birth and at 1, 2, 3, 4, 5, 7, 10, 14, 21, 28, 42, 56, and 84 d of age. Each foal received cosyntropin (0.1 microg/kg) intravenously. Plasma cortisol concentrations were measured before (baseline), and 30, and 60 min after cosyntropin administration at birth and at 3, 5, 7, 10, 14, 21, 28, 42, 56, and 84 d of age. Compared with baseline, cortisol concentration increased significantly 30 min after administration of cosyntropin on all days. Cortisol concentration was highest at birth, measured at 30 and 60 min after cosyntropin administration, compared with all other days. With the exception of birth measurements, cortisol concentration was significantly higher on day 84, measured at 30 and 60 min after cosyntropin administration, when compared with all other days. Baseline plasma ACTH was lowest at birth when compared with concentrations on days 2, 3, 4, 5, 7, 10, 14, 42, 56, and 84. Administration of 0.1 microg/kg of cosyntropin, IV, reliably induces cortisol secretion in healthy foals. Differences in the magnitude of response to cosyntropin are observed depending on the age of the foal. These data should serve as a reference for the ACTH stimulation test in foals and should be useful in subsequent studies to evaluate the hypothalamic-pituitary-adrenal axis in healthy and critically ill foals. [\hyperlink{Cosyntropin}{PMID: 19337398}, David M Wong et al., 2009]

\hypertarget{pmid_24392295}{R}ecombinant human growth hormone (rhGH) is effective and safe when used to treat growth hormone deficiency (GHD) in children. However, it has been suggested that switching between different types of rhGH can have a detrimental effect on patients. The current analysis assessed the efficacy and safety of rhGH in children who received continuous Omnitrope® (Sandoz GmbH, Kundl, Austria) therapy either with lyophilized powder for solution or ready-to-use solution, with children who received 9 months of treatment with Genotropin® (Pfizer Limited, Sandwich, UK) followed by Omnitrope solution thereafter. Changes to height, height SD score (SDS), height velocity SDS, insulin-like growth factor (IGF-1) levels, and IGF binding protein (IGFBP-3) levels were assessed using data from three trials. Baseline demographics of the three study groups were similar. Over an 18-month period there were no observable differences between the three groups with respect to height, height SDS, height velocity SDS, IGF-1 levels, and IGFBP-3 levels. This result was corroborated by the model data, whereby most data points for Omnitrope-treated children fell within the defined limits of the prediction model based on Genotropin data. Few adverse drug reactions (ADRs) occurred. Switching from Genotropin to Omnitrope solution has no impact on efficacy or safety in children with GHD, and the various rhGH preparations are well tolerated. [\hyperlink{Cosyntropin}{PMID: 24392295}, Tomasz Romer et al., 2011]

\hypertarget{pmid_19441606}{I}ntranasal corticosteroids (INSs) are the most effective treatment for allergic rhinitis (AR). However, available INS safety and efficacy data in children younger than 6 years are limited. To report the first well-controlled study assessing the safety and efficacy of an INS in children aged 2 to 5 years with perennial AR. In a 4-week, multicenter, double-blind, parallel-group study, patients were randomized to receive triamcinolone acetonide aqueous nasal spray (TAA AQ), 110 microg once daily, or placebo. A subset of children continued into a 6-month, open-label phase. Efficacy end points included total nasal symptom scores. Safety measures included reports of adverse events, morning serum cortisol levels before and after cosyntropin infusion, and growth as measured using office stadiometry. A total of 474 patients were randomized to receive TAA AQ (n = 236) or placebo (n = 238); 436 entered the open-label extension phase. Adjusted mean (SE) changes from baseline during the double-blind period in instantaneous and reflective total nasal symptom scores were -2.28 (0.16) and -2.31 (0.15), respectively, in the TAA AQ group (P = .09) vs -1.92 (0.16) and -1.87 (0.15) in the placebo group (P = .03). Adverse event rates were comparable between treatment groups. There was no significant change from baseline in serum cortisol levels after cosyntropin infusion at study end. The distribution of children by stature-for-age percentile remained stable during the study. Use of TAA AQ, 110 microg once daily, for up to 6 months offers a favorable efficacy to safety ratio in children aged 2 to 5 years with perennial AR. [\hyperlink{Cosyntropin}{PMID: 19441606}, Steven Weinstein et al., 2009]

\hypertarget{pmid_31321320}{A}zithromycin is widely used in children not only in the treatment of individual children with infectious diseases, but also as mass drug administration (MDA) within a community to eradicate or control specific tropical diseases. MDA has also been reported to have a beneficial effect on child mortality and morbidity. However, concerns have been raised about the safety of azithromycin, especially in young children. The aim of this review is to systematically identify the safety of azithromycin in children of all ages. MEDLINE, PubMed, Cochrane Central Register of Controlled Trials, Embase, CINAHL, International Pharmaceutical Abstracts and adverse drug reaction (ADR) monitoring systems will be systematically searched for randomised controlled trials (RCTs), cohort studies, case-control studies, cross-sectional studies, case series and case reports evaluating the safety of azithromycin in children. The Cochrane risk of bias tool, Newcastle-Ottawa and quality assessment tools, and The Joanna Briggs Institute Critical Appraisal tools will be used for quality assessment. Meta-analyses will be conducted to the incidence of ADRs from RCTs if appropriate. Subgroup analyses will be performed in different age and azithromycin dosage groups. Formal ethical approval is not required as no primary data are collected. This systematic review will be disseminated through a peer-reviewed publication. CRD42018112629. [\hyperlink{Cosyntropin}{PMID: 31321320}, Peipei Xu et al., 2019]

\hypertarget{pmid_15592080}{W}e evaluated the efficacy and safety of oxybutynin in children with detrusor hyperreflexia due to neurological conditions. Study 1--A prospective, open label trial of 3 formulations of oxybutynin (tablets, syrup and extended release tablets) was conducted for 24 weeks in children 6 to 15 years old with detrusor hyperreflexia who used oxybutynin and clean intermittent catheterization. The effect of treatment on average urine volume per catheterization and on secondary urodynamic outcomes was evaluated. Study 2--The efficacy and safety of oxybutynin syrup were evaluated urodynamically in an open label study of children 1 to 5 years old with detrusor hyperreflexia who used oxybutynin and clean intermittent catheterization. Study 1--Mean urine volume per catheterization (+/- SEM) increased by 25.5 +/- 5.9 ml (p <0.001). Maximal cystometric capacity increased by 75.4 +/- 9.8 ml (p <0.001). Mean detrusor and intravesical pressures were significantly decreased by -9.2 +/- 2.3 (p < or =0.001) and -7.5 +/- 2.5 cm H2O (p <0.004), respectively, at week 24. Of 61 children with uninhibited detrusor contractions 15 cm H2O or greater at baseline 34 did not have them at week 24 (p <0.001). Improvements in bladder function were consistent across all oxybutynin formulations. Study 2--Mean maximal cystometric capacity increased significantly by 71.5 +/- 21.99 ml (p = 0.005). At study end only 12.5\% of patients had uninhibited detrusor contractions 15 cm H2O or greater compared with 68.8\% at baseline (p = 0.004). Oxybutynin was well tolerated in both studies. There were no serious treatment related adverse events. All 3 formulations of oxybutynin are safe and effective in children with neurogenic bladder dysfunction. [\hyperlink{Cosyntropin}{PMID: 15592080}, Israel Franco et al., 2005]

\hypertarget{pmid_11568544}{A} prospective, randomized, double-blinded, placebo-controlled protocol. An academic, tertiary care referral center. Forty randomly selected children, ages 3 to 13 years, scheduled for adenotonsillectomy without other simultaneous procedures. A single, oral dose of dextromethorphan pediatric cough syrup (1 mg/kg) or placebo given 30 minutes before surgery. Total dose requirement of intravenous morphine within a 6-hour postoperative observation period. During routine postoperative observation, significantly fewer patients in the dextromethorphan group required no intravenous morphine compared with the placebo group (P =.03). Of those children requiring morphine, the mean dose requirement was significantly lower in the dextromethorphan group (P =.02). There was no known drug-related morbidity. Dextromethorphan syrup is a safe, non-narcotic medication that significantly reduced the requirement of intravenous morphine after pediatric adenotonsillectomy. Its routine use in this manner is recommended. [\hyperlink{Cosyntropin}{PMID: 11568544}, G S Dawson et al., 2001]

\hypertarget{pmid_25708686}{T}o clarify the efficacy and safety of fosphenytoin for seizures in children with benign convulsions and mild gastroenteritis. Using the mailing list of the Annual Zao Conference on Pediatric Neurology, we recruited patients who met the following criteria: (1) clinical diagnosis of benign convulsions with mild gastroenteritis and (2) treatment with intravenous fosphenytoin. Benign convulsions with mild gastroenteritis were defined as a condition of (a) seizures associated with gastroenteritis without electrolyte imbalance, hypoglycemia, or dehydration in patients (b) between 6 months and 3 years of age with (c) no preexisting neurological disorders, (d) no impaired consciousness, and (e) a body temperature less than 38.0 °C before and after the seizures. The efficacy of fosphenytoin was categorized as effective when cessation of seizures was achieved. Data from 16 child patients were obtained (median age, 20 months). Seizures were completely controlled after the initial dose of fosphenytoin in 14 of 16 patients. The median loading dose of fosphenytoin was 22.5 mg/kg. In 10 patients, fosphenytoin was administered after other antiepileptic drugs such as diazepam and midazolam were used. Adverse effects of fosphenytoin, excessive sedation, or intravenous fluid incompatibility were not observed in any patients. Fosphenytoin is effective and well tolerated among children with benign convulsions with mild gastroenteritis. [\hyperlink{Cosyntropin}{PMID: 25708686}, Mika Nakazawa et al., 2015]

\hypertarget{pmid_19301727}{V}arious trials showed benefit of the prophylactic agent ketotifen in prevention of recurrent wheezing in young children, but no such clinical trial with loratadine or comparison trial is available. To study the efficacy and safety of loratadine syrup compared with ketotifen and placebo in prevention of recurrent wheezing in young children. Randomized double-blind placebo controlled trial on 90 recurrent wheezing children aged less than 6 years old was done. Children were randomized to receive loratadine, ketotifen syrup, or placebo with dose of 0.25 cc/kg once a day for four months. Blood biochemistry (CBC, LFT) and EKG were performed pre and post treatment period. Assessment of symptoms--wheezing and night cough including use of bronchodilators was done daily via patient diary card. Subjects were asked to do monthly visits to the clinic for physical examination. At those visits, the doctors questioned the patients about adverse event. Of the 90 children enrolled, 12 dropped out. Thus, 27 children remained in the loratadine, 26 in the placebo, and 25 in the ketotifen group. The demographic data were comparable among the three treatment groups. It was noted that wheezing decreased significantly at 2 months in the ketotifen (p = 0.008) and at 3 months in the loratadine (p = 0.029) but not in the placebo group. Coughing at night decreased significantly at 3 months in both the loratadine (p = 0.005) and the ketotifen (p = 0.036) group. The use of bronchodilator drug was significantly decreased at 2 months in the ketotifen (p = 0.028) and placebo (p = 0.025) group, and at 3 months in the loratadine (p = 0.009) group. Only a few patients had mild adverse events in all groups. Loratadine and ketotifen are safe and effective significantly in prevention of recurrent wheezing in young children. [\hyperlink{Cosyntropin}{PMID: 19301727}, Jarungchit Ngamphaiboon et al., 2009]

\hypertarget{pmid_29151100}{T}he cosyntropin stimulation study (CSS) measures the patient's ability to adequately mount a cortisol response. Clinically, CSS results may not be used to guide hydrocortisone use. The objective of this study was to examine how the CSS results are associated with clinical parameters, mortality/disease severity, and use of glucocorticoids in pediatric patients with catecholamine- and fluid-resistant shock. This was a retrospective cohort study of patients who had a CSS during 2009-2014 in the intensive care unit at a children's hospital. Data collected included clinical variables, mortality, biochemical studies, and glucocorticoid use. PRISM III scores were used to determine the association between CSS results and disease severity. Adequate response to cosyntropin was defined as peak cortisol of 18 µg/dL or higher. Of the 76 patients that underwent CSS, 68 (89\%) had an adequate response to cosyntropin. There was a positive correlation between peak cortisol and PRISM III score (r = 0.45, r2 = 0.2). Glucocorticoid was administered in 52/76 (68\%) despite several patients with normal CSS results. Sicker patients were more likely to have an adequate response to CSS. Clinically, glucocorticoid supplementation was not based on CSS results. Further prospective studies are needed to elucidate if CSS is a valuable clinical tool. [\hyperlink{Cosyntropin}{PMID: 29151100}, Pallavi Iyer et al., 2018]

\hypertarget{pmid_20225908}{K}etoprofen is a highly effective NSAID with antipyretic and analgesic properties for the symptomatic management of pain and fever in both adults and children. To compare three dose levels of ketoprofen with paracetamol (acetaminophen) in the management of fever in children. Two prospective, randomized, single-blind, comparator-controlled, single-dose, multicentre, phase II studies with four parallel groups in each study were conducted in primary-care outpatient clinics. Children aged 6-24 months and 2-6 years presenting with a febrile condition (rectal body temperature > or =39 degrees C) were included in the studies. Patients were treated with either ketoprofen syrup 0.25 mg/kg, 0.5 mg/kg or 1 mg/kg, or paracetamol drinkable solution 15 mg/kg, both administered orally. The primary outcome measure was the maximal reduction in body temperature before re-medication compared with baseline during the 6-hour study period. In the ketoprofen groups, the mean maximal temperature decreases in the younger/older age groups were 1.6/1.6 degrees C, 2.0/1.9 degrees C and 1.9/2.2 degrees C with doses of 0.25 mg/kg, 0.5 mg/kg and 1 mg/kg of ketoprofen, respectively, compared with 1.8/1.8 degrees C with paracetamol 15 mg/kg. In the older children, ketoprofen provided antipyretic efficacy in a dose-dependent manner. Ketoprofen was found to have a significant antipyretic efficacy in children. The lowest dose of ketoprofen syrup that provided a meaningful antipyretic effect in both groups was 0.5 mg/kg. At this dose the antipyretic efficacy was equal to that of paracetamol 15 mg/kg. Based on these data, a dose of 0.5 mg/kg of ketoprofen was selected for future evaluation in phase III studies in the symptomatic management of fever in children. [\hyperlink{Cosyntropin}{PMID: 20225908}, Hannu Kokki et al., 2010]

\hypertarget{pmid_17848841}{W}e report 24-month interim results of two multicenter phase III studies in previously untreated children with growth failure secondary to GH deficiency (GHD) that were paramount to the development of a new recombinant human GH (rh- GH, somatropin), approved as the first 'biosimilar' in Europe. Study 1 consisted of 3 parts performed in 89 children. The objective was to compare efficacy and safety of the lyophilized formulation of the new somatropin [Somatropin Powder (Sandoz)] with a licensed reference rhGH preparation and the liquid formulation of the new somatropin [Somatropin Solution (Sandoz)] and to assess long-term efficacy and safety of this ready-to-use Somatropin Solution. Study 2 was performed in 51 children and designed to demonstrate efficacy and safety of Somatropin Powder and to confirm its low immunogenic potential; rhGH was given sc at a daily dose of 0.03 mg/kg. Primary [body height, height SD score (HSDS), height velocity, and height velocity (HV) SD score (HVSDS)] and secondary [IGF-I and IGF binding protein 3 (IGFBP-3)] efficacy endpoints and safety parameters were assessed regularly. In study 1, all treatments showed comparable increases in growth. The baseline-adjusted difference between Somatropin Powder and the reference rhGH product in mean HV was -0.20 cm/yr (95\% confidence interval (CI) [-1.34;0.94]) and in mean HVSDS was 0.76 (95\% CI [-0.57;2.10]) after 9 months. These very small differences demonstrate comparable therapeutic efficacy between the two treatments. The results of study 2 were consistent with those seen in study 1. Equivalent therapeutic efficacy and clinical comparability in terms of safety and immunogenicity between Somatropin Powder and the reference rhGH product and between Somatropin Powder and Somatropin Solution was demonstrated. The safety and immunogenicity profiles were similar and as expected from experience with rhGH preparations. [\hyperlink{Cosyntropin}{PMID: 17848841}, T Romer et al., ]

\hypertarget{pmid_37964810}{F}osinopril and amlodipine are commonly prescribed as first-line pharmacotherapeutic agents for pediatric hypertension, but there is a lack of comparative studies regarding the efficacy of these two drugs. We aimed to evaluate and compare the efficacy of fosinopril and amlodipine monotherapy in pediatric primary hypertension. This was a single-center, bidirectional observational study. A total of 175 children and adolescents with primary hypertension receiving antihypertensive monotherapy from July 2020 to February 2023 were enrolled. According to antihypertensive drugs, they were divided into the fosinopril group ( After 4 weeks of treatment, both groups achieved significant reductions in systolic BP (SBP) and diastolic BP (DBP) by more than 18 mmHg and 6 mmHg, respectively, with BP control rates of 61.5\% in the fosinopril group and 59.5\% in the amlodipine group, revealing no significant differences in the antihypertensive efficacy between the two groups except for DBP control rate (FDR adjusted  Fosinopril and amlodipine monotherapy were both effective in pediatric primary hypertension during a short-term follow-up. Fosinopril may be particularly effective in reducing BP in hypertensive patients of females, central obesity, IR, and hypo-HDL-cholesterolemia. These findings indicate that optimizing antihypertensive medication selection based on the individualized characteristics of children with hypertension may improve the efficacy of antihypertensive treatment. [\hyperlink{Cosyntropin}{PMID: 37964810}, Hui Wang et al., 2023]

\hypertarget{pmid_37864281}{L}isinopril, an angiotensin-converting enzyme inhibitor, is a frequently prescribed antihypertensive drug in the paediatric population, while being used off-label under the age of 6 years in the USA and for all paediatric patients globally. The SAFEPEDRUG project (IWT-130033) investigated lisinopril pharmacokinetics in hypertensive paediatric patients corresponding with the day-to-day clinical population. The dose-escalation pilot study included 13 children with primary and secondary hypertension who received oral lisinopril once daily in the morning; doses ranged from 0.05 to 0.2 mg kg A 1-compartment model with first-order absorption and first-order elimination optimally describes the data. Parameter estimates of absorption rate constant (0.075 h Lisinopril dose and regimen adjustments for paediatric patients should include eGFR on top of weight adjustments. An expanded model characterizing the pharmacodynamic effect is required to identify the optimal dose and dosing regimen. [\hyperlink{Cosyntropin}{PMID: 37864281}, Louis Sandra et al., 2023]

\hypertarget{pmid_21512499}{C}linical-neurophysiological investigation of the efficiency of complex metabolic neuroprotector Cytoflavin in monotherapy was performed in 64 children of 4-8 years old with the minimal cerebral dysfunction. During the treatment other preparations were not used. Neurologic status, neurophysiological inspection, neuropsychophysiological and speech testing were performed before and after the course of treatment. Efficiency of Cytoflavin was established in children with the minimal cerebral dysfunction of preschool and early school age. By increasing a cerebral blood flow, improving parameters EEG and decreasing intracranial pressure, the drug had positive influence on the psycho-emotional and cognitive spheres, decreased asthenia syndrome, improving social adaptation of the children. Side-effects of Cytoflavin administration were not observed. [\hyperlink{Cosyntropin}{PMID: 21512499}, S Iu Lavrik et al., 2011]

\hypertarget{pmid_3324636}{T}he efficacy and safety of Genotropin (recombinant somatropin, KabiVitrum AB, Sweden) was analysed in 199 children with hGH deficiency, comprising a combined series of four current multicentre trials. Stimulation of linear growth from pretreatment height velocities of 3-4 cm/year to about 10 cm/year was observed after 6 and 9 months of treatment. Statistical analysis revealed significantly greater height velocities (by 2-3 cm/year) when the weekly dose of the hormone was given in 6-7 injections rather than in 3 injections. Immunogenicity seems to be very low, with only about 2\% of the children having detectable antibodies during treatment. [\hyperlink{Cosyntropin}{PMID: 3324636}, R Gunnarsson et al., 1987]

\section*{Tranexamic Acid}
\subsection*{Result}
\subsubsection*{Answer}

Infants (<1 year): Yes  
Children (<3 years): Yes  
Children (3–16 years): Yes  
Adolescents (up to 17 years): Yes  
Pediatric trauma (all ages): Unknown  

\subsubsection*{{Explanation}}
\hypertarget{Tranexamic Acid}
Based on the abstracts reviewed, there are multiple targeted studies evaluating the safety of tranexamic acid (TXA) in children across various age ranges and surgical/trauma contexts. Below is a summary of the evidence by age group and indication:

Infants (<1 year):
- A pharmacokinetic study in infants <1 year undergoing cardiac surgery found that TXA dosing regimens could be optimized to avoid high peak concentrations and maintain therapeutic levels, with the aim of reducing the risk of seizures. No significant safety concerns were reported in this study [\hyperlink{pmid_28245519}{PMID: 28245519}, Ralph Gertler et al., 2017].
- A case report describes the use of TXA in two infants with choroid plexus papilloma, with no adverse events reported [\hyperlink{pmid_24924340}{PMID: 24924340}, Ji H Phi et al., 2014].

Toddlers and Young Children (<3 years):
- A randomized double-blind clinical trial in 80 children under 3 years undergoing cleft palate surgery found that TXA at various dosages significantly reduced blood loss, and the authors concluded it was safe to use the minimal effective dose to control bleeding [\hyperlink{pmid_33912417}{PMID: 33912417}, Amir Shafa et al.].
- A prospective, randomized, double-blind study in children (age not specified, but likely includes young children) undergoing cardiac surgery found no significant adverse effects [\hyperlink{pmid_9141920}{PMID: 9141920}, R W Reid et al., 1997].

Children (3–16 years):
- A retrospective review of 476 children aged 3–16 years undergoing tonsillectomy found that perioperative TXA reduced primary hemorrhage rates, with only two cases of minor bleeding and no further complications reported [\hyperlink{pmid_24780670}{PMID: 24780670}, P J Robb et al., 2014].
- A double-blind clinical trial in 64 children (age not specified, but likely school-aged) undergoing adenotonsillectomy compared two TXA dosages and found no significant difference in adverse events or recovery parameters [\hyperlink{pmid_36161261}{PMID: 36161261}, Amir Shafa et al., 2022].
- A prospective, double-blind, randomized, non-inferiority trial in children undergoing craniosynostosis surgery (age not specified, but typically infants and young children) found both high and low dose TXA regimens to be non-inferior in efficacy, with no significant safety concerns [\hyperlink{pmid_32620262}{PMID: 32620262}, Susan M Goobie et al., 2020].
- A prospective, double-blind, randomized, controlled trial in 400 children undergoing adenoidectomy found topical TXA significantly reduced blood loss and postoperative bleeding, with no reported adverse events [\hyperlink{pmid_23669000}{PMID: 23669000}, Osama A Albirmawy et al., 2013].
- A randomized, double-blind study in 88 children undergoing cardiac surgery found no significant adverse effects, and TXA reduced blood loss in cyanotic children [\hyperlink{pmid_8622323}{PMID: 8622323}, Z Zonis et al., 1996].
- A meta-analysis including pediatric patients undergoing spinal surgery found that intravenous TXA was safe and effective, with no significant difference in complications between TXA and non-TXA groups [\hyperlink{pmid_34930653}{PMID: 34930653}, Lingan Huang et al., 2022].
- A randomized trial in 41 children undergoing repeat sternotomy for congenital heart defects found TXA reduced blood loss and transfusion requirements, with no significant adverse effects [\hyperlink{pmid_9141920}{PMID: 9141920}, R W Reid et al., 1997].
- A randomized trial in 160 pediatric patients (cyanotic and acyanotic) undergoing cardiac surgery found TXA reduced blood loss, with no significant difference in adverse events [\hyperlink{pmid_21947753}{PMID: 21947753}, Kazuyoshi Shimizu et al., 2011].
- A randomized trial in 100 children undergoing endoscopic sinus surgery found TXA reduced bleeding and duration of surgery, with no reported adverse events [\hyperlink{pmid_24015121}{PMID: 24015121}, Ahmed A Eldaba et al., 2013].

Adolescents (up to 17 years):
- A prospective study in 163 patients aged ≤17 years with traumatic hyphema found TXA reduced secondary hemorrhage rates, with no significant ocular or systemic side effects [\hyperlink{pmid_1633590}{PMID: 1633590}, R Deans et al., 1992].
- A large administrative database study of 35,478 pediatric encounters (children <18 years) receiving TXA for various indications did not report significant safety concerns, but called for more research on indications and dosing [\hyperlink{pmid_27017532}{PMID: 27017532}, Daniel K Nishijima et al., 2016].
- A retrospective cohort study of prehospital TXA in pediatric trauma patients (median age 10, range 5–15) found no prehospital adverse events (upper 95\% CI for 0/70 is 4.3\%) [\hyperlink{pmid_36752620}{PMID: 36752620}, Sami Gendler et al., 2023].

General Pediatric Population:
- Several systematic reviews and meta-analyses (including [\hyperlink{pmid_22531271}{PMID: 22531271}, David Faraoni et al., 2012] and [\hyperlink{pmid_30893114}{PMID: 30893114}, Susan M Goobie et al., 2019]) note that while TXA is widely used and appears well-tolerated in pediatric surgery, the benefit-to-risk ratio is not fully defined, and more large-scale studies are needed, especially for rare adverse events.
- A case report of delayed anaphylaxis in a pediatric patient highlights that rare but serious adverse events can occur [\hyperlink{pmid_28321954}{PMID: 28321954}, Jennifer Chiem et al., 2017].

Pediatric Trauma:
- Multiple studies and reviews note that while TXA is used in pediatric trauma, there is a lack of large randomized controlled trials specifically addressing safety and efficacy in this context. Ongoing and planned trials are expected to provide more definitive data [\hyperlink{pmid_30376893}{PMID: 30376893}, Daniel K Nishijima et al., 2018; \hyperlink{pmid_34736498}{PMID: 34736498}, John M VanBuren et al., 2021].

Summary:
- For children undergoing elective surgery (cardiac, craniofacial, ENT, spinal), multiple randomized controlled trials and observational studies affirm the safety of TXA, with rare adverse events.
- For pediatric trauma, safety is less well established due to lack of large RCTs, but available data suggest a low rate of adverse events.
- In infants and children under 3 years, targeted studies support safety in surgical contexts.
- Rare serious adverse events (e.g., anaphylaxis) have been reported, but are extremely uncommon.

Therefore, based on the available abstracts, TXA is affirmed as safe for use in children for surgical indications (including infants, young children, and adolescents), with rare but possible serious adverse events. For pediatric trauma, safety is likely but not definitively established.

\subsection*{Abstracts}
\hypertarget{pmid_28321954}{T}ranexamic acid is an anti-fibrinolytic agent frequently used in pediatric surgery. Common side effects include nausea, flushing, and headache, but in rare instances, it may produce anaphylaxis; with only one previously reported case in a 72-year-old man. We report a case of a delayed anaphylactic reaction in a pediatric patient undergoing posterior spine fusion; and discuss the intraoperative management of the acute event, immunologic confirmation, and subsequent anesthetic approach. [\hyperlink{Tranexamic Acid}{PMID: 28321954}, Jennifer Chiem et al., 2017]

\hypertarget{pmid_24780670}{T}ranexamic acid has been used for many years to minimise blood loss during surgery and, more recently, to reduce morbidity after major trauma. While small studies have confirmed reduction in blood loss during tonsillectomy with its use, the rate of primary haemorrhage following tonsillectomy has not been reported. In the UK, less than 50\% of children having a tonsillectomy are managed as day cases, partly because of concerns about bleeding during the initial 24 hours following surgery. A retrospective review of clinical records between January 2007 and January 2013 produced 476 children between the ages of 3 and 16 years who underwent Coblation™ tonsillectomy, with or without adenoidectomy and/or insertion of ventilation tubes. All children were ASA (American Society of Anesthesiologists) grade 1 or 2 and anaesthetised using a standard day surgery protocol. Following induction of anaesthesia, all received intravenous tranexamic acid at a dose of 10-15 mg/kg. Two children (0.4\%) had minor bleeding within two hours of surgery. Both returned to theatre for haemostasis and were discharged home later the same day with no further complications. The expected rate for primary haemorrhage in the UK using this technique for tonsillectomy is 1\%. Perioperative tranexamic acid in a single, parenteral dose might reduce the incidence of primary haemorrhage following paediatric tonsillectomy, facilitating discharge on the day of surgery. The results from this observational study indicate a potential benefit and need for a large, prospective, multicentre, randomised controlled trial. [\hyperlink{Tranexamic Acid}{PMID: 24780670}, P J Robb et al., 2014]

\hypertarget{pmid_27017532}{T}he prevalence of tranexamic acid (TXA) use for trauma and other conditions in children is unknown. The objective of this study was to describe the use of TXA in United States (US) children's hospitals for children in general, and specifically for trauma. We conducted a secondary analysis of a large, administrative database of 36 US children's hospitals. We included children <18 years of age who received TXA (based on pharmacy charge codes) between 2009 and 2013. Patients were grouped into the following diagnostic categories: trauma, congenital heart surgery, scoliosis surgery, craniosynostosis/craniofacial surgery, and other, based on the International Classification of Diseases, Ninth Revision principal procedure and diagnostic codes. TXA administration and dosage, in-hospital clinical variables, and diagnostic and procedure codes were documented. A total of 35,478 pediatric encounters with a TXA charge were included in the study cohort. The proportions of children who received TXA were similar across the years 2009 to 2013. Only 110 encounters (0.31\%) were for traumatic conditions. Congenital heart surgery accounted for more than one-half of the encounters (22,863; 64\%). Overall, the median estimated weight-based dose of TXA was 22.4 mg/kg (interquartile range, 7.3-84.9 mg/kg). We identified a wide frequency of use and range of doses of TXA for several diagnostic conditions in children. The use of TXA among injured children, however, appears to be rare despite its common use and efficacy among injured adults. Additional work is needed to identify appropriate indications for TXA and provide dosage guidelines among children with a variety of conditions, including trauma. [\hyperlink{Tranexamic Acid}{PMID: 27017532}, Daniel K Nishijima et al., 2016]

\hypertarget{pmid_33417398}{A}lthough controversial, early administration of tranexamic acid (TXA) has been shown to reduce mortality in adult patients with major trauma. Tranexamic acid has also been successfully used in elective pediatric surgery, with significant reduction in blood loss and transfusion requirements. There are limited data to guide its use in pediatric trauma patients. We sought to determine the current practices for TXA administration in pediatric trauma patients in the United States. A survey was conducted of all the American College of Surgeons-verified Level I and II trauma centers in the United States. The survey data underwent quantitative analysis. Of the 363 Level I and II qualifying centers, we received responses from 220 for an overall response rate of 61\%. Eighty of 99 verified pediatric trauma centers responded for a pediatric trauma center response rate of 81\%. Of all responding centers, 148 (67\%) reported they care for pediatric trauma patients, with an average of 513 pediatric trauma patients annually. The pediatric trauma centers report caring for an average of 650 pediatric trauma patients annually. Of all centers caring for pediatric trauma, 52 (35\%) report using TXA, with the most common initial dosing being 15 mg/kg (68\%). A follow-up infusion was utilized by 45 (87\%) of the programs, most commonly dosed at 2 mg/kg/hr × 8 hr utilized by 24 centers (54\%). Although the clinical evidence for TXA in pediatric trauma patients is limited, we believe that consideration should be given for use in major trauma with hemodynamic instability or significant risk for ongoing hemorrhage. If available, resuscitation should be guided by thromboelastography to identify candidates who would most benefit from antithrombolytic administration. This represents a low-cost/low-risk and high-yield therapy for pediatric trauma patients. [\hyperlink{Tranexamic Acid}{PMID: 33417398}, Brian Cornelius et al., ]

\hypertarget{pmid_1633590}{I}n a prospective study 163 patients aged 17 years or less admitted to a children's hospital between April 1985 and December 1990 with traumatic hyphema were treated with tranexamic acid, 25 mg/kg given orally every 8 hours to a maximum of 1500 mg every 8 hours for 5 days. Secondary hemorrhage occurred in 5 patients (3\%), none of whom had more than one rebleeding episode. In contrast, 24 (8\%) of 316 patients aged 17 years or less admitted to the same hospital between January 1977 and March 1985 with traumatic hyphema who were not treated with antifibrinolytics had a secondary hemorrhage, several more than once, giving a rebleeding rate of 33/316 (10\%). The results suggest that tranexamic acid reduces the incidence and number of secondary hemorrhages in children, without significant ocular or systemic side effects. [\hyperlink{Tranexamic Acid}{PMID: 1633590}, R Deans et al., 1992]

\hypertarget{pmid_32090705}{E}vidence for tranexamic acid (TXA) in the pharmacologic management of trauma is largely derived from data in adults. Guidance on the use of TXA in pediatric patients comes from studies evaluating its use in cardiac and orthopedic surgery. There is minimal data describing TXA safety and efficacy in pediatric trauma. The purpose of this study is to describe the use of TXA in the management of pediatric trauma and to evaluate its efficacy and safety end points. This retrospective, observational analysis of pediatric trauma admissions at Hennepin County Medical Center from August 2011 to March 2019 compares patients who did and did not receive TXA. The primary end point is survival to hospital discharge. Secondary end points include surgical intervention, transfusion requirements, length of stay, thrombosis, and TXA dose administered. There were 48 patients aged ≤16 years identified for inclusion using a massive transfusion protocol order. Twenty-nine (60\%) patients received TXA. Baseline characteristics and results are presented as median (interquartile range) unless otherwise specified, with statistical significance defined as  TXA was utilized in 60\% of pediatric trauma admissions at a single level 1 trauma center, more commonly in older patients. Although limited by observational design, we found patients receiving TXA had no difference in mortality or thrombosis. [\hyperlink{Tranexamic Acid}{PMID: 32090705}, Julie M Thomson et al., 2021]

\hypertarget{pmid_9141920}{T}he antifibrinolytic drug, tranexamic acid, decreases blood loss in adult patients undergoing cardiac surgery. However, its efficacy has not been extensively studied in children. Using a prospective, randomized, double-blind study design, we examined 41 children undergoing repeat sternotomy for repair of congenital heart defects. After induction of anesthesia and prior to skin incision, patients received either tranexamic acid (100 mg/kg, followed by 10 mg.kg-1.h-1) or saline placebo. At the onset of cardiopulmonary bypass, a second bolus of tranexamic acid (100 mg/kg) or placebo was administered. Total blood loss and transfusion requirements during the period from protamine administration until 24 h after admission to the intensive care unit were recorded. Children who were treated with tranexamic acid had 24\% less total blood loss (26 +/- 7 vs 34 +/- 17 mL/kg) compared with children who received placebo (univariate analysis P = 0.03 and multivariate analysis P < 0.01). Additionally, the total transfusion requirements, total donor unit exposure, and financial cost of blood components were less in the tranexamic acid group. In conclusion, tranexamic acid can reduce perioperative blood loss in children undergoing repeat cardiac surgery. [\hyperlink{Tranexamic Acid}{PMID: 9141920}, R W Reid et al., 1997]

\hypertarget{pmid_32620262}{T}ranexamic acid (TXA) reduces blood loss and transfusion in paediatric craniosynostosis surgery. The hypothesis is that low-dose TXA, determined by pharmacokinetic modelling, is non-inferior to high-dose TXA in decreasing blood loss and transfusion in children. Children undergoing craniosynostosis surgery were enrolled in a two-centre, prospective, double-blind, randomised, non-inferiority controlled trial to receive high TXA (50 mg kg Sixty-eight children were included. Values were non-inferior regarding blood loss (39.4 [4.4] vs 40.3 [6.2] ml kg Tranexamic acid 10 mg kg NCT02188576. [\hyperlink{Tranexamic Acid}{PMID: 32620262}, Susan M Goobie et al., 2020]

\hypertarget{pmid_21947753}{T}he benefit of tranexamic acid (TXA) in pediatric cardiac surgery on postoperative bleeding has varied among studies. It is also unclear whether the effects of TXA differ between cyanotic patients and acyanotic patients. The aim of this study was to test the benefit of TXA in pediatric cardiac surgery in a well-balanced study population of cyanotic and acyanotic patients. A total of 160 pediatric patients undergoing cardiac surgery with cardiopulmonary bypass (81 cyanotic, 79 acyanotic) were included in this single-blinded, randomized trial at a tertiary care university-affiliated teaching hospital. Eighty-one children (41 cyanotic, 40 acyanotic) were randomly assigned to a TXA group, in which they received 50 mg/kg of TXA as a bolus followed by 15 mg/kg/h infusion and another 50 mg/kg into the bypass circuit. The other 79 patients were randomly assigned to a placebo group. The primary end point was the amount of 24-h blood loss. The amount of 24-h blood loss was significantly less in the TXA group than in the placebo group [mean (95\% confidence interval): 18.6 (15.8-21.4) vs. 23.5 (19.4-27.5) ml/kg, respectively; mean difference -4.9 (-9.7 to -0.01) ml/kg; p = 0.049]. This effect of TXA was already significant at 6 h [9.5 (7.5-11.5) vs. 13.2 (10.6-15.9) ml/kg, respectively; mean difference -3.47 (-7.0 to -0.4) ml/kg; p = 0.027]. However, there was no significant difference in the amount of blood transfusion between the groups. There was also no statistical difference in the effect of TXA in each cyanotic and acyanotic subgroup. TXA can reduce blood loss in pediatric cardiac surgery but not the transfusion requirement (http://ClinicalTrials.gov number NCT00994994). [\hyperlink{Tranexamic Acid}{PMID: 21947753}, Kazuyoshi Shimizu et al., 2011]

\hypertarget{pmid_36161261}{A}denotonsillectomy is a safe and common operation to remove adenoids and tonsils. Here we decided to compare the two dosages of tranexamic acid and their effects on hemodynamic changes and anesthesia-related indexes during surgical interventions. This is a double-blinded clinical trial performed in 2019-2020 on 64 children who were candidates for adenotonsillectomy. The patients were randomly divided into two groups of 32 based on the table of random numbers. Group A received 5 mg/kg slowly tranexamic acid for 10 minutes and group B received 10 mg/kg tranexamic acid slowly for 10 minutes. The study protocol was approved by the Research committee of Isfahan University of Medical Sciences and the Ethics Committee has confirmed it (Ethics code: IR.MUI.MED.REC.1398.639) (Iranian Registry of Clinical Trials (IRCT) code: IRCT20171030037093N33, https://en.irct.ir/trial/46553). The mean volume of intraoperative bleeding in children in group A is significantly higher than in children in group B (P < 0.05). However, no significant difference was observed between the length of stay in recovery and the duration of extubation and the mean dose of propofol in the two groups (P > 0.05). The mean arterial oxygen saturation of children in both groups increased significantly over time (P < 0.05). However, no significant difference was observed between the two groups (P > 0.05). According to the results, the mean HR in both groups decreased significantly over time (P < 0.05). In addition, the mean HR in children in the group B was significantly lower than children in the group A (P < 0.05). Administration of 10 mg/kg of tranexamic acid during tonsillectomy is associated with lower amounts of bleeding and lower heart rate than 5 mg/kg dosage. These results were in line with most previous studies. [\hyperlink{Tranexamic Acid}{PMID: 36161261}, Amir Shafa et al., 2022]

\hypertarget{pmid_23669000}{I}s to evaluate the efficacy of tranexamic acid when applied locally in children after primary isolated adenoidectomy with respect to intra-operative blood loss and post-operative bleeding. Prospective, double-blind, randomized, controlled trial. Otolaryngology department, Tanta University and Tiba Hospitals, Egypt. Over three years, 400 children underwent primary isolated adenoidectomy followed by topical application of tranexamic acid (tranexamic acid group, 200 children) or saline (Placebo group, 200 children) with at least two weeks' follow up. Intra-operative blood loss and post-operative hemorrhage were monitored. Both groups were almost equivalent in age and gender. The frequency of primary post-adenoidectomy hemorrhage as well as the rate of postnasal packing and blood transfusion required to manage severe bleeding were higher in placebo group. The volume of blood loss during surgery showed significant reduction in tranexamic acid group. Topical application of tranexamic acid after adenoidectomy led to a significant reduction in blood loss during surgery and decreasing in the rate of post-operative bleeding as well as the need for postnasal packing and blood transfusion. [\hyperlink{Tranexamic Acid}{PMID: 23669000}, Osama A Albirmawy et al., 2013]

\hypertarget{pmid_27660323}{T}rauma is the leading cause of death among children aged 1-18. Studies indicate that better control of bleeding could potentially prevent 10-20\% of trauma-related deaths. The antifibrinolytic agent tranexamic acid (TxA) has shown promise in haemorrhage control in adult trauma patients. However, information on the potential benefits of TxA in children remains sparse. This review proposes to evaluate the current uses, benefits and adverse effects of TxA in the bleeding paediatric trauma population. A structured search of bibliographic databases (eg, MEDLINE, EMBASE, PubMed, CINAHL, Cochrane CENTRAL) has been undertaken to retrieve randomised controlled trials and cohort studies that describe the use of TxA in paediatric trauma patients. To ensure that all relevant data were captured, the search did not contain any restrictions on language or publication time. After deduplication, citations will be screened independently by 2 authors, and selected for inclusion based on prespecified criteria. Data extraction and risk of bias assessment will be performed independently and in duplicate. Meta-analytic methods will be employed wherever appropriate. This study will not involve primary data collection, and formal ethical approval will therefore not be required. The findings of this study will be disseminated through a peer-reviewed publication and at relevant conference meetings. CRD42016038023. [\hyperlink{Tranexamic Acid}{PMID: 27660323}, Denisa Urban et al., 2016]

\hypertarget{pmid_8622323}{C}hildren undergoing cardiac operations in which cardiopulmonary bypass is used are at risk of significant postoperative blood loss. The acquired coagulopathy is complex but is thought to be due, in part, to excessive fibrinolysis. We examined the possibility of reducing postoperative blood loss in children by using the antifibrinolytic drug tranexamic acid. Using a prospective, randomized, double-blind study design, we administered a single dose of tranexamic acid (50 mg/kg intravenously) or saline placebo, before skin incision, in 88 children undergoing cardiac operations. Post-operative blood loss and fluid replacement were recorded for the next 24 hours. In addition, hemoglobin, platelet counts, and coagulation measures were recorded every 6 hours. When all patients were examined, there was no significant difference in postoperative blood loss between the treated and placebo groups (21.2 +/- 12 ml/kg per 24 hours, tranexamic acid, vs 27.2 +/- 20.3 mls/kg per 24 hours, placebo). However, when the children with cyanosis were analyzed separately, there was a highly significant difference in blood loss between the groups during the first 6 hours (11.2 +/- 3.7 ml/kg per 6 hours, tranexamic acid, vs 27.2 +/- 11.4 mls/kg per 6 hours, placebo; p < 0.002), as well as the overall 24 hour study period (23.7 +/- 7.5 mls/kg per 24 hours, tranexamic acid, vs 48.9 +/- 27.6 mls/kg per 24 hours, placebo; p < 0.02). Also significantly less blood and blood products were administered to the treated cyanosed group. Tranexamic acid produced a significant reduction in postoperative blood loss and blood product requirements in children with cyanosis undergoing heart operations. The drug had no effect in children without cyanosis or those requiring a second thoracotomy. [\hyperlink{Tranexamic Acid}{PMID: 8622323}, Z Zonis et al., 1996]

\hypertarget{pmid_24637029}{T}he aim of this study was to evaluate the clinical effects of intraoperative tranexamic acid administration in cardiac surgery without blood transfusion (bloodless cardiac surgery) in children. Seventy-one consecutive patients weighing less than 20 kg, who underwent bloodless cardiac surgery for simple atrial or ventricular septal defects at Kobe Children's Hospital from January 2011 to June 2013, were enrolled in this retrospective study. Tranexamic acid was administered during surgery from January 2012 (TXA group; n = 31), whereas it was not administered before January 2012 (control group; n = 40). Perioperative variables were compared between the TXA and control groups. There were no significant differences in patient characteristics or preoperative data between the 2 groups. Serial changes in perioperative hemoglobin and hematocrit levels, mixed venous oxygen saturation, and regional cerebral oxygenation during cardiopulmonary bypass were significantly higher in the TXA group compared to the control group. There were significant reductions in operative time, dopamine dose, peak serum lactate level, intubation time, chest tube drainage and duration, and hospital stay in the TXA group. Intraoperative tranexamic acid administration was effective for blood conservation, and improved postoperative clinical outcomes in pediatric bloodless cardiac surgery. [\hyperlink{Tranexamic Acid}{PMID: 24637029}, Tomomi Hasegawa et al., 2014]

\hypertarget{pmid_30376893}{T}rauma is the leading cause of morbidity and mortality in children in the United States. The antifibrinolytic drug tranexamic acid (TXA) improves survival in adults with traumatic hemorrhage, however, the drug has not been evaluated in a clinical trial in severely injured children. We designed the Traumatic Injury Clinical Trial Evaluating Tranexamic Acid in Children (TIC-TOC) trial to evaluate the feasibility of conducting a confirmatory clinical trial that evaluates the effects of TXA in children with severe trauma and hemorrhagic injuries. Children with severe trauma and evidence of hemorrhagic torso or brain injuries will be randomized to one of three arms: (1) TXA dose A (15 mg/kg bolus dose over 20 min, followed by 2 mg/kg/hr infusion over 8 h), (2) TXA dose B (30 mg/kg bolus dose over 20 min, followed by 4 mg/kg/hr infusion over 8 h), or (3) placebo. We will use permuted-block randomization by injury type: hemorrhagic brain injury, hemorrhagic torso injury, and combined hemorrhagic brain and torso injury. The trial will be conducted at four pediatric Level I trauma centers. We will collect the following outcome measures: global functioning as measured by the Pediatric Quality of Life (PedsQL) and Pediatric Glasgow Outcome Scale Extended (GOS-E Peds), working memory (digit span test), total amount of blood products transfused in the initial 48 h, intracranial hemorrhage progression at 24 h, coagulation biomarkers, and adverse events (specifically thromboembolic events and seizures). This multicenter trial will provide important preliminary data and assess the feasibility of conducting a confirmatory clinical trial that evaluates the benefits of TXA in children with severe trauma and hemorrhagic injuries to the torso and/or brain. ClinicalTrials.gov registration number: NCT02840097 . Registered on 14 July 2016. [\hyperlink{Tranexamic Acid}{PMID: 30376893}, Daniel K Nishijima et al., 2018]

\hypertarget{pmid_30893114}{P}erioperative bleeding and blood product transfusion are associated with significant morbidity and mortality. Prevention and optimal management of bleeding decreases risk and lowers costs. Tranexamic acid (TXA) is an antifibrinolytic agent that reduces bleeding and transfusion in a broad number of adult and pediatric surgeries, as well as in trauma and obstetrics. This review highlights the current pediatric indications and contraindications of TXA. The efficacy and safety profile, given current and evolving research, will be covered. Based on the published evidence, prophylactic or therapeutic TXA administration is a well-tolerated and effective strategy to reduce bleeding, decrease allogeneic blood product transfusion, and improve pediatric patients' outcomes. TXA is now recommended in recent guidelines as an important part of pediatric blood management protocols. Based on TXA pharmacokinetics, the authors recommend a dosing regimen of between 10 to 30 mg/kg loading dose followed by 5 to 10 mg/kg/h maintenance infusion rate for pediatric trauma and surgery. Maximal efficacy and minimal side-effects with this dosage regime will have to be determined in larger prospective trials including high-risk groups. Furthermore, future research should focus on determining the ideal TXA plasma therapeutic concentration for maximum efficacy and minimal side-effects. [\hyperlink{Tranexamic Acid}{PMID: 30893114}, Susan M Goobie et al., 2019]

\hypertarget{pmid_34930653}{A}s the number of fusion levels increases, the complexity of spinal correction surgery also increases. Thus, we conducted this study to determine the safety and efficacy of tranexamic acid (TXA) involving eight or more spinal fusion levels. According to the Preferred Reporting Items for Systematic Reviews and Meta-Analyses (PRISMA) and Assessing the Methodological Quality of Systematic Reviews (AMSTAR) guidelines, a search of the PubMed, Embase, CENTRAL, Web of Science, and ClinicalTrials.gov databases was conducted for relevant studies published prior to May 30, 2019. The primary outcomes, including blood loss and transfusion requirement, and the secondary outcomes, including general indices, postoperative hemoglobin, and coagulation function, were analyzed using Rev Man 5.3.5 software and STATA version 12.0. Eight randomized controlled trials (473 participants) were included in the study. Compared to the control treatments, TXA reduced intraoperative blood loss, total blood loss, transfusion volume, and prothrombin time. There were no significant differences between the TXA and non-TXA groups in transfusion rate, operative time, hospital stay, complications, hemoglobin level, and other coagulation function parameters. In the pediatric subgroup analysis, TXA additionally improved hemoglobin levels, platelet count, and prothrombin time international normalized ratio. The present meta-analysis showed that TXA reduced blood loss and transfusion volume in both adults and children. In pediatric patients, TXA led to a greater benefit in postoperative hemoglobin levels and coagulation function. Intravenous TXA is safe and effective in children with eight or more spinal corrective levels. [\hyperlink{Tranexamic Acid}{PMID: 34930653}, Lingan Huang et al., 2022]

\hypertarget{pmid_28245519}{T}ranexamic acid (TXA) continues to be one of the antifibrinolytics of choice during paediatric cardiac surgery. However, in infants less than 1 year of age, the optimal dosing based on pharmacokinetic (PK) considerations is still under discussion. Forty-three children less than 1 year of age were enrolled, of whom 37 required the use of cardiopulmonary bypass (CPB) and six were operated on without CPB. Administration of 50 mg kg A two-compartment model was fitted, with the main covariates being allometrically scaled bodyweight, CPB, postmenstrual age (PMA). Intercompartmental clearance (Q), peripheral volume (V2), systemic clearance, (CL) and the central volume (V1) were calculated. Typical values of the PK parameter estimates were as follows: CL = 3.78 [95 \% confidence interval (CI) 2.52, 5.05] l h The introduction of a modified dosing regimen using a starting bolus followed by an infusion and a CPB prime bolus would prohibit the potential risk of seizures caused by high peak concentrations and also maintain therapeutic plasma concentration above 20 μg ml [\hyperlink{Tranexamic Acid}{PMID: 28245519}, Ralph Gertler et al., 2017] The benefit-to-risk ratio of using tranexamic acid (TXA) in paediatric cardiac surgery has not yet been determined. This systematic review evaluated studies that compared TXA to placebo in children undergoing cardiac surgery. A systematic search was conducted in all relevant randomized controlled trials. The following information was extracted from the studies and analysed if relevant: demographic data, TXA dose and regimen of administration, cardiopulmonary bypass time, blood loss and blood product transfusion at 24 h. From the studies screened, only 8 (848 patients) were included in the analysis. Most data were heterogeneously distributed and could not be analysed. Further, transfusion policies were not well defined for each study. TXA reduced the need for red blood cell transfusion by 6.4 ml kg(-1) day(-1) (I(2) = 0\%, P = 0.45), platelet transfusion by 3.7 ml kg(-1) day(-1) (I(2) = 0\%, P = 0.46) and fresh frozen plasma transfusion by 5.4 ml kg(-1) day(-1) (I(2) = 0\%, P = 0.53). The number of children who avoided all blood product transfusions was not reported in most of the studies. Evaluation of the side effects associated with TXA use and the effects of the agent on postoperative morbidity and mortality was not possible from the data. There was marked variability in the dosage and infusion schemes used in different studies. This systematic review showed that in paediatric cardiac surgery, the benefit-to-risk ratio associated with the use of TXA cannot be adequately defined. Evidence supporting the routine use of TXA in paediatric cardiac surgery remains weak. [\hyperlink{Tranexamic Acid}{PMID: 28245519}, David Faraoni et al., 2012]

\hypertarget{pmid_33912417}{T}he aim of the present study was to evaluate and select the optimal dosage of tranexamic acid (TXA) to reduce blood loss during cleft palate surgery in children. This randomized double-blind clinical trial was performed on 80 children under 3 years of age that were candidates for cleft palate surgery. These children were divided into four groups as follows: the first, second, and third groups received 5, 7.5, and 10 mg/kg of TXA, respectively. Moreover, the fourth group was considered as the control group. Before induction of anesthesia and then every 15 min during the surgery, some parameters such as mean arterial pressure, heart rate, SpO The amount of blood loss during the surgery in TXA groups receiving dosages of 5, 7.5, and 10 mg/kg with the means of 63.75 ± 10.62, 61.25 ± 15.03, and 61.00 ± 14.29, respectively, was significantly lower than that of the control group with the mean of 92.25 ± 19.83 ( According to the results of the present study, all three dosages of TXA had a significant role in reducing blood loss in cleft palate surgery. Given the potential for increased risk of side effects from the drug, it seems safe to use the minimal dosage of this drug to control and reduce blood loss during cleft palate surgery in children <3 years of age. [\hyperlink{Tranexamic Acid}{PMID: 33912417}, Amir Shafa et al., ]

\hypertarget{pmid_36752620}{T}ranexamic acid (TXA) administration confers a survival benefit in bleeding trauma patients; however, data regarding its use in pediatric patients are limited. This study evaluates the prehospital treatment with TXA in pediatric trauma patients treated by the Israel Defense Forces Medical Corps (IDF-MC). Retrospective, cohort study using the Israel Defense Forces registry, 2011-2021. Pediatric trauma patients less than 18 years old. We excluded patients pronounced dead at the scene. None. All cases of pediatric trauma in the registry were assessed for treatment with TXA. Propensity score matching was used to assess the association between prehospital TXA administration and mortality. Overall, 911 pediatric trauma patients were treated with TXA by the IDF-MC teams; the median (interquartile) age was 10 years (5-15 yr), and 72.8\% were male. Seventy patients (7.6\%) received TXA, with 52 of 70 (74\%) receiving a 1,000 mg dose (range 200-1,000 mg). There were no prehospital adverse events associated with the use of TXA (upper limit of 95\% CI for 0/70 is 4.3\%). Compared with pediatric patients who did not receive TXA, patients receiving TXA were more likely to suffer from shock (40\% vs 10.7\%; p < 0.001), sustain more penetrating injuries (72.9\% vs 31.7\%; p < 0.001), be treated with plasma or crystalloids (62.9\% vs 11.4\%; p < 0.001), and undergo more lifesaving interventions (24.3\% vs 6.2\%; p < 0.001). The propensity score matching failed to identify an association between TXA and lesser odds of mortality, although a lack of effect (or even adverse effect) could not be excluded (non-TXA: 7.1\% vs TXA: 4.3\%, odds ratio = 0.584; 95\% CI 0.084-3.143; p = 0.718). Although prehospital TXA administration in the pediatric population is feasible with adverse event rate under 5\%, more research is needed to determine the appropriate approach to pediatric hemostatic resuscitation and the role of TXA in this population. [\hyperlink{Tranexamic Acid}{PMID: 36752620}, Sami Gendler et al., 2023]

\hypertarget{pmid_24924340}{C}horoid plexus papilloma (CPP) is a highly vascular tumor of infancy. Reducing blood loss is the key to successful surgical removal of CPPs. Tranexamic acid (TXA) is efficacious in reducing bleeding in craniofacial surgery for infants. This report demonstrates the potential utility of TXA for decreasing blood loss in the removal of vascular tumors in infants. We administered tranexamic acid to two infants with CPP during surgical removal to potentially aid hemostasis and therefore lessen intra-operative bleeding. Gross total surgical resection was accomplished; the patients were hemodynamically stable perioperatively, and the total calculated blood loss was minimal at <20\% of the patients' total circulating blood volume. This is the first report of tranexamic acid administration for CPP surgery in children. TXA is an easily administered hemostatic agent and may merit further study as an agent to help reduce intra-operative blood loss in this vulnerable population.  [\hyperlink{Tranexamic Acid}{PMID: 24924340}, Ji H Phi et al., 2014] Trauma is the leading cause of death and disability in children in the USA. Tranexamic acid (TXA) reduces the blood transfusion requirements in adults and children during surgery. Several studies have evaluated TXA in adults with hemorrhagic trauma, but no randomized controlled trials have occurred in children with trauma. We propose a Bayesian adaptive clinical trial to investigate TXA in children with brain and/or torso hemorrhagic trauma. We designed a double-blind, Bayesian adaptive clinical trial that will enroll up to 2000 patients. We extend the traditional E This trial design evaluating TXA in pediatric hemorrhagic trauma allows for three separate injury populations to be analyzed and compared within a single study framework. Individual conclusions regarding optimal dosing of TXA can be made within each injury group. Identifying the optimal dose of TXA, if any, for various injury types in childhood may reduce death and disability. [\hyperlink{Tranexamic Acid}{PMID: 24924340}, John M VanBuren et al., 2021]

\hypertarget{pmid_25043066}{T}rauma is a leading cause of death in pediatrics. Currently, no medical treatment exists to reduce mortality in the setting of pediatric trauma; however, this evidence does exist in adults. Bleeding and coagulopathy after trauma increases mortality in both adults and children. Clinical research has demonstrated a reduction in mortality with early use of tranexamic acid in adult trauma patients in both civilian and military settings. Tranexamic acid used in the perioperative setting safely reduces transfusion requirements in children. This article compares the hematologic response to trauma between children and adults, and explores the potential use of tranexamic acid in pediatric hemorrhagic trauma. [\hyperlink{Tranexamic Acid}{PMID: 25043066}, Suzanne Beno et al., 2014]

\hypertarget{pmid_24015121}{T}his study was conducted to evaluate the effect of tranexamic acid (TA) on the intra-operative bleeding during the functional endoscopic sinus surgery (FESS) in children. A total of 100 children recruited to undergo FESS were randomized into two groups. Group I: Was given just after induction, intra-venous 25 mg/kg TA diluted in 10 ml of normal saline. Group II: Was given 10 ml of normal saline. Non-invasive blood pressure, heart rate, and quality of the surgical field were estimated every 15 min. Volume of bleeding and duration of the surgical procedure were recorded. Surgical field quality after 15 min revealed that seven patients in group I had minimal bleeding versus no one in group II, P=0.006. Meanwhile, 35 patients in group I had mild bleeding versus 26 patients in group II, P=0.064. Higher number of patients in group II than in group I had moderate bleeding, P=0006. Also, at 30 min, revealed that 10 patients in group I had minimal bleeding versus one patient in group II, P=0.004. Meanwhile, 37 patients in group I had mild bleeding versus 28 patients in group II, P=0.059. Higher number of patients in group II than in group I had moderate bleeding, P<0001. Duration of the surgeries and volume of bleeding were significantly less in tranexamic group than the placebo group, P<0.0001. Single intra-venous bolus dose of tranexamic in children during the FESS improves quality of surgical field, reduces intra-operative bleeding, and duration of surgery. [\hyperlink{Tranexamic Acid}{PMID: 24015121}, Ahmed A Eldaba et al., 2013]

\section*{Daunorubicin Hydrochloride}
\subsection*{Result}
\subsubsection*{Answer}

Infants (<1 year): Unknown  
Children (1–16 years): Yes (with significant, well-characterized risks, especially cardiotoxicity, requiring monitoring)

\subsubsection*{{Explanation}}
\hypertarget{Daunorubicin Hydrochloride}
A review of the available abstracts reveals multiple targeted studies of Daunorubicin Hydrochloride (including its liposomal form) in children, primarily for the treatment of childhood cancers such as acute lymphoblastic leukemia (ALL) and brain tumors. The studies include children across a range of ages, from infants (<1 year) to adolescents.

Key findings by age range:

Infants (<1 year):
- One pharmacokinetic study included infants as young as 0.05 years (about 18 days) and up to 1.88 years. The study found no age-dependency in pharmacokinetics after body surface area normalization, and toxicities were similar to older children, though infants <6 months experienced more infections. No unique safety concerns were identified, but the sample size was small and the focus was on pharmacokinetics and toxicity, not comprehensive safety [\hyperlink{pmid_19731319}{PMID: 19731319}, Georg Hempel et al., 2010].

Children (1–16 years):
- Multiple studies specifically evaluated safety and toxicity in children receiving Daunorubicin for leukemia and brain tumors. These studies consistently report that Daunorubicin is effective but associated with dose-dependent cardiotoxicity, including both acute and chronic cardiac dysfunction. Some studies report mild to moderate toxicity at standard doses, while others note significant rates of cardiac dysfunction, especially at higher cumulative doses or in combination with other anthracyclines [\hyperlink{pmid_10100273}{PMID: 10100273}, R J Lippens et al.; \hyperlink{pmid_24259388}{PMID: 24259388}, Abdul Sattar Shaikh et al., 2013; \hyperlink{pmid_7758517}{PMID: 7758517}, K Schmitt et al., 1995; \hyperlink{pmid_34048168}{PMID: 34048168}, Sunny Mariana Samosir et al., 2021; \hyperlink{pmid_27858183}{PMID: 27858183}, Hiroki Hori et al., 2017].

- One long-term follow-up study (median 22 years) of survivors treated with low-dose Daunorubicin in childhood (diagnosed at ages 2–14) found no clinically relevant deterioration of cardiac function [\hyperlink{pmid_18648960}{PMID: 18648960}, C A J Brouwer et al., 2007].

- Another large cohort study (n=15,815) found that Daunorubicin is less cardiotoxic than doxorubicin in childhood cancer survivors, but still carries a risk of heart failure [\hyperlink{pmid_26304888}{PMID: 26304888}, Elizabeth A M Feijen et al., 2015].

- Several studies emphasize the importance of monitoring for cardiotoxicity and adjusting doses based on risk factors (age, sex, cumulative dose) [\hyperlink{pmid_19555570}{PMID: 19555570}, Yi-Na Sun et al., 2009; \hyperlink{pmid_34048168}{PMID: 34048168}, Sunny Mariana Samosir et al., 2021].

Summary:
- There is clear evidence from targeted pediatric studies that Daunorubicin Hydrochloride is used in children and infants for cancer treatment, with established dosing and monitoring protocols.
- The main safety concern is dose-dependent cardiotoxicity, which is well-documented and monitored in clinical practice.
- No studies affirm that Daunorubicin is "safe" in the sense of being free from significant risk; rather, its use is justified by the severity of the underlying disease and the balance of risks and benefits.
- There is no evidence from these abstracts that Daunorubicin is unsafe in children in the sense of being contraindicated or causing unacceptable harm when used as indicated and monitored.
- For infants (<1 year), data are more limited, but available studies do not indicate unique safety concerns beyond those seen in older children.

Therefore, based on the abstracts, Daunorubicin Hydrochloride is not "safe" in the sense of being without significant risk, but it is used in children with established protocols and known, manageable toxicities. Its safety profile is well-characterized, especially regarding cardiotoxicity, and it is not contraindicated in children or infants based on the available evidence.

\subsection*{Abstracts}
\hypertarget{pmid_10100273}{L}iposomal daunorubicin (DaunoXome = DNX) has been used in 14 children with recurrent or progressive growing brain tumor. DNX was given as a 1-h intravenous infusion with a dose of 60 mg/m2, once every 4 weeks, up to a cumulative dose of 600 mg/m2. At 3-month intervals the tumor process was evaluated on MRI or CT scan. Tumor response and toxicity of DNX were recorded according to the WHO guidelines. In 6 of the children a response has been established: 2 had complete responses, of which one relapsed again after 3 months; in 3 children a partial response was found. Two children showed stable disease. In 6 children the tumors grew progressively. In all responding children a remarkable subjective response was found. The toxicity of DNX at this dose was mild with a mild bone marrow depression and a slight but certain cardiotoxicity in 3 children. For the whole group the left ventricular function decreased with 13.8\%. In 1 child the DNX treatment was stopped because of a decrease of the shortening fraction to 20\%. In 4 children some hair loss was observed at the end of the treatment. In 3 children mental depression occurred that was associated with the administration of DNX. DNX is a well-tolerated and effective drug in the treatment of slowly progressive or recurrent brain tumors in children. [\hyperlink{Daunorubicin Hydrochloride}{PMID: 10100273}, R J Lippens et al., ]

\hypertarget{pmid_23146307}{D}aunorubicin is a chemotherapeutic antibiotic of the anthracycline family used for the treatment of many type of cancers when doxorubicin or other less effective drugs cannot be used. The aim of the present study was labeling of Daunorubicin with (99m)Tc, quality control, characterization, and biodistribution of radiolabeled Daunorubicin. Labeling efficiency was determined by ascending paper chromatography. All the experiments were performed at room temperature (25°C±2°C). More than 96\% labeling efficiency with (99m)Tc was achieved at pH 5-6, 2-4 μg stannous chloride and 300 μg of ligand in few minutes. The characterization of the compound was performed by using HPLC, electrophoresis and shake flask assay. Electrophoresis indicates that Tc-99m-Daunorubicin is neutral, HPLC confirms the single specie of the labeled compound, while shake flask assay confirms high lipophilicity. The biodistribution studies of (99m)Tc-Daunorubicin were performed in rats. Significantly higher accumulation of (99m)Tc-Daunorubicin was seen in brain of normal rats. Scintigraphy was also indicating higher accumulation of (99m)Tc-Daunorubicin in brain of normal rabbits. [\hyperlink{Daunorubicin Hydrochloride}{PMID: 23146307}, A R Faheem et al., 2013]

\hypertarget{pmid_7758517}{D}oxorubicin and daunorubicin are effective anticancer agents in children, however, their therapeutic value is limited by myocardial cardiotoxicity. In 14 children (median age 5.0 years, range 3-12) prospective studies were performed using pulsed Doppler echocardiography to assess the changes in left ventricular systolic and diastolic filling dynamics. None of these children developed cardiomyopathy. M-mode echocardiographic systolic parameters and Doppler transmitral flow velocities were analysed at baseline, after a cumulative anthracycline dose of 138 +/- 26 mg/m2 (second examination) and after 240 +/- 15 mg/m2 (third examination). At the second examination the acceleration time/ejection time ratio was significantly reduced (P < 0.01), but this was no longer evident at the third examination. There was no significant change of peak velocity over aortic valve, pre-ejection period and change of velocity over time. In contrast, three diastolic parameters changed significantly; the late over early inflow velocity (P < 0.05), mitral valve late time velocity integral (P < 0.01 at the second and P < 0.05 at the third examination) and the ratio A-TVI/TVI (P < 0.025 and P < 0.01). At the third examination the velocity of the A wave was also significantly increased. CONCLUSION In anthracycline treated children left ventricular diastolic function deteriorates before systolic function. Diastolic function parameters should be used rather than systolic parameters to monitor these patients. [\hyperlink{Daunorubicin Hydrochloride}{PMID: 7758517}, K Schmitt et al., 1995]

\hypertarget{pmid_17525906}{D}aunorubicin (DNR) is one of the most important drugs in treatment of acute lymphoblastic leukemia (ALL). Prolonged infusions of anthracyclines are less cardiotoxic but it has not been investigated whether the in vivo leukemic cell kill is equivalent to short-term infusions. In the cooperative treatment study COALL-92 for childhood ALL 178 patients were randomized to receive in a therapeutic window a single dose of 36 mg/m (2) DNR either as a 1-h (85 patients) or 24-h infusion (93 patients). Daily measurements of white blood cell count (WBC) and peripheral blood smears for seven days could be evaluated centrally in 101 patients (1-h: 43 patients, 24-h: 58 patients). The proportional decline of blasts at day 7 after DNR infusion showed no statistically significant difference between the two treatment arms. At day 3 the median percentage of blasts was less than 10\%, at day 7 less than 2\% for either the 1-h or 24-h infusion. Twelve patients (1-h: 5 patients, 24-h: 7 patients) had an absolute number of more than 1000 blasts per mul peripheral blood (PB) at day 7 after DNR infusion (DNR poor responders). Kaplan-Meier analysis showed an equal probability of EFS for the short- and long-term infusion group (24-h: 83\%+/-5; 1-h: 81+/-6) after a median observation time of 12.3 years. We conclude that in children with ALL a 24-h infusion of DNR has the same in vivo cytotoxicity for leukemic cells as a 1-h infusion. This offers the possibility to use prolonged infusions with hopefully less cardiotoxicity without loss of efficacy. [\hyperlink{Daunorubicin Hydrochloride}{PMID: 17525906}, G Escherich et al., ]

\hypertarget{pmid_24259388}{T}o identify anthracycline-induced acute (within 1 month) and early-onset chronic progressive (within 1 year) cardiotoxicity in children younger than 16 years of age with childhood malignancies at a tertiary care centre of Pakistan. Prospective cohort study. Aga Khan University, Karachi, Pakistan. 110 children (aged 1 month-16 years). Anthracycline (doxorubicin and/or daunorubicin). All children who received anthracycline as chemotherapy and three echocardiographic evaluations (baseline, 1 month and 1 year) between July 2010 and June 2012 were prospectively analysed for cardiac dysfunction. Statistical analysis including systolic and diastolic functions at baseline, 1 month and 1 year was carried out by repeated measures analysis of variance. Mean age was 74±44 months and 75 (68.2\%) were males. Acute lymphoblastic leukaemia was seen in 70 (64\%) patients. Doxorubicin alone was used in 59 (54\%) and combination therapy was used in 35 (32\%). A cumulative dose of anthracycline <300 mg/m(2) was used in 95 (86\%). Fifteen (14\%) children developed cardiac dysfunction within a month and 28 (25\%) children within a year. Of these 10/15 (66.6\%) and 12/28 (43\%) had isolated diastolic dysfunction, respectively, while 5/15 (33.3\%) and 16/28 (57\%) had combined systolic and diastolic dysfunction. Seven (6.4\%) patients expired due to severe cardiac dysfunction. Eight of 59 (13.5\%) children showed dose-related cardiotoxicity (p=<0.001). Cardiotoxicity was also high when the combination of doxorubicin and daunorubicin was used (p=0.004). Incidence of anthracycline-induced cardiotoxicity is high. Long-term follow-up is essential to diagnose its late manifestations. [\hyperlink{Daunorubicin Hydrochloride}{PMID: 24259388}, Abdul Sattar Shaikh et al., 2013]

\hypertarget{pmid_26304888}{C}umulative anthracycline dose is one of the strongest predictors of heart failure (HF) after cancer treatment. However, the differential risk for cardiotoxicity between daunorubicin and doxorubicin has not been rigorously evaluated among survivors of childhood cancer. These risks, which are based on hematologic toxicity, are currently assumed to be approximately equivalent. Data from 15,815 survivors of childhood cancer who survived at least 5 years were used. Survivors were from the Emma Children's Hospital/Academic Medical Center (n = 1,349), the National Wilms Tumor Study (n = 364), the St Jude Lifetime Cohort Study (n = 1,695), and the Childhood Cancer Survivor Study (n = 12,407). The hazard ratio (HR) for clinical HF through age 40 years for doses of daunorubicin and doxorubicin (per 100-mg/m(2) increments) was estimated by using Cox regression adjusted for sex, age at diagnosis, treatment with other anthracycline agents and chest radiation, and cohort membership. In total, 5,144 (32.5\%) patients received doxorubicin as part of their cancer treatment, whereas 2,243 (14.7\%) received daunorubicin. On the basis of 271 occurrences of HF during a median follow-up time after cohort entry of 17.3 years (range, 0.0 to 35.0 years), the cumulative incidence of HF at age 40 years was 3.2\% (95\% CI, 2.8\% to 3.7\%). The average ratio of HRs for daunorubicin to doxorubicin was 0.45 (95\% CI, 0.23 to 0.73). A similar ratio was obtained by using a linear dose-response model, which yielded an HR of 0.49 (95\% CI, 0.28 to 0.70). Compared with doxorubicin, daunorubicin was less cardiotoxic among survivors of childhood cancer than most current guidelines suggest. This may have implications for follow-up guidelines. The feasibility of substitution of doxorubicin with daunorubicin in childhood cancer treatment protocols to reduce cardiotoxicity should be additionally investigated. [\hyperlink{Daunorubicin Hydrochloride}{PMID: 26304888}, Elizabeth A M Feijen et al., 2015]

\hypertarget{pmid_19731319}{T}here is an extreme paucity of pharmacokinetic data for anticancer agents in infants. Therefore, we aimed at characterizing the pharmacokinetics for daunorubicin in infants and examined their relationship to age, body weight, and body surface area. Leukemia patients treated according to the Interfant 99 protocol received 30 mg/m(2) daunorubicin, with dose reduction to 3/4 for patients 6-12 months old and 2/3 for patients <6 months, respectively. Plasma samples from 21 patients (aged 0.05-1.88 years) were collected and analyzed for daunorubicin and daunorubicinol. Samples from 12 children (age 1.6-18.8 years), who received daunorubicin in an earlier investigation, were used for pharmacokinetic model building using the software NONMEM. Plasma concentration time profiles could be described using a two compartment model. Daunorubicin clearance was 43.9 L hr(-1) m(-2) +/- 65\% and central volume of distribution 16.4 L m(-2) +/- 46\%, whereas apparent clearance of daunorubicinol was 19.1 L hr(-1) m(-2) +/- 32\% and apparent volume of distribution 228 L m(-2) +/- 80\% (mean +/- interindividual variability). No age-dependency in any of the BSA-normalized pharmacokinetic parameters was observed. Consequently, due to the empirical dose reduction in infants the overall exposure to daunorubicinol in infants was smaller than would be expected from older children. Patients aged <6 months experienced more infections in the induction phase than the group aged 6-12 months at diagnosis. Other toxicities were similar in both groups. We observed no indication of an age-dependency in the pharmacokinetics of daunorubicin. Pediatr Blood Cancer 2010;54:355-360. [\hyperlink{Daunorubicin Hydrochloride}{PMID: 19731319}, Georg Hempel et al., 2010]

\hypertarget{pmid_19555570}{T}o study relationship between daunorubicin (DNR) pharmacokinetics and efficacy and toxicity in children with acute leukemia. (1) The concentration of DNR in plasma of children with acute leukemia was determined by high performance liquid chromatography (HPLC)-fluorescence detection method. Plasma was sampled frequently from the start of the infusion till the end of 24 h. DNR pharmacokinetics was studied by determination of the concentrations. (2) Efficacy and toxicity were monitored in each period after chemotherapy. Laboratory studies included examination of bone marrow, white blood cell count, electrocardiogram, echocardiogram, myocardial enzymogram, the liver and kidney function. (1) DNR was eliminated from plasma in a two-compartment manner. The maximum concentration was seen 1 - 3 h after infusion. Cmax was 63.50 microg/L. Tmax was 1.45 h. The concentration decreased quickly to a low level of about 11.52 microg/L from the end of 2 hours infusion. There was a large inter-individual difference in pharmacokinetic parameters of DNR. The difference of CL was 9-fold, AUC was 8-fold, Cmax was 5-fold. (2) CL of male patients [57.99 L/(h.m(2))] was significantly lower than that of female patients [93.71 L/(h.m(2))] (P < 0.05). Tmax of children older than 6 years was 1.1 h, and that of children younger than 6 years was 1.8 h (P < 0.05); Cmax of children older than 6 years was 90.77 microg/L, younger than 6 years was 57.44 microg/L (P < 0.05). (1) There is a large inter-individual difference in pharmacokinetic parameters of DNR in children. It may predict individual variety of efficacy and toxicity. Therapeutic drug monitoring is important. (2) Male patients and children older than 6 years had a higher bioavailability and lower metabolism, toxicity may easily occur in such children, therefore they may need lower dose. [\hyperlink{Daunorubicin Hydrochloride}{PMID: 19555570}, Yi-Na Sun et al., 2009]

\hypertarget{pmid_7779709}{D}aunorubicin (DNR) is a major front-line drug in the treatment of childhood acute lymphoblastic leukaemia (ALL). Previously, we showed that in vitro resistance to DNR at diagnosis is related to a poor long-term clinical outcome in childhood ALL and that relapsed ALL samples are more resistant to DNR than untreated ALL samples. In cell line studies, idarubicin (IDR), aclarubicin (ACR) and mitoxantrone (MIT) showed a (partial) lack of cross-resistance to the conventional anthracyclines DNR and doxorubicin (DOX), but clinical studies in childhood ALL have been inconclusive about the suggested lack of cross-resistance. In the present study we determined the in vitro cross-resistance pattern between DNR, DOX, IDR, ACR and MIT in 48 untreated and 39 relapsed samples from children with ALL using the MTT assay. The relapsed ALL group was about twice as resistant to DNR, DOX, IDR, ACR and MTT as the untreated ALL group. Thus, resistance developed to all five drugs. We found a significant cross-resistance between DNR, DOX, IDR, ACR and MIT, although in some individual cases in vitro anthracycline cross-resistance was less pronounced. We conclude that IDR, ACR and MIT cannot circumvent in vitro resistance to DNR in childhood ALL. Clinical studies may still prove whether IDR, ACR or MIT has a more favourable toxicity profile than DNR. [\hyperlink{Daunorubicin Hydrochloride}{PMID: 7779709}, E Klumper et al., 1995]

\hypertarget{pmid_29794839}{E}vidence supports a significant reduction in the incidence of intraventricular hemorrhage (IVH) in preterm infants receiving delayed umbilical cord clamping (DCC). This study evaluated clinical feasibility, efficacy, and safety outcomes in preterm infants (<36 weeks' gestational age) who received DCC following a practice change implementation intended to reduce the incidence of IVH. Infants receiving DCC (45-60 seconds) were compared with a sample of infants receiving immediate umbilical cord clamping (<15 seconds) in a retrospective chart review (N = 354). The primary outcome measure was the prevalence of IVH. Secondary safety outcome measures of 1- and 5-minute Apgar scores, axillary temperature on neonatal intensive care unit admission, and initial 24-hour bilirubin level were also evaluated. Gestational age was examined for its effect on outcomes. Although the small number of infants with IVH precluded the ability to detect statistical significance, our raw data suggest DCC is efficacious in reducing the risk for IVH. For infants 29 or less weeks' gestational age, admission axillary temperature was significantly higher in those who received DCC. No differences were found in 1- and 5-minute Apgar scores, 24-hour bilirubin level, or hematocrit level between the two groups. Infants more than 29 weeks' gestational age who received DCC had significantly higher 1-minute Apgar scores, temperature, and 24-hour bilirubin level. Clinicians should advocate for the implementation of DCC as part of the resuscitative process for preterm neonates. Future studies are needed to evaluate the effect of DCC on other clinical outcomes and to investigate umbilical cord milking as an alternative approach to DCC. [\hyperlink{Daunorubicin Hydrochloride}{PMID: 29794839}, Christen Fenton et al., 2018]

\hypertarget{pmid_2522789}{T}he neuromuscular and cardiovascular effects of doxacurium chloride (BW A938U) were evaluated in 27 children (2-12 yr) anaesthetized with 1\% halothane and nitrous oxide in oxygen. In nine children the incremental technique was used to establish a cumulative dose-response curve by train-of-four stimulation. The remaining children received either 30 or 50 micrograms kg-1 of the drug as a single bolus. The median ED50 and ED95 of doxacurium in children were 19 and 32 micrograms kg-1, respectively. No clinically significant change in heart rate or arterial pressure occurred. Following doxacurium 30 micrograms kg-1 and 50 micrograms kg-1, recovery to 25\% of control occurred in 25 (SEM 6) and 44 (3) min, respectively. The recovery index (25-75\% of control) was 27 (2) min. The duration of action of doxacurium is similar to that of tubocurarine and dimethyl-tubocurarine in children. Compared with adults, children seem to require more doxacurium (microgram kg-1) to achieve a comparable degree of neuromuscular depression, and they recover more rapidly. [\hyperlink{Daunorubicin Hydrochloride}{PMID: 2522789}, N G Goudsouzian et al., 1989]

\hypertarget{pmid_24211979}{D}oxorubicin hydrochloride is widely used to treat various types of cancers. Its therapeutic and side effects are well documented. However, the developmental toxicity of doxorubicin has not been previously described. Lethal and sublethal effects on embryo-larval stages of zebrafish in a study of the developmental toxicity of doxorubicin were observed. Zebrafish embryos were exposed to different concentrations (0-100 mg/L) of doxorubicin between 4 and 120 h post fertilization, and zebrafish larvae were exposed to different concentrations (0-200 mg/L) of doxorubicin for 96 h. The markers about the development toxicity of doxorubicin in zebrafish were observed under a stereomicroscope. Higher doxorubicin concentrations mainly caused acute lethal effects, and lower doxorubicin concentrations mainly caused sublethal effects, such as multiple malformations in embryos and larvae. Moreover, with the increase of doxorubicin concentration, the malformation rate increased. The heart rate of embryos was accelerated at lower concentrations of doxorubicin (≤ 10 mg/L) and decelerated at higher concentrations (≥ 25 mg/L). The hatching rate and body length were inhibited at higher concentrations of doxorubicin (≥ 25 mg/L).In conclusion, doxorubicin has serious developmental toxicity and this raises a concern for developmental effects of doxorubicin in clinical practice.  [\hyperlink{Daunorubicin Hydrochloride}{PMID: 24211979}, Can Chang et al., 2014] The anthracyclines daunorubicin (DNR) and doxorubicin (DOX) are among the most important drugs in the treatment of childhood acute lymphoblastic leukemia, however there are conflicting in vitro data about the comparative efficacy and equivalent doses of both anthracyclines. To address the question of in vivo efficacy of both anthracyclines, patients enrolled in the CoALL 07-03 trial were randomized to receive one single dose of either doxorubicin 30 mg/m(2) , daunorubicin 30 mg/m(2) , or daunorubicin 40 mg/m(2) upfront induction therapy. Children with newly diagnosed B-Precursor ALL or T-ALL were eligible for the randomized comparison. From the percentage of blasts and the white blood cell count (WBC) the absolute number of leukemic cells per µl peripheral blood (PB) was calculated and the initial value before DOX/DNR infusion equated as 100\%. Main target criterion of this study was the leukemic cell decrease from Day 0 to Day 7. Seven hundred forty three patients were randomized: 247 to the DOX; 252 to the DNR 30 mg/m(2) ; and DNR to the 40 mg/m(2) arm. The in vivo response was similar in all three treatment arms with a comparable blast decline in the peripheral blood. The percentages of patients with a clear non-response (M3 marrow) and moreover, the level of minimal residual disease (MRD) on Day 15 or at the end of induction were similar. In vivo efficacy of a single dose daunorubicin 30 or 40 mg/m(2) is similar to that of doxorubicin given in a dose of 30 mg/m(2) . [\hyperlink{Daunorubicin Hydrochloride}{PMID: 24211979}, Gabriele Escherich et al., 2013]

\hypertarget{pmid_34048168}{D}aunorubicine, a type of anthracycline, is a drug commonly used in cancer chemotherapy that increases survival rate but consequently compromises with cardiovascular outcomes in some patients. Thus, preventing the early progression of cardiotoxicity is important to improve the treatment outcome in childhood acute lymhoblastic leukemia (ALL). The present study aimed to identify the risk factors in anthracycline-induced early cardiotoxicity in childhood ALL. This retrospective study was conducted by observing ALL-diagnosed children from 2014 to 2019 in Dr. Soetomo General Hospital. There were 49 patients who met the inclusion criteria and were treated with chemotherapy using Indonesian Childhood ALL Protocol 2013. Echocardiography was performed by pediatric cardiologists to compare before and at any given time after anthracycline therapy. Early cardiotoxicity was defined as a decline of left ventricle ejection fraction (LVEF) greater than 10\% with a final LVEF < 53\% during the first year of anthracycline administration.  Risk factors such as sex, age, risk stratification group, and cumulative dose were identified by using multiple logistic regression. Diagnostic performance of cumulative anthracycline dose was evaluated by receiver operating characteristic (ROC) curve. Early anthracycline-induced cardiotoxicity was observed in 5 out of 49 patients. The median cumulative dose of anthracycline was 143.69±72.68 mg/m2. Thirty-three patients experienced a decreasing LVEF. The factors associated with early cardiomyopathy were age of ≥ 4 years (PR= 1.128; 95\% CI: 1.015-1.254; p= 0.001), high risk group (PR= 1.135; 95\% CI: 1.016-1.269; p= 0.001), and cumulative dose of ≥120 mg / m2 (CI= 1.161; 95\% CI:1.019-1.332). Age of ≥ 4 years, risk group, and cumulative dose of ≥120 mg/m2 are significant risk factors for early cardiomyopathy in childhood ALL. [\hyperlink{Daunorubicin Hydrochloride}{PMID: 34048168}, Sunny Mariana Samosir et al., 2021]

\hypertarget{pmid_30231396}{A}nthracyclines (doxorubicin, daunorubicin, epirubicin, and idarubicin) are among the most potent chemotherapeutic agents and have truly revolutionized the management of childhood cancer. They form the backbone of chemotherapy regimens used to treat childhood acute lymphoblastic leukemia, acute myeloid leukemia, Hodgkin lymphoma, Ewing sarcoma, osteosarcoma, and neuroblastoma. More than 50\% of children with cancer are treated with anthracyclines. The clinical utility of anthracyclines is compromised by dose-dependent cardiotoxicity, manifesting initially as asymptomatic cardiac dysfunction and evolving irreversibly to congestive heart failure. Childhood cancer survivors are at a five- to 15-fold increased risk for congestive heart failure compared with the general population. Once diagnosed with congestive heart failure, the 5-year survival rate is less than 50\%. Prediction models have been developed for childhood cancer survivors (i.e., after exposure to anthracyclines) to identify those at increased risk for cardiotoxicity. Studies are currently under way to test risk-reducing strategies. There remains a critical need to identify patients with childhood cancer at diagnosis (i.e., prior to anthracycline exposure) such that noncardiotoxic therapies can be contemplated. [\hyperlink{Daunorubicin Hydrochloride}{PMID: 30231396}, Saro Armenian et al., 2018]

\hypertarget{pmid_17242696}{T}his review systematically assessed the evidence on the clinical and cost-effectiveness of cardioprotection against the toxic effects of anthracyclines given to children with cancer. We searched eight electronic databases, including Medline and the Cochrane Library, from inception to January 2006 for systematic reviews and randomised controlled trials that reported death, heart failure, arrhythmias or measures of cardiac performance associated with cardioprotective technologies compared with standard treatment in children treated for cancer with anthracyclines. Economic evaluations were also sought. Inclusion criteria, data extraction and quality assessment were undertaken by standard methodology. Four randomised controlled trials met the inclusion criteria of the review; each had methodological limitations. No economic evaluations were identified. Studies were combined through narrative synthesis. One trial found that continuous infusion of doxorubicin did not offer any cardioprotection over rapid infusion. One suggested that continuous infusion of daunorubicin provoked less cardiotoxicity than rapid infusion. One concluded that dexrazoxane reduces cardiac injury during doxorubicin therapy and one reported a protective effect of coenzyme Q(10) on cardiac function during anthracycline therapy. The evidence on the effectiveness of cardioprotective technologies in children is limited in quality and quantity thus making conclusions difficult. This is surprising given the importance of anthracycline use in children with cancer. Further long-term research, which includes relevant outcome measures, is needed to determine whether technologies influence the development of cardiac damage without limiting the antitumour efficacy of anthracyclines. [\hyperlink{Daunorubicin Hydrochloride}{PMID: 17242696}, J Bryant et al., 2007]

\hypertarget{pmid_36174614}{S}urvivors of childhood cancer are at risk of anthracycline-induced cardiotoxicity, which might be prevented by dexrazoxane. However, concerns exist about the safety of dexrazoxane, and little guidance is available on its use in children. To facilitate global consensus, a working group within the International Late Effects of Childhood Cancer Guideline Harmonization Group reviewed the existing literature and used evidence-based methodology to develop a guideline for dexrazoxane administration in children with cancer who are expected to receive anthracyclines. Recommendations were made in consideration of evidence supporting the balance of potential benefits and harms, and clinical judgement by the expert panel. Given the dose-dependent risk of anthracycline-induced cardiotoxicity, we concluded that the benefits of dexrazoxane probably outweigh the risk of subsequent neoplasms when the cumulative doxorubicin or equivalent dose is at least 250 mg/m [\hyperlink{Daunorubicin Hydrochloride}{PMID: 36174614}, Esmée C de Baat et al., 2022] The killed oral cholera vaccine Dukoral is recommended for adults and only children over 2 years of age, although cholera is seen frequently in younger children and there is an urgent need for a vaccine for them. Since decreased immunogenicity of oral vaccines in children in developing countries is a critical problem, we tested interventions to enhance responses to Dukoral. We evaluated the effect on the immune responses by temporarily withholding breast-feeding or by giving zinc supplementation. Two doses of Dukoral consisting of killed cholera vibrios and cholera B subunit were given to 6-18 months old Bangladeshi children (n=340) and safety and immunogenicity studied. Our results showed that two doses of the vaccine were safe and induced antibacterial (vibriocidal) antibody responses in 57\% and antitoxin responses in 85\% of the children. Immune responses were comparable after intake of one and two doses. Temporary withholding breast-feeding for 3 h before immunization or supplementation with 20 mg of zinc per day for 42 days resulted in increased magnitude of vibriocidal antibodies (77\% and 79\% responders, respectively). Administration of vaccines without buffer or in water did not result in reduction of vibriocidal responses. This study demonstrates that the vaccine is safe and immunogenic in children under 2 years of age and that simple interventions can enhance immune responses in young children. [\hyperlink{Daunorubicin Hydrochloride}{PMID: 36174614}, Tanvir Ahmed et al., 2009]

\hypertarget{pmid_23166343}{D}oxorubicin, effective against many malignancies, is limited by cardiotoxicity. Continuous-infusion doxorubicin, compared with bolus-infusion, reduces early cardiotoxicity in adults. Its effectiveness in reducing late cardiotoxicity in children remains uncertain. We determined continuous-infusion doxorubicin cardioprotective efficacy in long-term survivors of childhood acute lymphoblastic leukemia (ALL). The Dana-Farber Cancer Institute ALL Consortium Protocol 91-01 enrolled pediatric patients between 1991 and 1995. Newly diagnosed high-risk patients were randomly assigned to receive a total of 360 mg/m(2) of doxorubicin in 30 mg/m(2) doses every 3 weeks, by either continuous (over 48 hours) or bolus-infusion (within 15 minutes). Echocardiograms at baseline, during, and after doxorubicin therapy were blindly remeasured centrally. Primary outcomes were late left ventricular (LV) structure and function. A total of 102 children were randomized to each treatment group. We analyzed 484 serial echocardiograms from 92 patients (n = 49 continuous; n = 43 bolus) with ≥1 echocardiogram ≥3 years after assignment. Both groups had similar demographics and normal baseline LV characteristics. Cardiac follow-up after randomization (median, 8 years) showed changes from baseline within the randomized groups (depressed systolic function, systolic dilation, reduced wall thickness, and reduced mass) at 3, 6, and 8 years; there were no statistically significant differences between randomized groups. Ten-year ALL event-free survival rates did not differ between the 2 groups (continuous-infusion, 83\% versus bolus-infusion, 78\%; P = .24). In survivors of childhood high-risk ALL, continuous-infusion doxorubicin, compared with bolus-infusion, provided no long-term cardioprotection or improvement in ALL event-free survival, hence provided no benefit over bolus-infusion. [\hyperlink{Daunorubicin Hydrochloride}{PMID: 23166343}, Steven E Lipshultz et al., 2012]

\hypertarget{pmid_27858183}{A}nthracyclines are used to treat childhood acute lymphoblastic leukemia (ALL). Even when administered at low doses, these agents are reported to cause progressive cardiac dysfunction. We conducted a clinical trial comparing the toxicities of two anthracyclines, pirarubicin (THP) and daunorubicin (DNR), in the treatment of childhood ALL. The results from our study that relate to acute and late toxicities are reported here. 276 children with B-ALL were enrolled in the trial from April 1997 to March 2002 and were randomly assigned to receive a regimen including either THP (25 mg/m Acute hematological toxicity in the early phase was more significant in the THP arm. Based on ultrasound cardiography, cardiac function was impaired in both groups during the follow-up period, but there was no significant difference between the groups except for a greater decline in fractional shortening on ultrasound cardiography in the DNR arm. While acute hematological toxicity was more significant in the THP arm, THP also appeared to be less cardiotoxic. However, the evaluation of late cardiotoxicity was limited because only a few subjects were followed beyond 10 years after ALL onset. Considering that the THP regimen produced an EFS rate comparable with that of the DNR regimen, the efficacy and toxicity of THP at reduced doses should be studied in order to identify potentially safer regimens. [\hyperlink{Daunorubicin Hydrochloride}{PMID: 27858183}, Hiroki Hori et al., 2017]

\hypertarget{pmid_20819318}{A}llergic rhinitis (AR) and chronic idiopathic urticaria (CIU) are common causes of substantial illness and disability in preschool children. Antihistamines are commonly used to treat preschool children with these conditions, but their use is based mostly on extrapolated efficacy from adult populations; it is thus important to characterize the safety of antihistamines in the pediatric population. This study was designed to assess the safety of levocetirizine dihydrochloride oral liquid drops in infants and children with AR or CIU. Two multicenter, double-blind, randomized, parallel-group studies randomized infants aged 6-11 months (study 1, n = 69) and children aged 1-5 years (study 2, n = 173) to levocetirizine, 1.25 mg (q.d. or b.i.d., respectively), or placebo for 2 weeks, using a 2:1 ratio. Safety evaluations included treatment-emergent adverse events (TEAEs), vital signs, electrocardiographic (ECG) assessments, and laboratory tests. The overall incidence of TEAEs was similar between levocetirizine and placebo in both studies. Most TEAEs were mild or moderate in intensity. TEAEs prompted discontinuation of therapy in three patients receiving levocetirizine in study 1. No clinically relevant changes from baseline in vital signs or laboratory parameters were apparent in either study; changes from baseline in these evaluations were similar between groups. No significant changes were observed in ECG parameters, including corrected QT interval. Levocetirizine, 1.25 and 2.5 mg/day, was well tolerated in infants aged 6-11 months and in children aged 1-5 years, respectively, with AR or CIU. [\hyperlink{Daunorubicin Hydrochloride}{PMID: 20819318}, Frank Hampel et al., ]

\hypertarget{pmid_36961611}{C}ardiotoxicity is a major concern following doxorubicin (DOX) use in the treatment of malignancies. We aimed to investigate whether deferoxamine (DFO) can prevent acute cardiotoxicity in children with cancer who were treated with DOX as part of their chemotherapy. Sixty-two newly-diagnosed pediatric cancer patients aged 2-18 years with DOX as part of their treatment regimens were assigned to three groups: group 1 (no intervention, n = 21), group II (Deferoxamine (DFO) 10 times DOX dose, n = 20), and group III (DFO 50 mg/kg, n = 21). Patients in the intervention groups were pretreated with DFO 8-h intravenous infusion in each chemotherapy course during and after completion of DOX infusion. Conventional and tissue Doppler echocardiography, serum concentrations of human brain natriuretic peptide (BNP), and cardiac troponin I (cTnI) were checked after the last course of chemotherapy. Sixty patients were analyzed. The level of cTnI was < 0.01 in all patients. Serum BNP was significantly lower in group 3 compared to control subjects (P = 0.036). No significant differences were observed in the parameters of Doppler echocardiography. Significant lower values of tissue Doppler late diastolic velocity at the lateral annulus of the tricuspid valve were noticed in group 3 in comparison with controls. By using Pearson analysis, tissue Doppler systolic velocity of the septum showed a marginally significant negative correlation with DOX dose (P = 0.05, r = - 0.308). No adverse effect was reported in the intervention groups. High-dose DFO (50 mg/kg) may serve as a promising cardioprotective agent at least at the molecular level in cancer patients treated with DOX. Further multicenter trials with longer follow-ups are needed to investigate its protective role in delayed DOX-induced cardiac damage. Trial registration IRCT, IRCT2016080615666N5. Registered 6 September 2016, http://www.irct.ir/IRCT2016080615666N5 . [\hyperlink{Daunorubicin Hydrochloride}{PMID: 36961611}, Kosar Rahimi et al., 2023]

\hypertarget{pmid_18648960}{I}n children with cancer a well-known risk factor for cardiotoxicity is a high cumulative dose of anthracyclines, but little is known about cardiac function in low-dose anthracycline-treated survivors. Also, it is unclear if a safe anthracycline-dose exists at all. Cardiac function was assessed in 23 long-term ALL-survivors with a median follow-up of 22 years (range 19.5-24.5) post-treatment. Age at diagnosis and current age were 5.0 (2.0-14.0) and 29.0 (24.0-39.0) years. All 23 survivors were treated according to DCLSG protocol ALL-5, including 18-25 Gy cranial irradiation. Thirteen of them received 4 x 25 mg/m(2) daunorubicin by randomization. Cardiac evaluation included blood pressure measurement, echocardiography, and (24 h-) electrocardiogram. Results were compared with an earlier assessment at median 12 years post-treatment. None of the survivors had cardiac abnormalities. Cardiac status of daunorubicin-treated survivors showed no deterioration compared with the previous assessment in 1995. CONCLUSION AND IMPLICATION FOR CANCER SURVIVORS: After prolonged follow-up (more than 20 years post-treatment), ALL-survivors treated with low dose daunorubicin had no clinical relevant deterioration of cardiac function. [\hyperlink{Daunorubicin Hydrochloride}{PMID: 18648960}, C A J Brouwer et al., 2007]

\hypertarget{pmid_34431211}{A}BVD (doxorubicin, bleomycin,vinblastine, and dacarbazine) is not a standard regimen in children due to concerns regarding late effects. However, no studies have evaluated long-term toxicities of ABVD in children. Total 154 pediatric Hodgkin lymphoma (HL) survivors uniformly treated with ABVD were clinically followed up as per institutional protocol. All participants were evaluated for cardiac, pulmonary, and thyroid function abnormalities by multigated acquisition scan (MUGA) scan, spirometry with diffusion capacity of lung for the uptake of carbon monoxide (DLCO), and thyroid profile test, respectively, at a single time point. Predictors of toxicity were also analyzed. The median duration of follow-up of the cohort was 10.3 years (6.04-16.8). No secondary malignant neoplasm (SMN) or symptomatic cardiac/pulmonary toxicities were detected. Nine patients (5.9\%) had left ventricular ejection fraction (LVEF) <55\%. Subclinical and overt hypothyroidism were observed in 78 (50.6\%) and 16 (10.4\%) survivors, respectively. Abnormal spirometry and reduced DLCO was observed in 43.2\% and 42.0\% survivors, respectively. Receiving neck radiation was significantly associated with thyroid dysfunction (odds ratio [OR] 16.04, p < .001); age ≥10 years predicted reduced DLCO (OR 4.12, p = .001). Sixty-three and 33 patients had one and two late adverse effects, respectively; receiving neck radiation predicted development of multiple late effects (proportional OR 4.72, p < 0.001). Cumulative dose of chemotherapy did not predict toxicity. Overall, ABVD appears safe in children at a relatively short follow-up. Long-term safety data are required before it can be adopted for treating pediatric HL patients. Children receiving neck radiation require close follow-up. [\hyperlink{Daunorubicin Hydrochloride}{PMID: 34431211}, Abhenil Mittal et al., 2021]

\hypertarget{pmid_11493818}{C}ontrolled intubation in the pediatric emergency department (ED) requires a paralytic agent that is safe, efficacious, and of rapid onset. The safety of succinylcholine has been challenged, leading some clinicians to use vecuronium as an alternative. Rocuronium's onset is similar to that of succinylcholine. To evaluate the safety and efficacy of rocuronium for controlled intubation with paralysis (CIP) in the pediatric ED. A retrospective, observational study reviewed the records of patients less than 15 years of age, who received controlled intubation with paralytics at two Dallas EDs. The patients received either vecuronium or rocuronium. The study included 84 patients (vecuronium 19, rocuronium 65). Complications were similar between the two groups. Rocuronium had a shorter time from administration to intubation when compared to vecuronium (P < 0.05). Rocuronium is as safe and efficacious as vecuronium for CIP in the pediatric ED. [\hyperlink{Daunorubicin Hydrochloride}{PMID: 11493818}, D R Mendez et al., 2001]

\section*{Methylprednisolone Acetate}
\subsection*{Result}
\subsubsection*{Answer}

Yes (Neonates, Infants, Children, Adolescents)

\subsubsection*{{Explanation}}
\hypertarget{Methylprednisolone Acetate}
A review of the available abstracts reveals multiple targeted studies evaluating the safety of Methylprednisolone Acetate (or its closely related forms, such as methylprednisolone sodium succinate or unspecified methylprednisolone) in children across various age ranges and indications. Below is a summary by age group and indication:

Infants and Young Children (0–2 years):
- Several studies have evaluated methylprednisolone in infants and young children, particularly for conditions such as infantile spasms (West syndrome), acute respiratory distress syndrome (ARDS), and severe infections.
    - In a case report, a 21-month-old child with ARDS was treated with methylprednisolone and improved without reported adverse effects, suggesting safety in this instance [\hyperlink{pmid_16891686}{PMID: 16891686}, Lokesh Guglani et al., 2006].
    - Multiple studies compared pulse methylprednisolone to ACTH in infants with West syndrome (3–24 months), finding methylprednisolone to be as effective as ACTH with no significant or persistent adverse effects reported [\hyperlink{pmid_35087722}{PMID: 35087722}, Kanij Fatema et al., 2021; \hyperlink{pmid_33103229}{PMID: 33103229}, Madan Rajpurohit et al., 2021].
    - Another study in infants with infantile spasms (age 3 months–2 years) found methylprednisolone therapy to be relatively safe, with hypertension less common than with ACTH [\hyperlink{pmid_33103229}{PMID: 33103229}, Madan Rajpurohit et al., 2021].
    - In neonates (≤30 days) undergoing heart surgery, a large multicenter observational study found no mortality benefit and suggested a possible increased risk of infection with methylprednisolone, but did not report other major safety concerns [\hyperlink{pmid_22271697}{PMID: 22271697}, Sara K Pasquali et al., 2012].
    - In very preterm infants (<30 weeks gestation), methylprednisolone was compared to dexamethasone for chronic lung disease. Methylprednisolone was as effective and had fewer side effects, but the authors called for further randomized trials to confirm safety [\hyperlink{pmid_11126262}{PMID: 11126262}, P André et al., 2000].

Children (2–18 years):
- Studies in older children have evaluated methylprednisolone for various indications:
    - In children with longitudinal extensive transverse myelitis (age not specified, but "children"), methylprednisolone was used with no significant adverse effects reported [\hyperlink{pmid_32292451}{PMID: 32292451}, Muhammad Azeem Ashfaq et al.].
    - In children with intractable epilepsy (ages 2–14), intravenous methylprednisolone was used, and no major side effects were noted [\hyperlink{pmid_24507698}{PMID: 24507698}, Kholoud H Almaabdi et al., 2014].
    - In children with chronic rhinosinusitis (ages 6–17), oral methylprednisolone was well tolerated, with no difference in adverse events compared to placebo [\hyperlink{pmid_21624649}{PMID: 21624649}, Fadıl Ozturk et al., 2011].
    - In children with juvenile chronic arthritis (ages 3–14 years, mean 9.7), pulse methylprednisolone provided benefit with no serious side effects noted [\hyperlink{pmid_9851278}{PMID: 9851278}, A O Adebajo et al., 1998].
    - In children with second and third degree esophageal burns, high-dose methylprednisolone was used, and the study focused on efficacy rather than safety, but no specific safety concerns were reported [\hyperlink{pmid_15686192}{PMID: 15686192}, Sonia Mazigh Mrad et al., 2004].
    - In children undergoing cardiac surgery (ages 1–18 months), both high and low doses of methylprednisolone were compared. The higher dose was associated with more frequent hyperglycemia, but no other major safety issues were reported [\hyperlink{pmid_27154159}{PMID: 27154159}, Juho Keski-Nisula et al., 2016].

Adverse Reactions:
- One case series reported hypersensitivity (anaphylactic/anaphylactoid) reactions to methylprednisolone sodium succinate in three children (ages 5, 7, and 8) with asthma, highlighting the potential for rare but serious allergic reactions [\hyperlink{pmid_19444032}{PMID: 19444032}, Amit Nahum et al., 2009].

Summary:
- Across multiple studies and indications, methylprednisolone (including acetate and other forms) has been studied in children from neonates to adolescents. Most studies report that methylprednisolone is generally safe and well tolerated in children, with adverse effects comparable to other corticosteroids or placebo, except for rare hypersensitivity reactions and increased risk of infection in certain subgroups (e.g., neonates after heart surgery).
- No studies were found that specifically identified methylprednisolone acetate as unsafe in children.
- The safety profile is supported by targeted studies in infants, children, and adolescents, though some studies call for further randomized trials for confirmation in specific populations (e.g., very preterm infants).

Therefore, based on the available abstracts, methylprednisolone (including acetate and other forms) appears to be safe for use in children across a range of ages and indications, with rare exceptions (e.g., hypersensitivity).

\subsection*{Abstracts}
\hypertarget{pmid_16891686}{W}e report successful use of methyl-prednisolone in a 21-month old child with ARDS that did not improve with conventional therapy. The child improved and could be extubated after 10 days of methylprednisolone therapy. Subsequently, the child was weaned off supplementary oxygen. Methylprednisolone appeared to be safe and effective in a child with unresolving ARDS. [\hyperlink{Methylprednisolone Acetate}{PMID: 16891686}, Lokesh Guglani et al., 2006]

\hypertarget{pmid_31903560}{T}o compare the efficacy and safety of prednisolone/prednisone and adrenocorticotropic hormone (ACTH) in the treatment of infantile spasms using a meta-analysis of randomized controlled trials (RCTs). In a systematic literature search of electronic databases (MEDLINE, Embase, the Cochrane Library), we identified RCTs that assessed prednisolone/prednisone compared with ACTH/tetracosactide in patients with infantile spasms. The electroclinical response and adverse events were evaluated. Six RCTs (616 participants) were included in the meta-analysis. Compared with prednisolone/prednisone, ACTH/tetracosactide was not superior in terms of cessation of spasms at day 14 (relative risk 1.19, 95\% confidence interval [CI] 0.74-1.92), day 42 (relative risk 1.02, 95\% CI 0.63-1.65), and resolution of hypsarrhythmia on electroencephalogram (relative risk 1.14, 95\% CI 0.71-1.81); the incidences of common adverse reactions caused by ACTH/tetracosactide were not lower than that of prednisolone/prednisone for irritability (relative risk 0.79, 95\% CI 0.57-1.10), increased appetite (relative risk 0.78, 95\% CI 0.57-1.08), weight gain (relative risk 0.86, 95\% CI 0.56-1.32), and gastrointestinal upset (relative risk 0.60, 95\% CI 0.35-1.02), though it seemed less frequent. Prednisolone/prednisone elicits a similar electroclinical response as ACTH for infantile spasms, which indicates that it can be an alternative to ACTH for treating infantile spasms. What this paper adds Prednisolone/prednisone is as effective as adrenocorticotropic hormone (ACTH) in electroclinical response of infantile spasms. Prednisolone/prednisone and ACTH cause similar and tolerable adverse effects, whose incidences are comparable. High-dose prednisone/prednisolone might be preferable to low dose for achieving freedom from spasms. [\hyperlink{Methylprednisolone Acetate}{PMID: 31903560}, Shaojun Li et al., 2020]

\hypertarget{pmid_32851257}{S}evere  A randomized, single-blind, parallel-controlled, multicenter clinical trial, methylprednisolone for children with severe  This is the first randomized clinical trial designed to evaluate the safety and efficacy of low- versus high-dose methylprednisolone for reducing long-term pulmonary outcomes in pediatric patients with severe MPP. The results of this study will provide scientific evidence to guide clinical practice for the treatment of severe MPP. Trial registration: This study is registered at ClinicalTrials.gov (NCT02303587). [\hyperlink{Methylprednisolone Acetate}{PMID: 32851257}, Baoping Xu et al., 2018]

\hypertarget{pmid_32292451}{T}he role of methyl prednisolone in longitudinal extensive transverse myelitis in children is not completely discovered in developing country like Pakistan. So this is the first study which aimed to evaluate the efficacy of methyl prednisolone in longitudinal extensive transverse myelitis in children. This is quasi experimental hospital based descriptive prospective study. The data was collected from 34 children admitted in Paediatric Neurology department through Outpatient/emergency department in Children's Hospital and the Institute of Child Health, Lahore for period of one year from January 2018 to December 2018. The children full filling the inclusion criteria were observed before and after giving injection methyl prednisolone 30mg/kg/dose (maximum dose one Gram irrespective of the body weight) once daily for five days in the form of intravenous infusion. Complete recovery was seen in 41.2\% while 58.8\% showed partial recovery. The correlation of response to treatment (recovery) with gender, area of spinal cord involvement, muscle power and autonomic dysfunction is found at significance level of five percent according to Chi square test. Early consideration and administration of methyl prednisolone in longitudinally extensive transverse myelitis in children can be beneficial and can help to reduce the morbidity. [\hyperlink{Methylprednisolone Acetate}{PMID: 32292451}, Muhammad Azeem Ashfaq et al., ]

\hypertarget{pmid_6308515}{W}e treated 116 children with ACTH or prednisone. Fifty-two had infantile spasms with hypsarhythmia, and 64 had other types of intractable seizures. ACTH completely controlled seizures in all patients with infantile spasms and hypsarhythmia and 74\% of those with other types of seizures. Prednisone controlled 51\% of patients with infantile spasms and none with other seizures. Serious side effects were minimal for both drugs, and recurrent seizures occurred in 40 to 50\% of patients within 4 to 14 months after completion of therapy. [\hyperlink{Methylprednisolone Acetate}{PMID: 6308515}, O C Snead et al., 1983]

\hypertarget{pmid_33103229}{T}o assess the feasibility, effectiveness, and safety of pulse methylprednisolone in comparison with intramuscular adrenocorticotropic hormone (ACTH) therapy in children with West syndrome (WS). This open-label, pilot study with a parallel-group assignment included 44 recently diagnosed children with WS. Methylprednisolone therapy was given as intravenous infusion at a dose of 30 mg/kg/d for five days followed by oral steroids 1 mg/kg gradually tapered over 5-6 wk. The efficacy outcomes included a cessation of epileptic spasms (as per caregiver reporting) and resolution of hypsarrhythmia on electroencephalogram; safety outcome was the frequency of various adverse effects. By day 14 of therapy, 6/18 (33.3\%) children in the methylprednisolone group and 10/26 (38.5\%) children in the ACTH group achieved cessation of epileptic spasms [group difference - 5.2\%; confidence interval (CI) -30.7 to 22.8; p = 0.73]. However, by six weeks of therapy, 4/18 (22.2\%) children in the methylprednisolone group and 11/26 (42.3\%) children in the ACTH group had cessation of epileptic spasms (group difference - 20.1\%; CI -43.0 to 8.4; p = 0.17). Hypertension was more commonly observed in the ACTH group (10 children) than in the methylprednisolone group (2 children; p = 0.046). Pulse methylprednisolone therapy was relatively safe. The study observed limited effectiveness of both ACTH and pulse methylprednisolone therapy, which may partially be due to preponderance of structural etiology and a long treatment lag. However, pulse methylprednisolone therapy appeared to be safe, tolerable, and feasible for management of WS. [\hyperlink{Methylprednisolone Acetate}{PMID: 33103229}, Madan Rajpurohit et al., 2021]

\hypertarget{pmid_22271697}{R}ecent studies have called into question the benefit of perioperative corticosteroids in children undergoing heart surgery, but have been limited by the lack of placebo control, limited power, and grouping of various steroid regimens together in analysis. We evaluated outcomes across methylprednisolone regimens versus no steroids in a large cohort of neonates. Clinical data from the Society of Thoracic Surgeons Database were linked to medication data from the Pediatric Health Information Systems Database for neonates (≤30 days) undergoing heart surgery (2004-2008) at 25 participating centers. Multivariable analysis adjusting for patient and center characteristics, surgical risk category, and within-center clustering was used to evaluate the association of methylprednisolone regimen with outcome. A total of 3180 neonates were included: 22\% received methylprednisolone on both the day before and day of surgery, 12\% on the day before surgery only, and 28\% on the day of surgery only; 38\% did not receive any perioperative steroids. In multivariable analysis, there was no significant mortality or length-of-stay benefit associated with any methylprednisolone regimen versus no steroids, and no difference in postoperative infection. In subgroup analysis by surgical-risk group, there was a significant association of methylprednisolone with infection consistent across all regimens (overall odds ratio 2.6, 95\% confidence interval 1.3-5.2) in the lower-surgical-risk group. This multicenter observational analysis did not find any benefit associated with methylprednisolone in neonates undergoing heart surgery and suggested increased infection in certain subgroups. These data reinforce the need for a large randomized trial in this population. [\hyperlink{Methylprednisolone Acetate}{PMID: 22271697}, Sara K Pasquali et al., 2012]

\hypertarget{pmid_3665340}{P}rednisolone and methylprednisolone pharmacokinetic parameters were evaluated in asthmatic children receiving concomitant anticonvulsant therapy. On separate study days, 15 children receiving either phenobarbital, carbamazepine, phenytoin, or combination anticonvulsant therapy were administered an intravenous dose of prednisolone or methylprednisolone and compared with a pediatric population not receiving anticonvulsant therapy. Plasma clearance of prednisolone in subjects receiving phenobarbital, carbamazepine, and phenytoin and in control subjects was 302.7 +/- 74.6, 383.2 +/- 53.8, 378.9 +/- 50.7, and 214.0 +/- 28.8 ml/min/1.73 m2 (mean +/- 1 SD), whereas plasma clearance of methylprednisolone was 1179.1 +/- 519.4, 1687.0 +/- 109.9, 2209.5 +/- 473.8, and 381.7 +/- 98.4 ml/min/1.73 m2, respectively. Bioavailability of prednisolone after the oral administration of prednisone and methylprednisolone ranged from 86\% to 104\% during anticonvulsant therapy. Three individuals reevaluated 13 to 20 days after discontinuing anticonvulsant therapy demonstrated pharmacokinetic parameters similar to those of the control group. Limited studies performed in patients receiving combination anticonvulsant therapy did not demonstrate an additive effect on prednisolone elimination. Differences in the degree of enhancement of prednisolone and methylprednisolone disposition in all three anticonvulsant study groups suggest that different metabolic pathways may be involved. [\hyperlink{Methylprednisolone Acetate}{PMID: 3665340}, M Bartoszek et al., 1987]

\hypertarget{pmid_11126262}{T}o evaluate the benefits and the medium-term side effects of methylprednisolone in very preterm infants at risk of chronic lung disease. Forty-five consecutive preterm infants (< 30 weeks' gestation) at risk of chronic lung disease were treated at a mean postnatal age of 16 days with a tapering course of methylprednisolone. The outcome of treatment was assessed by comparison with 45 consecutive historical cases of infants treated with dexamethasone; the infants did not differ in baseline characteristics. There were no differences between groups in the rate of survivors without chronic lung disease. Infants treated with methylprednisolone had a higher rate of body weight gain during the treatment period (median 120 g, range 0 to 190, vs. 70 g, range -110 to 210, P = 0.01) and between birth and the age of 40 weeks (median 1660 g, range 1170-2520, vs. 1580 g, range 1,040 to 2,120, P = 0.02). The incidence of both glucose intolerance requiring insulin (0 \% vs. 18 \%, P = 0.006) and cystic periventricular leukomalacia (2 \% vs. 18\%, P = 0.03) was lower among methylprednisolone-treated infants. Our observations confirm methylprednisolone to be as effective as dexamethasone and to have fewer side effects. A randomized control trial is needed to further study the efficacy and safety of methylprednisolone in very premature infants at risk of chronic lung disease. [\hyperlink{Methylprednisolone Acetate}{PMID: 11126262}, P André et al., 2000]

\hypertarget{pmid_26858095}{S}edation is increasingly used to facilitate procedures on children in emergency departments (EDs). This overview of systematic reviews (SRs) examines the safety and efficacy of sedative agents commonly used for procedural sedation in children in the ED or similar settings. We followed standard SR methods: comprehensive search; dual study selection, quality assessment, data extraction. We included SRs of children (1 month to 18 years) where the indication for sedation was procedure-related and performed in the ED. Fourteen SRs were included (210 primary studies). The most data were available for propofol (six reviews/50,472 sedations) followed by ketamine (7/8,238), nitrous oxide (5/8,220), and midazolam (4/4,978). Inconsistent conclusions for propofol were reported across six reviews. Half concluded that propofol was sufficiently safe; three reviews noted a higher occurrence of adverse events, particularly respiratory depression (upper estimate 1.1\%; 5.4\% for hypotension requiring intervention). Efficacy of propofol was considered in four reviews and found adequate in three. Five reviews found ketamine to be efficacious and seven reviews showed it to be safe. All five reviews of nitrous oxide concluded it is safe (0.1\% incidence of respiratory events); most found it effective in cooperative children. Four reviews of midazolam made varying recommendations. To be effective, midazolam should be combined with another agent that increases the risk of adverse events (upper estimate 9.1\% for desaturation, 0.1\% for hypotension requiring intervention). This comprehensive examination of an extensive body of literature shows consistent safety and efficacy for nitrous oxide and ketamine, with very rare significant adverse events for propofol. There was considerable heterogeneity in outcomes and reporting across studies and previous reviews. Standardized outcome sets and reporting should be encouraged to facilitate evidence-based recommendations for care. [\hyperlink{Methylprednisolone Acetate}{PMID: 26858095}, Lisa Hartling et al., 2016]

\hypertarget{pmid_24507698}{S}teroids have been used for the treatment of certain epilepsy types, such as infantile spasms; however, the use in the treatment of other intractable epilepsies has received limited study. We report our experience with intravenous methylprednisolone in children with epilepsy refractory to multiple antiepileptic drugs. A series of consecutive children were analyzed retrospectively. Patients with infantile spasms, progressive degenerative, or metabolic disorders were excluded. Seventeen children aged 2-14 (mean 5.3) years were included. Associated cognitive and motor deficits were recognized in 82\%. Most children (88\%) had daily seizures and 13 (76\%) were admitted previously with status epilepticus. The epilepsy was cryptogenic (unknown etiology) in 47\% and the seizures were mixed in 41\%. Intravenous methylprednisolone was given at 15 mg/kg per day followed by a weaning dose of oral prednisolone for 2-8 weeks (mean 3 weeks). Children were followed for 6-24 months (mean 18). Six (35\%) children became completely seizure free; however, three of them later developed recurrent seizures. At 6 months posttreatment, improved seizure control was noted in 10 (59\%) children. Children with mixed seizures were more likely to have a favorable response than those with one seizure type (49\% vs 31\%, P = 0.02). No major side effects were noted, and 35\% of the parents reported improvements in their child's alertness and appetite. Add-on steroid treatment for children with intractable epilepsy is safe and may be effective in some children when used in a short course. [\hyperlink{Methylprednisolone Acetate}{PMID: 24507698}, Kholoud H Almaabdi et al., 2014]

\hypertarget{pmid_16183305}{A}drenocorticotrophic hormone (ACTH) and prednisone are both used to treat infantile spasms (IS) in West syndrome. In many countries, ACTH is expensive and difficult to obtain whereas, prednisone or prednisolone are cheap, given orally and easily available. The purpose of this retrospective data analysis was to compare the efficacy and cost of ACTH and prednisolone in the treatment of IS from the perspective of a developing country. Patients admitted with West syndrome in Children's Hospital, Islamabad, between January 1995 and December 2001 were included in the analysis. The diagnosis was made after eliciting a history of characteristic seizures and detecting hypsarrhythmia on the EEG. Parents were offered the use of either ACTH administered by intramuscular injection or prednisolone given orally. ACTH was expensive and difficult to obtain whereas prednisolone was cheap and easily available. One hundred and five children were included in the study. Sixty-three were boys and their age ranged from 2 months to 3 years with a mean of 11 months. Thirty-three children received ACTH injections; 27 showed improvement and 11 remained spasms free after discontinuation of injections. Seventy-two patients were given oral prednisolone, 51 responded and 17 remained spasms free after oral steroids were stopped. Overall outcome was similar in both groups. The cost of ACTH injection was more than 100 times the cost of oral prednisolone. No significant difference was seen in the final outcome in both treatment groups. Since prednisolone is inexpensive, easily available and given orally, it is the preferred mode of therapy. [\hyperlink{Methylprednisolone Acetate}{PMID: 16183305}, Matloob Azam et al., 2005]

\hypertarget{pmid_15686192}{W}e reviewed the case histories of 28 children seen at children hospital from 31 December 1991 to 31 December 2001. These children has second and third degree oesophageal burns and they were treated by systemic Methylprednisolone (1000mg/1, 73/m2 SC). We divided the 26 children in four groups according to the time we began the steroids (before or beyond the 24th hours of the accident and according the number of steroids's bolus (less or more than 21 bolus). We analysed the number and the treatment of stricture in each group. High doses of methyl prednisolone seem to decrease the risk of oesophageal stricture. We found no difference between the children treated before the 24th hours and those treated after the 24 hours and those treated with less than 21 bolus and those with more than 21 bolus. [\hyperlink{Methylprednisolone Acetate}{PMID: 15686192}, Sonia Mazigh Mrad et al., 2004]

\hypertarget{pmid_19435579}{T}he ideal treatment of infantile spasms is unclear, but many studies advocate hormonal treatment. In the United States, intramuscular ACTH is most widely used, despite the problematic financial cost and side effect profile. Since September 2007, we have replaced ACTH with high-dose oral prednisolone (40-60 mg/day) according to the 2004 United Kingdom Infantile Spasms Study (UKISS). Ten of 15 (67\%) infants with new-onset and previously treated infantile spasms became spasm free within 2 weeks; 4 later recurred. More children with an idiopathic etiology for infantile spasms were spasm free than were symptomatic cases (88\% vs 43\%, P=0.10). Spasm freedom was equivalent to our most recent 15 infants receiving ACTH, with 13 (87\%) responding, P=0.16. Oral prednisolone had fewer adverse effects (53\% vs 80\%, P=0.10) and was less expensive (\$200 vs approximately \$70,000) than ACTH. We now routinely recommend oral prednisolone to all families of children with infantile spasms. [\hyperlink{Methylprednisolone Acetate}{PMID: 19435579}, Eric H Kossoff et al., 2009] 83 prednisolone-treated and 57 control children of a previous study demonstrating benefits of antenatal steroid prophylaxis of IRDS were reexamined at the age of 6 years. The prednisolone children were slightly higher and heavier, but no significant differences between steroid and control groups were noted in any other characteristics including IQ, psychosocial development, EEG, ophthalmology and audiology. THe results confirm the finding that antenatal steroid prophylaxis has no long-term hazards to the fetus. [\hyperlink{Methylprednisolone Acetate}{PMID: 19435579}, I Horváth et al., 1984]

\hypertarget{pmid_35087722}{W}est syndrome is an epileptic encephalopathy of infancy. According to guidelines, adrenocorticotrophic hormone (ACTH) is probably effective for the short-term management of infantile spasm, but there is little uniformity in treatment due to variable response. This study has been done to evaluate the efficacy of pulse methylprednisolone as compared to ACTH in children with West syndrome. Children between 3 months to 24 months with the diagnosis of West syndrome were included and ACTH and pulse methyl prednisolone followed by oral prednisolone were given after randomization. Total duration of treatment was 6 weeks in both groups. Total 87 children were enrolled; 12 patients lost in follow up. Finally, 43 received ACTH and 32 received pulse methylprednisolone. In pulse methylprednisolone group, 28.13\% showed 50-80\% response, 28.13\% showed 80-99\% response and 21.87\% patients showed 100\% response. In ACTH group, 41.86\% showed 50-80\% response, 25.58\% showed 80-99\% response and only 3 (6.97\%) patients showed 100\% response. Methylprednisolone treatment regimen did not cause significant or persistent adverse effects. Pulse methylprednisolone followed by oral prednisolone for 6 weeks is as effective as ACTH. Thus, methylprednisolone therapy can be an important alternative to ACTH. [\hyperlink{Methylprednisolone Acetate}{PMID: 35087722}, Kanij Fatema et al., 2021]

\hypertarget{pmid_18245869}{P}alatability is an important factor in medication compliance for children where the acceptability of a liquid medication and its ease of administration will be greatly affected by its taste. The objective of this study was to determine which, if any of two steroid preparations, oral dexamethasone or oral prednisolone, was more palatable to children requiring steroid treatment for asthma. A single-blind taste test of 2 different steroid suspensions, liquid prednisolone (1mg/ml) versus liquid dexamethasone (1mg/ml), was conducted in children aged 5-12 years, presenting to the pediatric emergency department with an exacerbation of asthma requiring steroid treatment. Children received 2.5mls of either prednisolone or dexamethasone and were asked to score their impression of taste on a 10 cm visual analog scale. After cleansing of the palate they were given the other steroid and scored its taste. Thirty-nine children (54\% male) were enrolled in the study. The mean age was 7.1 years (SD=2.0). The median visual analog scale measurement for dexamethasone was 8.2 cm (IQR= 5.2) whilst the median measurement for prednisolone was 5.0 cm (IQR= 7.3), p=0.03. Male children were more likely to prefer dexamethasone than females with a median score of 9.9 cm (IQR=3.8) for males vs. 5.9 cm (IQR=9.3) for females, p=0.005. There was no gender preference for prednisolone. There was a statistically significant difference between the taste of dexamethasone and prednisolone, with dexamethasone being the preferred steroid among pediatric patients with asthma. Males were much more likely to prefer dexamethasone than females. [\hyperlink{Methylprednisolone Acetate}{PMID: 18245869}, Heather Hames et al., 2008]

\hypertarget{pmid_27154159}{T}he optimal dose of methylprednisolone during pediatric open heart surgical procedures is unknown. This study compared the antiinflammatory and cardioprotective effects of high and lower doses of methylprednisolone in children undergoing cardiac operations. Thirty children, between 1 and 18 months old and undergoing total correction of tetralogy of Fallot, were randomized in double-blind fashion to receive either 5 or 30 mg/kg of intravenous methylprednisolone after anesthesia induction. Plasma concentrations of methylprednisolone, interleukin-6 (IL-6), IL-8, and IL-10, troponin T, and glucose were measured at anesthesia induction before administration of the study drug, at 30 minutes on cardiopulmonary bypass (CPB), just after weaning from CPB, and at 6 hours after CPB. Troponin T and blood glucose were also measured on the first postoperative morning. Significantly higher methylprednisolone concentrations were measured in patients receiving 30 mg/kg of methylprednisolone at 30 minutes on CBP, after weaning from CPB and at 6 hours after CPB (p < 0.001). No differences were detected in IL-6, IL-8, IL-10, or troponin concentrations at any time point. Blood glucose levels were significantly higher in patients receiving 30 mg/kg of methylprednisolone at 6 hours after CPB (p = 0.04) and on the first postoperative morning (p = 0.02). Based on the measured concentrations of interleukins or troponin T, a 30 mg/kg dose of methylprednisolone during pediatric open heart operations does not offer any additional antiinflammatory or cardioprotective benefit over a 5 mg/kg dose. Higher dose of methylprednisolone exposes patients more frequently to hyperglycemia. [\hyperlink{Methylprednisolone Acetate}{PMID: 27154159}, Juho Keski-Nisula et al., 2016]

\hypertarget{pmid_31397941}{A}lthough steroids are suggested as the treatment of choice for infantile spasms, the mechanism of action is still unclear. Using a rat model of malformation of cortical development with refractory infantile spasms, we evaluated the efficacy of methylprednisolone on spasms susceptibility and behaviors. Additionally, we investigated the in vivo electrophysiological and neurochemical changes of the brain after methylprednisolone treatment. Infant rats with prenatal exposure of methylazoxymethanol at gestational day 15 were used. After a single dose of methylprednisolone or three different doses of methylprednisolone for 3 days, spasms were triggered by intraperitoneal injection of N-methyl-d-aspartic acid. In rats with 3 days of methylprednisolone pretreatment and their controls, behavioral testing was performed at postnatal day 15. In vivo magnetic resonance imaging was conducted at postnatal day 15 after 3 days of methylprednisolone treatment. The rats with single methylprednisolone pretreatment showed significantly delayed onset of spasms and multiple doses of methylprednisolone significantly suppressed the development of spasms in a dose-dependent manner. After multiple methylprednisolone pretreatment and a cluster of N-methyl-d-aspartic acid-induced spasms, the rats showed significantly increased freezing behaviors to conditioned stimuli. Glutamate-weighted chemical exchange saturation transfer revealed significant elevation of glutamate concentration in the cortices of the rats with multiple methylprednisolone pretreatments. Methylprednisolone pretreatment could attenuate N-methyl-d-aspartic acid-induced spasms with in vivo neurochemical and electrophysiological changes, which indicates this steroid's action on the brain and in epilepsy. [\hyperlink{Methylprednisolone Acetate}{PMID: 31397941}, Minyoung Lee et al., 2019]

\hypertarget{pmid_25048310}{T}his study aimed to test the hypothesis that high-dose prednisolone (4 mg/kg/day) may be more efficacious than usual-dose (2 mg/kg/day) prednisolone for spasm resolution at 14-days in children with infantile spasms. This was a randomized, open-label-trial conducted at a tertiary-level-hospital from February-2012 to March-2013. Children aged 3-months to 2-years presenting with infantile spasms in clusters (at least 1 cluster/day) with hypsarrhythmia or its variants on EEG were enrolled. The study participants were randomized to receive either high-dose prednisolone (4 mg/kg/day) or the usual-dose (2 mg/kg/day) prednisolone. The primary outcome measure was the proportion of children who achieved spasm freedom for 48-h at day-14 after treatment initiation as per parental reports in both the groups. The adverse effects were also monitored. The study was registered with the clinicaltrials.gov (ClinicalTrials.gov Identifier: NCT01575639). Sixty-three children were randomized into the two groups with comparable baseline characteristics. The proportion of children with spasm cessation on day-14 was significantly higher in the high-dose group as compared to the usual-dose group (51.6\% vs. 25\%, p=0.03). The absolute risk reduction was 26.6\% (95\% confidence interval 11.5-41.7\%) with number needed to treat being 4. The adverse effects were comparable in both the groups. High-dose prednisolone (4 mg/kg/d) was more effective than low-dose prednisolone (2mg/kg/d) in achieving spasm cessation at 14-days (as per parental reports) in children with infantile spasms. [\hyperlink{Methylprednisolone Acetate}{PMID: 25048310}, Prabaharan Chellamuthu et al., 2014]

\hypertarget{pmid_24446954}{T}his study investigated the short-term response to a standardized hormonal therapy protocol for treatment of infantile spasms. Twenty-seven children with video electroencephalography (EEG)-confirmed infantile spasms received very high dose (8 mg/kg/day, max 60 mg/day) oral prednisolone for 2 weeks. Response (absence of both hypsarrhythmia and spasms) to prednisolone was ascertained by repeat overnight video-EEG. Responders were tapered over 2 weeks and nonresponders were immediately transitioned to high dose (150 IU/m(2)/day) intramuscular adrenocorticotropic hormone (ACTH) for two additional weeks. Response was again determined by overnight video-EEG after ACTH therapy. Sixty-three percent (17/27) of patients responded completely to prednisolone. Subsequently, 40\% (4/10) of prednisolone nonresponders exhibited a complete response after an additional 2-week course with ACTH. Among 27 subjects with median follow-up of 13.5 months (interquartile range [IQR] 4.8-25.9), 12\% (2/17) of prednisolone responders and 50\% (2/4) of ACTH responders experienced a relapse between 2 and 9 months after initial response. Very high dose prednisolone demonstrated significantly higher efficacy than previously reported for lower doses in prior studies. High dose ACTH may be superior to very high dose prednisolone, and in lieu of a definitive clinical trial, the choice between prednisolone and ACTH for initial treatment of infantile spasms remains controversial. [\hyperlink{Methylprednisolone Acetate}{PMID: 24446954}, Shaun A Hussain et al., 2014]

\hypertarget{pmid_21624649}{T}he place of systemic corticosteroids in the treatment of children with chronic rhinosinusitis (CRS) remains unclear. We sought to assess the effectiveness and tolerability of oral methylprednisolone as an anti-inflammatory adjunct in the treatment of CRS in children. Forty-eight children (age, 6-17 years) with clinically and radiologically proved CRS were included. Patients were randomly assigned to either oral amoxicillin/clavulanate (AMX/C) and methylprednisolone or AMX/C and placebo twice daily for 30 days. Oral methylprednisolone was administered for the first 15 days with a tapering schedule. Primary parameters were mean change in symptom and sinus computed tomographic (CT) scan scores after treatment. Secondary study parameters were mean changes in individual symptom scores after treatment, relapse rate, and tolerability. Forty-five patients completed the study: 22 received AMX/C and methylprednisolone, and 23 received AMX/C and placebo. Both groups demonstrated significant improvements in symptom and sinus CT scores when comparing baseline values with end-of-treatment values (P < .001). Methylprednisolone as an adjunct was significantly more effective than placebo in reducing CT scores (P = .004), total rhinosinusitis symptoms (P = .001), and individual symptoms of nasal obstruction (P = .001), postnasal discharge (P = .007), and cough (P = .009). At the end of treatment, 48\% of the children in the placebo group still had abnormal findings on CT scans versus 14\% in the methylprednisolone group (P = .013). Therapy-related adverse events were not different between groups. Although insignificant, the incidence of clinical relapses was also less in the methylprednisolone group (25\%) compared with that in the placebo group (43\%, P = .137). Oral methylprednisolone is well tolerated and provides added benefit to treatment with antibiotics for children with CRS. [\hyperlink{Methylprednisolone Acetate}{PMID: 21624649}, Fadıl Ozturk et al., 2011]

\hypertarget{pmid_9851278}{A}n open prospective study using i.v. methylprednisolone in children with juvenile chronic arthritis (JCA) who had had a systemic exacerbation of disease is described. Eighteen children aged from 3 to 14 yr and 9 months (mean 9.7 yr) were treated. Ten patients (55\%) had a loss of all systemic features 1 month after the pulse, and eight (45\%) had a reduction in the active joint count. At this time, five of the patients on oral prednisolone had achieved a reduction in dosage. Also at 1 month, a reduction in erythrocyte sedimentation rate was observed in 11 patients (61\%) and of C-reactive protein in 11 of 16 (72\%). Altogether, 13 patients (72\%) had a good response, while a further three (16\%) went into remission. Our conclusions are that pulse methylprednisolone provides good short-term benefit in patients with systemic-onset JCA; no serious side-effects were noted. Further long-term studies are warranted. [\hyperlink{Methylprednisolone Acetate}{PMID: 9851278}, A O Adebajo et al., 1998]

\hypertarget{pmid_19444032}{T}o present 3 children with hypersensitivity reaction to methylprednisolone sodium-succinate and review the literature regarding such reactions. Data on the clinical features were obtained from the children's files. Skin prick tests were performed with a panel of corticosteroid preparations. Three patients (5, 7, and 8 years) with asthma who were treated with intravenous methylprednisolone succinate at the emergency department developed hypersensitivity reactions initially consider to be due to their primary disease. Two had a positive skin prick test to methylprednisolone sodium succinate but not to other corticosteroids or to the succinate ester. Skin prick tests to different corticosteroids, performed 4 years after the event in the third patient, were negative. Methylprednisolone sodium-succinate may cause anaphylactic/anaphylactoid reactions in children. Our patients' histories emphasize the importance of awareness to corticosteroid-induced reactions, especially in children with asthma in whom the symptoms may be considered as an exacerbation of their primary illness. [\hyperlink{Methylprednisolone Acetate}{PMID: 19444032}, Amit Nahum et al., 2009]

\hypertarget{pmid_20142465}{T}he authors report their experience with intravenous methylprednisolone for the treatment of infantile spasms. A pulse dose of 20 mg/kg intravenous methylprednisolone on each of 3 successive days, followed by a 2-month oral prednisolone taper, led to the rapid remission (range, 2-6 days) of infantile spasms in 5 of 10 (50\%) infants. In the subgroup of infants treated within 1 month of onset, 5 of 6 (83\%) experienced remission within 6 days. The authors estimate the medication cost of intravenous methylprednisolone with prednisolone taper to be less than \$200. In comparison, the cost of a typical course of adrenocorticotropic hormone in the United States can exceed \$70,000. Initial treatment with intravenous methylprednisolone and/or oral corticosteroids is a reasonable cost-effective approach to infantile spasms. The lack of serious side effects, low cost, availability, ease of administration, and comparable efficacy suggests that intravenous methylprednisolone merits consideration for study in randomized prospective trials. [\hyperlink{Methylprednisolone Acetate}{PMID: 20142465}, John R Mytinger et al., 2010]

\section*{Dopamine Hydrochloride}
\subsection*{Result}
\subsubsection*{Answer}

Unknown

\subsubsection*{{Explanation}}
\hypertarget{Dopamine Hydrochloride}
A review of the available abstracts reveals several that discuss the use of dopamine hydrochloride in children, including neonates, infants, and older children. However, the majority of these abstracts focus on the pharmacologic effects, clinical use, and adverse effects, rather than specifically designed safety studies.

1. One abstract [\hyperlink{pmid_2682552}{PMID: 2682552}, V Bhatt-Mehta et al., 1989] provides a detailed review of dopamine hydrochloride's pharmacology and clinical use in neonates, infants, and children. It notes that dopamine is widely used in these populations for shock and cardiac failure, and describes dose-dependent effects and potential adverse reactions (e.g., tachycardia, arrhythmias, especially at high doses in neonates). However, this is a review and not a targeted safety study.

2. Another abstract [\hyperlink{pmid_875819}{PMID: 875819}, J S Wright et al., 1977] describes the use of dopamine hydrochloride in postoperative management of open-heart surgical patients, including children. It reports a very low incidence of undesirable responses and a high frequency of circulatory improvement, but again, this is not a targeted safety study and does not specify age ranges or systematically assess safety.

3. A further abstract [\hyperlink{pmid_2358894}{PMID: 2358894}, V Bhatt-Mehta et al., 1990] examines the physical compatibility and chemical stability of dopamine when co-administered with other drugs in newborn infants. While it confirms stability, it does not address clinical safety outcomes.

4. An abstract on docarpamine, a dopamine prodrug, [\hyperlink{pmid_10870004}{PMID: 10870004}, S Watarida et al., 2000] reports a targeted safety and efficacy study in 11 children (age not specified) after open heart surgery. The study found no significant adverse effects, arrhythmias, or ECG changes, and concluded that docarpamine is safe and effective in this context. However, this is not a study of dopamine hydrochloride itself.

5. An animal study [\hyperlink{pmid_10453184}{PMID: 10453184}, H Miyaguchi et al., 1999] investigates dopamine hydrochloride in infant rats, but this does not provide evidence for safety in human children.

In summary, while dopamine hydrochloride is widely used in pediatric populations and some abstracts report low incidence of adverse effects, there are no abstracts among those reviewed that describe a targeted, prospective safety study of dopamine hydrochloride in children (neonates, infants, or older children) that definitively affirms its safety. Therefore, based on the abstracts available, the safety of dopamine hydrochloride in children remains unknown according to the strict criteria provided.

\subsection*{Abstracts}
\hypertarget{pmid_2682552}{D}opamine hydrochloride is widely used to increase blood pressure, cardiac output, urine output, and peripheral perfusion in neonates, infants, and older children with shock and cardiac failure. Its pharmacologic effects are dose dependent, and at low, intermediate, and high dosages include dilation of renal, mesenteric, and cerebral vasculature; inotropic response in the myocardium; and increases in peripheral and renal vascular resistance, respectively. The inotropic response is diminished in neonates compared with older children and adults due to maturational differences in norepinephrine stores. The clearance of dopamine varies widely in the pediatric population, depending on age. Its elimination half-life is approximately 2 minutes in full-term neonates and older children, and may be as long as 4-5 minutes in preterm infants. Due to immaturity of the autonomic nervous system, the drug may produce some adverse respiratory responses at high dose in neonates, the most common being tachycardia and cardiac arrhythmias. Dobutamine resembles dopamine chemically and is an analog of isoproterenol. It is relatively cardioselective at dosages used in clinical practice, with its main action being on beta 1-adrenergic receptors. Unlike dopamine, it does not have any effect on specific dopaminergic receptors. Dobutamine is used to increase cardiac output in infants and children with circulatory failure. Its elimination half-life is about 2 minutes in adults and older children. No information is available about its pharmacokinetics in neonates and infants. Adverse effects such as an increase in heart rate usually occur at high dosages. [\hyperlink{Dopamine Hydrochloride}{PMID: 2682552}, V Bhatt-Mehta et al., 1989]

\hypertarget{pmid_17614751}{D}oxapram hydrochloride, a respiratory stimulant, has several undesirable side effects during high-dose administration, including second-degree atrioventricular (AV) block and QT prolongation. In Japan, this drug is contraindicated for newborn infants. Recent studies, however, have demonstrated the efficacy and safety of doxapram therapy for apnea of prematurity (AOP) using lower doses than those previously tested. As a result, approximately 60\% of Japanese neonatologists continue to use this drug. This study used surface ECG recordings to assess the cardiac safety of low-dose doxapram hydrochloride (0.2 mg/kg/h) in fifteen premature very-low-birth-weight infants with idiopathic AOP. Cardiac intervals and number of apnea episodes were compared before and after drug administration. Low-dose doxapram hydrochloride resulted in approximately 90\% reduction in the frequency of apnea without side effects. None of the infants developed QT or PR prolongation, arrhythmia, or other conduction disorders. In addition, there was no change in the slope of QT/RR before versus after administration of doxapram hydrochloride. We conclude that low-dose administration of doxapram hydrochloride did not have any undesirable effects on myocardial depolarization and repolarization. [\hyperlink{Dopamine Hydrochloride}{PMID: 17614751}, Masafumi Miyata et al., 2007]

\hypertarget{pmid_875819}{D}opamine hydrochloride (Intropin) has been used in the postoperative management of open-heart surgical patients, both children and adults. It has been associated with a very low incidence of undesirable responses and a high frequency of circulatory improvement associated with diuresis. Its administration requires careful control of infusion rate and measurement of response. In the dosages employed, it has been free of most of the undesirable side effects of other commonly used catecholamines. [\hyperlink{Dopamine Hydrochloride}{PMID: 875819}, J S Wright et al., 1977]

\hypertarget{pmid_2358894}{D}opamine hydrochloride is widely used to increase blood pressure, cardiac output, and peripheral perfusion in critically ill newborn infants and children with shock and congestive heart failure. These patients often require numerous other intravenous drugs such as dobutamine, tolazoline, and theophylline concurrently, but have limited venous access. As a result, two or more of these drugs may be administered through the same intravenous site. The objective of our study was to determine the physical compatibility and chemical stability of dopamine with dobutamine, tolazoline, and theophylline using simulated conditions encountered in the neonatal intensive care unit. Dopamine, dobutamine, tolazoline, and theophylline were studied at concentrations of 120 mg/100 mL, 120 mg/100 mL, 400 mg/100 mL, and 400 mg/500 mL, respectively, in 5\% dextrose injection. The flow rate of dopamine was 0.3 mL/h while all combination drugs were run at 1 mL/h. Aliquots were collected at hourly intervals for 5 hours. A simultaneous static experiment was also performed by mixing dopamine with each combination drug in a ratio of 1:3 and allowing these to stand at room temperature. Samples were obtained at 0.5-hour intervals for 3 hours. Each aliquot was inspected visually for any change in color and clarity and analyzed in triplicate for dopamine content using high performance liquid chromatographic technique with electrochemical detection. Linear regression analysis was performed on the mean values of dopamine concentrations to assess its degradation. Dopamine was found to be physically and chemically stable with dobutamine, tolazoline, and theophylline. Thus, dopamine can be infused concurrently with any of these drugs in 5\% dextrose injected at frequently used concentrations in newborn infants. [\hyperlink{Dopamine Hydrochloride}{PMID: 2358894}, V Bhatt-Mehta et al., 1990]

\hypertarget{pmid_28142338}{N}euroleptic drug molindone hydrochloride is a dopamine D2/D5 receptor antagonist and it is in late stage development for the treatment of impulsive aggression in children and adolescents who have attention deficient/hyperactivity disorder (ADHD). This new indication for this drug would expand the target population to include younger patients, and therefore, toxicity assessments in juvenile animals were undertaken in order to determine susceptibility differences, if any, between this age group and the adult rats. Adult rats were administered molindone by oral gavage for 13 weeks at dose levels of 0, 5, 20, or 60 mg/kg-bw/day. Juvenile rats were dosed for 8 weeks by oral gavage at dose levels of 0, 5, 25, 50, or 75 mg/kg-bw/day. Standard toxicological assessments were made using relevant study designs in consultation with FDA. Treatment-related elevation in serum cholesterol and triglycerides and decreases in glucose levels were observed in both the age groups. Organ weight changes included increases in liver, adrenal gland and seminal vesicles/prostate weights. Decreases in uterine weights were also observed in adult females exposed to the top two doses with sufficient exposure. In juveniles, sexual maturity parameters secondary to decreased body weights were observed, but, were reversed. In conclusion, the adverse effects noted in reproductive tissues of adults were attributed to hyperprolactinemia and these changes were not considered to be relevant to humans due to species differences in hormonal regulation of reproduction. On the whole, there were no remarkable differences in the toxicity profile of the drug between the two age groups. [\hyperlink{Dopamine Hydrochloride}{PMID: 28142338}, Gopala Krishna et al., 2017]

\hypertarget{pmid_942230}{K}etamine hydrochloride 2 mg/kg, together with atropine 0.2 mg, has been given intravenously on 100 occasions on a general paediatric ward. No serious side effects occurred. Dreams followed in 4 children but did not reduce acceptability of the drug. In our hands it has greatly reduced the pain and distress of children undergoing many routine medical procedures, particularly the dread which builds up when these have to be repeated in the same child. It has also produced close to ideal conditions for the operator, and probably increased his efficiency by reducing the emotional strain which occurs when doing painful things to a frightened patient. [\hyperlink{Dopamine Hydrochloride}{PMID: 942230}, E Elliott et al., 1976]

\hypertarget{pmid_26499007}{T}o evaluate the efficacy and safety of Drotaverine hydrochroride in children with recurrent abdominal pain. Double blind, randomized placebo-controlled trial. Pediatric Gastroenterology clinic of a teaching hospital. 132 children (age 4-12 y) with recurrent abdominal pain (Apley Criteria) randomized to receivedrotaverine (n=66) or placebo (n=66) orally. Children between 4-6 years of age received 10 mL syrup orally (20 mg drotaverine hydrochloride or placebo) thrice daily for 4 weeks while children >6 years of age received one tablet orally (40 mg drotaverine hydrochloride or placebo) thrice daily for 4 weeks. Primary: Number of episodes of pain during 4 weeks of use of drug/placebo and number of pain-free days. Secondary: Number of school days missed during the study period, parental satisfaction (on a Likert scale), and occurrence of solicited adverse effects. Reduction in number of episodes of abdominal pain [mean (SD) number of episodes 10.3 (14) vs 21.6 (32.4); P=0.01] and lesser school absence [mean (SD) number of school days missed 0.25 (0.85) vs 0.71 (1.59); P=0.05] was noticed in children receiving drotaverine in comparison to those who received placebo. The number of pain-free days, were comparable in two groups [17.4 (8.2) vs 15.6 (8.7); P=0.23]. Significant improvement in parental satisfaction score was noticed on Likert scale by estimation of mood, activity, alertness, comfort and fluid intake. Frequency of adverse events during follow-up period was comparable between children receiving drotaverine or placebo (46.9\% vs 46.7\%; P=0.98). Drotaverine hydrochloride is an effective and safe pharmaceutical agent in the management of recurrent abdominal pain in children. [\hyperlink{Dopamine Hydrochloride}{PMID: 26499007}, Manish Narang et al., 2015]

\hypertarget{pmid_17542008}{T}here is growing evidence to support the use of early central cholinergic enhancement to improve cognitive functioning in individuals with Down syndrome (DS). This report summarizes preliminary safety and cognitive efficacy data for seven children (8-13 years) with DS who participated in a 22-week, open-label trial of donepezil hydrochloride. Donepezil was dosed once daily at 2.5 mg and, based on tolerability, increased to 5 mg/day. Safety assessments were conducted at Week 1 (baseline), Week 8 (2.5 mg donepezil), Week 16 (5 mg) and Week 22 (after the donepezil had been discontinued). Measures of cognitive function were administered at each visit, encompassing the following domains: memory; attention; mood; and adaptive functioning. Donepezil was well tolerated at the 2.5 and 5 mg doses. The side effects were mild, transient, and consistent with the adverse events noted with cholinesterase inhibitors. Some children showed improvement on measures of memory (NEPSY Memory for Names and Narrative Memory) and sustained attention to tasks (Conners' Parent Rating Scales), although increased irritability and/or assertiveness were noted in some patients. Overall, this clinical report series adds to our initial findings of language gains in children with DS treated with donepezil. It also supports the need for larger, double-blind studies of the safety and efficacy of donepezil and other cholinesterase inhibitors for children with DS. [\hyperlink{Dopamine Hydrochloride}{PMID: 17542008}, Gail A Spiridigliozzi et al., 2007]

\hypertarget{pmid_10453184}{T}he purpose of this study was to evaluate whether or not dopamine (DA) can penetrate to the central nervous system (CNS) from the blood in the infantile period in rats. In a preliminary experiment, we administered a 50 mg/kg dose of DA hydrochloride, intraperitoneally, to 7-day-old rats (DA 50 mg/kg group), obtaining cerebrospinal fluid (CSF) both before and at 5, 10, 20, 30, 60 and 120 min after administration. The CSF levels of DA and its main metabolites, 3,4-dihydroxyphenylacetic acid (DOPAC) and homovanillic acid (HVA), were then measured. Next, we investigated the DA transfer from blood to the CNS by administering doses of 1, 5, 10 and 30 mg/kg DA hydrochloride (DA 1, 5, 10 and 30 mg/kg groups). In these groups, CSF samples were obtained only at 10 and/or 60 min after DA administration, based on the results of the DA 50 mg/kg group. The DA concentrations in CSF significantly increased compared with values before DA administration in the DA 50 mg/kg group. The DA concentrations in the DA 30 mg/kg group, DOPAC concentrations in the DA 5, 10 and 30 mg/kg groups, and HVA concentrations in all groups were significantly higher than in the control (saline injection) group. These findings suggest easy DA transfer from blood to the CNS and immaturity of the blood-brain barrier for DA in the infantile period in rats. [\hyperlink{Dopamine Hydrochloride}{PMID: 10453184}, H Miyaguchi et al., 1999]

\hypertarget{pmid_2522789}{T}he neuromuscular and cardiovascular effects of doxacurium chloride (BW A938U) were evaluated in 27 children (2-12 yr) anaesthetized with 1\% halothane and nitrous oxide in oxygen. In nine children the incremental technique was used to establish a cumulative dose-response curve by train-of-four stimulation. The remaining children received either 30 or 50 micrograms kg-1 of the drug as a single bolus. The median ED50 and ED95 of doxacurium in children were 19 and 32 micrograms kg-1, respectively. No clinically significant change in heart rate or arterial pressure occurred. Following doxacurium 30 micrograms kg-1 and 50 micrograms kg-1, recovery to 25\% of control occurred in 25 (SEM 6) and 44 (3) min, respectively. The recovery index (25-75\% of control) was 27 (2) min. The duration of action of doxacurium is similar to that of tubocurarine and dimethyl-tubocurarine in children. Compared with adults, children seem to require more doxacurium (microgram kg-1) to achieve a comparable degree of neuromuscular depression, and they recover more rapidly. [\hyperlink{Dopamine Hydrochloride}{PMID: 2522789}, N G Goudsouzian et al., 1989]

\hypertarget{pmid_16224736}{T}o conduct an updated review of the mechanisms of action, pharmacokinetics, clinical effectiveness and safety of atomoxetine in the treatment of the symptoms of ADHD. Atomoxetine is the first of the group of non-stimulant drugs to be approved by the US Food and Drug Administration to treat this disorder in children, adolescents and adults. Atomoxetine has a direct effect on noradrenalin and dopamine concentrations by exerting a strong and highly selective inhibiting action on the pre-synaptic noradrenalin transporter, with a minimum affinity for other transporters and receptors. After adjustment of the dosage for body weight, the pharmacokinetic parameters are similar across all age and gender groups. Maximal plasma concentration is reached one to two hours after oral administration. Data concerning the effectiveness and safety from the clinical trials and studies reported in the literature are discussed. Atomoxetine is an effective and well-tolerated drug when used for the pharmacological treatment of ADHD symptoms. Despite being a drug that has only recently been developed, evidence from the large number of comparative studies that have been carried out endorse its widespread use in the treatment of this syndrome. [\hyperlink{Dopamine Hydrochloride}{PMID: 16224736}, J D Velásquez-Tirado et al., ]

\hypertarget{pmid_11392342}{A} growing body of literature implicates interactions between glutamatergic and neostriatal dopaminergic neurotransmitter systems in the development and expression of impulsivity, hyperactivity, and stereotypy. Amantadine hydrochloride, a drug used in young children for influenza prophylaxis, acts both as an indirect dopamine agonist as well as an N-methyl-D-aspartate (NMDA) receptor antagonist. Thus an open clinical trial of this medication for the treatment of symptoms of impulse control disorders in children was performed. A total of eight children (seven with neurodevelopmental disorders and all inpatients) with target behaviors refractory to other treatments were selected after parental informed consent. All patients were male and ranged in age from 4 to 12 years. Outcome was based on subjective consensus clinical ratings by the multidisciplinary treatment team. For four of the children, amantadine was associated with marked clinical improvement. In the remainder, improvement was also observed. Amantadine was well tolerated. On the basis of this experience, it appears that amantadine hydrochloride or related NMDA antagonists may warrant additional study in this and related populations. [\hyperlink{Dopamine Hydrochloride}{PMID: 11392342}, B H King et al., 2001]

\hypertarget{pmid_14749146}{A}ttention-deficit/hyperactivity disorder (ADHD) occurs in approximately 3\% to 10\% of the pediatric population. Most of the drugs typically used to treat ADHD are stimulants, which, because of their addictive properties and potential for abuse, are controlled substances. Although these drugs are the mainstay of treatment for ADHD, nearly one third of patients may not respond to or be able to tolerate them. Atomoxetine hydrochloride, a nonstimulant approved by the US Food and Drug Administration for the treatment of ADHD, may provide an alternative to the use of stimulants. The goal of this review was to describe the chemistry, mechanism of action, pharmacokinetics, drug interactions, and efficacy and safety profiles of atomoxetine in pediatric and adult patients with ADHD, as well as relevant pharmacoeconomic considerations. Relevant publications were identified through a search of the English-language literature indexed on PreMEDLINE and MEDLINE (1966-May 2003) using the search terms atomoxetine, tomoxetine, and LY139603. These terms were also applied to the Google search engine. All articles were reviewed for suitability for inclusion. The manufacturer of atomoxetine provided both published and unpublished data. In the data reviewed, atomoxetine was more efficacious than placebo in patients with ADHD (P<0.05 to P<0.01). Therapeutic doses ranged from 45 mg or placebo (P<0.05). These results add support to the hypothesis that atomoxetine may not cause the increase in dopamine concentrations in the nucleus accumbens that is associated with pleasurable effects and abuse potential. [\hyperlink{Dopamine Hydrochloride}{PMID: 14749146}, Joshua Caballero et al., 2003]

\hypertarget{pmid_15951862}{D}iagnostic and therapeutic procedures in children are made easier using sedation. However, there is no consensus about which drug should be used to achieve this. Furthermore, none of the drugs used for sedation are risk free. The aim of this work is to study sedation indications, effectiveness, and safety at our center. A prospective observational study conducted at the Pediatric Day Care Unit, King Fahad National Guard Hospital, Riyadh, Saudi Arabia. The study covered 17.5 weeks in 2 periods: May 9th 1999 to June 13th 1999 and October 31st 2001 to February 11th 2002. Children <12 years were included. Collected data included demographics, indication, drug dosing and outcome. Data were reported as mean +/- SD. We included 148 patients, age 38 +/- 30 months. Adequate sedation was achieved in 79\% after initial chloral hydrate (CH) dose of 56.9 +/- 9.3 mg/kg, in 95\% after adding 18.5 +/- 6.4 mg/kg CH and in 96\% after adding second drug. Compared to nonrespondents, first CH dose respondents were younger and lower in weight. The CH side effects were few and mild. Chloral hydrate is a safe and effective agent for sedation in children with an age and weight dependent response. [\hyperlink{Dopamine Hydrochloride}{PMID: 15951862}, Omar M Hijazi et al., 2005]

\hypertarget{pmid_10870004}{D}ocarpamine is a dopamine prodrug which has been selected from a large number of dopamine derivatives in order to develop an orally effective dopamine. The pharmacokinetics after oral administration of docarpamine have not yet been studied in children undergoing open heart surgery. This study examined the effects of docarpamine on hemodynamics and evaluated its safety in 11 children undergoing open heart surgery for congenital heart disease. This study began when the patientOs postoperative condition was stabilized by continuous dopamine infusion into the vein at a rate of 5 micro g/kg/min. The patients were administered 40 mg/kg of docarpamine every 8 hours, and hemodynamics were measured every 4 hours for 16 hours after the initial docarpamine administration. Immediately after the initial docarpamine administration, the dose of dopamine was reduced to 3 micro g/kg/min. Infusion of dopamine was stopped 8 hours after the initial docarpamine administration. Systemic systolic and diastolic blood pressure and heart rate showed no significant changes. Mean right atrial pressure decreased 4 hours after docarpamine administration. Mixed venous oxygen saturation and mean velocity of circumferential fiber shortening increased significantly after docarpamine administration. No significant changes were observed in urine volume. All patients could be weaned from dopamine within 8 hours. No changes were observed in ECG, and no arrhythmia-inducing action was noted. Our study indicates that 40 mg/kg oral doses of docarpamine produce plasma dopamine concentration equivalent to those of a 3 to 5 micro g/kg/min dopamine infusion. Our data suggest that docarpamine is a safe and effective drug for children who have undergone open heart surgery. [\hyperlink{Dopamine Hydrochloride}{PMID: 10870004}, S Watarida et al., 2000]

\hypertarget{pmid_2402648}{C}hloral hydrate has been used extensively to sedate children, but at Brooke Army Medical Center, other drug combinations were becoming increasingly popular due to a perception that chloral hydrate had a high rate of failure, especially with younger or neurologically impaired children. Therefore, 50 children were given the drug before a diagnostic study, and patient data and a sedation score were recorded on a worksheet. Of 50 children, 43 (86\%) were "successfully sedated" on the first attempt with no side effects. Children with neurologic disorders had a much greater (27\% vs 4\%) failure rate than "normal" children. The sedation rate did not significantly differ by age, sex, or initial drug dosage. The study suggest that chloral hydrate is a safe and effective oral sedative but that children with neurologic disorders may need alternative drugs for sedation. [\hyperlink{Dopamine Hydrochloride}{PMID: 2402648}, P D Rumm et al., 1990]

\hypertarget{pmid_8703459}{T}o evaluate neuromuscular potency of doxacurium during balanced anesthesia in pediatric patients. Prospective, consecutive sample trial. Operating room at a university hospital. 15 infants (1 to 11 months), 20 children (3 to 10 years), and 20 adolescents (13 to 19 years). Anesthesia was induced and maintained with thiopental, alfentanil, and nitrous oxide in oxygen. No volatile drugs were used at any time during the study. The neuromuscular function was recorded as adductor pollicis electromyography evoked by a train-of-four stimulation at 20-second intervals. A cumulative log-dose probit-response curve of doxacurium was established for every patient. ED50 and ED95 doses of doxacurium (14 micrograms/kg and 25 micrograms/kg in infants, 26 micrograms/kg and 53 micrograms/kg in children, and 20 micrograms/kg and 41 micrograms/kg in adolescents, respectively) were smallest in infants and greatest in children (p < 0.05 between each pair of groups by analysis of variance and Scheffe's F-test). Potency of doxacurium was greatest in infants and least in children. We suggest that doxacurium can be administered safely in infants, and with dosages close to those reported in adults. Children's dose requirement was almost 50\% greater than that of infants. [\hyperlink{Dopamine Hydrochloride}{PMID: 8703459}, T R Taivainen et al., 1996]

\hypertarget{pmid_20040824}{T}o assess the long-term safety and tolerability of atomoxetine hydrochloride in children and adolescents with attention-deficit/hyperactivity disorder treated for > or = 3 years. Data from 13 double-blind, placebo-controlled trials and 3 open-label extension studies were pooled. Outcome measures were patient-reported treatment-emergent adverse events (AEs); discontinuations due to AEs, serious AEs, and changes in body weight, height, vital signs, electrocardiogram, and hepatic function tests. In total, 714 patients were treated with atomoxetine for > or = 3 years (mean follow-up 4.8 years [SD 1.1 years]), including a subset of 508 treated for > or = 4 years (mean follow-up 5.3 years [SD 0.8 years]). Most subjects were younger than 12 years at entry (73.8\%), male (78.4\%), and white (88.9\%). The mean final daily dose of atomoxetine was 1.35 mg/kg (SD 0.37 mg/kg). No new or unexpected AEs were observed compared with acute-phase treatment. Less than 6\% of patients exhibited aggressive/hostile behaviors, and less than 1.6\% reported suicidal ideation/behavior. No clinically significant effects were seen on growth rate, vital signs, or electrocardiographic parameters, and < or = 2\% of patients showed potentially clinically significant hepatic changes. Atomoxetine was safe and well tolerated for children and adolescents with > or = 3 and/or > or = 4 years of treatment. [\hyperlink{Dopamine Hydrochloride}{PMID: 20040824}, Craig Donnelly et al., 2009]

\hypertarget{pmid_8708264}{P}emoline, a dopamine agonist, is effective in children with attention deficit hyperactivity disorder (ADHD), but its efficacy in adults is unknown. The authors studied the efficacy and safety of pemoline, using retrospective chart review of treated students with ADHD over a 2-year period. Forty students met diagnostic and treatment criteria; pemoline was associated with much improved or very much improved Clinical Global Impression symptoms scores in 70\% of the students during a treatment period of 14 or more days. Severity of illness scores dropped from 4.11 to 3.01 between baseline and subsequent evaluation. Nine evaluable patients had adverse events, most commonly headaches, insomnia, and decreased appetite. Five additional students, who failed to meet the treatment-duration criterion, terminated because of severe initial insomnia. The authors concluded that pemoline is effective and safe in students with ADHD and has a lower abuse potential than methylphenidate and dextroamphetamine, the other two widely used, structurally dissimilar compounds, but controlled studies may be necessary before any final conclusions are reached. [\hyperlink{Dopamine Hydrochloride}{PMID: 8708264}, E Heiligenstein et al., 1996]

\hypertarget{pmid_22246409}{C}hloral hydrate (CH) is safe and effective for sedation of suitable children. The purpose of this study was to assess whether adequate sedation is achieved with reduced CH doses. We retrospectively recorded outpatient CH sedations over 1 year. We defined standard doses of CH as 50 mg/kg (infants) and 75 mg/kg (children >1 year). A reduced dose was defined as at least 20\% lower than the standard dose. In total, 653 children received CH sedation (age, 1 month-3 years 10 months), 42\% were given a reduced initial dose. Augmentation dose was required in 10.9\% of all children, and in a higher proportion of children >1 year (15.7\%) compared to infants (5.7\%; P < 0.001). Sedation was successful in 96.7\%, and more frequently successful in infants (98.3\%) than children >1 year (95.3\%; P = 0.03). A reduced initial dose had no negative effect on outcome (P = 0.19) or time to sedation. No significant complications were seen. We advocate sedation with reduced CH doses (40 mg/kg for infants; 60 mg/kg for children >1 year of age) for outpatient imaging procedures when the child is judged to be quiet or sleepy on arrival. [\hyperlink{Dopamine Hydrochloride}{PMID: 22246409}, Jennifer Bracken et al., 2012]

\hypertarget{pmid_20644039}{P}ropranolol hydrochloride has been prescribed for decades in the pediatric population for a variety of disorders, but its effectiveness in the treatment of infantile hemangiomas (IHs) was only recently discovered. Since then, the use of propranolol for IHs has exploded because it is viewed as a safer alternative to traditional therapy. We report the cases of 3 patients who developed symptomatic hypoglycemia during treatment with propranolol for their IHs and review the literature to identify other reports of propranolol-associated hypoglycemia in children to highlight this rare adverse effect. Although propranolol has a long history of safe and effective use in infants and children, understanding and recognition of deleterious adverse effects is critical for physicians and caregivers. This is especially important when new medical indications evolve as physicians who may not be as familiar with propranolol and its adverse effects begin to recommend it as therapy. [\hyperlink{Dopamine Hydrochloride}{PMID: 20644039}, Kristen E Holland et al., 2010]

\hypertarget{pmid_31230222}{T}he safety and efficacy of a novel combination treatment of AChE inhibitors and choline supplement was initiated and evaluated in children and adolescents with autism spectrum disorder (ASD). Safety and efficacy were evaluated on 60 children and adolescents with ASD during a 9-month randomized, double-blind, placebo-controlled trial comprising 12 weeks of treatment preceded by baseline evaluation, and followed by 6 months of washout, with subsequent follow-up evaluations. The primary exploratory measure was language, and secondary measures included core autism symptoms, sleep and behavior. Significant improvement was found in receptive language skills 6 months after the end of treatment as compared to placebo. The percentage of gastrointestinal disturbance reported as a side effect during treatment was higher in the treatment group as compared to placebo. The treatment effect was enhanced in the younger subgroup (younger than 10 years), occurred already at the end of the treatment phase, and was sustained at 6 months post treatment. No significant side effects were found in the younger subgroup. In the adolescent subgroup, no significant improvement was found, and irritability was reported statistically more often in the adolescent subgroup as compared to placebo. Combined treatment of donepezil hydrochloride with choline supplement demonstrates a sustainable effect on receptive language skills in children with ASD for 6 months after treatment, with a more significant effect in those under the age of 10 years. [\hyperlink{Dopamine Hydrochloride}{PMID: 31230222}, Lidia V Gabis et al., 2019]

\hypertarget{pmid_17803435}{T}he aim of this study was to evaluate the safety of olopatadine hydrochloride ophthalmic solution 0.2\% in children and adolescents 3-17 years of age. In this 6-week, randomized, double-masked safety evaluation, eligible subjects with asymptomatic eyes underwent in-office visits at weeks 1, 3, and 6 and were contacted by telephone at weeks 2, 4, and 5. Qualified subjects were assigned randomly in a 2:1 ratio of olopatadine 0.2\% to vehicle (identical formation without the active ingredient) for dosing on a once-daily schedule. Safety parameters assessed included adverse events, visual acuity, ocular signs (slit-lamp assessments), dilated fundus examinations, intraocular pressure (IOP), pulse, and blood pressure. An evaluation of 126 subjects (age range, 3-17) revealed no clinically relevant treatment-related changes in visual acuity, IOP, slit-lamp assessments, fundus examinations, or cardiovascular parameters. All adverse events reported were mild or moderate. Olopatadine 0.2\% administered once-daily for 6 weeks is safe and well tolerated in children and adolescent patients. [\hyperlink{Dopamine Hydrochloride}{PMID: 17803435}, Steven J Lichtenstein et al., 2007]

\hypertarget{pmid_404123}{A}lthough the FDA recommends imipramine hydrochloride (IMI) only for temporary relief of symptoms of enuresis nocturna (EN), the drug has been applied to a number of other pediatric situations, including the Hyperkinetic Syndrome (HS), childhood depression, somnambulism and pavor nocturnus, school phobia, petit mal epilepsy, allergies, autism, encorpresis and head-banging. We have reviewed the literature, with particular attention to the pharmacokinetics of IMI in children, and its putative mechanisms of action. The drug probably works through a number of different actions, and the futher delineation of these will be of considerable heuristic value. We review the toxic effects of IMI treatment and IMI poisoning in children, and the pediatric literature concerning other antidepressant drugs and lithium carbonate (Li). [\hyperlink{Dopamine Hydrochloride}{PMID: 404123}, C T Gualtieri et al., 1977]

\hypertarget{pmid_2295577}{F}luoxetine hydrochloride is the first selective serotonin uptake inhibitor introduced commercially in the United States. This report describes preliminary clinical experience with fluoxetine in 10 children and adolescents, aged 8 to 15 years, with primary obsessive compulsive disorder (OCD) or Tourette's syndrome (TS) plus OCD. In general, fluoxetine, which was administered from 4 to 20 weeks at a dosage of 10 or 40 mg per day, was well tolerated. Adverse effects included behavioral agitation/activation in four patients and mild gastrointestinal symptoms in two patients. No abnormalities were noted in the seven children who had follow-up EKGs. Five of the 10 patients (50\%) were considered responders; their obsessive-compulsive symptoms decreased substantially during treatment with fluoxetine. Responder rates were similar in the primary OCD (two of four, 50\%) and TS + OCD (three of six, 50\%) groups. In conclusion, short-term fluoxetine administration appears to be safe in children and adolescents. Placebo-controlled trials are needed to further assess the efficacy of fluoxetine. [\hyperlink{Dopamine Hydrochloride}{PMID: 2295577}, M A Riddle et al., 1990]

\section*{Erythromycin Lactobionate}
\subsection*{Result}
\subsubsection*{Answer}

Unknown (for all pediatric age ranges, including preterm neonates, infants, children, and adolescents)

\subsubsection*{{Explanation}}
\hypertarget{Erythromycin Lactobionate}
To determine if Erythromycin Lactobionate is safe for use in children, I reviewed all available abstracts for targeted studies on its safety in pediatric populations. The following summarizes the relevant evidence by age group:

1. **Preterm Neonates (≤15 days old, birth weight ≤1500g):**
   - A targeted study administered erythromycin lactobionate intravenously at two different dosages to 14 preterm neonates for 10 days. The study specifically monitored for adverse effects and found that "no adverse effects thought to be related to administration of erythromycin were observed." The authors concluded that erythromycin lactobionate was well-tolerated and had favorable pharmacokinetics in this population, but also noted these were preliminary findings and recommended further investigation [\hyperlink{pmid_8036045}{PMID: 8036045}, K B Waites et al., 1994].

2. **Critically Ill Children (age not specified, but not neonates):**
   - A randomized controlled trial evaluated the use of intravenous erythromycin lactobionate (10 mg/kg) in 74 critically ill children to facilitate feeding tube placement. The study did not report any adverse effects related to erythromycin lactobionate, but the primary focus was on efficacy rather than safety. The absence of reported adverse effects is noted, but the study was not primarily designed to assess safety [\hyperlink{pmid_11334066}{PMID: 11334066}, V Gharpure et al.].

3. **General Pediatric Use (including infants and children):**
   - Several studies and reviews discuss the safety of erythromycin in children, including oral and intravenous forms. One review notes that "no serious adverse effects have been reported in studies in which erythromycin has been used for its prokinetic effects, although fatal reactions have followed the intravenous administration of erythromycin to neonates in antibiotic doses" [\hyperlink{pmid_11328252}{PMID: 11328252}, J I Curry et al., 2001]. However, this is a general statement and not based on a targeted safety study of erythromycin lactobionate in children.
   - Another study in young adults (not children) found a high incidence of severe systemic side effects with intravenous erythromycin lactobionate, but this cannot be extrapolated to children [\hyperlink{pmid_6885176}{PMID: 6885176}, R Putzi et al.].

**Summary:**
- For preterm neonates, there is a targeted study indicating erythromycin lactobionate is well-tolerated with no observed adverse effects in a small sample, but the authors recommend further research.
- For other pediatric age groups, there is no targeted safety study of erythromycin lactobionate; available studies either do not focus on safety or do not specify the age range, or they use other erythromycin formulations.
- There is a general warning that fatal reactions have occurred with intravenous erythromycin in neonates at antibiotic doses, but this is not from a targeted safety study.

Therefore, the safety of erythromycin lactobionate is:
- Supported by preliminary evidence in preterm neonates (≤15 days old, ≤1500g), but not definitively established.
- Unknown in other pediatric age groups due to lack of targeted safety studies.

\subsection*{Abstracts}
\hypertarget{pmid_11328252}{E}rythromycin has been used as an antibiotic for more than four decades, but only in the last 10 years have other therapeutic benefits of this agent been exploited. Animal and human studies have demonstrated a prokinetic effect on the gastrointestinal tract at sub-antimicrobial doses (typically a quarter or less of the antibiotic dose). A limited number of studies have been performed in children to investigate this action. A review of this literature is particularly pertinent given the frequency of clinical problems related to gastrointestinal dysmotility in children and the limited availability of prokinetic agents in paediatric practice, compounded by the recent withdrawal of cisapride. The prokinetic effects of erythromycin have been investigated in infants with dysmotility associated with prematurity, in low birth-weight infants recovering from abdominal surgery, and in older children with a variety of other gastrointestinal disorders. Only one randomized placebo-controlled trial has been conducted. All except one of these studies have shown a beneficial effect of erythromycin in either promoting tolerance of enteral feeds or enhancing a measured index of gastrointestinal motility. Erythromycin appears to be equally effective when given orally (as ethylsuccinate or estolate) or intravenously (as lactobionate). Significantly, no serious adverse effects have been reported in studies in which erythromycin has been used for its prokinetic effects, although fatal reactions have followed the intravenous administration of erythromycin to neonates in antibiotic doses. [\hyperlink{Erythromycin Lactobionate}{PMID: 11328252}, J I Curry et al., 2001]

\hypertarget{pmid_6349401}{A} double-blind placebo-controlled trial of erythromycin ethylsuccinate was conducted in 65 infants and young children hospitalized with acute nonspecific gastroenteritis. Etiologic agents included rotaviruses (29\%), Campylobacter jejuni (17\%), "classical" enteropathogenic Escherichia coli (12\%), enterotoxigenic E. coli (11\%), Salmonella (9\%), Shigella (2\%), and Giardia lamblia (2\%). No pathogens were obtained from 25 (38\%) children. Treatment with erythromycin had no effect on the course of the illness in terms of the time required for hydration, stool frequency and temperature to return to normal, or for vomiting to be abolished. Children treated with erythromycin, however, experienced a marginally, but significantly (P less than 0.05), shorter period of abnormal stool consistency compared with control subjects. This effect was most pronounced in children from whom no enteropathogens were isolated. [\hyperlink{Erythromycin Lactobionate}{PMID: 6349401}, R M Robins-Browne et al., 1983]

\hypertarget{pmid_14502372}{T}he efficacy of erythromycin was assessed in the treatment of 14 children aged 4 to 13 years with refractory chronic constipation, and presenting megarectum and fecal impaction. A double-blind, placebo- controlled, crossover study was conducted at the Pediatric Gastroenterology Outpatient Clinic of the University Hospital. The patients were randomized to receive placebo for 4 weeks followed by erythromycin estolate, 20 mg kg-1 day-1, divided into four oral doses for another 4 weeks, or vice versa. Patient outcome was assessed according to a clinical score from 12 (most severe clinical condition) to 0 (complete recovery). At enrollment in the study and on the occasion of follow-up medical visits at two-week intervals, patient score and laxative requirements were recorded. During the first 30 days, the mean SD clinical score for the erythromycin group (N = 6) decreased from 8.2+/-2.3 to 2.2+/-1.0 while the score for the placebo group (N = 8) decreased from 7.8+/-2.1 to 2.9+/-2.8. During the second crossover phase, the score for patients on erythromycin ranged from 2.9+/-2.8 to 2.4+/-2.1 and the score for the patients on placebo worsened from 2.2+/-1.0 to 4.3+/-2.3. There was a significant improvement in score when patients were on erythromycin (P < 0.01). Mean laxative requirement was lower when patients ingested erythromycin (P < 0.05). No erythromycin-related side effects occurred. Erythromycin was useful in this group of severely constipated children. A larger trial is needed to fully ascertain the prokinetic efficacy of this drug as an adjunct in the treatment of severe constipation in children. [\hyperlink{Erythromycin Lactobionate}{PMID: 14502372}, M A Bellomo-Brandão et al., 2003]

\hypertarget{pmid_8036045}{E}rythromycin is receiving renewed attention as an alternative for treatment of neonatal infections caused by Ureaplasma urealyticum because of recently proved abilities of this organism to produce systemic disease in this population. Although erythromycin has been used clinically for almost 40 years, very little is known about its activity in the preterm neonate. Fourteen neonates, birth weights < or = 1500 g and < or = 15 days of age, from whom U. urealyticum was isolated from the lower respiratory tract were randomized to receive erythromycin lactobionate either 25 or 40 mg/kg/day in four divided doses at 6-hour intervals scheduled for a total of 10 days. Blood samples collected at multiple time points after initial and steady state doses were assayed for erythromycin by liquid chromatography. Minimal inhibitory concentrations (MICs) of erythromycin for the U. urealyticum isolates were determined. MICs ranged from 0.031 to 2 micrograms/ml; MIC90 = 2 micrograms/ml. Serum erythromycin concentrations met or exceeded most MICs, with peak values of 3.05 to 3.69 and 1.92 to 2.9 micrograms/ml for the 40- and 25-mg/kg/day dosage groups, respectively. Pharmacokinetic parameters were calculated after the initial dose and at steady state for both dosage groups and compared. No adverse effects thought to be related to administration of erythromycin were observed. These preliminary findings showed that erythromycin is well-tolerated, has favorable pharmacokinetic activity in the preterm neonate and should be further investigated for treatment of ureaplasmal infections. [\hyperlink{Erythromycin Lactobionate}{PMID: 8036045}, K B Waites et al., 1994]

\hypertarget{pmid_8504017}{R}esults of studies conducted to characterise local, systemic, reproductive, and mutagenic effects indicate that the new macrolide antimicrobial clarithromycin is well tolerated within reasonable multiples of the intended clinical dose. No adverse effects of clarithyromycin on male or female fertility, perinatal, or postnatal reproduction were indicated by data from rabbits, mice, rats and macaques. No evidence of mutagenic potential was revealed from various in vitro and in vivo study methodologies. Evidence of low potential for ototoxicity, oculotoxicity, hepatotoxicity and nephrotoxicity was provided in studies involving rats, dogs and primates. In agreement with studies with other macrolides, venous irritation potential for the intravenous lactobionate salt formulation was substantial in rabbit studies. In addition, the safety profile of this agent has been evaluated on the basis of adverse reactions and abnormal laboratory values seen in phase I, II and III international clinical trials conducted in adults. The most frequently reported adverse reactions occurring in 3768 patients receiving clarithromycin in phase II and III trials were nausea (3.8\%), diarrhoea (3.0\%), abdominal pain (1.9\%) and headache (1.7\%). Adverse reactions were not serious and were usually rapidly reversible. The incidence of adverse reactions did not vary with gender, race or age. Adverse reaction rates were comparable to or less than those of comparator beta-lactams and macrolides. Overall, clarithromycin appears to be a safe and well-tolerated macrolide antimicrobial agent. [\hyperlink{Erythromycin Lactobionate}{PMID: 8504017}, D R Guay et al., 1993]

\hypertarget{pmid_18789096}{C}hronic bullous disease of childhood is the commonest acquired blistering disorder of children. Erythromycin has been reported to be beneficial for this condition. A three question survey was e-mailed to all members of the British Society for Paediatric Dermatology to assess the incidence, preferred treatments and experience of oral erythromycin in treating chronic bullous disease of childhood. A second, more detailed questionnaire was sent to members who had used erythromycin. Forty patients were reported to have been treated over the previous 2 years. The preferred treatment was dapsone. Erythromycin alone had been used in five children as first-line oral treatment. In three of these patients the initial improvement was graded as either "good" or "complete resolution." This benefit was only sustained in one child, with the other two relapsing between 4 and 12 weeks. In a further eight children, erythromycin had been used with other oral agents. In five of these children, erythromycin was associated with long-term benefit. These results suggest that erythromycin is unlikely to produce sustained improvement in chronic bullous disease of childhood when used as a sole first-line agent. However, erythromycin can cause an initial improvement, which may be useful whilst awaiting results of diagnostic tests and may confer benefit when used with other systemic treatments. [\hyperlink{Erythromycin Lactobionate}{PMID: 18789096}, Paul Farrant et al., ]

\hypertarget{pmid_20687891}{T}he number of children suffering from atopic eczema has increased over the past 30 years especially in children between the ages of 2 and 5 years. These is a significant group of eczematous children that are resistant to standard therapy. Babies and children with eczema suffer pain, irritation and disfigurement from the dermatitis. In this study, we have followed 14 cases of pediatric patients (ages of 8 months to 64 months) with a history of resistant eczema for a period of at least six months. All of these children received 300 mg to 500 mg standardized Lactobacillus rhamnosus cell lysate daily as an immunobiotic supplement. The results of this open label non-randomized clinical observation showed a substantial improvement in quality of life, skin symptoms and day- and nighttime irritation scores in children with the supplementation of Lactobacillus rhamnosus lysate. There were no intolerance or adverse reactions observed in these children. Lactobacillus rhamnosus cell lysate may thus be used as a safe and effective immunobiotic for the treatment and prevention of childhood eczema and possible other types of atopy (allergic diseases). [\hyperlink{Erythromycin Lactobionate}{PMID: 20687891}, Ba X Hoang et al., 2010]

\hypertarget{pmid_36827282}{E}rythromycin is a macrolide antibiotic that is also prescribed off-label in premature neonates as a prokinetic agent. There is no oral formulation with dosage and/or excipients adapted for these high-risk patients. Clinical studies of erythromycin as a prokinetic agent were reviewed. Capsules of 20 milligrams of erythromycin were compounded with microcrystalline cellulose. Erythromycin capsules were analyzed using the chromatographic method described in the United States Pharmacopoeia which was found to be stability-indicating. The stability of 20 mg erythromycin capsules stored protected from light at room temperature was studied for one year. 20 mg erythromycin capsules have a beyond use date not lower than one year. 20 milligrams erythromycin capsules can be compounded in batches of 300 unities in hospital pharmacy with a beyond-use-date of one year at ambient temperature protected from light. [\hyperlink{Erythromycin Lactobionate}{PMID: 36827282}, Patrick Thevin et al., 2023]

\hypertarget{pmid_792407}{E}rythromycin continues to be a valuable and useful antimicrobial agent in children. Its low index of toxicity, freedom from sensitization, and reliable absorption and when administered orally contribute to make it an attractive agent in the treatment of a variety of minor respiratory and skin infections, especially in those situations where real or potential allergy to penicillin exists. Additional major uses are in the eradication of the carrier state in whooping cough and in diphtheria, especially in those instances when oral therapy can be tolerated. Dispite use over more than two decades, resistance developing in formerly susceptible organisms has not been a problem and thus seems unlikely to become so in the future. [\hyperlink{Erythromycin Lactobionate}{PMID: 792407}, C M Ginsburg et al., 1976]

\hypertarget{pmid_31321320}{A}zithromycin is widely used in children not only in the treatment of individual children with infectious diseases, but also as mass drug administration (MDA) within a community to eradicate or control specific tropical diseases. MDA has also been reported to have a beneficial effect on child mortality and morbidity. However, concerns have been raised about the safety of azithromycin, especially in young children. The aim of this review is to systematically identify the safety of azithromycin in children of all ages. MEDLINE, PubMed, Cochrane Central Register of Controlled Trials, Embase, CINAHL, International Pharmaceutical Abstracts and adverse drug reaction (ADR) monitoring systems will be systematically searched for randomised controlled trials (RCTs), cohort studies, case-control studies, cross-sectional studies, case series and case reports evaluating the safety of azithromycin in children. The Cochrane risk of bias tool, Newcastle-Ottawa and quality assessment tools, and The Joanna Briggs Institute Critical Appraisal tools will be used for quality assessment. Meta-analyses will be conducted to the incidence of ADRs from RCTs if appropriate. Subgroup analyses will be performed in different age and azithromycin dosage groups. Formal ethical approval is not required as no primary data are collected. This systematic review will be disseminated through a peer-reviewed publication. CRD42018112629. [\hyperlink{Erythromycin Lactobionate}{PMID: 31321320}, Peipei Xu et al., 2019]

\hypertarget{pmid_22348492}{E}rythromycin is generally used as a prokinetic agent for the treatment of feeding intolerance in preterm infants; however, results from previous studies significantly vary due to different medication dosages, routes of administration, and therapy durations. The effectiveness and safety of intermediate-dose oral erythromycin in very low birth weight (VLBW) infants with feeding intolerance was examined in this study. Between November 2007 and August 2009, 45 VLBW infants with feeding intolerance, who were all at least 14 days old, were randomly allocated to a treatment group and administered 5mg/kg oral erythromycin every 6hours for 14 days (n=19). Another set of randomly selected infants was allocated to the control group, which was not administered erythromycin (n=26). The number of days required to achieve full enteral feeding (36.5±7.4 vs. 54.7±23.3 days, respectively; p=0.01), the duration of parenteral nutrition (p<0.05), and the time required to achieve a body weight ≥2500g (p<0.05) were significantly shorter in the erythromycin group compared with the control group. The incidence of parenteral nutrition-associated cholestasis (PNAC) and necrotizing enterocolitis (NEC) ≥ stage II after 14 days of treatment were significantly lower (p<0.05) in the erythromycin group. No significant differences were observed in terms of the incidences of sepsis, bronchopulmonary dysplasia, or retinopathy of prematurity. No adverse effects were associated with erythromycin treatment. Intermediate-dose oral erythromycin is effective and safe for the treatment of feeding intolerance in VLBW infants. The incidences of PNAC and ≥ stage II NEC were significant lower in the erythromycin group. [\hyperlink{Erythromycin Lactobionate}{PMID: 22348492}, Yan-Yan Ng et al., 2012]

\hypertarget{pmid_8295812}{C}larithromycin is a new macrolide antibiotic that is active in vitro against a variety of organisms that are responsible for acute otitis media in children. The parent compound is metabolized to microbiologically active 14-hydroxy clarithromycin, which is especially active against Haemophilus influenzae. The safety and efficacy of clarithromycin and amoxicillin suspensions were compared in the treatment of acute otitis media in children 1 to 12 years of age inclusive. This was a Phase III, single blind (investigator-blind), randomized, multicenter clinical trial. Clarithromycin oral suspension was given in a dose of 7.5 mg/kg (maximum, 500 mg) twice daily, and amoxicillin suspension in a dose of 20 mg/kg (maximum, 750 mg) was given twice daily for 7 to 10 days in a 1:1 ratio. Clinical evaluations were performed pretreatment, within 48 hours posttreatment and 10 to 14 days posttreatment. Myringotomy was performed in every child to obtain a microbiologic sample pretreatment and at subsequent visits as clinically indicated. A total of 79 children were enrolled, 39 in the clarithromycin and 40 in the amoxicillin treatment group. Thirty-two children were excluded from the efficacy analysis for various reasons. Clinical success (cure and improvement) rates at 0 to 4 days posttreatment were 93\% for clarithromycin and 90\% for amoxicillin (P > 0.999). Altogether 17 children (10 receiving clarithromycin, 7 receiving amoxicillin) experienced some adverse event, with gastrointestinal disorders being the most common complaint. No clinically significant differences in laboratory tests were found between the groups.(ABSTRACT TRUNCATED AT 250 WORDS) [\hyperlink{Erythromycin Lactobionate}{PMID: 8295812}, J S Pukander et al., 1993] We evaluated 260 previously healthy children ages 3 through 12 years who had clinical signs and symptoms of pneumonia, radiographically confirmed. Patients were randomized 1:1 to a 10-day course of either clarithromycin suspension 15 mg/kg/day divided twice a day or erythromycin suspension 40 mg/kg/day divided twice a day or three times a day. Evidence of infection with Chlamydia pneumoniae was detected in 28\% (74) of patients: 13\% (34) by nasopharyngeal culture and 18\% (48) by serology with the microimmunofluorescence assay. Evidence of infection with Mycoplasma pneumoniae was detected in 27\% (69) of patients: 20\% (53) by nasopharyngeal culture or polymerase chain reaction and 17\% (44) by serology with the use of enzyme-linked immunosorbent assay. Serologic confirmation of infection was observed in 23\% (8) and 53\% (28) of patients with bacteriologically detected C. pneumoniae and M. pneumoniae, respectively. Treatment with clarithromycin vs. erythromycin, respectively, yielded the following outcomes: clinical success 98\% (121 of 124) vs. 95\% (105 of 110); radiologic success 98\% (109 of 111) vs. 94\% (92 of 110); and eradication by pathogen, C. pneumoniae 79\% (15 of 19) vs. 86\% (12 of 14) and M. pneumoniae 100\% (9 of 9) vs. 100\% (4 of 4). Adverse events were primarily gastrointestinal occurring in almost one-fourth of patients in both groups, and were mild to moderate in severity. Clarithromycin and erythromycin were similarly effective and safe for the treatment of radiographically proved, community-acquired pneumonia in children older than 2 years old.(ABSTRACT TRUNCATED AT 250 WORDS) [\hyperlink{Erythromycin Lactobionate}{PMID: 8295812}, S Block et al., 1995] Studies were conducted on in vivo pharmacokinetics of clarithromycin (TE-031, A-56268) in children and also on the efficacy and the safety of this macrolide antibiotic in the treatment of bacterial infections in children. The results obtained are summarized as follows: 1. TE-031 granules were orally administered to 5 children in a dosage of 5 mg/kg before meal. Maximum drug concentrations (range: 0.29-2.0 micrograms/ml) in the serum occurred during a period from 30 minutes to 1 hour after administration, but there were clear differences in blood concentrations among the individuals. 2. TE-031 granules were orally administered in a average dosage of 20 mg/kg/day to a total of 17 patients, consisting of 14 children with respiratory tract infections and 3 children with intestinal infections. The clinical efficacy evaluation resulted in 10 excellent cases, 6 good cases and 1 fair case, for an efficacy rate of 94.1\%. 3. Studies on the bacterial efficacy were carried out for 10 cases. The TE-031 bacteriological efficacy evaluation showed elimination in 7 cases, a decreased bacterial count in 2 cases, and no change in 1 case. The elimination rate was, thus, 70.0\%. Elimination rates according to different species of bacteria were 66.7\% (2 of 3 strains) for Staphylococcus aureus, 100\% for both Streptococcus pneumoniae (3 of 3) and Streptococcus pyogenes (1 of 1), and 42.9\% (3 of 7) for Haemophilus influenzae. 4. There were no symptoms which were attributable to side effects of the TE-031 therapy. The only laboratory test abnormality detected was eosinophilia in 1 patient. [\hyperlink{Erythromycin Lactobionate}{PMID: 8295812}, M Hayashi et al., 1989]

\hypertarget{pmid_6885176}{A} double-blind study was designed to test the hypothesis that local side-effects during i. v. administration of erythromycin lactobionate depend on the drug concentration and that they can therefore be minimized by dissolving erythromycin in a larger infusion volume. Forty healthy students were assigned in a randomized sequence to four 30 min infusions: 120 and 250 ml of erythromycin lactobionate (1 g in 0.9\% NaCl) and 120 and 250 ml of placebo (0.9\% NaCl). An unexpectedly high incidence (95\% and 80\% for the infusion volumes of 120 and 250 ml, respectively) of severe systemic side-effects was observed during the first 79 infusions. Because all of these systemic side-effects were associated with the infusion of erythromycin, the study was terminated at this point. Side-effects included abdominal cramps, nausea, vomiting, dizziness and profuse sweating. The postulated positive effect of lower erythromycin concentrations in the infusion on local side-effects (pain at the infusion site, erythema) was marginal (63\% vs. 45\%). Compared to the systemic side-effects, the problem of local tolerance is less important. In young adults, 30 min infusions of 1 g erythromycin lactobionate are associated with a high incidence of systemic side-effects which may be due to an age-dependent effect of the drug on smooth muscle. [\hyperlink{Erythromycin Lactobionate}{PMID: 6885176}, R Putzi et al., ]

\hypertarget{pmid_10724028}{C}hildren infected with Chlamydia pneumoniae sometimes experience lower respiratory tract infections such as pneumonia and bronchitis. Although numerous anti-microbial compounds have been reported to be active against the organism, most of them have not been in a clinical trial in infants and children with C. pneumoniae infection. Clarithromycin has been shown to express anti-chlamydial effects in vitro. In this study, we evaluated the clinical anti-C. pneumoniae properties of clarithromycin in children with mainly lower respiratory tract infection. We administered clarithromycin orally to 21 infants and children at a dose of 10-15 mg/kg/day divided into two or three doses for 4-21 days. Clinical symptoms, roentgenographic and laboratory abnormal findings improved. The overall clinical efficacy rate was 85.7\% (18 of 21 cases). Administration of clarithromycin was considered to be a suitable treatment for improving lower respiratory infections in infants and children caused by C. pneumoniae. [\hyperlink{Erythromycin Lactobionate}{PMID: 10724028}, K Numazaki et al., 2000]

\hypertarget{pmid_3534749}{E}rythromycin ethyl succinate is an antibiotic frequently administered in pediatrics. According to some authors, this drug sharply decreases the fecal count of enterobacteria. The fecal flora of 12 infants less than one year old, treated by erythromycin ethyl succinate for 7 to 10 days was studied by differential count. A variable effect was observed on enterobacteria: a 10(3) to 10(5) fold reduction in 9 cases with a final count superior or equal to 10(4) per gram of feces, with or without coming back to the initial count; in 3 cases no modification. MIC of enterobacteria and concentrations of erythromycin in feces were not predictives of flora variation. Anaerobic flora was weakly modified. No implantation of potentially-pathogenic bacteria or multi-resistant or highly erythromycin resistant enterobacteria occurred. Thus, erythromycin ethyl succinate is valuable in pediatrics as it does not disturb barrier effects. But its use for selective decontamination of gut must be discussed depending on pharmacologic form and posology administered. [\hyperlink{Erythromycin Lactobionate}{PMID: 3534749}, M J Butel et al., 1986]

\hypertarget{pmid_7049959}{F}ollowing a study in which the etiology of nearly 70\% of 142 cases of pneumonia in children could be determined using a combination of bacteriological and serological methods, the effect of erythromycin ethylsuccinate was compared with that of amoxicillin in a randomized study on 120 cases of pneumonia. We first examined the tracheal secretion microbiologically and determined other serological parameters and clinical data. The tracheal secretion was sterile in only 19\% of the cases. We were able to identify the etiology in 64\% of the cases using a combination of microbiological and serological methods. A discontinuation of therapy and acceptable side-effects were considerably more frequent with amoxicillin than with erythromycin ethylsuccinate (75 mg/kg body weight). The advantages of erythromycin, especially for the initial therapy of pneumonia, and the improvements in diagnosis resulting from the examination of the tracheal secretion will be discussed. [\hyperlink{Erythromycin Lactobionate}{PMID: 7049959}, H Ruhrmann et al., 1982]

\hypertarget{pmid_21269858}{A} randomized, double-blind, double-dummy, multicenter international study was conducted to assess the clinical and bacteriologic response, safety, and compliance of a single 60-mg/kg dose of azithromycin extended-release (ER) versus a 10-day regimen of amoxicillin/clavulanate 90/6.4 mg/kg per day in children with acute otitis media at high risk of persistent or recurrent middle ear infection. Children aged 3 to 48 months were enrolled and stratified into two age groups (≤ 24 months and >24 months). Pretreatment tympanocentesis was performed at all sites and was repeated during treatment at selected sites. The primary endpoint, clinical response at the test-of-cure visit in the bacteriologic eligible population, was achieved in 80.5\% of children in the azithromycin ER group and 84.5\% of children in the amoxicillin/clavulanate group (difference-3.9\%; 95\% confidence interval-10.4, 2.6). Bacteriologic eradication was 82.6\% in the azithromycin ER group and 92\% in the amoxicillin/clavulanate group (p=0.050). Children who received amoxicillin/clavulanate had significantly higher rates of dermatitis and diarrhea, a greater burden of adverse events, and a lower rate of compliance to study drug compared to those who received azithromycin ER. A single 60-mg/kg dose of azithromycin ER provides near equivalent effectiveness to a 10-day regimen of amoxicillin/clavulanate 90/6.4 mg/kg per day in the treatment of children with acute otitis media. [\hyperlink{Erythromycin Lactobionate}{PMID: 21269858}, Adriano Arguedas et al., 2011]

\hypertarget{pmid_6975059}{W}e studied the pharmacokinetics of erythromycin estolate and ethylsuccinate suspensions in infants under 4 months of age who were being treated for chlamydial infections or pertussis. We conducted our studies after the initial dose of 10 mg/kg and subsequently during steady-state treatment. The estolate preparation resulted in higher peak concentrations in sera, and its absorption and elimination half-lives were longer. Peak concentrations occurred 3 h after a dose with the estolate preparation and 1 h after a dose with the ethylsuccinate preparation. The area under the curve for the estolate preparation was about three times greater than that for the ethylsuccinate preparation. Based on these findings, we recommend that erythromycin estolate suspensions be given to young infants at 8- or 12-h intervals (30 mg/kg per day in three divided doses or 20 mg/kg per day in two divided doses) and that erythromycin ethylsuccinate is best given at 6-h intervals (40 mg/kg per day in four divided doses). [\hyperlink{Erythromycin Lactobionate}{PMID: 6975059}, P Patamasucon et al., 1981]

\hypertarget{pmid_26569091}{H}ematopoietic cell transplantation (HCT) has become a standard treatment for many adult and pediatric conditions. Emerging evidence suggests that perturbations in the microbiota diversity increase recipients' susceptibilities to gut-mediated conditions such as diarrhea, infection and acute GvHD. Probiotics preserve the microbiota and may minimize the risk of developing a gut-mediated condition; however, their safety has not been evaluated in the setting of HCT. We evaluated the safety and feasibility of the probiotic, Lactobacillus plantarum (LBP), in children and adolescents undergoing allogeneic HCT. Participants received once-daily supplementation with LBP beginning on day -8 or -7 and continued until day +14. Outcomes were compliance with daily administration and incidence of LBP bacteremia. Administration of LBP was feasible with 97\% (30/31, 95\% confidence interval (CI) 83-100\%) of children receiving at least 50\% of the probiotic dose (median 97\%; range 50-100\%). We did not observe any case of LBP bacteremia (0\% (0/30) with 95\% CI 0-12\%). There were not any unexpected adverse events related to LBP. Our study provides preliminary evidence that administration of LBP is safe and feasible in children and adolescents undergoing HCT. Future steps include the conduct of an approved randomized, controlled trial through Children's Oncology Group. [\hyperlink{Erythromycin Lactobionate}{PMID: 26569091}, E J Ladas et al., 2016]

\hypertarget{pmid_1778860}{A} randomized single-blind study of the effects of erythromycin and roxithromycin on chlamydial conjunctivitis was performed on a group of patients, comprising 28 newborns and 27 adults. Treatment used was either 200 mg of erythromycin ethylsuccinate or 50 mg of roxithromycin daily, divided into two doses for the neonatal group or for the adult group, 1000 mg of erythromycin stearate or 300 mg of roxithromycin daily divided into two doses. All patients were treated for ten days. Clinically nine of the neonates and 13 of the adults had unilateral conjunctivitis, whilst the remaining cases were bilateral. At follow-up one month after commencing therapy, all but one (erythromycin-treated) of the 28 neonates and three (two of whom were erythromycin-treated) of the 27 adults were cured. However, 16 (nine neonates and seven adults) were culture-positive for Chlamydia trachomatis in samples from eye and/or nasopharynx. The culture-positive group comprised ten cases (four neonates and six adults) who had been treated with erythromycin and six (five neonates and one adult) with roxithromycin. No major side effects of the therapy were seen. The study indicates that there was no difference in the clinical cure rate for the two drugs either in neonates or in adults. However, the isolation rate of chlamydiae in the adult group differed, with 12 (92\%) of the 13 roxithromycin-treated cases becoming culture-negative, whilst this was true for only eight (57\%) of the 14 erythromycin-treated cases (P less than 0.007). [\hyperlink{Erythromycin Lactobionate}{PMID: 1778860}, K Stenberg et al., 1991]

\hypertarget{pmid_11334066}{E}rythromycin enhances gastric emptying and has been suggested to facilitate nasoenteric feeding tube placement in adults. Our primary objective was to evaluate the effect of erythromycin on the transpyloric passage of feeding tubes in critically ill children, and second, to evaluate the effect of erythromycin on the distal migration of duodenal feeding tubes. Seventy-four children were randomly assigned to receive erythromycin lactobionate (10 mg/kg) IV or equal volume of saline placebo 60 minutes before passage of a flexible weighted tip feeding tube. Abdominal radiographs were obtained 4 hours later to assess tube placement. If the tube was proximal to the third part of the duodenum, two additional doses of erythromycin/placebo were administered 6 hours apart. Those receiving additional doses had repeat radiographs 14 to 18 hours after tube placement. The number of postpyloric feeding tubes was similar in the erythromycin and placebo treated groups 4 hours after tube insertion (23/37 vs 27/37, p = .5). Of those with prepyloric tubes at 4 hours, none in the erythromycin group and 3 in the placebo group had the tube migrate to the postpyloric position by 14 to 18 hours (p < .05). Of those with postpyloric tubes proximal to the third part of the duodenum at 4 hours, additional doses of erythromycin did not cause more tubes to advance further into the intestine than did placebo (p = .6). Erythromycin does not facilitate transpyloric passage of feeding tubes in critically ill children. The distal migration of duodenal tubes further into the small bowel is also not enhanced by erythromycin. [\hyperlink{Erythromycin Lactobionate}{PMID: 11334066}, V Gharpure et al., ]

\hypertarget{pmid_2526241}{A} new oral macrolide, clarithromycin (TE-031, A-56268), was evaluated for its safety, efficacy and pharmacokinetics in 33 children. TE-031 was effective in all cases of mycoplasmal pneumonia, pneumococcal pneumonia, streptococcal pharyngitis, pertussis and Campylobacter gastroenteritis. The pharmacokinetic availability of TE-031 granule and tablets was much better than the older macrolides; serum half-lives of TE-031 averaged 3.2 +/- 0.25 hours (for the granule preparation). No clinical adverse reaction was encountered, but cases of transient mild elevation of the serum GPT (2 cases) and eosinophilia (2 cases) were encountered. From these preliminary data, TE-031 seems to have a place in the treatment of pediatric infectious diseases. [\hyperlink{Erythromycin Lactobionate}{PMID: 2526241}, H Meguro et al., 1989]

\hypertarget{pmid_7359612}{E}rythromycin is considered one of the safest antibiotics in common use today. In its otolaryngologic use, the authors have found it effective in treating acute suppurative sinusitis and occasionally otitis media, when combined with sulfonamides. There are few complications of erythromycin administration. Probably the least generally acknowledged of these is ototoxicity. There have been three reports of six cases with ototoxic complications from erythromycin, primarily from administration of its intravenous form. The authors present a case study of an 18 year old girl in severe renal failure, who suffered a reversible sensorineural hearing loss from high doses of an oral erythromycin preparation. The clinical manifestations of this case are compared to those previously reported. [\hyperlink{Erythromycin Lactobionate}{PMID: 7359612}, P Thompson et al., 1980]

\section*{Fomepizole}
\subsection*{Result}
\subsubsection*{Answer}

Unknown

\subsubsection*{{Explanation}}
\hypertarget{Fomepizole}
To determine if Fomepizole is safe for use in children, I reviewed all available abstracts for evidence of targeted pediatric safety studies or case series. Here is a summary of the relevant findings, organized by age range:

Infants (<1 year):
- One case report describes a 5-month-old boy with severe ethylene glycol poisoning treated with Fomepizole (seven doses, no hemodialysis). The infant made a complete recovery with no change in renal function. The authors state, "Fomepizole seemed safe and effective in a case of severe ethylene glycol poisoning, without the need for hemodialysis" but also note it is "not yet approved for this indication in the child" [\hyperlink{pmid_15329167}{PMID: 15329167}, Thierry Detaille et al., 2004]. This is a single case report, not a targeted safety study.

- Another case report describes a 7-month-old female with acetaminophen toxicity treated with Fomepizole and NAC. She fully recovered, and the authors state, "Although randomized trials are lacking, this case suggests that fomepizole may safely provide additional benefit in pediatric patients at risk for severe acetaminophen toxicity" [\hyperlink{pmid_37681263}{PMID: 37681263}, Lesley Pepin et al., 2023]. Again, this is a single case, not a targeted safety study.

Toddlers and Young Children (1–6 years):
- A case report describes a 2-year-old child with ethylene glycol poisoning managed with Fomepizole, who fully recovered [\hyperlink{pmid_22605809}{PMID: 22605809}, Gayle Hann et al., 2012]. This is a single case, not a targeted safety study.

- Another case report describes a 3-year-old boy with methanol poisoning treated with Fomepizole, with an uneventful course and no side effects. The authors provide "an overview...of all cases of paediatric poisoning in which fomepizole was used" and conclude, "Fomepizole seems to be a safe and valid alternative to ethanol in cases of paediatric methanol poisoning" [\hyperlink{pmid_15578220}{PMID: 15578220}, Nathalie De Brabander et al., 2005]. However, this is still based on case reports, not a targeted safety study.

- A 6-year-old girl with ethylene glycol poisoning was treated with Fomepizole and developed transient nystagmus but recovered uneventfully. The authors state, "Fomepizole has been incompletely evaluated in the pediatric population, and the nature and occurrence of adverse events have not been described adequately. The use of fomepizole appeared safe in this patient although she developed transient nystagmus" [\hyperlink{pmid_11192468}{PMID: 11192468}, J G Benitez et al., 2000].

Older Children (7–18 years):
- No targeted safety studies or case series specifically in this age group were found in the abstracts. Some reviews and case series mention pediatric use in general, but without age-specific safety data.

General Pediatric Population:
- Several review articles and case series mention that Fomepizole is being used in children and appears to be safe and effective, but all note the lack of targeted pediatric safety studies:
    - "Little experience exists with this drug, however, in the pediatric population. We present a case..." [\hyperlink{pmid_11134452}{PMID: 11134452}, E W Boyer et al., 2001].
    - "Fomepizole is both efficacious and safe in the pediatric population, but is not recommended during pregnancy" [\hyperlink{pmid_27147840}{PMID: 27147840}, Bruno Mégarbane et al., 2010]. However, this is a review and does not cite a targeted pediatric safety study.
    - "Data on comparative efficacy of fomepizole versus ethanol and data on administration of fomepizole in children are limited" [\hyperlink{pmid_11899949}{PMID: 11899949}, Deon P Druteika et al., 2002].

Summary:
- There are multiple case reports and small case series describing the use of Fomepizole in infants, toddlers, and young children, with generally positive outcomes and few reported adverse effects. However, there are no targeted, prospective safety studies or large pediatric case series that definitively affirm the safety of Fomepizole in children of any age group.
- Therefore, based on the abstracts, the safety of Fomepizole in children remains unknown due to the lack of targeted pediatric safety studies.

\subsection*{Abstracts}
\hypertarget{pmid_11134452}{F}omepizole (4-methylpyrazole, Antizol) is being increasingly used in the treatment of ethylene glycol toxicity in adults. Little experience exists with this drug, however, in the pediatric population. We present a case of ethylene glycol poisoning in a child where use of fomepizole averted intravenous ethanol infusion and hemodialysis, limited the duration of intensive care monitoring, and decreased the overall cost of treatment. [\hyperlink{Fomepizole}{PMID: 11134452}, E W Boyer et al., 2001]

\hypertarget{pmid_11581485}{F}omepizole (4-methylpyrazole; Antizol) is used increasingly in the treatment of methanol toxicity in adults. Little experience exists with this drug in the pediatric population, however. We present a case of methanol poisoning in a child in whom the use of fomepizole averted intravenous ethanol infusion and the attendant side effects of this therapy. [\hyperlink{Fomepizole}{PMID: 11581485}, M J Brown et al., 2001]

\hypertarget{pmid_11192468}{F}omepizole is an alcohol dehydrogenase inhibitor used to treat ethylene glycol poisoning in adults, with only one report describing the use of fomepizole in the pediatric population. We report a case of nystagmus associated with fomepizole treatment of a 6-year-old female who ingested ethylene glycol 15 hours prior to admission. A previously healthy 6-year-old presented to the emergency department mottled, comatose, and with Kussmaul respirations. Initial arterial blood gases: pH 7.11, PO2 200, HCO3 2, base excess -29, and within 20 minutes her pH dropped to 7.03. The patient was responsive to pain only. Initially, crystalluria without fluorescence was observed in the emergency department; 2 hours after admission, the urine fluoresced under Wood's light. Laboratory data were significant for increased anion and osmolar gaps. She was fluid-resuscitated, NaHCO3, thiamine, and pyridoxine were administered, and she was admitted to the pediatric intensive care unit. Within 4 hours of admission, a loading dose of fomepizole (15 mg/kg) was infused due to the severity of the patient's clinical status. Hemodialysis was initiated but discontinued temporarily due to catheter thrombus formation. The initial (3-hour postadmission) ethylene glycol concentration was 13 mg/dL. She developed coarse vertical nystagmus within 2 hours of fomepizole infusion. The ethylene glycol concentration was 5 mg/dL 3 hours after hemodialysis which then was discontinued. No further fomepizole was administered and the child recovered uneventfully. There was no evidence of the more frequently cited adverse events, such as headache, nausea, and dizziness. Fomepizole has been incompletely evaluated in the pediatric population, and the nature and occurrence of adverse events have not been described adequately. The use of fomepizole appeared safe in this patient although she developed transient nystagmus. [\hyperlink{Fomepizole}{PMID: 11192468}, J G Benitez et al., 2000]

\hypertarget{pmid_22891985}{F}omepizole has been utilized with remarkable success for ethylene glycol and methanol poisonings in children and adults. However, very little information is available regarding the safe and effective use of fomepizole in pregnancy. The goal of this research was to utilize an animal model to investigate the kinetics of fomepizole in pregnancy. Male and pregnant Sprague-Dawley rats, which were obtained at 19 days gestation, were administered fomepizole 15 mg/kg intraperitoneally. The animals were anesthetized and blood, liver, kidney, and fetus samples were collected at 1-24 hours post administration. Tissue samples were homogenized, deproteinized and prepared by solid phase extraction. Fomepizole concentrations from tissue and serum samples were analyzed using high pressure liquid chromatography. Between males and pregnant females, tissue and serum fomepizole levels were similar. Fomepizole concentrations in whole fetal tissue were similar to those in the maternal liver and kidney tissue. Fetal fomepizole concentrations were fivefold higher than maternal serum concentrations. The zero order elimination rate of fomepizole from maternal serum was 7.6 mol/L/h, which was slightly slower than the elimination rate in male rats (12.9 mol/L/h). Elimination of fomepizole from the fetus followed a similar time course to that in the maternal tissues. Elevated concentrations of fomepizole were detected in the fetus following maternal administration. Although the levels of fomepizole in the fetal tissue would imply significant protection against fetal formation of toxic alcohol metabolites, further research is needed to explore the long-term effects of fomepizole on the fetus. [\hyperlink{Fomepizole}{PMID: 22891985}, Rebeca Gracia et al., 2012]

\hypertarget{pmid_15329167}{T}o report a case of a massive ingestion of ethylene glycol in an infant successfully treated by fomepizole without hemodialysis. Descriptive case report. Pediatric intensive care unit. A 5-mo-old boy who ingested 200 mL of an antifreeze solution. Antidotal therapy with a total of seven doses of fomepizole administered intravenously with an interval of 12 hrs (15 mg/kg as loading dose, then 10 mg/kg). Hemodialysis was not performed. Iterative determination of ethylene glycol concentration was obtained in blood and urine. Kinetics were calculated for ethylene glycol and fomepizole elimination. The infant made a complete recovery with no change in renal function. Although not yet approved for this indication in the child, fomepizole seemed safe and effective in a case of severe ethylene glycol poisoning, without the need for hemodialysis. [\hyperlink{Fomepizole}{PMID: 15329167}, Thierry Detaille et al., 2004]

\hypertarget{pmid_34697779}{F}omepizole is an anti-metabolite therapy that is used to diminish the toxicity from methanol or ethylene glycol. Although its elimination kinetics have been well described in healthy human subjects, the elimination in poisoned patients have only been described in a few isolated cases. This study was designed to relate the elimination of fomepizole in a series of poisoned patients to that in healthy humans. Plasma samples from 26 patients in the clinical trials of the use of fomepizole for methanol and ethylene glycol poisoning were analyzed for fomepizole concentrations. The elimination of fomepizole was assessed after individual doses, both during and without intermittent hemodialysis. In methanol- and ethylene glycol-poisoned patients, fomepizole had a volume of distribution of 0.66-0.68 L/kg. After repeated doses of fomepizole, the minimum trough concentration averaged 86-109 µmol/L, which is 10 times higher than the minimum therapeutic concentration. In healthy human subjects, fomepizole elimination follows Michaelis-Menten kinetics and has been calculated as zero-order elimination rates. Zero-order elimination rates averaged 13 and 17 μmol/L/h in methanol and ethylene glycol patients, respectively, compared to 6-19 μmol/L/h in healthy subjects. Elimination during intermittent hemodialysis followed first-order kinetics, with a half-life of 3 h. Plasma concentrations during the repeated dosing confirmed that the recommended dosing schedule, with and without intermittent hemodialysis, maintained therapeutic concentrations throughout the treatments. Fomepizole elimination in poisoned patients at therapeutic plasma concentrations appears be similar to that reported previously in healthy human subjects. [\hyperlink{Fomepizole}{PMID: 34697779}, Kenneth McMartin et al., 2022]

\hypertarget{pmid_10349109}{E}thylene glycol is a serious toxin that children frequently ingest. Diagnosis and treatment of this poisoning are challenging and frequently involve the use of novel therapies. In the past year, fomepizole (4-methylpyrazole) has been approved for use as an antidote in the treatment of ethylene glycol poisoning in adults, and the first article reporting the use of fomepizole in a pediatric ethylene glycol exposure was published. As a result, the therapy of ethylene glycol poisoning in children is likely to change from the traditional approach of ethanol administration coupled with hemodialysis to the administration of fomepizole with or without hemodialysis. [\hyperlink{Fomepizole}{PMID: 10349109}, J F Wiley et al., 1999]

\hypertarget{pmid_27147840}{E}thylene glycol (EG) and methanol are responsible for life-threatening poisonings. Fomepizole, a potent alcohol dehydrogenase (ADH) inhibitor, is an efficient and safe antidote that prevents or reduces toxic EG and methanol metabolism. Although no study has compared its efficacy with ethanol, fomepizole is recommended as a first-line antidote. Treatment should be started as soon as possible, based on history and initial findings including anion gap metabolic acidosis, while awaiting measurement of alcohol concentration. Administration is easy (15 mg/kg-loading dose, either intravenously or orally, independent of alcohol concentration, followed by intermittent 10 mg/kg-doses every 12 hours until alcohol concentrations are <30 mg/dL). There is no need to monitor fomepizole concentrations. Administered early, fomepizole prevents EG-related renal failure and methanol-related visual and neurological injuries. When administered prior to the onset of significant acidosis or organ injury, fomepizole may obviate the need for hemodialysis. When dialysis is indicated, 1 mg/kg/h-continuous infusion should be provided to compensate for its elimination. Side-effects are rarely serious and with a lower occurrence than ethanol. Fomepizole is contraindicated in case of allergy to pyrazoles. It is both efficacious and safe in the pediatric population, but is not recommended during pregnancy. In conclusion, fomepizole is an effective and safe first-line antidote for EG and methanol intoxications.  [\hyperlink{Fomepizole}{PMID: 27147840}, Bruno Mégarbane et al., 2010] Orphan Medical has developed fomepizole as a potential treatment for both ethylene glycol and methanol poisoning. The drug was launched as Antizol in January 1998 for the treatment of ethylene glycol poisoning [273949] after US marketing approval was grantedin December 1997 [271563]. It has also received US approval for methanol poisoning [393217] and UK approval for ethylene glycol poisoning [329495]. In 1999, Orphan Medical's partner, Cambridge Laboratories, intended to pursue European approval under the mutual recognition procedure [329495]. However, by September 2000, Cambridge Laboratories had discontinued their involvement with fomepizole and IDIS World Medicines had licensed the rights to distribute the drug in the UK [412142]. In February 2000, the Canadian Therapeutic Products Programme (TPP) granted fomepizole Priority Review, provided that an NDA was submitted by March 14, 2000 [354665]. In August 2000, the TPP accepted this NDA and set a target date for approval in the fourth quarter of 2000 [379474]. The TPP granted fomepizole a Notice of Compliance permitting the sale of fomepizole in Canada in December 2000. The company's marketing partner in Canada, Paladin Labs had launched fomepizole by January 2001 [396953]. In June 2000, Tucker Anthony Cleary Gull stated that the Orphan Drug status which Orphan Medical had obtained for fomepizole would provide marketing exclusivity through December 2004. The analysts also stated that fomepizole had accounted for 40\% of Orphan Medical's revenue in financial year 1999, although +/- 30\% of sales were estimated to be due to stockpiling [409606]. [\hyperlink{Fomepizole}{PMID: 27147840}, P Hantson et al., 2001]

\hypertarget{pmid_18344099}{F}omepizole is available intravenously (i.v.) for the treatment of methanol and ethylene glycol poisoning. Few studies demonstrate that fomepizole achieves effective serum concentrations after i.v. or oral (p.o.) use. The objective was to describe the comparative pharmacokinetics of fomepizole after a single p.o. and i.v. dose. This was a prospective, randomized, crossover trial in 10 healthy volunteers. Each received 15 mg/kg fomepizole, p.o. and by 30 minute i.v. infusion. Serum was collected at 0, 0.25, 0.5, 1, 2, 4, 7, 12, 24, 36, and 48 hours (h) and stored at -70 degrees C. Candidate models were fit to the i.v. and p.o. data, simultaneously, using iterative 2-stage analysis weighted by the estimated inverse observation variance. Time above the MEC (T>MEC) was determined by numeric integration of the fitted functions using 10 micromoles/L as the minimum effective concentration (MEC). Seven females and 3 males were enrolled. Sole complaints included headache and dizziness in 3 subjects and 10/10 reported an unpleasant taste. The final PK model was 2-compartment with 0-order i.v. and 1(st)-order p.o. input (following a fitted TLag) and Michaelis-Menten elimination. p.o. fomepizole was rapidly absorbed with a bioavailability of approximately 100\%. The Km was 0.935+/-0.98 micromoles/L and the Vmax was 18.57+/-9.58 micromoles/L/h. T>MEC was 32 h with agreement between p.o. and i.v. dosing. This is the first study that effectively determines a human Vmax and Km for p.o. and i.v. fomepizole. p.o. and i.v. administration of fomepizole result in similar pharmacokinetic parameters. [\hyperlink{Fomepizole}{PMID: 18344099}, Jeanna Marraffa et al., 2008]

\hypertarget{pmid_15578220}{M}ethanol poisoning is not frequently observed in children; however, without treatment, serious intoxication can be complicated by visual impairment, coma, metabolic acidosis, respiratory and circulatory insufficiency and death. Treatment in a paediatric intensive care is therefore compulsory. Methanol is metabolised in the liver by alcohol dehydrogenase to the toxic metabolites formaldehyde and formic acid. Classically, ethanol is given as a competitive inhibitor in order to avoid the formation of these compounds. We report on the use of fomepizole (4-methylpyrazole),a new and potent inhibitor of alcohol dehydrogenase, in a 3-year-old boy after the intake of a toxic amount of methanol. The course was uneventful and the use of fomepizole was not accompanied by any side-effects. An overview is given of all cases of paediatric poisoning in which fomepizole was used. Fomepizole seems to be a safe and valid alternative to ethanol in cases of paediatric methanol poisoning. [\hyperlink{Fomepizole}{PMID: 15578220}, Nathalie De Brabander et al., 2005]

\hypertarget{pmid_1494233}{C}efprozil (CFPZ, BMY-28100) was evaluated for its efficacy, safety and pharmacokinetics in children. CFPZ was effective against streptococcal pharyngitis, pneumococcal lower respiratory tract infections, staphylococcal skin infections and Escherichia coli urinary tract infections, but was less effective against lower respiratory tract infections and otitis media due to Haemophilus influenzae. No adverse reactions were encountered in 46 cases treated with CFPZ. With a premeal administration of 7.5 mg/kg, the Cmax was approximately 3.2 micrograms/ml and the T 1/2 beta was 1.4 hours. From the present study, CFPZ appears to be safe and effective against community-acquired childhood infections. [\hyperlink{Fomepizole}{PMID: 1494233}, H Meguro et al., 1992]

\hypertarget{pmid_10485727}{F}omepizole is an effective alternative to ethanol in the treatment of ethylene glycol poisoning. In a series of 38 acute poisonings without renal failure, fomepizole obviated the need for haemodialysis. [\hyperlink{Fomepizole}{PMID: 10485727}, S W Borron et al., 1999]

\hypertarget{pmid_25251104}{T}o evaluate the efficacy and safety of piperacillin/tazobactam (PIPC/TAZ) or cefepime (CFPM) monotherapy for febrile neutropenia (FN) in children, a total of 53 patients with 213 febrile episodes were randomly treated with either PIPC/TAZ 337.5 mg/kg/day, or CFPM 100 mg/kg/day. No significant differences were observed in the success rates of the PIPC/TAZ and CFPM treatments (62.1\% vs. 59.1\%, P = 0.650). Furthermore, no differences were noted in the rates of new infection and mortality, and no serious adverse effects occurred in either of groups. Both PIPC/TAZ and CFPM were effective and safe as an empirical therapy for FN in children. Pediatr Blood Cancer 2015;62:356-358. © 2014 Wiley Periodicals, Inc. [\hyperlink{Fomepizole}{PMID: 25251104}, Hirozumi Sano et al., 2015]

\hypertarget{pmid_37681263}{A}cetaminophen overdose is common in the pediatric population. N-acetylcysteine (NAC) is effective at preventing liver injury in most patients when started shortly after the overdose. Delays to therapy increase risk of hepatotoxicity and liver failure that may necessitate organ transplant. Animal studies have demonstrated fomepizole may provide added benefit in acetaminophen overdose because of its ability to block the metabolic pathway that produces the toxic acetaminophen metabolite and downstream inhibition of oxidative stress pathways that lead to cell death. Several adult case reports describe use of fomepizole in patients at higher risk for poor outcomes despite NAC. We describe a case of a 7-month-old female who presented in acute liver failure with persistently elevated acetaminophen concentration secondary to repeated supratherapeutic doses of acetaminophen to manage fever. Fomepizole and NAC antidotes were used in the management of the patient. She fully recovered despite demonstrating multiple markers of poor outcome on initial presentation. Although randomized trials are lacking, this case suggests that fomepizole may safely provide additional benefit in pediatric patients at risk for severe acetaminophen toxicity. [\hyperlink{Fomepizole}{PMID: 37681263}, Lesley Pepin et al., 2023]

\hypertarget{pmid_18818954}{A} randomized, open, coordinated multi-center trial compared the bacteriological and clinical efficacy and safety of orally administered ceftibuten and trimethoprim-sulfamethoxazole (TMP-SMX) in children with febrile urinary tract infection (UTI). Children aged 1 month to 12 years presenting with presumptive first-time febrile UTI were eligible for enrollment. A 2:1 assignment to treatment with ceftibuten 9 mg/kg once daily (n = 368) or TMP-SMX (3 mg + 15 mg)/kg twice daily (n = 179) for 10 days was performed. Escherichia coli was recovered in 96\% of the cases. Among the E. coli isolates, 14\% were resistant to TMP-SMX but none to ceftibuten. In the modified intention-to-treat population, the bacteriological elimination rates at follow-up did not differ significantly between patients treated with ceftibuten and those treated with TMP-SMX [91 vs. 95\%, with a 95\% confidence interval (CI) for difference of -9.7 to 1.0]. However, the clinical cure rate was significantly higher among those treated with ceftibuten (93 vs. 83\%, with a 95\% CI for difference of 2.4 to 17.0). Adverse events were similar for both regimens and consisted mainly of gastrointestinal disturbances. In conclusion, ceftibuten is a safe and effective drug for the empirical treatment of febrile UTI in young children. [\hyperlink{Fomepizole}{PMID: 18818954}, Staffan Mårild et al., 2009]

\hypertarget{pmid_25482738}{D}uring an outbreak of mass methanol poisonings in the Czech Republic in 2012-2013, fomepizole was applied as an alternative antidote to ethanol. We present the laboratory data, clinical features, adverse reactions, and treatment outcomes in all patients treated with fomepizole. Combined retrospective and prospective case series study in 25 patients, median age 50 (16-73) years, 18 males and 7 females. There were 24\% fatalities, 36\% survivors without health impairment, and 40\% survivors with sequelae. All the patients who died were comatose on admission; the mortality was 50\% among patients in a coma. The median intensive care unit length of stay was six (2-22) days. The median total dose of fomepizole was 2 (1-9) g. Complications were observed in 7/25 cases: aspiration pneumonia (4), sepsis (2), bleeding (2), malignant arrhythmia (1), delirium tremens (1), and rebound of acidosis (1). The patients who survived without impairment were less acidotic than those who died or survived with sequelae (P<0.01). No difference in serum methanol and formate was found between the three groups. There is no evidence whether fomepizole is a more efficient antidote than ethanol with regards to the hospital mortality. The possibility of delirium tremens in the patients with a history of chronic alcohol abuse has to be taken in consideration. The benefits of fomepizole were indirect: no need to monitor serum ethanol's level during the hemodialysis in severely poisoned patients and less working overload on ICU doctors treating several poisoned patients simultaneously. [\hyperlink{Fomepizole}{PMID: 25482738}, Sergey Zakharov et al., 2014]

\hypertarget{pmid_10497633}{F}omepizole (4-methylpyrazole, 4-MP, Antizol) is a potent inhibitor of alcohol dehydrogenase that was approved recently by the US Food and Drug Administration (FDA) for the treatment of ethylene glycol poisoning. Although ethanol is the traditional antidote for ethylene glycol poisoning, it has not been studied prospectively. Furthermore, the FDA has not approved the use of ethanol for this purpose. Case reports and a prospective case series indicate that the intravenous (i.v.) administration of fomepizole every 12 hours prevents renal damage and metabolic abnormalities associated with the conversion of ethylene glycol to toxic metabolites. Currently, there are insufficient data to define the relative role of fomepizole and ethanol in the treatment of ethylene glycol poisoning. Fomepizole has clear advantages over ethanol in terms of validated efficacy, predictable pharmacokinetics, ease of administration, and lack of adverse effects, whereas ethanol has clear advantages over fomepizole in terms of long-term clinical experience and acquisition cost. The overall comparative cost of medical treatment using each antidote requires further study. [\hyperlink{Fomepizole}{PMID: 10497633}, D G Barceloux et al., 1999]

\hypertarget{pmid_30360666}{C}hronic idiopathic nausea (CIN) and functional dyspepsia (FD) cause considerable strain on many children's lives and their families. Areas covered: This study aims to systematically assess the evidence on efficacy and safety of pharmacological treatments for CIN or FD in children. CENTRAL, EMBASE, and Medline were searched for Randomized Controlled Trials (RCTs) investigating pharmacological treatments of CIN and FD in children (4-18 years). Cochrane risk of bias tool was used to assess methodological quality of the included articles. Expert commentary: Three RCTs (256 children with FD, 2-16 years) were included. No studies were found for CIN. All studies showed considerable risk of bias, therefore results should be interpreted with caution. Compared with baseline, successful relief of dyspeptic symptoms was found for omeprazole (53.8\%), famotidine (44.4\%), ranitidine (43.2\%) and cimetidine (21.6\%) (p = 0.024). Compared with placebo, famotidine showed benefit in global symptom improvement (OR 11.0; 95\% CI 1.6-75.5; p = 0.02). Compared with baseline, mosapride versus pantoprazole reduced global symptoms (p = 0.011; p = 0.009). One study reported no occurrence of adverse events. This systematic review found no evidence to support the use of pharmacological drugs to treat CIN or FD in children. More high-quality clinical trials are needed. AP-FGID: Abdominal Pain Related Functional Gastrointestinal Disorders; BART: Biofeedback-Assisted Relaxation Training; CIN: Chronic Idiopathic Nausea; COS: Core Outcomes Sets; EPS: Epigastric Pain Syndrome; ESPGHAN: European Society for Pediatric Gastroenterology Hepatology and Nutrition; FAP: Functional Abdominal Pain; FD: Functional Dyspepsia; GERD: Gastroesophageal Reflux Disease; GES: Gastric Electrical Stimulation; H [\hyperlink{Fomepizole}{PMID: 30360666}, Pamela D Browne et al., 2018] To assess the efficacy and safety of fomepizole, a competitive alcohol dehydrogenase inhibitor, in methanol poisoning and to test the hypothesis that fomepizole obviates the need for hemodialysis in selected patients. Retrospective clinical study in three intensive care units in university-affiliated teaching hospitals. All methanol-poisoned patients admitted to these ICUs and treated with fomepizole from 1987-1999 (n=14). The median plasma methanol concentration was 50 mg/dl (range 4-146), anion gap 22.1 mmol/l (11.8-42.2), arterial pH 7.34 (7.11-7.51), and bicarbonate 17.5 mmol/l (3.0-25.0). Patients received oral or intravenous fomepizole until blood methanol was undetectable. The median cumulative dose was 1250 mg (500-6000); the median number of twice daily doses was 2 (1-16). Four patients underwent hemodialysis for visual impairment present on admission. Four patients with plasma methanol concentrations of 50 mg/dl or higher and treated without hemodialysis recovered fully. Patients without pretreatment visual disturbances recovered, with no sequelae in any case. There were no deaths. Fomepizole was safe and well tolerated, even in the case of prolonged treatment. Analysis of methanol toxicokinetics in five patients demonstrated that fomepizole was effective in blocking methanol's toxic metabolism. Fomepizole appears safe and effective in the treatment of methanol-poisoned patients. If our results are confirmed in prospective analyses, hemodialysis may prove unnecessary in patients presenting without visual impairment or severe acidosis. [\hyperlink{Fomepizole}{PMID: 30360666}, B Mégarbane et al., 2001]

\hypertarget{pmid_22605809}{T}his case report describes the presentation and management of a 2-year-old child who ingested a potentially fatal amount of ethylene glycol (EG). There are few published cases worldwide of EG poisoning in children managed with fomepizole. All cases described in the literature were managed in a paediatric intensive care unit. In this case, the child presented irritable, pale and confused with high anion gap metabolic acidosis. As there were no paediatric intensive care beds available in the region, the child was successfully managed in a high dependency area in our district general hospital. The child fully recovered and was discharged home in 7 days. The authors believe that multi-disciplinary team management and the use of fomepizole contributed to the positive outcome and this case raised many useful learning points. [\hyperlink{Fomepizole}{PMID: 22605809}, Gayle Hann et al., 2012]

\hypertarget{pmid_27057183}{F}ebrile seizure is the most common neurologic problem in children between 3 months to 5 years old. Two to five percent of children aged less than five yr old will experience it at least one time. This type of seizure is age dependent and its recurrence rate is about 33\% overalls and 50\% in children less than one yr old. The prophylactic treatment is still controversial, so we conducted a randomized controlled clinical trial to find out the effectiveness of continuous phenobarbital versus intermittent diazepam for febrile seizure. This clinical trial was conducted in the Department of Pediatric Neurology, Babol University of Medical Sciences, Babol, Iran between March 2008 and October 2010. All children from 6 month to 5 yr old referred to Amirkola Children's Hospital, Babol, Iran were enrolled in the study. Children with febrile seizure that had indication for prophylaxis but did not receive any prophylaxis previously were enrolled in the study. For prophylactic anti convulsion therapy, patients were divided randomly in two groups. One group received continuous phenobarbital and another treated with intermittent diazepam whenever the children experienced an episode of febrile illness for up to one year after their last convulsion. Of all 145 studied cases, the recurrent rate in children under prophylaxis with diazepam was 11/71 and in phenobarbital group was 17/74. There was no significant difference in the recurrence rate in both groups. There was no significant difference in the effectiveness of phenobarbital and diazepam in prevention of recurrent in febrile seizure and we think that in respect of lower complication rate in diazepam administration, it cloud be better choice than phenobarbital. [\hyperlink{Fomepizole}{PMID: 27057183}, Mohammadreza Salehiomran et al., 2016]

\hypertarget{pmid_34585641}{F}omepizole is the preferred antidote for treatment of methanol and ethylene glycol poisoning, acting by inhibiting the formation of the toxic metabolites. Although very effective, the price is high and the availability is limited. Its availability is further challenged in situations with mass poisonings. Therefore, a 50\% reduced maintenance dose for fomepizole during continuous renal replacement therapy (CRRT) was suggested in 2016, based on pharmacokinetic data only. Our aim was to study whether this new dosing for fomepizole during CRRT gave plasma concentrations above the required 10 µmol/L. Secondly, we wanted to study the elimination kinetics of fomepizole during CRRT, which has never been studied before. Prospective observational study of adult patients treated with fomepizole and CRRT. We collected samples from arterial line (pre-filter) = plasma concentration, post-filter and dialysate for fomepizole measurements. Fomepizole was measured using high-pressure liquid chromatography with a reverse phase column. Ten patients were included in the study. Seven were treated with continuous veno-venous hemodialysis (CVVHD) and three with continuous veno-venous hemodiafiltration (CVVHDF). Ninety-eight percent of the plasma samples were above the minimum plasma concentration of 10 µmol/L. Fomepizole was removed during CRRT with a median saturation/sieving coefficient of 0.85 and dialysis clearance of 28 mL/min. Fomepizole was eliminated during CCRT. The new dosing recommendations for fomepizole and CRRT appeared safe, by maintaining the plasma concentration above the minimum value of 10 µmol/L. Based on these data, the fomepizole maintenance dose during CRRT could be reduced to half as compared to intermittent hemodialysis. [\hyperlink{Fomepizole}{PMID: 34585641}, Yvonne E Lao et al., 2022]

\hypertarget{pmid_25449223}{T}o systematically review literature assessing efficacy and safety of pharmacologic treatments in children with abdominal pain-related functional gastrointestinal disorders (AP-FGIDs). MEDLINE and Cochrane Database were searched for systematic reviews and randomized controlled trials investigating efficacy and safety of pharmacologic agents in children aged 4-18 years with AP-FGIDs. Quality of evidence was assessed using Grades of Recommendation, Assessment, Development and Evaluation approach. We included 6 studies with 275 children (aged 4.5-18 years) evaluating antispasmodic, antidepressant, antireflux, antihistaminic, and laxative agents. Overall quality of evidence was very low. Compared with placebo, some evidence was found for peppermint oil in improving symptoms (OR 3.3 (95\% CI 0.9-12.0) and for cyproheptadine in reducing pain frequency (relative risk [RR] 2.43, 95\% CI 1.17-5.04) and pain intensity (RR 3.03, 95\% CI 1.29-7.11). Compared with placebo, amitriptyline showed 15\% improvement in overall quality of life score (P = .007) and famotidine only provides benefit in global symptom improvement (OR 11.0; 95\% CI 1.6-75.5; P = .02). Polyethylene glycol with tegaserod significantly decreased pain intensity compared with polyethylene glycol only (RR 3.60, 95\% CI 1.54-8.40). No serious adverse effects were reported. No studies were found concerning antidiarrheal agents, antibiotics, pain medication, anti-emetics, or antimigraine agents. Because of the lack of high-quality, placebo-controlled trials of pharmacologic treatment for pediatric AP-FGIDs, there is no evidence to support routine use of any pharmacologic therapy. Peppermint oil, cyproheptadine, and famotidine might be potential interventions, but well-designed randomized controlled trials are needed. [\hyperlink{Fomepizole}{PMID: 25449223}, Judith J Korterink et al., 2015]

\hypertarget{pmid_11899949}{T}o systematically review English-language articles on fomepizole administration in patients with ethylene glycol poisoning. MEDLINE, EMBASE, Current Contents, and PubMed. Search terms were fomepizole, 4-methylpyrazole, and ethylene glycol. The search was supplemented with a bibliographic review of all relevant articles. All published reports of fomepizole administration in patients with ethylene glycol poisoning were reviewed, irrespective of study design. We identified one clinical trial and subsequent pharmacokinetic study, one case series, and 13 case reports. Fomepizole has been investigated in 70 patients in open, unblinded studies. Most patients received an intravenous loading dose, with subsequent variable maintenance doses every 12 hours until plasma ethylene glycol levels became undetectable. Additional hemodialysis treatment generally was administered when patients had renal insufficiency or ethylene glycol levels above 50 mg/dl. Many patients had detectable ethanol levels either because of coadministration or as a result of adjunctive treatment at a referring center. Poorer patient outcomes, such as death and renal insufficiency, were associated with later clinical presentation time after ingestion. At therapeutic fomepizole levels (> 8.6 mg/ml), the half-life of ethylene glycol was prolonged to over 19 hours. Fomepizole appeared to be well tolerated by most patients. Fomepizole is an effective alcohol dehydrogenase inhibitor that decreases production of ethylene glycol metabolites. Reduced mortality and morbidity are undetermined because of the small number of patients evaluated to date. Data on comparative efficacy of fomepizole versus ethanol and data on administration of fomepizole in children are limited. [\hyperlink{Fomepizole}{PMID: 11899949}, Deon P Druteika et al., 2002]

\section*{Gadoteridol}
\subsection*{Result}
\subsubsection*{Answer}

Yes (10–18 years)
Unknown (<10 years)

\subsubsection*{{Explanation}}
\hypertarget{Gadoteridol}
Based on the available abstracts, several targeted studies have evaluated the safety of gadoteridol in children for use as a contrast agent in MRI, specifically for central nervous system (CNS) imaging.

1. Children (General Pediatric Population, including specific age ranges):
   - A phase III open-label clinical trial included 103 children (age range not specified, but described as "children") with suspected neurologic disease. Only minor, self-resolving adverse events were reported, and the study concluded "excellent safety and efficacy for use of gadoteridol in children with suspected intracranial or spinal disease" [\hyperlink{pmid_8430186}{PMID: 8430186}, W S Ball et al., 1993].
   - Another study enrolled 13 children aged 10 to 18 years, with no minor or major reactions to gadoteridol injection observed. The study concluded that gadoteridol injection is a "safe and excellent contrast agent for use in MRI" [\hyperlink{pmid_8496990}{PMID: 8496990}, S E Byrd et al., 1993].
   - A phase IIIB open-label multicenter clinical trial included 22 pediatric patients (age not specified, but described as "children") with CNS neoplasms. No clinically relevant changes in vital signs or laboratory values were attributed to gadoteridol, and no systemic complaints were reported. The study concluded that intravenous administration of gadoteridol was "found to be safe in children" [\hyperlink{pmid_1501959}{PMID: 1501959}, J F Debatin et al., 1992].
   - Another multicenter clinical trial involved 101 pediatric patients (age not specified, but described as "children") and found gadoteridol suitable for enhanced MRI detection, localization, and characterization of CNS pathology in children [\hyperlink{pmid_1506153}{PMID: 1506153}, W S Ball et al., 1992].
   - A study on high-dose gadoteridol (up to 0.3 mmol/kg) included 67 patients (age not specified, but context suggests inclusion of children) and reported only minor, self-resolving adverse effects, concluding that gadoteridol can be safely administered up to this dose [\hyperlink{pmid_8073972}{PMID: 8073972}, W T Yuh et al., 1994].

2. Preclinical and animal studies:
   - Preclinical safety studies in animals (mice, rats, dogs, rabbits) found no serious effects or teratogenicity at doses much higher than those used clinically, and in vitro studies showed no hemolytic potential [\hyperlink{pmid_1506157}{PMID: 1506157}, R A Soltys et al., 1992]. However, these do not directly address safety in children.

3. Age Ranges:
   - The abstracts specifically mention children as young as 10 years old (\hyperlink{pmid_8496990}{PMID: 8496990}), and several studies refer to "children" or "pediatric patients" without specifying lower age limits. There is no evidence from these abstracts regarding safety in infants or children under 10 years old.

Summary:
- For children aged 10–18 years, there is direct evidence from targeted studies affirming the safety of gadoteridol.
- For the broader pediatric population (age not always specified), multiple studies in "children" or "pediatric patients" found gadoteridol to be safe, but the lower age limit is not always clear.
- There is no targeted safety data in the abstracts for children under 10 years old or for infants.

Therefore, based on the abstracts, gadoteridol is affirmed as safe for use in children aged 10–18 years, and likely safe in the general pediatric population, but the safety in children under 10 years old (especially infants and toddlers) is unknown due to lack of specific data.

\subsection*{Abstracts}
\hypertarget{pmid_8430186}{A} phase III open-label clinical trial was conducted at 11 institutions to determine the safety and efficacy of gadoteridol in children suspected of having neurologic disease. One hundred three children were included in the safety analysis; 92 were evaluated for efficacy (76 intracranial and 16 spinal examinations). Three adverse events were reported in two children. All adverse events were considered minor and resolved spontaneously without treatment or sequelae. In a comparison of enhanced T1-weighted magnetic resonance images with unenhanced T1- and T2-weighted images, enhancement of disease was noted in 70\% of the intracranial and 38\% of the spinal examinations. Additional diagnostic information was reported in 82\% of the postcontrast intracranial examinations and 62\% of the spinal examinations. Use of this additional information contributed to a potential modification of patient diagnosis in 48\% of intracranial and 20\% of spinal cases with additional information. These results indicate excellent safety and efficacy for use of gadoteridol in children with suspected intracranial or spinal disease. [\hyperlink{Gadoteridol}{PMID: 8430186}, W S Ball et al., 1993]

\hypertarget{pmid_26045036}{G}adoteric acid is a paramagnetic gadolinium macrocyclic contrast agent approved for use in MRI of cerebral and spinal lesions and for body imaging. To investigate the safety and efficacy of gadoteric acid in children by extensively reviewing clinical and post-marketing observational studies. Data were collected from 3,810 children (ages 3 days to 17 years) investigated in seven clinical trials of central nervous system (CNS) imaging (n = 141) and six post-marketing observational studies of CNS, musculoskeletal and whole-body MR imaging (n = 3,669). Of these, 3,569 children were 2-17 years of age and 241 were younger than 2 years. Gadoteric acid was generally administered at a dose of 0.1 mmol/kg. We evaluated image quality, lesion detection and border delineation, and the safety of gadoteric acid. We also reviewed post-marketing pharmacovigilance experience. Consistent with findings in adults, gadoteric acid was effective in children for improving image quality compared with T1-W unenhanced sequences, providing diagnostic improvement, and often influencing the therapeutic approach, resulting in treatment modifications. In studies assessing neurological tumors, gadoteric acid improved border delineation, internal morphology and contrast enhancement compared to unenhanced MR imaging. Gadoteric acid has a well-established safety profile. Among all studies, a total of 10 children experienced 20 adverse events, 7 of which were thought to be related to gadoteric acid. No serious adverse events were reported in any study. Post-marketing pharmacovigilance experience did not find any specific safety concern. Gadoteric acid was associated with improved lesion detection and delineation and is an effective and well-tolerated contrast agent for use in children. [\hyperlink{Gadoteridol}{PMID: 26045036}, Csilla Balassy et al., 2015]

\hypertarget{pmid_21786126}{T}here is a paucity of evidence with regard to the safety of contrast medium administration at MRI in neonates and infants. To assess immediate adverse reactions in children younger than 18 months of age during routine clinical utilization of gadoteric acid (Gd-DOTA) in a cohort of patients with nonselected indications. One hundred and four neonates and infants were enrolled in a postmarketing survey with Gd-DOTA (Dotarem, Guerbet, Roissy, France) from a single pediatric hospital. A standardized questionnaire was used to collect the patient information. All included children, ages 3 days to 18 months, received one injection of Gd-DOTA (volume 0.6-4 ml). No immediate adverse event was reported. This postmarketing study involving neonates and infants suggests a favorable safety profile of Gd-DOTA in routine practice. [\hyperlink{Gadoteridol}{PMID: 21786126}, Sophie Emond et al., 2011]

\hypertarget{pmid_1506157}{T}o support clinical use of gadoteridol (0.5 M) injection, a battery of in vitro and in vivo safety studies was conducted. In mice, the acute intravenous LD50 for gadoteridol (0.5 M) injection was 11 to 14 mmol/kg, and the intravenous minimal lethal dose in rats was greater than 10 mmol/kg. In 2-week studies with gadoteridol, no serious effects were observed in mice given 3 mmol/kg or dogs given 1.5 mmol/kg daily. In a series of reproduction studies, no treatment-related adverse effects on fertility, reproductive performance, or postnatal development were seen in rats at doses of 1.5 mmol/kg or less, and no teratogenic effects were observed at doses as high as 6 mmol/kg in rabbits and 10 mmol/kg in rats. In an in vitro test, gadoteridol did not demonstrate any potential to hemolyze human erythrocytes when incubated in high concentrations with whole blood, suggesting there is little probability gadoteridol will cause hemolysis in vivo. A substantial margin of safety exists for the clinical use of gadoteridol in magnetic resonance imaging procedures. [\hyperlink{Gadoteridol}{PMID: 1506157}, R A Soltys et al., 1992]

\hypertarget{pmid_25114540}{G}adobutrol is a 1-molar gadolinium-based contrast agent with well-characterized safety and efficacy for magnetic resonance imaging (MRI) in adults and children ≥ 2 years old. This observational study assessed gadobutrol-enhanced MRI in children < 2 years of age. Sixty infants (mean age 11.1 months) underwent MRI using gadobutrol at standard dose of 0.1 mL/kg (0.1 mmol/kg) body weight. MRI examinations included brain, spine, and neck (n = 24), subcutaneous soft tissues (n = 14), chest, abdomen, and pelvis (n = 12), musculoskeletal system (n = 7) and vascular system (n = 3). No patients experienced adverse events related to gadobutrol injection. In 57 patients with confirmed diagnoses, gadobutrol-enhanced MRI provided findings consistent with confirmed pathologies. This study indicates that gadobutrol at a standard dose for MRI is safe in patients aged < 2 years and provides diagnostic information for multiple pathologies.  [\hyperlink{Gadoteridol}{PMID: 25114540}, Ravi Bhargava et al., 2013] Gadobutrol is a gadolinium-based contrast agent, uniquely formulated at 1.0 mmol/ml. Although there is extensive safety evidence on the use of gadobutrol in adults, few studies have addressed the safety and tolerability of gadobutrol in pediatric patients. This subanalysis of data from the GARDIAN study evaluated the safety and use of gadobutrol in pediatric patients (age <18 years). The GARDIAN study was a large phase IV non-interventional prospective multicenter post-authorization safety study performed in Europe, Asia, North America and Africa. A total of 23,708 patients were included who were scheduled to undergo cranial or spinal MRI, liver or kidney MRI, or MR angiography with gadobutrol enhancement. The primary study endpoint was the overall incidence of adverse drug reactions (ADRs) and serious adverse events (SAEs) following gadobutrol administration. The GARDIAN study included 1,142 children (age <18 years) who received gadobutrol at a mean dose of 0.13 (range 0.04-0.50) mmol/kg body weight. Gadobutrol was well tolerated in these children, with low rates of ADRs (0.5\%) and no SAEs, consistent with results in adults enrolled in the GARDIAN study. Rates of adverse events and ADRs were unrelated to pediatric age or gadobutrol weight-adjusted dose. There were no symptoms suggestive of nephrogenic systemic fibrosis. Investigators rated the contrast quality of gadobutrol-enhanced images as good or excellent in 97.8\% of pediatric patients, similar to the main study population. Gadobutrol is very well tolerated and provides excellent contrast quality at the recommended weight-adjusted dose in children (age <18 years), similar to the profile in adults. [\hyperlink{Gadoteridol}{PMID: 25114540}, Katja Glutig et al., 2016]

\hypertarget{pmid_8496990}{T}his article reports the results of clinical testing in pediatric patients of a new contrast agent, gadoteridol injection (ProHance), developed by Squibb Diagnostic as a nonionic gadolinium agent for use in magnetic resonance imaging (MRI). Thirteen children (four girls and nine boys) ranging in age from 10 to 18 years were enrolled in the study. The children had MR studies of the brain and/or spine with T1-weighted, T2-weighted, and postgadoteridol injection T1-weighted sequences. Five children had primary brain or spine neoplasms, three children had metastatic disease to the central nervous system, one child had a recurrent brain neoplasm and spinal canal metastasis, one child had an arteriovenous malformation, and two children were normal on the MRI studies. No minor or major reactions to gadoteridol injection developed in the 13 patients. Gadoteridol injection provided excellent delineation and enhancement of the arteriovenous malformation and all of the primary and secondary neoplasms of the central nervous system except for one case of a grade 1 glioma of the midbrain. Gadoteridol injection is a safe and excellent contrast agent for use in MRI. [\hyperlink{Gadoteridol}{PMID: 8496990}, S E Byrd et al., 1993]

\hypertarget{pmid_1501959}{T}wenty-two pediatric patients with known CNS neoplasms underwent magnetic resonance (MR) imaging before and after intravenous injection of 0.1 mmol/kg gadoteridol injection as part of a Phase IIIB open label multicenter clinical trial. Intravenous administration of this neutral, nonionic contrast agent was found to be safe in children. No clinically relevant changes in vital signs or laboratory values (including complete blood count, blood chemistry, serum electrolytes, thyroid and metabolic panel and clotting function) were attributed to the administration of gadoteridol injection. There were no systemic complaints. The imaging characteristics of gadoteridol in pediatric CNS disease appeared similar to those of gadopentetate dimeglumine. Contrast enhancement was present in 17 of 22 patients (77\%). The administration of gadoteridol injection provided additional clinically relevant information including improved visualization and delineation of the primary lesion, detection of additional lesions, determination of tumor recurrence and narrowing the list of differential considerations in all 17 enhancing studies as well as in 2 of 5 studies without signal intensity enhancement. The very low toxicity, inherent to this nonionic low osmolal paramagnetic contrast formulation may allow administration of increased doses at increased infusion rates for an increased number of indications with improved sensitivity. [\hyperlink{Gadoteridol}{PMID: 1501959}, J F Debatin et al., 1992]

\hypertarget{pmid_6937455}{H}aloperidol is safe and effective in children for relieving psychotic symptoms associated with childhood autism, schizophrenia and mental retardation. It is the drug of choice for Tourette's syndrome, and may be useful in nonpsychotic hyperactive or aggressive children to control acute episodes, or when the stimulants normally useful in hyperactive children are ineffective. Such children taking haloperidol not only become calmer, but are often better able to respond to other modalities of therapy and to school instruction. Dosage, initially low, is increased gradually to minimize drowsiness and extrapyramidal symptoms, the most common side effects. Haloperidol in children is usually well-tolerated. [\hyperlink{Gadoteridol}{PMID: 6937455}, A C Serrano et al., 1981]

\hypertarget{pmid_8929382}{T}he safety and efficacy of intravenous gadodiamide injection, 0.1 mmol/kg body weight, have been evaluated in an open label, non-comparative as to drug, phase III clinical trial in 50 children from 6 months to 13 years of age, referred for MRI requiring the injection of a contrast medium. The central nervous system and other body areas were examined with T1 sequences before and after intravenous injection of the contrast medium. Overall safety was very good and no clinically relevant changes were evident as regards heart rate and venous blood oxygen saturation after injection. No adverse event or discomfort was experienced by conscious patients that could with certainty be related to the contrast medium, but slight movements were observed in two sedated patients that could be related to the injection. Comparing pre- and post-injection images, additional diagnostic information could be obtained from the latter in 41 patients (82\%). In these images, the number of lesions detected increased and they were generally better delineated and their size more easily estimated. The results of this trial indicate that gadodiamide injection is safe and effective for MRI examinations in children. [\hyperlink{Gadoteridol}{PMID: 8929382}, S Hanquinet et al., 1996]

\hypertarget{pmid_27315460}{I}n the mouse, when a tympanic perforation is present, gadoteridol does not seem to cause ototoxicity. Gadodiamide may cause mild ototoxicity other than toxicity to the outer hair cells of the cochlea. Endolymphatic hydrops have been visualized through intra-tympanic injection of gadolinium-based contrast agents (GBCAs) and three-dimensional fluid-attenuated inversion recovery (3-D FLAIR) magnetic resonance imaging. However, reports on the safety of GBCAs are limited. This study aimed to assess ototoxicity of gadoteridol and gadodiamide. In a prospective, randomized, controlled trial, myringotomies in the left ear were performed in 20 male C57 BL/6 mice. After testing the baseline auditory brainstem response (ABR) (range = 8-32 kHz), the test solution (gadoteridol, gadodiamide, saline, or cisplatin) was injected into the left ear. ABR testing was repeated 14 days after test solution application. In morphological experiments, images of post-mortem surface preparations were assessed for cochlear hair cell status. At 14 days following gadoteridol application, there was no significant change in ABR thresholds at 8, 16, or 32 kHz. Gadodiamide application caused a significant change in the ABR threshold at 8 kHz. Apparent cochlear hair cell loss was not observed in the surface preparation after gadoteridol or gadodiamide application. [\hyperlink{Gadoteridol}{PMID: 27315460}, Hiroshi Nonoyama et al., 2016]

\hypertarget{pmid_1506153}{T}his study assesses the efficacy of gadoteridol for contrast-enhanced magnetic resonance imaging (MRI) in children. Patients were examined by MRI before and after receiving 0.10 mmol/kg gadoteridol. Blinded and unblinded readers analyzed brain and spine MRI studies from a multicenter clinical trial involving 101 patients at 11 sites. Ninety-two cases (76 brain, 16 spine) were evaluated by unblinded investigators, and 91 cases (76 brain, 15 spine) were evaluated by three neuroradiologists unaffiliated with any investigational site and blinded to clinical information. Unblinded readers noted enhancement of brain pathology in 70\% of cases versus 50\% to 67\% among blinded readers. Unblinded readers determined that additional diagnostic information was available after contrast in 82\% of brain studies (average, 64\% for blinded readers) and would have changed patient diagnoses in 48\% of these studies (average, 46\% for blinded readers). In spine cases, enhancement of pathology was noted in 38\% (unblinded) and 33\% to 40\% (blinded). Additional diagnostic information was available after contrast in 63\% of spine studies (unblinded), or an average of 58\% (blinded), and patient diagnoses would have changed in 20\% (unblinded), or an average of 59\% (blinded). Gadoteridol is suitable for enhanced MRI detection, localization, and characterization of central nervous system pathology in children. [\hyperlink{Gadoteridol}{PMID: 1506153}, W S Ball et al., 1992]

\hypertarget{pmid_1506151}{T}he use of paramagnetic contrast agents has improved the diagnostic sensitivity and specificity of magnetic resonance imaging (MRI) for evaluating diseases of the central nervous system. To assess the safety and imaging properties of the nonionic, gadolinium-based MRI contrast agent gadoteridol, 151 patients and controls were evaluated for safety, and 118 patients with cerebral or spinal pathology were evaluated for imaging efficacy. Precontrast T1- and T2-weighted spin-echo images and postcontrast (0.10 mmol/kg) T1-weighted spin-echo images were read by unblinded investigators at each site. The rate of adverse events possibly or probably related to gadoteridol was 4.0\% (vasodilation [facial flushing], 1 patient; nausea, 3 patients; urticaria, 2 patients). Laboratory changes were reported in 6.0\%. None of these events or changes was considered to be clinically significant. Contrast enhancement was noted in 75\% of cases with brain pathology and 64\% of cases involving spine lesions. Gadoteridol is safe in routine clinical use at a dose of 0.10 mmol/kg and provides improved lesion detection compared to plain MRI. [\hyperlink{Gadoteridol}{PMID: 1506151}, M Seiderer et al., 1992]

\hypertarget{pmid_28932122}{G}adobutrol is a gadolinium (Gd)-based contrast agent for magnetic resonance imaging (MRI). In India, gadobutrol is approved for MRI of the central nervous system (CNS), liver, kidneys, breast and for MR angiography for patients 2 years and older. The standard dose for all age groups is 0.1 mmol/kg body weight. The safety profile has been demonstrated in 42 clinical phase 2 to 4 studies (>6800 patients), 7 observational studies, and by assessing pharmacovigilance data of 29 million applications. Furthermore, studies in children, adults, and elderly and in patients with impaired liver or kidney function did not show any increased adverse event rate. Diagnostic efficacy was demonstrated in numerous studies and various indications, such as diseases of the CNS, peripheral and supra-aortic vessels, kidneys, liver, and breast. [\hyperlink{Gadoteridol}{PMID: 28932122}, Jan Endrikat et al., 2017]

\hypertarget{pmid_16028153}{B}ecause of concerns about arthrotoxicity, fluoroquinolones are restricted for use in children. This study describes the safety and efficacy of gatifloxacin when used for treatment of children with recurrent acute otitis media (ROM) or acute otitis media (AOM) treatment failure (AOMTF). We performed an analysis of 867 children included in 4 clinical trials who had ROM and/or AOMTF and were treated with gatifloxacin (10 mg/kg once daily for 10 days). Gatifloxacin had adverse event rates that were similar overall to those of a comparator antibiotic (amoxicillin-clavulanate), except for increased diarrhea in children <2 years old receiving amoxicillin-clavulanate. There was no evidence of arthrotoxicity, hepatotoxicity, alteration of glucose homeostasis, or central nervous system toxicity acutely or during 1 year follow-up in any child. Regarding efficacy, in 2 noncomparative trials, the gatifloxacin cure rate of AOM was 89\% (95\% confidence interval [CI], 83\%-95\%) at the test of cure (TOC) visit, 3-10 days after completion of therapy. In 2 comparative trials of gatifloxacin versus amoxicillin-clavulanate, the efficacy of gatifloxacin was 88\% (95\% CI, 82\%-94\%). Gatifloxacin led to better clinical outcomes than amoxicillin-clavulanate for AOMTF (91\% vs. 81\%; P=.029), for AOMTF and age <2 years old (89\% vs. 69\%; P=.009), and for severe AOM in children <2 years old (90\% vs. 75\%; P=.012). Among children with AOMTF previously treated with amoxicillin-clavulanate or ceftriaxone injections, gatifloxacin cure rates were high (88\% and 75\%, respectively). Gatifloxacin appears to be safe for children, with no evidence of producing arthrotoxicity in 867 children exposed to the antibiotic when used as treatment for ROM and AOMTF. [\hyperlink{Gadoteridol}{PMID: 16028153}, Michael E Pichichero et al., 2005]

\hypertarget{pmid_30450703}{S}edation is often required for young children during transthoracic echocardiography. Dexmedetomidine and ketamine are two sedatives that are commonly used in children for procedural sedation, but they have some disadvantages when they are used alone. The aim of this retrospective study was to analyze the effects and safety of intranasal sedation with a combination of dexmedetomidine and ketamine during transthoracic echocardiography in young children and to analyze risk factors for sedation failure. After IRB approval, we retrospectively evaluated data on patients who underwent echocardiography between May 2016 and August 2017 utilizing a combination of dexmedetomidine 2 μg/kg and ketamine 1 mg/kg. We collected information including heart rate, pulse oxygen saturation, sedation onset time, exam time, recovery time, and adverse reactions. Stepwise logistic regression analyses were performed to analyze the risk factors for sedation failure. Sedation was successful in 2212 patients (96\%) and took effect in 15.7 (IQR: 10-23) min, while sedation failed in 92 patients. Cyanotic heart disease, history of sedation failure, history of congenital heart disease surgery, and fever were independent risk factors for sedation failure. Most of the patients in this study had an American Society of Anesthesiologists (ASA) grade of II to III, but no severe adverse reactions were observed. Intranasal sedation with a combination of dexmedetomidine and ketamine is effective and appears to have an acceptable safety profile for young children during transthoracic echocardiography. [\hyperlink{Gadoteridol}{PMID: 30450703}, Jianxia Liu et al., 2019]

\hypertarget{pmid_26858095}{S}edation is increasingly used to facilitate procedures on children in emergency departments (EDs). This overview of systematic reviews (SRs) examines the safety and efficacy of sedative agents commonly used for procedural sedation in children in the ED or similar settings. We followed standard SR methods: comprehensive search; dual study selection, quality assessment, data extraction. We included SRs of children (1 month to 18 years) where the indication for sedation was procedure-related and performed in the ED. Fourteen SRs were included (210 primary studies). The most data were available for propofol (six reviews/50,472 sedations) followed by ketamine (7/8,238), nitrous oxide (5/8,220), and midazolam (4/4,978). Inconsistent conclusions for propofol were reported across six reviews. Half concluded that propofol was sufficiently safe; three reviews noted a higher occurrence of adverse events, particularly respiratory depression (upper estimate 1.1\%; 5.4\% for hypotension requiring intervention). Efficacy of propofol was considered in four reviews and found adequate in three. Five reviews found ketamine to be efficacious and seven reviews showed it to be safe. All five reviews of nitrous oxide concluded it is safe (0.1\% incidence of respiratory events); most found it effective in cooperative children. Four reviews of midazolam made varying recommendations. To be effective, midazolam should be combined with another agent that increases the risk of adverse events (upper estimate 9.1\% for desaturation, 0.1\% for hypotension requiring intervention). This comprehensive examination of an extensive body of literature shows consistent safety and efficacy for nitrous oxide and ketamine, with very rare significant adverse events for propofol. There was considerable heterogeneity in outcomes and reporting across studies and previous reviews. Standardized outcome sets and reporting should be encouraged to facilitate evidence-based recommendations for care. [\hyperlink{Gadoteridol}{PMID: 26858095}, Lisa Hartling et al., 2016]

\hypertarget{pmid_26197466}{G}onadotropin-releasing hormone analogues are generally regarded as safe drugs. Gonadorelin acetate has been widely used for the diagnosis of central precocious puberty, and life-threatening reactions to gonadorelin acetate are extremely rare. Herein, we described - to the best of our knowledge - the first pediatric case in which severe anaphylaxis was encountered after intravenous gonadorelin acetate administration. An 8-year-old girl who was diagnosed with central precocious puberty was receiving triptorelin acetate treatment uneventfully for 6 months. In order to evaluate the efficacy of the treatment, an LH-RH stimulation test with gonadorelin acetate was planned. Within 3 min after intravenous administration of gonadorelin acetate, she lost consciousness and tonic seizures began in her hands and feet. She was immediately treated with epinephrine, diphenhydramine, and fluids. Her vital signs recovered within 30 min. Based on the results, anaphylaxis should be anticipated and the administration of these drugs should be performed in a setting that is equipped to deal with systemic reactions.  [\hyperlink{Gadoteridol}{PMID: 26197466}, Onur Akın et al., 2015] The European Medicine Agency recommendations limiting codeine use in children have created a void in managing moderate pain. We review the evidence on the pharmacokinetic, pharmacodynamic and safety profile of tramadol, a possible substitute for codeine. Tramadol appears to be safe in both paediatric inpatients and outpatients. It may be appropriate to limit the current use of tramadol to monitored settings in children with risk factors for respiratory depression, subject to further safety evidence. [\hyperlink{Gadoteridol}{PMID: 26197466}, Pierluigi Marzuillo et al., 2014]

\hypertarget{pmid_8073972}{T}o assess the efficacy and safety profile of high-dose (0.3 mmol/kg cumulative dose) gadoteridol in patients with suspected central nervous system metastatic disease. We studied 67 patients using an incremental-dose technique. Patient monitoring included a medical history, physical examination, vital signs, and extensive laboratory tests within 24 hours before and after the MR examination. Precontrast T1- and T2-weighted spin-echo studies were performed, followed by intravenous injection of 0.1 mmol/kg of gadoteridol. T1-weighted images were acquired immediately after and at 10 and 20 minutes after injection. At 30 minutes an additional 0.2 mmol/kg of gadoteridol was administered (0.3-mmol/kg cumulative dose), and T1-weighted images were acquired. Cases demonstrating abnormal MR findings were assessed for efficacy by unblinded and blinded reviewers and were analyzed quantitatively. Three adverse effects in two patients were considered to be related to gadoteridol administration. No adverse effects were serious; all self-resolved. Forty-nine cases showed abnormal MR findings and were included in the efficacy analysis. A significantly greater number of lesions was seen on the high-dose as opposed to the standard-dose images. Blinded and unblinded readers identified 5 and 8 patients, respectively, with solitary lesions on standard-dose examination and multiple lesions on high-dose examination. Two patients who had normal standard-dose findings had lesions identified on high-dose studies. Quantitative analysis of 133 lesions in 45 patients demonstrated significant increases in lesion signal intensity on high-dose studies when compared with standard-dose studies. Gadoteridol can be safely administered up to a cumulative dose of 0.3 mmol/kg. High-dose contrast studies provide improved lesion detectability and additional diagnostic information over studies performed in the same patients with a 0.1-mmol/kg dose and aid in patient diagnosis and treatment. High-dose gadoteridol study may facilitate the care of patients with suspected central nervous system metastasis. [\hyperlink{Gadoteridol}{PMID: 8073972}, W T Yuh et al., 1994]

\hypertarget{pmid_34698441}{T}here is a paucity of data regarding the safety of the practice of sedation for oro-dental trauma in paediatric emergency departments (ED). A previous study reported the safety of intramuscular ketamine administered as a single agent. In the paediatric ED of a tertiary trauma centre in Israel, one of two ketamine-based regimens is used for sedating children with intraoral injuries according to the physician's discretion: a single dose of intramuscular ketamine or a combination of ketamine and propofol (KP) intravenously. The aim of this study was to assess the safety of KP sedation in children undergoing emergency treatment of oro-dental injuries in this paediatric ED. The primary outcome was sedation adverse events that required intervention (SAERI): prolonged oxygen desaturation and apnoea, laryngospasm, hypotension, bradycardia, partial or complete airway obstruction, and pulmonary aspiration. During the 2 years study period, 17 children were sedated with KP, 20 with intramuscular ketamine and 29 with nitrous oxide. Patients who were treated with ketamine-based sedation or with nitrous oxide sedation had a median (interquartile range, IQR) age of 3 (2-4) years and 7 (5-9) years, respectively. No SAERI occurred in patients who were sedated with intramuscular ketamine. One (3.4\%) SAERI was reported in a patient who was sedated with N [\hyperlink{Gadoteridol}{PMID: 34698441}, Leon Bilder et al., 2022] To demonstrate that gadodiamide injection is a safe and efficient contrast agent for MRI in infants younger than 6 months of age. The authors designed a phase III multicenter nonrandomized study using a control group. Gadodiamide injection at a dosage of 0.1 mmol/kg body weight was administered to 39 children; 20 received no contrast. The mean age was 10.6 weeks in the contrast group and 9.3 weeks in the control group. MR examinations, blood (serum creatinine, S-ASAT, S-ALAT, S-bilirubin, alkaline phosphatase) and urine (proteins, blood, others) sampling before sedation and after examination, heart rate (electrocardiography) and oxygen saturation (pulse oximetry) during examination, adverse events, and efficacy parameters were analyzed. In the contrast group, 18 (51.4\%) children had 31 abnormal changes in one or more of the safety parameters and vital signs. In the control group there were 16 (80.0\%) children with 19 abnormal changes. Gadodiamide injection had no negative influence on the safety parameters. No serious adverse events occurred, and only three clinically relevant adverse events (elevation of S-ALAT and S-ASAT, elevation of bilirubin) in two patients in the contrast group and one event (vomiting) in one patient in the control group were documented. The benefit of the contrast medium was clearly shown for all evaluated parameters. Gadodiamide injection is safe, well tolerated, and effective in infants younger than 6 months of age. [\hyperlink{Gadoteridol}{PMID: 34698441}, L Martí-Bonmatí et al., 2000]

\hypertarget{pmid_25746065}{S}ildenafil (Revatio®) and tadalafil (Adcirca®) are specific inhibitors of the phosphodiesterase-5 enzyme and produce pulmonary vasodilation by inhibiting the breakdown of cyclic guanosine monophosphate (cGMP) in the walls of pulmonary arterioles. We focus on the efficacy and safety of sildenafil and tadalafil in the treatment of pulmonary hypertension (PH) in children through a PubMed literature search. Although used since 1999 in the treatment of PH in children, it is only in the past few years that robust evidence for the use of sildenafil has emerged principally in the pivotal STARTS-1 study. The open-label extension of this study, STARTS-2, has revealed safety concerns substantiated by FDA post marketing surveillance leading to recommendations to use lower doses. More recently, tadalafil has been introduced allowing once daily dosing with apparently similar efficacy to sildenafil in children. Recently there have been suggestions that sildenafil and tadalafil may have a place in treating muscular dystrophy. [\hyperlink{Gadoteridol}{PMID: 25746065}, Alan G Magee et al., 2015]

\hypertarget{pmid_7772422}{I}n a prospective, randomized, blind study, we assessed the effectiveness of droperidol 20 micrograms kg-1 i.v., given at induction of anaesthesia, in preventing postoperative vomiting in paediatric day-case patients. We studied 270 children, aged 1-15 yr, undergoing body surface surgery. There was a significant reduction in the incidence of vomiting in the recovery room (1.4\% vs 9.2\%, P < 0.005) and in the day ward (9.4\% vs 18.3\%, P < 0.05) in patients receiving droperidol. There was no significant difference on the journey home (9.5\% vs 17.83\%, ns) or at home (16.7\% vs 10.3\%, ns). There was also a reduction in the severity of vomiting in the droperidol group. There were no adverse side effects. [\hyperlink{Gadoteridol}{PMID: 7772422}, D V Lunn et al., 1995]

\hypertarget{pmid_37812485}{T}his review describes the pharmacokinetics, efficacy, and safety of gadopiclenol, a new macrocyclic gadolinium-based contrast agent (GBCA) recently approved by the Food and Drug Administration at the dose of 0.05 mmol/kg. Gadopiclenol is a high relaxivity contrast agent that shares similar pharmacokinetic characteristics with other macrocyclic GBCAs, including a predominant renal excretion. In pediatric patients aged 2-17 years, the pharmacokinetic parameters (assessed through a population pharmacokinetics model) were comparable to those observed in adults, indicating no need for age-based dose adjustment. For contrast-enhanced magnetic resonance imaging (MRI) of the central nervous system (CNS) and body indications, gadopiclenol at 0.05 mmol/kg was shown to be noninferior to gadobutrol at 0.1 mmol/kg in terms of 3 lesion visualization parameters (ie, lesion border delineation, internal morphology, and contrast enhancement). Moreover, for contrast-enhanced MRI of the CNS, compared with gadobenate dimeglumine at 0.1 mmol/kg, gadopiclenol exhibited superior contrast-to-noise ratio at 0.1 mmol/kg and comparable contrast-to-noise ratio at 0.05 mmol/kg. A pooled safety analysis of 1047 participants showed a favorable safety profile for gadopiclenol. Comparative studies showed that the incidence and nature of adverse drug reactions with gadopiclenol were comparable to those observed with other GBCAs. Importantly, no significant safety concerns were identified in pediatric and elderly patients, as well as in patients with renal impairment. Overall, these findings support the clinical utility and safety of gadopiclenol for MRI in adult and pediatric patients aged 2 years and older in CNS and body indications. [\hyperlink{Gadoteridol}{PMID: 37812485}, Jing Hao et al., 2023]

\section*{Glycopyrrolate}
\subsection*{Result}
\subsubsection*{Answer}

Ages 3–16 years: Yes  
Ages under 3 years: Unknown  
Ages 1–18 years (perioperative/hyperhidrosis): Unknown  

\subsubsection*{{Explanation}}
\hypertarget{Glycopyrrolate}
Based on the abstracts reviewed, there are multiple targeted studies evaluating the safety of glycopyrrolate in children for various indications, primarily for sialorrhea (drooling) associated with neurological conditions, as well as for hyperhidrosis and perioperative use. The following summarizes the evidence by age range:

Ages 3–16 years:
- Several randomized, double-blind, placebo-controlled studies specifically evaluated glycopyrrolate in children aged 3–16 years with neurologic conditions causing drooling. These studies consistently found glycopyrrolate to be effective and generally well tolerated, with adverse events typical of anticholinergic agents (e.g., dry mouth, constipation). A minority (about 20–28\%) discontinued due to side effects, but no severe or unexpected safety concerns were reported. These studies affirm safety in this age group for the treatment of sialorrhea [\hyperlink{pmid_22298950}{PMID: 22298950}, Zeller et al., 2012; \hyperlink{pmid_22646067}{PMID: 22646067}, Garnock-Jones et al., 2012; \hyperlink{pmid_22003294}{PMID: 22003294}, Evatt et al., 2011; \hyperlink{pmid_11115305}{PMID: 11115305}, Mier et al., 2000; \hyperlink{pmid_8790123}{PMID: 8790123}, Blasco et al., 1996; \hyperlink{pmid_9069045}{PMID: 9069045}, Stern et al., 1997; \hyperlink{pmid_9729704}{PMID: 9729704}, Bachrach et al., 1998].

Ages 1–18 years:
- Studies of glycopyrrolate for perioperative use and for hyperhidrosis included children as young as 1 month and up to 18 years. These studies did not identify significant safety concerns, and glycopyrrolate was generally well tolerated. However, some studies were not primarily designed to assess long-term safety, and some were retrospective or had small sample sizes [\hyperlink{pmid_17242078}{PMID: 17242078}, Tait et al., 2007; \hyperlink{pmid_22405644}{PMID: 22405644}, Paller et al., 2012; \hyperlink{pmid_24266878}{PMID: 24266878}, Kumar et al., 2013].

Ages under 3 years:
- There are pharmacokinetic and perioperative studies including children under 3 years of age, and even as young as 1 month. These studies did not report significant adverse effects or safety concerns, but the sample sizes were small and the studies were not primarily designed to assess long-term safety for chronic use [\hyperlink{pmid_8060629}{PMID: 8060629}, Rautakorpi et al., 1994; \hyperlink{pmid_17242078}{PMID: 17242078}, Tait et al., 2007]. There is also a small case series (n=4) suggesting possible benefit and safety in infants with breath-holding spells, but this is not sufficient to definitively affirm safety in this age group [\hyperlink{pmid_25869368}{PMID: 25869368}, Williams et al., 2015].

Summary:
- For children aged 3–16 years, there is strong evidence from targeted studies affirming the safety of glycopyrrolate for sialorrhea due to neurologic conditions.
- For children under 3 years, and for infants, the safety of glycopyrrolate is not definitively established due to limited and less robust data.
- For children and adolescents with hyperhidrosis, small studies suggest safety, but larger, targeted safety studies are lacking.

\subsection*{Abstracts}
\hypertarget{pmid_22003294}{E}xcessive drooling may complicate the care of children with chronic neurological conditions by socially isolating both patients and families and by causing secondary dermatitis and infection. Normal control of saliva requires normal integrity of oral structures, normal oropharyngeal sensation, and motor functioning, as well as normal cognitive awareness and rate of salivary production. Glycopyrrolate is an anticholinergic medication with a quaternary structure that recently received Food and Drug Administration approval to treat sialorrhea due to neurological problems in children ages 3-16 years. This review summarizes the few published studies of safety and efficacy of glycopyrrolate for drooling in children with chronic neurological conditions. [\hyperlink{Glycopyrrolate}{PMID: 22003294}, Marian L Evatt et al., 2011]

\hypertarget{pmid_11115305}{T}o determine the safety and efficacy of glycopyrrolate in the treatment of developmentally disabled children with sialorrhea. Placebo-controlled, double-blind, crossover dose-ranging study. Outpatient facilities in 2 pediatric hospitals. Thirty-nine children with both developmental disabilities and excessive and bothersome sialorrhea. Parent and investigator evaluation of change in sialorrhea and adverse effects. Glycopyrrolate in doses of 0.10 mg/kg per dose is effective at controlling sialorrhea. Even at low doses, 20\% of children may exhibit adverse effects severe enough to require discontinuation. Glycopyrrolate is effective in the control of excessive sialorrhea in children with developmental disabilities. Approximately 20\% of children given glycopyrrolate may experience substantial adverse effects, enough to require discontinuation of medication. Arch Pediatr Adolesc Med. 2000;154:1214-1218. [\hyperlink{Glycopyrrolate}{PMID: 11115305}, R J Mier et al., 2000]

\hypertarget{pmid_9069045}{A} study was undertaken to assess the efficacy of an oral anticholinergic drug, glycopyrrolate, in the management of drooling in children and young adults with disabilities. Glycopyrrolate was used by 24 children and young adults for up to 28 months. Parents/carers were asked to complete a questionnaire on the effects of the drug on severity and frequency of drooling and to report any side-effects. Twenty-two questionnaires were returned. There was a statistically significant decrease in both severity and frequency of drooling with minimal side-effects reported. In this preliminary study, glycopyrrolate was found to be an effective and well-tolerated addition to the management of drooling in children with disabilities. [\hyperlink{Glycopyrrolate}{PMID: 9069045}, L M Stern et al., 1997]

\hypertarget{pmid_24266878}{P}rimary hyperhidrosis is a common disorder affecting children and adolescents, and it can have a significant negative psychosocial effect. Treatment for pediatric hyperhidrosis tends to be limited by low efficacy, low adherence, and poor tolerance. Oral glycopyrrolate is emerging as a potential second-line treatment option, but experience with safety, efficacy, and dosing is especially limited in children. We present an institutional review of 12 children with severe, refractory hyperhidrosis treated with oral glycopyrrolate; 11 (92\%) noted improvement and 9 (75\%) would recommend oral glycopyrrolate to their friends. No significant side effects were noted. Our retrospective analysis suggests that oral glycopyrrolate is safe and effective in children with hyperhidrosis.  [\hyperlink{Glycopyrrolate}{PMID: 24266878}, Monique G Kumar et al., ] Chronic drooling (sialorrhea) is a common dysfunction in children with neurologic disorders such as cerebral palsy. Glycopyrrolate oral solution, an anticholinergic agent, is the first drug treatment approved in the US for drooling in children with neurologic conditions. This article reviews the clinical efficacy and tolerability of glycopyrrolate oral solution in pediatric patients with neurologic conditions and provides an overview of the pharmacological properties of the drug. In a phase III, randomized, double-blind, multicenter trial, children (aged 3-16 years; n = 36) with problem drooling associated with neurologic conditions and receiving glycopyrrolate oral solution had a significantly (p < 0.01) greater modified Teacher's Drooling Scale (mTDS) response rate at 8 weeks (primary endpoint) than those receiving placebo (73.7\% vs 17.6\%). At 24 weeks in an additional, noncomparative, phase III study, 52.3\% of glycopyrrolate oral solution recipients (aged 3-18 years; n = 137) had an mTDS response (primary endpoint); the response rate was consistently above 50\% at all 4-weekly timepoints, aside from the first assessment at week 4 (40.3\%). In general, glycopyrrolate oral solution was well tolerated in clinical trials. The majority of adverse events were within expectations as characteristic anticholinergic outcomes. [\hyperlink{Glycopyrrolate}{PMID: 24266878}, Karly P Garnock-Jones et al., 2012]

\hypertarget{pmid_8060629}{T}o investigate the pharmacokinetics of glycopyrrolate in children. Open study with three parallel groups. Pediatric surgery department at a university hospital. 26 healthy ASA physical status I children undergoing minor surgery. Patients were assigned to 1 of 3 groups: under 1 year of age (Group 1, n = 8), between 1 and 3 years of age (Group 2, n = 7), and over 3 years of age (Group 3, n = 11). Glycopyrrolate 5 micrograms/kg was given as a single intravenous (i.v.) injection before induction of general anesthesia. Blood samples (for determination of drug concentrations in plasma) were collected via venous cannula inserted into the contralateral antecubital vein. ECG was observed continuously, blood pressure was measured with an automatic noninvasive device, and blood samples were taken just before and at 2, 4, 6, 10, 15, 30, 60, 120, 180, 240, 360, and 480 minutes after injection of glycopyrrolate. Glycopyrrolate concentrations in plasma were determined with a radioreceptor assay. The only significant difference in the pharmacokinetic parameters was the shortened elimination half-life in patients between 1 and 3 years of age. Glycopyrrolate 5 micrograms/kg i.v. did not cause any significant alterations in heart rate. There were no significant changes in the distribution volume or clearance of glycopyrrolate in children of different ages. The shortened elimination half-life in children between 1 and 3 years of age is of minor clinical importance. [\hyperlink{Glycopyrrolate}{PMID: 8060629}, P Rautakorpi et al., ]

\hypertarget{pmid_17242078}{T}wo recent studies have identified copious secretions as an independent risk factor for perioperative adverse events in children who present for elective surgery in the presence of an upper respiratory tract infection (URI). We designed this study, therefore, to determine whether the administration of the anticholinergic drug, glycopyrrolate, to children with URIs would reduce the incidence of adverse perioperative respiratory events. One hundred thirty children (1 mo to 18 yr of age) who presented for elective surgery with a URI were randomized to receive either 0.01 mg/kg glycopyrrolate or placebo and were followed for the appearance and severity of any perioperative respiratory adverse events. The two groups were similar with respect to demographics, presenting URI symptoms, anesthetic management, and surgical procedure. In the intention-to-treat analysis, there were no statistical differences in the incidence or severity of perioperative respiratory adverse events between the glycopyrrolate and placebo groups (45.2\% vs 37.5\% respectively, P = NS). Furthermore, there were no differences in outcome between the two groups when children with congestion and secretions were analyzed separately (45.0\% vs 37.0\%, respectively). However, compared with the placebo group, children in the glycopyrrolate group had significantly shorter discharge times (83.9 min vs 111.4 min, P = 0.024), and significantly less postoperative nausea and vomiting (10.7\% vs 33.3\%, P = 0.005). These results suggest that glycopyrrolate, administered after induction of anesthesia to children with URIs, does not reduce the incidence of perioperative respiratory adverse events, and thus may not be clinically indicated for routine use in this population. [\hyperlink{Glycopyrrolate}{PMID: 17242078}, Alan R Tait et al., 2007]

\hypertarget{pmid_6859501}{G}lycopyrrolate is a quaternary ammonium compound with indications for use similar to those for atropine. Because of the quaternary nature, it is poorly absorbed when taken orally and penetrates neither placental nor blood-brain barriers. When given by the parenteral route, the cardio-vagal blocking action of glycopyrrolate is twice that of atropine while inhibition of salivation is 5-6 times greater. The use of glycopyrrolate for premedication provides a therapeutic margin 2-3 times wider than that of atropine. Glycopyrrolate administered with neostigmine to antagonise the residual neuromuscular blockade of non-depolarising relaxants has advantages over atropine because the pharmacodynamic profile is more suited to that of neostigmine. The abrupt changes in cardiac rate, therefore, become minimal. If glycopyrrolate, 5 micrograms/kg-1, is injected intravenously just before the induction of anaesthesia, severe bradycardia is inhibited when repeated doses of succinylcholine are used. Although the alkalinising effect on gastric secretions has not been substantially verified, glycopyrrolate does provide long lasting bronchodilatation from its blocking action on smooth muscle. Only a few studies with glycopyrrolate in children have yet been published. However, it appears that this drug provides no real advantages over atropine when used in paediatric anaesthesia. [\hyperlink{Glycopyrrolate}{PMID: 6859501}, D A Cozanitis et al., 1983]

\hypertarget{pmid_9729704}{F}ifty-four parents/caretakers of children with cerebral palsy were surveyed regarding their use of antisialorrheic medication for excessive drooling. Glycopyrrolate was used by 37 of 41 respondents, with significant improvement in drooling noted in the vast majority (95\%) of cases as indicated by a five-point rating scale. Side effects (dry mouth, thick secretions, urinary retention, or flushing) surfaced in almost half (44\%) of the patients but necessitated discontinuation of pharmacologic treatment in less than a third. While larger clinical studies are needed, our preliminary data indicate a trial of glycopyrrolate should be considered in children with cerebral palsy where drooling is a significant problem. [\hyperlink{Glycopyrrolate}{PMID: 9729704}, S J Bachrach et al., 1998]

\hypertarget{pmid_9783332}{B}ased on plasma levels determined with a radioreceptor assay and following a single oral (50 micrograms/kg) and intravenous (5 micrograms/kg) administration of glycopyrrolate in six healthy children operated twice during a several weeks period, a negligible and variable oral bioavailability was found (3.3; 1.3-13.3\%) (median;range). No significant changes in heart rate after oral or intravenous administration of the drug could be seen. Oral glycopyrrolate appears to have no place in paediatric premedication. [\hyperlink{Glycopyrrolate}{PMID: 9783332}, P Rautakorpi et al., 1998]

\hypertarget{pmid_8481249}{G}lycopyrrolate is an anticholinergic agent used to dry oral secretions and has been advocated for routine use with transesophageal echocardiography (TEE). To evaluate the safety and efficacy of glycopyrrolate for this unique application, a prospective double-blind placebo-controlled study of glycopyrrolate was performed in 61 patients who were awake while undergoing TEE. Thirty patients were randomized to the standard dose of glycopyrrolate (0.2 mg intravenously), and 31 patients received 1 ml of saline solution as placebo. Intravenous midazolam was used for sedation in all but one patient. Heart rate, electrocardiogram, blood pressure, and oxygen saturation were continuously monitored before, during, and after TEE. The patients scored their comfort immediately after TEE and were interviewed at 24 hours for side effects. The operator scored the ease of performing the TEE. No complications occurred in either group. Changes in vital signs and oxygen saturation were similar in both groups. The operator ease and patient comfort was similar in both groups. A significantly higher incidence of the following side effects was observed at 24 hours in patients who received glycopyrrolate versus those who received placebo: sore throat, 63\% versus 19\%; dry mouth, 43\% versus 6\%; and urinary retention, 16\% versus 0\% (p < 0.05 for all). No benefit from glycopyrrolate was noted in operator ease or patient comfort. In conclusion, glycopyrrolate is not recommended for routine use when performing TEE on patients who are awake. [\hyperlink{Glycopyrrolate}{PMID: 8481249}, J Gorcsan et al., ]

\hypertarget{pmid_22298950}{T}o evaluate the efficacy of glycopyrrolate oral solution (1 mg/5 mL) in managing problem drooling associated with cerebral palsy and other neurologic conditions. Thirty-eight patients aged 3-23 years weighing at least 27 lb (12.2 kg) with severe drooling (clothing damp 5-7 days/week) were randomized to glycopyrrolate (n = 20), 0.02-0.1 mg/kg three times a day, or matching placebo (n = 18). Primary efficacy endpoint was responder rate, defined as percentage showing ≥3-point change on the modified Teacher's Drooling Scale (mTDS). Responder rate was significantly higher for the glycopyrrolate (14/19; 73.7\%) than for the placebo (3/17; 17.6\%) group (P = 0.0011), with improvements starting 2 weeks after treatment initiation. Mean improvements in mTDS at week 8 were significantly greater in the glycopyrrolate than in the placebo group (3.94 ± 1.95 vs 0.71 ± 2.14 points; P < 0.0001). In addition, 84\% of physicians and 100\% of parents/caregivers regarded glycopyrrolate as worthwhile compared with 41\% and 56\%, respectively, for placebo (P ≤ 0.014). Most frequently reported treatment-emergent adverse events (glycopyrrolate vs placebo) were dry mouth, constipation, and vomiting. Children aged 3-16 years with problem drooling due to neurologic conditions showed a significantly better response, as assessed by mTDS, to glycopyrrolate than to placebo.  CLINICALTRIALS.GOV IDENTIFIER: NCT00425087. [\hyperlink{Glycopyrrolate}{PMID: 22298950}, Robert S Zeller et al., 2012]

\hypertarget{pmid_7446931}{A}tropine and glycopyrrolate were compared when given in a mixture with neostigmine for the reversal of non-depolarising neuromuscular block in children. Glycopyrrolate was an effective antimuscarinic agent and could be safely used as an alternative to atropine, although the advantages in this age group were not as marked as have been observed in adults. [\hyperlink{Glycopyrrolate}{PMID: 7446931}, G W Black et al., 1980]

\hypertarget{pmid_33433785}{G}lycopyrronium tosylate (GT; Qbrexza The objective of this study was to compare the pharmacokinetics and safety of GT to oral glycopyrrolate (phase I study) and assess the relationship between glycopyrronium pharmacokinetics and anticholinergic-related adverse events or efficacy with population pharmacokinetics using data from two phase II studies. In the phase I study, study staff applied GT to axillae of patients with primary axillary hyperhidrosis (aged 9-65 years) once daily (5 days); oral glycopyrrolate was administered to healthy adults (aged 18-65 years) every 8 hours (15 days). In the phase II studies (NCT02016885 [20 December, 2013], NCT02129660 [2 May, 2014]), adults with primary axillary hyperhidrosis applied topical glycopyrronium (0.8-3.2\%) or vehicle to axillae once daily (4 weeks). Pharmacokinetic and adverse event data were collected in all studies. Glycopyrronium pharmacokinetic parameters were similar between adult and pediatric patients treated with GT; there was no evidence of accumulation. Systemic absorption of glycopyrronium was lower with GT vs oral glycopyrrolate. No anticholinergic-related adverse events occurred with GT in the phase I study, while dry mouth and nasal dryness occurred with oral glycopyrrolate; anticholinergic adverse events occurred in the phase II studies. In the population pharmacokinetic analysis, frequency/severity of anticholinergic-related adverse events increased with higher glycopyrronium concentration; no relationship was observed between efficacy and pharmacokinetic measures. These studies indicate limited absorption of GT compared to oral glycopyrrolate and a low risk of anticholinergic adverse events with proper GT administration when following instructions for use (wipe each underarm once with same cloth, wash hands, avoid ocular contact). [\hyperlink{Glycopyrrolate}{PMID: 33433785}, David M Pariser et al., 2021]

\hypertarget{pmid_8790123}{T}o describe the use of glycopyrrolate in the control of drooling in children and young adults with cerebral palsy and related neurodevelopmental disabilities. Prospective, open-label study of drug dosage parameters, response to therapy, and side effects. Follow-up ranged from 8 months to 4 years. Outpatient clinic of a rehabilitation hospital that is a regional referral center for children with disabilities. Forty children and young adults with motor and/or cognitive disabilities who were experiencing drooling to a severe degree. Treatment with oral glycopyrrolate. Change in the quantity of drooling and side effects associated with treatment. Thirty-six patients (90\%) had reduced drooling in response to medication; 2 (5\%) could not be assessed and 2 (5\%) received no benefit. Side effects resulted in discontinuation of treatment in 11 (28\%). Overall, 26 (65\%) continued to receive drug therapy because of the perceived benefit. The final effective dose ranged widely from 0.01 to 0.82 mg/kg per day. Glycopyrrolate therapy safely and effectively decreased but rarely abolished drooling in patients with cerebral palsy and related neurodevelopmental disabilities. The dose range was surprisingly broad. Side effects, although generally minor and predictable, often led to discontinuation of drug therapy. [\hyperlink{Glycopyrrolate}{PMID: 8790123}, P A Blasco et al., 1996]

\hypertarget{pmid_8720721}{G}lycopyrrolate, an anticholinergic agent that does not cross the blood-brain barrier, has several indications, but its mydriatic effect has never been tested. This study was carried out in order to compare the mydriatic effect of glycopyrrolate 0.5\% to that of atropine sulfate 1\%. Glycopyrrolate 0.5\% and atropine 1.0\% were instilled separately in the eyes of albino rabbits. Pupil diameter and intra-ocular pressure were monitored. Mydriasis was noted within 5 min of glycopyrrolate instillation, reached near-maximal level at 15 min and persisted for 1 week. Glycopyrrolate 0.5\% showed a faster, stronger and more persistent mydriatic effect than atropine 1.0\%. Administration of glycopyrrolate 0.5\% solution b.i.d. for 1 week did not affect intra-ocular pressure or produce any adverse reaction. Glycopyrrolate solution has the potential to deliver an ocular anticholinergic effect without causing associated central anticholinergic hazards. [\hyperlink{Glycopyrrolate}{PMID: 8720721}, D Varssano et al., 1996]

\hypertarget{pmid_10365009}{W}e have tested the hypotheses that glycopyrrolate, administered immediately before induction of subarachnoid anaesthesia for elective Caesarean section, reduces the incidence and severity of nausea, with no adverse effects on neonatal Apgar scores, in a double-blind, randomized, controlled study. Fifty women received either glycopyrrolate 200 micrograms or saline (placebo) i.v. during fluid preload, before induction of spinal anaesthesia with 2.5 ml of 0.5\% isobaric bupivacaine. Patients were questioned directly regarding nausea at 3-min intervals throughout operation and asked to report symptoms as they arose. The severity of nausea was assessed using a verbal scoring system and was treated with increments of i.v. ephedrine and fluids. Patients in the group pretreated with glycopyrrolate reported a reduction in the frequency (P = 0.02) and severity (P = 0.03) of nausea. Glycopyrrolate also reduced the severity of hypotension, as evidenced by reduced ephedrine requirements (P = 0.02). There were no differences in neonatal Apgar scores between groups. [\hyperlink{Glycopyrrolate}{PMID: 10365009}, D Ure et al., 1999]

\hypertarget{pmid_22405644}{P}rimary focal hyperhidrosis not uncommonly begins during the first two decades of life, and can have a profound effect on quality of life. Few treatment options have been studied in children. We sought to evaluate the response to oral glycopyrrolate in pediatric patients. Records of pediatric patients with hyperhidrosis seen at a pediatric hospital in a 10-year period were reviewed retrospectively and, if possible, parents and patients were also interviewed. The efficacy and adverse effects of oral glycopyrrolate were assessed. In all, 31 children took at least one dose of oral glycopyrrolate. All had daily hyperhidrosis that affected their quality of life and were resistant or intolerant of aluminum salts. The mean age of hyperhidrosis onset was 10.3 years, and mean age of initiation of glycopyrrolate was 14.8 years. At a mean dosage of 2 mg daily, 90\% of patients experienced improvement, which was major in 71\% of responders. Improvement occurred within hours of administration and disappeared within a day of discontinuation. Duration of treatment averaged 2.1 years (range to 10 years). Side effects were noted by 29\% of children, most commonly dry mouth (26\%) and eyes (10\%), and were dose-related. One patient developed blurred vision, which resolved with dosing below 5 mg/d; one patient experienced palpitations and discontinued the medication. This was a retrospective analysis of a limited number of pediatric patients. Oral glycopyrrolate is a cost-effective, painless second-line therapy for children and adolescents with primary focal hyperhidrosis that impacts their quality of life. [\hyperlink{Glycopyrrolate}{PMID: 22405644}, Amy S Paller et al., 2012]

\hypertarget{pmid_27354782}{T}he purpose of this study was to confirm the efficacy and safety of twice-daily glycopyrrolate 15.6 µg, a long-acting muscarinic antagonist, in patients with stable, symptomatic, chronic obstructive pulmonary disease (COPD) with moderate-to-severe airflow limitation. The GEM1 study was a 12-week, multicenter, double-blind, parallel-group, placebo-controlled study that randomized patients with stable, symptomatic COPD with moderate-to-severe airflow limitation to twice-daily glycopyrrolate 15.6 µg or placebo (1:1) via the Neohaler(®) device. The primary objective was to demonstrate superiority of glycopyrrolate versus placebo in terms of forced expiratory volume in 1 second area under the curve between 0 and 12 hours post morning dose at week 12. Other outcomes included additional spirometric end points, transition dyspnea index, St George's Respiratory Questionnaire, COPD Assessment Test, rescue medication use, and symptoms reported by patients via electronic diary. Safety was also assessed during the study. Of the 441 patients randomized (glycopyrrolate, n=222; placebo, n=219), 96\% of patients completed the planned treatment phase. Glycopyrrolate demonstrated statistically significant (P<0.001) improvements in lung function versus placebo. Glycopyrrolate showed statistically significant improvement in the transition dyspnea index focal score, St George's Respiratory Questionnaire total score, COPD Assessment Test score, rescue medication use, and daily total symptom score versus placebo at week 12. Safety was comparable between the treatment groups. Significant improvement in lung function, dyspnea, COPD symptoms, health status, and rescue medication use suggests that glycopyrrolate is a safe and effective treatment option as maintenance bronchodilator in patients with stable, symptomatic COPD with moderate-to-severe airflow limitation. [\hyperlink{Glycopyrrolate}{PMID: 27354782}, Craig LaForce et al., 2016]

\hypertarget{pmid_7137551}{A}tropine 15 micrograms/kg and glycopyrrolate 5 or 10 microgram/kg were studied as anticholinergic premedicants in groups of 20 children each. A control group of 20 children did not receive anticholinergic premedication. Both atropine and the higher dose of glycopyrrolate produced significant increases in heart rate prior to induction of anaesthesia. The subsequent increase during the process of induction was less than in those who had not received an anticholinergic drug or glycopyrrolate 5 micrograms/kg. Dysrhythmias during induction of anaesthesia occurred slightly less frequently in the patients given atropine or the higher dose of glycopyrrolate. Although the incidence was similar in these two groups, ventricular ectopic beats occurred less frequently following the use of glycopyrrolate. The control of secretions was also superior with this anticholinergic premedicant. [\hyperlink{Glycopyrrolate}{PMID: 7137551}, R K Mirakhur et al., 1982]

\hypertarget{pmid_6871778}{T}o determine whether intravenous atropine and glycopyrrolate are equally effective in preventing succinylcholine-induced heart rate changes, we studied the heart rate during the first 78 seconds of anaesthesia in 40 children anaesthetized with either thiopentone, atropine (0.02 mg X kg-1) and succinylcholine (2 mg X kg-1), or thiopentone, glycopyrrolate (0.01 mg X kg-1) and succinylcholine (2 mg X kg-1). Each treatment group was divided into four subgroups which differed only in the interval (6, 10, 15, 20 seconds) between injection of atropine or glycopyrrolate and succinylcholine. During the 54 seconds after succinylcholine, the mean heart rate of each subgroup decreased transiently and then returned to the pre-induction heart rate or higher. There was no difference in either the magnitude or the duration of the decrease in heart rate or the subsequent increase in heart rate between respective subgroups. Bradycardia occurred in only two patients, both of whom received glycopyrrolate. We conclude that atropine (0.02 mg X kg-1) and glycopyrrolate (0.01 mg X kg-1) are equally effective in attenuating succinylcholine-induced changes in heart rate in children. [\hyperlink{Glycopyrrolate}{PMID: 6871778}, J Lerman et al., 1983]

\hypertarget{pmid_7126399}{T}he frequency and nature of the oculocardiac reflex and its prevention by atropine or glycopyrrolate (i.m. and i.v.) has been studied in 160 children undergoing surgery for the correction of squint. Ninety per cent of those given no anticholinergic premedication exhibited the reflex. This was decreased to about 50\% in those receiving the drugs i.m. Glycopyrrolate 7.5 micrograms kg-1 and atropine 15 micrograms kg-1 i.v. were effective in most instances, the latter being slightly better. However, glycopyrrolate was associated with tachycardia of smaller magnitude. The reflex was observed more often following traction on the medial rectus muscle. [\hyperlink{Glycopyrrolate}{PMID: 7126399}, R K Mirakhur et al., 1982]

\hypertarget{pmid_20385892}{S}ialorrhea affects approximately 75\% of patients with Parkinson disease (PD). Sialorrhea is often treated with anticholinergics, but central side effects limit their usefulness. Glycopyrrolate (glycopyrronium bromide) is an anticholinergic drug with a quaternary ammonium structure not able to cross the blood-brain barrier in considerable amounts. Therefore, glycopyrrolate exhibits minimal central side effects, which may be an advantage in patients with PD, of whom a significant portion already experience cognitive deficits. To determine the efficacy and safety of glycopyrrolate in the treatment of sialorrhea in patients with PD. We conducted a 4-week, randomized, double-blind, placebo-controlled, crossover trial with oral glycopyrrolate 1 mg 3 times daily in 23 patients with PD. The severity of the sialorrhea was scored on a daily basis by the patients or a caregiver with a sialorrhea scoring scale ranging from 1 (no sialorrhea) to 9 (profuse sialorrhea). The mean (SD) sialorrhea score improved from 4.6 (1.7) with placebo to 3.8 (1.6) with glycopyrrolate (p = 0.011). Nine patients (39.1\%) with glycopyrrolate had a clinically relevant improvement of at least 30\% vs 1 patient (4.3\%) with placebo (p = 0.021). There were no significant differences in adverse events between glycopyrrolate and placebo treatment. Oral glycopyrrolate 1 mg 3 times daily is an effective and safe therapy for sialorrhea in Parkinson disease. This study provides Class I evidence that glycopyrrolate 1 mg 3 times daily is more effective than placebo in reducing sialorrhea in patients with Parkinson disease during a 4-week study. [\hyperlink{Glycopyrrolate}{PMID: 20385892}, M E L Arbouw et al., 2010]

\hypertarget{pmid_25869368}{B}reath-holding spells are a common childhood disorder that typically present before 12 months of age. Whereas most cases are benign, some patients have very severe cases associated with bradycardia that can progress from asystole to syncope and seizures. Treatment studies have implicated the use of several therapies, such as oral iron, fluoxetine, and pacemaker implantation. This is a retrospective study of patients treated with glycopyrrolate for pallid breath-holding spells. Clinical data from 4 patients referred to pediatric cardiology who saw therapeutic benefit from treatment using glycopyrrolate were reviewed to evaluate for clinical response to the drug. Two twin patients, whose symptoms began at 5 months of age, experienced a decrease in breath-holding frequency after 1 month. A patient diagnosed at 7 months of age experienced a decrease in frequency of spells. A patient diagnosed at 10 months of age reported cessation of syncope shortly after initiation of glycopyrrolate and complete resolution of breath-holding spells during prolonged treatment. This case study of 4 patients with pallid breath-holding offers evidence that glycopyrrolate may be beneficial in treating breath-holding spells and has a safer side-effect profile than pacemaker implantation.  [\hyperlink{Glycopyrrolate}{PMID: 25869368}, Jason Williams et al., 2015] Remifentanil is recommended for use in procedures with painful intraoperative stimuli but minimal postoperative pain. However, bradycardia and hypotension are known side-effects. We evaluated haemodynamic effects of i.v. glycopyrrolate during remifentanil-sevoflurane anaesthesia for cardiac catheterization of children with congenital heart disease. Forty-five children undergoing general anaesthesia with remifentanil and sevoflurane were randomly allocated to receive either saline, glycopyrrolate 6 microg kg(-1) or glycopyrrolate 12 microg kg(-1). After induction of anaesthesia with sevoflurane, i.v. placebo or glycopyrrolate was administered. An infusion of remifentanil at the rate of 0.15 microg kg(-1) min(-1) was started, sevoflurane continued at 0.6 MAC and cisatracurium 0.2 mg kg(-1) was given. Heart rate (HR) and non-invasive arterial pressures were monitored and noted every minute for the first 10 min and then every 2.5 min for subsequent maximum of 45 min. Baseline HR [mean (SD)] of 117 (20) beats min(-1) decreased significantly from 12.5 min onwards after starting the remifentanil infusion in the control group [106 (18) at 12.5 min and 99 (16) beats min(-1) at 45 min]. In the groups receiving glycopyrrolate, no significant decrease in HR was noticed. Glycopyrrolate at 12 microg kg(-1) induced tachycardia between 5 and 9 min after administration. Systolic and diastolic arterial pressures decreased gradually, but there were no significant differences in the pressures between groups. I.V. glycopyrrolate 6 microg kg(-1) prevents bradycardia during general anaesthesia with remifentanil and sevoflurane for cardiac catheterization in children with congenital heart disease. Administering 12 microg kg(-1) of glycopyrrolate temporarily induces tachycardia and offers no additional advantage. [\hyperlink{Glycopyrrolate}{PMID: 25869368}, K Reyntjens et al., 2005]

\section*{Granisetron Hydrochloride}
\subsection*{Result}
\subsubsection*{Answer}

Ages <1 year: Unknown  
Ages 1–18 years: Yes  
Ages >18 years: Unknown (not addressed in pediatric studies)

\subsubsection*{{Explanation}}
\hypertarget{Granisetron Hydrochloride}
A review of the available abstracts reveals multiple targeted studies evaluating the safety of Granisetron Hydrochloride in children across various age ranges and clinical scenarios. Below is a summary of the evidence, organized by age group:

Ages 1–16 years:
- A multicenter, double-blind, dose-ranging study in 294 children aged 1–16 years receiving chemotherapy found both 20 and 40 microg/kg oral granisetron to be effective and safe, with no significant safety concerns reported [\hyperlink{pmid_10779814}{PMID: 10779814}, M Mabro et al., 2000].
- Another study in children aged 1–23 years (median 7.7 years) receiving chemotherapy found both 10 and 40 microg/kg IV granisetron to be well tolerated, with no significant adverse events [\hyperlink{pmid_17372773}{PMID: 17372773}, Su G Berrak et al., 2007].

Ages 1–18 years:
- A study in 30 children aged 3–18 years receiving chemotherapy reported no serious adverse events with IV granisetron (20 microg/kg/dose) [\hyperlink{pmid_8037341}{PMID: 8037341}, S J Jacobson et al., 1994].
- A study in 88 children aged 2–17 years receiving ifosfamide therapy found granisetron (20 microg/kg IV) to be superior to chlorpromazine-dexamethasone, with fewer adverse effects and no extrapyramidal reactions [\hyperlink{pmid_7844684}{PMID: 7844684}, K Hählen et al., 1995].

Ages 1–10 years:
- Multiple randomized, double-blind, placebo-controlled studies in children aged 4–10 years undergoing surgery (including tonsillectomy, hernia repair, and strabismus repair) found oral and IV granisetron at doses of 20–80 microg/kg to be effective for preventing postoperative vomiting, with no clinically important adverse events reported [\hyperlink{pmid_9861127}{PMID: 9861127}, Y Fujii et al., 1998; \hyperlink{pmid_10084103}{PMID: 10084103}, Y Fujii et al., 1999; \hyperlink{pmid_11903942}{PMID: 11903942}, Yoshitaka Fujii et al., 2002; \hyperlink{pmid_8807169}{PMID: 8807169}, Y Fujii et al., 1996; \hyperlink{pmid_10434165}{PMID: 10434165}, Y Fujii et al., 1999; \hyperlink{pmid_11210871}{PMID: 11210871}, Y Fujii et al., 2001].
- A study in 80 children aged 2–5 years found IV granisetron (10 µg/kg) effective for preventing shivering after spinal anesthesia, with no significant adverse events [\hyperlink{pmid_22313076}{PMID: 22313076}, Ahmed A Eldaba et al., 2012].

Ages 1 year (case report):
- A case report describes safe use of granisetron in a 1-year-old girl after ondansetron anaphylaxis, following negative challenge tests [\hyperlink{pmid_20921907}{PMID: 20921907}, Hacı Ahmet Demir et al., 2010].

General pediatric oncology:
- Several studies in children with cancer (ages not always specified, but including children as young as 1 year) consistently report granisetron (20–40 microg/kg) as effective and well tolerated, with no serious adverse events [\hyperlink{pmid_7752999}{PMID: 7752999}, A W Craft et al., 1995; \hyperlink{pmid_10478183}{PMID: 10478183}, T Sawada et al., 1999; \hyperlink{pmid_10533454}{PMID: 10533454}, Y Komada et al., 1999; \hyperlink{pmid_8037342}{PMID: 8037342}, Y Miyajima et al., 1994].

Special considerations:
- One study in 22 children found that IV granisetron (40 microg/kg) caused transient, asymptomatic changes in ECG (QT and QTc dispersion), but no clinical symptoms were observed [\hyperlink{pmid_15803017}{PMID: 15803017}, Mustafa Buyukavci et al., 2005].
- A laboratory study on human fetal tissue (16–19 weeks gestation) found no toxicity at 3 ng/mL, but up to 10\% apoptosis in cardiac tissue at 30 ng/mL; clinical significance is unclear and further research is needed [\hyperlink{pmid_25813292}{PMID: 25813292}, Judith A Smith et al., 2015].

Summary:
There is robust evidence from multiple targeted studies affirming the safety of Granisetron Hydrochloride in children aged 1–18 years for the prevention of chemotherapy-induced and postoperative nausea and vomiting. The studies consistently report no serious adverse events, and granisetron is well tolerated at commonly used doses (10–40 microg/kg, occasionally up to 80–100 microg/kg). For infants under 1 year, only a single case report exists, which is insufficient to establish safety in this age group. For fetal exposure, safety is unknown and requires further research.

\subsection*{Abstracts}
\hypertarget{pmid_10779814}{T}his multicentric double-blind, dose-ranging study was to compare efficacy and safety of two oral doses of granisetron solution in the prevention of chemotherapy-induced emesis in children with malignant diseases : 294 children, aged 1 to 16, treated with a moderately or highly emetogenic chemotherapy were randomly assigned to receive oral granisetron either 20 microg/kg (n = 143) or 40 microg/kg (n = 151) before and 6 to 12 hours after the start of chemotherapy. Fifty-one percent of patients treated with 20 microg/kg bd of oral granisetron solution achieved a complete response (no vomiting, no worse than mild nausea, no rescue therapy and no withdrawal during the specified period) and 59\% achieved a major response (no more than one episode of vomiting, no worse than mild nausea, no rescue therapy and no withdrawal during the specified period). There was no difference between the two oral doses of granisetron. Treatment was rated as good or very good by investigators in 70\% of cases. In conclusion, oral granisetron suspension either at 20 microg/kg bd or at 40 microg/kg bd showed good efficacy and safety in the prevention of chemotherapy-induced emesis in children with malignant diseases. Oral granisetron solution can be used as prophylaxis of emesis in children receiving moderately or highly emetogenic chemotherapy. [\hyperlink{Granisetron Hydrochloride}{PMID: 10779814}, M Mabro et al., 2000]

\hypertarget{pmid_7752999}{T}he safety and efficacy of the new 5HT-3 antagonist granisetron as an antiemetic in children with cancer was evaluated in 40 children at a single dose of 40 micrograms/kg. No adverse affects attributable to the granisetron were noted. The overall major and complete response rate was 82.5\% and this was highest in the younger children. Only 2 patients showed no response. Pharmacokinetic studies showed associations between some pharmacokinetic parameters and age which were no longer apparent after normalisation for body weight. Granisetron is an effective and very well-tolerated antiemetic and appears to be an important addition to the supportive care available for children with cancer. [\hyperlink{Granisetron Hydrochloride}{PMID: 7752999}, A W Craft et al., 1995]

\hypertarget{pmid_8037341}{T}his study was undertaken to evaluate the safety and efficacy of granisetron (a 5-hydroxytryptamine. antagonist) in children with malignant disease who had previously experienced unacceptable nausea and vomiting and/or adverse effects associated with standard antiemetic therapy. Thirty children 3-18 years of age who were receiving anticancer chemotherapy were enrolled in the study. Patients received a prophylactic dose of granisetron before chemotherapy and two subsequent doses as needed. If further antiemetics were required, standard therapy was given and those patients were classified as treatment failures. Patients received granisetron during one to three cycles of chemotherapy; a total of 66 courses were given. Eighty-seven percent of patients had good control of nausea and vomiting with granisetron alone; 90\% of patients elected to receive granisetron with subsequent chemotherapy. No loss of efficacy was noted with repeated cycles in 21 patients. No serious adverse events occurred. Intravenous granisetron (20 micrograms/kg/dose) appears to be a safe and effective drug for pediatric patients receiving emetogenic chemotherapy. [\hyperlink{Granisetron Hydrochloride}{PMID: 8037341}, S J Jacobson et al., 1994]

\hypertarget{pmid_20921907}{A} 1-year-old girl with stage-IV neuroblastoma developed ondansetron hydrochloride anaphylaxis. Safe use of granisetron as an antiemetic after skin prick, oral, and intravenous challenge tests is presented. We present this case to emphasize that ondansetron hydrochloride may cause a serious anaphylactic reaction. In such a case, granisetron may be given to patients as an antiemetic after some provocative tests performed. [\hyperlink{Granisetron Hydrochloride}{PMID: 20921907}, Hacı Ahmet Demir et al., 2010]

\hypertarget{pmid_10087313}{T}he efficacy of granisetron hydrochloride 20 microg/kg and 40 microg/kg were compared using a cross-over method to determine the optimal dose in children with solid tumors receiving high-dose chemotherapy. Granisetron controlled the onset of vomiting in 17 of 23 patients (73.9\%) who were given 40 microg/kg of granisetron, while 8 of 21 patients (38.1\%) were free of vomiting in the 20 microg/kg group. The average frequency of vomiting was 7.22 times in the 20 microg/kg dose versus 4.44 times in the 40 microg/kg dose. No safety problems were associated with either dose. The 40 microg/kg dose of granisetron appears to be more optimal. [\hyperlink{Granisetron Hydrochloride}{PMID: 10087313}, Y Tsuchida et al., 1999]

\hypertarget{pmid_22313076}{T}his study evaluates the effect of prophylactic granisetron on the incidence of postoperative shivering after spinal anaesthesia in children. Eighty children, American Society of Anesthesiologists physical status I to II and aged two to five years were scheduled for surgery of the lower limb under spinal anaesthesia. The children were randomised to receive 10 µg/kg granisetron diluted in 10 ml saline 0.9\% intravenously (group 1, n=40) or placebo (10 ml 0.9\% saline, group 2, n=40) to be given over five minutes just before spinal puncture. Shivering, core temperature and the levels of motor and sensory block were assessed. No patients shivered in group 1. However, six patients shivered in Group 2 (P=0.025). There were no significant differences in the other measured variables between the groups. Granisetron is an effective agent to prevent shivering after spinal anaesthesia in children from two to five years of age. [\hyperlink{Granisetron Hydrochloride}{PMID: 22313076}, Ahmed A Eldaba et al., 2012]

\hypertarget{pmid_9861127}{W}e have studied the efficacy of granisetron, a selective 5-hydroxytryptamine type 3 receptor antagonist, administered orally for the prevention of postoperative vomiting after tonsillectomy in children. In a randomized, double-blind, placebo-controlled study, 160 paediatric patients, ASA 1, aged 4-10 yr, received placebo or granisetron (20, 40 or 80 micrograms kg-1) (n = 40 each) orally, 1 h before surgery. A standard general anaesthetic technique was used throughout. A complete response, defined as no emesis and no need for another rescue antiemetic during the first 24 h after anaesthesia, occurred in 40\%, 48\%, 85\% and 90\% of patients who had received placebo, or granisetron 20, 40 or 80 micrograms kg-1, respectively (P < 0.05; overall Fisher's exact probability test). There were no clinically important adverse events. We conclude that preoperative oral granisetron, in doses more than 40 micrograms kg-1, was effective for the prevention of postoperative vomiting in children. [\hyperlink{Granisetron Hydrochloride}{PMID: 9861127}, Y Fujii et al., 1998]

\hypertarget{pmid_10478183}{G}ranisetron has been used widely for the prevention and treatment of nausea and vomiting associated with anticancer drugs in adult patients with cancer. This multi-center open study was conducted to study the efficacy, safety and usefulness of granisetron in children with cancer. Among 166 evaluable patients, the efficacy rate (percentage of "remarkably effective" or "effective") was 84.9\% and the usefulness rate (percentage of "extremely useful" or "useful") was 87.3\%. No serious adverse effects were observed. As granisetron 40 micrograms/kg had an excellent antiemetic effect and a high degree of safety against nausea and vomiting associated with anticancer drugs, it was shown to be useful for children with cancer. [\hyperlink{Granisetron Hydrochloride}{PMID: 10478183}, T Sawada et al., 1999]

\hypertarget{pmid_17372773}{G}ranisetron is a safe and effective prophylaxis for nausea and vomiting associated with moderate to highly emetogenic chemotherapy. Few trials have been conducted to determine the optimal effective dose of granisetron in children with cancer. The objective of this report was to compare two doses of granisetron in patients with optic pathway tumors receiving moderately emetogenic doses of carboplatin. In this double-blind, crossover, randomized study, antiemetic efficacy and tolerability of two dose levels (10 and 40 microg/kg) of granisetron in the prevention of acute and delayed nausea/emesis were compared in children and young adults. A total of 18 patients (13 boys) aged 1-23 years (median 7.7 years) treated with a moderately emetogenic dose of carboplatin were randomly assigned to receive either 10 or 40 microg/kg of slow granisetron intravenous (i.v.) infusions at alternating cycles of chemotherapy in a blinded fashion until the end of the study period or until their chemotherapy regimen ended. In this way, the patients acted as their own controls. Patients in the granisetron 10 and 40 microg/kg groups received 104 and 121 cycles of chemotherapy, respectively. There was no significant difference in antiemetic efficacy in terms of nausea and emesis between the dose groups in the first 5 days of chemotherapy. The treatment was well tolerated. We conclude that granisetron 10 and 40 microg/kg have comparable efficacy in controlling carboplatin-induced acute and delayed nausea/emesis and is well tolerated in children and young adults. [\hyperlink{Granisetron Hydrochloride}{PMID: 17372773}, Su G Berrak et al., 2007]

\hypertarget{pmid_11210871}{G}ranisetron, a selective 5-hydroxytryptamine type 3 receptor antagonist, is effective for the prevention of vomiting after tonsillectomy in children. Ramosetron (Nasea; Yamanouchi; Tokyo, Japan), another new antagonist of 5-hydroxytryptamione type 3 receptor, has more potent and longer-acting properties than granisetron (Kytril; Smith Kline Beecham, London, UK) against cisplatin-induced emesis. This study was undertaken to compare the efficacy and safety of granisetron and ramosetron for the prevention of vomiting after pediatric tonsillectomy. Prospective, randomized, double-blinded study. Ninety pediatric patients, aged 4 to 10 years, received intravenously granisetron 40 microg/kg or ramosetron 6 microg/kg (n = 45 each) at the end of surgery. The same standard general anesthetic technique and postoperative analgesia were used throughout. Emetic episodes and safety assessment were performed during the first 24-hour period and the next 24-hour period after anesthesia. The rates of patients being emesis-free during the period from 0 to 24 hours after anesthesia were 89\% with granisetron and 93\% with ramosetron, respectively (P = .357); the corresponding rates during the period from 24 to 48 hours after anesthesia were 71\% and 93\%, respectively (P = .006). No clinically serious adverse events attributable to the study drugs were observed in any of the groups. Ramosetron is a better antiemetic than granisetron for the long-term prevention of postoperative vomiting in children undergoing general anesthesia for tonsillectomy. [\hyperlink{Granisetron Hydrochloride}{PMID: 11210871}, Y Fujii et al., 2001]

\hypertarget{pmid_8037342}{I}n a prospective crossover study, we evaluated the safety and antiemetic activity of granisetron, a 5-hydroxytryptamine3 (5-HT3) receptor antagonist, compared with conventional antiemetics regimen, including metoclopramide, in pediatric cancer patients. Twenty-two children with malignant diseases were enrolled. The chemotherapy included cytarabine 3 g/m2 (regimen A), cisplatin 90 mg/m2 (regimen B), and actinomycin D 900 micrograms/m2 plus ifosfamide 3 g/m2 (regimen C). Granisetron 40 micrograms/kg was infused over 30 min just before each chemotherapy treatment. A complete response was obtained more often with granisetron than with conventional antiemetics (59.1\% vs. 0\%, p < 0.001). In terms of efficacy by chemotherapy type, complete response with granisetron was obtained in eight of 10 patients with regimen A, three of eight with regimen B, and two of four with regimen C. Major efficacy (vomiting fewer than two times) was also obtained more with granisetron than with conventional antiemetics (81.8\% vs. 4.6\%, p < 0.001). The number of vomiting episodes in the first 24 h was less with granisetron than with conventional antiemetics (1.1 +/- 1.46 vs. 9.0 +/- 4.97, p < 0.001). Normal appetite and activity were retained in more patients with granisetron than with conventional antiemetics. Extrapyramidal reactions, akathisia, and sedation were not seen in any case with granisetron. Granisetron 40 micrograms/kg is well tolerated and more effective than are conventional antiemetic regimens containing metoclopramide for children receiving cancer chemotherapy. [\hyperlink{Granisetron Hydrochloride}{PMID: 8037342}, Y Miyajima et al., 1994]

\hypertarget{pmid_10925688}{W}e investigated the antiemetic effect, safety and usefulness of granisetron hydrochloride tablets on nausea and vomiting induced by oral anticancer drugs used in chemotherapy for gastric cancer and colorectal cancer. In the present trial, oral administration of granisetron hydrochloride was performed during 5 days after nausea or vomiting. 1) Clinically, the effective rate of granisetron hydrochloride (the percentage of cases in which the drug was assessed as "Remarkably effective" or "Effective") was more than 75\% on each day of administration. There were no adverse events or abnormal laboratory tests. 2) In terms of usefulness, granisetron hydrochloride was rated "Extremely useful" or "Useful" in 17 out of 23 cases (78.2\%). The above results have shown that granisetron hydrochloride tablets, administrated orally once daily at a dose of 2 mg, have an excellent antiemetic effect, and that this is a safe and useful drug. [\hyperlink{Granisetron Hydrochloride}{PMID: 10925688}, S Togo et al., 2000] 5-HT3 receptor antagonists, including granisetron and ondansetron, are widely used in the prophylactic treatment of chemotherapy-induced nausea and vomiting. Although the cardiac safety of granisetron and ondansetron has been investigated in several adult studies, there is no report investigating the effects of those agents on electrocardiography (ECG) in children. The effects of intravenously infused (over 30 seconds) 0.1 mg/kg ondansetron and 40 microg/kg granisetron on ECG were assessed in 22 children receiving high-dose methotrexate therapy for acute lymphoblastic leukemia. The ECG recording was obtained at before and just after the infusion, and repeated at 1, 3, 6, and 24 hours of treatment. Granisetron administration resulted in a statistically significant decrease of mean heart rate at 1 and 3 hours, and significant prolongation of mean QT and QTc dispersions at 1 hour of infusion. In patients treated with ondansetron, no meaningful change was observed. In conclusion, intravenous granisetron but not ondansetron causes clinically asymptomatic and transient changes on ECG measurements in children receiving high-dose methotrexate therapy. [\hyperlink{Granisetron Hydrochloride}{PMID: 10925688}, Mustafa Buyukavci et al., 2005]

\hypertarget{pmid_8546476}{W}e, in the Department of Obstetrics and Gynecology, Kansai Medical College, conducted an evaluation of the usefulness and safety of granisetron hydrochloride used for nausea and vomiting due to chemotherapy in patients with gynecological malignant tumors, with an additional study of the efficacy of different regimens. The subjects were 9 patients in whom 16 courses of CAP therapy were given (group A) and 13 patients in whom 24 courses of CAP therapy were given (group B). Granisetron hydrochloride 3 mg/body was administered by intravenous drip in the two groups before chemotherapy. Clinical symptoms of nausea, vomiting, and anorexia were observed for 2 days after anticancer drugs were administered in order to evaluate its efficacy. The percentage of patients who responded as "effective" or better was 90.0\%. In different regimens, the efficacy was 93.8\% in group A and 87.5\% in group B. These results indicated clinically high usefulness in both groups. No side effects related to granisetron hydrochloride were found in this study. [\hyperlink{Granisetron Hydrochloride}{PMID: 8546476}, M Kitada et al., 1996]

\hypertarget{pmid_10084103}{T}his study was undertaken to determine the minimum effective dose of granisetron, 5-hydroxytryptamine type 3 receptor antagonist, for the prevention of post-operative vomiting in children undergoing general inhalational anaesthesia for surgery (inguinal hernia and phimosis). In a randomized, double-blind manner, 120 children, ASA physical status I, aged 4-10 years, were assigned to receive placebo (saline) or granisetron at three different doses (20 micrograms kg-1, 40 micrograms kg-1, 100 micrograms kg-1) intravenously immediately after inhalation induction of anaesthesia (n = 30 of each). A complete response, defined as no emesis and no need for another rescue antiemetic during the first 24 h after anaesthesia, occurred in 57\% with placebo, 67\% with granisetron 20 micrograms kg-1, 90\% with granisetron 40 micrograms kg-1 and 90\% with granisetron 100 micrograms kg-1 respectively (P < 0.05; overall Fisher's exact probability test). No clinically important adverse events were observed in any of the groups. Our results suggest that granisetron 40 micrograms kg-1 is the minimum effective dose for the prevention of emesis after paediatric surgery, and that increasing its dose to 100 micrograms kg-1 provides no demonstrable benefit. [\hyperlink{Granisetron Hydrochloride}{PMID: 10084103}, Y Fujii et al., 1999]

\hypertarget{pmid_7844684}{T}o compare the efficacy and safety of intravenously administered granisetron with those of chlorpromazine plus dexamethosone in the prevention of ifosmamide-induced emesis in children with malignant disease. Eighty-eight children, aged 2 to 17 years, were scheduled for ifosfamide therapy (> or = 3 gm/m2) for 2 or 3 consecutive days. On each day, children received granisetron, 20 microgram/kg intravenously, before ifosfamide therapy, plus up to two more doses within 24 hours if required, or chlorpromazine, 0.3 to 0.5 mg/kg intravenously, every 4 to 6 hours, plus dexamethasone, 2 mg/m2 intravenously every 8 hours. During the initial 24 hours, significantly fewer episodes of vomiting were recorded after granisetron administration (median number, 1.5 vs 7.0; p = 0.001), and the percentages of children having no more than one vomiting episode (51\% granisetron vs 21\% chlorpromazine-dexamethasone) and no worse than mild nausea (67\% granisetron vs 38\% chlorpromazine-dexamethasone) were lower after granisetron therapy (p < 0.01). Fewer children had sedation with granisetron (2 vs 19; p < 0.001); there were no extrapyramidal reactions during granisetron therapy compared with two during control therapy. Granisetron was superior to chlorpromazine-dexamethasone antiemetic therapy for children receiving ifosfamide therapy and deserves further study during other chemotherapy regimens. [\hyperlink{Granisetron Hydrochloride}{PMID: 7844684}, K Hählen et al., 1995]

\hypertarget{pmid_10434165}{T}his study was undertaken to compare the efficacy and safety of granisetron, a 5-hydroxytryptamine type 3 receptor antagonist, and dexamethasone and each drug alone for the prevention of post-operative vomiting by children, with no history of motion sickness and/or previous post-operative vomiting, undergoing general inhalational anaesthesia for surgery (inguinal hernia and phimosis). In a randomized, double-blind manner, 150 children, ASA physical status 1, aged 4-10 years, were assigned to receive granisetron 40 mg kg-1, dexamethasone 150 mg kg-1, or granisetron 40 mg kg-1 plus dexamethasone 150 mg kg-1 intravenously immediately after inhalation induction of anaesthesia (n = 50 of each). A complete response, defined as no emesis and no need for another rescue anti-emetic during the first 24 h after anaesthesia, was 86\% with granisetron, 68\% with dexamethasone and 98\% with granisetron plus dexamethasone, respectively (P < 0.05; overall Fisher's exact probability test). No clinically serious adverse events were observed in any of the groups. In conclusion, prophylactic therapy with combined granisetron and dexamethasone was more effective than was each anti-emetic alone for the prevention of vomiting after paediatric surgery. [\hyperlink{Granisetron Hydrochloride}{PMID: 10434165}, Y Fujii et al., 1999]

\hypertarget{pmid_9512674}{T}he safety and efficacy of granisetron (10 micrograms/kg and 40 micrograms/kg) were evaluated during a second (n = 393) and third (n = 200) cycle of chemotherapy in this multicenter, double-blind, randomized, parallel-group study. Granisetron was administered as a single intravenous dose before the start of cisplatin chemotherapy (> or = 60 mg/m2). Total control (no vomiting, no retching, no nausea, and no use of antiemetic rescue medication) after the first 24 hr following chemotherapy was achieved in 40\% and 49\% of patients in Cycles 2 and 3, respectively, for the 10 micrograms/kg group, and in 42\% and 38\% of patients in Cycles 2 and 3, respectively, for the 40 micrograms/kg group. Both dose levels of granisetron were well tolerated. The results demonstrate comparable efficacy between the 10 micrograms/kg and 40 micrograms/kg doses of granisetron in preventing nausea and vomiting during repeat cycles of high-dose cisplatin-based chemotherapy. The results of this study show that granisetron 10 micrograms/kg is safe and well tolerated, and remains effective with repeat cycle use. [\hyperlink{Granisetron Hydrochloride}{PMID: 9512674}, H L Ritter et al., 1998]

\hypertarget{pmid_25813292}{T}he objective of this study was to elucidate the possible toxic effects on the fetal tissues after exposure to two clinically relevant concentrations of granisetron. Primary cells were isolated from human fetal organs of 16-19 weeks gestational age and treated with 3 ng/mL or 30 ng/mL of granisetron. Cell cycle progression was evaluated by flow cytometry. ELISA was used to detect alterations in major apoptotic proteins. Up to 10\% apoptosis in cardiac tissue was observed following treatment with 30 ng/mL granisetron. Neither concentration of granisetron caused alteration in cell cycle progression or alterations in apoptotic proteins in any of the other tissues. At 30 ng/mL granisetron concentration had the potential to induce up to 10\% apoptosis in cardiac tissue; clinical significance needs further evaluation. At granisetron 3 ng/mL there was no detectable toxicity or on any fetal tissue in this study. Further research is needed to confirm these preliminary findings and determine if clinically significant. [\hyperlink{Granisetron Hydrochloride}{PMID: 25813292}, Judith A Smith et al., 2015]

\hypertarget{pmid_10533454}{T}his randomised study was performed to assess the anti-emetic efficacy and tolerability of two-dose regimens of granisetron in children with leukaemia. 49 children with leukaemia were treated with three consecutive courses of high-dose methotrexate or cytarabine regimen. During the first course, patients were evaluated regarding the emetogenicity of each regimen. They were randomised in a crossover manner to receive 20 or 40 micrograms/kg of granisetron before the second and third course of chemotherapy. Neither emesis nor severe appetite loss were observed in over 80\% of patients within the first 24 h in all treatment groups. There was no significant difference in the anti-emetic efficacy between the two-dose regimens of granisetron. However, complete protection was achieved less frequently on days 2 and 3. Older children and girls appeared to be less well protected. No adverse events attributable to granisetron were observed. Granisetron dose regimens of 20 and 40 micrograms/kg are, comparably, well tolerated and effective in controlling chemotherapy-induced emesis in the first 24 h, though this protection fails thereafter, particularly in older patients and girls. [\hyperlink{Granisetron Hydrochloride}{PMID: 10533454}, Y Komada et al., 1999]

\hypertarget{pmid_7552896}{T}he stability and sterility of granisetron hydrochloride in 5\% dextrose injection or 0.9\% sodium chloride injection when stored in a disposable elastomeric infusion device were studied. Granisetron was diluted to 0.02 mg/mL (as the hydrochloride salt) in 5\% dextrose chloride injection. The solution was placed in the drug reservoir of a disposable elastomeric infusion device and refrigerated at 4 degrees C for 14 days. A total of eight pumps were prepared, four containing granisetron 0.02 mg/mL in 5\% dextrose injection and four containing granisetron 0.02 mg/mL in 0.9\% sodium chloride injection. The solutions were assayed for granisetron concentration by stability-indicating high-performance liquid chromatography at 0 hours, 24 hours, 48 hours, 7 days, and 14 days. Each solution was inspected for clarity, color, and precipitation, and sterility testing was performed. Throughout the study, the mean concentration of granisetron remaining was more than 92\% of the initial concentration both in 5\% dextrose injection and in 0.9\% sodium chloride injection. Individual solutions in 0.9\% sodium chloride injection consistently maintained more than 90\% of the initial drug concentration for only seven days. No microbial growth was detected. No precipitation, color change, or haziness was seen. Granisetron 0.02 mg/mL (as the hydrochloride salt) was stable and free of microbial growth in 0.9\% sodium chloride injection for up to 7 days and stable and free of microbial growth in 5\% dextrose injection for up to 14 days when stored at 4 degrees C in a disposable elastomeric infusion device. [\hyperlink{Granisetron Hydrochloride}{PMID: 7552896}, K C Chung et al., 1995]

\hypertarget{pmid_9117791}{T}he compatibility of granisetron hydrochloride with selected other drugs during simulated Y-site administration was studied. Five milliliters of granisetron 50 micrograms/mL (as the hydrochloride) in 5\% dextrose injection was combined with 5 mL of each of 91 secondary additives, including antineoplastics, anti-infectives, and supportive care drugs, in 5\% dextrose injection or (if necessary to avoid incompatibility with the diluent) 0.9\% sodium chloride injection. Visual examinations were performed with the unaided eye in fluorescent light and in high-intensity monodirectional light to enhance visualization of small particles and low-level turbidity. The turbidity of each solution was measured as well. Particle sizing and counting were performed for selected solutions. Evaluations were performed initially and at one and four hours. Nearly all the test drugs were compatible with granisetron during the four-hour observation period. The granisetron-amphotericin B combination had an unacceptable increase in turbidity upon being mixed. During simulated Y-site administration, granisetron 50 micrograms/mL (as the hydrochloride) in 5\% dextrose injection was compatible with 90 of 91 drugs and combination drugs for four hours at room temperature; the exception was amphotericin B. [\hyperlink{Granisetron Hydrochloride}{PMID: 9117791}, L A Trissel et al., 1997]

\hypertarget{pmid_8807169}{T}his study was to identify the minimum effective dose of granisetron, a selective 5-hydroxytryptamine type 3 receptor antagonist, to prevent postoperative vomiting in children who have undergone strabismus repair, tonsillectomy or tonsillectomy with adenoidectomy. In a randomized, double-blind fashion, 80 healthy children aged 4-10 yr were assigned to receive either placebo (saline) or granisetron in a dose of 20, 40 or 80 micrograms.kg-1 iv immediately following the induction of anaesthesia. All subjects received a standardized anaesthetic, which consisted of sevoflurane in nitrous oxide and oxygen. Rescue antiemetics were administered if two or more episodes of vomiting occurred. Postoperative pain was treated with acetaminophene pr or pentazocine iv. During the first 24 hr after anaesthesia, the frequencies of retching and vomiting were recorded in a standardized fashion by nursing staff while subjects were in a hospital. There were no differences among four treatment groups with regard to subject characteristics, surgical procedures, anaesthetic and postoperative management or adverse effects. The frequencies of these symptoms were as follows: 65\%, 60\%, 20\% and 15\% after administration of placebo, granisetron 20, 40 or 80 micrograms.kg-1. Three children who had received either placebo or granisetron 20 micrograms.kg-1 required another rescue antiemetic drug, whereas none who had received granisetron 40 or 80 micrograms.kg-1 needed rescue drugs. Granisetron 40 micrograms.kg-1 is an effective antiemetic for preventing retching and vomiting following strabismus repair and tonsillectomy in children. Increasing the dose to 80 micrograms.kg-1 provided no demonstrable benefit in reducing postoperative emesis. [\hyperlink{Granisetron Hydrochloride}{PMID: 8807169}, Y Fujii et al., 1996]

\hypertarget{pmid_11903942}{W}e evaluated the efficacy of granisetron, 5-hydroxytryptamine type 3 receptor antagonist, given orally, preoperatively, for the prevention of postoperative vomiting in children undergoing general anaesthesia for surgery (inguinal hernia, phimosis-circumcision). In a randomized, double-blinded manner, 100 children, ASA physical status I, aged 4-11 years, received orally placebo or granisetron at three different doses (20 microg x kg(-1), 40 microg x kg(-1), 80 microg x kg(-1)) 60 min before surgery (n=25 of each). The same standard general anaesthetic technique was used. The percentage of patients being emesis-free during 0-6 h after anaesthesia was 56\% with placebo, 64\% with graniseron 20 microg x kg(-1) (P=0.773), 88\% with granisetron 40 microg x kg(-1) (P=0.027) and 92\% with granisetron 80 microg x kg(-1) (P=0.01); the corresponding rate during 6-24 h after anaesthesia was 60\%, 68\% (P=0.768), 92\% (P=0.02) and 92\% (P=0.02) (P-values versus placebo). No clinically serious adverse events were observed in any of the groups. In summary, preoperative oral granisetron 40 microg x kg(-1) is effective for the prevention of vomiting following paediatric surgery (inguinal hernia, phimosis-circumcision). Increasing the doses to 80 microg x kg(-1) provides no demonstrable additional benefit. [\hyperlink{Granisetron Hydrochloride}{PMID: 11903942}, Yoshitaka Fujii et al., 2002]

\hypertarget{pmid_9233110}{A} high incidence of vomiting after tonsillectomy is consistently reported in the literature. Multiple trials with different pharmacological agents and alternative anesthetic techniques have been used in an attempt to decrease this, but results have been inconclusive and adverse effects occur frequently. The purpose of this study was to examine the safety and efficacy of granisetron (Kytril) in the prevention of postoperative vomiting in children undergoing tonsillectomy. Fifty-four patients were included in this study. The study population was divided into an experimental (n = 28) and control group (n = 26) using a randomized double-blinded technique. The experimental group was treated with granisetron 10 micrograms/kg intravenously (IV) and the control group received saline solution IV. Episodes of vomiting were recorded throughout the hospital stay and for the 24 hours after surgery. A reduction in the incidence and severity of vomiting was shown to be significant in the experimental group, both in the hospital and at 24 hours postoperatively. High patient satisfaction was also reported in this group. Clinically, the prophylactic administration of granisetron intraoperatively in pediatric tonsillectomy patients should be considered to reduce the incidence and severity of postoperative vomiting. [\hyperlink{Granisetron Hydrochloride}{PMID: 9233110}, D Carnahan et al., 1997]

\section*{Ifosfamide}
\subsection*{Result}
\subsubsection*{Answer}

Infants and Young Children (<5 years): No  
School-Aged Children and Adolescents (5–18 years): No  
General Pediatric Population (up to 18 years): No  

\subsubsection*{{Explanation}}
\hypertarget{Ifosfamide}
A review of the available abstracts reveals that Ifosfamide has been extensively studied in children, including infants, toddlers, school-aged children, and adolescents, primarily as a chemotherapeutic agent for various pediatric cancers. The studies consistently report that Ifosfamide is effective in treating pediatric tumors, but its use is associated with significant risks, particularly nephrotoxicity (kidney damage) and neurotoxicity (brain/nerve damage), with the risk and severity often related to cumulative dose and younger age at treatment.

Key findings by age group:

Infants and Young Children (<5 years, including infants and toddlers):
- Multiple studies specifically address safety in this age group. Nephrotoxicity, especially proximal tubular toxicity, is more severe in children under 5 years, and particularly under 4 years, compared to older children. This toxicity can lead to serious complications such as rickets and impaired growth and development [\hyperlink{pmid_16628552}{PMID: 16628552}, W Stöhr et al., 2007; \hyperlink{pmid_1503924}{PMID: 1503924}, R Skinner et al., 1992; \hyperlink{pmid_8774570}{PMID: 8774570}, R Skinner et al., 1996; \hyperlink{pmid_23912821}{PMID: 23912821}, Arul P Lionel et al., 2014]. Neurotoxicity, including encephalopathy and, in rare cases, irreversible brain damage, has also been reported in infants [\hyperlink{pmid_8058012}{PMID: 8058012}, C S Bruggers et al., 1994]. The risk of severe, sometimes permanent, toxicity is higher in this age group, especially at higher cumulative doses.

School-Aged Children and Adolescents (5–18 years):
- Ifosfamide is widely used in this group, with studies showing moderate to severe nephrotoxicity in a minority of patients, especially at higher cumulative doses (>60 g/m\textasciicircum{}2) [\hyperlink{pmid_19826134}{PMID: 19826134}, Odile Oberlin et al., 2009; \hyperlink{pmid_16628552}{PMID: 16628552}, W Stöhr et al., 2007; \hyperlink{pmid_8774570}{PMID: 8774570}, R Skinner et al., 1996]. Neurotoxicity, including encephalopathy and seizures, is also reported but is usually reversible [\hyperlink{pmid_33779369}{PMID: 33779369}, Eda Ataseven et al., 2021; \hyperlink{pmid_19953647}{PMID: 19953647}, Bethany Ames et al., 2010]. Most studies recommend careful monitoring and dose limitation to reduce risk.

General Pediatric Population (up to 18 years):
- Across all ages, Ifosfamide is considered an important and effective chemotherapeutic agent, but its use is limited by the risk of nephrotoxicity and neurotoxicity. The risk is dose-dependent and higher in younger children. Some studies report that with appropriate monitoring, dose adjustment, and supportive care (e.g., mesna for bladder protection, hydration, and possibly N-acetylcysteine for kidney protection), Ifosfamide can be administered safely in many children, but the risk of serious, sometimes permanent, toxicity remains [\hyperlink{pmid_2503259}{PMID: 2503259}, S M Davies et al., 1989; \hyperlink{pmid_8157032}{PMID: 8157032}, M S Ashraf et al., 1994; \hyperlink{pmid_14586158}{PMID: 14586158}, M Carli et al., 2003].

Summary:
- There are targeted studies on the safety of Ifosfamide in children, including infants, toddlers, and adolescents. These studies affirm that while Ifosfamide is effective and can be used in children, it carries a significant risk of serious toxicity, especially in younger children and at higher doses. The studies do not affirm that Ifosfamide is "safe" in the sense of being free from significant risk; rather, they highlight the need for careful risk-benefit assessment, monitoring, and dose limitation. Therefore, based on the evidence in the abstracts, Ifosfamide cannot be considered "safe" for use in children as defined by the criteria (i.e., a targeted study affirming safety).

\subsection*{Abstracts}
\hypertarget{pmid_1411617}{I}fosfamide is an effective drug for the treatment of solid tumors in children and most blastomas and sarcomas react favorably. It is administered in combination chemotherapy protocols. In young children, nephrotoxicity is a serious side effect. In older boys, gonadal damage is the major late side effect. [\hyperlink{Ifosfamide}{PMID: 1411617}, P A Voûte et al., 1992]

\hypertarget{pmid_2758568}{I}fosfamide, alone or in combination, is used in a variety of childhood tumours. Soft-tissue sarcomas are especially sensitive, with a 78\% actuarial survival. Hyperaminoaciduria and a brief transient decrease in the plasma level of certain amino acids are the earliest signs of tubular toxicity. [\hyperlink{Ifosfamide}{PMID: 2758568}, J de Kraker et al., 1989]

\hypertarget{pmid_16628552}{I}fosfamide is widely used in paediatric oncology, but its use is limited by nephrotoxic side effects. The aim of this study was to evaluate the incidence and risk factors of tubulopathy, with special emphasis on the influence of age, where different findings have been published so far. Five hundred ninety three children and adolescents treated for Ewing, osteo- or soft-tissue sarcoma (median age at diagnosis: 11.7 years) were prospectively investigated for nephrotoxicity in the Late Effects Surveillance System (LESS) study. Tubulopathy was diagnosed in case of continuing hypophosphatemia and proteinuria. After a median follow up of 19 months, 27 patients (4.6\%; 95\% CI: 3.0-6.6\%) had newly developed tubulopathy. This incidence was 0.4\% (95\% CI: 0-2.4\%) in patients treated with a cumulative ifosfamide dose of < or =24 g/m2, 6.5\% (95\% CI: 3.6-10.7\%) after 24-60 g/m2, and 8.0\% (95\% CI: 4.2-13.6\%) after > or = 60 g/m2. In multivariate analysis, children younger than 4 years at time of diagnosis had an 8.7-fold (95\% CI: 3.5-21.8) higher risk for tubulopathy than older patients. Neither carboplatin treatment nor abdominal irradiation showed any significant influence. Ifosfamide-induced nephrotoxicity was found in 4.6\% of patients. Risk factors were the cumulative ifosfamide dose and young age at treatment. [\hyperlink{Ifosfamide}{PMID: 16628552}, W Stöhr et al., 2007]

\hypertarget{pmid_9673760}{I}fosfamide is an active drug in the therapy of paediatric tumours such as rhabdomyosarcoma, Ewings' sarcoma, Wilms' tumour, neuroblastoma, germ cell tumours and lymphomas. Myelosuppression is the major toxicity along with haemorrhagic cystitis. The latter is largely prevented by the use of concomitant mesna. [\hyperlink{Ifosfamide}{PMID: 9673760}, S H Advani et al., 1998]

\hypertarget{pmid_1810941}{I}n oncology, particularly in pediatric malignancies, high doses (5-10 g/m2) of the oxazaphosphorine ifosfamide play an important role in the treatment of sarcomas. Pharmacokinetic data of ifosfamide and its metabolites in these cases are scanty. Considering the special demands of the determination of ifosfamide in plasma of young children, a very sensitive capillary gas chromatographic method, requiring only 50 microliters of plasma, has been developed. This bioanalysis of ifosfamide shows good linearity and accuracy in the concentration range 10 ng to 100 micrograms per ml of plasma and 25 ng to 1 mg per ml of urine. The absolute limits of detection in plasma and urine are 2 ng/ml and 5 ng/ml, respectively. The stability of various solutions of ifosfamide and trofosfamide was tested and proved to be satisfactory, except for ifosfamide in plasma and urine kept in the refrigerator. The validity of the method for pharmacokinetic purposes is shown in the case of one patient. [\hyperlink{Ifosfamide}{PMID: 1810941}, G P Kaijser et al., 1991]

\hypertarget{pmid_14586158}{P}hase II studies conducted in Europe and the USA on pediatric solid tumors have shown that ifosfamide, as a single agent, is an active drug against a variety of neoplasms - rhabdomyosarcoma (RMS), some non-RMS soft tissue sarcomas, Wilms' tumor, bone sarcomas and neuroblastoma. Furthermore, an increase in tumor response rate has been observed when ifosfamide has been used in combination with other drugs. The usual dose of ifosfamide varies from 1.8 to 3 g/m(2)/day for 2-5 days according to the different regimens. Some controversies still exist on the modality of drug administration and more precisely on the time of infusion, however in pediatric practice, short infusion (e.g. 3 h) is usually preferred because of the reduced neurotoxicity in comparison to lengthier administration (e.g. 24 h). Ifosfamide is currently included in the standard therapy of pediatric bone and soft tissue sarcomas. It is also used in a selected high-risk group of patients with Wilms' tumor, neuroblastoma and germ cell tumors. [\hyperlink{Ifosfamide}{PMID: 14586158}, M Carli et al., 2003]

\hypertarget{pmid_33779369}{I}fosfamide is an alkylating agent, mostly used against variety of solid tumors in pediatric oncology practice. Although hemorrhagic cystitis is known as a common adverse effect, encephalopathy is the another one that should be kept in mind. It may occur in 2-5\% of the children, and manifested by different clinical spectrums such as somnolence, lethargy, irritability, excitement, disorientation, confusion, weakness, hallucinations, seizures, movement disorders, and coma. Herein, we present two patients who developed generalized seizure activity and one who developed coma during ifosfamide infusion. These cases highlight that pediatric oncologists and hematologists should be aware of possibility of severe neurological toxicity after administration of ifosfamide in adolescent patients. Apart from seizure, clinicians should also be prepared to notice drowsiness during ifosfamide infusions in children. Most of the time cessation of ifosfamide and hydration is enough. However, in severe toxicities there is a risk of irreversible neurological damage, and for these patients methylene blue (MB) and thiamine treatment should be kept in mind. [\hyperlink{Ifosfamide}{PMID: 33779369}, Eda Ataseven et al., 2021]

\hypertarget{pmid_9606250}{I}fosfamide is widely used in the treatment of pediatric solid tumors. Its main adverse effects are various forms of renal tubular and glomerular damage. Many risk factors have been proposed to play a role in the development and severity of nephrotoxicity in children receiving ifosfamide, among which are 1) patient's age, 2) cumulative ifosfamide dose, 3) concurrent administration of cis or carboplatinum, 4) unilateral nephrectomy, and 5) method of ifosfamide administration. However, presently there is no consensus regarding the weight of each one of them. Therefore, we critically reviewed the major studies that have evaluated the different risk factors in an attempt to determine the relative importance of each. Cumulative ifosfamide doses of >/=60 g/m appears to be the most consistent independent predictor for both the development and the severity of nephrotoxicity, whereas a younger age (<5 years of age) was associated primarily with the more severe and chronic forms of proximal tubulopathy. Comparable incidence and severity forms of proximal tubulopathy among children who had been treated with cis platinum in addition to ifosfamide and those who had not indicate that platinums probably potentiate ifosfamide-induced renal damage rather than act as a major independent risk factor. Finally, although unilateral nephrectomy has been proposed as a significant risk factor in different studies, the relatively small number of nephrectomized children in these cohorts limit the strength of this association. To reduce the frequency and severity of ifosfamide-induced nephrotoxicity, it appears that cumulative doses of 60 g/m should be considered carefully, especially in children <5 years of age. [\hyperlink{Ifosfamide}{PMID: 9606250}, R Loebstein et al., 1998]

\hypertarget{pmid_2503259}{I}fosfamide has been shown to be an active agent in the treatment of several childhood cancers. However, the optimal dose and method of administration remains to be established. The dose/response relationship of ifosfamide suggests that a maximum tolerable, fractionated dose be given, and to reduce hospitalisation this dose should be given in the shortest possible time. A total of 20 patients aged 1-23 years received 124 courses (mean, 6 courses/patient; range, 1-16); 9 subjects had either relapsed or resistant disease, and all of these had previously received cyclophosphamide. A dose of 3 g/m2 ifosfamide was given for 2 (five patients) or 3 (15 patients) successive days. In all, 9 patients received the drug twice daily as a bolus and 11 were given a continuous infusion. All patients received 3 g/m2 mesna per day with ifosfamide and for 12 h there after, and hydration was maintained with 3 l/m2 fluid daily. Myelosuppression occurred in all patients but was mild and reversible, with no toxic deaths. On four occasions in three patients treatment had to be delayed due to myelosuppression. Seven episodes of fever and neutropaenia were successfully treated with antibiotics. The mean glomerular filtration rate in 13 patients at the start of treatment was 104 ml/min per 1.73 m2 and at the end was 92 ml/min per 1.73 m2. In all, 19 patients had microscopic and 1 macroscopic haematuria, with no clinical sequelae. Two patients with grossly impaired renal function following previous cisplatin therapy may have been precipitated into terminal renal failure by the ifosfamide therapy. Only one person developed neurotoxicity, which recurred on further treatment with ifosfamide but was fully reversible. All patients had moderate to severe vomiting, which was controlled with anti-emetics. No abnormalities of liver or cardiac function were detected. We conclude that ifosfamide given by this schedule is safe in patients with normal renal function. [\hyperlink{Ifosfamide}{PMID: 2503259}, S M Davies et al., 1989]

\hypertarget{pmid_17641745}{P}urpose. Ifosfamide is a drug commonly used in the management of sarcomas and other solid tumours. One potential toxicity of its use is renal tubular damage, which can lead to skeletal abnormalities; rickets in children and osteomalacia in adults. We aimed to characterise this rare complication in adults. Patients. Three illustrative patient cases treated in our institution are presented. All were treated for sarcoma, and received varying doses of ifosfamide during their therapy. Methods. We performed a review of the literature on the renal tubular and skeletal complications of ifosfamide in adults. Papers were identified by searches of PubMed using the terms "osteomalacia," "nephrotoxicity," "Fanconi syndrome," "ifosfamide," and "chemotherapy" for articles published between 1970 and 2006. Additional papers were identified from review of references of relevant articles. Results. There are only four case reports of skeletal toxicity secondary to ifosfamide in adults; the majority of data refer to children. Risk factors for development of renal tubular dysfunction and osteodystrophy include platinum chemotherapy, increasing cumulative ifosfamide dose, and reduced nephron mass. The natural history of ifosfamide-induced renal damage is variable, dysfunction may not become apparent until some months after treatment, and may improve or worsen with time. Discussion. Ifosfamide-induced osteomalacia is seldom described in adults. Clinicians should be vigilant for its development, as timely intervention may minimise complications. [\hyperlink{Ifosfamide}{PMID: 17641745}, D N Church et al., 2007]

\hypertarget{pmid_23912821}{I}fosfamide is commonly used as a chemotherapeutic agent in children. The authors report a 4-y-old boy who developed proximal renal tubulopathy with florid rickets a year after completion of ifosfamide therapy for Ewing's sarcoma. After initiation of treatment, there was complete healing of rickets and he did not need supplements beyond 18 mo. Growth monitoring and musculoskeletal system examination is important in all children who have received ifosfamide therapy. Routine monitoring for nephrotoxicity during and after ifosfamide therapy helps in early identification and intervention. [\hyperlink{Ifosfamide}{PMID: 23912821}, Arul P Lionel et al., 2014]

\hypertarget{pmid_19826134}{I}fosfamide is widely used in pediatric oncology but its nephrotoxicity may become a significant issue in survivors. This study is aimed at evaluating the incidence of late renal toxicity of ifosfamide and its risk factors. Of the 183 patients prospectively investigated for renal function, 77 treated for rhabdomyosarcoma, 39 for other soft tissue sarcoma, 39 for Ewing's sarcoma, and 28 for osteosarcoma were investigated at least 5 years after treatment. No patients had received cisplatin and/or carboplatin. Glomerular and tubular functions were graded according to the Skinner system. The median dose of ifosfamide was 54 g/m(2) (range, 18 to 117 g/m(2)). After a median follow-up of 10 years, 89.5\% of patients had normal tubular function, and 78.5\% had normal glomerular function rate (GFR). Serum bicarbonate and calcium were normal in all patients. Hypomagnesemia was observed in 1.2\% and hypophosphatemia in 1\%. The tubular threshold for phosphate was reduced in 24\% of the patients (grade 1 in 15\%, grade 2 in 8\%, and grade 3 in 0.5\%). Glycosuria was detected in 37\% of the patients but was more than 0.5 g/24 hours in only 5\%. Proteinuria was observed in 12\%. Ifosfamide dose and interval from therapy to investigations were predictors of tubulopathy in univariate and multivariate analysis. In a multivariate analysis, an older age at diagnosis and the length of interval since treatment had independent impacts on the risk of abnormal GFR. Renal toxicity is moderate with a moderate dose of ifosfamide. However, since it can be permanent and can get worse with time, repeated long-term evaluations are important, and this risk should be balanced against efficacy. [\hyperlink{Ifosfamide}{PMID: 19826134}, Odile Oberlin et al., 2009]

\hypertarget{pmid_9815588}{A}lthough both cyclophosphamide (CP) and ifosfamide (IF) are used in the treatment of central nervous system tumors, little is known about the concentration of either drug or their metabolites in the cerebrospinal fluid (CSF) of children. The concentrations of the parent oxazaphosphorine and its principal metabolites were measured simultaneously in the plasma and CSF of 25 children. Twenty-one patients received CP for the treatment of either acute lymphoblastic leukemia, non-Hodgkin's lymphoma, or medulloblastoma, and 4 children received IF for the treatment of rhabdomyosarcoma. A high degree of interpatient variation was seen in terms of the CSF concentration of CP and the CSF:plasma ratio. The CSF:plasma ratio was greater for IF than for CP (P < 0.001). In contrast to IF, where the majority of metabolites was measured in the CSF, no child receiving CP had detectable metabolites. Children receiving dexamethasone had lower concentrations of CP in the CSF (P = 0.04). The CSF:plasma ratio for isophosphoramide mustard was greater than that for either parent drug or any other metabolite. These results demonstrate that IF enters the CSF to a greater extent than CP in children. The ability of both IF and CP and their metabolites to cross the blood-brain barrier may be reduced by dexamethasone. [\hyperlink{Ifosfamide}{PMID: 9815588}, S M Yule et al., 1997]

\hypertarget{pmid_1503924}{N}ephrotoxicity is an important adverse effect of chemotherapy in children. Renal function after treatment with either ifosfamide or cisplatinum in children aged 5 years or less ('younger children') was compared with that in those over 5 years ('older children'). Eighteen children (six younger, 12 older) given ifosfamide were studied after completion of chemotherapy, and 28 patients (16 younger, 12 older) were evaluated after cisplatinum. Glomerular filtration rate was measured from the plasma clearance of 51chromium-labelled edetic acid. Proximal tubular function was assessed by determination of plasma and urine calcium, phosphate, magnesium and glucose concentration; calculations of their fractional excretions, and of the renal threshold for phosphate; and measurement of urinary excretion of beta 2-microglobulin. Distal tubular function was evaluated by measurement of the early morning urine osmolality. Younger children had more severe proximal tubular toxicity than older children treated with ifosfamide, with significantly lower plasma phosphate concentrations and higher fractional excretions of glucose. However, there was no evidence of any such difference in glomerular or distal tubular damage after ifosfamide, and no difference in any aspect of renal function between younger and older children treated with cisplatinum. Increased proximal tubular toxicity after ifosfamide in younger children may have serious implications for future growth and development. [\hyperlink{Ifosfamide}{PMID: 1503924}, R Skinner et al., 1992]

\hypertarget{pmid_8348510}{I}fosfamide has previously been shown to be active as a single agent and in combination with doxorubicin, etoposide, and teniposide in pediatric solid tumors and adult acute leukemia. The authors performed a dose-escalation trial of ifosfamide with a fixed dosage of etoposide, with mesna uroprotection, in children with multiply recurrent acute leukemia. Chemotherapy was administered daily for 5 days. Etoposide 100 mg/m2 was followed by ifosfamide at an initial dosage of 1.6 g/m2. The ifosfamide was escalated in 20\% increments to the maximum tolerated dosage in cohorts of three patients. Mesna 400 mg/m2 was given immediately before the ifosfamide and then at 3 and 6 hours after ifosfamide in the initial patients. Subsequent patients were treated with mesna 400 mg/m2 just before ifosfamide, and then every 2 hours to a total dosage equal to the ifosfamide dosage. Forty-four heavily pretreated patients were entered on study. Forty were evaluable for toxicity and 36 for response as well. The maximum tolerated dosage of ifosfamide was 4.0 g/m2/d for 5 days (20 g/m2/course). Overall, 10 patients achieved complete remission, and 3 achieved partial remission. Remissions were brief, although four patients went on to bone marrow transplant while in remission. One patient is still alive. The combination of etoposide and ifosfamide with mesna uroprotection showed promising activity in children with multiply recurrent acute leukemia. [\hyperlink{Ifosfamide}{PMID: 8348510}, M L Bernstein et al., 1993]

\hypertarget{pmid_19953647}{I}fosfamide is a widely used chemotherapeutic agent for the treatment of a broad spectrum of solid tumors. CNS toxicity is a well-described side effect of ifosfamide, but the mechanism of ifosfamide-induced neurotoxicity remains poorly defined. We present two pediatric cases of hemiballismic limb movements in the setting of ifosfamide-associated encephalopathy. To our knowledge, there have been no prior reports of ifosfamide-induced hemiballism in pediatric patients. One of our patient's encephalopathy and abnormal movements may have improved after the administration of methylene blue and thiamine. [\hyperlink{Ifosfamide}{PMID: 19953647}, Bethany Ames et al., 2010]

\hypertarget{pmid_8774570}{R}isk factors for long-term nephrotoxicity after ifosfamide for childhood cancers are not fully known. We have studied patient-related and treatment-related risk factors for chronic ifosfamide nephrotoxicity. A group of 23 children who had received ifosfamide at age 2.1-16.2 years (median 6.9) for various cancers were assessed for nephrotoxicity, at 1-28 (2) months after the end of treatment, by renal function testing, laboratory values, and a grading score (none, mild, moderate, severe). No patient had received cisplatin or undergone nephrectomy. 13 children were reassessed at 10-26 (23) months; eight had died and two were not evaluable. The median total ifosfamide dose was 100.8 (9.0-160.4) g/m2 over a median of 15 courses every 3 weeks as a 48-72 h continuous intravenous infusion (in 22 cases), with mesna and hydration. Glomerular filtration rate was below normal in ten (45\%) of 22 evaluable children; their rate was 61-85 mL/min per 1.73 m2. Proximal tubular toxicity led to hypophosphataemic rickets and/or renal tubular acidosis in six children, and distal tubular toxicity caused nephrogenic diabetes insipidus in one. Of the risk factors analysed by multiple regression, only total ifosfamide dose was associated with proximal tubular toxicity. Only two of ten evaluable patients who received under 100 g/m2 developed moderate nephrotoxicity, whereas six of ten who received over this dose had moderate or severe nephrotoxicity. High total ifosfamide dose was the only risk factor we identified. Although inter-patient variability was high, cumulative doses of 100 g/m2 or higher should be avoided in children with cancer. [\hyperlink{Ifosfamide}{PMID: 8774570}, R Skinner et al., 1996]

\hypertarget{pmid_8157032}{I}fosfamide is an alkylating agent which has been incorporated into frontline therapy for a number of malignant paediatric tumours. Recent data appears to suggest that tubular dysfunction may result from incorporation of this drug into chemotherapy schedules and that toxicity may be dose related. A detailed investigation of renal function was performed in a group of patients, ranging in age from 8 months to 15.9 years (median 8.6 years) with rhabdomyosarcoma (n = 11) and Ewing's sarcoma (n = 9) who were currently receiving (n = 4) or had completed ifosfamide (n = 16) therapy a mean of 16 months at the time of study. All but one patient demonstrated some degree of renal dysfunction and toxicity did not necessarily appear to be dose related. Implications for incorporation of this agent into future schedules for childhood sarcomas, which can expect to cure more than 60\% of such children, must be addressed. The importance of ongoing monitoring is emphasised. [\hyperlink{Ifosfamide}{PMID: 8157032}, M S Ashraf et al., 1994]

\hypertarget{pmid_21609276}{I}fosfamide-induced nephrotoxicity is a serious adverse effect in children undergoing chemotherapy. Our previous cell and rodent models have shown that the antioxidant N-acetylcysteine (NAC), used extensively as an antidote for acetaminophen poisoning, protects renal tubular cells from ifosfamide-induced nephrotoxicity at a clinically relevant concentration. For the use of NAC to be clinically relevant in preventing ifosfamide nephrotoxicity, we must ensure there is no effect of NAC on the antitumor activity of ifosfamide. Common pediatric tumors that are sensitive to ifosfamide, human neuroblastoma SK-N-BE(2) and rhabdomyosarcoma RD114-B cells, received either no pretreatment or pretreatment with 400 µmol/L of NAC, followed by concurrent treatment with NAC and either ifosfamide or the active agent ifosfamide mustard. Ifosfamide mustard significantly decreased the growth of both cancer cell lines in a dose-dependent manner (p < 0.001). The different combined treatments of NAC alone, sodium 2-mercaptoethanesulfonate alone, or NAC plus sodium 2-mercaptoethanesulfonate did not significantly interfere with the tumor cytotoxic effect of ifosfamide mustard. These observations suggest that NAC may improve the risk/benefit ratio of ifosfamide by decreasing ifosfamide-induced nephrotoxicity without interfering with its antitumor effect in cancer cells clinically treated with ifosfamide. [\hyperlink{Ifosfamide}{PMID: 21609276}, Nancy Chen et al., 2011]

\hypertarget{pmid_15049014}{I}fosfamide-induced nephrotoxicity is well recognized in children, although it has also been reported in adults. Whether ifosfamide nephrotoxicity is more common in children than in adults is not known. Medical records of adults and children diagnosed with sarcoma whom received ifosfamide with a cumulative dose >20 g/m(2) were evaluated. Twenty-five children (</=18-years of age) and 28 adults were identified. National Cancer Institute Common Toxicity Criteria grade 3-4 ifosfamide-induced nephrotoxicity was present in 24 and 17\% of children and adults, respectively (P = 0.58). Cumulative ifosfamide doses were similar between the two populations, with the median (range) of 70.2 g/m(2) (22.4-72) for children and 59 g/m(2) (20.8-146) for adults (P = 0.25). Logistic regression analysis indicated that neither age or cumulative ifosfamide dose were associated with grade 3-4 ifosfamide-induced nephrotoxicity (P = 0.36). Children and adults receiving >20 g/m(2) of ifosfamide have similar susceptibility to ifosfamide-induced nephrotoxicity. Factors other than age and cumulative dose should be considered for understanding the inter-individual variation in nephrotoxicity. [\hyperlink{Ifosfamide}{PMID: 15049014}, Jeannine S McCune et al., 2004]

\hypertarget{pmid_8711502}{N}euroblastomas, nephroblastomas, malignant mesenchymal tumors, including rhabdomyosarcomas, Ewing's sarcomas, osteosarcomas, brain tumors, and non-Hodgkin's lymphomas respond to ifosfamide monotherapy. Ifosfamide has found an established place in the treatment of these pediatric malignancies. Ifosfamide has been shown to be an especially valuable agent in brain tumors in patients with a poor-prognosis medulloblastoma. The combination of ifosfamide with other cytotoxic drugs is more effective than ifosfamide medication alone. Ifosfamide has to be given with mesna to prevent bladder toxicity, but other toxic side effects, such as neurotoxicity, nephrotoxicity, and gonadal damage, are more difficult to prevent. Ifosfamide has proven to be an important asset in the treatment of cancer in children. [\hyperlink{Ifosfamide}{PMID: 8711502}, P A Voûte et al., 1996]

\hypertarget{pmid_18278066}{I}fosfamide nephrotoxicity is a serious adverse effect for children undergoing cancer chemotherapy. Our recent in vitro studies have shown that the antioxidant N-acetylcysteine (NAC), which is used extensively as an antidote for paracetamol (acetaminophen) poisoning in children, protects renal tubular cells from ifosfamide-induced toxicity at a clinically relevant concentration. To further validate this observation, an animal model of ifosfamide-induced nephrotoxicity was used to determine the protective effect of NAC. Male Wistar albino rats were injected intraperitoneally with saline, ifosfamide (50 or 80 mg kg(-1) daily for 5 days), NAC (1.2 g kg(-1) daily for 6 days) or ifosfamide+NAC (for 6 days). Twenty-four hours after the last injection, rats were killed and serum and urine were collected for biochemical analysis. Kidney tissues were obtained for analysis of glutathione, glutathione S-transferase and lipid peroxide levels as well as histology analysis. NAC markedly reduces the severity of renal dysfunction induced by ifosfamide with a significant decrease in elevations of serum creatinine (57.8+/-2.3 vs 45.25+/-2.1 micromol l(-1)) as well as a reduced elevation of beta2-microglobulin excretion (25.44+/-3.3 vs 8.83+/-1.3 nmol l(-1)) and magnesium excretion (19.5+/-1.5 vs 11.16+/-1.5 mmol l(-1)). Moreover, NAC significantly improved the ifosfamide-induced glutathione depletion and the decrease of glutathione S-transferase activity, lowered the elevation of lipid peroxides and prevented typical morphological damages in renal tubules and glomeruli. Our results suggest a potential therapeutic role for NAC in paediatric patients in preventing ifosfamide nephrotoxicity. [\hyperlink{Ifosfamide}{PMID: 18278066}, N Chen et al., 2008]

\hypertarget{pmid_33023384}{I}fosfamide (IFO) is an alkylating agent used to treat broad range of malignancies. One of the life-threatening toxic effects is reversible neurotoxicity. In this report; we presented a case report of ifosfamide induced encephalopathy (IIE) in a child with osteosarcoma in order to emphize that it is important to continue ifosfamide treatment as well as the importance of this potentially fatal complication. Following the 20th week of ifosfamide treatment, the patient's follow-up with the diagnosis of osteosarcoma developed neurological findings. Laboratory analyzes before and after ifosfamide infusion were normal. No pathological findings were seen on MR imaging. Hypoglycemia, electrolyte disturbances, encephalitis, meningitis, metastasis and posterior reversible encephalopathy syndrome (PRES) were not considered. Electroencephalography was found compatible with neuronal hyperexcitability originating from the left hemisphere. With the diagnosis of ifosfamide induced encephalopathy, prophylaxis with methylene blue was received before the next infusion of ifosfamide. Neurological findings were not observed in the patient's follow-up. Patients who develop IIE can continue their treatment protocol with methylene blue prophylaxis and supportive therapy. [\hyperlink{Ifosfamide}{PMID: 33023384}, Hakan Sarbay et al., 2021]

\hypertarget{pmid_8058012}{I}fosfamide, a nitrogen mustard derived alkylating agent commonly used in the treatment of solid tumors, has been associated with neurotoxicity in 5-33\% of treated patients. Encephalopathy most often occurs during or shortly following drug administration, with increased drowsiness or irritability, confusion, hallucinations, visual blurring, extrapyramidal dysfunction, cranial nerve abnormalities, incontinence, generalized muscle twitching, seizures, and coma reported in infants, children, and older adults. While most reported neurologic abnormalities associated with ifosfamide have been reversible, encephalopathy resulting in death has occurred. We now report an infant who developed ifosfamide-induced encephalopathy, loss of developmental milestones, progressive brain atrophy, and cessation of cranial growth. This is the first case of cerebral atrophy and loss of developmental milestones that has been reported in a pediatric patient treated with ifosfamide. Given the efficacy of this anti-neoplastic agent and its increasing use in pediatrics, further investigation is indicated, especially in infants where brain growth is ongoing. [\hyperlink{Ifosfamide}{PMID: 8058012}, C S Bruggers et al., 1994]

\hypertarget{pmid_11793120}{I}fosfamide has been in use as an effective antineoplastic agent for solid tumors in both children and adults since the late 1960s. Although some adverse effects (e.g. hemorrhagic cystitis) can be overcome by the co-administration of 2-mercaptoethanesulfonate (MESNA), others such as nephrotoxicity cannot. There is a consensus that factors such as the cumulative dose of ifosfamide and concomitant cisplatin administration may influence not only the incidence but also the severity of ifosfamide-induced renal toxicity. Several preliminary studies suggested young age as a risk factor for nephrotoxicity; however, there is little agreement on this. The reasons for this uncertainty may include sample size, study design, dose and differences in renal function assessment. In this review we examine the two largest cohort studies conducted in pediatric patients. One study suggests that ifosfamide-induced renal toxicity is age- related, whereas analysis of the other failed to show age as an important predictor for ifosfamide-induced renal toxicity. The studies differed in design, end-points of toxicity and concomitant drug therapy. Due to the effectiveness of ifosfamide as an antineoplastic agent, it is important that an understanding of the factors that predispose pediatric patients to ifosfamide-induced nephrotoxicity be obtained. [\hyperlink{Ifosfamide}{PMID: 11793120}, K Aleksa et al., 2001]

\section*{Isoproterenol Hydrochloride}
\subsection*{Result}
\subsubsection*{Answer}

Infants (under 1 year): Yes  
Children (1–6 years): Yes  
Older children (7–16 years): Yes  

\subsubsection*{{Explanation}}
\hypertarget{Isoproterenol Hydrochloride}
A review of the available abstracts reveals several studies that specifically address the safety of Isoproterenol Hydrochloride in children, particularly in the context of asthma treatment and as a diagnostic agent during anesthesia. Below is a summary of the relevant evidence, organized by age range where possible:

1. **Children with Severe Asthma Attacks (General Pediatric Population, including under 6 years):**
   - A study of 22 children (age not specified, but described as "children") with 32 episodes of severe asthma attacks treated with continuous isoproterenol inhalation found the therapy to be effective and safe, especially in the early stage of severe attacks. Heart rates decreased as symptoms improved, and no significant adverse effects were detected [\hyperlink{pmid_1444819}{PMID: 1444819}, Y Adachi et al., 1992].
   - Another study compared continuous isoproterenol inhalation therapy in 31 children, divided into two groups: under 6 years old (n=20) and over 7 years old (n=22). The therapy was effective in both groups, though younger children required higher doses and had a slightly lower degree of efficacy. No significant safety concerns were reported [\hyperlink{pmid_1492792}{PMID: 1492792}, Y Adachi et al., 1992].
   - A retrospective review of 11 children (8 months to 15 years) with respiratory failure due to status asthmaticus treated with intravenous isoproterenol reported that all cases were successfully treated without complications, and the therapy was described as safe and effective [\hyperlink{pmid_2013558}{PMID: 2013558}, M S Victoria et al., 1991].
   - A study of 23 asthmatic children (age not specified) compared three methods of administering isoproterenol hydrochloride for severe bronchospasm and found all methods similarly effective, with no mention of significant adverse effects [\hyperlink{pmid_328233}{PMID: 328233}, M Loren et al., 1977].

2. **Children Undergoing Anesthesia (Ages 2 months to 10 years, and infants):**
   - Several studies evaluated the use of intravenous isoproterenol as a marker for epidural test dosing in children under anesthesia. In a study of 72 children (mean age 2.8 ± 1.7 years), isoproterenol produced expected hemodynamic responses with only one transient benign dysrhythmia reported, suggesting a favorable safety profile in this context [\hyperlink{pmid_8712442}{PMID: 8712442}, S Kozek-Langenecker et al., 1996].
   - Another study in 44 children aged 2 months to 10 years found no episodes of hypotension or arrhythmia with isoproterenol doses up to 0.075 mcg/kg, but noted that full evaluation of neurotoxicity was still needed [\hyperlink{pmid_8418721}{PMID: 8418721}, M Perillo et al., 1993].
   - A dose-response study in 36 children aged 0.5–8 years under anesthesia found no significant safety concerns, but focused on chronotropic response rather than adverse events [\hyperlink{pmid_9728825}{PMID: 9728825}, S A Kozek-Langenecker et al., 1998].

3. **Cardiotoxicity Concerns:**
   - One prospective study of 20 admissions for severe childhood asthma found evidence of cardiotoxicity (elevated CPK-MB and ECG changes) in all six admissions where isoproterenol infusion was used, but not in those where it was not used. The authors recommend serial monitoring for cardiotoxicity in children receiving isoproterenol infusions for severe asthma [\hyperlink{pmid_1956735}{PMID: 1956735}, J F Maguire et al., 1991].

4. **Comparative Efficacy and Safety:**
   - A study comparing albuterol and isoproterenol in 11 asthmatic children (9–16 years) found that albuterol produced greater bronchodilation and did not produce overt cardiovascular effects, while isoproterenol was less effective, but no significant adverse effects were reported for isoproterenol [\hyperlink{pmid_6846920}{PMID: 6846920}, M R Littner et al., 1983].

**Summary by Age Range:**
- **Infants (under 1 year):** Included in studies [\hyperlink{pmid_2013558}{PMID: 2013558}, 8418721, 8712442, 1492792], with no significant adverse effects reported in most, but one study [\hyperlink{pmid_1956735}{PMID: 1956735}] raises concern for cardiotoxicity with intravenous infusion.
- **Children (1–6 years):** Multiple studies [\hyperlink{pmid_1444819}{PMID: 1444819}, 1492792, 8712442, 8418721, 2013558] support safety for inhaled or intravenous isoproterenol in acute asthma, with the caveat of potential cardiotoxicity with infusion.
- **Older Children (7–16 years):** Included in studies [\hyperlink{pmid_1492792}{PMID: 1492792}, 328233, 6846920], with similar findings—generally safe, but with isolated reports of cardiotoxicity.

**Conclusion:**
There are targeted studies in children (including infants, young children, and older children) that affirm the safety of Isoproterenol Hydrochloride for short-term use in acute asthma attacks, especially via inhalation. However, there is evidence of potential cardiotoxicity with intravenous infusion in severe asthma, warranting monitoring. The safety profile for use as a diagnostic agent during anesthesia appears favorable, but long-term safety and neurotoxicity have not been fully evaluated.

\subsection*{Abstracts}
\hypertarget{pmid_1444819}{T}he aim of this study was to evaluate the efficacy and safety of continuous isoproterenol inhalation therapy for asthma attacks in children. We used l-body isoproterenol (Proternol L) in 22 children with 32 episodes of severe attacks. One of them did not respond to this therapy, and two had complications (atelectasis and pneumothorax). Twenty-nine cases were divided into three subgroups according to their clinical scores; A) scores less than or equal to 4, which meant that they were in the early stage of severe attack (n = 9), B) scores 5-6, which meant impending respiratory failure (n = 17), C) scores greater than or equal to 7, which meant respiratory failure (n = 3). The values of SpO2 at the start of this therapy were 94.8, 91.5, 82.0\%, respectively. The more severe their attacks were, the lower their SpO2 levels were. The periods until their scores became zero were 0.78, 6.3, 17.2 hours, respectively. There were significant differences between each period respectively (p less than 0.001, p less than 0.01). Heart rates decreased when their symptoms improved, and other adverse effects were not detected. These results suggest that this therapy is effective and safe for children with severe asthma attacks, especially in the early stage. [\hyperlink{Isoproterenol Hydrochloride}{PMID: 1444819}, Y Adachi et al., 1992]

\hypertarget{pmid_8712442}{A}n epidural test dose containing epinephrine does not reliably produce hemodynamic responses in children under halothane anesthesia. The purpose of this study was to determine hemodynamic responses to intravenous isoproterenol in both awake and halothane-anesthetized children. After obtaining institutional review board approval and parental informed consent, 72 ASA physical status 1 or 2 children (2.8 +/- 1.7 yr) undergoing elective minor surgery were studied before and during anesthesia with 1.2 minimum alveolar concentration halothane. A bolus containing 0.25 mg/ kg bupivacaine and 0.05 microgram/kg, 0.075 microgram/kg, or 0.1 microgram/kg isoproterenol, or bupivacaine and saline was injected via a peripheral arm vein to simulate intravascular injection of an epidural test dose. Before induction of anesthesia, all patients showed a positive test response after isoproterenol injection (heart rate increase > or = 20 beats/min). During anesthesia, 79\% of patients receiving 0.05 microgram/kg, 89\% of patients receiving 0.075 microgram/kg, and 100\% of patients receiving 0.1 microgram/kg met the criterion of a positive test response. Among each treatment group, all infants showed a positive test response. Blood pressure did not differ among the groups at any time. Transient benign dysrhythmias occurred in only one patient under halothane anesthesia receiving 0.075 microgram/kg isoproterenol. Isoproterenol at a dose of 0.1 microgram/kg is a sensitive indicator for intravascular injection of a test dose in children anesthetized with halothane and nitrous oxide. Isoproterenol at a dose of 0.05 microgram/kg approximates a minimal effective dose in awake children and in infants. After detailed studies on neural toxicity, isoproterenol could be of value as an epidural test agent in children. [\hyperlink{Isoproterenol Hydrochloride}{PMID: 8712442}, S Kozek-Langenecker et al., 1996]

\hypertarget{pmid_9728825}{I}soproterenol has been suggested as an alternative marker for epidural test dosing in children receiving halothane anesthesia. The purpose of this prospective, randomized, double-blind study was to determine the chronotropic response to IV isoproterenol in sevoflurane-anesthetized children. Thirty-six ASA physical status I children (0.5-8 yr) were anesthetized with either halothane or sevoflurane at 1 minimum alveolar anesthetic concentration adjusted for age in 70\% nitrous oxide. Patients received incremental IV injections of isoproterenol until their heart rate increased > or = 20 bpm above baseline. The minimal effective dose of isoproterenol required to produce an increase of > or = 20 bpm was 55 ng/kg (42-72 ng/kg; 95\% confidence interval) in sevoflurane-anesthetized children and 32 ng/kg (26-38 ng/kg; 95\% confidence interval) in halothane-anesthetized children (P < 0.05). This dose-response study suggests that sevoflurane antagonizes beta-adrenergic-mediated chronotropic responses to isoproterenol more than halothane. These observations also suggest that larger doses of isoproterenol will be necessary for epidural test dosing in children receiving sevoflurane rather than halothane anesthesia. Isoproterenol has been suggested as an alternative marker for epidural test dosing in children receiving halothane anesthesia. This isoproterenol dose-response study indicates that larger doses of isoproterenol will be necessary for epidural test dosing in children undergoing sevoflurane rather than halothane anesthesia. [\hyperlink{Isoproterenol Hydrochloride}{PMID: 9728825}, S A Kozek-Langenecker et al., 1998]

\hypertarget{pmid_8418721}{T}he purpose of this study was to determine if isoproterenol would be an effective marker of intravascular injection in anesthetized children. Forty-four ASA 1 children, aged 2 mo to 10 yr, were randomly assigned to two groups. Children in group 1 (n = 21) received 0.05 microgram/kg isoproterenol, and children in group 2 (n = 23) received 0.075 microgram/kg isoproterenol. A blinded observer continuously recorded heart rate and arterial blood pressure. Measurements were recorded before the surgical incision at steady-state halothane concentration of 1.2 minimum alveolar concentration adjusted for age. Isoproterenol produced a graded increase in heart rate with mean maximum increases of 16.5 +/- 8.7 beats/min in group 1 and 21.5 +/- 9.2 beats/min in group 2. No episodes of hypotension and arrhythmia were noted. Isoproterenol, 0.075 microgram/kg, is more sensitive but still is an imperfect marker of an intravascular injection. It produces a heart rate increase in 96\% of children anesthetized with halothane and nitrous oxide in 50\% oxygen. The application of isoproterenol as an epidural test dose appears promising, but cannot be recommended until its full reliability and neurotoxicity are evaluated. [\hyperlink{Isoproterenol Hydrochloride}{PMID: 8418721}, M Perillo et al., 1993]

\hypertarget{pmid_619057}{T}o determine the effectiveness of oral propranolol in children, we administered 0.5 to 4.0 mg/kg/day of the drug to 64 children (age one day to 20 years); 41 with cardiac dysrhythmias, six with isiopathic hypertrophic subaortic stenosis, and 17 with paroxysmal hypoxemic spells associated with right ventricular infundibular obstruction. A new liquid form of propranolol (10 mg/ml) was administered to 37 of the younger patients, and tablets were given to the other 27. Propranolol improved the dysrhythmia in 31 of 41 patients, being notably effective in supraventricular tachycardia and ventricular tachycardia associated with a prolonged QT interval. The drug also eliminated symptoms attributed to IHSS in six of six patients and abolished hypoxemic spells in 12 of 17. The liquid and tablets were equally effective; and the liquid had the advantage of allowing for accurate dose changes in younger children. We conclude that oral propranolol is an excellent drug for use in pediatric patients with certain types of cardiac disease. [\hyperlink{Isoproterenol Hydrochloride}{PMID: 619057}, P Gillette et al., 1978]

\hypertarget{pmid_21788220}{P}ropranolol hydrochloride is a safe and effective medication for treating infantile hemangiomas (IHs), with decreases in IH volume, color, and elevation. Forty children between the ages of 9 weeks and 5 years with facial IHs or IHs in sites with the potential for disfigurement were randomly assigned to receive propranolol or placebo oral solution 2 mg/kg per day divided 3 times daily for 6 months. Baseline electrocardiogram, echocardiogram, and laboratory evaluations were performed. Monitoring of heart rate, blood pressure, and blood glucose was performed at each visit. Children younger than 6 months were admitted to the hospital for monitoring after their first dose at weeks 1 and 2. Efficacy was assessed by performing blinded volume measurements at weeks 0, 4, 8, 12, 16, 20, and 24 and blinded investigator scoring of photographs at weeks 0, 12, and 24. IH growth stopped by week 4 in the propranolol group. Significant differences in the percent change in volume were seen between groups, with the largest difference at week 12. Significant decrease in IH redness and elevation occurred in the propranolol group at weeks 12 and 24 (P = .01 and .001, respectively). No significant hypoglycemia, hypotension, or bradycardia occurred. One child discontinued the study because of an upper respiratory tract infection. Other adverse events included bronchiolitis, gastroenteritis, streptococcal infection, cool extremities, dental caries, and sleep disturbance. Propranolol hydrochloride administered orally at 2 mg/kg per day reduced the volume, color, and elevation of focal and segmental IH in infants younger than 6 months and children up to 5 years of age. [\hyperlink{Isoproterenol Hydrochloride}{PMID: 21788220}, Marcia Hogeling et al., 2011]

\hypertarget{pmid_20644039}{P}ropranolol hydrochloride has been prescribed for decades in the pediatric population for a variety of disorders, but its effectiveness in the treatment of infantile hemangiomas (IHs) was only recently discovered. Since then, the use of propranolol for IHs has exploded because it is viewed as a safer alternative to traditional therapy. We report the cases of 3 patients who developed symptomatic hypoglycemia during treatment with propranolol for their IHs and review the literature to identify other reports of propranolol-associated hypoglycemia in children to highlight this rare adverse effect. Although propranolol has a long history of safe and effective use in infants and children, understanding and recognition of deleterious adverse effects is critical for physicians and caregivers. This is especially important when new medical indications evolve as physicians who may not be as familiar with propranolol and its adverse effects begin to recommend it as therapy. [\hyperlink{Isoproterenol Hydrochloride}{PMID: 20644039}, Kristen E Holland et al., 2010]

\hypertarget{pmid_2013558}{B}etween January 1985 and December 1988, a total of 701 children were admitted to The Methodist Hospital, Brooklyn, NY for treatment of acute asthma. Eleven of these patients (age range between 8 months and 15 years) went into respiratory failure. All cases of respiratory failure were successfully treated with intravenous isoproterenol. Only one patient needed mechanical ventilation. Treatment with isoproterenol was safe and effective without any complications. We present our experience in the use of isoproterenol in treating children with respiratory failure secondary to status asthmaticus. A brief review of the literature is included. [\hyperlink{Isoproterenol Hydrochloride}{PMID: 2013558}, M S Victoria et al., 1991]

\hypertarget{pmid_1492792}{T}he aim of this study was to evaluate the efficacy of continuous isoproterenol inhalation therapy for severe asthma attacks in younger children, compared with its efficacy in older children. We used l-body isoproterenol (Proternol L) in 31 children with 42 episodes of severe attacks. They were divided into two group according to age: 20 cases under 6 years old (Group A), and 22 cases over 7 years old (Group B). All of the patients except for one in Group B, eventually improved with this therapy. Wood's clinical scores for Group A were significantly higher than those for group B (p < 0.01). In 22 cases whose scores were 5-6, their SpO2 values at the onset of this therapy were 90.8 +/- 3.17 in group A and 92.4 +/- 3.82\% in group B. The improvement time of group A (13.6 +/- 16.2 hours) was significantly longer than that of group B (2.5 +/- 5.66, p < 0.01). The nebulized isoproternol doses for group A were 0.47 +/- 0.168 and for group B 0.26 +/- 0.096 mg/kg/saline 500 ml. The dose for group A was significantly higher than that for group B (p < 0.01). We concluded that continuous isoproterenol inhalation therapy was effective even in younger children. But the degree of efficacy was slightly lower in younger children, although they inhaled higher doses of isoproterenal than older children. [\hyperlink{Isoproterenol Hydrochloride}{PMID: 1492792}, Y Adachi et al., 1992]

\hypertarget{pmid_21156771}{R}apid anterograde conduction in the setting of ventricular preexcitation is associated with an increased risk of sudden cardiac death. The effect of isoproterenol in this setting is unclear, particularly in younger anesthetized patients. The aim of this study was to determine the effect of isoproterenol on accessory-pathway conduction in children undergoing general anesthesia and its role in the risk-stratification process. The records of 151 pediatric patients with preexcitation undergoing electrophysiologic study under propofol anesthesia during a 5-year period were reviewed. Data included accessory-pathway effective refractory period, minimum 1:1 accessory pathway conduction with atrial pacing, and shortest preexcited R-R interval in atrial fibrillation. Measurements were repeated after administration of low-dose isoproterenol (mean, 0.013 μg/kg per min; range, 0.003 to 0.027). All accessory-pathway characteristics were significantly shortened with isoproterenol (P<0.001). Accessory-pathway effective refractory period increased modestly with age, both in the baseline state (r=0.172, P=0.04) and with isoproterenol (r=0.267, P<0.01) as did minimum 1:1 accessory pathway conduction with atrial pacing (r=0.178, P=0.034, and r=0.175, P<0.01, respectively). Accessory-pathway effective refractory period ≤250 ms was observed in only 5\% of patients at baseline vs 25\% after isoproterenol, and Shortest preexcited R-R interval in atrial fibrillation ≤250 ms was noted in 16\% vs 41\%. Tachycardia was induced in 48 of 151 patients before and in 102 of 151 after isoproterenol. In anesthetized children with ventricular preexcitation, accessory pathways display shorter conduction properties at younger ages and important adrenergic sensitivity at all ages. Use of low-dose isoproterenol resulted in a substantial increase in the number of patients who would otherwise meet typical criteria for ablation. [\hyperlink{Isoproterenol Hydrochloride}{PMID: 21156771}, Jeremy P Moore et al., 2011]

\hypertarget{pmid_30659785}{P}ropranolol is an effective method of treatment for infantile hemangiomas (IH). A recent concern is a shift of the therapy into outpatient settings. The aim of the study was to evaluate the safety of initiating and maintaining propranolol therapy for IH. The study involved 55 consecutive children with IH being treated with propranolol. The patients were assessed in the hospital at the initiation of the therapy and later in outpatient settings during and after the therapy. Each time, the following monitoring methods were used: physical examination, cardiac ultrasound (ECHO), electrocardiography (ECG), blood pressure (BP), heart rate (HR), and biochemical parameters: blood count, blood glucose, aspartate transaminase (AST), alanine transaminase (ALT), and ionogram. The therapeutic dose of propranolol was 2.0 mg/kg/day divided into 2 doses. Four children were excluded during the qualification or the initiation of propranolol; a total of 51 patients were subject to the final analysis. All the children presented clinical improvement. There was a significant reduction in the mean HR values only at the initiation of propranolol. There were no changes in HR during the course of the therapy. Blood pressure values were within normal limits. Both systolic and diastolic values decreased in the first 3 months. Bradycardia and hypotension were observed sporadically, and they were asymptomatic. Electrocardiography did not show significant deviations. The pathological findings of the ECHO scans were not a contraindication to continuing the therapy. There were no changes in biochemical parameters. Apart from 1 symptomatic case of hypoglycemia, other low glucose episodes were asymptomatic and sporadic. The observed adverse effects were mild and the propranolol dose had to be adjusted in only 6 cases. Propranolol is effective, safe and well-tolerated by children with IH. The positive results of the safety assessment support the strategy of initiating propranolol in outpatient settings. Future studies are needed to assess the benefits of the therapy in ambulatory conditions. [\hyperlink{Isoproterenol Hydrochloride}{PMID: 30659785}, Lidia Babiak-Choroszczak et al., 2019]

\hypertarget{pmid_328233}{T}weinty-three asthmatic children had severe sudden bronchospasm due to numerous factors. Baseline values for the peak expiratory flow rate were less than 25 percent of predicted. Utilizing an analysis of variance, three methods of administering isoproterenol hydrochloride (hand-held Freon-propelled nebulization, continuous nebulization, and intermittent positive-pressure breathing [IPPB]) were compared and found to be similar in reversing the bronchospasm (F = 1.56; degrees of freedom, 2/44; P = 0.22). There were no patients whose condition consistently improved with IPPB over the methods of therapy using simple nebulization. Therapy with IPPB did not offer any advantage over simple nebulization in patients with severe, reversible airway obstruction. [\hyperlink{Isoproterenol Hydrochloride}{PMID: 328233}, M Loren et al., 1977]

\hypertarget{pmid_25009634}{T}he aim of the present study was to investigate the effects of propranolol and isoproterenol on the growth curve of infantile hemangioma endothelial cells (IHECs)  [\hyperlink{Isoproterenol Hydrochloride}{PMID: 25009634}, Yalin Zhu et al., 2014] Propranolol hydrochloride is a nonselective β-blocker that is used for the treatment of hypertension, arrhythmia, and angina pectoris. In Japan, it was recently approved for the treatment of childhood arrhythmia. It has been observed to produce drastic involution of infantile hemangiomas. The aim of this prospective study was to examine propranolol's superiority to classical therapy with pulsed dye laser and/or cryosurgery in treating proliferating infantile hemangiomas. Fifteen patients between the ages of 1 and 4 months with proliferating infantile hemangiomas received grinded propranolol tablets 2 mg/kg per day divided in three doses. Twelve patients with proliferating infantile hemangiomas receiving pulsed dye laser and/or cryosurgery were enrolled as controls. Baseline electrocardiogram, echocardiogram, and chest x-ray were performed. Monitoring of heart rate, blood pressure, and blood glucose was performed every 2 weeks. Efficacy was assessed by performing blinded volume measurements and taking photographs at every visit. Propranolol induced significantly earlier involution and redness reduction of infantile hemangiomas, compared to pulsed dye laser and cryosurgery. Adverse effects such as hypoglycemia, hypotension, or bradycardia did not occur. The dramatic response of infantile hemangiomas to propranolol and few side effects suggest that early treatment of infantile hemangiomas could result in decreased disfigurement. Propranolol should be considered as a first-line treatment of infantile hemangiomas. [\hyperlink{Isoproterenol Hydrochloride}{PMID: 25009634}, Shinji Kagami et al., 2013]

\hypertarget{pmid_1956735}{W}e prospectively evaluated 20 patient admissions for severe exacerbation of childhood asthma at The Children's Hospital, Boston, to detect evidence of cardiotoxicity. Evidence of cardiotoxicity was found in all six patient admissions for which isoproterenol infusion was utilized. This included marked elevation of serum creatine phosphokinase isoenzyme (CPK-MB) levels and electrocardiogram abnormalities consistent with transient myocardial ischemia. Peak serum CPK-MB levels were significantly lower and electrocardiogram abnormalities were significantly less frequent during 14 patient admissions for which isoproterenol infusion was not utilized. Risk factors associated with cardiotoxicity included tachycardia, hypercapnia, acidosis, and intravenous isoproterenol therapy. We conclude that cardiotoxicity is not infrequent during therapy for severe exacerbations of childhood asthma. Electrocardiograms and measurement of serum CPK-MB levels are sensitive, useful, and readily obtained indicators of cardiotoxicity. Abnormalities of these studies may detect cardiotoxicity prior to the occurrence of more blatant or catastrophic manifestations of cardiotoxicity. We therefore recommend serial monitoring of serum CPK-MB levels and electrocardiograms for all children requiring an admission to the intensive care unit for management of severe asthmatic exacerbation. [\hyperlink{Isoproterenol Hydrochloride}{PMID: 1956735}, J F Maguire et al., 1991]

\hypertarget{pmid_31266444}{P}ropranolol hydrochloride is the first-line agent recommended for the treatment of infantile hemangiomas (IH). Serious adverse effects of propranolol therapy for hemangiomas are infrequent. We report a case presented in deep hypoglycemic coma during his treatment with propranolol for IH. Through our case report and the review of the literature, we aimed to underline the importance of recognizing adverse effects during propranolol therapy. Although propranolol has a long history of safe and effective use in infants and children, pediatricians should be aware that life-threatening adverse effects can happen during propranolol therapy for IH. Early identification of these adverse effects can be of great importance for patient management and prognosis. It must certainly be noted that not just early identification among doctors, but education for parents is crucial. [\hyperlink{Isoproterenol Hydrochloride}{PMID: 31266444}, Ilirjana Bakalli et al., 2019]

\hypertarget{pmid_3712718}{A} previously well 2-year-old child presented with seizures and ventricular tachycardia shortly after playing with an aerosol can of a well-known proprietary deodorant. She required intensive care and survived without sequelae. The propellants used in this product were isobutane, n-butane, and propane. The propellants have been thought to be safer than the previously used Freons, which were known to be cardiotoxic and neurotoxic. Significant exposure was confirmed by the detection of n-butane and isobutane in the patient's serum. We conclude that unintentional exposure to non-Freon aerosol propellants in a nonconfined space can be hazardous to children. Aerosol cans should be considered to represent toxic hazards and should be kept out of reach of children. [\hyperlink{Isoproterenol Hydrochloride}{PMID: 3712718}, S Wason et al., 1986]

\hypertarget{pmid_19706583}{I}nfantile hemangiomas (IHs) are the most-common soft-tissue tumors of infancy. We report the use of propranolol to control the growth phase of IHs. Propranolol was given to 32 children (21 girls; mean age at onset of treatment: 4.2 months) after clinical and ultrasound evaluations. After electrocardiographic and echocardiographic evaluations, propranolol was administered with a starting dose of 2 to 3 mg/kg per day, given in 2 or 3 divided doses. Blood pressure and heart rate were monitored during the first 6 hours of treatment. In the absence of side effects, treatment was continued at home and the child was reevaluated after 10 days of treatment and then every month. Ultrasound measurements were performed after 60 days of treatment. Immediate effects on color and growth were noted in all cases and were especially dramatic in cases of dyspnea, hemodynamic compromise, or palpebral occlusion. In ulcerated IHs, complete healing occurred in <2 months. Objective clinical and ultrasound evidence of longer-term regression was seen in 2 months. Systemic corticosteroid treatment could be stopped within a few weeks. Treatment was administered for a mean total duration of 6.1 months. Relapses were mild and responded to retreatment. Side effects were limited and mild. One patient discontinued treatment because of wheezing. Propranolol administered orally at 2 to 3 mg/kg per day has a consistent, rapid, therapeutic effect, leading to considerable shortening of the natural course of IHs, with good clinical tolerance. [\hyperlink{Isoproterenol Hydrochloride}{PMID: 19706583}, Véronique Sans et al., 2009]

\hypertarget{pmid_23271298}{T}he clinical efficacy and safety of topical propranolol hydrochloride gel in the treatment of superficial infantile hemangiomas (IHs) were assessed. Fifty-one cases of IHs from Oct. 2010 to Sept. 2011 were subjected to the topical propranolol hydrochloride gel intervention in Fuzhou General Hospital of Nanjing Military Commands, China. Changes in size, texture, color, peak systolic velocity of the hemangiomas, resistance index and adverse effects were observed. The results were evaluated by using Achauer system, and responses of IHs to pranpronolol were considered scale II (poor) in 4 patients (17.24\%), scale II (moderate) in 18 patients (24.14\%), scale III (good) in 22 patients (44.83\%) and scale IV (excellent) in 7 patients (13.79\%). The response of superficial hemangiomas was significantly better than other hemangiomas (P<0.05), and no differences in response were found among different primary sites (P>0.05). Our study indicates that topical application of 3\% propranolol hydrochloride gel is effective and safe in treating IHs. [\hyperlink{Isoproterenol Hydrochloride}{PMID: 23271298}, Lie Wang et al., 2012]

\hypertarget{pmid_23578266}{T}he aim of this study is to formulate an extemporaneous pediatric oral solution of propranolol hydrochloride (PRO) 2 mg/ml for the therapy of infantile haemangioma or hypertension in a target age group of 1 month to school children and to evaluate its stability. A citric acid solution and/or a citrate-phosphate buffer solution, respectively, were used as the vehicles to achieve pH value of about 3, optimal for the stability of PRO. In order to mask the bitter taste of PRO, simple syrup was used as the sweetener. All solutions were stored in tightly closed brown glass bottles at 5 ± 3 °C and/or 25 ± 3 °C, respectively. The validated HPLC method was used to evaluate the concentration of PRO and a preservative, sodium benzoate, at time intervals of 0-180 days. All preparations were stable at both storage temperatures with pH values in the range of 2.8-3.2. According to pharmacopoeial requirements, the efficacy of sodium benzoate 0.05 \% w/v was proved (Ph.Eur., 5.1.3). The preparation formulated with the citrate-phosphate buffer, in our experience, had better palatability than that formulated with the citric acid solution. propranolol hydrochloride pediatric preparation extemporaneous preparation solution stability testing HPLC. [\hyperlink{Isoproterenol Hydrochloride}{PMID: 23578266}, Sylva Klovrzová et al., 2013]

\hypertarget{pmid_32532590}{I}nfantile hepatic hemangioendothelioma (IHHE) is a benign liver tumor, associated with hypothyroidism and vascular malformations along the skin, brain, digestive tract and other organs. Here, we determined a single-center patient cohort by evaluating the effectiveness and safety of propranolol and sirolimus for the treatment of IHHE. We performed a monocentric and observational study, based on clinical data obtained from 20 cases of IHHE treated with oral propranolol and sirolimus at the Shanghai Children's Medical Center (SCMC), between December 2017 and April 2019. All cases were confirmed by abdominal enhanced CT examination (18/20, 90\%) and sustained decrease of alpha fetoprotein (AFP) (2/20, 10\%). Propranolol treatment was standardized as once a day at 1.0mg/kg for patients younger than 2 months, and twice a day at 1.0mg/kg (per dose) for patients older than 2 months. Sirolimus was used to treat refractory IHHE patients after 6 months of propranolol treatment, and initial dosing was at 0.8mg/m The effective rate of propranolol for the treatment of children with IHHE was 85\% (17/20). In most cases, the AFP levels gradually decreased into the normal range. A complete response (CR) was achieved in 3 cases, partial response (PR) for 14 cases, progressive disease (PD) for 2 cases and stable disease (SD) was only detected once. Lesions decreased in two PD patients after administration of oral sirolimus. No serious adverse reactions were observed. This study indicates that both propranolol and sirolimus were effective drugs for the treatment of children with IHHE at SCMC. [\hyperlink{Isoproterenol Hydrochloride}{PMID: 32532590}, Ruicheng Tian et al., ]

\hypertarget{pmid_21601311}{I}nfantile hemangioma (IH) is a frequently encountered tumor with a potentially complicated course. Recently, propranolol was discovered to be an effective treatment option. To describe the effects and side effects of propranolol treatment in 28 children with (complicated) IH. A protocol for treatment of IH with propranolol was designed and implemented. Propranolol was administered to 28 children (21 girls and 7 boys, mean age at onset of treatment: 8.8 months). All 28 patients had a good response. In two patients, systemic corticosteroid therapy was tapered successfully after propranolol was initiated. Propranolol was also an effective treatment for hemangiomas in 4 patients older than 1 year of age. Side effects that needed intervention and/or close monitoring were not dose dependent and included symptomatic hypoglycemia (n = 2; 1 patient also taking prednisone), hypotension (n = 16, of which 1 is symptomatic), and bronchial hyperreactivity (n = 3). Restless sleep (n = 8), constipation (n = 3) and cold extremities (n = 3) were observed. Clinical studies are necessary to evaluate the incidence of side effects of propranolol treatment of IH. Propranolol appears to be an effective treatment option for IH even in the nonproliferative phase and after the first year of life. Potentially harmful adverse effects include hypoglycemia, bronchospasm, and hypotension. [\hyperlink{Isoproterenol Hydrochloride}{PMID: 21601311}, Marlies de Graaf et al., 2011]

\hypertarget{pmid_16542772}{A}nticholinergic treatment combined with intermittent catheterisation is the cornerstone of the conservative treatment strategy in children with neurogenic detrusor overactivity, which in most cases is due to congenital causes. Efficacy, tolerability and safety of propiverine hydrochloride were evaluated retrospectively in these children. At four specialized outpatient clinics, all children's records were scrutinized for first-line propiverine hydrochloride treatment, or second- or third-line treatment after failure of a non-selective alpha-blocker (phenoxybenzamine) and/or other anticholinergics (oxybutynin, trospium chloride). The primary efficacy outcomes were urodynamic parameters, with clinical symptoms as secondary outcomes. Statistical analysis was performed by paired t-tests (significance level p < 0.05). Altogether 74 children and adolescents (40 boys, 34 girls; age range 11 months-19 years) were treated with propiverine hydrochloride (average duration 2 years and approximately 4 months; individual dose range 5-75 mg). The primary efficacy outcome parameters improved significantly: maximum cystometric capacity 161.2 [standard deviation (SD) 97.3] to 252.2 ml (SD 117.2), p < 0.001; maximum detrusor pressure 43.8 (SD 39.2) to 27.1 cm H(2)O (SD 26.4), p = 0.002; bladder compliance 7.6 (SD 6.4) to 17.0 ml/cm H(2)O (SD 16.2), p < 0.001. Phasic detrusor overactivity was abolished by 63\%; incontinence resolved by 54\%. One patient spontaneously reported a typical anticholinergic adverse event, which resolved after dose reduction. No safety concerns were documented. Propiverine hydrochloride is effective in neurogenic detrusor overactivity in children and adolescents, even in some of those cases unresponsive to other anticholinergics. The low incidence rate (<1.5\%) of adverse events evidences a favourable risk-benefit profile of propiverine hydrochloride, considering in particular the total documented treatment and surveillance period of 171 patient years and nine months. [\hyperlink{Isoproterenol Hydrochloride}{PMID: 16542772}, Ute Grigoleit et al., 2006]

\hypertarget{pmid_6846920}{A}lthough albuterol aerosol is an effective bronchodilator in adult asthmatics its efficacy and safety are not fully established in asthmatic children. Albuterol (100 and 200 micrograms), isoproterenol (70 and 140 micrograms) and placebo were administered from metered dose aerosol cannisters to 11 asthmatic children (9 to 16 years of age). Our findings indicate that in asthmatic children albuterol aerosol causes bronchodilation for up to six hours and greater bronchodilation than comparable doses of inhaled isoproterenol and does not produce overt effects on the cardiovascular system, significant changes in arterial partial pressure of oxygen or excessive tremor. [\hyperlink{Isoproterenol Hydrochloride}{PMID: 6846920}, M R Littner et al., 1983]

\hypertarget{pmid_29149854}{P}ropranolol has become the first-line treatment for complicated Infantile Hemangioma (IH), showing so far a good risk-benefit profile. We report the case of a toddler, on propranolol, who suffered cardiac arrest during an acute viral infection. She had a neurally-mediated syncope that progressed to asystole, probably because of concurrent factors as dehydration, beta-blocking and probably individual susceptibility to vaso-vagal phenomena. In fact a significant history of breath-holding spells was consistent with vagal hyperactivity. The number of patients treated with propranolol for IHs will increase and sharing experience will help to better define the safety profile of this drug. [\hyperlink{Isoproterenol Hydrochloride}{PMID: 29149854}, Alvise Tosoni et al., 2017]

\section*{Norepinephrine Bitartrate}
\subsection*{Result}
\subsubsection*{Answer}

Neonates (including preterm and full-term): Yes  
Infants and Children (up to 18 years): Yes  
Children undergoing liver transplantation: No  

\subsubsection*{{Explanation}}
\hypertarget{Norepinephrine Bitartrate}
A review of the available abstracts reveals several studies specifically addressing the use of norepinephrine (including norepinephrine bitartrate) in pediatric populations, with varying age ranges and clinical contexts. Below is a summary of the relevant evidence:

1. **Neonates (including preterm and full-term):**
   - A retrospective cohort study evaluated norepinephrine in 92 neonates (76\% preterm) with dopamine-resistant shock. The study found that norepinephrine improved hemodynamic parameters in both preterm and full-term neonates. However, the study was underpowered to determine the association between norepinephrine use and mortality in preterm neonates. No specific safety concerns were highlighted, but the authors note the need for further research to confirm safety and efficacy in this group [\hyperlink{pmid_37045542}{PMID: 37045542}, Pei Lu et al., 2023].
   - Another randomized controlled trial in neonates with upper gastrointestinal hemorrhage compared conventional therapy to conventional therapy plus norepinephrine and cimetidine. The group receiving norepinephrine had shorter recovery times and no reported adverse reactions, leading the authors to conclude that norepinephrine was safe and effective in this context [\hyperlink{pmid_36415239}{PMID: 36415239}, Xuyan Dong et al.].

2. **Infants and Children (up to 18 years):**
   - A retrospective study of 37 pediatric patients (median age 1.8 years, range up to 18 years) who received norepinephrine via peripheral catheter in the prehospital setting found only one case (2.7\%) of suspected extravasation with local skin hypoperfusion, which resolved without sequelae. The authors concluded that peripheral norepinephrine administration in children was associated with a low risk of harm, but recommended further multicenter studies before making firm recommendations [\hyperlink{pmid_33718927}{PMID: 33718927}, Ramy C Charbel et al., 2021].
   - An open-label randomized controlled study compared norepinephrine plus dobutamine versus epinephrine as first-line agents in children aged 2 months to less than 18 years with fluid refractory cold septic shock. The study found no significant difference in shock resolution at 1 hour and did not report any new safety concerns, but noted that further studies are needed to confirm these findings [\hyperlink{pmid_36600781}{PMID: 36600781}, Kiran Kumar Banothu et al., 2023].

3. **Children undergoing liver transplantation:**
   - A retrospective study of 430 pediatric patients (median age 7 months) undergoing living donor liver transplantation found that intraoperative norepinephrine infusion was associated with a higher probability of mortality compared to those not receiving norepinephrine. The authors recommend caution in the use of norepinephrine in this specific context [\hyperlink{pmid_32922292}{PMID: 32922292}, Jiang-Long Chen et al., 2020].

**Summary by Age Range:**
- **Neonates (including preterm and full-term):** There is evidence from targeted studies suggesting norepinephrine can be used safely in neonates for certain indications (shock, upper GI hemorrhage), but the data are limited and further research is needed, especially in preterm neonates.
- **Infants and Children (up to 18 years):** Targeted studies suggest norepinephrine is generally safe for use in children with shock, with a low incidence of adverse events, but larger studies are needed for definitive conclusions.
- **Children undergoing liver transplantation:** One study suggests possible increased risk, so safety in this specific context is questionable.

**Conclusion:** There are targeted studies in neonates, infants, and children (up to 18 years) that affirm the safety of norepinephrine (bitartrate) for certain indications (mainly shock and upper GI hemorrhage), but the evidence is not comprehensive for all pediatric uses and further research is warranted. In the context of pediatric liver transplantation, safety is not affirmed.

\subsection*{Abstracts}
\hypertarget{pmid_28827252}{C}eftriaxone is widely used in children in the treatment of sepsis. However, concerns have been raised about the safety of ceftriaxone, especially in young children. The aim of this review is to systematically evaluate the safety of ceftriaxone in children of all age groups. MEDLINE, PubMed, Cochrane Central Register of Controlled Trials, EMBASE, CINAHL, International Pharmaceutical Abstracts and adverse drug reaction (ADR) monitoring systems will be systematically searched for randomised controlled trials (RCTs), cohort studies, case-control studies, cross-sectional studies, case series and case reports evaluating the safety of ceftriaxone in children. The Cochrane risk of bias tool, Newcastle-Ottawa and quality assessment tools developed by the National Institutes of Health will be used for quality assessment. Meta-analysis of the incidence of ADRs from RCTs and prospective studies will be done. Subgroup analyses will be performed for age and dosage regimen. Formal ethical approval is not required as no primary data are collected. This systematic review will be disseminated through a peer-reviewed publication and at conference meetings. CRD42017055428. [\hyperlink{Norepinephrine Bitartrate}{PMID: 28827252}, Linan Zeng et al., 2017]

\hypertarget{pmid_36053397}{N}on-steroidal anti-inflammatory drugs (NSAIDs) are commonly used in infants, children, and adolescents worldwide; however, despite sufficient evidence of the beneficial effects of NSAIDs in children and adolescents, there is a lack of comprehensive data in infants. The present review summarizes the current knowledge on the safety and efficacy of various NSAIDs used in infants for which data are available, and includes ibuprofen, dexibuprofen, ketoprofen, flurbiprofen, naproxen, diclofenac, ketorolac, indomethacin, niflumic acid, meloxicam, celecoxib, parecoxib, rofecoxib, acetylsalicylic acid, and nimesulide. The efficacy of NSAIDs has been documented for a variety of conditions, such as fever and pain. NSAIDs are also the main pillars of anti-inflammatory treatment, such as in pediatric inflammatory rheumatic diseases. Limited data are available on the safety of most NSAIDs in infants. Adverse drug reactions may be renal, gastrointestinal, hematological, or immunologic. Since NSAIDs are among the most frequently used drugs in the pediatric population, safety and efficacy studies can be performed as part of normal clinical routine, even in young infants. Available data sources, such as (electronic) medical records, should be used for safety and efficacy analyses. On a larger scale, existing data sources, e.g. adverse drug reaction programs/networks, spontaneous national reporting systems, and electronic medical records should be assessed with child-specific methods in order to detect safety signals pertinent to certain pediatric age groups or disease entities. To improve the safety of NSAIDs in infants, treatment needs to be initiated with the lowest age-appropriate or weight-based dose. Duration of treatment and amount of drug used should be regularly evaluated and maximum dose limits and other recommendations by the manufacturer or expert committees should be followed. Treatment for non-chronic conditions such as fever and acute (postoperative) pain should be kept as short as possible. Patients with chronic conditions should be regularly monitored for possible adverse effects of NSAIDs. [\hyperlink{Norepinephrine Bitartrate}{PMID: 36053397}, Victoria C Ziesenitz et al., 2022]

\hypertarget{pmid_11793132}{C}oncerns regarding the safety of nifedipine emerged in 1995 with the report of an increased risk of myocardial infarction associated with adult patients receiving short-acting calcium channel blockers. There have been few case reports of adverse events in children. The purpose of this study is to investigate the effect on blood pressure (BP) and the incidence of adverse events associated with nifedipine in our pediatric population. We conducted a retrospective chart review of pediatric patients who received nifedipine. We recorded the dose administered, all BP measurements and all adverse events reported within six hours of a nifedipine dose regardless of the likelihood that those events were related to the nifedipine dose. 1,746 doses of nifedipine in 166 pediatric patients were reviewed. Systolic BP decreased by a mean of 17\% and a maximum of 63\%. Diastolic BP decreased by a mean of 28\% and a maximum of 89\%. Adverse events included: a) change in neurologic status, six cases; b) hypotension, two cases; c) oxygen desaturation, 16 cases. Neurologic events occurred in 33\% of patients with acute CNS injury and 3.6\% of all patients. Short-acting nifedipine is an important and effective oral antihypertensive agent which can be safely used for the treatment of hypertensive emergencies in children. It should be used with caution in children with acute CNS injury. [\hyperlink{Norepinephrine Bitartrate}{PMID: 11793132}, David W Egger et al., 2002]

\hypertarget{pmid_33718927}{I}n prehospital and emergency settings, vasoactive medications may need to be started through a peripheral intravenous catheter. Fear of extravasation and skin injury, with norepinephrine specifically, may prevent or delay peripheral vasopressor initiation, though studies from adults suggest the actual risk is low. We sought to study the risk of extravasation and skin injury with peripheral administration of norepinephrine in children in the prehospital setting. We performed a retrospective study of pediatric patients (≤18 years) who received a vasopressor during prehospital transport. We collected data from retrieval and hospital records from 2 pediatric medical retrieval teams in the Paris/Ile-de-France region. Patients were eligible if they had documentation of distributive or obstructive shock and administration of norepinephrine through a peripheral catheter (intravenous or intraosseous) during retrieval. The primary outcomes were the occurrence of extravasation and evidence of skin injury. We also examined approach to norepinephrine administration (concentration, duration, proximal vs distal site) and hospital outcomes. Over a 3-year-period, 37 pediatric patients received norepinephrine through a peripheral catheter (33 intravenous, 4 intraosseous). Median patient age was 1.8 years. Thirty-two patients (86.5\%) had septic shock. The median total duration of norepinephrine infusion was almost 4 hours. One patient (2.7\%, 95\% confidence interval 0.5\%, 13.8\%) had suspected extravasation from a 24-gauge intravenous catheter in the hand, with local skin hypoperfusion. Skin changes were noted after 135 minutes of norepinephrine infusion. Perfusion normalized after catheter removal, and there were no other sequelae. In a 3-year sample of pediatric patients from a large metropolitan area, we found only 1 patient with evidence of any harm with peripheral administration of norepinephrine. This finding is consistent with the adult literature but requires multicenter and multiyear investigation before a firm recommendation for this practice can be made. [\hyperlink{Norepinephrine Bitartrate}{PMID: 33718927}, Ramy C Charbel et al., 2021]

\hypertarget{pmid_36415239}{T}o investigate the clinical efficacy of norepinephrine combined with cimetidine in the treatment of neonatal upper gastrointestinal hemorrhage and its adverse reactions. A total of 68 cases of neonatal upper gastrointestinal hemorrhage in Huangshi Maternal and Child Health Care Hospital from please mention dates October 2018 to February 2020 were selected and randomly divided into treatment group and control group by coin tossing, with 34 infants in each group. The control group received conventional therapy, and the treatment group was additionally treated with norepinephrine combined with cimetidine. The efficacy and safety were compared between the two groups. The time when the bleeding stops, the time of fecal occult blood turning negative and hospital stay of the treatment group were shorter than those of the control group ( Norepinephrine combined with cimetidine in the treatment of neonatal upper gastrointestinal hemorrhage can shorten the recovery time of symptoms, improve efficacy and reduce stress reaction. It is safe, effective and worthy of use in clinical practice. [\hyperlink{Norepinephrine Bitartrate}{PMID: 36415239}, Xuyan Dong et al., ]

\hypertarget{pmid_28570317}{T}his pilot study was conducted to profile safety of nebulized racemic epinephrine when used as a therapy for smoke inhalation injury in severely burned children. We enrolled 16 patients who were 7 to 19 years of age ([mean ± SD], 12 ± 4 years) with burns covering more than 30\% of the TBSA (55 ± 17\%) and smoke inhalation injury, as diagnosed by bronchoscopy at burn center admission. Patients were randomized to receive either standard of care (n = 8), which consisted of nebulized acetylcysteine, nebulized heparin, and nebulized albuterol, or to receive standard of care plus nebulized epinephrine (n = 8). Primary endpoints were death, chest pain, and adverse changes in cardiopulmonary hemodynamics (arrhythmia, arterial blood pressure, electrocardiographic [ST segment] changes, and peak inspiratory pressure). Additional endpoints included total days on ventilator, pulmonary function, and physiological cardiopulmonary measurements at intensive care unit discharge. No adverse events were observed during or after the nebulization of epinephrine, and no deaths were reported that were attributable to the administration of nebulized epinephrine. The groups did not significantly differ with regard to age, sex, burn size, days on ventilator, pulmonary function, or cardiopulmonary fitness. Results of this pilot trial indicate epinephrine to be safe when administered to pediatric burn patients with smoke inhalation injury. Current data warrant future efficacy studies with a greater number of patients. [\hyperlink{Norepinephrine Bitartrate}{PMID: 28570317}, Guillermo Foncerrada et al., ]

\hypertarget{pmid_24665311}{M}igraine is the most common acute intermittent primary headache in children and prophylactic therapy is indicated in children with frequent or disabling headaches. The purpose of this study was to evaluate the efficacy and safety of topiramate (TPM) for migraine prophylaxis in children. In a quasi-experimental study, monthly frequency, severity and duration of headache, migraine disability, and side-effects were evaluated in 100 children who were referred to the Pediatric Neurology Clinic of Shahid Sadoughi University of Medical Sciences, Yazd, Iran from April 2011 to March 2012, and were treated with 3 mg/kg/day of TPM for three months. Fifty eight (57.4\%) girls and 42 (41.6\%) boys with the mean age of 10.46±2.11 years were evaluated. Monthly frequency, severity, and duration of headache decreased with treatment from 15.34±7.28 to 6.07±3.16 attacks, from 6.21±1.74 to 3.15±2.22, and from 2.28±1.55 to 0.94±0.35 hours, respectively, and the Pediatric Migraine Disability Assessment score reduced with TPM from 32.48±9.33 to 15.54±6.16. Transient side-effects were seen in 21\% of the patients, including hyperthermia in 11\%, anorexia and weight loss in 6\%, and drowsiness in 4\%. No serious side-effects were reported. TPM could be considered as a safe and effective drug in pediatric migraine prophylaxis. [\hyperlink{Norepinephrine Bitartrate}{PMID: 24665311}, Razieh Fallah et al., 2013]

\hypertarget{pmid_9374563}{T}o test the hypothesis that short-term use of ibuprofen increases the risk of impaired renal function in children. Randomized, double-blind acetaminophen-controlled clinical trial. Children with a febrile illness were enrolled from outpatient pediatric and family medicine practices and randomly assigned to receive either acetaminophen suspension or one of two dosages of ibuprofen suspension (5 mg/kg or 10 mg/kg) for fever control. Mean blood urea nitrogen levels on admission among children admitted to hospital and assigned ibuprofen 5 mg/kg (n = 96), ibuprofen 10 mg/kg (n = 102), and acetaminophen 12 mg/kg (n = 87) were 4.1, 3.8, and 3.9 mmol/L, respectively. The corresponding creatinine levels were 43, 41, and 43 micromol/L, respectively. The prevalence of a creatinine level >62 micromol/L was 9.5\% overall and did not vary by antipyretic assignment. Among 83 children hospitalized with dehydration, the mean creatinine level was 44 micromol/L, and the prevalence of an elevated creatinine was 14\%; neither measure varied by antipyretic assignment. Although renal failure in children has been reported after ibuprofen use, these data suggest that for short-term use the risk of less severe renal impairment, as reflected by blood urea nitrogen and creatinine levels, is small and not significantly greater than that after acetaminophen use. [\hyperlink{Norepinephrine Bitartrate}{PMID: 9374563}, S M Lesko et al., 1997]

\hypertarget{pmid_9501886}{W}e evaluated the efficacy and safety of buspirone in the management of anxiety and irritability in children with pervasive developmental disorders (PDD). Twenty-two subjects, 6 to 17 years old, with DSM-III-R diagnosed PDD-NOS (N = 20) or autistic disorder (N = 2), were included. They were treated with buspirone in dosages ranging from 15 to 45 mg/day in an open-label trial lasting 6 to 8 weeks. Responders continued buspirone treatment and were followed up for up to 12 months. Nine subjects had a marked therapeutic response and 7 subjects a moderate response on the Clinical Global Impressions (CGI) scale after 6 to 8 weeks of treatment. Side effects were minimal, except for 1 patient who developed abnormal involuntary movements. These results suggest that buspirone may be useful for treating symptoms of anxiety and irritability in children with PDD. [\hyperlink{Norepinephrine Bitartrate}{PMID: 9501886}, J K Buitelaar et al., 1998]

\hypertarget{pmid_7843952}{A}n open, prospective study was undertaken to assess the efficacy and safety of subcutaneous sumatriptan in 17 children, ages 6 to 16 years, with severe, recurrent migraine. A 6-mg dose was used in 15 patients and relieved headache within 1 hour in six and by 2 hours in five others. Two smaller children received a 3-mg dose and both were headache-free within 2 hours. Most also reported marked improvement in associated symptoms such as nausea and photophobia. Four subjects had no clinical improvement after a 6-mg dose. Side effects, such as neck pressure, were brief and mild. These findings suggest that subcutaneous sumatriptan can be both effective and safe as an abortive agent in juvenile migraine, but the appropriate dose in smaller children will need further investigation. [\hyperlink{Norepinephrine Bitartrate}{PMID: 7843952}, J T MacDonald et al., ]

\hypertarget{pmid_7158752}{H}eart rate and rhythm were studied in conscious children and children under nitrous oxide and halothane anaesthesia following intravenous administration of atropine or glycopyrrolate. Both drugs produced a similar increase in heart rate when the potency of glycopyrrolate is considered twice that of atropine. There is no difference in the response of anaesthetised and awake children. Junctional rhythm is the main dysrhythmia observed which appears to occur more frequently in anaesthetised children. The administration of both drugs is safe in paediatric patients. [\hyperlink{Norepinephrine Bitartrate}{PMID: 7158752}, R K Mirakhur et al., 1982]

\hypertarget{pmid_17428708}{A}bout 4-10\% of children and adolescents suffer from migraine. In the last few years, several studies have been performed to assess the efficacy and safety of triptans for the acute treatment of migraine in children and adolescents. Only sumatriptan nasal spray has been approved for the treatment of acute migraine with or without aura in adolescents aged 12-17 years in Europe. This review describes the results of the studies with sumatriptan nasal spray that have been performed in children and adolescents, including a study performed in the Netherlands. [\hyperlink{Norepinephrine Bitartrate}{PMID: 17428708}, Petra M C Callenbach et al., 2007]

\hypertarget{pmid_4000140}{T}he efficacy of Nifedipine (N) as an antihypertensive drug was assessed in 4 children aged 6-12 years with acute severe hypertension. In one child with a hypertensive encephalopathy N 10 mg administered sublingually in addition to other antihypertensive drugs caused a prompt fall in blood pressure followed by a rapid clinical improvement. In the other 3 children N 10 mg reduced systolic and diastolic blood pressure by 13.7\% and 16.4\% respectively. This antihypertensive action lasted for about 3-4 hours and was associated with an increase in heart rate by 11.5\%. The antihypertensive effects of N are positively related to pretreatment blood pressure. These results provide support that N is a safe and effective drug for controlling blood pressure also in hypertensive emergencies of children. [\hyperlink{Norepinephrine Bitartrate}{PMID: 4000140}, H Stopfkuchen et al., 1985]

\hypertarget{pmid_19818173}{T}his article reviews the comprehensive data on the safety and tolerability from over 6,300 patients who have taken artemether/lumefantrine (Coartem) as part of Novartis-sponsored or independently-sponsored clinical trials. The majority of the reported adverse events seen in these studies are mild or moderate in severity and tend to affect the gastrointestinal or nervous systems. These adverse events, which are common in both adults and children, are also typical of symptoms of malaria or concomitant infections present in these patients. The wealth of safety data on artemether/lumefantrine has not identified any neurological, cardiac or haematological safety concerns. In addition, repeated administration is not associated with an increased risk of adverse drug reactions including neurological adverse events. This finding is especially relevant for children from regions with high malaria transmission rates who often receive many courses of anti-malarial medications during their lifetime. Data are also available to show that there were no clinically relevant differences in pregnancy outcomes in women exposed to artemether/lumefantrine compared with sulphadoxine-pyrimethamine during pregnancy. The six-dose regimen of artemether/lumefantrine is therefore well tolerated in a wide range of patient populations. In addition, post-marketing experience, based on the delivery of 250 million treatments as of July 2009, has not identified any new safety concerns for artemether/lumefantrine apart from hypersensitivity and allergies, known class effects of artemisinin derivatives. [\hyperlink{Norepinephrine Bitartrate}{PMID: 19818173}, Catherine Falade et al., 2009]

\hypertarget{pmid_31143123}{P}ediatric drug development, especially in disease areas that only affect children, can be stimulated by using juvenile animal models not only for general safety studies, but also to gain knowledge on the pharmacokinetic and pharmacodynamic properties of the drug. Recently, the conventional growing piglet has been suggested as juvenile animal model. However, more studies with different classes of drugs are warranted to make a thorough evaluation whether the juvenile pig might be a suitable preclinical animal model. Ibuprofen is one of the most widely used non-steroidal anti-inflammatory drugs in human. The present study determined the PK parameters, gastro-intestinal and renal safety of 5 mg/kg BW ibuprofen after single intravenous, single oral and multiple oral administration to each time eight pigs (four males, four females) aging 1, 4, 8 weeks and 6-7 months. Oral administration was performed via a gastrostomy button. A jugular catheter was used for intravenous administration and blood sampling. To assess NSAID induced renal toxicity, renal function was evaluated using iohexol and  [\hyperlink{Norepinephrine Bitartrate}{PMID: 31143123}, Joske Millecam et al., 2019] Ibuprofen is the most widely used non-steroidal anti-inflammatory drug (NSAID) for the treatment of inflammation, mild-to-moderate pain and fever in children, and is the only NSAID approved for use in children aged ≥3 months. Its efficacy and safety profile have led to its increasing use in paediatric care, even without medical prescription. However, an increase of suspected adverse reactions to ibuprofen has been noted in concomitance with the raised, often medically unsupervised, consumption of the drug. The purpose of this work was a critical review of the paediatric literature over the last 15 years on side effects and adverse events associated with ibuprofen, in order to highlight circumstances associated with higher risks and to promote safe and appropriate use of this drug. The literature from 2000 to date demonstrates that gastrointestinal events are rare, but (when they occur) include both upper and lower digestive tract lesions. Dehydration plays an important role in triggering renal damage, so ibuprofen should not be given to patients with diarrhoea and vomiting, with or without fever. Likewise, ibuprofen should never be administered to patients who are sensitive to it or to other NSAIDs. It is contraindicated in neonates and in children with wheezing and persistent asthma and/or during varicella. Most of the analysed studies reported adverse events when ibuprofen was being used for fever symptoms or flu-like syndrome. Ibuprofen should not be used as an antipyretic, except in rare cases. Ibuprofen remains the drug of first choice in the treatment of inflammatory pain in children. [\hyperlink{Norepinephrine Bitartrate}{PMID: 31143123}, Maurizio de Martino et al., 2017]

\hypertarget{pmid_37045542}{N}orepinephrine (NE) is recommended for children and full-term neonates (born at >37 gestational weeks) with septic shock. Meanwhile, data on the effectiveness of NE in preterm neonates are still limited. This study aimed to evaluate the clinical efficacy of NE in preterm neonates with dopamine-resistant shock compared with that in full-term neonates. This was a single-centre, retrospective (January 2010-December 2020) cohort study of neonates with persistent shock despite adequate fluid resuscitation and dopamine or dobutamine administration at ≥10 μg/kg/min. Medical records of neonates treated with NE were retrospectively reviewed to collect respiratory and haemodynamic parameters and results of arterial blood gas (ABG) tests before and 8 hours after NE infusion. The effectiveness of NE was assessed using changes in clinical parameters and multiple regression models for mortality among subgroups of preterm and full-term neonates. Ninety-two neonates (76\% preterm) who received NE infusion were included in the study. NE infusion was started after a median of 7 hours (IQR 2-19 hours) after shock onset. Among the preterm neonates, the maximum dose of NE infusion was 0.5 (IQR 0.3-1.0) µg/kg/min with a median duration of 45 (IQR 24.0-84.5) hours. Haemodynamic dysfunction was ameliorated with increased blood pressure, decreased heart rate and improved ABG results. Preterm neonates with septic shock tended to have a reduced response to NE; however, preterm neonates with persistent pulmonary hypertension of the newborn tended to have a better response. Thirty-four (37\%) neonates died in our cohort. The timing, dose and duration of NE use were not associated with neonatal mortality. Although using NE effectively improves clinical parameters in preterm neonates with dopamine-resistant shock, our study is underpowered to identify the association between NE infusion and mortality in preterm neonates with dopamine-resistant shock. [\hyperlink{Norepinephrine Bitartrate}{PMID: 37045542}, Pei Lu et al., 2023]

\hypertarget{pmid_36600781}{O}ur objective was to compare norepinephrine plus dobutamine versus epinephrine as the first-line agent in children with fluid refractory cold septic shock. Open-label randomized controlled study. A single-center PICU from North India. Children 2 months to less than 18 years old with fluid refractory cold septic shock. In the intervention group, norepinephrine and dobutamine were started and in the control group, epinephrine was started as the first-line vasoactive agent. The primary outcome was the proportion attaining shock resolution (attaining all the therapeutic endpoints) at 1 hour of therapy. We enrolled 67 children: 34 in the norepinephrine plus dobutamine group (intervention) and 33 in the epinephrine group (control). There was no difference in shock resolution at 1 hour (17.6\% vs 9\%; risk ratio [RR], 2.0; 95\% CI, 0.54-7.35;  In children with fluid refractory cold septic shock, with use of norepinephrine plus dobutamine as first-line agents, the difference in the proportion of children attaining shock resolution at 1 hour between the groups was inconclusive. However, the time to shock resolution was earlier in the norepinephrine plus dobutamine group. Also, fewer children in the intervention group were refractory to treatment. Further studies powered to detect (or exclude) an important difference would be required to test this intervention. [\hyperlink{Norepinephrine Bitartrate}{PMID: 36600781}, Kiran Kumar Banothu et al., 2023]

\hypertarget{pmid_10506264}{R}ecently ibuprofen has been introduced as a nonprescription analgesic/antipyretic for use in children. To compare the incidence of serious adverse clinical events among children <2 years old given ibuprofen and acetaminophen to control fever. A practitioner-based, randomized clinical trial. A total of 27 065 febrile children were randomized to receive acetaminophen (12 mg/kg), ibuprofen (5 mg/kg), or ibuprofen (10 mg/kg). Rates of hospitalization for acute gastrointestinal bleeding, acute renal failure, anaphylaxis, Reye's syndrome, asthma, bronchiolitis, and vomiting/gastritis were compared by randomization group. The risk of hospitalization with any diagnosis in the 4 weeks after enrollment was 1.4\% (95\% confidence interval, 1. 3\%-1.6\%) and did not vary by antipyretic assignment. No children were hospitalized for acute renal failure, anaphylaxis, or Reye's syndrome. Three children were hospitalized with gastrointestinal bleeding; all 3 had been assigned to treatment with ibuprofen. The risk of hospitalization with gastrointestinal bleeding among children randomized to ibuprofen was 17 per 100 000 (95\% confidence interval, 3.5-49 per 100 000) but was not significantly greater than the risk among children given acetaminophen. The risk of hospitalization with asthma, bronchiolitis, or vomiting/gastritis did not differ by antipyretic assignment. The risk of serious adverse clinical events among children <2 years old receiving short-term treatment with either acetaminophen or ibuprofen suspension was small and did not vary by choice of medication. These data do not provide any information on the safety of these medications when used for prolonged periods or when used together, regardless of duration. [\hyperlink{Norepinephrine Bitartrate}{PMID: 10506264}, S M Lesko et al., 1999]

\hypertarget{pmid_2405737}{T}he effect of an intravenous (iv) injection of lidocaine with epinephrine was studied to determine if such a test dose would cause a reliably detectable increase in heart rate and systemic blood pressure in children anesthetized with halothane and nitrous oxide. The effect of the injection of atropine before the test dose on these parameters was also determined. Sixty-five children 1 month to 11 yr of age and weighing 3.9-35 kg were studied. The children were assigned to one of four groups, each of which was anesthetized with 1\% halothane and 50\% nitrous oxide. Group 1 (n = 20) received 10 micrograms/kg atropine followed 5 min later by an iv dose of 0.1 ml/kg 1\% lidocaine with 1/200,000 epinephrine (0.5 micrograms/kg) to simulate an intravascularly administered epidural test dose. Group 2 (n = 21) was identical to group 1 but did not receive atropine prior to the simulated intravascular test dose. Groups 3 (n = 12) and 4 (n = 11) were identical to groups 1 and 2, but the simulated intravascular test dose did not contain epinephrine: group 3 received atropine prior to the test dose and group 4 did not. The simulated intravascular test dose increased heart rate in group 1 (with atropine) at each time period from 15 to 120 s, but only at 45 and 60 s in group 2 (without atropine). Following the iv test dose, 6 of 21 children in group 2 had an increase in heart rate of less than 10 beats/min, while only one child in group 1 had an increase in heart rate of less than 10 beats/min. Intravenous test doses that did not contain epinephrine (groups 3 and 4) had no effect on heart rate or blood pressure. Atropine, 10 micrograms/kg, improves the reliability of an epidural test dose in children anesthetized with halothane and nitrous oxide but does not ensure total reliability in detecting an intravascular injection. [\hyperlink{Norepinephrine Bitartrate}{PMID: 2405737}, J Desparmet et al., 1990]

\hypertarget{pmid_24400233}{T}his review evaluates 17 clinical studies from 18 selected publications concerning the safety, tolerability, and additional effects of the phytotherapeutic drug, Canephron® N (CAN, containing the medicinal plants, Centaurium erythraea, Levisticum officinale, and Rosmarinus officinalis) as standard therapy in various clinical settings. Its role in the prophylaxis and treatment of urinary tract infections in adults and in children, therapy and prophylaxis in adult patients with renal stones, treatment and prevention of urinary tract infections and other gestational diseases in pregnancy, and also its safety and tolerability. The dosage was as recommended and over a varying duration. Overall, CAN was shown to be effective in the treatment and prophylaxis of UTI compared with standard therapy, both in adults and children, and there was a reduced number of relapses. Children undergoing surgical correction of vesicoureteral reflux benefited from a prophylactic course of CAN. Ten-day add on therapy increased the rate of spontaneous elimination of kidney stones compared with standard therapy alone and may also have had a positive effect on stone prevention. Pregnant women showed earlier relief of symptoms and normalization of pyuria on additional treatment with the herbal combination. Only one adverse effect was reported (skin rash) in the 3115 patients included in this review. No teratogenic, embryotoxic, or fetotoxic effects, or negative interference with the psychological development or health of children born of mothers treated with the drug were reported. Because some of the studies were not well designed, their statistical significance remains unclear.  [\hyperlink{Norepinephrine Bitartrate}{PMID: 24400233}, Kurt G Naber et al., 2013] Although epinephrine is used in the neonatal intensive care unit, few data exist on efficacy of doses <0.05 mcg/kg/min. This study evaluates the efficacy and safety of low-dose epinephrine continuous infusion at doses <0.05 mcg/kg/min in infants. Single-center, retrospective review of hypotensive infants from 2011-2018. Charts were reviewed for initial and maximum epinephrine doses, additional vasoactive agents, short-term efficacy, and adverse effects. The primary outcome was percentage of patients initiated on low-dose epinephrine whose dose did not require titration to ≥0.05 mcg/kg/min. A total of 115 patients met study criteria with 131 distinct occurrences of low-dose epinephrine initiation. Most patients were unresponsive to other vasopressors at the time of epinephrine initiation. The median (IQR) starting dose of low-dose epinephrine was 0.01 (0.01-0.04) mcg/kg/min and median (IQR) maximum dose was 0.04 (0.02-0.08) mcg/kg/min. Fifty-five percent were responders. Patients in this cohort demonstrated significant improvement of blood pressure and urine output (p < 0.001) without adverse effects. Low-dose epinephrine infusion may be considered as an alternative treatment to standard starting doses in hypotensive neonatal intensive care unit patients. [\hyperlink{Norepinephrine Bitartrate}{PMID: 24400233}, Gloria Lee et al., 2021]

\hypertarget{pmid_31767042}{P}ain control is an important element of care for patients after surgery, leading to better outcomes, quicker transitions to recovery, and improvement in quality of life. The purpose of this study was to evaluate the safety and efficacy of non-steroidal anti-inflammatory drugs in children after cardiac surgery. Patients between the ages of 1 month and 18 years of age, who received intravenous or oral non-steroidal anti-inflammataory drugs after cardiac surgery, from November 2015 until September 2017 were included in this study. The primary endpoints were non-steroidal anti-inflammataory drug-associated renal dysfunction and post-operative bleeding. Secondary endpoints examined the effect of non-steroidal anti-inflammataory drug use on total daily dose of narcotics, number of intravenous PRN narcotic doses received, and pain assessment score. Data were analysed using descriptive statistics for frequencies and ranges. Multivariate analysis was performed to measure the association of all predictors and outcomes. Wilcoxon singed-rank test was performed for secondary outcomes. There was no association between the incidence of renal dysfunction and the use of or duration of non-steroidal anti-inflammataory drugs; in addition no association was found with increased chest tube output. There was a statistically significant reduction of patients' median Face, Legs, Activity, Cry, Consolability (FLACC) scores (2-0; p = 0.003), seen within first 24 hours after initiation of ketorolac, and a significant reduction of morphine requirements seen from day 1 to day 2 (0.3 mg/kg versus 0.1 mg/kg; p < 0.001) and number of as-needed doses. Non-steroidal anti-inflammataory drugs in paediatric cardiac surgery patients are safe and effective for post-operative pain management. [\hyperlink{Norepinephrine Bitartrate}{PMID: 31767042}, Dimitrios A Savva et al., 2019]

\hypertarget{pmid_32922292}{N}orepinephrine (NE) is often administered during the perioperative period of liver transplantation to address hemodynamic instability and to improve organ perfusion and oxygen supply. However, its role and safety profile have yet to be evaluated in pediatric living donor liver transplantation (LDLT). We hypothesized that intraoperative NE infusion might affect pediatric LDLT outcomes. A retrospective study of 430 pediatric patients (median [interquartile range] age, 7 [6.10] months; 189 [43.9\%] female) receiving LDLT between 2014 and 2016 at Renji Hospital was conducted. We evaluated patient survival among recipients who received intraoperative NE infusion (NE group, 85 recipients) and those that did not (non-NE group, 345 recipients). The number of children aged over 24 months and weighing more than 10 kg in NE group was more than that in non-NE group. And children in NE group had longer operative time, longer anhepatic phase time and more fluid infusion. After multivariate regression analysis and propensity score regression adjusting for confounding factors to determine the influence of intraoperative NE infusion on patient survival, the NE group had a 169\% more probability of dying. Although there was no difference in mean arterial pressure changes relative to the baseline between the two groups, we did observe increased heart rates in NE group compared with those of the non-NE group at anhepatic phase (P=0.025), neohepatic phase (P=0.012) and operation end phase (P=0.017) of the operation. In conclusion, intraoperative NE infusion was associated with a poorer prognosis for pediatric LDLT recipients. Therefore, we recommend the application of NE during pediatric LDLT should be carefully re-considered. [\hyperlink{Norepinephrine Bitartrate}{PMID: 32922292}, Jiang-Long Chen et al., 2020]

\hypertarget{pmid_25585913}{S}ubarachnoid hemorrhage is a rare, but life-threatening neurological emergency. Cerebral vasospasm is a complication of subarachnoid hemorrhage that contributes significantly to morbidity and mortality. Nimodipine has been used in adults to reduce the incidence of cerebral vasospasm after subarachnoid hemorrhage and improve long-term outcomes. There are, however, no data in children. Records of children with a confirmed diagnosis of subarachnoid hemorrhage who received nimodipine between January 1, 2005 and August 31, 2013 were reviewed. Dosing of nimodipine and associated hypotensive events were recorded. Transcranial Doppler ultrasonography, cranial computerized tomography, and angiography were followed as a measure of cerebral vasospasm, rebleeding, and subsequent infarction. Twelve children (average age 11.8 ± 3.3 years, age range 3.5 to 17.3 years) were included. Aneurysm was responsible for the highest percentage (41.7\%) of subarachnoid hemorrhage events. The mean dose of oral nimodipine was 1 mg/kg every 4 hours and was associated with a high rate of hypotension requiring intervention or dose modification. Clinical outcomes while on nimodipine therapy varied; evidence of vasospasm was observed in 67\%, new infarction in 33\%, and rebleeding in 17\%. Functional and cognitive deficits were minor in two-thirds and absent in the remaining individuals. All patients survived until hospital discharge. Oral nimodipine after subarachnoid hemorrhage in children does not eliminate vasospasm, rebleeding, or infarction and is associated with significant hypotension. Nevertheless, clinical outcomes appear favorable relative to the adult population who receive nimodipine. Further study, with dose titration, is warranted. [\hyperlink{Norepinephrine Bitartrate}{PMID: 25585913}, Josh Heffren et al., 2015]

\section*{Methylergonovine Maleate}
\subsection*{Result}
\subsubsection*{Answer}

Neonates (0-28 days): Unknown  
Infants (up to 1-3 years): Unknown  
Children (older than infants): Unknown  

\subsubsection*{{Explanation}}
\hypertarget{Methylergonovine Maleate}
A review of the available abstracts reveals the following evidence regarding the safety of Methylergonovine Maleate in children:

1. One abstract describes a prospective, controlled observational study of 29 breastfeeding mothers treated with methylergonovine and their infants, compared to a control group. The study found no effect of methylergonovine on lactation, no difference in neonatal complications, and no increase in adverse long-term outcomes in infants exposed to methylergonovine through breastfeeding. The infants' ages are not specified in detail, but the context suggests they are neonates and young infants (up to 1-3 years at follow-up). This study specifically addresses exposure through breast milk, not direct administration to children [\hyperlink{pmid_27846760}{PMID: 27846760}, Oded Gilad et al., 2017].

2. Another abstract reports a case of accidental direct intramuscular administration of methylergonovine to a term neonate. The infant developed transient oxygen desaturation requiring supplemental oxygen but recovered fully and was discharged in good condition after 72 hours. The abstract notes that ergot alkaloids can cause serious adverse events in neonates, including respiratory depression, seizures, and death, and emphasizes the need to avoid accidental administration [\hyperlink{pmid_27765939}{PMID: 27765939}, Bryan M Corbett et al., 2016].

3. No other abstracts provide targeted studies on the safety of methylergonovine maleate in children outside the context of breastfeeding exposure or accidental administration.

Summary by age range:
- Neonates (0-28 days): There is a case report of accidental administration resulting in transient adverse effects, and a warning about potential for serious harm. No targeted safety study affirms safety for direct use.
- Infants (up to 1-3 years): One prospective study found no adverse effects from exposure via breast milk, but this does not equate to direct administration safety.
- Children (older than infants): No abstracts provide targeted safety studies for methylergonovine maleate in this age group.

Conclusion: There are no targeted studies affirming the safety of direct use of methylergonovine maleate in children (neonates, infants, or older children). The only available evidence is for exposure via breast milk, which was not associated with adverse outcomes, but this does not establish safety for direct administration. There is also evidence of potential harm with direct administration in neonates.

\subsection*{Abstracts}
\hypertarget{pmid_27846760}{T}o evaluate maternal and breastfed infant's outcome following post-partum maternal use of methylergonovine. A prospective, controlled observational study design was used. Mothers who contacted Beilinson Teratology Information Service (BELTIS) were followed by phone interview. Data on lactation, neonatal symptoms and outcomes at the age of 1-3 years were obtained. Mothers' breastfeeding while treated with methylergonovine and their infants were compared to a matched control group of breastfeeding mothers using a drug known to be safe during lactation (amoxicillin). Follow-up was obtained for 38 of 42 women (90.5\%). Of whom, six stopped breastfeeding because of concerns regarding drug treatment and three refused to participate. The remaining 29 women and infant pairs were compared to a control group of 58 women and their infants. Comparison showed no effect of methylergonovine on lactation and similarly showed no difference in rate of neonatal complications (p = 1). At time of follow-up there were no differences in growth or in adverse neurodevelopment outcomes (p = 0.26). No increase in adverse long-term outcomes was found in infants exposed to methylergonovine through breastfeeding. Our data in conjunction with previous estimates of very low drug exposure support continuation of breastfeeding in women requiring treatment with methylergonovine. [\hyperlink{Methylergonovine Maleate}{PMID: 27846760}, Oded Gilad et al., 2017]

\hypertarget{pmid_27765939}{B}ACKGROUND Methylergonovine is an ergot alkaloid used to treat post-partum hemorrhage secondary to uterine atony. Mistaking methylergonovine for vitamin K with accidental administration to the neonate is a rare iatrogenic illness occurring almost exclusively in the delivery room setting. Complications of ergot alkaloids in neonates include respiratory depression, seizures, and death. CASE REPORT A term infant was inadvertently given 0.1 mg of methylergonovine intramuscularly in the right thigh. The error was only noted when the vial of medication was scanned, after administration, identifying it as methylergonovine rather than vitamin K. The local poison center was notified, and the infant was transferred to the neonatal intensive care unit for observation. Two hours after transfer, the infant was noted to have oxygen desaturations and required oxygen via nasal cannula. Supplemental oxygen was continued for 4 hours until the neonate was able to maintain normal oxygen saturations in room air. Feeding was started by 10 hours of life, and the infant was discharged home in good condition after a 72-hour stay without further complications. CONCLUSIONS Because of the potential for serious adverse events, vigilance is required to prevent accidental administration of methylergonovine to the neonate as a result of possible confusion with vitamin K in the early post-partum period. [\hyperlink{Methylergonovine Maleate}{PMID: 27765939}, Bryan M Corbett et al., 2016]

\hypertarget{pmid_26861518}{I}nfantile hemangioma is the most common benign vascular tumor of childhood that has a tendency for spontaneous involution. The aim of this study was to evaluate the efficacy of topical timolol maleate in the treatment of superficial infantile hemangioma and associated side effects during the course of treatment. Four boys and five girls with a median age of 5 months were reviewed at 2-week intervals for a period of 16 weeks. A decrease in size, color, and consistency were noted. Adverse effects caused by timolol maleate were noted and managed. Of nine cases, two patients showed excellent response, five showed good response, one showed partial response, and one had poor response. Topical timolol maleate is safe and effective in the treatment of infantile hemangioma.  [\hyperlink{Methylergonovine Maleate}{PMID: 26861518}, Abhijeet Kumar Jha et al., ] The antimetabolite methotrexate has been shown in placebo-controlled trials to be effective in adults with rheumatoid arthritis. Methotrexate may also be effective in children with resistant juvenile rheumatoid arthritis, but the supporting data are from uncontrolled trials. Centers in the United States and the Soviet Union participated in this randomized, controlled, double-blind trial designed to evaluate the effectiveness and safety of orally administered methotrexate. Patients received one of the following treatments each week for six months: 10 mg of methotrexate per square meter of body-surface area (low dose), 5 mg of methotrexate per square meter (very low dose), or placebo. The use of prednisone (less than or equal to 10 mg per day) and two nonsteroidal antiinflammatory drugs was also allowed. The 127 children (mean age, 10.1 years) had a mean duration of disease of 5.1 years; 114 qualified for the analysis of efficacy. According to a composite index of several response variables, 63 percent of the children who received low-dose methotrexate improved, as compared with 32 percent of those in the very-low-dose group and 36 percent of those in the placebo group (P = 0.013). As compared with the placebo group, the low-dose group also had significantly larger mean reductions from base line in the number of joints with pain on motion (-11.0 vs. -7.1), the pain-severity score (-19 vs. -11.5), the number of joints with limited motion (-5.4 vs. -0.7), and the erythrocyte sedimentation rate (-19.0 vs. -6 mm per hour). In the methotrexate groups only three children had the drug discontinued because of mild-to-moderate side effects; none had severe toxicity. Methotrexate given weekly in low doses is an effective treatment for children with resistant juvenile rheumatoid arthritis, and at least in the short term this regimen is safe. [\hyperlink{Methylergonovine Maleate}{PMID: 26861518}, E H Giannini et al., 1992]

\hypertarget{pmid_27588127}{T}he aim of the present study was to assess the efficacy and safety of topical timolol maleate combined with oral propranolol for parotid infantile hemangiomas. Between October 2012 and April 2014, propranolol was administered orally at a dose of 1.0-1.5 mg/kg/day to 22 infants with proliferating hemangiomas in the Department of Oral and Maxillofacial Surgery (Hospital of Stomatology, China Medical University, Shenyang, Liaoning, China). A small amount of 0.5\% timolol maleate eye drop solution was topically applied with medical cotton swabs to the area of the lesion twice a day, every 12 h. The study group consisted of 9 males and 13 females, aged 2-9 months, with a median age of 4.7 months. The lesions were all located in the parotid region, and measured between 3.5×4×0.5 and 7×8×3 cm in volume. The planned duration of therapy was 6-8 months, or the two drugs were stopped when complete regression of the lesions was obtained. The therapeutic outcomes and safety were assessed by the change in the size and color of the tumor, and the presence of adverse effects throughout the course of treatment. The mean duration of therapy was 21.1 weeks and ranged from 3 to 8 months. Of the 22 patients, 16 demonstrated an excellent response, 6 showed a good response and 2 displayed a moderate response. No major collateral effects were observed. Overall, oral propranolol combined with topical timolol maleate may be used as the first-line therapeutic choice in the treatment of infantile parotid mixed hemangioma. [\hyperlink{Methylergonovine Maleate}{PMID: 27588127}, Shuang Tong et al., 2016]

\hypertarget{pmid_30465439}{T}o investigate by meta-analysis the efficacy of gelatin tannate (GT), a mucosal barrier protector, in children with acute gastroenteritis. A comprehensive literature search was conducted. Studies were selected according to PICO: Participants: children aged 0-12 years with acute diarrhea; Intervention: GT; Comparison: oral rehydration solution and/or placebo; Outcomes: diarrhea-related outcomes. Three published randomized controlled trials were identified of pediatric diarrhea treated with GT (n = 203) or control (n = 204). GT significantly (p < 0.01) reduced stool frequency at 12 h in two randomized controlled trials. A significant treatment effect (risk ratio = 0.74; p < 0.01) in favor of GT was found for the exploratory composite outcome of 'diarrhea or liquid stools at 24 h' in three studies. Risk ratios in a single study which reported the percentage of patients with liquid stools at 12, 24 and 48 h favored GT at all time points. No significant differences were found between GT and control for patients with diarrhea at 12 or 24 h or for duration of diarrhea. GT improved stool frequency and stool consistency in children with acute diarrhea, although further well-controlled studies would be useful to confirm a beneficial treatment effect. [\hyperlink{Methylergonovine Maleate}{PMID: 30465439}, Marina Aloi et al., 2019]

\hypertarget{pmid_22238470}{C}hildren have a lower response rate to antimonial drugs and higher elimination rate of antimony (Sb) than adults. Oral miltefosine has not been evaluated for pediatric cutaneous leishmaniasis. A randomized, noninferiority clinical trial with masked evaluation was conducted at 3 locations in Colombia where Leishmania panamensis and Leishmania guyanensis predominated. One hundred sixteen children aged 2-12 years with parasitologically confirmed cutaneous leishmaniasis were randomized to directly observed treatment with meglumine antimoniate (20 mg Sb/kg/d for 20 days; intramuscular) (n = 58) or miltefosine (1.8-2.5 mg/kg/d for 28 days; by mouth) (n = 58). Primary outcome was treatment failure at or before week 26 after initiation of treatment. Miltefosine was noninferior if the proportion of treatment failures was ≤15\% higher than achieved with meglumine antimoniate (1-sided test, α = .05). Ninety-five percent of children (111/116) completed follow-up evaluation. By intention-to-treat analysis, failure rate was 17.2\% (98\% confidence interval [CI], 5.7\%-28.7\%) for miltefosine and 31\% (98\% CI, 16.9\%-45.2\%) for meglumine antimoniate. The difference between treatment groups was 13.8\%, (98\% CI, -4.5\% to 32\%) (P = .04). Adverse events were mild for both treatments. Miltefosine is noninferior to meglumine antimoniate for treatment of pediatric cutaneous leishmaniasis caused by Leishmania (Viannia) species. Advantages of oral administration and low toxicity favor use of miltefosine in children. NCT00487253. [\hyperlink{Methylergonovine Maleate}{PMID: 22238470}, Luisa Consuelo Rubiano et al., 2012]

\hypertarget{pmid_23850529}{T}he purpose of this study was to examine the risks of acute coronary syndrome (ACS) and acute myocardial infarction (AMI) that are associated with methylergonovine maleate (Methergine; Novartis Pharmaceuticals Corporation, Plantation, FL) use in a large database of inpatient delivery admissions in the United States. We conducted a retrospective cohort study using data from the Premier Perspective Database and identified 2,233,630 women who were hospitalized for delivery between 2007 and 2011 (approximately one-seventh of all US deliveries during this period). Exposure was defined by a charge code for methylergonovine during the delivery hospitalization. Study outcomes included ACS and AMI. Propensity score matching was used to address potential confounding. Methylergonovine was administered to 139,617 patients (6.3\%). Overall, 6 patients (0.004\%) who were exposed to methylergonovine and 52 patients (0.002\%) who were not exposed to methylergonovine had an ACS. Four patients (0.003\%) who were exposed to methylergonovine and 44 patients (0.002\%) in the not-exposed group had an AMI. After propensity score matching, the relative risk for ACS that was associated with methylergonovine exposure was 1.67 (95\% confidence interval [CI], 0.40-6.97), and the risk difference was 1.44 per 100,000 patients (95\% CI, -2.56 to 5.45); the relative risk for AMI that was associated with methylergonovine exposure was 1.00 (95\% CI, 0.20-4.95), and the risk difference was 0.00 per 100,000 patients (95\% CI, -3.47 to 3.47). Despite studying a very large proportion of US deliveries, we did not find a significant increase in the risk of ACS or AMI in women who received methylergonovine compared with those who did not; estimates were increased only modestly or not at all. The upper limit of the 95\% CI of our analysis suggests that treatment with methylergonovine would result in no more than 5 additional cases of ACS and 3 additional cases of AMI per 100,000 exposed patients. [\hyperlink{Methylergonovine Maleate}{PMID: 23850529}, Brian T Bateman et al., 2013]

\hypertarget{pmid_22187405}{W}ithin the last two decades low molecular weight heparins (LMWH) have gained increasing widespread use as anticoagulants in children. The use of LMWH has been implemented into clinical care even though there is a lack of firm evidence on the efficacy and safety of LMWH in this population due to the absence of sufficiently powered randomized controlled trials. In the absence of clinical trials, we performed a meta-analysis of available single-arm studies using LMWH in children. A systematic search of electronic databases (Medline, EMBASE, OVID, Web of Science, The Cochrane Library) for studies published from 1980 to 2010 was conducted using keywords in combination both as MeSH terms and text words. Two authors independently screened citations and those meeting a priori defined inclusion criteria were retained. Data on year of publication, study design, country of origin, number of patients, ethnicity, venous thromboembolic events type, and frequency of recurrence and major bleedings were abstracted. Pooled incidence rates (IR) including 95\% confidence intervals (95\% CIs) on efficacy and safety data of LMWH administration on primary prophylaxis, as well as on secondary prophylaxis in children following symptomatic thromboembolism (TE) were shown. We included 2251 pediatric patients derived from 35 single-arm studies from 12 study countries who were eligible for analysis in the present systematic review. Pooled incidence rates (95\% CI) to develop first TE on primary prophylaxis, further TE event on LMWH secondary prophylaxis, or a major bleeding event on LMWH were 0.047 (0.023 to 0.091), 0.052 (0.037 to 0.073) for efficacy, and 0.054 (0.039 to 0.074) for safety (treatment data only), respectively. Efficacy and safety data are comparable with adult data. The present systematic review suggests that use of LMWH in children as primary prophylaxis and in treatment of symptomatic thrombosis is effective and safe. However, properly designed randomized controlled trials are needed. [\hyperlink{Methylergonovine Maleate}{PMID: 22187405}, Christoph Bidlingmaier et al., 2011]

\hypertarget{pmid_31573668}{P}rilocaine/lidocaine is widely used as local anesthetic in children for cannulation and minor surgical procedures. Usually it is unproblematic but it is important to adhere to recommended dose to avoid serious complications. Excessive amount of prilocaine/lidocaine, large application area, prolonged application time or repeated application can, especially in infants, cause methemoglobinemia with clinical symptoms. In severe cases intensive care and antidote treatment with Methylene blue may be required. We report three infants who were overdosed with prilocaine/lidocaine, two of them due to incorrect use after circumcision and one premature baby where prilocaine/lidocaine was not removed in time. Two of the babies had MetHb levels > 33\% and were seriously affected with hypoxia, tachycardia and fatigue. After Methylene blue was given the infants recovered within 15 minutes and MetHb levels returned to normal. [\hyperlink{Methylergonovine Maleate}{PMID: 31573668}, Cornelia Kjellgard et al., 2019]

\hypertarget{pmid_32453920}{S}hock refractory to fluid and catecholamine therapy has significant morbidity and mortality in children. The use of methylene blue to treat refractory shock in children is not well described. We aim to collect and summarize the literature and define physicians' practice patterns regarding the use of methylene blue to treat shock in children. We conducted a systematic search of MEDLINE, Embase, PubMed, Web of Science, Cochrane for studies involving the use of methylene blue for catecholamine-refractory shock from database inception to 2019. Collected studies were analyzed qualitatively. To describe practice patterns of methylene blue use, we electronically distributed a survey to U.S.-based pediatric critical care physicians. We assessed physician knowledge and experience with methylene blue. Survey responses were quantitatively and qualitatively evaluated. Pediatric critical and cardiac care units. Patients less than or equal to 25 years old with refractory shock treated with methylene blue. None. One-thousand two-hundred ninety-three abstracts met search criteria, 139 articles underwent full-text review, and 24 studies were included. Studies investigated refractory shock induced by a variety of etiologies and found that methylene blue was generally safe and increased mean arterial blood pressure. There is overall lack of studies, low number of study patients, and low quality of studies identified. Our survey had a 22.5\% response rate, representing 125 institutions. Similar proportions of physicians reported using (40\%) or never even considering (43\%) methylene blue for shock. The most common reasons for not using methylene blue were unfamiliarity with this drug, its proper dosing, and lack of evidentiary support. Methylene blue appears safe and may benefit children with refractory shock. There is a stark divide in familiarity and practice patterns regarding its use among physicians. Studies to formally assess safety and efficacy of methylene blue in treating pediatric shock are warranted. [\hyperlink{Methylergonovine Maleate}{PMID: 32453920}, Andrea V Otero Luna et al., 2020]

\hypertarget{pmid_35176737}{T}he scientific evidence of methotrexate (MTX) in children with severe plaque psoriasis is scarce. To retrospectively evaluate the efficacy and safety of oral MTX in children with severe plaque psoriasis in a single center in China. We enrolled 42 children with severe plaque psoriasis who were administrated MTX. Efficacy was evaluated by the psoriasis area and severity index (PASI) score, physician global assessment (PGA) score, and body surface area (BSA) score. The Children's Dermatology Life Quality Index (CDLQI) score and safety data were recorded. Among 42 children (22 males, 20 females), the mean age was 11.2 years old. The initial weight-based dosage of oral MTX ranged from 0.1 to 0.3 mg/kg weekly. Overall, 80.6 and 47.2\% of patients achieved PASI75 (at least 75\% improvement from baseline in PASI score) and PASI90 (at least 90\% improvement from baseline in PASI score) at week 12, respectively. 72.2\% of patients achieved PGA 0/1 at week 12. BSA and PGA scores significantly decreased from baseline from week 4, accompanied by CDLQI score improvement from week 8. The steady effect of MTX could be reached at week 16. Elevated liver enzymes (28.6\%) and infections (28.6\%) were the most common side effects. Relapse was recorded in 9 (30.0\%) of 30 patients, and the mean posttherapy disease-free interval was 7.2 months. MTX is an effective and safe option for children with severe plaque psoriasis with adequate monitoring. [\hyperlink{Methylergonovine Maleate}{PMID: 35176737}, Zhaoyang Wang et al., 2022]

\hypertarget{pmid_32292451}{T}he role of methyl prednisolone in longitudinal extensive transverse myelitis in children is not completely discovered in developing country like Pakistan. So this is the first study which aimed to evaluate the efficacy of methyl prednisolone in longitudinal extensive transverse myelitis in children. This is quasi experimental hospital based descriptive prospective study. The data was collected from 34 children admitted in Paediatric Neurology department through Outpatient/emergency department in Children's Hospital and the Institute of Child Health, Lahore for period of one year from January 2018 to December 2018. The children full filling the inclusion criteria were observed before and after giving injection methyl prednisolone 30mg/kg/dose (maximum dose one Gram irrespective of the body weight) once daily for five days in the form of intravenous infusion. Complete recovery was seen in 41.2\% while 58.8\% showed partial recovery. The correlation of response to treatment (recovery) with gender, area of spinal cord involvement, muscle power and autonomic dysfunction is found at significance level of five percent according to Chi square test. Early consideration and administration of methyl prednisolone in longitudinally extensive transverse myelitis in children can be beneficial and can help to reduce the morbidity. [\hyperlink{Methylergonovine Maleate}{PMID: 32292451}, Muhammad Azeem Ashfaq et al., ]

\hypertarget{pmid_16284689}{M}ethotrexate (MTX) remains a mainstay in the treatment of children with hematological malignancies. The availability of an antidote/rescue agent, leucovorin (LV) has allowed escalation of MTX doses to achieve enormous plasma concentrations, compared with plasma folate. However, a recent review of more than 40 trials for children with ALL concluded that the addition of high dose MTX (HDMTX) in many different doses and schedules did not improve CNS therapy and made only minor improvements in systemic therapy for children with ALL [11]. Some assessment suggested that by HDMTX benefits only limited amount of children with ALL. Recent treatment schedules vary markedly in terms of timing, dosing and scheduling of MTX and/or leukovorin, which may leave us uncertain with ideas such as "how should we best use HDMTX and LV?" or "why are we still using such by industry recommended doses of MTX?" The answer of how best to incorporate HDMTX and/or LV into ALL treatment plans is still not known and further clinical and pharmacological studies dealing with still controversial systemic MTX issue are actual even now, after more than 5 decades of clinical experiences with the MTX in pediatric oncology. [\hyperlink{Methylergonovine Maleate}{PMID: 16284689}, J Sterba et al., 2005]

\hypertarget{pmid_21372841}{M}ercaptopurine has been used in continuing treatment of childhood acute lymphoblastic leukaemia since the mid 1950s. Recent advances in the understanding of thiopurine pharmacology indicated that thioguanine (TG) might be more effective than mercaptopurine (MP). The US and UK cooperative groups began randomised thiopurine trials and agreed prospectively to a meta-analysis. All randomised trials of TG versus MP were sought, and data on individual patients were analysed by standard methods. Combining three trials (from US, UK and Germany), the overall event-free survival (EFS) was not significantly improved with TG (odds ratio (OR)=0.89; 95\% confidence interval 0.78-1.03). Apparent differences in results between trials may be partly explained by the different types of patients studied. The larger treatment effect reported in males in the US trial was confirmed in the other trials. There was heterogeneity between sex/age subgroups (P=0.001), with significant EFS benefit of TG only observed for males aged <10 years old (OR=0.70; 0.58-0.84), although this did not result in a significant difference in overall survival (OR=0.83; 0.62-1.10). Additional toxicity occurs with TG. Mercaptopurine remains the standard thiopurine of choice, but further study of TG may be warranted to determine whether it could benefit particular subgroups. [\hyperlink{Methylergonovine Maleate}{PMID: 21372841}, G Escherich et al., 2011]

\hypertarget{pmid_34397332}{T}he present review was carried out to describe publications on the use of methylene blue (MB) in pediatrics and neonatology, discussing dose, infusion rate, action characteristics, and possible benefits for a pediatric patient group. The research was performed on the data sources PubMed, BioMed Central, and Embase (updated on Aug 31, 2020) by two independent investigators. The selected articles included human studies that evaluated MB use in pediatric or neonatal patients with vasoplegia due to any cause, regardless of the applied methodology. The MB use and 0 to 18-years-old patients with vasodilatory shock were the adopted criteria. Exclusion criteria were the use of MB in patients without vasoplegia and patients ≥ 18-years-old. The primary endpoint was the increase in mean arterial pressure (MAP). Side effects and dose were also evaluated. Eleven studies were found, of which 10 were case reports, and 1 was a randomized clinical study. Only two of these studies were with neonatal patients (less than 28 days-old), reporting a small number of cases (1 and 6). All studies described the positive action of MB on MAP, allowing the decrease of vasoactive amines in several of them. No severe side effects or death related to the use of the medication were reported. The maximum dose used was 2 mg/kg, but there was no consensus on the infusion rate and drug administration timing. Finally, no theoretical or experimental basis sustains the decision to avoid MB in children claiming it can cause pulmonary hypertension. The same goes for the concern of a possible deleterious effect on inflammatory distress syndrome. [\hyperlink{Methylergonovine Maleate}{PMID: 34397332}, Walusa A Gonçalves-Ferri et al., 2022]

\hypertarget{pmid_11224849}{A} child with malaria from a chloroquine-resistant area received an accidental overdose of chloroquine administered by a parent. Application of pharmacokinetics permitted definitive treatment with mefloquine in a safe and effective manner. [\hyperlink{Methylergonovine Maleate}{PMID: 11224849}, J A Lowry et al., 2001]

\hypertarget{pmid_23136875}{M}ethylphenidate is a centrally acting sympathomimetic used for the treatment of attention deficit/hyperactivity disorder in children and adolescents and for narcolepsy in adults. Despite the growing use among adult women, no reliable data on the prevalence of use during pregnancy have been published, and safety during pregnancy has not been established. We systematically reviewed available data on birth outcome after human in utero exposure to methylphenidate. Systematic searches in PubMed/Embase were performed from origin to August 2012, and data from Michigan Medicaid recipients, The Collaborative Perinatal Project and the Swedish Birth Registry were evaluated. Excluding three case reports, a total of 180 children exposed to methylphenidate in utero during first trimester were identified, among whom, four children with major malformations were observed. Methylphenidate exposure during pregnancy does not appear to be associated with a substantially (i.e. more than twofold) increased risk of congenital malformations. [\hyperlink{Methylergonovine Maleate}{PMID: 23136875}, Dorthe Dideriksen et al., 2013]

\hypertarget{pmid_26582874}{C}hildren account for 7\%-20\% of cutaneous leishmaniasis cases in Iran, but there are few safety data to guide pediatric antiparasitic therapy. We evaluated the clinical and laboratory tolerance of the systemic pentavalent antimonial compound meglumine antimoniate, in 70 Iranian children with cutaneous leishmaniasis. Adverse effects were similar to those seen in adults.  [\hyperlink{Methylergonovine Maleate}{PMID: 26582874}, Pouran Layegh et al., 2015] The use of antihistamine therapy in children for the management of upper respiratory tract infections remains a topic of debate. In this study, we focused on evaluating the effectiveness of promethazine (Phenergan), a first-generation H1 receptor antagonist and sedative, in addressing preoperative and intra-operative sequelae in cleft surgeries. A single-centered, parallel, randomized, double-blinded controlled clinical trial was conducted on 128 children aged 2 to 4 years undergoing cleft palate surgery under general anesthesia. The case group received Phenergan syrup orally twice a day for three days, while the control group received a placebo. Primary outcomes measured preoperative anxiety levels using a children's fear scale, while secondary outcomes assessed preoperative sleep quality and cough rate through objective scales. Intraoperative heart rate was monitored using an ECG connected to a monitor. The results demonstrated that the administration of promethazine resulted in a 34\% reduction in anxiety levels, a 46\% reduction in cold and cough, a 38\% improvement in sleep score, and stable heart rates throughout the surgery compared to the control group. Based on these findings, promethazine is considered a safe premedication option for children undergoing cleft palate surgeries; given its benefits outweigh its adverse effects. [\hyperlink{Methylergonovine Maleate}{PMID: 26582874}, Vedha Vivigdha A et al., 2023]

\hypertarget{pmid_17136402}{W}e performed a pharmacokinetic evaluation of methotrexate (MTX) in infants with acute lymphoblastic leukemia enrolled on the Pediatric Oncology Group (POG) 9407 Infant Leukemia Study to evaluate the effects of age on MTX pharmacokinetics and pharmacodynamics. A pharmacokinetic database of 61 patients was developed by combining MTX data obtained from 16 patients in a pharmacokinetic sub-study with data obtained for clinical care in other patients enrolled on the POG 9407 protocol. The data were analyzed for the first dose of MTX given to patients in induction/intensification therapy. Patients received MTX (4 g/m2) over 24 h at week 4 of therapy. Toxicity data were also reviewed to evaluate the incidence of common MTX toxicities during the first 6 weeks of therapy (the induction/intensification phase). Steady-state clearance (mean+/-standard deviation) for infants aged 0-6 months was 89+/-32 ml/min/m2 compared to 111+/-40 for infants aged 7-12 months (P=0.030). In the subgroup of infants aged 0-3 months the mean steady-state clearance was 84+/-30 ml/min/m2 (P=0.026 vs. the 7-12-month group). The incidence of renal toxicity (all grades) during induction/intensification therapy was 23\% in the 0-3 months age group compared to 0\% (for n=27) in the group 7-12 months of age (P=0.029). There were no significant differences in hepatoxicity or mucous membrane toxicity between age groups. A modest difference in steady-state MTX clearance is observed between younger infants (0-6 months) and older infants (7-12 months). Very young infants (0-3 months) also experienced a slightly higher incidence of renal toxicity during induction/intensification therapy. Steady-state clearance for the older infants is similar to values reported for children in other studies. [\hyperlink{Methylergonovine Maleate}{PMID: 17136402}, Patrick A Thompson et al., 2007]

\hypertarget{pmid_7792222}{O}ver the past four years terbinafine has become established as an effective systemic antimycotic agent with an excellent safety profile. However, experience with its use in children is very limited. We report the effective treatment of five children with oral terbinafine. [\hyperlink{Methylergonovine Maleate}{PMID: 7792222}, V Goulden et al., 1995]

\hypertarget{pmid_30414269}{I}nfantile haemangiomas (IH) are soft swellings of the skin that occur in 3-10\% of infants. When haemangiomas occur in high-risk areas or when complications develop, active intervention is necessary. To update a Cochrane Review assessing the interventions for the management of IH in children. We searched for randomized controlled trials in CENTRAL, MEDLINE, Embase, LILACS, AMED, PsycINFO, CINAHL and six trials registers up to February 2017. We included 28 trials (1728 participants) assessing 12 interventions. We downgraded evidence from high to moderate/low for issues related to risk of bias and imprecision. Oral propranolol (3 mg kg Our key results indicate that oral propranolol and topical timolol maleate are more beneficial than placebo in terms of clearance or other measures of resolution, or both, without an increase in harm. [\hyperlink{Methylergonovine Maleate}{PMID: 30414269}, M Novoa et al., 2019]

\hypertarget{pmid_24139067}{T}he purpose of this study was to assess the safety and efficacy of mitoxantrone (MX) in pediatric patients with aggressive multiple sclerosis (MS). A retrospective analysis on pediatric MS patients treated with MX was performed with regards to demographic/clinical parameters and magnetic resonance imaging (MRI) findings. 19 definite pediatric MS cases with mean ± SD age of 15.4 ± 2.8 years underwent 20 mg MX for control of their severe/frequent relapses, high EDSS score or new and active brain MRI lesions. After a median [IQR] follow-up period of 30[12-60] months, 14 cases (73\%) were relapse free; the EDSS score decreased by at least 0.5 in 16 cases (84.2\%); and gadolinium-enhancing lesion volume fell by 84.2\% in 16 cases. Adverse events included nausea and vomiting, fatigue, alopecia, palpitation, cardiomyopathy and mild leukopenia. All adverse events were mild and transient. Our results suggest MX is a good candidate for treatment of children with worsening RRMS and SPMS. Recommendations regarding patient selection, treatment administration, and close follow-up should be considered. Continuing research is needed to establish its efficacy and safety profile in a multinational collaboration with careful follow-up of adverse events. [\hyperlink{Methylergonovine Maleate}{PMID: 24139067}, Masoud Etemadifar et al., 2014]

\hypertarget{pmid_32229154}{T}he present study evaluated the exposure of children aged from one to 36 months to seven groups of mycotoxins, in the context of the infant French Total Diet Study (iTDS). Exposure was then compared to the health-based guidance values (HBGVs) for each mycotoxin. The value of the 90th percentile of exposure to nivalenol, patulin, fumonisins and zearalenone was less than 40\% of the HBGV considered relevant for children. On the other hand, a risk could not be excluded for ochratoxin A and aflatoxins as exposure was close to the HBGV for ochratoxin A and the margin of exposure was much lower than the critical margin of 10,000 for aflatoxins. The HBGVs for toxins T2 and HT2, and for deoxynivalenol (DON) and its acetylated compounds were exceeded. Five percent to 10\% of the children aged 5-12 months exceeded the HBGV considering the lower bound hypothesis for toxins T2 and HT2 and 7.5\%-27\% of the children aged 5 months and above exceeded the HBGV for DON. Consequently, the exposure of young children raises safety concerns for T2/HT2 and DON. Efforts should therefore be pursued to decrease their exposure to these molecules. [\hyperlink{Methylergonovine Maleate}{PMID: 32229154}, Karine Vin et al., 2020]

\section*{Indium In-111 Pentetreotide Kit}
\subsection*{Result}
\subsubsection*{Answer}
Unknown
\subsubsection*{{Explanation}}
\hypertarget{Indium In-111 Pentetreotide Kit}
After reviewing all available abstracts, none of them are about Indium In-111 Pentetreotide Kit or its use in children. There are no abstracts describing targeted safety studies of Indium In-111 Pentetreotide Kit in pediatric populations, nor are there any abstracts reporting on its safety or lack thereof in specific age ranges of children. Therefore, based on the abstracts provided, there is not enough data to determine whether Indium In-111 Pentetreotide Kit is safe or unsafe for use in children.

\subsection*{Abstracts}
\hypertarget{pmid_31862730}{I}ntranasal dexmedetomidine (IND) is an emerging agent for procedural distress in children. To explore the effectiveness of IND for procedural distress in children. We performed electronic searches of Medline (1946-2019), Embase (1980-2019), Google Scholar (2019), Cumulative Index to Nursing and Allied Health Literature (1981-2019), and Cochrane Central Register. We included randomized trials of IND for procedures in children. Methodologic quality of evidence was evaluated by using the Cochrane Collaboration's risk of bias tool and the Grading of Recommendations Assessment, Development, and Evaluation system, respectively. The primary outcome was the proportion of participants with adequate sedation. Among 19 trials ( The adequacy of sedation was subjective, which possibly led to biased outcome reporting. Given the methodologic limitations of included trials, IND is likely more effective at sedating children compared to oral chloral hydrate and oral midazolam. However, this must be weighed against the potential for adverse cardiovascular effects. [\hyperlink{Indium In-111 Pentetreotide Kit}{PMID: 31862730}, Naveen Poonai et al., 2020]

\hypertarget{pmid_32740502}{T}his study evaluated 2 doses of intranasal dexmedetomidine (IND) (3.0 and 3.5 µg/kg) as a procedural sedative for postoperative examination of children with glaucoma. A dose of 3.5 µg/kg was more efficacious and obviated the need for repeated general anesthesia. This study was carried out to determine the safety and effective dose of IND as a procedural sedative for postoperative follow-up examinations after glaucoma surgery in children in place of repeated examination under anesthesia. In this prospective randomized double-blinded interventional study, consecutive children aged 6 months to 6 years were randomized to receive 3.0 and 3.5 µg/kg IND using a mucosal atomizer device in the preoperative area of the operating room, under continuous monitoring of vital signs. Intranasal midazolam 0.25 mg/kg was used as a rescue agent in case of inadequate sedation, and general anesthesia was administered in case of persistent failure. All infants underwent a complete anterior and posterior segment evaluation including intraocular pressure and corneal diameter measurements. A total of 30 and 31 children aged 23.9±15.0 and 19.2±10.1 months, respectively, received 3.0 and 3.5 µg/kg IND. Adequate sedation was possible in 18 of 30 (60\%) children receiving 3.0 µg/kg and 24 of 31 (77.4\%) receiving 3.5 µg/kg IND alone (P=0.17). In combination with midazolam, successful sedations were 86.6\% versus 100\%, respectively (P=0.052). One patient in the 3.5 µg/kg group had ventricular arrhythmia, reversed with dextrose-saline infusion and injection glycopyrrolate. IND appears to be a safe and effective procedural sedative for postoperative follow-up examinations of pediatric glaucoma patients at doses of 3 and 3.5 µg/kg. The dose of 3.5 µg/kg was successful in more children. [\hyperlink{Indium In-111 Pentetreotide Kit}{PMID: 32740502}, Deepika Dhingra et al., 2020]

\hypertarget{pmid_27609854}{F}entanyl is the most widely studied intranasal (IN) analgesic in children. IN subdissociative (INSD) ketamine may offer a safe and efficacious alternative to IN fentanyl and may decrease overall opioid use during the emergency department (ED) stay. This study examines the feasibility of a larger, multicentre clinical trial comparing the safety and efficacy of INSD ketamine to IN fentanyl and the potential role for INSD ketamine in reducing total opioid medication usage. This double-blind, randomised controlled, pilot trial will compare INSD ketamine (1 mg/kg) to IN fentanyl (1.5 μg/kg) for analgesia in 80 children aged 4-17 years with acute pain from a suspected, single extremity fracture. The primary safety outcome for this pilot trial will be the frequency of cumulative side effects and adverse events at 60 min after drug administration. The primary efficacy outcome will be exploratory and will be the mean reduction of pain scale scores at 20 min. The study is not powered to examine efficacy. Secondary outcome measures will include the total dose of opioid pain medication in morphine equivalents/kg/hour (excluding study drug) required during the ED stay, number and reason for screen failures, time to consent, and the number and type of protocol deviations. Patients may receive up to 2 doses of study drug. This study was approved by the US Food and Drug Administration, the local institutional review board and the study data safety monitoring board. This study data will be submitted for publication regardless of results and will be used to establish feasibility for a multicentre, non-inferiority trial. NCT02521415. [\hyperlink{Indium In-111 Pentetreotide Kit}{PMID: 27609854}, Stacy L Reynolds et al., 2016]

\hypertarget{pmid_31331550}{I}nfantile colic is a common benign disease occurring in early infancy that may have a great impact on family life. In the present study, the effectiveness and safety of the complex homeopathic medicine Enterokind was compared with Simethicone for treating infantile colic. Current data were drawn from a prospective, multicenter, randomized, open-label, controlled clinical trial that was conducted in 2009 in 3 Russian outpatient clinics. Children received either Enterokind (Chamomilla D6, Cina D6, Colocynthis D6, Lac defloratum D6 and Magnesium chloratum D6) or Simethicone. Data from infants ≤ 6 months with infantile colic are presented here. The main outcomes assessments were the change of total complaints score (maximum 17 points) and total objective symptoms score (maximum 22 points) after 10 days of treatment. Data from 125 infants ≤ 6 months with infantile colic were analyzed. The differences in total complaints and objective symptoms scores between baseline and day 10, estimated from the ANCOVA model, were found to be highly significant (p < 0.0001; ITT) in favor of Enterokind, both for complaints (Δ=-2.38; 95\% confidence interval (CI): [-2.87; -1.89]) and for objective symptoms (Δ=-2.07; 95\% CI: [-2.65; -1.49]). 1 adverse event (AE), vomiting, occurred under Enterokind and was rated to be unlikely related to it; 4 AEs occurred under Simethicone. All AEs were non-serious. The current study indicates that Enterokind is an effective and safe homeopathic treatment for functional intestinal colic in infants ≤ 6 months. [\hyperlink{Indium In-111 Pentetreotide Kit}{PMID: 31331550}, Christa Raak et al., 2019]

\hypertarget{pmid_34820060}{P}ediatric patients feel significant fear and anxiety when undergoing surgeries. The ideal drug and its administration route have not been found yet. The aim of this study was to compare the efficacy and safety of intranasal (IN) ketamine and midazolam as premedication in children. We studied 71 eligible pediatric patients undergoing elective urologic surgeries, aged 2 to 6 years. The degree of sedation and separation scores was compared between the two groups. Additionally, hemodynamic parameters, before premedication, after induction of anesthesia, and during surgery were documented and compared between two groups. Postoperatively, any side effect was recorded as well. Finally, the data from 71 children were analyzed. Recovery time was significantly longer in group K (ketamine) compared to group M (midazolam); 27.86±4.42 vs 38.19± 6.67 minutes respectively (P=0.01). No significant difference was observed in terms of sedation score between two groups of K \& M; 3.29±0.78 vs 3 ±0.71 respectively (P=0.17), and not regarding separation score; 2.51±0.61 \& 2.31±0.52 respectively (P=0.01). Vital signs were kept within the physiological limits in both groups with no marked fluctuations. To produce sedation in young children, both midazolam and ketamine were effective and safe by IN route. [\hyperlink{Indium In-111 Pentetreotide Kit}{PMID: 34820060}, Hossein Khoshrang et al., 2021]

\hypertarget{pmid_24406329}{T}o establish the safety of an intranasal diamorphine (IND) spray in children. An open-label, single-dose pharmacovigilance trial. Emergency departments in eight UK hospitals. Children aged 2-16 years with a fracture or other trauma. Adverse events (AE) specifically related to nasal irritation, respiratory and central nervous system depression. 226 patients received 0.1 mg/kg IND. No serious or severe AEs occurred. The incidence of treatment-emergent AEs (TEAEs) was 26.5\% (95\% CI 20.9\% to 32.8\%), 93\% being mild. 89\% were related to treatment, all being known effects of the drug or route of administration except for three events in two patients. 20.4\% (95\% CI 15.3\% to 26.2\%) patients reported nasal irritation, all mild except one moderate and one 'unknown' severity. No respiratory depression was reported. Three AEs related to reduced Glasgow Coma Score (GCS) occurred, all mild. There were no safety concerns raised during the conduct of the study. In addition to expected side effects, IND can cause mild nasal irritation in a proportion of patients. 2009-014982-16. [\hyperlink{Indium In-111 Pentetreotide Kit}{PMID: 24406329}, Jason Kendall et al., 2015]

\hypertarget{pmid_23560967}{T}he present study aims to conduct a pilot study examining the effectiveness of intranasal (IN) ketamine as an analgesic for children in the ED. The present study used an observational study on a convenience sample of paediatric ED patients aged 3-13 years, with moderate to severe (≥6/10) pain from isolated limb injury. IN ketamine was administered at enrolment, with a supplementary dose after 15 min, if required. Primary outcome was change in median pain rating at 30 min. Secondary outcomes included change in median pain rating at 60 min, patient/parent satisfaction, need for additional analgesia and adverse events being reported. For the 28 children included in the primary analysis, median age was 9 years (interquartile range [IQR] 6-10). Twenty-three (82.1\%) were male. Eighteen (64\%) received only one dose of IN ketamine (mean dose 0.84 mg/kg), whereas 10 (36\%) required a second dose at 15 min (mean for second dose 0.54 mg/kg). The total mean dose for all patients was 1.0 mg/kg (95\% CI: 0.92-1.14). The median pain rating decreased from 74.5 mm (IQR 60-85) to 30 mm (IQR 12-51.5) at 30 min (P < 0.001, Mann-Whitney). For the 24 children who contributed data at 60 min, the median pain rating was 25 mm (IQR 4-44). Twenty (83\%) subjects were satisfied with their analgesia. Eight (33\%) were given additional opioid analgesia and the 28 reported adverse events were all transient and mild. In this population, an average dose of 1.0 mg/kg IN ketamine provided adequate analgesia by 30 min for most patients. [\hyperlink{Indium In-111 Pentetreotide Kit}{PMID: 23560967}, Fiona Yeaman et al., 2013]

\hypertarget{pmid_33447148}{P}re-hospital analgesic treatment of injured children is suboptimal, with very few children in pain receiving analgesia. Studies have identified a number of barriers to pre-hospital pain management in children which include the route of analgesia administration. The aim of this review is to critically evaluate the pre-hospital literature, exploring the safety and efficacy of intranasal (IN) analgesics for children suffering pain. We performed a rapid evidence review, searching from inception to 17 December 2018, CINAHL, MEDLINE and Google Scholar. We included studies of children < 18 years suffering pain who were administered any IN analgesic in the pre-hospital setting. Our outcomes were effective pain management, defined as a pain score reduction of ≥ 2 out of 10, safety and rates of analgesia administration. Screening and risk of bias assessments were performed in duplicate. We performed a narrative synthesis. From 310 articles screened, 23 received a full-text review resulting in 10 articles included. No interventional studies were found. Most papers reported on the use of intranasal fentanyl (INF) (n = 8) with one reporting IN ketamine and the other IN S-ketamine. Narrative synthesis showed that INF appeared safe and effective at reducing pain; however, its ability to increase analgesia administration rates was unclear. The effectiveness, safety and ability of IN ketamine and S-ketamine to increase analgesia administration rates were unclear. There was no evidence for IN diamorphine for children in this setting. Interventional studies are needed to determine with a higher confidence the effectiveness and safety of IN analgesics (fentanyl, ketamine, S-ketamine, diamorphine) for children in the pre-hospital setting. [\hyperlink{Indium In-111 Pentetreotide Kit}{PMID: 33447148}, Gregory Adam Whitley et al., 2019]

\hypertarget{pmid_6424996}{I}n-111 oxine labeled white cells were used to diagnose acute inflammatory conditions in 42 children and adolescents, aged 6 weeks to 19 years. In 43 scans where a clinical correlation could be made, the test had a sensitivity of 81\% and a specificity of 94\%. There were no adverse reactions. For children the dose of In-111 recommended is 10-12 mu Ci/kg body weight to a maximum of 500 mu Ci. [\hyperlink{Indium In-111 Pentetreotide Kit}{PMID: 6424996}, M A Gainey et al., 1984]

\hypertarget{pmid_24598046}{W}e performed a prospective, randomized, placebo-controlled study aimed to evaluate the efficacy and safety of a sedation protocol based on intranasal Ketamine and Midazolam (INKM) administered by a mucosal atomizer device in uncooperative children undergoing gastric aspirates for suspected tuberculosis. evaluation of Modified Objective Pain Score (MOPS) reduction in children undergoing INKM compared to the placebo group. evaluation of safety of INKM protocol, start time sedation effect, duration of sedation and evaluation of parents and doctors' satisfaction about the procedure. In the sedation group, 19 children, mean age 41.5 months, received intranasal Midazolam (0.5 mg/kg) and Ketamine (2 mg/kg). In the placebo group, 17 children received normal saline solution twice in each nostril. The child's degree of sedation was scored using the MOPS. A questionnaire was designed to evaluate the parents' and doctors' opinions on the procedures of both groups. Fifty-seven gastric washings were performed in the sedation-group, while in the placebo-group we performed 51 gastric aspirates. The degree of sedation achieved by INMK enabled all procedures to be completed without additional drugs. The mean duration of sedation was 71.5 min. Mean MOPS was 3.5 (range 1-8) in the sedation-group, 7.2 (range 4-9) in the placebo-group (p <0.0001). The questionnaire revealed high levels of satisfaction by both doctors and parents in the sedation-group compared to the placebo-group. The only side effect registered was post-sedation agitation in 6 procedures in the sedation group (10.5\%). Our experience suggests that atomized INKM makes gastric aspirates more acceptable and easy to perform in children. Unique trial Number: UMIN000010623; Receipt Number: R000012422. [\hyperlink{Indium In-111 Pentetreotide Kit}{PMID: 24598046}, Danilo Buonsenso et al., 2014]

\hypertarget{pmid_6240599}{I}n a double blind randomized study during a period of 2 weeks we compared the therapeutic effectiveness and side effects of IK-6-Inhaletten (0.1 mg Fenoterol + 0.04 mg Ipratropiumbromide) and SCH 1000-Inhaletten (0.2 mg Ipratropiumbromide) in 39 children (4-14 years) suffering from mild, moderate or severe asthma bronchiale. All measurements were performed with a whole body plethysmograph. In contrast to SCH 1000-inhalation after inhalation of IK-6-Inhaletten, we found a good improvement of the total airway resistance Rtot, the specific airway resistance SRaw and the forced exspiratory volume FEV1. Especially SRaw was significantly diminished compared to the less effective SCH 1000-inhalation. IK-6-inhalation allowed to decrease the amount of bronchospasmolytic therapy in our group of patients. We did not observe any severe side effects after inhalation of IK-6 or SCH 1000. In summary, we recommend the application of the IK-6-Inhaletten in children suffering from mild and moderate asthma bronchiale. [\hyperlink{Indium In-111 Pentetreotide Kit}{PMID: 6240599}, T Zimmermann et al., 1984]

\hypertarget{pmid_18496113}{P}rocedural sedation is increasingly more common in pediatric emergency departments. We report our experience with intranasal midazolam (INM) using a unique atomization delivery device, specifically the efficacy and safety of this method of sedation. We performed a retrospective chart review of children who received INM sedation in the emergency department from April 1, 2005, through June 30, 2005. All children aged 1 to 60 months who received INM as the initial means of sedation were eligible for the study. Patients were excluded if they were older than 60 months. There were 205 patients who received INM for sedation and who met the study criteria. The mean age was 31.3 +/- 13.2 months (range, 1.5-60 months). The mean and median initial INM dose was 0.4 mg/kg (range, 0.3-0.8 mg/kg). Laceration repair was the most common procedure necessitating sedation (89\%). The median degree-of-sedation score achieved was 2.0 (anxiolysis). Eleven patients (5.4\%; 95\% CI, 3\%-9\%) required an additional sedative to complete the procedure. Ten of the 11 patients received ketamine as the adjunctive sedative, and 1 patient required additional INM. The average time of last oral intake to start of sedation was 3.5 hours (range, 0.5-10.0 hours). Thirty six patients (18\%) were NPO for 2 hours or less. There was 1 adverse event (0.5\%; 95\% CI, 0\%-3\%). This was a minor desaturation episode following ketamine administration requiring brief blow by oxygen. There were no adverse events (0\%; 95\% CI, 0\%-2\%) in patients who received INM alone. We conclude that atomized INM is effective in providing anxiolysis to children undergoing minor procedures in the pediatric emergency department. We are encouraged that no adverse events occurred with the use of INM alone despite relatively short fasting times. [\hyperlink{Indium In-111 Pentetreotide Kit}{PMID: 18496113}, Roni D Lane et al., 2008]

\hypertarget{pmid_36658444}{T}he purpose of this study was to compare the efficacy of oral triclofos (TRI), intranasal midazolam (INM), and intranasal dexmedetomidine (IND) in achieving successful sedation in children undergoing MRI. This open-label, three-arm, randomized trial was conducted in a tertiary care teaching hospital over 18-month period. Children scheduled for MRI were enrolled. Rate of successful/adequate sedation was assessed using the Paediatric Sedation State Scale (PSSS). The primary outcome was the efficacy (successful sedation or sedation rate) of the three drugs. One-hundred and ninety-five children were included for the MRI procedure. IND was found to be superior in terms of achieving successful sedation. INM had a shorter onset and duration of sedation compared to IND and TRI, but with an increased failure rate (88.3\%). Keeping INM as the reference group, it was found that the odds of sedation increased 4.1 times on changing from INM to IND (p < 0.01), and 2.26 times on changing from INM to TRI (p < 0.01). Adverse events included nasal discomfort (18.3\%) in INM group; and self-limited tachycardia (4.6\%) and hypotension (10.8\%) in the IND group. IND was more efficacious than INM or TRI for procedural sedation in children undergoing MRI without any significant adverse events. CTRI/2019/01/017257; date registered: 25/01/2019. • Oral triclofos (TRI) and intranasal midazolam (INM) have been used for procedural sedation in children undergoing MRI with variable success; but the experience with intranasal dexmedetomidine (IND) is limited. • IND provides more effective sedation compared to INM or TRI for procedural sedation in children undergoing MRI, without any significant adverse events. [\hyperlink{Indium In-111 Pentetreotide Kit}{PMID: 36658444}, Shyam Chandrasekar et al., 2023]

\hypertarget{pmid_25095322}{T}o evaluate and compare the efficacy and safety of Intranasal (IN) Dexmedetomidine, Midazolam and Ketamine in producing moderate sedation among uncooperative pediatric dental patients. This randomized triple blind comparative study comprises of eighty four ASA grade I children of both sexes aged 4-14 years, who were uncooperative and could not be managed by conventional behavior management techniques. All the children were randomized to receive one of the four drug groups Dexmedetomidine 1 microg/ kg (D1), 1.5 microg/kg (D2), Midazolam 0.2 mg/kg (M1) and Ketamine 5 mg/kg (K1) through IN route. These drug groups were assessed for efficacy and safety by gauging overall success rate and by monitoring vital signs, respectively. The onset of sedation was significantly rapid with M1 and K1 as compared to D1 and D2 (p = < 0.001). The overall success rate was highest in D2 (85.7\%) followed by D1 (81\%), K1 (66.7\%) and M1 (61.9\%), however, the difference among them was not statistically significant (p = > 0.05). Even though all the vital signs were within physiological limits, there was significant reduction in pulse rate (PR) (p = < 0.001) and systolic blood pressure (SBP) (p = < 0.05) among D1 and D2 as compared to M1 and K1. D1, D2 and K1 produced greater intra- and post-operative analgesia as compared to M1. There were no significant adverse effects with any group. Dexmedetomidine, Midazolam and Ketamine, all the three drugs evaluated in the present study can be used safely and effectively through IN route in uncooperative pediatric dental patients for producing moderate sedation. [\hyperlink{Indium In-111 Pentetreotide Kit}{PMID: 25095322}, M Natarajan Surendar et al., 2014]

\hypertarget{pmid_16635173}{C}hildren often require relief of pain and anxiety when undergoing painful procedures. The purpose of this study is to evaluate the effectiveness and safety of painful pediatric procedures performed by pediatric intensivist, using the combination of intravenous ketamine and midazolam for sedation and analgesia. The records of the patients who received intravenous ketamine-midazolam combination for painful procedures in the pediatric sedation unit of a university hospital over a 3 year period were retrospectively reviewed to determine indications, dosing, assessment of the level of sedation, adverse events, and recovery time for each procedural sedation and analgesia. A total of 227 children aged 4 months to 18 years were admitted to the pediatric sedation unit for a total of 356 procedures. The indications for procedural sedation and analgesia included bone marrow aspiration or biopsy (50.8\%), central venous catheter insertion (27\%), and others (22\%). A total of 46 adverse events (12.9\%) were observed. These adverse events included SpO2 below 85\% without apnea (n = 14), apnea (n = 3), transient stridor (n = 2), hypertension and tachycardia (n = 8), hypersalivation (n = 6), vomiting (n = 5), hallucinatory emergence reaction (n = 4), and rash (n = 4). There were no adverse outcomes attributable to ketamine and midazolam combination. Skilled pediatric intensivists can safely and effectively administer ketamine and midazolam to facilitate painful procedures outside the operating room setting. [\hyperlink{Indium In-111 Pentetreotide Kit}{PMID: 16635173}, Bülent Karapinar et al., 2006]

\hypertarget{pmid_29288113}{T}o report current evidence regarding the safety of intramuscular botulinum toxin injection (BTI) in children with orthopedic- and neurologic-related musculoskeletal disorders >2 years of age. PubMed, Cochrane Library, and ScienceDirect, Google Scholar, and Web of Science. Two reviewers independently selected studies based on predetermined inclusion criteria. Data relating to the aim were extracted. Methodologic quality was graded independently by 2 reviewers using the Physiotherapy Evidence Database scale for randomized controlled trials (RCTs) and the Downs and Black evaluation tool for non-RCTs. Level of evidence was determined using the modified Sackett scale. Data of 473 infants were analyzed. Fifty-five infants had cerebral palsy, 112 had obstetric brachial plexus palsy, 257 had clubfoot, and 44 had congenital torticollis. No studies reported any severe adverse event that could be attributed to the BTI. The rate of mild to moderate adverse events reported varied from 5\% to 25\%. Results regarding efficacy were preliminary, dependent on the pathology, and limited by the small number of studies and their low levels of evidence. BTI is already widely used as an early treatment for this age group. The safety profile of BTI in infants appears similar to that of older children and risks appear more related to the severity of the pathology and the location of the injections than to the toxin itself. Regarding effectiveness, other studies with higher levels of evidence should be carried out for each specific pathology. [\hyperlink{Indium In-111 Pentetreotide Kit}{PMID: 29288113}, Jean-Sébastien Bourseul et al., 2018]

\hypertarget{pmid_21054515}{T}he aims of the study were to assess the long-term safety and compare neurodevelopmental outcomes in school-age children born prematurely who received inhaled nitric oxide or placebo during the first week of life in a randomized, double-blinded study. Children treated with inhaled nitric oxide had previously been shown to have decreased intraventricular haemorrhage and periventricular leukomalacia as newborns and decreased cognitive impairment at 2 years (L.W. Doyle and P.J. Anderson. (2005) Arch Dis Child Fetal Neonatal Ed, 90, F484-F8). It is follow-up study of medical outcomes, neurodevelopmental assessment and school readiness in 135 of 167 (81\%) surviving premature infants seen at 5.7±1.0 years. Compared to placebo-treated children (n=65), iNO-treated children (n=70) demonstrated no difference in growth parameters, school readiness or need for subsequent hospitalization. However, iNO-treated children were less likely to have multiple chronic morbidities or technology dependence (p=0.05). They also had less functional disability (p=0.05). These results demonstrate the long-term safety of iNO in premature infants. Furthermore, iNO treatment may improve health status by decreasing the incidence of severe ongoing morbidities and technology dependence and may also decrease the incidence of educational and community functional disability of premature infants at early school age. [\hyperlink{Indium In-111 Pentetreotide Kit}{PMID: 21054515}, Athena I Patrianakos-Hoobler et al., 2011]

\hypertarget{pmid_16326151}{I}ndomethacin (IND) is the drug of choice for the closure of a patent ductus arteriosus (PDA) in neonates. This paper describes a simple, sensitive, accurate and precise microscale HPLC method suitable for the analysis of IND in plasma of premature neonates. Samples were prepared by plasma protein precipitation with acetonitrile containing the methyl ester of IND as the internal standard (IS). Chromatography was performed on a Hypersil C(18) column. The mobile phase of methanol, water and orthophosphoric acid (70:29.5:0.5, v/v, respectively), was delivered at 1.5 mL/min and monitored at 270 nm. IND and the IS were eluted at 2.9 and 4.3 min, respectively. Calibrations were linear (r>0.999) from 25 to 2500 microg/L. The inter- and intra-day assay imprecision was less than 4.3 \% at 400-2000 microg/L, and less than 22.1\% at 35 microg/L. Inaccuracy ranged from -6.0\% to +1.0\% from 35 to 2000 microg/L. The absolute recovery of IND over this range was 93.0-113.3\%. The IS was stable for at least 36 h when added to plasma at ambient temperature. This method is suitable for pharmacokinetic studies of IND and has potential for monitoring therapy in infants with PDA when a target therapeutic range for IND has been validated. [\hyperlink{Indium In-111 Pentetreotide Kit}{PMID: 16326151}, M A Al Za'abi et al., 2006]

\hypertarget{pmid_21030365}{T}o evaluate the safety and efficacy of a sedation protocol based on intranasal lidocaine spray and midazolam (INM) in children who are anxious and uncooperative when undergoing minor painful or diagnostic procedures, such as peripheral line insertion, venipuncture, intramuscular injection, echocardiogram, CT scan, audiometry testing and dental examination and extractions. 46 children, aged 5-50 months, received INM (0.5 mg/kg) via a mucosal atomiser device. To avoid any nasal discomfort a puff of lidocaine spray (10 mg/puff) was administered before INM. The child's degree of sedation was scored using a modified Ramsay sedation scale. A questionnaire was designed to evaluate the parents' and doctors' opinions on the efficacy of the sedation. Statistical analysis was used to compare sedation times with children's age and weight. The degree of sedation achieved by INM enabled all procedures to be completed without additional drugs. Premedication with lidocaine spray prevented any nasal discomfort related to the INM. The mean duration of sedation was 23.1 min. The depth of sedation was 1 on the modified Ramsay scale. The questionnaire revealed high levels of satisfaction by both doctors and parents. Sedation start and end times were significantly correlated with age only. No side effects were recorded in the cohort of children studied. This study has shown that the combined use of lidocaine spray and atomised INM appears to be a safe and effective method to achieve short-term sedation in children to facilitate medical care and procedures. [\hyperlink{Indium In-111 Pentetreotide Kit}{PMID: 21030365}, Antonio Chiaretti et al., 2011]

\hypertarget{pmid_7590052}{T}o assess the safety and efficacy of intravenous sedation in pediatric upper endoscopy, all elective outpatient procedures performed during a 2-year period (January 1, 1991 through December 31, 1992) were retrospectively reviewed. Of 614 children, 553 received intravenous meperidine and midazolam; 61 received fentanyl and midazolam. The mean dose of meperidine was 1.5 +/- 0.7 mg/kg and of fentanyl 0.0031 +/- 0.0014 mg/kg. Less midazolam was needed for children receiving fentanyl than for those receiving meperidine (0.05 +/- 0.03 mg/kg versus 0.08 +/- 0.05 mg/kg, p < 002). Recovery time (minutes) was shorter for those receiving fentanyl (74.7 +/- 22.8 versus 95.1 +/- 23.0, p < .003). Side effects occurred in 117 patients (19.1\%), of which the majority were mild (83\%); all were transient with no residual sequelae. Inability to complete the procedure occurred in fewer than 1\%. We conclude that both combinations of medication are safe and effective for children of all ages. The use of fentanyl/midazolam results in a shorter recovery time and a lower dose of midazolam. [\hyperlink{Indium In-111 Pentetreotide Kit}{PMID: 7590052}, E Chuang et al., 1995]

\hypertarget{pmid_32198128}{P}lasma-derived C1-inhibitor (pdC1-INH) is a first-line therapy for hereditary angioedema (HAE) with C1-inhibitor deficiency (C1-INH-HAE) in pediatric patients. We intended to study the clinical characteristics and safety of treatment with pdC1-INH in this population. In the prospective, long-term survey, real-world data on pdC1-INH (Berinert, CSL Behring) use in pediatric patients, diagnosed and followed up at our Angioedema Reference Center, were analyzed for the period from 1986 to 2018. A total of 70 pediatric patients (31 boys and 39 girls) experienced a total of 3009 HAE attacks. The most common location of HAE attacks was subcutaneous. HAE attacks of any location were more frequent in girls versus boys, except for genital edema. Among the 70 patients, 37 received pdC1-INH for 456 HAE attacks, or as prophylaxis (69 vials). On average, 14.2 vials were administered per patient. The distribution of pdC1-INH use in the different age groups was as follows: no use (0-1 years), 0.11 vials/year (1-3 years), 0.7 vials/year (3-6 years), 1.26 vials/year (6-12 years), and 1.28 vials/year (12-18 years). No systemic allergic reactions, viral transmission, development of anti-C1-INH antibodies, or thromboembolic events occurred in relation to treatment with this drug. We confirmed that the clinical manifestations and the use of pdC1-INH are different in the various age groups of pediatric patients with C1-INH-HAE. Our long-term survey shows that the use of pdC1-INH is safe in this patient population. [\hyperlink{Indium In-111 Pentetreotide Kit}{PMID: 32198128}, Henriette Farkas et al., ]

\hypertarget{pmid_6800281}{T}he authors relate their experience of 61 inhalation anesthesia of children from 5 months to 15 years years old. Head-tent is usually employed for intensive care as a method to administrate pure oxygen. Children are often afraid of the face-mask and tolerate head-tent easier. Three different protocols were studied: Nitrous oxide and oxygen mixture at different level (50 p. cent oxygen, 50 p. cent nitrous-oxide; or 30 p. cent oxygen, 70 p. cent nitrous oxide). The authors also used halothan in the inhalated mixture. The rebreathing level of CO2 in the head-tent according to the gas flow was measured. No incident, nor accident are related. This new anesthetic apparatus is easy to use, well accepted by children. [\hyperlink{Indium In-111 Pentetreotide Kit}{PMID: 6800281}, J P Postel et al., 1981]

\hypertarget{pmid_36520326}{P}ENTAXIM™ (Sanofi), DTaP-IPV//Hib, a pentavalent combination vaccine for protection against diphtheria, tetanus, pertussis, poliomyelitis, and invasive infections caused by Haemophilus influenzae type b, has been licensed in South Korea by the Ministry of Food and Drug Safety (MFDS) on May 9, 2016, and is currently used in routine vaccination. The aim of this phase IV study, conducted as a post-licensure commitment in South Korea, was to evaluate the safety of the DTaP-IPV//Hib vaccine when administered in infants at 2, 4, and 6 months of age in the real-world clinical practice. This multicenter, observational, post-marketing surveillance (PMS) study was conducted in real-world practice in South Korea. Infants aged 2 months or older were enrolled across seven centers from July 31, 2018 to February 11, 2020. The study outcomes included occurrence, time to onset, duration, intensity, and causality assessment (for unsolicited adverse events [AEs] only) for several pre-listed solicited injection-site and systemic reactions, unsolicited AEs, and serious adverse events (SAEs). Data from 619 participants were included in the safety analysis. Overall, 618 AEs were reported by 273 (44.1\%) participants consisting of 121 solicited injection-site reactions (15.4\%), 344 solicited systemic reactions (24.6\%), and 153 unsolicited AEs (15.7\%) of which, 124 were unexpected AEs (12.9\%) (regardless of intensity). None of the unsolicited AEs were reported to have a causal relationship with the study vaccine. One SAE of pyrexia (solicited reaction) was reported. Most AEs were of mild intensity, and all participants recovered. This PMS study of the DTaP-IPV//Hib vaccine confirmed its safety profile in a real-life setting in South Korea and justified that the vaccine is well tolerated when used in infants aged 2 months or older for the primary series. [\hyperlink{Indium In-111 Pentetreotide Kit}{PMID: 36520326}, Kuhyun Yang et al., 2023]

\hypertarget{pmid_30394192}{I}n children, intravenous anesthetic premedication can be distressing. Intranasal (IN) ketamine offers a less invasive approach. We included randomized trials of IN ketamine in anesthetic premedication in children 0-19 years. We performed electronic searches of MEDLINE, EMBASE, Google Scholar, CINAHL, Cochrane Library, Web of Science, Scopus, clinical trial registries and conference proceedings. Among the 23 trials (n = 1680) included, IN ketamine adequately sedated 220/311 (70\%) for face mask application, 217/308 (70\%) for caregiver separation, 200/371 (54\%) for iv. insertion and 19/30 (63\%) for monitor application. Vomiting was the most common adverse effect (35/1579 [2.2\%]). There is a need for sufficiently powered, methodologically rigorous trials, using psychometrically evaluated, objective outcome measures to meaningfully inform practice. [\hyperlink{Indium In-111 Pentetreotide Kit}{PMID: 30394192}, Naveen Poonai et al., 2018]

\hypertarget{pmid_33884980}{P}rocedural sedation required to improve the quality of Transthoracic Echocardiography (TTE) in infants and children. The ideal drug and route for sedation in children should have a rapid and reliable onset, atraumatic, palatable with minimal side effects, and rapid recovery. So, the aim of our study to evaluate and compare the efficacy and safety of intranasal midazolam and intranasal dexmedetomidine in pediatric patients for sedation during TTE. Hundred children under three year of age, belonging to the American Society of Anaesthesiologists class-I and II, scheduled for TTE were divided into two groups by standard randomization technique. Patients in group-M received intranasal midazolam 0.2 mg/kg, whereas patients in group-D received intranasal dexmedetomidine 2 μg/kg prior to TTE under an adequately monitored anesthesia care. Onset and duration of sedation, heart rate, oxygen saturation, sonographer's, and parent's satisfaction scores were recorded. All patients were successfully sedated for TTE. The average onset time, sedation time, awakening time and total time for Group-M were 7.3, 18.8, 29.51, 51 min and group-D were 10.1, 14.2, 24.9, 46.3 min, respectively and all were statistically significant (P < 0.001). TTE scan time of Group-M is 8.84 min and Group-D is 9.18 min and was statistically significant. Sonographer's and Parent's average satisfaction score for Group-M was 9.88, 10 and for Group-D was 7.64, 8.76, respectively, which were statistically significant (P < 0.001). Intranasal midazolam and dexmedetomidine are safe and effective for sedation in TTE. Intranasal midazolam was found to be comparatively more effective in view of onset of action, sonographers, and parental satisfaction score, while sedation time, awakening time and total duration was significantly higher as compared to intranasal dexmedetomidine. [\hyperlink{Indium In-111 Pentetreotide Kit}{PMID: 33884980}, Suvendu Panda et al., ]

\section*{Pyridoxine Hydrochloride}
\subsection*{Result}
\subsubsection*{Answer}

Yes (3 months to adolescence)

\subsubsection*{{Explanation}}
\hypertarget{Pyridoxine Hydrochloride}
To determine if Pyridoxine Hydrochloride is safe for use in children, I reviewed all available abstracts for targeted studies on its safety in pediatric populations. The following relevant studies were identified:

1. **Children with Atopic Dermatitis (unspecified age, but pediatric):**
   - A randomized, double-blind, placebo-controlled trial in 48 children with moderate or severe atopic dermatitis administered pyridoxine hydrochloride 50 mg once daily for 4 weeks. The study found no statistically significant difference in clinical benefit between pyridoxine and placebo, but did not report any adverse effects or safety concerns, suggesting it was tolerated in this population [\hyperlink{pmid_8555030}{PMID: 8555030}, D C Mabin et al., 1995].

2. **Children with Febrile Convulsions (age not specified, but pediatric):**
   - In a randomized trial, 55 children received pyridoxine hydrochloride (20 mg twice daily) and 52 received placebo for up to one year. The study focused on efficacy for preventing recurrent febrile convulsions and did not report adverse effects, implying no major safety concerns during the study period [\hyperlink{pmid_7009019}{PMID: 7009019}, J McKiernan et al., 1981].

3. **Children with Recurrent Seizures (infants and children):**
   - A study of 90 infants and children with recurrent convulsions (primarily due to acute infectious diseases) compared high-dose pyridoxine (30 or 50 mg/kg/day IV) to controls. No adverse effects of pyridoxine were apparent during the observation period, and the authors concluded pyridoxine is "effective, safe, well-tolerated" as an adjunct for recurrent seizures in children [\hyperlink{pmid_9308977}{PMID: 9308977}, F Y Jiao et al., 1997].

4. **Children with Infantile Spasms (3 months to 3 years):**
   - A pilot, randomized, open-label trial compared high-dose pyridoxine (30 mg/kg/day) plus prednisolone to prednisolone alone in 62 children. The adverse effects were comparable in both groups, and the authors state "high dose pyridoxine may be safe in children with infantile spasms" [\hyperlink{pmid_29547159}{PMID: 29547159}, Vedavathi Kunnanayaka et al.].

5. **Children with Infantile Spasms (recently manifested, 17 children):**
   - High-dose pyridoxine-HCl (300 mg/kg/day orally) was used as initial treatment. No serious adverse reactions were noted; side effects were mainly reversible gastrointestinal symptoms. The authors suggest a controlled clinical trial is justified to further assess efficacy [\hyperlink{pmid_8330589}{PMID: 8330589}, J Pietz et al.].

6. **Case Report: Acute Poisoning in a 32-month-old Child:**
   - Pyridoxine IV was used effectively for seizures in a poisoning case. No safety concerns were reported in this context [\hyperlink{pmid_6866642}{PMID: 6866642}, E Hartemann et al.].

7. **Survey of Parenteral Pyridoxine Availability in Pediatric Hospitals:**
   - The abstract notes that pyridoxine is a recommended antidote for acute seizures secondary to pyridoxine dependency or responsiveness in children, but does not provide new safety data [\hyperlink{pmid_16160662}{PMID: 16160662}, Sidney M Gospe et al., 2005].

8. **Children with Asthma (76 children, age not specified):**
   - A double-blind study with 76 asthmatic children followed for five months indicated significant improvement with pyridoxine therapy (200 mg daily). The abstract notes a "lack of toxicity of this vitamin at relatively high doses (1 or 1.5 g)" but does not detail adverse events [\hyperlink{pmid_94242}{PMID: 94242}, M Garcia et al.].

Summary by Age Range:
- **Infants and Young Children (3 months to 3 years):** Multiple studies (including randomized trials) using high-dose pyridoxine hydrochloride for infantile spasms and seizures report no serious adverse effects, with only mild, reversible gastrointestinal symptoms in some cases [\hyperlink{pmid_29547159}{PMID: 29547159}; \hyperlink{pmid_8330589}{PMID: 8330589}].
- **Children (age not always specified, but pediatric):** Studies in children with atopic dermatitis, febrile convulsions, and asthma report no significant safety concerns or adverse effects with pyridoxine hydrochloride at various doses [\hyperlink{pmid_8555030}{PMID: 8555030}; \hyperlink{pmid_7009019}{PMID: 7009019}; \hyperlink{pmid_94242}{PMID: 94242}].
- **Case report in a 32-month-old:** No safety concerns reported in acute use [\hyperlink{pmid_6866642}{PMID: 6866642}].

Conclusion:
There are targeted studies in infants and children (including those as young as 3 months) that specifically address the safety of pyridoxine hydrochloride, with no serious adverse effects reported and only mild, reversible side effects at high doses. Therefore, based on the available abstracts, pyridoxine hydrochloride appears to be safe for use in children, including infants as young as 3 months, when used as studied.

\subsection*{Abstracts}
\hypertarget{pmid_8555030}{A} previous study has reported benefit when pyridoxine hydrochloride was given to patients with atopic dermatitis. To investigate this in children, we performed a randomized, double-blind, parallel-group, placebo-controlled trial. Forty-eight children with moderate or severe atopic dermatitis were recruited and, of those who completed the study, 19 received pyridoxine hydrochloride 50 mg once daily for 4 weeks and 22 received placebo. Disease activity was monitored by clinical severity scores measuring the extent and degrees of erythema recorded by the investigator and symptom scores (daytime itch and nocturnal sleep disturbance) recorded by parents. There was no statistically significant difference between the two groups at the end of treatment. We have been unable to demonstrate clinical benefit from pyridoxine supplementation in children with atopic dermatitis. [\hyperlink{Pyridoxine Hydrochloride}{PMID: 8555030}, D C Mabin et al., 1995]

\hypertarget{pmid_7009019}{A} total of 107 children who had been hospitalized following a febrile convulsion were enrolled into the trial. By random allocation, 55 children were treated with pyridoxine hydrochloride (20 mgs twice daily) and the remaining 52 children were treated with a placebo until there had been either a further convulsion or a year had passed without recurrence. Eighty children were adequately followed up and of these, 17 had a recurrent febrile convulsion while receiving medication. Recurrences occurrences occurred in 7 of the 38 children receiving pyridoxine and in 10 of the 42 children receiving placebo (X2 = .346, p greater than 0.5). Initial tryptophan load tests had been abnormal in 34 children, and of these, recurrences occurred in 3 of the 17 who received pyridoxine and in 3 of the 17 who received placebo. It has yet to be shown that pyridoxine supplementation protects children from recurrent febrile convulsions. [\hyperlink{Pyridoxine Hydrochloride}{PMID: 7009019}, J McKiernan et al., 1981]

\hypertarget{pmid_8010205}{T}he purpose of this prospective study was to evaluate the safety and efficacy of thioridazine as an adjunct to chloral hydrate sedation when children undergoing MR imaging are difficult to sedate. All 87 children in the study either could not be sedated with chloral hydrate alone or were mentally retarded. Thioridazine (2-4 mg/kg) was administered orally 2 hr before and chloral hydrate (50-100 mg/kg) was administered orally 30 min before the 104 MR examinations. All children were monitored by continuous pulse oximetry. All images were individually evaluated by pediatric radiologists and were graded acceptable if they contained only minimal motion artifact or no motion artifact. Studies were considered successful only when 95\% or more of the images were acceptable. MR imaging was successful in 93 (89\%) of 104 examinations. The success rate for children entered into the study because of prior failure of chloral hydrate sedation was not significantly different from the success rate for children with mental retardation. A tendency for increasing failure rate with age was not significant. No serious complications occurred during the study. The most common adverse reaction, transient reduced oxygen saturation, was seen in five children. Other adverse effects encountered were vomiting in four children, hyperactivity in two children, transient tachycardia in one child, and prolonged sedation in one child. No child required hospitalization because of an adverse reaction to sedation. The study indicates that thioridazine is a safe and effective adjunct to chloral hydrate when a child undergoing MR imaging is difficult to sedate. [\hyperlink{Pyridoxine Hydrochloride}{PMID: 8010205}, S B Greenberg et al., 1994]

\hypertarget{pmid_9308977}{T}o determine the efficacy of pyridoxine in treating seizures, 90 infants and children with recurrent convulsions primarily due to acute infectious diseases were enrolled in the present study. Forty patients were treated with high-dose pyridoxine (30 or 50 mg/kg/day) by intravenous infusion, and 50 subjects served as controls. Antiepileptic drugs and other therapies were similar in the two groups except for pyridoxine. Clinical efficacy criteria were based on the frequency of convulsions per day and on the duration of individual seizures after therapy was initiated. The results indicated that total response rates in the pyridoxine group and control group were 92.5\% and 64\%, respectively (chi-square = 14.68, P < .001). After initiation of therapy, seizures resolved after 2.4 +/- 1.4 days in the pyridoxine group and after 3.7 +/- 2.0 days in the control group (t = 3.67, P < .001). No adverse effects of pyridoxine were apparent during the observation period. We conclude that pyridoxine is an effective, safe, well-tolerated, and relatively inexpensive adjunct to routine antiepileptic drugs for treatment of recurrent seizures in children. [\hyperlink{Pyridoxine Hydrochloride}{PMID: 9308977}, F Y Jiao et al., 1997]

\hypertarget{pmid_10323625}{Y}oung children often appear bothered by ear pain during ascent and descent while traveling on commercial airplanes. While pseudoephedrine hydrochloride is effective in decreasing the risk for earache in adults with recurrent air travel-associated ear pain, such use in children has not been studied. To assess the efficacy and side effects of prophylactic pseudoephedrine in children traveling by air. A placebo-controlled, double-blind clinical trial. Children aged 6 months to 6 years were included in this study. Pseudoephedrine hydrochloride (1 mg/kg body weight) or placebo was administered 30 to 60 minutes prior to departure on commercial air flights. Caregivers noted historical details and the degree of apparent ear pain, drowsiness, and excitability with ascent and descent. Ninety-one flights involving 50 children were studied, with ear pain being reported in 13 (14\%) of flights. Ear pain was not associated with a history of air travel-associated ear pain, recent ear infection, or recent upper airway symptoms. Pseudoephedrine use was not associated with a decrease in ear pain during either ascent (4\% with pseudoephedrine vs 5\% with placebo; P approximately 1.00) or descent (12\% with pseudoephedrine vs. 13\% with placebo; P approximately 1.00). Pseudoephedrine use was, however, linked to drowsiness at takeoff (60\% with pseudoephedrine vs. 27\% with placebo; P = .003) but not at landing (P = .39). Treatment was not associated with excitability at takeoff (P = .09) or landing (P approximately 1.00). Ear pain is not uncommon in children traveling by commercial aircraft. The predeparture use of pseudoephedrine does not decrease the risk for in-flight ear pain in children but is associated with drowsiness. [\hyperlink{Pyridoxine Hydrochloride}{PMID: 10323625}, B J Buchanan et al., 1999]

\hypertarget{pmid_16160662}{P}yridoxine is a recommended antidote that should be available in emergency departments (EDs). A pediatric use of this preparation is the treatment of acute seizures secondary to pyridoxine dependency or responsiveness. Two cases of children with pyridoxine-dependent and pyridoxine-responsive seizures whose treatment was affected by the unavailability of pyridoxine in local EDs are presented. These cases prompted the development of a survey to ascertain the availability of parenteral pyridoxine in the pharmacies and EDs of both children's and general hospitals in the United States. A survey of 203 pharmacy directors in 100 pediatric hospitals (42 self-governing and 58 within a hospital) and 103 general hospitals was conducted. The questionnaire asked for the number of licensed beds and whether injectable pyridoxine was on the formulary and stocked by the ED. The overall response rate was 73\% (83\% pediatric and 64\% general hospitals). Injectable pyridoxine was on the formulary of 99\% of pediatric hospitals and 91\% of general hospitals (P = 0.044). Of those hospitals that had pyridoxine on the formulary, the availability of injectable pyridoxine in EDs was low in both pediatric (20.7\%) and general hospitals (16.7\%). Given the number of possible uses of parenteral pyridoxine in the ED, it is suggested that there is a case for all pediatric and general hospital pharmacies to have it on the formulary and further for all EDs in these hospitals to have injectable pyridoxine available for immediate use. [\hyperlink{Pyridoxine Hydrochloride}{PMID: 16160662}, Sidney M Gospe et al., 2005]

\hypertarget{pmid_7857353}{T}he efficacy and safety of pidotimod ((R)-3-[(S)-(5-oxo-2-pyrrolidinyl)carbonyl]-thiazolidine-4-carboxylic acid, PGT/1A, CAS 121808-62-6) were rated in a child population with a remote history of recurrent respiratory infections (RRI). This randomized double-blind multicenter clinical trial versus placebo, stratified by age groups, involved 748 children recruited in 69 Medical Centres. The trial consisted of a 60-day treatment period and a 90-day follow-up. At the end of the treatment period the pidotimod group showed a significant decrease in the number of RRI episodes and associated symptoms vs control group. As a consequence, there was a significant decrease in the number of days of absence from kindergarten or school and in the consumption of antibiotics and symptomatic drugs. Safety was good. The effect of the drug persisted after its withdrawal throughout the whole 90-day follow-up period. During this period there was a significantly lower RRI incidence rate in the pidotimod group than in the placebo group (p < 0.01). Because of its efficacy and safety, pidotimod may be rated as an excellent drug in the RRI management in children. [\hyperlink{Pyridoxine Hydrochloride}{PMID: 7857353}, P Careddu et al., 1994]

\hypertarget{pmid_16176855}{S}tudies of children indicate that exposure of the general population to low levels of polychlorinated dibenzo-p-dioxins and dibenzofurans (PCDD/Fs) does not result in any clinical evidence of disease, although accidental exposure to high levels either before or after birth have led to a number of developmental deficits. Breast-fed infants have higher exposures than formula-fed infants, but studies consistently find that breast-fed infants perform better on developmental neurologic tests than their formula-fed counterparts, supporting the well-recognized benefits of breast feeding. Children receive higher exposures to PCDD/Fs from food than adults on a body-weight basis but those exposures are below the World Health Organization's tolerable daily intake. Laboratory rodents appear to be at least an order of magnitude more sensitive than humans to the aryl hydrocarbon receptor-mediated effects of these substances, which makes them poor surrogates for predicting quantitative risks but makes them good models for establishing safe levels of human exposure by organizations mandated to protect public health. Any exposure limit for PCDD/Fs based on developmental toxicity in sensitive laboratory animals can be expected to be especially protective of human health, including the health of infants and children. Because body burdens and environmental levels continue to decline, it is unlikely that children alive today in the USA will experience exposures to PCDD/Fs that are injurious to their health. [\hyperlink{Pyridoxine Hydrochloride}{PMID: 16176855}, Gail Charnley et al., 2006]

\hypertarget{pmid_29547159}{W}est syndrome is a catastrophic epilepsy syndrome characterized by infantile spasms, hypsarrhythmia, and developmental arrest or regression. The aim of this study was to explore the role of pyridoxine in the management of infantile spasms. This was a pilot, randomized, open-label trial conducted at a tertiary level hospital from November 2012 to March 2014. Children aged 3 months to 3 years presenting with infantile spasms in clusters (at least 1 cluster/day) with hypsarrhythmia or its variants on electroencephalogram (EEG) were enrolled. The study participants were randomized to receive either oral prednisolone (4 mg/kg/day) alone or 30 mg/kg/day of pyridoxine with oral prednisolone. The primary outcome measure was the proportion of children who achieved spasm freedom for 48 h on day-14 after treatment initiation, as per parental reports, in both the groups. The adverse effects were also monitored. The study was registered with clinicaltrials.gov (ClinicalTrials.gov Identifier: NCT01828437). Sixty-two children were randomized into the two groups with comparable baseline characteristics. The proportion of children with spasm cessation on day-14 was similar in the two groups (39 vs. 37\%, P = 0.98). The adverse effects were comparable in both the groups. The combination of pyridoxine with oral prednisolone was not found to be a beneficial therapy as compared to prednisolone alone in the treatment of infantile spasms in this pilot study. However, high dose pyridoxine may be safe in children with infantile spasms. [\hyperlink{Pyridoxine Hydrochloride}{PMID: 29547159}, Vedavathi Kunnanayaka et al., ]

\hypertarget{pmid_7857354}{T}he efficacy and safety of a new synthetic immunostimulant pidotimod ((R)-3-[(S)-(5-oxo-2-pyrrolidinyl) carbonyl]-thiazolidine-4-carboxylic acid, PGT/1A, CAS 121808-62-6) in recurrent infections of the primary airways were assessed in a group of 416 children with a history of recurrent respiratory infections (RRI). This was a double-blind randomized trial of pidotimod vs. placebo, consisting of a treatment period of 60 days and a follow-up period of 3 months. A reduction in the duration and frequency of infectious episodes in the group of children treated with pidotimod (one 400 mg oral bottle daily) was observed which was statistically different from the placebo group. The protective effect produced by pidotimod was also confirmed by a series of recordings made over the five-month observation period, which showed a significant reduction in the number of days of fever, the severity of the signs and symptoms of acute episodes, use of antibiotics and antipyretic drugs and absence from school or nursery school. Safety was excellent. [\hyperlink{Pyridoxine Hydrochloride}{PMID: 7857354}, D Passali et al., 1994]

\hypertarget{pmid_28590988}{V}ilazodone hydrochloride is the first member in a new class of antidepressants called indolealkylamines and was approved for use in the United States in 2011 for major depressive disorder. It has a combined mechanism of action of a selective serotonin reuptake inhibitor and a partial agonist of serotonin 5-HT1A receptors. It has not been approved for use in the pediatric population, and toxicity from exploratory vilazodone ingestion has been rarely described to date. We describe 2 children with laboratory-confirmed vilazodone ingestions that led to significant toxicity including refractory status epilepticus in 1 patient and likely transient seizure activity in the other. Both patients required multiple doses of benzodiazepines; in the more severe case, barbiturates were added to control seizure activity. These children returned to baseline and had no prolonged neurologic complications. Pediatric experience with vilazodone is limited; however, the literature demonstrates 3 additional case reports of children experiencing seizure after vilazodone ingestion. With the 2 new cases presented here, it seems prudent to educate prescribers and families of the potential dangers of ingestion of vilazodone tablets by young children. [\hyperlink{Pyridoxine Hydrochloride}{PMID: 28590988}, Jeannine Del Pizzo et al., 2018]

\hypertarget{pmid_3686545}{P}yridoxine hydrochloride was gavaged to 2 groups of pregnant Wistar rats from day 0 to 13 and day 6 to 15 of gestation at doses of 100, 200, 400 and 800 mg/kg. A higher number of implantations, live pups and corpora lutea were observed in the treated rats, but a significant reduction in the body weights of the pups was noticed in the groups treated with 400 and 800 mg/kg. No other adverse effects on implantation and pregnancy were noticed. No evidence of dismorphogenic effects was seen. [\hyperlink{Pyridoxine Hydrochloride}{PMID: 3686545}, M R Marathe et al., 1987]

\hypertarget{pmid_94242}{P}yridoxine, one of the B vitamins, has been shown to be useful in the treatment of childhood bronchial asthma by Collip et al. (1975). A double-blind study with 76 asthmatic children followed for five months indicated significant improvement in asthma following pyridoxine therapy (200 mg daily) and a reduction in dosage of bronchodilators and cortisone. Other reports have shown that nicotinamide, another B vitamin shows inhibitory activity in rat mast cell degranulation and histamine release (Bekier et al. 1974, Wiczolkowska and Maslinski, 1975, 1976). These results induced us to investigate if pyridoxine, like nicotinamide or disodium cromoglycate, exhibits pharmacological inhibitory activity in rat mast cell degranulation and histamine release induced by antigen or other non-immunological stimulants. We found that pyridoxine at concentrations of 10 (-3) M, or greater significantly inhibited rat mast cell degranulation and histamine release induced by phospholipase A, compound 48/80, antigen (egg albumin) or a mixture of dextran and phosphatidyl serine, respectively. In these experimental models, pyridoxine shows a pharmacological profile similar to nicotinamide and disodium cromoglycate, although weaker than the latter. In spite of this, the lack of toxicity of this vitamin at relatively high doses (1 or 1.5 g), the possibility that other mechanisms of action may be involved, such as the improvement in tryptophan metabolism reported by Collip following pyridoxine therapy, suggest that this vitamine merits additional research. [\hyperlink{Pyridoxine Hydrochloride}{PMID: 94242}, M Garcia et al., ]

\hypertarget{pmid_21516020}{H}ydroxyurea is a safe and efficacious medication for children with sickle cell disease (SCD). Our objective was to compare health-related quality of life (HRQL) between children taking hydroxyurea and those not taking hydroxyurea. We conducted a retrospective cohort study of children with SCD who had completed the PedsQL 4.0 at Duke University Medical Center or the Midwest Sickle Cell Center. Our primary outcome was HRQL in children receiving hydroxyurea therapy compared with those not receiving hydroxyurea. One hundred and ninety-one children with SCD were included in the study. Children in the hydroxyurea group had higher self-reported Total PedsQL median scores than children in the no hydroxyurea group (P=0.04). Child self-reported physical functioning scores were significantly higher for children in the hydroxyurea group (P=0.01). In conclusion, children with SCD who received hydroxyurea therapy reported better overall HRQL and better physical HRQL than children who did not receive this therapy despite disease severity. Further research assessing the impact of hydroxyurea therapy on HRQL, such as prospective assessment over time, would aid in our understanding of the effectiveness of hydroxyurea for individual children. Ultimately, this may aid in decreasing the barriers to the use of hydroxyurea. [\hyperlink{Pyridoxine Hydrochloride}{PMID: 21516020}, Courtney D Thornburg et al., 2011]

\hypertarget{pmid_17941284}{T}he safety of fexofenadine has been examined extensively in adults and school-age children. However, the safety of fexofenadine in children younger than 6 years has not been reported to date. To compare the safety and tolerability of twice-daily fexofenadine hydrochloride, 30 mg, and placebo in preschool children aged 2 to 5 years with allergic rhinitis. This was a multicenter, double-blind, randomized, placebo-controlled, parallel-group study, conducted between February 29, 2000, and June 14, 2001. Participants were randomized to either fexofenadine hydrochloride, 30 mg, or placebo twice daily for a 2-week period. To facilitate dosing, capsule content was mixed with applesauce (approximately 10 mL). Safety assessments depended on date of entry into the study because of an amendment to the protocol. Before the amendment, assessments included physical examination, vital signs reporting (oral temperature, heart rate, and respiratory rate), and adverse event (AE) reporting. After the amendment, safety assessments included laboratory testing (blood chemistry and hematology profiles), physical examination, 12-lead electrocardiography, and vital signs (oral temperature, blood pressure, heart rate, and respiratory rate) and AE reporting. Treatment-emergent AEs were observed in 116 of 231 participants receiving placebo and 111 of 222 receiving fexofenadine. These AEs were possibly related to study medication in 19 (8.2\%) and 21 (9.5\%) of the participants receiving placebo and fexofenadine, respectively, and most frequently involved the digestive system. No clinically relevant differences in laboratory measures, vital signs, and physical examinations were observed. The findings show that fexofenadine hydrochloride, 30 mg, is well tolerated and has a good safety profile in children aged 2 to 5 years with allergic rhinitis. [\hyperlink{Pyridoxine Hydrochloride}{PMID: 17941284}, Henry Milgrom et al., 2007]

\hypertarget{pmid_11423814}{P}yridoxine hydrochloride, the antidote for isonicotinic acid hydrazide (INH)--induced seizures, is available in solution at a concentration of 100 mg/mL at a pH of less than 3. Pyridoxine is often infused rapidly in large doses for INH-induced seizures. Effects of pyridoxine infusion on base deficit in amounts given for INH poisoning have not been studied in human subjects. We hypothesized that this infusion would result in transient worsening of acidosis. We conducted a randomized, controlled crossover trial in human volunteers. Five healthy volunteers (mean age, 35 years; range, 29 to 43 years) were randomized to receive intravenous placebo (50 mL of normal saline solution) or 5 g of pyridoxine (50 mL) over 5 minutes. A peripheral intravenous catheter was established in each arm, and a heparinized venous blood sample was obtained for base deficit at baseline and 3, 6, 10, 20, and 30 minutes after infusion. After at least a 1-week washout period, the volunteers were assigned to the alternate arms of the experiments, thus acting as their own control subjects. Data were analyzed by using the 2-tailed paired t test, controlling for multiple comparisons. No difference was noted between groups at baseline. A statistically significant increased base deficit was noted after the pyridoxine infusion versus control at 3 to 20 minutes but not at 30 minutes (P =.1). Maximal mean increase in base deficit (2.74 mEq/L) was noted at 3 minutes. A transient increase in base deficit occurs after the infusion of 5 g of pyridoxine in normal volunteers. [\hyperlink{Pyridoxine Hydrochloride}{PMID: 11423814}, F Lovecchio et al., 2001]

\hypertarget{pmid_18365604}{P}seudoephedrine hydrochloride (PEH) is a sympathomimetic agent that is widely used in common cold disease in children. Though side effects of PEH are well known, it is preferred by many pediatricians in order to benefit from its symptomatic relief in common cold disease. A case of acute urinary retention due to PEH in a three-year-old boy is reported. The aim of this case report is to emphasize the clinical importance and differential diagnosis of PEH overdose in children and to discuss the appropriate treatment approach to PEH overdose in the emergency department. [\hyperlink{Pyridoxine Hydrochloride}{PMID: 18365604}, Tutku Soyer et al., ]

\hypertarget{pmid_2295577}{F}luoxetine hydrochloride is the first selective serotonin uptake inhibitor introduced commercially in the United States. This report describes preliminary clinical experience with fluoxetine in 10 children and adolescents, aged 8 to 15 years, with primary obsessive compulsive disorder (OCD) or Tourette's syndrome (TS) plus OCD. In general, fluoxetine, which was administered from 4 to 20 weeks at a dosage of 10 or 40 mg per day, was well tolerated. Adverse effects included behavioral agitation/activation in four patients and mild gastrointestinal symptoms in two patients. No abnormalities were noted in the seven children who had follow-up EKGs. Five of the 10 patients (50\%) were considered responders; their obsessive-compulsive symptoms decreased substantially during treatment with fluoxetine. Responder rates were similar in the primary OCD (two of four, 50\%) and TS + OCD (three of six, 50\%) groups. In conclusion, short-term fluoxetine administration appears to be safe in children and adolescents. Placebo-controlled trials are needed to further assess the efficacy of fluoxetine. [\hyperlink{Pyridoxine Hydrochloride}{PMID: 2295577}, M A Riddle et al., 1990]

\hypertarget{pmid_32925756}{B}romhexine hydrochloride tablets may be effective in the treatment of Coronavirus disease 2019 (COVID-19) in children. This study will further evaluate the efficacy and safety of bromhexine hydrochloride tablets in the treatment of COVID-19 in children. The following electronic databases will be searched, with all relevant randomized controlled trials (RCTs) up to August 2020 to be included: PubMed, Embase, Web of Science, the Cochrane Library, China National Knowledge Infrastructure (CNKI), the Chongqing VIP China Science and Technology Database (VIP), Wanfang, the Technology Periodical Database, and the Chinese Biomedical Literature Database (CBM). As well as the above, Baidu, the International Clinical Trials Registry Platform (ICTRP), Google Scholar, and the Chinese Clinical Trial Registry (ChiCTR) will also be searched to obtain more comprehensive data. Besides, the references of the included literature will also be traced to supplement our search results and to obtain all relevant literature. This systematic review will evaluate the current status of bromhexine hydrochloride in the treatment of COVID-19 in children, to evaluate its efficacy and safety. This study will provide the latest evidence for evaluating the efficacy and safety of bromhexine hydrochloride in the treatment of COVID-19 in children. CRD42020199805. The private information of individuals will not be published. This systematic review will also not involve endangering participant rights. Ethical approval is not available. The results may be published in peer-reviewed journals or disseminated at relevant conferences. [\hyperlink{Pyridoxine Hydrochloride}{PMID: 32925756}, Yuying Wang et al., 2020]

\hypertarget{pmid_17063023}{E}vidence on the caries-preventive effect of chlorhexidine (CHX) among high-risk children is inconclusive, possibly because obscured by fluoride exposure. We investigated the effect of CHX among initially 3-year-old subjects whose baseline d(3)ft was = 0 and whose only regular fluoride exposure came from toothpaste. The subjects were assigned to three groups: high-risk test (HRT, n = 70), high-risk control (HRC, n = 71), and low-risk control (LRC, n = 70). Risk classification was based on salivary mutans streptococcal levels (MS, </>or=1.0 x 10(5) cfu/ml). Basic measures (oral hygiene, dietary counselling every 4 months) were given to all groups. HRT also underwent CHX gel applications for 3 consecutive days at 3-month intervals for 15 months. Eighteen months after baseline d(3)ft increments and proportions of children with d(3)ft increment >or=1 (\%d(3)ft increment >or=1) among all groups were assessed. Anti-MS effect on high-risk children and caries-preventive effect on all children were statistically analysed by residual change analysis (MS), non-parametric tests and logistic regression analysis (caries). No differences were found between the groups in basic programme compliance. CHX significantly reduced MS levels. \%d(3)ft increment >or=1 and mean d(3)ft increments were 34.3\%, 0.56 (HRT), 32.4\%, 0.54 (HRC) and 11.4\%, 0.11 (LRC), with HRT/HRC values statistically significantly higher than LRC values and no significant difference between HRT and HRC. HRT children were not less likely to show new lesions than HRC children (OR = 1.09; 95\% confidence interval 0.54-2.19), while high-risk children were 4 times more likely to show new lesions than low-risk children (OR = 3.71; 95\% confidence interval 1.53-9.03). CHX gel applications showed moderate anti-MS effect but negligible caries-preventive effect. [\hyperlink{Pyridoxine Hydrochloride}{PMID: 17063023}, S Petti et al., 2006]

\hypertarget{pmid_10851644}{C}iprofloxacin clinical and bacteriological efficacies, as well as tolerability mainly with respect to chondrotoxicity were evaluated in the treatment of children with mucoviscidosis. The drug was shown to have high clinical and moderate bacteriological efficacies. As for its tolerability, adverse reactions chiefly associated with affection of the gastrointestinal tract, i.e. nausea, stomach pain, diarrhea, increased transaminase levels were recorded. The arthrotoxicity episode was single and transitory. The use of ciprofloxacin had no negative effect on the children growth. No chondrotoxic effect of ciprofloxacin in the treatment of children was observed which is explained in the paper. It is concluded that ciprofloxacin is in general an efficient and safe antibiotic useful for the treatment of children. [\hyperlink{Pyridoxine Hydrochloride}{PMID: 10851644}, S S Postnikov et al., 2000]

\hypertarget{pmid_8330589}{H}igh-dose vitamin B6 (pyridoxine-HCl, 300 mg/kg/day orally) was introduced as the initial treatment of recently manifested infantile spasms in 17 children (13 symptomatic cases with identified brain lesion and 4 cryptogenic cases). 5 of 17 children (2 cryptogenic, 2 with severe pre/perinatal brain damage and one with Sturge-Weber syndrome) were classified as responders to high-dose vitamin B6. In all 5 cases the response to vitamin B6 occurred within the first 2 weeks of treatment and within 4 weeks all patients were free of seizures. Two patients developed other seizures (partial seizures, etiologically unclear blinking attacks), but no relapse of infantile spasms was observed among the five responders to vitamin B6. No serious adverse reactions were noted. Side effects were mainly gastrointestinal symptoms, which were reversible after reduction of the dosage. Considering the life-threatening side effects of treatment with ACTH/corticosteroids or valproate, a controlled clinical trial with high-dose vitamin B6 would appear justified to either prove or disprove efficacy. [\hyperlink{Pyridoxine Hydrochloride}{PMID: 8330589}, J Pietz et al., ]

\hypertarget{pmid_34506721}{T}he contents of perchlorate and chlorate were determined in a total of 278 samples of infant formulas marketed in China. The associated health risk via infant and young child formulas consumption for 0-36 month old children in China was also assessed. The contents of perchlorate and chlorate were measured by a validated method with LC-MS and the limit of detection (LOD) was 1.5 μg kg [\hyperlink{Pyridoxine Hydrochloride}{PMID: 34506721}, Qing Liu et al., 2021] A case of acute poisoning in a 32 months old child, with generalised and uncontrollable seizures is reported. Pyridoxine IV is efficacely used in such poisoning. [\hyperlink{Pyridoxine Hydrochloride}{PMID: 34506721}, E Hartemann et al., ]

\hypertarget{pmid_28741653}{C}hloral hydrate is commonly used to sedate children for painless procedures. Children may recover more quickly after sedation with dexmedetomidine, which has a shorter half-life. We randomly allocated 196 children to chloral hydrate syrup 50 mg.kg [\hyperlink{Pyridoxine Hydrochloride}{PMID: 28741653}, V M Yuen et al., 2017]

\section*{Terbutaline Sulfate}
\subsection*{Result}
\subsubsection*{Answer}

Infants (<1 year): Unknown  
Children 1–5 years: Yes  
Children 3–10 years: Yes  
Children 6–14 years: Yes  
Children 1–16 years: Yes  

\subsubsection*{{Explanation}}
\hypertarget{Terbutaline Sulfate}
Based on the available abstracts, several targeted studies have evaluated the safety of Terbutaline Sulfate in children with asthma, across a range of ages. Here is a summary by age group:

Infants (<1 year):
- One case report describes a five-month-old girl who developed focal seizures after taking terbutaline sulfate syrup, with no other cause found. The seizures stopped after discontinuing the drug, suggesting a possible adverse effect [\hyperlink{pmid_37719479}{PMID: 37719479}, Falah Nadeem et al., 2023]. This is a single case report and not a controlled safety study, so definitive safety cannot be established for this age group.

Toddlers and Young Children (1–5 years):
- A randomized, controlled trial in children aged 2–5 years with acute asthma found that continuous nebulization of terbutaline sulfate was effective and showed evidence of absorption, but did not report significant adverse effects [\hyperlink{pmid_9677633}{PMID: 9677633}, J P Lotufo et al., 1998].
- Another study included children as young as 1 year and 7 months (up to 10 years) and found nebulized terbutaline (5 mg) to be safe and effective for acute asthma, with no significant changes in pulse rate or blood pressure [\hyperlink{pmid_2284942}{PMID: 2284942}, B F Lee et al.].
- A study in children aged 3–6 years compared terbutaline delivery methods and found both to be effective, with no significant adverse effects reported [\hyperlink{pmid_2677624}{PMID: 2677624}, J Pendergast et al., 1989].

Children (3–10 years):
- Two double-blind, placebo-controlled studies in children aged 3–10 years with mild to moderate asthma found that inhaled terbutaline sulfate was effective, with no significant adverse effects reported [\hyperlink{pmid_8875584}{PMID: 8875584}, E Ståhl et al., 1996].
- Another study in children aged 4–13 years found inhaled terbutaline to be safe and effective, with fewer cardiovascular side effects than injected terbutaline [\hyperlink{pmid_2751767}{PMID: 2751767}, P Phanichyakarn et al., 1989].
- A study in children aged 4.9–13.7 years found that terbutaline delivered via spacer was effective, with no significant adverse effects [\hyperlink{pmid_6374804}{PMID: 6374804}, K G Hidinger et al., 1984].

Older Children (6–14 years):
- A study in children aged 6–14 years with mild or moderate asthma found that both dry powder inhaler and nebulizer forms of terbutaline were effective and safe, with no significant changes in vital signs [\hyperlink{pmid_14689023}{PMID: 14689023}, D Solé et al.].
- A year-long study in children aged 7–14 years found no adverse effects on growth, bone marrow, liver function, or cardiovascular system with regular daily terbutaline aerosol therapy [\hyperlink{pmid_7308338}{PMID: 7308338}, M I Blackhall et al., 1981].
- A study in children aged 6–12 years using sustained-release terbutaline found a small, dose-related increase in peak expiratory flow, but no clinically useful protection against exercise-induced asthma; safety was not a concern [\hyperlink{pmid_7985255}{PMID: 7985255}, B Hertz et al., 1994].

General Pediatric Population (1–16 years):
- Several studies, including randomized controlled trials and pharmacokinetic studies, in children aged 1–16 years (and subgroups within this range) found terbutaline to be effective for acute asthma, with side effects generally mild and infrequent (e.g., tremor, mild tachycardia, hypokalemia in rare cases) [\hyperlink{pmid_334490}{PMID: 334490}, W J Davis et al., 1977; \hyperlink{pmid_3128951}{PMID: 3128951}, J Portnoy et al., 1988; \hyperlink{pmid_16203672}{PMID: 16203672}, Mamatha Kambalapalli et al., 2005; \hyperlink{pmid_25164315}{PMID: 25164315}, Sunit Singhi et al., 2014; \hyperlink{pmid_7211775}{PMID: 7211775}, F E Simons et al., 1981; \hyperlink{pmid_6838255}{PMID: 6838255}, R Dinwiddie et al., 1983; \hyperlink{pmid_6586481}{PMID: 6586481}, G Lönnerholm et al., 1984; \hyperlink{pmid_3789327}{PMID: 3789327}, G Wettrell et al., 1986].

Summary:
- For infants (<1 year), there is insufficient controlled safety data; a single case report suggests possible risk.
- For children aged 1–16 years, multiple controlled studies affirm the safety of terbutaline sulfate for short-term and, in some cases, long-term use in asthma, with side effects generally mild and manageable.
- For children aged 7–14 years, long-term use (up to 1 year) has been studied without evidence of significant adverse effects.
- For children aged 1–5 years, safety is supported by controlled studies, but data in infants (<1 year) is lacking or limited to case reports.

\subsection*{Abstracts}
\hypertarget{pmid_9677633}{T}his study investigated the efficacy of a system for continuous nebulization of terbutaline sulphate in the treatment of acute asthmatic crises in children. The equipment consisted of a condensation nebulizer attached to a 40 liter acrylic tent placed around the patient's head. A prospective, randomized and open clinical trial was conducted. Twenty eight children, 2 to 5 year-old, in acute asthmatic crises were selected. Fourteen were nebulized with terbutaline sulphate while in the control group the aerosolization was proceeded only with half diluted physiologic serum. All patients were administered aminophyline intravenously. The parameter used to evaluate the efficacy of the terbutaline sulphate nebulizing system was clinical improvement measured by the Wood-Downes Score. Two additional parameters indicating terbutaline sulphate absorption were used: reduction of potassium seric levels and positive chronotropic effect. The group treated with terbutaline sulphate showed greater clinical improvement than control group at the 12 hour protocol evaluation as well as lower seric potassium level. A positive chronotropic effect was also observed at the final protocol evaluation. The data showed, preliminarily, that (a) the system for continuous nebulization of terbutaline sulphate was effective in treatment of children's acute asthmatic crises, and (b) there was evidence attesting to the absorption of terbutaline sulphate by the children treatment with it. [\hyperlink{Terbutaline Sulfate}{PMID: 9677633}, J P Lotufo et al., 1998]

\hypertarget{pmid_37719479}{T}erbutaline sulfate is a beta-adrenergic receptor agonist. More specific for B2 receptors, it is used as a bronchodilator in asthma. Its known side effects can include dizziness, tremors, and tachycardia. However, seizures are not among the commonly reported side effects. This is the case of a five-month-old girl who presented with focal seizures after the intake of terbutaline sulfate syrup. Other causes of the seizures were excluded through history and investigations, including an EEG and electrolyte panel. The seizures stopped on cessation of the terbutaline sulfate with no recurrence, leading us to believe that the focal seizures were an adverse effect of the terbutaline sulfate. A high index of suspicion for drug-related adverse effects should therefore be kept for a child with new onset focal seizures. [\hyperlink{Terbutaline Sulfate}{PMID: 37719479}, Falah Nadeem et al., 2023]

\hypertarget{pmid_2751767}{T}hirty asthmatic children, aged 4 to 13 years, 22 boys and 8 girls, were studied during acute asthmatic attacks. Each group of 15 children received either a 0.01 mg/kg subcutaneous injection of terbutaline or 2 puffs from terbutaline pressurized aerosol (0.25 mg/puff) inhaler through a 750-ml volumetric spacer. A slightly greater increase in PEFR following injection compared with inhalation throughout the 6 hours study period was observed. Significant increases in systolic blood pressure and pulse rate were observed only after injection. Therefore, it was concluded that inhaled terbutaline is safe and effective for treating children over 4 years of age with acute bronchospasm and has less cardiovascular side effects than injected terbutaline. [\hyperlink{Terbutaline Sulfate}{PMID: 2751767}, P Phanichyakarn et al., 1989]

\hypertarget{pmid_8875584}{T}he purpose of this study was to investigate the relative effectiveness of 0.25 mg, 0.5 mg, and 1.0 mg of terbutaline, administered via Turbuhaler, in children with mild to moderate asthma, and to register peak inspiratory flow rates through Turbuhaler (PIFTBH). Thirty-seven children in Portugal (one center) and 45 children in Sweden (one center) aged 3-10 years participated in two separate, double-blind, placebo-controlled, crossover, and randomized studies of the same design. Because of differences in other therapies for asthma and climate, combination of the two studies into one metanalysis did not appear appropriate. The children inhaled 0.25 mg, 0.5 mg, and 1.0 mg terbutaline sulfate and placebo t.i.d. for consecutive 2-week periods without washout periods. Peak expiratory flow rates (PEF) were measured at home before and 15 minutes after each inhalation in the morning, afternoon, and evening. PIFTBH was measured twice at each of four clinic visits. At the Portuguese center the increases in mean morning PEF from before to after inhalation were 32 L/min after 0.25 mg, 35 L/min after 0.5 mg, and 40 L/min after 1.0 mg. The corresponding figures in Sweden were 26 L/min, 31 L/min, and 29 L/min after 0.25 mg, 0.5 mg, and 1.0 mg, respectively. For children 3-6 years, mean values for PIFTBH were 60 L/min in Portugal (n = 15), and 58 L/min in Sweden (n = 23). In the 7-10 year group the mean PIFTBH was 72 L/min (n = 22) in Portugal, and 68 L/min (n = 22) in Sweden. We conclude that inhalation of terbutaline sulfate via Turbuhaler at a small dose of 0.25 mg resulted in good bronchodilation and was comparable to inhalations of 0.5 mg and 1.0 mg in children aged 3-10 years with mild to moderate asthma. PIFTBH were comparable to values previously recorded in healthy 6-year-old and older children and in adult asthmatic patients. [\hyperlink{Terbutaline Sulfate}{PMID: 8875584}, E Ståhl et al., 1996]

\hypertarget{pmid_334490}{F}orty-eight children with known asthma (ranging in age from 2 to 16 years) were studied during an acute attack. Each received either terbutaline or epinephrine subcutaneously in a random double-blind fashion. Measurement of heart rate, respiratory rate, and systemic arterial systolic and diastolic blood pressures and careful clinical assessment of obstruction of the airway were made before and at 15, 30, and 60 minutes after the administration of the drugs. Appreciable and significant clinical improvement was noted in 19 of the 24 patients in both groups and was of comparable magnitude. A small, but significant, increase in heart rate was noted in those patients requiring only one injection of terbutaline, suggesting that the drug's selectivity for the lung is relative not absolute. The present study demonstrates that terbutaline is an effective bronchodilator drug in acute childhood asthma. [\hyperlink{Terbutaline Sulfate}{PMID: 334490}, W J Davis et al., 1977]

\hypertarget{pmid_7041287}{T}erbutaline sulphate (Bricanyl; Keatings) aerosol or placebo aerosol was administered in a randomized fashion to 26 asthmatic children with proven exercise-induced asthma. The children were then subjected to the modified standard exercise challenge test involving running on the level for 6 minutes. Terbutaline sulphate aerosol had a marked protective effect against exercise-induced asthma in these children. Compared with placebo, a significant reduction in exercise-induced bronchospasm was achieved. The improved design of the mouthpiece, incorporating a newly introduced 'misting tube' enabled the children to handle the apparatus easily. The need to synchronize the activation of the aerosol with inhalation was eliminated. Terbutaline aerosol can be recommended to protect children affected by exercise-induced asthma. The preparation can be given prior to the exercise challenge and will offer prolonged and adequate protection against exercise asthma. [\hyperlink{Terbutaline Sulfate}{PMID: 7041287}, E G Weinberg et al., 1982]

\hypertarget{pmid_7308338}{T}en asthmatic children received regular daily therapy with terbutaline aerosol for 50 weeks. No evidence was found for adverse effects of this drug on growth, bone marrow, liver function or the cardiovascular system. All children had improved respiratory function throughout the year. In acute experiments carried out in 12 symptom-free asthmatic children with 0.5 mg terbutaline, it was demonstrated that the improvement in respiratory function, i.e. increase in FEV1, MMEF25-75\%, FVC and PEF, was quick in onset, was maintained for at least 60 min and was not accompanied by effects on pulse rate. Thus, the bronchodilator aerosol, terbutaline, can be safely used as a regular daily therapy in asthmatic children aged 7 to 14 years. [\hyperlink{Terbutaline Sulfate}{PMID: 7308338}, M I Blackhall et al., 1981]

\hypertarget{pmid_18818954}{A} randomized, open, coordinated multi-center trial compared the bacteriological and clinical efficacy and safety of orally administered ceftibuten and trimethoprim-sulfamethoxazole (TMP-SMX) in children with febrile urinary tract infection (UTI). Children aged 1 month to 12 years presenting with presumptive first-time febrile UTI were eligible for enrollment. A 2:1 assignment to treatment with ceftibuten 9 mg/kg once daily (n = 368) or TMP-SMX (3 mg + 15 mg)/kg twice daily (n = 179) for 10 days was performed. Escherichia coli was recovered in 96\% of the cases. Among the E. coli isolates, 14\% were resistant to TMP-SMX but none to ceftibuten. In the modified intention-to-treat population, the bacteriological elimination rates at follow-up did not differ significantly between patients treated with ceftibuten and those treated with TMP-SMX [91 vs. 95\%, with a 95\% confidence interval (CI) for difference of -9.7 to 1.0]. However, the clinical cure rate was significantly higher among those treated with ceftibuten (93 vs. 83\%, with a 95\% CI for difference of 2.4 to 17.0). Adverse events were similar for both regimens and consisted mainly of gastrointestinal disturbances. In conclusion, ceftibuten is a safe and effective drug for the empirical treatment of febrile UTI in young children. [\hyperlink{Terbutaline Sulfate}{PMID: 18818954}, Staffan Mårild et al., 2009]

\hypertarget{pmid_3128951}{T}welve children with severe asthma were treated in an intensive care unit with continuously nebulized terbutaline at doses between 1.0 and 12.0 mg/hour. All patients showed improvement in blood gases, pulse, and respiratory rates. None experience significant side effects. The duration of therapy ranged from 1 to 24 hours (mean = 8.3 hours), and all were able to leave the intensive care unit within one day. The use of continuously nebulized terbutaline appears to be safe and effective for the treatment of severe asthma in children in this limited experience. [\hyperlink{Terbutaline Sulfate}{PMID: 3128951}, J Portnoy et al., 1988]

\hypertarget{pmid_14689023}{F}orty seven children (6-14 years), with an acute mild or moderate attack of asthma (clinical score 3 or FEV1 > 50\% of the predicted), were treated with terbutaline sulphate, by inhalation route with a dry powder inhaler (Turbuhaler - 0,5 mg - group T; N=27, or by a nebulizer 1\% solution-in saline-compressed air (6 l/min.) group S; N=20. The children were evaluated at 5, 15, 25 and 30 minutes after the initial treatment. In both groups a significant fall of the clinical score (starting at 15 minutes) (p < 0.05) and a significant improvement of the FEV(1), VC and FEF25-75\% (starting at 5 minutes), were observed (p < 0.05). There were no significant changes in heart rates, respiratory rates and blood pressure (p > 0.05). At the end of the first treatment, the number of patients with a FEV(1) < 80\% was similar in both groups (T = 13/27 and S = 10/20). The same treatment was repeated, and all the children showed a marked improvement, except for one boy of the group T was hospitalized. In conclusion, children with mild or moderate acute attacks of asthma can be treated up to a week with an inhalation of dry powder, resulting in adequate bronchodilatation without important side effects. [\hyperlink{Terbutaline Sulfate}{PMID: 14689023}, D Solé et al., ]

\hypertarget{pmid_6838255}{P}lasma terbutaline levels and peak expiratory flow rates were measured in 5 asthmatic children using doses of 0.25 and 0.075 mg/kg. The higher dose resulted in safe, non-toxic plasma levels and returned the peak expiratory flow rate to normal. This dose (maximum 5 mg) is safe in children. [\hyperlink{Terbutaline Sulfate}{PMID: 6838255}, R Dinwiddie et al., 1983] (1) To determine the effect of intravenous terbutaline in children with acute severe asthma on parameters like heart rate, blood pressure, electrocardiogram and serum electrolytes; (2) to assess the safety profile and to evaluate the outcome of children treated with intravenous terbutaline for acute severe asthma. Retrospective study of admission records of children admitted with acute severe asthma who needed intravenous terbutaline. Children's Hospital at the Leicester Royal Infirmary, UK. 77 children with acute severe asthma admitted between April 1999 and October 2002. There was a significant increase in heart rate and significant fall in diastolic blood pressure in this cohort. Four patients required inotropic support. None of the patients had cardiac arrhythmias. Potassium supplements were required in 10 patients due to hypokalaemia. All patients improved and none required initiation of ventilation after commencing terbutaline. There was no mortality in this cohort. Terbutaline was found to be safe for use in this patient group in doses ranging between 1 and 5 microg/kg/min. Intravenous terbutaline was found to be a useful adjunct in those who failed to respond to standard initial therapy. [\hyperlink{Terbutaline Sulfate}{PMID: 6838255}, Mamatha Kambalapalli et al., 2005]

\hypertarget{pmid_6374804}{A} bronchodilator aerosol, terbutaline sulphate, was administered to 18 asthmatic children, mean age 8.0 years (range 4.9-13.7 years) in a cross-over trial using either a common actuator or a 750-ml collapsible spacer. With the spacer, peak expiratory flow rate was significantly higher (p less than 0.05) at 5, 20, and 60 min after administration of 0.25 mg terbutaline sulphate. The mean maximum value during this period was 92\% of the predicted normal value with the spacer compared to 86\% with the common actuator. The difference was significant (p less than 0.01). This study confirms previous findings that a 750-ml spacer is beneficial in bronchodilator aerosol therapy in children with asthma. [\hyperlink{Terbutaline Sulfate}{PMID: 6374804}, K G Hidinger et al., 1984]

\hypertarget{pmid_6586480}{S}even asthmatic children (8-12 years) were given terbutaline sulphate intravenously (5.5 micrograms/kg) and orally (50 micrograms/kg) one week apart. Unchanged terbutaline was measured in plasma and urine. In urine, conjugates were also assayed. The intravenous plasma concentration-time curve declined in a multiexponential manner. The terminal half-life ranged from 8.8 to 15.8 (mean 12.1) h. Body clearance (mean +/- SD) was 3.76 +/- 0.86, renal clearance 2.42 +/- 0.49 mL/min/kg. The volume of distribution at steady state was 1.57 +/- 0.19 L/kg. The extrapolated recovery of intact terbutaline in urine was 65.4 +/- 6.5\% of the dose and the total recovery 80.8 +/- 8.4\%. After oral administration, the recovery of intact terbutaline in urine was 6.2 +/- 1.1\%. Absorption was on average 33\%, but because of a mean first-pass elimination of 70\%, bioavailability was 9.5 +/- 2.4\%. It seems that children as a group have shorter terminal half-lives than adults and slightly higher weight-corrected clearances. [\hyperlink{Terbutaline Sulfate}{PMID: 6586480}, C Hultquist et al., 1984]

\hypertarget{pmid_2284942}{T}o evaluate the therapeutic effect of nebulized terbutaline in children with acute asthma, 21 children, aged 1 year and 7 months to 10 years, with acute asthma, were enrolled into this study, during the period from July to December 1989. Each patient received nebulized terbutaline (Bricanyl) 5 mg/dose over 10 minutes. The respiratory rate, pulse rate, blood pressure, peak expiratory flow rate and clinical severity score were recorded before, and at 10 minutes after treatment. Comparing with the data before treatment, respiratory rate, peak expiratory flow rate and clinical severity score at 10 minutes after treatment showed significant improvement (p value less than 0.05; less than 0.0005; less than 0.0001, respectively), but pulse rate and blood pressure did not differ significantly. It was concluded that administration of nebulized terbutaline, at a dose of 5 mg, was both safe and effective in treating acute asthma, and may be used as the first line measure in treating acute asthma in children. [\hyperlink{Terbutaline Sulfate}{PMID: 2284942}, B F Lee et al., ]

\hypertarget{pmid_7985255}{W}e wanted to assess the protective effects on exercise-induced asthma as well as the clinical efficacy and safety of increasing doses of a new sustained-release formulation of terbutaline sulphate in 17 asthmatic children aged 6-12 years (mean 9 years). Placebo, 2, 4, and 6 mg terbutaline were given b.i.d. for 14 days in a randomized, double-blind, cross-over design. At the end of each two week period, an exercise test was performed and plasma terbutaline was measured. Compared with placebo, no significant effect was seen on asthma symptoms monitored at home, or on exercise-induced asthma. The percentage falls in FEV1 after the exercise test were 36, 35, 27 and 28\%, after placebo, 4, 8 and 12 mg terbutaline/day, respectively. A small but statistically significant dose-related increase was seen in morning and evening peak expiratory flow (PEF) recordings. It is concluded that continuous treatment, even with high doses or oral terbutaline, does not offer clinically useful protection against exercise-induced asthma. [\hyperlink{Terbutaline Sulfate}{PMID: 7985255}, B Hertz et al., 1994]

\hypertarget{pmid_7211775}{I}n a double-blind dose response study in 26 children, 3, 6, or 12 microgram/kg of terbutaline sulfate was compared with 10 microgram/kg of epinephrine administered subcutaneously. In the first hour after injection, all doses of terbutaline and epinephrine resulted in improvement in mean clinical score, mean forced vital capacity, mean forced expiratory volume in the first second, and mean forced expiratory flow from 25\% to 75\% of vital capacity. Terbutaline epinephrine. However, while adverse effects following terbutaline were clinically imperceptible, epinephrine produced unpleasant headache and excitement in a few patients. Terbutaline did not change mean PaO2 or PaCO2 significantly in a subgroup of patients. The 12 microgram/kg dose of terbutaline was superior to 3 or 6 microgram/kg in relieving obstruction to airflow measured at the midportion of the vital capacity. This dose caused tremor in some children, but the tremor was not apparent to patients or their parents. [\hyperlink{Terbutaline Sulfate}{PMID: 7211775}, F E Simons et al., 1981]

\hypertarget{pmid_15572812}{T}o compare the clinical efficacy and side effects of terbutaline and salbutamol administered by metered dose inhaler and holding chamber in the mild to moderate acute exacerbations of asthma in children. The study subjects were children in the age group of 5- 15 years who presented with a mild or moderate acute exacerbation of asthma. Baseline assessment included clinical parameters and spirometry. The children were then randomized to receive salbutamol or terbutaline. Three puffs each of either 100 mcg salbutamol or 250 mcg of terbutaline were administered using 750 ml holding chamber with valve. Thirty minutes after drug administration, the children were reevaluated for clinical parameters and spirometry. Of the total 60 subjects studied, 31 were administered terbutaline and 29 salbutamol. The baseline spirometric parameters were comparable. After drug administration, all the studied variables showed significant improvement within each group. However, there were no statistically significant differences when the two groups were compared with each other. There was no significant difference in the side effects between two groups. Terbutaline and salbutamol, when administered by MDI with holding chamber, are equally efficacious in children with mild or moderate acute exacerbation of asthma. [\hyperlink{Terbutaline Sulfate}{PMID: 15572812}, Prakash Chandra et al., 2004]

\hypertarget{pmid_6586481}{N}ine children, 10-15 years old with chronic asthma, were treated for weekly periods with gradually increasing doses of oral terbutaline sulphate: 45, 79, 118 and 166 micrograms/kg (mean values) 3 times daily. There was a linear relationship between dose and steady-state plasma concentration of terbutaline within patients, but the plasma levels varied 3-fold between patients taking similar doses. Symptom score, peak expiratory flow rate (PEFR) and volume of air expelled in the first second of forced expiration (FEV1) improved with increasing doses, and the need for inhalation therapy decreased. An increase in pulse rate and tremor was measurable at all dose levels, but reported side-effects were few and mild. Linear regression analysis showed a statistically significant relationship between the plasma concentration of terbutaline and the effect on FEV1 (p less than 0.01) and PEFR (p less than 0.05) within patients. [\hyperlink{Terbutaline Sulfate}{PMID: 6586481}, G Lönnerholm et al., 1984]

\hypertarget{pmid_3789327}{T}he effect of terbutaline sulphate in slow-release (SR) tablets (Bricanyl Depot), 5 mg twice daily, was compared with that of terbutaline sulphate in ordinary tablets (Bricanyl), 2.5 mg three times daily, in a double-blind, randomized, cross-over study during 2 consecutive weeks in 10 asthmatic children. Plasma concentrations and urinary excretion of terbutaline were measured at various times during both treatment periods. The SR tablets produced a higher mean plasma concentration in the morning and a smaller peak-trough variation over the day than the ordinary ones. No differences between the two treatments were observed concerning FEV1 (forced expiratory volume in 1 s). Tremor, measured with an opto-electronic tremorgraph, was about the same for two treatments and not significantly different from tremor seen in healthy children. The reported side effects were less frequent in the SR tablet period. [\hyperlink{Terbutaline Sulfate}{PMID: 3789327}, G Wettrell et al., 1986]

\hypertarget{pmid_2019938}{T}o test whether nebulized salbutamol (albuterol) is safe and efficacious for the treatment of young children with acute bronchiolitis, we enrolled 83 children (median age 6 months, range 1 to 21 months) in a randomized, double-blind clinical trial. Participants received two treatments at 30-minute intervals of either nebulized salbutamol (0.10 mg/kg in 2 ml 0.9\% saline solution) or a similar volume of 0.9\% saline solution placebo. Outcome measures were the respiratory rate, pulse oximetry, and a clinical score based on the degree of wheezing and retractions. Patients in the salbutamol arm had significantly greater improvement in clinical scores after the initial treatment (p = 0.04). There was no difference between the groups in oxygen saturation (p = 0.74); patients treated with salbutamol had a small increase in heart rate after two treatments (159 +/- 16 vs 151 +/- 16; p = 0.03). We conclude that salbutamol is safe and effective for the initial treatment of young children with acute bronchiolitis. [\hyperlink{Terbutaline Sulfate}{PMID: 2019938}, T P Klassen et al., 1991]

\hypertarget{pmid_15247700}{M}any children with urological disease require long-term treatment with antibiotics. In many cases the choice of medical instead of surgical management hinges on the implied safety of certain drugs. Recently some groups have advocated subureteral injection procedures to avoid long-term antibiotics for low grade reflux. We present a concise and relevant review on the use and adverse reactions of nitrofurantoin, trimethoprim and sulfamethoxazole in children. We reviewed the literature regarding the safety and toxicity of these drugs. Information regarding absorption, excretion and dosing was also gathered to explain better the mechanisms of toxicity. Adverse reactions in children reported in the literature related to nitrofurantoin are gastrointestinal disturbance (4.4/100 person-years at risk), cutaneous reactions (2\% to 3\%), pulmonary toxicity (9 patients), hepatoxicity (12 patients and 3 deaths), hematological toxicity (12 patients), neurotoxicity and an increased rate of sister chromatid exchanges. Adverse reactions in children related to trimethoprim/sulfamethoxazole are almost exclusively due to the sulfamethoxazole component, including cutaneous reactions (1.4 to 7.4 events per 100 person-years at risk), hematological toxicity (0\% to 72\% of patients) and hepatotoxicity (5 patients). The majority of adverse reactions were found in children on full dose therapy and not prophylaxis. The use of nitrofurantoin, trimethoprim and sulfamethoxazole is safe in children for long-term prophylactic therapy. The antibiotic safety issue should not be misconstrued as an argument for surgical therapy, whether minimally invasive or not. Adverse reactions exist to these medicines but they are less common than seen in adults, presumably because of the lower dose used for therapy, and the lack of significant comorbidities and drug interactions in children. Serious side effects are extremely rare and most are reversible by discontinuing therapy. The extremely low potential for significant adverse reactions should be discussed with parents. [\hyperlink{Terbutaline Sulfate}{PMID: 15247700}, Edward Karpman et al., 2004]

\hypertarget{pmid_2677624}{W}e compared the use of terbutaline sulphate that was delivered by a nebulizer with its delivery by a Nebuhaler at two dose levels in 27 children (nine children per group) of between three and six years of age with acute asthma. No significant difference was found in the mean baseline clinical score among the three groups, and a significant decline occurred in the mean clinical scores in all groups by 15 minutes which was maintained to 60 minutes after the dose was administered. The decline that was achieved with delivery of the drug by way of a Nebuhaler (at either dose level) was not significantly different from that with a nebulizer, although cooperation with Nebuhaler usage was not universal in the age-group. [\hyperlink{Terbutaline Sulfate}{PMID: 2677624}, J Pendergast et al., 1989]

\hypertarget{pmid_25164315}{T}his study compared the efficacy of intravenous magnesium sulphate, terbutaline and aminophylline for children with acute, severe asthma poorly responsive to standard initial treatment. We enrolled 100 children, aged one to 12 years, who had failed to respond to initial standard treatment for acute, severe asthma, in this randomised controlled trial. They received either intravenous magnesium sulphate, terbutaline or aminophylline. Responses were monitored using a modified Clinical Asthma Severity (CAS) score. The primary outcome was treatment success, defined as a reduction in the CAS of four points or more 1 h after starting the intervention. The magnesium sulphate group had higher treatment success (33/34, 97\%) than the terbutaline and aminophylline groups (both 23/33, 70\%) (p = 0.006) and faster resolution of retractions, wheeze and dyspnoea (p < 0.001). No adverse events occurred among patients receiving magnesium sulphate, but two patients receiving terbutaline had hypokalemia and nine patients receiving aminophylline had nausea and, or, vomiting. Adding a single dose of Intravenous magnesium sulphate to inhaled beta2-agonists and corticosteroids was more effective, and safer, than using terbutaline or aminophylline when treating a child with acute severe asthma poorly responsive to initial treatment. [\hyperlink{Terbutaline Sulfate}{PMID: 25164315}, Sunit Singhi et al., 2014]

\hypertarget{pmid_18611612}{T}he safety and efficacy of cefetamet pivoxil, an oral cephalosporin of the third generation, have been studied in open, prospective, randomized comparative, clinical trials including 301 toddlers (children aged 1 to 2 years) with upper and lower respiratory tract infections, and urinary tract infections. Cefetamet pivoxil (CAT) syrup formulation was given to 177 toddlers either in the standard dose of 10 mg/kg b.i.d. [n = 116] or 20 mg/kg b.i.d. [n = 61] and 124 toddlers have been treated with comparator drugs [cefaclor, n = 98; phenoxymethylpenicillin, n = 18; amoxicillin plus clavulanic acid; n = 8]. The treatment period was 7 days mainly, except for pharyngotonsillitis for which the treatment duration was 7 or 10 days. The assessment of treatment was based on clinical signs and symptoms primarily in infections of lower respiratory tract and acute otitis media, whereas in patients with pharyngotonsillitis and acute urinary tract infections the bacteriological findings were the main evaluation criteria. The overall therapeutic outcome was successful in 148 (95.4\%) of the 155 toddlers to whom CAT was administered and in 87 (85.3\%) out of 102 toddlers receiving standard drugs. Adverse events of mild to moderate severity, mainly of gastro-intestinal type (vomiting or diarrhoea) occurred in 14.7\% in the patient group receiving CAT, 11.2\% in the toddlers receiving the standard dose of CAT, and in 12.9\% with the comparator drugs. From the data presented it is concluded that cefetamet pivoxil is efficient and safe in toddlers presenting with community-acquired respiratory and urinary infections mainly caused by S. pneumoniae, H. influenzae, Group A beta-haemolytic streptococci, M. catarrhalis, E. coli, Proteus spp. and K. pneumoniae. [\hyperlink{Terbutaline Sulfate}{PMID: 18611612}, A Chibante et al., 1994]

\section*{Zinc Chloride}
\subsection*{Result}
\subsubsection*{Answer}

Unknown

\subsubsection*{{Explanation}}
\hypertarget{Zinc Chloride}
A review of the available abstracts reveals the following evidence regarding the safety of zinc chloride in children:

- One abstract specifically addresses the ingestion of zinc chloride in a child: a 10-year-old girl accidentally ingested a zinc chloride solution (30\% to <60\%) used as an acid soldering flux. The main effects were severe gastric corrosion (local caustic injury), leading to an antral stricture of the stomach that required surgical intervention. No adverse systemic effects were observed, and chelation therapy was not needed. However, the local corrosive injury was significant, and long-term follow-up was recommended due to the risk of malignancy in the damaged stomach [\hyperlink{pmid_9574776}{PMID: 9574776}, A Yamataka et al., 1998].

- Another case report describes a child who ingested zinc chloride solution used in lead soldering. The child experienced bleeding from esophageal erosions and hematemesis, but no adverse systemic effects. Serum zinc levels were markedly elevated, and a single dose of calcium disodium edetate normalized the zinc level [\hyperlink{pmid_6784612}{PMID: 6784612}, J L Potter et al., 1981].

- A recent abstract (2023) mentions the effects of locally applied zinc chloride (ZnCl...), but the abstract is truncated and does not provide safety data or age information [\hyperlink{pmid_36515300}{PMID: 36515300}, Kevin Innella et al., 2023].

No abstracts report targeted studies evaluating the safety of zinc chloride as a supplement, medication, or therapeutic agent in children. The only available data are from accidental ingestion or local application, both of which resulted in significant local tissue injury (corrosive effects) but not systemic toxicity at the reported doses. There are no studies affirming the safety of zinc chloride for intentional use in children for any indication.

In contrast, many abstracts report on the safety and efficacy of other zinc salts (zinc sulfate, zinc acetate, zinc gluconate, zinc amino acid chelate) in children, with multiple studies affirming their safety and efficacy for supplementation or treatment of specific conditions. However, these findings cannot be extrapolated to zinc chloride.

In summary, based on the abstracts available, there are no targeted studies affirming the safety of zinc chloride for use in children for any age range. The only pediatric data are from accidental exposures, which resulted in significant local injury. Therefore, the safety of zinc chloride for use in children is unknown.

\subsection*{Abstracts}
\hypertarget{pmid_11241029}{T}he objectives were to evaluate appropriate doses of zinc acetate and its efficacy for the maintenance management of Wilson's disease in pediatric cases. Pediatric patients of 1 to 5 years of age were given 25 mg of zinc twice daily; patients of 6 to 15 years of age, if under 125 pounds body weight, were given 25 mg of zinc three times daily; and patients 16 years of age or older were given 50 mg of zinc three times daily. Patients were followed for efficacy (or over-treatment) until their 19th birthday by measuring levels of urine and plasma copper, urine and plasma zinc and through liver function tests and quantitative speech and neurologic scores. Patients were followed for toxicity by measures of blood counts, blood biochemistries, urinalysis, and clinical follow-up. Thirty-four patients, ranging in ages from 3.2 to 17.7 years of age, were included in the study. All doses met efficacy objectives of copper control, zinc levels, neurologic improvement, and maintenance of liver function except for episodes of poor compliance. No instance of over-treatment was encountered. Four patients exhibited mild and transient gastric disturbance from the zinc. Zinc therapy in pediatric patients appears to have a mildly adverse effect on the high-density lipoprotein/total cholesterol ratio, contrary to results of an earlier large study of primarily adults. In conclusion, zinc is effective and safe for the maintenance management of pediatric cases of Wilson's disease. Our data are strongest in children above 10 years of age. More work needs to be done in very young children, and the cholesterol observations need to be studied further. [\hyperlink{Zinc Chloride}{PMID: 11241029}, G J Brewer et al., 2001]

\hypertarget{pmid_18326612}{M}ultiple studies have shown the benefits of zinc supplementation among young children in high-risk populations. However, the optimal dose and safe upper level of zinc have not been determined. The objectives of this study were to measure the effects of different doses of supplemental zinc on the plasma zinc concentration, morbidity, and growth of young children; to detect any adverse effects of 10 mg supplemental Zn on markers of copper or iron status; and to determine whether any adverse effects are alleviated by providing copper with zinc. This randomized, double-masked, community-based intervention trial was conducted in 631 Ecuadorian children who were 12-30 mo old at baseline and who had initial length-for-age z scores <-1.3. Children received 1 of 5 daily supplements for 6 mo: 3, 7, or 10 mg Zn as zinc sulfate, 10 mg Zn + 0.5 mg Cu as copper sulfate, or placebo. The change in plasma zinc concentration from baseline was positively related to the zinc dose (P < 0.001). Zinc supplementation, including doses as low as 3 mg/d, reduced the incidence of diarrhea by 21-42\% (P < 0.01). There were no other significant group-wise differences. Zinc supplementation with a dose as low as 3 mg/d increased plasma zinc concentrations and reduced diarrhea incidence in the study population. There were no observed adverse effects of 10 mg Zn/d on indicators of copper or iron status. The current tolerable upper level of zinc recommended by the Institute of Medicine should be reassessed for young children. [\hyperlink{Zinc Chloride}{PMID: 18326612}, Sara E Wuehler et al., 2008]

\hypertarget{pmid_20335624}{Z}inc supplementation has proven beneficial in the treatment of acute child diarrhea and appears to enhance linear growth. There is a theoretical risk of anemia in zinc-supplemented children due to inhibited iron transport via decreased copper absorption. Although many zinc supplementation trials have included hematological measures, the potential effect of zinc on these outcomes has not been quantitatively evaluated in a comprehensive review. We performed a systematic review of randomized trials that examined the effect of zinc supplementation on hemoglobin concentrations in apparently healthy children ages 0-15 y and conducted a random effects meta-analysis of weighted mean differences (WMD) of change in hemoglobin concentrations before and after supplementation. Twenty-one randomized, controlled trials representing 3869 participants were included in the meta-analysis. The duration of treatment ranged from 4 to 15 mo; doses were typically 10-20 mg/d. Zinc supplementation did not affect changes in hemoglobin concentrations (pooled WMD: 0.8 g/L; 95\% CI: -0.6, 2.2; P = 0.27). There was no evidence for effect modification by age, zinc dosage, duration of treatment, type of control, baseline hemoglobin status, geographical or healthcare setting, or quality of the studies. These results suggest that zinc supplementation at doses typically used in randomized trials is a safe intervention with regards to hemoglobin concentrations. Some benefits might exist among children with severe anemia or zinc deficiency, which warrant further evaluation. [\hyperlink{Zinc Chloride}{PMID: 20335624}, Louise H Dekker et al., 2010]

\hypertarget{pmid_6784612}{T}his brief report describes the clinical course and management of a child who ingested a zinc chloride solution used in a lead soldering process. Injury was limited to bleeding from esophageal erosions and hematemesis. No adverse systemic effects were observed, although serum zinc levels were markedly elevated. A single small dosage of calcium disodium edetate (150 mg dissolved in 75 ml 1:5 normal saline) was effective in normalizing the serum zinc level. [\hyperlink{Zinc Chloride}{PMID: 6784612}, J L Potter et al., 1981]

\hypertarget{pmid_675401}{S}uccessful therapy with zinc sulphate is reported in 3 children suffering from acroedematitis enteropathica. [\hyperlink{Zinc Chloride}{PMID: 675401}, I L Rubin et al., 1978]

\hypertarget{pmid_19472600}{Z}inc supplementation trials carried out among children have produced variable results, depending on the specific outcomes considered and the initial characteristics of the children who were enrolled. We completed a series of meta-analyses to examine the impact of preventive zinc supplementation on morbidity; mortality; physical growth; biochemical indicators of zinc, iron, and copper status; and indicators of behavioral development, along with possible modifying effects of the intervention results. Zinc supplementation reduced the incidence of diarrhea by approximately 20\%, but the impact was limited to studies that enrolled children with a mean initial age greater than 12 months. Among the subset of studies that enrolled children with mean initial age greater than 12 months, the relative risk of diarrhea was reduced by 27\%. Zinc supplementation reduced the incidence of acute lower respiratory tract infections by approximately 15\%. Zinc supplementation yielded inconsistent impacts on malaria incidence, and too few trials are currently available to allow definitive conclusions to be drawn. Zinc supplementation had a marginal 6\% impact on overall child mortality, but there was an 18\% reduction in deaths among zinc-supplemented children older than 12 months of age. Zinc supplementation increased linear growth and weight gain by a small, but highly significant, amount. The interventions yielded a consistent, moderately large increase in mean serum zinc concentrations, and they had no significant adverse effects on indicators of iron and copper status. There were no significant effects on children's behavioral development, although the number of available studies is relatively small. The available evidence supports the need for intervention programs to enhance zinc status to reduce child morbidity and mortality and to enhance child growth. Possible strategies for delivering preventive zinc supplements are discussed. [\hyperlink{Zinc Chloride}{PMID: 19472600}, Kenneth H Brown et al., 2009]

\hypertarget{pmid_9574776}{Z}inc chloride is a powerful corrosive agent. Reports of zinc chloride ingestion are uncommon, and there is little information about its toxicity and management. The authors report the clinical course of a 10-year-old girl who accidentally ingested an acid soldering flux solution (pH, 3.0; zinc chloride, 30\% to < 60\%). Systemic effects after the ingestion were unremarkable except for lethargy. Thus, chelation therapy was not considered. Severe gastric corrosion was caused by local caustic action. An antral stricture of the stomach approximately 3 weeks after the ingestion developed, and she underwent a modified Heineke-Mikulicz antropyloroplasty. Postoperatively, she made an uneventful recovery. On follow-up, although she was tolerating a normal diet, results of a barium meal showed her stomach to be totally aperistaltic. Results of a nuclear medicine study showed moderately delayed gastric emptying. Careful long-term follow-up is necessary, because there is potential risk for malignancy in the damaged stomach. [\hyperlink{Zinc Chloride}{PMID: 9574776}, A Yamataka et al., 1998]

\hypertarget{pmid_24475083}{T}here is no official consensus regarding zinc therapy in pre-symptomatic children with Wilson Disease (WD); more data is needed. To investigate the safety and efficacy of zinc gluconate therapy for Chinese children with pre-symptomatic WD. We retrospectively analyzed pre-symptomatic children receiving zinc gluconate in a single Chinese center specialized in pediatric hepatology. Short-term follow-up data on safety and efficacy were presented, and effects of different zinc dosages were compared. 30 children (21 males) aged 2.7 to 16.8 years were followed for up to 4.4 years; 26 (87\%) children had abnormal ALT at baseline. Most patients (73\%) received higher than the currently recommended dose of elemental zinc. Zinc gluconate significantly reduced mean ALT (p<0.0001), AST (p<0.0001), GGT (p<0.0001) levels after 1 month, and urinary copper excretion after 6 months (p<0.0054). Mean direct bilirubin levels dropped significantly at 1 month (p = 0.0175), 3 months (p = 0.0010), and 6 months (p = 0.0036). Serum zinc levels gradually increased and reached a significantly higher level after 6 months (p<0.0026), reflecting good compliance with the therapy. Complete blood count parameters did not change throughout the analysis period. 8 children experienced mild and transient gastrointestinal side effects. The higher zinc dose did not affect treatment response and was not associated with different or increased side effects when compared to conventional zinc dose. In our cohort, zinc gluconate therapy for Chinese children with pre-symptomatic WD was effective, and higher initial dose of elemental zinc had the same level of efficacy as the conventional dose. [\hyperlink{Zinc Chloride}{PMID: 24475083}, Kuerbanjiang Abuduxikuer et al., 2014]

\hypertarget{pmid_36515300}{T}he effects of locally applied zinc chloride (ZnCl [\hyperlink{Zinc Chloride}{PMID: 36515300}, Kevin Innella et al., 2023] Zinc phosphide is a chemical compound that is frequently used as a rodenticide; it is a highly toxic product that is widely used, among other spaces, at home. Given that it is a highly commercialized pesticide and that there is no antidote, it is mandatory to establish favorably the clinical manifestations of the intoxication. The aim was to describe the epidemiological and clinical profile of children intoxicated with zinc phosphide attended in a toxicological center of a tertiary referral hospital. Cross-sectional, retrospective and observational study based on the medical records of 36 pediatric patients attended from 2005 to 2015 at the Centro de Información y Atención Toxicológica from Hospital General "Dr. Gaudencio González Garza", which belongs to the Instituto Mexicano del Seguro Social. The study didn't show a prevalence of gender; 66\% of patients were children between ages 1 and 2. 96\% of patients were healthy and three adolescents used the product with suicidal purposes. Zinc phosphide exposure occurred at home. Toxicity was characterized by hypotension, hypoglycemia, metabolic acidosis, abdominal pain, nausea, and vomiting; none of the patients died. In addition, neither required mechanical ventilation nor hemodialysis. The lack of knowledge of the potential toxicity of zinc phosphide and the fact that is easily reached at home allow the exposure to this product; it is an absolutely preventable risk. [\hyperlink{Zinc Chloride}{PMID: 36515300}, María Carmen Socorro Sánchez-Villegas et al., 2017]

\hypertarget{pmid_27872827}{T}o evaluate the role of zinc as add on treatment to the "recommended treatment" of nephrotic syndrome (NS) in children. All the published literature through the major databases including Medline/Pubmed, Embase, and Google Scholar were searched till 31 Of 54 citations retrieved, a total of 6 RCTs were included. Zinc was used at a dose of 10-20 mg/d, for the duration that varied from 6-12 mo. Compared to placebo, zinc reduced the frequency of relapses, induced sustained remission/no relapse, reduced the proportion of infection episodes associated with relapse with a mild adverse event in the form of metallic taste. The GRADE evidence generated was of "very low-quality". Zinc may be a useful additive in the treatment of childhood NS. The evidence generated mostly was of "very low-quality". We need more good quality RCTs in different country setting as well different subgroups of children before any firm recommendation can be made. [\hyperlink{Zinc Chloride}{PMID: 27872827}, Girish Chandra Bhatt et al., 2016]

\hypertarget{pmid_22392179}{D}iarrhea and pneumonia are the leading causes of illness and death in children <5 years of age. Zinc supplementation is effective for treatment of acute diarrhea and can prevent pneumonia. In this trial, we measured the efficacy of zinc when given to children hospitalized and treated with antibiotics for severe pneumonia. We enrolled 610 children aged 2 to 35 months who presented with severe pneumonia defined by the World Health Organization as cough and/or difficult breathing combined with lower chest indrawing. All children received standard antibiotic treatment and were randomized to receive zinc (10 mg in 2- to 11-month-olds and 20 mg in older children) or placebo daily for up to 14 days. The primary outcome was time to cessation of severe pneumonia. Zinc recipients recovered marginally faster, but this difference was not statistically significant (hazard ratio = 1.10, 95\% CI 0.94-1.30). Similarly, the risk of treatment failure was slightly but not significantly lower in those who received zinc (risk ratio = 0.88 95\% CI 0.71-1.10). Adjunct treatment with zinc reduced the time to cessation of severe pneumonia and the risk of treatment failure only marginally, if at all, in hospitalized children. [\hyperlink{Zinc Chloride}{PMID: 22392179}, Sudha Basnet et al., 2012]

\hypertarget{pmid_25825293}{Z}inc deficiency has been estimated to result in more than 450,000 child deaths annually by increasing the risk of diarrhea and pneumonia mortality. Trials of daily supplemental zinc have shown preventive benefits in childhood diarrhea with a 20\% reduction in incidence. Use of zinc in treatment of diarrhea has also been successful in shortening the duration of the episode by 10\% and reducing the number of prolonged episodes. The World Health Organization recommends that zinc supplements be used for 10-14 days for every episode of childhood diarrhea along with oral hydration and feeding. Large-scale effectiveness trials of these recommendations in Bangladesh and India have found a reduction in hospitalizations due to diarrhea and pneumonia and in child mortality. Trials have also demonstrated a reduction in the incidence childhood pneumonia with zinc supplements and some, but not all, studies have found a therapeutic benefit of zinc as adjunctive treatment along with antibiotics as well. Preventive zinc also improves the growth of children in developing countries and reduces total deaths in 1-to 4-year-old children by 18\%. Zinc supplementation is an intervention with proven effectiveness and broad application to address pneumonia and diarrhea, the two most important childhood infectious diseases globally.  [\hyperlink{Zinc Chloride}{PMID: 25825293}, Robert E Black et al., 2012] To determine whether continuing with zinc supplementation after zinc treatment (ZT) of an acute diarrhoea episode will result in additional clinical benefits beyond ZT alone. Children 6-23 months of age, living in an urban slum in Dhaka, Bangladesh with acute childhood diarrhoea (ACD), were enrolled in a randomized, double-blind field trial. All children received 10 days of ZT (20 mg/day) and were then randomized to zinc (10 mg/day) or placebo supplementation for 3 months. Weekly follow-up of all children occurred over a period of 9 months. A total of 353 subjects were enrolled, with 93\% of the zinc supplemented and 96\% of the placebo children followed for 9 months. The incidence density of ACD among those receiving zinc supplementation compared to those receiving placebo was reduced by 28\% (2.64 vs.3.66 episodes/p-y follow-up) over the 3 months while on supplementation and by 21\% (2.05 vs.2.59 episodes/p-y follow-up) over the 9 months of follow-up. There was no observed effect on the incidence of acute respiratory infections (ARIs) or on growth. Zinc supplementation after treatment provides additional preventive ACD benefits to children in early childhood. Larger, effectiveness trials of this strategy are warranted. [\hyperlink{Zinc Chloride}{PMID: 25825293}, Charles P Larson et al., 2010]

\hypertarget{pmid_15951862}{D}iagnostic and therapeutic procedures in children are made easier using sedation. However, there is no consensus about which drug should be used to achieve this. Furthermore, none of the drugs used for sedation are risk free. The aim of this work is to study sedation indications, effectiveness, and safety at our center. A prospective observational study conducted at the Pediatric Day Care Unit, King Fahad National Guard Hospital, Riyadh, Saudi Arabia. The study covered 17.5 weeks in 2 periods: May 9th 1999 to June 13th 1999 and October 31st 2001 to February 11th 2002. Children <12 years were included. Collected data included demographics, indication, drug dosing and outcome. Data were reported as mean +/- SD. We included 148 patients, age 38 +/- 30 months. Adequate sedation was achieved in 79\% after initial chloral hydrate (CH) dose of 56.9 +/- 9.3 mg/kg, in 95\% after adding 18.5 +/- 6.4 mg/kg CH and in 96\% after adding second drug. Compared to nonrespondents, first CH dose respondents were younger and lower in weight. The CH side effects were few and mild. Chloral hydrate is a safe and effective agent for sedation in children with an age and weight dependent response. [\hyperlink{Zinc Chloride}{PMID: 15951862}, Omar M Hijazi et al., 2005]

\hypertarget{pmid_24906347}{Z}inc is an essential micronutrient important for growth and for normal function of the immune system. Many children in developing countries have inadequate zinc nutrition. Routine zinc supplementation reduces the risk of respiratory infections and diarrhea, the two leading causes of morbidity and mortality in young children worldwide. In childhood diarrhea oral zinc also reduces illness duration and risk of persistent episodes. Oral zinc is therefore recommended for the treatment of acute diarrhea in young children. The results from the studies that have measured the therapeutic effect of zinc on acute respiratory infections, however, are conflicting. Moreover, the results of therapeutic zinc for childhood malaria also are so far not promising.This paper gives a brief outline of the current evidence from clinical trials on therapeutic effect of oral zinc on childhood respiratory infections, pneumonia and malaria and also of new evidence of the effect on serious bacterial illness in young infants.  [\hyperlink{Zinc Chloride}{PMID: 24906347}, Sudha Basnet et al., 2015] In a previous study, children aged 2-5 years old in Bangladesh were supplemented orally with a single dose of Vitamin A (200,000 IU) and a placebo for zinc (zinc equivalent to 20 mg of elemental zinc) everyday for 42 days (group A), zinc and a placebo for Vitamin A (group Z), zinc and Vitamin A (group AZ) or both placebos (group P). All children were orally immunised with two doses of the killed cholera vaccine containing whole cells and a recombinant B subunit of cholera toxin (CT). The number of children who responded with > or = 4-fold vibriocidal antibody (a proxy indicator of protection against cholera) was significantly greater among the zinc-supplemented groups than among the non-zinc-supplemented groups, while Vitamin A supplementation did not appear to have any effect. The sera from these children were assayed for antibody to CT. Antibody to CT is known to exert a synergistic protective effect against cholera in animal studies, and offer significantly higher short-term protection against cholera and significant short-term protection against enterotoxigenic Escherichia coli diarrhoea in humans on oral immunisation with the cholera vaccine. Children who received zinc had significantly reduced levels of serum antibodies to CT than children who received placebos only. Factorial analysis showed a trend for zinc showing a reduction in the number of children responding with CT-antibody, while Vitamin A did not appear to have any effect. Thus, zinc enhanced vibriocidal antibody response, but suppressed CT-antibody response, suggesting that zinc supplementation has different modulating effects on vibriocidal antibody response and CT-antibody response. [\hyperlink{Zinc Chloride}{PMID: 24906347}, Firdausi Qadri et al., 2004]

\hypertarget{pmid_12324294}{Z}inc supplementation in young children has been associated with reductions in the incidence and severity of diarrheal diseases, acute respiratory infections, and malaria. The objective was to evaluate the potential role of zinc as an adjunct in the treatment of acute, uncomplicated falciparum malaria; a multicenter, double-blind, randomized placebo-controlled clinical trial was undertaken. Children (n = 1087) aged 6 mo to 5 y were enrolled at sites in Ecuador, Ghana, Tanzania, Uganda, and Zambia. Children with fever and >or=2000 asexual forms of Plasmodium falciparum/ micro L in a thick blood smear received chloroquine and were randomly assigned to receive zinc (20 mg/d for infants, 40 mg/d for older children) or placebo for 4 d. There was no effect of zinc on the median time to reduction of fever (zinc group: 24.2 h; placebo group: 24.0 h; P = 0.37), a >or=75\% reduction in parasitemia from baseline in the first 72 h in 73.4\% of the zinc group and in 77.6\% of the placebo group (P = 0.11), and no significant change in hemoglobin concentration during the 3-d period of hospitalization and the 4 wk of follow-up. Mean plasma zinc concentrations were low in all children at baseline (zinc group: 8.54 +/- 3.93 micro mol/L; placebo group: 8.34 +/- 3.25 micro mol/L), but children who received zinc supplementation had higher plasma zinc concentrations at 72 h than did those who received placebo (10.95 +/- 3.63 compared with 10.16 +/- 3.25 micro mol/L, P < 0.001). Zinc does not appear to provide a beneficial effect in the treatment of acute, uncomplicated falciparum malaria in preschool children. [\hyperlink{Zinc Chloride}{PMID: 12324294},  et al., 2002]

\hypertarget{pmid_1792743}{Z}inc sulfate-enriched lactic acid lactobacterin was used in the combined treatment of 23 children with celiac disease, aged from 1 to 10 years. A group of 23 children with celiac disease who received lactic acid lactobacterin without zinc were used as control. The patients treated with lactobacterin containing zinc showed a higher increase in body mass, total protein and zinc levels in the blood serum and elevated activity of metalloenzymes-ceruloplasmin and cytochrome oxidase. [\hyperlink{Zinc Chloride}{PMID: 1792743}, I D Uspenskaia et al., ]

\hypertarget{pmid_12052800}{T}o evaluate the effect of daily zinc supplementation in children on the incidence of acute lower respiratory tract infections and pneumonia. Double masked, randomised placebo controlled trial. A slum community in New Delhi, India. 2482 children aged 6 to 30 months. Daily elemental zinc, 10 mg to infants and 20 mg to older children or placebo for four months. Both groups received single massive dose of vitamin A (100 000 IU for infants and 200 000 IU for older children) at enrollment. All households were visited weekly. Any children with cough and lower chest indrawing or respiratory rate 5 breaths per minute less than the World Health Organization criteria for fast breathing were brought to study physicians. At four months the mean plasma zinc concentration was higher in the zinc group (19.8 (SD 10.1) v 9.3 (2.1) micromol/l, P<0.001). The proportion of children who had acute lower respiratory tract infection during follow up was no different in the two groups (absolute risk reduction -0.2\%, 95\% confidence interval -3.9\% to 3.6\%). Zinc supplementation resulted in a lower incidence of pneumonia than placebo (absolute risk reduction 2.5\%, 95\% confidence interval 0.4\% to 4.6\%). After correction for multiple episodes in the same child by generalised estimating equations analysis the odds ratio was 0.74, 95\% confidence interval 0.56 to 0.99. Zinc supplementation substantially reduced the incidence of pneumonia in children who had received vitamin A. [\hyperlink{Zinc Chloride}{PMID: 12052800}, Nita Bhandari et al., 2002]

\hypertarget{pmid_24967861}{Z}inc deficiency is common in children among populations in developing areas. Zinc deficiency alters the immune system and the resistance to infections. To evaluate the effect of two zinc compounds in the prevention of acute respiratory infection and acute diarrhea. Randomized triple-blind community trial with 301 children between 2-5 years of age from six child daycare centers in Medellin, Colombia. Children were distributed in three groups receiving zinc amino acid chelate, zinc sulfate and placebo five days a week for 16 weeks. Daily symptoms of respiratory infection, acute diarrhea and side effects were evaluated. The incidence of respiratory infection was lower with zinc amino acid chelate (1.42 per 1,000 child-days) compared with placebo (3.3 per 1,000 child-days) (RR=0.43, 95\% CI: 0.196 to 0.950, p=0.049) and with zinc sulfate (1.57 per 1,000 child-days) (RR=0.90, 95\% CI 0.382 to 2.153, p=0.999). The incidence of acute diarrhea with zinc amino acid chelate (0.15 per 1,000 child-days) was lower than with placebo (0.49 per 1,000 child-days) (RR=0.32, 95\% CI 0.006 to 3.990, p=0.346) and with zinc sulfate (0.78 per 1,000 child-days) (RR=0.20, 95\% CI: 0.0043 to 1.662, p=0.361). Zinc amino acid chelate had a better effect in reducing the incidence of acute respiratory infection and acute diarrhea in preschool children when compared with the other groups. [\hyperlink{Zinc Chloride}{PMID: 24967861}, Juliana Sánchez et al., ]

\hypertarget{pmid_21713083}{Z}inc supplementation is a critical new intervention for treating diarrheal episodes in children. Recent studies suggest that administration of zinc along with new low osmolarity oral rehydration solutions / salts (ORS), can reduce the duration and severity of diarrheal episodes for up to three months. The World Health Organization (WHO) and UNICEF recommend daily 20 mg zinc supplements for 10 - 14 days for children with acute diarrhea, and 10 mg per day for infants under six months old, to curtail the severity of the episode and prevent further occurrences in the ensuing -two to three months, thereby decreasing the morbidity considerably. This article reviews the available evidence on the efficacy and safety of zinc supplementation in pediatric diarrhea and convincingly concludes that zinc supplementation has a beneficial impact on the disease outcome. [\hyperlink{Zinc Chloride}{PMID: 21713083}, Chaitali Bajait et al., 2011]

\hypertarget{pmid_19888335}{Z}inc treatment of childhood diarrhea has the potential to save 400,000 under-five lives per year in lesser developed countries. In 2004 the World Health Organization (WHO)/UNICEF revised their clinical management of childhood diarrhea guidelines to include zinc. The aim of this study was to monitor the impact of the first national campaign to scale up zinc treatment of childhood diarrhea in Bangladesh. Between September 2006 to October 2008 seven repeated ecologic surveys were carried out in four representative population strata: mega-city urban slum and urban nonslum, municipal, and rural. Households of approximately 3,200 children with an active or recent case of diarrhea were enrolled in each survey round. Caretaker awareness of zinc as a treatment for childhood diarrhea by 10 mo following the mass media launch was attained in 90\%, 74\%, 66\%, and 50\% of urban nonslum, municipal, urban slum, and rural populations, respectively. By 23 mo into the campaign, approximately 25\% of urban nonslum, 20\% of municipal and urban slum, and 10\% of rural under-five children were receiving zinc for the treatment of diarrhea. The scale-up campaign had no adverse effect on the use of oral rehydration salt (ORS). Long-term monitoring of scale-up programs identifies important gaps in coverage and provides the information necessary to document that intended outcomes are being attained and unintended consequences avoided. The scale-up of zinc treatment of childhood diarrhea rapidly attained widespread awareness, but actual use has lagged behind. Disparities in zinc coverage favoring higher income, urban households were identified, but these were gradually diminished over the two years of follow-up monitoring. The scale up campaign has not had any adverse effect on the use of ORS. Please see later in the article for the Editors' Summary. [\hyperlink{Zinc Chloride}{PMID: 19888335}, Charles P Larson et al., 2009]

\hypertarget{pmid_9401251}{I}n a zinc supplementation trial (with a significant impact on diarrheal morbidity), to evaluate effect of zinc supplementation on cellular immune status before and after 120 days of supplementation. A double blind, randomized controlled trial with immune assessment at baseline and after 120 days on supplement. Community based study in an urban slum population. Randomly selected children (zinc 38, control 48), had a Multitest CMI skin test at both times. In 66 children (zinc 22, control 34), proportions of CD3, CD4, CD8, CD16, CD20 cells and the CD/CD8 ratio were also estimated using a whole blood lysis method and flowcytometry. Zinc gluconate to provide elemental zinc 10 mg daily and 20 mg during diarrhea. Regarding CMI, the percentage of anergic or hypoergic children (using induration score) decreased from 67\% to 47\% in the zinc group, while in the control group it remained unchanged (73\% vs 71\%) (p = 0.05). The percentage of children deteriorating between first and second tests was significantly lower in the zinc group (13\% vs 33\%, p = 0.03). Regarding lymphocyte subsets, the zinc group had a significantly higher rise in the geometric means of CD3 (25\%, p = 0.02), CD4 (64\% p = 0.001), and CD4/CD8 ratio (73\% p = 0.004) with no difference in CD8 and CD20. The rise in CD4 was significantly higher in the zinc as compared to the control group; the ratio of geometric means was 1.45 (95\% CI, 1.03-2.01). Zinc supplementation improves cellular immune status, which may have been one of the mechanisms for observed impact of zinc supplementation on diarrheal morbidity. [\hyperlink{Zinc Chloride}{PMID: 9401251}, S Sazawal et al., 1997]

\hypertarget{pmid_16155274}{R}ecent studies reported that zinc significantly reduced the duration and volume of acute watery diarrhea in children aged > or = 4 mo, but there were no data specifically on infants aged < 6 mo. This study investigated the effect of zinc on the duration of illness and the stool quantity in acute watery diarrhea of infants aged 1-6 mo by comparing a 20 mg Zn/d dose with a 5 mg Zn/d dose. Infants hospitalized with at least some dehydration (by World Health Organization classification) were enrolled in a double-blind, randomized, placebo-controlled trial. Infants were randomly assigned to receive 20 mg Zn (acetate)/d, 5 mg Zn/d, or placebo for the duration of illness. Two hundred seventy-five infants were enrolled between 20 September 1998 and 18 December 2000. Neither diarrhea duration nor mean stool volume differed between groups. There were no significant differences in fluid intake, the need for unscheduled intravenous fluid, weight gain, or vomiting rates between the groups. Zinc supplementation did not affect diarrhea duration or stool volume in young infants. Young infants tolerated both zinc doses. A beneficial effect on subsequent illness cannot be ruled out. [\hyperlink{Zinc Chloride}{PMID: 16155274}, W Abdullah Brooks et al., 2005]

\section*{Nelarabine}
\subsection*{Result}
\subsubsection*{Answer}

Yes (children and adolescents, ages 1–21 years)
Unknown (infants, <1 year)

\subsubsection*{{Explanation}}
\hypertarget{Nelarabine}
A review of the available abstracts reveals multiple targeted studies evaluating the safety of nelarabine in children and adolescents with T-cell acute lymphoblastic leukemia (T-ALL) or T-cell lymphoblastic lymphoma (T-LBL). Below is a summary of the most relevant evidence, organized by age range where possible:

1. **Children and Adolescents (≤21 years):**
   - A phase 4, multicenter, single-arm, observational, open-label trial specifically evaluated nelarabine in children and young adults ≤21 years (mean age 11.5 years, N=28). Adverse events (AEs) were reported in 46\% of patients, with treatment-related AEs in 21\%. Neurological AEs were reported in four patients, but there were no AE-related treatment discontinuations or withdrawals. The safety profile was consistent with previous reports, and the study concluded that nelarabine was safe and effective in this population [\hyperlink{pmid_28771663}{PMID: 28771663}, Christian Michel Zwaan et al., 2017].
   - The Children's Oncology Group AALL00P2 study assessed the feasibility and safety of adding nelarabine to chemotherapy in children with newly diagnosed T-ALL. The only significant difference in toxicities was a decrease in neutropenic infections in the nelarabine group. The addition of nelarabine was well tolerated [\hyperlink{pmid_22734022}{PMID: 22734022}, Kimberly P Dunsmore et al., 2012].
   - The AALL0434 trial (N=1,562, ages 1-31 years) randomized intermediate- and high-risk patients to receive nelarabine or not. The addition of nelarabine improved disease-free survival without increased toxicity, indicating it was well tolerated in children and young adults [\hyperlink{pmid_32813610}{PMID: 32813610}, Kimberly P Dunsmore et al., 2020].
   - A phase II study enrolled 121 patients (children and young adults) with refractory or recurrent T-cell disease. At the final dose levels, 18\% of patients experienced grade 3 or higher neurologic adverse events, but the study concluded that nelarabine is active and the most significant adverse events are neurologic. Further studies were planned to optimize use [\hyperlink{pmid_15908649}{PMID: 15908649}, Stacey L Berg et al., 2005].
   - A phase I study determined the maximum tolerated dose in children and adults, finding dose-limiting toxicity to be neurologic in both groups. Myelosuppression and other significant organ toxicities did not occur. The study encouraged further trials in T-cell malignancies [\hyperlink{pmid_15908652}{PMID: 15908652}, J Kurtzberg et al., 2005].
   - A Japanese study included 6 pediatric patients and found that nelarabine was well tolerated, with most adverse events being hematologic and no dose-limiting toxicities observed [\hyperlink{pmid_21737993}{PMID: 21737993}, Keizo Horibe et al., 2011].
   - A small phase 1/2 trial (N=5) of nelarabine in combination with fludarabine and etoposide in children with relapsed T-ALL reported no dose-limiting toxicity and no neurotoxicity [\hyperlink{pmid_32761462}{PMID: 32761462}, Tadashi Kumamoto et al., 2020].

2. **General Pediatric Population:**
   - Multiple reviews and summaries of clinical trials confirm that nelarabine has been studied in pediatric populations, with the main dose-limiting toxicity being neurotoxicity. However, these studies generally affirm that nelarabine is safe for use in children with relapsed or refractory T-ALL/T-LBL, provided that patients are monitored for neurologic adverse events [\hyperlink{pmid_17000665}{PMID: 17000665}, Martin H Cohen et al., 2006; \hyperlink{pmid_16988579}{PMID: 16988579}, Varsha Gandhi et al., 2006; \hyperlink{pmid_20616909}{PMID: 20616909}, Kelly M Reilly et al., 2009; \hyperlink{pmid_15870141}{PMID: 15870141}, David F Kisor et al., 2005].

**Summary by Age Range:**
- For children and adolescents (up to 21 years), multiple targeted studies affirm that nelarabine is safe for use, with the main risk being neurotoxicity, which is dose-limiting but manageable with appropriate monitoring.
- For infants (<1 year), the abstracts do not provide specific safety data, so safety in this age group is unknown.

\subsection*{Abstracts}
\hypertarget{pmid_21737993}{T}he safety, tolerability, pharmacokinetics and efficacy of nelarabine were evaluated in adult and pediatric patients with relapsed or refractory T-ALL/T-LBL. Adult patients received nelarabine i.v. over 2 hours on days 1, 3 and 5 in every 21 days, and pediatric patients received this regimen over 1 hour for 5 consecutive days in every 21 days. Safety was evaluated in 7 adult and 6 pediatric patients. Adverse events (AEs) were reported in all patients. Most frequently reported AEs included somnolence and nausea in adult patients and leukopenia and lymphocytopenia in pediatric patients. Five grade 3/4 AEs were reported in both adult and pediatric patients, most of which were hematologic events. There were no dose-limiting toxicities. Efficacy was evaluated in 7 adult and 4 pediatric patients. Complete response was noted in 1 adult and 2 pediatric patients. Higher intracellular ara-GTP concentrations were suggested to be associated with efficacy. Japanese adult and pediatric patients with T-ALL/T-LBL well tolerated nelarabine treatment, warranting further investigation. [\hyperlink{Nelarabine}{PMID: 21737993}, Keizo Horibe et al., 2011]

\hypertarget{pmid_28771663}{N}elarabine is an antineoplastic agent approved for the treatment of relapsed/refractory T-lineage acute lymphoblastic leukaemia (T-ALL) or T-lineage acute lymphoblastic lymphoma (T-LBL). The purpose of this phase 4, multicentre, single-arm, observational, open-label trial was to provide additional data on the safety and efficacy of nelarabine under licensed conditions of use in children and young adults ≤21 years of age. Patients (N = 28) had a mean ± standard deviation age of 11·5 ± 4·6 years; 71\% were male and 61\% had a diagnosis of T-ALL. Adverse events (AEs) and treatment-related AEs were experienced by 46\% and 21\%, respectively, and included few haematological AEs and no haematological serious AEs. Neurological AEs from one of four predefined categories (peripheral and central nervous systems, mental status change and uncategorized) were reported in four patients. There were no AE-related treatment discontinuations/withdrawals. The overall response rate was 39.3\%: complete response (CR), 35.7\%; CR without full haematological recovery (CR*), 3.6\%. Post-treatment stem cell transplantation was performed for 46\% of the cohort. Median overall survival (OS) was 3·35 months for non-responders and not reached for responders (CR + CR*). The response rate, median OS, and safety profile of nelarabine in this disease setting and population were consistent with those reported previously. [\hyperlink{Nelarabine}{PMID: 28771663}, Christian Michel Zwaan et al., 2017]

\hypertarget{pmid_16922610}{N}elarabine is a nucleoside analog prodrug of 9-beta-D-arabinofuranosylguanine. Nelarabine is indicated for the treatment of adult and pediatric patients with T-cell acute lymphoblastic leukemia or T-cell lymphoblastic lymphoma whose disease has not responded to, or has relapsed after treatment with, at least two chemotherapy regimens. Nelarabine was granted Orphan Medical Product Status in Europe in June 2005, and accelerated approval by the US FDA in October 2005. The most common adverse events associated with nelarabine are neurotoxic in nature and have been dose-limiting. [\hyperlink{Nelarabine}{PMID: 16922610}, Andrew M Roecker et al., 2006]

\hypertarget{pmid_22734022}{C}hildren's Oncology Group study AALL00P2 was designed to assess the feasibility and safety of adding nelarabine to a BFM 86-based chemotherapy regimen in children with newly diagnosed T-cell acute lymphoblastic leukemia (T-ALL). In stage one of the study, eight patients with a slow early response (SER) by prednisone poor response (PPR; ≥ 1,000 peripheral blood blasts on day 8 of prednisone prephase) received chemotherapy plus six courses of nelarabine 400 mg/m(2) once per day; four patients with SER by high minimal residual disease (MRD; ≥ 1\% at day 36 of induction) received chemotherapy plus five courses of nelarabine; 16 patients with a rapid early response (RER) received chemotherapy without nelarabine. In stage two, all patients received six 5-day courses of nelarabine at 650 mg/m(2) once per day (10 SER patients [one by MRD, nine by PPR]) or 400 mg/m(2) once per day (38 RER patients; 12 SER patients [three by MRD, nine by PPR]). The only significant difference in toxicities was decreased neutropenic infections in patients treated with nelarabine (42\% with v 81\% without nelarabine). Five-year event-free survival (EFS) rates were 73\% for 11 stage one SER patients and 67\% for 22 stage two SER patients treated with nelarabine versus 69\% for 16 stage one RER patients treated without nelarabine and 74\% for 38 stage two RER patients treated with nelarabine. Five-year EFS for all patients receiving nelarabine (n = 70) was 73\% versus 69\% for those treated without nelarabine (n = 16). Addition of nelarabine to a BFM 86-based chemotherapy regimen was well tolerated and produced encouraging results in pediatric patients with T-ALL, particularly those with a SER, who have historically fared poorly. [\hyperlink{Nelarabine}{PMID: 22734022}, Kimberly P Dunsmore et al., 2012]

\hypertarget{pmid_17000665}{T}o describe the clinical studies, chemistry manufacturing and controls, and clinical pharmacology and toxicology that led to Food and Drug Administration approval of nelarabine (Arranon) for the treatment of T-cell acute lymphoblastic leukemia/lymphoblastic lymphoma. Two phase 2 trials, one conducted in pediatric patients and the other in adult patients, were reviewed. The i.v. dose and schedule of nelarabine in the pediatric and adult studies was 650 mg/m2/d daily for 5 days and 1,500 mg/m2 on days 1, 3, and 5, respectively. Treatments were repeated every 21 days. Study end points were the rates of complete response (CR) and CR with incomplete hematologic or bone marrow recovery (CR*). The pediatric efficacy population consisted of 39 patients who had relapsed or had been refractory to two or more induction regimens. CR to nelarabine treatment was observed in 5 (13\%) patients and CR+CR* was observed in 9 (23\%) patients. The adult efficacy population consisted of 28 patients. CR to nelarabine treatment was observed in 5 (18\%) patients and CR+CR* was observed in 6 (21\%) patients. Neurologic toxicity was dose limiting for both pediatric and adult patients. Other severe toxicities included laboratory abnormalities in pediatric patients and gastrointestinal and pulmonary toxicities in adults. On October 28, 2005, the Food and Drug Administration granted accelerated approval for nelarabine for treatment of patients with relapsed or refractory T-cell acute lymphoblastic leukemia/lymphoblastic lymphoma after at least two prior regimens. This use is based on the induction of CRs. The applicant will conduct postmarketing clinical trials to show clinical benefit (e.g., survival prolongation). [\hyperlink{Nelarabine}{PMID: 17000665}, Martin H Cohen et al., 2006]

\hypertarget{pmid_15908649}{N}elarabine (compound 506U78), a water soluble prodrug of 9-b-d-arabinofuranosylguanine, is converted to ara-GTP in T lymphoblasts. We sought to define the response rate of nelarabine in children and young adults with refractory or recurrent T-cell disease. We performed a phase II study with patients stratified as follows: stratum 1: > or = 25\% bone marrow blasts in first relapse; stratum 2: > or = 25\% bone marrow blasts in > or = second relapse; stratum 3: positive CSF; stratum 4: extramedullary (non-CNS) relapse. The initial nelarabine dose was 1.2 g/m2 daily for 5 consecutive days every 3 weeks. There were two dose de-escalations due to neurotoxicity on this or other studies. The final dose was 650 mg/m2/d for strata 1 and two patients and 400 mg/m2/d for strata 3 and four patients. We enrolled 121 patients (106 assessable for response) at the final dose levels. Complete plus partial response rates at the final dose levels were: 55\% in stratum 1; 27\% in stratum 2; 33\% in stratum 3; and 14\% in stratum 4. There were 31 episodes of > or = grade 3 neurologic adverse events in 27 patients (18\% of patients). Nelarabine is active as a single agent in recurrent T-cell leukemia, with a response rate more than 50\% in first bone marrow relapse. The most significant adverse events associated with nelarabine administration are neurologic. Further studies are planned to determine whether the addition of nelarabine to front-line therapy for T-cell leukemia in children will improve survival. [\hyperlink{Nelarabine}{PMID: 15908649}, Stacey L Berg et al., 2005]

\hypertarget{pmid_36508268}{N}elarabine, an antimetabolite prodrug, is approved as monotherapy for children and adults with relapsed and refractory T-cell acute lymphoblastic leukemia and lymphoma (R/R T-ALL/LBL), although it is often used in combination regimens. We sought to understand differences in efficacy and toxicity when nelarabine is administered alone or in combination. We retrospectively analyzed 44 consecutive patients with R/R T-ALL/LBL; 29 of whom were treated with combination therapy, most with cyclophosphamide and etoposide (23, 79\%) and 15 with monotherapy. The median age was 19 years (range, 2-69), including 18 children (<18 years). After a median of 1 (range, 1-3) cycle of treatment, 24 patients (55\%) achieved complete remission, 62\% (18/29) with combination therapy and 40\% (6/15) with monotherapy (P = .21). Most responders (21, 88\%) pursued allogeneic stem cell transplant (alloSCT). Overall survival (OS) was 12.8 months (95\% confidence interval, 6.93-not reached) in the entire cohort and was higher in the combination therapy than in the monotherapy group (24-month OS, 53\% vs 8\%; P = .003). The rate of neurotoxicity was similar between groups (27\% vs 17\%; P = .46) and grade 3/4 anemia and thrombocytopenia were more frequent in the combination group (76\% vs 20\%; P < .001\% and 66\% vs 27\%; P = .014, respectively). In a multivariate analysis, nelarabine combination therapy and alloSCT post nelarabine were associated with improved OS (hazard ratio, 0.41; P = .04 and hazard ratio, 0.25; P = .008, respectively). In conclusion, compared with monotherapy, nelarabine combination therapy was well tolerated and associated with improved survival in pediatric and adult patients with R/R T-ALL/LBL. [\hyperlink{Nelarabine}{PMID: 36508268}, Shai Shimony et al., 2023]

\hypertarget{pmid_36448227}{N}elarabine is a purine nucleoside analogue prodrug approved for the treatment of relapsed and refractory T-cell acute lymphoblastic leukemia (R/R T-ALL) and lymphoblastic lymphoma (T-LBL). Although effective in R/R T-ALL, significant neurotoxicity is dose-limiting and such neurotoxicity associated with nucleoside analogues can be related to dosing schedule. The authors conducted a phase 1 study to evaluate the pharmacokinetics and toxicity of nelarabine administered as a continuous infusion (CI) for 5 days (120 hours), rather than the standard, short-infusion approach. Twenty-nine patients with R/R T-ALL/LBL or T-cell prolymphocytic leukemia (T-PLL) were treated, with escalating doses of nelarabine from 100 to 800 mg/m Preliminary evaluation of continuous infusion schedule of nelarabine suggests that the safety profile is acceptable for this patient population, with clinical activity observed even at low doses and could broaden the use of nelarabine both as single agent and in combinations by potentially mitigating the risk of central nervous system toxicities. [\hyperlink{Nelarabine}{PMID: 36448227}, Prajwal C Boddu et al., 2023]

\hypertarget{pmid_32813610}{N}elarabine is effective in inducing remission in patients with relapsed and refractory T-cell acute lymphoblastic leukemia (T-ALL) but has not been fully evaluated in those with newly diagnosed disease. From 2007 to 2014, Children's Oncology Group trial AALL0434 (ClinicalTrials.gov identifier: NCT00408005) enrolled 1,562 evaluable patients with T-ALL age 1-31 years who received the augmented Berlin-Frankfurt-Muenster (ABFM) regimen with a 2 × 2 pseudo-factorial randomization to receive escalating-dose methotrexate (MTX) without leucovorin rescue plus pegaspargase (C-MTX) or high-dose MTX (HDMTX) with leucovorin rescue. Intermediate- and high-risk patients were also randomly assigned after induction to receive or not receive six 5-day courses of nelarabine that was incorporated into ABFM. Patients who experienced induction failure were nonrandomly assigned to HDMTX plus nelarabine. Patients with overt CNS disease (CNS3; ≥ 5 WBCs/μL with blasts) received HDMTX and were randomly assigned to receive or not receive nelarabine. All patients, except those with low-risk disease, received cranial irradiation. The 5-year event-free and overall survival rates were 83.7\% ± 1.1\% and 89.5\% ± 0.9\%, respectively. The 5-year disease-free survival (DFS) rates for patients with T-ALL randomly assigned to nelarabine (n = 323) and no nelarabine (n = 336) were 88.2\% ± 2.4\% and 82.1\% ± 2.7\%, respectively ( The addition of nelarabine to ABFM therapy improved DFS for children and young adults with newly diagnosed T-ALL without increased toxicity. [\hyperlink{Nelarabine}{PMID: 32813610}, Kimberly P Dunsmore et al., 2020]

\hypertarget{pmid_18318562}{N}elarabine is an anticancer prodrug of arabinofuranosylguanine (ara-G), which is metabolized in cells to the cytotoxic metabolite ara-G triphosphate (ara-GTP). Ara-GTP competes with deoxyguanosine triphosphate for incorporation into DNA. Once incorporated, it inhibits DNA synthesis and leads to high molecular weight DNA fragmentation and cell death. In paediatric and adult patients with T-cell acute lymphoblastic leukaemia or T-cell lymphoblastic lymphoma, nelarabine induced a complete response, with or without complete haematological recovery, in approximately one-fifth of patients who had not responded to, or had relapsed following treatment with, two or more prior chemotherapy regimens. The median overall survival time was 13.1 and 20.6 weeks in paediatric and adult patients, with corresponding 1-year survival rates of 14\% and 29\%. Treatment-emergent adverse events were common, but non-haematological events were mostly of mild or moderate severity. Neurological events, which may be severe and irreversible, were the most likely adverse events to limit treatment. [\hyperlink{Nelarabine}{PMID: 18318562}, Mark Sanford et al., 2008]

\hypertarget{pmid_15870141}{T}o present the pharmacology and pharmacokinetics of nelarabine, 9-beta-D-arabinofuranosylguanine (ara-G) and intraleukemic cellular pharmacokinetics of 9-beta-D-arabinofuranosylguanine triphosphate (ara-GTP) generated from the administration of nelarabine, and clinical and safety information relative to nelarabine use in the treatment of hematologic malignancies. MEDLINE (1966-December 2004) was searched using the English-language key terms 2-amino-6-methoxypurine arabinoside, 506U78, and nelarabine. Data were also obtained from published abstracts. Clinical studies evaluating the pharmacokinetics of nelarabine, ara-G, and cellular ara-GTP and use of nelarabine, alone or in combination with other agents for the treatment of hematologic malignancies, were included in this review. Nelarabine is the water-soluble, 6-methoxy analog of 9-beta-D-ara-G. Nelarabine is readily converted to ara-G by endogenous adenosine deaminase. The half-life of nelarabine is approximately 15 minutes compared with 2-4 hours for ara-G. The clearance of ara-G is higher in children than in adults (0.312 vs 0.213 L x h(-1) x kg(-1)). Intracellular ara-GTP elimination is slow relative to nelarabine and ara-G. In pediatric and adult patients, neurologic toxicity is dose limiting. Severe myelosuppression was not consistently observed. Major responses were seen in patients with T-cell malignancies. Patients who responded had significantly higher intracellular ara-GTP concentrations compared with those who did not respond. Nelarabine is an effective ara-G prodrug. Nelarabine has significant activity against malignant T-cells and appears to be an important addition to treatments of various leukemias. [\hyperlink{Nelarabine}{PMID: 15870141}, David F Kisor et al., 2005]

\hypertarget{pmid_25866685}{N}elarabine (ara-G; Arranon; compound 506U78) is an antineoplastic purine analog used for the treatment of refractory or relapsed T-cell acute lymphoblastic leukemia (T-ALL) and T-cell lymphoblastic lymphoma (T-LBL). The drug was granted accelerated approval in October 2005 by the US Food and Drug Administration (FDA) given the efficacy (induction of complete responses) noted in 2 single-arm trials (one in pediatric setting and one in adult patient population). The main spectra of toxicities that have been reported in these clinical trials and subsequent studies are hematological and neurological. Nelarabine induced rhabdomyolysis and increased creatinine phosphokinase (CK; CPK) levels apparently have been reported and this side effect has been added as an adverse reaction in the product monograph from the drug company during postmarketing surveillance. However, the true extent and incidence of the myotoxicity from the drug are unclear. In this paper we report a grade IV CK elevation and rhabdomyolysis in a patient with T-ALL treated with nelarabine. Given the reported finding, we examined the literature further for myotoxicity, increased CK, and/or rhabdomyolysis associated with the use of the nelarabine and report our findings.  [\hyperlink{Nelarabine}{PMID: 25866685}, Mahnur Haider et al., 2015] Nelarabine (506U78) is a soluble pro-drug of 9-beta-D-arabinofuranosylguanine (ara-G), a deoxyguanosine derivative. We treated 26 patients with T-cell acute lymphoblastic leukemia (T-ALL) and 13 with T-cell lymphoblastic lymphoma (T-LBL) with nelarabine. All patients were refractory to at least one multiagent regimen or had relapsed after achieving a complete remission. Nelarabine was administered on an alternate day schedule (days 1, 3, and 5) at 1.5 g/m(2)/day. Cycles were repeated every 22 days. The median age was 34 years (range, 16-66 years); 32 (82\%) patients were male. The rate of complete remission was 31\% (95\% confidence interval [CI], 17\%, 48\%) and the overall response rate was 41\% (95\% CI, 26\%, 58\%). The principal toxicity was grade 3 or 4 neutropenia and thrombocytopenia, occurring in 37\% and 26\% of patients, respectively. There was only one grade 4 adverse event of the nervous system, which was a reversible depressed level of consciousness. The median disease-free survival (DFS) was 20 weeks (95\% CI, 11, 56), and the median overall survival was 20 weeks (95\% CI, 13, 36). The 1-year overall survival was 28\% (95\% CI, 15\%, 43\%). Nelarabine is well tolerated and has significant antitumor activity in relapsed or refractory T-ALL and T-LBL. [\hyperlink{Nelarabine}{PMID: 25866685}, Daniel J DeAngelo et al., 2007]

\hypertarget{pmid_21151585}{N}elarabine is a nucleoside analog indicated for the treatment of adult and pediatric patients with T-cell acute lymphoblastic leukemia (T-ALL) or T-cell lymphoblastic lymphoma (T-LBL) that is refractory or has relapsed after treatment with at least two chemotherapy regimens. After being first synthesized in the late 1970s and receiving FDA approval in 2005, the appropriate use of nelarabine for refractory hematologic malignancies is still being elucidated. Nelarabine is the prodrug of 9-β-D-arabinofuranosylguanine (ara-G) which when phosphorylated intracellularly to ara-G triphosphate (ara-GTP), preferentially accumulates in cancerous T-cells. Dose-dependent toxicities, including neurotoxicity and myelosuppression, have been documented and may, in turn, limit the ability to appropriately treat the diagnosed malignancy. This article will summarize the pharmacologic properties of nelarabine and will address the current place in therapy nelarabine holds based upon the results of the available clinical trials to date. [\hyperlink{Nelarabine}{PMID: 21151585}, Andrew M Roecker et al., 2010]

\hypertarget{pmid_36053397}{N}on-steroidal anti-inflammatory drugs (NSAIDs) are commonly used in infants, children, and adolescents worldwide; however, despite sufficient evidence of the beneficial effects of NSAIDs in children and adolescents, there is a lack of comprehensive data in infants. The present review summarizes the current knowledge on the safety and efficacy of various NSAIDs used in infants for which data are available, and includes ibuprofen, dexibuprofen, ketoprofen, flurbiprofen, naproxen, diclofenac, ketorolac, indomethacin, niflumic acid, meloxicam, celecoxib, parecoxib, rofecoxib, acetylsalicylic acid, and nimesulide. The efficacy of NSAIDs has been documented for a variety of conditions, such as fever and pain. NSAIDs are also the main pillars of anti-inflammatory treatment, such as in pediatric inflammatory rheumatic diseases. Limited data are available on the safety of most NSAIDs in infants. Adverse drug reactions may be renal, gastrointestinal, hematological, or immunologic. Since NSAIDs are among the most frequently used drugs in the pediatric population, safety and efficacy studies can be performed as part of normal clinical routine, even in young infants. Available data sources, such as (electronic) medical records, should be used for safety and efficacy analyses. On a larger scale, existing data sources, e.g. adverse drug reaction programs/networks, spontaneous national reporting systems, and electronic medical records should be assessed with child-specific methods in order to detect safety signals pertinent to certain pediatric age groups or disease entities. To improve the safety of NSAIDs in infants, treatment needs to be initiated with the lowest age-appropriate or weight-based dose. Duration of treatment and amount of drug used should be regularly evaluated and maximum dose limits and other recommendations by the manufacturer or expert committees should be followed. Treatment for non-chronic conditions such as fever and acute (postoperative) pain should be kept as short as possible. Patients with chronic conditions should be regularly monitored for possible adverse effects of NSAIDs. [\hyperlink{Nelarabine}{PMID: 36053397}, Victoria C Ziesenitz et al., 2022]

\hypertarget{pmid_20616909}{N}elarabine is the prodrug of 9-beta-arabinofuranosylguanine (ara-G) and is therapeutically classified as a purine nucleoside analog. Nelarabine is converted to ara-G by adenosine deaminase and transported into cells by a nucleoside transporter. Ara-G is subsequently phosphorylated to ara-G triphosphate (ara-GTP), thereby initiating the therapeutic effect by inhibiting DNA synthesis. Nelarabine has been extensively studied in regards to its pharmacokinetics, and the data have demonstrated that ara-GTP preferentially accumulates in malignant T-cells. Clinical responses to nelarabine have been demonstrated in various T-cell malignancies and appear to correlate with a relatively high intracellular concentration of ara-GTP compared to nonresponders. Therefore, this unique drug feature of nelarabine accounts for clinical utilization in treating adult and pediatric patients with relapsed or refractory T-cell acute lymphoblastic leukemia or T-cell lymphoblastic lymphoma. Neuropathy is the most predominant adverse effect associated with nelarabine and the incidence correlates with the dose administered. Myelosuppression has been observed, with thrombocytopenia and neutropenia as the most common hematologic complications. This article reviews the pharmacology, mechanism of action, and pharmacokinetic properties of nelarabine, as well as nelarabine's clinical efficacy in T-ALL, T-LBL, and other hematologic malignancies. The toxicity profile, dosage, and administration, and areas of ongoing and future research, are also presented. [\hyperlink{Nelarabine}{PMID: 20616909}, Kelly M Reilly et al., 2009]

\hypertarget{pmid_22105561}{I}n contrast to drugs established to treat neonatal seizures, levetiracetam shows little neurotoxicity in experimental animal models and has good safety records in adults and children. Here, we present results from a survey on the off-label use of levetiracetam in newborn infants among neonatologists and pediatric neurologists in German university hospitals. [\hyperlink{Nelarabine}{PMID: 22105561}, A Koppelstäetter et al., 2011]

\hypertarget{pmid_15908652}{A} phase I study was conducted to determine the maximum-tolerated dose (MTD), toxicity profile, and pharmacokinetics of a novel purine nucleoside, nelarabine, a soluble prodrug of 9-beta-D-arabinosylguanine (araG; Nelarabine), in pediatric and adult patients with refractory hematologic malignancies. Between April 1994 and April 1997, 93 patients with refractory hematologic malignancies were treated with one to 16 cycles of study drug. Nelarabine was administered daily, as a 1-hour intravenous infusion for 5 consecutive days, every 21 to 28 days. First-cycle pharmacokinetic data, including plasma nelarabine and araG levels, were obtained on all patients treated. Intracellular phosphorylation of araG was studied in samples of leukemic blasts from selected patients. The MTDs were defined at 60 mg/kg/dose and 40 mg/kg/dose daily x 5 days in children and adults, respectively. Dose-limiting toxicity (DLT) was neurologic in both children and adults. Myelosuppression and other significant organ toxicities did not occur. Pharmacokinetic parameters were similar in children and adults. Accumulation of araGTP in leukemic blasts was correlated with cytotoxic activity. The overall response rate was 31\%. Major responses were seen in patients with T-cell malignancies, with 54\% of patients with T-lineage acute lymphoblastic leukemia achieving a complete or partial response after one to two courses of drug. Nelarabine is a novel nucleoside with significant cytotoxic activity against malignant T cells. DLT is neurologic. Phase II and III trials in patients with T-cell malignancies are encouraged. [\hyperlink{Nelarabine}{PMID: 15908652}, J Kurtzberg et al., 2005]

\hypertarget{pmid_16953392}{N}elarabine is a water-soluble prodrug of the cytotoxic deoxyguanosine analog ara-G, to which it is rapidly converted in vivo by adenosine deaminase. Nelarabine has shown activity in the treatment of T-cell malignancies, especially T-cell acute lymphoblastic leukemia. Preliminary data suggested that nelarabine might penetrate into the CSF. We therefore studied the CSF penetration of nelarabine and ara-G in a nonhuman primate model that has been highly predictive of anticancer drug distribution in humans. Nelarabine (35 mg/kg, approximately 700 mg/m2) was administered over 1 h through a surgically implanted central venous catheter to four nonhuman primates. Blood (four animals) and ventricular CSF (three animals) samples were obtained at intervals for 24 h for determination of nelarabine concentrations, which were measured by HPLC-mass spectrometry. The nelarabine plasma AUC (median+/-s.d.) was 2,820+/-1,140 microM min and the ara-G plasma AUC was 20,000+/-8,100 microM min. The terminal half-life of nelarabine in plasma was 25+/-5.2 min and clearance was 42+/-61 ml/min/kg. The terminal half-life of ara-G in plasma was 182+/-45 min. In CSF the terminal half-life of nelarabine was 77+/-28 min and of ara-G was 232+/-79 min. The AUCcsf:AUCplasma was 29+/-11\% for nelarabine and 23+/-12\% for ara-G. The excellent CSF penetration of nelarabine and ara-G supports further study of the contribution of nelarabine to the prevention and treatment of CNS leukemia. [\hyperlink{Nelarabine}{PMID: 16953392}, Stacey L Berg et al., 2007]

\hypertarget{pmid_32761462}{N}elarabine is a key drug for T-cell acute lymphoblastic leukemia (T-ALL). Fludarabine and etoposide might have synergistic effect with nelarabine by inhibiting ribonucleotide reductase and by preparing cell cycle for G1/S phase, respectively. We had started phase 1/2 multicenter clinical trial of combination chemotherapy consisted of nelarabine, fludarabine, and etoposide (FLEND therapy) for children with relapsed/refractory T-ALL which has been conducted since October 2011. Although we could not complete this trial because of recruitment difficulties, we treated five children with first-relapsed T-ALL which were enrolled in the phase 1 dose escalation study of fludarabine and etoposide with nelarabine. No dose-limiting toxicity occurred, and frequent grade 3-4 toxicity was hematological toxicity and febrile neutropenia, as expected. There was no neurotoxicity. All 2 patients who received the target dose of FLEND, in which nelarabine (650 mg/m [\hyperlink{Nelarabine}{PMID: 32761462}, Tadashi Kumamoto et al., 2020] Nelarabine (compound 506U78), a novel purine nucleoside, is a soluble pro-drug of 9-beta-D-arabinofuranosylguanine (ara-G). Nelarabine is rapidly demethoxylated in blood by adenosine deaminase to ara-G. Pre-clinical and clinical studies have demonstrated the selective cytotoxicity of ara-G to T-lineage derived cells. CALGB Protocol 59901 was a Phase II study of nelarabine in patients with systemically untreated cutaneous T-cell lymphoma (CTCL) or refractory/relapsed systemic T-cell lymphoma (STCL). The objectives were to determine response rate, remission duration and safety profile associated with nelarabine given at 1.5 g m(-2) per day on days 1, 3 and 5 as an intravenous infusion every 21 days for a minimum of two cycles and to continue up to two cycles beyond CR up to a maximum of eight cycles. Nineteen patients were enrolled in the study: 11 CTCL and eight STCL patients. Grade 3 or 4 adverse events were documented in 50\% and 28\%, respectively. In particular, 33\% of patients experienced Grade 3 or 4 neurologic toxicities. There were two partial remissions lasting 3 months and 5.5 months, respectively. Median event-free survival was 1.2 months and median overall survival was 3 months. Due to lack of efficacy and excessive toxicity, nelarabine is not recommended as monotherapy in adult patients with CTCL and STCL at this dose schedule. [\hyperlink{Nelarabine}{PMID: 32761462}, Myron S Czuczman et al., 2007]

\hypertarget{pmid_18774740}{T}o evaluate the safety and efficacy of pregabalin in the management of chemotherapy-induced neuropathic pain in patients with childhood solid tumors and leukaemia. In an open-label study, 30 children (11 boys and 19 girls; mean age 13.5 years) who were treated for solid tumors and leukaemia, and developed a painful peripheral neuropathy, were medicated with pregabalin in the daily dose of 150-300 mg for 8 weeks. Twenty-eight patients completed the 8-week follow-up. A significant and long-lasting pain relief was noted in 86\% of these patients. Median VAS score decreased by 59\% at the 8th week from baseline. Adverse effects were infrequent and transient. The treatment with pregabalin resulted in a significant improvement in pain symptoms. The use of pregabalin in children is off-label so far. However, this drug seems to be a safe and effective remedy, which could significantly broaden the therapeutic spectrum in paediatric oncological patients suffering from neuropathic pain. [\hyperlink{Nelarabine}{PMID: 18774740}, Petr Vondracek et al., 2009]

\hypertarget{pmid_25023977}{I}n spite of the high occurrence of migraine headaches in school-age children, there are currently no approved and widely accepted pharmacologic agents for migraine prophylaxis in children. Our previous open-label study in children revealed the efficacy of cinnarizine, a calcium channel blocker, in migraine prophylaxis. A placebo-controlled trial was conducted to demonstrate the efficacy and safety of cinnarizine in the prophylaxis of migraine in children. A double-blind, placebo-controlled, parallel-group study conducted in a tertiary medical center in Tehran, Iran. Children (5-17 years) who experienced migraines with and without aura, as defined on the basis of 2004 International Headache Society criteria, were recruited into the study. Children were excluded if they had complicated migraine, epilepsy, or a history of use of migraine prophylactic agents. Each participant was randomly assigned to receive cinnarizine (a single 1.5 mg/kg/day dose in children weighing less than 30 kg and a single 50 mg dose in children weighing more than 30 kg, administered at bedtime) or placebo. The frequency, severity, and duration of headaches over the trial period were assessed and adverse effects were monitored. A total of 68 children (34 in each group) with migraine were enrolled and 62 participants completed the study. After 3 months of taking cinnarizine or placebo, children in both groups experienced significantly reduced frequency, severity, and duration of headaches compared with baseline measurements (P < 0.001). However, compared with 31.3\% of children in the placebo group, 60\% of children in the cinnarizine group reported more than 50\% reduction in monthly headache frequency (P = 0.023), suggesting that cinnarizine was significantly more effective than placebo in reducing the frequency of headaches. No serious adverse effects of the medications were observed in the treated children, including no abnormal weight gain or extrapyramidal signs. Our results indicate that the use of cinnarizine at doses administered in this study is effective and safe for prophylaxis of migraine headaches in children. [\hyperlink{Nelarabine}{PMID: 25023977}, Mahmoud Reza Ashrafi et al., 2014]

\hypertarget{pmid_18376287}{L}ow-molecular-weight heparins are increasingly used for treatment of pediatric venous thromboembolic disease (VTE). Pediatric data about therapeutic doses of nadroparin are not available. To evaluate pharmacodynamics and safety of therapeutic doses of nadroparin, consecutive patients (age 0 to 18 y) with objectively diagnosed VTE and treated with nadroparin were included in this single center study over a 12-year period. All patients started with 85.5 IU/kg of nadroparin twice daily. The target therapeutic range (TTR) was set at 0.5 to 1.0 anti-Xa IU/mL 4 hours postdose. Safety end points were major bleeding and therapy-related death. A total of 84 patients were enrolled, of whom 8 patients did not undergo measurement of anti-Xa levels. Fifty-four (71\%) of 76 patients achieved TTR. The maintenance dose (mean+/-SE) was 448+/-42 IU/kg/d in neonates (<2 mo, n=6), 253+/-22 IU/kg/d in infants (2 mo to 1 y, n=10), 214+/-8 IU/kg/d in children (2 to 11 y, n=13), and 183+/-5 IU/kg/d in adolescents (12 to 18 y, n=25). Neonates required significantly more dose adjustments and time to achieve TTR than adolescents. No major bleeding or therapy-related death occurred. In summary, an age-dependent response to nadroparin exists in pediatric patients. Nadroparin therapy seems to be safe for treatment of pediatric VTE. [\hyperlink{Nelarabine}{PMID: 18376287}, C Heleen van Ommen et al., 2008]

\hypertarget{pmid_16988579}{B}oth clofarabine and nelarabine recently received an accelerated approval by the US Food and Drug Administration for use in refractory or relapsed pediatric acute lymphoblastic leukemia and in refractory-relapsed T-cell acute lymphoblastic leukemia or T-cell lymphoblastic lymphoma. Both drugs have been studied for their metabolism and mechanisms of action in preclinical investigations and for their efficacy in clinical trials. This review will summarize these investigations and will suggest future plans with these nucleoside analogs. Clofarabine and nelarabine were designed based on preclinical and clinical findings with other nucleoside analogs or normal deoxynucleotides such as dGTP. Studies in cell lines have demonstrated that triphosphate is the active metabolite for both these purine nucleoside analogs. Pharmacokinetic and pharmacodynamic investigations during clinical trials have verified the importance of triphosphate levels in achieving clinical responses. Several phase I and II clinical explorations have suggested the utility of clofarabine in acute leukemias and nelarabine in T-cell diseases. Dose-limiting toxicities were nonhematologic: hepatotoxicity for clofarabine and neurotoxicity for nelarabine. Clofarabine is the first deoxyadenosine analog that shows promise in adult and pediatric acute leukemias without untoward toxicity. Nelarabine, as expected from its design, is a drug that may be directed to T-cell diseases. [\hyperlink{Nelarabine}{PMID: 16988579}, Varsha Gandhi et al., 2006]

\section*{Zoledronic Acid}
\subsection*{Result}
\subsubsection*{Answer}

Yes (Infants and children as young as 2.5 weeks and up to 20 years, based on available studies)

\subsubsection*{{Explanation}}
\hypertarget{Zoledronic Acid}
A review of the available abstracts reveals multiple targeted studies evaluating the safety of zoledronic acid in children across a range of ages and conditions. Below is a summary of the evidence, organized by age ranges where possible:

Infants and Young Children (as young as 2.5 weeks to 6 months):
- A systematic review included studies with patients as young as 2.5 weeks up to 16.8 years, all treated with zoledronic acid for osteogenesis imperfecta. Across these studies, zoledronic acid was well-tolerated, with the most common side effects being fever and flu-like reactions. No severe adverse events were reported [\hyperlink{pmid_37339526}{PMID: 37339526}, Ines Mahmoud et al., 2023].
- Another large retrospective review included patients from 0.5 to 20 years old, reporting that adverse events were typically mild and manageable, with no severe or long-term complications identified. The most common were hypophosphatemia, acute phase reactions, and hypocalcemia, especially after the first infusion [\hyperlink{pmid_26308295}{PMID: 26308295}, Sobenna George et al., 2015].

Children (ages 6 months to 14 years):
- A study of 14 children (6 months to 14 years) with osteogenesis imperfecta found that zoledronic acid decreased fractures and pain, improved function, and had no significant long-term adverse effects. Short-term side effects (fever, bone pain) resolved with paracetamol [\hyperlink{pmid_25946526}{PMID: 25946526}, Luz María Sánchez-Sánchez et al.].
- Another study of 22 children with osteogenesis imperfecta and related conditions (mean treatment duration 3.4 years) found no difference in safety compared to pamidronate, with no significant adverse effects reported [\hyperlink{pmid_19344075}{PMID: 19344075}, J J Brown et al., 2009].

Children and Adolescents (ages 3 to 18 years):
- A retrospective study of 46 patients aged 3 to 18 years with secondary osteoporosis found zoledronic acid reduced fracture rates and pain, with no long-term side effects reported [\hyperlink{pmid_27045392}{PMID: 27045392}, Abdulmoein E Al-Agha et al., 2016].
- A larger retrospective study of 131 children and adolescents (mean age 11.43 years) with primary and secondary osteoporosis found favorable effects on fracture rate and quality of life, with no documented long-term complications [\hyperlink{pmid_26593164}{PMID: 26593164}, Abdulmoein E Al-Agha et al., 2015].
- A study of 82 children with osteogenesis imperfecta (median duration of treatment 60 months) reported only mild short-term side effects (fever, flu-like illness) and no long-term adverse effects [\hyperlink{pmid_36089714}{PMID: 36089714}, Maira Riaz et al., 2022].

Adolescents (up to 20 years):
- A retrospective review of 81 patients aged 0.5 to 20 years found that adverse events were mild and manageable, with no severe or long-term complications [\hyperlink{pmid_26308295}{PMID: 26308295}, Sobenna George et al., 2015].
- A randomized, placebo-controlled trial in adolescents with Crohn’s disease and osteopenia (n=13, ages not specified but described as adolescents) found zoledronic acid was well-tolerated, with only transient fever, arthralgias, and nausea reported [\hyperlink{pmid_20524999}{PMID: 20524999}, Anne Marie Sbrocchi et al., 2010].

General Pediatric Population:
- Multiple large cohort and retrospective studies (sample sizes up to 309 children) report that zoledronic acid is generally well-tolerated in children with bone fragility, osteoporosis, and other bone disorders. The most common side effects are mild and transient, such as fever, flu-like symptoms, and hypocalcemia, especially after the first dose. No severe or long-term adverse effects were consistently reported [\hyperlink{pmid_33508822}{PMID: 33508822}, Angelina Lim et al., 2020; \hyperlink{pmid_21820091}{PMID: 21820091}, Peter J Simm et al., 2011].
- One case report described a severe, life-threatening reaction in a medically complex 7-year-old child, highlighting the need for careful monitoring in high-risk populations [\hyperlink{pmid_26892041}{PMID: 26892041}, S Trivedi et al., 2016]. However, this appears to be an isolated case.

Summary:
Across multiple targeted studies, including randomized controlled trials, retrospective reviews, and systematic reviews, zoledronic acid has been studied in children from infancy through adolescence. The evidence consistently shows that, when used for appropriate indications and with proper monitoring, zoledronic acid is generally safe for use in children, with adverse events being mostly mild, transient, and manageable. There is no evidence from these abstracts of long-term or severe adverse effects in the general pediatric population, though rare severe reactions may occur in medically complex children.

\subsection*{Abstracts}
\hypertarget{pmid_26892041}{Z}oledronic acid is an intravenous bisphosphonate used to increase bone mineral density and reduce the risk of fractures. Its safety profile compares well with pamidronate in pediatric patients. We describe an acute, severe, life-threatening, inflammatory reaction in a child. A 7-year-old boy with complex medical problems and chronic ventilator requirements was admitted to the pediatric intensive care unit (due to ventilator needs) for zoledronic acid infusion and subsequent monitoring. His history was significant for osteoporosis secondary to immobilization with multiple fractures since 2 years of age, hypoxic-ischemic encephalopathy, quadriplegic cerebral palsy, seizure disorder, ventilator dependence, and pulmonary hypertension. He had previously been treated with four cycles of pamidronate without adverse events. He received 0.013 mg/kg of zoledronic acid infused over 30 minutes. Beginning 3 hours after completion of the infusion, he developed progressive tachycardia, fever, hypotension requiring vasopressor infusion, and increasing oxygen requirements. Laboratory studies revealed leukopenia, thrombocytopenia, elevated C-reactive protein, abnormal coagulation profile, metabolic acidosis, and negative cultures. The following day, he developed moderate acute respiratory distress syndrome and pulmonary hemorrhage requiring higher ventilatory settings, and subsequently diarrhea and abdominal distension. Initial clinical resolution was noted from the third day onward, and he was discharged on the sixth day after zoledronate administration. Our pediatric patient demonstrated an acute, severe, life-threatening reaction to zoledronic acid requiring intensive cardiorespiratory support without an underlying pre-existing inflammatory disorder. Our case highlights the importance of careful monitoring of children following zoledronic acid therapy. We recommend inpatient observation after an initial infusion of zoledronic acid in medically complex children. Children and their parents should be thoroughly counseled on the potential risks of bisphosphonate treatment, which can sometimes be severe and life threatening. [\hyperlink{Zoledronic Acid}{PMID: 26892041}, S Trivedi et al., 2016]

\hypertarget{pmid_29184807}{Z}oledronic acid (ZA), a highly potent intravenous bisphosphonate (BP), has been increasingly used in children with primary and secondary osteoporosis due to its convenience of shorter infusion time and less frequent dosing compared to pamidronate. Many studies have also demonstrated beneficial effects of ZA in other conditions such as hypercalcemia of malignancy, fibrous dysplasia (FD), chemotherapy-related osteonecrosis (ON) and metastatic bone disease. This review summarizes pharmacologic properties, mechanism of action, dosing regimen, and therapeutic outcomes of ZA in a variety of metabolic bone disorders in children. Several potential novel uses of ZA are also discussed. Safety concerns and adverse effects are also highlighted. [\hyperlink{Zoledronic Acid}{PMID: 29184807}, Sasigarn A Bowden et al., 2017]

\hypertarget{pmid_21298747}{T}he third generation bisphosphonate zoledronic acid has demonstrated efficacy in reducing skeletal-related events in adult patients with multiple cancer types that have skeletal disease. The use of zoledronic acid in pediatric oncology patients with bone metastases for the purpose of reducing pain, improving bone strength and altering the progression of metastatic disease has not been thoroughly evaluated. From October 2005 to December 2008, 19 patients at the Aflac Cancer Center received one or more doses of zoledronic acid as part of their therapy. A retrospective review of these patients was performed and information was collected including indication for treatment, toxicities, and outcomes. Most patients (n = 15) received zoledronic acid following relapse of their malignancy with metastatic disease present in one or more bony sites. Hypocalcemia and hypophosphatemia were frequent, but did not result in clinical symptoms. More significant toxicities associated with zoledronic acid, including clinically apparent renal insufficiency and osteonecrosis of the jaw, were not seen. Overall, zoledronic acid was well tolerated in this population. The benefits of zoledronic acid seen in randomized trials of adults with bone metastases have sparked interest in its use for children with metastatic cancer. The administration of zoledronic acid in pediatric oncology appears safe, and may result in improved bone strength and pain control. Further evaluation is warranted to prospectively evaluate its efficacy and long-term safety in pediatric patients with cancer and skeletal metastases. [\hyperlink{Zoledronic Acid}{PMID: 21298747}, Keith J August et al., 2011]

\hypertarget{pmid_25946526}{Z}oledronic acid or zo/edronate is a potent bisphosphonate that recently has been used in children with osteoporosis and osteogenesis imperfecta (01), so it could be an option in the treatment of children with this terrible disease that virtually condemns them to a life of pain and prostration. The aim of this study was to evaluate the clinical and biochemical conditions of pediatric patients with 01 before and after treatment with zo /edronate. We included 14 patients, median age six years (6 months to 14 years), eight (57.1 \%) males and six (42 .9\%) females, weight 19 kg (5.8-45 kg). According to the type of 01, six (42.9\%) were type I, six (42.9\%) type Ill, and two (14.2\%) type IV The functional score (Bleck) previous to treatment was 4 (1-9) and 6 (2-9) after treatment (p = 0.001). Pain intensity prior to zo/edronate was 2 (1-9) and 0 (0-2) after (p = 0.008). Previous fractures five (1-15) and post-treatment one (0-2) (p = 0.001 ). There were no significant differences in calcium, phosphorus, alkaline phosphatase, and parathyroid hormone. Zoledronic acid decreases the number of bone fractures and pain in children with osteogenesis imperfect and improves functional status. The most common side effects were fever and bone pain within five days after the infusion,which disappear paracetamol. No adverse long-term effects such as hypocalcemia or hypoparathyroidism were reported. [\hyperlink{Zoledronic Acid}{PMID: 25946526}, Luz María Sánchez-Sánchez et al., ]

\hypertarget{pmid_20524999}{P}ediatric patients with Crohn's disease often have low bone mass (osteopenia) for age. No randomized, placebo-controlled trials using zoledronic acid have ever been performed in this population. The objective of this study was to assess the efficacy of zoledronic acid in children with Crohn's disease and osteopenia. A double-blind, randomized, placebo-controlled design was used. Thirteen adolescents received either a single intravenous dose of zoledronic acid (0.066 mg/kg, max 4 mg, n= 7) or saline placebo (n= 6). The primary outcome was change in lumbar spine bone mineral density (LSBMD) z-score at 6 months. Secondary outcomes included bone markers and adverse events. At 6 months, the change in LSBMD z-score was significantly higher in the zoledronic acid group compared to placebo (0.7 vs 0.1, P < 0.001). Volumetrically adjusted LSBMD z-score also significantly increased in the treated group. This significant difference persisted until 12 months. With zoledronic acid, urinary C-telopeptide excretion decreased by 50\% at 6 months and remained suppressed at 12 months (P= 0.02), but no changes were observed with placebo. Both groups had similar adverse events which included transient fever, arthralgias, and nausea (3/7 treated, 2/6 placebo, P= NS). In this study, zoledronic acid demonstrated a significant increase in LSBMD at 6 and 12 months following a well-tolerated infusion. [\hyperlink{Zoledronic Acid}{PMID: 20524999}, Anne Marie Sbrocchi et al., 2010]

\hypertarget{pmid_17574945}{P}revious reports have shown a high frequency of hypocalcaemia and flu-like symptoms following an initial first zoledronic acid dose of 0.02-0.025 mg/kg in children. We systematically evaluated the mineral status and symptomatology of 63 children with a variety of bone disorders treated with an initial zoledronic acid dose of 0.0125 mg/kg. The Mann-Whitney U test, chi-squared test and Fisher's exact test were used as appropriate. 0.0125 mg/kg zoledronic acid reduced the incidence and intensity of hypocalcaemia but not the incidence of the flu-like symptoms compared to higher doses. Within the low dose cohort, flu-like symptoms were associated with an acute inflammatory response. Children who became hypocalcaemic received a higher dose in relation to their body mass index and body surface area. Reducing the initial zoledronic acid dose in children decreased the incidence of hypocalcaemia and thus improved safety. Dosing on the basis of body mass index or body surface area instead of body weight may further reduce the incidence of hypocalcaemia. [\hyperlink{Zoledronic Acid}{PMID: 17574945}, Craig F Munns et al., 2007]

\hypertarget{pmid_33825940}{Z}oledronic acid (ZA) is an intravenous bisphosphonate used to treat pediatric osteoporosis. Adverse events including hypocalcemia and acute phase reaction (APR) are common following first-infusion. The purpose of this report is to describe implementation of a ZA clinical practice guideline and the subsequent process changes to improve adherence to aspects of the protocol related to safety and efficacy. Quality assurance was evaluated by chart review over a 5-year period to compare the prevalence of hypocalcemia and APR to published data. A quality improvement (QI) initiative consisting of process changes including the addition of an endocrine RN to coordinate infusions and a shift to patient/family self-scheduling of infusions was conducted. The effect of the interventions on safety (completion of pre- and post-infusion bloodwork) and efficacy (receipt of all prescribed infusions) outcomes was evaluated. Seventy-two patients received 244 infusions over the period. The frequency of hypocalcemia (22\%) and APR (31\%) was consistent with prior reports. 99\% of patients received pre-infusion bloodwork, 78\% received post-first-infusion bloodwork, and 47\% received all prescribed infusions. QI initiatives increased the percentage of patients receiving post-first-infusion bloodwork from 67 to 79\% and those receiving all infusions from 62 to 74\%, but fell short of the goal of 90\%. The implementation of a standardized protocol for ZA use in children was successful in confirming patient eligibility with pre-infusion bloodwork but failed to ensure that patients obtained post-first-infusion bloodwork and received all prescribed infusions. Further efforts to systematize the management of children on ZA are needed. [\hyperlink{Zoledronic Acid}{PMID: 33825940}, David R Weber et al., 2021]

\hypertarget{pmid_27045392}{O}steoporosis is a systemic disease characterized by decreased bone density and increased tendency to develop fractures. Osteoporosis in children and adolescents is a rare disease usually secondary to Medical conditions or medications given to children. The condition affects normal bone growth and development and carries with it multiple morbidities (physical and psychological) if not corrected promptly. This study aims to share our experience with Zoledronic Acid Therapy in Pediatric patients with secondary osteoporosis. A retrospective study which included 46 patients aged 3 to 18 years. All patients received specific doses of Zoledronic acid and were followed up at King Abdulaziz University Hospital (KAUH) in Jeddah, Saudi Arabia. Clinical and laboratory data were collected for each patient from their files. Adverse events were also recorded. The use of Zoledronic Acid in children and adolescents appears to be statically significant reduce fracture rate (p=0.005), bone turnover markers (Osteocalcin p= 0.003, CTX p= 0.008) and pain frequency in symptomatic individuals (p=0.000). Careful selection of cases is required to provide maximum benefits compared to risks associated with therapy. This study demonstrates that Zoledronic acid has positive effects on clinical outcome and bone marker level as well as quality of life for Pediatric patients with Osteoporosis and their families, with no long-term side effects. [\hyperlink{Zoledronic Acid}{PMID: 27045392}, Abdulmoein E Al-Agha et al., 2016]

\hypertarget{pmid_26593164}{T}o highlight the clinical benefit, efficacy, and safety of zoledronic acid (ZA) therapy in children and adolescents with primary and secondary osteoporosis. This is a retrospective observational study of 131 children and adolescents visiting the Pediatric Endocrine Clinic at King Abdulaziz University Hospital, Jeddah, Kingdom of Saudi Arabia, between January 2002 and January 2015. Clinical and laboratory data were collected for each patient and adverse events were evaluated. The mean patient age was 11.43 years. There was a significant decrease in the number of fractures after ZA treatment for primary osteoporosis (p=0.000) and in secondary osteoporosis (p=0.005). There was a significant decrease in both osteocalcin (p=0.001) and C-terminal telopeptide (p=0.003) in patients with primary osteoporosis, as well as osteocalcin (p=0.003) and C-terminal telopeptide (p=0.008) in patients with secondary osteoporosis after treatment. The use of ZA in children and adolescent appears to have favorable effects on fracture rate and quality of life, including pain and mobility in symptomatic individuals. Intravenous ZA is comparable to other bisphosphonate agents in its efficacy and safety and features a more convenient infusion protocol with no documented long-term complications, thus, we advise its use in pediatric population. [\hyperlink{Zoledronic Acid}{PMID: 26593164}, Abdulmoein E Al-Agha et al., 2015]

\hypertarget{pmid_33508822}{L}imited evidence is available concerning experience with use of zoledronic acid (ZA) and treatment for conditions other than primary bone fragility. A retrospective review of all Royal Children Hospital patients who had been administered at least 1 dose of intravenous ZA from 2002 to 2015 was undertaken. The audit included 309 children with 228 being treated for bone fragility conditions. Of the 228, 68 had height-adjusted lumbar spine bone mineral density Z-scores available over up to a 5-year period, and median increases were +2.0 SD (median absolute deviation = 0.9) (N = 36, p value for median increase of at least 0.5 in Z-score <0.001), for patients with osteogenesis imperfecta or other primary bone fragility disorders, +1.0 SD (0.9) (N = 14, p = 0.029), for immobility conditions, +0.5 SD (0.7) (N = 10, p = 0.399), and for glucocorticoid-induced secondary osteoporosis, +0.7 SD (0.6) (N = 8, p = 0.015). 81/309 children were treated for bone abnormality indications (e.g., avascular necrosis [AVN], fibrous dysplasia, and bone cysts). Of 39 with AVN, outcome data were available for 33, with joint integrity maintained for 24/33 from 6 to 24 months after last ZA, subjective reports (22/28) of reduced pain. Reduction in bone lesion size was seen in 2/4 patients with bone cysts within 12 months of ZA commencement. This is the largest cohort of reported outcomes of ZA use in a paediatric population. Results demonstrate a good efficacy profile and associated improved bone density for osteoporotic conditions and stabilization of non-traumatic AVN with a low rate of joint collapse. [\hyperlink{Zoledronic Acid}{PMID: 33508822}, Angelina Lim et al., 2020]

\hypertarget{pmid_26308295}{Z}oledronic acid (ZA) is increasingly used in young patients with bone disorders. However, data related to the safety of ZA administration in this population are limited. The study aimed to characterize the short-term safety profile of ZA and identify risk factors for ZA-related adverse events (AEs) in young patients. This was a retrospective chart review of inpatients and outpatients less than 21 years old who received at least one ZA infusion between July 2010 and January 2014 at The Children's Hospital of Philadelphia. Eighty-one patients (56\% male; median age, 12 y; age at first infusion, 0.5 to 20 y) with diverse skeletal disorders received a total of 204 infusions. The most common indications were osteoporosis (33\% of cohort) and osteogenesis imperfecta (27.2\%). The median ZA dose was 0.025 mg/kg (interquartile range, 0.025-0.05); the median dosing interval was 6 months (range, 1 to 25.6 mo). AEs were mild and more common after the first ZA infusion in patients with no previous bisphosphonate exposure: hypophosphatemia (25.2\% of infusions), acute phase reactions (19.1\%), and hypocalcemia (16.4\%). Symptomatic hypocalcemia requiring iv calcium occurred after two infusions. ZA dose was significantly associated with hypophosphatemia, but not other AEs. Hypocalcemia was more common in patients with high bone turnover as assessed by preinfusion alkaline phosphatase levels. AEs were not associated with diagnosis, baseline serum calcium, or calcium/calcitriol supplementation. Acute AEs related to ZA infusion in youths are common, occur principally after the first ZA infusion in bisphosphonate-naive patients, and are typically mild and easily managed. Future prospective studies are needed to determine the potential long-term risks, as well as benefits, of ZA therapy in the pediatric population. [\hyperlink{Zoledronic Acid}{PMID: 26308295}, Sobenna George et al., 2015]

\hypertarget{pmid_19344075}{T}o study the safety and efficacy of zoledronic acid treatment in children with osteoporotic bone disorders. Observational study in 22 patients with osteogenesis imperfecta and related conditions who were treated at our institution with zoledronic acid. These patients had initial treatment with pamidronate. Lumbar spine z-scores, annual change in areal bone mineral density, bone mineral adjusted density, fracture number and linear growth before and after zoledronic acid treatment was commenced were compared. Patients were treated for a mean of 3.4 years with zoledronic acid after a mean of 3.75 years of pamidronate therapy. There was no difference in areal bone mineral density accrual in the first year of zoledronic acid treatment compared to the preceding year of pamidronate treatment. Lumbar spine z-scores and bone mineral adjusted density continued to increase with zoledronic acid. Number of fractures during treatment was significantly reduced compared to baseline with either bisphosphonate, with no difference between treatments. Linear growth was not affected. Zoledronic acid is at least as effective as pamidronate as treatment for paediatric osteoporosis, and has a similar safety profile. [\hyperlink{Zoledronic Acid}{PMID: 19344075}, J J Brown et al., 2009]

\hypertarget{pmid_21671363}{Z}oledronic acid, a bisphosphonate, delays progression of bone metastases in adult malignancies. Bone is a common metastatic site of advanced neuroblastoma. We previously reported efficacy of zoledronic acid in a murine model of neuroblastoma bone invasion prompting this Phase I trial of zoledronic acid with cyclophosphamide in children with neuroblastoma and bone metastases. The primary objective was to determine recommended dosing of zoledronic acid for future trials. Escalating doses of intravenous zoledronic acid were given every 28 days with oral metronomic cyclophosphamide (25 mg/m(2)/day). Toxicity, response, zoledronic acid pharmacokinetics, bone turnover markers, serum IL-6, and sIL-6R were evaluated. Twenty-one patients, median age 7.5 (range 0.8-25.6) years were treated with 2 mg/m(2) (n = 4), 3 mg/m(2) (n = 3), or 4 mg/m(2) (n = 14) zoledronic acid. Fourteen patients were evaluable for dose escalation. A median of one (range 1-18) courses was given. Two dose limiting toxicities (grade 3 hypophosphatemia) occurred at 4 mg/m(2) zoledronic acid. Other grades 3-4 toxicities included hypocalcemia (n = 2), elevated transaminases (n = 1), neutropenia (n = 2), anemia (n = 1), lymphopenia (n = 1), and hypokalemia (n = 1). Osteosclerosis contributed to fractures in one patient after 18 courses. Responses in evaluable patients included 1 partial response, 9 stable disease (median 4.5 courses, range 3-18), and 10 progressions. Zoledronic acid pharmacokinetics were similar to adults. Markers of osteoclast activity and serum IL-6 levels decreased with therapy. Zoledronic acid with metronomic cyclophosphamide is well tolerated with clinical and biologic responses in recurrent/refractory neuroblastoma. The recommended dose of zoledronic acid is 4 mg/m(2) every 28 days. [\hyperlink{Zoledronic Acid}{PMID: 21671363}, Heidi V Russell et al., 2011]

\hypertarget{pmid_36089714}{T}o evaluate the efficacy and safety of zoledronic acid in children with osteogenesis imperfecta (OI). Descriptive Study. National Institute of Child Health, Department of Endocrine and Diabetes, Karachi, Pakistan, from January 2011 to December 2020. Children, with OI registered for the treatment, were included. Zoledronic acid was given to them by intravenous infusion over 30 minutes with a dose of 0.05 mg/Kg/day for a median duration of 60 (24-96) months. To ensure safety, patients were kept for 24 hours after dose administration to monitor any short-term side effects. The patients were assessed after every 3-6 months for frequency of fracture, bone pain, and BMD. Out of 82 children [40 females (48.8\%) and 42 males (51.2\%)], 11 patients (13.4\%) had fever and 2 patients (2.4\%) had flu-like illness. No other side effects were observed. The annual fracture rate decreased overall from 2.8±1.5 to 0.2±0.5 (˂0.001) in both males (2.6±1.3 to 0.1±0.4) and females (3.1±1.7 to 0.2±0.6). Z-score on DEXA scan showed improvement in BMD overall (-3.9±2.0 to 2.2 ±1.7), in males (-3.7±1.9 to -2.1±1.7) and in females (4.1±2.1 to -2.3±1.8). There were no other long-term side effects like ocular problems, osteonecrosis of the jaw, and delayed healing of the fractures. Zoledronate use in children is associated with minimal short-term and long-term side effects with a significant improvement in BMD and decline in fracture rate. Osteogenesis imperfecta (OI), Bisphosphonates (BPs), DEXA scan, Bone mineral density (BMD). [\hyperlink{Zoledronic Acid}{PMID: 36089714}, Maira Riaz et al., 2022]

\hypertarget{pmid_37339526}{O}steogenesis imperfecta is an inherited clinically heterogeneous disorder of bone metabolism characterized by bone and skeletal fragility and an increased risk of fractures. Pamidronate infusion was the standard treatment, but zoledronic acid is increasingly used to treat children with osteogenesis imperfecta. We conducted a systematic literature review to evaluate the efficacy and safety of intravenous zoledronic acid in the treatment of osteogenesis imperfecta in pediatric patients. A systematic review of the published literature was performed according to the Preferred Reporting Items for Systematic Reviews and Meta-Analyses guidelines. Eligible articles were clinical trials and observational studies including pediatric patients (<16 years) with osteogenesis imperfecta treated with zoledronic acid. We selected articles published during the 20 past years. The selected languages were English and French. We included articles with a minimum sample size of five patients. Six articles fulfilled the selection criteria. The majority of patients were Chinese (58\%). The predominant sex was male (65\%), and the age of included patients ranged from 2.5 weeks to 16.8 years. For all patients, zoledronic infusions were administrated intravenously. The zoledronic treatment duration ranged from 1 to 3 years. Densitometry parameters before and after zoledronic treatment were evaluated and showed significant improvement both in lumbar spine-bone mineral density Z-score and femoral neck-bone mineral density Z-scores. A significant decrease in fracture rate has also been noted both in vertebral and nonvertebral fracture incidence. The two most common side effects were fever and flu-like reactions. None of the patients presented severe adverse events. Zoledronic acid appeared to be well-tolerated and effective in the treatment of pediatric osteogenesis imperfecta. [\hyperlink{Zoledronic Acid}{PMID: 37339526}, Ines Mahmoud et al., 2023]

\hypertarget{pmid_36264042}{S}evere hypercalcemia associated with vitamin D intoxication or malignancy in children is a rare and life-threatening condition. There is little published experience with Zoledronic acid in the treatment of pediatric severe hypercalcemia. Here, we present two pediatric cases of severe hypercalcemia, one due to vitamin D intoxication and the second to malignancy, in which Zoledronic acid was used as the first-line bisphosphonate in the treatment. While both cases responded well to a single dose of Zoledronic acid, the second case experienced hypocalcemia requiring calcium treatment after Zoledronic acid infusion. Our report shows that Zoledronic acid may be an effective option in the treatment of severe pediatric hypercalcemia, although patients should be followed closely after infusion due to the risk of hypocalcemia. We provide additional published evidence for the effectiveness of Zoledronic acid in correcting severe pediatric hypercalcemia and hope this will encourage future studies with larger numbers of patients. [\hyperlink{Zoledronic Acid}{PMID: 36264042}, Fatih Kilci et al., 2022]

\hypertarget{pmid_15520785}{T}he clinical side effects of the potent new bisphosphonate zoledronic acid in children are unknown. In this study of 34 children with various bone disorders, the frequency of postinfusion flu-like illness, hypocalcemia, and hypophosphatemia was 85\%, 74\%, and 82\%, respectively. No renal side effects were detected after up to 3 consecutive infusions. [\hyperlink{Zoledronic Acid}{PMID: 15520785}, Wolfgang Högler et al., 2004]

\hypertarget{pmid_31377434}{Z}oledronic Acid is a bisphosphonate used in a 4-week schedule for the treatment of bone metastases. Some randomized trials supported its role also when administered every 12 weeks. we performed a systematic review and a meta-analysis in order to evaluate the two different schedules in terms of skeletal morbidity rate (SMR), skeletal related events (SRE) and adverse events (AEs). our results showed a clinical difference favouring the 12-week schedule in terms of AEs (RR 1.17, 95\% CI 1.06-1.29). No signifcant differences were found for SMR (RR 0.97, 95\% CI 0.84-1.13) and SRE (RR 1.02, 95\% CI 0.89-1.16). Our findings support in clinical practice the 12-week schedule an alternative to the standard 4-week schedule in advanced breast and prostate cancer, in particular when the clinical comorbidities of the patients suggest a higher risk of renal failure or hypocalcaemia. [\hyperlink{Zoledronic Acid}{PMID: 31377434}, Daniele Santini et al., 2019]

\hypertarget{pmid_21114419}{B}oth bone metastases and fragility fractures due to bone loss result in considerable morbidity affecting quality of life and independence as well as placing complex demands on healthcare resources. Zoledronic acid is a widely used intravenous bisphosphonate that reduces this skeletal morbidity in both benign and malignant conditions. The incidence, clinical importance and prevention strategies to minimize side effects associated with the use of zoledronic acid are discussed with a particular focus on use in oncology where intensive monthly scheduling is required. This potentially increases the risk for adverse events over the 6-12 monthly administration used to treat benign bone diseases. A detailed understanding of the generally favorable safety profile of zoledronic acid, but particularly the potential for renal dysfunction and osteonecrosis of the jaw. When compared to many other therapies, especially in the cancer setting, the severity of adverse events related to zoledronic acid is generally mild and, with the exception of the acute phase response causing transient fever, myalgia and bone pain, side effects are infrequent. Thus, the benefits of treatment with zoledronic acid within its licensed indications almost always outweigh the risks. [\hyperlink{Zoledronic Acid}{PMID: 21114419}, Robert Coleman et al., 2011]

\hypertarget{pmid_21820091}{T}here are limited data on the use of bisphosphonate therapy for secondary osteoporoses in childhood, and no previous reports of the use of zoledronic acid in this group. We report 20 children with a variety of underlying primary diagnoses with associated secondary osteoporosis, who were treated with 3 monthly zoledronic acid for 2 years (annualised dose 0.1mg/kg/year). There was a significant improvement in lumbar spine (by 1.88 SD±1.24 over first 12 months, p<0.001) and total bone mineral density as assessed by dual energy absorptiometry (DXA) scans, with a similar increase in bone mineral content for lean tissue mass (mean increase 1.34 SD in first 12 months, p<0.001). Bone turnover was reduced with a suppression of both osteocalcin and alkaline phosphatase in the first 12 months of treatment. Skeletal architecture was improved, with increased second metacarpal cortical thickness from 2.44mm to 2.72mm (p<0.001) and improved vertebral morphometry, with 7 patients who had vertebral wedging at baseline showing improved anterior (p=0.017) and middle (p=0.001) vertebral height ratios. Aside from well reported transient side effects with the first dose, there were no adverse effects reported. No adverse effects on anthropometric parameters were seen over the course of the study. Despite all patients having sustained fragility fractures prior to treatment, no fractures were reported during the study period. Further evidence is required to confirm efficacy, with long term follow up required to assess the impact of treatment on fracture risk. [\hyperlink{Zoledronic Acid}{PMID: 21820091}, Peter J Simm et al., 2011]

\hypertarget{pmid_18383897}{Z}oledronic acid (Zometa, Novartis, Basel, Switzerland) is a new generation of bisphosphonates (BPs) with demonstrated clinical benefit in breast and prostate cancer patients with bone metastases. The safety and efficacy of intravenous zoledronic acid in lung cancer patients was assessed. In 86 patients with newly diagnosed non-small cell lung cancer (NSCLC) or small cell lung cancer (SCLC) and bone metastases, 4 mg of zoledronic acid was administered with rapid 15-minute intravenous infusion every 3-4 weeks. A total of 414 infusions were administered over a 24-month period during which a statistically significant decrease in serum calcium levels (p = 0.03) was observed. Serum alkaline phosphatase (ALP) also decreased but not significantly. With regard to clinical efficacy, 55 of our patients stabilized or reduced their need for analgesic treatment. No significant side-effects, including fever, hemodynamic instability and renal dysfunction, were seen. We conclude that the rapid infusion of zoledronic acid is safe and convenient for lung cancer patients even after the 3rd and 6th months follow-up. [\hyperlink{Zoledronic Acid}{PMID: 18383897}, Elias Kotteas et al., ]

\hypertarget{pmid_17480179}{Z}oledronic acid has been marketed for the past 5 years for the treatment of hypercalcemia of malignancy and malignant bone disease in patients with multiple myeloma or a broad range of solid tumors. The safety profile of zoledronic acid in this patient population is well established from the databases of several large, randomized, Phase III trials and postmarketing clinical experience. Zoledronic acid is well tolerated, with predictable side effects, primarily transient flu-like symptoms, which are manageable with standard treatment. Renal monitoring is recommended, with dose reductions for patients with renal dysfunction. Monitoring for electrolyte imbalance is also important as is calcium and vitamin D supplementation. Ocular complications are rare, and osteonecrosis of the jaw is uncommon and might be avoidable with appropriate dental care. [\hyperlink{Zoledronic Acid}{PMID: 17480179}, Allan Lipton et al., 2007]

\hypertarget{pmid_18318568}{Z}oledronic acid (Zometa), a third-generation amino-bisphosphonate, has been approved in the US, the EU and many other countries worldwide for the prevention of skeletal-related events in patients with bone metastases of malignancy. In several well designed trials, zoledronic acid 4 mg administered as a 15-minute infusion every 3-4 weeks was effective in reducing the occurrence of skeletal complications in patients with bone metastases secondary to multiple myeloma, breast cancer or prostate cancer. Zoledronic acid was as effective as pamidronic acid in reducing the occurrence of skeletal complications in patients with multiple myeloma or breast cancer. In patients with solid tumours other than breast or prostate cancer, zoledronic acid did not show significant clinical benefit over placebo in terms of the primary endpoint; however, some benefit of therapy in terms of secondary endpoints was observed with zoledronic acid relative to placebo. Its efficacy in a broad range of tumours and short infusion time (15 minutes) are an advantage over other available bisphosphonates. Modelled pharmacoeconomic analyses in patients with breast cancer suggested that zoledronic acid therapy is cost effective relative to no therapy with regard to the cost per quality-adjusted life-year (QALY) gained; however, results were mixed when zoledronic acid was compared with other commonly used bisphosphonates. Zoledronic acid is generally well tolerated; the risk of osteonecrosis of the jaw may be minimized by adhering to recommendations regarding dental therapy. Additional efficacy and economic data are required to definitively position zoledronic acid with respect to other bisphosphonates. Nevertheless, available clinical data indicate that zoledronic acid is an effective treatment option for the management of bone metastases of malignancy. [\hyperlink{Zoledronic Acid}{PMID: 18318568}, Sohita Dhillon et al., 2008]

\hypertarget{pmid_17553207}{T}he aim of this study was to evaluate the efficacy and safety of zoledronic acid compared with placebo in preventing bone mineral density (BMD) loss and suppressing bone markers when initiated during the first year of androgen deprivation therapy in patients with locally advanced prostate cancer. Patients were randomized to receive zoledronic acid 4 mg or placebo intravenously every 3 months. Lumbar spine (LS) and total hip BMD was measured using dual-energy x-ray absorptiometry at baseline and at week 52. N-telopeptide (NTX) and bone-specific alkaline phosphatase (BSAP) were evaluated at baseline and every 12 weeks. Safety assessments were performed throughout the study. Efficacy analyses included 106 patients and 109 patients in the zoledronic acid and placebo groups, respectively. At week 52, the least squares mean BMD percentage differences were 6.7\% for LS and 3.7\% for total hip (P < 0.0001 for both). In the zoledronic acid group, decreases in NTX ((-)14\% to (-)28\%) and BSAP ((-)31\% to (-)37\%) levels were significant and sustained; changes in NTX levels and LS BMD (r = (-)0.25; P = 0.04) and in BSAP levels and hip BMD (r = (-)0.28; P = 0.02) were significantly correlated. Only traumatic fractures were reported for 2 and 3 patients receiving zoledronic acid and placebo, respectively. One patient in each group experienced acute renal failure. Osteonecrosis of the jaw was not reported. Zoledronic acid (4 mg intravenously every 3 months) was safe and effective in preventing bone loss and reducing bone turnover in patients with prostate cancer when initiated during the first year of androgen deprivation therapy; patients with low baseline BMD experienced the greatest benefit. [\hyperlink{Zoledronic Acid}{PMID: 17553207}, Ron S Israeli et al., 2007]

\hypertarget{pmid_16225510}{P}atients with metastatic bone disease often have severe bone pain and debilitating skeletal complications. Zoledronic acid is the only bisphosphonate shown to be safe and effective in reducing skeletal-related events (SREs), including pathological fractures, spinal cord compression, and radiation or surgery to bone in patients with bone metastases from advanced prostate cancer or renal cell carcinoma (RCC). In both tumour types, zoledronic acid significantly decreased the overall risk of developing an SRE, delayed their onset and significantly reduced the incidence of SREs compared with placebo. In patients with RCC, zoledronic acid also significantly delayed the time to progression of bone lesions by 5 months compared with placebo. Zoledronic acid is safe and well tolerated with long-term use. [\hyperlink{Zoledronic Acid}{PMID: 16225510}, Fred Saad et al., 2005]

\end{document}
